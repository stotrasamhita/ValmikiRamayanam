\sect{एकोनपञ्चाशः सर्गः — सीतापहरणम्}

\twolineshloka
{सीताया वचनं श्रुत्वा दशग्रीवः प्रतापवान्}
{हस्ते हस्तं समाहत्य चकार सुमहद् वपुः} %3-49-1

\twolineshloka
{स मैथिलीं पुनर्वाक्यं बभाषे वाक्यकोविदः}
{नोन्मत्तया श्रुतौ मन्ये मम वीर्यपराक्रमौ} %3-49-2

\twolineshloka
{उद्वहेयं भुजाभ्यां तु मेदिनीमम्बरे स्थितः}
{आपिबेयं समुद्रं च मृत्युं हन्यां रणे स्थितः} %3-49-3

\twolineshloka
{अर्कं तुद्यां शरैस्तीक्ष्णैर्विभिन्द्यां हि महीतलम्}
{कामरूपेण उन्मत्ते पश्य मां कामरूपिणम्} %3-49-4

\twolineshloka
{एवमुक्तवतस्तस्य रावणस्य शिखिप्रभे}
{क्रुद्धस्य हरिपर्यन्ते रक्ते नेत्रे बभूवतुः} %3-49-5

\twolineshloka
{सद्यः सौम्यं परित्यज्य तीक्ष्णरूपं स रावणः}
{स्वं रूपं कालरूपाभं भेजे वैश्रवणानुजः} %3-49-6

\twolineshloka
{संरक्तनयनः श्रीमांस्तप्तकाञ्चनभूषणः}
{क्रोधेन महताविष्टो नीलजीमूतसन्निभः} %3-49-7

\twolineshloka
{दशास्यो विंशतिभुजो बभूव क्षणदाचरः}
{स परिव्राजकच्छद्म महाकायो विहाय तत्} %3-49-8

\twolineshloka
{प्रतिपेदे स्वकं रूपं रावणो राक्षसाधिपः}
{रक्ताम्बरधरस्तस्थौ स्त्रीरत्नं प्रेक्ष्य मैथिलीम्} %3-49-9

\twolineshloka
{स तामसितकेशान्तां भास्करस्य प्रभामिव}
{वसनाभरणोपेतां मैथिलीं रावणोऽब्रवीत्} %3-49-10

\twolineshloka
{त्रिषु लोकेषु विख्यातं यदि भर्तारमिच्छसि}
{मामाश्रय वरारोहे तवाहं सदृशः पतिः} %3-49-11

\twolineshloka
{मां भजस्व चिराय त्वमहं श्लाघ्यः पतिस्तव}
{नैव चाहं क्वचिद् भद्रे करिष्ये तव विप्रियम्} %3-49-12

\twolineshloka
{त्यज्यतां मानुषो भावो मयि भावः प्रणीयताम्}
{राज्याच्च्युतमसिद्धार्थं रामं परिमितायुषम्} %3-49-13

\twolineshloka
{कैर्गुणैरनुरक्तासि मूढे पण्डितमानिनि}
{यः स्त्रियो वचनाद् राज्यं विहाय ससुहृज्जनम्} %3-49-14

\twolineshloka
{अस्मिन् व्यालानुचरिते वने वसति दुर्मतिः}
{इत्युक्त्वा मैथिलीं वाक्यं प्रियार्हां प्रियवादिनीम्} %3-49-15

\twolineshloka
{अभिगम्य सुदुष्टात्मा राक्षसः काममोहितः}
{जग्राह रावणः सीतां बुधः खे रोहिणीमिव} %3-49-16

\twolineshloka
{वामेन सीतां पद्माक्षीं मूर्धजेषु करेण सः}
{ऊर्वोस्तु दक्षिणेनैव परिजग्राह पाणिना} %3-49-17

\twolineshloka
{तं दृष्ट्वा गिरिशृङ्गाभं तीक्ष्णदंष्ट्रं महाभुजम्}
{प्राद्रवन् मृत्युसङ्काशं भयार्ता वनदेवताः} %3-49-18

\twolineshloka
{स च मायामयो दिव्यः खरयुक्तः खरस्वनः}
{प्रत्यदृश्यत हेमाङ्गो रावणस्य महारथः} %3-49-19

\twolineshloka
{ततस्तां परुषैर्वाक्यैरभितर्ज्य महास्वनः}
{अङ्केनादाय वैदेहीं रथमारोपयत् तदा} %3-49-20

\twolineshloka
{सा गृहीतातिचुक्रोश रावणेन यशस्विनी}
{रामेति सीता दुःखार्ता रामं दूरं गतं वने} %3-49-21

\twolineshloka
{तामकामां स कामार्तः पन्नगेन्द्रवधूमिव}
{विचेष्टमानामादाय उत्पपाताथ रावणः} %3-49-22

\twolineshloka
{ततः सा राक्षसेन्द्रेण ह्रियमाणा विहायसा}
{भृशं चुक्रोश मत्तेव भ्रान्तचित्ता यथातुरा} %3-49-23

\twolineshloka
{हा लक्ष्मण महाबाहो गुरुचित्तप्रसादक}
{ह्रियमाणां न जानीषे रक्षसा कामरूपिणा} %3-49-24

\twolineshloka
{जीवितं सुखमर्थं च धर्महेतोः परित्यजन्}
{ह्रियमाणामधर्मेण मां राघव न पश्यसि} %3-49-25

\twolineshloka
{ननु नामाविनीतानां विनेतासि परन्तप}
{कथमेवंविधं पापं न त्वं शाधि हि रावणम्} %3-49-26

\twolineshloka
{न तु सद्योऽविनीतस्य दृश्यते कर्मणः फलम्}
{कालोऽप्यङ्गीभवत्यत्र सस्यानामिव पक्तये} %3-49-27

\twolineshloka
{त्वं कर्म कृतवानेतत् कालोपहतचेतनः}
{जीवितान्तकरं घोरं रामाद् व्यसनमाप्नुहि} %3-49-28

\twolineshloka
{हन्तेदानीं सकामा तु कैकेयी बान्धवैः सह}
{ह्रियेयं धर्मकामस्य धर्मपत्नी यशस्विनः} %3-49-29

\twolineshloka
{आमन्त्रये जनस्थाने कर्णिकारांश्च पुष्पितान्}
{क्षिप्रं रामाय शंसध्वं सीतां हरति रावणः} %3-49-30

\twolineshloka
{हंससारससङ्घुष्टां वन्दे गोदावरीं नदीम्}
{क्षिप्रं रामाय शंस त्वं सीतां हरति रावणः} %3-49-31

\twolineshloka
{दैवतानि च यान्यस्मिन् वने विविधपादपे}
{नमस्करोम्यहं तेभ्यो भर्तुः शंसत मां हृताम्} %3-49-32

\twolineshloka
{यानि कानिचिदप्यत्र सत्त्वानि विविधानि च}
{सर्वाणि शरणं यामि मृगपक्षिगणानि वै} %3-49-33

\twolineshloka
{ह्रियमाणां प्रियां भर्तुः प्राणेभ्योऽपि गरीयसीम्}
{विवशा ते हृता सीता रावणेनेति शंसत} %3-49-34

\twolineshloka
{विदित्वा तु महाबाहुरमुत्रापि महाबलः}
{आनेष्यति पराक्रम्य वैवस्वतहृतामपि} %3-49-35

\twolineshloka
{सा तदा करुणा वाचो विलपन्ती सुदुःखिता}
{वनस्पतिगतं गृध्रं ददर्शायतलोचना} %3-49-36

\twolineshloka
{सा तमुद्वीक्ष्य सुश्रोणी रावणस्य वशङ्गता}
{समाक्रन्दद् भयपरा दुःखोपहतया गिरा} %3-49-37

\twolineshloka
{जटायो पश्य मामार्य ह्रियमाणामनाथवत्}
{अनेन राक्षसेन्द्रेणाकरुणं पापकर्मणा} %3-49-38

\twolineshloka
{नैष वारयितुं शक्यस्त्वया क्रूरो निशाचरः}
{सत्ववाञ्जितकाशी च सायुधश्चैव दुर्मतिः} %3-49-39

\twolineshloka
{रामाय तु यथातत्त्वं जटायो हरणं मम}
{लक्ष्मणाय च तत् सर्वमाख्यातव्यमशेषतः} %3-49-40


॥इत्यार्षे श्रीमद्रामायणे वाल्मीकीये आदिकाव्ये अरण्यकाण्डे सीतापहरणम् नाम एकोनपञ्चाशः सर्गः ॥३-४९॥
