\sect{चतुर्दशः सर्गः — जटायुस्सङ्गमः}

\twolineshloka
{अथ पञ्चवटीम् गच्चन्न् अन्तरा रघुनन्दनः}
{आससाद महाकायम् गृध्रम् भीम पराक्रमम्} %3-14-1

\twolineshloka
{तम् दृष्ट्वा तौ महाभागौ वनस्थम् राम लक्ष्मणौ}
{मेनाते राक्षसम् पक्षिम् ब्रुवाणौ को भवान् इति} %3-14-2

\twolineshloka
{स तौ मधुरया वाचा सौम्यया प्रीणयन्न् इव}
{उवाच वत्स माम् विद्धि वयस्यम् पितुर् आत्मनः} %3-14-3

\twolineshloka
{स तम् पितृ सखम् मत्वा पूजयामास राघवः}
{स तस्य कुलम् अव्यग्रम् अथ पप्रच्छ नाम च} %3-14-4

\twolineshloka
{रामस्य वचनम् श्रुत्वा कुलम् आत्मानम् एव च}
{आचचक्षे द्विजः तस्मै सर्वभूत समुद्भवम्} %3-14-5

\twolineshloka
{पूर्वकाले महाबाहो ये प्रजापतयो अभवन्}
{तान् मे निगदतः सर्वान् आदितः शृणु राघव} %3-14-6

\twolineshloka
{कर्दमः प्रथमः तेषाम् विकृतः तद् अनन्तरम्}
{शेषः च संश्रयः चैव बहु पुत्रः च वीर्यवान्} %3-14-7

\twolineshloka
{स्थाणुर् मरीचिर् अत्रिः च क्रतुः चैव महाबलः}
{पुलस्त्यः च अङ्गिराः चैव प्रचेताः पुलहः तथा} %3-14-8

\twolineshloka
{दक्षो विवस्वान् अपरो अरिष्टनेमिः च राघव}
{कश्यपः च महातेजाः तेषाम् आसीत् च पश्चिमः} %3-14-9

\twolineshloka
{प्रजापतेः तु दक्षस्य बभूवुर् इति विश्रुतम्}
{षष्टिर् दुहितरो राम यशस्विन्यो महायशः} %3-14-10

\twolineshloka
{कश्यपः प्रतिजग्राह तासाम् अष्टौ सुमध्यमाः}
{अदितिम् च दितिम् चैव दनूम् अपि च कालकाम्} %3-14-11

\twolineshloka
{ताम्राम् क्रोध वशाम् चैव मनुम् च अप्य् अनलाम् अपि}
{ताः तु कन्याः ततः प्रीतः कश्यपः पुनर् अब्रवीत्} %3-14-12

\twolineshloka
{पुत्रामः त्रैलोक्य भर्तॄन् वै जनयिष्यथ मत् समान्}
{अदितिः तन् मना राम दितिः च दनुर् एव च} %3-14-13

\twolineshloka
{कालका च महाबाहो शेषाः तु अमनसो अभवन्}
{अदित्याम् जज्ञिरे देवाः त्रयः त्रिंशत् अरिन्दम} %3-14-14

\twolineshloka
{आदित्या वसवो रुद्रा अश्विनौ च परन्तप}
{दितिः तु अजनयत् पुत्रान् दैत्याम् तात यशस्विनः} %3-14-15

\twolineshloka
{तेषाम् इयम् वसुमती पुरा आसीत् स वन अर्णवा}
{दनुः तु अजनयत् पुत्रम् अश्वग्रीवम् अरिन्दम} %3-14-16

\twolineshloka
{नरकम् कालकम् चैव कालका अपि व्यजायत}
{क्रौन्चीम् भासीम् तथा श्येनीम् धृतराष्ट्रीम् तथा शुकीम्} %3-14-17

\twolineshloka
{ताम्रा तु सुषुवे कन्याः पञ्च एता लोकविश्रुताः}
{उलूकान् जनयत् क्रौन्ची भासी भासान् व्यजायत} %3-14-18

\twolineshloka
{श्येनी श्येनाम् च गृध्राम च व्यजायत सुतेजसः}
{धृतराष्ट्री तु हंसाम् च कलहंसाम् च सर्वशः} %3-14-19

\twolineshloka
{चक्रवाकाम् च भद्रम् ते विजज्ञे सा अपि भामिनी}
{शुकी नताम् विजज्ञे तु नताया विनता सुता} %3-14-20

\twolineshloka
{दश क्रोधवशा राम विजज्ञे अपि आत्मसम्भवाः}
{मृगीम् च मृगमन्दाम् च हरीम् भद्रमदाम् अपि} %3-14-21

\twolineshloka
{मातङ्गीम् अथ शार्दूलीम् श्वेताम् च सुरभीम् तथा}
{सर्व लक्षण सम्पन्नाम् सुरसाम् कद्रुकाम् अपि} %3-14-22

\twolineshloka
{अपत्यम् तु मृगाः सर्वे मृग्या नरवरोत्तम}
{ऋक्षाः च मृगमन्दायाः सृमराः चमराः तथा} %3-14-23

\twolineshloka
{ततः तु इरावतीम् नाम जज्ञे भद्रमदा सुताम्}
{तस्याः तु ऐरावतः पुत्रो लोकनाथो महागजः} %3-14-24

\twolineshloka
{हर्याः च हरयो अपत्यम् वानराः च तपस्विनः}
{गोलाङ्गूलाः च शार्दूली व्याघ्राम् च अजनयत् सुतान्} %3-14-25

\twolineshloka
{मातङ्ग्याः तु अथ मातन्गाअपत्यम् मनुज ऋषभ}
{दिशागजम् तु श्वेत काकुत्स्थ श्वेता व्यजनयत् सुतम्} %3-14-26

\twolineshloka
{ततो दुहितरौ राम सुरभिर् द्वे वि अजायत}
{रोहिणीम् नाम भद्रम् ते गन्धर्वीम् च यशस्विनीम्} %3-14-27

\twolineshloka
{रोहिणि अजनयद् गावो गन्धर्वी वाजिनः सुतान्}
{सुरसा अजनयन् नागान् राम कद्रूः च पन्नगान्} %3-14-28

\twolineshloka
{मनुर् मनुष्यान् जनयत् कश्यपस्य महात्मनः}
{ब्राह्मणान् क्षत्रियान् वैश्यान् शूद्राम् च मनुजर्षभ} %3-14-29

\twolineshloka
{मुखतो ब्राह्मणा जाता उरसः क्षत्रियाः तथा}
{ऊरुभ्याम् जज्ञिरे वैश्याः पद्भ्याम् शूद्रा इति श्रुतिः} %3-14-30

\twolineshloka
{सर्वान् पुण्य फलान् वृक्षान् अनला अपि व्यजायत}
{विनता च शुकी पौत्री कद्रूः च सुरसा स्वसा} %3-14-31

\twolineshloka
{कद्रूर् नाग सहस्रम् तु विजज्ञे धरणीधरन्}
{द्वौ पुत्रौ विनतायाः तु गरुडो अरुण एव च} %3-14-32

\twolineshloka
{तस्मात् जातो अहम् अरुणात् सम्पातिः च मम अग्रजः}
{जटायुर् इति माम् विद्धि श्येनी पुत्रम् अरिन्दम} %3-14-33

\twolineshloka
{सो अहम् वास सहायः ते भविष्यामि यदि इच्छसि}
{इदम् दुर्गम् हि कान्तारम् मृग राक्षस सेवितम् सीताम् च तात रक्षिष्ये त्वयि याते सलक्ष्मणे} %3-14-34

\twolineshloka
{जटायुषम् तु प्रतिपूज्य राघवो मुदा परिष्वज्य च सन्नतो अभवत्}
{पितुर् हि शुश्राव सखित्वम् आत्मवान् जटायुषा सङ्कथितम् पुनः पुनः} %3-14-35

\twolineshloka
{स तत्र सीताम् परिदाय मैथिलीम् सह एव तेन अतिबलेन पक्षिणा}
{जगाम ताम् पञ्चवटीम् सलक्ष्मणो रिपून् दिधक्षन् शलभान् इव अनलः} %3-14-36


॥इत्यार्षे श्रीमद्रामायणे वाल्मीकीये आदिकाव्ये अरण्यकाण्डे जटायुस्सङ्गमः नाम चतुर्दशः सर्गः ॥३-१४॥
