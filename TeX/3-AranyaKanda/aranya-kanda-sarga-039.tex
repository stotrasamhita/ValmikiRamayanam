\sect{एकोनचत्वारिंशः सर्गः — साहाय्यकानभ्युपगमः}

\twolineshloka
{एवमस्मि तदा मुक्तः कथञ्चित् तेन संयुगे}
{इदानीमपि यद् वृत्तं तच्छृणुष्व यदुत्तरम्} %3-39-1

\twolineshloka
{राक्षसाभ्यामहं द्वाभ्यामनिर्विण्णस्तथाकृतः}
{सहितो मृगरूपाभ्यां प्रविष्टो दण्डकावने} %3-39-2

\twolineshloka
{दीप्तजिह्वो महादंष्ट्रस्तीक्ष्णशृङ्गो महाबलः}
{व्यचरन् दण्डकारण्यं मांसभक्षो महामृगः} %3-39-3

\twolineshloka
{अग्निहोत्रेषु तीर्थेषु चैत्यवृक्षेषु रावण}
{अत्यन्तघोरो व्यचरंस्तापसांस्तान् प्रधर्षयन्} %3-39-4

\twolineshloka
{निहत्य दण्डकारण्ये तापसान् धर्मचारिणः}
{रुधिराणि पिबंस्तेषां तन्मांसानि च भक्षयन्} %3-39-5

\twolineshloka
{ऋषिमांसाशनः क्रूरस्त्रासयन् वनगोचरान्}
{तदा रुधिरमत्तोऽहं व्यचरं दण्डकावनम्} %3-39-6

\twolineshloka
{तदाहं दण्डकारण्ये विचरन् धर्मदूषकः}
{आसादयं तदा रामं तापसं धर्ममाश्रितम्} %3-39-7

\twolineshloka
{वैदेहीं च महाभागां लक्ष्मणं च महारथम्}
{तापसं नियताहारं सर्वभूतहिते रतम्} %3-39-8

\twolineshloka
{सोऽहं वनगतं रामं परिभूय महाबलम्}
{तापसोऽयमिति ज्ञात्वा पूर्ववैरमनुस्मरन्} %3-39-9

\twolineshloka
{अभ्यधावं सुसङ्क्रुद्धस्तीक्ष्णशृङ्गो मृगाकृतिः}
{जिघांसुरकृतप्रज्ञस्तं प्रहारमनुस्मरन्} %3-39-10

\twolineshloka
{तेन त्यक्तास्त्रयो बाणाः शिताः शत्रुनिबर्हणाः}
{विकृष्य सुमहच्चापं सुपर्णानिलतुल्यगाः} %3-39-11

\twolineshloka
{ते बाणा वज्रसङ्काशाः सुघोरा रक्तभोजनाः}
{आजग्मुः सहिताः सर्वे त्रयः सन्नतपर्वणः} %3-39-12

\twolineshloka
{पराक्रमज्ञो रामस्य शठो दृष्टभयः पुरा}
{समुत्क्रान्तस्ततो मुक्तस्तावुभौ राक्षसौ हतौ} %3-39-13

\twolineshloka
{शरेण मुक्तो रामस्य कथञ्चित् प्राप्य जीवितम्}
{इह प्रव्राजितो युक्तस्तापसोऽहं समाहितः} %3-39-14

\twolineshloka
{वृक्षे वृक्षे हि पश्यामि चीरकृष्णाजिनाम्बरम्}
{गृहीतधनुषं रामं पाशहस्तमिवान्तकम्} %3-39-15

\twolineshloka
{अपि रामसहस्राणि भीतः पश्यामि रावण}
{रामभूतमिदं सर्वमरण्यं प्रतिभाति मे} %3-39-16

\twolineshloka
{राममेव हि पश्यामि रहिते राक्षसेश्वर}
{दृष्ट्वा स्वप्नगतं राममुद्भ्रमामि विचेतनः} %3-39-17

\twolineshloka
{रकारादीनि नामानि रामत्रस्तस्य रावण}
{रत्नानि च रथाश्चैव वित्रासं जनयन्ति मे} %3-39-18

\twolineshloka
{अहं तस्य प्रभावज्ञो न युद्धं तेन ते क्षमम्}
{बलिं वा नमुचिं वापि हन्याद्धि रघुनन्दनः} %3-39-19

\twolineshloka
{रणे रामेण युद्धस्व क्षमां वा कुरु रावण}
{न ते रामकथा कार्या यदि मां द्रष्टुमिच्छसि} %3-39-20

\twolineshloka
{बहवः साधवो लोके युक्ता धर्ममनुष्ठिताः}
{परेषामपराधेन विनष्टाः सपरिच्छदाः} %3-39-21

\twolineshloka
{सोऽहं परापराधेन विनशेयं निशाचर}
{कुरु यत् ते क्षमं तत्त्वमहं त्वां नानुयामि वै} %3-39-22

\twolineshloka
{रामश्च हि महातेजा महासत्त्वो महाबलः}
{अपि राक्षसलोकस्य भवेदन्तकरोऽपि हि} %3-39-23

\threelineshloka
{यदि शूर्पणखाहेतोर्जनस्थानगतः खरः}
{अतिवृत्तो हतः पूर्वं रामेणाक्लिष्टकर्मणा}
{अत्र ब्रूहि यथातत्त्वं को रामस्य व्यतिक्रमः} %3-39-24

\twolineshloka
{इदं वचो बन्धुहितार्थिना मया यथोच्यमानं यदि नाभिपत्स्यसे}
{सबान्धवस्त्यक्ष्यसि जीवितं रणे हतोऽद्य रामेण शरैरजिह्मगैः} %3-39-25


॥इत्यार्षे श्रीमद्रामायणे वाल्मीकीये आदिकाव्ये अरण्यकाण्डे साहाय्यकानभ्युपगमः नाम एकोनचत्वारिंशः सर्गः ॥३-३९॥
