\sect{पञ्चपञ्चाशः सर्गः — सीताविलोभनोद्यमः}

\twolineshloka
{संदिश्य राक्षसान् घोरान् रावणोऽष्टौ महाबलान्}
{आत्मानं बुद्धिवैक्लव्यात् कृत्कृत्यममन्यत} %3-55-1

\twolineshloka
{स चिन्तयानो वैदेहीं कामबाणैः प्रपीडितः}
{प्रविवेश गृहं रम्यं सीतां द्रष्टुमभित्वरन्} %3-55-2

\twolineshloka
{स प्रविश्य तु तद्वेश्म रावणो राक्षसाधिपः}
{अपश्यद् राक्षसीमध्ये सीतां दुःखपरायणाम्} %3-55-3

\twolineshloka
{अश्रुपूर्णमुखीं दीनां शोकभारावपीडिताम्}
{वायुवेगैरिवाक्रान्तां मज्जन्तीं नावमर्णवे} %3-55-4

\twolineshloka
{मृगयूथपरिभ्रष्टां मृगीं श्वभिरिवावृताम्}
{अधोगतमुखीं सीतां तामभ्येत्य निशाचरः} %3-55-5

\twolineshloka
{तां तु शोकवशाद् दीनामवशां राक्षसाधिपः}
{सबलाद् दर्शयामास गृहं देवगृहोपमम्} %3-55-6

\twolineshloka
{हर्म्यप्रासादसम्बाधं स्त्रीसहस्रनिषेवितम्}
{नानापक्षिगणैर्जुष्टं नानारत्नसमन्वितम्} %3-55-7

\twolineshloka
{दान्तकैस्तापनीयैश्च स्फाटिकै राजतैस्तथा}
{वज्रवैदूर्यचित्रैश्च स्तम्भैर्दृष्टिमनोरमैः} %3-55-8

\twolineshloka
{दिव्यदुन्दुभिनिर्घोषं तप्तकाञ्चनभूषणम्}
{सोपानं काञ्चनं चित्रमारुरोह तया सह} %3-55-9

\twolineshloka
{दान्तका राजताश्चैव गवाक्षाः प्रियदर्शनाः}
{हेमजालावृताश्चासंस्तत्र प्रासादपङ्क्तयः} %3-55-10

\twolineshloka
{सुधामणिविचित्राणि भूमिभागानि सर्वशः}
{दशग्रीवः स्वभवने प्रादर्शयत मैथिलीम्} %3-55-11

\twolineshloka
{दीर्घिकाः पुष्करिण्यश्च नानापुष्पसमावृताः}
{रावणो दर्शयामास सीतां शोकपरायणाम्} %3-55-12

\twolineshloka
{दर्शयित्वा तु वैदेहीं कृत्स्नं तद्भवनोत्तमम्}
{उवाच वाक्यं पापात्मा सीतां लोभितुमिच्छया} %3-55-13

\twolineshloka
{दश राक्षसकोट्यश्च द्वाविंशतिरथापराः}
{वर्जयित्वा जरावृद्धान् बालांश्च रजनीचरान्} %3-55-14

\twolineshloka
{तेषां प्रभुरहं सीते सर्वेषां भीमकर्मणाम्}
{सहस्रमेकमेकस्य मम कार्यपुरःसरम्} %3-55-15

\twolineshloka
{यदिदं राज्यतन्त्रं मे त्वयि सर्वं प्रतिष्ठितम्}
{जीवितं च विशालाक्षि त्वं मे प्राणैर्गरीयसी} %3-55-16

\twolineshloka
{बह्वीनामुत्तमस्त्रीणां मम योऽसौ परिग्रहः}
{तासां त्वमीश्वरी सीते मम भार्या भव प्रिये} %3-55-17

\twolineshloka
{साधु किं तेऽन्यथाबुद्ध्या रोचयस्व वचो मम}
{भजस्व माभितप्तस्य प्रसादं कर्तुमर्हसि} %3-55-18

\twolineshloka
{परिक्षिप्ता समुद्रेण लङ्केयं शतयोजना}
{नेयं धर्षयितुं शक्या सेन्द्रैरपि सुरासुरैः} %3-55-19

\twolineshloka
{न देवेषु न यक्षेषु न गन्धर्वेषु नर्षिषु}
{अहं पश्यामि लोकेषु यो मे वीर्यसमो भवेत्} %3-55-20

\twolineshloka
{राज्यभ्रष्टेन दीनेन तापसेन पदातिना}
{किं करिष्यसि रामेण मानुषेणाल्पतेजसा} %3-55-21

\twolineshloka
{भजस्व सीते मामेव भर्ताहं सदृशस्तव}
{यौवनं त्वध्रुवं भीरु रमस्वेह मया सह} %3-55-22

\twolineshloka
{दर्शने मा कृथा बुद्धिं राघवस्य वरानने}
{कास्य शक्तिरिहागन्तुमपि सीते मनोरथैः} %3-55-23

\twolineshloka
{न शक्यो वायुराकाशे पाशैर्बद्धुं महाजवः}
{दीप्यमानस्य वाप्यग्नेर्ग्रहीतुं विमलाः शिखाः} %3-55-24

\twolineshloka
{त्रयाणामपि लोकानां न तं पश्यामि शोभने}
{विक्रमेण नयेद् यस्त्वां मद्बाहुपरिपालिताम्} %3-55-25

\twolineshloka
{लङ्कायाः सुमहद्राज्यमिदं त्वमनुपालय}
{त्वत्प्रेष्या मद्विधाश्चैव देवाश्चापि चराचरम्} %3-55-26

\twolineshloka
{अभिषेकजलक्लिन्ना तुष्टा च रमयस्व च}
{दुष्कृतं यत्पुरा कर्म वनवासेन तद्गतम्} %3-55-27

\twolineshloka
{यच्च ते सुकृतं कर्म तस्येह फलमाप्नुहि}
{इह सर्वाणि माल्यानि दिव्यगन्धानि मैथिलि} %3-55-28

\twolineshloka
{भूषणानि च मुख्यानि तानि सेव मया सह}
{पुष्पकं नाम सुश्रोणि भ्रातुर्वैश्रवणस्य मे} %3-55-29

\twolineshloka
{विमानं सूर्यसंकाशं तरसा निर्जितं रणे}
{विशालं रमणीयं च तद्विमानं मनोजवम्} %3-55-30

\twolineshloka
{तत्र सीते मया सार्धं विहरस्व यथासुखम्}
{वदनं पद्मसंकाशं विमलं चारुदर्शनम्} %3-55-31

\twolineshloka
{शोकार्तं तु वरारोहे न भ्राजति वरानने}
{एवं वदति तस्मिन् सा वस्त्रान्तेन वराङ्गना} %3-55-32

\twolineshloka
{पिधायेन्दुनिभं सीता मन्दमश्रूण्यवर्तयत्}
{ध्यायन्तीं तामिवास्वस्थां सीतां चिन्ताहतप्रभाम्} %3-55-33

\twolineshloka
{उवाच वचनं वीरो रावणो रजनीचरः}
{अलं व्रीडेन वैदेहि धर्मलोपकृतेन ते} %3-55-34

\twolineshloka
{आर्षोऽयं देवि निष्पन्दो यस्त्वामभिभविष्यति}
{एतौ पादौ मया स्निग्धौ शिरोभिः परिपीडितौ} %3-55-35

\twolineshloka
{प्रसादं कुरु मे क्षिप्रं वश्यो दासोऽहमस्मि ते}
{इमाः शून्या मया वाचः शुष्यमाणेन भाषिताः} %3-55-36

\threelineshloka
{न चापि रावणः कांचिन्मूर्ध्ना स्त्रीं प्रणमेत ह}
{एवमुक्त्वा दशग्रीवो मैथिलीं जनकात्मजाम्}
{कृतान्तवशमापन्नो ममेयमिति मन्यते} %3-55-37


॥इत्यार्षे श्रीमद्रामायणे वाल्मीकीये आदिकाव्ये अरण्यकाण्डे सीताविलोभनोद्यमः नाम पञ्चपञ्चाशः सर्गः ॥३-५५॥
