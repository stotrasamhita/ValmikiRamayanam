\sect{एकत्रिंशः सर्गः — रावणखरवृत्तोपलम्भः}

\twolineshloka
{त्वरमाणस्ततो गत्वा जनस्थानादकम्पनः}
{प्रविश्य लङ्कां वेगेन रावणं वाक्यमब्रवीत्} %3-31-1

\twolineshloka
{जनस्थानस्थिता राजन् राक्षसा बहवो हताः}
{खरश्च निहतः सङ्ख्ये कथञ्चिदहमागतः} %3-31-2

\twolineshloka
{एवमुक्तो दशग्रीवः क्रुद्धः संरक्तलोचनः}
{अकम्पनमुवाचेदं निर्दहन्निव तेजसा} %3-31-3

\twolineshloka
{केन भीमं जनस्थानं हतं मम परासुना}
{को हि सर्वेषु लोकेषु गतिं नाधिगमिष्यति} %3-31-4

\twolineshloka
{न हि मे विप्रियं कृत्वा शक्यं मघवता सुखम्}
{प्राप्तुं वैश्रवणेनापि न यमेन च विष्णुना} %3-31-5

\twolineshloka
{कालस्य चाप्यहं कालो दहेयमपि पावकम्}
{मृत्युं मरणधर्मेण संयोजयितुमुत्सहे} %3-31-6

\twolineshloka
{वातस्य तरसा वेगं निहन्तुमपि चोत्सहे}
{दहेयमपि सङ्क्रुद्धस्तेजसाऽऽदित्यपावकौ} %3-31-7

\twolineshloka
{तथा क्रुद्धं दशग्रीवं कृताञ्जलिरकम्पनः}
{भयात् सन्दिग्धया वाचा रावणं याचतेऽभयम्} %3-31-8

\twolineshloka
{दशग्रीवोऽभयं तस्मै प्रददौ रक्षसां वरः}
{स विस्रब्धोऽब्रवीद् वाक्यमसन्दिग्धमकम्पनः} %3-31-9

\twolineshloka
{पुत्रो दशरथस्यास्ते सिंहसंहननो युवा}
{रामो नाम महास्कन्धो वृत्तायतमहाभुजः} %3-31-10

\twolineshloka
{श्यामः पृथुयशाः श्रीमानतुल्यबलविक्रमः}
{हतस्तेन जनस्थाने खरश्च सहदूषणः} %3-31-11

\twolineshloka
{अकम्पनवचः श्रुत्वा रावणो राक्षसाधिपः}
{नागेन्द्र इव निःश्वस्य इदं वचनमब्रवीत्} %3-31-12

\twolineshloka
{स सुरेन्द्रेण संयुक्तो रामः सर्वामरैः सह}
{उपयातो जनस्थानं ब्रूहि कच्चिदकम्पन} %3-31-13

\twolineshloka
{रावणस्य पुनर्वाक्यं निशम्य तदकम्पनः}
{आचचक्षे बलं तस्य विक्रमं च महात्मनः} %3-31-14

\twolineshloka
{रामो नाम महातेजाः श्रेष्ठः सर्वधनुष्मताम्}
{दिव्यास्त्रगुणसम्पन्नः परं धर्मं गतो युधि} %3-31-15

\twolineshloka
{तस्यानुरूपो बलवान् रक्ताक्षो दुन्दुभिस्वनः}
{कनीयाँल्लक्ष्मणो भ्राता राकाशशिनिभाननः} %3-31-16

\twolineshloka
{स तेन सह संयुक्तः पावकेनानिलो यथा}
{श्रीमान् राजवरस्तेन जनस्थानं निपातितम्} %3-31-17

\twolineshloka
{नैव देवा महात्मानो नात्र कार्या विचारणा}
{शरा रामेण तूत्सृष्टा रुक्मपुङ्खाः पतत्त्रिणः} %3-31-18

\twolineshloka
{सर्पाः पञ्चानना भूत्वा भक्षयन्ति स्म राक्षसान्}
{येन येन च गच्छन्ति राक्षसा भयकर्षिताः} %3-31-19

\twolineshloka
{तेन तेन स्म पश्यन्ति राममेवाग्रतः स्थितम्}
{इत्थं विनाशितं तेन जनस्थानं तवानघ} %3-31-20

\twolineshloka
{अकम्पनवचः श्रुत्वा रावणो वाक्यमब्रवीत्}
{गमिष्यामि जनस्थानं रामं हन्तुं सलक्ष्मणम्} %3-31-21

\twolineshloka
{अथैवमुक्ते वचने प्रोवाचेदमकम्पनः}
{शृणु राजन् यथावृत्तं रामस्य बलपौरुषम्} %3-31-22

\twolineshloka
{असाध्यः कुपितो रामो विक्रमेण महायशाः}
{आपगायास्तु पूर्णाया वेगं परिहरेच्छरैः} %3-31-23

\twolineshloka
{सताराग्रहनक्षत्रं नभश्चाप्यवसादयेत्}
{असौ रामस्तु सीदन्तीं श्रीमानभ्युद्धरेन्महीम्} %3-31-24

\twolineshloka
{भित्त्वा वेलां समुद्रस्य लोकानाप्लावयेद् विभुः}
{वेगं वापि समुद्रस्य वायुं वा विधमेच्छरैः} %3-31-25

\twolineshloka
{संहृत्य वा पुनर्लोकान् विक्रमेण महायशाः}
{शक्तः श्रेष्ठः स पुरुषः स्रष्टुं पुनरपि प्रजाः} %3-31-26

\twolineshloka
{नहि रामो दशग्रीव शक्यो जेतुं रणे त्वया}
{रक्षसां वापि लोकेन स्वर्गः पापजनैरिव} %3-31-27

\twolineshloka
{न तं वध्यमहं मन्ये सर्वैर्देवासुरैरपि}
{अयं तस्य वधोपायस्तन्ममैकमनाः शृणु} %3-31-28

\twolineshloka
{भार्या तस्योत्तमा लोके सीता नाम सुमध्यमा}
{श्यामा समविभक्ताङ्गी स्त्रीरत्नं रत्नभूषिता} %3-31-29

\twolineshloka
{नैव देवी न गन्धर्वी नाप्सरा न च पन्नगी}
{तुल्या सीमन्तिनी तस्या मानुषी तु कुतो भवेत्} %3-31-30

\twolineshloka
{तस्यापहर भार्यां त्वं तं प्रमथ्य महावने}
{सीतया रहितो रामो न चैव हि भविष्यति} %3-31-31

\twolineshloka
{अरोचयत तद्वाक्यं रावणो राक्षसाधिपः}
{चिन्तयित्वा महाबाहुरकम्पनमुवाच ह} %3-31-32

\twolineshloka
{बाढं कल्यं गमिष्यामि ह्येकः सारथिना सह}
{आनेष्यामि च वैदेहीमिमां हृष्टो महापुरीम्} %3-31-33

\twolineshloka
{तदेवमुक्त्वा प्रययौ खरयुक्तेन रावणः}
{रथेनादित्यवर्णेन दिशः सर्वाः प्रकाशयन्} %3-31-34

\twolineshloka
{स रथो राक्षसेन्द्रस्य नक्षत्रपथगो महान्}
{चञ्चूर्यमाणः शुशुभे जलदे चन्द्रमा इव} %3-31-35

\twolineshloka
{स दूरे चाश्रमं गत्वा ताटकेयमुपागमत्}
{मारीचेनार्चितो राजा भक्ष्यभोज्यैरमानुषैः} %3-31-36

\twolineshloka
{तं स्वयं पूजयित्वा तु आसनेनोदकेन च}
{अर्थोपहितया वाचा मारीचो वाक्यमब्रवीत्} %3-31-37

\twolineshloka
{कच्चित् सकुशलं राजँल्लोकानां राक्षसाधिप}
{आशङ्के नाधिजाने त्वं यतस्तूर्णमुपागतः} %3-31-38

\twolineshloka
{एवमुक्तो महातेजा मारीचेन स रावणः}
{ततः पश्चादिदं वाक्यमब्रवीद् वाक्यकोविदः} %3-31-39

\twolineshloka
{आरक्षो मे हतस्तात रामेणाक्लिष्टकारिणा}
{जनस्थानमवध्यं तत् सर्वं युधि निपातितम्} %3-31-40

\twolineshloka
{तस्य मे कुरु साचिव्यं तस्य भार्यापहारणे}
{राक्षसेन्द्रवचः श्रुत्वा मारीचो वाक्यमब्रवीत्} %3-31-41

\twolineshloka
{आख्याता केन वा सीता मित्ररूपेण शत्रुणा}
{त्वया राक्षसशार्दूल को न नन्दति नन्दितः} %3-31-42

\twolineshloka
{सीतामिहानयस्वेति को ब्रवीति ब्रवीहि मे}
{रक्षोलोकस्य सर्वस्य कः शृङ्गं छेत्तुमिच्छति} %3-31-43

\twolineshloka
{प्रोत्साहयति यश्च त्वां स च शत्रुरसंशयम्}
{आशीविषमुखाद् दंष्ट्रामुद्धर्तुं चेच्छति त्वया} %3-31-44

\twolineshloka
{कर्मणानेन केनासि कापथं प्रतिपादितः}
{सुखसुप्तस्य ते राजन् प्रहृतं केन मूर्धनि} %3-31-45

\twolineshloka
{विशुद्धवंशाभिजनाग्रहस्ततेजोमदः संस्थितदोर्विषाणः}
{उदीक्षितुं रावण नेह युक्तः स संयुगे राघवगन्धहस्ती} %3-31-46

\twolineshloka
{असौ रणान्तःस्थितिसन्धिवालो विदग्धरक्षोमृगहा नृसिंहः}
{सुप्तस्त्वया बोधयितुं न शक्यः शराङ्गपूर्णो निशितासिदंष्ट्रः} %3-31-47

\twolineshloka
{चापापहारे भुजवेगपङ्के शरोर्मिमाले सुमहाहवौघे}
{न रामपातालमुखेऽतिघोरे प्रस्कन्दितुं राक्षसराज युक्तम्} %3-31-48

\twolineshloka
{प्रसीद लङ्केश्वर राक्षसेन्द्र लङ्कां प्रसन्नो भव साधु गच्छ}
{त्वं स्वेषु दारेषु रमस्व नित्यं रामः सभार्यो रमतां वनेषु} %3-31-49

\twolineshloka
{एवमुक्तो दशग्रीवो मारीचेन स रावणः}
{न्यवर्तत पुरीं लङ्कां विवेश च गृहोत्तमम्} %3-31-50


॥इत्यार्षे श्रीमद्रामायणे वाल्मीकीये आदिकाव्ये अरण्यकाण्डे रावणखरवृत्तोपलम्भः नाम एकत्रिंशः सर्गः ॥३-३१॥
