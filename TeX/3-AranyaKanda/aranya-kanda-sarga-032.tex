\sect{द्वात्रिंशः सर्गः — शूर्पनखोद्यमः}

\twolineshloka
{ततः शूर्पणखा दृष्ट्वा सहस्राणि चतुर्दश}
{हतान्येकेन रामेण रक्षसां भीमकर्मणाम्} %3-32-1

\twolineshloka
{दूषणं च खरं चैव हतं त्रिशिरसं रणे}
{दृष्ट्वा पुनर्महानादान् ननाद जलदोपमा} %3-32-2

\twolineshloka
{सा दृष्ट्वा कर्म रामस्य कृतमन्यैः सुदुष्करम्}
{जगाम परमोद्विग्ना लङ्कां रावणपालिताम्} %3-32-3

\twolineshloka
{सा ददर्श विमानाग्रे रावणं दीप्ततेजसम्}
{उपोपविष्टं सचिवैर्मरुद्भिरिव वासवम्} %3-32-4

\twolineshloka
{आसीनं सूर्यसङ्काशे काञ्चने परमासने}
{रुक्मवेदिगतं प्राज्यं ज्वलन्तमिव पावकम्} %3-32-5

\twolineshloka
{देवगन्धर्वभूतानामृषीणां च महात्मनाम्}
{अजेयं समरे घोरं व्यात्ताननमिवान्तकम्} %3-32-6

\twolineshloka
{देवासुरविमर्देषु वज्राशनिकृतव्रणम्}
{ऐरावतविषाणाग्रैरुत्कृष्टकिणवक्षसम्} %3-32-7

\twolineshloka
{विंशद्भुजं दशग्रीवं दर्शनीयपरिच्छदम्}
{विशालवक्षसं वीरं राजलक्षणलक्षितम्} %3-32-8

\twolineshloka
{नद्धवैदूर्यसङ्काशं तप्तकाञ्चनभूषणम्}
{सुभुजं शुक्लदशनं महास्यं पर्वतोपमम्} %3-32-9

\twolineshloka
{विष्णुचक्रनिपातैश्च शतशो देवसंयुगे}
{अन्यैः शस्त्रैः प्रहारैश्च महायुद्धेषु ताडितम्} %3-32-10

\twolineshloka
{अहताङ्गैः समस्तैस्तं देवप्रहरणैस्तदा}
{अक्षोभ्याणां समुद्राणां क्षोभणं क्षिप्रकारिणम्} %3-32-11

\twolineshloka
{क्षेप्तारं पर्वताग्राणां सुराणां च प्रमर्दनम्}
{उच्छेत्तारं च धर्माणां परदाराभिमर्शनम्} %3-32-12

\twolineshloka
{सर्वदिव्यास्त्रयोक्तारं यज्ञविघ्नकरं सदा}
{पुरीं भोगवतीं गत्वा पराजित्य च वासुकिम्} %3-32-13

\twolineshloka
{तक्षकस्य प्रियां भार्यां पराजित्य जहार यः}
{कैलासं पर्वतं गत्वा विजित्य नरवाहनम्} %3-32-14

\twolineshloka
{विमानं पुष्पकं तस्य कामगं वै जहार यः}
{वनं चैत्ररथं दिव्यं नलिनीं नन्दनं वनम्} %3-32-15

\twolineshloka
{विनाशयति यः क्रोधाद् देवोद्यानानि वीर्यवान्}
{चन्द्रसूर्यौ महाभागावुत्तिष्ठन्तौ परन्तपौ} %3-32-16

\twolineshloka
{निवारयति बाहुभ्यां यः शैलशिखरोपमः}
{दशवर्षसहस्राणि तपस्तप्त्वा महावने} %3-32-17

\twolineshloka
{पुरा स्वयम्भुवे धीरः शिरांस्युपजहार यः}
{देवदानवगन्धर्वपिशाचपतगोरगैः} %3-32-18

\twolineshloka
{अभयं यस्य सङ्ग्रामे मृत्युतो मानुषादृते}
{मन्त्रैरभिष्टुतं पुण्यमध्वरेषु द्विजातिभिः} %3-32-19

\twolineshloka
{हविर्धानेषु यः सोममुपहन्ति महाबलः}
{प्राप्तयज्ञहरं दुष्टं ब्रह्मघ्नं क्रूरकारिणम्} %3-32-20

\twolineshloka
{कर्कशं निरनुक्रोशं प्रजानामहिते रतम्}
{रावणं सर्वभूतानां सर्वलोकभयावहम्} %3-32-21

\twolineshloka
{राक्षसी भ्रातरं क्रूरं सा ददर्श महाबलम्}
{तं दिव्यवस्त्राभरणं दिव्यमाल्योपशोभितम्} %3-32-22

\twolineshloka
{आसने सूपविष्टं तं काले कालमिवोद्यतम्}
{राक्षसेन्द्रं महाभागं पौलस्त्यकुलनन्दनम्} %3-32-23

\twolineshloka
{उपगम्याब्रवीद् वाक्यं राक्षसी भयविह्वला}
{रावणं शत्रुहन्तारं मन्त्रिभिः परिवारितम्} %3-32-24

\twolineshloka
{तमब्रवीद् दीप्तविशाललोचनं प्रदर्शयित्वा भयलोभमोहिता}
{सुदारुणं वाक्यमभीतचारिणी महात्मना शूर्पणखा विरूपिता} %3-32-25


॥इत्यार्षे श्रीमद्रामायणे वाल्मीकीये आदिकाव्ये अरण्यकाण्डे शूर्पनखोद्यमः नाम द्वात्रिंशः सर्गः ॥३-३२॥
