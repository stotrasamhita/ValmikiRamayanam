\sect{अष्टपञ्चाशः सर्गः — अनिमित्तदर्शनम्}

\twolineshloka
{स दृष्ट्वा लक्ष्मणं दीनं शून्यं दशरथात्मजः}
{पर्यपृच्छत धर्मात्मा वैदेहीमागतं विना} %3-58-1

\twolineshloka
{प्रस्थितं दण्डकारण्यं या मामनुजगाम ह}
{क्व सा लक्ष्मण वैदेही यां हित्वा त्वमिहागतः} %3-58-2

\twolineshloka
{राज्यभ्रष्टस्य दीनस्य दण्डकान् परिधावतः}
{क्व सा दुःखसहाया मे वैदेही तनुमध्यमा} %3-58-3

\twolineshloka
{यां विना नोत्सहे वीर मुहूर्तमपि जीवितुम्}
{क्व सा प्राणसहाया मे सीता सुरसुतोपमा} %3-58-4

\twolineshloka
{पतित्वममराणां हि पृथिव्याश्चापि लक्ष्मण}
{विना तां तपनीयाभां नेच्छेयं जनकात्मजाम्} %3-58-5

\twolineshloka
{कच्चिज्जीवति वैदेही प्राणैः प्रियतरा मम}
{कच्चित् प्रव्राजनं वीर न मे मिथ्या भविष्यति} %3-58-6

\twolineshloka
{सीतानिमित्तं सौमित्रे मृते मयि गते त्वयि}
{कच्चित् सकामा कैकेयी सुखिता सा भविष्यति} %3-58-7

\twolineshloka
{सपुत्रराज्यां सिद्धार्थां मृतपुत्रा तपस्विनी}
{उपस्थास्यति कौसल्या कच्चित् सौम्येन कैकयीम्} %3-58-8

\twolineshloka
{यदि जीवति वैदेही गमिष्याम्याश्रमं पुनः}
{संवृत्ता यदि वृत्ता सा प्राणांस्त्यक्ष्यामि लक्ष्मण} %3-58-9

\twolineshloka
{यदि मामाश्रमगतं वैदेही नाभिभाषते}
{पुरः प्रहसिता सीता विनशिष्यामि लक्ष्मण} %3-58-10

\twolineshloka
{ब्रूहि लक्ष्मण वैदेही यदि जीवति वा न वा}
{त्वयि प्रमत्ते रक्षोभिर्भक्षिता वा तपस्विनी} %3-58-11

\twolineshloka
{सुकुमारी च बाला च नित्यं चादुःखभागिनी}
{मद्वियोगेन वैदेही व्यक्तं शोचति दुर्मनाः} %3-58-12

\twolineshloka
{सर्वथा रक्षसा तेन जिह्मेन सुदुरात्मना}
{वदता लक्ष्मणेत्युच्चैस्तवापि जनितं भयम्} %3-58-13

\twolineshloka
{श्रुतश्च मन्ये वैदेह्या स स्वरः सदृशो मम}
{त्रस्तया प्रेषितस्त्वं च द्रष्टुं मां शीघ्रमागतः} %3-58-14

\twolineshloka
{सर्वथा तु कृतं कष्टं सीतामुत्सृजता वने}
{प्रतिकर्तुं नृशंसानां रक्षसां दत्तमन्तरम्} %3-58-15

\twolineshloka
{दुःखिताः खरघातेन राक्षसाः पिशिताशनाः}
{तैः सीता निहता घोरैर्भविष्यति न संशयः} %3-58-16

\twolineshloka
{अहोऽस्मि व्यसने मग्नः सर्वथा रिपुनाशन}
{किं त्विदानीं करिष्यामि शङ्के प्राप्तव्यमीदृशम्} %3-58-17

\twolineshloka
{इति सीतां वरारोहां चिन्तयन्नेव राघवः}
{आजगाम जनस्थानं त्वरया सहलक्ष्मणः} %3-58-18

\twolineshloka
{विगर्हमाणोऽनुजमार्तरूपं क्षुधाश्रमेणैव पिपासया च}
{विनिःश्वसन् शुष्कमुखो विषण्णः प्रतिश्रयं प्राप्य समीक्ष्य शून्यम्} %3-58-19

\twolineshloka
{स्वमाश्रमं स प्रविगाह्य वीरो विहारदेशाननुसृत्य कांश्चित्}
{एतत्तदित्येव निवासभूमौ प्रहृष्टरोमा व्यथितो बभूव} %3-58-20


॥इत्यार्षे श्रीमद्रामायणे वाल्मीकीये आदिकाव्ये अरण्यकाण्डे अनिमित्तदर्शनम् नाम अष्टपञ्चाशः सर्गः ॥३-५८॥
