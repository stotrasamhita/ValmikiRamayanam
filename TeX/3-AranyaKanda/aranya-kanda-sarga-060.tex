\sect{षष्ठितमः सर्गः — रामोन्मादः}

\twolineshloka
{भृशमाव्रजमानस्य तस्याधो वामलोचनम्}
{प्रास्फुरच्चास्खलद् रामो वेपथुश्चास्य जायते} %3-60-1

\twolineshloka
{उपालक्ष्य निमित्तानि सोऽशुभानि मुहुर्मुहुः}
{अपि क्षेमं तु सीताया इति वै व्याजहार ह} %3-60-2

\twolineshloka
{त्वरमाणो जगामाथ सीतादर्शनलालसः}
{शून्यमावसथं दृष्ट्वा बभूवोद्विग्नमानसः} %3-60-3

\twolineshloka
{उद्भ्रमन्निव वेगेन विक्षिपन् रघुनन्दनः}
{तत्र तत्रोटजस्थानमभिवीक्ष्य समन्ततः} %3-60-4

\twolineshloka
{ददर्श पर्णशालां च सीतया रहितां तदा}
{श्रिया विरहितां ध्वस्तां हेमन्ते पद्मिनीमिव} %3-60-5

\twolineshloka
{रुदन्तमिव वृक्षैश्च ग्लानपुष्पमृगद्विजम्}
{श्रिया विहीनं विध्वस्तं संत्यक्तं वनदैवतैः} %3-60-6

\twolineshloka
{विप्रकीर्णाजिनकुशं विप्रविद्धबृसीकटम्}
{दृष्ट्वा शून्योटजस्थानं विललाप पुनः पुनः} %3-60-7

\twolineshloka
{हृता मृता वा नष्टा वा भक्षिता वा भविष्यति}
{निलीनाप्यथवा भीरुरथवा वनमाश्रिता} %3-60-8

\twolineshloka
{गता विचेतुं पुष्पाणि फलान्यपि च वा पुनः}
{अथवा पद्मिनीं याता जलार्थं वा नदीं गता} %3-60-9

\twolineshloka
{यत्नान्मृगयमाणस्तु नाससाद वने प्रियाम्}
{शोकरक्तेक्षणः श्रीमानुन्मत्त इव लक्ष्यते} %3-60-10

\twolineshloka
{वृक्षाद् वृक्षं प्रधावन् स गिरींश्चापि नदीनदम्}
{बभ्राम विलपन् रामः शोकपङ्कार्णवप्लुतः} %3-60-11

\twolineshloka
{अस्ति कच्चित्त्वया दृष्टा सा कदम्बप्रिया प्रिया}
{कदम्ब यदि जानीषे शंस सीतां शुभाननाम्} %3-60-12

\twolineshloka
{स्निग्धपल्लवसंकाशां पीतकौशेयवासिनीम्}
{शंसस्व यदि सा दृष्टा बिल्व बिल्वोपमस्तनी} %3-60-13

\twolineshloka
{अथवार्जुन शंस त्वं प्रियां तामर्जुनप्रियाम्}
{जनकस्य सुता तन्वी यदि जीवति वा न वा} %3-60-14

\twolineshloka
{ककुभः ककुभोरुं तां व्यक्तं जानाति मैथिलीम्}
{लतापल्लवपुष्पाढ्यो भाति ह्येष वनस्पतिः} %3-60-15

\twolineshloka
{भ्रमरैरुपगीतश्च यथा द्रुमवरो ह्यसि}
{एष व्यक्तं विजानाति तिलकस्तिलकप्रियाम्} %3-60-16

\twolineshloka
{अशोक शोकापनुद शोकोपहतचेतनम्}
{त्वन्नामानं कुरु क्षिप्रं प्रियासंदर्शनेन माम्} %3-60-17

\twolineshloka
{यदि ताल त्वया दृष्टा पक्वतालोपमस्तनी}
{कथयस्व वरारोहां कारुण्यं यदि ते मयि} %3-60-18

\twolineshloka
{यदि दृष्टा त्वया जम्बो जाम्बूनदसमप्रभा}
{प्रियां यदि विजानासि निःशङ्क कथयस्व मे} %3-60-19

\twolineshloka
{अहो त्वं कर्णिकाराद्य पुष्पितः शोभसे भृशम्}
{कर्णिकारप्रियां साध्वीं शंस दृष्टा यदि प्रिया} %3-60-20

\twolineshloka
{चूतनीपमहासालान् पनसान् कुरवान् धवान्}
{दाडिमानपि तान् गत्वा दृष्ट्वा रामो महायशाः} %3-60-21

\twolineshloka
{बकुलानथ पुन्नागांश्चन्दनान् केतकांस्तथा}
{पृच्छन् रामो वने भ्रान्त उन्मत्त इव लक्ष्यते} %3-60-22

\twolineshloka
{अथवा मृगशावाक्षीं मृग जानासि मैथिलीम्}
{मृगविप्रेक्षणी कान्ता मृगीभिः सहिता भवेत्} %3-60-23

\twolineshloka
{गज सा गजनासोरुर्यदि दृष्टा त्वया भवेत्}
{तां मन्ये विदितां तुभ्यमाख्याहि वरवारण} %3-60-24

\twolineshloka
{शार्दूल यदि सा दृष्टा प्रिया चन्द्रनिभानना}
{मैथिली मम विस्रब्धः कथयस्व न ते भयम्} %3-60-25

\twolineshloka
{किं धावसि प्रिये नूनं दृष्टासि कमलेक्षणे}
{वृक्षैराच्छाद्य चात्मानं किं मां न प्रतिभाषसे} %3-60-26

\twolineshloka
{तिष्ठ तिष्ठ वरारोहे न तेऽस्ति करुणा मयि}
{नात्यर्थं हास्यशीलासि किमर्थं मामुपेक्षसे} %3-60-27

\twolineshloka
{पीतकौशेयकेनासि सूचिता वरवर्णिनि}
{धावन्त्यपि मया दृष्टा तिष्ठ यद्यस्ति सौहृदम्} %3-60-28

\twolineshloka
{नैव सा नूनमथवा हिंसिता चारुहासिनी}
{कृच्छ्रं प्राप्तं न मां नूनं यथोपेक्षितुमर्हति} %3-60-29

\twolineshloka
{व्यक्तं सा भक्षिता बाला राक्षसैः पिशिताशनैः}
{विभज्याङ्गानि सर्वाणि मया विरहिता प्रिया} %3-60-30

\twolineshloka
{नूनं तच्छुभदन्तोष्ठं सुनासं शुभकुण्डलम्}
{पूर्णचन्द्रनिभं ग्रस्तं मुखं निष्प्रभतां गतम्} %3-60-31

\twolineshloka
{सा हि चम्पकवर्णाभा ग्रीवा ग्रैवेयकोचिता}
{कोमला विलपन्त्यास्तु कान्ताया भक्षिता शुभा} %3-60-32

\twolineshloka
{नूनं विक्षिप्यमाणौ तौ बाहू पल्लवकोमलौ}
{भक्षितौ वेपमानाग्रौ सहस्ताभरणाङ्गदौ} %3-60-33

\twolineshloka
{मया विरहिता बाला रक्षसां भक्षणाय वै}
{सार्थेनेव परित्यक्ता भक्षिता बहुबान्धवा} %3-60-34

\twolineshloka
{हा लक्ष्मण महाबाहो पश्यसे त्वं प्रियां क्वचित्}
{हा प्रिये क्व गता भद्रे हा सीतेति पुनः पुनः} %3-60-35

\twolineshloka
{इत्येवं विलपन् रामः परिधावन् वनाद् वनम्}
{क्वचिदुद्भ्रमते वेगात् क्वचिद् विभ्रमते बलात्} %3-60-36

\threelineshloka
{क्वचिन्मत्त इवाभाति कान्तान्वेषणतत्परः}
{स वनानि नदीः शैलान् गिरिप्रस्रवणानि च}
{काननानि च वेगेन भ्रमत्यपरिसंस्थितः} %3-60-37

\twolineshloka
{तदा स गत्वा विपुलं महद् वनं परीत्य सर्वं त्वथ मैथिलीं प्रति}
{अनिष्ठिताशः स चकार मार्गणे पुनः प्रियायाः परमं परिश्रमम्} %3-60-38


॥इत्यार्षे श्रीमद्रामायणे वाल्मीकीये आदिकाव्ये अरण्यकाण्डे रामोन्मादः नाम षष्ठितमः सर्गः ॥३-६०॥
