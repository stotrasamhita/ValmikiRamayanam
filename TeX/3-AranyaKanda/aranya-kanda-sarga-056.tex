\sect{षट्पञ्चाशः सर्गः — वत्सरावधिकरणम्}

\twolineshloka
{प्रवेशितायां सीतायां लङ्कां प्रति पितामहः}
{तदा प्रोवाच देवेन्द्रं परितुष्टं शतक्रतुम्} %3-56-1

\twolineshloka
{त्रैलोक्यस्य हितार्थाय रक्षसामहिताय च}
{लङ्कां प्रवेशिता सीता रावणेन दुरात्मना} %3-56-2

\twolineshloka
{पतिव्रता महाभागा नित्यं चैव सुखैधिता}
{अपश्यन्ती च भर्तारं पश्यन्ती राक्षसीजनम्} %3-56-3

\twolineshloka
{राक्षसीभिः परिवृता भर्तृदर्शनलालसा}
{निविष्टा हि पुरी लङ्का तीरे नदनदीपतेः} %3-56-4

\twolineshloka
{कथं ज्ञास्यति तां रामस्तत्रस्थां तामनिन्दिताम्}
{दुःखं सञ्चिन्तयन्ती सा बहुशः परिदुर्लभा} %3-56-5

\twolineshloka
{प्राणयात्रामकुर्वाणा प्राणांस्त्यक्ष्यत्यसंशयम्}
{स भूयः संशयो जातः सीतायाः प्राणसङ्क्षये} %3-56-6

\twolineshloka
{स त्वं शीघ्रमितो गत्वा सीतां पश्य शुभाननाम्}
{प्रविश्य नगरीं लङ्कां प्रयच्छ हविरुत्तमम्} %3-56-7

\twolineshloka
{एवमुक्तोऽथ देवेन्द्रः पुरीं रावणपालिताम्}
{आगच्छन्निद्रया सार्धं भगवान् पाकशासनः} %3-56-8

\twolineshloka
{निद्रां चोवाच गच्छ त्वं राक्षसान् सम्प्रमोहय}
{सा तथोक्ता मघवता देवी परमहर्षिता} %3-56-9

\twolineshloka
{देवकार्यार्थसिद्ध्यर्थं प्रामोहयत राक्षसान्}
{एतस्मिन्नन्तरे देवः सहस्राक्षः शचीपतिः} %3-56-10

\twolineshloka
{आससाद वनस्थां तां वचनं चेदमब्रवीत्}
{देवराजोऽस्मि भद्रं ते इह चास्मि शुचिस्मिते} %3-56-11

\twolineshloka
{अहं त्वां कार्यसिद्ध्यर्थं राघवस्य महात्मनः}
{साहाय्यं कल्पयिष्यामि मा शुचो जनकात्मजे} %3-56-12

\twolineshloka
{मत्प्रसादात् समुद्रं स तरिष्यति बलैः सह}
{मयैवेह च राक्षस्यो मायया मोहिताः शुभे} %3-56-13

\twolineshloka
{तस्मादन्नमिदं सीते हविष्यान्नमहं स्वयम्}
{स त्वां सङ्गृह्य वैदेहि आगतः सह निद्रया} %3-56-14

\twolineshloka
{एतदत्स्यसि मद्धस्तान्न त्वां बाधिष्यते शुभे}
{क्षुधा तृषा च रम्भोरु वर्षाणामयुतैरपि} %3-56-15

\twolineshloka
{एवमुक्ता तु देवेन्द्रमुवाच परिशङ्किता}
{कथं जानामि देवेन्द्रं त्वामिहस्थं शचीपतिम्} %3-56-16

\twolineshloka
{देवलिङ्गानि दृष्टानि रामलक्ष्मणसन्निधौ}
{तानि दर्शय देवेन्द्र यदि त्वं देवराट् स्वयम्} %3-56-17

\twolineshloka
{सीताया वचनं श्रुत्वा तथा चक्रे शचीपतिः}
{पृथिवीं नास्पृशत् पद्भ्यामनिमेषेक्षणानि च} %3-56-18

\twolineshloka
{अरजोऽम्बरधारी च न म्लानकुसुमस्तथा}
{तं ज्ञात्वा लक्षणैः सीता वासवं परिहर्षिता} %3-56-19

\twolineshloka
{उवाच वाक्यं रुदती भगवन् राघवं प्रति}
{सह भ्रात्रा महाबाहुर्दिष्ट्या मे श्रुतिमागतः} %3-56-20

\twolineshloka
{यथा मे श्वशुरो राजा यथा च मिथिलाधिपः}
{तथा त्वामद्य पश्यामि सनाथो मे पतिस्त्वया} %3-56-21

\twolineshloka
{तवाज्ञया च देवेन्द्र पयोभूतमिदं हविः}
{अशिष्यामि त्वया दत्तं रघूणां कुलवर्धनम्} %3-56-22

\twolineshloka
{इन्द्रहस्ताद् गृहीत्वा तत् पायसं सा शुचिस्मिता}
{न्यवेदयत भर्त्रे सा लक्ष्मणाय च मैथिली} %3-56-23

\twolineshloka
{यदि जीवति मे भर्ता सह भ्रात्रा महाबलः}
{इदमस्तु तयोर्भक्त्या तदाश्नात् पायसं स्वयम्} %3-56-24

\twolineshloka
{इतीव तत् प्राश्य हविर्वरानना जहौ क्षुधादुःखसमुद्भवं च तम्}
{इन्द्रात् प्रवृत्तिम् उपलभ्य जानकी काकुत्स्थयोः प्रीतमना बभूव} %3-56-25

\twolineshloka
{स चापि शक्रस्त्रिदिवालयं तदा प्रीतो ययौ राघवकार्यसिद्धये}
{आमन्त्र्य सीतां स ततो महात्मा जगाम निद्रासहितः स्वमालयम्} %3-56-26


॥इत्यार्षे श्रीमद्रामायणे वाल्मीकीये आदिकाव्ये अरण्यकाण्डे वत्सरावधिकरणम् नाम षट्पञ्चाशः सर्गः ॥३-५६॥
