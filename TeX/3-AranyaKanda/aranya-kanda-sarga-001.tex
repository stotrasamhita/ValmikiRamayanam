\sect{प्रथमः सर्गः — महर्षिसङ्गः}

\twolineshloka
{प्रविश्य तु महारण्यं दण्डकारण्यमात्मवान्}
{रामो ददर्श दुर्धर्षस्तापसाश्रममण्डलम्} %3-1-1

\twolineshloka
{कुशचीरपरिक्षिप्तं ब्राह्म्या लक्ष्म्या समावृतम्}
{यथा प्रदीप्तं दुर्दर्शं गगने सूर्यमण्डलम्} %3-1-2

\twolineshloka
{शरण्यं सर्वभूतानां सुसम्मृष्टाजिरं सदा}
{मृगैर्बहुभिराकीर्णं पक्षिसंघैः समावृतम्} %3-1-3

\twolineshloka
{पूजितं चोपनृत्तं च नित्यमप्सरसां गणैः}
{विशालैरग्निशरणैः स्रुग्भाण्डैरजिनैः कुशैः} %3-1-4

\twolineshloka
{समिद्भिस्तोयकलशैः फलमूलैश्च शोभितम्}
{आरण्यैश्च महावृक्षैः पुण्यैः स्वादुफलैर्वृतम्} %3-1-5

\twolineshloka
{बलिहोमार्चितं पुण्यं ब्रह्मघोषनिनादितम्}
{पुष्पैश्चान्यैः परिक्षिप्तं पद्मिन्या च सपद्मया} %3-1-6

\twolineshloka
{फलमूलाशनैर्दान्तैश्चीरकृष्णाजिनाम्बरैः}
{सूर्यवैश्वानराभैश्च पुराणैर्मुनिभिर्युतम्} %3-1-7

\twolineshloka
{पुण्यैश्च नियताहारैः शोभितं परमर्षिभिः}
{तद् ब्रह्मभवनप्रख्यं ब्रह्मघोषनिनादितम्} %3-1-8

\twolineshloka
{ब्रह्मविद्भिर्महाभागैर्ब्राह्मणैरुपशोभितम्}
{तद् दृष्ट्वा राघवः श्रीमांस्तापसाश्रममण्डलम्} %3-1-9

\twolineshloka
{अभ्यगच्छन्महातेजा विज्यं कृत्वा महद् धनुः}
{दिव्यज्ञानोपपन्नास्ते रामं दृष्ट्वा महर्षयः} %3-1-10

\twolineshloka
{अभिजग्मुस्तदा प्रीता वैदेहीं च यशस्विनीम्}
{ते तु सोममिवोद्यन्तं दृष्ट्वा वै धर्मचारिणम्} %3-1-11

\twolineshloka
{लक्ष्मणं चैव दृष्ट्वा तु वैदेहीं च यशस्विनीम्}
{मङ्गलानि प्रयुञ्जानाः प्रत्यगृह्णन् दृढव्रताः} %3-1-12

\twolineshloka
{रूपसंहननं लक्ष्मीं सौकुमार्यं सुवेषताम्}
{ददृशुर्विस्मिताकारा रामस्य वनवासिनः} %3-1-13

\twolineshloka
{वैदेहीं लक्ष्मणं रामं नेत्रैरनिमिषैरिव}
{आश्चर्यभूतान् ददृशुः सर्वे ते वनवासिनः} %3-1-14

\twolineshloka
{अत्रैनं हि महाभागाः सर्वभूतहिते रताः}
{अतिथिं पर्णशालायां राघवं संन्यवेशयन्} %3-1-15

\twolineshloka
{ततो रामस्य सत्कृत्य विधिना पावकोपमाः}
{आजह्रुस्ते महाभागाः सलिलं धर्मचारिणः} %3-1-16

\twolineshloka
{मङ्गलानि प्रयुञ्जाना मुदा परमया युताः}
{मूलं पुष्पं फलं सर्वमाश्रमं च महात्मनः} %3-1-17

\twolineshloka
{निवेदयित्वा धर्मज्ञास्ते तु प्राञ्जलयोऽब्रुवन्}
{धर्मपालो जनस्यास्य शरण्यश्च महायशाः} %3-1-18

\twolineshloka
{पूजनीयश्च मान्यश्च राजा दण्डधरो गुरुः}
{इन्द्रस्यैव चतुर्भागः प्रजा रक्षति राघव} %3-1-19

\threelineshloka
{राजा तस्माद् वरान् भोगान् रम्यान् भुङ्क्ते नमस्कृतः}
{ते वयं भवता रक्ष्या भवद्विषयवासिनः}
{नगरस्थो वनस्थो वा त्वं नो राजा जनेश्वरः} %3-1-20

\twolineshloka
{न्यस्तदण्डा वयं राजञ्जितक्रोधा जितेन्द्रियाः}
{रक्षणीयास्त्वया शश्वद् गर्भभूतास्तपोधनाः} %3-1-21

\twolineshloka
{एवमुक्त्वा फलैर्मूलैः पुष्पैरन्यैश्च राघवम्}
{वन्यैश्च विविधाहारैः सलक्ष्मणमपूजयन्} %3-1-22

\twolineshloka
{तथान्ये तापसाः सिद्धा रामं वैश्वानरोपमाः}
{न्यायवृत्ता यथान्यायं तर्पयामासुरीश्वरम्} %3-1-23


॥इत्यार्षे श्रीमद्रामायणे वाल्मीकीये आदिकाव्ये अरण्यकाण्डे महर्षिसङ्गः नाम प्रथमः सर्गः ॥३-१॥
