\sect{पञ्चचत्वारिंशः सर्गः — सीतापारुष्यम्}

\twolineshloka
{आर्तस्वरं तु तं भर्तुर्विज्ञाय सदृशं वने}
{उवाच लक्ष्मणं सीता गच्छ जानीहि राघवम्} %3-45-1

\twolineshloka
{नहि मे जीवितं स्थाने हृदयं वावतिष्ठते}
{क्रोशतः परमार्तस्य श्रुतः शब्दो मया भृशम्} %3-45-2

\twolineshloka
{आक्रन्दमानं तु वने भ्रातरं त्रातुमर्हसि}
{तं क्षिप्रमभिधाव त्वं भ्रातरं शरणैषिणम्} %3-45-3

\twolineshloka
{रक्षसां वशमापन्नं सिंहानामिव गोवृषम्}
{न जगाम तथोक्तस्तु भ्रातुराज्ञाय शासनम्} %3-45-4

\twolineshloka
{तमुवाच ततस्तत्र क्षुभिता जनकात्मजा}
{सौमित्रे मित्ररूपेण भ्रातुस्त्वमसि शत्रुवत्} %3-45-5

\twolineshloka
{यस्त्वमस्यामवस्थायां भ्रातरं नाभिपद्यसे}
{इच्छसि त्वं विनश्यन्तं रामं लक्ष्मण मत्कृते} %3-45-6

\twolineshloka
{लोभात्तु मत्कृते नूनं नानुगच्छसि राघवम्}
{व्यसनं ते प्रियं मन्ये स्नेहो भ्रातरि नास्ति ते} %3-45-7

\twolineshloka
{तेन तिष्ठसि विस्रब्धं तमपश्यन् महाद्युतिम्}
{किं हि संशयमापन्ने तस्मिन्निह मया भवेत्} %3-45-8

\twolineshloka
{कर्तव्यमिह तिष्ठन्त्या यत्प्रधानस्त्वमागतः}
{एवं ब्रुवाणां वैदेहीं बाष्पशोकसमन्विताम्} %3-45-9

\twolineshloka
{अब्रवील्लक्ष्मणस्त्रस्तां सीतां मृगवधूमिव}
{पन्नगासुरगन्धर्वदेवदानवराक्षसैः} %3-45-10

\twolineshloka
{अशक्यस्तव वैदेहि भर्ता जेतुं न संशयः}
{देवि देवमनुष्येषु गन्धर्वेषु पतत्रिषु} %3-45-11

\twolineshloka
{राक्षसेषु पिशाचेषु किन्नरेषु मृगेषु च}
{दानवेषु च घोरेषु न स विद्येत शोभने} %3-45-12

\twolineshloka
{यो रामं प्रतियुध्येत समरे वासवोपमम्}
{अवध्यः समरे रामो नैवं त्वं वक्तुमर्हसि} %3-45-13

\twolineshloka
{न त्वामस्मिन् वने हातुमुत्सहे राघवं विना}
{अनिवार्यं बलं तस्य बलैर्बलवतामपि} %3-45-14

\twolineshloka
{त्रिभिर्लोकैः समुदितैः सेश्वरैः सामरैरपि}
{हृदयं निर्वृतं तेऽस्तु संतापस्त्यज्यतां तव} %3-45-15

\twolineshloka
{आगमिष्यति ते भर्ता शीघ्रं हत्वा मृगोत्तमम्}
{न स तस्य स्वरो व्यक्तं न कश्चिदपि दैवतः} %3-45-16

\twolineshloka
{गन्धर्वनगरप्रख्या माया तस्य च रक्षसः}
{न्यासभूतासि वैदेहि न्यस्ता मयि महात्मना} %3-45-17

\twolineshloka
{रामेण त्वं वरारोहे न त्वां त्यक्तुमिहोत्सहे}
{कृतवैराश्च कल्याणि वयमेतैर्निशाचरैः} %3-45-18

\twolineshloka
{खरस्य निधने देवि जनस्थानवधं प्रति}
{राक्षसा विविधा वाचो व्याहरन्ति महावने} %3-45-19

\twolineshloka
{हिंसाविहारा वैदेहि न चिन्तयितुमर्हसि}
{लक्ष्मणेनैवमुक्ता तु क्रुद्धा संरक्तलोचना} %3-45-20

\twolineshloka
{अब्रवीत् परुषं वाक्यं लक्ष्मणं सत्यवादिनम्}
{अनार्याकरुणारम्भ नृशंस कुलपांसन} %3-45-21

\twolineshloka
{अहं तव प्रियं मन्ये रामस्य व्यसनं महत्}
{रामस्य व्यसनं दृष्ट्वा तेनैतानि प्रभाषसे} %3-45-22

\twolineshloka
{नैव चित्रं सपत्नेषु पापं लक्ष्मण यद् भवेत्}
{त्वद्विधेषु नृशंसेषु नित्यं प्रच्छन्नचारिषु} %3-45-23

\twolineshloka
{सुदुष्टस्त्वं वने राममेकमेकोऽनुगच्छसि}
{मम हेतोः प्रतिच्छन्नः प्रयुक्तो भरतेन वा} %3-45-24

\twolineshloka
{तन्न सिध्यति सौमित्रे तवापि भरतस्य वा}
{कथमिन्दीवरश्यामं रामं पद्मनिभेक्षणम्} %3-45-25

\twolineshloka
{उपसंश्रित्य भर्तारं कामयेयं पृथग्जनम्}
{समक्षं तव सौमित्रे प्राणांस्त्यक्ष्याम्यसंशयम्} %3-45-26

\twolineshloka
{रामं विना क्षणमपि नैव जीवामि भूतले}
{इत्युक्तः परुषं वाक्यं सीतया रोमहर्षणम्} %3-45-27

\twolineshloka
{अब्रवील्लक्ष्मणः सीतां प्राञ्जलिः स जितेन्द्रियः}
{उत्तरं नोत्सहे वक्तुं दैवतं भवती मम} %3-45-28

\twolineshloka
{वाक्यमप्रतिरूपं तु न चित्रं स्त्रीषु मैथिलि}
{स्वभावस्त्वेष नारीणामेषु लोकेषु दृश्यते} %3-45-29

\twolineshloka
{विमुक्तधर्माश्चपलास्तीक्ष्णा भेदकराः स्त्रियः}
{न सहे हीदृशं वाक्यं वैदेहि जनकात्मजे} %3-45-30

\twolineshloka
{श्रोत्रयोरुभयोर्मध्ये तप्तनाराचसंनिभम्}
{उपशृण्वन्तु मे सर्वे साक्षिणो हि वनेचराः} %3-45-31

\twolineshloka
{न्यायवादी यथा वाक्यमुक्तोऽहं परुषं त्वया}
{धिक् त्वामद्य विनश्यन्तीं यन्मामेवं विशङ्कसे} %3-45-32

\twolineshloka
{स्त्रीत्वाद् दुष्टस्वभावेन गुरुवाक्ये व्यवस्थितम्}
{गच्छामि यत्र काकुत्स्थः स्वस्ति तेऽस्तु वरानने} %3-45-33

\threelineshloka
{रक्षन्तु त्वां विशालाक्षि समग्रा वनदेवताः}
{निमित्तानि हि घोराणि यानि प्रादुर्भवन्ति मे}
{अपि त्वां सह रामेण पश्येयं पुनरागतः} %3-45-34

\twolineshloka
{लक्ष्मणेनैवमुक्ता तु रुदती जनकात्मजा}
{प्रत्युवाच ततो वाक्यं तीव्रबाष्पपरिप्लुता} %3-45-35

\twolineshloka
{गोदावरीं प्रवेक्ष्यामि हीना रामेण लक्ष्मण}
{आबन्धिष्येऽथवा त्यक्ष्ये विषमे देहमात्मनः} %3-45-36

\twolineshloka
{पिबामि वा विषं तीक्ष्णं प्रवेक्ष्यामि हुताशनम्}
{न त्वहं राघवादन्यं कदापि पुरुषं स्पृशे} %3-45-37

\twolineshloka
{इति लक्ष्मणमाश्रुत्य सीता शोकसमन्विता}
{पाणिभ्यां रुदती दुःखादुदरं प्रजघान ह} %3-45-38

\twolineshloka
{तामार्तरूपां विमना रुदन्तीं सौमित्रिरालोक्य विशालनेत्राम्}
{आश्वासयामास न चैव भर्तुस्तं भ्रातरं किंचिदुवाच सीता} %3-45-39

\twolineshloka
{ततस्तु सीतामभिवाद्य लक्ष्मणः कृताञ्जलिः किंचिदभिप्रणम्य}
{अवेक्षमाणो बहुशः स मैथिलीं जगाम रामस्य समीपमात्मवान्} %3-45-40


॥इत्यार्षे श्रीमद्रामायणे वाल्मीकीये आदिकाव्ये अरण्यकाण्डे सीतापारुष्यम् नाम पञ्चचत्वारिंशः सर्गः ॥३-४५॥
