\sect{सप्तषष्ठितमः सर्गः — गृध्रराजदर्शनम्}

\twolineshloka
{पूर्वजोऽप्युक्तमात्रस्तु लक्ष्मणेन सुभाषितम्}
{सारग्राही महासारं प्रतिजग्राह राघवः} %3-67-1

\twolineshloka
{स निगृह्य महाबाहुः प्रवृद्धं रोषमात्मनः}
{अवष्टभ्य धनुश्चित्रं रामो लक्ष्मणमब्रवीत्} %3-67-2

\twolineshloka
{किं करिष्यावहे वत्स क्व वा गच्छाव लक्ष्मण}
{केनोपायेन पश्यावः सीतामिह विचिन्तय} %3-67-3

\twolineshloka
{तं तथा परितापार्तं लक्ष्मणो वाक्यमब्रवीत्}
{इदमेव जनस्थानं त्वमन्वेषितुमर्हसि} %3-67-4

\twolineshloka
{राक्षसैर्बहुभिः कीर्णं नानाद्रुमलतायुतम्}
{सन्तीह गिरिदुर्गाणि निर्दराः कन्दराणि च} %3-67-5

\twolineshloka
{गुहाश्च विविधा घोरा नानामृगगणाकुलाः}
{आवासाः किन्नराणां च गन्धर्वभवनानि च} %3-67-6

\twolineshloka
{तानि युक्तो मया सार्धं समन्वेषितुमर्हसि}
{त्वद्विधा बुद्धिसम्पन्ना महात्मानो नरर्षभाः} %3-67-7

\twolineshloka
{आपत्सु न प्रकम्पन्ते वायुवेगैरिवाचलाः}
{इत्युक्तस्तद् वनं सर्वं विचचार सलक्ष्मणः} %3-67-8

\twolineshloka
{क्रुद्धो रामः शरं घोरं सन्धाय धनुषि क्षुरम्}
{ततः पर्वतकूटाभं महाभागं द्विजोत्तमम्} %3-67-9

\twolineshloka
{ददर्श पतितं भूमौ क्षतजार्द्रं जटायुषम्}
{तं दृष्ट्वा गिरिशृङ्गाभं रामो लक्ष्मणमब्रवीत्} %3-67-10

\twolineshloka
{अनेन सीता वैदेही भक्षिता नात्र संशयः}
{गृध्ररूपमिदं व्यक्तं रक्षो भ्रमति काननम्} %3-67-11

\twolineshloka
{भक्षयित्वा विशालाक्षीमास्ते सीतां यथासुखम्}
{एनं वधिष्ये दीप्ताग्रैः शरैर्घोरैरजिह्मगैः} %3-67-12

\twolineshloka
{इत्युक्त्वाभ्यपतद् द्रष्टुं सन्धाय धनुषि क्षुरम्}
{क्रुद्धो रामः समुद्रान्तां चालयन्निव मेदिनीम्} %3-67-13

\twolineshloka
{तं दीनदीनया वाचा सफेनं रुधिरं वमन्}
{अभ्यभाषत पक्षी स रामं दशरथात्मजम्} %3-67-14

\twolineshloka
{यामोषधीमिवायुष्मन्नन्वेषसि महावने}
{सा देवी मम च प्राणा रावणेनोभयं हृतम्} %3-67-15

\twolineshloka
{त्वया विरहिता देवी लक्ष्मणेन च राघव}
{ह्रियमाणा मया दृष्टा रावणेन बलीयसा} %3-67-16

\twolineshloka
{सीतामभ्यवपन्नोऽहं रावणश्च रणे प्रभो}
{विध्वंसितरथच्छत्रः पतितो धरणीतले} %3-67-17

\twolineshloka
{एतदस्य धनुर्भग्नमेते चास्य शरास्तथा}
{अयमस्य रणे राम भग्नः साङ्ग्रामिको रथः} %3-67-18

\twolineshloka
{अयं तु सारथिस्तस्य मत्पक्षनिहतो भुवि}
{परिश्रान्तस्य मे पक्षौ छित्त्वा खड्गेन रावणः} %3-67-19

\twolineshloka
{सीतामादाय वैदेहीमुत्पपात विहायसम्}
{रक्षसा निहतं पूर्वं मां न हन्तुं त्वमर्हसि} %3-67-20

\twolineshloka
{रामस्तस्य तु विज्ञाय सीतासक्तां प्रियां कथाम्}
{गृध्रराजं परिष्वज्य परित्यज्य महद् धनुः} %3-67-21

\twolineshloka
{निपपातावशो भूमौ रुरोद सहलक्ष्मणः}
{द्विगुणीकृततापार्तो रामो धीरतरोऽपि सन्} %3-67-22

\twolineshloka
{एकमेकायने कृच्छ्रे निःश्वसन्तं मुहुर्मुहुः}
{समीक्ष्य दुःखितो रामः सौमित्रिमिदमब्रवीत्} %3-67-23

\twolineshloka
{राज्यं भ्रष्टं वने वासः सीता नष्टा मृतो द्विजः}
{ईदृशीयं ममालक्ष्मीर्दहेदपि हि पावकम्} %3-67-24

\twolineshloka
{सम्पूर्णमपि चेदद्य प्रतरेयं महोदधिम्}
{सोऽपि नूनं ममालक्ष्म्या विशुष्येत् सरितां पतिः} %3-67-25

\twolineshloka
{नास्त्यभाग्यतरो लोके मत्तोऽस्मिन् स चराचरे}
{येनेयं महती प्राप्ता मया व्यसनवागुरा} %3-67-26

\twolineshloka
{अयं पितुर्वयस्यो मे गृध्रराजो महाबलः}
{शेते विनिहतो भूमौ मम भाग्यविपर्ययात्} %3-67-27

\twolineshloka
{इत्येवमुक्त्वा बहुशो राघवः सहलक्ष्मणः}
{जटायुषं च पस्पर्श पितृस्नेहं निदर्शयन्} %3-67-28

\twolineshloka
{निकृत्तपक्षं रुधिरावसिक्तं तं गृध्रराजं परिगृह्य राघवः}
{क्व मैथिली प्राणसमा गतेति विमुच्य वाचं निपपात भूमौ} %3-67-29


॥इत्यार्षे श्रीमद्रामायणे वाल्मीकीये आदिकाव्ये अरण्यकाण्डे गृध्रराजदर्शनम् नाम सप्तषष्ठितमः सर्गः ॥३-६७॥
