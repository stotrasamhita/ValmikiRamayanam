\sect{एकविंशः सर्गः — खरसन्धुक्षणम्}

\twolineshloka
{स पुनः पतितां दृष्ट्वा क्रोधाच्छूर्पणखां पुनः}
{उवाच व्यक्तया वाचा तामनर्थार्थमागताम्} %3-21-1

\twolineshloka
{मया त्विदानीं शूरास्ते राक्षसाः पिशिताशनाः}
{त्वत्प्रियार्थं विनिर्दिष्टाः किमर्थं रुद्यते पुनः} %3-21-2

\twolineshloka
{भक्ताश्चैवानुरक्ताश्च हिताश्च मम नित्यशः}
{हन्यमाना न हन्यन्ते न न कुर्युर्वचो मम} %3-21-3

\twolineshloka
{किमेतच्छ्रोतुमिच्छामि कारणं यत्कृते पुनः}
{हा नाथेति विनर्दन्ती सर्पवच्चेष्टसे क्षितौ} %3-21-4

\twolineshloka
{अनाथवद् विलपसि किं नु नाथे मयि स्थिते}
{उत्तिष्ठोत्तिष्ठ मा मैवं वैक्लव्यं त्यज्यतामिति} %3-21-5

\twolineshloka
{इत्येवमुक्ता दुर्धर्षा खरेण परिसान्त्विता}
{विमृज्य नयने सास्रे खरं भ्रातरमब्रवीत्} %3-21-6

\twolineshloka
{अस्मीदानीमहं प्राप्ता हतश्रवणनासिका}
{शोणितौघपरिक्लिन्ना त्वया च परिसान्त्विता} %3-21-7

\twolineshloka
{प्रेषिताश्च त्वया शूरा राक्षसास्ते चतुर्दश}
{निहन्तुं राघवं घोरं मत्प्रियार्थं सलक्ष्मणम्} %3-21-8

\twolineshloka
{ते तु रामेण सामर्षाः शूलपट्टिशपाणयः}
{समरे निहताः सर्वे सायकैर्मर्मभेदिभिः} %3-21-9

\twolineshloka
{तान् भूमौ पतितान् दृष्ट्वा क्षणेनैव महाजवान्}
{रामस्य च महत्कर्म महांस्त्रासोऽभवन्मम} %3-21-10

\twolineshloka
{सास्मि भीता समुद्विग्ना विषण्णा च निशाचर}
{शरणं त्वां पुनः प्राप्ता सर्वतो भयदर्शिनी} %3-21-11

\twolineshloka
{विषादनक्राध्युषिते परित्रासोर्मिमालिनि}
{किं मां न त्रायसे मग्नां विपुले शोकसागरे} %3-21-12

\twolineshloka
{एते च निहता भूमौ रामेण निशितैः शरैः}
{ये च मे पदवीं प्राप्ता राक्षसाः पिशिताशनाः} %3-21-13

\twolineshloka
{मयि ते यद्यनुक्रोशो यदि रक्षःसु तेषु च}
{रामेण यदि शक्तिस्ते तेजो वास्ति निशाचर} %3-21-14

\twolineshloka
{दण्डकारण्यनिलयं जहि राक्षसकण्टकम्}
{यदि रामममित्रघ्नं न त्वमद्य वधिष्यसि} %3-21-15

\twolineshloka
{तव चैवाग्रतः प्राणांस्त्यक्ष्यामि निरपत्रपा}
{बुद्ध्याहमनुपश्यामि न त्वं रामस्य संयुगे} %3-21-16

\twolineshloka
{स्थातुं प्रतिमुखे शक्तः सबलोऽपि महारणे}
{शूरमानी न शूरस्त्वं मिथ्यारोपितविक्रमः} %3-21-17

\twolineshloka
{अपयाहि जनस्थानात् त्वरितः सहबान्धवः}
{जहि त्वं समरे मूढान्यथा तु कुलपांसन} %3-21-18

\twolineshloka
{मानुषौ तौ न शक्नोषि हन्तुं वै रामलक्ष्मणौ}
{निःसत्त्वस्याल्पवीर्यस्य वासस्ते कीदृशस्त्विह} %3-21-19

\twolineshloka
{रामतेजोऽभिभूतो हि त्वं क्षिप्रं विनशिष्यसि}
{स हि तेजःसमायुक्तो रामो दशरथात्मजः} %3-21-20

\twolineshloka
{भ्राता चास्य महावीर्यो येन चास्मि विरूपिता}
{एवं विलप्य बहुशो राक्षसी प्रदरोदरी} %3-21-21

\twolineshloka
{भ्रातुः समीपे शोकार्ता नष्टसंज्ञा बभूव ह}
{कराभ्यामुदरं हत्वा रुरोद भृशदुःखिता} %3-21-22


॥इत्यार्षे श्रीमद्रामायणे वाल्मीकीये आदिकाव्ये अरण्यकाण्डे खरसन्धुक्षणम् नाम एकविंशः सर्गः ॥३-२१॥
