\sect{षष्ठः सर्गः — रक्षोवधप्रतिज्ञानम्}

\twolineshloka
{शरभङ्गे दिवं प्राप्ते मुनिसङ्घाः समागताः}
{अभ्यगच्छन्त काकुत्स्थं रामं ज्वलिततेजसम्} %3-6-1

\twolineshloka
{वैखानसा वालखिल्याः सम्प्रक्षाला मरीचिपाः}
{अश्मकुट्टाश्च बहवः पत्राहाराश्च तापसाः} %3-6-2

\twolineshloka
{दन्तोलूखलिनश्चैव तथैवोन्मज्जकाः परे}
{गात्रशय्या अशय्याश्च तथैवानवकाशिकाः} %3-6-3

\twolineshloka
{मुनयः सलिलाहारा वायुभक्षास्तथापरे}
{आकाशनिलयाश्चैव तथा स्थण्डिलशायिनः} %3-6-4

\twolineshloka
{तथोर्ध्ववासिनो दान्तास्तथाऽऽर्द्रपटवाससः}
{सजपाश्च तपोनिष्ठास्तथा पञ्चतपोऽन्विताः} %3-6-5

\twolineshloka
{सर्वे ब्राह्म्या श्रिया युक्ता दृढयोगसमाहिताः}
{शरभङ्गाश्रमे राममभिजग्मुश्च तापसाः} %3-6-6

\twolineshloka
{अभिगम्य च धर्मज्ञा रामं धर्मभृतां वरम्}
{ऊचुः परमधर्मज्ञमृषिसङ्घाः समागताः} %3-6-7

\twolineshloka
{त्वमिक्ष्वाकुकुलस्यास्य पृथिव्याश्च महारथः}
{प्रधानश्चापि नाथश्च देवानां मघवानिव} %3-6-8

\twolineshloka
{विश्रुतस्त्रिषु लोकेषु यशसा विक्रमेण च}
{पितृव्रतत्वं सत्यं च त्वयि धर्मश्च पुष्कलः} %3-6-9

\twolineshloka
{त्वामासाद्य महात्मानं धर्मज्ञं धर्मवत्सलम्}
{अर्थित्वान्नाथ वक्ष्यामस्तच्च नः क्षन्तुमर्हसि} %3-6-10

\twolineshloka
{अधर्मः सुमहान् नाथ भवेत् तस्य तु भूपतेः}
{यो हरेद् बलिषड्भागं न च रक्षति पुत्रवत्} %3-6-11

\twolineshloka
{युञ्जानः स्वानिव प्राणान् प्राणैरिष्टान् सुतानिव}
{नित्ययुक्तः सदा रक्षन् सर्वान् विषयवासिनः} %3-6-12

\twolineshloka
{प्राप्नोति शाश्वतीं राम कीर्तिं स बहुवार्षिकीम्}
{ब्रह्मणः स्थानमासाद्य तत्र चापि महीयते} %3-6-13

\twolineshloka
{यत् करोति परं धर्मं मुनिर्मूलफलाशनः}
{तत्र राज्ञश्चतुर्भागः प्रजा धर्मेण रक्षतः} %3-6-14

\twolineshloka
{सोऽयं ब्राह्मणभूयिष्ठो वानप्रस्थगणो महान्}
{त्वन्नाथोऽनाथवद् राम राक्षसैर्हन्यते भृशम्} %3-6-15

\twolineshloka
{एहि पश्य शरीराणि मुनीनां भावितात्मनाम्}
{हतानां राक्षसैर्घोरैर्बहूनां बहुधा वने} %3-6-16

\twolineshloka
{पम्पानदीनिवासानामनुमन्दाकिनीमपि}
{चित्रकूटालयानां च क्रियते कदनं महत्} %3-6-17

\twolineshloka
{एवं वयं न मृष्यामो विप्रकारं तपस्विनाम्}
{क्रियमाणं वने घोरं रक्षोभिर्भीमकर्मभिः} %3-6-18

\twolineshloka
{ततस्त्वां शरणार्थं च शरण्यं समुपस्थिताः}
{परिपालय नो राम वध्यमानान् निशाचरैः} %3-6-19

\twolineshloka
{परा त्वत्तो गतिर्वीर पृथिव्यां नोपपद्यते}
{परिपालय नः सर्वान् राक्षसेभ्यो नृपात्मज} %3-6-20

\twolineshloka
{एतच्छ्रुत्वा तु काकुत्स्थस्तापसानां तपस्विनाम्}
{इदं प्रोवाच धर्मात्मा सर्वानेव तपस्विनः} %3-6-21

\twolineshloka
{नैवमर्हथ मां वक्तुमाज्ञाप्योऽहं तपस्विनाम्}
{केवलेन स्वकार्येण प्रवेष्टव्यं वनं मया} %3-6-22

\twolineshloka
{विप्रकारमपाक्रष्टुं राक्षसैर्भवतामिमम्}
{पितुस्तु निर्देशकरः प्रविष्टोऽहमिदं वनम्} %3-6-23

\twolineshloka
{भवतामर्थसिद्ध्यर्थमागतोऽहं यदृच्छया}
{तस्य मेऽयं वने वासो भविष्यति महाफलः} %3-6-24

\twolineshloka
{तपस्विनां रणे शत्रून् हन्तुमिच्छामि राक्षसान्}
{पश्यन्तु वीर्यमृषयः सभ्रातुर्मे तपोधनाः} %3-6-25

\twolineshloka
{दत्त्वा वरं चापि तपोधनानां धर्मे धृतात्मा सह लक्ष्मणेन}
{तपोधनैश्चापि सहार्यदत्तः सुतीक्ष्णमेवाभिजगाम वीरः} %3-6-26


॥इत्यार्षे श्रीमद्रामायणे वाल्मीकीये आदिकाव्ये अरण्यकाण्डे रक्षोवधप्रतिज्ञानम् नाम षष्ठः सर्गः ॥३-६॥
