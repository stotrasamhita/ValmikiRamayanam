\sect{सप्तत्रिंशः सर्गः — अप्रियपथ्यवचनम्}

\twolineshloka
{तच्छ्रुत्वा राक्षसेन्द्रस्य वाक्यं वाक्यविशारदः}
{प्रत्युवाच महातेजा मारीचो राक्षसेश्वरम्} %3-37-1

\twolineshloka
{सुलभाः पुरुषा राजन् सततं प्रियवादिनः}
{अप्रियस्य च पथ्यस्य वक्ता श्रोता च दुर्लभः} %3-37-2

\twolineshloka
{न नूनं बुध्यसे रामं महावीर्यगुणोन्नतम्}
{अयुक्तचारश्चपलो महेन्द्रवरुणोपमम्} %3-37-3

\twolineshloka
{अपि स्वस्ति भवेत् तात सर्वेषामपि रक्षसाम्}
{अपि रामो न सङ्क्रुद्धः कुर्याल्लोकानराक्षसान्} %3-37-4

\twolineshloka
{अपि ते जीवितान्ताय नोत्पन्ना जनकात्मजा}
{अपि सीतानिमित्तं च न भवेद् व्यसनं महत्} %3-37-5

\twolineshloka
{अपि त्वामीश्वरं प्राप्य कामवृत्तं निरङ्कुशम्}
{न विनश्येत् पुरी लङ्का त्वया सह सराक्षसा} %3-37-6

\twolineshloka
{त्वद्विधः कामवृत्तो हि दुःशीलः पापमन्त्रितः}
{आत्मानं स्वजनं राष्ट्रं स राजा हन्ति दुर्मतिः} %3-37-7

\twolineshloka
{न च पित्रा परित्यक्तो नामर्यादः कथञ्चन}
{न लुब्धो न च दुःशीलो न च क्षत्रियपांसनः} %3-37-8

\twolineshloka
{न च धर्मगुणैर्हीनः कौसल्यानन्दवर्धनः}
{न च तीक्ष्णो हि भूतानां सर्वभूतहिते रतः} %3-37-9

\twolineshloka
{वञ्चितं पितरं दृष्ट्वा कैकेय्या सत्यवादिनम्}
{करिष्यामीति धर्मात्मा ततः प्रव्रजितो वनम्} %3-37-10

\twolineshloka
{कैकेय्याः प्रियकामार्थं पितुर्दशरथस्य च}
{हित्वा राज्यं च भोगांश्च प्रविष्टो दण्डकावनम्} %3-37-11

\twolineshloka
{न रामः कर्कशस्तात नाविद्वान् नाजितेन्द्रियः}
{अनृतं न श्रुतं चैव नैव त्वं वक्तुमर्हसि} %3-37-12

\twolineshloka
{रामो विग्रहवान् धर्मः साधुः सत्यपराक्रमः}
{राजा सर्वस्य लोकस्य देवानामिव वासवः} %3-37-13

\twolineshloka
{कथं नु तस्य वैदेहीं रक्षितां स्वेन तेजसा}
{इच्छसे प्रसभं हर्तुं प्रभामिव विवस्वतः} %3-37-14

\twolineshloka
{शरार्चिषमनाधृष्यं चापखड्गेन्धनं रणे}
{रामाग्निं सहसा दीप्तं न प्रवेष्टुं त्वमर्हसि} %3-37-15

\twolineshloka
{धनुर्व्यादितदीप्तास्यं शरार्चिषममर्षणम्}
{चापबाणधरं तीक्ष्णं शत्रुसेनापहारिणम्} %3-37-16

\twolineshloka
{राज्यं सुखं च सन्त्यज्य जीवितं चेष्टमात्मनः}
{नात्यासादयितुं तात रामान्तकमिहार्हसि} %3-37-17

\twolineshloka
{अप्रमेयं हि तत्तेजो यस्य सा जनकात्मजा}
{न त्वं समर्थस्तां हर्तुं रामचापाश्रयां वने} %3-37-18

\twolineshloka
{तस्य वै नरसिंहस्य सिंहोरस्कस्य भामिनी}
{प्राणेभ्योऽपि प्रियतरा भार्या नित्यमनुव्रता} %3-37-19

\twolineshloka
{न सा धर्षयितुं शक्या मैथिल्योजस्विनः प्रिया}
{दीप्तस्येव हुताशस्य शिखा सीता सुमध्यमा} %3-37-20

\twolineshloka
{किमुद्यमं व्यर्थमिमं कृत्वा ते राक्षसाधिप}
{दृष्टश्चेत् त्वं रणे तेन तदन्तमुपजीवितम्} %3-37-21

\twolineshloka
{जीवितं च सुखं चैव राज्यं चैव सुदुर्लभम्}
{यदीच्छसि चिरं भोक्तुं मा कृथा रामविप्रियम्} %3-37-22

\threelineshloka
{स सर्वैः सचिवैः सार्धं विभीषणपुरस्कृतैः}
{मन्त्रयित्वा स धर्मिष्ठैः कृत्वा निश्चयमात्मनः}
{दोषाणां च गुणानां च सम्प्रधार्य बलाबलम्} %3-37-23

\twolineshloka
{आत्मनश्च बलं ज्ञात्वा राघवस्य च तत्त्वतः}
{हितं हि तव निश्चित्य क्षमं त्वं कर्तुमर्हसि} %3-37-24

\twolineshloka
{अहं तु मन्ये तव न क्षमं रणे समागमं कोसलराजसूनुना}
{इदं हि भूयः शृणु वाक्यमुत्तमं क्षमं च युक्तं च निशाचराधिप} %3-37-25


॥इत्यार्षे श्रीमद्रामायणे वाल्मीकीये आदिकाव्ये अरण्यकाण्डे अप्रियपथ्यवचनम् नाम सप्तत्रिंशः सर्गः ॥३-३७॥
