\sect{एकोनत्रिंशः सर्गः — खरगदाभेदनम्}

\twolineshloka
{खरं तु विरथं रामो गदापाणिमवस्थितम्}
{मृदुपूर्वं महातेजाः परुषं वाक्यमब्रवीत्} %3-29-1

\twolineshloka
{गजाश्वरथसम्बाधे बले महति तिष्ठता}
{कृतं ते दारुणं कर्म सर्वलोकजुगुप्सितम्} %3-29-2

\twolineshloka
{उद्वेजनीयो भूतानां नृशंसः पापकर्मकृत्}
{त्रयाणामपि लोकानामीश्वरोऽपि न तिष्ठति} %3-29-3

\twolineshloka
{कर्म लोकविरुद्धं तु कुर्वाणं क्षणदाचर}
{तीक्ष्णं सर्वजनो हन्ति सर्पं दुष्टमिवागतम्} %3-29-4

\twolineshloka
{लोभात् पापानि कुर्वाणः कामाद् वा यो न बुध्यते}
{हृष्टः पश्यति तस्यान्तं ब्राह्मणी करकादिव} %3-29-5

\twolineshloka
{वसतो दण्डकारण्ये तापसान् धर्मचारिणः}
{किं नु हत्वा महाभागान् फलं प्राप्स्यसि राक्षस} %3-29-6

\twolineshloka
{न चिरं पापकर्माणः क्रूरा लोकजुगुप्सिताः}
{ऐश्वर्यं प्राप्य तिष्ठन्ति शीर्णमूला इव द्रुमाः} %3-29-7

\twolineshloka
{अवश्यं लभते कर्ता फलं पापस्य कर्मणः}
{घोरं पर्यागते काले द्रुमः पुष्पमिवार्तवम्} %3-29-8

\twolineshloka
{नचिरात् प्राप्यते लोके पापानां कर्मणां फलम्}
{सविषाणामिवान्नानां भुक्तानां क्षणदाचर} %3-29-9

\twolineshloka
{पापमाचरतां घोरं लोकस्याप्रियमिच्छताम्}
{अहमासादितो राज्ञा प्राणान् हन्तुं निशाचर} %3-29-10

\twolineshloka
{अद्य भित्त्वा मया मुक्ताः शराः काञ्चनभूषणाः}
{विदार्यातिपतिष्यन्ति वल्मीकमिव पन्नगाः} %3-29-11

\twolineshloka
{ये त्वया दण्डकारण्ये भक्षिता धर्मचारिणः}
{तानद्य निहतः सङ्ख्ये ससैन्योऽनुगमिष्यसि} %3-29-12

\twolineshloka
{अद्य त्वां निहतं बाणैः पश्यन्तु परमर्षयः}
{निरयस्थं विमानस्था ये त्वया निहताः पुरा} %3-29-13

\twolineshloka
{प्रहरस्व यथाकामं कुरु यत्नं कुलाधम}
{अद्य ते पातयिष्यामि शिरस्तालफलं यथा} %3-29-14

\twolineshloka
{एवमुक्तस्तु रामेण क्रुद्धः संरक्तलोचनः}
{प्रत्युवाच ततो रामं प्रहसन् क्रोधमूर्च्छितः} %3-29-15

\twolineshloka
{प्राकृतान् राक्षसान् हत्वा युद्धे दशरथात्मज}
{आत्मना कथमात्मानमप्रशस्यं प्रशंससि} %3-29-16

\twolineshloka
{विक्रान्ता बलवन्तो वा ये भवन्ति नरर्षभाः}
{कथयन्ति न ते किञ्चित् तेजसा चातिगर्विताः} %3-29-17

\twolineshloka
{प्राकृतास्त्वकृतात्मानो लोके क्षत्रियपांसनाः}
{निरर्थकं विकत्थन्ते यथा राम विकत्थसे} %3-29-18

\twolineshloka
{कुलं व्यपदिशन् वीरः समरे कोऽभिधास्यति}
{मृत्युकाले तु सम्प्राप्ते स्वयमप्रस्तवे स्तवम्} %3-29-19

\twolineshloka
{सर्वथा तु लघुत्वं ते कत्थनेन विदर्शितम्}
{सुवर्णप्रतिरूपेण तप्तेनेव कुशाग्निना} %3-29-20

\twolineshloka
{न तु मामिह तिष्ठन्तं पश्यसि त्वं गदाधरम्}
{धराधरमिवाकम्प्यं पर्वतं धातुभिश्चितम्} %3-29-21

\twolineshloka
{पर्याप्तोऽहं गदापाणिर्हन्तुं प्राणान् रणे तव}
{त्रयाणामपि लोकानां पाशहस्त इवान्तकः} %3-29-22

\twolineshloka
{कामं बह्वपि वक्तव्यं त्वयि वक्ष्यामि न त्वहम्}
{अस्तं प्राप्नोति सविता युद्धविघ्नस्ततो भवेत्} %3-29-23

\twolineshloka
{चतुर्दश सहस्राणि राक्षसानां हतानि ते}
{त्वद्विनाशात् करोम्यद्य तेषामश्रुप्रमार्जनम्} %3-29-24

\twolineshloka
{इत्युक्त्वा परमक्रुद्धः स गदां परमाङ्गदाम्}
{खरश्चिक्षेप रामाय प्रदीप्तामशनिं यथा} %3-29-25

\twolineshloka
{खरबाहुप्रमुक्ता सा प्रदीप्ता महती गदा}
{भस्म वृक्षांश्च गुल्मांश्च कृत्वागात् तत्समीपतः} %3-29-26

\twolineshloka
{तामापतन्तीं महतीं मृत्युपाशोपमां गदाम्}
{अन्तरिक्षगतां रामश्चिच्छेद बहुधा शरैः} %3-29-27

\twolineshloka
{सा विशीर्णा शरैर्भिन्ना पपात धरणीतले}
{गदा मन्त्रौषधिबलैर्व्यालीव विनिपातिता} %3-29-28


॥इत्यार्षे श्रीमद्रामायणे वाल्मीकीये आदिकाव्ये अरण्यकाण्डे खरगदाभेदनम् नाम एकोनत्रिंशः सर्गः ॥३-२९॥
