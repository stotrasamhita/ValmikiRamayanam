\sect{द्विचत्वारिंशः सर्गः — स्वर्णमृगप्रेक्षणम्}

\twolineshloka
{एवमुक्त्वा तु परुषं मारीचो रावणं ततः}
{गच्छावेत्यब्रवीद् दीनो भयाद् रात्रिंचरप्रभोः} %3-42-1

\twolineshloka
{दृष्टश्चाहं पुनस्तेन शरचापासिधारिणा}
{मद्वधोद्यतशस्त्रेण निहतं जीवितं च मे} %3-42-2

\twolineshloka
{नहि रामं पराक्रम्य जीवन् प्रतिनिवर्तते}
{वर्तते प्रतिरूपोऽसौ यमदण्डहतस्य ते} %3-42-3

\twolineshloka
{किं नु कर्तुं मया शक्यमेवं त्वयि दुरात्मनि}
{एष गच्छाम्यहं तात स्वस्ति तेऽस्तु निशाचर} %3-42-4

\twolineshloka
{प्रहृष्टस्त्वभवत् तेन वचनेन स राक्षसः}
{परिष्वज्य सुसंश्लिष्टमिदं वचनमब्रवीत्} %3-42-5

\twolineshloka
{एतच्छौटीर्ययुक्तं ते मच्छन्दवशवर्तिनः}
{इदानीमसि मारीचः पूर्वमन्यो हि राक्षसः} %3-42-6

\twolineshloka
{आरुह्यतामयं शीघ्रं खगो रत्नविभूषितः}
{मया सह रथो युक्तः पिशाचवदनैः खरैः} %3-42-7

\twolineshloka
{प्रलोभयित्वा वैदेहीं यथेष्टं गन्तुमर्हसि}
{तां शून्ये प्रसभं सीतामानयिष्यामि मैथिलीम्} %3-42-8

\twolineshloka
{ततस्तथेत्युवाचैनं रावणं ताटकासुतः}
{ततो रावणमारीचौ विमानमिव तं रथम्} %3-42-9

\twolineshloka
{आरुह्याययतुः शीघ्रं तस्मादाश्रममण्डलात्}
{तथैव तत्र पश्यन्तौ पत्तनानि वनानि च} %3-42-10

\twolineshloka
{गिरींश्च सरितः सर्वा राष्ट्राणि नगराणि च}
{समेत्य दण्डकारण्यं राघवस्याश्रमं ततः} %3-42-11

\twolineshloka
{ददर्श सहमारीचो रावणो राक्षसाधिपः}
{अवतीर्य रथात् तस्मात् ततः काञ्चनभूषणात्} %3-42-12

\twolineshloka
{हस्ते गृहीत्वा मारीचं रावणो वाक्यमब्रवीत्}
{एतद् रामाश्रमपदं दृश्यते कदलीवृतम्} %3-42-13

\twolineshloka
{क्रियतां तत् सखे शीघ्रं यदर्थं वयमागताः}
{स रावणवचः श्रुत्वा मारीचो राक्षसस्तदा} %3-42-14

\twolineshloka
{मृगो भूत्वाऽऽश्रमद्वारि रामस्य विचचार ह}
{स तु रूपं समास्थाय महदद्भुतदर्शनम्} %3-42-15

\twolineshloka
{मणिप्रवरशृङ्गाग्रः सितासितमुखाकृतिः}
{रक्तपद्मोत्पलमुख इन्द्रनीलोत्पलश्रवाः} %3-42-16

\twolineshloka
{किंचिदभ्युन्नतग्रीव इन्द्रनीलनिभोदरः}
{मधूकनिभपार्श्वश्च कञ्जकिञ्जल्कसंनिभः} %3-42-17

\twolineshloka
{वैदूर्यसंकाशखुरस्तनुजङ्घः सुसंहतः}
{इन्द्रायुधसवर्णेन पुच्छेनोर्ध्वं विराजितः} %3-42-18

\twolineshloka
{मनोहरस्निग्धवर्णो रत्नैर्नानाविधैर्वृतः}
{क्षणेन राक्षसो जातो मृगः परमशोभनः} %3-42-19

\twolineshloka
{वनं प्रज्वलयन् रम्यं रामाश्रमपदं च तत्}
{मनोहरं दर्शनीयं रूपं कृत्वा स राक्षसः} %3-42-20

\twolineshloka
{प्रलोभनार्थं वैदेह्या नानाधातुविचित्रितम्}
{विचरन् गच्छते सम्यक् शाद्वलानि समन्ततः} %3-42-21

\twolineshloka
{रौप्यैर्बिन्दुशतैश्चित्रं भूत्वा च प्रियदर्शनः}
{विटपीनां किसलयान् भक्षयन् विचचार ह} %3-42-22

\twolineshloka
{कदलीगृहकं गत्वा कर्णिकारानितस्ततः}
{समाश्रयन् मन्दगतिं सीतासंदर्शनं ततः} %3-42-23

\twolineshloka
{राजीवचित्रपृष्ठः स विरराज महामृगः}
{रामाश्रमपदाभ्याशे विचचार यथासुखम्} %3-42-24

\twolineshloka
{पुनर्गत्वा निवृत्तश्च विचचार मृगोत्तमः}
{गत्वा मुहूर्तं त्वरया पुनः प्रतिनिवर्तते} %3-42-25

\twolineshloka
{विक्रीडंश्च क्वचिद् भूमौ पुनरेव निषीदति}
{आश्रमद्वारमागम्य मृगयूथानि गच्छति} %3-42-26

\twolineshloka
{मृगयूथैरनुगतः पुनरेव निवर्तते}
{सीतादर्शनमाकांक्षन् राक्षसो मृगतां गतः} %3-42-27

\twolineshloka
{परिभ्रमति चित्राणि मण्डलानि विनिष्पतन्}
{समुद्वीक्ष्य च सर्वे तं मृगा येऽन्ये वनेचराः} %3-42-28

\twolineshloka
{उपगम्य समाघ्राय विद्रवन्ति दिशो दश}
{राक्षसः सोऽपि तान् वन्यान् मृगान् मृगवधे रतः} %3-42-29

\twolineshloka
{प्रच्छादनार्थं भावस्य न भक्षयति संस्पृशन्}
{तस्मिन् नेव ततः काले वैदेही शुभलोचना} %3-42-30

\twolineshloka
{कुसुमापचये व्यग्रा पादपानत्यवर्तत}
{कर्णिकारानशोकांश्च चूतांश्च मदिरेक्षणा} %3-42-31

\twolineshloka
{कुसुमान्यपचिन्वन्ती चचार रुचिरानना}
{अनर्हा वनवासस्य सा तं रत्नमयं मृगम्} %3-42-32

\twolineshloka
{मुक्तामणिविचित्राङ्गं ददर्श परमाङ्गना}
{तं वै रुचिरदन्तोष्ठं रूप्यधातुतनूरुहम्} %3-42-33

\twolineshloka
{विस्मयोत्फुल्लनयना सस्नेहं समुदैक्षत}
{स च तां रामदयितां पश्यन् मायामयो मृगः} %3-42-34

\threelineshloka
{विचचार ततस्तत्र दीपयन्निव तद् वनम्}
{अदृष्टपूर्वं दृष्ट्वा तं नानारत्नमयं मृगम्}
{विस्मयं परमं सीता जगाम जनकात्मजा} %3-42-35


॥इत्यार्षे श्रीमद्रामायणे वाल्मीकीये आदिकाव्ये अरण्यकाण्डे स्वर्णमृगप्रेक्षणम् नाम द्विचत्वारिंशः सर्गः ॥३-४२॥
