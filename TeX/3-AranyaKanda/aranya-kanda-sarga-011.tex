\sect{एकादशः सर्गः — अगस्त्याश्रमः}

\twolineshloka
{अग्रतः प्रययौ रामः सीता मध्ये सुशोभना}
{पृष्ठतस्तु धनुष्पाणिर्लक्ष्मणोऽनुजगाम ह} %3-11-1

\twolineshloka
{तौ पश्यमानौ विविधान् शैलप्रस्थान् वनानि च}
{नदीश्च विविधा रम्या जग्मतुः सह सीतया} %3-11-2

\twolineshloka
{सारसांश्चक्रवाकांश्च नदीपुलिनचारिणः}
{सरांसि च सपद्मानि युतानि जलजैः खगैः} %3-11-3

\twolineshloka
{यूथबद्धांश्च पृषतान् मदोन्मत्तान् विषाणिनः}
{महिषांश्च वराहांश्च गजांश्च द्रुमवैरिणः} %3-11-4

\twolineshloka
{ते गत्वा दूरमध्वानं लम्बमाने दिवाकरे}
{ददृशुः सहिता रम्यं तटाकं योजनायुतम्} %3-11-5

\twolineshloka
{पद्मपुष्करसम्बाधं गजयूथैरलंकृतम्}
{सारसैर्हंसकादम्बैः संकुलं जलजातिभिः} %3-11-6

\twolineshloka
{प्रसन्नसलिले रम्ये तस्मिन् सरसि शुश्रुवे}
{गीतवादित्रनिर्घोषो न तु कश्चन दृश्यते} %3-11-7

\twolineshloka
{ततः कौतूहलाद् रामो लक्ष्मणश्च महारथः}
{मुनिं धर्मभृतं नाम प्रष्टुं समुपचक्रमे} %3-11-8

\twolineshloka
{इदमत्यद्भुतं श्रुत्वा सर्वेषां नो महामुने}
{कौतूहलं महज्जातं किमिदं साधु कथ्यताम्} %3-11-9

\twolineshloka
{तेनैवमुक्तो धर्मात्मा राघवेण मुनिस्तदा}
{प्रभावं सरसः क्षिप्रमाख्यातुमुपचक्रमे} %3-11-10

\twolineshloka
{इदं पञ्चाप्सरो नाम तटाकं सार्वकालिकम्}
{निर्मितं तपसा राम मुनिना माण्डकर्णिना} %3-11-11

\twolineshloka
{स हि तेपे तपस्तीव्रं माण्डकर्णिर्महामुनिः}
{दशवर्षसहस्राणि वायुभक्षो जलाशये} %3-11-12

\twolineshloka
{ततः प्रव्यथिताः सर्वे देवाः साग्निपुरोगमाः}
{अब्रुवन् वचनं सर्वे परस्परसमागताः} %3-11-13

\twolineshloka
{अस्माकं कस्यचित् स्थानमेष प्रार्थयते मुनिः}
{इति संविग्नमनसः सर्वे तत्र दिवौकसः} %3-11-14

\twolineshloka
{ततः कर्तुं तपोविघ्नं सर्वदेवैर्नियोजिताः}
{प्रधानाप्सरसः पञ्च विद्युच्चलितवर्चसः} %3-11-15

\twolineshloka
{अप्सरोभिस्ततस्ताभिर्मुनिर्दृष्टपरावरः}
{नीतो मदनवश्यत्वं देवानां कार्यसिद्धये} %3-11-16

\twolineshloka
{ताश्चैवाप्सरसः पञ्च मुनेः पत्नीत्वमागताः}
{तटाके निर्मितं तासां तस्मिन्नन्तर्हितं गृहम्} %3-11-17

\twolineshloka
{तत्रैवाप्सरसः पञ्च निवसन्त्यो यथासुखम्}
{रमयन्ति तपोयोगान्मुनिं यौवनमास्थितम्} %3-11-18

\twolineshloka
{तासां संक्रीडमानानामेष वादित्रनिःस्वनः}
{श्रूयते भूषणोन्मिश्रो गीतशब्दो मनोहरः} %3-11-19

\twolineshloka
{आश्चर्यमिति तस्यैतद् वचनं भावितात्मनः}
{राघवः प्रतिजग्राह सह भ्रात्रा महायशाः} %3-11-20

\twolineshloka
{एवं कथयमानः स ददर्शाश्रममण्डलम्}
{कुशचीरपरिक्षिप्तं ब्राह्म्या लक्ष्म्या समावृतम्} %3-11-21

\twolineshloka
{प्रविश्य सह वैदेह्या लक्ष्मणेन च राघवः}
{तदा तस्मिन् स काकुत्स्थः श्रीमत्याश्रममण्डले} %3-11-22

\twolineshloka
{उषित्वा स सुखं तत्र पूज्यमानो महर्षिभिः}
{जगाम चाश्रमांस्तेषां पर्यायेण तपस्विनाम्} %3-11-23

\twolineshloka
{येषामुषितवान् पूर्वं सकाशे स महास्त्रवित्}
{क्वचित् परिदशान् मासानेकसंवत्सरं क्वचित्} %3-11-24

\twolineshloka
{क्वचिच्च चतुरो मासान् पञ्च षट् च परान् क्वचित्}
{अपरत्राधिकान् मासानध्यर्धमधिकं क्वचित्} %3-11-25

\twolineshloka
{त्रीन् मासानष्टमासांश्च राघवो न्यवसत् सुखम्}
{तत्र संवसतस्तस्य मुनीनामाश्रमेषु वै} %3-11-26

\twolineshloka
{रमतश्चानुकूल्येन ययुः संवत्सरा दश}
{परिसृत्य च धर्मज्ञो राघवः सह सीतया} %3-11-27

\twolineshloka
{सुतीक्ष्णस्याश्रमपदं पुनरेवाजगाम ह}
{स तमाश्रममागम्य मुनिभिः परिपूजितः} %3-11-28

\twolineshloka
{तत्रापि न्यवसद् रामः किंचित् कालमरिंदमः}
{अथाश्रमस्थो विनयात् कदाचित् तं महामुनिम्} %3-11-29

\twolineshloka
{उपासीनः स काकुत्स्थः सुतीक्ष्णमिदमब्रवीत्}
{अस्मिन्नरण्ये भगवन्नगस्त्यो मुनिसत्तमः} %3-11-30

\twolineshloka
{वसतीति मया नित्यं कथाः कथयतां श्रुतम्}
{न तु जानामि तं देशं वनस्यास्य महत्तया} %3-11-31

\twolineshloka
{कुत्राश्रमपदं रम्यं महर्षेस्तस्य धीमतः}
{प्रसादार्थं भगवतः सानुजः सह सीतया} %3-11-32

\twolineshloka
{अगस्त्यमधिगच्छेयमभिवादयितुं मुनिम्}
{मनोरथो महानेष हृदि सम्परिवर्तते} %3-11-33

\twolineshloka
{यदहं तं मुनिवरं शुश्रूषेयमपि स्वयम्}
{इति रामस्य स मुनिः श्रुत्वा धर्मात्मनो वचः} %3-11-34

\twolineshloka
{सुतीक्ष्णः प्रत्युवाचेदं प्रीतो दशरथात्मजम्}
{अहमप्येतदेव त्वां वक्तुकामः सलक्ष्मणम्} %3-11-35

\twolineshloka
{अगस्त्यमभिगच्छेति सीतया सह राघव}
{दिष्ट्या त्विदानीमर्थेऽस्मिन् स्वयमेव ब्रवीषि माम्} %3-11-36

\threelineshloka
{अयमाख्यामि ते राम यत्रागस्त्यो महामुनिः}
{योजनान्याश्रमात् तात याहि चत्वारि वै ततः}
{दक्षिणेन महान् श्रीमानगस्त्यभ्रातुराश्रमः} %3-11-37

\twolineshloka
{स्थलीप्रायवनोद्देशे पिप्पलीवनशोभिते}
{बहुपुष्पफले रम्ये नानाविहगनादिते} %3-11-38

\twolineshloka
{पद्मिन्यो विविधास्तत्र प्रसन्नसलिलाशयाः}
{हंसकारण्डवाकीर्णाश्चक्रवाकोपशोभिताः} %3-11-39

\twolineshloka
{तत्रैकां रजनीं व्युष्य प्रभाते राम गम्यताम्}
{दक्षिणां दिशमास्थाय वनखण्डस्य पार्श्वतः} %3-11-40

\twolineshloka
{तत्रागस्त्याश्रमपदं गत्वा योजनमन्तरम्}
{रमणीये वनोद्देशे बहुपादपशोभिते} %3-11-41

\twolineshloka
{रंस्यते तत्र वैदेही लक्ष्मणश्च त्वया सह}
{स हि रम्यो वनोद्देशो बहुपादपसंयुतः} %3-11-42

\twolineshloka
{यदि बुद्धिः कृता द्रष्टुमगस्त्यं तं महामुनिम्}
{अद्यैव गमने बुद्धिं रोचयस्व महामते} %3-11-43

\twolineshloka
{इति रामो मुनेः श्रुत्वा सह भ्रात्राभिवाद्य च}
{प्रतस्थेऽगस्त्यमुद्दिश्य सानुगः सह सीतया} %3-11-44

\twolineshloka
{पश्यन् वनानि चित्राणि पर्वतांश्चाभ्रसंनिभान्}
{सरांसि सरितश्चैव पथि मार्गवशानुगान्} %3-11-45

\twolineshloka
{सुतीक्ष्णेनोपदिष्टेन गत्वा तेन पथा सुखम्}
{इदं परमसंहृष्टो वाक्यं लक्ष्मणमब्रवीत्} %3-11-46

\twolineshloka
{एतदेवाश्रमपदं नूनं तस्य महात्मनः}
{अगस्त्यस्य मुनेर्भ्रातुर्दृश्यते पुण्यकर्मणः} %3-11-47

\twolineshloka
{यथा हीमे वनस्यास्य ज्ञाताः पथि सहस्रशः}
{संनताः फलभारेण पुष्पभारेण च द्रुमाः} %3-11-48

\twolineshloka
{पिप्पलीनां च पक्वानां वनादस्मादुपागतः}
{गन्धोऽयं पवनोत्क्षिप्तः सहसा कटुकोदयः} %3-11-49

\twolineshloka
{तत्र तत्र च दृश्यन्ते संक्षिप्ताः काष्ठसंचयाः}
{लूनाश्च परिदृश्यन्ते दर्भा वैदूर्यवर्चसः} %3-11-50

\twolineshloka
{एतच्च वनमध्यस्थं कृष्णाभ्रशिखरोपमम्}
{पावकस्याश्रमस्थस्य धूमाग्रं सम्प्रदृश्यते} %3-11-51

\twolineshloka
{विविक्तेषु च तीर्थेषु कृतस्नाना द्विजातयः}
{पुष्पोपहारं कुर्वन्ति कुसुमैः स्वयमर्जितैः} %3-11-52

\twolineshloka
{ततः सुतीक्ष्णवचनं यथा सौम्य मया श्रुतम्}
{अगस्त्यस्याश्रमो भ्रातुर्नूनमेष भविष्यति} %3-11-53

\twolineshloka
{निगृह्य तरसा मृत्युं लोकानां हितकाम्यया}
{यस्य भ्रात्रा कृतेयं दिक्शरण्या पुण्यकर्मणा} %3-11-54

\twolineshloka
{इहैकदा किल क्रूरो वातापिरपि चेल्वलः}
{भ्रातरौ सहितावास्तां ब्राह्मणघ्नौ महासुरौ} %3-11-55

\twolineshloka
{धारयन् ब्राह्मणं रूपमिल्वलः संस्कृतं वदन्}
{आमन्त्रयति विप्रान् स श्राद्धमुद्दिश्य निर्घृणः} %3-11-56

\twolineshloka
{भ्रातरं संस्कृतं कृत्वा ततस्तं मेषरूपिणम्}
{तान् द्विजान् भोजयामास श्राद्धदृष्टेन कर्मणा} %3-11-57

\twolineshloka
{ततो भुक्तवतां तेषां विप्राणामिल्वलोऽब्रवीत्}
{वातापे निष्क्रमस्वेति स्वरेण महता वदन्} %3-11-58

\twolineshloka
{ततो भ्रातुर्वचः श्रुत्वा वातापिर्मेषवन्नदन्}
{भित्त्वा भित्त्वा शरीराणि ब्राह्मणानां विनिष्पतत्} %3-11-59

\twolineshloka
{ब्राह्मणानां सहस्राणि तैरेवं कामरूपिभिः}
{विनाशितानि संहत्य नित्यशः पिशिताशनैः} %3-11-60

\twolineshloka
{अगस्त्येन तदा देवैः प्रार्थितेन महर्षिणा}
{अनुभूय किल श्राद्धे भक्षितः स महासुरः} %3-11-61

\twolineshloka
{ततः सम्पन्नमित्युक्त्वा दत्त्वा हस्तेऽवनेजनम्}
{भ्रातरं निष्क्रमस्वेति चेल्वलः समभाषत} %3-11-62

\twolineshloka
{स तदा भाषमाणं तु भ्रातरं विप्रघातिनम्}
{अब्रवीत् प्रहसन् धीमानगस्त्यो मुनिसत्तमः} %3-11-63

\twolineshloka
{कुतो निष्क्रमितुं शक्तिर्मया जीर्णस्य रक्षसः}
{भ्रातुस्तु मेषरूपस्य गतस्य यमसादनम्} %3-11-64

\twolineshloka
{अथ तस्य वचः श्रुत्वा भ्रातुर्निधनसंश्रितम्}
{प्रधर्षयितुमारेभे मुनिं क्रोधान्निशाचरः} %3-11-65

\twolineshloka
{सोऽभ्यद्रवद् द्विजेन्द्रं तं मुनिना दीप्ततेजसा}
{चक्षुषानलकल्पेन निर्दग्धो निधनं गतः} %3-11-66

\twolineshloka
{तस्यायमाश्रमो भ्रातुस्तटाकवनशोभितः}
{विप्रानुकम्पया येन कर्मेदं दुष्करं कृतम्} %3-11-67

\twolineshloka
{एवं कथयमानस्य तस्य सौमित्रिणा सह}
{रामस्यास्तं गतः सूर्यः संध्याकालोऽभ्यवर्तत} %3-11-68

\twolineshloka
{उपास्य पश्चिमां संध्यां सह भ्रात्रा यथाविधि}
{प्रविवेशाश्रमपदं तमृषिं चाभ्यवादयत्} %3-11-69

\twolineshloka
{सम्यक्प्रतिगृहीतस्तु मुनिना तेन राघवः}
{न्यवसत् तां निशामेकां प्राश्य मूलफलानि च} %3-11-70

\twolineshloka
{तस्यां रात्र्यां व्यतीतायामुदिते रविमण्डले}
{भ्रातरं तमगस्त्यस्य आमन्त्रयत राघवः} %3-11-71

\twolineshloka
{अभिवादये त्वां भगवन् सुखमस्म्युषितो निशाम्}
{आमन्त्रये त्वां गच्छामि गुरुं ते द्रष्टुमग्रजम्} %3-11-72

\twolineshloka
{गम्यतामिति तेनोक्तो जगाम रघुनन्दनः}
{यथोद्दिष्टेन मार्गेण वनं तच्चावलोकयन्} %3-11-73

\twolineshloka
{नीवारान् पनसान् सालान् वञ्जुलांस्तिनिशांस्तथा}
{चिरिबिल्वान् मधूकांश्च बिल्वानथ च तिन्दुकान्} %3-11-74

\twolineshloka
{पुष्पितान् पुष्पिताग्राभिर्लताभिरुपशोभितान्}
{ददर्श रामः शतशस्तत्र कान्तारपादपान्} %3-11-75

\twolineshloka
{हस्तिहस्तैर्विमृदितान् वानरैरुपशोभितान्}
{मत्तैः शकुनिसङ्घैश्च शतशः प्रतिनादितान्} %3-11-76

\twolineshloka
{ततोऽब्रवीत् समीपस्थं रामो राजीवलोचनः}
{पृष्ठतोऽनुगतं वीरं लक्ष्मणं लक्ष्मिवर्धनम्} %3-11-77

\twolineshloka
{स्निग्धपत्रा यथा वृक्षा यथा क्षान्ता मृगद्विजाः}
{आश्रमो नातिदूरस्थो महर्षेर्भावितात्मनः} %3-11-78

\twolineshloka
{अगस्त्य इति विख्यातो लोके स्वेनैव कर्मणा}
{आश्रमो दृश्यते तस्य परिश्रान्तश्रमापहः} %3-11-79

\twolineshloka
{प्राज्यधूमाकुलवनश्चीरमालापरिष्कृतः}
{प्रशान्तमृगयूथश्च नानाशकुनिनादितः} %3-11-80

\twolineshloka
{निगृह्य तरसा मृत्युं लोकानां हितकाम्यया}
{दक्षिणा दिक् कृता येन शरण्या पुण्यकर्मणा} %3-11-81

\twolineshloka
{तस्येदमाश्रमपदं प्रभावाद् यस्य राक्षसैः}
{दिगियं दक्षिणा त्रासाद् दृश्यते नोपभुज्यते} %3-11-82

\twolineshloka
{यदाप्रभृति चाक्रान्ता दिगियं पुण्यकर्मणा}
{तदाप्रभृति निर्वैराः प्रशान्ता रजनीचराः} %3-11-83

\twolineshloka
{नाम्ना चेयं भगवतो दक्षिणा दिक्प्रदक्षिणा}
{प्रथिता त्रिषु लोकेषु दुर्धर्षा क्रूरकर्मभिः} %3-11-84

\twolineshloka
{मार्गं निरोद्धुं सततं भास्करस्याचलोत्तमः}
{संदेशं पालयंस्तस्य विन्ध्यशैलो न वर्धते} %3-11-85

\twolineshloka
{अयं दीर्घायुषस्तस्य लोके विश्रुतकर्मणः}
{अगस्त्यस्याश्रमः श्रीमान् विनीतमृगसेवितः} %3-11-86

\twolineshloka
{एष लोकार्चितः साधुर्हिते नित्यं रतः सताम्}
{अस्मानधिगतानेष श्रेयसा योजयिष्यति} %3-11-87

\twolineshloka
{आराधयिष्याम्यत्राहमगस्त्यं तं महामुनिम्}
{शेषं च वनवासस्य सौम्य वत्स्याम्यहं प्रभो} %3-11-88

\twolineshloka
{अत्र देवाः सगन्धर्वाः सिद्धाश्च परमर्षयः}
{अगस्त्यं नियताहाराः सततं पर्युपासते} %3-11-89

\twolineshloka
{नात्र जीवेन्मृषावादी क्रूरो वा यदि वा शठः}
{नृशंसः पापवृत्तो वा मुनिरेष तथाविधः} %3-11-90

\twolineshloka
{अत्र देवाश्च यक्षाश्च नागाश्च पतगैः सह}
{वसन्ति नियताहारा धर्ममाराधयिष्णवः} %3-11-91

\twolineshloka
{अत्र सिद्धा महात्मानो विमानैः सूर्यसंनिभैः}
{त्यक्त्वा देहान् नवैर्देहैः स्वर्याताः परमर्षयः} %3-11-92

\twolineshloka
{यक्षत्वममरत्वं च राज्यानि विविधानि च}
{अत्र देवाः प्रयच्छन्ति भूतैराराधिताः शुभैः} %3-11-93

\twolineshloka
{आगताः स्माश्रमपदं सौमित्रे प्रविशाग्रतः}
{निवेदयेह मां प्राप्तमृषये सह सीतया} %3-11-94


॥इत्यार्षे श्रीमद्रामायणे वाल्मीकीये आदिकाव्ये अरण्यकाण्डे अगस्त्याश्रमः नाम एकादशः सर्गः ॥३-११॥
