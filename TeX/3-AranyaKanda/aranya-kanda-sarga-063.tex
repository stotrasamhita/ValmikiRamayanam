\sect{त्रिषष्ठितमः सर्गः — दुःखानुचिन्तनम्}

\twolineshloka
{स राजपुत्रः प्रियया विहीनः शोकेन मोहेन च पीड्यमानः}
{विषादयन् भ्रातरमार्तरूपो भूयो विषादं प्रविवेश तीव्रम्} %3-63-1

\twolineshloka
{स लक्ष्मणं शोकवशाभिपन्नं शोके निमग्नो विपुले तु रामः}
{उवाच वाक्यं व्यसनानुरूपमुष्णं विनिःश्वस्य रुदन् सशोकम्} %3-63-2

\twolineshloka
{न मद्विधो दुष्कृतकर्मकारी मन्ये द्वितीयोऽस्ति वसुन्धरायाम्}
{शोकानुशोको हि परम्पराया मामेति भिन्दन् हृदयं मनश्च} %3-63-3

\twolineshloka
{पूर्वं मया नूनमभीप्सितानि पापानि कर्माण्यसकृत्कृतानि}
{तत्रायमद्यापतितो विपाको दुःखेन दुःखं यदहं विशामि} %3-63-4

\twolineshloka
{राज्यप्रणाशः स्वजनैर्वियोगः पितुर्विनाशो जननीवियोगः}
{सर्वाणि मे लक्ष्मण शोकवेगमापूरयन्ति प्रविचिन्तितानि} %3-63-5

\twolineshloka
{सर्वं तु दुःखं मम लक्ष्मणेदं शान्तं शरीरे वनमेत्य क्लेशम्}
{सीतावियोगात् पुनरप्युदीर्णं काष्ठैरिवाग्निः सहसोपदीप्तः} %3-63-6

\twolineshloka
{सा नूनमार्या मम राक्षसेन ह्यभ्याहृता खं समुपेत्य भीरुः}
{अपस्वरं सुस्वरविप्रलापा भयेन विक्रन्दितवत्यभीक्ष्णम्} %3-63-7

\twolineshloka
{तौ लोहितस्य प्रियदर्शनस्य सदोचितावुत्तमचन्दनस्य}
{वृत्तौ स्तनौ शोणितपङ्कदिग्धौ नूनं प्रियाया मम नाभिपातः} %3-63-8

\twolineshloka
{तच्छ्लक्ष्णसुव्यक्तमृदुप्रलापं तस्या मुखं कुञ्चितकेशभारम्}
{रक्षोवशं नूनमुपागताया न भ्राजते राहुमुखे यथेन्दुः} %3-63-9

\twolineshloka
{तां हारपाशस्य सदोचितान्तां ग्रीवां प्रियाया मम सुव्रतायाः}
{रक्षांसि नूनं परिपीतवन्ति शून्ये हि भित्त्वा रुधिराशनानि} %3-63-10

\twolineshloka
{मया विहीना विजने वने सा रक्षोभिराहृत्य विकृष्यमाणा}
{नूनं विनादं कुररीव दीना सा मुक्तवत्यायतकान्तनेत्रा} %3-63-11

\twolineshloka
{अस्मिन् मया सार्धमुदारशीला शिलातले पूर्वमुपोपविष्टा}
{कान्तस्मिता लक्ष्मण जातहासा त्वामाह सीता बहुवाक्यजातम्} %3-63-12

\twolineshloka
{गोदावरीयं सरितां वरिष्ठा प्रिया प्रियाया मम नित्यकालम्}
{अप्यत्र गच्छेदिति चिन्तयामि नैकाकिनी याति हि सा कदाचित्} %3-63-13

\twolineshloka
{पद्मानना पद्मपलाशनेत्रा पद्मानि वानेतुमभिप्रयाता}
{तदप्ययुक्तं नहि सा कदाचिन्मया विना गच्छति पङ्कजानि} %3-63-14

\twolineshloka
{कामं त्विदं पुष्पितवृक्षषण्डं नानाविधैः पक्षिगणैरुपेतम्}
{वनं प्रयाता नु तदप्ययुक्तमेकाकिनी सातिबिभेति भीरुः} %3-63-15

\twolineshloka
{आदित्य भो लोककृताकृतज्ञ लोकस्य सत्यानृतकर्मसाक्षिन्}
{मम प्रिया सा क्व गता हृता वा शंसस्व मे शोकहतस्य सर्वम्} %3-63-16

\twolineshloka
{लोकेषु सर्वेषु न नास्ति किञ्चिद् यत् ते न नित्यं विदितं भवेत् तत्}
{शंसस्व वायो कुलपालिनीं तां मृता हृता वा पथि वर्तते वा} %3-63-17

\twolineshloka
{इतीव तं शोकविधेयदेहं रामं विसंज्ञं विलपन्तमेव}
{उवाच सौमित्रिरदीनसत्त्वो न्याय्ये स्थितः कालयुतं च वाक्यम्} %3-63-18

\twolineshloka
{शोकं विसृज्याद्य धृतिं भजस्व सोत्साहता चास्तु विमार्गणेऽस्याः}
{उत्साहवन्तो हि नरा न लोके सीदन्ति कर्मस्वतिदुष्करेषु} %3-63-19

\twolineshloka
{इतीव सौमित्रिमुदग्रपौरुषं ब्रुवन्तमार्तो रघुवंशवर्धनः}
{न चिन्तयामास धृतिं विमुक्तवान् पुनश्च दुःखं महदभ्युपागमत्} %3-63-20


॥इत्यार्षे श्रीमद्रामायणे वाल्मीकीये आदिकाव्ये अरण्यकाण्डे दुःखानुचिन्तनम् नाम त्रिषष्ठितमः सर्गः ॥३-६३॥
