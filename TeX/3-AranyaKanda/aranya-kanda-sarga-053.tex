\sect{त्रिपञ्चाशः सर्गः — रावणभर्त्सनम्}

\twolineshloka
{खमुत्पतन्तं तं दृष्ट्वा मैथिली जनकात्मजा}
{दुःखिता परमोद्विग्ना भये महति वर्तिनी} %3-53-1

\twolineshloka
{रोषरोदनताम्राक्षी भीमाक्षं राक्षसाधिपम्}
{रुदती करुणं सीता ह्रियमाणा तमब्रवीत्} %3-53-2

\twolineshloka
{न व्यपत्रपसे नीच कर्मणानेन रावण}
{ज्ञात्वा विरहितां यो मां चोरयित्वा पलायसे} %3-53-3

\twolineshloka
{त्वयैव नूनं दुष्टात्मन् भीरुणा हर्तुमिच्छता}
{ममापवाहितो भर्ता मृगरूपेण मायया} %3-53-4

\twolineshloka
{यो हि मामुद्यतस्त्रातुं सोऽप्ययं विनिपातितः}
{गृध्रराजः पुराणोऽसौ श्वशुरस्य सखा मम} %3-53-5

\twolineshloka
{परमं खलु ते वीर्यं दृश्यते राक्षसाधम}
{विश्राव्य नामधेयं हि युद्धे नास्मि जिता त्वया} %3-53-6

\twolineshloka
{ईदृशं गर्हितं कर्म कथं कृत्वा न लज्जसे}
{स्त्रियाश्चाहरणं नीच रहिते च परस्य च} %3-53-7

\twolineshloka
{कथयिष्यन्ति लोकेषु पुरुषाः कर्म कुत्सितम्}
{सुनृशंसमधर्मिष्ठं तव शौटीर्यमानिनः} %3-53-8

\twolineshloka
{धिक् ते शौर्यं च सत्त्वं च यत्त्वया कथितं तदा}
{कुलाक्रोशकरं लोके धिक् ते चारित्रमीदृशम्} %3-53-9

\twolineshloka
{किं शक्यं कर्तुमेवं हि यज्जवेनैव धावसि}
{मुहूर्तमपि तिष्ठ त्वं न जीवन् प्रतियास्यसि} %3-53-10

\twolineshloka
{नहि चक्षुःपथं प्राप्य तयोः पार्थिवपुत्रयोः}
{ससैन्योऽपि समर्थस्त्वं मुहूर्तमपि जीवितुम्} %3-53-11

\twolineshloka
{न त्वं तयोः शरस्पर्शं सोढुं शक्तः कथञ्चन}
{वने प्रज्वलितस्येव स्पर्शमग्नेर्विहङ्गमः} %3-53-12

\twolineshloka
{साधु कृत्वाऽऽत्मनः पथ्यं साधु मां मुञ्च रावण}
{मत्प्रधर्षणसङ्क्रुद्धो भ्रात्रा सह पतिर्मम} %3-53-13

\twolineshloka
{विधास्यति विनाशाय त्वं मां यदि न मुञ्चसि}
{येन त्वं व्यवसायेन बलान्मां हर्तुमिच्छसि} %3-53-14

\twolineshloka
{व्यवसायस्तु ते नीच भविष्यति निरर्थकः}
{नह्यहं तमपश्यन्ती भर्तारं विबुधोपमम्} %3-53-15

\twolineshloka
{उत्सहे शत्रुवशगा प्राणान् धारयितुं चिरम्}
{न नूनं चात्मनः श्रेयः पथ्यं वा समवेक्षसे} %3-53-16

\twolineshloka
{मृत्युकाले यथा मर्त्यो विपरीतानि सेवते}
{मुमूर्षूणां तु सर्वेषां यत् पथ्यं तन्न रोचते} %3-53-17

\twolineshloka
{पश्यामीह हि कण्ठे त्वां कालपाशावपाशितम्}
{यथा चास्मिन् भयस्थाने न बिभेषि निशाचर} %3-53-18

\twolineshloka
{व्यक्तं हिरण्मयांस्त्वं हि सम्पश्यसि महीरुहान्}
{नदीं वैतरणीं घोरां रुधिरौघविवाहिनीम्} %3-53-19

\twolineshloka
{खड्गपत्रवनं चैव भीमं पश्यसि रावण}
{तप्तकाञ्चनपुष्पां च वैदूर्यप्रवरच्छदाम्} %3-53-20

\twolineshloka
{द्रक्ष्यसे शाल्मलीं तीक्ष्णामायसैः कण्टकैश्चिताम्}
{नहि त्वमीदृशं कृत्वा तस्यालीकं महात्मनः} %3-53-21

\twolineshloka
{धारितुं शक्ष्यसि चिरं विषं पीत्वेव निर्घृण}
{बद्धस्त्वं कालपाशेन दुर्निवारेण रावण} %3-53-22

\twolineshloka
{क्व गतो लप्स्यसे शर्म मम भर्तुर्महात्मनः}
{निमेषान्तरमात्रेण विना भ्रातरमाहवे} %3-53-23

\twolineshloka
{राक्षसा निहता येन सहस्राणि चतुर्दश}
{कथं स राघवो वीरः सर्वास्त्रकुशलो बली} %3-53-24

\threelineshloka
{न त्वां हन्याच्छरैस्तीक्ष्णैरिष्टभार्यापहारिणम्}
{एतच्चान्यच्च परुषं वैदेही रावणाङ्कगा}
{भयशोकसमाविष्टा करुणं विललाप ह} %3-53-25

\twolineshloka
{तदा भृशार्तां बहु चैव भाषिणीं विलापपूर्वं करुणं च भामिनीम्}
{जहार पापस्तरुणीं विचेष्टतीं नृपात्मजामागतगात्रवेपथुः} %3-53-26


॥इत्यार्षे श्रीमद्रामायणे वाल्मीकीये आदिकाव्ये अरण्यकाण्डे रावणभर्त्सनम् नाम त्रिपञ्चाशः सर्गः ॥३-५३॥
