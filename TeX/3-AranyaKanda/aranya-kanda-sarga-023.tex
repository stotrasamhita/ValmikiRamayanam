\sect{त्रयोविंशः सर्गः — उत्पातदर्शनम्}

\twolineshloka
{तत्प्रयातं बलं घोरमशिवं शोणितोदकम्}
{अभ्यवर्षन्महाघोरस्तुमुलो गर्दभारुणः} %3-23-1

\twolineshloka
{निपेतुस्तुरगास्तस्य रथयुक्ता महाजवाः}
{समे पुष्पचिते देशे राजमार्गे यदृच्छया} %3-23-2

\twolineshloka
{श्यामं रुधिरपर्यन्तं बभूव परिवेषणम्}
{अलातचक्रप्रतिमं प्रतिगृह्य दिवाकरम्} %3-23-3

\twolineshloka
{ततो ध्वजमुपागम्य हेमदण्डं समुच्छ्रितम्}
{समाक्रम्य महाकायस्तस्थौ गृध्रः सुदारुणः} %3-23-4

\twolineshloka
{जनस्थानसमीपे च समाक्रम्य खरस्वनाः}
{विस्वरान् विविधान् नादान् मांसादा मृगपक्षिणः} %3-23-5

\twolineshloka
{व्याजह्रुरभिदीप्तायां दिशि वै भैरवस्वनम्}
{अशिवं यातुधानानां शिवा घोरा महास्वनाः} %3-23-6

\twolineshloka
{प्रभिन्नगजसङ्काशास्तोयशोणितधारिणः}
{आकाशं तदनाकाशं चक्रुर्भीमाम्बुवाहकाः} %3-23-7

\twolineshloka
{बभूव तिमिरं घोरमुद्धतं रोमहर्षणम्}
{दिशो वा प्रदिशो वापि सुव्यक्तं न चकाशिरे} %3-23-8

\twolineshloka
{क्षतजार्द्रसवर्णाभा सन्ध्या कालं विना बभौ}
{खरं चाभिमुखं नेदुस्तदा घोरा मृगाः खगाः} %3-23-9

\twolineshloka
{कङ्कगोमायुगृध्राश्च चुक्रुशुर्भयशंसिनः}
{नित्याशिवकरा युद्धे शिवा घोरनिदर्शनाः} %3-23-10

\twolineshloka
{नेदुर्बलस्याभिमुखं ज्वालोद्गारिभिराननैः}
{कबन्धः परिघाभासो दृश्यते भास्करान्तिके} %3-23-11

\twolineshloka
{जग्राह सूर्यं स्वर्भानुरपर्वणि महाग्रहः}
{प्रवाति मारुतः शीघ्रं निष्प्रभोऽभूद् दिवाकरः} %3-23-12

\twolineshloka
{उत्पेतुश्च विना रात्रिं ताराः खद्योतसप्रभाः}
{संलीनमीनविहगा नलिन्यः शुष्कपङ्कजाः} %3-23-13

\twolineshloka
{तस्मिन् क्षणे बभूवुश्च विना पुष्पफलैर्द्रुमाः}
{उद्धूतश्च विना वातं रेणुर्जलधरारुणः} %3-23-14

\twolineshloka
{चीचीकूचीति वाश्यन्त्यो बभूवुस्तत्र सारिकाः}
{उल्काश्चापि सनिर्घोषा निपेतुर्घोरदर्शनाः} %3-23-15

\twolineshloka
{प्रचचाल मही चापि सशैलवनकानना}
{खरस्य च रथस्थस्य नर्दमानस्य धीमतः} %3-23-16

\twolineshloka
{प्राकम्पत भुजः सव्यः स्वरश्चास्यावसज्जत}
{सास्रा सम्पद्यते दृष्टिः पश्यमानस्य सर्वतः} %3-23-17

\twolineshloka
{ललाटे च रुजो जाता न च मोहान्न्यवर्तत}
{तान् समीक्ष्य महोत्पातानुत्थितान् रोमहर्षणान्} %3-23-18

\twolineshloka
{अब्रवीद् राक्षसान् सर्वान् प्रहसन् स खरस्तदा}
{महोत्पातानिमान् सर्वानुत्थितान् घोरदर्शनान्} %3-23-19

\twolineshloka
{न चिन्तयाम्यहं वीर्याद् बलवान् दुर्बलानिव}
{तारा अपि शरैस्तीक्ष्णैः पातयेयं नभस्तलात्} %3-23-20

\twolineshloka
{मृत्युं मरणधर्मेण सङ्क्रुद्धो योजयाम्यहम्}
{राघवं तं बलोत्सिक्तं भ्रातरं चापि लक्ष्मणम्} %3-23-21

\twolineshloka
{अहत्वा सायकैस्तीक्ष्णैर्नोपावर्तितुमुत्सहे}
{यन्निमित्तं तु रामस्य लक्ष्मणस्य विपर्ययः} %3-23-22

\twolineshloka
{सकामा भगिनीमेऽस्तु पीत्वा तु रुधिरं तयोः}
{न क्वचित् प्राप्तपूर्वो मे संयुगेषु पराजयः} %3-23-23

\twolineshloka
{युष्माकमेतत् प्रत्यक्षं नानृतं कथयाम्यहम्}
{देवराजमपि क्रुद्धो मत्तैरावतगामिनम्} %3-23-24

\twolineshloka
{वज्रहस्तं रणे हन्यां किं पुनस्तौ च मानवौ}
{सा तस्य गर्जितं श्रुत्वा राक्षसानां महाचमूः} %3-23-25

\twolineshloka
{प्रहर्षमतुलं लेभे मृत्युपाशावपाशिता}
{समेयुश्च महात्मानो युद्धदर्शनकाङ्क्षिणः} %3-23-26

\twolineshloka
{ऋषयो देवगन्धर्वाः सिद्धाश्च सह चारणैः}
{समेत्य चोचुः सहितास्तेऽन्योन्यं पुण्यकर्मणः} %3-23-27

\twolineshloka
{स्वस्ति गोब्राह्मणेभ्यस्तु लोकानां ये च सम्मताः}
{जयतां राघवो युद्धे पौलस्त्यान् रजनीचरान्} %3-23-28

\twolineshloka
{चक्रहस्तो यथा विष्णुः सर्वानसुरसत्तमान्}
{एतच्चान्यच्च बहुशो ब्रुवाणाः परमर्षयः} %3-23-29

\twolineshloka
{जातकौतूहलास्तत्र विमानस्थाश्च देवताः}
{ददृशुर्वाहिनीं तेषां राक्षसानां गतायुषाम्} %3-23-30

\twolineshloka
{रथेन तु खरो वेगात् सैन्यस्याग्राद् विनिःसृतः}
{श्येनगामी पृथुग्रीवो यज्ञशत्रुर्विहङ्गमः} %3-23-31

\twolineshloka
{दुर्जयः करवीराक्षः परुषः कालकार्मुकः}
{हेममाली महामाली सर्पास्यो रुधिराशनः} %3-23-32

\threelineshloka
{द्वादशैते महावीर्याः प्रतस्थुरभितः खरम्}
{महाकपालः स्थूलाक्षः प्रमाथस्त्रिशिरास्तथा}
{चत्वार एते सेनाग्रे दूषणं पृष्ठतोऽन्वयुः} %3-23-33

\twolineshloka
{सा भीमवेगा समराभिकाङ्क्षिणी सुदारुणा राक्षसवीरसेना}
{तौ राजपुत्रौ सहसाभ्युपेता माला ग्रहाणामिव चन्द्रसूर्यौ} %3-23-34


॥इत्यार्षे श्रीमद्रामायणे वाल्मीकीये आदिकाव्ये अरण्यकाण्डे उत्पातदर्शनम् नाम त्रयोविंशः सर्गः ॥३-२३॥
