\sect{चत्वारिंशः सर्गः — मायामृगरूपपरिग्रहनिर्बन्धः}

\twolineshloka
{मारीचस्य तु तद् वाक्यं क्षमं युक्तं च रावणः}
{उक्तो न प्रतिजग्राह मर्तुकाम इवौषधम्} %3-40-1

\twolineshloka
{तं पथ्यहितवक्तारं मारीचं राक्षसाधिपः}
{अब्रवीत् परुषं वाक्यमयुक्तं कालचोदितः} %3-40-2

\twolineshloka
{दुष्कुलैतदयुक्तार्थं मारीच मयि कथ्यते}
{वाक्यं निष्फलमत्यर्थं बीजमुप्तमिवोषरे} %3-40-3

\twolineshloka
{त्वद्वाक्यैर्न तु मां शक्यं भेत्तुं रामस्य संयुगे}
{मूर्खस्य पापशीलस्य मानुषस्य विशेषतः} %3-40-4

\twolineshloka
{यस्त्यक्त्वा सुहृदो राज्यं मातरं पितरं तथा}
{स्त्रीवाक्यं प्राकृतं श्रुत्वा वनमेकपदे गतः} %3-40-5

\twolineshloka
{अवश्यं तु मया तस्य संयुगे खरघातिनः}
{प्राणैः प्रियतरा सीता हर्तव्या तव सन्निधौ} %3-40-6

\twolineshloka
{एवं मे निश्चिता बुद्धिर्हृदि मारीच विद्यते}
{न व्यावर्तयितुं शक्या सेन्द्रैरपि सुरासुरैः} %3-40-7

\twolineshloka
{दोषं गुणं वा सम्पृष्टस्त्वमेवं वक्तुमर्हसि}
{अपायं वा उपायं वा कार्यस्यास्य विनिश्चये} %3-40-8

\twolineshloka
{सम्पृष्टेन तु वक्तव्यं सचिवेन विपश्चिता}
{उद्यताञ्जलिना राज्ञो य इच्छेद् भूतिमात्मनः} %3-40-9

\twolineshloka
{वाक्यमप्रतिकूलं तु मृदुपूर्वं शुभं हितम्}
{उपचारेण वक्तव्यो युक्तं च वसुधाधिपः} %3-40-10

\twolineshloka
{सावमर्दं तु यद्वाक्यमथवा हितमुच्यते}
{नाभिनन्देत तद् राजा मानार्थी मानवर्जितम्} %3-40-11

\twolineshloka
{पञ्च रूपाणि राजानो धारयन्त्यमितौजसः}
{अग्नेरिन्द्रस्य सोमस्य यमस्य वरुणस्य च} %3-40-12

\twolineshloka
{औष्ण्यं तथा विक्रमं च सौम्यं दण्डं प्रसन्नताम्}
{धारयन्ति महात्मानो राजानः क्षणदाचर} %3-40-13

\twolineshloka
{तस्मात् सर्वास्ववस्थासु मान्याः पूज्याश्च नित्यदा}
{त्वं तु धर्ममविज्ञाय केवलं मोहमाश्रितः} %3-40-14

\twolineshloka
{अभ्यागतं तु दौरात्म्यात् परुषं वदसीदृशम्}
{गुणदोषौ न पृच्छामि क्षेमं चात्मनि राक्षस} %3-40-15

\twolineshloka
{मयोक्तमपि चैतावत् त्वां प्रत्यमितविक्रम}
{अस्मिंस्तु स भवान् कृत्ये साहाय्यं कर्तुमर्हसि} %3-40-16

\twolineshloka
{शृणु तत्कर्म साहाय्ये यत्कार्यं वचनान्मम}
{सौवर्णस्त्वं मृगो भूत्वा चित्रो रजतबिन्दुभिः} %3-40-17

\twolineshloka
{आश्रमे तस्य रामस्य सीतायाः प्रमुखे चर}
{प्रलोभयित्वा वैदेहीं यथेष्टं गन्तुमर्हसि} %3-40-18

\twolineshloka
{त्वां हि मायामयं दृष्ट्वा काञ्चनं जातविस्मया}
{आनयैनमिति क्षिप्रं रामं वक्ष्यति मैथिली} %3-40-19

\twolineshloka
{अपक्रान्ते च काकुत्स्थे दूरं गत्वाप्युदाहर}
{हा सीते लक्ष्मणेत्येवं रामवाक्यानुरूपकम्} %3-40-20

\twolineshloka
{तच्छ्रुत्वा रामपदवीं सीतया च प्रचोदितः}
{अनुगच्छति सम्भ्रान्तः सौमित्रिरपि सौहृदात्} %3-40-21

\twolineshloka
{अपक्रान्ते च काकुत्स्थे लक्ष्मणे च यथासुखम्}
{आहरिष्यामि वैदेहीं सहस्राक्षः शचीमिव} %3-40-22

\twolineshloka
{एवं कृत्वा त्विदं कार्यं यथेष्टं गच्छ राक्षस}
{राज्यस्यार्धं प्रदास्यामि मारीच तव सुव्रत} %3-40-23

\twolineshloka
{गच्छ सौम्य शिवं मार्गं कार्यस्यास्य विवृद्धये}
{अहं त्वानुगमिष्यामि सरथो दण्डकावनम्} %3-40-24

\twolineshloka
{प्राप्य सीतामयुद्धेन वञ्चयित्वा तु राघवम्}
{लङ्कां प्रति गमिष्यामि कृतकार्यः सह त्वया} %3-40-25

\threelineshloka
{नो चेत् करोषि मारीच हन्मि त्वामहमद्य वै}
{एतत् कार्यमवश्यं मे बलादपि करिष्यसि}
{राज्ञो विप्रतिकूलस्थो न जातु सुखमेधते} %3-40-26

\twolineshloka
{आसाद्य तं जीवितसंशयस्ते मृत्युर्ध्रुवो ह्यद्य मया विरुध्यतः}
{एतद् यथावत् परिगण्य बुद्ध्या यदत्र पथ्यं कुरु तत्तथा त्वम्} %3-40-27


॥इत्यार्षे श्रीमद्रामायणे वाल्मीकीये आदिकाव्ये अरण्यकाण्डे मायामृगरूपपरिग्रहनिर्बन्धः नाम चत्वारिंशः सर्गः ॥३-४०॥
