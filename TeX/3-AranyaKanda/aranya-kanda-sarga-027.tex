\sect{सप्तविंशः सर्गः — त्रिशिरोवधः}

\twolineshloka
{खरं तु रामाभिमुखं प्रयान्तं वाहिनीपतिः}
{राक्षसस्त्रिशिरा नाम संनिपत्येदमब्रवीत्} %3-27-1

\twolineshloka
{मां नियोजय विक्रान्तं त्वं निवर्तस्व साहसात्}
{पश्य रामं महाबाहुं संयुगे विनिपातितम्} %3-27-2

\twolineshloka
{प्रतिजानामि ते सत्यमायुधं चाहमालभे}
{यथा रामं वधिष्यामि वधार्हं सर्वरक्षसाम्} %3-27-3

\twolineshloka
{अहं वास्य रणे मृत्युरेष वा समरे मम}
{विनिवर्त्य रणोत्साहं मुहूर्तं प्राश्निको भव} %3-27-4

\twolineshloka
{प्रहृष्टो वा हते रामे जनस्थानं प्रयास्यसि}
{मयि वा निहते रामं संयुगाय प्रयास्यसि} %3-27-5

\twolineshloka
{खरस्त्रिशिरसा तेन मृत्युलोभात् प्रसादितः}
{गच्छ युध्येत्यनुज्ञातो राघवाभिमुखो ययौ} %3-27-6

\twolineshloka
{त्रिशिरास्तु रथेनैव वाजियुक्तेन भास्वता}
{अभ्यद्रवद् रणे रामं त्रिशृङ्ग इव पर्वतः} %3-27-7

\twolineshloka
{शरधारासमूहान् स महामेघ इवोत्सृजन्}
{व्यसृजत् सदृशं नादं जलार्द्रस्येव दुन्दुभेः} %3-27-8

\twolineshloka
{आगच्छन्तं त्रिशिरसं राक्षसं प्रेक्ष्य राघवः}
{धनुषा प्रतिजग्राह विधुन्वन् सायकान् शितान्} %3-27-9

\twolineshloka
{स सम्प्रहारस्तुमुलो रामत्रिशिरसोस्तदा}
{सम्बभूवातिबलिनोः सिंहकुञ्जरयोरिव} %3-27-10

\twolineshloka
{ततस्त्रिशिरसा बाणैर्ललाटे ताडितस्त्रिभिः}
{अमर्षी कुपितो रामः संरब्ध इदमब्रवीत्} %3-27-11

\twolineshloka
{अहो विक्रमशूरस्य राक्षसस्येदृशं बलम्}
{पुष्पैरिव शरैर्योऽहं ललाटेऽस्मि परिक्षतः} %3-27-12

\twolineshloka
{ममापि प्रतिगृह्णीष्व शरांश्चापगुणाच्च्युतान्}
{एवमुक्त्वा सुसंरब्धः शरानाशीविषोपमान्} %3-27-13

\twolineshloka
{त्रिशिरोवक्षसि क्रुद्धो निजघान चतुर्दश}
{चतुर्भिस्तुरगानस्य शरैः संनतपर्वभिः} %3-27-14

\twolineshloka
{न्यपातयत तेजस्वी चतुरस्तस्य वाजिनः}
{अष्टभिः सायकैः सूतं रथोपस्थे न्यपातयत्} %3-27-15

\twolineshloka
{रामश्चिच्छेद बाणेन ध्वजं चास्य समुच्छ्रितम्}
{ततो हतरथात् तस्मादुत्पतन्तं निशाचरम्} %3-27-16

\twolineshloka
{चिच्छेद रामस्तं बाणैर्हृदये सोऽभवज्जडः}
{सायकैश्चाप्रमेयात्मा सामर्षस्तस्य रक्षसः} %3-27-17

\twolineshloka
{शिरांस्यपातयत् त्रीणि वेगवद्भिस्त्रिभिः शरैः}
{स धूमशोणितोद्गारी रामबाणाभिपीडितः} %3-27-18

\twolineshloka
{न्यपतत् पतितैः पूर्वं समरस्थो निशाचरः}
{हतशेषास्ततो भग्ना राक्षसाः खरसंश्रयाः} %3-27-19

\threelineshloka
{द्रवन्ति स्म न तिष्ठन्ति व्याघ्रत्रस्ता मृगा इव}
{तान् खरो द्रवतो दृष्ट्वा निवर्त्य रुषितस्त्वरन्}
{राममेवाभिदुद्राव राहुश्चन्द्रमसं यथा} %3-27-20


॥इत्यार्षे श्रीमद्रामायणे वाल्मीकीये आदिकाव्ये अरण्यकाण्डे त्रिशिरोवधः नाम सप्तविंशः सर्गः ॥३-२७॥
