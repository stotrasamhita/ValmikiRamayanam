\sect{पञ्चसप्ततितमः सर्गः — पम्पादर्शनम्}

\twolineshloka
{दिवं तु तस्यां यातायां शबर्यां स्वेन तेजसा}
{लक्ष्मणेन सह भ्रात्रा चिन्तयामास राघवः} %3-75-1

\twolineshloka
{चिन्तयित्वा तु धर्मात्मा प्रभावं तं महात्मनाम्}
{हितकारिणमेकाग्रं लक्ष्मणं राघवोऽब्रवीत्} %3-75-2

\twolineshloka
{दृष्टो मयाऽऽश्रमः सौम्य बह्वाश्चर्यः कृतात्मनाम्}
{विश्वस्तमृगशार्दूलो नानाविहगसेवितः} %3-75-3

\twolineshloka
{सप्तानां च समुद्राणां तेषां तीर्थेषु लक्ष्मण}
{उपस्पृष्टं च विधिवत् पितरश्चापि तर्पिताः} %3-75-4

\twolineshloka
{प्रणष्टमशुभं यन्नः कल्याणं समुपस्थितम्}
{तेन त्वेतत् प्रहृष्टं मे मनो लक्ष्मण सम्प्रति} %3-75-5

\twolineshloka
{हृदये मे नरव्याघ्र शुभमाविर्भविष्यति}
{तदागच्छ गमिष्यावः पम्पां तां प्रियदर्शनाम्} %3-75-6

\twolineshloka
{ऋष्यमूको गिरिर्यत्र नातिदूरे प्रकाशते}
{यस्मिन् वसति धर्मात्मा सुग्रीवोंऽशुमतः सुतः} %3-75-7

\twolineshloka
{नित्यं वालिभयात् त्रस्तश्चतुर्भिः सह वानरैः}
{अहं त्वरे च तं द्रष्टुं सुग्रीवं वानरर्षभम्} %3-75-8

\twolineshloka
{तदधीनं हि मे कार्यं सीतायाः परिमार्गणम्}
{इति ब्रुवाणं तं वीरं सौमित्रिरिदमब्रवीत्} %3-75-9

\twolineshloka
{गच्छावस्त्वरितं तत्र ममापि त्वरते मनः}
{आश्रमात्तु ततस्तस्मान्निष्क्रम्य स विशाम्पतिः} %3-75-10

\twolineshloka
{आजगाम ततः पम्पां लक्ष्मणेन सह प्रभुः}
{समीक्षमाणः पुष्पाढ्यं सर्वतो विपुलद्रुमम्} %3-75-11

\twolineshloka
{कोयष्टिभिश्चार्जुनकैः शतपत्रैश्च कीरकैः}
{एतैश्चान्यैश्च बहुभिर्नादितं तद् वनं महत्} %3-75-12

\twolineshloka
{स रामो विविधान् वृक्षान् सरांसि विविधानि च}
{पश्यन् कामाभिसंतप्तो जगाम परमं ह्रदम्} %3-75-13

\twolineshloka
{स तामासाद्य वै रामो दूरात् पानीयवाहिनीम्}
{मतङ्गसरसं नाम ह्रदं समवगाहत} %3-75-14

\twolineshloka
{तत्र जग्मतुरव्यग्रौ राघवौ हि समाहितौ}
{स तु शोकसमाविष्टो रामो दशरथात्मजः} %3-75-15

\twolineshloka
{विवेश नलिनीं रम्यां पंकजैश्च समावृताम्}
{तिलकाशोकपुंनागबकुलोद्दालकाशिनीम्} %3-75-16

\twolineshloka
{रम्योपवनसम्बाधां पद्मसम्पीडितोदकाम्}
{स्फटिकोपमतोयां तां श्लक्ष्णवालुकसंतताम्} %3-75-17

\twolineshloka
{मत्स्यकच्छपसम्बाधां तीरस्थद्रुमशोभिताम्}
{सखीभिरिव संयुक्तां लताभिरनुवेष्टिताम्} %3-75-18

\twolineshloka
{किंनरोरगगन्धर्वयक्षराक्षससेविताम्}
{नानाद्रुमलताकीर्णां शीतवारिनिधिं शुभाम्} %3-75-19

\twolineshloka
{पद्मसौगन्धिकैस्ताम्रां शुक्लां कुमुदमण्डलैः}
{नीलां कुवलयोद्घाटैर्बहुवर्णां कुथामिव} %3-75-20

\twolineshloka
{अरविन्दोत्पलवतीं पद्मसौगन्धिकायुताम्}
{पुष्पिताम्रवणोपेतां बर्हिणोद्घुष्टनादिताम्} %3-75-21

\twolineshloka
{स तां दृष्ट्वा ततः पम्पां रामः सौमित्रिणा सह}
{विललाप च तेजस्वी रामो दशरथात्मजः} %3-75-22

\twolineshloka
{तिलकैर्बीजपूरैश्च वटैः शुक्लद्रुमैस्तथा}
{पुष्पितैः करवीरैश्च पुंनागैश्च सुपुष्पितैः} %3-75-23

\twolineshloka
{मालतीकुन्दगुल्मैश्च भण्डीरैर्निचुलैस्तथा}
{अशोकैः सप्तपर्णैश्च कतकैरतिमुक्तकैः} %3-75-24

\twolineshloka
{अन्यैश्च विविधैर्वृक्षैः प्रमदामिव शोभिताम्}
{अस्यास्तीरे तु पूर्वोक्तः पर्वतो धातुमण्डितः} %3-75-25

\twolineshloka
{ऋष्यमूक इति ख्यातश्चित्रपुष्पितपादपः}
{हरिर्ऋक्षरजोनाम्नः पुत्रस्तस्य महात्मनः} %3-75-26

\twolineshloka
{अध्यास्ते तु महावीर्यः सुग्रीव इति विश्रुतः}
{सुग्रीवमभिगच्छ त्वं वानरेन्द्रं नरर्षभ} %3-75-27

\twolineshloka
{इत्युवाच पुनर्वाक्यं लक्ष्मणं सत्यविक्रमः}
{कथं मया विना सीतां शक्यं लक्ष्मण जीवितुम्} %3-75-28

\twolineshloka
{इत्येवमुक्त्वा मदनाभिपीडितः स लक्ष्मणं वाक्यमनन्यचेतनः}
{विवेश पम्पां नलिनीमनोरमां तमुत्तमं शोकमुदीरयाणः} %3-75-29

\twolineshloka
{क्रमेण गत्वा प्रविलोकयन् वनं ददर्श पम्पां शुभदर्शकाननाम्}
{अनेकनानाविधपक्षिसंकुलां विवेश रामः सह लक्ष्मणेन} %3-75-30


॥इत्यार्षे श्रीमद्रामायणे वाल्मीकीये आदिकाव्ये अरण्यकाण्डे पम्पादर्शनम् नाम पञ्चसप्ततितमः सर्गः ॥३-७५॥
