\sect{पञ्चविंशः सर्गः — खरसैन्यावमर्दः}

\twolineshloka
{अवष्टब्धधनुं रामं क्रुद्धं तं रिपुघातिनम्}
{ददर्शाश्रममागम्य खरः सह पुरःसरैः} %3-25-1

\twolineshloka
{तं दृष्ट्वा सगुणं चापमुद्यम्य खरनिःस्वनम्}
{रामस्याभिमुखं सूतं चोद्यतामित्यचोदयत्} %3-25-2

\twolineshloka
{स खरस्याज्ञया सूतस्तुरगान् समचोदयत्}
{यत्र रामो महाबाहुरेको धुन्वन् धनुः स्थितः} %3-25-3

\twolineshloka
{तं तु निष्पतितं दृष्ट्वा सर्वतो रजनीचराः}
{मुञ्चमाना महानादं सचिवाः पर्यवारयन्} %3-25-4

\twolineshloka
{स तेषां यातुधानानां मध्ये रथगतः खरः}
{बभूव मध्ये ताराणां लोहिताङ्ग इवोदितः} %3-25-5

\twolineshloka
{ततः शरसहस्रेण राममप्रतिमौजसम्}
{अर्दयित्वा महानादं ननाद समरे खरः} %3-25-6

\twolineshloka
{ततस्तं भीमधन्वानं क्रुद्धाः सर्वे निशाचराः}
{रामं नानाविधैः शस्त्रैरभ्यवर्षन्त दुर्जयम्} %3-25-7

\twolineshloka
{मुद्गरैरायसैः शूलैः प्रासैः खड्गैः परश्वधैः}
{राक्षसाः समरे शूरं निजघ्नू रोषतत्पराः} %3-25-8

\twolineshloka
{ते बलाहकसङ्काशा महाकाया महाबलाः}
{अभ्यधावन्त काकुत्स्थं रथैर्वाजिभिरेव च} %3-25-9

\twolineshloka
{गजैः पर्वतकूटाभै रामं युद्धे जिघांसवः}
{ते रामे शरवर्षाणि व्यसृजन् रक्षसां गणाः} %3-25-10

\twolineshloka
{शैलेन्द्रमिव धाराभिर्वर्षमाणा महाघनाः}
{सर्वैः परिवृतो रामो राक्षसैः क्रूरदर्शनैः} %3-25-11

\twolineshloka
{तिथिष्विव महादेवो वृतः पारिषदां गणैः}
{तानि मुक्तानि शस्त्राणि यातुधानैः स राघवः} %3-25-12

\twolineshloka
{प्रतिजग्राह विशिखैर्नद्योघानिव सागरः}
{स तैः प्रहरणैर्घोरैर्भिन्नगात्रो न विव्यथे} %3-25-13

\twolineshloka
{रामः प्रदीप्तैर्बहुभिर्वज्रैरिव महाचलः}
{स विद्धः क्षतजादिग्धः सर्वगात्रेषु राघवः} %3-25-14

\twolineshloka
{बभूव रामः सन्ध्याभ्रैर्दिवाकर इवावृतः}
{विषेदुर्देवगन्धर्वाः सिद्धाश्च परमर्षयः} %3-25-15

\twolineshloka
{एकं सहस्रैर्बहुभिस्तदा दृष्ट्वा समावृतम्}
{ततो रामस्तु सङ्क्रुद्धो मण्डलीकृतकार्मुकः} %3-25-16

\twolineshloka
{ससर्ज निशितान् बाणान् शतशोऽथ सहस्रशः}
{दुरावारान् दुर्विषहान् कालपाशोपमान् रणे} %3-25-17

\twolineshloka
{मुमोच लीलया कङ्कपत्रान् काञ्चनभूषणान्}
{ते शराः शत्रुसैन्येषु मुक्ता रामेण लीलया} %3-25-18

\twolineshloka
{आददू रक्षसां प्राणान् पाशाः कालकृता इव}
{भित्त्वा राक्षसदेहांस्तांस्ते शरा रुधिराप्लुताः} %3-25-19

\twolineshloka
{अन्तरिक्षगता रेजुर्दीप्ताग्निसमतेजसः}
{असङ्ख्येयास्तु रामस्य सायकाश्चापमण्डलात्} %3-25-20

\twolineshloka
{विनिष्पेतुरतीवोग्रा रक्षःप्राणापहारिणः}
{तैर्धनूंषि ध्वजाग्राणि चर्माणि कवचानि च} %3-25-21

\twolineshloka
{बाहून् सहस्ताभरणानूरून् करिकरोपमान्}
{चिच्छेद रामः समरे शतशोऽथ सहस्रशः} %3-25-22

\twolineshloka
{हयान् काञ्चनसन्नाहान् रथयुक्तान् ससारथीन्}
{गजांश्च सगजारोहान् सहयान् सादिनस्तदा} %3-25-23

\twolineshloka
{चिच्छिदुर्बिभिदुश्चैव रामबाणा गुणच्युताः}
{पदातीन् समरे हत्वा ह्यनयद् यमसादनम्} %3-25-24

\twolineshloka
{ततो नालीकनाराचैस्तीक्ष्णाग्रैश्च विकर्णिभिः}
{भीममार्तस्वरं चक्रुश्छिद्यमाना निशाचराः} %3-25-25

\twolineshloka
{तत्सैन्यं विविधैर्बाणैरर्दितं मर्मभेदिभिः}
{न रामेण सुखं लेभे शुष्कं वनमिवाग्निना} %3-25-26

\twolineshloka
{केचिद् भीमबलाः शूराः प्रासान् शूलान् परश्वधान्}
{चिक्षिपुः परमक्रुद्धा रामाय रजनीचराः} %3-25-27

\twolineshloka
{तेषां बाणैर्महाबाहुः शस्त्राण्यावार्य वीर्यवान्}
{जहार समरे प्राणांश्चिच्छेद च शिरोधरान्} %3-25-28

\twolineshloka
{ते छिन्नशिरसः पेतुश्छिन्नचर्मशरासनाः}
{सुपर्णवातविक्षिप्ता जगत्यां पादपा यथा} %3-25-29

\twolineshloka
{अवशिष्टाश्च ये तत्र विषण्णास्ते निशाचराः}
{खरमेवाभ्यधावन्त शरणार्थं शराहताः} %3-25-30

\twolineshloka
{तान् सर्वान् धनुरादाय समाश्वास्य च दूषणः}
{अभ्यधावत् सुसङ्क्रुद्धः क्रुद्धं क्रुद्ध इवान्तकः} %3-25-31

\twolineshloka
{निवृत्तास्तु पुनः सर्वे दूषणाश्रयनिर्भयाः}
{राममेवाभ्यधावन्त सालतालशिलायुधाः} %3-25-32

\twolineshloka
{शूलमुद्गरहस्ताश्च पाशहस्ता महाबलाः}
{सृजन्तः शरवर्षाणि शस्त्रवर्षाणि संयुगे} %3-25-33

\twolineshloka
{द्रुमवर्षाणि मुञ्चन्तः शिलावर्षाणि राक्षसाः}
{तद् बभूवाद्भुतं युद्धं तुमुलं रोमहर्षणम्} %3-25-34

\twolineshloka
{रामस्यास्य महाघोरं पुनस्तेषां च रक्षसाम्}
{ते समन्तादभिक्रुद्धा राघवं पुनरार्दयन्} %3-25-35

\twolineshloka
{ततः सर्वा दिशो दृष्ट्वा प्रदिशश्च समावृताः}
{राक्षसैः सर्वतः प्राप्तैः शरवर्षाभिरावृतः} %3-25-36

\twolineshloka
{स कृत्वा भैरवं नादमस्त्रं परमभास्वरम्}
{समयोजयद् गान्धर्वं राक्षसेषु महाबलः} %3-25-37

\twolineshloka
{ततः शरसहस्राणि निर्ययुश्चापमण्डलात्}
{सर्वा दश दिशो बाणैरापूर्यन्त समागतैः} %3-25-38

\twolineshloka
{नाददानं शरान् घोरान् विमुञ्चन्तं शरोत्तमान्}
{विकर्षमाणं पश्यन्ति राक्षसास्ते शरार्दिताः} %3-25-39

\twolineshloka
{शरान्धकारमाकाशमावृणोत् सदिवाकरम्}
{बभूवावस्थितो रामः प्रक्षिपन्निव तान् शरान्} %3-25-40

\twolineshloka
{युगपत्पतमानैश्च युगपच्च हतैर्भृशम्}
{युगपत्पतितैश्चैव विकीर्णा वसुधाभवत्} %3-25-41

\twolineshloka
{निहताः पतिताः क्षीणाश्छिन्ना भिन्ना विदारिताः}
{तत्र तत्र स्म दृश्यन्ते राक्षसास्ते सहस्रशः} %3-25-42

\twolineshloka
{सोष्णीषैरुत्तमाङ्गैश्च साङ्गदैर्बाहुभिस्तथा}
{ऊरुभिर्बाहुभिश्छिन्नैर्नानारूपैर्विभूषणैः} %3-25-43

\twolineshloka
{हयैश्च द्विपमुख्यैश्च रथैर्भिन्नैरनेकशः}
{चामरव्यजनैश्छत्रैर्ध्वजैर्नानाविधैरपि} %3-25-44

\twolineshloka
{रामेण बाणाभिहतैर्विच्छिन्नैः शूलपट्टिशैः}
{खड्गैः खण्डीकृतैः प्रासैर्विकीर्णैश्च परश्वधैः} %3-25-45

\twolineshloka
{चूर्णिताभिः शिलाभिश्च शरैश्चित्रैरनेकशः}
{विच्छिन्नैः समरे भूमिर्विस्तीर्णाभूद् भयङ्करा} %3-25-46

\twolineshloka
{तान् दृष्ट्वा निहतान् सर्वे राक्षसाः परमातुराः}
{न तत्र चलितुं शक्ता रामं परपुरञ्जयम्} %3-25-47


॥इत्यार्षे श्रीमद्रामायणे वाल्मीकीये आदिकाव्ये अरण्यकाण्डे खरसैन्यावमर्दः नाम पञ्चविंशः सर्गः ॥३-२५॥
