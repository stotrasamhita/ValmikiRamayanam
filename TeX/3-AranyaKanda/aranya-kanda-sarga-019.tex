\sect{एकोनविंशः सर्गः — खरक्रोधः}

\twolineshloka
{तां तथा पतितां दृष्ट्वा विरूपां शोणितोक्षिताम्}
{भगिनीं क्रोधसंतप्तः खरः पप्रच्छ राक्षसः} %3-19-1

\twolineshloka
{उत्तिष्ठ तावदाख्याहि प्रमोहं जहि सम्भ्रमम्}
{व्यक्तमाख्याहि केन त्वमेवंरूपा विरूपिता} %3-19-2

\twolineshloka
{कः कृष्णसर्पमासीनमाशीविषमनागसम्}
{तुदत्यभिसमापन्नमङ्गुल्यग्रेण लीलया} %3-19-3

\twolineshloka
{कालपाशं समासज्य कण्ठे मोहान्न बुध्यते}
{यस्त्वामद्य समासाद्य पीतवान् विषमुत्तमम्} %3-19-4

\twolineshloka
{बलविक्रमसम्पन्ना कामगा कामरूपिणी}
{इमामवस्थां नीता त्वं केनान्तकसमागता} %3-19-5

\twolineshloka
{देवगन्धर्वभूतानामृषीणां च महात्मनाम्}
{कोऽयमेवं महावीर्यस्त्वां विरूपां चकार ह} %3-19-6

\twolineshloka
{नहि पश्याम्यहं लोके यः कुर्यान्मम विप्रियम्}
{अमरेषु सहस्राक्षं महेन्द्रं पाकशासनम्} %3-19-7

\twolineshloka
{अद्याहं मार्गणैः प्राणानादास्ये जीवितान्तगैः}
{सलिले क्षीरमासक्तं निष्पिबन्निव सारसः} %3-19-8

\twolineshloka
{निहतस्य मया संख्ये शरसंकृत्तमर्मणः}
{सफेनं रुधिरं कस्य मेदिनी पातुमिच्छति} %3-19-9

\twolineshloka
{कस्य पत्ररथाः कायान्मांसमुत्कृत्य संगताः}
{प्रहृष्टा भक्षयिष्यन्ति निहतस्य मया रणे} %3-19-10

\twolineshloka
{तं न देवा न गन्धर्वा न पिशाचा न राक्षसाः}
{मयापकृष्टं कृपणं शक्तास्त्रातुं महाहवे} %3-19-11

\twolineshloka
{उपलभ्य शनैः संज्ञां तं मे शंसितुमर्हसि}
{येन त्वं दुर्विनीतेन वने विक्रम्य निर्जिता} %3-19-12

\twolineshloka
{इति भ्रातुर्वचः श्रुत्वा क्रुद्धस्य च विशेषतः}
{ततः शूर्पणखा वाक्यं सबाष्पमिदमब्रवीत्} %3-19-13

\twolineshloka
{तरुणौ रूपसम्पन्नौ सुकुमारौ महाबलौ}
{पुण्डरीकविशालाक्षौ चीरकृष्णाजिनाम्बरौ} %3-19-14

\twolineshloka
{फलमूलाशनौ दान्तौ तापसौ ब्रह्मचारिणौ}
{पुत्रौ दशरथस्यास्तां भ्रातरौ रामलक्ष्मणौ} %3-19-15

\twolineshloka
{गन्धर्वराजप्रतिमौ पार्थिवव्यञ्जनान्वितौ}
{देवौ वा दानवावेतौ न तर्कयितुमुत्सहे} %3-19-16

\twolineshloka
{तरुणी रूपसम्पन्ना सर्वाभरणभूषिता}
{दृष्टा तत्र मया नारी तयोर्मध्ये सुमध्यमा} %3-19-17

\twolineshloka
{ताभ्यामुभाभ्यां सम्भूय प्रमदामधिकृत्य ताम्}
{इमामवस्थां नीताहं यथानाथासती तथा} %3-19-18

\twolineshloka
{तस्याश्चानृजुवृत्तायास्तयोश्च हतयोरहम्}
{सफेनं पातुमिच्छामि रुधिरं रणमूर्धनि} %3-19-19

\twolineshloka
{एष मे प्रथमः कामः कृतस्तत्र त्वया भवेत्}
{तस्यास्तयोश्च रुधिरं पिबेयमहमाहवे} %3-19-20

\twolineshloka
{इति तस्यां ब्रुवाणायां चतुर्दश महाबलान्}
{व्यादिदेश खरः क्रुद्धो राक्षसानन्तकोपमान्} %3-19-21

\twolineshloka
{मानुषौ शस्त्रसम्पन्नौ चीरकृष्णाजिनाम्बरौ}
{प्रविष्टौ दण्डकारण्यं घोरं प्रमदया सह} %3-19-22

\twolineshloka
{तौ हत्वा तां च दुर्वृत्तामुपावर्तितुमर्हथ}
{इयं च भगिनी तेषां रुधिरं मम पास्यति} %3-19-23

\twolineshloka
{मनोरथोऽयमिष्टोऽस्या भगिन्या मम राक्षसाः}
{शीघ्रं सम्पाद्यतां गत्वा तौ प्रमथ्य स्वतेजसा} %3-19-24

\twolineshloka
{युष्माभिर्निहतौ दृष्ट्वा तावुभौ भ्रातरौ रणे}
{इयं प्रहृष्टा मुदिता रुधिरं युधि पास्यति} %3-19-25

\twolineshloka
{इति प्रतिसमादिष्टा राक्षसास्ते चतुर्दश}
{तत्र जग्मुस्तया सार्धं घना वातेरिता इव} %3-19-26


॥इत्यार्षे श्रीमद्रामायणे वाल्मीकीये आदिकाव्ये अरण्यकाण्डे खरक्रोधः नाम एकोनविंशः सर्गः ॥३-१९॥
