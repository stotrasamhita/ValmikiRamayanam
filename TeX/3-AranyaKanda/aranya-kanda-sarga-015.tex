\sect{पञ्चदशः सर्गः — पञ्चवटीपर्णशाला}

\twolineshloka
{ततः पञ्चवटीम् गत्वा नाना व्याल मृगायुताम्}
{उवाच भ्रातरम् रामो लक्ष्मणम् दीप्त तेजसम्} %3-15-1

\twolineshloka
{आगताः स्म यथा उद्दिष्टम् यम् देशम् मुनिः अब्रवीत्}
{अयम् पञ्चवटी देशः सौम्य पुष्पित काननः} %3-15-2

\twolineshloka
{सर्वतः चार्यताम् दृष्टिः कानने निपुणो हि असि}
{आश्रमः कतर अस्मिन् नः देशे भवति सम्मतः} %3-15-3

\twolineshloka
{रमते यत्र वैदेही त्वम् अहम् चैव लक्ष्मण}
{तादृशो दृश्यताम् देशः सन्निकृष्ट जलाशयः} %3-15-4

\twolineshloka
{वन रामण्यकम् यत्र जल रामण्यकम् तथा}
{सन्निकृष्टम् च यस्मिन् तु समित् पुष्प कुश उदकम्} %3-15-5

\twolineshloka
{एवम् उक्तः तु रामेण लक्मणः संयत अञ्जलिः}
{सीता समक्षम् काकुत्स्थम् इदम् वचनम् अब्रवीत्} %3-15-6

\twolineshloka
{परवान् अस्मि काकुत्स्थ त्वयि वर्ष शतम् स्थिते}
{स्वयम् तु रुचिरे देशे क्रियताम् इति माम् वद} %3-15-7

\twolineshloka
{सुप्रीतः तेन वाक्येन लक्ष्मणस्य महाद्युतिः}
{विमृशन् रोचयामास देशम् सर्व गुण अन्वितम्} %3-15-8

\twolineshloka
{स तम् रुचिरम् आक्रम्य देशम् आश्रम कर्मणि}
{हस्ते गृहीत्वा हस्तेन रामः सौमित्रिम् अब्रवीत्} %3-15-9

\twolineshloka
{अयम् देशः समः श्रीमान् पुष्पितैर् तरुभिर् वृतः}
{इह आश्रम पदम् सौम्य यथावत् कर्तुम् अर्हसि} %3-15-10

\twolineshloka
{इयम् आदित्य सङ्काशैः पद्मैः सुरभि गन्धिभिः}
{अदूरे दृश्यते रम्या पद्मिनी पद्म शोभिता} %3-15-11

\twolineshloka
{यथा आख्यातम् अगस्त्येन मुनिना भावितात्मना}
{इयम् गोदावरी रम्या पुष्पितैः तरुभिर् वृता} %3-15-12

\twolineshloka
{हंस कारण्डव आकीर्णा चक्रवाक उपशोभिता}
{न अतिदूरे न च आसन्ने मृग यूथ निपीडिता} %3-15-13

\twolineshloka
{मयूर नादिता रम्याः प्रांशवो बहु कन्दराः}
{दृश्यन्ते गिरयः सौम्य फुल्लैः तरुभिर् आवृताः} %3-15-14

\twolineshloka
{सौवर्णै राजतैः ताम्रैः देशे देशे च धातुभिः}
{गवाक्षिता इव आभान्ति गजाः परम भक्तिभिः} %3-15-15

\twolineshloka
{सालैः तालैः तमालैः च खर्जूरैः पनसैः द्रुमैः}
{नीवारैः तिनिशैः चैव पुन्नागैः च उपशोभिताः} %3-15-16

\twolineshloka
{चूतैर् अशोकैः तिलकैः केतकैर् अपि चम्पकैः}
{पुष्प गुल्म लता उपेतैः तैः तैः तरुभिर् आवृताः} %3-15-17

\twolineshloka
{स्यन्दनैः चन्दनैः नीपैः पर्णासैः लकुचैः अपि}
{धव अश्वकर्ण खदिरैः शमी किंशुक पाटलैः} %3-15-18

\twolineshloka
{इदम् पुण्यम् इदम् रम्यम् इदम् बहु मृग द्विजम्}
{इह वत्स्याम सौमित्रे सार्धम् एतेन पक्षिणा} %3-15-19

\twolineshloka
{एवम् उक्तः तु रामेण लक्ष्मणः परवीरहा}
{अचिरेण आश्रमम् भ्रातुः चकार सुमहाबलः} %3-15-20

\twolineshloka
{पर्णशालाम् सुविपुलाम् तत्र सङ्घात मृत्तिकाम्}
{सुस्तम्भाम् मस्करैर् दीर्घैः कृत वंशाम् सुशोभनाम्} %3-15-21

\twolineshloka
{शमी शाखाभिः आस्तीर्य धृढ पाशावपाशितम्}
{कुश काश शरैः पर्णैः सुपरिच्छादिताम् तथा} %3-15-22

\twolineshloka
{समीकृत तलाम् रम्याम् चकार सुमहाबलः}
{निवासम् राघवस्य अर्थे प्रेक्ष्णीयम् अनुत्तमम्} %3-15-23

\twolineshloka
{स गत्वा लक्ष्मणः श्रीमान् नदीम् गोदावरीम् तदा}
{स्नात्वा पद्मानि च आदाय सफलः पुनर् आगतः} %3-15-24

\twolineshloka
{ततः पुष्प बलिम् कृत्वा शान्तिम् च स यथाविधि}
{दर्शयामास रामाय तद् आश्रम पदम् कृतम्} %3-15-25

\twolineshloka
{स तम् दृष्ट्वा कृतम् सौम्यम् आश्रमम् सह सीतया}
{राघवः पर्णशालायाम् हर्षम् आहारयत् परम्} %3-15-26

\twolineshloka
{सुसंहृष्टः परिष्वज्य बाहुभ्याम् लक्ष्मणम् तदा}
{अति स्निग्धम् च गाढम् च वचनम् च इदम् अब्रवीत्} %3-15-27

\twolineshloka
{प्रीतो अस्मि ते महत् कर्म त्वया कृतम् इदम् प्रभो}
{प्रदेयो यन् निमित्तम् ते परिष्वङ्गो मया कृतः} %3-15-28

\twolineshloka
{भावज्ञेन कृतज्ञेन धर्मज्ञेन च लक्ष्मण}
{त्वया पुत्रेण धर्मात्मा न संवृत्तः पिता मम} %3-15-29

\twolineshloka
{एवम् लक्ष्मणम् उक्त्वा तु राघवो लक्ष्मिवर्धनः}
{तस्मिन् देशे बहु फले न्यवसत् स सुखम् सुखी} %3-15-30

\onelineshloka
{कञ्चित् कालम् स धर्मात्मा सीतया लक्ष्मणेन च अन्वास्यमानो न्यवसत् स्वर्ग लोके यथा अमरः} %3-15-31


॥इत्यार्षे श्रीमद्रामायणे वाल्मीकीये आदिकाव्ये अरण्यकाण्डे पञ्चवटीपर्णशाला नाम पञ्चदशः सर्गः ॥३-१५॥
