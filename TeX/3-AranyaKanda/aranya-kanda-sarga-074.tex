\sect{चतुःसप्ततितमः सर्गः — शबरीस्वर्गप्राप्तिः}

\twolineshloka
{तौ कबन्धेन तं मार्गं पम्पाया दर्शितं वने}
{आतस्थतुर्दिशं गृह्य प्रतीचीं नृवरात्मजौ} %3-74-1

\twolineshloka
{तौ शैलेष्वाचितानेकान् क्षौद्रपुष्पफलद्रुमान्}
{वीक्षन्तौ जग्मतुर्द्रष्टुं सुग्रीवं रामलक्ष्मणौ} %3-74-2

\twolineshloka
{कृत्वा तु शैलपृष्ठे तु तौ वासं रघुनन्दनौ}
{पम्पायाः पश्चिमं तीरं राघवावुपतस्थतुः} %3-74-3

\twolineshloka
{तौ पुष्करिण्याः पम्पायास्तीरमासाद्य पश्चिमम्}
{अपश्यतां ततस्तत्र शबर्या रम्यमाश्रमम्} %3-74-4

\twolineshloka
{तौ तमाश्रममासाद्य द्रुमैर्बहुभिरावृतम्}
{सुरम्यमभिवीक्षन्तौ शबरीमभ्युपेयतुः} %3-74-5

\twolineshloka
{तौ दृष्ट्वा तु तदा सिद्धा समुत्थाय कृताञ्जलिः}
{पादौ जग्राह रामस्य लक्ष्मणस्य च धीमतः} %3-74-6

\twolineshloka
{पाद्यमाचमनीयं च सर्वं प्रादाद् यथाविधि}
{तामुवाच ततो रामः श्रमणीं धर्मसंस्थिताम्} %3-74-7

\twolineshloka
{कच्चित्ते निर्जिता विघ्नाः कच्चित्ते वर्धते तपः}
{कच्चित्ते नियतः कोप आहारश्च तपोधने} %3-74-8

\twolineshloka
{कच्चित्ते नियमाः प्राप्ताः कच्चित्ते मनसः सुखम्}
{कच्चित्ते गुरुशुश्रूषा सफला चारुभाषिणि} %3-74-9

\twolineshloka
{रामेण तापसी पृष्टा सा सिद्धा सिद्धसम्मता}
{शशंस शबरी वृद्धा रामाय प्रत्यवस्थिता} %3-74-10

\twolineshloka
{अद्य प्राप्ता तपःसिद्धिस्तव संदर्शनान्मया}
{अद्य मे सफलं जन्म गुरवश्च सुपूजिताः} %3-74-11

\twolineshloka
{अद्य मे सफलं तप्तं स्वर्गश्चैव भविष्यति}
{त्वयि देववरे राम पूजिते पुरुषर्षभ} %3-74-12

\twolineshloka
{तवाहं चक्षुषा सौम्य पूता सौम्येन मानद}
{गमिष्याम्यक्षयांल्लोकांस्त्वत्प्रसादादरिंदम} %3-74-13

\twolineshloka
{चित्रकूटं त्वयि प्राप्ते विमानैरतुलप्रभैः}
{इतस्ते दिवमारूढा यानहं पर्यचारिषम्} %3-74-14

\twolineshloka
{तैश्चाहमुक्ता धर्मज्ञैर्महाभागैर्महर्षिभिः}
{आगमिष्यति ते रामः सुपुण्यमिममाश्रमम्} %3-74-15

\twolineshloka
{स ते प्रतिग्रहीतव्यः सौमित्रिसहितोऽतिथिः}
{तं च दृष्ट्वा वरांल्लोकानक्षयांस्त्वं गमिष्यसि} %3-74-16

\twolineshloka
{एवमुक्ता महाभागैस्तदाहं पुरुषर्षभ}
{मया तु संचितं वन्यं विविधं पुरुषर्षभ} %3-74-17

\twolineshloka
{तवार्थे पुरुषव्याघ्र पम्पायास्तीरसम्भवम्}
{एवमुक्तः स धर्मात्मा शबर्या शबरीमिदम्} %3-74-18

\twolineshloka
{राघवः प्राह विज्ञाने तां नित्यमबहिष्कृताम्}
{दनोः सकाशात् तत्त्वेन प्रभावं ते महात्मनाम्} %3-74-19

\twolineshloka
{श्रुतं प्रत्यक्षमिच्छामि संद्रष्टुं यदि मन्यसे}
{एतत्तु वचनं श्रुत्वा रामवक्त्रविनिःसृतम्} %3-74-20

\twolineshloka
{शबरी दर्शयामास तावुभौ तद्वनं महत्}
{पश्य मेघघनप्रख्यं मृगपक्षिसमाकुलम्} %3-74-21

\threelineshloka
{मतङ्गवनमित्येव विश्रुतं रघुनन्दन}
{इह ते भावितात्मानो गुरवो मे महाद्युते}
{जुहवांचक्रिरे नीडं मन्त्रवन्मन्त्रपूजितम्} %3-74-22

\twolineshloka
{इयं प्रत्यक्स्थली वेदी यत्र ते मे सुसत्कृताः}
{पुष्पोपहारं कुर्वन्ति श्रमादुद्वेपिभिः करैः} %3-74-23

\twolineshloka
{तेषां तपःप्रभावेण पश्याद्यापि रघूत्तम}
{द्योतयन्ती दिशः सर्वाः श्रिया वेद्यतुलप्रभा} %3-74-24

\twolineshloka
{अशक्नुवद्भिस्तैर्गन्तुमुपवासश्रमालसैः}
{चिन्तितेनागतान् पश्य समेतान् सप्त सागरान्} %3-74-25

\twolineshloka
{कृताभिषेकैस्तैर्न्यस्ता वल्कलाः पादपेष्विह}
{अद्यापि न विशुष्यन्ति प्रदेशे रघुनन्दन} %3-74-26

\twolineshloka
{देवकार्याणि कुर्वद्भिर्यानीमानि कृतानि वै}
{पुष्पैः कुवलयैः सार्धं म्लानत्वं न तु यान्ति वै} %3-74-27

\twolineshloka
{कृत्स्नं वनमिदं दृष्टं श्रोतव्यं च श्रुतं त्वया}
{तद् इच्छाम्य् अभ्यनुज्ञाता त्यक्ष्याम्य् एतत् कलेवरम्} %3-74-28

\twolineshloka
{तेषामिच्छाम्यहं गन्तुं समीपं भावितात्मनाम्}
{मुनीनामाश्रमो येषामहं च परिचारिणी} %3-74-29

\twolineshloka
{धर्मिष्ठं तु वचः श्रुत्वा राघवः सहलक्ष्मणः}
{प्रहर्षमतुलं लेभे आश्चर्यमिति चाब्रवीत्} %3-74-30

\twolineshloka
{तामुवाच ततो रामः शबरीं संशितव्रताम्}
{अर्चितोऽहं त्वया भद्रे गच्छ कामं यथासुखम्} %3-74-31

\twolineshloka
{इत्येवमुक्ता जटिला चीरकृष्णाजिनाम्बरा}
{अनुज्ञाता तु रामेण हुत्वाऽऽत्मानं हुताशने} %3-74-32

\twolineshloka
{ज्वलत्पावकसंकाशा स्वर्गमेव जगाम ह}
{दिव्याभरणसंयुक्ता दिव्यमाल्यानुलेपना} %3-74-33

\twolineshloka
{दिव्याम्बरधरा तत्र बभूव प्रियदर्शना}
{विराजयन्ती तं देशं विद्युत्सौदामनी यथा} %3-74-34

\twolineshloka
{यत्र ते सुकृतात्मानो विहरन्ति महर्षयः}
{तत् पुण्यं शबरी स्थानं जगामात्मसमाधिना} %3-74-35


॥इत्यार्षे श्रीमद्रामायणे वाल्मीकीये आदिकाव्ये अरण्यकाण्डे शबरीस्वर्गप्राप्तिः नाम चतुःसप्ततितमः सर्गः ॥३-७४॥
