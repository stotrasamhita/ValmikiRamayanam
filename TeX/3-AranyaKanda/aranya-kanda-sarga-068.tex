\sect{अष्टषष्ठितमः सर्गः — जटयुस्संस्कारः}

\twolineshloka
{रामः प्रेक्ष्य तु तं गृध्रं भुवि रौद्रेण पातितम्}
{सौमित्रिं मित्रसम्पन्नमिदं वचनमब्रवीत्} %3-68-1

\twolineshloka
{ममायं नूनमर्थेषु यतमानो विहङ्गमः}
{राक्षसेन हतः सङ्ख्ये प्राणांस्त्यजति मत्कृते} %3-68-2

\twolineshloka
{अतिखिन्नः शरीरेऽस्मिन् प्राणो लक्ष्मण विद्यते}
{तथा स्वरविहीनोऽयं विक्लवं समुदीक्षते} %3-68-3

\twolineshloka
{जटायो यदि शक्नोषि वाक्यं व्याहरितुं पुनः}
{सीतामाख्याहि भद्रं ते वधमाख्याहि चात्मनः} %3-68-4

\twolineshloka
{किन्निमित्तो जहारार्यां रावणस्तस्य किं मया}
{अपराधं तु यं दृष्ट्वा रावणेन हृता प्रिया} %3-68-5

\twolineshloka
{कथं तच्चन्द्रसङ्काशं मुखमासीन्मनोहरम्}
{सीतया कानि चोक्तानि तस्मिन् काले द्विजोत्तम} %3-68-6

\twolineshloka
{कथंवीर्यः कथंरूपः किङ्कर्मा स च राक्षसः}
{क्व चास्य भवनं तात ब्रूहि मे परिपृच्छतः} %3-68-7

\twolineshloka
{तमुद्वीक्ष्य स धर्मात्मा विलपन्तमनाथवत्}
{वाचा विक्लवया राममिदं वचनमब्रवीत्} %3-68-8

\twolineshloka
{सा हृता राक्षसेन्द्रेण रावणेन दुरात्मना}
{मायामास्थाय विपुलां वातदुर्दिनसङ्कुलाम्} %3-68-9

\twolineshloka
{परिक्लान्तस्य मे तात पक्षौ छित्त्वा निशाचरः}
{सीतामादाय वैदेहीं प्रयातो दक्षिणामुखः} %3-68-10

\twolineshloka
{उपरुध्यन्ति मे प्राणा दृष्टिर्भ्रमति राघव}
{पश्यामि वृक्षान् सौवर्णानुशीरकृतमूर्धजान्} %3-68-11

\twolineshloka
{येन याति मुहूर्तेन सीतामादाय रावणः}
{विप्रणष्टं धनं क्षिप्रं तत्स्वामी प्रतिपद्यते} %3-68-12

\threelineshloka
{विन्दो नाम मुहूर्तोऽसौ न च काकुत्स्थ सोऽबुधत्}
{त्वत्प्रियां जानकीं हृत्वा रावणो राक्षसेश्वरः}
{झषवद् बडिशं गृह्य क्षिप्रमेव विनश्यति} %3-68-13

\twolineshloka
{न च त्वया व्यथा कार्या जनकस्य सुतां प्रति}
{वैदेह्या रंस्यसे क्षिप्रं हत्वा तं रणमूर्धनि} %3-68-14

\twolineshloka
{असम्मूढस्य गृध्रस्य रामं प्रत्यनुभाषतः}
{आस्यात् सुस्राव रुधिरं म्रियमाणस्य सामिषम्} %3-68-15

\twolineshloka
{पुत्रो विश्रवसः साक्षाद् भ्राता वैश्रवणस्य च}
{इत्युक्त्वा दुर्लभान् प्राणान् मुमोच पतगेश्वरः} %3-68-16

\twolineshloka
{ब्रूहि ब्रूहीति रामस्य ब्रुवाणस्य कृताञ्जलेः}
{त्यक्त्वा शरीरं गृध्रस्य प्राणा जग्मुर्विहायसम्} %3-68-17

\twolineshloka
{स निक्षिप्य शिरो भूमौ प्रसार्य चरणौ तथा}
{विक्षिप्य च शरीरं स्वं पपात धरणीतले} %3-68-18

\twolineshloka
{तं गृध्रं प्रेक्ष्य ताम्राक्षं गतासुमचलोपमम्}
{रामः सुबहुभिर्दुःखैर्दीनः सौमित्रिमब्रवीत्} %3-68-19

\twolineshloka
{बहूनि रक्षसां वासे वर्षाणि वसता सुखम्}
{अनेन दण्डकारण्ये विशीर्णमिह पक्षिणा} %3-68-20

\twolineshloka
{अनेकवार्षिको यस्तु चिरकालसमुत्थितः}
{सोऽयमद्य हतः शेते कालो हि दुरतिक्रमः} %3-68-21

\twolineshloka
{पश्य लक्ष्मण गृध्रोऽयमुपकारी हतश्च मे}
{सीतामभ्यवपन्नो हि रावणेन बलीयसा} %3-68-22

\twolineshloka
{गृध्रराज्यं परित्यज्य पितृपैतामहं महत्}
{मम हेतोरयं प्राणान् मुमोच पतगेश्वरः} %3-68-23

\twolineshloka
{सर्वत्र खलु दृश्यन्ते साधवो धर्मचारिणः}
{शूराः शरण्याः सौमित्रे तिर्यग्योनिगतेष्वपि} %3-68-24

\twolineshloka
{सीताहरणजं दुःखं न मे सौम्य तथागतम्}
{यथा विनाशो गृध्रस्य मत्कृते च परन्तप} %3-68-25

\twolineshloka
{राजा दशरथः श्रीमान् यथा मम महायशाः}
{पूजनीयश्च मान्यश्च तथायं पतगेश्वरः} %3-68-26

\twolineshloka
{सौमित्रे हर काष्ठानि निर्मथिष्यामि पावकम्}
{गृध्रराजं दिधक्ष्यामि मत्कृते निधनं गतम्} %3-68-27

\twolineshloka
{नाथं पतगलोकस्य चितिमारोपयाम्यहम्}
{इमं धक्ष्यामि सौमित्रे हतं रौद्रेण रक्षसा} %3-68-28

\twolineshloka
{या गतिर्यज्ञशीलानामाहिताग्नेश्च या गतिः}
{अपरावर्तिनां या च या च भूमिप्रदायिनाम्} %3-68-29

\twolineshloka
{मया त्वं समनुज्ञातो गच्छ लोकाननुत्तमान्}
{गृध्रराज महासत्त्व संस्कृतश्च मया व्रज} %3-68-30

\twolineshloka
{एवमुक्त्वा चितां दीप्तामारोप्य पतगेश्वरम्}
{ददाह रामो धर्मात्मा स्वबन्धुमिव दुःखितः} %3-68-31

\twolineshloka
{रामोऽथ सहसौमित्रिर्वनं गत्वा स वीर्यवान्}
{स्थूलान् हत्वा महारोहीननुतस्तार तं द्विजम्} %3-68-32

\twolineshloka
{रोहिमांसानि चोद्धृत्य पेशीकृत्वा महायशाः}
{शकुनाय ददौ रामो रम्ये हरितशाद्वले} %3-68-33

\twolineshloka
{यत् तत् प्रेतस्य मर्त्यस्य कथयन्ति द्विजातयः}
{तत् स्वर्गगमनं पित्र्यं तस्य रामो जजाप ह} %3-68-34

\twolineshloka
{ततो गोदावरीं गत्वा नदीं नरवरात्मजौ}
{उदकं चक्रतुस्तस्मै गृध्रराजाय तावुभौ} %3-68-35

\twolineshloka
{शास्त्रदृष्टेन विधिना जलं गृध्राय राघवौ}
{स्नात्वा तौ गृध्रराजाय उदकं चक्रतुस्तदा} %3-68-36

\twolineshloka
{स गृध्रराजः कृतवान् यशस्करं सुदुष्करं कर्म रणे निपातितः}
{महर्षिकल्पेन च संस्कृतस्तदा जगाम पुण्यां गतिमात्मनः शुभाम्} %3-68-37

\twolineshloka
{कृतोदकौ तावपि पक्षिसत्तमे स्थिरां च बुद्धिं प्रणिधाय जग्मतुः}
{प्रवेश्य सीताधिगमे ततो मनो वनं सुरेन्द्राविव विष्णुवासवौ} %3-68-38


॥इत्यार्षे श्रीमद्रामायणे वाल्मीकीये आदिकाव्ये अरण्यकाण्डे जटयुस्संस्कारः नाम अष्टषष्ठितमः सर्गः ॥३-६८॥
