\sect{एकपञ्चाशः सर्गः — जटायुरावणयुद्धम्}

\twolineshloka
{इत्युक्तः क्रोधताम्राक्षस्तप्तकाञ्चनकुण्डलः}
{राक्षसेन्द्रोऽभिदुद्राव पतगेन्द्रममर्षणः} %3-51-1

\twolineshloka
{स सम्प्रहारस्तुमुलस्तयोस्तस्मिन् महामृधे}
{बभूव वातोद्धुतयोर्मेघयोर्गगने यथा} %3-51-2

\twolineshloka
{तद् बभूवाद्भुतं युद्धं गृध्रराक्षसयोस्तदा}
{सपक्षयोर्माल्यवतोर्महापर्वतयोरिव} %3-51-3

\twolineshloka
{ततो नालीकनाराचैस्तीक्ष्णाग्रैश्च विकर्णिभिः}
{अभ्यवर्षन्महाघोरैर्गृध्रराजं महाबलम्} %3-51-4

\twolineshloka
{स तानि शरजालानि गृध्रः पत्ररथेश्वरः}
{जटायुः प्रतिजग्राह रावणास्त्राणि संयुगे} %3-51-5

\twolineshloka
{तस्य तीक्ष्णनखाभ्यां तु चरणाभ्यां महाबलः}
{चकार बहुधा गात्रे व्रणान् पतगसत्तमः} %3-51-6

\twolineshloka
{अथ क्रोधाद् दशग्रीवो जग्राह दश मार्गणान्}
{मृत्युदण्डनिभान् घोरान् शत्रोर्निधनकांक्षया} %3-51-7

\twolineshloka
{स तैर्बाणैर्महावीर्यः पूर्णमुक्तैरजिह्मगैः}
{बिभेद निशितैस्तीक्ष्णैर्गृध्रं घोरैः शिलीमुखैः} %3-51-8

\twolineshloka
{स राक्षसरथे पश्यञ्जानकीं बाष्पलोचनाम्}
{अचिन्तयित्वा बाणांस्तान् राक्षसं समभिद्रवत्} %3-51-9

\twolineshloka
{ततोऽस्य सशरं चापं मुक्तामणिविभूषितम्}
{चरणाभ्यां महातेजा बभञ्ज पतगोत्तमः} %3-51-10

\twolineshloka
{ततोऽन्यद् धनुरादाय रावणः क्रोधमूर्च्छितः}
{ववर्ष शरवर्षाणि शतशोऽथ सहस्रशः} %3-51-11

\twolineshloka
{शरैरावारितस्तस्य संयुगे पतगेश्वरः}
{कुलायमभिसम्प्राप्तः पक्षिवच्च बभौ तदा} %3-51-12

\twolineshloka
{स तानि शरजालानि पक्षाभ्यां तु विधूय ह}
{चरणाभ्यां महातेजा बभञ्जास्य महद् धनुः} %3-51-13

\twolineshloka
{तच्चाग्निसदृशं दीप्तं रावणस्य शरावरम्}
{पक्षाभ्यां च महातेजा व्यधुनोत् पतगेश्वरः} %3-51-14

\twolineshloka
{काञ्चनोरश्छदान् दिव्यान् पिशाचवदनान् खरान्}
{तांश्चास्य जवसम्पन्नाञ्जघान समरे बली} %3-51-15

\twolineshloka
{अथ त्रिवेणुसम्पन्नं कामगं पावकार्चिषम्}
{मणिसोपानचित्राङ्गं बभञ्ज च महारथम्} %3-51-16

\twolineshloka
{पूर्णचन्द्रप्रतीकाशं छत्रं च व्यजनैः सह}
{पातयामास वेगेन ग्राहिभी राक्षसैः सह} %3-51-17

\twolineshloka
{सारथेश्चास्य वेगेन तुण्डेन च महच्छिरः}
{पुनर्व्यपहनच्छ्रीमान् पक्षिराजो महाबलः} %3-51-18

\twolineshloka
{स भग्नधन्वा विरथो हताश्वो हतसारथिः}
{अङ्केनादाय वैदेहीं पपात भुवि रावणः} %3-51-19

\twolineshloka
{दृष्ट्वा निपतितं भूमौ रावणं भग्नवाहनम्}
{साधु साध्विति भूतानि गृध्रराजमपूजयन्} %3-51-20

\twolineshloka
{परिश्रान्तं तु तं दृष्ट्वा जरया पक्षियूथपम्}
{उत्पपात पुनर्हृष्टो मैथिलीं गृह्य रावणः} %3-51-21

\twolineshloka
{तं प्रहृष्टं निधायाङ्के रावणं जनकात्मजाम्}
{गच्छन्तं खड्गशेषं च प्रणष्टहतसाधनम्} %3-51-22

\twolineshloka
{गृध्रराजः समुत्पत्य रावणं समभिद्रवत्}
{समावार्य महातेजा जटायुरिदमब्रवीत्} %3-51-23

\twolineshloka
{वज्रसंस्पर्शबाणस्य भार्यां रामस्य रावण}
{अल्पबुद्धे हरस्येनां वधाय खलु रक्षसाम्} %3-51-24

\twolineshloka
{समित्रबन्धुः सामात्यः सबलः सपरिच्छदः}
{विषपानं पिबस्येतत् पिपासित इवोदकम्} %3-51-25

\twolineshloka
{अनुबन्धमजानन्तः कर्मणामविचक्षणाः}
{शीघ्रमेव विनश्यन्ति यथा त्वं विनशिष्यसि} %3-51-26

\twolineshloka
{बद्धस्त्वं कालपाशेन क्व गतस्तस्य मोक्ष्यसे}
{वधाय बडिशं गृह्य सामिषं जलजो यथा} %3-51-27

\twolineshloka
{नहि जातु दुराधर्षौ काकुत्स्थौ तव रावण}
{धर्षणं चाश्रमस्यास्य क्षमिष्येते तु राघवौ} %3-51-28

\twolineshloka
{यथा त्वया कृतं कर्म भीरुणा लोकगर्हितम्}
{तस्कराचरितो मार्गो नैष वीरनिषेवितः} %3-51-29

\twolineshloka
{युद्ध्यस्व यदि शूरोऽसि मुहूर्तं तिष्ठ रावण}
{शयिष्यसे हतो भूमौ यथा भ्राता खरस्तथा} %3-51-30

\twolineshloka
{परेतकाले पुरुषो यत् कर्म प्रतिपद्यते}
{विनाशायात्मनोऽधर्म्यं प्रतिपन्नोऽसि कर्म तत्} %3-51-31

\twolineshloka
{पापानुबन्धो वै यस्य कर्मणः को नु तत् पुमान्}
{कुर्वीत लोकाधिपतिः स्वयंभूर्भगवानपि} %3-51-32

\twolineshloka
{एवमुक्त्वा शुभं वाक्यं जटायुस्तस्य रक्षसः}
{निपपात भृशं पृष्ठे दशग्रीवस्य वीर्यवान्} %3-51-33

\twolineshloka
{तं गृहीत्वा नखैस्तीक्ष्णैर्विददार समन्ततः}
{अधिरूढो गजारोहो यथा स्याद् दुष्टवारणम्} %3-51-34

\twolineshloka
{विददार नखैरस्य तुण्डं पृष्ठे समर्पयन्}
{केशांश्चोत्पाटयामास नखपक्षमुखायुधः} %3-51-35

\twolineshloka
{स तथा गृध्रराजेन क्लिश्यमानो मुहुर्मुहुः}
{अमर्षस्फुरितोष्ठः सन् प्राकम्पत च राक्षसः} %3-51-36

\twolineshloka
{सम्परिष्वज्य वैदेहीं वामेनाङ्केन रावणः}
{तलेनाभिजघानार्तो जटायुं क्रोधमूर्च्छितः} %3-51-37

\twolineshloka
{जटायुस्तमतिक्रम्य तुण्डेनास्य खगाधिपः}
{वामबाहून् दश तदा व्यपाहरदरिंदमः} %3-51-38

\twolineshloka
{संछिन्नबाहोः सद्यो वै बाहवः सहसाभवन्}
{विषज्वालावलीयुक्ता वल्मीकादिव पन्नगाः} %3-51-39

\twolineshloka
{ततः क्रोधाद् दशग्रीवः सीतामुत्सृज्य वीर्यवान्}
{मुष्टिभ्यां चरणाभ्यां च गृध्रराजमपोथयत्} %3-51-40

\twolineshloka
{ततो मुहूर्तं संग्रामो बभूवातुलवीर्ययोः}
{राक्षसानां च मुख्यस्य पक्षिणां प्रवरस्य च} %3-51-41

\twolineshloka
{तस्य व्यायच्छमानस्य रामस्यार्थे स रावणः}
{पक्षौ पादौ च पार्श्वौ च खड्गमुद्धृत्य सोऽच्छिनत्} %3-51-42

\twolineshloka
{स च्छिन्नपक्षः सहसा रक्षसा रौद्रकर्मणा}
{निपपात महागृध्रो धरण्यामल्पजीवितः} %3-51-43

\twolineshloka
{तं दृष्ट्वा पतितं भूमौ क्षतजार्द्रं जटायुषम्}
{अभ्यधावत वैदेही स्वबन्धुमिव दुःखिता} %3-51-44

\twolineshloka
{तं नीलजीमूतनिकाशकल्पं सपाण्डुरोरस्कमुदारवीर्यम्}
{ददर्श लङ्काधिपतिः पृथिव्यां जटायुषं शान्तमिवाग्निदावम्} %3-51-45

\twolineshloka
{ततस्तु तं पत्ररथं महीतले निपातितं रावणवेगमर्दितम्}
{पुनश्च संगृह्य शशिप्रभानना रुरोद सीता जनकात्मजा तदा} %3-51-46


॥इत्यार्षे श्रीमद्रामायणे वाल्मीकीये आदिकाव्ये अरण्यकाण्डे जटायुरावणयुद्धम् नाम एकपञ्चाशः सर्गः ॥३-५१॥
