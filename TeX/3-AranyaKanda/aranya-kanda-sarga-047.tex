\sect{सप्तचत्वारिंशः सर्गः — रावणाधिक्षेपः}

\twolineshloka
{रावणेन तु वैदेही तदा पृष्टा जिहीर्षुणा}
{परिव्राजकरूपेण शशंसात्मानमात्मना} %3-47-1

\twolineshloka
{ब्राह्मणश्चातिथिश्चैष अनुक्तो हि शपेत माम्}
{इति ध्यात्वा मुहूर्तं तु सीता वचनमब्रवीत्} %3-47-2

\twolineshloka
{दुहिता जनकस्याहं मैथिलस्य महात्मनः}
{सीता नाम्नास्मि भद्रं ते रामस्य महिषी प्रिया} %3-47-3

\twolineshloka
{उषित्वा द्वादश समा इक्ष्वाकूणां निवेशने}
{भुञ्जाना मानुषान् भोगान् सर्वकामसमृद्धिनी} %3-47-4

\twolineshloka
{तत्र त्रयोदशे वर्षे राजाऽमन्त्रयत प्रभुः}
{अभिषेचयितुं रामं समेतो राजमन्त्रिभिः} %3-47-5

\twolineshloka
{तस्मिन् सम्भ्रियमाणे तु राघवस्याभिषेचने}
{कैकेयी नाम भर्तारं ममार्या याचते वरम्} %3-47-6

\twolineshloka
{परिगृह्य तु कैकेयी श्वशुरं सुकृतेन मे}
{मम प्रव्राजनं भर्तुर्भरतस्याभिषेचनम्} %3-47-7

\twolineshloka
{द्वावयाचत भर्तारं सत्यसंधं नृपोत्तमम्}
{नाद्य भोक्ष्ये न च स्वप्स्ये न पास्ये न कदाचन} %3-47-8

\twolineshloka
{एष मे जीवितस्यान्तो रामो यदभिषिच्यते}
{इति ब्रुवाणां कैकेयीं श्वशुरो मे स पार्थिवः} %3-47-9

\twolineshloka
{अयाचतार्थैरन्वर्थैर्न च याच्ञां चकार सा}
{मम भर्ता महातेजा वयसा पञ्चविंशकः} %3-47-10

\twolineshloka
{अष्टादश हि वर्षाणि मम जन्मनि गण्यते}
{रामेति प्रथितो लोके सत्यवान् शीलवान् शुचिः} %3-47-11

\twolineshloka
{विशालाक्षो महाबाहुः सर्वभूतहिते रतः}
{कामार्तश्च महाराजः पिता दशरथः स्वयम्} %3-47-12

\twolineshloka
{कैकेय्याः प्रियकामार्थं तं रामं नाभ्यषेचयत्}
{अभिषेकाय तु पितुः समीपं राममागतम्} %3-47-13

\twolineshloka
{कैकेयी मम भर्तारमित्युवाच द्रुतं वचः}
{तव पित्रा समाज्ञप्तं ममेदं शृणु राघव} %3-47-14

\twolineshloka
{भरताय प्रदातव्यमिदं राज्यमकण्टकम्}
{त्वया तु खलु वस्तव्यं नव वर्षाणि पञ्च च} %3-47-15

\twolineshloka
{वने प्रव्रज काकुत्स्थ पितरं मोचयानृतात्}
{तथेत्युवाच तां रामः कैकेयीमकुतोभयः} %3-47-16

\twolineshloka
{चकार तद्वचः श्रुत्वा भर्ता मम दृढव्रतः}
{दद्यान्न प्रतिगृह्णीयात् सत्यं ब्रूयान्न चानृतम्} %3-47-17

\twolineshloka
{एतद् ब्राह्मण रामस्य व्रतं धृतमनुत्तमम्}
{तस्य भ्राता तु वैमात्रो लक्ष्मणो नाम वीर्यवान्} %3-47-18

\twolineshloka
{रामस्य पुरुषव्याघ्रः सहायः समरेऽरिहा}
{स भ्राता लक्ष्मणो नाम ब्रह्मचारी दृढव्रतः} %3-47-19

\twolineshloka
{अन्वगच्छद् धनुष्पाणिः प्रव्रजन्तं मया सह}
{जटी तापसरूपेण मया सह सहानुजः} %3-47-20

\twolineshloka
{प्रविष्टो दण्डकारण्यं धर्मनित्यो दृढव्रतः}
{ते वयं प्रच्युता राज्यात् कैकेय्यास्तु कृते त्रयः} %3-47-21

\twolineshloka
{विचराम द्विजश्रेष्ठ वनं गम्भीरमोजसा}
{समाश्वस मुहूर्तं तु शक्यं वस्तुमिह त्वया} %3-47-22

\twolineshloka
{आगमिष्यति मे भर्ता वन्यमादाय पुष्कलम्}
{रुरून् गोधान् वराहांश्च हत्वाऽऽदायामिषं बहु} %3-47-23

\twolineshloka
{स त्वं नाम च गोत्रं च कुलमाचक्ष्व तत्त्वतः}
{एकश्च दण्डकारण्ये किमर्थं चरसि द्विज} %3-47-24

\twolineshloka
{एवं ब्रुवत्यां सीतायां रामपत्न्यां महाबलः}
{प्रत्युवाचोत्तरं तीव्रं रावणो राक्षसाधिपः} %3-47-25

\twolineshloka
{येन वित्रासिता लोकाः सदेवासुरमानुषाः}
{अहं स रावणो नाम सीते रक्षोगणेश्वरः} %3-47-26

\twolineshloka
{त्वां तु काञ्चनवर्णाभां दृष्ट्वा कौशेयवासिनीम्}
{रतिं स्वकेषु दारेषु नाधिगच्छाम्यनिन्दिते} %3-47-27

\twolineshloka
{बह्वीनामुत्तमस्त्रीणामाहृतानामितस्ततः}
{सर्वासामेव भद्रं ते ममाग्रमहिषी भव} %3-47-28

\twolineshloka
{लङ्का नाम समुद्रस्य मध्ये मम महापुरी}
{सागरेण परिक्षिप्ता निविष्टा गिरिमूर्धनि} %3-47-29

\twolineshloka
{तत्र सीते मया सार्धं वनेषु विचरिष्यसि}
{न चास्य वनवासस्य स्पृहयिष्यसि भामिनि} %3-47-30

\twolineshloka
{पञ्च दास्यः सहस्राणि सर्वाभरणभूषिताः}
{सीते परिचरिष्यन्ति भार्या भवसि मे यदि} %3-47-31

\twolineshloka
{रावणेनैवमुक्ता तु कुपिता जनकात्मजा}
{प्रत्युवाचानवद्याङ्गी तमनादृत्य राक्षसम्} %3-47-32

\twolineshloka
{महागिरिमिवाकम्प्यं महेन्द्रसदृशं पतिम्}
{महोदधिमिवाक्षोभ्यमहं राममनुव्रता} %3-47-33

\twolineshloka
{सर्वलक्षणसम्पन्नं न्यग्रोधपरिमण्डलम्}
{सत्यसंधं महाभागमहं राममनुव्रता} %3-47-34

\twolineshloka
{महाबाहुं महोरस्कं सिंहविक्रान्तगामिनम्}
{नृसिंहं सिंहसंकाशमहं राममनुव्रता} %3-47-35

\twolineshloka
{पूर्णचन्द्राननं रामं राजवत्सं जितेन्द्रियम्}
{पृथुकीर्तिं महाबाहुमहं राममनुव्रता} %3-47-36

\twolineshloka
{त्वं पुनर्जम्बुकः सिंहीं मामिहेच्छसि दुर्लभाम्}
{नाहं शक्या त्वया स्प्रष्टुमादित्यस्य प्रभा यथा} %3-47-37

\twolineshloka
{पादपान् काञ्चनान् नूनं बहून् पश्यसि मन्दभाक्}
{राघवस्य प्रियां भार्यां यस्त्वमिच्छसि राक्षस} %3-47-38

\twolineshloka
{क्षुधितस्य च सिंहस्य मृगशत्रोस्तरस्विनः}
{आशीविषस्य वदनाद् दंष्ट्रामादातुमिच्छसि} %3-47-39

\twolineshloka
{मन्दरं पर्वतश्रेष्ठं पाणिना हर्तुमिच्छसि}
{कालकूटं विषं पीत्वा स्वस्तिमान् गन्तुमिच्छसि} %3-47-40

\twolineshloka
{अक्षि सूच्या प्रमृजसि जिह्वया लेढि च क्षुरम्}
{राघवस्य प्रियां भार्यामधिगन्तुं त्वमिच्छसि} %3-47-41

\twolineshloka
{अवसज्य शिलां कण्ठे समुद्रं तर्तुमिच्छसि}
{सूर्याचन्द्रमसौ चोभौ पाणिभ्यां हर्तुमिच्छसि} %3-47-42

\twolineshloka
{यो रामस्य प्रियां भार्यां प्रधर्षयितुमिच्छसि}
{अग्निं प्रज्वलितं दृष्ट्वा वस्त्रेणाहर्तुमिच्छसि} %3-47-43

\threelineshloka
{कल्याणवृत्तां यो भार्यां रामस्याहर्तुमिच्छसि}
{अयोमुखानां शूलानामग्रे चरितुमिच्छसि}
{रामस्य सदृशीं भार्यां योऽधिगन्तुं त्वमिच्छसि} %3-47-44

\twolineshloka
{यदन्तरं सिंहसृगालयोर्वने यदन्तरं स्यन्दनिकासमुद्रयोः}
{सुराग्र्यसौवीरकयोर्यदन्तरं तदन्तरं दाशरथेस्तवैव च} %3-47-45

\twolineshloka
{यदन्तरं काञ्चनसीसलोहयोर्यदन्तरं चन्दनवारिपङ्कयोः}
{यदन्तरं हस्तिबिडालयोर्वने तदन्तरं दाशरथेस्तवैव च} %3-47-46

\twolineshloka
{यदन्तरं वायसवैनतेययोर्यदन्तरं मद्गुमयूरयोरपि}
{यदन्तरं हंसकगृध्रयोर्वने तदन्तरं दाशरथेस्तवैव च} %3-47-47

\twolineshloka
{तस्मिन् सहस्राक्षसमप्रभावे रामे स्थिते कार्मुकबाणपाणौ}
{हृतापि तेऽहं न जरां गमिष्ये आज्यं यथा मक्षिकयावगीर्णम्} %3-47-48

\twolineshloka
{इतीव तद्वाक्यमदुष्टभावा सुदुष्टमुक्त्वा रजनीचरं तम्}
{गात्रप्रकम्पाद् व्यथिता बभूव वातोद्धता सा कदलीव तन्वी} %3-47-49

\twolineshloka
{तां वेपमानामुपलक्ष्य सीतां स रावणो मृत्युसमप्रभावः}
{कुलं बलं नाम च कर्म चात्मनः समाचचक्षे भयकारणार्थम्} %3-47-50


॥इत्यार्षे श्रीमद्रामायणे वाल्मीकीये आदिकाव्ये अरण्यकाण्डे रावणाधिक्षेपः नाम सप्तचत्वारिंशः सर्गः ॥३-४७॥
