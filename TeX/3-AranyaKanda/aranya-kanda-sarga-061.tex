\sect{एकषष्ठितमः सर्गः — सीतान्वेषणम्}

\twolineshloka
{दृष्ट्वाऽऽश्रमपदं शून्यं रामो दशरथात्मजः}
{रहितां पर्णशालां च प्रविद्धान्यासनानि च} %3-61-1

\twolineshloka
{अदृष्ट्वा तत्र वैदेहीं संनिरीक्ष्य च सर्वशः}
{उवाच रामः प्राक्रुश्य प्रगृह्य रुचिरौ भुजौ} %3-61-2

\twolineshloka
{क्व नु लक्ष्मण वैदेही कं वा देशमितो गता}
{केनाहृता वा सौमित्रे भक्षिता केन वा प्रिया} %3-61-3

\twolineshloka
{वृक्षेणावार्य यदि मां सीते हसितुमिच्छसि}
{अलं ते हसितेनाद्य मां भजस्व सुदुःखितम्} %3-61-4

\twolineshloka
{यैः परिक्रीडसे सीते विश्वस्तैर्मृगपोतकैः}
{एते हीनास्त्वया सौम्ये ध्यायन्त्यस्राविलेक्षणाः} %3-61-5

\twolineshloka
{सीतया रहितोऽहं वै नहि जीवामि लक्ष्मण}
{वृतं शोकेन महता सीताहरणजेन माम्} %3-61-6

\twolineshloka
{परलोके महाराजो नूनं द्रक्ष्यति मे पिता}
{कथं प्रतिज्ञां संश्रुत्य मया त्वमभियोजितः} %3-61-7

\twolineshloka
{अपूरयित्वा तं कालं मत्सकाशमिहागतः}
{कामवृत्तमनार्यं वा मृषावादिनमेव च} %3-61-8

\twolineshloka
{धिक् त्वामिति परे लोके व्यक्तं वक्ष्यति मे पिता}
{विवशं शोकसंतप्तं दीनं भग्नमनोरथम्} %3-61-9

\twolineshloka
{मामिहोत्सृज्य करुणं कीर्तिर्नरमिवानृजुम्}
{क्व गच्छसि वरारोहे मा मोत्सृज सुमध्यमे} %3-61-10

\twolineshloka
{त्वया विरहितश्चाहं त्यक्ष्ये जीवितमात्मनः}
{इतीव विलपन् रामः सीतादर्शनलालसः} %3-61-11

\twolineshloka
{न ददर्श सुदुःखार्तो राघवो जनकात्मजाम्}
{अनासादयमानं तं सीतां शोकपरायणम्} %3-61-12

\twolineshloka
{पङ्कमासाद्य विपुलं सीदन्तमिव कुञ्जरम्}
{लक्ष्मणो राममत्यर्थमुवाच हितकाम्यया} %3-61-13

\twolineshloka
{मा विषादं महाबुद्धे कुरु यत्नं मया सह}
{इदं गिरिवरं वीर बहुकन्दरशोभितम्} %3-61-14

\twolineshloka
{प्रियकाननसंचारा वनोन्मत्ता च मैथिली}
{सा वनं वा प्रविष्टा स्यान्नलिनीं वा सुपुष्पिताम्} %3-61-15

\twolineshloka
{सरितं वापि सम्प्राप्ता मीनवञ्जुलसेविताम्}
{वित्रासयितुकामा वा लीना स्यात् कानने क्वचित्} %3-61-16

\twolineshloka
{जिज्ञासमाना वैदेही त्वां मां च पुरुषर्षभ}
{तस्या ह्यन्वेषणे श्रीमन् क्षिप्रमेव यतावहे} %3-61-17

\twolineshloka
{वनं सर्वं विचिनुवो यत्र सा जनकात्मजा}
{मन्यसे यदि काकुत्स्थ मा स्म शोके मनः कृथाः} %3-61-18

\twolineshloka
{एवमुक्तः स सौहार्दाल्लक्ष्मणेन समाहितः}
{सह सौमित्रिणा रामो विचेतुमुपचक्रमे} %3-61-19

\twolineshloka
{तौ वनानि गिरींश्चैव सरितश्च सरांसि च}
{निखिलेन विचिन्वन्तौ सीतां दशरथात्मजौ} %3-61-20

\twolineshloka
{तस्य शैलस्य सानूनि शिलाश्च शिखराणि च}
{निखिलेन विचिन्वन्तौ नैव तामभिजग्मतुः} %3-61-21

\twolineshloka
{विचित्य सर्वतः शैलं रामो लक्ष्मणमब्रवीत्}
{नेह पश्यामि सौमित्रे वैदेहीं पर्वते शुभाम्} %3-61-22

\twolineshloka
{ततो दुःखाभिसंतप्तो लक्ष्मणो वाक्यमब्रवीत्}
{विचरन् दण्डकारण्यं भ्रातरं दीप्ततेजसम्} %3-61-23

\twolineshloka
{प्राप्स्यसे त्वं महाप्राज्ञ मैथिलीं जनकात्मजाम्}
{यथा विष्णुर्महाबाहुर्बलिं बद्ध्वा महीमिमाम्} %3-61-24

\twolineshloka
{एवमुक्तस्तु वीरेण लक्ष्मणेन स राघवः}
{उवाच दीनया वाचा दुःखाभिहतचेतनः} %3-61-25

\threelineshloka
{वनं सुविचितं सर्वं पद्मिन्यः फुल्लपङ्कजाः}
{गिरिश्चायं महाप्राज्ञ बहुकन्दरनिर्झरः}
{नहि पश्यामि वैदेहीं प्राणेभ्योऽपि गरीयसीम्} %3-61-26

\twolineshloka
{एवं स विलपन् रामः सीताहरणकर्षितः}
{दीनः शोकसमाविष्टो मुहूर्तं विह्वलोऽभवत्} %3-61-27

\twolineshloka
{स विह्वलितसर्वाङ्गो गतबुद्धिर्विचेतनः}
{निषसादातुरो दीनो निःश्वस्याशीतमायतम्} %3-61-28

\twolineshloka
{बहुशः स तु निःश्वस्य रामो राजीवलोचनः}
{हा प्रियेति विचुक्रोश बहुशो बाष्पगद्गदः} %3-61-29

\twolineshloka
{तं सान्त्वयामास ततो लक्ष्मणः प्रियबान्धवम्}
{बहुप्रकारं शोकार्तः प्रश्रितः प्रश्रिताञ्जलिः} %3-61-30

\twolineshloka
{अनादृत्य तु तद् वाक्यं लक्ष्मणोष्ठपुटच्युतम्}
{अपश्यंस्तां प्रियां सीतां प्राक्रोशत् स पुनः पुनः} %3-61-31


॥इत्यार्षे श्रीमद्रामायणे वाल्मीकीये आदिकाव्ये अरण्यकाण्डे सीतान्वेषणम् नाम एकषष्ठितमः सर्गः ॥३-६१॥
