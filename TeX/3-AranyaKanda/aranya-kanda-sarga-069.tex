\sect{एकोनसप्ततितमः सर्गः — कबन्धग्राहः}

\twolineshloka
{कृत्वैवमुदकं तस्मै प्रस्थितौ राघवौ तदा}
{अवेक्षन्तौ वने सीतां जग्मतुः पश्चिमां दिशम्} %3-69-1

\twolineshloka
{तां दिशं दक्षिणां गत्वा शरचापासिधारिणौ}
{अविप्रहतमैक्ष्वाकौ पन्थानं प्रतिपेदतुः} %3-69-2

\twolineshloka
{गुल्मैर्वृक्षैश्च बहुभिर्लताभिश्च प्रवेष्टितम्}
{आवृतं सर्वतो दुर्गं गहनं घोरदर्शनम्} %3-69-3

\twolineshloka
{व्यतिक्रम्य तु वेगेन गृहीत्वा दक्षिणां दिशम्}
{सुभीमं तन्महारण्यं व्यतियातौ महाबलौ} %3-69-4

\twolineshloka
{ततः परं जनस्थानात् त्रिकोशं गम्य राघवौ}
{क्रौञ्चारण्यं विविशतुर्गहनं तौ महौजसौ} %3-69-5

\twolineshloka
{नानामेघघनप्रख्यं प्रहृष्टमिव सर्वतः}
{नानावर्णैः शुभैः पुष्पैर्मृगपक्षिगणैर्युतम्} %3-69-6

\twolineshloka
{दिदृक्षमाणौ वैदेहीं तद् वनं तौ विचिक्यतुः}
{तत्र तत्रावतिष्ठन्तौ सीताहरणदुःखितौ} %3-69-7

\twolineshloka
{ततः पूर्वेण तौ गत्वा त्रिक्रोशं भ्रातरौ तदा}
{क्रौञ्चारण्यमतिक्रम्य मतङ्गाश्रममन्तरे} %3-69-8

\twolineshloka
{दृष्ट्वा तु तद् वनं घोरं बहुभीममृगद्विजम्}
{नानावृक्षसमाकीर्णं सर्वं गहनपादपम्} %3-69-9

\twolineshloka
{ददृशाते गिरौ तत्र दरीं दशरथात्मजौ}
{पातालसमगम्भीरां तमसा नित्यसंवृताम्} %3-69-10

\twolineshloka
{आसाद्य च नरव्याघ्रौ दर्यास्तस्याविदूरतः}
{ददर्शतुर्महारूपां राक्षसीं विकृताननाम्} %3-69-11

\twolineshloka
{भयदामल्पसत्त्वानां बीभत्सां रौद्रदर्शनाम्}
{लम्बोदरीं तीक्ष्णदंष्ट्रां करालीं परुषत्वचम्} %3-69-12

\twolineshloka
{भक्षयन्तीं मृगान् भीमान् विकटां मुक्तमूर्धजाम्}
{अवैक्षतां तु तौ तत्र भ्रातरौ रामलक्ष्मणौ} %3-69-13

\twolineshloka
{सा समासाद्य तौ वीरौ व्रजन्तं भ्रातुरग्रतः}
{एहि रंस्यावहेत्युक्त्वा समालम्भत लक्ष्मणम्} %3-69-14

\twolineshloka
{उवाच चैनं वचनं सौमित्रिमुपगुह्य च}
{अहं त्वयोमुखी नाम लाभस्ते त्वमसि प्रियः} %3-69-15

\twolineshloka
{नाथ पर्वतदुर्गेषु नदीनां पुलिनेषु च}
{आयुश्चिरमिदं वीर त्वं मया सह रंस्यसे} %3-69-16

\twolineshloka
{एवमुक्तस्तु कुपितः खड्गमुद्धृत्य लक्ष्मणः}
{कर्णनासस्तनं तस्या निचकर्तारिसूदनः} %3-69-17

\twolineshloka
{कर्णनासे निकृत्ते तु विस्वरं विननाद सा}
{यथागतं प्रदुद्राव राक्षसी घोरदर्शना} %3-69-18

\twolineshloka
{तस्यां गतायां गहनं व्रजन्तौ वनमोजसा}
{आसेदतुरमित्रघ्नौ भ्रातरौ रामलक्ष्मणौ} %3-69-19

\twolineshloka
{लक्ष्मणस्तु महातेजाः सत्त्ववाञ्छीलवाञ्छुचिः}
{अब्रवीत् प्राञ्जलिर्वाक्यं भ्रातरं दीप्ततेजसम्} %3-69-20

\twolineshloka
{स्पन्दते मे दृढं बाहुरुद्विग्नमिव मे मनः}
{प्रायशश्चाप्यनिष्टानि निमित्तान्युपलक्षये} %3-69-21

\twolineshloka
{तस्मात् सज्जीभवार्य त्वं कुरुष्व वचनं मम}
{ममैव हि निमित्तानि सद्यः शंसन्ति सम्भ्रमम्} %3-69-22

\twolineshloka
{एष वञ्जुलको नाम पक्षी परमदारुणः}
{आवयोर्विजयं युद्धे शंसन्निव विनर्दति} %3-69-23

\twolineshloka
{तयोरन्वेषतोरेवं सर्वं तद् वनमोजसा}
{सञ्जज्ञे विपुलः शब्दः प्रभञ्जन्निव तद् वनम्} %3-69-24

\twolineshloka
{संवेष्टितमिवात्यर्थं गहनं मातरिश्वना}
{वनस्य तस्य शब्दोऽभूद् वनमापूरयन्निव} %3-69-25

\twolineshloka
{तं शब्दं काङ्क्षमाणस्तु रामः खड्गी सहानुजः}
{ददर्श सुमहाकायं राक्षसं विपुलोरसम्} %3-69-26

\twolineshloka
{आसेदतुश्च तद्रक्षस्तावुभौ प्रमुखे स्थितम्}
{विवृद्धमशिरोग्रीवं कबन्धमुदरेमुखम्} %3-69-27

\twolineshloka
{रोमभिर्निशितैस्तीक्ष्णैर्महागिरिमिवोच्छ्रितम्}
{नीलमेघनिभं रौद्रं मेघस्तनितनिःस्वनम्} %3-69-28

\twolineshloka
{अग्निज्वालानिकाशेन ललाटस्थेन दीप्यता}
{महापक्षेण पिङ्गेन विपुलेनायतेन च} %3-69-29

\twolineshloka
{एकेनोरसि घोरेण नयनेन सुदर्शिना}
{महादंष्ट्रोपपन्नं तं लेलिहानं महामुखम्} %3-69-30

\twolineshloka
{भक्षयन्तं महाघोरानृक्षसिंहमृगद्विजान्}
{घोरौ भुजौ विकुर्वाणमुभौ योजनमायतौ} %3-69-31

\twolineshloka
{कराभ्यां विविधान् गृह्य ऋक्षान् पक्षिगणान् मृगान्}
{आकर्षन्तं विकर्षन्तमनेकान् मृगयूथपान्} %3-69-32

\twolineshloka
{स्थितमावृत्य पन्थानं तयोर्भ्रात्रोः प्रपन्नयोः}
{अथ तं समतिक्रम्य क्रोशमात्रं ददर्शतुः} %3-69-33

\twolineshloka
{महान्तं दारुणं भीमं कबन्धं भुजसंवृतम्}
{कबन्धमिव संस्थानादतिघोरप्रदर्शनम्} %3-69-34

\twolineshloka
{स महाबाहुरत्यर्थं प्रसार्य विपुलौ भुजौ}
{जग्राह सहितावेव राघवौ पीडयन् बलात्} %3-69-35

\twolineshloka
{खड्गिनौ दृढधन्वानौ तिग्मतेजौ महाभुजौ}
{भ्रातरौ विवशं प्राप्तौ कृष्यमाणौ महाबलौ} %3-69-36

\twolineshloka
{तत्र धैर्याच्च शूरस्तु राघवो नैव विव्यथे}
{बाल्यादनाश्रयाच्चैव लक्ष्मणस्त्वभिविव्यथे} %3-69-37

\twolineshloka
{उवाच च विषण्णः सन् राघवं राघवानुजः}
{पश्य मां विवशं वीर राक्षसस्य वशङ्गतम्} %3-69-38

\twolineshloka
{मयैकेन तु निर्युक्तः परिमुच्यस्व राघव}
{मां हि भूतबलिं दत्त्वा पलायस्व यथासुखम्} %3-69-39

\twolineshloka
{अधिगन्तासि वैदेहीमचिरेणेति मे मतिः}
{प्रतिलभ्य च काकुत्स्थ पितृपैतामहीं महीम्} %3-69-40

\twolineshloka
{तत्र मां राम राज्यस्थः स्मर्तुमर्हसि सर्वदा}
{लक्ष्मणेनैवमुक्तस्तु रामः सौमित्रिमब्रवीत्} %3-69-41

\twolineshloka
{मा स्म त्रासं वृथा वीर नहि त्वादृग् विषीदति}
{एतस्मिन्नन्तरे क्रूरो भ्रातरौ रामलक्ष्मणौ} %3-69-42

\twolineshloka
{तावुवाच महाबाहुः कबन्धो दानवोत्तमः}
{कौ युवां वृषभस्कन्धौ महाखड्गधनुर्धरौ} %3-69-43

\twolineshloka
{घोरं देशमिमं प्राप्तौ दैवेन मम चाक्षुषौ}
{वदतं कार्यमिह वां किमर्थं चागतौ युवाम्} %3-69-44

\twolineshloka
{इमं देशमनुप्राप्तौ क्षुधार्तस्येह तिष्ठतः}
{सबाणचापखड्गौ च तीक्ष्णशृङ्गाविवर्षभौ} %3-69-45

\twolineshloka
{मां तूर्णमनुसम्प्राप्तौ दुर्लभं जीवितं हि वाम्}
{तस्य तद् वचनं श्रुत्वा कबन्धस्य दुरात्मनः} %3-69-46

\twolineshloka
{उवाच लक्ष्मणं रामो मुखेन परिशुष्यता}
{कृच्छ्रात् कृच्छ्रतरं प्राप्य दारुणं सत्यविक्रम} %3-69-47

\twolineshloka
{व्यसनं जीवितान्ताय प्राप्तमप्राप्य तां प्रियाम्}
{कालस्य सुमहद् वीर्यं सर्वभूतेषु लक्ष्मण} %3-69-48

\twolineshloka
{त्वां च मां च नरव्याघ्र व्यसनैः पश्य मोहितौ}
{नहि भारोऽस्ति दैवस्य सर्वभूतेषु लक्ष्मण} %3-69-49

\twolineshloka
{शूराश्च बलवन्तश्च कृतास्त्राश्च रणाजिरे}
{कालाभिपन्नाः सीदन्ति यथा वालुकसेतवः} %3-69-50

\twolineshloka
{इति ब्रुवाणो दृढसत्यविक्रमो महायशा दाशरथिः प्रतापवान्}
{अवेक्ष्य सौमित्रिमुदग्रविक्रमः स्थिरां तदा स्वां मतिमात्मनाकरोत्} %3-69-51


॥इत्यार्षे श्रीमद्रामायणे वाल्मीकीये आदिकाव्ये अरण्यकाण्डे कबन्धग्राहः नाम एकोनसप्ततितमः सर्गः ॥३-६९॥
