\sect{चतुर्विंशः सर्गः — रामखरबलसन्निकर्षः}

\twolineshloka
{आश्रमं प्रतियाते तु खरे खरपराक्रमे}
{तानेवौत्पातिकान् रामः सह भ्रात्रा ददर्श ह} %3-24-1

\twolineshloka
{तानुत्पातान् महाघोरान् रामो दृष्ट्वात्यमर्षणः}
{प्रजानामहितान् दृष्ट्वा वाक्यं लक्ष्मणमब्रवीत्} %3-24-2

\twolineshloka
{इमान् पश्य महाबाहो सर्वभूतापहारिणः}
{समुत्थितान् महोत्पातान् संहर्तुं सर्वराक्षसान्} %3-24-3

\twolineshloka
{अमी रुधिरधारास्तु विसृजन्ते खरस्वनाः}
{व्योम्नि मेघा निवर्तन्ते परुषा गर्दभारुणाः} %3-24-4

\twolineshloka
{सधूमाश्च शराः सर्वे मम युद्धाभिनन्दिताः}
{रुक्मपृष्ठानि चापानि विचेष्टन्ते विचक्षण} %3-24-5

\twolineshloka
{यादृशा इह कूजन्ति पक्षिणो वनचारिणः}
{अग्रतो नोऽभयं प्राप्तं संशयो जीवितस्य च} %3-24-6

\twolineshloka
{सम्प्रहारस्तु सुमहान् भविष्यति न संशयः}
{अयमाख्याति मे बाहुः स्फुरमाणो मुहुर्मुहुः} %3-24-7

\twolineshloka
{सन्निकर्षे तु नः शूर जयं शत्रोः पराजयम्}
{सुप्रभं च प्रसन्नं च तव वक्त्रं हि लक्ष्यते} %3-24-8

\twolineshloka
{उद्यतानां हि युद्धार्थं येषां भवति लक्ष्मण}
{निष्प्रभं वदनं तेषां भवत्यायुः परिक्षयः} %3-24-9

\twolineshloka
{रक्षसां नर्दतां घोरः श्रूयतेऽयं महाध्वनिः}
{आहतानां च भेरीणां राक्षसैः क्रूरकर्मभिः} %3-24-10

\twolineshloka
{अनागतविधानं तु कर्तव्यं शुभमिच्छता}
{आपदं शङ्कमानेन पुरुषेण विपश्चिता} %3-24-11

\twolineshloka
{तस्माद् गृहीत्वा वैदेहीं शरपाणिर्धनुर्धरः}
{गुहामाश्रय शैलस्य दुर्गां पादपसङ्कुलाम्} %3-24-12

\twolineshloka
{प्रतिकूलितुमिच्छामि न हि वाक्यमिदं त्वया}
{शापितो मम पादाभ्यां गम्यतां वत्स मा चिरम्} %3-24-13

\twolineshloka
{त्वं हि शूरश्च बलवान् हन्या एतान् न संशयः}
{स्वयं निहन्तुमिच्छामि सर्वानेव निशाचरान्} %3-24-14

\twolineshloka
{एवमुक्तस्तु रामेण लक्ष्मणः सह सीतया}
{शरानादाय चापं च गुहां दुर्गां समाश्रयत्} %3-24-15

\twolineshloka
{तस्मिन् प्रविष्टे तु गुहां लक्ष्मणे सह सीतया}
{हन्त निर्युक्तमित्युक्त्वा रामः कवचमाविशत्} %3-24-16

\twolineshloka
{स तेनाग्निनिकाशेन कवचेन विभूषितः}
{बभूव रामस्तिमिरे महानग्निरिवोत्थितः} %3-24-17

\twolineshloka
{स चापमुद्यम्य महच्छरानादाय वीर्यवान्}
{सम्बभूवास्थितस्तत्र ज्यास्वनैः पूरयन् दिशः} %3-24-18

\twolineshloka
{ततो देवाः सगन्धर्वाः सिद्धाश्च सह चारणैः}
{समेयुश्च महात्मानो युद्धदर्शनकाङ्क्षया} %3-24-19

\twolineshloka
{ऋषयश्च महात्मानो लोके ब्रह्मर्षिसत्तमाः}
{समेत्य चोचुः सहितास्तेऽन्योन्यं पुण्यकर्मणः} %3-24-20

\twolineshloka
{स्वस्ति गोब्राह्मणानां च लोकानां चेति संस्थिताः}
{जयतां राघवो युद्धे पौलस्त्यान् रजनीचरान्} %3-24-21

\twolineshloka
{चक्रहस्तो यथा युद्धे सर्वानसुरपुङ्गवान्}
{एवमुक्त्वा पुनः प्रोचुरालोक्य च परस्परम्} %3-24-22

\twolineshloka
{चतुर्दश सहस्राणि रक्षसां भीमकर्मणाम्}
{एकश्च रामो धर्मात्मा कथं युद्धं भविष्यति} %3-24-23

\twolineshloka
{इति राजर्षयः सिद्धाः सगणाश्च द्विजर्षभाः}
{जातकौतूहलास्तस्थुर्विमानस्थाश्च देवताः} %3-24-24

\twolineshloka
{आविष्टं तेजसा रामं सङ्ग्रामशिरसि स्थितम्}
{दृष्ट्वा सर्वाणि भूतानि भयाद् विव्यथिरे तदा} %3-24-25

\twolineshloka
{रूपमप्रतिमं तस्य रामस्याक्लिष्टकर्मणः}
{बभूव रूपं क्रुद्धस्य रुद्रस्येव महात्मनः} %3-24-26

\twolineshloka
{इति सम्भाष्यमाणे तु देवगन्धर्वचारणैः}
{ततो गम्भीरनिर्ह्रादं घोरचर्मायुधध्वजम्} %3-24-27

\twolineshloka
{अनीकं यातुधानानां समन्तात् प्रत्यपद्यत}
{वीरालापान् विसृजतामन्योन्यमभिगच्छताम्} %3-24-28

\twolineshloka
{चापानि विस्फारयतां जृम्भतां चाप्यभीक्ष्णशः}
{विप्रघुष्टस्वनानां च दुन्दुभींश्चापि निघ्नताम्} %3-24-29

\twolineshloka
{तेषां सुतुमुलः शब्दः पूरयामास तद् वनम्}
{तेन शब्देन वित्रस्ताः श्वापदा वनचारिणः} %3-24-30

\twolineshloka
{दुद्रुवुर्यत्र निःशब्दं पृष्ठतो नावलोकयन्}
{तच्चानीकं महावेगं रामं समनुवर्तत} %3-24-31

\twolineshloka
{धृतनानाप्रहरणं गम्भीरं सागरोपमम्}
{रामोऽपि चारयंश्चक्षुः सर्वतो रणपण्डितः} %3-24-32

\twolineshloka
{ददर्श खरसैन्यं तद् युद्धायाभिमुखो गतः}
{वितत्य च धनुर्भीमं तूण्याश्चोद्धृत्य सायकान्} %3-24-33

\twolineshloka
{क्रोधमाहारयत् तीव्रं वधार्थं सर्वरक्षसाम्}
{दुष्प्रेक्ष्यश्चाभवत् क्रुद्धो युगान्ताग्निरिव ज्वलन्} %3-24-34

\threelineshloka
{तं दृष्ट्वा तेजसाऽऽविष्टं प्राव्यथन् वनदेवताः}
{तस्य रुष्टस्य रूपं तु रामस्य ददृशे तदा}
{दक्षस्येव क्रतुं हन्तुमुद्यतस्य पिनाकिनः} %3-24-35

\twolineshloka
{तत्कार्मुकैराभरणै रथैश्च तद्वर्मभिश्चाग्निसमानवर्णैः}
{बभूव सैन्यं पिशिताशनानां सूर्योदये नीलमिवाभ्रजालम्} %3-24-36


॥इत्यार्षे श्रीमद्रामायणे वाल्मीकीये आदिकाव्ये अरण्यकाण्डे रामखरबलसन्निकर्षः नाम चतुर्विंशः सर्गः ॥३-२४॥
