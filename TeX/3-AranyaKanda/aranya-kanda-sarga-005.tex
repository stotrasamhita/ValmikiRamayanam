\sect{पञ्चमः सर्गः — शरभङ्गब्रह्मलोकप्रस्थानम्}

\twolineshloka
{हत्वा तु तं भीमबलं विराधं राक्षसं वने}
{ततः सीतां परिष्वज्य समाश्वास्य च वीर्यवान्} %3-5-1

\twolineshloka
{अब्रवीद् भ्रातरं रामो लक्ष्मणं दीप्ततेजसम्}
{कष्टं वनमिदं दुर्गं न च स्मो वनगोचराः} %3-5-2

\twolineshloka
{अभिगच्छामहे शीघ्रं शरभङ्गं तपोधनम्}
{आश्रमं शरभङ्गस्य राघवोऽभिजगाम ह} %3-5-3

\twolineshloka
{तस्य देवप्रभावस्य तपसा भावितात्मनः}
{समीपे शरभङ्गस्य ददर्श महदद्भुतम्} %3-5-4

\twolineshloka
{विभ्राजमानं वपुषा सूर्यवैश्वानरप्रभम्}
{रथप्रवरमारूढमाकाशे विबुधानुगम्} %3-5-5

\twolineshloka
{असंस्पृशन्तं वसुधां ददर्श विबुधेश्वरम्}
{सम्प्रभाभरणं देवं विरजोऽम्बरधारिणम्} %3-5-6

\twolineshloka
{तद्विधैरेव बहुभिः पूज्यमानं महात्मभिः}
{हरितैर्वाजिभिर्युक्तमन्तरिक्षगतं रथम्} %3-5-7

\twolineshloka
{ददर्शादूरतस्तस्य तरुणादित्यसंनिभम्}
{पाण्डुराभ्रघनप्रख्यं चन्द्रमण्डलसंनिभम्} %3-5-8

\twolineshloka
{अपश्यद् विमलं छत्रं चित्रमाल्योपशोभितम्}
{चामरव्यजने चाग्र्ये रुक्मदण्डे महाधने} %3-5-9

\twolineshloka
{गृहीते वरनारीभ्यां धूयमाने च मूर्धनि}
{गन्धर्वामरसिद्धाश्च बहवः परमर्षयः} %3-5-10

\twolineshloka
{अन्तरिक्षगतं देवं गीर्भिरग्र्या भिरैडयन्}
{सह सम्भाषमाणे तु शरभङ्गेन वासवे} %3-5-11

\twolineshloka
{दृष्ट्वा शतक्रतुं तत्र रामो लक्ष्मणमब्रवीत्}
{रामोऽथ रथमुद्दिश्य भ्रातुर्दर्शयताद्भुतम्} %3-5-12

\twolineshloka
{अर्चिष्मन्तं श्रिया जुष्टमद्भुतं पश्य लक्ष्मण}
{प्रतपन्तमिवादित्यमन्तरिक्षगतं रथम्} %3-5-13

\twolineshloka
{ये हयाः पुरुहूतस्य पुरा शक्रस्य नः श्रुताः}
{अन्तरिक्षगता दिव्यास्त इमे हरयो ध्रुवम्} %3-5-14

\twolineshloka
{इमे च पुरुषव्याघ्र ये तिष्ठन्त्यभितो दिशम्}
{शतं शतं कुण्डलिनो युवानः खड्गपाणयः} %3-5-15

\twolineshloka
{विस्तीर्णविपुलोरस्काः परिघायतबाहवः}
{शोणांशुवसनाः सर्वे व्याघ्रा इव दुरासदाः} %3-5-16

\twolineshloka
{उरोदेशेषु सर्वेषां हारा ज्वलनसंनिभाः}
{रूपं बिभ्रति सौमित्रे पञ्चविंशतिवार्षिकम्} %3-5-17

\twolineshloka
{एतद्धि किल देवानां वयो भवति नित्यदा}
{यथेमे पुरुषव्याघ्रा दृश्यन्ते प्रियदर्शनाः} %3-5-18

\twolineshloka
{इहैव सह वैदेह्या मुहूर्तं तिष्ठ लक्ष्मण}
{यावज्जानाम्यहं व्यक्तं क एष द्युतिमान् रथे} %3-5-19

\twolineshloka
{तमेवमुक्त्वा सौमित्रिमिहैव स्थीयतामिति}
{अभिचक्राम काकुत्स्थः शरभङ्गाश्रमं प्रति} %3-5-20

\twolineshloka
{ततः समभिगच्छन्तं प्रेक्ष्य रामं शचीपतिः}
{शरभङ्गमनुज्ञाप्य विबुधानिदमब्रवीत्} %3-5-21

\twolineshloka
{इहोपयात्यसौ रामो यावन्मां नाभिभाषते}
{निष्ठां नयत तावत् तु ततो माद्रष्टुमर्हति} %3-5-22

\twolineshloka
{जितवन्तं कृतार्थं हि तदाहमचिरादिमम्}
{कर्म ह्यनेन कर्तव्यं महदन्यैः सुदुष्करम्} %3-5-23

\twolineshloka
{अथ वज्री तमामन्त्र्य मानयित्वा च तापसम्}
{रथेन हययुक्तेन ययौ दिवमरिंदमः} %3-5-24

\twolineshloka
{प्रयाते तु सहस्राक्षे राघवः सपरिच्छदः}
{अग्निहोत्रमुपासीनं शरभङ्गमुपागमत्} %3-5-25

\twolineshloka
{तस्य पादौ च संगृह्य रामः सीता च लक्ष्मणः}
{निषेदुस्तदनुज्ञाता लब्धवासा निमन्त्रिताः} %3-5-26

\twolineshloka
{ततः शक्रोपयानं तु पर्यपृच्छत राघवः}
{शरभङ्गश्च तत् सर्वं राघवाय न्यवेदयत्} %3-5-27

\twolineshloka
{मामेष वरदो राम ब्रह्मलोकं निनीषति}
{जितमुग्रेण तपसा दुष्प्रापमकृतात्मभिः} %3-5-28

\twolineshloka
{अहं ज्ञात्वा नरव्याघ्र वर्तमानमदूरतः}
{ब्रह्मलोकं न गच्छामि त्वामदृष्ट्वा प्रियातिथिम्} %3-5-29

\twolineshloka
{त्वयाहं पुरुषव्याघ्र धार्मिकेण महात्मना}
{समागम्य गमिष्यामि त्रिदिवं चावरं परम्} %3-5-30

\twolineshloka
{अक्षया नरशार्दूल जिता लोका मया शुभाः}
{ब्राह्म्याश्च नाकपृष्ठ्याश्च प्रतिगृह्णीष्व मामकान्} %3-5-31

\twolineshloka
{एवमुक्तो नरव्याघ्रः सर्वशास्त्रविशारदः}
{ऋषिणा शरभङ्गेन राघवो वाक्यमब्रवीत्} %3-5-32

\twolineshloka
{अहमेवाहरिष्यामि सर्वाल्ँ लोकान् महामुने}
{आवासं त्वहमिच्छामि प्रदिष्टमिह कानने} %3-5-33

\twolineshloka
{राघवेणैवमुक्तस्तु शक्रतुल्यबलेन वै}
{शरभङ्गो महाप्राज्ञः पुनरेवाब्रवीद् वचः} %3-5-34

\twolineshloka
{इह राम महातेजाः सुतीक्ष्णो नाम धार्मिकः}
{वसत्यरण्ये नियतः स ते श्रेयो विधास्यति} %3-5-35

\twolineshloka
{सुतीक्ष्णमभिगच्छ त्वं शुचौ देशे तपस्विनम्}
{रमणीये वनोद्देशे स ते वासं विधास्यति} %3-5-36

\twolineshloka
{इमां मन्दाकिनीं राम प्रतिस्रोतामनुव्रज}
{नदीं पुष्पोडुपवहां ततस्तत्र गमिष्यसि} %3-5-37

\twolineshloka
{एष पन्था नरव्याघ्र मुहूर्तं पश्य तात माम्}
{यावज्जहामि गात्राणि जीर्णां त्वचमिवोरगः} %3-5-38

\twolineshloka
{ततोऽग्निं स समाधाय हुत्वा चाज्येन मन्त्रवत्}
{शरभङ्गो महातेजाः प्रविवेश हुताशनम्} %3-5-39

\twolineshloka
{तस्य रोमाणि केशांश्च तदा वह्निर्महात्मनः}
{जीर्णां त्वचं तदस्थीनि यच्च मांसं च शोणितम्} %3-5-40

\twolineshloka
{स च पावकसंकाशः कुमारः समपद्यत}
{उत्थायाग्निचयात् तस्माच्छरभङ्गो व्यरोचत} %3-5-41

\twolineshloka
{स लोकानाहिताग्नीनामृषीणां च महात्मनाम्}
{देवानां च व्यतिक्रम्य ब्रह्मलोकं व्यरोहत} %3-5-42

\twolineshloka
{स पुण्यकर्मा भुवने द्विजर्षभः पितामहं सानुचरं ददर्श ह}
{पितामहश्चापि समीक्ष्य तं द्विजं ननन्द सुस्वागतमित्युवाच ह} %3-5-43


॥इत्यार्षे श्रीमद्रामायणे वाल्मीकीये आदिकाव्ये अरण्यकाण्डे शरभङ्गब्रह्मलोकप्रस्थानम् नाम पञ्चमः सर्गः ॥३-५॥
