\sect{द्वितीयः सर्गः — विराधसंरोधः}

\twolineshloka
{कृतातिथ्योऽथ रामस्तु सूर्यस्योदयनं प्रति}
{आमन्त्र्य स मुनीन् सर्वान् वनमेवान्वगाहत} %3-2-1

\twolineshloka
{नानामृगगणाकीर्णमृक्षशार्दूलसेवितम्}
{ध्वस्तवृक्षलतागुल्मं दुर्दर्शसलिलाशयम्} %3-2-2

\twolineshloka
{निष्कूजमानशकुनिं झिल्लिकागणनादितम्}
{लक्ष्मणानुचरो रामो वनमध्यं ददर्श ह} %3-2-3

\twolineshloka
{सीतया सह काकुत्स्थस्तस्मिन् घोरमृगायुते}
{ददर्श गिरिशृङ्गाभं पुरुषादं महास्वनम्} %3-2-4

\twolineshloka
{गभीराक्षं महावक्त्रं विकटं विकटोदरम्}
{बीभत्सं विषमं दीर्घं विकृतं घोरदर्शनम्} %3-2-5

\twolineshloka
{वसानं चर्म वैयाघ्रं वसार्द्रं रुधिरोक्षितम्}
{त्रासनं सर्वभूतानां व्यादितास्यमिवान्तकम्} %3-2-6

\twolineshloka
{त्रीन् सिंहांश्चतुरो व्याघ्रान् द्वौ वृकौ पृषतान् दश}
{सविषाणं वसादिग्धं गजस्य च शिरो महत्} %3-2-7

\twolineshloka
{अवसज्यायसे शूले विनदन्तं महास्वनम्}
{स रामं लक्ष्मणं चैव सीतां दृष्ट्वा च मैथिलीम्} %3-2-8

\onelineshloka
{अभ्यधावत् सुसङ्क्रुद्धः प्रजाः काल इवान्तकः स कृत्वा भैरवं नादं चालयन्निव मेदिनीम्} %3-2-9

\twolineshloka
{अङ्केनादाय वैदेहीमपक्रम्य तदाब्रवीत्}
{युवां जटाचीरधरौ सभार्यौ क्षीणजीवितौ} %3-2-10

\twolineshloka
{प्रविष्टौ दण्डकारण्यं शरचापासिपाणिनौ}
{कथं तापसयोर्वां च वासः प्रमदया सह} %3-2-11

\twolineshloka
{अधर्मचारिणौ पापौ कौ युवां मुनिदूषकौ}
{अहं वनमिदं दुर्गं विराधो नाम राक्षसः} %3-2-12

\twolineshloka
{चरामि सायुधो नित्यमृषिमांसानि भक्षयन्}
{इयं नारी वरारोहा मम भार्या भविष्यति} %3-2-13

\twolineshloka
{युवयोः पापयोश्चाहं पास्यामि रुधिरं मृधे}
{तस्यैवं ब्रुवतो दुष्टं विराधस्य दुरात्मनः} %3-2-14

\twolineshloka
{श्रुत्वा सगर्वितं वाक्यं सम्भ्रान्ता जनकात्मजा}
{सीता प्रवेपितोद्वेगात् प्रवाते कदली यथा} %3-2-15

\twolineshloka
{तां दृष्ट्वा राघवः सीतां विराधाङ्कगतां शुभाम्}
{अब्रवील्लक्ष्मणं वाक्यं मुखेन परिशुष्यता} %3-2-16

\twolineshloka
{पश्य सौम्य नरेन्द्रस्य जनकस्यात्मसम्भवाम्}
{मम भार्यां शुभाचारां विराधाङ्के प्रवेशिताम्} %3-2-17

\twolineshloka
{अत्यन्तसुखसंवृद्धां राजपुत्रीं यशस्विनीम्}
{यदभिप्रेतमस्मासु प्रियं वरवृतं च यत्} %3-2-18

\twolineshloka
{कैकेय्यास्तु सुसंवृत्तं क्षिप्रमद्यैव लक्ष्मण}
{या न तुष्यति राज्येन पुत्रार्थे दीर्घदर्शिनी} %3-2-19

\twolineshloka
{ययाहं सर्वभूतानां प्रियः प्रस्थापितो वनम्}
{अद्येदानीं सकामा सा या माता मध्यमा मम} %3-2-20

\twolineshloka
{परस्पर्शात् तु वैदेह्या न दुःखतरमस्ति मे}
{पितुर्विनाशात् सौमित्रे स्वराज्य हरणात् तथा} %3-2-21

\twolineshloka
{इति ब्रुवति काकुत्स्थे बाष्पशोकपरिप्लुतः}
{अब्रवील्लक्ष्मणः क्रुद्धो रुद्धो नाग इव श्वसन्} %3-2-22

\twolineshloka
{अनाथ इव भूतानां नाथस्त्वं वासवोपमः}
{मया प्रेष्येण काकुत्स्थ किमर्थं परितप्यसे} %3-2-23

\twolineshloka
{शरेण निहतस्याद्य मया क्रुद्धेन रक्षसः}
{विराधस्य गतासोर्हि मही पास्यति शोणितम्} %3-2-24

\twolineshloka
{राज्यकामे मम क्रोधो भरते यो बभूव ह}
{तं विराधे विमोक्ष्यामि वज्री वज्रमिवाचले} %3-2-25

\twolineshloka
{मम भुजबलवेगवेगितः पततु शरोऽस्य महान् महोरसि}
{व्यपनयतु तनोश्च जीवितं पततु ततश्च महीं विघूर्णितः} %3-2-26


॥इत्यार्षे श्रीमद्रामायणे वाल्मीकीये आदिकाव्ये अरण्यकाण्डे विराधसंरोधः नाम द्वितीयः सर्गः ॥३-२॥
