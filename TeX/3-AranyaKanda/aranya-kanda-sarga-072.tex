\sect{द्विसप्ततितमः सर्गः — सीताधिगमोपायः}

\twolineshloka
{एवमुक्तौ तु तौ वीरौ कबन्धेन नरेश्वरौ}
{गिरिप्रदरमासाद्य पावकं विससर्जतुः} %3-72-1

\twolineshloka
{लक्ष्मणस्तु महोल्काभिर्ज्वलिताभिः समन्ततः}
{चितामादीपयामास सा प्रजज्वाल सर्वतः} %3-72-2

\twolineshloka
{तच्छरीरं कबन्धस्य घृतपिण्डोपमं महत्}
{मेदसा पच्यमानस्य मन्दं दहत पावकः} %3-72-3

\twolineshloka
{सविधूय चितामाशु विधूमोऽग्निरिवोत्थितः}
{अरजे वाससी बिभ्रन्माल्यं दिव्यं महाबलः} %3-72-4

\twolineshloka
{ततश्चिताया वेगेन भास्वरो विरजाम्बरः}
{उत्पपाताशु संहृष्टः सर्वप्रत्यङ्गभूषणः} %3-72-5

\twolineshloka
{विमाने भास्वरे तिष्ठन् हंसयुक्ते यशस्करे}
{प्रभया च महातेजा दिशो दश विराजयन्} %3-72-6

\twolineshloka
{सोऽन्तरिक्षगतो वाक्यं कबन्धो राममब्रवीत्}
{शृणु राघव तत्त्वेन यथा सीतामवाप्स्यसि} %3-72-7

\twolineshloka
{राम षड् युक्तयो लोके याभिः सर्वं विमृश्यते}
{परिमृष्टो दशान्तेन दशाभागेन सेव्यते} %3-72-8

\twolineshloka
{दशाभागगतो हीनस्त्वं हि राम सलक्ष्मणः}
{यत्कृते व्यसनं प्राप्तं त्वया दारप्रधर्षणम्} %3-72-9

\twolineshloka
{तदवश्यं त्वया कार्यः स सुहृत् सुहृदां वर}
{अकृत्वा नहि ते सिद्धिमहं पश्यामि चिन्तयन्} %3-72-10

\twolineshloka
{श्रूयतां राम वक्ष्यामि सुग्रीवो नाम वानरः}
{भ्रात्रा निरस्तः क्रुद्धेन वालिना शक्रसूनुना} %3-72-11

\twolineshloka
{ऋष्यमूके गिरिवरे पम्पापर्यन्तशोभिते}
{निवसत्यात्मवान् वीरश्चतुर्भिः सह वानरैः} %3-72-12

\twolineshloka
{वानरेन्द्रो महावीर्यस्तेजोवानमितप्रभः}
{सत्यसन्धो विनीतश्च धृतिमान् मतिमान् महान्} %3-72-13

\twolineshloka
{दक्षः प्रगल्भो द्युतिमान् महाबलपराक्रमः}
{भ्रात्रा विवासितो वीर राज्यहेतोर्महात्मना} %3-72-14

\twolineshloka
{स ते सहायो मित्रं च सीतायाः परिमार्गणे}
{भविष्यति हि ते राम मा च शोके मनः कृथाः} %3-72-15

\twolineshloka
{भवितव्यं हि तच्चापि न तच्छक्यमिहान्यथा}
{कर्तुमिक्ष्वाकुशार्दूल कालो हि दुरतिक्रमः} %3-72-16

\twolineshloka
{गच्छ शीघ्रमितो वीर सुग्रीवं तं महाबलम्}
{वयस्यं तं कुरु क्षिप्रमितो गत्वाद्य राघव} %3-72-17

\twolineshloka
{अद्रोहाय समागम्य दीप्यमाने विभावसौ}
{न च ते सोऽवमन्तव्यः सुग्रीवो वानराधिपः} %3-72-18

\twolineshloka
{कृतज्ञः कामरूपी च सहायार्थी च वीर्यवान्}
{शक्तौ ह्यद्य युवां कर्तुं कार्यं तस्य चिकीर्षितम्} %3-72-19

\twolineshloka
{कृतार्थो वाकृतार्थो वा तव कृत्यं करिष्यति}
{स ऋक्षरजसः पुत्रः पम्पामटति शङ्कितः} %3-72-20

\twolineshloka
{भास्करस्यौरसः पुत्रो वालिना कृतकिल्बिषः}
{सन्निधायायुधं क्षिप्रमृष्यमूकालयं कपिम्} %3-72-21

\twolineshloka
{कुरु राघव सत्येन वयस्यं वनचारिणम्}
{स हि स्थानानि कात्स्न्र्येन सर्वाणि कपिकुञ्जरः} %3-72-22

\twolineshloka
{नरमांसाशिनां लोके नैपुण्यादधिगच्छति}
{न तस्याविदितं लोके किञ्चिदस्ति हि राघव} %3-72-23

\twolineshloka
{यावत् सूर्यः प्रतपति सहस्रांशुः परन्तप}
{स नदीर्विपुलान् शैलान् गिरिदुर्गाणि कन्दरान्} %3-72-24

\twolineshloka
{अन्विष्य वानरैः सार्धं पत्नीं तेऽधिगमिष्यति}
{वानरांश्च महाकायान् प्रेषयिष्यति राघव} %3-72-25

\twolineshloka
{दिशो विचेतुं तां सीतां त्वद्वियोगेन शोचतीम्}
{अन्वेष्यति वरारोहां मैथिलीं रावणालये} %3-72-26

\twolineshloka
{स मेरुशृङ्गाग्रगतामनिन्दितां प्रविश्य पातालतलेऽपि वाश्रिताम्}
{प्लवङ्गमानामृषभस्तव प्रियां निहत्य रक्षांसि पुनः प्रदास्यति} %3-72-27


॥इत्यार्षे श्रीमद्रामायणे वाल्मीकीये आदिकाव्ये अरण्यकाण्डे सीताधिगमोपायः नाम द्विसप्ततितमः सर्गः ॥३-७२॥
