\sect{द्विपञ्चाशः सर्गः — सीताविक्रोशः}

\twolineshloka
{सा तु ताराधिपमुखी रावणेन निरीक्ष्य तम्}
{गृध्रराजं विनिहतं विललाप सुदुःखिता} %3-52-1

\twolineshloka
{निमित्तं लक्षणं स्वप्नं शकुनिस्वरदर्शनम्}
{अवश्यं सुखदुःखेषु नराणां परिदृश्यते} %3-52-2

\twolineshloka
{न नूनं राम जानासि महद्व्यसनमात्मनः}
{धावन्ति नूनं काकुत्स्थ मदर्थं मृगपक्षिणः} %3-52-3

\twolineshloka
{अयं हि कृपया राम मां त्रातुमिह सङ्गतः}
{शेते विनिहतो भूमौ ममाभाग्याद् विहङ्गमः} %3-52-4

\twolineshloka
{त्राहि मामद्य काकुत्स्थ लक्ष्मणेति वराङ्गना}
{सुसन्त्रस्ता समाक्रन्दच्छृण्वतां तु यथान्तिके} %3-52-5

\twolineshloka
{तां क्लिष्टमाल्याभरणां विलपन्तीमनाथवत्}
{अभ्यधावत वैदेहीं रावणो राक्षसाधिपः} %3-52-6

\twolineshloka
{तां लतामिव वेष्टन्तीमालिङ्गन्तीं महाद्रुमान्}
{मुञ्च मुञ्चेति बहुशः प्राप तां राक्षसाधिपः} %3-52-7

\twolineshloka
{क्रोशन्तीं राम रामेति रामेण रहितां वने}
{जीवितान्ताय केशेषु जग्राहान्तकसन्निभः} %3-52-8

\twolineshloka
{प्रधर्षितायां वैदेह्यां बभूव सचराचरम्}
{जगत् सर्वममर्यादं तमसान्धेन संवृतम्} %3-52-9

\twolineshloka
{न वाति मारुतस्तत्र निष्प्रभोऽभूद् दिवाकरः}
{दृष्ट्वा सीतां परामृष्टां देवो दिव्येन चक्षुषा} %3-52-10

\twolineshloka
{कृतं कार्यमिति श्रीमान् व्याजहार पितामहः}
{प्रहृष्टा व्यथिताश्चासन् सर्वे ते परमर्षयः} %3-52-11

\twolineshloka
{दृष्ट्वा सीतां परामृष्टां दण्डकारण्यवासिनः}
{रावणस्य विनाशं च प्राप्तं बुद्ध्वा यदृच्छया} %3-52-12

\twolineshloka
{स तु तां राम रामेति रुदतीं लक्ष्मणेति च}
{जगामादाय चाकाशं रावणो राक्षसेश्वरः} %3-52-13

\twolineshloka
{तप्ताभरणवर्णाङ्गी पीतकौशेयवासिनी}
{रराज राजपुत्री तु विद्युत्सौदामनी यथा} %3-52-14

\twolineshloka
{उद्धूतेन च वस्त्रेण तस्याः पीतेन रावणः}
{अधिकं परिबभ्राज गिरिर्दीप्त इवाग्निना} %3-52-15

\twolineshloka
{तस्याः परमकल्याण्यास्ताम्राणि सुरभीणि च}
{पद्मपत्राणि वैदेह्या अभ्यकीर्यन्त रावणम्} %3-52-16

\twolineshloka
{तस्याः कौशेयमुद्धूतमाकाशे कनकप्रभम्}
{बभौ चादित्यरागेण ताम्रमभ्रमिवातपे} %3-52-17

\twolineshloka
{तस्यास्तद् विमलं वक्त्रमाकाशे रावणाङ्कगम्}
{न रराज विना रामं विनालमिव पङ्कजम्} %3-52-18

\twolineshloka
{बभूव जलदं नीलं भित्त्वा चन्द्र इवोदितः}
{सुललाटं सुकेशान्तं पद्मगर्भाभमव्रणम्} %3-52-19

\twolineshloka
{शुक्लैः सुविमलैर्दन्तैः प्रभावद्भिरलङ्कृतम्}
{तस्याः सुनयनं वक्त्रमाकाशे रावणाङ्कगम्} %3-52-20

\twolineshloka
{रुदितं व्यपमृष्टास्रं चन्द्रवत् प्रियदर्शनम्}
{सुनासं चारुताम्रोष्ठमाकाशे हाटकप्रभम्} %3-52-21

\twolineshloka
{राक्षसेन्द्रसमाधूतं तस्यास्तद् वदनं शुभम्}
{शुशुभे न विना रामं दिवा चन्द्र इवोदितः} %3-52-22

\twolineshloka
{सा हेमवर्णा नीलाङ्गं मैथिली राक्षसाधिपम्}
{शुशुभे काञ्चनी काञ्ची नीलं गजमिवाश्रिता} %3-52-23

\twolineshloka
{सा पद्मपीता हेमाभा रावणं जनकात्मजा}
{विद्युद् घनमिवाविश्य शुशुभे तप्तभूषणा} %3-52-24

\twolineshloka
{तस्या भूषणघोषेण वैदेह्या राक्षसेश्वरः}
{बभूव विमलो नीलः सघोष इव तोयदः} %3-52-25

\twolineshloka
{उत्तमाङ्गच्युता तस्याः पुष्पवृष्टिः समन्ततः}
{सीताया ह्रियमाणायाः पपात धरणीतले} %3-52-26

\twolineshloka
{सा तु रावणवेगेन पुष्पवृष्टिः समन्ततः}
{समाधूता दशग्रीवं पुनरेवाभ्यवर्तत} %3-52-27

\twolineshloka
{अभ्यवर्तत पुष्पाणां धारा वैश्रवणानुजम्}
{नक्षत्रमाला विमला मेरुं नगमिवोन्नतम्} %3-52-28

\twolineshloka
{चरणान्नूपुरं भ्रष्टं वैदेह्या रत्नभूषितम्}
{विद्युन्मण्डलसङ्काशं पपात धरणीतले} %3-52-29

\twolineshloka
{तरुप्रवालरक्ता सा नीलाङ्गं राक्षसेश्वरम्}
{प्रशोभयत वैदेही गजं कक्ष्येव काञ्चनी} %3-52-30

\twolineshloka
{तां महोल्कामिवाकाशे दीप्यमानां स्वतेजसा}
{जहाराकाशमाविश्य सीतां वैश्रवणानुजः} %3-52-31

\twolineshloka
{तस्यास्तान्यग्निवर्णानि भूषणानि महीतले}
{सघोषाण्यवशीर्यन्त क्षीणास्तारा इवाम्बरात्} %3-52-32

\twolineshloka
{तस्याः स्तनान्तराद् भ्रष्टो हारस्ताराधिपद्युतिः}
{वैदेह्या निपतन् भाति गङ्गेव गगनच्युता} %3-52-33

\twolineshloka
{उत्पातवाताभिरता नानाद्विजगणायुताः}
{मा भैरिति विधूताग्रा व्याजह्रुरिव पादपाः} %3-52-34

\twolineshloka
{नलिन्यो ध्वस्तकमलास्त्रस्तमीनजलेचराः}
{सखीमिव गतोत्साहां शोचन्तीव स्म मैथिलीम्} %3-52-35

\twolineshloka
{समन्तादभिसम्पत्य सिंहव्याघ्रमृगद्विजाः}
{अन्वधावंस्तदा रोषात् सीताच्छायानुगामिनः} %3-52-36

\twolineshloka
{जलप्रपातास्रमुखाः शृङ्गैरुच्छ्रितबाहुभिः}
{सीतायां ह्रियमाणायां विक्रोशन्तीव पर्वताः} %3-52-37

\twolineshloka
{ह्रियमाणां तु वैदेहीं दृष्ट्वा दीनो दिवाकरः}
{प्रविध्वस्तप्रभः श्रीमानासीत् पाण्डुरमण्डलः} %3-52-38

\twolineshloka
{नास्ति धर्मः कुतः सत्यं नार्जवं नानृशंसता}
{यत्र रामस्य वैदेहीं सीतां हरति रावणः} %3-52-39

\twolineshloka
{इति भूतानि सर्वाणि गणशः पर्यदेवयन्}
{वित्रस्तका दीनमुखा रुरुदुर्मृगपोतकाः} %3-52-40

\twolineshloka
{उद्वीक्ष्योद्वीक्ष्य नयनैर्भयादिव विलक्षणैः}
{सुप्रवेपितगात्राश्च बभूवुर्वनदेवताः} %3-52-41

\twolineshloka
{विक्रोशन्तीं दृढं सीतां दृष्ट्वा दुःखं तथा गताम्}
{तां तु लक्ष्मण रामेति क्रोशन्तीं मधुरस्वराम्} %3-52-42

\threelineshloka
{अवेक्षमाणां बहुशो वैदेहीं धरणीतलम्}
{स तामाकुलकेशान्तां विप्रमृष्टविशेषकाम्}
{जहारात्मविनाशाय दशग्रीवो मनस्विनीम्} %3-52-43

\twolineshloka
{ततस्तु सा चारुदती शुचिस्मिता विनाकृता बन्धुजनेन मैथिली}
{अपश्यती राघवलक्ष्मणावुभौ विवर्णवक्त्रा भयभारपीडिता} %3-52-44


॥इत्यार्षे श्रीमद्रामायणे वाल्मीकीये आदिकाव्ये अरण्यकाण्डे सीताविक्रोशः नाम द्विपञ्चाशः सर्गः ॥३-५२॥
