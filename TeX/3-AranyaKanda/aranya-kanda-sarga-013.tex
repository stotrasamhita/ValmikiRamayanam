\sect{त्रयोदशः सर्गः — पञ्चवटीगमनम्}

\twolineshloka
{राम प्रीतोऽस्मि भद्रं ते परितुष्टोऽस्मि लक्ष्मण}
{अभिवादयितुं यन्मां प्राप्तौ स्थः सह सीतया} %3-13-1

\twolineshloka
{अध्वश्रमेण वां खेदो बाधते प्रचुरश्रमः}
{व्यक्तमुत्कण्ठते वापि मैथिली जनकात्मजा} %3-13-2

\twolineshloka
{एषा च सुकुमारी च खेदैश्च न विमानिता}
{प्राज्यदोषं वनं प्राप्ता भर्तृस्नेहप्रचोदिता} %3-13-3

\twolineshloka
{यथैषा रमते राम इह सीता तथा कुरु}
{दुष्करं कृतवत्येषा वने त्वामभिगच्छती} %3-13-4

\twolineshloka
{एषा हि प्रकृतिः स्त्रीणामासृष्टे रघुनन्दन}
{समस्थमनुरज्यन्ते विषमस्थं त्यजन्ति च} %3-13-5

\twolineshloka
{शतह्रदानां लोलत्वं शस्त्राणां तीक्ष्णतां तथा}
{गरुडानिलयोः शैघ्र्यमनुगच्छन्ति योषितः} %3-13-6

\twolineshloka
{इयं तु भवतो भार्या दोषैरेतैर्विवर्जिता}
{श्लाघ्या च व्यपदेश्या च यथा देवीष्वरुन्धती} %3-13-7

\twolineshloka
{अलङ्कृतोऽयं देशश्च यत्र सौमित्रिणा सह}
{वैदेह्या चानया राम वत्स्यसि त्वमरिन्दम} %3-13-8

\twolineshloka
{एवमुक्तस्तु मुनिना राघवः संयताञ्जलिः}
{उवाच प्रश्रितं वाक्यमृषिं दीप्तमिवानलम्} %3-13-9

\twolineshloka
{धन्योऽस्म्यनुगृहीतोऽस्मि यस्य मे मुनिपुङ्गवः}
{गुणैः सभ्रातृभार्यस्य गुरुर्नः परितुष्यति} %3-13-10

\twolineshloka
{किं तु व्यादिश मे देशं सोदकं बहुकाननम्}
{यत्राश्रमपदं कृत्वा वसेयं निरतः सुखम्} %3-13-11

\twolineshloka
{ततोऽब्रवीन्मुनिश्रेष्ठः श्रुत्वा रामस्य भाषितम्}
{ध्यात्वा मुहूर्तं धर्मात्मा ततोवाच वचः शुभम्} %3-13-12

\twolineshloka
{इतो द्वियोजने तात बहुमूलफलोदकः}
{देशो बहुमृगः श्रीमान् पञ्चवट्यभिविश्रुतः} %3-13-13

\twolineshloka
{तत्र गत्वाऽऽश्रमपदं कृत्वा सौमित्रिणा सह}
{रमस्व त्वं पितुर्वाक्यं यथोक्तमनुपालयन्} %3-13-14

\twolineshloka
{विदितो ह्येष वृत्तान्तो मम सर्वस्तवानघ}
{तपसश्च प्रभावेण स्नेहाद् दशरथस्य च} %3-13-15

\twolineshloka
{हृदयस्थं च ते च्छन्दो विज्ञातं तपसा मया}
{इह वासं प्रतिज्ञाय मया सह तपोवने} %3-13-16

\twolineshloka
{अतश्च त्वामहं ब्रूमि गच्छ पञ्चवटीमिति}
{स हि रम्यो वनोद्देशो मैथिली तत्र रंस्यते} %3-13-17

\twolineshloka
{स देशः श्लाघनीयश्च नातिदूरे च राघव}
{गोदावर्याः समीपे च मैथिली तत्र रंस्यते} %3-13-18

\twolineshloka
{प्राज्यमूलफलैश्चैव नानाद्विजगणैर्युतः}
{विविक्तश्च महाबाहो पुण्यो रम्यस्तथैव च} %3-13-19

\twolineshloka
{भवानपि सदाचारः शक्तश्च परिरक्षणे}
{अपि चात्र वसन् राम तापसान् पालयिष्यसि} %3-13-20

\twolineshloka
{एतदालक्ष्यते वीर मधूकानां महावनम्}
{उत्तरेणास्य गन्तव्यं न्यग्रोधमपि गच्छता} %3-13-21

\twolineshloka
{ततः स्थलमुपारुह्य पर्वतस्याविदूरतः}
{ख्यातः पञ्चवटीत्येव नित्यपुष्पितकाननः} %3-13-22

\twolineshloka
{अगस्त्येनैवमुक्तस्तु रामः सौमित्रिणा सह}
{सत्कृत्यामन्त्रयामास तमृषिं सत्यवादिनम्} %3-13-23

\twolineshloka
{तौ तु तेनाभ्यनुज्ञातौ कृतपादाभिवन्दनौ}
{तमाश्रमं पञ्चवटीं जग्मतुः सह सीतया} %3-13-24

\twolineshloka
{गृहीतचापौ तु नराधिपात्मजौ विषक्ततूणी समरेष्वकातरौ}
{यथोपदिष्टेन पथा महर्षिणा प्रजग्मतुः पञ्चवटीं समाहितौ} %3-13-25


॥इत्यार्षे श्रीमद्रामायणे वाल्मीकीये आदिकाव्ये अरण्यकाण्डे पञ्चवटीगमनम् नाम त्रयोदशः सर्गः ॥३-१३॥
