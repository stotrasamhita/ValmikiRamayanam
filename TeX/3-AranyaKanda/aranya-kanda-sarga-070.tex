\sect{सप्ततितमः सर्गः — कबन्धबाहुच्छेदः}

\twolineshloka
{तौ तु तत्र स्थितौ दृष्ट्वा भ्रातरौ रामलक्ष्मणौ}
{बाहुपाशपरिक्षिप्तौ कबन्धो वाक्यमब्रवीत्} %3-70-1

\twolineshloka
{तिष्ठतः किं नु मां दृष्ट्वा क्षुधार्तं क्षत्रियर्षभौ}
{आहारार्थं तु संदिष्टौ दैवेन हतचेतनौ} %3-70-2

\twolineshloka
{तच्छ्रुत्वा लक्ष्मणो वाक्यं प्राप्तकालं हितं तदा}
{उवाचार्तिसमापन्नो विक्रमे कृतनिश्चयः} %3-70-3

\twolineshloka
{त्वां च मां च पुरा तूर्णमादत्ते राक्षसाधमः}
{तस्मादसिभ्यामस्याशु बाहू छिन्दावहे गुरू} %3-70-4

\twolineshloka
{भीषणोऽयं महाकायो राक्षसा भुजविक्रमः}
{लोकं ह्यतिजितं कृत्वा ह्यावां हन्तुमिहेच्छति} %3-70-5

\twolineshloka
{निश्चेष्टानां वधो राजन् कुत्सितो जगतीपतेः}
{क्रतुमध्योपनीतानां पशूनामिव राघव} %3-70-6

\twolineshloka
{एतत् संजल्पितं श्रुत्वा तयोः क्रुद्धस्तु राक्षसः}
{विदार्यास्यं ततो रौद्रं तौ भक्षयितुमारभत्} %3-70-7

\twolineshloka
{ततस्तौ देशकालज्ञौ खड्गाभ्यामेव राघवौ}
{अच्छिन्दन्तां सुसंहृष्टौ बाहू तस्यांसदेशतः} %3-70-8

\twolineshloka
{दक्षिणो दक्षिणं बाहुमसक्तमसिना ततः}
{चिच्छेद रामो वेगेन सव्यं वीरस्तु लक्ष्मणः} %3-70-9

\twolineshloka
{स पपात महाबाहुश्छिन्नबाहुर्महास्वनः}
{खं च गां च दिशश्चैव नादयञ्जलदो यथा} %3-70-10

\twolineshloka
{स निकृत्तौ भुजौ दृष्ट्वा शोणितौघपरिप्लुतः}
{दीनः पप्रच्छ तौ वीरौ कौ युवामिति दानवः} %3-70-11

\twolineshloka
{इति तस्य ब्रुवाणस्य लक्ष्मणः शुभलक्षणः}
{शशंस तस्य काकुस्त्थं कबन्धस्य महाबलः} %3-70-12

\twolineshloka
{अयमिक्ष्वाकुदायादो रामो नाम जनैः श्रुतः}
{तस्यैवावरजं विद्धि भ्रातरं मां च लक्ष्मणम्} %3-70-13

\twolineshloka
{मात्रा प्रतिहते राज्ये रामः प्रव्राजितो वनम्}
{मया सह चरत्येष भार्यया च महद् वनम्} %3-70-14

\twolineshloka
{अस्य देवप्रभावस्य वसतो विजने वने}
{रक्षसापहृता भार्या यामिच्छन्ताविहागतौ} %3-70-15

\twolineshloka
{त्वं तु को वा किमर्थं वा कबन्धसदृशो वने}
{आस्येनोरसि दीप्तेन भग्नजङ्घो विचेष्टसे} %3-70-16

\twolineshloka
{एवमुक्तः कबन्धस्तु लक्ष्मणेनोत्तरं वचः}
{उवाच वचनं प्रीतस्तदिन्द्रवचनं स्मरन्} %3-70-17

\twolineshloka
{स्वागतं वां नरव्याघ्रौ दिष्ट्या पश्यामि वामहम्}
{दिष्ट्या चेमौ निकृत्तौ मे युवाभ्यां बाहुबन्धनौ} %3-70-18

\twolineshloka
{विरूपं यच्च मे रूपं प्राप्तं ह्यविनयाद् यथा}
{तन्मे शृणु नरव्याघ्र तत्त्वतः शंसतस्तव} %3-70-19


॥इत्यार्षे श्रीमद्रामायणे वाल्मीकीये आदिकाव्ये अरण्यकाण्डे कबन्धबाहुच्छेदः नाम सप्ततितमः सर्गः ॥३-७०॥
