\sect{अष्टाविंशः सर्गः — खररामसम्प्रहारः}

\twolineshloka
{निहतं दूषणं दृष्ट्वा रणे त्रिशिरसा सह}
{खरस्याप्यभवत् त्रासो दृष्ट्वा रामस्य विक्रमम्} %3-28-1

\twolineshloka
{स दृष्ट्वा राक्षसं सैन्यमविषह्यं महाबलम्}
{हतमेकेन रामेण दूषणस्त्रिशिरा अपि} %3-28-2

\twolineshloka
{तद्बलं हतभूयिष्ठं विमनाः प्रेक्ष्य राक्षसः}
{आससाद खरो रामं नमुचिर्वासवं यथा} %3-28-3

\twolineshloka
{विकृष्य बलवच्चापं नाराचान् रक्तभोजनान्}
{खरश्चिक्षेप रामाय क्रुद्धानाशीविषानिव} %3-28-4

\twolineshloka
{ज्यां विधुन्वन् सुबहुशः शिक्षयास्त्राणि दर्शयन्}
{चचार समरे मार्गान् शरै रथगतः खरः} %3-28-5

\twolineshloka
{स सर्वाश्च दिशो बाणैः प्रदिशश्च महारथः}
{पूरयामास तं दृष्ट्वा रामोऽपि सुमहद् धनुः} %3-28-6

\twolineshloka
{स सायकैर्दुर्विषहैर्विस्फुलिङ्गैरिवाग्निभिः}
{नभश्चकाराविवरं पर्जन्य इव वृष्टिभिः} %3-28-7

\twolineshloka
{तद् बभूव शितैर्बाणैः खररामविसर्जितैः}
{पर्याकाशमनाकाशं सर्वतः शरसङ्कुलम्} %3-28-8

\twolineshloka
{शरजालावृतः सूर्यो न तदा स्म प्रकाशते}
{अन्योन्यवधसंरम्भादुभयोः सम्प्रयुध्यतोः} %3-28-9

\twolineshloka
{ततो नालीकनाराचैस्तीक्ष्णाग्रैश्च विकर्णिभिः}
{आजघान रणे रामं तोत्रैरिव महाद्विपम्} %3-28-10

\twolineshloka
{तं रथस्थं धनुष्पाणिं राक्षसं पर्यवस्थितम्}
{ददृशुः सर्वभूतानि पाशहस्तमिवान्तकम्} %3-28-11

\twolineshloka
{हन्तारं सर्वसैन्यस्य पौरुषे पर्यवस्थितम्}
{परिश्रान्तं महासत्त्वं मेने रामं खरस्तदा} %3-28-12

\twolineshloka
{तं सिंहमिव विक्रान्तं सिंहविक्रान्तगामिनम्}
{दृष्ट्वा नोद्विजते रामः सिंहः क्षुद्रमृगं यथा} %3-28-13

\twolineshloka
{ततः सूर्यनिकाशेन रथेन महता खरः}
{आससादाथ तं रामं पतङ्ग इव पावकम्} %3-28-14

\twolineshloka
{ततोऽस्य सशरं चापं मुष्टिदेशे महात्मनः}
{खरश्चिच्छेद रामस्य दर्शयन् हस्तलाघवम्} %3-28-15

\twolineshloka
{स पुनस्त्वपरान् सप्त शरानादाय मर्मणि}
{निजघान रणे क्रुद्धः शक्राशनिसमप्रभान्} %3-28-16

\twolineshloka
{ततः शरसहस्रेण राममप्रतिमौजसम्}
{अर्दयित्वा महानादं ननाद समरे खरः} %3-28-17

\twolineshloka
{ततस्तत्प्रहतं बाणैः खरमुक्तैः सुपर्वभिः}
{पपात कवचं भूमौ रामस्यादित्यवर्चसम्} %3-28-18

\twolineshloka
{स शरैरर्पितः क्रुद्धः सर्वगात्रेषु राघवः}
{रराज समरे रामो विधूमोऽग्निरिव ज्वलन्} %3-28-19

\twolineshloka
{ततो गम्भीरनिर्ह्रादं रामः शत्रुनिबर्हणः}
{चकारान्ताय स रिपोः सज्यमन्यन्महद्धनुः} %3-28-20

\twolineshloka
{सुमहद् वैष्णवं यत् तदतिसृष्टं महर्षिणा}
{वरं तद् धनुरुद्यम्य खरं समभिधावत} %3-28-21

\twolineshloka
{ततः कनकपुङ्खैस्तु शरैः सन्नतपर्वभिः}
{चिच्छेद रामः सङ्क्रुद्धः खरस्य समरे ध्वजम्} %3-28-22

\twolineshloka
{स दर्शनीयो बहुधा विच्छिन्नः काञ्चनो ध्वजः}
{जगाम धरणीं सूर्यो देवतानामिवाज्ञया} %3-28-23

\twolineshloka
{तं चतुर्भिः खरः क्रुद्धो रामं गात्रेषु मार्गणैः}
{विव्याध हृदि मर्मज्ञो मातङ्गमिव तोमरैः} %3-28-24

\twolineshloka
{स रामो बहुभिर्बाणैः खरकार्मुकनिःसृतैः}
{विद्धो रुधिरसिक्ताङ्गो बभूव रुषितो भृशम्} %3-28-25

\twolineshloka
{स धनुर्धन्विनां श्रेष्ठः सङ्गृह्य परमाहवे}
{मुमोच परमेष्वासः षट् शरानभिलक्षितान्} %3-28-26

\twolineshloka
{शिरस्येकेन बाणेन द्वाभ्यां बाह्वोरथार्पयत्}
{त्रिभिश्चन्द्रार्धवक्त्रैश्च वक्षस्यभिजघान ह} %3-28-27

\twolineshloka
{ततः पश्चान्महातेजा नाराचान् भास्करोपमान्}
{जघान राक्षसं क्रुद्धस्त्रयोदश शिलाशितान्} %3-28-28

\twolineshloka
{रथस्य युगमेकेन चतुर्भिः शबलान् हयान्}
{षष्ठेन च शिरः सङ्ख्ये चिच्छेद खरसारथेः} %3-28-29

\twolineshloka
{त्रिभिस्त्रिवेणून् बलवान् द्वाभ्यामक्षं महाबलः}
{द्वादशेन तु बाणेन खरस्य सशरं धनुः} %3-28-30

\twolineshloka
{छित्त्वा वज्रनिकाशेन राघवः प्रहसन्निव}
{त्रयोदशेनेन्द्रसमो बिभेद समरे खरम्} %3-28-31

\twolineshloka
{प्रभग्नधन्वा विरथो हताश्वो हतसारथिः}
{गदापाणिरवप्लुत्य तस्थौ भूमौ खरस्तदा} %3-28-32

\twolineshloka
{तत् कर्म रामस्य महारथस्य समेत्य देवाश्च महर्षयश्च}
{अपूजयन् प्राञ्जलयः प्रहृष्टास्तदा विमानाग्रगताः समेताः} %3-28-33


॥इत्यार्षे श्रीमद्रामायणे वाल्मीकीये आदिकाव्ये अरण्यकाण्डे खररामसम्प्रहारः नाम अष्टाविंशः सर्गः ॥३-२८॥
