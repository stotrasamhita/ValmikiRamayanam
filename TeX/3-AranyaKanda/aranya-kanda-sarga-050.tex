\sect{पञ्चाशः सर्गः — जटायुरभियोगः}

\twolineshloka
{तं शब्दमवसुप्तस्तु जटायुरथ शुश्रुवे}
{निरैक्षद् रावणं क्षिप्रं वैदेहीं च ददर्श सः} %3-50-1

\twolineshloka
{ततः पर्वतशृङ्गाभस्तीक्ष्णतुण्डः खगोत्तमः}
{वनस्पतिगतः श्रीमान् व्याजहार शुभां गिरम्} %3-50-2

\twolineshloka
{दशग्रीव स्थितो धर्मे पुराणे सत्यसंश्रयः}
{भ्रातस्त्वं निन्दितं कर्म कर्तुं नार्हसि साम्प्रतम्} %3-50-3

\twolineshloka
{जटायुर्नाम नाम्नाहं गृध्रराजो महाबलः}
{राजा सर्वस्य लोकस्य महेन्द्रवरुणोपमः} %3-50-4

\twolineshloka
{लोकानां च हिते युक्तो रामो दशरथात्मजः}
{तस्यैषा लोकनाथस्य धर्मपत्नी यशस्विनी} %3-50-5

\twolineshloka
{सीता नाम वरारोहा यां त्वं हर्तुमिहेच्छसि}
{कथं राजा स्थितो धर्मे परदारान् परामृशेत्} %3-50-6

\twolineshloka
{रक्षणीया विशेषेण राजदारा महाबल}
{निवर्तय गतिं नीचां परदाराभिमर्शनात्} %3-50-7

\twolineshloka
{न तत् समाचरेद् धीरो यत् परोऽस्य विगर्हयेत्}
{यथाऽऽत्मनस्तथान्येषां दारा रक्ष्या विमर्शनात्} %3-50-8

\twolineshloka
{अर्थं वा यदि वा कामं शिष्टाः शास्त्रेष्वनागतम्}
{व्यवस्यन्त्यनुराजानं धर्मं पौलस्त्यनन्दन} %3-50-9

\twolineshloka
{राजा धर्मश्च कामश्च द्रव्याणां चोत्तमो निधिः}
{धर्मः शुभं वा पापं वा राजमूलं प्रवर्तते} %3-50-10

\twolineshloka
{पापस्वभावश्चपलः कथं त्वं रक्षसां वर}
{ऐश्वर्यमभिसम्प्राप्तो विमानमिव दुष्कृती} %3-50-11

\twolineshloka
{कामस्वभावो यःसोऽसौ न शक्यस्तं प्रमार्जितुम्}
{नहि दुष्टात्मनामार्यमावसत्यालये चिरम्} %3-50-12

\twolineshloka
{विषये वा पुरे वा ते यदा रामो महाबलः}
{नापराध्यति धर्मात्मा कथं तस्यापराध्यसि} %3-50-13

\twolineshloka
{यदि शूर्पणखाहेतोर्जनस्थानगतः खरः}
{अतिवृत्तो हतः पूर्वं रामेणाक्लिष्टकर्मणा} %3-50-14

\twolineshloka
{अत्र ब्रूहि यथातत्त्वं को रामस्य व्यतिक्रमः}
{यस्य त्वं लोकनाथस्य हृत्वा भार्यां गमिष्यसि} %3-50-15

\twolineshloka
{क्षिप्रं विसृज वैदेहीं मा त्वा घोरेण चक्षुषा}
{दहेद् दहनभूतेन वृत्रमिन्द्राशनिर्यथा} %3-50-16

\twolineshloka
{सर्पमाशीविषं बद्ध्वा वस्त्रान्ते नावबुध्यसे}
{ग्रीवायां प्रतिमुक्तं च कालपाशं न पश्यसि} %3-50-17

\twolineshloka
{स भारः सौम्य भर्तव्यो यो नरं नावसादयेत्}
{तदन्नमपि भोक्तव्यं जीर्यते यदनामयम्} %3-50-18

\twolineshloka
{यत् कृत्वा न भवेद् धर्मो न कीर्तिर्न यशो ध्रुवम्}
{शरीरस्य भवेत् खेदः कस्तत् कर्म समाचरेत्} %3-50-19

\twolineshloka
{षष्टिवर्षसहस्राणि जातस्य मम रावण}
{पितृपैतामहं राज्यं यथावदनुतिष्ठतः} %3-50-20

\twolineshloka
{वृद्धोऽहं त्वं युवा धन्वी सरथः कवची शरी}
{न चाप्यादाय कुशली वैदेहीं मे गमिष्यसि} %3-50-21

\twolineshloka
{न शक्तस्त्वं बलाद्धर्तुं वैदेहीं मम पश्यतः}
{हेतुभिर्न्यायसंयुक्तैर्ध्रुवां वेदश्रुतीमिव} %3-50-22

\twolineshloka
{युध्यस्व यदि शूरोऽसि मुहूर्तं तिष्ठ रावण}
{शयिष्यसे हतो भूमौ यथा पूर्वं खरस्तथा} %3-50-23

\twolineshloka
{असकृत्संयुगे येन निहता दैत्यदानवाः}
{न चिराच्चीरवासास्त्वां रामो युधि वधिष्यति} %3-50-24

\twolineshloka
{किं नु शक्यं मया कर्तुं गतौ दूरं नृपात्मजौ}
{क्षिप्रं त्वं नश्यसे नीच तयोर्भीतो न संशयः} %3-50-25

\twolineshloka
{नहि मे जीवमानस्य नयिष्यसि शुभामिमाम्}
{सीतां कमलपत्राक्षीं रामस्य महिषीं प्रियाम्} %3-50-26

\twolineshloka
{अवश्यं तु मया कार्यं प्रियं तस्य महात्मनः}
{जीवितेनापि रामस्य तथा दशरथस्य च} %3-50-27

\threelineshloka
{तिष्ठ तिष्ठ दशग्रीव मुहूर्तं पश्य रावण}
{वृन्तादिव फलं त्वां तु पातयेयं रथोत्तमात्}
{युद्धातिथ्यं प्रदास्यामि यथाप्राणं निशाचर} %3-50-28


॥इत्यार्षे श्रीमद्रामायणे वाल्मीकीये आदिकाव्ये अरण्यकाण्डे जटायुरभियोगः नाम पञ्चाशः सर्गः ॥३-५०॥
