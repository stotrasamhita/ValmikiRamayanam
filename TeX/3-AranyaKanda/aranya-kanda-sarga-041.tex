\sect{एकचत्वारिंशः सर्गः — रावणनिन्दा}

\twolineshloka
{आज्ञप्तो रावणेनेत्थं प्रतिकूलं च राजवत्}
{अब्रवीत् परुषं वाक्यं निःशङ्को राक्षसाधिपम्} %3-41-1

\twolineshloka
{केनायमुपदिष्टस्ते विनाशः पापकर्मणा}
{सपुत्रस्य सराज्यस्य सामात्यस्य निशाचर} %3-41-2

\twolineshloka
{कस्त्वया सुखिना राजन् नाभिनन्दति पापकृत्}
{केनेदमुपदिष्टं ते मृत्युद्वारमुपायतः} %3-41-3

\twolineshloka
{शत्रवस्तव सुव्यक्तं हीनवीर्या निशाचर}
{इच्छन्ति त्वां विनश्यन्तमुपरुद्धं बलीयसा} %3-41-4

\twolineshloka
{केनेदमुपदिष्टं ते क्षुद्रेणाहितबुद्धिना}
{यस्त्वामिच्छति नश्यन्तं स्वकृतेन निशाचर} %3-41-5

\twolineshloka
{वध्याः खलु न वध्यन्ते सचिवास्तव रावण}
{ये त्वामुत्पथमारूढं न निगृह्णन्ति सर्वशः} %3-41-6

\twolineshloka
{अमात्यैः कामवृत्तो हि राजा कापथमाश्रितः}
{निग्राह्यः सर्वथा सद्भिः स निग्राह्यो न गृह्यसे} %3-41-7

\twolineshloka
{धर्ममर्थं च कामं च यशश्च जयतां वर}
{स्वामिप्रसादात् सचिवाः प्राप्नुवन्ति निशाचर} %3-41-8

\twolineshloka
{विपर्यये तु तत्सर्वं व्यर्थं भवति रावण}
{व्यसनं स्वामिवैगुण्यात् प्राप्नुवन्तीतरे जनाः} %3-41-9

\twolineshloka
{राजमूलो हि धर्मश्च यशश्च जयतां वर}
{तस्मात् सर्वास्ववस्थासु रक्षितव्या नराधिपाः} %3-41-10

\twolineshloka
{राज्यं पालयितुं शक्यं न तीक्ष्णेन निशाचर}
{न चातिप्रतिकूलेन नाविनीतेन राक्षस} %3-41-11

\twolineshloka
{ये तीक्ष्णमन्त्राः सचिवा भुज्यन्ते सह तेन वै}
{विषमेषु रथाः शीघ्रं मन्दसारथयो यथा} %3-41-12

\twolineshloka
{बहवः साधवो लोके युक्तधर्ममनुष्ठिताः}
{परेषामपराधेन विनष्टाः सपरिच्छदाः} %3-41-13

\twolineshloka
{स्वामिना प्रतिकूलेन प्रजास्तीक्ष्णेन रावण}
{रक्ष्यमाणा न वर्धन्ते मेषा गोमायुना यथा} %3-41-14

\twolineshloka
{अवश्यं विनशिष्यन्ति सर्वे रावण राक्षसाः}
{येषां त्वं कर्कशो राजा दुर्बुद्धिरजितेन्द्रियः} %3-41-15

\twolineshloka
{तदिदं काकतालीयं घोरमासादितं मया}
{अत्र त्वं शोचनीयोऽसि ससैन्यो विनशिष्यसि} %3-41-16

\twolineshloka
{मां निहत्य तु रामोऽसावचिरात् त्वां वधिष्यति}
{अनेन कृतकृत्योऽस्मि म्रिये चाप्यरिणा हतः} %3-41-17

\twolineshloka
{दर्शनादेव रामस्य हतं मामवधारय}
{आत्मानं च हतं विद्धि हृत्वा सीतां सबान्धवम्} %3-41-18

\twolineshloka
{आनयिष्यसि चेत् सीतामाश्रमात् सहितो मया}
{नैव त्वमपि नाहं वै नैव लङ्का न राक्षसाः} %3-41-19

\twolineshloka
{निवार्यमाणस्तु मया हितैषिणा न मृष्यसे वाक्यमिदं निशाचर}
{परेतकल्पा हि गतायुषो नरा हितं न गृह्णन्ति सुहृद्भिरीरितम्} %3-41-20


॥इत्यार्षे श्रीमद्रामायणे वाल्मीकीये आदिकाव्ये अरण्यकाण्डे रावणनिन्दा नाम एकचत्वारिंशः सर्गः ॥३-४१॥
