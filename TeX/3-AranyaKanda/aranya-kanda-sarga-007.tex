\sect{सप्तमः सर्गः — सुतीक्ष्णाश्रमः}

\twolineshloka
{रामस्तु सहितो भ्रात्रा सीतया च परन्तपः}
{सुतीक्ष्णस्याश्रमपदं जगाम सह तैर्द्विजैः} %3-7-1

\twolineshloka
{स गत्वा दूरमध्वानं नदीस्तीर्त्वा बहूदकाः}
{ददर्श विमलं शैलं महामेरुमिवोन्नतम्} %3-7-2

\twolineshloka
{ततस्तदिक्ष्वाकुवरौ सततं विविधैर्द्रुमैः}
{काननं तौ विविशतुः सीतया सह राघवौ} %3-7-3

\twolineshloka
{प्रविष्टस्तु वनं घोरं बहुपुष्पफलद्रुमम्}
{ददर्शाश्रममेकान्ते चीरमालापरिष्कृतम्} %3-7-4

\twolineshloka
{तत्र तापसमासीनं मलपङ्कजधारिणम्}
{रामः सुतीक्ष्णं विधिवत् तपोधनमभाषत} %3-7-5

\twolineshloka
{रामोऽहमस्मि भगवन् भवन्तं द्रष्टुमागतः}
{तन्माभिवद धर्मज्ञ महर्षे सत्यविक्रम} %3-7-6

\twolineshloka
{स निरीक्ष्य ततो धीरो रामं धर्मभृतां वरम्}
{समाश्लिष्य च बाहुभ्यामिदं वचनमब्रवीत्} %3-7-7

\twolineshloka
{स्वागतं ते रघुश्रेष्ठ राम सत्यभृतां वर}
{आश्रमोऽयं त्वयाऽऽक्रान्तः सनाथ इव साम्प्रतम्} %3-7-8

\twolineshloka
{प्रतीक्षमाणस्त्वामेव नारोहेऽहं महायशः}
{देवलोकमितो वीर देहं त्यक्त्वा महीतले} %3-7-9

\twolineshloka
{चित्रकूटमुपादाय राज्यभ्रष्टोऽसि मे श्रुतः}
{इहोपयातः काकुत्स्थ देवराजः शतक्रतुः} %3-7-10

\twolineshloka
{उपागम्य च मे देवो महादेवः सुरेश्वरः}
{सर्वाल्ँ लोकाञ्जितानाह मम पुण्येन कर्मणा} %3-7-11

\twolineshloka
{तेषु देवर्षिजुष्टेषु जितेषु तपसा मया}
{मत्प्रसादात् सभार्यस्त्वं विहरस्व सलक्ष्मणः} %3-7-12

\twolineshloka
{तमुग्रतपसं दीप्तं महर्षिं सत्यवादिनम्}
{प्रत्युवाचात्मवान् रामो ब्रह्माणमिव वासवः} %3-7-13

\twolineshloka
{अहमेवाहरिष्यामि स्वयं लोकान् महामुने}
{आवासं त्वहमिच्छामि प्रदिष्टमिह कानने} %3-7-14

\twolineshloka
{भवान् सर्वत्र कुशलः सर्वभूतहिते रतः}
{आख्यातं शरभङ्गेन गौतमेन महात्मना} %3-7-15

\twolineshloka
{एवमुक्तस्तु रामेण महर्षिर्लोकविश्रुतः}
{अब्रवीन्मधुरं वाक्यं हर्षेण महता युतः} %3-7-16

\twolineshloka
{अयमेवाश्रमो राम गुणवान् रम्यतामिति}
{ऋषिसङ्घानुचरितः सदा मूलफलैर्युतः} %3-7-17

\twolineshloka
{इममाश्रममागम्य मृगसङ्घा महीयसः}
{अहत्वा प्रतिगच्छन्ति लोभयित्वाकुतोभयाः} %3-7-18

\twolineshloka
{नान्यो दोषो भवेदत्र मृगेभ्योऽन्यत्र विद्धि वै}
{तच्छ्रुत्वा वचनं तस्य महर्षेर्लक्ष्मणाग्रजः} %3-7-19

\twolineshloka
{उवाच वचनं धीरो विगृह्य सशरं धनुः}
{तानहं सुमहाभाग मृगसङ्घान् समागतान्} %3-7-20

\twolineshloka
{हन्यां निशितधारेण शरेणानतपर्वणा}
{भवांस्तत्राभिषज्येत किं स्यात् कृच्छ्रतरं ततः} %3-7-21

\twolineshloka
{एतस्मिन्नाश्रमे वासं चिरं तु न समर्थये}
{तमेवमुक्त्वोपरमं रामः सन्ध्यामुपागमत्} %3-7-22

\twolineshloka
{अन्वास्य पश्चिमां सन्ध्यां तत्र वासमकल्पयत्}
{सुतीक्ष्णस्याश्रमे रम्ये सीतया लक्ष्मणेन च} %3-7-23

\twolineshloka
{ततः शुभं तापसयोग्यमन्नं स्वयं सुतीक्ष्णः पुरुषर्षभाभ्याम्}
{ताभ्यां सुसत्कृत्य ददौ महात्मा सन्ध्यानिवृत्तौ रजनीं समीक्ष्य} %3-7-24


॥इत्यार्षे श्रीमद्रामायणे वाल्मीकीये आदिकाव्ये अरण्यकाण्डे सुतीक्ष्णाश्रमः नाम सप्तमः सर्गः ॥३-७॥
