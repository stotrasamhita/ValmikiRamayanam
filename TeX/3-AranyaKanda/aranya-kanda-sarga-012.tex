\sect{द्वादशः सर्गः — अगस्त्यदर्शनम्}

\twolineshloka
{स प्रविश्याश्रमपदं लक्ष्मणो राघवानुजः}
{अगस्त्यशिष्यमासाद्य वाक्यमेतदुवाच ह} %3-12-1

\twolineshloka
{राजा दशरथो नाम ज्येष्ठस्तस्य सुतो बली}
{रामः प्राप्तो मुनिं द्रष्टुं भार्यया सह सीतया} %3-12-2

\twolineshloka
{लक्ष्मणो नाम तस्याहं भ्राता त्ववरजो हितः}
{अनुकूलश्च भक्तश्च यदि ते श्रोत्रमागतः} %3-12-3

\twolineshloka
{ते वयं वनमत्युग्रं प्रविष्टाः पितृशासनात्}
{द्रष्टुमिच्छामहे सर्वे भगवन्तं निवेद्यताम्} %3-12-4

\twolineshloka
{तस्य तद् वचनं श्रुत्वा लक्ष्मणस्य तपोधनः}
{तथेत्युक्त्वाग्निशरणं प्रविवेश निवेदितुम्} %3-12-5

\twolineshloka
{स प्रविश्य मुनिश्रेष्ठं तपसा दुष्प्रधर्षणम्}
{कृताञ्जलिरुवाचेदं रामागमनमञ्जसा} %3-12-6

\twolineshloka
{यथोक्तं लक्ष्मणेनैव शिष्योऽगस्त्यस्य सम्मतः}
{पुत्रौ दशरथस्येमौ रामो लक्ष्मण एव च} %3-12-7

\twolineshloka
{प्रविष्टावाश्रमपदं सीतया सह भार्यया}
{द्रष्टुं भवन्तमायातौ शुश्रूषार्थमरिन्दमौ} %3-12-8

\twolineshloka
{यदत्रानन्तरं तत् त्वमाज्ञापयितुमर्हसि}
{ततः शिष्यादुपश्रुत्य प्राप्तं रामं सलक्ष्मणम्} %3-12-9

\twolineshloka
{वैदेहीं च महाभागामिदं वचनमब्रवीत्}
{दिष्ट्या रामश्चिरस्याद्य द्रष्टुं मां समुपागतः} %3-12-10

\twolineshloka
{मनसा काङ्क्षितं ह्यस्य मयाप्यागमनं प्रति}
{गम्यतां सत्कृतो रामः सभार्यः सहलक्ष्मणः} %3-12-11

\twolineshloka
{प्रवेश्यतां समीपं मे किमसौ न प्रवेशितः}
{एवमुक्तस्तु मुनिना धर्मज्ञेन महात्मना} %3-12-12

\twolineshloka
{अभिवाद्याब्रवीच्छिष्यस्तथेति नियताञ्जलिः}
{तदा निष्क्रम्य सम्भ्रान्तः शिष्यो लक्ष्मणमब्रवीत्} %3-12-13

\twolineshloka
{कोऽसौ रामो मुनिं द्रष्टुमेतु प्रविशतु स्वयम्}
{ततो गत्वाऽऽश्रमपदं शिष्येण सह लक्ष्मणः} %3-12-14

\twolineshloka
{दर्शयामास काकुत्स्थं सीतां च जनकात्मजाम्}
{तं शिष्यः प्रश्रितं वाक्यमगस्त्यवचनं ब्रुवन्} %3-12-15

\twolineshloka
{प्रावेशयद् यथान्यायं सत्कारार्हं सुसत्कृतम्}
{प्रविवेश ततो रामः सीतया सह लक्ष्मणः} %3-12-16

\twolineshloka
{प्रशान्तहरिणाकीर्णमाश्रमं ह्यवलोकयन्}
{स तत्र ब्रह्मणः स्थानमग्नेः स्थानं तथैव च} %3-12-17

\twolineshloka
{विष्णोः स्थानं महेन्द्रस्य स्थानं चैव विवस्वतः}
{सोमस्थानं भगस्थानं स्थानं कौबेरमेव च} %3-12-18

\twolineshloka
{धातुर्विधातुः स्थानं च वायोः स्थानं तथैव च}
{स्थानं च पाशहस्तस्य वरुणस्य महात्मनः} %3-12-19

\twolineshloka
{स्थानं तथैव गायत्र्या वसूनां स्थानमेव च}
{स्थानं च नागराजस्य गरुडस्थानमेव च} %3-12-20

\twolineshloka
{कार्तिकेयस्य च स्थानं धर्मस्थानं च पश्यति}
{ततः शिष्यैः परिवृतो मुनिरप्यभिनिष्पतत्} %3-12-21

\twolineshloka
{तं ददर्शाग्रतो रामो मुनीनां दीप्ततेजसाम्}
{अब्रवीद् वचनं वीरो लक्ष्मणं लक्ष्मिवर्धनम्} %3-12-22

\twolineshloka
{बहिर्लक्ष्मण निष्क्रामत्यगस्त्यो भगवानृषिः}
{औदार्येणावगच्छामि निधानं तपसामिमम्} %3-12-23

\twolineshloka
{एवमुक्त्वा महाबाहुरगस्त्यं सूर्यवर्चसम्}
{जग्राहापततस्तस्य पादौ च रघुनन्दनः} %3-12-24

\twolineshloka
{अभिवाद्य तु धर्मात्मा तस्थौ रामः कृताञ्जलिः}
{सीतया सह वैदेह्या तदा रामः सलक्ष्मणः} %3-12-25

\twolineshloka
{प्रतिगृह्य च काकुत्स्थमर्चयित्वाऽऽसनोदकैः}
{कुशलप्रश्नमुक्त्वा च आस्यतामिति सोऽब्रवीत्} %3-12-26

\twolineshloka
{अग्निं हुत्वा प्रदायार्घ्यमतिथीन् प्रतिपूज्य च}
{वानप्रस्थेन धर्मेण स तेषां भोजनं ददौ} %3-12-27

\twolineshloka
{प्रथमं चोपविश्याथ धर्मज्ञो मुनिपुङ्गवः}
{उवाच राममासीनं प्राञ्जलिं धर्मकोविदम्} %3-12-28

\threelineshloka
{अग्निं हुत्वा प्रदायार्घ्यमतिथिं प्रतिपूजयेत्}
{अन्यथा खलु काकुत्स्थ तपस्वी समुदाचरन्}
{दुःसाक्षीव परे लोके स्वानि मांसानि भक्षयेत्} %3-12-29

\twolineshloka
{राजा सर्वस्य लोकस्य धर्मचारी महारथः}
{पूजनीयश्च मान्यश्च भवान् प्राप्तः प्रियातिथिः} %3-12-30

\twolineshloka
{एवमुक्त्वा फलैर्मूलैः पुष्पैश्चान्यैश्च राघवम्}
{पूजयित्वा यथाकामं ततोऽगस्त्यस्तमब्रवीत्} %3-12-31

\twolineshloka
{इदं दिव्यं महच्चापं हेमवज्रविभूषितम्}
{वैष्णवं पुरुषव्याघ्र निर्मितं विश्वकर्मणा} %3-12-32

\twolineshloka
{अमोघः सूर्यसङ्काशो ब्रह्मदत्तः शरोत्तमः}
{दत्तौ मम महेन्द्रेण तूणी चाक्षय्यसायकौ} %3-12-33

\twolineshloka
{सम्पूर्णौ निशितैर्बाणैर्ज्वलद्भिरिव पावकैः}
{महाराजतकोशोऽयमसिर्हेमविभूषितः} %3-12-34

\twolineshloka
{अनेन धनुषा राम हत्वा सङ्ख्ये महासुरान्}
{आजहार श्रियं दीप्तां पुरा विष्णुर्दिवौकसाम्} %3-12-35

\twolineshloka
{तद्धनुस्तौ च तूणी च शरं खड्गं च मानद}
{जयाय प्रतिगृह्णीष्व वज्रं वज्रधरो यथा} %3-12-36

\twolineshloka
{एवमुक्त्वा महातेजाः समस्तं तद्वरायुधम्}
{दत्त्वा रामाय भगवानगस्त्यः पुनरब्रवीत्} %3-12-37


॥इत्यार्षे श्रीमद्रामायणे वाल्मीकीये आदिकाव्ये अरण्यकाण्डे अगस्त्यदर्शनम् नाम द्वादशः सर्गः ॥३-१२॥
