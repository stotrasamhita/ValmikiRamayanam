\sect{त्रयस्त्रिंशः सर्गः — रावणनिन्दा}

\twolineshloka
{ततः शूर्पणखा दीना रावणं लोकरावणम्}
{अमात्यमध्ये संक्रुद्धा परुषं वाक्यमब्रवीत्} %3-33-1

\twolineshloka
{प्रमत्तः कामभोगेषु स्वैरवृत्तो निरङ्कुशः}
{समुत्पन्नं भयं घोरं बोद्धव्यं नावबुध्यसे} %3-33-2

\twolineshloka
{सक्तं ग्राम्येषु भोगेषु कामवृत्तं महीपतिम्}
{लुब्धं न बहु मन्यन्ते श्मशानाग्निमिव प्रजाः} %3-33-3

\twolineshloka
{स्वयं कार्याणि यः काले नानुतिष्ठति पार्थिवः}
{स तु वै सह राज्येन तैश्च कार्यैर्विनश्यति} %3-33-4

\twolineshloka
{अयुक्तचारं दुर्दर्शमस्वाधीनं नराधिपम्}
{वर्जयन्ति नरा दूरान्नदीपङ्कमिव द्विपाः} %3-33-5

\twolineshloka
{ये न रक्षन्ति विषयमस्वाधीनं नराधिपाः}
{ते न वृद्ध्या प्रकाशन्ते गिरयः सागरे यथा} %3-33-6

\twolineshloka
{आत्मवद्भिर्विगृह्य त्वं देवगन्धर्वदानवैः}
{अयुक्तचारश्चपलः कथं राजा भविष्यसि} %3-33-7

\twolineshloka
{त्वं तु बालस्वभावश्च बुद्धिहीनश्च राक्षस}
{ज्ञातव्यं तन्न जानीषे कथं राजा भविष्यसि} %3-33-8

\twolineshloka
{येषां चाराश्च कोशश्च नयश्च जयतां वर}
{अस्वाधीना नरेन्द्राणां प्राकृतैस्ते जनैः समाः} %3-33-9

\twolineshloka
{यस्मात् पश्यन्ति दूरस्थान् सर्वानर्थान् नराधिपाः}
{चारेण तस्मादुच्यन्ते राजानो दीर्घचक्षुषः} %3-33-10

\twolineshloka
{अयुक्तचारं मन्ये त्वां प्राकृतैः सचिवैर्युतः}
{स्वजनं च जनस्थानं निहतं नावबुध्यसे} %3-33-11

\twolineshloka
{चतुर्दश सहस्राणि रक्षसां भीमकर्मणाम्}
{हतान्येकेन रामेण खरश्च सहदूषणः} %3-33-12

\twolineshloka
{ऋषीणामभयं दत्तं कृतक्षेमाश्च दण्डकाः}
{धर्षितं च जनस्थानं रामेणाक्लिष्टकारिणा} %3-33-13

\twolineshloka
{त्वं तु लुब्धः प्रमत्तश्च पराधीनश्च राक्षस}
{विषये स्वे समुत्पन्नं यद् भयं नावबुध्यसे} %3-33-14

\twolineshloka
{तीक्ष्णमल्पप्रदातारं प्रमत्तं गर्वितं शठम्}
{व्यसने सर्वभूतानि नाभिधावन्ति पार्थिवम्} %3-33-15

\twolineshloka
{अतिमानिनमग्राह्यमात्मसम्भावितं नरम्}
{क्रोधनं व्यसने हन्ति स्वजनोऽपि नराधिपम्} %3-33-16

\twolineshloka
{नानुतिष्ठति कार्याणि भयेषु न बिभेति च}
{क्षिप्रं राज्याच्च्युतो दीनस्तृणैस्तुल्यो भवेदिह} %3-33-17

\twolineshloka
{शुष्ककाष्ठैर्भवेत् कार्यं लोष्ठैरपि च पांसुभिः}
{न तु स्थानात् परिभ्रष्टैः कार्यं स्याद् वसुधाधिपैः} %3-33-18

\twolineshloka
{उपभुक्तं यथा वासः स्रजो वा मृदिता यथा}
{एवं राज्यात् परिभ्रष्टः समर्थोऽपि निरर्थकः} %3-33-19

\twolineshloka
{अप्रमत्तश्च यो राजा सर्वज्ञो विजितेन्द्रियः}
{कृतज्ञो धर्मशीलश्च स राजा तिष्ठते चिरम्} %3-33-20

\twolineshloka
{नयनाभ्यां प्रसुप्तो वा जागर्ति नयचक्षुषा}
{व्यक्तक्रोधप्रसादश्च स राजा पूज्यते जनैः} %3-33-21

\twolineshloka
{त्वं तु रावण दुर्बुद्धिर्गुणैरेतैर्विवर्जितः}
{यस्य तेऽविदितश्चारै रक्षसां सुमहान् वधः} %3-33-22

\twolineshloka
{परावमन्ता विषयेषु सङ्गवान् न देशकालप्रविभागतत्त्ववित्}
{अयुक्तबुद्धिर्गुणदोषनिश्चये विपन्नराज्यो न चिराद् विपत्स्यसे} %3-33-23

\twolineshloka
{इति स्वदोषान् परिकीर्तितांस्तथा समीक्ष्य बुद्ध्या क्षणदाचरेश्वरः}
{धनेन दर्पेण बलेन चान्वितो विचिन्तयामास चिरं स रावणः} %3-33-24


॥इत्यार्षे श्रीमद्रामायणे वाल्मीकीये आदिकाव्ये अरण्यकाण्डे रावणनिन्दा नाम त्रयस्त्रिंशः सर्गः ॥३-३३॥
