\sect{षड्त्रिंशः सर्गः — सहायैषणा}

\twolineshloka
{मारीच श्रूयतां तात वचनं मम भाषतः}
{आर्तोऽस्मि मम चार्तस्य भवान् हि परमा गतिः} %3-36-1

\twolineshloka
{जानीषे त्वं जनस्थानं भ्राता यत्र खरो मम}
{दूषणश्च महाबाहुः स्वसा शूर्पणखा च मे} %3-36-2

\twolineshloka
{त्रिशिराश्च महाबाहू राक्षसः पिशिताशनः}
{अन्ये च बहवः शूरा लब्धलक्षा निशाचराः} %3-36-3

\twolineshloka
{वसन्ति मन्नियोगेन अधिवासं च राक्षसाः}
{बाधमाना महारण्ये मुनीन् ये धर्मचारिणः} %3-36-4

\twolineshloka
{चतुर्दश सहस्राणि रक्षसां भीमकर्मणाम्}
{शूराणां लब्धलक्षाणां खरचित्तानुवर्तिनाम्} %3-36-5

\twolineshloka
{ते त्विदानीं जनस्थाने वसमाना महाबलाः}
{सङ्गताः परमायत्ता रामेण सह संयुगे} %3-36-6

\twolineshloka
{नानाशस्त्रप्रहरणाः खरप्रमुखराक्षसाः}
{तेन संजातरोषेण रामेण रणमूर्धनि} %3-36-7

\twolineshloka
{अनुक्त्वा परुषं किंचिच्छरैर्व्यापारितं धनुः}
{चतुर्दश सहस्राणि रक्षसामुग्रतेजसाम्} %3-36-8

\twolineshloka
{निहतानि शरैर्दीप्तैर्मानुषेण पदातिना}
{खरश्च निहतः संख्ये दूषणश्च निपातितः} %3-36-9

\twolineshloka
{हत्वा त्रिशिरसं चापि निर्भया दण्डकाः कृताः}
{पित्रा निरस्तः क्रुद्धेन सभार्यः क्षीणजीवितः} %3-36-10

\twolineshloka
{स हन्ता तस्य सैन्यस्य रामः क्षत्रियपांसनः}
{अशीलः कर्कशस्तीक्ष्णो मूर्खो लुब्धोऽजितेन्द्रियः} %3-36-11

\twolineshloka
{त्यक्तधर्मा त्वधर्मात्मा भूतानामहिते रतः}
{येन वैरं विनारण्ये सत्त्वमास्थाय केवलम्} %3-36-12

\twolineshloka
{कर्णनासापहारेण भगिनी मे विरूपिता}
{अस्य भार्यां जनस्थानात् सीतां सुरसुतोपमाम्} %3-36-13

\twolineshloka
{आनयिष्यामि विक्रम्य सहायस्तत्र मे भव}
{त्वया ह्यहं सहायेन पार्श्वस्थेन महाबल} %3-36-14

\twolineshloka
{भ्रातृभिश्च सुरान् सर्वान् नाहमत्राभिचिन्तये}
{तत्सहायो भव त्वं मे समर्थो ह्यसि राक्षस} %3-36-15

\twolineshloka
{वीर्ये युद्धे च दर्पे च न ह्यस्ति सदृशस्तव}
{उपायतो महान् शूरो महामायाविशारदः} %3-36-16

\twolineshloka
{एतदर्थमहं प्राप्तस्त्वत्समीपं निशाचर}
{शृणु तत् कर्म साहाय्ये यत् कार्यं वचनान्मम} %3-36-17

\twolineshloka
{सौवर्णस्त्वं मृगो भूत्वा चित्रो रजतबिन्दुभिः}
{आश्रमे तस्य रामस्य सीतायाः प्रमुखे चर} %3-36-18

\twolineshloka
{त्वां तु निःसंशयं सीता दृष्ट्वा तु मृगरूपिणम्}
{गृह्यतामिति भर्तारं लक्ष्मणं चाभिधास्यति} %3-36-19

\twolineshloka
{ततस्तयोरपाये तु शून्ये सीतां यथासुखम्}
{निराबाधो हरिष्यामि राहुश्चन्द्रप्रभामिव} %3-36-20

\twolineshloka
{ततः पश्चात् सुखं रामे भार्याहरणकर्शिते}
{विश्रब्धं प्रहरिष्यामि कृतार्थेनान्तरात्मना} %3-36-21

\twolineshloka
{तस्य रामकथां श्रुत्वा मारीचस्य महात्मनः}
{शुष्कं समभवद् वक्त्रं परित्रस्तो बभूव च} %3-36-22

\twolineshloka
{ओष्ठौ परिलिहन् शुष्कौ नेत्रैरनिमिषैरिव}
{मृतभूत इवार्तस्तु रावणं समुदैक्षत} %3-36-23

\twolineshloka
{स रावणं त्रस्तविषण्णचेता महावने रामपराक्रमज्ञः}
{कृताञ्जलिस्तत्त्वमुवाच वाक्यं हितं च तस्मै हितमात्मनश्च} %3-36-24


॥इत्यार्षे श्रीमद्रामायणे वाल्मीकीये आदिकाव्ये अरण्यकाण्डे सहायैषणा नाम षड्त्रिंशः सर्गः ॥३-३६॥
