\sect{द्विषष्ठितमः सर्गः — राघवविलापः}

\twolineshloka
{सीतामपश्यन् धर्मात्मा शोकोपहतचेतनः}
{विललाप महाबाहू रामः कमललोचनः} %3-62-1

\twolineshloka
{पश्यन्निव च तां सीतामपश्यन्मन्मथार्दितः}
{उवाच राघवो वाक्यं विलापाश्रयदुर्वचम्} %3-62-2

\twolineshloka
{त्वमशोकस्य शाखाभिः पुष्पप्रियतरा प्रिये}
{आवृणोषि शरीरं ते मम शोकविवर्धनी} %3-62-3

\twolineshloka
{कदलीकाण्डसदृशौ कदल्या संवृतावुभौ}
{ऊरू पश्यामि ते देवि नासि शक्ता निगूहितुम्} %3-62-4

\twolineshloka
{कर्णिकारवनं भद्रे हसन्ती देवि सेवसे}
{अलं ते परिहासेन मम बाधावहेन वै} %3-62-5

\twolineshloka
{विशेषेणाश्रमस्थाने हासोऽयं न प्रशस्यते}
{अवगच्छामि ते शीलं परिहासप्रियं प्रिये} %3-62-6

\twolineshloka
{आगच्छ त्वं विशालाक्षि शून्योऽयमुटजस्तव}
{सुव्यक्तं राक्षसैः सीता भक्षिता वा हृतापि वा} %3-62-7

\twolineshloka
{न हि सा विलपन्तं मामुपसम्प्रैति लक्ष्मण}
{एतानि मृगयूथानि साश्रुनेत्राणि लक्ष्मण} %3-62-8

\twolineshloka
{शंसन्तीव हि मे देवीं भक्षितां रजनीचरैः}
{हा ममार्ये क्व यातासि हा साध्वि वरवर्णिनि} %3-62-9

\twolineshloka
{हा सकामाद्य कैकेयी देवि मेऽद्य भविष्यति}
{सीतया सह निर्यातो विना सीतामुपागतः} %3-62-10

\twolineshloka
{कथं नाम प्रवेक्ष्यामि शून्यमन्तःपुरं मम}
{निर्वीर्य इति लोको मां निर्दयश्चेति वक्ष्यति} %3-62-11

\twolineshloka
{कातरत्वं प्रकाशं हि सीतापनयनेन मे}
{निवृत्तवनवासश्च जनकं मिथिलाधिपम्} %3-62-12

\twolineshloka
{कुशलं परिपृच्छन्तं कथं शक्ष्ये निरीक्षितुम्}
{विदेहराजो नूनं मां दृष्ट्वा विरहितं तया} %3-62-13

\twolineshloka
{सुताविनाशसंतप्तो मोहस्य वशमेष्यति}
{अथवा न गमिष्यामि पुरीं भरतपालिताम्} %3-62-14

\twolineshloka
{स्वर्गोऽपि हि तया हीनः शून्य एव मतो मम}
{तन्मामुत्सृज्य हि वने गच्छायोध्यापुरीं शुभाम्} %3-62-15

\twolineshloka
{न त्वहं तां विना सीतां जीवेयं हि कथंचन}
{गाढमाश्लिष्य भरतो वाच्यो मद्वचनात् त्वया} %3-62-16

\twolineshloka
{अनुज्ञातोऽसि रामेण पालयेति वसुंधराम्}
{अम्बा च मम कैकेयी सुमित्रा च त्वया विभो} %3-62-17

\twolineshloka
{कौसल्या च यथान्यायमभिवाद्या ममाज्ञया}
{रक्षणीया प्रयत्नेन भवता सूक्तचारिणा} %3-62-18

\twolineshloka
{सीतायाश्च विनाशोऽयं मम चामित्रसूदन}
{विस्तरेण जनन्या मे विनिवेद्यस्त्वया भवेत्} %3-62-19

\twolineshloka
{इति विलपति राघवे तु दीने वनमुपगम्य तया विना सुकेश्या}
{भयविकलमुखस्तु लक्ष्मणोऽपि व्यथितमना भृशमातुरो बभूव} %3-62-20


॥इत्यार्षे श्रीमद्रामायणे वाल्मीकीये आदिकाव्ये अरण्यकाण्डे राघवविलापः नाम द्विषष्ठितमः सर्गः ॥३-६२॥
