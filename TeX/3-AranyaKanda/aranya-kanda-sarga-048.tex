\sect{अष्टचत्वारिंशः सर्गः — रवणविकत्थनम्}

\twolineshloka
{एवं ब्रुवत्यां सीतायां संरब्धः परुषं वचः}
{ललाटे भ्रुकुटिं कृत्वा रावणः प्रत्युवाच ह} %3-48-1

\twolineshloka
{भ्राता वैश्रवणस्याहं सापत्नो वरवर्णिनि}
{रावणो नाम भद्रं ते दशग्रीवः प्रतापवान्} %3-48-2

\twolineshloka
{यस्य देवाः सगन्धर्वाः पिशाचपतगोरगाः}
{विद्रवन्ति सदा भीता मृत्योरिव सदा प्रजाः} %3-48-3

\twolineshloka
{येन वैश्रवणो भ्राता वैमात्राः कारणान्तरे}
{द्वन्द्वमासादितः क्रोधाद् रणे विक्रम्य निर्जितः} %3-48-4

\twolineshloka
{मद्भयार्तः परित्यज्य स्वमधिष्ठानमृद्धिमत्}
{कैलासं पर्वतश्रेष्ठमध्यास्ते नरवाहनः} %3-48-5

\twolineshloka
{यस्य तत् पुष्पकं नाम विमानं कामगं शुभम्}
{वीर्यादावर्जितं भद्रे येन यामि विहायसम्} %3-48-6

\twolineshloka
{मम संजातरोषस्य मुखं दृष्ट्वैव मैथिलि}
{विद्रवन्ति परित्रस्ताः सुराः शक्रपुरोगमाः} %3-48-7

\twolineshloka
{यत्र तिष्ठाम्यहं तत्र मारुतो वाति शङ्कितः}
{तीव्रांशुः शिशिरांशुश्च भयात् सम्पद्यते दिवि} %3-48-8

\twolineshloka
{निष्कम्पपत्रास्तरवो नद्यश्च स्तिमितोदकाः}
{भवन्ति यत्र तत्राहं तिष्ठामि च चरामि च} %3-48-9

\twolineshloka
{मम पारे समुद्रस्य लङ्का नाम पुरी शुभा}
{सम्पूर्णा राक्षसैर्घोरैर्यथेन्द्रस्यामरावती} %3-48-10

\twolineshloka
{प्राकारेण परिक्षिप्ता पाण्डुरेण विराजिता}
{हेमकक्ष्या पुरी रम्या वैदूर्यमयतोरणा} %3-48-11

\twolineshloka
{हस्त्यश्वरथसम्बाधा तूर्यनादविनादिता}
{सर्वकामफलैर्वृक्षैः संकुलोद्यानभूषिता} %3-48-12

\twolineshloka
{तत्र त्वं वस हे सीते राजपुत्रि मया सह}
{न स्मरिष्यसि नारीणां मानुषीणां मनस्विनि} %3-48-13

\twolineshloka
{भुञ्जाना मानुषान् भोगान् दिव्यांश्च वरवर्णिनि}
{न स्मरिष्यसि रामस्य मानुषस्य गतायुषः} %3-48-14

\twolineshloka
{स्थापयित्वा प्रियं पुत्रं राज्ये दशरथो नृपः}
{मन्दवीर्यस्ततो ज्येष्ठः सुतः प्रस्थापितो वनम्} %3-48-15

\twolineshloka
{तेन किं भ्रष्टराज्येन रामेण गतचेतसा}
{करिष्यसि विशालाक्षि तापसेन तपस्विना} %3-48-16

\twolineshloka
{रक्ष राक्षसभर्तारं कामय स्वयमागतम्}
{न मन्मथशराविष्टं प्रत्याख्यातुं त्वमर्हसि} %3-48-17

\twolineshloka
{प्रत्याख्याय हि मां भीरु पश्चात्तापं गमिष्यसि}
{चरणेनाभिहत्येव पुरूरवसमुर्वशी} %3-48-18

\twolineshloka
{अङ्गुल्या न समो रामो मम युद्धे स मानुषः}
{तव भाग्येन सम्प्राप्तं भजस्व वरवर्णिनि} %3-48-19

\twolineshloka
{एवमुक्ता तु वैदेही क्रुद्धा संरक्तलोचना}
{अब्रवीत् परुषं वाक्यं रहिते राक्षसाधिपम्} %3-48-20

\twolineshloka
{कथं वैश्रवणं देवं सर्वदेवनमस्कृतम्}
{भ्रातरं व्यपदिश्य त्वमशुभं कर्तुमिच्छसि} %3-48-21

\twolineshloka
{अवश्यं विनशिष्यन्ति सर्वे रावण राक्षसाः}
{येषां त्वं कर्कशो राजा दुर्बुद्धिरजितेन्द्रियः} %3-48-22

\twolineshloka
{अपहृत्य शचीं भार्यां शक्यमिन्द्रस्य जीवितुम्}
{नहि रामस्य भार्यां मामानीय स्वस्तिमान् भवेत्} %3-48-23

\twolineshloka
{जीवेच्चिरं वज्रधरस्य पश्चाच्छचीं प्रधृष्याप्रतिरूपरूपाम्}
{न मादृशीं राक्षस धर्षयित्वा पीतामृतस्यापि तवास्ति मोक्षः} %3-48-24


॥इत्यार्षे श्रीमद्रामायणे वाल्मीकीये आदिकाव्ये अरण्यकाण्डे रवणविकत्थनम् नाम अष्टचत्वारिंशः सर्गः ॥३-४८॥
