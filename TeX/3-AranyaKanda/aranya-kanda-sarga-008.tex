\sect{अष्टमः सर्गः — सुतीक्ष्णाभ्यनुज्ञा}

\twolineshloka
{रामस्तु सहसौमित्रिः सुतीक्ष्णेनाभिपूजितः}
{परिणाम्य निशां तत्र प्रभाते प्रत्यबुध्यत} %3-8-1

\twolineshloka
{उत्थाय च यथाकालं राघवः सह सीतया}
{उपस्पृश्य सुशीतेन तोयेनोत्पलगन्धिना} %3-8-2

\twolineshloka
{अथ तेऽग्निं सुरांश्चैव वैदेही रामलक्ष्मणौ}
{काल्यं विधिवदभ्यर्च्य तपस्विशरणे वने} %3-8-3

\twolineshloka
{उदयन्तं दिनकरं दृष्ट्वा विगतकल्मषाः}
{सुतीक्ष्णमभिगम्येदं श्लक्ष्णं वचनमब्रुवन्} %3-8-4

\twolineshloka
{सुखोषिताः स्म भगवंस्त्वया पूज्येन पूजिताः}
{आपृच्छामः प्रयास्यामो मुनयस्त्वरयन्ति नः} %3-8-5

\twolineshloka
{त्वरामहे वयं द्रष्टुं कृत्स्नमाश्रममण्डलम्}
{ऋषीणां पुण्यशीलानां दण्डकारण्यवासिनाम्} %3-8-6

\twolineshloka
{अभ्यनुज्ञातुमिच्छामः सहैभिर्मुनिपुङ्गवैः}
{धर्मनित्यैस्तपोदान्तैर्विशिखैरिव पावकैः} %3-8-7

\twolineshloka
{अविषह्यातपो यावत् सूर्यो नातिविराजते}
{अमार्गेणागतां लक्ष्मीं प्राप्येवान्वयवर्जितः} %3-8-8

\twolineshloka
{तावदिच्छामहे गन्तुमित्युक्त्वा चरणौ मुनेः}
{ववन्दे सहसौमित्रिः सीतया सह राघवः} %3-8-9

\twolineshloka
{तौ संस्पृशन्तौ चरणावुत्थाप्य मुनिपुङ्गवः}
{गाढमाश्लिष्य सस्नेहमिदं वचनमब्रवीत्} %3-8-10

\twolineshloka
{अरिष्टं गच्छ पन्थानं राम सौमित्रिणा सह}
{सीतया चानया सार्धं छाययेवानुवृत्तया} %3-8-11

\twolineshloka
{पश्याश्रमपदं रम्यं दण्डकारण्यवासिनाम्}
{एषां तपस्विनां वीर तपसा भावितात्मनाम्} %3-8-12

\twolineshloka
{सुप्राज्यफलमूलानि पुष्पितानि वनानि च}
{प्रशस्तमृगयूथानि शान्तपक्षिगणानि च} %3-8-13

\twolineshloka
{फुल्लपङ्कजखण्डानि प्रसन्नसलिलानि च}
{कारण्डवविकीर्णानि तटाकानि सरांसि च} %3-8-14

\twolineshloka
{द्रक्ष्यसे दृष्टिरम्याणि गिरिप्रस्रवणानि च}
{रमणीयान्यरण्यानि मयूराभिरुतानि च} %3-8-15

\twolineshloka
{गम्यतां वत्स सौमित्रे भवानपि च गच्छतु}
{आगन्तव्यं च ते दृष्ट्वा पुनरेवाश्रमं प्रति} %3-8-16

\twolineshloka
{एवमुक्तस्तथेत्युक्त्वा काकुत्स्थः सहलक्ष्मणः}
{प्रदक्षिणं मुनिं कृत्वा प्रस्थातुमुपचक्रमे} %3-8-17

\twolineshloka
{ततः शुभतरे तूणी धनुषी चायतेक्षणा}
{ददौ सीता तयोर्भ्रात्रोः खड्गौ च विमलौ ततः} %3-8-18

\twolineshloka
{आबध्य च शुभे तूणी चापे चादाय सस्वने}
{निष्क्रान्तावाश्रमाद् गन्तुमुभौ तौ रामलक्ष्मणौ} %3-8-19

\twolineshloka
{शीघ्रं तौ रूपसम्पन्नावनुज्ञातौ महर्षिणा}
{प्रस्थितौ धृतचापासी सीतया सह राघवौ} %3-8-20


॥इत्यार्षे श्रीमद्रामायणे वाल्मीकीये आदिकाव्ये अरण्यकाण्डे सुतीक्ष्णाभ्यनुज्ञा नाम अष्टमः सर्गः ॥३-८॥
