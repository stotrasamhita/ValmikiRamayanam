\sect{एकसप्ततितमः सर्गः — कबन्धशापाख्यानम्}

\twolineshloka
{पुरा राम महाबाहो महाबलपराक्रमम्}
{रूपमासीन्ममाचिन्त्यं त्रिषु लोकेषु विश्रुतम्} %3-71-1

\twolineshloka
{यथा सूर्यस्य सोमस्य शक्रस्य च यथा वपुः}
{सोऽहं रूपमिदं कृत्वा लोकवित्रासनं महत्} %3-71-2

\twolineshloka
{ऋषीन् वनगतान् राम त्रासयामि ततस्ततः}
{ततः स्थूलशिरा नाम महर्षिः कोपितो मया} %3-71-3

\twolineshloka
{स चिन्वन् विविधं वन्यं रूपेणानेन धर्षितः}
{तेनाहमुक्तः प्रेक्ष्यैवं घोरशापाभिधायिना} %3-71-4

\twolineshloka
{एतदेवं नृशंसं ते रूपमस्तु विगर्हितम्}
{स मया याचितः क्रुद्धः शापस्यान्तो भवेदिति} %3-71-5

\twolineshloka
{अभिशापकृतस्येति तेनेदं भाषितं वचः}
{यदा छित्त्वा भुजौ रामस्त्वां दहेद् विजने वने} %3-71-6

\twolineshloka
{तदा त्वं प्राप्स्यसे रूपं स्वमेव विपुलं शुभम्}
{श्रिया विराजितं पुत्रं दनोस्त्वं विद्धि लक्ष्मण} %3-71-7

\twolineshloka
{इन्द्रकोपादिदं रूपं प्राप्तमेवं रणाजिरे}
{अहं हि तपसोग्रेण पितामहमतोषयम्} %3-71-8

\twolineshloka
{दीर्घमायुः स मे प्रादात् ततो मां विभ्रमोऽस्पृशत्}
{दीर्घमायुर्मया प्राप्तं किं मां शक्रः करिष्यति} %3-71-9

\twolineshloka
{इत्येवं बुद्धिमास्थाय रणे शक्रमधर्षयम्}
{तस्य बाहुप्रमुक्तेन वज्रेण शतपर्वणा} %3-71-10

\twolineshloka
{सक्थिनी च शिरश्चैव शरीरे सम्प्रवेशितम्}
{स मया याच्यमानः सन् नानयद् यमसादनम्} %3-71-11

\twolineshloka
{पितामहवचः सत्यं तदस्त्विति ममाब्रवीत्}
{अनाहारः कथं शक्तो भग्नसक्थिशिरोमुखः} %3-71-12

\twolineshloka
{वज्रेणाभिहतः कालं सुदीर्घमपि जीवितुम्}
{स एवमुक्तः शक्रो मे बाहू योजनमायतौ} %3-71-13

\twolineshloka
{तदा चास्यं च मे कुक्षौ तीक्ष्णदंष्ट्रमकल्पयत्}
{सोऽहं भुजाभ्यां दीर्घाभ्यां सङ्क्षिप्यास्मिन् वनेचरान्} %3-71-14

\twolineshloka
{सिंहद्वीपिमृगव्याघ्रान् भक्षयामि समन्ततः}
{स तु मामब्रवीदिन्द्रो यदा रामः सलक्ष्मणः} %3-71-15

\twolineshloka
{छेत्स्यते समरे बाहू तदा स्वर्गं गमिष्यसि}
{अनेन वपुषा तात वनेऽस्मिन् राजसत्तम} %3-71-16

\twolineshloka
{यद् यत् पश्यामि सर्वस्य ग्रहणं साधु रोचये}
{अवश्यं ग्रहणं रामो मन्येऽहं समुपैष्यति} %3-71-17

\twolineshloka
{इमां बुद्धिं पुरस्कृत्य देहन्यासकृतश्रमः}
{स त्वं रामोऽसि भद्रं ते नाहमन्येन राघव} %3-71-18

\twolineshloka
{शक्यो हन्तुं यथा तत्त्वमेवमुक्तं महर्षिणा}
{अहं हि मतिसाचिव्यं करिष्यामि नरर्षभ} %3-71-19

\twolineshloka
{मित्रं चैवोपदेक्ष्यामि युवाभ्यां संस्कृतोऽग्निना}
{एवमुक्तस्तु धर्मात्मा दनुना तेन राघवः} %3-71-20

\twolineshloka
{इदं जगाद वचनं लक्ष्मणस्य च पश्यतः}
{रावणेन हृता भार्या सीता मम यशस्विनी} %3-71-21

\twolineshloka
{निष्क्रान्तस्य जनस्थानात् सह भ्रात्रा यथासुखम्}
{नाममात्रं तु जानामि न रूपं तस्य रक्षसः} %3-71-22

\twolineshloka
{निवासं वा प्रभावं वा वयं तस्य न विद्महे}
{शोकार्तानामनाथानामेवं विपरिधावताम्} %3-71-23

\twolineshloka
{कारुण्यं सदृशं कर्तुमुपकारेण वर्तताम्}
{काष्ठान्यानीय भग्नानि काले शुष्काणि कुञ्जरैः} %3-71-24

\twolineshloka
{धक्ष्यामस्त्वां वयं वीर श्वभ्रे महति कल्पिते}
{स त्वं सीतां समाचक्ष्व येन वा यत्र वा हृता} %3-71-25

\twolineshloka
{कुरु कल्याणमत्यर्थं यदि जानासि तत्त्वतः}
{एवमुक्तस्तु रामेण वाक्यं दनुरनुत्तमम्} %3-71-26

\twolineshloka
{प्रोवाच कुशलो वक्ता वक्तारमपि राघवम्}
{दिव्यमस्ति न मे ज्ञानं नाभिजानामि मैथिलीम्} %3-71-27

\twolineshloka
{यस्तां वक्ष्यति तं वक्ष्ये दग्धः स्वं रूपमास्थितः}
{योऽभिजानाति तद्रक्षस्तद् वक्ष्ये राम तत्परम्} %3-71-28

\twolineshloka
{अदग्धस्य हि विज्ञातुं शक्तिरस्ति न मे प्रभो}
{राक्षसं तु महावीर्यं सीता येन हृता तव} %3-71-29

\twolineshloka
{विज्ञानं हि महद् भ्रष्टं शापदोषेण राघव}
{स्वकृतेन मया प्राप्तं रूपं लोकविगर्हितम्} %3-71-30

\twolineshloka
{किं तु यावन्न यात्यस्तं सविता श्रान्तवाहनः}
{तावन्मामवटे क्षिप्त्वा दह राम यथाविधि} %3-71-31

\twolineshloka
{दग्धस्त्वयाहमवटे न्यायेन रघुनन्दन}
{वक्ष्यामि तं महावीर यस्तं वेत्स्यति राक्षसम्} %3-71-32

\twolineshloka
{तेन सख्यं च कर्तव्यं न्याय्यवृत्तेन राघव}
{कल्पयिष्यति ते वीर साहाय्यं लघुविक्रम} %3-71-33

\twolineshloka
{नहि तस्यास्त्यविज्ञातं त्रिषु लोकेषु राघव}
{सर्वान् परिवृतो लोकान् पुरा वै कारणान्तरे} %3-71-34


॥इत्यार्षे श्रीमद्रामायणे वाल्मीकीये आदिकाव्ये अरण्यकाण्डे कबन्धशापाख्यानम् नाम एकसप्ततितमः सर्गः ॥३-७१॥
