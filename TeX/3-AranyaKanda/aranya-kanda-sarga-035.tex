\sect{पञ्चत्रिंशः सर्गः — मारिचाश्रमपुनर्गमनम्}

\twolineshloka
{ततः शूर्पणखावाक्यं तच्छ्रुत्वा रोमहर्षणम्}
{सचिवानभ्यनुज्ञाय कार्यं बुद्ध्वा जगाम ह} %3-35-1

\twolineshloka
{तत् कार्यमनुगम्यान्तर्यथावदुपलभ्य च}
{दोषाणां च गुणानां च सम्प्रधार्य बलाबलम्} %3-35-2

\twolineshloka
{इति कर्तव्यमित्येव कृत्वा निश्चयमात्मनः}
{स्थिरबुद्धिस्ततो रम्यां यानशालां जगाम ह} %3-35-3

\twolineshloka
{यानशालां ततो गत्वा प्रच्छन्नं राक्षसाधिपः}
{सूतं संचोदयामास रथः संयुज्यतामिति} %3-35-4

\twolineshloka
{एवमुक्तः क्षणेनैव सारथिर्लघुविक्रमः}
{रथं संयोजयामास तस्याभिमतमुत्तमम्} %3-35-5

\twolineshloka
{कामगं रथमास्थाय काञ्चनं रत्नभूषितम्}
{पिशाचवदनैर्युक्तं खरैः कनकभूषणैः} %3-35-6

\twolineshloka
{मेघप्रतिमनादेन स तेन धनदानुजः}
{राक्षसाधिपतिः श्रीमान् ययौ नदनदीपतिम्} %3-35-7

\twolineshloka
{स श्वेतवालव्यजनः श्वेतच्छत्रो दशाननः}
{स्निग्धवैदूर्यसंकाशस्तप्तकाञ्चनभूषणः} %3-35-8

\twolineshloka
{दशग्रीवो विंशतिभुजो दर्शनीयपरिच्छदः}
{त्रिदशारिर्मुनीन्द्रघ्नो दशशीर्ष इवाद्रिराट्} %3-35-9

\twolineshloka
{कामगं रथमास्थाय शुशुभे राक्षसाधिपः}
{विद्युन्मण्डलवान् मेघः सबलाक इवाम्बरे} %3-35-10

\twolineshloka
{सशैलसागरानूपं वीर्यवानवलोकयन्}
{नानापुष्पफलैर्वृक्षैरनुकीर्णं सहस्रशः} %3-35-11

\twolineshloka
{शीतमङ्गलतोयाभिः पद्मिनीभिः समन्ततः}
{विशालैराश्रमपदैर्वेदिमद्भिरलंकृतम्} %3-35-12

\twolineshloka
{कदल्यटविसंशोभं नारिकेलोपशोभितम्}
{सालैस्तालैस्तमालैश्च तरुभिश्च सुपुष्पितैः} %3-35-13

\twolineshloka
{अत्यन्तनियताहारैः शोभितं परमर्षिभिः}
{नागैः सुपर्णैर्गन्धर्वैः किंनरैश्च सहस्रशः} %3-35-14

\twolineshloka
{जितकामैश्च सिद्धैश्च चारणैश्चोपशोभितम्}
{आजैर्वैखानसैर्माषैर्वालखिल्यैर्मरीचिपैः} %3-35-15

\twolineshloka
{दिव्याभरणमाल्याभिर्दिव्यरूपाभिरावृतम्}
{क्रीडारतविधिज्ञाभिरप्सरोभिः सहस्रशः} %3-35-16

\twolineshloka
{सेवितं देवपत्नीभिः श्रीमतीभिरुपासितम्}
{देवदानवसङ्घैश्च चरितं त्वमृताशिभिः} %3-35-17

\twolineshloka
{हंसक्रौञ्चप्लवाकीर्णं सारसैः सम्प्रसादितम्}
{वैदूर्यप्रस्तरं स्निग्धं सान्द्रं सागरतेजसा} %3-35-18

\twolineshloka
{पाण्डुराणि विशालानि दिव्यमाल्ययुतानि च}
{तूर्यगीताभिजुष्टानि विमानानि समन्ततः} %3-35-19

\twolineshloka
{तपसा जितलोकानां कामगान्यभिसम्पतन्}
{गन्धर्वाप्सरसश्चैव ददर्श धनदानुजः} %3-35-20

\twolineshloka
{निर्यासरसमूलानां चन्दनानां सहस्रशः}
{वनानि पश्यन् सौम्यानि घ्राणतृप्तिकराणि च} %3-35-21

\twolineshloka
{अगुरूणां च मुख्यानां वनान्युपवनानि च}
{तक्कोलानां च जात्यानां फलिनां च सुगन्धिनाम्} %3-35-22

\twolineshloka
{पुष्पाणि च तमालस्य गुल्मानि मरिचस्य च}
{मुक्तानां च समूहानि शुष्यमाणानि तीरतः} %3-35-23

\twolineshloka
{शैलानि प्रवरांश्चैव प्रवालनिचयांस्तथा}
{काञ्चनानि च शृङ्गाणि राजतानि तथैव च} %3-35-24

\twolineshloka
{प्रस्रवाणि मनोज्ञानि प्रसन्नान्यद्भुतानि च}
{धनधान्योपपन्नानि स्त्रीरत्नैरावृतानि च} %3-35-25

\twolineshloka
{हस्त्यश्वरथगाढानि नगराणि विलोकयन्}
{तं समं सर्वतः स्निग्धं मृदुसंस्पर्शमारुतम्} %3-35-26

\twolineshloka
{अनूपे सिन्धुराजस्य ददर्श त्रिदिवोपमम्}
{तत्रापश्यत् स मेघाभं न्यग्रोधं मुनिभिर्वृतम्} %3-35-27

\twolineshloka
{समन्ताद् यस्य ताः शाखाः शतयोजनमायताः}
{यस्य हस्तिनमादाय महाकायं च कच्छपम्} %3-35-28

\twolineshloka
{भक्षार्थं गरुडः शाखामाजगाम महाबलः}
{तस्य तां सहसा शाखां भारेण पतगोत्तमः} %3-35-29

\twolineshloka
{सुपर्णः पर्णबहुलां बभञ्जाथ महाबलः}
{तत्र वैखानसा माषा वालखिल्या मरीचिपाः} %3-35-30

\twolineshloka
{आजा बभूवुर्धूम्राश्च संगताः परमर्षयः}
{तेषां दयार्थं गरुडस्तां शाखां शतयोजनाम्} %3-35-31

\twolineshloka
{भग्नामादाय वेगेन तौ चोभौ गजकच्छपौ}
{एकपादेन धर्मात्मा भक्षयित्वा तदामिषम्} %3-35-32

\twolineshloka
{निषादविषयं हत्वा शाखया पतगोत्तमः}
{प्रहर्षमतुलं लेभे मोक्षयित्वा महामुनीन्} %3-35-33

\twolineshloka
{स तु तेन प्रहर्षेण द्विगुणीकृतविक्रमः}
{अमृतानयनार्थं वै चकार मतिमान् मतिम्} %3-35-34

\twolineshloka
{अयोजालानि निर्मथ्य भित्त्वा रत्नगृहं वरम्}
{महेन्द्रभवनाद् गुप्तमाजहारामृतं ततः} %3-35-35

\twolineshloka
{तं महर्षिगणैर्जुष्टं सुपर्णकृतलक्षणम्}
{नाम्ना सुभद्रं न्यग्रोधं ददर्श धनदानुजः} %3-35-36

\twolineshloka
{तं तु गत्वा परं पारं समुद्रस्य नदीपतेः}
{ददर्शाश्रममेकान्ते पुण्ये रम्ये वनान्तरे} %3-35-37

\twolineshloka
{तत्र कृष्णाजिनधरं जटामण्डलधारिणम्}
{ददर्श नियताहारं मारीचं नाम राक्षसम्} %3-35-38

\twolineshloka
{स रावणः समागम्य विधिवत् तेन रक्षसा}
{मारीचेनार्चितो राजा सर्वकामैरमानुषैः} %3-35-39

\twolineshloka
{तं स्वयं पूजयित्वा च भोजनेनोदकेन च}
{अर्थोपहितया वाचा मारीचो वाक्यमब्रवीत्} %3-35-40

\twolineshloka
{कच्चित्ते कुशलं राजन् लङ्कायां राक्षसेश्वर}
{केनार्थेन पुनस्त्वं वै तूर्णमेव इहागतः} %3-35-41

\twolineshloka
{एवमुक्तो महातेजा मारीचेन स रावणः}
{ततः पश्चादिदं वाक्यमब्रवीद् वाक्यकोविदः} %3-35-42


॥इत्यार्षे श्रीमद्रामायणे वाल्मीकीये आदिकाव्ये अरण्यकाण्डे मारिचाश्रमपुनर्गमनम् नाम पञ्चत्रिंशः सर्गः ॥३-३५॥
