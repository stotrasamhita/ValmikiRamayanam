\sect{चतुस्त्रिंशः सर्गः — सीताहरणोपदेशः}

\twolineshloka
{ततः शूर्पणखां दृष्ट्वा ब्रुवन्तीं परुषं वचः}
{अमात्यमध्ये सङ्क्रुद्धः परिपप्रच्छ रावणः} %3-34-1

\twolineshloka
{कश्च रामः कथंवीर्यः किंरूपः किम्पराक्रमः}
{किमर्थं दण्डकारण्यं प्रविष्टश्च सुदुस्तरम्} %3-34-2

\twolineshloka
{आयुधं किं च रामस्य येन ते राक्षसा हताः}
{खरश्च निहतः सङ्ख्ये दूषणस्त्रिशिरास्तथा} %3-34-3

\twolineshloka
{तत्त्वं ब्रूहि मनोज्ञाङ्गि केन त्वं च विरूपिता}
{इत्युक्ता राक्षसेन्द्रेण राक्षसी क्रोधमूर्च्छिता} %3-34-4

\twolineshloka
{ततो रामं यथान्यायमाख्यातुमुपचक्रमे}
{दीर्घबाहुर्विशालाक्षश्चीरकृष्णाजिनाम्बरः} %3-34-5

\twolineshloka
{कन्दर्पसमरूपश्च रामो दशरथात्मजः}
{शक्रचापनिभं चापं विकृष्य कनकाङ्गदम्} %3-34-6

\twolineshloka
{दीप्तान् क्षिपति नाराचान् सर्पानिव महाविषान्}
{नाददानं शरान् घोरान् विमुञ्चन्तं महाबलम्} %3-34-7

\twolineshloka
{न कार्मुकं विकर्षन्तं रामं पश्यामि संयुगे}
{हन्यमानं तु तत्सैन्यं पश्यामि शरवृष्टिभिः} %3-34-8

\twolineshloka
{इन्द्रेणेवोत्तमं सस्यमाहतं त्वश्मवृष्टिभिः}
{रक्षसां भीमवीर्याणां सहस्राणि चतुर्दश} %3-34-9

\twolineshloka
{निहतानि शरैस्तीक्ष्णैस्तेनैकेन पदातिना}
{अर्धाधिकमुहूर्तेन खरश्च सहदूषणः} %3-34-10

\onelineshloka
{ऋषीणामभयं दत्तं कृतक्षेमाश्च दण्डकाः} %3-34-11

\twolineshloka
{एका कथञ्चिन्मुक्ताहं परिभूय महात्मना}
{स्त्रीवधं शङ्कमानेन रामेण विदितात्मना} %3-34-12

\twolineshloka
{भ्राता चास्य महातेजा गुणतस्तुल्यविक्रमः}
{अनुरक्तश्च भक्तश्च लक्ष्मणो नाम वीर्यवान्} %3-34-13

\twolineshloka
{अमर्षी दुर्जयो जेता विक्रान्तो बुद्धिमान् बली}
{रामस्य दक्षिणो बाहुर्नित्यं प्राणो बहिश्चरः} %3-34-14

\twolineshloka
{रामस्य तु विशालाक्षी पूर्णेन्दुसदृशानना}
{धर्मपत्नी प्रिया नित्यं भर्तुः प्रियहिते रता} %3-34-15

\twolineshloka
{सा सुकेशी सुनासोरूः सुरूपा च यशस्विनी}
{देवतेव वनस्यास्य राजते श्रीरिवापरा} %3-34-16

\twolineshloka
{तप्तकाञ्चनवर्णाभा रक्ततुङ्गनखी शुभा}
{सीता नाम वरारोहा वैदेही तनुमध्यमा} %3-34-17

\twolineshloka
{नैव देवी न गन्धर्वी न यक्षी न च किन्नरी}
{तथारूपा मया नारी दृष्टपूर्वा महीतले} %3-34-18

\twolineshloka
{यस्य सीता भवेद् भार्या यं च हृष्टा परिष्वजेत्}
{अभिजीवेत् स सर्वेषु लोकेष्वपि पुरन्दरात्} %3-34-19

\twolineshloka
{सा सुशीला वपुःश्लाघ्या रूपेणाप्रतिमा भुवि}
{तवानुरूपा भार्या सा त्वं च तस्याः पतिर्वरः} %3-34-20

\twolineshloka
{तां तु विस्तीर्णजघनां पीनोत्तुङ्गपयोधराम्}
{भार्यार्थे तु तवानेतुमुद्यताहं वराननाम्} %3-34-21

\twolineshloka
{विरूपितास्मि क्रूरेण लक्ष्मणेन महाभुज}
{तां तु दृष्ट्वाद्य वैदेहीं पूर्णचन्द्रनिभाननाम्} %3-34-22

\threelineshloka
{मन्मथस्य शराणां च त्वं विधेयो भविष्यसि}
{यदि तस्यामभिप्रायो भार्यात्वे तव जायते}
{शीघ्रमुद्ध्रियतां पादो जयार्थमिह दक्षिणः} %3-34-23

\twolineshloka
{रोचते यदि ते वाक्यं ममैतद् राक्षसेश्वर}
{क्रियतां निर्विशङ्केन वचनं मम रावण} %3-34-24

\twolineshloka
{विज्ञायैषामशक्तिं च क्रियतां च महाबल}
{सीता तवानवद्याङ्गी भार्यात्वे राक्षसेश्वर} %3-34-25

\twolineshloka
{निशम्य रामेण शरैरजिह्मगैर्हताञ्जनस्थानगतान् निशाचरान्}
{खरं च दृष्ट्वा निहतं च दूषणं त्वमद्य कृत्यं प्रतिपत्तुमर्हसि} %3-34-26


॥इत्यार्षे श्रीमद्रामायणे वाल्मीकीये आदिकाव्ये अरण्यकाण्डे सीताहरणोपदेशः नाम चतुस्त्रिंशः सर्गः ॥३-३४॥
