\sect{षट्षष्ठितमः सर्गः — औचित्यप्रबोधनम्}

\twolineshloka
{तं तथा शोकसन्तप्तं विलपन्तमनाथवत्}
{मोहेन महता युक्तं परिद्यूनमचेतसम्} %3-66-1

\twolineshloka
{ततः सौमित्रिराश्वस्य मुहूर्तादिव लक्ष्मणः}
{रामं सम्बोधयामास चरणौ चाभिपीडयन्} %3-66-2

\twolineshloka
{महता तपसा चापि महता चापि कर्मणा}
{राज्ञा दशरथेनासील्लब्धोऽमृतमिवामरैः} %3-66-3

\twolineshloka
{तव चैव गुणैर्बद्धस्त्वद्वियोगान्महीपतिः}
{राजा देवत्वमापन्नो भरतस्य यथा श्रुतम्} %3-66-4

\twolineshloka
{यदि दुःखमिदं प्राप्तं काकुत्स्थ न सहिष्यसे}
{प्राकृतश्चाल्पसत्त्वश्च इतरः कः सहिष्यति} %3-66-5

\twolineshloka
{आश्वसिहि नरश्रेष्ठ प्राणिनः कस्य नापदः}
{संस्पृशन्त्यग्निवद् राजन् क्षणेन व्यपयान्ति च} %3-66-6

\twolineshloka
{दुःखितो हि भवाँल्लोकांस्तेजसा यदि धक्ष्यते}
{आर्ताः प्रजा नरव्याघ्र क्व नु यास्यन्ति निर्वृतिम्} %3-66-7

\twolineshloka
{लोकस्वभाव एवैष ययातिर्नहुषात्मजः}
{गतः शक्रेण सालोक्यमनयस्तं समस्पृशत्} %3-66-8

\twolineshloka
{महर्षिर्यो वसिष्ठस्तु यः पितुर्नः पुरोहितः}
{अह्ना पुत्रशतं जज्ञे तथैवास्य पुनर्हतम्} %3-66-9

\twolineshloka
{या चेयं जगतो माता सर्वलोकनमस्कृता}
{अस्याश्च चलनं भूमेर्दृश्यते कोसलेश्वर} %3-66-10

\twolineshloka
{यौ धर्मौ जगतो नेत्रौ यत्र सर्वं प्रतिष्ठितम्}
{आदित्यचन्द्रौ ग्रहणमभ्युपेतौ महाबलौ} %3-66-11

\twolineshloka
{सुमहान्त्यपि भूतानि देवाश्च पुरुषर्षभ}
{न दैवस्य प्रमुञ्चन्ति सर्वभूतानि देहिनः} %3-66-12

\twolineshloka
{शक्रादिष्वपि देवेषु वर्तमानौ नयानयौ}
{श्रूयेते नरशार्दूल न त्वं शोचितुमर्हसि} %3-66-13

\twolineshloka
{मृतायामपि वैदेह्यां नष्टायामपि राघव}
{शोचितुं नार्हसे वीर यथान्यः प्राकृतस्तथा} %3-66-14

\twolineshloka
{त्वद्विधा नहि शोचन्ति सततं सर्वदर्शनाः}
{सुमहत्स्वपि कृच्छ्रेषु रामानिर्विण्णदर्शनाः} %3-66-15

\twolineshloka
{तत्त्वतो हि नरश्रेष्ठ बुद्ध्या समनुचिन्तय}
{बुद्ध्या युक्ता महाप्राज्ञा विजानन्ति शुभाशुभे} %3-66-16

\twolineshloka
{अदृष्टगुणदोषाणामध्रुवाणां तु कर्मणाम्}
{नान्तरेण क्रियां तेषां फलमिष्टं च वर्तते} %3-66-17

\twolineshloka
{मामेवं हि पुरा वीर त्वमेव बहुशोक्तवान्}
{अनुशिष्याद्धि को नु त्वामपि साक्षाद् बृहस्पतिः} %3-66-18

\twolineshloka
{बुद्धिश्च ते महाप्राज्ञ देवैरपि दुरन्वया}
{शोकेनाभिप्रसुप्तं ते ज्ञानं सम्बोधयाम्यहम्} %3-66-19

\twolineshloka
{दिव्यं च मानुषं चैवमात्मनश्च पराक्रमम्}
{इक्ष्वाकुवृषभावेक्ष्य यतस्व द्विषतां वधे} %3-66-20

\twolineshloka
{किं ते सर्वविनाशेन कृतेन पुरुषर्षभ}
{तमेव तु रिपुं पापं विज्ञायोद्धर्तुमर्हसि} %3-66-21


॥इत्यार्षे श्रीमद्रामायणे वाल्मीकीये आदिकाव्ये अरण्यकाण्डे औचित्यप्रबोधनम् नाम षट्षष्ठितमः सर्गः ॥३-६६॥
