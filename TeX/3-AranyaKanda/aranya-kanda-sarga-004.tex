\sect{चतुर्थः सर्गः — विराधनिखननम्}

\twolineshloka
{ह्रियमाणौ तु काकुत्स्थौ दृष्ट्वा सीता रघूत्तमौ}
{उच्चैः स्वरेण चुक्रोश प्रगृह्य सुमहाभुजौ} %3-4-1

\twolineshloka
{एष दाशरथी रामः सत्यवाञ्छीलवान् शुचिः}
{रक्षसा रौद्ररूपेण ह्रियते सहलक्ष्मणः} %3-4-2

\twolineshloka
{मामृक्षा भक्षयिष्यन्ति शार्दूलद्वीपिनस्तथा}
{मां हरोत्सृज काकुत्स्थौ नमस्ते राक्षसोत्तम} %3-4-3

\twolineshloka
{तस्यास्तद् वचनं श्रुत्वा वैदेह्या रामलक्ष्मणौ}
{वेगं प्रचक्रतुर्वीरौ वधे तस्य दुरात्मनः} %3-4-4

\twolineshloka
{तस्य रौद्रस्य सौमित्रिः सव्यं बाहुं बभञ्ज ह}
{रामस्तु दक्षिणं बाहुं तरसा तस्य रक्षसः} %3-4-5

\twolineshloka
{स भग्नबाहुः संविग्नः पपाताशु विमूर्च्छितः}
{धरण्यां मेघसंकाशो वज्रभिन्न इवाचलः} %3-4-6

\twolineshloka
{मुष्टिभिर्बाहुभिः पद्भिः सूदयन्तौ तु राक्षसम्}
{उद्यम्योद्यम्य चाप्येनं स्थण्डिले निष्पिपेषतुः} %3-4-7

\twolineshloka
{स विद्धौ बहुभिर्बाणैः खड्गाभ्यां च परिक्षतः}
{निष्पिष्टो बहुधा भूमौ न ममार स राक्षसः} %3-4-8

\twolineshloka
{तं प्रेक्ष्य रामः सुभृशमवध्यमचलोपमम्}
{भयेष्वभयदः श्रीमानिदं वचनमब्रवीत्} %3-4-9

\twolineshloka
{तपसा पुरुषव्याघ्र राक्षसोऽयं न शक्यते}
{शस्त्रेण युधि निर्जेतुं राक्षसं निखनावहे} %3-4-10

\twolineshloka
{कुञ्जरस्येव रौद्रस्य राक्षसस्यास्य लक्ष्मण}
{वनेऽस्मिन् सुमहच्छ्वभ्रं खन्यतां रौद्रवर्चसः} %3-4-11

\twolineshloka
{इत्युक्त्वा लक्ष्मणं रामः प्रदरः खन्यतामिति}
{तस्थौ विराधमाक्रम्य कण्ठे पादेन वीर्यवान्} %3-4-12

\twolineshloka
{तच्छ्रुत्वा राघवेणोक्तं राक्षसः प्रश्रितं वचः}
{इदं प्रोवाच काकुत्स्थं विराधः पुरुषर्षभम्} %3-4-13

\twolineshloka
{हतोऽहं पुरुषव्याघ्र शक्रतुल्यबलेन वै}
{मया तु पूर्वं त्वं मोहान्न ज्ञातः पुरुषर्षभ} %3-4-14

\twolineshloka
{कौसल्या सुप्रजास्तात रामस्त्वं विदितो मया}
{वैदेही च महाभागा लक्ष्मणश्च महायशाः} %3-4-15

\twolineshloka
{अभिशापादहं घोरां प्रविष्टो राक्षसीं तनुम्}
{तुम्बुरुर्नाम गन्धर्वः शप्तो वैश्रवणेन हि} %3-4-16

\twolineshloka
{प्रसाद्यमानश्च मया सोऽब्रवीन्मां महायशाः}
{यदा दाशरथी रामस्त्वां वधिष्यति संयुगे} %3-4-17

\twolineshloka
{तदा प्रकृतिमापन्नो भवान् स्वर्गं गमिष्यति}
{अनुपस्थीयमानो मां स क्रुद्धो व्याजहार ह} %3-4-18

\twolineshloka
{इति वैश्रवणो राजा रम्भासक्तमुवाच ह}
{तव प्रसादान्मुक्तोऽहमभिशापात् सुदारुणात्} %3-4-19

\twolineshloka
{भुवनं स्वं गमिष्यामि स्वस्ति वोऽस्तु परंतप}
{इतो वसति धर्मात्मा शरभङ्गः प्रतापवान्} %3-4-20

\twolineshloka
{अध्यर्धयोजने तात महर्षिः सूर्यसंनिभः}
{तं क्षिप्रमभिगच्छ त्वं स ते श्रेयोऽभिधास्यति} %3-4-21

\twolineshloka
{अवटे चापि मां राम निक्षिप्य कुशली व्रज}
{रक्षसां गतसत्त्वानामेष धर्मः सनातनः} %3-4-22

\twolineshloka
{अवटे ये निधीयन्ते तेषां लोकाः सनातनाः}
{एवमुक्त्वा तु काकुत्स्थं विराधः शरपीडितः} %3-4-23

\twolineshloka
{बभूव स्वर्गसम्प्राप्तो न्यस्तदेहो महाबलः}
{तच्छ्रुत्वा राघवो वाक्यं लक्ष्मणं व्यादिदेश ह} %3-4-24

\twolineshloka
{कुञ्जरस्येव रौद्रस्य राक्षसस्यास्य लक्ष्मण}
{वनेऽस्मिन्सुमहान् श्वभ्रः खन्यतां रौद्रकर्मणः} %3-4-25

\twolineshloka
{इत्युक्त्वा लक्ष्मणं रामः प्रदरः खन्यतामिति}
{तस्थौ विराधमाक्रम्य कण्ठे पादेन वीर्यवान्} %3-4-26

\twolineshloka
{ततः खनित्रमादाय लक्ष्मणः श्वभ्रमुत्तमम्}
{अखनत् पार्श्वतस्तस्य विराधस्य महात्मनः} %3-4-27

\twolineshloka
{तं मुक्तकण्ठमुत्क्षिप्य शङ्कुकर्णं महास्वनम्}
{विराधं प्राक्षिपच्छ्वभ्रे नदन्तं भैरवस्वनम्} %3-4-28

\twolineshloka
{तमाहवे दारुणमाशुविक्रमौ स्थिरावुभौ संयति रामलक्ष्मणौ}
{मुदान्वितौ चिक्षिपतुर्भयावहं नदन्तमुत्क्षिप्य बलेन राक्षसम्} %3-4-29

\twolineshloka
{अवध्यतां प्रेक्ष्य महासुरस्य तौ शितेन शस्त्रेण तदा नरर्षभौ}
{समर्थ्य चात्यर्थविशारदावुभौ बिले विराधस्य वधं प्रचक्रतुः} %3-4-30

\twolineshloka
{स्वयं विराधेन हि मृत्युमात्मनः प्रसह्य रामेण यथार्थमीप्सितः}
{निवेदितः काननचारिणा स्वयं न मे वधः शस्त्रकृतो भवेदिति} %3-4-31

\twolineshloka
{तदेव रामेण निशम्य भाषितं कृता मतिस्तस्य बिलप्रवेशने}
{बिलं च तेनातिबलेन रक्षसा प्रवेश्यमानेन वनं विनादितम्} %3-4-32

\twolineshloka
{प्रहृष्टरूपाविव रामलक्ष्मणौ विराधमुर्व्यां प्रदरे निपात्य तम्}
{ननन्दतुर्वीतभयौ महावने शिलाभिरन्तर्दधतुश्च राक्षसम्} %3-4-33

\twolineshloka
{ततस्तु तौ काञ्चनचित्रकार्मुकौ निहत्य रक्षः परिगृह्य मैथिलीम्}
{विजह्रतुस्तौ मुदितौ महावने दिवि स्थितौ चन्द्रदिवाकराविव} %3-4-34


॥इत्यार्षे श्रीमद्रामायणे वाल्मीकीये आदिकाव्ये अरण्यकाण्डे विराधनिखननम् नाम चतुर्थः सर्गः ॥३-४॥
