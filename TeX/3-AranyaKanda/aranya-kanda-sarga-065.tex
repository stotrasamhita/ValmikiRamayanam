\sect{पञ्चषष्ठितमः सर्गः — क्रोधसंहारप्रार्थना}

\twolineshloka
{तप्यमानं तदा रामं सीताहरणकर्शितम्}
{लोकानामभवे युक्तं सांवर्तकमिवानलम्} %3-65-1

\twolineshloka
{वीक्षमाणं धनुः सज्यं निःश्वसन्तं पुनः पुनः}
{दग्धुकामं जगत् सर्वं युगान्ते च यथा हरम्} %3-65-2

\twolineshloka
{अदृष्टपूर्वं सङ्क्रुद्धं दृष्ट्वा रामं स लक्ष्मणः}
{अब्रवीत् प्राञ्जलिर्वाक्यं मुखेन परिशुष्यता} %3-65-3

\twolineshloka
{पुरा भूत्वा मृदुर्दान्तः सर्वभूतहिते रतः}
{न क्रोधवशमापन्नः प्रकृतिं हातुमर्हसि} %3-65-4

\twolineshloka
{चन्द्रे लक्ष्मीः प्रभा सूर्ये गतिर्वायौ भुवि क्षमा}
{एतच्च नियतं नित्यं त्वयि चानुत्तमं यशः} %3-65-5

\twolineshloka
{एकस्य नापराधेन लोकान् हन्तुं त्वमर्हसि}
{ननु जानामि कस्यायं भग्नः साङ्ग्रामिको रथः} %3-65-6

\twolineshloka
{केन वा कस्य वा हेतोः सयुगः सपरिच्छदः}
{खुरनेमिक्षतश्चायं सिक्तो रुधिरबिन्दुभिः} %3-65-7

\twolineshloka
{देशो निर्वृत्तसङ्ग्रामः सुघोरः पार्थिवात्मज}
{एकस्य तु विमर्दोऽयं न द्वयोर्वदतां वर} %3-65-8

\twolineshloka
{नहि वृत्तं हि पश्यामि बलस्य महतः पदम्}
{नैकस्य तु कृते लोकान् विनाशयितुमर्हसि} %3-65-9

\twolineshloka
{युक्तदण्डा हि मृदवः प्रशान्ता वसुधाधिपाः}
{सदा त्वं सर्वभूतानां शरण्यः परमा गतिः} %3-65-10

\twolineshloka
{को नु दारप्रणाशं ते साधु मन्येत राघव}
{सरितः सागराः शैला देवगन्धर्वदानवाः} %3-65-11

\twolineshloka
{नालं ते विप्रियं कर्तुं दीक्षितस्येव साधवः}
{येन राजन् हृता सीता तमन्वेषितुमर्हसि} %3-65-12

\twolineshloka
{मद्द्वितीयो धनुष्पाणिः सहायैः परमर्षिभिः}
{समुद्रं वा विचेष्यामः पर्वतांश्च वनानि च} %3-65-13

\twolineshloka
{गुहाश्च विविधा घोराः पद्मिन्यो विविधास्तथा}
{देवगन्धर्वलोकांश्च विचेष्यामः समाहिताः} %3-65-14

\threelineshloka
{यावन्नाधिगमिष्यामस्तव भार्यापहारिणम्}
{न चेत् साम्ना प्रदास्यन्ति पत्नीं ते त्रिदशेश्वराः}
{कोसलेन्द्र ततः पश्चात् प्राप्तकालं करिष्यसि} %3-65-15

\twolineshloka
{शीलेन साम्ना विनयेन सीतां नयेन न प्राप्स्यसि चेन्नरेन्द्र}
{ततः समुत्सादय हेमपुङ्खैर्महेन्द्रवज्रप्रतिमैः शरौघैः} %3-65-16


॥इत्यार्षे श्रीमद्रामायणे वाल्मीकीये आदिकाव्ये अरण्यकाण्डे क्रोधसंहारप्रार्थना नाम पञ्चषष्ठितमः सर्गः ॥३-६५॥
