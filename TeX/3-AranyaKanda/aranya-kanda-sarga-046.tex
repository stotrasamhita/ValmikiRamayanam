\sect{षड्चत्वारिंशः सर्गः — रावणभिक्षुसत्कारः}

\twolineshloka
{तया परुषमुक्तस्तु कुपितो राघवानुजः}
{स विकांक्षन् भृशं रामं प्रतस्थे नचिरादिव} %3-46-1

\twolineshloka
{तदासाद्य दशग्रीवः क्षिप्रमन्तरमास्थितः}
{अभिचक्राम वैदेहीं परिव्राजकरूपधृक्} %3-46-2

\twolineshloka
{श्लक्ष्णकाषायसंवीतः शिखी छत्री उपानही}
{वामे चांसेऽवसज्याथ शुभे यष्टिकमण्डलू} %3-46-3

\twolineshloka
{परिव्राजकरूपेण वैदेहीमन्ववर्तत}
{तामाससादातिबलो भ्रातृभ्यां रहितां वने} %3-46-4

\twolineshloka
{रहितां सूर्यचन्द्राभ्यां संध्यामिव महत्तमः}
{तामपश्यत् ततो बालां राजपुत्रीं यशस्विनीम्} %3-46-5

\twolineshloka
{रोहिणीं शशिना हीनां ग्रहवद् भृशदारुणः}
{तमुग्रं पापकर्माणं जनस्थानगता द्रुमाः} %3-46-6

\twolineshloka
{संदृश्य न प्रकम्पन्ते न प्रवाति च मारुतः}
{शीघ्रस्रोताश्च तं दृष्ट्वा वीक्षन्तं रक्तलोचनम्} %3-46-7

\twolineshloka
{स्तिमितं गन्तुमारेभे भयाद् गोदावरी नदी}
{रामस्य त्वन्तरं प्रेप्सुर्दशग्रीवस्तदन्तरे} %3-46-8

\twolineshloka
{उपतस्थे च वैदेहीं भिक्षुरूपेण रावणः}
{अभव्यो भव्यरूपेण भर्तारमनुशोचतीम्} %3-46-9

\twolineshloka
{अभ्यवर्तत वैदेहीं चित्रामिव शनैश्चरः}
{सहसा भव्यरूपेण तृणैः कूप इवावृतः} %3-46-10

\twolineshloka
{अतिष्ठत् प्रेक्ष्य वैदेहीं रामपत्नीं यशस्विनीम्}
{तिष्ठन् सम्प्रेक्ष्य च तदा पत्नीं रामस्य रावणः} %3-46-11

\twolineshloka
{शुभां रुचिरदन्तोष्ठीं पूर्णचन्द्रनिभाननाम्}
{आसीनां पर्णशालायां बाष्पशोकाभिपीडिताम्} %3-46-12

\twolineshloka
{स तां पद्मपलाशाक्षीं पीतकौशेयवासिनीम्}
{अभ्यगच्छत वैदेहीं हृष्टचेता निशाचरः} %3-46-13

\twolineshloka
{दृष्ट्वा कामशराविद्धो ब्रह्मघोषमुदीरयन्}
{अब्रवीत् प्रश्रितं वाक्यं रहिते राक्षसाधिपः} %3-46-14

\twolineshloka
{तामुत्तमां त्रिलोकानां पद्महीनामिव श्रियम्}
{विभ्राजमानां वपुषा रावणः प्रशशंस ह} %3-46-15

\twolineshloka
{रौप्यकाञ्चनवर्णाभे पीतकौशेयवासिनि}
{कमलानां शुभां मालां पद्मिनीव च बिभ्रती} %3-46-16

\twolineshloka
{ह्रीः श्रीः कीर्तिः शुभा लक्ष्मीरप्सरा वा शुभानने}
{भूतिर्वा त्वं वरारोहे रतिर्वा स्वैरचारिणी} %3-46-17

\twolineshloka
{समाः शिखरिणः स्निग्धाः पाण्डुरा दशनास्तव}
{विशाले विमले नेत्रे रक्तान्ते कृष्णतारके} %3-46-18

\twolineshloka
{विशालं जघनं पीनमूरू करिकरोपमौ}
{एतावुपचितौ वृत्तौ संहतौ सम्प्रगल्भितौ} %3-46-19

\twolineshloka
{पीनोन्नतमुखौ कान्तौ स्निग्धतालफलोपमौ}
{मणिप्रवेकाभरणौ रुचिरौ ते पयोधरौ} %3-46-20

\twolineshloka
{चारुस्मिते चारुदति चारुनेत्रे विलासिनि}
{मनो हरसि मे रामे नदीकूलमिवाम्भसा} %3-46-21

\twolineshloka
{करान्तमितमध्यासि सुकेशे संहतस्तनि}
{नैव देवी न गन्धर्वी न यक्षी न च किंनरी} %3-46-22

\twolineshloka
{नैवंरूपा मया नारी दृष्टपूर्वा महीतले}
{रूपमग्र्यं च लोकेषु सौकुमार्यं वयश्च ते} %3-46-23

\twolineshloka
{इह वासश्च कान्तारे चित्तमुन्माथयन्ति मे}
{सा प्रतिक्राम भद्रं ते न त्वं वस्तुमिहार्हसि} %3-46-24

\twolineshloka
{राक्षसानामयं वासो घोराणां कामरूपिणाम्}
{प्रासादाग्राणि रम्याणि नगरोपवनानि च} %3-46-25

\twolineshloka
{सम्पन्नानि सुगन्धीनि युक्तान्याचरितुं त्वया}
{वरं माल्यं वरं गन्धं वरं वस्त्रं च शोभने} %3-46-26

\twolineshloka
{भर्तारं च वरं मन्ये त्वद्युक्तमसितेक्षणे}
{का त्वं भवसि रुद्राणां मरुतां वा शुचिस्मिते} %3-46-27

\twolineshloka
{वसूनां वा वरारोहे देवता प्रतिभासि मे}
{नेह गच्छन्ति गन्धर्वा न देवा न च किन्नराः} %3-46-28

\twolineshloka
{राक्षसानामयं वासः कथं तु त्वमिहागता}
{इह शाखामृगाः सिंहा द्वीपिव्याघ्रमृगा वृकाः} %3-46-29

\twolineshloka
{ऋक्षास्तरक्षवः कङ्काः कथं तेभ्यो न बिभ्यसे}
{मदान्वितानां घोराणां कुञ्जराणां तरस्विनाम्} %3-46-30

\twolineshloka
{कथमेका महारण्ये न बिभेषि वरानने}
{कासि कस्य कुतश्च त्वं किं निमित्तं च दण्डकान्} %3-46-31

\twolineshloka
{एका चरसि कल्याणि घोरान् राक्षससेवितान्}
{इति प्रशस्ता वैदेही रावणेन महात्मना} %3-46-32

\twolineshloka
{द्विजातिवेषेण हि तं दृष्ट्वा रावणमागतम्}
{सर्वैरतिथिसत्कारैः पूजयामास मैथिली} %3-46-33

\twolineshloka
{उपानीयासनं पूर्वं पाद्येनाभिनिमन्त्र्य च}
{अब्रवीत् सिद्धमित्येव तदा तं सौम्यदर्शनम्} %3-46-34

\twolineshloka
{द्विजातिवेषेण समीक्ष्य मैथिली समागतं पात्रकुसुम्भधारिणम्}
{अशक्यमुद्द्वेष्टुमुपायदर्शनान्न्यमन्त्रयद् ब्राह्मणवत् तथागतम्} %3-46-35

\twolineshloka
{इयं बृसी ब्राह्मण काममास्यतामिदं च पाद्यं प्रतिगृह्यतामिति}
{इदं च सिद्धं वनजातमुत्तमं त्वदर्थमव्यग्रमिहोपभुज्यताम्} %3-46-36

\twolineshloka
{निमन्त्र्यमाणः प्रतिपूर्णभाषिणीं नरेन्द्रपत्नीं प्रसमीक्ष्य मैथिलीम्}
{प्रसह्य तस्या हरणे दृढं मनः समर्पयामास वधाय रावणः} %3-46-37

\twolineshloka
{ततः सुवेषं मृगयागतं पतिं प्रतीक्षमाणा सहलक्ष्मणं तदा}
{निरीक्षमाणा हरितं ददर्श तन्महद् वनं नैव तु रामलक्ष्मणौ} %3-46-38


॥इत्यार्षे श्रीमद्रामायणे वाल्मीकीये आदिकाव्ये अरण्यकाण्डे रावणभिक्षुसत्कारः नाम षड्चत्वारिंशः सर्गः ॥३-४६॥
