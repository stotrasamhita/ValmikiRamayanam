\sect{नवमः सर्गः — सीताधर्मावेदनम्}

\twolineshloka
{सुतीक्ष्णेनाभ्यनुज्ञातं प्रस्थितं रघुनन्दनम्}
{हृद्यया स्निग्धया वाचा भर्तारमिदमब्रवीत्} %3-9-1

\twolineshloka
{अधर्मं तु सुसूक्ष्मेण विधिना प्राप्यते महान्}
{निवृत्तेन च शक्योऽयं व्यसनात् कामजादिह} %3-9-2

\twolineshloka
{त्रीण्येव व्यसनान्यत्र कामजानि भवन्त्युत}
{मिथ्यावाक्यं तु परमं तस्माद् गुरुतरावुभौ} %3-9-3

\twolineshloka
{परदाराभिगमनं विना वैरं च रौद्रता}
{मिथ्यावाक्यं न ते भूतं न भविष्यति राघव} %3-9-4

\twolineshloka
{कुतोऽभिलषणं स्त्रीणां परेषां धर्मनाशनम्}
{तव नास्ति मनुष्येन्द्र न चाभूत् ते कदाचन} %3-9-5

\twolineshloka
{मनस्यपि तथा राम न चैतद् विद्यते क्वचित्}
{स्वदारनिरतश्चैव नित्यमेव नृपात्मज} %3-9-6

\twolineshloka
{धर्मिष्ठः सत्यसन्धश्च पितुर्निर्देशकारकः}
{त्वयि धर्मश्च सत्यं च त्वयि सर्वं प्रतिष्ठितम्} %3-9-7

\twolineshloka
{तच्च सर्वं महाबाहो शक्यं वोढुं जितेन्द्रियैः}
{तव वश्येन्द्रियत्वं च जानामि शुभदर्शन} %3-9-8

\twolineshloka
{तृतीयं यदिदं रौद्रं परप्राणाभिहिंसनम्}
{निर्वैरं क्रियते मोहात् तच्च ते समुपस्थितम्} %3-9-9

\twolineshloka
{प्रतिज्ञातस्त्वया वीर दण्डकारण्यवासिनाम्}
{ऋषीणां रक्षणार्थाय वधः संयति रक्षसाम्} %3-9-10

\twolineshloka
{एतन्निमित्तं च वनं दण्डका इति विश्रुतम्}
{प्रस्थितस्त्वं सह भ्रात्रा धृतबाणशरासनः} %3-9-11

\twolineshloka
{ततस्त्वां प्रस्थितं दृष्ट्वा मम चिन्ताकुलं मनः}
{त्वद्धृत्तं चिन्तयन्त्या वै भवेन्निःश्रेयसं हितम्} %3-9-12

\twolineshloka
{नहि मे रोचते वीर गमनं दण्डकान् प्रति}
{कारणं तत्र वक्ष्यामि वदन्त्याः श्रूयतां मम} %3-9-13

\twolineshloka
{त्वं हि बाणधनुष्पाणिर्भ्रात्रा सह वनं गतः}
{दृष्ट्वा वनचरान् सर्वान् कच्चित् कुर्याः शरव्ययम्} %3-9-14

\twolineshloka
{क्षत्रियाणामिह धनुर्हुताशस्येन्धनानि च}
{समीपतः स्थितं तेजोबलमुच्छ्रयते भृशम्} %3-9-15

\twolineshloka
{पुरा किल महाबाहो तपस्वी सत्यवान् शुचिः}
{कस्मिंश्चिदभवत् पुण्ये वने रतमृगद्विजे} %3-9-16

\twolineshloka
{तस्यैव तपसो विघ्नं कर्तुमिन्द्रः शचीपतिः}
{खड्गपाणिरथागच्छदाश्रमं भटरूपधृक्} %3-9-17

\twolineshloka
{तस्मिंस्तदाश्रमपदे निहितः खड्ग उत्तमः}
{स न्यासविधिना दत्तः पुण्ये तपसि तिष्ठतः} %3-9-18

\twolineshloka
{स तच्छस्त्रमनुप्राप्य न्यासरक्षणतत्परः}
{वने तु विचरत्येव रक्षन् प्रत्ययमात्मनः} %3-9-19

\twolineshloka
{यत्र गच्छत्युपादातुं मूलानि च फलानि च}
{न विना याति तं खड्गं न्यासरक्षणतत्परः} %3-9-20

\twolineshloka
{नित्यं शस्त्रं परिवहन् क्रमेण स तपोधनः}
{चकार रौद्रीं स्वां बुद्धिं त्यक्त्वा तपसि निश्चयम्} %3-9-21

\twolineshloka
{ततः स रौद्राभिरतः प्रमत्तोऽधर्मकर्षितः}
{तस्य शस्त्रस्य संवासाज्जगाम नरकं मुनिः} %3-9-22

\twolineshloka
{एवमेतत् पुरावृत्तं शस्त्रसंयोगकारणम्}
{अग्निसंयोगवद्धेतुः शस्त्रसंयोग उच्यते} %3-9-23

\twolineshloka
{स्नेहाच्च बहुमानाच्च स्मारये त्वां तु शिक्षये}
{न कथञ्चन सा कार्या गृहीतधनुषा त्वया} %3-9-24

\twolineshloka
{बुद्धिर्वैरं विना हन्तुं राक्षसान् दण्डकाश्रितान्}
{अपराधं विना हन्तुं लोको वीर न मंस्यते} %3-9-25

\twolineshloka
{क्षत्रियाणां तु वीराणां वनेषु नियतात्मनाम्}
{धनुषा कार्यमेतावदार्तानामभिरक्षणम्} %3-9-26

\twolineshloka
{क्व च शस्त्रं क्व च वनं क्व च क्षात्रं तपः क्व च}
{व्याविद्धमिदमस्माभिर्देशधर्मस्तु पूज्यताम्} %3-9-27

\twolineshloka
{कदर्यकलुषा बुद्धिर्जायते शस्त्रसेवनात्}
{पुनर्गत्वा त्वयोध्यायां क्षत्रधर्मं चरिष्यसि} %3-9-28

\twolineshloka
{अक्षया तु भवेत् प्रीतिः श्वश्रूश्वशुरयोर्मम}
{यदि राज्यं हि सन्न्यस्य भवेस्त्वं निरतो मुनिः} %3-9-29

\twolineshloka
{धर्मादर्थः प्रभवति धर्मात् प्रभवते सुखम्}
{धर्मेण लभते सर्वं धर्मसारमिदं जगत्} %3-9-30

\twolineshloka
{आत्मानं नियमैस्तैस्तैः कर्षयित्वा प्रयत्नतः}
{प्राप्तये निपुणैर्धर्मो न सुखाल्लभते सुखम्} %3-9-31

\twolineshloka
{नित्यं शुचिमतिः सौम्य चर धर्मं तपोवने}
{सर्वं तु विदितं तुभ्यं त्रैलोक्यामपि तत्त्वतः} %3-9-32

\twolineshloka
{स्त्रीचापलादेतदुपाहृतं मे धर्मं च वक्तुं तव कः समर्थः}
{विचार्य बुद्ध्या तु सहानुजेन यद् रोचते तत् कुरु माचिरेण} %3-9-33


॥इत्यार्षे श्रीमद्रामायणे वाल्मीकीये आदिकाव्ये अरण्यकाण्डे सीताधर्मावेदनम् नाम नवमः सर्गः ॥३-९॥
