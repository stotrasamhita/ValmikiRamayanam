\sect{त्रिसप्ततितमः सर्गः — ऋष्यमूकमार्गकथनम्}

\twolineshloka
{दर्शयित्वा तु रामाय सीतायाः परिमार्गणे}
{वाक्यमन्वर्थमर्थज्ञः कबन्धः पुनरब्रवीत्} %3-73-1

\twolineshloka
{एष राम शिवः पन्था यत्रैते पुष्पिता द्रुमाः}
{प्रतीचीं दिशमाश्रित्य प्रकाशन्ते मनोरमाः} %3-73-2

\twolineshloka
{जम्बूप्रियालपनसा न्यग्रोधप्लक्षतिन्दुकाः}
{अश्वत्थाः कर्णिकाराश्च चूताश्चान्ये च पादपाः} %3-73-3

\twolineshloka
{धन्वना नागवृक्षाश्च तिलका नक्तमालकाः}
{नीलाशोकाः कदम्बाश्च करवीराश्च पुष्पिताः} %3-73-4

\twolineshloka
{अग्निमुख्या अशोकाश्च सुरक्ताः पारिभद्रकाः}
{तानारुह्याथवा भूमौ पातयित्वा च तान् बलात्} %3-73-5

\twolineshloka
{फलान्यमृतकल्पानि भक्षयित्वा गमिष्यथः}
{तदतिक्रम्य काकुत्स्थ वनं पुष्पितपादपम्} %3-73-6

\twolineshloka
{नन्दनप्रतिमं चान्यत् कुरवस्तूत्तरा इव}
{सर्वकालफला यत्र पादपा मधुरस्रवाः} %3-73-7

\twolineshloka
{सर्वे च ऋतवस्तत्र वने चैत्ररथे यथा}
{फलभारनतास्तत्र महाविटपधारिणः} %3-73-8

\twolineshloka
{शोभन्ते सर्वतस्तत्र मेघपर्वतसन्निभाः}
{तानारुह्याथवा भूमौ पातयित्वाथवा सुखम्} %3-73-9

\twolineshloka
{फलान्यमृतकल्पानि लक्ष्मणस्ते प्रदास्यति}
{चङ्क्रमन्तौ वरान् शैलान् शैलाच्छैलं वनाद् वनम्} %3-73-10

\twolineshloka
{ततः पुष्करिणीं वीरौ पम्पां नाम गमिष्यथः}
{अशर्करामविभ्रंशां समतीर्थामशैवलाम्} %3-73-11

\twolineshloka
{राम सञ्जातवालूकां कमलोत्पलशोभिताम्}
{तत्र हंसाः प्लवाः क्रौञ्चाः कुरराश्चैव राघव} %3-73-12

\twolineshloka
{वल्गुस्वरा निकूजन्ति पम्पासलिलगोचराः}
{नोद्विजन्ते नरान् दृष्ट्वा वधस्याकोविदाः शुभाः} %3-73-13

\twolineshloka
{घृतपिण्डोपमान् स्थूलांस्तान् द्विजान् भक्षयिष्यथः}
{रोहितान् वक्रतुण्डांश्च नलमीनांश्च राघव} %3-73-14

\twolineshloka
{पम्पायामिषुभिर्मत्स्यांस्तत्र राम वरान् हतान्}
{निस्त्वक्पक्षानयस्तप्तानकृशानैककण्टकान्} %3-73-15

\twolineshloka
{तव भक्त्या समायुक्तो लक्ष्मणः सम्प्रदास्यति}
{भृशं तान् खादतो मत्स्यान् पम्पायाः पुष्पसञ्चये} %3-73-16

\twolineshloka
{पद्मगन्धि शिवं वारि सुखशीतमनामयम्}
{उद्धृत्य स तदाक्लिष्टं रूप्यस्फटिकसन्निभम्} %3-73-17

\twolineshloka
{अथ पुष्करपर्णेन लक्ष्मणः पाययिष्यति}
{स्थूलान् गिरिगुहाशय्यान् वानरान् वनचारिणः} %3-73-18

\twolineshloka
{सायाह्ने विचरन् राम दर्शयिष्यति लक्ष्मणः}
{अपां लोभादुपावृत्तान् वृषभानिव नर्दतः} %3-73-19

\twolineshloka
{स्थूलान् पीतांश्च पम्पायां द्रक्ष्यसि त्वं नरोत्तम}
{सायाह्ने विचरन् राम विटपी माल्यधारिणः} %3-73-20

\twolineshloka
{शिवोदकं च पम्पायां दृष्ट्वा शोकं विहास्यसि}
{सुमनोभिश्चितास्तत्र तिलका नक्तमालकाः} %3-73-21

\twolineshloka
{उत्पलानि च फुल्लानि पङ्कजानि च राघव}
{न तानि कश्चिन्माल्यानि तत्रारोपयिता नरः} %3-73-22

\twolineshloka
{न च वै म्लानतां यान्ति न च शीर्यन्ति राघव}
{मतङ्गशिष्यास्तत्रासन्नृषयः सुसमाहिताः} %3-73-23

\twolineshloka
{तेषां भाराभितप्तानां वन्यमाहरतां गुरोः}
{ये प्रपेतुर्महीं तूर्णं शरीरात् स्वेदबिन्दवः} %3-73-24

\twolineshloka
{तानि माल्यानि जातानि मुनीनां तपसा तदा}
{स्वेदबिन्दुसमुत्थानि न विनश्यन्ति राघव} %3-73-25

\twolineshloka
{तेषां गतानामद्यापि दृश्यते परिचारिणी}
{श्रमणी शबरी नाम काकुत्स्थ चिरजीविनी} %3-73-26

\twolineshloka
{त्वां तु धर्मे स्थिता नित्यं सर्वभूतनमस्कृतम्}
{दृष्ट्वा देवोपमं राम स्वर्गलोकं गमिष्यति} %3-73-27

\twolineshloka
{ततस्तद्राम पम्पायास्तीरमाश्रित्य पश्चिमम्}
{आश्रमस्थानमतुलं गुह्यं काकुत्स्थ पश्यसि} %3-73-28

\twolineshloka
{न तत्राक्रमितुं नागाः शक्नुवन्ति तदाश्रमे}
{ऋषेस्तस्य मतङ्गस्य विधानात् तच्च काननम्} %3-73-29

\twolineshloka
{मतङ्गवनमित्येव विश्रुतं रघुनन्दन}
{तस्मिन् नन्दनसङ्काशे देवारण्योपमे वने} %3-73-30

\twolineshloka
{नानाविहगसङ्कीर्णे रंस्यसे राम निर्वृतः}
{ऋष्यमूकस्तु पम्पायाः पुरस्तात् पुष्पितद्रुमः} %3-73-31

\twolineshloka
{सुदुःखारोहणश्चैव शिशुनागाभिरक्षितः}
{उदारो ब्रह्मणा चैव पूर्वकालेऽभिनिर्मितः} %3-73-32

\twolineshloka
{शयानः पुरुषो राम तस्य शैलस्य मूर्धनि}
{यत् स्वप्नं लभते वित्तं तत् प्रबुद्धोऽधिगच्छति} %3-73-33

\twolineshloka
{यस्त्वेनं विषमाचारः पापकर्माधिरोहति}
{तत्रैव प्रहरन्त्येनं सुप्तमादाय राक्षसाः} %3-73-34

\twolineshloka
{तत्रापि शिशुनागानामाक्रन्दः श्रूयते महान्}
{क्रीडतां राम पम्पायां मतङ्गाश्रमवासिनाम्} %3-73-35

\twolineshloka
{सक्ता रुधिरधाराभिः संहत्य परमद्विपाः}
{प्रचरन्ति पृथक्कीर्णा मेघवर्णास्तरस्विनः} %3-73-36

\twolineshloka
{ते तत्र पीत्वा पानीयं विमलं चारु शोभनम्}
{अत्यन्तसुखसंस्पर्शं सर्वगन्धसमन्वितम्} %3-73-37

\twolineshloka
{निर्वृत्ताः संविगाहन्ते वनानि वनगोचराः}
{ऋक्षांश्च द्वीपिनश्चैव नीलकोमलकप्रभान्} %3-73-38

\twolineshloka
{रुरूनपेतानजयान् दृष्ट्वा शोकं प्रहास्यसि}
{राम तस्य तु शैलस्य महती शोभते गुहा} %3-73-39

\threelineshloka
{शिलापिधाना काकुत्स्थ दुःखं चास्याः प्रवेशनम्}
{तस्या गुहायाः प्राग्द्वारे महान् शीतोदको ह्रदः बहुमूलफलो रम्यो नानानगसमाकुलः}
{तस्यां वसति धर्मात्मा सुग्रीवः सह वानरैः} %3-73-40

\twolineshloka
{कदाचिच्छिखरे तस्य पर्वतस्यापि तिष्ठति}
{कबन्धस्त्वनुशास्यैवं तावुभौ रामलक्ष्मणौ} %3-73-41

\twolineshloka
{स्रग्वी भास्करवर्णाभः खे व्यरोचत वीर्यवान्}
{तं तु खस्थं महाभागं तावुभौ रामलक्ष्मणौ} %3-73-42

\twolineshloka
{प्रस्थितौ त्वं व्रजस्वेति वाक्यमूचतुरन्तिके}
{गम्यतां कार्यसिद्ध्यर्थमिति तावब्रवीत् स च} %3-73-43

\onelineshloka
{सुप्रीतौ तावनुज्ञाप्य कबन्धः प्रस्थितस्तदा} %3-73-44

\twolineshloka
{स तत् कबन्धः प्रतिपद्य रूपं वृतः श्रिया भास्वरसर्वदेहः}
{निदर्शयन् राममवेक्ष्य खस्थः सख्यं कुरुष्वेति तदाभ्युवाच} %3-73-45


॥इत्यार्षे श्रीमद्रामायणे वाल्मीकीये आदिकाव्ये अरण्यकाण्डे ऋष्यमूकमार्गकथनम् नाम त्रिसप्ततितमः सर्गः ॥३-७३॥
