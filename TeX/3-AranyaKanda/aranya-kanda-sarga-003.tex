\sect{तृतीयः सर्गः — विराधप्रहारः}

\twolineshloka
{अथोवाच पुनर्वाक्यं विराधः पूरयन् वनम्}
{पृच्छतो मम हि ब्रूतं कौ युवां क्व गमिष्यथः} %3-3-1

\twolineshloka
{तमुवाच ततो रामो राक्षसं ज्वलिताननम्}
{पृच्छन्तं सुमहातेजा इक्ष्वाकुकुलमात्मनः} %3-3-2

\twolineshloka
{क्षत्रियौ वृत्तसम्पन्नौ विद्धि नौ वनगोचरौ}
{त्वां तु वेदितुमिच्छावः कस्त्वं चरसि दण्डकान्} %3-3-3

\twolineshloka
{तमुवाच विराधस्तु रामं सत्यपराक्रमम्}
{हन्त वक्ष्यामि ते राजन् निबोध मम राघव} %3-3-4

\twolineshloka
{पुत्रः किल जवस्याहं माता मम शतह्रदा}
{विराध इति मामाहुः पृथिव्यां सर्वराक्षसाः} %3-3-5

\twolineshloka
{तपसा चाभिसम्प्राप्ता ब्रह्मणो हि प्रसादजा}
{शस्त्रेणावध्यता लोकेऽच्छेद्याभेद्यत्वमेव च} %3-3-6

\twolineshloka
{उत्सृज्य प्रमदामेनामनपेक्षौ यथागतम्}
{त्वरमाणौ पलायेथां न वां जीवितमाददे} %3-3-7

\twolineshloka
{तं रामः प्रत्युवाचेदं कोपसंरक्तलोचनः}
{राक्षसं विकृताकारं विराधं पापचेतसम्} %3-3-8

\twolineshloka
{क्षुद्र धिक् त्वां तु हीनार्थं मृत्युमन्वेषसे ध्रुवम्}
{रणे प्राप्स्यसि सन्तिष्ठ न मे जीवन् विमोक्ष्यसे} %3-3-9

\twolineshloka
{ततः सज्यं धनुः कृत्वा रामः सुनिशितान् शरान्}
{सुशीघ्रमभिसन्धाय राक्षसं निजघान ह} %3-3-10

\twolineshloka
{धनुषा ज्यागुणवता सप्त बाणान् मुमोच ह}
{रुक्मपुङ्खान् महावेगान् सुपर्णानिलतुल्यगान्} %3-3-11

\twolineshloka
{ते शरीरं विराधस्य भित्त्वा बर्हिणवाससः}
{निपेतुः शोणितादिग्धा धरण्यां पावकोपमाः} %3-3-12

\twolineshloka
{स विद्धो न्यस्य वैदेहीं शूलमुद्यम्य राक्षसः}
{अभ्यद्रवत् सुसङ्क्रुद्धस्तदा रामं सलक्ष्मणम्} %3-3-13

\twolineshloka
{स विनद्य महानादं शूलं शक्रध्वजोपमम्}
{प्रगृह्याशोभत तदा व्यात्तानन इवान्तकः} %3-3-14

\twolineshloka
{अथ तौ भ्रातरौ दीप्तं शरवर्षं ववर्षतुः}
{विराधे राक्षसे तस्मिन् कालान्तकयमोपमे} %3-3-15

\twolineshloka
{स प्रहस्य महारौद्रः स्थित्वाजृम्भत राक्षसः}
{जृम्भमाणस्य ते बाणाः कायान्निष्पेतुराशुगाः} %3-3-16

\twolineshloka
{स्पर्शात् तु वरदानेन प्राणान् संरोध्य राक्षसः}
{विराधः शूलमुद्यम्य राघवावभ्यधावत} %3-3-17

\twolineshloka
{तच्छूलं वज्रसङ्काशं गगने ज्वलनोपमम्}
{द्वाभ्यां शराभ्यां चिच्छेद रामः शस्त्रभृतां वरः} %3-3-18

\twolineshloka
{तद् रामविशिखैश्छिन्नं शूलं तस्यापतद् भुवि}
{पपाताशनिना छिन्नं मेरोरिव शिलातलम्} %3-3-19

\twolineshloka
{तौ खड्गौ क्षिप्रमुद्यम्य कृष्णसर्पाविवोद्यतौ}
{तूर्णमापेततुस्तस्य तदा प्रहरतां बलात्} %3-3-20

\twolineshloka
{स वध्यमानः सुभृशं भुजाभ्यां परिगृह्य तौ}
{अप्रकम्प्यौ नरव्याघ्रौ रौद्रः प्रस्थातुमैच्छत} %3-3-21

\twolineshloka
{तस्याभिप्रायमाज्ञाय रामो लक्ष्मणमब्रवीत्}
{वहत्वयमलं तावत् पथानेन तु राक्षसः} %3-3-22

\twolineshloka
{यथा चेच्छति सौमित्रे तथा वहतु राक्षसः}
{अयमेव हि नः पन्था येन याति निशाचरः} %3-3-23

\twolineshloka
{स तु स्वबलवीर्येण समुत्क्षिप्य निशाचरः}
{बालाविव स्कन्धगतौ चकारातिबलोद्धतः} %3-3-24

\twolineshloka
{तावारोप्य ततः स्कन्धं राघवौ रजनीचरः}
{विराधो विनदन् घोरं जगामाभिमुखो वनम्} %3-3-25

\twolineshloka
{वनं महामेघनिभं प्रविष्टो द्रुमैर्महद्भिर्विविधैरुपेतम्}
{नानाविधैः पक्षिकुलैर्विचित्रं शिवायुतं व्यालमृगैर्विकीर्णम्} %3-3-26


॥इत्यार्षे श्रीमद्रामायणे वाल्मीकीये आदिकाव्ये अरण्यकाण्डे विराधप्रहारः नाम तृतीयः सर्गः ॥३-३॥
