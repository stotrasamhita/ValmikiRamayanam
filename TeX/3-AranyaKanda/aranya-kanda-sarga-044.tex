\sect{चतुश्चत्वारिंशः सर्गः — मारीचवञ्चना}

\twolineshloka
{तथा तु तं समादिश्य भ्रातरं रघुनन्दनः}
{बबन्धासिं महातेजा जाम्बूनदमयत्सरुम्} %3-44-1

\twolineshloka
{ततस्त्रिविनतं चापमादायात्मविभूषणम्}
{आबध्य च कपालौ द्वौ जगामोदग्रविक्रमः} %3-44-2

\twolineshloka
{तं वन्यराजो राजेन्द्रमापतन्तं निरीक्ष्य वै}
{बभूवान्तर्हितस्त्रासात् पुनः सन्दर्शनेऽभवत्} %3-44-3

\twolineshloka
{बद्धासिर्धनुरादाय प्रदुद्राव यतो मृगः}
{तं स्म पश्यति रूपेण द्योतयन्तमिवाग्रतः} %3-44-4

\twolineshloka
{अवेक्ष्यावेक्ष्य धावन्तं धनुष्पाणिर्महावने}
{अतिवृत्तमिवोत्पाताल्लोभयानं कदाचन} %3-44-5

\twolineshloka
{शङ्कितं तु समुद्भ्रान्तमुत्पतन्तमिवाम्बरम्}
{दृश्यमानमदृश्यं च वनोद्देशेषु केषुचित्} %3-44-6

\twolineshloka
{छिन्नाभ्रैरिव संवीतं शारदं चन्द्रमण्डलम्}
{मुहूर्तादेव ददृशे मुहुर्दूरात् प्रकाशते} %3-44-7

\twolineshloka
{दर्शनादर्शनेनैव सोऽपाकर्षत राघवम्}
{स दूरमाश्रमस्यास्य मारीचो मृगतां गतः} %3-44-8

\twolineshloka
{आसीत् क्रुद्धस्तु काकुत्स्थो विवशस्तेन मोहितः}
{अथावतस्थे सुश्रान्तश्छायामाश्रित्य शाद्वले} %3-44-9

\twolineshloka
{स तमुन्मादयामास मृगरूपो निशाचरः}
{मृगैः परिवृतोऽथान्यैरदूरात् प्रत्यदृश्यत} %3-44-10

\twolineshloka
{ग्रहीतुकामं दृष्ट्वा तं पुनरेवाभ्यधावत}
{तत्क्षणादेव सन्त्रासात् पुनरन्तर्हितोऽभवत्} %3-44-11

\twolineshloka
{पुनरेव ततो दूराद् वृक्षखण्डाद् विनिःसृतः}
{दृष्ट्वा रामो महातेजास्तं हन्तुं कृतनिश्चयः} %3-44-12

\twolineshloka
{भूयस्तु शरमुद्धृत्य कुपितस्तत्र राघवः}
{सूर्यरश्मिप्रतीकाशं ज्वलन्तमरिमर्दनम्} %3-44-13

\twolineshloka
{सन्धाय सुदृढे चापे विकृष्य बलवद्बली}
{तमेव मृगमुद्दिश्य श्वसन्तमिव पन्नगम्} %3-44-14

\twolineshloka
{मुमोच ज्वलितं दीप्तमस्त्रं ब्रह्मविनिर्मितम्}
{शरीरं मृगरूपस्य विनिर्भिद्य शरोत्तमः} %3-44-15

\twolineshloka
{मारीचस्यैव हृदयं बिभेदाशनिसन्निभः}
{तालमात्रमथोत्प्लुत्य न्यपतत् स भृशातुरः} %3-44-16

\twolineshloka
{व्यनदद् भैरवं नादं धरण्यामल्पजीवितः}
{म्रियमाणस्तु मारीचो जहौ तां कृत्रिमां तनुम्} %3-44-17

\twolineshloka
{स्मृत्वा तद्वचनं रक्षो दध्यौ केन तु लक्ष्मणम्}
{इह प्रस्थापयेत् सीता तां शून्ये रावणो हरेत्} %3-44-18

\twolineshloka
{स प्राप्तकालमाज्ञाय चकार च ततः स्वनम्}
{सदृशं राघवस्येव हा सीते लक्ष्मणेति च} %3-44-19

\twolineshloka
{तेन मर्मणि निर्विद्धं शरेणानुपमेन हि}
{मृगरूपं तु तत् त्यक्त्वा राक्षसं रूपमास्थितः} %3-44-20

\twolineshloka
{चक्रे स सुमहाकायं मारीचो जीवितं त्यजन्}
{तं दृष्ट्वा पतितं भूमौ राक्षसं भीमदर्शनम्} %3-44-21

\twolineshloka
{रामो रुधिरसिक्ताङ्गं चेष्टमानं महीतले}
{जगाम मनसा सीतां लक्ष्मणस्य वचः स्मरन्} %3-44-22

\twolineshloka
{मारीचस्य तु मायैषा पूर्वोक्तं लक्ष्मणेन तु}
{तत् तथा ह्यभवच्चाद्य मारीचोऽयं मया हतः} %3-44-23

\twolineshloka
{हा सीते लक्ष्मणेत्येवमाक्रुश्य तु महास्वनम्}
{ममार राक्षसः सोऽयं श्रुत्वा सीता कथं भवेत्} %3-44-24

\twolineshloka
{लक्ष्मणश्च महाबाहुः कामवस्थां गमिष्यति}
{इति सञ्चिन्त्य धर्मात्मा रामो हृष्टतनूरुहः} %3-44-25

\twolineshloka
{तत्र रामं भयं तीव्रमाविवेश विषादजम्}
{राक्षसं मृगरूपं तं हत्वा श्रुत्वा च तत्स्वनम्} %3-44-26

\twolineshloka
{निहत्य पृषतं चान्यं मांसमादाय राघवः}
{त्वरमाणो जनस्थानं ससाराभिमुखं तदा} %3-44-27


॥इत्यार्षे श्रीमद्रामायणे वाल्मीकीये आदिकाव्ये अरण्यकाण्डे मारीचवञ्चना नाम चतुश्चत्वारिंशः सर्गः ॥३-४४॥
