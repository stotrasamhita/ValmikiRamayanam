\sect{षोडशः सर्गः — हेमन्तवर्णनम्}

\twolineshloka
{अथ पञ्चवटीं गच्छन्नन्तरा रघुनन्दनः}
{आससाद महाकायं गृध्रं भीमपराक्रमम्} %3-16-1

\twolineshloka
{तं दृष्ट्वा तौ महाभागौ वनस्थं रामलक्ष्मणौ}
{मेनाते राक्षसं पक्षिं ब्रुवाणौ को भवानिति} %3-16-2

\twolineshloka
{ततो मधुरया वाचा सौम्यया प्रीणयन्निव}
{उवाच वत्स मां विद्धि वयस्यं पितुरात्मनः} %3-16-3

\twolineshloka
{स तं पितृसखं मत्वा पूजयामास राघवः}
{स तस्य कुलमव्यग्रमथ पप्रच्छ नाम च} %3-16-4

\twolineshloka
{रामस्य वचनं श्रुत्वा कुलमात्मानमेव च}
{आचचक्षे द्विजस्तस्मै सर्वभूतसमुद्भवम्} %3-16-5

\twolineshloka
{पूर्वकाले महाबाहो ये प्रजापतयोऽभवन्}
{तान् मे निगदतः सर्वानादितः शृणु राघव} %3-16-6

\twolineshloka
{कर्दमः प्रथमस्तेषां विकृतस्तदनन्तरम्}
{शेषश्च संश्रयश्चैव बहुपुत्रश्च वीर्यवान्} %3-16-7

\twolineshloka
{स्थाणुर्मरीचरत्रिश्च क्रतुश्चैव महाबलः}
{पुलस्त्यश्चाङ्गिराश्चैव प्रचेताः पुलहस्तथा} %3-16-8

\twolineshloka
{दक्षो विवस्वानपरोऽरिष्टनेमिश्च राघव}
{कश्यपश्च महातेजास्तेषामासीच्च पश्चिमः} %3-16-9

\twolineshloka
{प्रजापतेस्तु दक्षस्य बभूवुरिति विश्रुताः}
{षष्टिर्दुहितरो राम यशस्विन्यो महायशः} %3-16-10

\twolineshloka
{कश्यपः प्रतिजग्राह तासामष्टौ सुमध्यमाः}
{अदितिं च दितिं चैव दनूमपि च कालकाम्} %3-16-11

\twolineshloka
{ताम्रां क्रोधवशां चैव मनुं चाप्यनलामपि}
{तास्तु कन्यास्ततः प्रीतः कश्यपः पुनरब्रवीत्} %3-16-12

\twolineshloka
{पुत्रांस्त्रैलोक्यभर्तॄन् वै जनयिष्यथ मत्समान्}
{अदितिस्तन्मना राम दितिश्च दनुरेव च} %3-16-13

\twolineshloka
{कालका च महाबाहो शेषास्त्वमनसोऽभवन्}
{अदित्यां जज्ञिरे देवास्त्रयस्त्रिंशदरिंदम} %3-16-14

\twolineshloka
{आदित्या वसवो रुद्रा अश्विनौ च परंतप}
{दितिस्त्वजनयत् पुत्रान् दैत्यांस्तात यशस्विनः} %3-16-15

\twolineshloka
{तेषामियं वसुमती पुराऽऽसीत् सवनार्णवा}
{दनुस्त्वजनयत् पुत्रमश्वग्रीवमरिंदम} %3-16-16

\twolineshloka
{नरकं कालकं चैव कालकापि व्यजायत}
{क्रौञ्चीं भासीं तथा श्येनीं धृतराष्ट्रीं तथा शुकीम्} %3-16-17

\twolineshloka
{ताम्रा तु सुषुवे कन्याः पञ्चैता लोकविश्रुताः}
{उलूकाञ्जनयत् क्रौञ्ची भासी भासान् व्यजायत} %3-16-18

\twolineshloka
{श्येनी श्येनांश्च गृध्रांश्च व्यजायत सुतेजसः}
{धृतराष्ट्री तु हंसांश्च कलहंसाश्च सर्वशः} %3-16-19

\twolineshloka
{चक्रवाकांश्च भद्रं ते विजज्ञे सापि भामिनी}
{शुकी नतां विजज्ञे तु नतायां विनता सुता} %3-16-20

\twolineshloka
{दश क्रोधवशा राम विजज्ञेऽप्यात्मसंभवाः}
{मृगीं च मृगमन्दां च हरीं भद्रमदामपि} %3-16-21

\twolineshloka
{मातङ्गीमथ शार्दूलीं श्वेतां च सुरभीं तथा}
{सर्वलक्षणसम्पन्नां सुरसां कद्रुकामपि} %3-16-22

\twolineshloka
{अपत्यं तु मृगाः सर्वे मृग्या नरवरोत्तम}
{ऋक्षाश्च मृगमन्दायाः सृमराश्चमरास्तथा} %3-16-23

\twolineshloka
{ततस्त्विरावतीं नाम जज्ञे भद्रमदा सुताम्}
{तस्यास्त्वैरावतः पुत्रो लोकनाथो महागजः} %3-16-24

\twolineshloka
{हर्याश्च हरयोऽपत्यं वानराश्च तपस्विनः}
{गोलाङ्गूलाश्च शार्दूली व्याघ्रांश्चाजनयत् सुतान्} %3-16-25

\twolineshloka
{मातङ्ग्यास्त्वथ मातङ्गा अपत्यं मनुजर्षभ}
{दिशागजं तु काकुत्स्थ श्वेता व्यजनयत् सुतम्} %3-16-26

\twolineshloka
{ततो दुहितरौ राम सुरभिर्द्वे व्यजायत}
{रोहिणीं नाम भद्रं ते गन्धर्वीं च यशस्विनीम्} %3-16-27

\twolineshloka
{रोहिण्यजनयद् गावो गन्धर्वी वाजिनः सुतान्}
{सुरसाजनयन्नागान् राम कद्रूश्च पन्नगान्} %3-16-28

\twolineshloka
{मनुर्मनुष्याञ्जनयत् कश्यपस्य महात्मनः}
{ब्राह्मणान् क्षत्रियान् वैश्यान् शूद्रांश्च मनुजर्षभ} %3-16-29

\twolineshloka
{मुखतो ब्राह्मणा जाता उरसः क्षत्रियास्तथा}
{ऊरुभ्यां जज्ञिरे वैश्याः पद्भ्यां शूद्रा इति श्रुतिः} %3-16-30

\twolineshloka
{सर्वान् पुण्यफलान् वृक्षाननलापि व्यजायत}
{विनता च शुकीपौत्री कद्रूश्च सुरसास्वसा} %3-16-31

\twolineshloka
{कद्रूर्नागसहस्रं तु विजज्ञे धरणीधरान्}
{द्वौ पुत्रौ विनतायास्तु गरुडोऽरुण एव च} %3-16-32

\twolineshloka
{तस्माज्जातोऽहमरुणात् सम्पातिश्च ममाग्रजः}
{जटायुरिति मां विद्धि श्येनीपुत्रमरिंदम} %3-16-33

\threelineshloka
{सोऽहं वाससहायस्ते भविष्यामि यदीच्छसि}
{इदं दुर्गं हि कान्तारं मृगराक्षससेवितम्}
{सीतां च तात रक्षिष्ये त्वयि याते सलक्ष्मणे} %3-16-34

\twolineshloka
{जटायुषं तु प्रतिपूज्य राघवो मुदा परिष्वज्य च संनतोऽभवत्}
{पितुर्हि शुश्राव सखित्वमात्मवाञ्जटायुषा संकथितं पुनः पुनः} %3-16-35

\twolineshloka
{स तत्र सीतां परिदाय मैथिलीं सहैव तेनातिबलेन पक्षिणा}
{जगाम तां पञ्चवटीं सलक्ष्मणो रिपून् दिधक्षन् शलभानिवानलः} %3-16-36


॥इत्यार्षे श्रीमद्रामायणे वाल्मीकीये आदिकाव्ये अरण्यकाण्डे हेमन्तवर्णनम् नाम षोडशः सर्गः ॥३-१६॥
