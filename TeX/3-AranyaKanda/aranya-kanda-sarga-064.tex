\sect{चतुःषष्ठितमः सर्गः — रामक्रोधः}

\twolineshloka
{स दीनो दीनया वाचा लक्ष्मणं वाक्यमब्रवीत्}
{शीघ्रं लक्ष्मण जानीहि गत्वा गोदावरीं नदीम्} %3-64-1

\twolineshloka
{अपि गोदावरीं सीता पद्मान्यानयितुं गता}
{एवमुक्तस्तु रामेण लक्ष्मणः पुनरेव हि} %3-64-2

\twolineshloka
{नदीं गोदावरीं रम्यां जगाम लघुविक्रमः}
{तां लक्ष्मणस्तीर्थवतीं विचित्वा राममब्रवीत्} %3-64-3

\twolineshloka
{नैनां पश्यामि तीर्थेषु क्रोशतो न शृणोति मे}
{कं नु सा देशमापन्ना वैदेही क्लेशनाशिनी} %3-64-4

\twolineshloka
{नहि तं वेद्मि वै राम यत्र सा तनुमध्यमा}
{लक्ष्मणस्य वचः श्रुत्वा दीनः संतापमोहितः} %3-64-5

\twolineshloka
{रामः समभिचक्राम स्वयं गोदावरीं नदीम्}
{स तामुपस्थितो रामः क्व सीतेत्येवमब्रवीत्} %3-64-6

\twolineshloka
{भूतानि राक्षसेन्द्रेण वधार्हेण हृतामपि}
{न तां शशंसू रामाय तथा गोदावरी नदी} %3-64-7

\twolineshloka
{ततः प्रचोदिता भूतैः शंस चास्मै प्रियामिति}
{न च सा ह्यवदत् सीतां पृष्टा रामेण शोचता} %3-64-8

\twolineshloka
{रावणस्य च तद्रूपं कर्मापि च दुरात्मनः}
{ध्यात्वा भयात् तु वैदेहीं सा नदी न शशंस ह} %3-64-9

\twolineshloka
{निराशस्तु तया नद्या सीताया दर्शने कृतः}
{उवाच रामः सौमित्रिं सीतादर्शनकर्शितः} %3-64-10

\twolineshloka
{एषा गोदावरी सौम्य किंचिन्न प्रतिभाषते}
{किं नु लक्ष्मण वक्ष्यामि समेत्य जनकं वचः} %3-64-11

\twolineshloka
{मातरं चैव वैदेह्या विना तामहमप्रियम्}
{या मे राज्यविहीनस्य वने वन्येन जीवतः} %3-64-12

\twolineshloka
{सर्वं व्यपानयच्छोकं वैदेही क्व नु सा गता}
{ज्ञातिवर्गविहीनस्य वैदेहीमप्यपश्यतः} %3-64-13

\twolineshloka
{मन्ये दीर्घा भविष्यन्ति रात्रयो मम जाग्रतः}
{मन्दाकिनीं जनस्थानमिमं प्रस्रवणं गिरिम्} %3-64-14

\twolineshloka
{सर्वाण्यनुचरिष्यामि यदि सीता हि लभ्यते}
{एते महामृगा वीर मामीक्षन्ते पुनः पुनः} %3-64-15

\twolineshloka
{वक्तुकामा इह हि मे इङ्गितान्युपलक्षये}
{तांस्तु दृष्ट्वा नरव्याघ्रो राघवः प्रत्युवाच ह} %3-64-16

\twolineshloka
{क्व सीतेति निरीक्षन् वै बाष्पसंरुद्धया गिरा}
{एवमुक्ता नरेन्द्रेण ते मृगाः सहसोत्थिताः} %3-64-17

\twolineshloka
{दक्षिणाभिमुखाः सर्वे दर्शयन्तो नभःस्थलम्}
{मैथिली ह्रियमाणा सा दिशं यामभ्यपद्यत} %3-64-18

\twolineshloka
{तेन मार्गेण गच्छन्तो निरीक्षन्ते नराधिपम्}
{येन मार्गं च भूमिं च निरीक्षन्ते स्म ते मृगाः} %3-64-19

\twolineshloka
{पुनर्नदन्तो गच्छन्ति लक्ष्मणेनोपलक्षिताः}
{तेषां वचनसर्वस्वं लक्षयामास चेङ्गितम्} %3-64-20

\twolineshloka
{उवाच लक्ष्मणो धीमान् ज्येष्ठं भ्रातरमार्तवत्}
{क्व सीतेति त्वया पृष्टा यथेमे सहसोत्थिताः} %3-64-21

\twolineshloka
{दर्शयन्ति क्षितिं चैव दक्षिणां च दिशं मृगाः}
{साधु गच्छावहे देव दिशमेतां च नैर्ऋतीम्} %3-64-22

\twolineshloka
{यदि तस्यागमः कश्चिदार्या वा साथ लक्ष्यते}
{बाढमित्येव काकुत्स्थः प्रस्थितो दक्षिणां दिशम्} %3-64-23

\twolineshloka
{लक्ष्मणानुगतः श्रीमान् वीक्षमाणो वसुंधराम्}
{एवं सम्भाषमाणौ तावन्योन्यं भ्रातरावुभौ} %3-64-24

\twolineshloka
{वसुंधरायां पतितपुष्पमार्गमपश्यताम्}
{पुष्पवृष्टिं निपतितां दृष्ट्वा रामो महीतले} %3-64-25

\twolineshloka
{उवाच लक्ष्मणं वीरो दुःखितो दुःखितं वचः}
{अभिजानामि पुष्पाणि तानीमानीह लक्ष्मण} %3-64-26

\twolineshloka
{अपिनद्धानि वैदेह्या मया दत्तानि कानने}
{मन्ये सूर्यश्च वायुश्च मेदिनी च यशस्विनी} %3-64-27

\twolineshloka
{अभिरक्षन्ति पुष्पाणि प्रकुर्वन्तो मम प्रियम्}
{एवमुक्त्वा महाबाहुर्लक्ष्मणं पुरुषर्षभम्} %3-64-28

\twolineshloka
{उवाच रामो धर्मात्मा गिरिं प्रस्रवणाकुलम्}
{कच्चित् क्षितिभृतां नाथ दृष्टा सर्वाङ्गसुन्दरी} %3-64-29

\twolineshloka
{रामा रम्ये वनोद्देशे मया विरहिता त्वया}
{क्रुद्धोऽब्रवीद् गिरिं तत्र सिंहः क्षुद्रमृगं यथा} %3-64-30

\twolineshloka
{तां हेमवर्णां हेमाङ्गीं सीतां दर्शय पर्वत}
{यावत् सानूनि सर्वाणि न ते विध्वंसयाम्यहम्} %3-64-31

\twolineshloka
{एवमुक्तस्तु रामेण पर्वतो मैथिलीं प्रति}
{दर्शयन्निव तां सीतां नादर्शयत राघवे} %3-64-32

\twolineshloka
{ततो दाशरथी राम उवाच च शिलोच्चयम्}
{मम बाणाग्निनिर्दग्धो भस्मीभूतो भविष्यसि} %3-64-33

\twolineshloka
{असेव्यः सर्वतश्चैव निस्तृणद्रुमपल्लवः}
{इमां वा सरितं चाद्य शोषयिष्यामि लक्ष्मण} %3-64-34

\twolineshloka
{यदि नाख्याति मे सीतामद्य चन्द्रनिभाननाम्}
{एवं प्ररुषितो रामो दिधक्षन्निव चक्षुषा} %3-64-35

\twolineshloka
{ददर्श भूमौ निष्क्रान्तं राक्षसस्य पदं महत्}
{त्रस्ताया रामकांक्षिण्याः प्रधावन्त्या इतस्ततः} %3-64-36

\twolineshloka
{राक्षसेनानुसृप्ताया वैदेह्याश्च पदानि तु}
{स समीक्ष्य परिक्रान्तं सीताया राक्षसस्य च} %3-64-37

\twolineshloka
{भग्नं धनुश्च तूणी च विकीर्णं बहुधा रथम्}
{सम्भ्रान्तहृदयो रामः शशंस भ्रातरं प्रियम्} %3-64-38

\twolineshloka
{पश्य लक्ष्मण वैदेह्या कीर्णाः कनकबिन्दवः}
{भूषणानां हि सौमित्रे माल्यानि विविधानि च} %3-64-39

\twolineshloka
{तप्तबिन्दुनिकाशैश्च चित्रैः क्षतजबिन्दुभिः}
{आवृतं पश्य सौमित्रे सर्वतो धरणीतलम्} %3-64-40

\twolineshloka
{मन्ये लक्ष्मण वैदेही राक्षसैः कामरूपिभिः}
{भित्त्वा भित्त्वा विभक्ता वा भक्षिता वा भविष्यति} %3-64-41

\twolineshloka
{तस्या निमित्तं सीताया द्वयोर्विवदमानयोः}
{बभूव युद्धं सौमित्रे घोरं राक्षसयोरिह} %3-64-42

\twolineshloka
{मुक्तामणिचितं चेदं रमणीयं विभूषितम्}
{धरण्यां पतितं सौम्य कस्य भग्नं महद् धनुः} %3-64-43

\twolineshloka
{राक्षसानामिदं वत्स सुराणामथवापि वा}
{तरुणादित्यसंकाशं वैदूर्यगुलिकाचितम्} %3-64-44

\twolineshloka
{विशीर्णं पतितं भूमौ कवचं कस्य काञ्चनम्}
{छत्रं शतशलाकं च दिव्यमाल्योपशोभितम्} %3-64-45

\twolineshloka
{भग्नदण्डमिदं सौम्य भूमौ कस्य निपातितम्}
{काञ्चनोरश्छदाश्चेमे पिशाचवदनाः खराः} %3-64-46

\twolineshloka
{भीमरूपा महाकायाः कस्य वा निहता रणे}
{दीप्तपावकसंकाशो द्युतिमान् समरध्वजः} %3-64-47

\twolineshloka
{अपविद्धश्च भग्नश्च कस्य साङ्ग्रामिको रथः}
{रथाक्षमात्रा विशिखास्तपनीयविभूषणाः} %3-64-48

\twolineshloka
{कस्येमे निहता बाणाः प्रकीर्णा घोरदर्शनाः}
{शरावरौ शरैः पूर्णौ विध्वस्तौ पश्य लक्ष्मण} %3-64-49

\twolineshloka
{प्रतोदाभीषुहस्तोऽयं कस्य वा सारथिर्हतः}
{पदवी पुरुषस्यैषा व्यक्तं कस्यापि रक्षसः} %3-64-50

\twolineshloka
{वैरं शतगुणं पश्य मम तैर्जीवितान्तकम्}
{सुघोरहृदयैः सौम्य राक्षसैः कामरूपिभिः} %3-64-51

\twolineshloka
{हृता मृता वा वैदेही भक्षिता वा तपस्विनी}
{न धर्मस्त्रायते सीतां ह्रियमाणां महावने} %3-64-52

\twolineshloka
{भक्षितायां हि वैदेह्यां हृतायामपि लक्ष्मण}
{के हि लोके प्रियं कर्तुं शक्ताः सौम्य ममेश्वराः} %3-64-53

\twolineshloka
{कर्तारमपि लोकानां शूरं करुणवेदिनम्}
{अज्ञानादवमन्येरन् सर्वभूतानि लक्ष्मण} %3-64-54

\twolineshloka
{मृदुं लोकहिते युक्तं दान्तं करुणवेदिनम्}
{निर्वीर्य इति मन्यन्ते नूनं मां त्रिदशेश्वराः} %3-64-55

\twolineshloka
{मां प्राप्य हि गुणो दोषः संवृत्तः पश्य लक्ष्मण}
{अद्यैव सर्वभूतानां रक्षसामभवाय च} %3-64-56

\twolineshloka
{संहृत्यैव शशिज्योत्स्नां महान् सूर्य इवोदितः}
{संहृत्यैव गुणान् सर्वान् मम तेजः प्रकाशते} %3-64-57

\twolineshloka
{नैव यक्षा न गन्धर्वा न पिशाचा न राक्षसाः}
{किंनरा वा मनुष्या वा सुखं प्राप्स्यन्ति लक्ष्मण} %3-64-58

\twolineshloka
{ममास्त्रबाणसम्पूर्णमाकाशं पश्य लक्ष्मण}
{असम्पातं करिष्यामि ह्यद्य त्रैलोक्यचारिणाम्} %3-64-59

\twolineshloka
{संनिरुद्धग्रहगणमावारितनिशाकरम्}
{विप्रणष्टानलमरुद्भास्करद्युतिसंवृतम्} %3-64-60

\twolineshloka
{विनिर्मथितशैलाग्रं शुष्यमाणजलाशयम्}
{ध्वस्तद्रुमलतागुल्मं विप्रणाशितसागरम्} %3-64-61

\twolineshloka
{त्रैलोक्यं तु करिष्यामि संयुक्तं कालकर्मणा}
{न ते कुशलिनीं सीतां प्रदास्यन्ति ममेश्वराः} %3-64-62

\twolineshloka
{अस्मिन् मुहूर्ते सौमित्रे मम द्रक्ष्यन्ति विक्रमम्}
{नाकाशमुत्पतिष्यन्ति सर्वभूतानि लक्ष्मण} %3-64-63

\twolineshloka
{मम चापगुणोन्मुक्तैर्बाणजालैर्निरन्तरम्}
{मर्दितं मम नाराचैर्ध्वस्तभ्रान्तमृगद्विजम्} %3-64-64

\twolineshloka
{समाकुलममर्यादं जगत् पश्याद्य लक्ष्मण}
{आकर्णपूर्णैरिषुभिर्जीवलोकदुरावरैः} %3-64-65

\twolineshloka
{करिष्ये मैथिलीहेतोरपिशाचमराक्षसम्}
{मम रोषप्रयुक्तानां विशिखानां बलं सुराः} %3-64-66

\twolineshloka
{द्रक्ष्यन्त्यद्य विमुक्तानाममर्षाद् दूरगामिनाम्}
{नैव देवा न दैतेया न पिशाचा न राक्षसाः} %3-64-67

\twolineshloka
{भविष्यन्ति मम क्रोधात् त्रैलोक्ये विप्रणाशिते}
{देवदानवयक्षाणां लोका ये रक्षसामपि} %3-64-68

\twolineshloka
{बहुधा निपतिष्यन्ति बाणौघैः शकलीकृताः}
{निर्मर्यादानिमाँल्लोकान् करिष्याम्यद्य सायकैः} %3-64-69

\twolineshloka
{हृतां मृतां वा सौमित्रे न दास्यन्ति ममेश्वराः}
{तथारूपां हि वैदेहीं न दास्यन्ति यदि प्रियाम्} %3-64-70

\twolineshloka
{नाशयामि जगत् सर्वं त्रैलोक्यं सचराचरम्}
{यावद् दर्शनमस्या वै तापयामि च सायकैः} %3-64-71

\twolineshloka
{इत्युक्त्वा क्रोधताम्राक्षः स्फुरमाणोष्ठसम्पुटः}
{वल्कलाजिनमाबद्ध्य जटाभारमबन्धयत्} %3-64-72

\twolineshloka
{तस्य क्रुद्धस्य रामस्य तथाभूतस्य धीमतः}
{त्रिपुरं जघ्नुषः पूर्वं रुद्रस्येव बभौ तनुः} %3-64-73

\twolineshloka
{लक्ष्मणादथ चादाय रामो निष्पीड्य कार्मुकम्}
{शरमादाय संदीप्तं घोरमाशीविषोपमम्} %3-64-74

\twolineshloka
{संदधे धनुषि श्रीमान् रामः परपुरञ्जयः}
{युगान्ताग्निरिव क्रुद्ध इदं वचनमब्रवीत्} %3-64-75

\threelineshloka
{यथा जरा यथा मृत्युर्यथा कालो यथा विधिः}
{नित्यं न प्रतिहन्यन्ते सर्वभूतेषु लक्ष्मण}
{तथाहं क्रोधसंयुक्तो न निवार्योऽस्म्यसंशयम्} %3-64-76

\twolineshloka
{पुरेव मे चारुदतीमनिन्दितां दिशन्ति सीतां यदि नाद्य मैथिलीम्}
{सदेवगन्धर्वमनुष्यपन्नगं जगत् सशैलं परिवर्तयाम्यहम्} %3-64-77


॥इत्यार्षे श्रीमद्रामायणे वाल्मीकीये आदिकाव्ये अरण्यकाण्डे रामक्रोधः नाम चतुःषष्ठितमः सर्गः ॥३-६४॥
