\sect{अष्टादशः सर्गः — शूर्पनखाविरूपणम्}

\twolineshloka
{तां तु शूर्पणखां रामः कामपाशावपाशिताम्}
{स्वेच्छया श्लक्ष्णया वाचा स्मितपूर्वमथाब्रवीत्} %3-18-1

\twolineshloka
{कृतदारोऽस्मि भवति भार्येयं दयिता मम}
{त्वद्विधानां तु नारीणां सुदुःखा ससपत्नता} %3-18-2

\twolineshloka
{अनुजस्त्वेष मे भ्राता शीलवान् प्रियदर्शनः}
{श्रीमानकृतदारश्च लक्ष्मणो नाम वीर्यवान्} %3-18-3

\twolineshloka
{अपूर्वी भार्यया चार्थी तरुणः प्रियदर्शनः}
{अनुरूपश्च ते भर्ता रूपस्यास्य भविष्यति} %3-18-4

\twolineshloka
{एनं भज विशालाक्षि भर्तारं भ्रातरं मम}
{असपत्ना वरारोहे मेरुमर्कप्रभा यथा} %3-18-5

\twolineshloka
{इति रामेण सा प्रोक्ता राक्षसी काममोहिता}
{विसृज्य रामं सहसा ततो लक्ष्मणमब्रवीत्} %3-18-6

\twolineshloka
{अस्य रूपस्य ते युक्ता भार्याहं वरवर्णिनी}
{मया सह सुखं सर्वान् दण्डकान् विचरिष्यसि} %3-18-7

\twolineshloka
{एवमुक्तस्तु सौमित्री राक्षस्या वाक्यकोविदः}
{ततः शूर्पनखीं स्मित्वा लक्ष्मणो युक्तमब्रवीत्} %3-18-8

\twolineshloka
{कथं दासस्य मे दासी भार्या भवितुमिच्छसि}
{सोऽहमार्येण परवान् भ्रात्रा कमलवर्णिनि} %3-18-9

\twolineshloka
{समृद्धार्थस्य सिद्धार्था मुदितामलवर्णिनी}
{आर्यस्य त्वं विशालाक्षि भार्या भव यवीयसी} %3-18-10

\twolineshloka
{एतां विरूपामसतीं करालां निर्णतोदरीम्}
{भार्यां वृद्धां परित्यज्य त्वामेवैष भजिष्यति} %3-18-11

\twolineshloka
{को हि रूपमिदं श्रेष्ठं संत्यज्य वरवर्णिनि}
{मानुषीषु वरारोहे कुर्याद् भावं विचक्षणः} %3-18-12

\twolineshloka
{इति सा लक्ष्मणेनोक्ता कराला निर्णतोदरी}
{मन्यते तद्वचः सत्यं परिहासाविचक्षणा} %3-18-13

\twolineshloka
{सा रामं पर्णशालायामुपविष्टं परंतपम्}
{सीतया सह दुर्धर्षमब्रवीत् काममोहिता} %3-18-14

\twolineshloka
{इमां विरूपामसतीं करालां निर्णतोदरीम्}
{वृद्धां भार्यामवष्टभ्य न मां त्वं बहु मन्यसे} %3-18-15

\twolineshloka
{अद्येमां भक्षयिष्यामि पश्यतस्तव मानुषीम्}
{त्वया सह चरिष्यामि निःसपत्ना यथासुखम्} %3-18-16

\twolineshloka
{इत्युक्त्वा मृगशावाक्षीमलातसदृशेक्षणा}
{अभ्यगच्छत् सुसंक्रुद्धा महोल्का रोहिणीमिव} %3-18-17

\twolineshloka
{तां मृत्युपाशप्रतिमामापतन्तीं महाबलः}
{विगृह्य रामः कुपितस्ततो लक्ष्मणमब्रवीत्} %3-18-18

\twolineshloka
{क्रूरैरनार्यैः सौमित्रे परिहासः कथंचन}
{न कार्यः पश्य वैदेहीं कथंचित् सौम्य जीवतीम्} %3-18-19

\twolineshloka
{इमां विरूपामसतीमतिमत्तां महोदरीम्}
{राक्षसीं पुरुषव्याघ्र विरूपयितुमर्हसि} %3-18-20

\twolineshloka
{इत्युक्तो लक्ष्मणस्तस्याः क्रुद्धो रामस्य पश्यतः}
{उद्धृत्य खड्गं चिच्छेद कर्णनासे महाबलः} %3-18-21

\twolineshloka
{निकृत्तकर्णनासा तु विस्वरं सा विनद्य च}
{यथागतं प्रदुद्राव घोरा शूर्पणखा वनम्} %3-18-22

\twolineshloka
{सा विरूपा महाघोरा राक्षसी शोणितोक्षिता}
{ननाद विविधान् नादान् यथा प्रावृषि तोयदः} %3-18-23

\twolineshloka
{सा विक्षरन्ती रुधिरं बहुधा घोरदर्शना}
{प्रगृह्य बाहू गर्जन्ती प्रविवेश महावनम्} %3-18-24

\twolineshloka
{ततस्तु सा राक्षससङ्घसंवृतं खरं जनस्थानगतं विरूपिता}
{उपेत्य तं भ्रातरमुग्रतेजसं पपात भूमौ गगनाद् यथाशनिः} %3-18-25

\twolineshloka
{ततः सभार्यं भयमोहमूर्च्छिता सलक्ष्मणं राघवमागतं वनम्}
{विरूपणं चात्मनि शोणितोक्षिता शशंस सर्वं भगिनी खरस्य सा} %3-18-26


॥इत्यार्षे श्रीमद्रामायणे वाल्मीकीये आदिकाव्ये अरण्यकाण्डे शूर्पनखाविरूपणम् नाम अष्टादशः सर्गः ॥३-१८॥
