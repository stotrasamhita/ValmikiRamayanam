\sect{द्वाविंशः सर्गः — खरसन्नाहः}

\twolineshloka
{एवमाधर्षितः शूरः शूर्पणख्या खरस्ततः}
{उवाच रक्षसां मध्ये खरः खरतरं वचः} %3-22-1

\twolineshloka
{तवापमानप्रभवः क्रोधोऽयमतुलो मम}
{न शक्यते धारयितुं लवणाम्भ इवोल्बणम्} %3-22-2

\twolineshloka
{न रामं गणये वीर्यान्मानुषं क्षीणजीवितम्}
{आत्मदुश्चरितैः प्राणान् हतो योऽद्य विमोक्ष्यते} %3-22-3

\twolineshloka
{बाष्पः संधार्यतामेष सम्भ्रमश्च विमुच्यताम्}
{अहं रामं सह भ्रात्रा नयामि यमसादनम्} %3-22-4

\twolineshloka
{परश्वधहतस्याद्य मन्दप्राणस्य भूतले}
{रामस्य रुधिरं रक्तमुष्णं पास्यसि राक्षसि} %3-22-5

\twolineshloka
{सम्प्रहृष्टा वचः श्रुत्वा खरस्य वदनाच्च्युतम्}
{प्रशशंस पुनर्मौर्ख्याद् भ्रातरं रक्षसां वरम्} %3-22-6

\twolineshloka
{तया परुषितः पूर्वं पुनरेव प्रशंसितः}
{अब्रवीद् दूषणं नाम खरः सेनापतिं तदा} %3-22-7

\twolineshloka
{चतुर्दश सहस्राणि मम चित्तानुवर्तिनाम्}
{रक्षसां भीमवेगानां समरेष्वनिवर्तिनाम्} %3-22-8

\twolineshloka
{नीलजीमूतवर्णानां लोकहिंसाविहारिणाम्}
{सर्वोद्योगमुदीर्णानां रक्षसां सौम्य कारय} %3-22-9

\twolineshloka
{उपस्थापय मे क्षिप्रं रथं सौम्य धनूंषि च}
{शरांश्च चित्रान् खड्गांश्च शक्तीश्च विविधाः शिताः} %3-22-10

\twolineshloka
{अग्रे निर्यातुमिच्छामि पौलस्त्यानां महात्मनाम्}
{वधार्थं दुर्विनीतस्य रामस्य रणकोविद} %3-22-11

\twolineshloka
{इति तस्य ब्रुवाणस्य सूर्यवर्णं महारथम्}
{सदश्वैः शबलैर्युक्तमाचचक्षेऽथ दूषणः} %3-22-12

\twolineshloka
{तं मेरुशिखराकारं तप्तकाञ्चनभूषणम्}
{हेमचक्रमसम्बाधं वैदूर्यमयकूबरम्} %3-22-13

\twolineshloka
{मत्स्यैः पुष्पैर्द्रुमैः शैलैश्चन्द्रसूर्यैश्च काञ्चनैः}
{माङ्गल्यैः पक्षिसङ्घैश्च ताराभिश्च समावृतम्} %3-22-14

\twolineshloka
{ध्वजनिस्त्रिंशसम्पन्नं किंकिणीवरभूषितम्}
{सदश्वयुक्तं सोऽमर्षादारुरोह खरस्तदा} %3-22-15

\twolineshloka
{खरस्तु तन्महत्सैन्यं रथचर्मायुधध्वजम्}
{निर्यातेत्यब्रवीत् प्रेक्ष्य दूषणः सर्वराक्षसान्} %3-22-16

\twolineshloka
{ततस्तद् राक्षसं सैन्यं घोरचर्मायुधध्वजम्}
{निर्जगाम जनस्थानान्महानादं महाजवम्} %3-22-17

\twolineshloka
{मुद्गरैः पट्टिशैः शूलैः सुतीक्ष्णैश्च परश्वधैः}
{खड्गैश्चक्रैश्च हस्तस्थैर्भ्राजमानैः सतोमरैः} %3-22-18

\twolineshloka
{शक्तिभिः परिघैर्घोरैरतिमात्रैश्च कार्मुकैः}
{गदासिमुसलैर्वज्रैर्गृहीतैर्भीमदर्शनैः} %3-22-19

\twolineshloka
{राक्षसानां सुघोराणां सहस्राणि चतुर्दश}
{निर्यातानि जनस्थानात् खरचित्तानुवर्तिनाम्} %3-22-20

\twolineshloka
{तांस्तु निर्धावतो दृष्ट्वा राक्षसान् भीमदर्शनान्}
{खरस्याथ रथः किंचिज्जगाम तदनन्तरम्} %3-22-21

\twolineshloka
{ततस्ताञ्छबलानश्वांस्तप्तकाञ्चनभूषितान्}
{खरस्य मतमाज्ञाय सारथिः पर्यचोदयत्} %3-22-22

\twolineshloka
{संचोदितो रथः शीघ्रं खरस्य रिपुघातिनः}
{शब्देनापूरयामास दिशः सप्रदिशस्तथा} %3-22-23

\twolineshloka
{प्रवृद्धमन्युस्तु खरः खरस्वरो रिपोर्वधार्थं त्वरितो यथान्तकः}
{अचूचुदत् सारथिमुन्नदन् पुनर्महाबलो मेघ इवाश्मवर्षवान्} %3-22-24


॥इत्यार्षे श्रीमद्रामायणे वाल्मीकीये आदिकाव्ये अरण्यकाण्डे खरसन्नाहः नाम द्वाविंशः सर्गः ॥३-२२॥
