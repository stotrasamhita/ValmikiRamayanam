\sect{सप्तदशः सर्गः — शूर्पनखाभावाविष्करणम्}

\twolineshloka
{कृताभिषेको रामस्तु सीता सौमित्रिरेव च}
{तस्माद् गोदावरीतीरात् ततो जग्मुः स्वमाश्रमम्} %3-17-1

\twolineshloka
{आश्रमं तमुपागम्य राघवः सहलक्ष्मणः}
{कृत्वा पौर्वाह्णिकं कर्म पर्णशालामुपागमत्} %3-17-2

\twolineshloka
{उवास सुखितस्तत्र पूज्यमानो महर्षिभिः}
{स रामः पर्णशालायामासीनः सह सीतया} %3-17-3

\twolineshloka
{विरराज महाबाहुश्चित्रया चन्द्रमा इव}
{लक्ष्मणेन सह भ्रात्रा चकार विविधाः कथाः} %3-17-4

\twolineshloka
{तदासीनस्य रामस्य कथासंसक्तचेतसः}
{तं देशं राक्षसी काचिदाजगाम यदृच्छया} %3-17-5

\twolineshloka
{सा तु शूर्पणखा नाम दशग्रीवस्य रक्षसः}
{भगिनी राममासाद्य ददर्श त्रिदशोपमम्} %3-17-6

\twolineshloka
{दीप्तास्यं च महाबाहुं पद्मपत्रायतेक्षणम्}
{गजविक्रान्तगमनं जटामण्डलधारिणम्} %3-17-7

\twolineshloka
{सुकुमारं महासत्त्वं पार्थिवव्यञ्जनान्वितम्}
{राममिन्दीवरश्यामं कंदर्पसदृशप्रभम्} %3-17-8

\twolineshloka
{बभूवेन्द्रोपमं दृष्ट्वा राक्षसी काममोहिता}
{सुमुखं दुर्मुखी रामं वृत्तमध्यं महोदरी} %3-17-9

\twolineshloka
{विशालाक्षं विरूपाक्षी सुकेशं ताम्रमूर्धजा}
{प्रियरूपं विरूपा सा सुस्वरं भैरवस्वना} %3-17-10

\twolineshloka
{तरुणं दारुणा वृद्धा दक्षिणं वामभाषिणी}
{न्यायवृत्तं सुदुर्वृत्ता प्रियमप्रियदर्शना} %3-17-11

\twolineshloka
{शरीरजसमाविष्टा राक्षसी राममब्रवीत्}
{जटी तापसवेषेण सभार्यः शरचापधृक्} %3-17-12

\twolineshloka
{आगतस्त्वमिमं देशं कथं राक्षससेवितम्}
{किमागमनकृत्यं ते तत्त्वमाख्यातुमर्हसि} %3-17-13

\twolineshloka
{एवमुक्तस्तु राक्षस्या शूर्पणख्या परंतपः}
{ऋजुबुद्धितया सर्वमाख्यातुमुपचक्रमे} %3-17-14

\twolineshloka
{आसीद् दशरथो नाम राजा त्रिदशविक्रमः}
{तस्याहमग्रजः पुत्रो रामो नाम जनैः श्रुतः} %3-17-15

\twolineshloka
{भ्रातायं लक्ष्मणो नाम यवीयान् मामनुव्रतः}
{इयं भार्या च वैदेही मम सीतेति विश्रुता} %3-17-16

\twolineshloka
{नियोगात् तु नरेन्द्रस्य पितुर्मातुश्च यन्त्रितः}
{धर्मार्थं धर्मकांक्षी च वनं वस्तुमिहागतः} %3-17-17

\twolineshloka
{त्वां तु वेदितुमिच्छामि कस्य त्वं कासि कस्य वा}
{त्वं हि तावन्मनोज्ञाङ्गी राक्षसी प्रतिभासि मे} %3-17-18

\twolineshloka
{इह वा किंनिमित्तं त्वमागता ब्रूहि तत्त्वतः}
{साब्रवीद् वचनं श्रुत्वा राक्षसी मदनार्दिता} %3-17-19

\twolineshloka
{श्रूयतां राम तत्त्वार्थं वक्ष्यामि वचनं मम}
{अहं शूर्पणखा नाम राक्षसी कामरूपिणी} %3-17-20

\twolineshloka
{अरण्यं विचरामीदमेका सर्वभयंकरा}
{रावणो नाम मे भ्राता यदि ते श्रोत्रमागतः} %3-17-21

\twolineshloka
{वीरो विश्रवसः पुत्रो यदि ते श्रोत्रमागतः}
{प्रवृद्धनिद्रश्च सदा कुम्भकर्णो महाबलः} %3-17-22

\twolineshloka
{विभीषणस्तु धर्मात्मा न तु राक्षसचेष्टितः}
{प्रख्यातवीर्यौ च रणे भ्रातरौ खरदूषणौ} %3-17-23

\twolineshloka
{तानहं समतिक्रान्तां राम त्वापूर्वदर्शनात्}
{समुपेतास्मि भावेन भर्तारं पुरुषोत्तमम्} %3-17-24

\twolineshloka
{अहं प्रभावसम्पन्ना स्वच्छन्दबलगामिनी}
{चिराय भव भर्ता मे सीतया किं करिष्यसि} %3-17-25

\twolineshloka
{विकृता च विरूपा च न सेयं सदृशी तव}
{अहमेवानुरूपा ते भार्यारूपेण पश्य माम्} %3-17-26

\twolineshloka
{इमां विरूपामसतीं करालां निर्णतोदरीम्}
{अनेन सह ते भ्रात्रा भक्षयिष्यामि मानुषीम्} %3-17-27

\twolineshloka
{ततः पर्वतशृङ्गाणि वनानि विविधानि च}
{पश्यन् सह मया कामी दण्डकान् विचरिष्यसि} %3-17-28

\twolineshloka
{इत्येवमुक्तः काकुत्स्थः प्रहस्य मदिरेक्षणाम्}
{इदं वचनमारेभे वक्तुं वाक्यविशारदः} %3-17-29


॥इत्यार्षे श्रीमद्रामायणे वाल्मीकीये आदिकाव्ये अरण्यकाण्डे शूर्पनखाभावाविष्करणम् नाम सप्तदशः सर्गः ॥३-१७॥
