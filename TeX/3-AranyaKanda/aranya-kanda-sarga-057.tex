\sect{सप्तपञ्चाशः सर्गः — रामप्रत्यागमनम्}

\twolineshloka
{राक्षसं मृगरूपेण चरन्तं कामरूपिणम्}
{निहत्य रामो मारीचं तूर्णं पथि न्यवर्तत} %3-57-1

\twolineshloka
{तस्य सन्त्वरमाणस्य द्रष्टुकामस्य मैथिलीम्}
{क्रूरस्वनोऽथ गोमायुर्विननादास्य पृष्ठतः} %3-57-2

\twolineshloka
{स तस्य स्वरमाज्ञाय दारुणं रोमहर्षणम्}
{चिन्तयामास गोमायोः स्वरेण परिशङ्कितः} %3-57-3

\twolineshloka
{अशुभं बत मन्येऽहं गोमायुर्वाश्यते यथा}
{स्वस्ति स्यादपि वैदेह्या राक्षसैर्भक्षणं विना} %3-57-4

\twolineshloka
{मारीचेन तु विज्ञाय स्वरमालक्ष्य मामकम्}
{विक्रुष्टं मृगरूपेण लक्ष्मणः शृणुयाद् यदि} %3-57-5

\twolineshloka
{स सौमित्रिः स्वरं श्रुत्वा तां च हित्वाथ मैथिलीम्}
{तयैव प्रहितः क्षिप्रं मत्सकाशमिहैष्यति} %3-57-6

\twolineshloka
{राक्षसैः सहितैर्नूनं सीताया ईप्सितो वधः}
{काञ्चनश्च मृगो भूत्वा व्यपनीयाश्रमात्तु माम्} %3-57-7

\twolineshloka
{दूरं नीत्वाथ मारीचो राक्षसोऽभूच्छराहतः}
{हा लक्ष्मण हतोऽस्मीति यद्वाक्यं व्याजहार ह} %3-57-8

\twolineshloka
{अपि स्वस्ति भवेद् द्वाभ्यां रहिताभ्यां मया वने}
{जनस्थाननिमित्तं हि कृतवैरोऽस्मि राक्षसैः} %3-57-9

\twolineshloka
{निमित्तानि च घोराणि दृश्यन्तेऽद्य बहूनि च}
{इत्येवं चिन्तयन् रामः श्रुत्वा गोमायुनिःस्वनम्} %3-57-10

\twolineshloka
{निवर्तमानस्त्वरितो जगामाश्रममात्मवान्}
{आत्मनश्चापनयनं मृगरूपेण रक्षसा} %3-57-11

\twolineshloka
{आजगाम जनस्थानं राघवः परिशङ्कितः}
{तं दीनमानसं दीनमासेदुर्मृगपक्षिणः} %3-57-12

\threelineshloka
{सव्यं कृत्वा महात्मानं घोरांश्च ससृजुः स्वरान्}
{तानि दृष्ट्वा निमित्तानि महाघोराणि राघवः}
{न्यवर्तताथ त्वरितो जवेनाश्रममात्मनः} %3-57-13

\twolineshloka
{ततो लक्ष्मणमायान्तं ददर्श विगतप्रभम्}
{ततोऽविदूरे रामेण समीयाय स लक्ष्मणः} %3-57-14

\twolineshloka
{विषण्णः सन् विषण्णेन दुःखितो दुःखभागिना}
{स जगर्हेऽथ तं भ्राता दृष्ट्वा लक्ष्मणमागतम्} %3-57-15

\twolineshloka
{विहाय सीतां विजने वने राक्षससेविते}
{गृहीत्वा च करं सव्यं लक्ष्मणं रघुनन्दनः} %3-57-16

\twolineshloka
{उवाच मधुरोदर्कमिदं परुषमार्तवत्}
{अहो लक्ष्मण गर्ह्यं ते कृतं यत् त्वं विहाय ताम्} %3-57-17

\twolineshloka
{सीतामिहागतः सौम्य कच्चित् स्वस्ति भवेदिति}
{न मेऽस्ति संशयो वीर सर्वथा जनकात्मजा} %3-57-18

\twolineshloka
{विनष्टा भक्षिता वापि राक्षसैर्वनचारिभिः}
{अशुभान्येव भूयिष्ठं यथा प्रादुर्भवन्ति मे} %3-57-19

\twolineshloka
{अपि लक्ष्मण सीतायाः सामग्र्यं प्राप्नुयामहे}
{जीवन्त्याः पुरुषव्याघ्र सुताया जनकस्य वै} %3-57-20

\threelineshloka
{यथा वै मृगसङ्घाश्च गोमायुश्चैव भैरवम्}
{वाश्यन्ते शकुनाश्चापि प्रदीप्तामभितो दिशम्}
{अपि स्वस्ति भवेत् तस्या राजपुत्र्या महाबल} %3-57-21

\twolineshloka
{इदं हि रक्षो मृगसन्निकाशं प्रलोभ्य मां दूरमनुप्रयातम्}
{हतं कथञ्चिन्महता श्रमेण स राक्षसोऽभून्म्रियमाण एव} %3-57-22

\twolineshloka
{मनश्च मे दीनमिहाप्रहृष्टं चक्षुश्च सव्यं कुरुते विकारम्}
{असंशयं लक्ष्मण नास्ति सीता हृता मृता वा पथि वर्तते वा} %3-57-23


॥इत्यार्षे श्रीमद्रामायणे वाल्मीकीये आदिकाव्ये अरण्यकाण्डे रामप्रत्यागमनम् नाम सप्तपञ्चाशः सर्गः ॥३-५७॥
