\sect{विंशः सर्गः — चतुर्दशरक्षोवधः}

\twolineshloka
{ततः शूर्पणखा घोरा राघवाश्रममागता}
{राक्षसानाचचक्षे तौ भ्रातरौ सह सीतया} %3-20-1

\twolineshloka
{ते रामं पर्णशालायामुपविष्टं महाबलम्}
{ददृशुः सीतया सार्धं लक्ष्मणेनापि सेवितम्} %3-20-2

\twolineshloka
{तां दृष्ट्वा राघवः श्रीमानागतांस्तांश्च राक्षसान्}
{अब्रवीद् भ्रातरं रामो लक्ष्मणं दीप्ततेजसम्} %3-20-3

\twolineshloka
{मुहूर्तं भव सौमित्रे सीतायाः प्रत्यनन्तरः}
{इमानस्या वधिष्यामि पदवीमागतानिह} %3-20-4

\twolineshloka
{वाक्यमेतत् ततः श्रुत्वा रामस्य विदितात्मनः}
{तथेति लक्ष्मणो वाक्यं राघवस्य प्रपूजयन्} %3-20-5

\twolineshloka
{राघवोऽपि महच्चापं चामीकरविभूषितम्}
{चकार सज्यं धर्मात्मा तानि रक्षांसि चाब्रवीत्} %3-20-6

\twolineshloka
{पुत्रौ दशरथस्यावां भ्रातरौ रामलक्ष्मणौ}
{प्रविष्टौ सीतया सार्धं दुश्चरं दण्डकावनम्} %3-20-7

\twolineshloka
{फलमूलाशनौ दान्तौ तापसौ ब्रह्मचारिणौ}
{वसन्तौ दण्डकारण्ये किमर्थमुपहिंसथ} %3-20-8

\twolineshloka
{युष्मान् पापात्मकान् हन्तुं विप्रकारान् महाहवे}
{ऋषीणां तु नियोगेन सम्प्राप्तः सशरासनः} %3-20-9

\twolineshloka
{तिष्ठतैवात्र सन्तुष्टा नोपवर्तितुमर्हथ}
{यदि प्राणैरिहार्थो वो निवर्तध्वं निशाचराः} %3-20-10

\twolineshloka
{तस्य तद् वचनं श्रुत्वा राक्षसास्ते चतुर्दश}
{ऊचुर्वाचं सुसङ्क्रुद्धा ब्रह्मघ्नाः शूलपाणयः} %3-20-11

\twolineshloka
{संरक्तनयना घोरा रामं संरक्तलोचनम्}
{परुषा मधुराभाषं हृष्टा दृष्टपराक्रमम्} %3-20-12

\twolineshloka
{क्रोधमुत्पाद्य नो भर्तुः खरस्य सुमहात्मनः}
{त्वमेव हास्यसे प्राणान् सद्योऽस्माभिर्हतो युधि} %3-20-13

\twolineshloka
{का हि ते शक्तिरेकस्य बहूनां रणमूर्धनि}
{अस्माकमग्रतः स्थातुं किं पुनर्योद्धुमाहवे} %3-20-14

\twolineshloka
{एभिर्बाहुप्रयुक्तैश्च परिघैः शूलपट्टिशैः}
{प्राणांस्त्यक्ष्यसि वीर्यं च धनुश्च करपीडितम्} %3-20-15

\twolineshloka
{इत्येवमुक्त्वा संरब्धा राक्षसास्ते चतुर्दश}
{उद्यतायुधनिस्त्रिंशा राममेवाभिदुद्रुवुः} %3-20-16

\twolineshloka
{चिक्षिपुस्तानि शूलानि राघवं प्रति दुर्जयम्}
{तानि शूलानि काकुत्स्थः समस्तानि चतुर्दश} %3-20-17

\twolineshloka
{तावद्भिरेव चिच्छेद शरैः काञ्चनभूषितैः}
{ततः पश्चान्महातेजा नाराचान् सूर्यसन्निभान्} %3-20-18

\twolineshloka
{जग्राह परमक्रुद्धश्चतुर्दश शिलाशितान्}
{गृहीत्वा धनुरायम्य लक्ष्यानुद्दिश्य राक्षसान्} %3-20-19

\twolineshloka
{मुमोच राघवो बाणान् वज्रानिव शतक्रतुः}
{ते भित्त्वा रक्षसां वेगाद् वक्षांसि रुधिरप्लुताः} %3-20-20

\twolineshloka
{विनिष्पेतुस्तदा भूमौ वल्मीकादिव पन्नगाः}
{तैर्भग्नहृदया भूमौ छिन्नमूला इव द्रुमाः} %3-20-21

\twolineshloka
{निपेतुः शोणितस्नाता विकृता विगतासवः}
{तान् भूमौ पतितान् दृष्ट्वा राक्षसी क्रोधमूर्छिता} %3-20-22

\twolineshloka
{उपगम्य खरं सा तु किञ्चित्संशुष्कशोणिता}
{पपात पुनरेवार्ता सनिर्यासेव वल्लरी} %3-20-23

\twolineshloka
{भ्रातुः समीपे शोकार्ता ससर्ज निनदं महत्}
{सस्वरं मुमुचे बाष्पं विवर्णवदना तदा} %3-20-24

\twolineshloka
{निपातितान् प्रेक्ष्य रणे तु राक्षसान् प्रधाविता शूर्पणखा पुनस्ततः}
{वधं च तेषां निखिलेन रक्षसां शशंस सर्वं भगिनी खरस्य सा} %3-20-25


॥इत्यार्षे श्रीमद्रामायणे वाल्मीकीये आदिकाव्ये अरण्यकाण्डे चतुर्दशरक्षोवधः नाम विंशः सर्गः ॥३-२०॥
