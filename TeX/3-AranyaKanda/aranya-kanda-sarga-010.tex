\sect{दशमः सर्गः — रक्षोवधसमर्थनम्}

\twolineshloka
{वाक्यमेतत् तु वैदेह्या व्याहृतं भर्तृभक्तया}
{श्रुत्वा धर्मे स्थितो रामः प्रत्युवाचाथ जानकीम्} %3-10-1

\twolineshloka
{हितमुक्तं त्वया देवि स्निग्धया सदृशं वचः}
{कुलं व्यपदिशन्त्या च धर्मज्ञे जनकात्मजे} %3-10-2

\twolineshloka
{किं नु वक्ष्याम्यहं देवि त्वयैवोक्तमिदं वचः}
{क्षत्रियैर्धार्यते चापो नार्तशब्दो भवेदिति} %3-10-3

\twolineshloka
{ते चार्ता दण्डकारण्ये मुनयः संशितव्रताः}
{मां सीते स्वयमागम्य शरण्यं शरणं गताः} %3-10-4

\twolineshloka
{वसन्तः कालकालेषु वने मूलफलाशनाः}
{न लभन्ते सुखं भीरु राक्षसैः क्रूरकर्मभिः} %3-10-5

\twolineshloka
{भक्ष्यन्ते राक्षसैर्भीमैर्नरमांसोपजीविभिः}
{ते भक्ष्यमाणा मुनयो दण्डकारण्यवासिनः} %3-10-6

\twolineshloka
{अस्मानभ्यवपद्येति मामूचुर्द्विजसत्तमाः}
{मया तु वचनं श्रुत्वा तेषामेवं मुखाच्च्युतम्} %3-10-7

\twolineshloka
{कृत्वा वचनशुश्रूषां वाक्यमेतदुदाहृतम्}
{प्रसीदन्तु भवन्तो मे ह्रीरेषा तु ममातुला} %3-10-8

\twolineshloka
{यदीदृशैरहं विप्रैरुपस्थेयैरुपस्थितः}
{किं करोमीति च मया व्याहृतं द्विजसन्निधौ} %3-10-9

\twolineshloka
{सर्वैरेव समागम्य वागियं समुदाहृता}
{राक्षसैर्दण्डकारण्ये बहुभिः कामरूपिभिः} %3-10-10

\twolineshloka
{अर्दिताः स्म भृशं राम भवान् नस्तत्र रक्षतु}
{होमकाले तु सम्प्राप्ते पर्वकालेषु चानघ} %3-10-11

\twolineshloka
{धर्षयन्ति सुदुर्धर्षा राक्षसाः पिशिताशनाः}
{राक्षसैर्धर्षितानां च तापसानां तपस्विनाम्} %3-10-12

\twolineshloka
{गतिं मृगयमाणानां भवान् नः परमा गतिः}
{कामं तपःप्रभावेण शक्ता हन्तुं निशाचरान्} %3-10-13

\twolineshloka
{चिरार्जितं न चेच्छामस्तपः खण्डयितुं वयम्}
{बहुविघ्नं तपो नित्यं दुश्चरं चैव राघव} %3-10-14

\twolineshloka
{तेन शापं न मुञ्चामो भक्ष्यमाणाश्च राक्षसैः}
{तदर्द्यमानान् रक्षोभिर्दण्डकारण्यवासिभिः} %3-10-15

\twolineshloka
{रक्ष नस्त्वं सह भ्रात्रा त्वन्नाथा हि वयं वने}
{मया चैतद्वचः श्रुत्वा कात्स्न्र्येन परिपालनम्} %3-10-16

\twolineshloka
{ऋषीणां दण्डकारण्ये संश्रुतं जनकात्मजे}
{संश्रुत्य च न शक्ष्यामि जीवमानः प्रतिश्रवम्} %3-10-17

\twolineshloka
{मुनीनामन्यथा कर्तुं सत्यमिष्टं हि मे सदा}
{अप्यहं जीवितं जह्यां त्वां वा सीते सलक्ष्मणाम्} %3-10-18

\twolineshloka
{न तु प्रतिज्ञा संश्रुत्य ब्राह्मणेभ्यो विशेषतः}
{तदवश्यं मया कार्यमृषीणां परिपालनम्} %3-10-19

\twolineshloka
{अनुक्तेनापि वैदेहि प्रतिज्ञाय कथं पुनः}
{मम स्नेहाच्च सौहार्दादिदमुक्तं त्वया वचः} %3-10-20

\threelineshloka
{परितुष्टोऽस्म्यहं सीते न ह्यनिष्टोऽनुशास्यते}
{सदृशं चानुरूपं च कुलस्य तव शोभने}
{सधर्मचारिणी मे त्वं प्राणेभ्योऽपि गरीयसी} %3-10-21

\twolineshloka
{इत्येवमुक्त्वा वचनं महात्मा सीतां प्रियां मैथिलराजपुत्रीम्}
{रामो धनुष्मान् सह लक्ष्मणेन जगाम रम्याणि तपोवनानि} %3-10-22


॥इत्यार्षे श्रीमद्रामायणे वाल्मीकीये आदिकाव्ये अरण्यकाण्डे रक्षोवधसमर्थनम् नाम दशमः सर्गः ॥३-१०॥
