\sect{चतुःपञ्चाशः सर्गः — लङ्काप्रापणम्}

\twolineshloka
{ह्रियमाणा तु वैदेही कञ्चिन्नाथमपश्यती}
{ददर्श गिरिशृङ्गस्थान् पञ्च वानरपुङ्गवान्} %3-54-1

\twolineshloka
{तेषां मध्ये विशालाक्षी कौशेयं कनकप्रभम्}
{उत्तरीयं वरारोहा शुभान्याभरणानि च} %3-54-2

\twolineshloka
{मुमोच यदि रामाय शंसेयुरिति भामिनी}
{वस्त्रमुत्सृज्य तन्मध्ये निक्षिप्तं सहभूषणम्} %3-54-3

\twolineshloka
{सम्भ्रमात् तु दशग्रीवस्तत्कर्म च न बुद्धवान्}
{पिङ्गाक्षास्तां विशालाक्षीं नेत्रैरनिमिषैरिव} %3-54-4

\twolineshloka
{विक्रोशन्तीं तदा सीतां ददृशुर्वानरोत्तमाः}
{स च पम्पामतिक्रम्य लङ्कामभिमुखः पुरीम्} %3-54-5

\twolineshloka
{जगाम मैथिलीं गृह्य रुदतीं राक्षसेश्वरः}
{तां जहार सुसंहृष्टो रावणो मृत्युमात्मनः} %3-54-6

\twolineshloka
{उत्सङ्गेनैव भुजगीं तीक्ष्णदंष्ट्रां महाविषाम्}
{वनानि सरितः शैलान् सरांसि च विहायसा} %3-54-7

\twolineshloka
{स क्षिप्रं समतीयाय शरश्चापादिव च्युतः}
{तिमिनक्रनिकेतं तु वरुणालयमक्षयम्} %3-54-8

\twolineshloka
{सरितां शरणं गत्वा समतीयाय सागरम्}
{सम्भ्रमात् परिवृत्तोर्मी रुद्धमीनमहोरगः} %3-54-9

\twolineshloka
{वैदेह्यां ह्रियमाणायां बभूव वरुणालयः}
{अन्तरिक्षगता वाचः ससृजुश्चारणास्तदा} %3-54-10

\twolineshloka
{एतदन्तो दशग्रीव इति सिद्धास्तथाब्रुवन्}
{स तु सीतां विचेष्टन्तीमङ्केनादाय रावणः} %3-54-11

\twolineshloka
{प्रविवेश पुरीं लङ्कां रूपिणीं मृत्युमात्मनः}
{सोऽभिगम्य पुरीं लङ्कां सुविभक्तमहापथाम्} %3-54-12

\twolineshloka
{संरूढकक्ष्यां बहुलां स्वमन्तःपुरमाविशत्}
{तत्र तामसितापाङ्गीं शोकमोहसमन्विताम्} %3-54-13

\twolineshloka
{निदधे रावणः सीतां मयो मायामिवासुरीम्}
{अब्रवीच्च दशग्रीवः पिशाचीर्घोरदर्शनाः} %3-54-14

\twolineshloka
{यथा नैनां पुमान् स्त्री वा सीतां पश्यत्यसम्मतः}
{मुक्तामणिसुवर्णानि वस्त्राण्याभरणानि च} %3-54-15

\twolineshloka
{यद् यदिच्छेत् तदैवास्या देयं मच्छन्दतो यथा}
{या च वक्ष्यति वैदेहीं वचनं किञ्चिदप्रियम्} %3-54-16

\twolineshloka
{अज्ञानाद् यदि वा ज्ञानान्न तस्या जीवितं प्रियम्}
{तथोक्त्वा राक्षसीस्तास्तु राक्षसेन्द्रः प्रतापवान्} %3-54-17

\twolineshloka
{निष्क्रम्यान्तःपुरात् तस्मात् किं कृत्यमिति चिन्तयन्}
{ददर्शाष्टौ महावीर्यान् राक्षसान् पिशिताशनान्} %3-54-18

\twolineshloka
{स तान् दृष्ट्वा महावीर्यो वरदानेन मोहितः}
{उवाच तानिदं वाक्यं प्रशस्य बलवीर्यतः} %3-54-19

\twolineshloka
{नानाप्रहरणाः क्षिप्रमितो गच्छत सत्वराः}
{जनस्थानं हतस्थानं भूतपूर्वं खरालयम्} %3-54-20

\twolineshloka
{तत्रास्यतां जनस्थाने शून्ये निहतराक्षसे}
{पौरुषं बलमाश्रित्य त्रासमुत्सृज्य दूरतः} %3-54-21

\twolineshloka
{बहुसैन्यं महावीर्यं जनस्थाने निवेशितम्}
{सदूषणखरं युद्धे निहतं रामसायकैः} %3-54-22

\twolineshloka
{ततः क्रोधो ममापूर्वो धैर्यस्योपरि वर्धते}
{वैरं च सुमहज्जातं रामं प्रति सुदारुणम्} %3-54-23

\twolineshloka
{निर्यातयितुमिच्छामि तच्च वैरं महारिपोः}
{नहि लप्स्याम्यहं निद्रामहत्वा संयुगे रिपुम्} %3-54-24

\twolineshloka
{तं त्विदानीमहं हत्वा खरदूषणघातिनम्}
{रामं शर्मोपलप्स्यामि धनं लब्ध्वेव निर्धनः} %3-54-25

\twolineshloka
{जनस्थाने वसद्भिस्तु भवद्भी राममाश्रिता}
{प्रवृत्तिरुपनेतव्या किं करोतीति तत्त्वतः} %3-54-26

\twolineshloka
{अप्रमादाच्च गन्तव्यं सर्वैरेव निशाचरैः}
{कर्तव्यश्च सदा यत्नो राघवस्य वधं प्रति} %3-54-27

\twolineshloka
{युष्माकं तु बलं ज्ञातं बहुशो रणमूर्धनि}
{अतश्चास्मिञ्जनस्थाने मया यूयं निवेशिताः} %3-54-28

\twolineshloka
{ततः प्रियं वाक्यमुपेत्य राक्षसा महार्थमष्टावभिवाद्य रावणम्}
{विहाय लङ्कां सहिताः प्रतस्थिरे यतो जनस्थानमलक्ष्यदर्शनाः} %3-54-29

\twolineshloka
{ततस्तु सीतामुपलभ्य रावणः सुसम्प्रहृष्टः परिगृह्य मैथिलीम्}
{प्रसज्य रामेण च वैरमुत्तमं बभूव मोहान्मुदितः स रावणः} %3-54-30


॥इत्यार्षे श्रीमद्रामायणे वाल्मीकीये आदिकाव्ये अरण्यकाण्डे लङ्काप्रापणम् नाम चतुःपञ्चाशः सर्गः ॥३-५४॥
