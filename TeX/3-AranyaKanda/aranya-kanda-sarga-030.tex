\sect{त्रिंशः सर्गः — खरसंहारः}

\twolineshloka
{भित्त्वा तु तां गदां बाणै राघवो धर्मवत्सलः}
{स्मयमान इदं वाक्यं संरब्धमिदमब्रवीत्} %3-30-1

\twolineshloka
{एतत् ते बलसर्वस्वं दर्शितं राक्षसाधम}
{शक्तिहीनतरो मत्तो वृथा त्वमुपगर्जसि} %3-30-2

\twolineshloka
{एषा बाणविनिर्भिन्ना गदा भूमितलं गता}
{अभिधानप्रगल्भस्य तव प्रत्ययघातिनी} %3-30-3

\twolineshloka
{यत् त्वयोक्तं विनष्टानामिदमश्रुप्रमार्जनम्}
{राक्षसानां करोमीति मिथ्या तदपि ते वचः} %3-30-4

\twolineshloka
{नीचस्य क्षुद्रशीलस्य मिथ्यावृत्तस्य रक्षसः}
{प्राणानपहरिष्यामि गरुत्मानमृतं यथा} %3-30-5

\twolineshloka
{अद्य ते भिन्नकण्ठस्य फेनबुद्बुदभूषितम्}
{विदारितस्य मद्बाणैर्मही पास्यति शोणितम्} %3-30-6

\twolineshloka
{पांसुरूषितसर्वाङ्गः स्रस्तन्यस्तभुजद्वयः}
{स्वप्स्यसे गां समाश्लिष्य दुर्लभां प्रमदामिव} %3-30-7

\twolineshloka
{प्रवृद्धनिद्रे शयिते त्वयि राक्षसपांसने}
{भविष्यन्ति शरण्यानां शरण्या दण्डका इमे} %3-30-8

\twolineshloka
{जनस्थाने हतस्थाने तव राक्षस मच्छरैः}
{निर्भया विचरिष्यन्ति सर्वतो मुनयो वने} %3-30-9

\twolineshloka
{अद्य विप्रसरिष्यन्ति राक्षस्यो हतबान्धवाः}
{बाष्पार्द्रवदना दीना भयादन्यभयावहाः} %3-30-10

\twolineshloka
{अद्य शोकरसज्ञास्ता भविष्यन्ति निरर्थिकाः}
{अनुरूपकुलाः पत्न्यो यासां त्वं पतिरीदृशः} %3-30-11

\twolineshloka
{नृशंसशील क्षुद्रात्मन् नित्यं ब्राह्मणकण्टक}
{त्वत्कृते शङ्कितैरग्नौ मुनिभिः पात्यते हविः} %3-30-12

\twolineshloka
{तमेवमभिसंरब्धं ब्रुवाणं राघवं वने}
{खरो निर्भर्त्सयामास रोषात् खरतरस्वरः} %3-30-13

\twolineshloka
{दृढं खल्ववलिप्तोऽसि भयेष्वपि च निर्भयः}
{वाच्यावाच्यं ततो हि त्वं मृत्योर्वश्यो न बुध्यसे} %3-30-14

\twolineshloka
{कालपाशपरिक्षिप्ता भवन्ति पुरुषा हि ये}
{कार्याकार्यं न जानन्ति ते निरस्तषडिन्द्रियाः} %3-30-15

\twolineshloka
{एवमुक्त्वा ततो रामं संरुध्य भृकुटिं ततः}
{स ददर्श महासालमविदूरे निशाचरः} %3-30-16

\twolineshloka
{रणे प्रहरणस्यार्थे सर्वतो ह्यवलोकयन्}
{स तमुत्पाटयामास संदष्टदशनच्छदम्} %3-30-17

\twolineshloka
{तं समुत्क्षिप्य बाहुभ्यां विनर्दित्वा महाबलः}
{राममुद्दिश्य चिक्षेप हतस्त्वमिति चाब्रवीत्} %3-30-18

\twolineshloka
{तमापतन्तं बाणौघैश्छित्त्वा रामः प्रतापवान्}
{रोषमाहारयत् तीव्रं निहन्तुं समरे खरम्} %3-30-19

\twolineshloka
{जातस्वेदस्ततो रामो रोषरक्तान्तलोचनः}
{निर्बिभेद सहस्रेण बाणानां समरे खरम्} %3-30-20

\twolineshloka
{तस्य बाणान्तराद् रक्तं बहु सुस्राव फेनिलम्}
{गिरेः प्रस्रवणस्येव धाराणां च परिस्रवः} %3-30-21

\twolineshloka
{विकलः स कृतो बाणैः खरो रामेण संयुगे}
{मत्तो रुधिरगन्धेन तमेवाभ्यद्रवद् द्रुतम्} %3-30-22

\twolineshloka
{तमापतन्तं संक्रुद्धं कृतास्त्रो रुधिराप्लुतम्}
{अपासर्पद् द्वित्रिपदं किंचित्त्वरितविक्रमः} %3-30-23

\twolineshloka
{ततः पावकसंकाशं वधाय समरे शरम्}
{खरस्य रामो जग्राह ब्रह्मदण्डमिवापरम्} %3-30-24

\twolineshloka
{स तद् दत्तं मघवता सुरराजेन धीमता}
{संदधे च स धर्मात्मा मुमोच च खरं प्रति} %3-30-25

\twolineshloka
{स विमुक्तो महाबाणो निर्घातसमनिःस्वनः}
{रामेण धनुरायम्य खरस्योरसि चापतत्} %3-30-26

\twolineshloka
{स पपात खरो भूमौ दह्यमानः शराग्निना}
{रुद्रेणेव विनिर्दग्धः श्वेतारण्ये यथान्धकः} %3-30-27

\twolineshloka
{स वृत्र इव वज्रेण फेनेन नमुचिर्यथा}
{बलो वेन्द्राशनिहतो निपपात हतः खरः} %3-30-28

\twolineshloka
{एतस्मिन्नन्तरे देवाश्चारणैः सह संगताः}
{दुन्दुभींश्चाभिनिघ्नन्तः पुष्पवर्षं समन्ततः} %3-30-29

\twolineshloka
{रामस्योपरि संहृष्टा ववर्षुर्विस्मितास्तदा}
{अर्धाधिकमुहूर्तेन रामेण निशितैः शरैः} %3-30-30

\twolineshloka
{चतुर्दश सहस्राणि रक्षसां कामरूपिणाम्}
{खरदूषणमुख्यानां निहतानि महामृधे} %3-30-31

\twolineshloka
{अहो बत महत्कर्म रामस्य विदितात्मनः}
{अहो वीर्यमहो दार्ढ्यं विष्णोरिव हि दृश्यते} %3-30-32

\twolineshloka
{इत्येवमुक्त्वा ते सर्वे ययुर्देवा यथागतम्}
{ततो राजर्षयः सर्वे संगताः परमर्षयः} %3-30-33

\twolineshloka
{सभाज्य मुदिता रामं सागस्त्या इदमब्रुवन्}
{एतदर्थं महातेजा महेन्द्रः पाकशासनः} %3-30-34

\twolineshloka
{शरभङ्गाश्रमं पुण्यमाजगाम पुरंदरः}
{आनीतस्त्वमिमं देशमुपायेन महर्षिभिः} %3-30-35

\twolineshloka
{एषां वधार्थं शत्रूणां रक्षसां पापकर्मणाम्}
{तदिदं नः कृतं कार्यं त्वया दशरथात्मज} %3-30-36

\twolineshloka
{स्वधर्मं प्रचरिष्यन्ति दण्डकेषु महर्षयः}
{एतस्मिन्नन्तरे वीरो लक्ष्मणः सह सीतया} %3-30-37

\onelineshloka
{गिरिदुर्गाद् विनिष्क्रम्य संविवेशाश्रमे सुखी ततो रामस्तु विजयी पूज्यमानो महर्षिभिः} %3-30-38

\twolineshloka
{प्रविवेशाश्रमं वीरो लक्ष्मणेनाभिपूजितः}
{तं दृष्ट्वा शत्रुहन्तारं महर्षीणां सुखावहम्} %3-30-39

\threelineshloka
{बभूव हृष्टा वैदेही भर्तारं परिषस्वजे}
{मुदा परमया युक्ता दृष्ट्वा रक्षोगणान् हतान्}
{रामं चैवाव्ययं दृष्ट्वा तुतोष जनकात्मजा} %3-30-40

\twolineshloka
{ततस्तु तं राक्षससङ्घमर्दनं सम्पूज्यमानं मुदितैर्महात्मभिः}
{पुनः परिष्वज्य मुदान्वितानना बभूव हृष्टा जनकात्मजा तदा} %3-30-41


॥इत्यार्षे श्रीमद्रामायणे वाल्मीकीये आदिकाव्ये अरण्यकाण्डे खरसंहारः नाम त्रिंशः सर्गः ॥३-३०॥
