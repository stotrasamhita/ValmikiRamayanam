\sect{षड्विंशः सर्गः — दूषणादिवधः}

\twolineshloka
{दूषणस्तु स्वकं सैन्यं हन्यमानं विलोक्य च}
{सन्दिदेश महाबाहुर्भीमवेगान् दुरासदान्} %3-26-1

\twolineshloka
{राक्षसान् पञ्चसाहस्रान् समरेष्वनिवर्तिनः}
{ते शूलैः पट्टिशैः खड्गैः शिलावर्षैर्द्रुमैरपि} %3-26-2

\twolineshloka
{शरवर्षैरविच्छिन्नं ववर्षुस्तं समन्ततः}
{तद् द्रुमाणां शिलानां च वर्षं प्राणहरं महत्} %3-26-3

\twolineshloka
{प्रतिजग्राह धर्मात्मा राघवस्तीक्ष्णसायकैः}
{प्रतिगृह्य च तद् वर्षं निमीलित इवर्षभः} %3-26-4

\twolineshloka
{रामः क्रोधं परं लेभे वधार्थं सर्वरक्षसाम्}
{ततः क्रोधसमाविष्टः प्रदीप्त इव तेजसा} %3-26-5

\twolineshloka
{शरैरभ्यकिरत् सैन्यं सर्वतः सहदूषणम्}
{ततः सेनापतिः क्रुद्धो दूषणः शत्रुदूषणः} %3-26-6

\twolineshloka
{शरैरशनिकल्पैस्तं राघवं समवारयत्}
{ततो रामः सुसङ्क्रुद्धः क्षुरेणास्य महद् धनुः} %3-26-7

\twolineshloka
{चिच्छेद समरे वीरश्चर्तुभिश्चतुरो हयान्}
{हत्वा चाश्वान् शरैस्तीक्ष्णैरर्धचन्द्रेण सारथेः} %3-26-8

\twolineshloka
{शिरो जहार तद्रक्षस्त्रिभिर्विव्याध वक्षसि}
{स च्छिन्नधन्वा विरथो हताश्वो हतसारथिः} %3-26-9

\twolineshloka
{जग्राह गिरिशृङ्गाभं परिघं रोमहर्षणम्}
{वेष्टितं काञ्चनैः पट्टैर्देवसैन्याभिमर्दनम्} %3-26-10

\twolineshloka
{आयसैः शङ्कुभिस्तीक्ष्णैः कीर्णं परवसोक्षितम्}
{वज्राशनिसमस्पर्शं परगोपुरदारणम्} %3-26-11

\twolineshloka
{तं महोरगसङ्काशं प्रगृह्य परिघं रणे}
{दूषणोऽभ्यपतद् रामं क्रूरकर्मा निशाचरः} %3-26-12

\twolineshloka
{तस्याभिपतमानस्य दूषणस्य च राघवः}
{द्वाभ्यां शराभ्यां चिच्छेद सहस्ताभरणौ भुजौ} %3-26-13

\twolineshloka
{भ्रष्टस्तस्य महाकायः पपात रणमूर्धनि}
{परिघश्छिन्नहस्तस्य शक्रध्वज इवाग्रतः} %3-26-14

\twolineshloka
{कराभ्यां च विकीर्णाभ्यां पपात भुवि दूषणः}
{विषाणाभ्यां विशीर्णाभ्यां मनस्वीव महागजः} %3-26-15

\twolineshloka
{दृष्ट्वा तं पतितं भूमौ दूषणं निहतं रणे}
{साधु साध्विति काकुत्स्थं सर्वभूतान्यपूजयन्} %3-26-16

\twolineshloka
{एतस्मिन्नन्तरे क्रुद्धास्त्रयः सेनाग्रयायिनः}
{संहत्याभ्यद्रवन् रामं मृत्युपाशावपाशिताः} %3-26-17

\twolineshloka
{महाकपालः स्थूलाक्षः प्रमाथी च महाबलः}
{महाकपालो विपुलं शूलमुद्यम्य राक्षसः} %3-26-18

\twolineshloka
{स्थूलाक्षः पट्टिशं गृह्य प्रमाथी च परश्वधम्}
{दृष्ट्वैवापततस्तांस्तु राघवः सायकैः शितैः} %3-26-19

\twolineshloka
{तीक्ष्णाग्रैः प्रतिजग्राह सम्प्राप्तानतिथीनिव}
{महाकपालस्य शिरश्चिच्छेद रघुनन्दनः} %3-26-20

\twolineshloka
{असङ्ख्येयैस्तु बाणौघैः प्रममाथ प्रमाथिनम्}
{स्थूलाक्षस्याक्षिणी स्थूले पूरयामास सायकैः} %3-26-21

\twolineshloka
{स पपात हतो भूमौ विटपीव महाद्रुमः}
{दूषणस्यानुगान् पञ्चसाहस्रान् कुपितः क्षणात्} %3-26-22

\twolineshloka
{हत्वा तु पञ्चसाहस्रैरनयद् यमसादनम्}
{दूषणं निहतं श्रुत्वा तस्य चैव पदानुगान्} %3-26-23

\twolineshloka
{व्यादिदेश खरः क्रुद्धः सेनाध्यक्षान् महाबलान्}
{अयं विनिहतः सङ्ख्ये दूषणः सपदानुगः} %3-26-24

\twolineshloka
{महत्या सेनया सार्धं युद्ध्वा रामं कुमानुषम्}
{शस्त्रैर्नानाविधाकारैर्हनध्वं सर्वराक्षसाः} %3-26-25

\twolineshloka
{एवमुक्त्वा खरः क्रुद्धो राममेवाभिदुद्रुवे}
{श्येनगामी पृथुग्रीवो यज्ञशत्रुर्विहङ्गमः} %3-26-26

\twolineshloka
{दुर्जयः करवीराक्षः परुषः कालकार्मुकः}
{हेममाली महामाली सर्पास्यो रुधिराशनः} %3-26-27

\twolineshloka
{द्वादशैते महावीर्या बलाध्यक्षाः ससैनिकाः}
{राममेवाभ्यधावन्त विसृजन्तः शरोत्तमान्} %3-26-28

\twolineshloka
{ततः पावकसङ्काशैर्हेमवज्रविभूषितैः}
{जघान शेषं तेजस्वी तस्य सैन्यस्य सायकैः} %3-26-29

\twolineshloka
{ते रुक्मपुङ्खा विशिखाः सधूमा इव पावकाः}
{निजघ्नुस्तानि रक्षांसि वज्रा इव महाद्रुमान्} %3-26-30

\twolineshloka
{रक्षसां तु शतं रामः शतेनैकेन किर्णना}
{सहस्रं तु सहस्रेण जघान रणमूर्धनि} %3-26-31

\twolineshloka
{तैर्भिन्नवर्माभरणाश्छिन्नभिन्नशरासनाः}
{निपेतुः शोणितादिग्धा धरण्यां रजनीचराः} %3-26-32

\twolineshloka
{तैर्मुक्तकेशैः समरे पतितैः शोणितोक्षितैः}
{विस्तीर्णा वसुधा कृत्स्ना महावेदिः कुशैरिव} %3-26-33

\twolineshloka
{तत्क्षणे तु महाघोरं वनं निहतराक्षसम्}
{बभूव निरयप्रख्यं मांसशोणितकर्दमम्} %3-26-34

\twolineshloka
{चतुर्दशसहस्राणि रक्षसां भीमकर्मणाम्}
{हतान्येकेन रामेण मानुषेण पदातिना} %3-26-35

\twolineshloka
{तस्य सैन्यस्य सर्वस्य खरः शेषो महारथः}
{राक्षसस्त्रिशिराश्चैव रामश्च रिपुसूदनः} %3-26-36

\twolineshloka
{शेषा हता महावीर्या राक्षसा रणमूर्धनि}
{घोरा दुर्विषहाः सर्वे लक्ष्मणस्याग्रजेन ते} %3-26-37

\twolineshloka
{ततस्तु तद्भीमबलं महाहवे समीक्ष्य रामेण हतं बलीयसा}
{रथेन रामं महता खरस्ततः समाससादेन्द्र इवोद्यताशनिः} %3-26-38


॥इत्यार्षे श्रीमद्रामायणे वाल्मीकीये आदिकाव्ये अरण्यकाण्डे दूषणादिवधः नाम षड्विंशः सर्गः ॥३-२६॥
