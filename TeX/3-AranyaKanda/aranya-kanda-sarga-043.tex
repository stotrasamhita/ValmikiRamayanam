\sect{त्रिचत्वारिशः सर्गः — लक्ष्मणशङ्काप्रतिसमाधानम्}

\twolineshloka
{सा तं सम्प्रेक्ष्य सुश्रोणी कुसुमानि विचिन्वती}
{हेमराजतवर्णाभ्यां पार्श्वाभ्यामुपशोभितम्} %3-43-1

\twolineshloka
{प्रहृष्टा चानवद्याङ्गी मृष्टहाटकवर्णिनी}
{भर्तारमपि चक्रन्द लक्ष्मणं चैव सायुधम्} %3-43-2

\twolineshloka
{आहूयाहूय च पुनस्तं मृगं साधु वीक्षते}
{आगच्छागच्छ शीघ्रं वै आर्यपुत्र सहानुज} %3-43-3

\twolineshloka
{तावाहूतौ नरव्याघ्रौ वैदेह्या रामलक्ष्मणौ}
{वीक्षमाणौ तु तं देशं तदा ददृशतुर्मृगम्} %3-43-4

\twolineshloka
{शङ्कमानस्तु तं दृष्ट्वा लक्ष्मणो वाक्यमब्रवीत्}
{तमेवैनमहं मन्ये मारीचं राक्षसं मृगम्} %3-43-5

\twolineshloka
{चरन्तो मृगयां हृष्टाः पापेनोपाधिना वने}
{अनेन निहता राम राजानः कामरूपिणा} %3-43-6

\twolineshloka
{अस्य मायाविदो माया मृगरूपमिदं कृतम्}
{भानुमत् पुरुषव्याघ्र गन्धर्वपुरसंनिभम्} %3-43-7

\twolineshloka
{मृगो ह्येवंविधो रत्नविचित्रो नास्ति राघव}
{जगत्यां जगतीनाथ मायैषा हि न संशयः} %3-43-8

\twolineshloka
{एवं ब्रुवाणं काकुत्स्थं प्रतिवार्य शुचिस्मिता}
{उवाच सीता संहृष्टा छद्मना हृतचेतना} %3-43-9

\twolineshloka
{आर्यपुत्राभिरामोऽसौ मृगो हरति मे मनः}
{आनयैनं महाबाहो क्रीडार्थं नो भविष्यति} %3-43-10

\twolineshloka
{इहाश्रमपदेऽस्माकं बहवः पुण्यदर्शनाः}
{मृगाश्चरन्ति सहिताश्चमराः सृमरास्तथा} %3-43-11

\twolineshloka
{ऋक्षाः पृषतसङ्घाश्च वानराः किन्नरास्तथा}
{विहरन्ति महाबाहो रूपश्रेष्ठा महाबलाः} %3-43-12

\twolineshloka
{न चान्यः सदृशो राजन् दृष्टः पूर्वं मृगो मया}
{तेजसा क्षमया दीप्त्या यथायं मृगसत्तमः} %3-43-13

\twolineshloka
{नानावर्णविचित्राङ्गो रत्नभूतो ममाग्रतः}
{द्योतयन् वनमव्यग्रं शोभते शशिसंनिभः} %3-43-14

\twolineshloka
{अहो रूपमहो लक्ष्मीः स्वरसम्पच्च शोभना}
{मृगोऽद्भुतो विचित्राङ्गो हृदयं हरतीव मे} %3-43-15

\twolineshloka
{यदि ग्रहणमभ्येति जीवन् नेव मृगस्तव}
{आश्चर्यभूतं भवति विस्मयं जनयिष्यति} %3-43-16

\twolineshloka
{समाप्तवनवासानां राज्यस्थानां च नः पुनः}
{अन्तःपुरे विभूषार्थो मृग एष भविष्यति} %3-43-17

\twolineshloka
{भरतस्यार्यपुत्रस्य श्वश्रूणां मम च प्रभो}
{मृगरूपमिदं दिव्यं विस्मयं जनयिष्यति} %3-43-18

\twolineshloka
{जीवन्न यदि तेऽभ्येति ग्रहणं मृगसत्तमः}
{अजिनं नरशार्दूल रुचिरं तु भविष्यति} %3-43-19

\twolineshloka
{निहतस्यास्य सत्त्वस्य जाम्बूनदमयत्वचि}
{शष्पबृस्यां विनीतायामिच्छाम्यहमुपासितुम्} %3-43-20

\twolineshloka
{कामवृत्तमिदं रौद्रं स्त्रीणामसदृशं मतम्}
{वपुषा त्वस्य सत्त्वस्य विस्मयो जनितो मम} %3-43-21

\twolineshloka
{तेन काञ्चनरोम्णा तु मणिप्रवरशृङ्गिणा}
{तरुणादित्यवर्णेन नक्षत्रपथवर्चसा} %3-43-22

\twolineshloka
{बभूव राघवस्यापि मनो विस्मयमागतम्}
{इति सीतावचः श्रुत्वा दृष्ट्वा च मृगमद्भुतम्} %3-43-23

\twolineshloka
{लोभितस्तेन रूपेण सीतया च प्रचोदितः}
{उवाच राघवो हृष्टो भ्रातरं लक्ष्मणं वचः} %3-43-24

\twolineshloka
{पश्य लक्ष्मण वैदेह्याः स्पृहामुल्लसितामिमाम्}
{रूपश्रेष्ठतया ह्येष मृगोऽद्य न भविष्यति} %3-43-25

\twolineshloka
{न वने नन्दनोद्देशे न चैत्ररथसंश्रये}
{कुतः पृथिव्यां सौमित्रे योऽस्य कश्चित् समो मृगः} %3-43-26

\twolineshloka
{प्रतिलोमानुलोमाश्च रुचिरा रोमराजयः}
{शोभन्ते मृगमाश्रित्य चित्राः कनकबिन्दुभिः} %3-43-27

\twolineshloka
{पश्यास्य जृम्भमाणस्य दीप्तामग्निशिखोपमाम्}
{जिह्वां मुखान्निःसरन्तीं मेघादिव शतह्रदाम्} %3-43-28

\twolineshloka
{मसारगल्वर्कमुखः शङ्खमुक्तानिभोदरः}
{कस्य नामानिरूप्योऽसौ न मनो लोभयेन्मृगः} %3-43-29

\twolineshloka
{कस्य रूपमिदं दृष्ट्वा जाम्बूनदमयप्रभम्}
{नानारत्नमयं दिव्यं न मनो विस्मयं व्रजेत्} %3-43-30

\twolineshloka
{मांसहेतोरपि मृगान् विहारार्थं च धन्विनः}
{घ्नन्ति लक्ष्मण राजानो मृगयायां महावने} %3-43-31

\twolineshloka
{धनानि व्यवसायेन विचीयन्ते महावने}
{धातवो विविधाश्चापि मणिरत्नसुवर्णिनः} %3-43-32

\twolineshloka
{तत् सारमखिलं नॄणां धनं निचयवर्धनम्}
{मनसा चिन्तितं सर्वं यथा शुक्रस्य लक्ष्मण} %3-43-33

\twolineshloka
{अर्थी येनार्थकृत्येन संव्रजत्यविचारयन्}
{तमर्थमर्थशास्त्रज्ञाः प्राहुरर्थ्याः सुलक्ष्मण} %3-43-34

\twolineshloka
{एतस्य मृगरत्नस्य परार्घ्ये काञ्चनत्वचि}
{उपवेक्ष्यति वैदेही मया सह सुमध्यमा} %3-43-35

\twolineshloka
{न कादली न प्रियकी न प्रवेणी न चाविकी}
{भवेदेतस्य सदृशी स्पर्शेऽनेनेति मे मतिः} %3-43-36

\twolineshloka
{एष चैव मृगः श्रीमान् यश्च दिव्यो नभश्चरः}
{उभावेतौ मृगौ दिव्यौ तारामृगमहीमृगौ} %3-43-37

\twolineshloka
{यदि वायं तथा यन्मां भवेद् वदसि लक्ष्मण}
{मायैषा राक्षसस्येति कर्तव्योऽस्य वधो मया} %3-43-38

\twolineshloka
{एतेन हि नृशंसेन मारीचेनाकृतात्मना}
{वने विचरता पूर्वं हिंसिता मुनिपुंगवाः} %3-43-39

\twolineshloka
{उत्थाय बहवोऽनेन मृगयायां जनाधिपाः}
{निहताः परमेष्वासास्तस्माद् वध्यस्त्वयं मृगः} %3-43-40

\twolineshloka
{पुरस्तादिह वातापिः परिभूय तपस्विनः}
{उदरस्थो द्विजान् हन्ति स्वगर्भोऽश्वतरीमिव} %3-43-41

\twolineshloka
{स कदाचिच्चिराल्लोभादाससाद महामुनिम्}
{अगस्त्यं तेजसा युक्तं भक्ष्यस्तस्य बभूव ह} %3-43-42

\twolineshloka
{समुत्थाने च तद्रूपं कर्तुकामं समीक्ष्य तम्}
{उत्स्मयित्वा तु भगवान् वातापिमिदमब्रवीत्} %3-43-43

\twolineshloka
{त्वयाविगण्य वातापे परिभूताश्च तेजसा}
{जीवलोके द्विजश्रेष्ठास्तस्मादसि जरां गतः} %3-43-44

\twolineshloka
{तद् रक्षो न भवेदेव वातापिरिव लक्ष्मण}
{मद्विधं योऽतिमन्येत धर्मनित्यं जितेन्द्रियम्} %3-43-45

\twolineshloka
{भवेद्धतोऽयं वातापिरगस्त्येनेव मा गतः}
{इह त्वं भव संनद्धो यन्त्रितो रक्ष मैथिलीम्} %3-43-46

\twolineshloka
{अस्यामायत्तमस्माकं यत् कृत्यं रघुनन्दन}
{अहमेनं वधिष्यामि ग्रहीष्याम्यथवा मृगम्} %3-43-47

\twolineshloka
{यावद् गच्छामि सौमित्रे मृगमानयितुं द्रुतम्}
{पश्य लक्ष्मण वैदेह्या मृगत्वचि गतां स्पृहाम्} %3-43-48

\twolineshloka
{त्वचा प्रधानया ह्येष मृगोऽद्य न भविष्यति}
{अप्रमत्तेन ते भाव्यमाश्रमस्थेन सीतया} %3-43-49

\twolineshloka
{यावत् पृषतमेकेन सायकेन निहन्म्यहम्}
{हत्वैतच्चर्म चादाय शीघ्रमेष्यामि लक्ष्मण} %3-43-50

\twolineshloka
{प्रदक्षिणेनातिबलेन पक्षिणा जटायुषा बुद्धिमता च लक्ष्मण}
{भवाप्रमत्तः प्रतिगृह्य मैथिलीं प्रतिक्षणं सर्वत एव शङ्कितः} %3-43-51


॥इत्यार्षे श्रीमद्रामायणे वाल्मीकीये आदिकाव्ये अरण्यकाण्डे लक्ष्मणशङ्काप्रतिसमाधानम् नाम त्रिचत्वारिशः सर्गः ॥३-४३॥
