\sect{अष्टात्रिंशः सर्गः — रामास्त्रमहिमा}

\twolineshloka
{कदाचिदप्यहं वीर्यात् पर्यटन् पृथिवीमिमाम्}
{बलं नागसहस्रस्य धारयन् पर्वतोपमः} %3-38-1

\twolineshloka
{नीलजीमूतसंकाशस्तप्तकाञ्चनकुण्डलः}
{भयं लोकस्य जनयन् किरीटी परिघायुधः} %3-38-2

\twolineshloka
{व्यचरन् दण्डकारण्यमृषिमांसानि भक्षयन्}
{विश्वामित्रोऽथ धर्मात्मा मद्वित्रस्तो महामुनिः} %3-38-3

\twolineshloka
{स्वयं गत्वा दशरथं नरेन्द्रमिदमब्रवीत्}
{अयं रक्षतु मां रामः पर्वकाले समाहितः} %3-38-4

\twolineshloka
{मारीचान्मे भयं घोरं समुत्पन्नं नरेश्वर}
{इत्येवमुक्तो धर्मात्मा राजा दशरथस्तदा} %3-38-5

\twolineshloka
{प्रत्युवाच महाभागं विश्वामित्रं महामुनिम्}
{ऊनद्वादशवर्षोऽयमकृतास्त्रश्च राघवः} %3-38-6

\twolineshloka
{कामं तु मम तत् सैन्यं मया सह गमिष्यति}
{बलेन चतुरङ्गेण स्वयमेत्य निशाचरम्} %3-38-7

\twolineshloka
{वधिष्यामि मुनिश्रेष्ठ शत्रुं तव यथेप्सितम्}
{एवमुक्तः स तु मुनी राजानमिदमब्रवीत्} %3-38-8

\twolineshloka
{रामान्नान्यद् बलं लोके पर्याप्तं तस्य रक्षसः}
{देवतानामपि भवान् समरेष्वभिपालकः} %3-38-9

\twolineshloka
{आसीत् तव कृतं कर्म त्रिलोकविदितं नृप}
{काममस्ति महत् सैन्यं तिष्ठत्विह परंतप} %3-38-10

\twolineshloka
{बालोऽप्येष महातेजाः समर्थस्तस्य निग्रहे}
{गमिष्ये राममादाय स्वस्ति तेऽस्तु परंतप} %3-38-11

\twolineshloka
{इत्येवमुक्त्वा स मुनिस्तमादाय नृपात्मजम्}
{जगाम परमप्रीतो विश्वामित्रः स्वमाश्रमम्} %3-38-12

\twolineshloka
{तं तथा दण्डकारण्ये यज्ञमुद्दिश्य दीक्षितम्}
{बभूवोपस्थितो रामश्चित्रं विस्फारयन् धनुः} %3-38-13

\twolineshloka
{अजातव्यञ्जनः श्रीमान् बालः श्यामः शुभेक्षणः}
{एकवस्त्रधरो धन्वी शिखी कनकमालया} %3-38-14

\twolineshloka
{शोभयन् दण्डकारण्यं दीप्तेन स्वेन तेजसा}
{अदृश्यत तदा रामो बालचन्द्र इवोदितः} %3-38-15

\twolineshloka
{ततोऽहं मेघसंकाशस्तप्तकाञ्चनकुण्डलः}
{बली दत्तवरो दर्पादाजगामाश्रमान्तरम्} %3-38-16

\twolineshloka
{तेन दृष्टः प्रविष्टोऽहं सहसैवोद्यतायुधः}
{मां तु दृष्ट्वा धनुः सज्यमसम्भ्रान्तश्चकार ह} %3-38-17

\twolineshloka
{अवजानन्नहं मोहाद् बालोऽयमिति राघवम्}
{विश्वामित्रस्य तां वेदिमभ्यधावं कृतत्वरः} %3-38-18

\twolineshloka
{तेन मुक्तस्ततो बाणः शितः शत्रुनिबर्हणः}
{तेनाहं ताडितः क्षिप्तः समुद्रे शतयोजने} %3-38-19

\twolineshloka
{नेच्छता तात मां हन्तुं तदा वीरेण रक्षितः}
{रामस्य शरवेगेन निरस्तो भ्रान्तचेतनः} %3-38-20

\twolineshloka
{पातितोऽहं तदा तेन गम्भीरे सागराम्भसि}
{प्राप्य संज्ञां चिरात् तात लङ्कां प्रति गतः पुरीम्} %3-38-21

\twolineshloka
{एवमस्मि तदा मुक्तः सहायास्ते निपातिताः}
{अकृतास्त्रेण रामेण बालेनाक्लिष्टकर्मणा} %3-38-22

\twolineshloka
{तन्मया वार्यमाणस्तु यदि रामेण विग्रहम्}
{करिष्यस्यापदां घोरां क्षिप्रं प्राप्य न शिष्यसि} %3-38-23

\twolineshloka
{क्रीडारतिविधिज्ञानां समाजोत्सवदर्शिनाम्}
{रक्षसां चैव संतापमनर्थं चाहरिष्यसि} %3-38-24

\twolineshloka
{हर्म्यप्रासादसम्बाधां नानारत्नविभूषिताम्}
{द्रक्ष्यसि त्वं पुरीं लङ्कां विनष्टां मैथिलीकृते} %3-38-25

\twolineshloka
{अकुर्वन्तोऽपि पापानि शुचयः पापसंश्रयात्}
{परपापैर्विनश्यन्ति मत्स्या नागह्रदे यथा} %3-38-26

\twolineshloka
{दिव्यचन्दनदिग्धाङ्गान् दिव्याभरणभूषितान्}
{द्रक्ष्यस्यभिहतान् भूमौ तव दोषात् तु राक्षसान्} %3-38-27

\twolineshloka
{हृतदारान् सदारांश्च दश विद्रवतो दिशः}
{हतशेषानशरणान् द्रक्ष्यसि त्वं निशाचरान्} %3-38-28

\twolineshloka
{शरजालपरिक्षिप्तामग्निज्वालासमावृताम्}
{प्रदग्धभवनां लङ्कां द्रक्ष्यसि त्वमसंशयम्} %3-38-29

\twolineshloka
{परदाराभिमर्शात् तु नान्यत् पापतरं महत्}
{प्रमदानां सहस्राणि तव राजन् परिग्रहे} %3-38-30

\twolineshloka
{भव स्वदारनिरतः स्वकुलं रक्ष राक्षसान्}
{मानं वृद्धिं च राज्यं च जीवितं चेष्टमात्मनः} %3-38-31

\twolineshloka
{कलत्राणि च सौम्यानि मित्रवर्गं तथैव च}
{यदीच्छसि चिरं भोक्तुं मा कृथा रामविप्रियम्} %3-38-32

\twolineshloka
{निवार्यमाणः सुहृदा मया भृशं प्रसह्य सीतां यदि धर्षयिष्यसि}
{गमिष्यसि क्षीणबलः सबान्धवो यमक्षयं रामशरास्तजीवितः} %3-38-33


॥इत्यार्षे श्रीमद्रामायणे वाल्मीकीये आदिकाव्ये अरण्यकाण्डे रामास्त्रमहिमा नाम अष्टात्रिंशः सर्गः ॥३-३८॥
