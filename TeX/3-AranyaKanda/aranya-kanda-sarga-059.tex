\sect{एकोनषष्ठितमः सर्गः — लक्ष्मणागमनविगर्हणम्}

\twolineshloka
{अथाश्रमादुपावृत्तमन्तरा रघुनन्दनः}
{परिपप्रच्छ सौमित्रिं रामो दुःखादिदं वचः} %3-59-1

\twolineshloka
{तमुवाच किमर्थं त्वमागतोऽपास्य मैथिलीम्}
{यदा सा तव विश्वासाद् वने विरहिता मया} %3-59-2

\twolineshloka
{दृष्ट्वैवाभ्यागतं त्वां मे मैथिलीं त्यज्य लक्ष्मण}
{शङ्कमानं महत् पापं यत्सत्यं व्यथितं मनः} %3-59-3

\twolineshloka
{स्फुरते नयनं सव्यं बाहुश्च हृदयं च मे}
{दृष्ट्वा लक्ष्मण दूरे त्वां सीताविरहितं पथि} %3-59-4

\twolineshloka
{एवमुक्तस्तु सौमित्रिर्लक्ष्मणः शुभलक्षणः}
{भूयो दुःखसमाविष्टो दुःखितं राममब्रवीत्} %3-59-5

\twolineshloka
{न स्वयं कामकारेण तां त्यक्त्वाहमिहागतः}
{प्रचोदितस्तयैवोग्रैस्त्वत्सकाशमिहागतः} %3-59-6

\twolineshloka
{आर्येणेव परिक्रुष्टं लक्ष्मणेति सुविस्वरम्}
{परित्राहीति यद्वाक्यं मैथिल्यास्तच्छ्रुतिं गतम्} %3-59-7

\twolineshloka
{सा तमार्तस्वरं श्रुत्वा तव स्नेहेन मैथिली}
{गच्छ गच्छेति मामाशु रुदती भयविक्लवा} %3-59-8

\twolineshloka
{प्रचोद्यमानेन मया गच्छेति बहुशस्तया}
{प्रत्युक्ता मैथिली वाक्यमिदं तत् प्रत्ययान्वितम्} %3-59-9

\twolineshloka
{न तत् पश्याम्यहं रक्षो यदस्य भयमावहेत्}
{निर्वृता भव नास्त्येतत् केनाप्येतदुदाहृतम्} %3-59-10

\twolineshloka
{विगर्हितं च नीचं च कथमार्योऽभिधास्यति}
{त्राहीति वचनं सीते यस्त्रायेत् त्रिदशानपि} %3-59-11

\twolineshloka
{किन्निमित्तं तु केनापि भ्रातुरालम्ब्य मे स्वरम्}
{विस्वरं व्याहृतं वाक्यं लक्ष्मण त्राहि मामिति} %3-59-12

\twolineshloka
{राक्षसेनेरितं वाक्यं त्रासात् त्राहीति शोभने}
{न भवत्या व्यथा कार्या कुनारीजनसेविता} %3-59-13

\twolineshloka
{अलं विक्लवतां गन्तुं स्वस्था भव निरुत्सुका}
{न चास्ति त्रिषु लोकेषु पुमान् यो राघवं रणे} %3-59-14

\twolineshloka
{जातो वा जायमानो वा संयुगे यः पराजयेत्}
{अजेयो राघवो युद्धे देवैः शक्रपुरोगमैः} %3-59-15

\twolineshloka
{एवमुक्ता तु वैदेही परिमोहितचेतना}
{उवाचाश्रूणि मुञ्चन्ती दारुणं मामिदं वचः} %3-59-16

\twolineshloka
{भावो मयि तवात्यर्थं पाप एव निवेशितः}
{विनष्टे भ्रातरि प्राप्तुं न च त्वं मामवाप्स्यसे} %3-59-17

\twolineshloka
{सङ्केताद् भरतेन त्वं रामं समनुगच्छसि}
{क्रोशन्तं हि यथात्यर्थं नैनमभ्यवपद्यसे} %3-59-18

\twolineshloka
{रिपुः प्रच्छन्नचारी त्वं मदर्थमनुगच्छसि}
{राघवस्यान्तरं प्रेप्सुस्तथैनं नाभिपद्यसे} %3-59-19

\twolineshloka
{एवमुक्तस्तु वैदेह्या संरब्धो रक्तलोचनः}
{क्रोधात् प्रस्फुरमाणोष्ठ आश्रमादभिनिर्गतः} %3-59-20

\twolineshloka
{एवं ब्रुवाणं सौमित्रिं रामः सन्तापमोहितः}
{अब्रवीद् दुष्कृतं सौम्य तां विना त्वमिहागतः} %3-59-21

\twolineshloka
{जानन्नपि समर्थं मां रक्षसामपवारणे}
{अनेन क्रोधवाक्येन मैथिल्या निर्गतो भवान्} %3-59-22

\twolineshloka
{नहि ते परितुष्यामि त्यक्त्वा यदसि मैथिलीम्}
{क्रुद्धायाः परुषं श्रुत्वा स्त्रिया यत् त्वमिहागतः} %3-59-23

\twolineshloka
{सर्वथा त्वपनीतं ते सीतया यत् प्रचोदितः}
{क्रोधस्य वशमागम्य नाकरोः शासनं मम} %3-59-24

\twolineshloka
{असौ हि राक्षसः शेते शरेणाभिहतो मया}
{मृगरूपेण येनाहमाश्रमादपवाहितः} %3-59-25

\twolineshloka
{विकृष्य चापं परिधाय सायकं सलीलबाणेन च ताडितो मया}
{मार्गीं तनुं त्यज्य च विक्लवस्वरो बभूव केयूरधरः स राक्षसः} %3-59-26

\twolineshloka
{शराहतेनैव तदार्तया गिरा स्वरं ममालम्ब्य सुदूरसुश्रवम्}
{उदाहृतं तद् वचनं सुदारुणं त्वमागतो येन विहाय मैथिलीम्} %3-59-27


॥इत्यार्षे श्रीमद्रामायणे वाल्मीकीये आदिकाव्ये अरण्यकाण्डे लक्ष्मणागमनविगर्हणम् नाम एकोनषष्ठितमः सर्गः ॥३-५९॥
