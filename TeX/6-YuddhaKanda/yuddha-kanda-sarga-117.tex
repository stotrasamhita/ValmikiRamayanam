\sect{सप्तदशाधिकशततमः सर्गः — सीताभर्तृमुखोदीक्षणम्}

\twolineshloka
{तमुवाच महाप्राज्ञः सोऽभिवाद्य प्लवङ्गमः}
{रामं कमलपत्राक्षं वरं सर्वधनुष्मताम्} %6-117-1

\twolineshloka
{यन्निमित्तोऽयमारम्भः कर्मणां यः फलोदयः}
{तां देवीं शोकसन्तप्तां द्रष्टुमर्हसि मैथिलीम्} %6-117-2

\twolineshloka
{सा हि शोकसमाविष्टा बाष्पपर्याकुलेक्षणा}
{मैथिली विजयं श्रुत्वा द्रष्टुं त्वामभिकाङ्क्षति} %6-117-3

\twolineshloka
{पूर्वकात् प्रत्ययाच्चाहमुक्तो विश्वस्तया तया}
{द्रष्टुमिच्छामि भर्तारमिति पर्याकुलेक्षणा} %6-117-4

\twolineshloka
{एवमुक्तो हनुमता रामो धर्मभृतां वरः}
{आगच्छत् सहसा ध्यानमीषद्बाष्पपरिप्लुतः} %6-117-5

\twolineshloka
{स दीर्घमभिनिःश्वस्य जगतीमवलोकयन्}
{उवाच मेघसङ्काशं विभीषणमुपस्थितम्} %6-117-6

\twolineshloka
{दिव्याङ्गरागां वैदेहीं दिव्याभरणभूषिताम्}
{इह सीतां शिरःस्नातामुपस्थापय मा चिरम्} %6-117-7

\twolineshloka
{एवमुक्तस्तु रामेण त्वरमाणो विभीषणः}
{प्रविश्यान्तःपुरं सीतां स्त्रीभिः स्वाभिरचोदयत्} %6-117-8

\twolineshloka
{ततः सीतां महाभागां दृष्ट्वोवाच विभीषणः}
{मूर्ध्नि बद्धाञ्जलिः श्रीमान् विनीतो राक्षसेश्वरः} %6-117-9

\twolineshloka
{दिव्याङ्गरागा वैदेहि दिव्याभरणभूषिता}
{यानमारोह भद्रं ते भर्ता त्वां द्रष्टुमिच्छति} %6-117-10

\twolineshloka
{एवमुक्ता तु वैदेही प्रत्युवाच विभीषणम्}
{अस्नात्वा द्रष्टुमिच्छामि भर्तारं राक्षसेश्वर} %6-117-11

\twolineshloka
{तस्यास्तद् वचनं श्रुत्वा प्रत्युवाच विभीषणः}
{यथाऽऽह रामो भर्ता ते तत् तथा कर्तुमर्हसि} %6-117-12

\twolineshloka
{तस्य तद् वचनं श्रुत्वा मैथिली पतिदेवता}
{भर्तृभक्त्यावृता साध्वी तथेति प्रत्यभाषत} %6-117-13

\twolineshloka
{ततः सीतां शिरःस्नातां संयुक्तां प्रतिकर्मणा}
{महार्हाभरणोपेतां महार्हाम्बरधारिणीम्} %6-117-14

\twolineshloka
{आरोप्य शिबिकां दीप्तां परार्घ्याम्बरसंवृताम्}
{रक्षोभिर्बहुभिर्गुप्तामाजहार विभीषणः} %6-117-15

\twolineshloka
{सोऽभिगम्य महात्मानं ज्ञात्वापि ध्यानमास्थितम्}
{प्रणतश्च प्रहृष्टश्च प्राप्तां सीतां न्यवेदयत्} %6-117-16

\twolineshloka
{तामागतामुपश्रुत्य रक्षोगृहचिरोषिताम्}
{रोषं हर्षं च दैन्यं च राघवः प्राप शत्रुहा} %6-117-17

\twolineshloka
{ततो यानगतां सीतां सविमर्शं विचारयन्}
{विभीषणमिदं वाक्यमहृष्टो राघवोऽब्रवीत्} %6-117-18

\twolineshloka
{राक्षसाधिपते सौम्य नित्यं मद्विजये रत}
{वैदेही सन्निकर्षं मे क्षिप्रं समभिगच्छतु} %6-117-19

\twolineshloka
{तस्य तद् वचनं श्रुत्वा राघवस्य विभीषणः}
{तूर्णमुत्सारणं तत्र कारयामास धर्मवित्} %6-117-20

\twolineshloka
{कञ्चुकोष्णीषिणस्तत्र वेत्रझर्झरपाणयः}
{उत्सारयन्तस्तान् योधान् समन्तात् परिचक्रमुः} %6-117-21

\twolineshloka
{ऋक्षाणां वानराणां च राक्षसानां च सर्वशः}
{वृन्दान्युत्सार्यमाणानि दूरमुत्तस्थुरन्ततः} %6-117-22

\twolineshloka
{तेषामुत्सार्यमाणानां निःस्वनः सुमहानभूत्}
{वायुनोद्धूयमानस्य सागरस्येव निःस्वनः} %6-117-23

\twolineshloka
{उत्सार्यमाणांस्तान् दृष्ट्वा समन्ताज्जातसम्भ्रमान्}
{दाक्षिण्यात्तदमर्षाच्च वारयामास राघवः} %6-117-24

\twolineshloka
{संरम्भाच्चाब्रवीद् रामश्चक्षुषा प्रदहन्निव}
{विभीषणं महाप्राज्ञं सोपालम्भमिदं वचः} %6-117-25

\twolineshloka
{किमर्थं मामनादृत्य क्लिश्यतेऽयं त्वया जनः}
{निवर्तयैनमुद्वेगं जनोऽयं स्वजनो मम} %6-117-26

\twolineshloka
{न गृहाणि न वस्त्राणि न प्राकारस्तिरस्क्रिया}
{नेदृशा राजसत्कारा वृत्तमावरणं स्त्रियाः} %6-117-27

\twolineshloka
{व्यसनेषु न कृच्छ्रेषु न युद्धेषु स्वयंवरे}
{न क्रतौ नो विवाहे वा दर्शनं दूष्यते स्त्रियाः} %6-117-28

\twolineshloka
{सैषा विपद्गता चैव कृच्छ्रेण च समन्विता}
{दर्शने नास्ति दोषोऽस्या मत्समीपे विशेषतः} %6-117-29

\twolineshloka
{विसृज्य शिबिकां तस्मात् पद्भ्यामेवापसर्पतु}
{समीपे मम वैदेहीं पश्यन्त्वेते वनौकसः} %6-117-30

\twolineshloka
{एवमुक्तस्तु रामेण सविमर्शो विभीषणः}
{रामस्योपानयत् सीतां सन्निकर्षं विनीतवत्} %6-117-31

\twolineshloka
{ततो लक्ष्मणसुग्रीवौ हनूमांश्च प्लवङ्गमः}
{निशम्य वाक्यं रामस्य बभूवुर्व्यथिता भृशम्} %6-117-32

\twolineshloka
{कलत्रनिरपेक्षैश्च इङ्गितैरस्य दारुणैः}
{अप्रीतमिव सीतायां तर्कयन्ति स्म राघवम्} %6-117-33

\twolineshloka
{लज्जया त्ववलीयन्ती स्वेषु गात्रेषु मैथिली}
{विभीषणेनानुगता भर्तारं साभ्यवर्तत} %6-117-34

\twolineshloka
{विस्मयाच्च प्रहर्षाच्च स्नेहाच्च पतिदेवता}
{उदैक्षत मुखं भर्तुः सौम्यं सौम्यतरानना} %6-117-35

\twolineshloka
{अथ समपनुदन्मनःक्लमं सा सुचिरमदृष्टमुदीक्ष्य वै प्रियस्य}
{वदनमुदितपूर्णचन्द्रकान्तं विमलशशाङ्कनिभानना तदाऽऽसीत्} %6-117-36


॥इत्यार्षे श्रीमद्रामायणे वाल्मीकीये आदिकाव्ये युद्धकाण्डे सीताभर्तृमुखोदीक्षणम् नाम सप्तदशाधिकशततमः सर्गः ॥६-११७॥
