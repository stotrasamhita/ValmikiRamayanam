\sect{षोडशाधिकशततमः सर्गः — मैथिलीप्रियनिवेदनम्}

\twolineshloka
{इति प्रतिसमादिष्टो हनूमान् मारुतात्मजः}
{प्रविवेश पुरीं लङ्कां पूज्यमानो निशाचरैः} %6-116-1

\twolineshloka
{प्रविश्य च पुरीं लङ्कामनुज्ञाप्य विभीषणम्}
{ततस्तेनाभ्यनुज्ञातो हनूमान् वृक्षवाटिकाम्} %6-116-2

\twolineshloka
{सम्प्रविश्य यथान्यायं सीताया विदितो हरिः}
{ददर्श मृजया हीनां सातङ्कां रोहिणीमिव} %6-116-3

\twolineshloka
{वृक्षमूले निरानन्दां राक्षसीभिः परीवृताम्}
{निभृतः प्रणतः प्रह्वः सोऽभिगम्याभिवाद्य च} %6-116-4

\twolineshloka
{दृष्ट्वा तमागतं देवी हनूमन्तं महाबलम्}
{तूष्णीमास्त तदा दृष्ट्वा स्मृत्वा हृष्टाभवत् तदा} %6-116-5

\twolineshloka
{सौम्यं तस्या मुखं दृष्ट्वा हनूमान् प्लवगोत्तमः}
{रामस्य वचनं सर्वमाख्यातुमुपचक्रमे} %6-116-6

\twolineshloka
{वैदेहि कुशली रामः सहसुग्रीवलक्ष्मणः}
{कुशलं चाह सिद्धार्थो हतशत्रुरमित्रजित्} %6-116-7

\twolineshloka
{विभीषणसहायेन रामेण हरिभिः सह}
{निहतो रावणो देवि लक्ष्मणेन च वीर्यवान्} %6-116-8

\twolineshloka
{प्रियमाख्यामि ते देवि भूयश्च त्वां सभाजये}
{तव प्रभावाद् धर्मज्ञे महान् रामेण संयुगे} %6-116-9

\twolineshloka
{लब्धोऽयं विजयः सीते स्वस्था भव गतज्वरा}
{रावणश्च हतः शत्रुर्लङ्का चैव वशीकृता} %6-116-10

\twolineshloka
{मया ह्यलब्धनिद्रेण धृतेन तव निर्जये}
{प्रतिज्ञैषा विनिस्तीर्णा बद्ध्वा सेतुं महोदधौ} %6-116-11

\twolineshloka
{सम्भ्रमश्च न कर्तव्यो वर्तन्त्या रावणालये}
{विभीषणविधेयं हि लङ्कैश्वर्यमिदं कृतम्} %6-116-12

\twolineshloka
{तदाश्वसिहि विस्रब्धं स्वगृहे परिवर्तसे}
{अयं चाभ्येति संहृष्टस्त्वद्दर्शनसमुत्सुकः} %6-116-13

\twolineshloka
{एवमुक्ता तु सा देवी सीता शशिनिभानना}
{प्रहर्षेणावरुद्धा सा व्याहर्तुं न शशाक ह} %6-116-14

\twolineshloka
{ततोऽब्रवीद्धरिवरः सीतामप्रतिजल्पतीम्}
{किं त्वं चिन्तयसे देवि किं च मां नाभिभाषसे} %6-116-15

\twolineshloka
{एवमुक्ता हनुमता सीता धर्मपथे स्थिता}
{अब्रवीत् परमप्रीता बाष्पगद्गदया गिरा} %6-116-16

\twolineshloka
{प्रियमेतदुपश्रुत्य भर्तुर्विजयसंश्रितम्}
{प्रहर्षवशमापन्ना निर्वाक्यास्मि क्षणान्तरम्} %6-116-17

\twolineshloka
{नहि पश्यामि सदृशं चिन्तयन्ती प्लवङ्गम}
{आख्यानकस्य भवतो दातुं प्रत्यभिनन्दनम्} %6-116-18

\twolineshloka
{न हि पश्यामि तत् सौम्य पृथिव्यामपि वानर}
{सदृशं यत्प्रियाख्याने तव दत्त्वा भवेत् सुखम्} %6-116-19

\twolineshloka
{हिरण्यं वा सुवर्णं वा रत्नानि विविधानि च}
{राज्यं वा त्रिषु लोकेषु एतन्नार्हति भाषितम्} %6-116-20

\twolineshloka
{एवमुक्तस्तु वैदेह्या प्रत्युवाच प्लवङ्गमः}
{प्रगृहीताञ्जलिर्हर्षात् सीतायाः प्रमुखे स्थितः} %6-116-21

\twolineshloka
{भर्तुः प्रियहिते युक्ते भर्तुर्विजयकाङ्क्षिणि}
{स्निग्धमेवंविधं वाक्यं त्वमेवार्हस्यनिन्दिते} %6-116-22

\twolineshloka
{तवैतद् वचनं सौम्ये सारवत् स्निग्धमेव च}
{रत्नौघाद् विविधाच्चापि देवराज्याद् विशिष्यते} %6-116-23

\twolineshloka
{अर्थतश्च मया प्राप्ता देवराज्यादयो गुणाः}
{हतशत्रुं विजयिनं रामं पश्यामि सुस्थितम्} %6-116-24

\twolineshloka
{तस्य तद् वचनं श्रुत्वा मैथिली जनकात्मजा}
{ततः शुभतरं वाक्यमुवाच पवनात्मजम्} %6-116-25

\twolineshloka
{अतिलक्षणसम्पन्नं माधुर्यगुणभूषणम्}
{बुद्ध्या ह्यष्टाङ्गया युक्तं त्वमेवार्हसि भाषितुम्} %6-116-26

\twolineshloka
{श्लाघनीयोऽनिलस्य त्वं सुतः परमधार्मिकः}
{बलं शौर्यं श्रुतं सत्त्वं विक्रमो दाक्ष्यमुत्तमम्} %6-116-27

\twolineshloka
{तेजः क्षमा धृतिः स्थैर्यं विनीतत्वं न संशयः}
{एते चान्ये च बहवो गुणास्त्वय्येव शोभनाः} %6-116-28

\twolineshloka
{अथोवाच पुनः सीतामसम्भ्रान्तो विनीतवत्}
{प्रगृहीताञ्जलिर्हर्षात् सीतायाः प्रमुखे स्थितः} %6-116-29

\twolineshloka
{इमास्तु खलु राक्षस्यो यदि त्वमनुमन्यसे}
{हन्तुमिच्छामि ताः सर्वा याभिस्त्वं तर्जिता पुरा} %6-116-30

\twolineshloka
{क्लिश्यन्तीं पतिदेवां त्वामशोकवनिकां गताम्}
{घोररूपसमाचाराः क्रूराः क्रूरतरेक्षणाः} %6-116-31

\twolineshloka
{इह श्रुता मया देवि राक्षस्यो विकृताननाः}
{असकृत्परुषैर्वाक्यैर्वदन्त्यो रावणाज्ञया} %6-116-32

\twolineshloka
{विकृता विकृताकाराः क्रूराः क्रूरकचेक्षणाः}
{इच्छामि विविधैर्घातैर्हन्तुमेताः सुदारुणाः} %6-116-33

\twolineshloka
{राक्षस्यो दारुणकथा वरमेतत् प्रयच्छ मे}
{मुष्टिभिः पार्ष्णिघातैश्च विशालैश्चैव बाहुभिः} %6-116-34

\twolineshloka
{जङ्घाजानुप्रहारैश्च दन्तानां चैव पीडनैः}
{कर्तनैः कर्णनासानां केशानां लुञ्चनैस्तथा} %6-116-35

\twolineshloka
{निपात्य हन्तुमिच्छामि तव विप्रियकारिणीः}
{एवं प्रहारैर्बहुभिः सम्प्रहार्य यशस्विनि} %6-116-36

\twolineshloka
{घातये तीव्ररूपाभिर्याभिस्त्वं तर्जिता पुरा}
{इत्युक्ता सा हनुमता कृपणा दीनवत्सला} %6-116-37

\twolineshloka
{हनूमन्तमुवाचेदं चिन्तयित्वा विमृश्य च}
{राजसंश्रयवश्यानां कुर्वतीनां पराज्ञया} %6-116-38

\twolineshloka
{विधेयानां च दासीनां कः कुप्येद् वानरोत्तम}
{भाग्यवैषम्यदोषेण पुरस्ताद्दुष्कृतेन च} %6-116-39

\twolineshloka
{मयैतत् प्राप्यते सर्वं स्वकृतं ह्युपभुज्यते}
{मैवं वद महाबाहो दैवी ह्येषा परा गतिः} %6-116-40

\twolineshloka
{प्राप्तव्यं तु दशायोगान्मयैतदिति निश्चितम्}
{दासीनां रावणस्याहं मर्षयामीह दुर्बला} %6-116-41

\twolineshloka
{आज्ञप्ता राक्षसेनेह राक्षस्यस्तर्जयन्ति माम्}
{हते तस्मिन् न कुर्वन्ति तर्जनं मारुतात्मज} %6-116-42

\twolineshloka
{अयं व्याघ्रसमीपे तु पुराणो धर्मसंहितः}
{ऋक्षेण गीतः श्लोकोऽस्ति तं निबोध प्लवङ्गम} %6-116-43

\twolineshloka
{न परः पापमादत्ते परेषां पापकर्मणाम्}
{समयो रक्षितव्यस्तु सन्तश्चारित्रभूषणाः} %6-116-44

\twolineshloka
{पापानां वा शुभानां वा वधार्हाणामथापि वा}
{कार्यं कारुण्यमार्येण न कश्चिन्नापराध्यति} %6-116-45

\twolineshloka
{लोकहिंसाविहाराणां क्रूराणां पापकर्मणाम्}
{कुर्वतामपि पापानि नैव कार्यमशोभनम्} %6-116-46

\twolineshloka
{एवमुक्तस्तु हनुमान् सीतया वाक्यकोविदः}
{प्रत्युवाच ततः सीतां रामपत्नीमनिन्दिताम्} %6-116-47

\twolineshloka
{युक्ता रामस्य भवती धर्मपत्नी गुणान्विता}
{प्रतिसन्दिश मां देवि गमिष्ये यत्र राघवः} %6-116-48

\twolineshloka
{एवमुक्ता हनुमता वैदेही जनकात्मजा}
{साब्रवीद् द्रष्टुमिच्छामि भर्तारं भक्तवत्सलम्} %6-116-49

\twolineshloka
{तस्यास्तद् वचनं श्रुत्वा हनूमान् मारुतात्मजः}
{हर्षयन् मैथिलीं वाक्यमुवाचेदं महामतिः} %6-116-50

\twolineshloka
{पूर्णचन्द्रमुखं रामं द्रक्ष्यस्यद्य सलक्ष्मणम्}
{स्थितमित्रं हतामित्रं शचीवेन्द्रं सुरेश्वरम्} %6-116-51

\twolineshloka
{तामेवमुक्त्वा भ्राजन्तीं सीतां साक्षादिव श्रियम्}
{आजगाम महातेजा हनूमान् यत्र राघवः} %6-116-52

\twolineshloka
{सपदि हरिवरस्ततो हनूमान् प्रतिवचनं जनकेश्वरात्मजायाः}
{कथितमकथयद् यथाक्रमेण त्रिदशवरप्रतिमाय राघवाय} %6-116-53


॥इत्यार्षे श्रीमद्रामायणे वाल्मीकीये आदिकाव्ये युद्धकाण्डे मैथिलीप्रियनिवेदनम् नाम षोडशाधिकशततमः सर्गः ॥६-११६॥
