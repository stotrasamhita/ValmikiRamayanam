\sect{द्वादशाधिकशततमः सर्गः — विभीषणविलापः}

\twolineshloka
{भ्रातरं निहतं दृष्ट्वा शयानं निर्जितं रणे}
{शोकवेगपरीतात्मा विललाप विभीषणः} %6-112-1

\twolineshloka
{वीरविक्रान्त विख्यात प्रवीण नयकोविद}
{महार्हशयनोपेत किं शेषे निहतो भुवि} %6-112-2

\twolineshloka
{निक्षिप्य दीर्घौ निश्चेष्टौ भुजावङ्गदभूषितौ}
{मुकुटेनापवृत्तेन भास्कराकारवर्चसा} %6-112-3

\twolineshloka
{तदिदं वीर सम्प्राप्तं यन्मया पूर्वमीरितम्}
{काममोहपरीतस्य यत् तन्न रुचितं तव} %6-112-4

\threelineshloka
{यन्न दर्पात् प्रहस्तो वा नेन्द्रजिन्नापरे जनाः}
{न कुम्भकर्णोऽतिरथो नातिकायो नरान्तकः}
{न स्वयं बहु मन्येथास्तस्योदर्कोऽयमागतः} %6-112-5

\twolineshloka
{गतः सेतुः सुनीतानां गतो धर्मस्य विग्रहः}
{गतः सत्त्वस्य संक्षेपः सुहस्तानां गतिर्गता} %6-112-6

\threelineshloka
{आदित्यः पतितो भूमौ मग्नस्तमसि चन्द्रमाः}
{चित्रभानुः प्रशान्तार्चिर्व्यवसायो निरुद्यमः}
{अस्मिन् निपतिते वीरे भूमौ शस्त्रभृतां वरे} %6-112-7

\twolineshloka
{किं शेषमिहलोकस्य गतसत्त्वस्य सम्प्रति}
{रणे राक्षसशार्दूले प्रसुप्त इव पांसुषु} %6-112-8

\twolineshloka
{धृतिप्रवालः प्रसभाग्र्यपुष्पस्तपोबलः शौर्यनिबद्धमूलः}
{रणे महान् राक्षसराजवृक्षः सम्मर्दितो राघवमारुतेन} %6-112-9

\twolineshloka
{तेजोविषाणः कुलवंशवंशः कोपप्रसादापरगात्रहस्तः}
{इक्ष्वाकुसिंहावगृहीतदेहः सुप्तः क्षितौ रावणगन्धहस्ती} %6-112-10

\twolineshloka
{पराक्रमोत्साहविजृम्भितार्चिर्निःश्वासधूमः स्वबलप्रतापः}
{प्रतापवान् संयति राक्षसाग्निर्निर्वापितो रामपयोधरेण} %6-112-11

\twolineshloka
{सिंहर्क्षलाङ्गूलककुद्विषाणः पराभिजिद्गन्धनगन्धवाहः}
{रक्षोवृषश्चापलकर्णचक्षुः क्षितीश्वरव्याघ्रहतोऽवसन्नः} %6-112-12

\twolineshloka
{वदन्तं हेतुमद्वाक्यं परिदृष्टार्थनिश्चयम्}
{रामः शोकसमाविष्टमित्युवाच विभीषणम्} %6-112-13

\twolineshloka
{नायं विनष्टो निश्चेष्टः समरे चण्डविक्रमः}
{अत्युन्नतमहोत्साहः पतितोऽयमशङ्कितः} %6-112-14

\twolineshloka
{नैवं विनष्टाः शोचन्ते क्षत्रधर्मव्यवस्थिताः}
{वृद्धिमाशंसमाना ये निपतन्ति रणाजिरे} %6-112-15

\twolineshloka
{येन सेन्द्रास्त्रयो लोकास्त्रासिता युधि धीमता}
{तस्मिन् कालसमायुक्ते न कालः परिशोचितुम्} %6-112-16

\twolineshloka
{नैकान्तविजयो युद्धे भूतपूर्वः कदाचन}
{परैर्वा हन्यते वीरः परान् वा हन्ति संयुगे} %6-112-17

\twolineshloka
{इयं हि पूर्वैः संदिष्टा गतिः क्षत्रियसम्मता}
{क्षत्रियो निहतः संख्ये न शोच्य इति निश्चयः} %6-112-18

\twolineshloka
{तदेवं निश्चयं दृष्ट्वा तत्त्वमास्थाय विज्वरः}
{यदिहानन्तरं कार्यं कल्प्यं तदनुचिन्तय} %6-112-19

\twolineshloka
{तमुक्तवाक्यं विक्रान्तं राजपुत्रं विभीषणः}
{उवाच शोकसंतप्तो भ्रातुर्हितमनन्तरम्} %6-112-20

\twolineshloka
{योऽयं विमर्देष्वविभग्नपूर्वः सुरैः समस्तैरपि वासवेन}
{भवन्तमासाद्य रणे विभग्नो वेलामिवासाद्य यथा समुद्रः} %6-112-21

\twolineshloka
{अनेन दत्तानि वनीपकेषु भुक्ताश्च भोगा निभृताश्च भृत्याः}
{धनानि मित्रेषु समर्पितानि वैराण्यमित्रेषु च यापितानि} %6-112-22

\twolineshloka
{एषोऽहिताग्निश्च महातपाश्च वेदान्तगः कर्मसु चाग्र्यशूरः}
{एतस्य यत् प्रेतगतस्य कृत्यं तत् कर्तुमिच्छामि तव प्रसादात्} %6-112-23

\twolineshloka
{स तस्य वाक्यैः करुणैर्महात्मा सम्बोधितः साधु विभीषणेन}
{आज्ञापयामास नरेन्द्रसूनुः स्वर्गीयमाधानमदीनसत्त्वः} %6-112-24

\twolineshloka
{मरणान्तानि वैराणि निर्वृत्तं नः प्रयोजनम्}
{क्रियतामस्य संस्कारो ममाप्येष यथा तव} %6-112-25


॥इत्यार्षे श्रीमद्रामायणे वाल्मीकीये आदिकाव्ये युद्धकाण्डे विभीषणविलापः नाम द्वादशाधिकशततमः सर्गः ॥६-११२॥
