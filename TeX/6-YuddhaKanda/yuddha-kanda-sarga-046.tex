\sect{षड्चत्वारिंशः सर्गः — सुग्रीवाद्यनुशोकः}

\twolineshloka
{ततो द्यां पृथिवीं चैव वीक्षमाणा वनौकसः}
{ददृशुः संततौ बाणैर्भ्रातरौ रामलक्ष्मणौ} %6-46-1

\twolineshloka
{वृष्ट्वेवोपरते देवे कृतकर्मणि राक्षसे}
{आजगामाथ तं देशं ससुग्रीवो विभीषणः} %6-46-2

\twolineshloka
{नीलश्च द्विविदो मैन्दः सुषेणः कुमुदोऽङ्गदः}
{तूर्णं हनुमता सार्धमन्वशोचन्त राघवौ} %6-46-3

\twolineshloka
{अचेष्टौ मन्दनिःश्वासौ शोणितेन परिप्लुतौ}
{शरजालाचितौ स्तब्धौ शयानौ शरतल्पगौ} %6-46-4

\twolineshloka
{निःश्वसन्तौ यथा सर्पौ निश्चेष्टौ मन्दविक्रमौ}
{रुधिरस्रावदिग्धाङ्गौ तपनीयाविव ध्वजौ} %6-46-5

\twolineshloka
{तौ वीरशयने वीरौ शयानौ मन्दचेष्टितौ}
{यूथपैः स्वैः परिवृतौ बाष्पव्याकुललोचनैः} %6-46-6

\twolineshloka
{राघवौ पतितौ दृष्ट्वा शरजालसमन्वितौ}
{बभूवुर्व्यथिताः सर्वे वानराः सविभीषणाः} %6-46-7

\twolineshloka
{अन्तरिक्षं निरीक्षन्तो दिशः सर्वाश्च वानराः}
{न चैनं मायया छन्नं ददृशू रावणिं रणे} %6-46-8

\threelineshloka
{तं तु मायाप्रतिच्छन्नं माययैव विभीषणः}
{वीक्षमाणो ददर्शाग्रे भ्रातुः पुत्रमवस्थितम्}
{तमप्रतिमकर्माणमप्रतिद्वन्द्वमाहवे} %6-46-9

\twolineshloka
{ददर्शान्तर्हितं वीरं वरदानाद् विभीषणः}
{तेजसा यशसा चैव विक्रमेण च संयुतः} %6-46-10

\twolineshloka
{इन्द्रजित् त्वात्मनः कर्म तौ शयानौ समीक्ष्य च}
{उवाच परमप्रीतो हर्षयन् सर्वराक्षसान्} %6-46-11

\twolineshloka
{दूषणस्य च हन्तारौ खरस्य च महाबलौ}
{सादितौ मामकैर्बाणैर्भ्रातरौ रामलक्ष्मणौ} %6-46-12

\twolineshloka
{नेमौ मोक्षयितुं शक्यावेतस्मादिषुबन्धनात्}
{सर्वैरपि समागम्य सर्षिसङ्घैः सुरासुरैः} %6-46-13

\twolineshloka
{यत्कृते चिन्तयानस्य शोकार्तस्य पितुर्मम}
{अस्पृष्ट्वा शयनं गात्रैस्त्रियामा याति शर्वरी} %6-46-14

\twolineshloka
{कृत्स्नेयं यत्कृते लङ्का नदी वर्षास्विवाकुला}
{सोऽयं मूलहरोऽनर्थः सर्वेषां शमितो मया} %6-46-15

\twolineshloka
{रामस्य लक्ष्मणस्यैव सर्वेषां च वनौकसाम्}
{विक्रमा निष्फलाः सर्वे यथा शरदि तोयदाः} %6-46-16

\twolineshloka
{एवमुक्त्वा तु तान् सर्वान् राक्षसान् परिपश्यतः}
{यूथपानपि तान् सर्वांस्ताडयत् स च रावणिः} %6-46-17

\twolineshloka
{नीलं नवभिराहत्य मैन्दं सद्विविदं तथा}
{त्रिभिस्त्रिभिरमित्रघ्नस्तताप परमेषुभिः} %6-46-18

\twolineshloka
{जाम्बवन्तं महेष्वासो विद्ध्वा बाणेन वक्षसि}
{हनूमतो वेगवतो विससर्ज शरान् दश} %6-46-19

\twolineshloka
{गवाक्षं शरभं चैव तावप्यमितविक्रमौ}
{द्वाभ्यां द्वाभ्यां महावेगो विव्याध युधि रावणिः} %6-46-20

\twolineshloka
{गोलाङ्गूलेश्वरं चैव वालिपुत्रमथाङ्गदम्}
{विव्याध बहुभिर्बाणैस्त्वरमाणोऽथ रावणिः} %6-46-21

\twolineshloka
{तान् वानरवरान् भित्त्वा शरैरग्निशिखोपमैः}
{ननाद बलवांस्तत्र महासत्त्वः स रावणिः} %6-46-22

\twolineshloka
{तानर्दयित्वा बाणौघैस्त्रसयित्वा च वानरान्}
{प्रजहास महाबाहुर्वचनं चेदमब्रवीत्} %6-46-23

\twolineshloka
{शरबन्धेन घोरेण मया बद्धौ चमूमुखे}
{सहितौ भ्रातरावेतौ निशामयत राक्षसाः} %6-46-24

\twolineshloka
{एवमुक्तास्तु ते सर्वे राक्षसाः कूटयोधिनः}
{परं विस्मयमापन्नाः कर्मणा तेन हर्षिताः} %6-46-25

\twolineshloka
{विनेदुश्च महानादान् सर्वे ते जलदोपमाः}
{हतो राम इति ज्ञात्वा रावणिं समपूजयन्} %6-46-26

\twolineshloka
{निष्पन्दौ तु तदा दृष्ट्वा भ्रातरौ रामलक्ष्मणौ}
{वसुधायां निरुच्छ्वासौ हतावित्यन्वमन्यत} %6-46-27

\twolineshloka
{हर्षेण तु समाविष्ट इन्द्रजित् समितिञ्जयः}
{प्रविवेश पुरीं लङ्कां हर्षयन् सर्वनैर्ऋतान्} %6-46-28

\twolineshloka
{रामलक्ष्मणयोर्दृष्ट्वा शरीरे सायकैश्चिते}
{सर्वाणि चाङ्गोपाङ्गानि सुग्रीवं भयमाविशत्} %6-46-29

\twolineshloka
{तमुवाच परित्रस्तं वानरेन्द्रं विभीषणः}
{सबाष्पवदनं दीनं शोकव्याकुललोचनम्} %6-46-30

\twolineshloka
{अलं त्रासेन सुग्रीव बाष्पवेगो निगृह्यताम्}
{एवंप्रायाणि युद्धानि विजयो नास्ति नैष्ठिकः} %6-46-31

\twolineshloka
{सभाग्यशेषतास्माकं यदि वीर भविष्यति}
{मोहमेतौ प्रहास्येते महात्मानौ महाबलौ} %6-46-32

\twolineshloka
{पर्यवस्थापयात्मानमनाथं मां च वानर}
{सत्यधर्माभिरक्तानां नास्ति मृत्युकृतं भयम्} %6-46-33

\twolineshloka
{एवमुक्त्वा ततस्तस्य जलक्लिन्नेन पाणिना}
{सुग्रीवस्य शुभे नेत्रे प्रममार्ज विभीषणः} %6-46-34

\twolineshloka
{ततः सलिलमादाय विद्यया परिजप्य च}
{सुग्रीवनेत्रे धर्मात्मा प्रममार्ज विभीषणः} %6-46-35

\twolineshloka
{विमृज्य वदनं तस्य कपिराजस्य धीमतः}
{अब्रवीत् कालसम्प्राप्तमसम्भ्रान्तमिदं वचः} %6-46-36

\twolineshloka
{न कालः कपिराजेन्द्र वैक्लव्यमवलम्बितुम्}
{अतिस्नेहोऽपि कालेऽस्मिन् मरणायोपकल्पते} %6-46-37

\twolineshloka
{तस्मादुत्सृज्य वैक्लव्यं सर्वकार्यविनाशनम्}
{हितं रामपुरोगाणां सैन्यानामनुचिन्तय} %6-46-38

\twolineshloka
{अथ वा रक्ष्यतां रामो यावत्संज्ञाविपर्ययः}
{लब्धसंज्ञौ हि काकुत्स्थौ भयं नौ व्यपनेष्यतः} %6-46-39

\twolineshloka
{नैतत् किंचन रामस्य न च रामो मुमूर्षति}
{नह्येनं हास्यते लक्ष्मीर्दुर्लभा या गतायुषाम्} %6-46-40

\twolineshloka
{तस्मादाश्वासयात्मानं बलं चाश्वासय स्वकम्}
{यावत् सैन्यानि सर्वाणि पुनः संस्थापयाम्यहम्} %6-46-41

\twolineshloka
{एते हि फुल्लनयनास्त्रासादागतसाध्वसाः}
{कर्णे कर्णे प्रकथिता हरयो हरिसत्तम} %6-46-42

\twolineshloka
{मां तु दृष्ट्वा प्रधावन्तमनीकं सम्प्रहर्षितम्}
{त्यजन्तु हरयस्त्रासं भुक्तपूर्वामिव स्रजम्} %6-46-43

\twolineshloka
{समाश्वास्य तु सुग्रीवं राक्षसेन्द्रो विभीषणः}
{विद्रुतं वानरानीकं तत् समाश्वासयत् पुनः} %6-46-44

\twolineshloka
{इन्द्रजित् तु महामायः सर्वसैन्यसमावृतः}
{विवेश नगरीं लङ्कां पितरं चाभ्युपागमत्} %6-46-45

\twolineshloka
{तत्र रावणमासाद्य अभिवाद्य कृताञ्जलिः}
{आचचक्षे प्रियं पित्रे निहतौ रामलक्ष्मणौ} %6-46-46

\twolineshloka
{उत्पपात ततो हृष्टः पुत्रं च परिषस्वजे}
{रावणो रक्षसां मध्ये श्रुत्वा शत्रू निपातितौ} %6-46-47

\twolineshloka
{उपाघ्राय च तं मूर्ध्नि पप्रच्छ प्रीतमानसः}
{पृच्छते च यथावृत्तं पित्रे तस्मै न्यवेदयत्} %6-46-48

\onelineshloka
{यथा तौ शरबन्धेन निश्चेष्टौ निष्प्रभौ कृतौ} %6-46-49

\twolineshloka
{स हर्षवेगानुगतान्तरात्मा श्रुत्वा गिरं तस्य महारथस्य}
{जहौ ज्वरं दाशरथेः समुत्थं प्रहृष्टवाचाभिननन्द पुत्रम्} %6-46-50


॥इत्यार्षे श्रीमद्रामायणे वाल्मीकीये आदिकाव्ये युद्धकाण्डे सुग्रीवाद्यनुशोकः नाम षड्चत्वारिंशः सर्गः ॥६-४६॥
