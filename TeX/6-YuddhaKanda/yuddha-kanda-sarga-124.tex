\sect{चतुर्विंशत्यधिकशततमः सर्गः — पुष्पकोपस्थापनम्}

\twolineshloka
{तां रात्रिमुषितं रामं सुखोदितमरिंदमम्}
{अब्रवीत् प्राञ्जलिर्वाक्यं जयं पृष्ट्वा विभीषणः} %6-124-1

\twolineshloka
{स्नानानि चाङ्गरागाणि वस्त्राण्याभरणानि च}
{चन्दनानि च माल्यानि दिव्यानि विविधानि च} %6-124-2

\twolineshloka
{अलंकारविदश्चैता नार्यः पद्मनिभेक्षणाः}
{उपस्थितास्त्वां विधिवत् स्नापयिष्यन्ति राघव} %6-124-3

\twolineshloka
{एवमुक्तस्तु काकुत्स्थः प्रत्युवाच विभीषणम्}
{हरीन् सुग्रीवमुख्यांस्त्वं स्नानेनोपनिमन्त्रय} %6-124-4

\twolineshloka
{स तु ताम्यति धर्मात्मा मम हेतोः सुखोचितः}
{सुकुमारो महाबाहुर्भरतः सत्यसंश्रयः} %6-124-5

\twolineshloka
{तं विना कैकयीपुत्रं भरतं धर्मचारिणम्}
{न मे स्नानं बहु मतं वस्त्राण्याभरणानि च} %6-124-6

\twolineshloka
{एतत् पश्य यथा क्षिप्रं प्रतिगच्छाम तां पुरीम्}
{अयोध्यां गच्छतो ह्येष पन्थाः परमदुर्गमः} %6-124-7

\twolineshloka
{एवमुक्तस्तु काकुत्स्थं प्रत्युवाच विभीषणः}
{अह्ना त्वां प्रापयिष्यामि तां पुरीं पार्थिवात्मज} %6-124-8

\twolineshloka
{पुष्पकं नाम भद्रं ते विमानं सूर्यसंनिभम्}
{मम भ्रातुः कुबेरस्य रावणेन बलीयसा} %6-124-9

\twolineshloka
{हृतं निर्जित्य संग्रामे कामगं दिव्यमुत्तमम्}
{त्वदर्थं पालितं चेदं तिष्ठत्यतुलविक्रम} %6-124-10

\twolineshloka
{तदिदं मेघसंकाशं विमानमिह तिष्ठति}
{येन यास्यसि यानेन त्वमयोध्यां गतज्वरः} %6-124-11

\twolineshloka
{अहं ते यद्यनुग्राह्यो यदि स्मरसि मे गुणान्}
{वस तावदिह प्राज्ञ यद्यस्ति मयि सौहृदम्} %6-124-12

\twolineshloka
{लक्ष्मणेन सह भ्रात्रा वैदेह्या भार्यया सह}
{अर्चितः सर्वकामैस्त्वं ततो राम गमिष्यसि} %6-124-13

\twolineshloka
{प्रीतियुक्तस्य विहितां ससैन्यः ससुहृद्गणः}
{सत्क्रियां राम मे तावद् गृहाण त्वं मयोद्यताम्} %6-124-14

\twolineshloka
{प्रणयाद् बहुमानाच्च सौहार्देन च राघव}
{प्रसादयामि प्रेष्योऽहं न खल्वाज्ञापयामि ते} %6-124-15

\twolineshloka
{एवमुक्तस्ततो रामः प्रत्युवाच विभीषणम्}
{रक्षसां वानराणां च सर्वेषामेव शृण्वताम्} %6-124-16

\twolineshloka
{पूजितोऽस्मि त्वया वीर साचिव्येन परेण च}
{सर्वात्मना च चेष्टाभिः सौहार्देन परेण च} %6-124-17

\twolineshloka
{न खल्वेतन्न कुर्यां ते वचनं राक्षसेश्वर}
{तं तु मे भ्रातरं द्रष्टुं भरतं त्वरते मनः} %6-124-18

\twolineshloka
{मां निवर्तयितुं योऽसौ चित्रकूटमुपागतः}
{शिरसा याचतो यस्य वचनं न कृतं मया} %6-124-19

\twolineshloka
{कौसल्यां च सुमित्रां च कैकेयीं च यशस्विनीम्}
{गुहं च सुहृदं चैव पौराञ्जानपदैः सह} %6-124-20

\twolineshloka
{अनुजानीहि मां सौम्य पूजितोऽस्मि विभीषण}
{मन्युर्न खलु कर्तव्यः सखे त्वां चानुमानये} %6-124-21

\twolineshloka
{उपस्थापय मे शीघ्रं विमानं राक्षसेश्वर}
{कृतकार्यस्य मे वासः कथं स्यादिह सम्मतः} %6-124-22

\twolineshloka
{एवमुक्तस्तु रामेण राक्षसेन्द्रो विभीषणः}
{विमानं सूर्यसंकाशमाजुहाव त्वरान्वितः} %6-124-23

\twolineshloka
{ततः काञ्चनचित्राङ्गं वैदूर्यमणिवेदिकम्}
{कूटागारैः परिक्षिप्तं सर्वतो रजतप्रभम्} %6-124-24

\twolineshloka
{पाण्डुराभिः पताकाभिर्ध्वजैश्च समलंकृतम्}
{शोभितं काञ्चनैर्हर्म्यैर्हेमपद्मविभूषितैः} %6-124-25

\twolineshloka
{प्रकीर्णं किङ्किणीजालैर्मुक्तामणिगवाक्षकम्}
{घण्टाजालैः परिक्षिप्तं सर्वतो मधुरस्वनम्} %6-124-26

\twolineshloka
{तं मेरुशिखराकारं निर्मितं विश्वकर्मणा}
{बृहद्भिर्भूषितं हर्म्यैर्मुक्तारजतशोभितैः} %6-124-27

\twolineshloka
{तलैः स्फटिकचित्राङ्गैर्वैदूर्यैश्च वरासनैः}
{महार्हास्तरणोपेतैरुपपन्नं महाधनैः} %6-124-28

\twolineshloka
{उपस्थितमनाधृष्यं तद् विमानं मनोजवम्}
{निवेदयित्वा रामाय तस्थौ तत्र विभीषणः} %6-124-29

\twolineshloka
{तत् पुष्पकं कामगमं विमानमुपस्थितं भूधरसंनिकाशम्}
{दृष्ट्वा तदा विस्मयमाजगाम रामः ससौमित्रिरुदारसत्त्वः} %6-124-30


॥इत्यार्षे श्रीमद्रामायणे वाल्मीकीये आदिकाव्ये युद्धकाण्डे पुष्पकोपस्थापनम् नाम चतुर्विंशत्यधिकशततमः सर्गः ॥६-१२४॥
