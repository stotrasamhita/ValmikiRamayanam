\sect{नवाधिकशततमः सर्गः — रावणध्वजोन्मथनम्}

\twolineshloka
{ततः प्रवृत्तं सुक्रूरं रामरावणयोस्तदा}
{सुमहद् द्वैरथं युद्धं सर्वलोकभयावहम्} %6-109-1

\twolineshloka
{ततो राक्षससैन्यं च हरीणां च महद्बलम्}
{प्रगृहीतप्रहरणं निश्चेष्टं समवर्तत} %6-109-2

\twolineshloka
{सम्प्रयुद्धौ तु तौ दृष्ट्वा बलवन्नरराक्षसौ}
{व्याक्षिप्तहृदयाः सर्वे परं विस्मयमागताः} %6-109-3

\twolineshloka
{नानाप्रहरणैर्व्यग्रैर्भुजैर्विस्मितबुद्धयः}
{तस्थुः प्रेक्ष्य च संग्रामं नाभिजग्मुः परस्परम्} %6-109-4

\twolineshloka
{रक्षसां रावणं चापि वानराणां च राघवम्}
{पश्यतां विस्मिताक्षाणां सैन्यं चित्रमिवाबभौ} %6-109-5

\twolineshloka
{तौ तु तत्र निमित्तानि दृष्ट्वा राघवरावणौ}
{कृतबुद्धी स्थिरामर्षौ युयुधाते ह्यभीतवत्} %6-109-6

\twolineshloka
{जेतव्यमिति काकुत्स्थो मर्तव्यमिति रावणः}
{धृतौ स्ववीर्यसर्वस्वं युद्धेऽदर्शयतां तदा} %6-109-7

\twolineshloka
{ततः क्रोधाद् दशग्रीवः शरान् संधाय वीर्यवान्}
{मुमोच ध्वजमुद्दिश्य राघवस्य रथे स्थितम्} %6-109-8

\twolineshloka
{ते शरास्तमनासाद्य पुरंदररथध्वजम्}
{रथशक्तिं परामृश्य निपेतुर्धरणीतले} %6-109-9

\twolineshloka
{ततो रामोऽपि संक्रुद्धश्चापमाकृष्य वीर्यवान्}
{कृतप्रतिकृतं कर्तुं मनसा सम्प्रचक्रमे} %6-109-10

\twolineshloka
{रावणध्वजमुद्दिश्य मुमोच निशितं शरम्}
{महासर्पमिवासह्यं ज्वलन्तं स्वेन तेजसा} %6-109-11

\twolineshloka
{रामश्चिक्षेप तेजस्वी केतुमुद्दिश्य सायकम्}
{जगाम स महीं छित्त्वा दशग्रीवध्वजं शरः} %6-109-12

\twolineshloka
{स निकृत्तोऽपतद् भूमौ रावणस्यन्दनध्वजः}
{ध्वजस्योन्मथनं दृष्ट्वा रावणः स महाबलः} %6-109-13

\twolineshloka
{सम्प्रदीप्तोऽभवत् क्रोधादमर्षात् प्रदहन्निव}
{स रोषवशमापन्नः शरवर्षं ववर्ष ह} %6-109-14

\twolineshloka
{रामस्य तुरगान् दीप्तैः शरैर्विव्याध रावणः}
{ते दिव्या हरयस्तत्र नास्खलन्नापि बभ्रमुः} %6-109-15

\twolineshloka
{बभूवुः स्वस्थहृदयाः पद्मनालैरिवाहताः}
{तेषामसम्भ्रमं दृष्ट्वा वाजिनां रावणस्तदा} %6-109-16

\twolineshloka
{भूय एव सुसंक्रुद्धः शरवर्षं मुमोच ह}
{गदाश्च परिघांश्चैव चक्राणि मुसलानि च} %6-109-17

\threelineshloka
{गिरिशृङ्गाणि वृक्षांश्च तथा शूलपरश्वधान्}
{मायाविहितमेतत् तु शस्त्रवर्षमपातयत्}
{सहस्रशस्तदा बाणानश्रान्तहृदयोद्यमः} %6-109-18

\twolineshloka
{तुमुलं त्रासजननं भीमं भीमप्रतिस्वनम्}
{तद् वर्षमभवद् युद्धे नैकशस्त्रमयं महत्} %6-109-19

\twolineshloka
{विमुच्य राघवरथं समन्ताद् वानरे बले}
{सायकैरन्तरिक्षं च चकार सुनिरन्तरम्} %6-109-20

\twolineshloka
{मुमोच च दशग्रीवो निःसङ्गेनान्तरात्मना}
{व्यायच्छमानं तं दृष्ट्वा तत्परं रावणं रणे} %6-109-21

\twolineshloka
{प्रहसन्निव काकुत्स्थः संदधे निशितान् शरान्}
{स मुमोच ततो बाणान् शतशोऽथ सहस्रशः} %6-109-22

\twolineshloka
{तान् दृष्ट्वा रावणश्चक्रे स्वशरैः खं निरन्तरम्}
{ताभ्यां नियुक्तेन तदा शरवर्षेण भास्वता} %6-109-23

\twolineshloka
{शरबद्धमिवाभाति द्वितीयं भास्वदम्बरम्}
{नानिमित्तोऽभवद् बाणो नानिर्भेत्ता न निष्फलः} %6-109-24

\twolineshloka
{अन्योन्यमभिसंहत्य निपेतुर्धरणीतले}
{तथा विसृजतोर्बाणान् रामरावणयोर्मृधे} %6-109-25

\twolineshloka
{प्रायुध्येतामविच्छिन्नमस्यन्तौ सव्यदक्षिणम्}
{चक्रतुश्च शरैर्घोरैर्निरुच्छ्वासमिवाम्बरम्} %6-109-26

\twolineshloka
{रावणस्य हयान् रामो हयान् रामस्य रावणः}
{जघ्नतुस्तौ तदान्योन्यं कृतानुकृतकारिणौ} %6-109-27

\twolineshloka
{एवं तु तौ सुसंक्रुद्धौ चक्रतुर्युद्धमुत्तमम्}
{मुहूर्तमभवद् युद्धं तुमुलं रोमहर्षणम्} %6-109-28


॥इत्यार्षे श्रीमद्रामायणे वाल्मीकीये आदिकाव्ये युद्धकाण्डे रावणध्वजोन्मथनम् नाम नवाधिकशततमः सर्गः ॥६-१०९॥
