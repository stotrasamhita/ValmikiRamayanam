\sect{षड्त्रिंशः सर्गः — पुरद्वाररक्षा}

\twolineshloka
{तत् तु माल्यवतो वाक्यं हितमुक्तं दशाननः}
{न मर्षयति दुष्टात्मा कालस्य वशमागतः} %6-36-1

\twolineshloka
{स बद्ध्वा भ्रुकुटिं वक्त्रे क्रोधस्य वशमागतः}
{अमर्षात् परिवृत्ताक्षो माल्यवन्तमथाब्रवीत्} %6-36-2

\twolineshloka
{हितबुद्ध्या यदहितं वचः परुषमुच्यते}
{परपक्षं प्रविश्यैव नैतच्छ्रोत्रगतं मम} %6-36-3

\twolineshloka
{मानुषं कृपणं राममेकं शाखामृगाश्रयम्}
{समर्थं मन्यसे केन त्यक्तं पित्रा वनाश्रयम्} %6-36-4

\twolineshloka
{रक्षसामीश्वरं मां च देवानां च भयंकरम्}
{हीनं मां मन्यसे केन अहीनं सर्वविक्रमैः} %6-36-5

\twolineshloka
{वीरद्वेषेण वा शङ्के पक्षपातेन वा रिपोः}
{त्वयाहं परुषाण्युक्तो परप्रोत्साहनेन वा} %6-36-6

\twolineshloka
{प्रभवन्तं पदस्थं हि परुषं कोऽभिभाषते}
{पण्डितः शास्त्रतत्त्वज्ञो विना प्रोत्साहनेन वा} %6-36-7

\twolineshloka
{आनीय च वनात् सीतां पद्महीनामिव श्रियम्}
{किमर्थं प्रतिदास्यामि राघवस्य भयादहम्} %6-36-8

\twolineshloka
{वृतं वानरकोटीभिः ससुग्रीवं सलक्ष्मणम्}
{पश्य कैश्चिदहोभिश्च राघवं निहतं मया} %6-36-9

\twolineshloka
{द्वन्द्वे यस्य न तिष्ठन्ति दैवतान्यपि संयुगे}
{स कस्माद् रावणो युद्धे भयमाहारयिष्यति} %6-36-10

\twolineshloka
{द्विधा भज्येयमप्येवं न नमेयं तु कस्यचित्}
{एष मे सहजो दोषः स्वभावो दुरतिक्रमः} %6-36-11

\twolineshloka
{यदि तावत् समुद्रे तु सेतुर्बद्धो यदृच्छया}
{रामेण विस्मयः कोऽत्र येन ते भयमागतम्} %6-36-12

\twolineshloka
{स तु तीर्त्वार्णवं रामः सह वानरसेनया}
{प्रतिजानामि ते सत्यं न जीवन् प्रतियास्यति} %6-36-13

\twolineshloka
{एवं ब्रुवाणं संरब्धं रुष्टं विज्ञाय रावणम्}
{व्रीडितो माल्यवान् वाक्यं नोत्तरं प्रत्यपद्यत} %6-36-14

\twolineshloka
{जयाशिषा तु राजानं वर्धयित्वा यथोचितम्}
{माल्यवानभ्यनुज्ञातो जगाम स्वं निवेशनम्} %6-36-15

\twolineshloka
{रावणस्तु सहामात्यो मन्त्रयित्वा विमृश्य च}
{लङ्कायास्तु तदा गुप्तिं कारयामास राक्षसः} %6-36-16

\twolineshloka
{व्यादिदेश च पूर्वस्यां प्रहस्तं द्वारि राक्षसम्}
{दक्षिणस्यां महावीर्यौ महापार्श्वमहोदरौ} %6-36-17

\twolineshloka
{पश्चिमायामथ द्वारि पुत्रमिन्द्रजितं तदा}
{व्यादिदेश महामायं राक्षसैर्बहुभिर्वृतम्} %6-36-18

\twolineshloka
{उत्तरस्यां पुरद्वारि व्यादिश्य शुकसारणौ}
{स्वयं चात्र गमिष्यामि मन्त्रिणस्तानुवाच ह} %6-36-19

\twolineshloka
{राक्षसं तु विरूपाक्षं महावीर्यपराक्रमम्}
{मध्यमेऽस्थापयद् गुल्मे बहुभिः सह राक्षसैः} %6-36-20

\twolineshloka
{एवं विधानं लङ्कायां कृत्वा राक्षसपुंगवः}
{कृतकृत्यमिवात्मानं मन्यते कालचोदितः} %6-36-21

\twolineshloka
{विसर्जयामास ततः स मन्त्रिणो विधानमाज्ञाप्य पुरस्य पुष्कलम्}
{जयाशिषा मन्त्रिगणेन पूजितो विवेश सोऽन्तःपुरमृद्धिमन्महत्} %6-36-22


॥इत्यार्षे श्रीमद्रामायणे वाल्मीकीये आदिकाव्ये युद्धकाण्डे पुरद्वाररक्षा नाम षड्त्रिंशः सर्गः ॥६-३६॥
