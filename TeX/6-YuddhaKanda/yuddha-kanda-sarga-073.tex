\sect{त्रिसप्ततितमः सर्गः — इन्द्रजिन्मायायुद्धम्}

\twolineshloka
{ततो हतान् राक्षसपुङ्गवांस्तान् देवान्तकादित्रिशिरोऽतिकायान्}
{रक्षोगणास्तत्र हतावशिष्टास्ते रावणाय त्वरिताः शशंसुः} %6-73-1

\twolineshloka
{ततो हतांस्तान् सहसा निशम्य राजा महाबाष्पपरिप्लुताक्षः}
{पुत्रक्षयं भ्रातृवधं च घोरं विचिन्त्य राजा विपुलं प्रदध्यौ} %6-73-2

\twolineshloka
{ततस्तु राजानमुदीक्ष्य दीनं शोकार्णवे सम्परिपुप्लुवानम्}
{रथर्षभो राक्षसराजसूनुस्तमिन्द्रजिद् वाक्यमिदं बभाषे} %6-73-3

\twolineshloka
{न तात मोहं परिगन्तुमर्हसे यत्रेन्द्रजिज्जीवति नैर्ऋतेश}
{नेन्द्रारिबाणाभिहतो हि कश्चित् प्राणान् समर्थः समरेऽभिपातुम्} %6-73-4

\twolineshloka
{पश्याद्य रामं सह लक्ष्मणेन मद्बाणनिर्भिन्नविकीर्णदेहम्}
{गतायुषं भूमितले शयानं शितैः शरैराचितसर्वगात्रम्} %6-73-5

\twolineshloka
{इमां प्रतिज्ञां शृणु शक्रशत्रोः सुनिश्चितां पौरुषदैवयुक्ताम्}
{अद्यैव रामं सह लक्ष्मणेन सन्तर्पयिष्यामि शरैरमोघैः} %6-73-6

\twolineshloka
{अद्येन्द्रवैवस्वतविष्णुरुद्रसाध्याश्च वैश्वानरचन्द्रसूर्याः}
{द्रक्ष्यन्ति मे विक्रममप्रमेयं विष्णोरिवोग्रं बलियज्ञवाटे} %6-73-7

\twolineshloka
{स एवमुक्त्वा त्रिदशेन्द्रशत्रुरापृच्छ्य राजानमदीनसत्त्वः}
{समारुरोहानिलतुल्यवेगं रथं खरश्रेष्ठसमाधियुक्तम्} %6-73-8

\twolineshloka
{समास्थाय महातेजा रथं हरिरथोपमम्}
{जगाम सहसा तत्र यत्र युद्धमरिन्दमः} %6-73-9

\twolineshloka
{तं प्रस्थितं महात्मानमनुजग्मुर्महाबलाः}
{संहर्षमाणा बहवो धनुःप्रवरपाणयः} %6-73-10

\twolineshloka
{गजस्कन्धगताः केचित् केचित् परमवाजिभिः}
{व्याघ्रवृश्चिकमार्जारखरोष्ट्रैश्च भुजङ्गमैः} %6-73-11

\twolineshloka
{वराहैः श्वापदैः सिंहैर्जम्बुकैः पर्वतोपमैः}
{काकहंसमयूरैश्च राक्षसा भीमविक्रमाः} %6-73-12

\twolineshloka
{प्रासपट्टिशनिस्त्रिंशपरश्वधगदाधराः}
{भुशुण्डिमुद्गरायष्टिशतघ्नीपरिघायुधाः} %6-73-13

\twolineshloka
{स शङ्खनिनदैः पूर्णैर्भेरीणां चापि निःस्वनैः}
{जगाम त्रिदशेन्द्रारिराजिं वेगेन वीर्यवान्} %6-73-14

\twolineshloka
{स शङ्खशशिवर्णेन छत्रेण रिपुसूदनः}
{रराज प्रतिपूर्णेन नभश्चन्द्रमसा यथा} %6-73-15

\twolineshloka
{वीज्यमानस्ततो वीरो हैमैर्हेमविभूषणः}
{चारुचामरमुख्यैश्च मुख्यः सर्वधनुष्मताम्} %6-73-16

\twolineshloka
{स तु दृष्ट्वा विनिर्यान्तं बलेन महता वृतम्}
{राक्षसाधिपतिः श्रीमान् रावणः पुत्रमब्रवीत्} %6-73-17

\twolineshloka
{त्वमप्रतिरथः पुत्र त्वया वै वासवो जितः}
{किं पुनर्मानुषं धृष्यं निहनिष्यसि राघवम्} %6-73-18

\twolineshloka
{तथोक्तो राक्षसेन्द्रेण प्रत्यगृह्णान्महाशिषः}
{ततस्त्विन्द्रजिता लङ्का सूर्यप्रतिमतेजसा} %6-73-19

\twolineshloka
{रराजाप्रतिवीर्येण द्यौरिवार्केण भास्वता}
{स सम्प्राप्य महातेजा युद्धभूमिमरिन्दमः} %6-73-20

\twolineshloka
{स्थापयामास रक्षांसि रथं प्रति समन्ततः}
{ततस्तु हुतभोक्तारं हुतभुक्सदृशप्रभः} %6-73-21

\twolineshloka
{जुहुवे राक्षसश्रेष्ठो विधिवन्मन्त्रसत्तमैः}
{स हविर्लाजसत्कारैर्माल्यगन्धपुरस्कृतैः} %6-73-22

\twolineshloka
{जुहुवे पावकं तत्र राक्षसेन्द्रः प्रतापवान्}
{शस्त्राणि शरपत्राणि समिधोऽथ बिभीतकाः} %6-73-23

\twolineshloka
{लोहितानि च वासांसि स्रुवं कार्ष्णायसं तथा}
{स तत्राग्निं समास्तीर्य शरपत्रैः सतोमरैः} %6-73-24

\twolineshloka
{छागस्य कृष्णवर्णस्य गलं जग्राह जीवतः}
{सकृदेव समिद्धस्य विधूमस्य महार्चिषः} %6-73-25

\twolineshloka
{बभूवुस्तानि लिङ्गानि विजयं यान्यदर्शयन्}
{प्रदक्षिणावर्तशिखस्तप्तकाञ्चनसन्निभः} %6-73-26

\twolineshloka
{हविस्तत् प्रतिजग्राह पावकः स्वयमुत्थितः}
{सोऽस्त्रमाहारयामास ब्राह्ममस्त्रविशारदः} %6-73-27

\threelineshloka
{धनुश्चात्मरथं चैव सर्वं तत्राभ्यमन्त्रयत्}
{तस्मिन्नाहूयमानेऽस्त्रे हूयमाने च पावके}
{सार्कग्रहेन्दुनक्षत्रं वितत्रास नभस्थलम्} %6-73-28

\twolineshloka
{स पावकं पावकदीप्ततेजा हुत्वा महेन्द्रप्रतिमप्रभावः}
{सचापबाणासिरथाश्वसूतः खेऽन्तर्दधेऽऽत्मानमचिन्त्यवीर्यः} %6-73-29

\twolineshloka
{ततो हयरथाकीर्णं पताकाध्वजशोभितम्}
{निर्ययौ राक्षसबलं नर्दमानं युयुत्सया} %6-73-30

\twolineshloka
{ते शरैर्बहुभिश्चित्रैस्तीक्ष्णवेगैरलङ्कृतैः}
{तोमरैरङ्कुशैश्चापि वानराञ्जघ्नुराहवे} %6-73-31

\twolineshloka
{रावणिस्तु सुसङ्क्रुद्धस्तान् निरीक्ष्य निशाचरान्}
{हृष्टा भवन्तो युध्यन्तु वानराणां जिघांसया} %6-73-32

\twolineshloka
{ततस्ते राक्षसाः सर्वे गर्जन्तो जयकाङ्क्षिणः}
{अभ्यवर्षंस्ततो घोरं वानरान् शरवृष्टिभिः} %6-73-33

\twolineshloka
{स तु नालीकनाराचैर्गदाभिर्मुसलैरपि}
{रक्षोभिः संवृतः सङ्ख्ये वानरान् विचकर्ष ह} %6-73-34

\twolineshloka
{ते वध्यमानाः समरे वानराः पादपायुधाः}
{अभ्यवर्षन्त सहसा रावणिं शैलपादपैः} %6-73-35

\twolineshloka
{इन्द्रजित् तु तदा क्रुद्धो महातेजा महाबलः}
{वानराणां शरीराणि व्यधमद् रावणात्मजः} %6-73-36

\twolineshloka
{शरेणैकेन च हरीन् नव पञ्च च सप्त च}
{बिभेद समरे क्रुद्धो राक्षसान् सम्प्रहर्षयन्} %6-73-37

\twolineshloka
{स शरैः सूर्यसङ्काशैः शातकुम्भविभूषणैः}
{वानरान् समरे वीरः प्रममाथ सुदुर्जयः} %6-73-38

\twolineshloka
{ते भिन्नगात्राः समरे वानराः शरपीडिताः}
{पेतुर्मथितसङ्कल्पाः सुरैरिव महासुराः} %6-73-39

\twolineshloka
{ते तपन्तमिवादित्यं घोरैर्बाणगभस्तिभिः}
{अभ्यधावन्त सङ्क्रुद्धाः संयुगे वानरर्षभाः} %6-73-40

\twolineshloka
{ततस्तु वानराः सर्वे भिन्नदेहा विचेतसः}
{व्यथिता विद्रवन्ति स्म रुधिरेण समुक्षिताः} %6-73-41

\twolineshloka
{रामस्यार्थे पराक्रम्य वानरास्त्यक्तजीविताः}
{नर्दन्तस्तेऽनिवृत्तास्तु समरे सशिलायुधाः} %6-73-42

\twolineshloka
{ते द्रुमैः पर्वताग्रैश्च शिलाभिश्च प्लवङ्गमाः}
{अभ्यवर्षन्त समरे रावणिं समवस्थिताः} %6-73-43

\twolineshloka
{तं द्रुमाणां शिलानां च वर्षं प्राणहरं महत्}
{व्यपोहत महातेजा रावणिः समितिञ्जयः} %6-73-44

\twolineshloka
{ततः पावकसङ्काशैः शरैराशीविषोपमैः}
{वानराणामनीकानि बिभेद समरे प्रभुः} %6-73-45

\twolineshloka
{अष्टादशशरैस्तीक्ष्णैः स विद्ध्वा गन्धमादनम्}
{विव्याध नवभिश्चैव नलं दूरादवस्थितम्} %6-73-46

\twolineshloka
{सप्तभिस्तु महावीर्यो मैन्दं मर्मविदारणैः}
{पञ्चभिर्विशिखैश्चैव गजं विव्याध संयुगे} %6-73-47

\twolineshloka
{जाम्बवन्तं तु दशभिर्नीलं त्रिंशद्भिरेव च}
{सुग्रीवमृषभं चैव सोऽङ्गदं द्विविदं तथा} %6-73-48

\twolineshloka
{घोरैर्दत्तवरैस्तीक्ष्णैर्निष्प्राणानकरोत् तदा}
{अन्यानपि तथा मुख्यान् वानरान् बहुभिः शरैः} %6-73-49

\twolineshloka
{अर्दयामास सङ्क्रुद्धः कालाग्निरिव मूर्च्छितः}
{स शरैः सूर्यसङ्काशैः सुमुक्तैः शीघ्रगामिभिः} %6-73-50

\twolineshloka
{वानराणामनीकानि निर्ममन्थ महारणे}
{आकुलां वानरीं सेनां शरजालेन पीडिताम्} %6-73-51

\twolineshloka
{हृष्टः स परया प्रीत्या ददर्श क्षतजोक्षिताम्}
{पुनरेव महातेजा राक्षसेन्द्रात्मजो बली} %6-73-52

\twolineshloka
{संसृज्य बाणवर्षं च शस्त्रवर्षं च दारुणम्}
{ममर्द वानरानीकं परितस्त्विन्द्रजिद् बली} %6-73-53

\twolineshloka
{स्वसैन्यमुत्सृज्य समेत्य तूर्णं महाहवे वानरवाहिनीषु}
{अदृश्यमानः शरजालमुग्रं ववर्ष नीलाम्बुधरो यथाम्बु} %6-73-54

\twolineshloka
{ते शक्रजिद्बाणविशीर्णदेहा मायाहता विस्वरमुन्नदन्तः}
{रणे निपेतुर्हरयोऽद्रिकल्पा यथेन्द्रवज्राभिहता नगेन्द्राः} %6-73-55

\twolineshloka
{ते केवलं सन्ददृशुः शिताग्रान् बाणान् रणे वानरवाहिनीषु}
{मायाविगूढं च सुरेन्द्रशत्रुं न चात्र तं राक्षसमप्यपश्यन्} %6-73-56

\twolineshloka
{ततः स रक्षोधिपतिर्महात्मा सर्वा दिशो बाणगणैः शिताग्रैः}
{प्रच्छादयामास रविप्रकाशैर्विदारयामास च वानरेन्द्रान्} %6-73-57

\twolineshloka
{स शूलनिस्त्रिंशपरश्वधानि व्याविद्धदीप्तानलसप्रभाणि}
{सविस्फुलिङ्गोज्ज्वलपावकानि ववर्ष तीव्रं प्लवगेन्द्रसैन्ये} %6-73-58

\twolineshloka
{ततो ज्वलनसङ्काशैर्बाणैर्वानरयूथपाः}
{ताडिताः शक्रजिद्बाणैः प्रफुल्ला इव किंशुकाः} %6-73-59

\twolineshloka
{तेऽन्योन्यमभिसर्पन्तो निनदन्तश्च विस्वरम्}
{राक्षसेन्द्रास्त्रनिर्भिन्ना निपेतुर्वानरर्षभाः} %6-73-60

\twolineshloka
{उदीक्षमाणा गगनं केचिन्नेत्रेषु ताडिताः}
{शरैर्विविशुरन्योन्यं पेतुश्च जगतीतले} %6-73-61

\twolineshloka
{हनूमन्तं च सुग्रीवमङ्गदं गन्धमादनम्}
{जाम्बवन्तं सुषेणं च वेगदर्शिनमेव च} %6-73-62

\twolineshloka
{मैन्दं च द्विविदं नीलं गवाक्षं गवयं तथा}
{केसरिं हरिलोमानं विद्युद्दंष्ट्रं च वानरम्} %6-73-63

\twolineshloka
{सूर्याननं ज्योतिर्मुखं तथा दधिमुखं हरिम्}
{पावकाक्षं नलं चैव कुमुदं चैव वानरम्} %6-73-64

\twolineshloka
{प्रासैः शूलैः शितैर्बाणैरिन्द्रजिन्मन्त्रसंहितैः}
{विव्याध हरिशार्दूलान् सर्वांस्तान् राक्षसोत्तमः} %6-73-65

\twolineshloka
{स वै गदाभिर्हरियूथमुख्यान् निर्भिद्य बाणैस्तपनीयवर्णैः}
{ववर्ष रामं शरवृष्टिजालैः सलक्ष्मणं भास्कररश्मिकल्पैः} %6-73-66

\twolineshloka
{स बाणवर्षैरभिवृष्यमाणो धारानिपातानिव तानचिन्त्य}
{समीक्षमाणः परमाद्भुतश्रीरामस्तदा लक्ष्मणमित्युवाच} %6-73-67

\twolineshloka
{असौ पुनर्लक्ष्मण राक्षसेन्द्रो ब्रह्मास्त्रमाश्रित्य सुरेन्द्रशत्रुः}
{निपातयित्वा हरिसैन्यमस्मान्शितैः शरैरर्दयति प्रसक्तम्} %6-73-68

\twolineshloka
{स्वयम्भुवा दत्तवरो महात्मा समाहितोऽन्तर्हितभीमकायः}
{कथं नु शक्यो युधि नष्टदेहो निहन्तुमद्येन्द्रजिदुद्यतास्त्रः} %6-73-69

\twolineshloka
{मन्ये स्वयम्भूर्भगवानचिन्त्यस्तस्यैतदस्त्रं प्रभवश्च योऽस्य}
{बाणावपातं त्वमिहाद्य धीमन् मया सहाव्यग्रमनाः सहस्व} %6-73-70

\twolineshloka
{प्रच्छादयत्येष हि राक्षसेन्द्रः सर्वा दिशः सायकवृष्टिजालैः}
{एतच्च सर्वं पतिताग्र्यशूरं न भ्राजते वानरराजसैन्यम्} %6-73-71

\twolineshloka
{आवां तु दृष्ट्वा पतितौ विसंज्ञौ निवृत्तयुद्धौ हतहर्षरोषौ}
{ध्रुवं प्रवेक्ष्यत्यमरारिवासमसौ समासाद्य रणाग्र्यलक्ष्मीम्} %6-73-72

\twolineshloka
{ततस्तु ताविन्द्रजितोऽस्त्रजालैर्बभूवतुस्तत्र तदा विशस्तौ}
{स चापि तौ तत्र विषादयित्वा ननाद हर्षाद् युधि राक्षसेन्द्रः} %6-73-73

\threelineshloka
{ततस्तदा वानरसैन्यमेवं रामं च सङ्ख्ये सह लक्ष्मणेन}
{विषादयित्वा सहसा विवेश पुरीं दशग्रीवभुजाभिगुप्ताम्}
{संस्तूयमानः स तु यातुधानैः पित्रे च सर्वं हृषितोऽभ्युवाच} %6-73-74


॥इत्यार्षे श्रीमद्रामायणे वाल्मीकीये आदिकाव्ये युद्धकाण्डे इन्द्रजिन्मायायुद्धम् नाम त्रिसप्ततितमः सर्गः ॥६-७३॥
