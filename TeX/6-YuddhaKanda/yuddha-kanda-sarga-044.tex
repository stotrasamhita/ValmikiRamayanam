\sect{चतुश्चत्वारिंशः सर्गः — निशायुद्धम्}

\twolineshloka
{युध्यतामेव तेषां तु तदा वानररक्षसाम्}
{रविरस्तं गतो रात्रिः प्रवृत्ता प्राणहारिणी} %6-44-1

\twolineshloka
{अन्योन्यं बद्धवैराणां घोराणं जयमिच्छताम्}
{सम्प्रवृत्तं निशायुद्धं तदा वानररक्षसाम्} %6-44-2

\twolineshloka
{राक्षसोऽसीति हरयो वानरोऽसीति राक्षसाः}
{अन्योन्यं समरे जघ्नुस्तस्मिंस्तमसि दारुणे} %6-44-3

\twolineshloka
{हत दारय चैहीति कथं विद्रवसीति च}
{एवं सुतुमुलः शब्दस्तस्मिन् सैन्ये तु शुश्रुवे} %6-44-4

\twolineshloka
{कालाः काञ्चनसंनाहास्तस्मिंस्तमसि राक्षसाः}
{सम्प्रदृश्यन्त शैलेन्द्रा दीप्तौषधिवना इव} %6-44-5

\twolineshloka
{तस्मिंस्तमसि दुष्पारे राक्षसाः क्रोधमूिर्च्छताः}
{परिपेतुर्महावेगा भक्षयन्तः प्लवङ्गमान्} %6-44-6

\twolineshloka
{ते हयान् काञ्चनापीडान् ध्वजांश्चाशीविषोपमान्}
{आप्लुत्य दशनैस्तीक्ष्णैर्भीमकोपा व्यदारयन्} %6-44-7

\twolineshloka
{वानरा बलिनो युद्धेऽक्षोभयन् राक्षसीं चमूम्}
{कुञ्जरान् कुञ्जरारोहान् पताकाध्वजिनो रथान्} %6-44-8

\twolineshloka
{चकर्षुश्च ददंशुश्च दशनैः क्रोधमूिर्च्छताः}
{लक्ष्मणश्चापि रामश्च शरैराशीविषोपमैः} %6-44-9

\twolineshloka
{दृश्यादृश्यानि रक्षांसि प्रवराणि निजघ्नतुः}
{तुरंगखुरविध्वस्तं रथनेमिसमुत्थितम्} %6-44-10

\threelineshloka
{रुरोध कर्णनेत्राणि युध्यतां धरणीरजः}
{वर्तमाने तथा घोरे संग्रामे लोमहर्षणे}
{रुधिरौघा महाघोरा नद्यस्तत्र विसुस्रुवुः} %6-44-11

\twolineshloka
{ततो भेरीमृदङ्गानां पणवानां च निःस्वनः}
{शङ्खनेमिस्वनोन्मिश्रः सम्बभूवाद्भुतोपमः} %6-44-12

\twolineshloka
{हतानां स्तनमानानां राक्षसानां च निःस्वनः}
{शस्तानां वानराणां च सम्बभूवात्र दारुणः} %6-44-13

\twolineshloka
{हतैर्वानरमुख्यैश्च शक्तिशूलपरश्वधैः}
{निहतैः पर्वताकारै राक्षसैः कामरूपिभिः} %6-44-14

\twolineshloka
{शस्त्रपुष्पोपहारा च तत्रासीद् युद्धमेदिनी}
{दुर्ज्ञेया दुर्निवेशा च शोणितास्त्रावकर्दमा} %6-44-15

\twolineshloka
{सा बभूव निशा घोरा हरिराक्षसहारिणी}
{कालरात्रीव भूतानां सर्वेषां दुरतिक्रमा} %6-44-16

\twolineshloka
{ततस्ते राक्षसास्तत्र तस्मिंस्तमसि दारुणे}
{राममेवाभ्यवर्तन्त संहृष्टाः शरवृष्टिभिः} %6-44-17

\twolineshloka
{तेषामापततां शब्दः क्रुद्धानामपि गर्जताम्}
{उद्वर्त इव सप्तानां समुद्राणामभूत् स्वनः} %6-44-18

\twolineshloka
{तेषां रामः शरैः षड्भिः षड् जघान निशाचरान्}
{निमेषान्तरमात्रेण शरैरग्निशिखोपमैः} %6-44-19

\twolineshloka
{यज्ञशत्रुश्च दुर्धर्षो महापार्श्वमहोदरौ}
{वज्रदंष्ट्रो महाकायस्तौ चोभौ शुकसारणौ} %6-44-20

\twolineshloka
{ते तु रामेण बाणौघैः सर्वमर्मसु ताडिताः}
{युद्धादपसृतास्तत्र सावशेषायुषोऽभवन्} %6-44-21

\twolineshloka
{निमेषान्तरमात्रेण घोरैरग्निशिखोपमैः}
{दिशश्चकार विमलाः प्रदिशश्च महारथः} %6-44-22

\twolineshloka
{ये त्वन्ये राक्षसा वीरा रामस्याभिमुखे स्थिताः}
{तेऽपि नष्टाः समासाद्य पतङ्गा इव पावकम्} %6-44-23

\twolineshloka
{सुवर्णपुङ्खैर्विशिखैः सम्पतद्भिः समन्ततः}
{बभूव रजनी चित्रा खद्योतैरिव शारदी} %6-44-24

\twolineshloka
{राक्षसानां च निनदैर्भेरीणां चैव निःस्वनैः}
{सा बभूव निशा घोरा भूयो घोरतराभवत्} %6-44-25

\twolineshloka
{तेन शब्देन महता प्रवृद्धेन समन्ततः}
{त्रिकूटः कंदराकीर्णः प्रव्याहरदिवाचलः} %6-44-26

\twolineshloka
{गोलाङ्गूला महाकायास्तमसा तुल्यवर्चसः}
{सम्परिष्वज्य बाहुभ्यां भक्षयन् रजनीचरान्} %6-44-27

\twolineshloka
{अङ्गदस्तु रणे शत्रून् निहन्तुं समुपस्थितः}
{रावणिं निजघानाशु सारथिं च हयानपि} %6-44-28

\twolineshloka
{इन्द्रजित् तु रथं त्यक्त्वा हताश्वो हतसारथिः}
{अङ्गदेन महाकायस्तत्रैवान्तरधीयत} %6-44-29

\twolineshloka
{तत् कर्म वालिपुत्रस्य सर्वे देवाः सहर्षिभिः}
{तुष्टुवुः पूजनार्हस्य तौ चोभौ रामलक्ष्मणौ} %6-44-30

\twolineshloka
{प्रभावं सर्वभूतानि विदुरिन्द्रजितो युधि}
{ततस्ते तं महात्मानं दृष्ट्वा तुष्टाः प्रधर्षितम्} %6-44-31

\twolineshloka
{ततः प्रहृष्टाः कपयः ससुग्रीवविभीषणाः}
{साधुसाध्विति नेदुश्च दृष्ट्वा शत्रुं पराजितम्} %6-44-32

\twolineshloka
{इन्द्रजित् तु तदानेन निर्जितो भीमकर्मणा}
{संयुगे वालिपुत्रेण क्रोधं चक्रे सुदारुणम्} %6-44-33

\twolineshloka
{सोऽन्तर्धानगतः पापो रावणी रणकर्शितः}
{ब्रह्मदत्तवरो वीरो रावणिः क्रोधमूर्च्छितः} %6-44-34

\twolineshloka
{अदृश्यो निशितान् बाणान् मुमोचाशनिवर्चसः}
{रामं च लक्ष्मणं चैव घोरैर्नागमयैः शरैः} %6-44-35

\twolineshloka
{बिभेद समरे क्रुद्धः सर्वगात्रेषु राक्षसः}
{मायया संवृतस्तत्र मोहयन् राघवौ युधि} %6-44-36

\twolineshloka
{अदृश्यः सर्वभूतानां कूटयोधी निशाचरः}
{बबन्ध शरबन्धेन भ्रातरौ रामलक्ष्मणौ} %6-44-37

\twolineshloka
{तौ तेन पुरुषव्याघ्रौ क्रुद्धेनाशीविषैः शरैः}
{सहसाभिहतौ वीरौ तदा प्रेक्षन्त वानराः} %6-44-38

\twolineshloka
{प्रकाशरूपस्तु यदा न शक्तस्तौ बाधितुं राक्षसराजपुत्रः}
{मायां प्रयोक्तुं समुपाजगाम बबन्ध तौ राजसुतौ दुरात्मा} %6-44-39


॥इत्यार्षे श्रीमद्रामायणे वाल्मीकीये आदिकाव्ये युद्धकाण्डे निशायुद्धम् नाम चतुश्चत्वारिंशः सर्गः ॥६-४४॥
