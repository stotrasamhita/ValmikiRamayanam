\sect{त्रयोदशः सर्गः — महापार्श्ववचोऽभिनन्दनम्}

\twolineshloka
{रावणं क्रुद्धमाज्ञाय महापार्श्वो महाबलः}
{मुहूर्तमनुसञ्चिन्त्य प्राञ्जलिर्वाक्यमब्रवीत्} %6-13-1

\twolineshloka
{यः खल्वपि वनं प्राप्य मृगव्यालनिषेवितम्}
{न पिबेन्मधु सम्प्राप्य स नरो बालिशो भवेत्} %6-13-2

\twolineshloka
{ईश्वरस्येश्वरः कोऽस्ति तव शत्रुनिबर्हण}
{रमस्व सह वैदेह्या शत्रूनाक्रम्य मूर्धसु} %6-13-3

\twolineshloka
{बलात् कुक्कुटवृत्तेन प्रवर्तस्व महाबल}
{आक्रम्याक्रम्य सीतां वै तां भुङ्क्ष्व च रमस्व च} %6-13-4

\twolineshloka
{लब्धकामस्य ते पश्चादागमिष्यति किं भयम्}
{प्राप्तमप्राप्तकालं वा सर्वं प्रतिविधास्यसे} %6-13-5

\twolineshloka
{कुम्भकर्णः सहास्माभिरिन्द्रजिच्च महाबलः}
{प्रतिषेधयितुं शक्तौ सवज्रमपि वज्रिणम्} %6-13-6

\twolineshloka
{उपप्रदानं सान्त्वं वा भेदं वा कुशलैः कृतम्}
{समतिक्रम्य दण्डेन सिद्धिमर्थेषु रोचये} %6-13-7

\twolineshloka
{इह प्राप्तान् वयं सर्वाञ्छत्रूंस्तव महाबल}
{वशे शस्त्रप्रतापेन करिष्यामो न संशयः} %6-13-8

\twolineshloka
{एवमुक्तस्तदा राजा महापार्श्वेन रावणः}
{तस्य सम्पूजयन् वाक्यमिदं वचनमब्रवीत्} %6-13-9

\twolineshloka
{महापार्श्व निबोध त्वं रहस्यं किञ्चिदात्मनः}
{चिरवृत्तं तदाख्यास्ये यदवाप्तं पुरा मया} %6-13-10

\twolineshloka
{पितामहस्य भवनं गच्छन्तीं पुञ्जिकस्थलाम्}
{चञ्चूर्यमाणामद्राक्षमाकाशेऽग्निशिखामिव} %6-13-11

\twolineshloka
{सा प्रसह्य मया भुक्ता कृता विवसना ततः}
{स्वयम्भूभवनं प्राप्ता लोलिता नलिनी यथा} %6-13-12

\twolineshloka
{तच्च तस्य तथा मन्ये ज्ञातमासीन्महात्मनः}
{अथ सङ्कुपितो वेधा मामिदं वाक्यमब्रवीत्} %6-13-13

\twolineshloka
{अद्यप्रभृति यामन्यां बलान्नारीं गमिष्यसि}
{तदा ते शतधा मूर्धा फलिष्यति न संशयः} %6-13-14

\twolineshloka
{इत्यहं तस्य शापस्य भीतः प्रसभमेव ताम्}
{नारोहये बलात् सीतां वैदेहीं शयने शुभे} %6-13-15

\twolineshloka
{सागरस्येव मे वेगो मारुतस्येव मे गतिः}
{नैतद् दाशरथिर्वेद ह्यासादयति तेन माम्} %6-13-16

\twolineshloka
{को हि सिंहमिवासीनं सुप्तं गिरिगुहाशये}
{क्रुद्धं मृत्युमिवासीनं प्रबोधयितुमिच्छति} %6-13-17

\twolineshloka
{न मत्तो निर्गतान् बाणान् द्विजिह्वान् पन्नगानिव}
{रामः पश्यति सङ्ग्रामे तेन मामभिगच्छति} %6-13-18

\twolineshloka
{क्षिप्रं वज्रसमैर्बाणैः शतधा कार्मुकच्युतैः}
{राममादीपयिष्यामि उल्काभिरिव कुञ्जरम्} %6-13-19

\twolineshloka
{तच्चास्य बलमादास्ये बलेन महता वृतः}
{उदितः सविता काले नक्षत्राणां प्रभामिव} %6-13-20

\twolineshloka
{न वासवेनापि सहस्रचक्षुषा युधास्मि शक्यो वरुणेन वा पुनः}
{मया त्वियं बाहुबलेन निर्जिता पुरा पुरी वैश्रवणेन पालिता} %6-13-21


॥इत्यार्षे श्रीमद्रामायणे वाल्मीकीये आदिकाव्ये युद्धकाण्डे महापार्श्ववचोऽभिनन्दनम् नाम त्रयोदशः सर्गः ॥६-१३॥
