\sect{चतुर्विंशः सर्गः — रावणप्रतिज्ञा}

\twolineshloka
{सा वीरसमिती राज्ञा विरराज व्यवस्थिता}
{शशिना शुभनक्षत्रा पौर्णमासीव शारदी} %6-24-1

\twolineshloka
{प्रचचाल च वेगेन त्रस्ता चैव वसुंधरा}
{पीड्यमाना बलौघेन तेन सागरवर्चसा} %6-24-2

\twolineshloka
{ततः शुश्रुवुराक्रुष्टं लङ्कायां काननौकसः}
{भेरीमृदङ्गसंघुष्टं तुमुलं लोमहर्षणम्} %6-24-3

\twolineshloka
{बभूवुस्तेन घोषेण संहृष्टा हरियूथपाः}
{अमृष्यमाणास्तद् घोषं विनेदुर्घोषवत्तरम्} %6-24-4

\twolineshloka
{राक्षसास्तत् प्लवंगानां शुश्रुवुस्तेऽपि गर्जितम्}
{नर्दतामिव दृप्तानां मेघानामम्बरे स्वनम्} %6-24-5

\twolineshloka
{दृष्ट्वा दाशरथिर्लङ्कां चित्रध्वजपताकिनीम्}
{जगाम मनसा सीतां दूूयमानेन चेतसा} %6-24-6

\twolineshloka
{अत्र सा मृगशावाक्षी रावणेनोपरुध्यते}
{अभिभूता ग्रहेणेव लोहिताङ्गेन रोहिणी} %6-24-7

\twolineshloka
{दीर्घमुष्णं च निःश्वस्य समुद्वीक्ष्य च लक्ष्मणम्}
{उवाच वचनं वीरस्तत्कालहितमात्मनः} %6-24-8

\twolineshloka
{आलिखन्तीमिवाकाशमुत्थितां पश्य लक्ष्मण}
{मनसेव कृतां लङ्कां नगाग्रे विश्वकर्मणा} %6-24-9

\twolineshloka
{विमानैर्बहुभिर्लङ्का संकीर्णा रचिता पुरा}
{विष्णोः पदमिवाकाशं छादितं पाण्डुभिर्घनैः} %6-24-10

\twolineshloka
{पुष्पितैः शोभिता लङ्का वनैश्चित्ररथोपमैः}
{नानापतगसंघुष्टफलपुष्पोपगैः शुभैः} %6-24-11

\twolineshloka
{पश्य मत्तविहंगानि प्रलीनभ्रमराणि च}
{कोकिलाकुलखण्डानि दोधवीति शिवोऽनिलः} %6-24-12

\twolineshloka
{इति दाशरथी रामो लक्ष्मणं समभाषत}
{बलं च तत्र विभजच्छास्त्रदृष्टेन कर्मणा} %6-24-13

\twolineshloka
{शशास कपिसेनां तां बलादादाय वीर्यवान्}
{अङ्गदः सह नीलेन तिष्ठेदुरसि दुर्जयः} %6-24-14

\twolineshloka
{तिष्ठेद् वानरवाहिन्या वानरौघसमावृतः}
{आश्रितो दक्षिणं पार्श्वमृषभो नाम वानरः} %6-24-15

\twolineshloka
{गन्धहस्तीव दुर्धर्षस्तरस्वी गन्धमादनः}
{तिष्ठेद् वानरवाहिन्याः सव्यं पार्श्वमधिष्ठितः} %6-24-16

\twolineshloka
{मूर्ध्नि स्थास्याम्यहं यत्तो लक्ष्मणेन समन्वितः}
{जाम्बवांश्च सुषेणश्च वेगदर्शी च वानरः} %6-24-17

\threelineshloka
{ऋक्षमुख्या महात्मानः कुक्षिं रक्षन्तु ते त्रयः}
{जघनं कपिसेनायाः कपिराजोऽभिरक्षतु}
{पश्चार्धमिव लोकस्य प्रचेतास्तेजसा वृतः} %6-24-18

\twolineshloka
{सुविभक्तमहाव्यूहा महावानररक्षिता}
{अनीकिनी सा विबभौ यथा द्यौः साभ्रसम्प्लवा} %6-24-19

\twolineshloka
{प्रगृह्य गिरिशृङ्गाणि महतश्च महीरुहान्}
{आसेदुर्वानरा लङ्कां मिमर्दयिषवो रणे} %6-24-20

\twolineshloka
{शिखरैर्विकिरामैनां लङ्कां मुष्टिभिरेव वा}
{इति स्म दधिरे सर्वे मनांसि हरिपुङ्गवाः} %6-24-21

\twolineshloka
{ततो रामो महातेजाः सुग्रीवमिदमब्रवीत्}
{सुविभक्तानि सैन्यानि शुक एष विमुच्यताम्} %6-24-22

\twolineshloka
{रामस्य तु वचः श्रुत्वा वानरेन्द्रो महाबलः}
{मोचयामास तं दूतं शुकं रामस्य शासनात्} %6-24-23

\twolineshloka
{मोचितो रामवाक्येन वानरैश्च निपीडितः}
{शुकः परमसंत्रस्तो रक्षोधिपमुपागमत्} %6-24-24

\twolineshloka
{रावणः प्रहसन्नेव शुकं वाक्यमुवाच ह}
{किमिमौ ते सितौ पक्षौ लूनपक्षश्च दृश्यसे} %6-24-25

\threelineshloka
{कच्चिन्नानेकचित्तानां तेषां त्वं वशमागतः}
{ततः स भयसंविग्नस्तेन राज्ञाभिचोदितः}
{वचनं प्रत्युवाचेदं राक्षसाधिपमुत्तमम्} %6-24-26

\twolineshloka
{सागरस्योत्तरे तीरेऽब्रुवं ते वचनं तथा}
{यथा संदेशमक्लिष्टं सान्त्वयन् श्लक्ष्णया गिरा} %6-24-27

\twolineshloka
{क्रुद्धैस्तैरहमुत्प्लुत्य दृष्टमात्रः प्लवंगमैः}
{गृहीतोऽस्म्यपि चारब्धो हन्तुं लोप्तुं च मुष्टिभिः} %6-24-28

\twolineshloka
{न ते संभाषितुं शक्याः सम्प्रश्नोऽत्र न विद्यते}
{प्रकृत्या कोपनास्तीक्ष्णा वानरा राक्षसाधिप} %6-24-29

\twolineshloka
{स च हन्ता विराधस्य कबन्धस्य खरस्य च}
{सुग्रीवसहितो रामः सीतायाः पदमागतः} %6-24-30

\twolineshloka
{स कृत्वा सागरे सेतुं तीर्त्वा च लवणोदधिम्}
{एष रक्षांसि निर्धूय धन्वी तिष्ठति राघवः} %6-24-31

\twolineshloka
{ऋक्षवानरसङ्घानामनीकानि सहस्रशः}
{गिरिमेघनिकाशानां छादयन्ति वसुंधराम्} %6-24-32

\twolineshloka
{राक्षसानां बलौघस्य वानरेन्द्रबलस्य च}
{नैतयोर्विद्यते संधिर्देवदानवयोरिव} %6-24-33

\twolineshloka
{पुरा प्राकारमायान्ति क्षिप्रमेकतरं कुरु}
{सीतां चास्मै प्रयच्छाशु युद्धं वापि प्रदीयताम्} %6-24-34

\twolineshloka
{शुकस्य वचनं श्रुत्वा रावणो वाक्यमब्रवीत्}
{रोषसंरक्तनयनो निर्दहन्निव चक्षुषा} %6-24-35

\twolineshloka
{यदि मां प्रति युद्धेरन् देवगन्धर्वदानवाः}
{नैव सीतां प्रदास्यामि सर्वलोकभयादपि} %6-24-36

\twolineshloka
{कदा समभिधावन्त मामका राघवं शराः}
{वसन्ते पुष्पितं मत्ता भ्रमरा इव पादपम्} %6-24-37

\twolineshloka
{कदा शोणितदिग्धाङ्गं दीप्तैः कार्मुकविच्युतैः}
{शरैरादीपयिष्यामि उल्काभिरिव कुञ्जरम्} %6-24-38

\twolineshloka
{तच्चास्य बलमादास्ये बलेन महता वृतः}
{ज्योतिषामिव सर्वेषां प्रभामुद्यन् दिवाकरः} %6-24-39

\twolineshloka
{सागरस्येव मे वेगो मारुतस्येव मे बलम्}
{न च दाशरथिर्वेद तेन मां योद्धुमिच्छति} %6-24-40

\twolineshloka
{न मे तूणीशयान् बाणान् सविषानिव पन्नगान्}
{रामः पश्यति संग्रामे तेन मां योद्धुमिच्छति} %6-24-41

\twolineshloka
{न जानाति पुरा वीर्यं मम युद्धे स राघवः}
{मम चापमयीं वीणां शरकोणैः प्रवादिताम्} %6-24-42

\threelineshloka
{ज्याशब्दतुमुलां घोरामार्तगीतमहास्वनाम्}
{नाराचतलसंनादां नदीमहितवाहिनीम्}
{अवगाह्य महारङ्गं वादयिष्याम्यहं रणे} %6-24-43

\twolineshloka
{न वासवेनापि सहस्रचक्षुषा युद्धेऽस्मि शक्यो वरुणेन वा स्वयम्}
{यमेन वा धर्षयितुं शराग्निना महाहवे वैश्रवणेन वा पुनः} %6-24-44


॥इत्यार्षे श्रीमद्रामायणे वाल्मीकीये आदिकाव्ये युद्धकाण्डे रावणप्रतिज्ञा नाम चतुर्विंशः सर्गः ॥६-२४॥
