\sect{द्वादशः सर्गः — कुम्भकर्णमतिः}

\twolineshloka
{स तां परिषदं कृत्स्नां समीक्ष्य समितिंजयः}
{प्रचोदयामास तदा प्रहस्तं वाहिनीपतिम्} %6-12-1

\twolineshloka
{सेनापते यथा ते स्युः कृतविद्याश्चतुर्विधाः}
{योधा नगररक्षायां तथा व्यादेष्टुमर्हसि} %6-12-2

\twolineshloka
{स प्रहस्तः प्रणीतात्मा चिकीर्षन् राजशासनम्}
{विनिक्षिपद् बलं सर्वं बहिरन्तश्च मन्दिरे} %6-12-3

\twolineshloka
{ततो विनिक्षिप्य बलं सर्वं नगरगुप्तये}
{प्रहस्तः प्रमुखे राज्ञो निषसाद जगाद च} %6-12-4

\twolineshloka
{विहितं बहिरन्तश्च बलं बलवतस्तव}
{कुरुष्वाविमनाः क्षिप्रं यदभिप्रेतमस्ति ते} %6-12-5

\twolineshloka
{प्रहस्तस्य वचः श्रुत्वा राजा राज्यहितैषिणः}
{सुखेप्सुः सुहृदां मध्ये व्याजहार स रावणः} %6-12-6

\twolineshloka
{प्रियाप्रिये सुखे दुःखे लाभालाभे हिताहिते}
{धर्मकामार्थकृच्छ्रेषु यूयमर्हथ वेदितुम्} %6-12-7

\twolineshloka
{सर्वकृत्यानि युष्माभिः समारब्धानि सर्वदा}
{मन्त्रकर्मनियुक्तानि न जातु विफलानि मे} %6-12-8

\twolineshloka
{ससोमग्रहनक्षत्रैर्मरुद्भिरिव वासवः}
{भवद्भिरहमत्यर्थं वृतः श्रियमवाप्नुयाम्} %6-12-9

\twolineshloka
{अहं तु खलु सर्वान् वः समर्थयितुमुद्यतः}
{कुम्भकर्णस्य तु स्वप्नान् नेममर्थमचोदयम्} %6-12-10

\twolineshloka
{अयं हि सुप्तः षण्मासान् कुम्भकर्णो महाबलः}
{सर्वशस्त्रभृतां मुख्यः स इदानीं समुत्थितः} %6-12-11

\twolineshloka
{इयं च दण्डकारण्याद् रामस्य महिषी प्रिया}
{रक्षोभिश्चरितोद्देशादानीता जनकात्मजा} %6-12-12

\twolineshloka
{सा मे न शय्यामारोढुमिच्छत्यलसगामिनी}
{त्रिषु लोकेषु चान्या मे न सीतासदृशी तथा} %6-12-13

\twolineshloka
{तनुमध्या पृथुश्रोणी शरदिन्दुनिभानना}
{हेमबिम्बनिभा सौम्या मायेव मयनिर्मिता} %6-12-14

\twolineshloka
{सुलोहिततलौ श्लक्ष्णौ चरणौ सुप्रतिष्ठितौ}
{दृष्ट्वा ताम्रनखौ तस्या दीप्यते मे शरीरजः} %6-12-15

\twolineshloka
{हुताग्नेरर्चिसंकाशामेनां सौरीमिव प्रभाम्}
{उन्नसं विमलं वल्गु वदनं चारुलोचनम्} %6-12-16

\twolineshloka
{पश्यंस्तदवशस्तस्याः कामस्य वशमेयिवान्}
{क्रोधहर्षसमानेन दुर्वर्णकरणेन च} %6-12-17

\twolineshloka
{शोकसंतापनित्येन कामेन कलुषीकृतः}
{सा तु संवत्सरं कालं मामयाचत भामिनी} %6-12-18

\twolineshloka
{प्रतीक्षमाणा भर्तारं राममायतलोचना}
{तन्मया चारुनेत्रायाः प्रतिज्ञातं वचः शुभम्} %6-12-19

\twolineshloka
{श्रान्तोऽहं सततं कामाद् यातो हय इवाध्वनि}
{कथं सागरमक्षोभ्यं तरिष्यन्ति वनौकसः} %6-12-20

\twolineshloka
{बहुसत्त्वझषाकीर्णं तौ वा दशरथात्मजौ}
{अथवा कपिनैकेन कृतं नः कदनं महत्} %6-12-21

\twolineshloka
{दुर्ज्ञेयाः कार्यगतयो ब्रूत यस्य यथामति}
{मानुषान्नो भयं नास्ति तथापि तु विमृश्यताम्} %6-12-22

\twolineshloka
{तदा देवासुरे युद्धे युष्माभिः सहितोऽजयम्}
{ते मे भवन्तश्च तथा सुग्रीवप्रमुखान् हरीन्} %6-12-23

\twolineshloka
{परे पारे समुद्रस्य पुरस्कृत्य नृपात्मजौ}
{सीतायाः पदवीं प्राप्य सम्प्राप्तौ वरुणालयम्} %6-12-24

\twolineshloka
{अदेया च यथा सीता वध्यौ दशरथात्मजौ}
{भवद्भिर्मन्त्र्यतां मन्त्रः सुनीतं चाभिधीयताम्} %6-12-25

\twolineshloka
{नहि शक्तिं प्रपश्यामि जगत्यन्यस्य कस्यचित्}
{सागरं वानरैस्तीर्त्वा निश्चयेन जयो मम} %6-12-26

\twolineshloka
{तस्य कामपरीतस्य निशम्य परिदेवितम्}
{कुम्भकर्णः प्रचुक्रोध वचनं चेदमब्रवीत्} %6-12-27

\twolineshloka
{यदा तु रामस्य सलक्ष्मणस्य प्रसह्य सीता खलु सा इहाहृता}
{सकृत् समीक्ष्यैव सुनिश्चितं तदा भजेत चित्तं यमुनेव यामुनम्} %6-12-28

\twolineshloka
{सर्वमेतन्महाराज कृतमप्रतिमं तव}
{विधीयेत सहास्माभिरादावेवास्य कर्मणः} %6-12-29

\twolineshloka
{न्यायेन राजकार्याणि यः करोति दशानन}
{न स संतप्यते पश्चान्निश्चितार्थमतिर्नृपः} %6-12-30

\twolineshloka
{अनुपायेन कर्माणि विपरीतानि यानि च}
{क्रियमाणानि दुष्यन्ति हवींष्यप्रयतेष्विव} %6-12-31

\twolineshloka
{यः पश्चात् पूर्वकार्याणि कर्माण्यभिचिकीर्षति}
{पूर्वं चापरकार्याणि स न वेद नयानयौ} %6-12-32

\twolineshloka
{चपलस्य तु कृत्येषु प्रसमीक्ष्याधिकं बलम्}
{छिद्रमन्ये प्रपद्यन्ते क्रौञ्चस्य खमिव द्विजाः} %6-12-33

\twolineshloka
{त्वयेदं महदारब्धं कार्यमप्रतिचिन्तितम्}
{दिष्ट्या त्वां नावधीद् रामो विषमिश्रमिवामिषम्} %6-12-34

\twolineshloka
{तस्मात् त्वया समारब्धं कर्म ह्यप्रतिमं परैः}
{अहं समीकरिष्यामि हत्वा शत्रूंस्तवानघ} %6-12-35

\threelineshloka
{अहमुत्सादयिष्यामि शत्रूंस्तव निशाचर}
{यदि शक्रविवस्वन्तौ यदि पावकमारुतौ}
{तावहं योधयिष्यामि कुबेरवरुणावपि} %6-12-36

\twolineshloka
{गिरिमात्रशरीरस्य महापरिघयोधिनः}
{नर्दतस्तीक्ष्णदंष्ट्रस्य बिभीयाद् वै पुरंदरः} %6-12-37

\twolineshloka
{पुनर्मां स द्वितीयेन शरेण निहनिष्यति}
{ततोऽहं तस्य पास्यामि रुधिरं काममाश्वस} %6-12-38

\twolineshloka
{वधेन वै दाशरथेः सुखावहं जयं तवाहर्तुमहं यतिष्ये}
{हत्वा च रामं सह लक्ष्मणेन खादामि सर्वान् हरियूथमुख्यान्} %6-12-39

\twolineshloka
{रमस्व कामं पिब चाग्र्यवारुणीं कुरुष्व कार्याणि हितानि विज्वरः}
{मया तु रामे गमिते यमक्षयं चिराय सीता वशगा भविष्यति} %6-12-40


॥इत्यार्षे श्रीमद्रामायणे वाल्मीकीये आदिकाव्ये युद्धकाण्डे कुम्भकर्णमतिः नाम द्वादशः सर्गः ॥६-१२॥
