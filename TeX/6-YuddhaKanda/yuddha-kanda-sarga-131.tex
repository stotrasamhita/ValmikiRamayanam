\sect{एकत्रिंशदधिकशततमः सर्गः — श्रीरामपट्टाभिषेकः}

\twolineshloka
{शिरस्यञ्जलिमाधाय कैकेय्यानन्दवर्धनः}
{बभाषे भरतो ज्येष्ठं रामं सत्यपराक्रमम्} %6-131-1

\twolineshloka
{पूजिता मामिका माता दत्तं राज्यमिदं मम}
{तद् ददामि पुनस्तुभ्यं यथा त्वमददा मम} %6-131-2

\twolineshloka
{धुरमेकाकिना न्यस्तां वृषभेण बलीयसा}
{किशोरवद् गुरुं भारं न वोढुमहमुत्सहे} %6-131-3

\twolineshloka
{वारिवेगेन महता भिन्नः सेतुरिव क्षरन्}
{दुर्बन्धनमिदं मन्ये राज्यच्छिद्रमसंवृतम्} %6-131-4

\twolineshloka
{गतिं खर इवाश्वस्य हंसस्येव च वायसः}
{नान्वेतुमुत्सहे वीर तव मार्गमरिंदम} %6-131-5

\twolineshloka
{यथा चारोपितो वृक्षो जातश्चान्तर्निवेशने}
{महानपि दुरारोहो महास्कन्धः प्रशाखवान्} %6-131-6

\twolineshloka
{शीर्येत पुष्पितो भूत्वा न फलानि प्रदर्शयन्}
{तस्य नानुभवेदर्थं यस्य हेतोः स रोपितः} %6-131-7

\twolineshloka
{एषोपमा महाबाहो त्वमर्थं वेत्तुमर्हसि}
{यद्यस्मान् मनुजेन्द्र त्वं भर्ता भृत्यान् न शाधि हि} %6-131-8

\twolineshloka
{जगदद्याभिषिक्तं त्वामनुपश्यतु राघव}
{प्रतपन्तमिवादित्यं मध्याह्ने दीप्ततेजसम्} %6-131-9

\twolineshloka
{तूर्यसंघातनिर्घोषैः काञ्चीनूपुरनिःस्वनैः}
{मधुरैर्गीतशब्दैश्च प्रतिबुध्यस्व शेष्व च} %6-131-10

\twolineshloka
{यावदावर्तते चक्रं यावती च वसुंधरा}
{तावत् त्वमिह लोकस्य स्वामित्वमनुवर्तय} %6-131-11

\twolineshloka
{भरतस्य वचः श्रुत्वा रामः परपुरञ्जयः}
{तथेति प्रतिजग्राह निषसादासने शुभे} %6-131-12

\twolineshloka
{ततः शत्रुघ्नवचनान्निपुणाः श्मश्रुवर्धनाः}
{सुखहस्ताः सुशीघ्राश्च राघवं पर्यवारयन्} %6-131-13

\twolineshloka
{पूर्वं तु भरते स्नाते लक्ष्मणे च महाबले}
{सुग्रीवे वानरेन्द्रे च राक्षसेन्द्रे विभीषणे} %6-131-14

\twolineshloka
{विशोधितजटः स्नातश्चित्रमाल्यानुलेपनः}
{महार्हवसनोपेतस्तस्थौ तत्र श्रिया ज्वलन्} %6-131-15

\twolineshloka
{प्रतिकर्म च रामस्य कारयामास वीर्यवान्}
{लक्ष्मणस्य च लक्ष्मीवानिक्ष्वाकुकुलवर्धनः} %6-131-16

\twolineshloka
{प्रतिकर्म च सीतायाः सर्वा दशरथस्त्रियः}
{आत्मनैव तदा चक्रुर्मनस्विन्यो मनोहरम्} %6-131-17

\twolineshloka
{ततो वानरपत्नीनां सर्वासामेव शोभनम्}
{चकार यत्नात् कौसल्या प्रहृष्टा पुत्रवत्सला} %6-131-18

\twolineshloka
{ततः शत्रुघ्नवचनात् सुमन्त्रो नाम सारथिः}
{योजयित्वाभिचक्राम रथं सर्वाङ्गशोभनम्} %6-131-19

\twolineshloka
{अग्न्यर्कामलसंकाशं दिव्यं दृष्ट्वा रथं स्थितम्}
{आरुरोह महाबाहू रामः परपुरंजयः} %6-131-20

\twolineshloka
{सुग्रीवो हनुमांश्चैव महेन्द्रसदृशद्युती}
{स्नातौ दिव्यनिभैर्वस्त्रैर्जग्मतुः शुभकुण्डलौ} %6-131-21

\twolineshloka
{सर्वाभरणजुष्टाश्च ययुस्ताः शुभकुण्डलाः}
{सुग्रीवपत्न्यः सीता च द्रष्टुं नगरमुत्सुकाः} %6-131-22

\twolineshloka
{अयोध्यायां च सचिवा राज्ञो दशरथस्य च}
{पुरोहितं पुरस्कृत्य मन्त्रयामासुरर्थवत्} %6-131-23

\twolineshloka
{अशोको विजयश्चैव सिद्धार्थश्च समाहिताः}
{मन्त्रयन् रामवृद्ध्यर्थमृद्ध्यर्थं नगरस्य च} %6-131-24

\twolineshloka
{सर्वमेवाभिषेकार्थं जयार्हस्य महात्मनः}
{कर्तुमर्हथ रामस्य यद् यन्मङ्गलपूर्वकम्} %6-131-25

\twolineshloka
{इति ते मन्त्रिणः सर्वे संदिश्य च पुरोहितः}
{नगरान्निर्ययुस्तूर्णं रामदर्शनबुद्धयः} %6-131-26

\twolineshloka
{हरियुक्तं सहस्राक्षो रथमिन्द्र इवानघः}
{प्रययौ रथमास्थाय रामो नगरमुत्तमम्} %6-131-27

\twolineshloka
{जग्राह भरतो रश्मीन् शत्रुघ्नश्छत्रमाददे}
{लक्ष्मणो व्यजनं तस्य मूर्ध्नि संवीजयंस्तदा} %6-131-28

\twolineshloka
{श्वेतं च वालव्यजनं जगृहे परितः स्थितः}
{अपरं चन्द्रसंकाशं राक्षसेन्द्रो विभीषणः} %6-131-29

\twolineshloka
{ऋषिसङ्घैस्तदाऽऽकाशे देवैश्च समरुद्गणैः}
{स्तूयमानस्य रामस्य शुश्रुवे मधुरध्वनिः} %6-131-30

\twolineshloka
{ततः शत्रुञ्जयं नाम कुञ्जरं पर्वतोपमम्}
{आरुरोह महातेजाः सुग्रीवः प्लवगर्षभः} %6-131-31

\twolineshloka
{नव नागसहस्राणि ययुरास्थाय वानराः}
{मानुषं विग्रहं कृत्वा सर्वाभरणभूषिताः} %6-131-32

\twolineshloka
{शङ्खशब्दप्रणादैश्च दुन्दुभीनां च निःस्वनैः}
{प्रययौ पुरुषव्याघ्रस्तां पुरीं हर्म्यमालिनीम्} %6-131-33

\twolineshloka
{ददृशुस्ते समायान्तं राघवं सपुरःसरम्}
{विराजमानं वपुषा रथेनातिरथं तदा} %6-131-34

\twolineshloka
{ते वर्धयित्वा काकुत्स्थं रामेण प्रतिनन्दिताः}
{अनुजग्मुर्महात्मानं भ्रातृभिः परिवारितम्} %6-131-35

\twolineshloka
{अमात्यैर्ब्राह्मणैश्चैव तथा प्रकृतिभिर्वृतः}
{श्रिया विरुरुचे रामो नक्षत्रैरिव चन्द्रमाः} %6-131-36

\twolineshloka
{स पुरोगामिभिस्तूर्यैस्तालस्वस्तिकपाणिभिः}
{प्रव्याहरद्भिर्मुदितैर्मङ्गलानि वृतो ययौ} %6-131-37

\twolineshloka
{अक्षतं जातरूपं च गावः कन्याः सहद्विजाः}
{नरा मोदकहस्ताश्च रामस्य पुरतो ययुः} %6-131-38

\twolineshloka
{सख्यं च रामः सुग्रीवे प्रभावं चानिलात्मजे}
{वानराणां च तत् कर्म ह्याचचक्षेऽथ मन्त्रिणाम्} %6-131-39

\threelineshloka
{श्रुत्वा च विस्मयं जग्मुरयोध्यापुरवासिनः}
{वानराणां च तत् कर्म राक्षसानां च तद् बलम्}
{विभीषणस्य संयोगमाचचक्षेऽथ मन्त्रिणाम्} %6-131-40

\twolineshloka
{द्युतिमानेतदाख्याय रामो वानरसंयुतः}
{हृष्टपुष्टजनाकीर्णामयोध्यां प्रविवेश सः} %6-131-41

\twolineshloka
{ततो ह्यभ्युच्छ्रयन् पौराः पताकाश्च गृहे गृहे}
{ऐक्ष्वाकाध्युषितं रम्यमाससाद पितुर्गृहम्} %6-131-42

\twolineshloka
{अथाब्रवीद् राजपुत्रो भरतं धर्मिणां वरम्}
{अर्थोपहितया वाचा मधुरं रघुनन्दनः} %6-131-43

\twolineshloka
{पितुर्भवनमासाद्य प्रविश्य च महात्मनः}
{कौसल्यां च सुमित्रां च कैकेयीमभिवाद्य च} %6-131-44

\twolineshloka
{तच्च मद्भवनं श्रेष्ठं साशोकवनिकं महत्}
{मुक्तावैदूर्यसंकीर्णं सुग्रीवाय निवेदय} %6-131-45

\twolineshloka
{तस्य तद् वचनं श्रुत्वा भरतः सत्यविक्रमः}
{हस्ते गृहीत्वा सुग्रीवं प्रविवेश तमालयम्} %6-131-46

\twolineshloka
{ततस्तैलप्रदीपांश्च पर्यङ्कास्तरणानि च}
{गृहीत्वा विविशुः क्षिप्रं शत्रुघ्नेन प्रचोदिताः} %6-131-47

\twolineshloka
{उवाच च महातेजाः सुग्रीवं राघवानुजः}
{अभिषेकाय रामस्य दूतानाज्ञापय प्रभो} %6-131-48

\twolineshloka
{सौवर्णान् वानरेन्द्राणां चतुर्णां चतुरो घटान्}
{ददौ क्षिप्रं स सुग्रीवः सर्वरत्नविभूषितान्} %6-131-49

\twolineshloka
{तथा प्रत्यूषसमये चतुर्णां सागराम्भसाम्}
{पूर्णैर्घटैः प्रतीक्षध्वं तथा कुरुत वानराः} %6-131-50

\twolineshloka
{एवमुक्ता महात्मानो वानरा वारणोपमाः}
{उत्पेतुर्गगनं शीघ्रं गरुडा इव शीघ्रगाः} %6-131-51

\twolineshloka
{जाम्बवांश्च हनूमांश्च वेगदर्शी च वानरः}
{ऋषभश्चैव कलशाञ्जलपूर्णानथानयन्} %6-131-52

\twolineshloka
{नदीशतानां पञ्चानां जलं कुम्भैरुपाहरन्}
{पूर्वात् समुद्रात् कलशं जलपूर्णमथानयत्} %6-131-53

\twolineshloka
{सुषेणः सत्त्वसम्पन्नः सर्वरत्नविभूषितम्}
{ऋषभो दक्षिणात्तूर्णं समुद्राज्जलमानयत्} %6-131-54

\twolineshloka
{रक्तचन्दनकर्पूरैः संवृतं काञ्चनं घटम्}
{गवयः पश्चिमात् तोयमाजहार महार्णवात्} %6-131-55

\twolineshloka
{रत्नकुम्भेन महता शीतं मारुतविक्रमः}
{उत्तराच्च जलं शीघ्रं गरुडानिलविक्रमः} %6-131-56

\twolineshloka
{आजहार स धर्मात्मानिलः सर्वगुणान्वितः}
{ततस्तैर्वानरश्रेष्ठैरानीतं प्रेक्ष्य तज्जलम्} %6-131-57

\twolineshloka
{अभिषेकाय रामस्य शत्रुघ्नः सचिवैः सह}
{पुरोहिताय श्रेष्ठाय सुहृद्भ्यश्च न्यवेदयत्} %6-131-58

\twolineshloka
{ततः स प्रयतो वृद्धो वसिष्ठो ब्राह्मणैः सह}
{रामं रत्नमये पीठे ससीतं संन्यवेशयत्} %6-131-59

\twolineshloka
{वसिष्ठो वामदेवश्च जाबालिरथ काश्यपः}
{कात्यायनः सुयज्ञश्च गौतमो विजयस्तथा} %6-131-60

\twolineshloka
{अभ्यषिञ्चन्नरव्याघ्रं प्रसन्नेन सुगन्धिना}
{सलिलेन सहस्राक्षं वसवो वासवं यथा} %6-131-61

\twolineshloka
{ऋत्विग्भिर्ब्राह्मणैः पूर्वं कन्याभिर्मन्त्रिभिस्तथा}
{योधैश्चैवाभ्यषिञ्चस्ते सम्प्रहृष्टैः सनैगमैः} %6-131-62

\twolineshloka
{सर्वौषधिरसैश्चापि दैवतैर्नभसि स्थितैः}
{चतुर्भिर्लोकपालैश्च सर्वैर्देवैश्च संगतैः} %6-131-63

\twolineshloka
{ब्रह्मणा निर्मितं पूर्वं किरीटं रत्नशोभितम्}
{अभिषिक्तः पुरा येन मनुस्तं दीप्ततेजसम्} %6-131-64

\twolineshloka
{तस्यान्ववाये राजानः क्रमाद् येनाभिषेचिताः}
{सभायां हेमक्लृप्तायां शोभितायां महाधनैः} %6-131-65

\twolineshloka
{रत्नैर्नानाविधैश्चैव चित्रितायां सुशोभनैः}
{नानारत्नमये पीठे कल्पयित्वा यथाविधि} %6-131-66

\twolineshloka
{किरीटेन ततः पश्चाद् वसिष्ठेन महात्मना}
{ऋत्विग्भिर्भूषणैश्चैव समयोक्ष्यत राघवः} %6-131-67

\twolineshloka
{छत्रं तस्य च जग्राह शत्रुघ्नः पाण्डुरं शुभम्}
{श्वेतं च वालव्यजनं सुग्रीवो वानरेश्वरः} %6-131-68

\twolineshloka
{अपरं चन्द्रसंकाशं राक्षसेन्द्रो विभीषणः}
{मालां ज्वलन्तीं वपुषा काञ्चनीं शतपुष्कराम्} %6-131-69

\twolineshloka
{राघवाय ददौ वायुर्वासवेन प्रचोदितः}
{सर्वरत्नसमायुक्तं मणिभिश्च विभूषितम्} %6-131-70

\twolineshloka
{मुक्ताहारं नरेन्द्राय ददौ शक्रप्रचोदितः}
{प्रजगुर्देवगन्धर्वा ननृतुश्चाप्सरोगणाः} %6-131-71

\twolineshloka
{अभिषेके तदर्हस्य तदा रामस्य धीमतः}
{भूमिः सस्यवती चैव फलवन्तश्च पादपाः} %6-131-72

\twolineshloka
{गन्धवन्ति च पुष्पाणि बभूवू राघवोत्सवे}
{सहस्रशतमश्वानां धेनूनां च गवां तथा} %6-131-73

\twolineshloka
{ददौ शतवृषान् पूर्वं द्विजेभ्यो मनुजर्षभः}
{त्रिंशत्कोटीर्हिरण्यस्य ब्राह्मणेभ्यो ददौ पुनः} %6-131-74

\twolineshloka
{नानाभरणवस्त्राणि महार्हाणि च राघवः}
{अर्करश्मिप्रतीकाशां काञ्चनीं मणिविग्रहाम्} %6-131-75

\twolineshloka
{सुग्रीवाय स्रजं दिव्यां प्रायच्छन्मनुजाधिपः}
{वैदूर्यमयचित्रे च चन्द्ररश्मिविभूषिते} %6-131-76

\twolineshloka
{वालिपुत्राय धृतिमानङ्गदायाङ्गदे ददौ}
{मणिप्रवरजुष्टं तं मुक्ताहारमनुत्तमम्} %6-131-77

\twolineshloka
{सीतायै प्रददौ रामश्चन्द्ररश्मिसमप्रभम्}
{अरजे वाससी दिव्ये शुभान्याभरणानि च} %6-131-78

\twolineshloka
{अवेक्षमाणा वैदेही प्रददौ वायुसूनवे}
{अवमुच्यात्मनः कण्ठाद्धारं जनकनन्दिनी} %6-131-79

\twolineshloka
{अवैक्षत हरीन् सर्वान् भर्तारं च मुहुर्मुहुः}
{तामिङ्गितज्ञः सम्प्रेक्ष्य बभाषे जनकात्मजाम्} %6-131-80

\twolineshloka
{प्रदेहि सुभगे हारं यस्य तुष्टासि भामिनि}
{अथ सा वायुपुत्राय तं हारमसितेक्षणा} %6-131-81

\twolineshloka
{तेजो धृतिर्यशो दाक्ष्यं सामर्थ्यं विनयो नयः}
{पौरुषं विक्रमो बुद्धिर्यस्मिन् नेतानि नित्यदा} %6-131-82

\twolineshloka
{हनूमांस्तेन हारेण शुशुभे वानरर्षभः}
{चन्द्रांशुचयगौरेण श्वेताभ्रेण यथाचलः} %6-131-83

\twolineshloka
{सर्वे वानरवृद्धाश्च ये चान्ये वानरोत्तमाः}
{वासोभिर्भूषणैश्चैव यथार्हं प्रतिपूजिताः} %6-131-84

\twolineshloka
{विभीषणोऽथ सुग्रीवो हनूमाञ्जाम्बवांस्तथा}
{सर्वे वानरमुख्याश्च रामेणाक्लिष्टकर्मणा} %6-131-85

\twolineshloka
{यथार्हं पूजिताः सर्वे कामै रत्नैश्च पुष्कलैः}
{प्रहृष्टमनसः सर्वे जग्मुरेव यथागतम्} %6-131-86

\twolineshloka
{ततो द्विविदमैन्दाभ्यां नीलाय च परंतपः}
{सर्वान् कामगुणान् वीक्ष्य प्रददौ वसुधाधिपः} %6-131-87

\twolineshloka
{दृष्ट्वा सर्वे महात्मानस्ततस्ते वानरर्षभाः}
{विसृष्टाः पार्थिवेन्द्रेण किष्किन्धां समुपागमन्} %6-131-88

\twolineshloka
{सुग्रीवो वानरश्रेष्ठो दृष्ट्वा रामाभिषेचनम्}
{पूजितश्चैव रामेण किष्किन्धां प्राविशत् पुरीम्} %6-131-89

\twolineshloka
{विभीषणोऽपि धर्मात्मा सह तैर्नैर्ऋतर्षभैः}
{लब्ध्वा कुलधनं राजा लङ्कां प्रायान्महायशाः} %6-131-90

\threelineshloka
{स राज्यमखिलं शासन्निहतारिर्महायशाः}
{राघवः परमोदारः शशास परया मुदा}
{उवाच लक्ष्मणं रामो धर्मज्ञं धर्मवत्सलः} %6-131-91

\twolineshloka
{आतिष्ठ धर्मज्ञ मया सहेमां गां पूर्वराजाध्युषितां बलेन}
{तुल्यं मया त्वं पितृभिर्धृता या तां यौवराज्ये धुरमुद्वहस्व} %6-131-92

\twolineshloka
{सर्वात्मना पर्यनुनीयमानो यदा न सौमित्रिरुपैति योगम्}
{नियुज्यमानो भुवि यौवराज्ये ततोऽभ्यषिञ्चद् भरतं महात्मा} %6-131-93

\twolineshloka
{पौण्डरीकाश्वमेधाभ्यां वाजपेयेन चासकृत्}
{अन्यैश्च विविधैर्यज्ञैरयजत् पार्थिवात्मजः} %6-131-94

\twolineshloka
{राज्यं दशसहस्राणि प्राप्य वर्षाणि राघवः}
{शताश्वमेधानाजह्रे सदश्वान् भूरिदक्षिणान्} %6-131-95

\twolineshloka
{आजानुलम्बिबाहुः स महावक्षाः प्रतापवान्}
{लक्ष्मणानुचरो रामः शशास पृथिवीमिमाम्} %6-131-96

\twolineshloka
{राघवश्चापि धर्मात्मा प्राप्य राज्यमनुत्तमम्}
{ईजे बहुविधैर्यज्ञैः ससुहृज्ज्ञातिबान्धवः} %6-131-97

\twolineshloka
{न पर्यदेवन् विधवा न च व्यालकृतं भयम्}
{न व्याधिजं भयं चासीद् रामे राज्यं प्रशासति} %6-131-98

\twolineshloka
{निर्दस्युरभवल्लोको नानर्थं कश्चिदस्पृशत्}
{न च स्म वृद्धा बालानां प्रेतकार्याणि कुर्वते} %6-131-99

\twolineshloka
{सर्वं मुदितमेवासीत् सर्वो धर्मपरोऽभवत्}
{राममेवानुपश्यन्तो नाभ्यहिंसन् परस्परम्} %6-131-100

\twolineshloka
{आसन् वर्षसहस्राणि तथा पुत्रसहस्रिणः}
{निरामया विशोकाश्च रामे राज्यं प्रशासति} %6-131-101

\twolineshloka
{रामो रामो राम इति प्रजानामभवन् कथाः}
{रामभूतं जगदभूद् रामे राज्यं प्रशासति} %6-131-102

\twolineshloka
{नित्यमूला नित्यफलास्तरवस्तत्र पुष्पिताः}
{कामवर्षी च पर्जन्यः सुखस्पर्शश्च मारुतः} %6-131-103

\twolineshloka
{ब्राह्मणाः क्षत्रिया वैश्याः शूद्रा लोभविवर्जिताः}
{स्वकर्मसु प्रवर्तन्ते तुष्टाः स्वैरेव कर्मभिः} %6-131-104

\twolineshloka
{आसन् प्रजा धर्मपरा रामे शासति नानृताः}
{सर्वे लक्षणसम्पन्नाः सर्वे धर्मपरायणाः} %6-131-105

\twolineshloka
{दशवर्षसहस्राणि दशवर्षशतानि च}
{भ्रातृभिः सहितः श्रीमान् रामो राज्यमकारयत्} %6-131-106

\twolineshloka
{धर्म्यं यशस्यमायुष्यं राज्ञां च विजयावहम्}
{आदिकाव्यमिदं चार्षं पुरा वाल्मीकिना कृतम्} %6-131-107

\twolineshloka
{यः शृणोति सदा लोके नरः पापात् प्रमुच्यते}
{पुत्रकामश्च पुत्रान् वै धनकामो धनानि च} %6-131-108

\twolineshloka
{लभते मनुजो लोके श्रुत्वा रामाभिषेचनम्}
{महीं विजयते राजा रिपूंश्चाप्यधितिष्ठति} %6-131-109

\twolineshloka
{राघवेण यथा माता सुमित्रा लक्ष्मणेन च}
{भरतेन च कैकेयी जीवपुत्रास्तथा स्त्रियः} %6-131-110

\twolineshloka
{भविष्यन्ति सदानन्दाः पुत्रपौत्रसमन्विताः}
{श्रुत्वा रामायणमिदं दीर्घमायुश्च विन्दति} %6-131-111

\twolineshloka
{रामस्य विजयं चेमं सर्वमक्लिष्टकर्मणः}
{शृणोति य इदं काव्यं पुरा वाल्मीकिना कृतम्} %6-131-112

\twolineshloka
{श्रद्दधानो जितक्रोधो दुर्गाण्यतितरत्यसौ}
{समागम्य प्रवासान्ते रमन्ते सह बान्धवैः} %6-131-113

\twolineshloka
{शृण्वन्ति य इदं काव्यं पुरा वाल्मीकिना कृतम्}
{ते प्रार्थितान् वरान् सर्वान् प्राप्नुवन्तीह राघवात्} %6-131-114

\twolineshloka
{श्रवणेन सुराः सर्वे प्रीयन्ते सम्प्रशृण्वताम्}
{विनायकाश्च शाम्यन्ति गृहे तिष्ठन्ति यस्य वै} %6-131-115

\twolineshloka
{विजयेत महीं राजा प्रवासी स्वस्तिमान् भवेत्}
{स्त्रियो रजस्वलाः श्रुत्वा पुत्रान् सूयुरनुत्तमान्} %6-131-116

\twolineshloka
{पूजयंश्च पठंश्चैनमितिहासं पुरातनम्}
{सर्वपापैः प्रमुच्येत दीर्घमायुरवाप्नुयात्} %6-131-117

\twolineshloka
{प्रणम्य शिरसा नित्यं श्रोतव्यं क्षत्रियैर्द्विजात्}
{ऐश्वर्यं पुत्रलाभश्च भविष्यति न संशयः} %6-131-118

\twolineshloka
{रामायणमिदं कृत्स्नं शृण्वतः पठतः सदा}
{प्रीयते सततं रामः स हि विष्णुः सनातनः} %6-131-119

\twolineshloka
{आदिदेवो महाबाहुर्हरिर्नारायणः प्रभुः}
{साक्षाद् रामो रघुश्रेष्ठः शेषो लक्ष्मण उच्यते} %6-131-120

\twolineshloka
{एवमेतत् पुरावृत्तमाख्यानं भद्रमस्तु वः}
{प्रव्याहरत विस्रब्धं बलं विष्णोः प्रवर्धताम्} %6-131-121

\twolineshloka
{देवाश्च सर्वे तुष्यन्ति ग्रहणाच्छ्रवणात् तथा}
{रामायणस्य श्रवणे तृप्यन्ति पितरः सदा} %6-131-122

\twolineshloka
{भक्त्या रामस्य ये चेमां संहितामृषिणा कृताम्}
{ये लिखन्तीह च नरास्तेषां वासस्त्रिविष्टपे} %6-131-123

\twolineshloka
{कुटुम्बवृद्धिं धनधान्यवृद्धिं स्त्रियश्च मुख्याः सुखमुत्तमं च}
{श्रुत्वा शुभं काव्यमिदं महार्थं प्राप्नोति सर्वां भुवि चार्थसिद्धिम्} %6-131-124

\twolineshloka
{आयुष्यमारोग्यकरं यशस्यं सौभ्रातृकं बुद्धिकरं शुभं च}
{श्रोतव्यमेतन्नियमेन सद्भिराख्यानमोजस्करमृद्धिकामैः} %6-131-125


॥इत्यार्षे श्रीमद्रामायणे वाल्मीकीये आदिकाव्ये युद्धकाण्डे श्रीरामपट्टाभिषेकः नाम एकत्रिंशदधिकशततमः सर्गः ॥६-१३१॥
