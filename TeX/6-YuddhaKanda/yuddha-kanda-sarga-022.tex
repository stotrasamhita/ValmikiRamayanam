\sect{द्वाविंशः सर्गः — सेतुबन्धः}

\twolineshloka
{अथोवाच रघुश्रेष्ठः सागरं दारुणं वचः}
{अद्य त्वां शोषयिष्यामि सपातालं महार्णव} %6-22-1

\twolineshloka
{शरनिर्दग्धतोयस्य परिशुष्कस्य सागर}
{मया निहतसत्त्वस्य पांसुरुत्पद्यते महान्} %6-22-2

\twolineshloka
{मत्कार्मुकविसृष्टेन शरवर्षेण सागर}
{परं तीरं गमिष्यन्ति पद्भिरेव प्लवंगमाः} %6-22-3

\twolineshloka
{विचिन्वन्नाभिजानासि पौरुषं नापि विक्रमम्}
{दानवालय संतापं मत्तो नाम गमिष्यसि} %6-22-4

\twolineshloka
{ब्राह्मेणास्त्रेण संयोज्य ब्रह्मदण्डनिभं शरम्}
{संयोज्य धनुषि श्रेष्ठे विचकर्ष महाबलः} %6-22-5

\twolineshloka
{तस्मिन् विकृष्टे सहसा राघवेण शरासने}
{रोदसी सम्पफालेव पर्वताश्च चकम्पिरे} %6-22-6

\twolineshloka
{तमश्च लोकमावव्रे दिशश्च न चकाशिरे}
{प्रतिचुक्षुभिरे चाशु सरांसि सरितस्तथा} %6-22-7

\twolineshloka
{तिर्यक् च सह नक्षत्रैः संगतौ चन्द्रभास्करौ}
{भास्करांशुभिरादीप्तं तमसा च समावृतम्} %6-22-8

\twolineshloka
{प्रचकाशे तदाऽऽकाशमुल्काशतविदीपितम्}
{अन्तरिक्षाच्च निर्घाता निर्जग्मुरतुलस्वनाः} %6-22-9

\twolineshloka
{वपुःप्रकर्षेण ववुर्दिव्यमारुतपङ्क्तयः}
{बभञ्ज च तदा वृक्षाञ्जलदानुद्वहन्मुहुः} %6-22-10

\twolineshloka
{आरुजंश्चैव शैलाग्रान् शिखराणि बभञ्ज च}
{दिवि च स्म महामेघाः संहताः समहास्वनाः} %6-22-11

\twolineshloka
{मुमुचुर्वैद्युतानग्नींस्ते महाशनयस्तदा}
{यानि भूतानि दृश्यानि चुक्रुशुश्चाशनेः समम्} %6-22-12

\twolineshloka
{अदृश्यानि च भूतानि मुमुचुर्भैरवस्वनम्}
{शिश्यिरे चाभिभूतानि संत्रस्तान्युद्विजन्ति च} %6-22-13

\twolineshloka
{सम्प्रविव्यथिरे चापि न च पस्पन्दिरे भयात्}
{सह भूतैः सतोयोर्मिः सनागः सहराक्षसः} %6-22-14

\twolineshloka
{सहसाभूत् ततो वेगाद् भीमवेगो महोदधिः}
{योजनं व्यतिचक्राम वेलामन्यत्र सम्प्लवात्} %6-22-15

\twolineshloka
{तं तथा समतिक्रान्तं नातिचक्राम राघवः}
{समुद्धतममित्रघ्नो रामो नदनदीपतिम्} %6-22-16

\twolineshloka
{ततो मध्यात् समुद्रस्य सागरः स्वयमुत्थितः}
{उदयाद्रिमहाशैलान्मेरोरिव दिवाकरः} %6-22-17

\twolineshloka
{पन्नगैः सह दीप्तास्यैः समुद्रः प्रत्यदृश्यत}
{स्निग्धवैदूर्यसंकाशो जाम्बूनदविभूषणः} %6-22-18

\twolineshloka
{रक्तमाल्याम्बरधरः पद्मपत्रनिभेक्षणः}
{सर्वपुष्पमयीं दिव्यां शिरसा धारयन् स्रजम्} %6-22-19

\twolineshloka
{जातरूपमयैश्चैव तपनीयविभूषणैः}
{आत्मजानां च रत्नानां भूषितो भूषणोत्तमैः} %6-22-20

\twolineshloka
{धातुभिर्मण्डितः शैलो विविधैर्हिमवानिव}
{एकावलीमध्यगतं तरलं पाण्डरप्रभम्} %6-22-21

\twolineshloka
{विपुलेनोरसा बिभ्रत्कौस्तुभस्य सहोदरम्}
{आघूर्णिततरङ्गौघः कालिकानिलसंकुलः} %6-22-22

\twolineshloka
{गङ्गासिन्धुप्रधानाभिरापगाभिः समावृतः}
{उद्वर्तितमहाग्राहः सम्भ्रान्तोरगराक्षसः} %6-22-23

\twolineshloka
{देवतानां सुरूपाभिर्नानारूपाभिरीश्वरः}
{सागरः समुपक्रम्य पूर्वमामन्त्र्य वीर्यवान्} %6-22-24

\onelineshloka
{अब्रवीत् प्राञ्जलिर्वाक्यं राघवं शरपाणिनम्} %6-22-25

\twolineshloka
{पृथिवी वायुराकाशमापो ज्योतिश्च राघव}
{स्वभावे सौम्य तिष्ठन्ति शाश्वतं मार्गमाश्रिताः} %6-22-26

\twolineshloka
{तत्स्वभावो ममाप्येष यदगाधोऽहमप्लवः}
{विकारस्तु भवेद् गाध एतत् ते प्रवदाम्यहम्} %6-22-27

\twolineshloka
{न कामान्न च लोभाद् वा न भयात् पार्थिवात्मज}
{ग्राहनक्राकुलजलं स्तम्भयेयं कथंचन} %6-22-28

\threelineshloka
{विधास्ये येन गन्तासि विषहिष्येऽप्यहं तथा}
{न ग्राहा विधमिष्यन्ति यावत्सेना तरिष्यति}
{हरीणां तरणे राम करिष्यामि यथा स्थलम्} %6-22-29

\twolineshloka
{तमब्रवीत् तदा रामः शृणु मे वरुणालय}
{अमोघोऽयं महाबाणः कस्मिन् देशे निपात्यताम्} %6-22-30

\twolineshloka
{रामस्य वचनं श्रुत्वा तं च दृष्ट्वा महाशरम्}
{महोदधिर्महातेजा राघवं वाक्यमब्रवीत्} %6-22-31

\twolineshloka
{उत्तरेणावकाशोऽस्ति कश्चित् पुण्यतरो मम}
{द्रुमकुल्य इति ख्यातो लोके ख्यातो यथा भवान्} %6-22-32

\twolineshloka
{उग्रदर्शनकर्माणो बहवस्तत्र दस्यवः}
{आभीरप्रमुखाः पापाः पिबन्ति सलिलं मम} %6-22-33

\twolineshloka
{तैर्न तत्स्पर्शनं पापं सहेयं पापकर्मभिः}
{अमोघः क्रियतां राम अयं तत्र शरोत्तमः} %6-22-34

\twolineshloka
{तस्य तद् वचनं श्रुत्वा सागरस्य महात्मनः}
{मुमोच तं शरं दीप्तं परं सागरदर्शनात्} %6-22-35

\twolineshloka
{तेन तन्मरुकान्तारं पृथिव्यां किल विश्रुतम्}
{निपातितः शरो यत्र वज्राशनिसमप्रभः} %6-22-36

\twolineshloka
{ननाद च तदा तत्र वसुधा शल्यपीडिता}
{तस्माद् व्रणमुखात् तोयमुत्पपात रसातलात्} %6-22-37

\twolineshloka
{स बभूव तदा कूपो व्रण इत्येव विश्रुतः}
{सततं चोत्थितं तोयं समुद्रस्येव दृश्यते} %6-22-38

\twolineshloka
{अवदारणशब्दश्च दारुणः समपद्यत}
{तस्मात् तद् बाणपातेन अपः कुक्षिष्वशोषयत्} %6-22-39

\twolineshloka
{विख्यातं त्रिषु लोकेषु मरुकान्तारमेव च}
{शोषयित्वा तु तं कुक्षिं रामो दशरथात्मजः} %6-22-40

\onelineshloka
{वरं तस्मै ददौ विद्वान् मरवेऽमरविक्रमः} %6-22-41

\twolineshloka
{पशव्यश्चाल्परोगश्च फलमूलरसायुतः}
{बहुस्नेहो बहुक्षीरः सुगन्धिर्विविधौषधिः} %6-22-42

\twolineshloka
{एवमेतैश्च संयुक्तो बहुभिः संयुतो मरुः}
{रामस्य वरदानाच्च शिवः पन्था बभूव ह} %6-22-43

\twolineshloka
{तस्मिन् दग्धे तदा कुक्षौ समुद्रः सरितां पतिः}
{राघवं सर्वशास्त्रज्ञमिदं वचनमब्रवीत्} %6-22-44

\twolineshloka
{अयं सौम्य नलो नाम तनयो विश्वकर्मणः}
{पित्रा दत्तवरः श्रीमान् प्रीतिमान् विश्वकर्मणः} %6-22-45

\twolineshloka
{एष सेतुं महोत्साहः करोतु मयि वानरः}
{तमहं धारयिष्यामि यथा ह्येष पिता तथा} %6-22-46

\twolineshloka
{एवमुक्त्वोदधिर्नष्टः समुत्थाय नलस्ततः}
{अब्रवीद् वानरश्रेष्ठो वाक्यं रामं महाबलम्} %6-22-47

\twolineshloka
{अहं सेतुं करिष्यामि विस्तीर्णे मकरालये}
{पितुः सामर्थ्यमासाद्य तत्त्वमाह महोदधिः} %6-22-48

\twolineshloka
{दण्ड एव वरो लोके पुरुषस्येति मे मतिः}
{धिक् क्षमामकृतज्ञेषु सान्त्वं दानमथापि वा} %6-22-49

\twolineshloka
{अयं हि सागरो भीमः सेतुकर्मदिदृक्षया}
{ददौ दण्डभयाद् गाधं राघवाय महोदधिः} %6-22-50

\twolineshloka
{मम मातुर्वरो दत्तो मन्दरे विश्वकर्मणा}
{मया तु सदृशः पुत्रस्तव देवि भविष्यति} %6-22-51

\threelineshloka
{औरसस्तस्य पुत्रोऽहं सदृशो विश्वकर्मणा}
{स्मारितोऽस्म्यहमेतेन तत्त्वमाह महोदधिः}
{न चाप्यहमनुक्तो वः प्रब्रूयामात्मनो गुणान्} %6-22-52

\twolineshloka
{समर्थश्चाप्यहं सेतुं कर्तुं वै वरुणालये}
{तस्मादद्यैव बध्नन्तु सेतुं वानरपुङ्गवाः} %6-22-53

\twolineshloka
{ततो विसृष्टा रामेण सर्वतो हरिपुङ्गवाः}
{उत्पेततुर्महारण्यं हृष्टाः शतसहस्रशः} %6-22-54

\twolineshloka
{ते नगान् नगसंकाशाः शाखामृगगणर्षभाः}
{बभञ्जुः पादपांस्तत्र प्रचकर्षुश्च सागरम्} %6-22-55

\twolineshloka
{ते सालैश्चाश्वकर्णैश्च धवैर्वंशैश्च वानराः}
{कुटजैरर्जुनैस्तालैस्तिलकैस्तिनिशैरपि} %6-22-56

\twolineshloka
{बिल्वकैः सप्तपर्णैश्च कर्णिकारैश्च पुष्पितैः}
{चूतैश्चाशोकवृक्षैश्च सागरं समपूरयन्} %6-22-57

\twolineshloka
{समूलांश्च विमूलांश्च पादपान् हरिसत्तमाः}
{इन्द्रकेतूनिवोद्यम्य प्रजह्रुर्वानरास्तरून्} %6-22-58

\twolineshloka
{तालान् दाडिमगुल्मांश्च नारिकेलविभीतकान्}
{करीरान् बकुलान् निम्बान् समाजह्रुरितस्ततः} %6-22-59

\twolineshloka
{हस्तिमात्रान् महाकायाः पाषाणांश्च महाबलाः}
{पर्वतांश्च समुत्पाट्य यन्त्रैः परिवहन्ति च} %6-22-60

\twolineshloka
{प्रक्षिप्यमाणैरचलैः सहसा जलमुद्धृतम्}
{समुत्ससर्प चाकाशमवासर्पत् ततः पुनः} %6-22-61

\twolineshloka
{समुद्रं क्षोभयामासुर्निपतन्तः समन्ततः}
{सूत्राण्यन्ये प्रगृह्णन्ति ह्यायतं शतयोजनम्} %6-22-62

\twolineshloka
{नलश्चक्रे महासेतुं मध्ये नदनदीपतेः}
{स तदा क्रियते सेतुर्वानरैर्घोरकर्मभिः} %6-22-63

\twolineshloka
{दण्डानन्ये प्रगृह्णन्ति विचिन्वन्ति तथापरे}
{वानरैः शतशस्तत्र रामस्याज्ञापुरःसरैः} %6-22-64

\twolineshloka
{मेघाभैः पर्वताभैश्च तृणैः काष्ठैर्बबन्धिरे}
{पुष्पिताग्रैश्च तरुभिः सेतुं बघ्नन्ति वानराः} %6-22-65

\twolineshloka
{पाषाणांश्च गिरिप्रख्यान् गिरीणां शिखराणि च}
{दृश्यन्ते परिधावन्तो गृह्य दानवसंनिभाः} %6-22-66

\twolineshloka
{शिलानां क्षिप्यमाणानां शैलानां तत्र पात्यताम्}
{बभूव तुमुलः शब्दस्तदा तस्मिन् महोदधौ} %6-22-67

\twolineshloka
{कृतानि प्रथमेनाह्ना योजनानि चतुर्दश}
{प्रहृष्टैर्गजसंकाशैस्त्वरमाणैः प्लवङ्गमैः} %6-22-68

\twolineshloka
{द्वितीयेन तथैवाह्ना योजनानि तु विंशतिः}
{कृतानि प्लवगैस्तूर्णं भीमकायैर्महाबलैः} %6-22-69

\twolineshloka
{अह्ना तृतीयेन तथा योजनानि तु सागरे}
{त्वरमाणैर्महाकायैरेकविंशतिरेव च} %6-22-70

\twolineshloka
{चतुर्थेन तथा चाह्ना द्वाविंशतिरथापि वा}
{योजनानि महावेगैः कृतानि त्वरितैस्ततः} %6-22-71

\twolineshloka
{पञ्चमेन तथा चाह्ना प्लवगैः क्षिप्रकारिभिः}
{योजनानि त्रयोविंशत् सुवेलमधिकृत्य वै} %6-22-72

\twolineshloka
{स वानरवरः श्रीमान् विश्वकर्मात्मजो बली}
{बबन्ध सागरे सेतुं यथा चास्य पिता तथा} %6-22-73

\twolineshloka
{स नलेन कृतः सेतुः सागरे मकरालये}
{शुशुभे सुभगः श्रीमान् स्वातीपथ इवाम्बरे} %6-22-74

\twolineshloka
{ततो देवाः सगन्धर्वाः सिद्धाश्च परमर्षयः}
{आगम्य गगने तस्थुर्द्रष्टुकामास्तदद्भुतम्} %6-22-75

\twolineshloka
{दशयोजनविस्तीर्णं शतयोजनमायतम्}
{ददृशुर्देवगन्धर्वा नलसेतुं सुदुष्करम्} %6-22-76

\twolineshloka
{आप्लवन्तः प्लवन्तश्च गर्जन्तश्च प्लवंगमाः}
{तमचिन्त्यमसह्यं च ह्यद्भुतं लोमहर्षणम्} %6-22-77

\twolineshloka
{ददृशुः सर्वभूतानि सागरे सेतुबन्धनम्}
{तानि कोटिसहस्राणि वानराणां महौजसाम्} %6-22-78

\twolineshloka
{बध्नन्तः सागरे सेतुं जग्मुः पारं महोदधेः}
{विशालः सुकृतः श्रीमान् सुभूमिः सुसमाहितः} %6-22-79

\twolineshloka
{अशोभत महान् सेतुः सीमन्त इव सागरे}
{ततः पारे समुद्रस्य गदापाणिर्विभीषणः} %6-22-80

\twolineshloka
{परेषामभिघातार्थमतिष्ठत् सचिवैः सह}
{सुग्रीवस्तु ततः प्राह रामं सत्यपराक्रमम्} %6-22-81

\twolineshloka
{हनूमन्तं त्वमारोह अङ्गदं त्वथ लक्ष्मणः}
{अयं हि विपुलो वीर सागरो मकरालयः} %6-22-82

\twolineshloka
{वैहायसौ युवामेतौ वानरौ धारयिष्यतः}
{अग्रतस्तस्य सैन्यस्य श्रीमान् रामः सलक्ष्मणः} %6-22-83

\twolineshloka
{जगाम धन्वी धर्मात्मा सुग्रीवेण समन्वितः}
{अन्ये मध्येन गच्छन्ति पार्श्वतोऽन्ये प्लवंगमाः} %6-22-84

\twolineshloka
{सलिलं प्रपतन्त्यन्ये मार्गमन्ये प्रपेदिरे}
{केचिद् वैहायसगताः सुपर्णा इव पुप्लुवुः} %6-22-85

\twolineshloka
{घोषेण महता घोषं सागरस्य समुच्छ्रितम्}
{भीममन्तर्दधे भीमा तरन्ती हरिवाहिनी} %6-22-86

\twolineshloka
{वानराणां हि सा तीर्णा वाहिनी नलसेतुना}
{तीरे निविविशे राज्ञो बहुमूलफलोदके} %6-22-87

\twolineshloka
{तदद्भुतं राघवकर्म दुष्करं समीक्ष्य देवाः सह सिद्धचारणैः}
{उपेत्य रामं सहसा महर्षिभिस्तमभ्यषिञ्चन् सुशुभैर्जलैः पृथक्} %6-22-88

\twolineshloka
{जयस्व शत्रून् नरदेव मेदिनीं ससागरां पालय शाश्वतीः समाः}
{इतीव रामं नरदेवसत्कृतं शुभैर्वचोभिर्विविधैरपूजयन्} %6-22-89


॥इत्यार्षे श्रीमद्रामायणे वाल्मीकीये आदिकाव्ये युद्धकाण्डे सेतुबन्धः नाम द्वाविंशः सर्गः ॥६-२२॥
