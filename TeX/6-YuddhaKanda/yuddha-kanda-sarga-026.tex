\sect{षड्विंशः सर्गः — कपिबलावेक्षणम्}

\twolineshloka
{तद्वचः सत्यमक्लीबं सारणेनाभिभाषितम्}
{निशम्य रावणो राजा प्रत्यभाषत सारणम्} %6-26-1

\twolineshloka
{यदि मामभियुञ्जीरन् देवगन्धर्वदानवाः}
{नैव सीतामहं दद्यां सर्वलोकभयादपि} %6-26-2

\twolineshloka
{त्वं तु सौम्य परित्रस्तो हरिभिः पीडितो भृशम्}
{प्रतिप्रदानमद्यैव सीतायाः साधु मन्यसे} %6-26-3

\twolineshloka
{को हि नाम सपत्नो मां समरे जेतुमर्हति}
{इत्युक्त्वा परुषं वाक्यं रावणो राक्षसाधिपः} %6-26-4

\twolineshloka
{आरुरोह ततः श्रीमान् प्रासादं हिमपाण्डुरम्}
{बहुतालसमुत्सेधं रावणोऽथ दिदृक्षया} %6-26-5

\twolineshloka
{ताभ्यां चराभ्यां सहितो रावणः क्रोधमूर्च्छितः}
{पश्यमानः समुद्रं तं पर्वतांश्च वनानि च} %6-26-6

\twolineshloka
{ददर्श पृथिवीदेशं सुसम्पूर्णं प्लवङ्गमैः}
{तदपारमसह्यं च वानराणां महाबलम्} %6-26-7

\twolineshloka
{आलोक्य रावणो राजा परिपप्रच्छ सारणम्}
{एषां के वानरा मुख्याः के शूराः के महाबलाः} %6-26-8

\twolineshloka
{के पूर्वमभिवर्तन्ते महोत्साहाः समन्ततः}
{केषां शृणोति सुग्रीवः के वा यूथपयूथपाः} %6-26-9

\twolineshloka
{सारणाचक्ष्व मे सर्वं किम्प्रभावाः प्लवङ्गमाः}
{सारणो राक्षसेन्द्रस्य वचनं परिपृच्छतः} %6-26-10

\twolineshloka
{आबभाषेऽथ मुख्यज्ञो मुख्यांस्तत्र वनौकसः}
{एष योऽभिमुखो लङ्कां नर्दंस्तिष्ठति वानरः} %6-26-11

\twolineshloka
{यूथपानां सहस्राणां शतेन परिवारितः}
{यस्य घोषेण महता सप्राकारा सतोरणा} %6-26-12

\twolineshloka
{लङ्का प्रतिहता सर्वा सशैलवनकानना}
{सर्वशाखामृगेन्द्रस्य सुग्रीवस्य महात्मनः} %6-26-13

\twolineshloka
{बलाग्रे तिष्ठते वीरो नीलो नामैष यूथपः}
{बाहू प्रगृह्य यः पद्भ्यां महीं गच्छति वीर्यवान्} %6-26-14

\twolineshloka
{लङ्कामभिमुखः कोपादभीक्ष्णं च विजृम्भते}
{गिरिशृङ्गप्रतीकाशः पद्मकिञ्जल्कसन्निभः} %6-26-15

\twolineshloka
{स्फोटयत्यतिसंरब्धो लाङ्गूलं च पुनः पुनः}
{यस्य लाङ्गूलशब्देन स्वनन्ति प्रदिशो दश} %6-26-16

\twolineshloka
{एष वानरराजेन सुग्रीवेणाभिषेचितः}
{युवराजोऽङ्गदो नाम त्वामाह्वयति संयुगे} %6-26-17

\twolineshloka
{वालिनः सदृशः पुत्रः सुग्रीवस्य सदा प्रियः}
{राघवार्थे पराक्रान्तः शक्रार्थे वरुणो यथा} %6-26-18

\twolineshloka
{एतस्य सा मतिः सर्वा यद् दृष्टा जनकात्मजा}
{हनूमता वेगवता राघवस्य हितैषिणा} %6-26-19

\twolineshloka
{बहूनि वानरेन्द्राणामेष यूथानि वीर्यवान्}
{परिगृह्याभियाति त्वां स्वेनानीकेन मर्दितुम्} %6-26-20

\twolineshloka
{अनुवालिसुतस्यापि बलेन महता वृतः}
{वीरस्तिष्ठति सङ्ग्रामे सेतुहेतुरयं नलः} %6-26-21

\twolineshloka
{ये तु विष्टभ्य गात्राणि क्ष्वेडयन्ति नदन्ति च}
{उत्थाय च विजृम्भन्ते क्रोधेन हरिपुङ्गवाः} %6-26-22

\threelineshloka
{एते दुष्प्रसहा घोराश्चण्डाश्चण्डपराक्रमाः}
{अष्टौ शतसहस्राणि दशकोटिशतानि च}
{य एनमनुगच्छन्ति वीराश्चन्दनवासिनः} %6-26-23

\twolineshloka
{एषैवाशंसते लङ्कां स्वेनानीकेन मर्दितुम्}
{श्वेतो रजतसङ्काशश्चपलो भीमविक्रमः} %6-26-24

\twolineshloka
{बुद्धिमान् वानरः शूरस्त्रिषु लोकेषु विश्रुतः}
{तूर्णं सुग्रीवमागम्य पुनर्गच्छति वानरः} %6-26-25

\twolineshloka
{विभजन् वानरीं सेनामनीकानि प्रहर्षयन्}
{यः पुरा गोमतीतीरे रम्यं पर्येति पर्वतम्} %6-26-26

\twolineshloka
{नाम्ना संरोचनो नाम नानानगयुतो गिरिः}
{तत्र राज्यं प्रशास्त्येष कुमुदो नाम यूथपः} %6-26-27

\twolineshloka
{योऽसौ शतसहस्राणि सहर्षं परिकर्षति}
{यस्य वाला बहुव्यामा दीर्घलाङ्गूलमाश्रिताः} %6-26-28

\threelineshloka
{ताम्राः पीताः सिताः श्वेताः प्रकीर्णा घोरदर्शनाः}
{अदीनो वानरश्चण्डः सङ्ग्राममभिकाङ्क्षति}
{एषोऽप्याशंसते लङ्कां स्वेनानीकेन मर्दितुम्} %6-26-29

\twolineshloka
{यस्त्वेष सिंहसङ्काशः कपिलो दीर्घकेसरः}
{निभृतः प्रेक्षते लङ्कां दिधक्षन्निव चक्षुषा} %6-26-30

\threelineshloka
{विन्ध्यं कृष्णगिरिं सह्यं पर्वतं च सुदर्शनम्}
{राजन् सततमध्यास्ते स रम्भो नाम यूथपः}
{शतं शतसहस्राणां त्रिंशच्च हरिपुङ्गवाः} %6-26-31

\twolineshloka
{यं यान्तं वानरा घोराश्चण्डाश्चण्डपराक्रमाः}
{परिवार्यानुगच्छन्ति लङ्कां मर्दितुमोजसा} %6-26-32

\twolineshloka
{यस्तु कर्णौ विवृणुते जृम्भते च पुनः पुनः}
{न तु संविजते मृत्योर्न च सेनां प्रधावति} %6-26-33

\twolineshloka
{प्रकम्पते च रोषेण तिर्यक् च पुनरीक्षते}
{पश्य लाङ्गूलविक्षेपं क्ष्वेडत्येष महाबलः} %6-26-34

\twolineshloka
{महाजवो वीतभयो रम्यं साल्वेयपर्वतम्}
{राजन् सततमध्यास्ते शरभो नाम यूथपः} %6-26-35

\twolineshloka
{एतस्य बलिनः सर्वे विहारा नाम यूथपाः}
{राजन् शतसहस्राणि चत्वारिंशत्तथैव च} %6-26-36

\twolineshloka
{यस्तु मेघ इवाकाशं महानावृत्य तिष्ठति}
{मध्ये वानरवीराणां सुराणामिव वासवः} %6-26-37

\twolineshloka
{भेरीणामिव सन्नादो यस्यैष श्रूयते महान्}
{घोषः शाखामृगेन्द्राणां सङ्ग्राममभिकाङ्क्षताम्} %6-26-38

\twolineshloka
{एष पर्वतमध्यास्ते पारियात्रमनुत्तमम्}
{युद्धे दुष्प्रसहो नित्यं पनसो नाम यूथपः} %6-26-39

\twolineshloka
{एनं शतसहस्राणां शतार्धं पर्युपासते}
{यूथपा यूथपश्रेष्ठं येषां यूथानि भागशः} %6-26-40

\twolineshloka
{यस्तु भीमां प्रवल्गन्तीं चमूं तिष्ठति शोभयन्}
{स्थितां तीरे समुद्रस्य द्वितीय इव सागरः} %6-26-41

\twolineshloka
{एष दर्दुरसङ्काशो विनतो नाम यूथपः}
{पिबंश्चरति यो वेणां नदीनामुत्तमां नदीम्} %6-26-42

\twolineshloka
{षष्टिः शतसहस्राणि बलमस्य प्लवङ्गमाः}
{त्वामाह्वयति युद्धाय क्रोधनो नाम वानरः} %6-26-43

\twolineshloka
{विक्रान्ता बलवन्तश्च यथा यूथानि भागशः}
{यस्तु गैरिकवर्णाभं वपुः पुष्यति वानरः} %6-26-44

\twolineshloka
{अवमत्य सदा सर्वान् वानरान् बलदर्पितः}
{गवयो नाम तेजस्वी त्वां क्रोधादभिवर्तते} %6-26-45

\twolineshloka
{एनं शतसहस्राणि सप्ततिः पर्युपासते}
{एषैवाशंसते लङ्कां स्वेनानीकेन मर्दितुम्} %6-26-46

\twolineshloka
{एते दुष्प्रसहा वीरा येषां सङ्ख्या न विद्यते}
{यूथपा यूथपश्रेष्ठास्तेषां यूथानि भागशः} %6-26-47


॥इत्यार्षे श्रीमद्रामायणे वाल्मीकीये आदिकाव्ये युद्धकाण्डे कपिबलावेक्षणम् नाम षड्विंशः सर्गः ॥६-२६॥
