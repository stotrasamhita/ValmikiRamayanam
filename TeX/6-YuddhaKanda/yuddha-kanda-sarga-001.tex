\sect{प्रथमः सर्गः — हनूमत्प्रशंसनम्}

\twolineshloka
{श्रुत्वा हनूमतो वाक्यं यथावदभिभाषितम्}
{रामः प्रीतिसमायुक्तो वाक्यमुत्तरमब्रवीत्} %6-1-1

\twolineshloka
{कृतं हनूमता कार्यं सुमहद् भुवि दुर्लभम्}
{मनसापि यदन्येन न शक्यं धरणीतले} %6-1-2

\twolineshloka
{नहि तं परिपश्यामि यस्तरेत महोदधिम्}
{अन्यत्र गरुडाद् वायोरन्यत्र च हनूमतः} %6-1-3

\twolineshloka
{देवदानवयक्षाणां गन्धर्वोरगरक्षसाम्}
{अप्रधृष्यां पुरीं लङ्कां रावणेन सुरक्षिताम्} %6-1-4

\twolineshloka
{प्रविष्टः सत्त्वमाश्रित्य जीवन् को नाम निष्क्रमेत्}
{को विशेत् सुदुराधर्षां राक्षसैश्च सुरक्षिताम्} %6-1-5

\threelineshloka
{यो वीर्यबलसम्पन्नो न समः स्याद्धनूमतः}
{भृत्यकार्यं हनुमता सुग्रीवस्य कृतं महत्}
{एवं विधाय स्वबलं सदृशं विक्रमस्य च} %6-1-6

\twolineshloka
{यो हि भृत्यो नियुक्तः सन् भर्त्रा कर्मणि दुष्करे}
{कुर्यात् तदनुरागेण तमाहुः पुरुषोत्तमम्} %6-1-7

\twolineshloka
{यो नियुक्तः परं कार्यं न कुर्यान्नृपतेः प्रियम्}
{भृत्यो युक्तः समर्थश्च तमाहुर्मध्यमं नरम्} %6-1-8

\twolineshloka
{नियुक्तो नृपतेः कार्यं न कुर्याद् यः समाहितः}
{भृत्यो युक्तः समर्थश्च तमाहुः पुरुषाधमम्} %6-1-9

\twolineshloka
{तन्नियोगे नियुक्तेन कृतं कृत्यं हनूमता}
{न चात्मा लघुतां नीतः सुग्रीवश्चापि तोषितः} %6-1-10

\twolineshloka
{अहं च रघुवंशश्च लक्ष्मणश्च महाबलः}
{वैदेह्या दर्शनेनाद्य धर्मतः परिरक्षिताः} %6-1-11

\twolineshloka
{इदं तु मम दीनस्य मनो भूयः प्रकर्षति}
{यदिहास्य प्रियाख्यातुर्न कुर्मि सदृशं प्रियम्} %6-1-12

\twolineshloka
{एष सर्वस्वभूतस्तु परिष्वङ्गो हनूमतः}
{मया कालमिमं प्राप्य दत्तस्तस्य महात्मनः} %6-1-13

\twolineshloka
{इत्युक्त्वा प्रीतिहृष्टाङ्गो रामस्तं परिषस्वजे}
{हनूमन्तं कृतात्मानं कृतकार्यमुपागतम्} %6-1-14

\twolineshloka
{ध्यात्वा पुनरुवाचेदं वचनं रघुसत्तमः}
{हरीणामीश्वरस्यापि सुग्रीवस्योपशृण्वतः} %6-1-15

\twolineshloka
{सर्वथा सुकृतं तावत् सीतायाः परिमार्गणम्}
{सागरं तु समासाद्य पुनर्नष्टं मनो मम} %6-1-16

\twolineshloka
{कथं नाम समुद्रस्य दुष्पारस्य महाम्भसः}
{हरयो दक्षिणं पारं गमिष्यन्ति समागताः} %6-1-17

\twolineshloka
{यद्यप्येष तु वृत्तान्तो वैदेह्या गदितो मम}
{समुद्रपारगमने हरीणां किमिवोत्तरम्} %6-1-18

\twolineshloka
{इत्युक्त्वा शोकसम्भ्रान्तो रामः शत्रुनिबर्हणः}
{हनूमन्तं महाबाहुस्ततो ध्यानमुपागमत्} %6-1-19


॥इत्यार्षे श्रीमद्रामायणे वाल्मीकीये आदिकाव्ये युद्धकाण्डे हनूमत्प्रशंसनम् नाम प्रथमः सर्गः ॥६-१॥
