\sect{अष्टात्रिंशः सर्गः — सुवेलारोहणम्}

\twolineshloka
{स तु कृत्वा सुवेलस्य मतिमारोहणं प्रति}
{लक्ष्मणानुगतो रामः सुग्रीवमिदमब्रवीत्} %6-38-1

\twolineshloka
{विभीषणं च धर्मज्ञमनुरक्तं निशाचरम्}
{मन्त्रज्ञं च विधिज्ञं च श्लक्ष्णया परया गिरा} %6-38-2

\twolineshloka
{सुवेलं साधु शैलेन्द्रमिमं धातुशतैश्चितम्}
{अध्यारोहामहे सर्वे वत्स्यामोऽत्र निशामिमाम्} %6-38-3

\twolineshloka
{लङ्कां चालोकयिष्यामो निलयं तस्य रक्षसः}
{येन मे मरणान्ताय हृता भार्या दुरात्मना} %6-38-4

\twolineshloka
{येन धर्मो न विज्ञातो न वृत्तं न कुलं तथा}
{राक्षस्या नीचया बुद्ध्या येन तद् गर्हितं कृतम्} %6-38-5

\twolineshloka
{तस्मिन् मे वर्तते रोषः कीर्तिते राक्षसाधमे}
{यस्यापराधान्नीचस्य वधं द्रक्ष्यामि रक्षसाम्} %6-38-6

\twolineshloka
{एको हि कुरुते पापं कालपाशवशं गतः}
{नीचेनात्मापचारेण कुलं तेन विनश्यति} %6-38-7

\twolineshloka
{एवं सम्मन्त्रयन् नेव सक्रोधो रावणं प्रति}
{रामः सुवेलं वासाय चित्रसानुमुपारुहत्} %6-38-8

\twolineshloka
{पृष्ठतो लक्ष्मणश्चैनमन्वगच्छत् समाहितः}
{सशरं चापमुद्यम्य सुमहद्विक्रमे रतः} %6-38-9

\twolineshloka
{तमन्वारोहत् सुग्रीवः सामात्यः सविभीषणः}
{हनुमानङ्गदो नीलो मैन्दो द्विविद एव च} %6-38-10

\twolineshloka
{गजो गवाक्षो गवयः शरभो गन्धमादनः}
{पनसः कुमुदश्चैव हरो रम्भश्च यूथपः} %6-38-11

\twolineshloka
{जाम्बवांश्च सुषेणश्च ऋषभश्च महामतिः}
{दुर्मुखश्च महातेजास्तथा शतवलिः कपिः} %6-38-12

\twolineshloka
{एते चान्ये च बहवो वानराः शीघ्रगामिनः}
{ते वायुवेगप्रवणास्तं गिरिं गिरिचारिणः} %6-38-13

\twolineshloka
{अध्यारोहन्त शतशः सुवेलं यत्र राघवः}
{ते त्वदीर्घेण कालेन गिरिमारुह्य सर्वतः} %6-38-14

\twolineshloka
{ददृशुः शिखरे तस्य विषक्तामिव खे पुरीम्}
{तां शुभां प्रवरद्वारां प्राकारवरशोभिताम्} %6-38-15

\twolineshloka
{लङ्कां राक्षससम्पूर्णां ददृशुर्हरियूथपाः}
{प्राकारवरसंस्थैश्च तथा नीलैश्च राक्षसैः} %6-38-16

\onelineshloka
{ददृशुस्ते हरिश्रेष्ठाः प्राकारमपरं कृतम्} %6-38-17

\twolineshloka
{ते दृष्ट्वा वानराः सर्वे राक्षसान् युद्धकाङ्क्षिणः}
{मुमुचुर्विविधान् नादांस्तस्य रामस्य पश्यतः} %6-38-18

\twolineshloka
{ततोऽस्तमगमत् सूर्यः सन्ध्यया प्रतिरञ्जितः}
{पूर्णचन्द्रप्रदीप्ता च क्षपा समतिवर्तत} %6-38-19

\twolineshloka
{ततः स रामो हरिवाहिनीपतिर्विभीषणेन प्रतिनन्द्य सत्कृतः}
{सलक्ष्मणो यूथपयूथसंयुतः सुवेलपृष्ठे न्यवसद् यथासुखम्} %6-38-20


॥इत्यार्षे श्रीमद्रामायणे वाल्मीकीये आदिकाव्ये युद्धकाण्डे सुवेलारोहणम् नाम अष्टात्रिंशः सर्गः ॥६-३८॥
