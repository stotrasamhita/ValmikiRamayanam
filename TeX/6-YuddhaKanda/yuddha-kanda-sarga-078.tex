\sect{अष्टसप्ततितमः सर्गः — मकराक्षाभिषेणनम्}

\twolineshloka
{निकुम्भं निहतं श्रुत्वा कुम्भं च विनिपातितम्}
{रावणः परमामर्षी प्रजज्वालानलो यथा} %6-78-1

\twolineshloka
{नैर्ऋतः क्रोधशोकाभ्यां द्वाभ्यां तु परिमूर्च्छितः}
{खरपुत्रं विशालाक्षं मकराक्षमचोदयत्} %6-78-2

\twolineshloka
{गच्छ पुत्र मयाऽऽज्ञप्तो बलेनाभिसमन्वितः}
{राघवं लक्ष्मणं चैव जहि तौ सवनौकसौ} %6-78-3

\twolineshloka
{रावणस्य वचः श्रुत्वा शूरमानी खरात्मजः}
{बाढमित्यब्रवीद्धृष्टो मकराक्षो निशाचरम्} %6-78-4

\twolineshloka
{सोऽभिवाद्य दशग्रीवं कृत्वा चापि प्रदक्षिणम्}
{निर्जगाम गृहाच्छुभ्राद् रावणस्याज्ञया बली} %6-78-5

\twolineshloka
{समीपस्थं बलाध्यक्षं खरपुत्रोऽब्रवीद् वचः}
{रथमानीयतां तूर्णं सैन्यं त्वानीयतां त्वरात्} %6-78-6

\twolineshloka
{तस्य तद् वचनं श्रुत्वा बलाध्यक्षो निशाचरः}
{स्यन्दनं च बलं चैव समीपं प्रत्यपादयत्} %6-78-7

\twolineshloka
{प्रदक्षिणं रथं कृत्वा समारुह्य निशाचरः}
{सूतं संचोदयामास शीघ्रं वै रथमावह} %6-78-8

\twolineshloka
{अथ तान् राक्षसान् सर्वान् मकराक्षोऽब्रवीदिदम्}
{यूयं सर्वे प्रयुध्यध्वं पुरस्तान्मम राक्षसाः} %6-78-9

\twolineshloka
{अहं राक्षसराजेन रावणेन महात्मना}
{आज्ञप्तः समरे हन्तुं तावुभौ रामलक्ष्मणौ} %6-78-10

\twolineshloka
{अद्य रामं वधिष्यामि लक्ष्मणं च निशाचराः}
{शाखामृगं च सुग्रीवं वानरांश्च शरोत्तमैः} %6-78-11

\twolineshloka
{अद्य शूलनिपातैश्च वानराणां महाचमूम्}
{प्रदहिष्यामि सम्प्राप्तां शुष्केन्धनमिवानलः} %6-78-12

\twolineshloka
{मकराक्षस्य तच्छ्रुत्वा वचनं ते निशाचराः}
{सर्वे नानायुधोपेता बलवन्तः समाहिताः} %6-78-13

\twolineshloka
{ते कामरूपिणः क्रूरा दंष्ट्रिणः पिङ्गलेक्षणाः}
{मातंगा इव नर्दन्तो ध्वस्तकेशा भयावहाः} %6-78-14

\twolineshloka
{परिवार्य महाकाया महाकायं खरात्मजम्}
{अभिजग्मुस्ततो हृष्टाश्चालयन्तो वसुन्धराम्} %6-78-15

\twolineshloka
{शङ्खभेरीसहस्राणामाहतानां समन्ततः}
{क्ष्वेलितास्फोटितानां च तत्र शब्दो महानभूत्} %6-78-16

\twolineshloka
{प्रभ्रष्टोऽथ करात् तस्य प्रतोदः सारथेस्तदा}
{पपात सहसा दैवाद् ध्वजस्तस्य तु रक्षसः} %6-78-17

\twolineshloka
{तस्य ते रथसंयुक्ता हया विक्रमवर्जिताः}
{चरणैराकुलैर्गत्वा दीनाः सास्रमुखा ययुः} %6-78-18

\twolineshloka
{प्रवाति पवनस्तस्मिन् सपांसुः खरदारुणः}
{निर्याणे तस्य रौद्रस्य मकराक्षस्य दुर्मतेः} %6-78-19

\twolineshloka
{तानि दृष्ट्वा निमित्तानि राक्षसा वीर्यवत्तमाः}
{अचिन्त्य निर्गताः सर्वे यत्र तौ रामलक्ष्मणौ} %6-78-20

\twolineshloka
{घनगजमहिषाङ्गतुल्यवर्णाः समरमुखेष्वसकृद्गदासिभिन्नाः}
{अहमहमिति युद्धकौशलास्ते रजनिचराः परिबभ्रमुर्मुहुस्ते} %6-78-21


॥इत्यार्षे श्रीमद्रामायणे वाल्मीकीये आदिकाव्ये युद्धकाण्डे मकराक्षाभिषेणनम् नाम अष्टसप्ततितमः सर्गः ॥६-७८॥
