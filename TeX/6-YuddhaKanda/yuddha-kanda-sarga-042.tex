\sect{द्विचत्वारिंशः सर्गः — युद्धारम्भः}

\twolineshloka
{ततस्ते राक्षसास्तत्र गत्वा रावणमन्दिरम्}
{न्यवेदयन् पुरीं रुद्धां रामेण सह वानरैः} %6-42-1

\twolineshloka
{रुद्धां तु नगरीं श्रुत्वा जातक्रोधो निशाचरः}
{विधानं द्विगुणं कृत्वा प्रासादं चाप्यरोहत} %6-42-2

\twolineshloka
{स ददर्श वृतां लङ्कां सशैलवनकाननाम्}
{असंख्येयैर्हरिगणैः सर्वतो युद्धकाङ्क्षिभिः} %6-42-3

\twolineshloka
{स दृष्ट्वा वानरैः सर्वैर्वसुधां कपिलीकृताम्}
{कथं क्षपयितव्याः स्युरिति चिन्तापरोऽभवत्} %6-42-4

\twolineshloka
{स चिन्तयित्वा सुचिरं धैर्यमालम्ब्य रावणः}
{राघवं हरियूथांश्च ददर्शायतलोचनः} %6-42-5

\twolineshloka
{राघवः सह सैन्येन मुदितो नाम पुप्लुवे}
{लङ्कां ददर्श गुप्तां वै सर्वतो राक्षसैर्वृताम्} %6-42-6

\twolineshloka
{दृष्ट्वा दाशरथिर्लङ्कां चित्रध्वजपताकिनीम्}
{जगाम सहसा सीतां दूयमानेन चेतसा} %6-42-7

\twolineshloka
{अत्र सा मृगशावाक्षी मत्कृते जनकात्मजा}
{पीड्यते शोकसंतप्ता कृशा स्थण्डिलशायिनी} %6-42-8

\twolineshloka
{निपीड्यमानां धर्मात्मा वैदेहीमनुचिन्तयन्}
{क्षिप्रमाज्ञापयद् रामो वानरान् द्विषतां वधे} %6-42-9

\twolineshloka
{एवमुक्ते तु वचसि रामेणाक्लिष्टकर्मणा}
{संघर्षमाणाः प्लवगाः सिंहनादैरनादयन्} %6-42-10

\twolineshloka
{शिखरैर्विकिरामैतां लङ्कां मुष्टिभिरेव वा}
{इति स्म दधिरे सर्वे मनांसि हरियूथपाः} %6-42-11

\twolineshloka
{उद्यम्य गिरिशृङ्गाणि महान्ति शिखराणि च}
{तरूंश्चोत्पाट्य विविधांस्तिष्ठन्ति हरियूथपाः} %6-42-12

\twolineshloka
{प्रेक्षतो राक्षसेन्द्रस्य तान्यनीकानि भागशः}
{राघवप्रियकामार्थं लङ्कामारुरुहुस्तदा} %6-42-13

\twolineshloka
{ते ताम्रवक्त्रा हेमाभा रामार्थे त्यक्तजीविताः}
{लङ्कामेवाभ्यवर्तन्त सालभूधरयोधिनः} %6-42-14

\twolineshloka
{ते द्रुमैः पर्वताग्रैश्च मुष्टिभिश्च प्लवंगमाः}
{प्राकाराग्राण्यसंख्यानि ममन्थुस्तोरणानि च} %6-42-15

\twolineshloka
{परिखान् पूरयन्तश्च प्रसन्नसलिलाशयान्}
{पांसुभिः पर्वताग्रैश्च तृणैः काष्ठैश्च वानराः} %6-42-16

\twolineshloka
{ततः सहस्रयूथाश्च कोटियूथाश्च यूथपाः}
{कोटियूथशताश्चान्ये लङ्कामारुरुहुस्तदा} %6-42-17

\twolineshloka
{काञ्चनानि प्रमर्दन्तस्तोरणानि प्लवंगमाः}
{कैलासशिखराग्राणि गोपुराणि प्रमथ्य च} %6-42-18

\twolineshloka
{आप्लवन्तः प्लवन्तश्च गर्जन्तश्च प्लवंगमाः}
{लङ्कां तामभिधावन्ति महावारणसंनिभाः} %6-42-19

\twolineshloka
{जयत्युरुबलो रामो लक्ष्मणश्च महाबलः}
{राजा जयति सुग्रीवो राघवेणाभिपालितः} %6-42-20

\twolineshloka
{इत्येवं घोषयन्तश्च गर्जन्तश्च प्लवंगमाः}
{अभ्यधावन्त लङ्कायाः प्राकारं कामरूपिणः} %6-42-21

\threelineshloka
{वीरबाहुः सुबाहुश्च नलश्च पनसस्तथा}
{निपीड्योपनिविष्टास्ते प्राकारं हरियूथपाः}
{एतस्मिन्नन्तरे चक्रुः स्कन्धावारनिवेशनम्} %6-42-22

\twolineshloka
{पूर्वद्वारं तु कुमुदः कोटिभिर्दशभिर्वृतः}
{आवृत्य बलवांस्तस्थौ हरिभिर्जितकाशिभिः} %6-42-23

\twolineshloka
{सहायार्थे तु तस्यैव निविष्टः प्रघसो हरिः}
{पनसश्च महाबाहुर्वानरैरभिसंवृतः} %6-42-24

\twolineshloka
{दक्षिणद्वारमासाद्य वीरः शतबलिः कपिः}
{आवृत्य बलवांस्तस्थौ विंशत्या कोटिभिर्वृतः} %6-42-25

\twolineshloka
{सुषेणः पश्चिमद्वारं गत्वा तारापिता बली}
{आवृत्य बलवांस्तस्थौ कोटिकोटिभिरावृतः} %6-42-26

\twolineshloka
{उत्तरद्वारमागम्य रामः सौमित्रिणा सह}
{आवृत्य बलवांस्तस्थौ सुग्रीवश्च हरीश्वरः} %6-42-27

\twolineshloka
{गोलाङ्गूलो महाकायो गवाक्षो भीमदर्शना}
{वृतः कोट्या महावीर्यस्तस्थौ रामस्य पार्श्वतः} %6-42-28

\twolineshloka
{ऋक्षाणां भीमकोपानां धूम्रः शत्रुनिबर्हणः}
{वृतः कोट्या महावीर्यस्तस्थौ रामस्य पार्श्वतः} %6-42-29

\twolineshloka
{संनद्धस्तु महावीर्यो गदापाणिर्विभीषणः}
{वृतो यत्तैस्तु सचिवैस्तस्थौ यत्र महाबलः} %6-42-30

\twolineshloka
{गजो गवाक्षो गवयः शरभो गन्धमादनः}
{समन्तात् परिधावन्तो ररक्षुर्हरिवाहिनीम्} %6-42-31

\twolineshloka
{ततः कोपपरीतात्मा रावणो राक्षसेश्वरः}
{निर्याणं सर्वसैन्यानां द्रुतमाज्ञापयत् तदा} %6-42-32

\twolineshloka
{एतच्छ्रुत्वा तदा वाक्यं रावणस्य मुखेरितम्}
{सहसा भीमनिर्घोषमुद्घुष्टं रजनीचरैः} %6-42-33

\twolineshloka
{ततः प्रबोधिता भेर्यश्चन्द्रपाण्डुरपुष्कराः}
{हेमकोणैरभिहता राक्षसानां समन्ततः} %6-42-34

\twolineshloka
{विनेदुश्च महाघोषाः शङ्खाः शतसहस्रशः}
{राक्षसानां सुघोराणां मुखमारुतपूरिताः} %6-42-35

\twolineshloka
{ते बभुः शुभनीलाङ्गाः सशङ्खा रजनीचराः}
{विद्युन्मण्डलसंनद्धाः सबलाका इवाम्बुदाः} %6-42-36

\twolineshloka
{निष्पतन्ति ततः सैन्या हृष्टा रावणचोदिताः}
{समये पूर्यमाणस्य वेगा इव महोदधेः} %6-42-37

\twolineshloka
{ततो वानरसैन्येन मुक्तो नादः समन्ततः}
{मलयः पूरितो येन ससानुप्रस्थकन्दरः} %6-42-38

\twolineshloka
{शङ्खदुन्दुभिनिर्घोषः सिंहनादस्तरस्विनाम्}
{पृथिवीं चान्तरिक्षं च सागरं चाभ्यनादयत्} %6-42-39

\twolineshloka
{गजानां बृंहितैः सार्धं हयानां ह्रेषितैरपि}
{रथानां नेमिनिर्घोषै रक्षसां वदनस्वनैः} %6-42-40

\twolineshloka
{एतस्मिन्नन्तरे घोरः संग्रामः समपद्यत}
{रक्षसां वानराणां च यथा देवासुरे पुरा} %6-42-41

\twolineshloka
{ते गदाभिः प्रदीप्ताभिः शक्तिशूलपरश्वधैः}
{निजघ्नुर्वानरान् सर्वान् कथयन्तः स्वविक्रमान्} %6-42-42

\twolineshloka
{तथा वृक्षैर्महाकायाः पर्वताग्रैश्च वानराः}
{निजघ्नुस्तानि रक्षांसि नखैर्दन्तैश्च वेगिनः} %6-42-43

\twolineshloka
{राजा जयति सुग्रीव इति शब्दो महानभूत्}
{राजञ्जयजयेत्युक्त्वा स्वस्वनामकथां ततः} %6-42-44

\twolineshloka
{राक्षसास्त्वपरे भीमाः प्राकारस्था महीं गतान्}
{वानरान् भिन्दिपालैश्च शूलैश्चैव व्यदारयन्} %6-42-45

\twolineshloka
{वानराश्चापि संक्रुद्धाः प्राकारस्थान् महीं गताः}
{राक्षसान् पातयामासुः खमाप्लुत्य स्वबाहुभिः} %6-42-46

\twolineshloka
{स सम्प्रहारस्तुमुलो मांसशोणितकर्दमः}
{रक्षसां वानराणां च सम्बभूवाद्भुतोपमः} %6-42-47


॥इत्यार्षे श्रीमद्रामायणे वाल्मीकीये आदिकाव्ये युद्धकाण्डे युद्धारम्भः नाम द्विचत्वारिंशः सर्गः ॥६-४२॥
