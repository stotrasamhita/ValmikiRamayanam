\sect{त्रयोदशाधिकशततमः सर्गः — रावाणान्तःपुरपरिदेवनम्}

\twolineshloka
{रावणं निहतं श्रुत्वा राघवेण महात्मना}
{अन्तःपुराद् विनिष्पेतू राक्षस्यः शोककर्शिताः} %6-113-1

\twolineshloka
{वार्यमाणाः सुबहुशो वेष्टन्त्यः क्षितिपांसुषु}
{विमुक्तकेश्यः शोकार्ता गावो वत्सहता इव} %6-113-2

\twolineshloka
{उत्तरेण विनिष्क्रम्य द्वारेण सह राक्षसैः}
{प्रविश्यायोधनं घोरं विचिन्वन्त्यो हतं पतिम्} %6-113-3

\twolineshloka
{आर्यपुत्रेति वादिन्यो हा नाथेति च सर्वशः}
{परिपेतुः कबन्धाङ्कां महीं शोणितकर्दमाम्} %6-113-4

\twolineshloka
{ता बाष्पपरिपूर्णाक्ष्यो भर्तृशोकपराजिताः}
{करिण्य इव नर्दन्त्यः करेण्वो हतयूथपाः} %6-113-5

\twolineshloka
{ददृशुस्ता महाकायं महावीर्यं महाद्युतिम्}
{रावणं निहतं भूमौ नीलाञ्जनचयोपमम्} %6-113-6

\twolineshloka
{ताः पतिं सहसा दृष्ट्वा शयानं रणपांसुषु}
{निपेतुस्तस्य गात्रेषु च्छिन्ना वनलता इव} %6-113-7

\twolineshloka
{बहुमानात् परिष्वज्य काचिदेनं रुरोद ह}
{चरणौ काचिदालम्ब्य काचित् कण्ठेऽवलम्ब्य च} %6-113-8

\twolineshloka
{उत्क्षिप्य च भुजौ काचिद् भूमौ सुपरिवर्तते}
{हतस्य वदनं दृष्ट्वा काचिन्मोहमुपागमत्} %6-113-9

\twolineshloka
{काचिदङ्के शिरः कृत्वा रुरोद मुखमीक्षती}
{स्नापयन्ती मुखं बाष्पैस्तुषारैरिव पङ्कजम्} %6-113-10

\twolineshloka
{एवमार्ताः पतिं दृष्ट्वा रावणं निहतं भुवि}
{चुक्रुशुर्बहुधा शोकाद् भूयस्ताः पर्यदेवयन्} %6-113-11

\twolineshloka
{येन वित्रासितः शक्रो येन वित्रासितो यमः}
{येन वैश्रवणो राजा पुष्पकेण वियोजितः} %6-113-12

\twolineshloka
{गन्धर्वाणामृषीणां च सुराणां च महात्मनाम्}
{भयं येन रणे दत्तं सोऽयं शेते रणे हतः} %6-113-13

\twolineshloka
{असुरेभ्यः सुरेभ्यो वा पन्नगेभ्योऽपि वा तथा}
{भयं यो न विजानाति तस्येदं मानुषाद् भयम्} %6-113-14

\twolineshloka
{अवध्यो देवतानां यस्तथा दानवरक्षसाम्}
{हतः सोऽयं रणे शेते मानुषेण पदातिना} %6-113-15

\twolineshloka
{यो न शक्यः सुरैर्हन्तुं न यक्षैर्नासुरैस्तथा}
{सोऽयं कश्चिदिवासत्त्वो मृत्युं मर्त्येन लम्भितः} %6-113-16

\twolineshloka
{एवं वदन्त्यो रुरुदुस्तस्य ता दुःखिताः स्त्रियः}
{भूय एव च दुःखार्ता विलेपुश्च पुनः पुनः} %6-113-17

\threelineshloka
{अशृण्वता तु सुहृदां सततं हितवादिनाम्}
{मरणायाहृता सीता राक्षसाश्च निपातिताः}
{एताः सममिदानीं ते वयमात्मा च पातितः} %6-113-18

\twolineshloka
{ब्रुवाणोऽपि हितं वाक्यमिष्टो भ्राता विभीषणः}
{दृष्टं परुषितो मोहात् त्वयाऽऽत्मवधकाङ्क्षिणा} %6-113-19

\twolineshloka
{यदि निर्यातिता ते स्यात् सीता रामाय मैथिली}
{न नः स्याद् व्यसनं घोरमिदं मूलहरं महत्} %6-113-20

\twolineshloka
{वृत्तकामो भवेद् भ्राता रामो मित्रकुलं भवेत्}
{वयं चाविधवाः सर्वाः सकामा न च शत्रवः} %6-113-21

\twolineshloka
{त्वया पुनर्नृशंसेन सीतां संरुन्धता बलात्}
{राक्षसा वयमात्मा च त्रयं तुल्यं निपातितम्} %6-113-22

\twolineshloka
{न कामकारः कामं वा तव राक्षसपुङ्गव}
{दैवं चेष्टयते सर्वं हतं दैवेन हन्यते} %6-113-23

\twolineshloka
{वानराणां विनाशोऽयं राक्षसानां च ते रणे}
{तव चैव महाबाहो दैवयोगादुपागतः} %6-113-24

\twolineshloka
{नैवार्थेन च कामेन विक्रमेण न चाज्ञया}
{शक्या दैवगतिर्लोके निवर्तयितुमुद्यता} %6-113-25

\twolineshloka
{विलेपुरेवं दीनास्ता राक्षसाधिपयोषितः}
{कुरर्य इव दुःखार्ता बाष्पपर्याकुलेक्षणाः} %6-113-26


॥इत्यार्षे श्रीमद्रामायणे वाल्मीकीये आदिकाव्ये युद्धकाण्डे रावाणान्तःपुरपरिदेवनम् नाम त्रयोदशाधिकशततमः सर्गः ॥६-११३॥
