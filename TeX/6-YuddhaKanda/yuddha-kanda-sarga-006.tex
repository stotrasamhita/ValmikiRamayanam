\sect{षष्ठः सर्गः — रावणमन्त्रणम्}

\threelineshloka
{लङ्कायां तु कृतं कर्म घोरं दृष्ट्वा भयावहम्}
{राक्षसेन्द्रो हनुमता शक्रेणेव महात्मना}
{अब्रवीद् राक्षसान् सर्वान् ह्रिया किंचिदवाङ्मुखः} %6-6-1

\twolineshloka
{धर्षिता च प्रविष्टा च लङ्का दुष्प्रसहा पुरी}
{तेन वानरमात्रेण दृष्टा सीता च जानकी} %6-6-2

\twolineshloka
{प्रासादो धर्षितश्चैत्यः प्रवरा राक्षसा हताः}
{आविला च पुरी लङ्का सर्वा हनुमता कृता} %6-6-3

\twolineshloka
{किं करिष्यामि भद्रं वः किं वो युक्तमनन्तरम्}
{उच्यतां नः समर्थं यत् कृतं च सुकृतं भवेत्} %6-6-4

\twolineshloka
{मन्त्रमूलं च विजयं प्रवदन्ति मनस्विनः}
{तस्माद् वै रोचये मन्त्रं रामं प्रति महाबलाः} %6-6-5

\twolineshloka
{त्रिविधाः पुरुषा लोके उत्तमाधममध्यमाः}
{तेषां तु समवेतानां गुणदोषौ वदाम्यहम्} %6-6-6

\twolineshloka
{मन्त्रस्त्रिभिर्हि संयुक्तः समर्थैर्मन्त्रनिर्णये}
{मित्रैर्वापि समानार्थैर्बान्धवैरपि वाधिकैः} %6-6-7

\twolineshloka
{सहितो मन्त्रयित्वा यः कर्मारम्भान् प्रवर्तयेत्}
{दैवे च कुरुते यत्नं तमाहुः पुरुषोत्तमम्} %6-6-8

\twolineshloka
{एकोऽर्थं विमृशेदेको धर्मे प्रकुरुते मनः}
{एकः कार्याणि कुरुते तमाहुर्मध्यमं नरम्} %6-6-9

\twolineshloka
{गुणदोषौ न निश्चित्य त्यक्त्वा दैवव्यपाश्रयम्}
{करिष्यामीति यः कार्यमुपेक्षेत् स नराधमः} %6-6-10

\twolineshloka
{यथेमे पुरुषा नित्यमुत्तमाधममध्यमाः}
{एवं मन्त्रोऽपि विज्ञेय उत्तमाधममध्यमः} %6-6-11

\twolineshloka
{ऐकमत्यमुपागम्य शास्त्रदृष्टेन चक्षुषा}
{मन्त्रिणो यत्र निरतास्तमाहुर्मन्त्रमुत्तमम्} %6-6-12

\twolineshloka
{बह्वीरपि मतीर्गत्वा मन्त्रिणामर्थनिर्णयः}
{पुनर्यत्रैकतां प्राप्तः स मन्त्रो मध्यमः स्मृतः} %6-6-13

\twolineshloka
{अन्योन्यमतिमास्थाय यत्र सम्प्रतिभाष्यते}
{न चैकमत्ये श्रेयोऽस्ति मन्त्रः सोऽधम उच्यते} %6-6-14

\twolineshloka
{तस्मात् सुमन्त्रितं साधु भवन्तो मतिसत्तमाः}
{कार्यं सम्प्रतिपद्यन्तमेतत् कृत्यं मतं मम} %6-6-15

\twolineshloka
{वानराणां हि धीराणां सहस्रैः परिवारितः}
{रामोऽभ्येति पुरीं लङ्कामस्माकमुपरोधकः} %6-6-16

\twolineshloka
{तरिष्यति च सुव्यक्तं राघवः सागरं सुखम्}
{तरसा युक्तरूपेण सानुजः सबलानुगः} %6-6-17

\threelineshloka
{समुद्रमुच्छोषयति वीर्येणान्यत्करोति वा}
{तस्मिन्नेवंविधे कार्ये विरुद्धे वानरैः सह}
{हितं पुरे च सैन्ये च सर्वं सम्मन्त्र्यतां मम} %6-6-18


॥इत्यार्षे श्रीमद्रामायणे वाल्मीकीये आदिकाव्ये युद्धकाण्डे रावणमन्त्रणम् नाम षष्ठः सर्गः ॥६-६॥
