\sect{एकोनत्रिंशः सर्गः — शार्दूलादिचारप्रेषणम्}

\twolineshloka
{शुकेन तु समादिष्टान् दृष्ट्वा स हरियूथपान्}
{लक्ष्मणं च महावीर्यं भुजं रामस्य दक्षिणम्} %6-29-1

\twolineshloka
{समीपस्थं च रामस्य भ्रातरं च विभीषणम्}
{सर्ववानरराजं च सुग्रीवं भीमविक्रमम्} %6-29-2

\twolineshloka
{अङ्गदं चापि बलिनं वज्रहस्तात्मजात्मजम्}
{हनूमन्तं च विक्रान्तं जाम्बवन्तं च दुर्जयम्} %6-29-3

\twolineshloka
{सुषेणं कुमुदं नीलं नलं च प्लवगर्षभम्}
{गजं गवाक्षं शरभं मैन्दं च द्विविदं तथा} %6-29-4

\twolineshloka
{किंचिदाविग्नहृदयो जातक्रोधश्च रावणः}
{भर्त्सयामास तौ वीरौ कथान्ते शुकसारणौ} %6-29-5

\twolineshloka
{अधोमुखौ तौ प्रणतावब्रवीच्छुकसारणौ}
{रोषगद्गदया वाचा संरब्धं परुषं तथा} %6-29-6

\twolineshloka
{न तावत् सदृशं नाम सचिवैरुपजीविभिः}
{विप्रियं नृपतेर्वक्तुं निग्रहे प्रग्रहे प्रभोः} %6-29-7

\twolineshloka
{रिपूणां प्रतिकूलानां युद्धार्थमभिवर्तताम्}
{उभाभ्यां सदृशं नाम वक्तुमप्रस्तवे स्तवम्} %6-29-8

\twolineshloka
{आचार्या गुरवो वृद्धा वृथा वां पर्युपासिताः}
{सारं यद् राजशास्त्राणामनुजीव्यं न गृह्यते} %6-29-9

\twolineshloka
{गृहीतो वा न विज्ञातो भारोऽज्ञानस्य वाह्यते}
{ईदृशैः सचिवैर्युक्तो मूर्खैर्दिष्ट्या धराम्यहम्} %6-29-10

\twolineshloka
{किं नु मृत्योर्भयं नास्ति मां वक्तुं परुषं वचः}
{यस्य मे शासतो जिह्वा प्रयच्छति शुभाशुभम्} %6-29-11

\twolineshloka
{अप्येव दहनं स्पृष्ट्वा वने तिष्ठन्ति पादपाः}
{राजदण्डपरामृष्टास्तिष्ठन्ते नापराधिनः} %6-29-12

\twolineshloka
{हन्यामहं त्विमौ पापौ शत्रुपक्षप्रशंसिनौ}
{यदि पूर्वोपकारैर्मे क्रोधो न मृदुतां व्रजेत्} %6-29-13

\threelineshloka
{अपध्वंसत नश्यध्वं संनिकर्षादितो मम}
{नहि वां हन्तुमिच्छामि स्मराम्युपकृतानि वाम्}
{हतावेव कृतघ्नौ द्वौ मयि स्नेहपराङ्मुखौ} %6-29-14

\twolineshloka
{एवमुक्तौ तु सव्रीडौ तौ दृष्ट्वा शुकसारणौ}
{रावणं जयशब्देन प्रतिनन्द्याभिनिःसृतौ} %6-29-15

\threelineshloka
{अब्रवीच्च दशग्रीवः समीपस्थं महोदरम्}
{उपस्थापय मे शीघ्रं चारानिति निशाचरः}
{महोदरस्तथोक्तस्तु शीघ्रमाज्ञापयच्चरान्} %6-29-16

\twolineshloka
{ततश्चाराः संत्वरिताः प्राप्ताः पार्थिवशासनात्}
{उपस्थिताः प्राञ्जलयो वर्धयित्वा जयाशिषः} %6-29-17

\twolineshloka
{तानब्रवीत् ततो वाक्यं रावणो राक्षसाधिपः}
{चारान् प्रत्यायिकान् शूरान् धीरान् विगतसाध्वसान्} %6-29-18

\twolineshloka
{इतो गच्छत रामस्य व्यवसायं परीक्षितुम्}
{मन्त्रेष्वभ्यन्तरा येऽस्य प्रीत्या तेन समागताः} %6-29-19

\twolineshloka
{कथं स्वपिति जागर्ति किमद्य च करिष्यति}
{विज्ञाय निपुणं सर्वमागन्तव्यमशेषतः} %6-29-20

\twolineshloka
{चारेण विदितः शत्रुः पण्डितैर्वसुधाधिपैः}
{युद्धे स्वल्पेन यत्नेन समासाद्य निरस्यते} %6-29-21

\twolineshloka
{चारास्तु ते तथेत्युक्त्वा प्रहृष्टा राक्षसेश्वरम्}
{शार्दूलमग्रतः कृत्वा ततश्चक्रुः प्रदक्षिणम्} %6-29-22

\twolineshloka
{ततस्तं तु महात्मानं चारा राक्षससत्तमम्}
{कृत्वा प्रदक्षिणं जग्मुर्यत्र रामः सलक्ष्मणः} %6-29-23

\twolineshloka
{ते सुवेलस्य शैलस्य समीपे रामलक्ष्मणौ}
{प्रच्छन्ना ददृशुर्गत्वा ससुग्रीवविभीषणौ} %6-29-24

\twolineshloka
{प्रेक्षमाणाश्चमूं तां च बभूवुर्भयविह्वलाः}
{ते तु धर्मात्मना दृष्टा राक्षसेन्द्रेण राक्षसाः} %6-29-25

\twolineshloka
{विभीषणेन तत्रस्था निगृहीता यदृच्छया}
{शार्दूलो ग्राहितस्त्वेकः पापोऽयमिति राक्षसः} %6-29-26

\twolineshloka
{मोचितः सोऽपि रामेण वध्यमानः प्लवंगमैः}
{आनृशंस्येन रामेण मोचिता राक्षसाः परे} %6-29-27

\twolineshloka
{वानरैरर्दितास्ते तु विक्रान्तैर्लघुविक्रमैः}
{पुनर्लङ्कामनुप्राप्ताः श्वसन्तो नष्टचेतसः} %6-29-28

\twolineshloka
{ततो दशग्रीवमुपस्थितास्ते चारा बहिर्नित्यचरा निशाचराः}
{गिरेः सुवेलस्य समीपवासिनं न्यवेदयन् रामबलं महाबलाः} %6-29-29


॥इत्यार्षे श्रीमद्रामायणे वाल्मीकीये आदिकाव्ये युद्धकाण्डे शार्दूलादिचारप्रेषणम् नाम एकोनत्रिंशः सर्गः ॥६-२९॥
