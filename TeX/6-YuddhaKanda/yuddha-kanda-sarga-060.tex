\sect{षष्ठितमः सर्गः — कुम्भकर्णप्रबोधः}

\twolineshloka
{स प्रविश्य पुरीं लङ्कां रामबाणभयार्दितः}
{भग्नदर्पस्तदा राजा बभूव व्यथितेन्द्रियः} %6-60-1

\twolineshloka
{मातंग इव सिंहेन गरुडेनेव पन्नगः}
{अभिभूतोऽभवद् राजा राघवेण महात्मना} %6-60-2

\twolineshloka
{ब्रह्मदण्डप्रतीकानां विद्युच्चलितवर्चसाम्}
{स्मरन् राघवबाणानां विव्यथे राक्षसेश्वरः} %6-60-3

\twolineshloka
{स काञ्चनमयं दिव्यमाश्रित्य परमासनम्}
{विप्रेक्षमाणो रक्षांसि रावणो वाक्यमब्रवीत्} %6-60-4

\twolineshloka
{सर्वं तत् खलु मे मोघं यत् तप्तं परमं तपः}
{यत् समानो महेन्द्रेण मानुषेण विनिर्जितः} %6-60-5

\twolineshloka
{इदं तद् ब्रह्मणो घोरं वाक्यं मामभ्युपस्थितम्}
{मानुषेभ्यो विजानीहि भयं त्वमिति तत्तथा} %6-60-6

\twolineshloka
{देवदानवगन्धर्वैर्यक्षराक्षसपन्नगैः}
{अवध्यत्वं मया प्रोक्तं मानुषेभ्यो न याचितम्} %6-60-7

\twolineshloka
{तमिमं मानुषं मन्ये रामं दशरथात्मजम्}
{इक्ष्वाकुकुलजातेन अनरण्येन यत् पुरा} %6-60-8

\twolineshloka
{उत्पत्स्यति हि मद्वंशपुरुषो राक्षसाधम}
{यस्त्वां सपुत्रं सामात्यं सबलं साश्वसारथिम्} %6-60-9

\twolineshloka
{निहनिष्यति संग्रामे त्वां कुलाधम दुर्मते}
{शप्तोऽहं वेदवत्या च यथा सा धर्षिता पुरा} %6-60-10

\twolineshloka
{सेयं सीता महाभागा जाता जनकनन्दिनी}
{उमा नन्दीश्वरश्चापि रम्भा वरुणकन्यका} %6-60-11

\twolineshloka
{यथोक्तास्तन्मया प्राप्तं न मिथ्या ऋषिभाषितम्}
{एतदेव समागम्य यत्नं कर्तुमिहार्हथ} %6-60-12

\twolineshloka
{राक्षसाश्चापि तिष्ठन्तु चर्यागोपुरमूर्धसु}
{स चाप्रतिमगाम्भीर्यो देवदानवदर्पहा} %6-60-13

\twolineshloka
{ब्रह्मशापाभिभूतस्तु कुम्भकर्णो विबोध्यताम्}
{समरे जितमात्मानं प्रहस्तं च निषूदितम्} %6-60-14

\twolineshloka
{ज्ञात्वा रक्षोबलं भीममादिदेश महाबलः}
{द्वारेषु यत्नः क्रियतां प्राकारश्चाधिरुह्यताम्} %6-60-15

\twolineshloka
{निद्रावशसमाविष्टः कुम्भकर्णो विबोध्यताम्}
{सुखं स्वपिति निश्चिन्तः कामोपहतचेतनः} %6-60-16

\twolineshloka
{नव सप्त दशाष्टौ च मासान् स्वपिति राक्षसः}
{मन्त्रं कृत्वा प्रसुप्तोऽयमितस्तु नवमेऽहनि} %6-60-17

\threelineshloka
{तं तु बोधयत क्षिप्रं कुम्भकर्णं महाबलम्}
{स हि संख्ये महाबाहुः ककुदं सर्वरक्षसाम्}
{वानरान् राजपुत्रौ च क्षिप्रमेव हनिष्यति} %6-60-18

\twolineshloka
{एष केतुः परं संख्ये मुख्यो वै सर्वरक्षसाम्}
{कुम्भकर्णः सदा शेते मूढो ग्राम्यसुखे रतः} %6-60-19

\twolineshloka
{रामेणाभिनिरस्तस्य संग्रामेऽस्मिन् सुदारुणे}
{भविष्यति न मे शोकः कुम्भकर्णे विबोधिते} %6-60-20

\twolineshloka
{किं करिष्याम्यहं तेन शक्रतुल्यबलेन हि}
{ईदृशे व्यसने घोरे यो न साह्याय कल्पते} %6-60-21

\twolineshloka
{ते तु तद् वचनं श्रुत्वा राक्षसेन्द्रस्य राक्षसाः}
{जग्मुः परमसम्भ्रान्ताः कुम्भकर्णनिवेशनम्} %6-60-22

\twolineshloka
{ते रावणसमादिष्टा मांसशोणितभोजनाः}
{गन्धं माल्यं महद्भक्ष्यमादाय सहसा ययुः} %6-60-23

\twolineshloka
{तां प्रविश्य महाद्वारां सर्वतो योजनायताम्}
{कुम्भकर्णगुहां रम्यां पुष्पगन्धप्रवाहिनीम्} %6-60-24

\twolineshloka
{कुम्भकर्णस्य निःश्वासादवधूता महाबलाः}
{प्रतिष्ठमानाः कृच्छ्रेण यत्नात् प्रविविशुर्गुहाम्} %6-60-25

\twolineshloka
{तां प्रविश्य गुहां रम्यां रत्नकाञ्चनकुट्टिमाम्}
{ददृशुर्नैर्ऋतव्याघ्राः शयानं भीमविक्रमम्} %6-60-26

\twolineshloka
{ते तु तं विकृतं सुप्तं विकीर्णमिव पर्वतम्}
{कुम्भकर्णं महानिद्रं समेताः प्रत्यबोधयन्} %6-60-27

\twolineshloka
{ऊर्ध्वलोमाञ्चिततनुं श्वसन्तमिव पन्नगम्}
{भ्रामयन्तं विनिःश्वासैः शयानं भीमविक्रमम्} %6-60-28

\twolineshloka
{भीमनासापुटं तं तु पातालविपुलाननम्}
{शयने न्यस्तसर्वाङ्गं मेदोरुधिरगन्धिनम्} %6-60-29

\twolineshloka
{काञ्चनाङ्गदनद्धाङ्गं किरीटेनार्कवर्चसम्}
{ददृशुर्नैर्ऋतव्याघ्रं कुम्भकर्णमरिंदमम्} %6-60-30

\twolineshloka
{ततश्चक्रुर्महात्मानः कुम्भकर्णस्य चाग्रतः}
{भूतानां मेरुसंकाशं राशिं परमतर्पणम्} %6-60-31

\twolineshloka
{मृगाणां महिषाणां च वराहाणां च संचयान्}
{चक्रुर्नैर्ऋतशार्दूला राशिमन्नस्य चाद्भुतम्} %6-60-32

\twolineshloka
{ततः शोणितकुम्भांश्च मांसानि विविधानि च}
{पुरस्तात् कुम्भकर्णस्य चक्रुस्त्रिदशशत्रवः} %6-60-33

\twolineshloka
{लिलिपुश्च परार्घ्येन चन्दनेन परंतपम्}
{दिव्यैराश्वासयामासुर्माल्यैर्गन्धैश्च गन्धिभिः} %6-60-34

\twolineshloka
{धूपगन्धांश्च ससृजुस्तुष्टुवुश्च परंतपम्}
{जलदा इव चानेदुर्यातुधानास्ततस्ततः} %6-60-35

\twolineshloka
{शङ्खांश्च पूरयामासुः शशाङ्कसदृशप्रभान्}
{तुमुलं युगपच्चापि विनेदुश्चाप्यमर्षिताः} %6-60-36

\twolineshloka
{नेदुरास्फोटयामासुश्चिक्षिपुस्ते निशाचराः}
{कुम्भकर्णविबोधार्थं चक्रुस्ते विपुलं स्वरम्} %6-60-37

\twolineshloka
{सशङ्खभेरीपणवप्रणादं सास्फोटितक्ष्वेलितसिंहनादम्}
{दिशो द्रवन्तस्त्रिदिवं किरन्तः श्रुत्वा विहंगाः सहसा निपेतुः} %6-60-38

\twolineshloka
{यदा भृशं तैर्निनदैर्महात्मा न कुम्भकर्णो बुबुधे प्रसुप्तः}
{ततो भुशुण्डीर्मुसलानि सर्वे रक्षोगणास्ते जगृहुर्गदाश्च} %6-60-39

\twolineshloka
{तं शैलशृङ्गैर्मुसलैर्गदाभिर्वक्षःस्थले मुद्गरमुष्टिभिश्च}
{सुखप्रसुप्तं भुवि कुम्भकर्णं रक्षांस्युदग्राणि तदा निजघ्नुः} %6-60-40

\twolineshloka
{तस्य निःश्वासवातेन कुम्भकर्णस्य रक्षसः}
{राक्षसाः कुम्भकर्णस्य स्थातुं शेकुर्न चाग्रतः} %6-60-41

\twolineshloka
{ततः परिहिता गाढं राक्षसा भीमविक्रमाः}
{मृदङ्गपणवान् भेरीः शङ्खकुम्भगणांस्तथा} %6-60-42

\twolineshloka
{दश राक्षससाहस्रं युगपत्पर्यवारयत्}
{नीलाञ्जनचयाकारं ते तु तं प्रत्यबोधयन्} %6-60-43

\twolineshloka
{अभिघ्नन्तो नदन्तश्च न च सम्बुबुधे तदा}
{यदा चैनं न शेकुस्ते प्रतिबोधयितुं तदा} %6-60-44

\twolineshloka
{ततो गुरुतरं यत्नं दारुणं समुपाक्रमन्}
{अश्वानुष्ट्रान् खरान् नागाञ्जघ्नुर्दण्डकशाङ्कुशैः} %6-60-45

\twolineshloka
{भेरीशङ्खमृदङ्गांश्च सर्वप्राणैरवादयन्}
{निजघ्नुश्चास्य गात्राणि महाकाष्ठकटंकरैः} %6-60-46

\threelineshloka
{मुद्गरैर्मुसलैश्चापि सर्वप्राणसमुद्यतैः}
{तेन नादेन महता लङ्का सर्वा प्रपूरिता}
{सपर्वतवना सर्वा सोऽपि नैव प्रबुध्यते} %6-60-47

\twolineshloka
{ततो भेरीसहस्रं तु युगपत् समहन्यत}
{मृष्टकाञ्चनकोणानामसक्तानां समन्ततः} %6-60-48

\twolineshloka
{एवमप्यतिनिद्रस्तु यदा नैव प्रबुध्यते}
{शापस्य वशमापन्नस्ततः क्रुद्धा निशाचराः} %6-60-49

\twolineshloka
{ततः कोपसमाविष्टाः सर्वे भीमपराक्रमाः}
{तद् रक्षो बोधयिष्यन्तश्चक्रुरन्ये पराक्रमम्} %6-60-50

\twolineshloka
{अन्ये भेरीः समाजघ्नुरन्ये चक्रुर्महास्वनम्}
{केशानन्ये प्रलुलुपुः कर्णानन्ये दशन्ति च} %6-60-51

\twolineshloka
{उदकुम्भशतानन्ये समसिञ्चन्त कर्णयोः}
{न कुम्भकर्णः पस्पन्दे महानिद्रावशं गतः} %6-60-52

\twolineshloka
{अन्ये च बलिनस्तस्य कूटमुद्गरपाणयः}
{मूर्ध्नि वक्षसि गात्रेषु पातयन् कूटमुद्गरान्} %6-60-53

\twolineshloka
{रज्जुबन्धनबद्धाभिः शतघ्नीभिश्च सर्वतः}
{वध्यमानो महाकायो न प्राबुध्यत राक्षसः} %6-60-54

\twolineshloka
{वारणानां सहस्रं च शरीरेऽस्य प्रधावितम्}
{कुम्भकर्णस्तदा बुद्ध्वा स्पर्शं परमबुध्यत} %6-60-55

\twolineshloka
{स पात्यमानैर्गिरिशृङ्गवृक्षैरचिन्तयंस्तान् विपुलान् प्रहारान्}
{निद्राक्षयात् क्षुद्भयपीडितश्च विजृम्भमाणः सहसोत्पपात} %6-60-56

\twolineshloka
{स नागभोगाचलशृङ्गकल्पौ विक्षिप्य बाहू जितवज्रसारौ}
{विवृत्य वक्त्रं वडवामुखाभं निशाचरोऽसौ विकृतं जजृम्भे} %6-60-57

\twolineshloka
{तस्य जाजृम्भमाणस्य वक्त्रं पातालसंनिभम्}
{ददृशे मेरुशृङ्गाग्रे दिवाकर इवोदितः} %6-60-58

\twolineshloka
{स जृम्भमाणोऽतिबलः प्रबुद्धस्तु निशाचरः}
{निःश्वासश्चास्य संजज्ञे पर्वतादिव मारुतः} %6-60-59

\twolineshloka
{रूपमुत्तिष्ठतस्तस्य कुम्भकर्णस्य तद् बभौ}
{युगान्ते सर्वभूतानि कालस्येव दिधक्षतः} %6-60-60

\twolineshloka
{तस्य दीप्ताग्निसदृशे विद्युत्सदृशवर्चसी}
{ददृशाते महानेत्रे दीप्ताविव महाग्रहौ} %6-60-61

\twolineshloka
{ततस्त्वदर्शयन् सर्वान् भक्ष्यांश्च विविधान् बहून्}
{वराहान् महिषांश्चैव बभक्ष स महाबलः} %6-60-62

\twolineshloka
{आदद् बुभुक्षितो मांसं शोणितं तृषितोऽपिबत्}
{मेदःकुम्भांश्च मद्यांश्च पपौ शक्ररिपुस्तदा} %6-60-63

\twolineshloka
{ततस्तृप्त इति ज्ञात्वा समुत्पेतुर्निशाचराः}
{शिरोभिश्च प्रणम्यैनं सर्वतः पर्यवारयन्} %6-60-64

\twolineshloka
{निद्राविशदनेत्रस्तु कलुषीकृतलोचनः}
{चारयन् सर्वतो दृष्टिं तान् ददर्श निशाचरान्} %6-60-65

\twolineshloka
{स सर्वान् सान्त्वयामास नैर्ऋतान् नैर्ऋतर्षभः}
{बोधनाद् विस्मितश्चापि राक्षसानिदमब्रवीत्} %6-60-66

\twolineshloka
{किमर्थमहमादृत्य भवद्भिः प्रतिबोधितः}
{कच्चित् सुकुशलं राज्ञो भयं वा नेह किंचन} %6-60-67

\twolineshloka
{अथवा ध्रुवमन्येभ्यो भयं परमुपस्थितम्}
{यदर्थमेव त्वरितैर्भवद्भिः प्रतिबोधितः} %6-60-68

\twolineshloka
{अद्य राक्षसराजस्य भयमुत्पाटयाम्यहम्}
{दारयिष्ये महेन्द्रं वा शीतयिष्ये तथानलम्} %6-60-69

\twolineshloka
{न ह्यल्पकारणे सुप्तं बोधयिष्यति मादृशम्}
{तदाख्यातार्थतत्त्वेन मत्प्रबोधनकारणम्} %6-60-70

\twolineshloka
{एवं ब्रुवाणं संरब्धं कुम्भकर्णमरिंदमम्}
{यूपाक्षः सचिवो राज्ञः कृताञ्जलिरभाषत} %6-60-71

\twolineshloka
{न नो देवकृतं किंचिद् भयमस्ति कदाचन}
{मानुषान्नो भयं राजंस्तुमुलं सम्प्रबाधते} %6-60-72

\twolineshloka
{न दैत्यदानवेभ्यो वा भयमस्ति न नः क्वचित्}
{यादृशं मानुषं राजन् भयमस्मानुपस्थितम्} %6-60-73

\twolineshloka
{वानरैः पर्वताकारैर्लङ्केयं परिवारिता}
{सीताहरणसंतप्ताद् रामान्नस्तुमुलं भयम्} %6-60-74

\twolineshloka
{एकेन वानरेणेयं पूर्वं दग्धा महापुरी}
{कुमारो निहतश्चाक्षः सानुयात्रः सकुञ्जरः} %6-60-75

\twolineshloka
{स्वयं रक्षोधिपश्चापि पौलस्त्यो देवकण्टकः}
{व्रजेति संयुगे मुक्तो रामेणादित्यवर्चसा} %6-60-76

\twolineshloka
{यन्न देवैः कृतो राजा नापि दैत्यैर्न दानवैः}
{कृतः स इह रामेण विमुक्तः प्राणसंशयात्} %6-60-77

\twolineshloka
{स यूपाक्षवचः श्रुत्वा भ्रातुर्युधि पराभवम्}
{कुम्भकर्णो विवृत्ताक्षो यूपाक्षमिदमब्रवीत्} %6-60-78

\twolineshloka
{सर्वमद्यैव यूपाक्ष हरिसैन्यं सलक्ष्मणम्}
{राघवं च रणे जित्वा ततो द्रक्ष्यामि रावणम्} %6-60-79

\twolineshloka
{राक्षसांस्तर्पयिष्यामि हरीणां मांसशोणितैः}
{रामलक्ष्मणयोश्चापि स्वयं पास्यामि शोणितम्} %6-60-80

\twolineshloka
{तत् तस्य वाक्यं ब्रुवतो निशम्य सगर्वितं रोषविवृद्धदोषम्}
{महोदरो नैर्ऋतयोधमुख्यः कृताञ्जलिर्वाक्यमिदं बभाषे} %6-60-81

\twolineshloka
{रावणस्य वचः श्रुत्वा गुणदोषौ विमृश्य च}
{पश्चादपि महाबाहो शत्रून् युधि विजेष्यसि} %6-60-82

\twolineshloka
{महोदरवचः श्रुत्वा राक्षसैः परिवारितः}
{कुम्भकर्णो महातेजाः सम्प्रतस्थे महाबलः} %6-60-83

\twolineshloka
{सुप्तमुत्थाप्य भीमाक्षं भीमरूपपराक्रमम्}
{राक्षसास्त्वरिता जग्मुर्दशग्रीवनिवेशनम्} %6-60-84

\twolineshloka
{तेऽभिगम्य दशग्रीवमासीनं परमासने}
{ऊचुर्बद्धाञ्जलिपुटाः सर्व एव निशाचराः} %6-60-85

\twolineshloka
{कुम्भकर्णः प्रबुद्धोऽसौ भ्राता ते राक्षसेश्वर}
{कथं तत्रैव निर्यातु द्रक्ष्यसे तमिहागतम्} %6-60-86

\twolineshloka
{रावणस्त्वब्रवीद्धृष्टो राक्षसांस्तानुपस्थितान्}
{द्रष्टुमेनमिहेच्छामि यथान्यायं च पूज्यताम्} %6-60-87

\twolineshloka
{तथेत्युक्त्वा तु ते सर्वे पुनरागम्य राक्षसाः}
{कुम्भकर्णमिदं वाक्यमूचू रावणचोदिताः} %6-60-88

\twolineshloka
{द्रष्टुं त्वां काङ्क्षते राजा सर्वराक्षसपुङ्गवः}
{गमने क्रियतां बुद्धिर्भ्रातरं सम्प्रहर्षय} %6-60-89

\twolineshloka
{कुम्भकर्णस्तु दुर्धर्षो भ्रातुराज्ञाय शासनम्}
{तथेत्युक्त्वा महावीर्यः शयनादुत्पपात ह} %6-60-90

\twolineshloka
{प्रक्षाल्य वदनं हृष्टः स्नातः परमहर्षितः}
{पिपासुस्त्वरयामास पानं बलसमीरणम्} %6-60-91

\twolineshloka
{ततस्ते त्वरितास्तत्र राक्षसा रावणाज्ञया}
{मद्यं भक्ष्यांश्च विविधान् क्षिप्रमेवोपहारयन्} %6-60-92

\twolineshloka
{पीत्वा घटसहस्रे द्वे गमनायोपचक्रमे}
{ईषत्समुत्कटो मत्तस्तेजोबलसमन्वितः} %6-60-93

\threelineshloka
{कुम्भकर्णो बभौ रुष्टः कालान्तकयमोपमः}
{भ्रातुः स भवनं गच्छन् रक्षोबलसमन्वितः}
{कुम्भकर्णः पदन्यासैरकम्पयत मेदिनीम्} %6-60-94

\twolineshloka
{स राजमार्गं वपुषा प्रकाशयन् सहस्ररश्मिर्धरणीमिवांशुभिः}
{जगाम तत्राञ्जलिमालया वृतः शतक्रतुर्गेहमिव स्वयंभुवः} %6-60-95

\twolineshloka
{तं राजमार्गस्थममित्रघातिनं वनौकसस्ते सहसा बहिःस्थिताः}
{दृष्ट्वाप्रमेयं गिरिशृङ्गकल्पं वितत्रसुस्ते सह यूथपालैः} %6-60-96

\twolineshloka
{केचिच्छरण्यं शरणं स्म रामं व्रजन्ति केचिद् व्यथिताः पतन्ति}
{केचिद् दशश्च व्यथिताः पतन्ति केचिद् भयार्ता भुवि शेरते स्म} %6-60-97

\twolineshloka
{तमद्रिशृङ्गप्रतिमं किरीटिनं स्पृशन्तमादित्यमिवात्मतेजसा}
{वनौकसः प्रेक्ष्य विवृद्धमद्भुतं भयार्दिता दुद्रुविरे यतस्ततः} %6-60-98


॥इत्यार्षे श्रीमद्रामायणे वाल्मीकीये आदिकाव्ये युद्धकाण्डे कुम्भकर्णप्रबोधः नाम षष्ठितमः सर्गः ॥६-६०॥
