\sect{सप्ताशीतितमः सर्गः — विभीषणरावणिपरस्परनिन्दा}

\twolineshloka
{एवमुक्त्वा तु सौमित्रिं जातहर्षो विभीषणः}
{धनुष्पाणिं तमादाय त्वरमाणो जगाम सः} %6-87-1

\twolineshloka
{अविदूरं ततो गत्वा प्रविश्य तु महद् वनम्}
{अदर्शयत तत्कर्म लक्ष्मणाय विभीषणः} %6-87-2

\twolineshloka
{नीलजीमूतसंकाशं न्यग्रोधं भीमदर्शनम्}
{तेजस्वी रावणभ्राता लक्ष्मणाय न्यवेदयत्} %6-87-3

\twolineshloka
{इहोपहारं भूतानां बलवान् रावणात्मजः}
{उपहृत्य ततः पश्चात् संग्राममभिवर्तते} %6-87-4

\twolineshloka
{अदृश्यः सर्वभूतानां ततो भवति राक्षसः}
{निहन्ति समरे शत्रून् बध्नाति च शरोत्तमैः} %6-87-5

\twolineshloka
{तमप्रविष्टं न्यग्रोधं बलिनं रावणात्मजम्}
{विध्वंसय शरैर्दीप्तैः सरथं साश्वसारथिम्} %6-87-6

\twolineshloka
{तथेत्युक्त्वा महातेजाः सौमित्रिर्मित्रनन्दनः}
{बभूवावस्थितस्तत्र चित्रं विस्फारयन् धनुः} %6-87-7

\twolineshloka
{स रथेनाग्निवर्णेन बलवान् रावणात्मजः}
{इन्द्रजित् कवची खड्गी सध्वजः प्रत्यदृश्यत} %6-87-8

\twolineshloka
{तमुवाच महातेजाः पौलस्त्यमपराजितम्}
{समाह्वये त्वां समरे सम्यग् युद्धं प्रयच्छ मे} %6-87-9

\twolineshloka
{एवमुक्तो महातेजा मनस्वी रावणात्मजः}
{अब्रवीत् परुषं वाक्यं तत्र दृष्ट्वा विभीषणम्} %6-87-10

\twolineshloka
{इह त्वं जातसंवृद्धः साक्षात् भ्राता पितुर्मम}
{कथं द्रुह्यसि पुत्रस्य पितृव्यो मम राक्षस} %6-87-11

\twolineshloka
{न ज्ञातित्वं न सौहार्दं न जातिस्तव दुर्मते}
{प्रमाणं न च सौदर्यं न धर्मो धर्मदूषण} %6-87-12

\twolineshloka
{शोच्यस्त्वमसि दुर्बुद्धे निन्दनीयश्च साधुभिः}
{यस्त्वं स्वजनमुत्सृज्य परभृत्यत्वमागतः} %6-87-13

\twolineshloka
{नैतच्छिथिलया बुद्ध्या त्वं वेत्सि महदन्तरम्}
{क्व च स्वजनसंवासः क्व च नीच पराश्रयः} %6-87-14

\twolineshloka
{गुणवान् वा परजनः स्वजनो निर्गुणोऽपि वा}
{निर्गुणः स्वजनः श्रेयान् यः परः पर एव सः} %6-87-15

\twolineshloka
{यः स्वपक्षं परित्यज्य परपक्षं निषेवते}
{स स्वपक्षे क्षयं याते पश्चात् तैरेव हन्यते} %6-87-16

\twolineshloka
{निरनुक्रोशता चेयं यादृशी ते निशाचर}
{स्वजनेन त्वया शक्यं पौरुषं रावणानुज} %6-87-17

\twolineshloka
{इत्युक्तो भ्रातृपुत्रेण प्रत्युवाच विभीषणः}
{अजानन्निव मच्छीलं किं राक्षस विकत्थसे} %6-87-18

\threelineshloka
{राक्षसेन्द्रसुतासाधो पारुष्यं त्यज गौरवात्}
{कुले यद्यप्यहं जातो रक्षसां क्रूरकर्मणाम्}
{गुणो यः प्रथमो नॄणां तन्मे शीलमराक्षसम्} %6-87-19

\twolineshloka
{न रमे दारुणेनाहं न चाधर्मेण वै रमे}
{भ्रात्रा विषमशीलोऽपि कथं भ्राता निरस्यते} %6-87-20

\twolineshloka
{धर्मात् प्रच्युतशीलं हि पुरुषं पापनिश्चयम्}
{त्यक्त्वा सुखमवाप्नोति हस्तादाशीविषं यथा} %6-87-21

\twolineshloka
{परस्वहरणे युक्तं परदाराभिमर्शकम्}
{त्याज्यमाहुर्दुरात्मानं वेश्म प्रज्वलितं यथा} %6-87-22

\twolineshloka
{परस्वानां च हरणं परदाराभिमर्शनम्}
{सुहृदामतिशङ्का च त्रयो दोषाः क्षयावहाः} %6-87-23

\twolineshloka
{महर्षीणां वधो घोरः सर्वदेवैश्च विग्रहः}
{अभिमानश्च रोषश्च वैरित्वं प्रतिकूलता} %6-87-24

\twolineshloka
{एते दोषा मम भ्रातुर्जीवितैश्वर्यनाशनाः}
{गुणान् प्रच्छादयामासुः पर्वतानिव तोयदाः} %6-87-25

\twolineshloka
{दोषैरेतैः परित्यक्तो मया भ्राता पिता तव}
{नेयमस्ति पुरी लङ्का न च त्वं न च ते पिता} %6-87-26

\twolineshloka
{अतिमानश्च बालश्च दुर्विनीतश्च राक्षस}
{बद्धस्त्वं कालपाशेन ब्रूहि मां यद् यदिच्छसि} %6-87-27

\twolineshloka
{अद्येह व्यसनं प्राप्तं यन्मां परुषमुक्तवान्}
{प्रवेष्टुं न त्वया शक्यं न्यग्रोधं राक्षसाधम} %6-87-28

\threelineshloka
{धर्षयित्वा च काकुत्स्थं न शक्यं जीवितुं त्वया}
{युध्यस्व नरदेवेन लक्ष्मणेन रणे सह}
{हतस्त्वं देवताकार्यं करिष्यसि यमक्षयम्} %6-87-29

\twolineshloka
{निदर्शयस्वात्मबलं समुद्यतं कुरुष्व सर्वायुधसायकव्ययम्}
{न लक्ष्मणस्यैत्य हि बाणगोचरं त्वमद्य जीवन् सबलो गमिष्यसि} %6-87-30


॥इत्यार्षे श्रीमद्रामायणे वाल्मीकीये आदिकाव्ये युद्धकाण्डे विभीषणरावणिपरस्परनिन्दा नाम सप्ताशीतितमः सर्गः ॥६-८७॥
