\sect{पञ्चविंशः सर्गः — शुकसारणप्रषेणादिकम्}

\twolineshloka
{सबले सागरं तीर्णे रामे दशरथात्मजे}
{अमात्यौ रावणः श्रीमानब्रवीच्छुकसारणौ} %6-25-1

\twolineshloka
{समग्रं सागरं तीर्णं दुस्तरं वानरं बलम्}
{अभूतपूर्वं रामेण सागरे सेतुबन्धनम्} %6-25-2

\twolineshloka
{सागरे सेतुबन्धं तं न श्रद्दध्यां कथंचन}
{अवश्यं चापि संख्येयं तन्मया वानरं बलम्} %6-25-3

\twolineshloka
{भवन्तौ वानरं सैन्यं प्रविश्यानुपलक्षितौ}
{परिमाणं च वीर्यं च ये च मुख्याः प्लवंगमाः} %6-25-4

\twolineshloka
{मन्त्रिणो ये च रामस्य सुग्रीवस्य च सम्मताः}
{ये पूर्वमभिवर्तन्ते ये च शूराः प्लवंगमाः} %6-25-5

\twolineshloka
{स च सेतुर्यथा बद्धः सागरे सलिलार्णवे}
{निवेशं च यथा तेषां वानराणां महात्मनाम्} %6-25-6

\twolineshloka
{रामस्य व्यवसायं च वीर्यं प्रहरणानि च}
{लक्ष्मणस्य च वीरस्य तत्त्वतो ज्ञातुमर्हथः} %6-25-7

\twolineshloka
{कश्च सेनापतिस्तेषां वानराणां महात्मनाम्}
{तच्च ज्ञात्वा यथातत्त्वं शीघ्रमागन्तुमर्हथः} %6-25-8

\twolineshloka
{इति प्रतिसमादिष्टौ राक्षसौ शुकसारणौ}
{हरिरूपधरौ वीरौ प्रविष्टौ वानरं बलम्} %6-25-9

\twolineshloka
{ततस्तद् वानरं सैन्यमचिन्त्यं लोमहर्षणम्}
{संख्यातुं नाध्यगच्छेतां तदा तौ शुकसारणौ} %6-25-10

\threelineshloka
{तत् स्थितं पर्वताग्रेषु निर्झरेषु गुहासु च}
{समुद्रस्य च तीरेषु वनेषूपवनेषु च}
{तरमाणं च तीर्णं च तर्तुकामं च सर्वशः} %6-25-11

\twolineshloka
{निविष्टं निविशच्चैव भीमनादं महाबलम्}
{तद्बलार्णवमक्षोभ्यं ददृशाते निशाचरौ} %6-25-12

\twolineshloka
{तौ ददर्श महातेजाः प्रतिच्छन्नौ विभीषणः}
{आचचक्षे स रामाय गृहीत्वा शुकसारणौ} %6-25-13

\twolineshloka
{तस्यैतौ राक्षसेन्द्रस्य मन्त्रिणौ शुकसारणौ}
{लङ्कायाः समनुप्राप्तौ चारौ परपुरंजय} %6-25-14

\twolineshloka
{तौ दृष्ट्वा व्यथितौ रामं निराशौ जीविते तथा}
{कृताञ्जलिपुटौ भीतौ वचनं चेदमूचतुः} %6-25-15

\twolineshloka
{आवामिहागतौ सौम्य रावणप्रहितावुभौ}
{परिज्ञातुं बलं सर्वं तदिदं रघुनन्दन} %6-25-16

\twolineshloka
{तयोस्तद् वचनं श्रुत्वा रामो दशरथात्मजः}
{अब्रवीत् प्रहसन् वाक्यं सर्वभूतहिते रतः} %6-25-17

\twolineshloka
{यदि दृष्टं बलं सर्वं वयं वा सुसमाहिताः}
{यथोक्तं वा कृतं कार्यं छन्दतः प्रतिगम्यताम्} %6-25-18

\twolineshloka
{अथ किंचिददृष्टं वा भूयस्तद् द्रष्टुमर्हथः}
{विभीषणो वा कात्स्र्न्येन पुनः संदर्शयिष्यति} %6-25-19

\twolineshloka
{न चेदं ग्रहणं प्राप्य भेतव्यं जीवितं प्रति}
{न्यस्तशस्त्रौ गृहीतौ च न दूतौ वधमर्हथः} %6-25-20

\twolineshloka
{प्रच्छन्नौ च विमुञ्चेमौ चारौ रात्रिंचरावुभौ}
{शत्रुपक्षस्य सततं विभीषण विकर्षिणौ} %6-25-21

\twolineshloka
{प्रविश्य महतीं लङ्कां भवद्भ्यां धनदानुजः}
{वक्तव्यो रक्षसां राजा यथोक्तं वचनं मम} %6-25-22

\twolineshloka
{यद् बलं त्वं समाश्रित्य सीतां मे हृतवानसि}
{तद् दर्शय यथाकामं ससैन्यश्च सबान्धवः} %6-25-23

\twolineshloka
{श्वः काल्ये नगरीं लङ्कां सप्राकारां सतोरणाम्}
{रक्षसां च बलं पश्य शरैर्विध्वंसितं मया} %6-25-24

\twolineshloka
{क्रोधं भीममहं मोक्ष्ये ससैन्ये त्वयि रावण}
{श्वः काल्ये वज्रवान् वज्रं दानवेष्विव वासवः} %6-25-25

\twolineshloka
{इति प्रतिसमादिष्टौ राक्षसौ शुकसारणौ}
{जयेति प्रतिनन्द्यैनं राघवं धर्मवत्सलम्} %6-25-26

\twolineshloka
{आगम्य नगरीं लङ्कामब्रूतां राक्षसाधिपम्}
{विभीषणगृहीतौ तु वधार्थं राक्षसेश्वर} %6-25-27

\twolineshloka
{दृष्ट्वा धर्मात्मना मुक्तौ रामेणामिततेजसा}
{एकस्थानगता यत्र चत्वारः पुरुषर्षभाः} %6-25-28

\twolineshloka
{लोकपालसमाः शूराः कृतास्त्रा दृढविक्रमाः}
{रामो दाशरथिः श्रीमाल्ँलक्ष्मणश्च विभीषणः} %6-25-29

\twolineshloka
{सुग्रीवश्च महातेजा महेन्द्रसमविक्रमः}
{एते शक्ताः पुरीं लङ्कां सप्राकारां सतोरणाम्} %6-25-30

\twolineshloka
{उत्पाट्य संक्रामयितुं सर्वे तिष्ठन्तु वानराः}
{यादृशं तद्धि रामस्य रूपं प्रहरणानि च} %6-25-31

\threelineshloka
{वधिष्यति पुरीं लङ्कामेकस्तिष्ठन्तु ते त्रयः}
{रामलक्ष्मणगुप्ता सा सुग्रीवेण च वाहिनी}
{बभूव दुर्धर्षतरा सर्वैरपि सुरासुरैः} %6-25-32

\twolineshloka
{प्रहृष्टयोधा ध्वजिनी महात्मनां वनौकसां सम्प्रति योद्धुमिच्छताम्}
{अलं विरोधेन शमो विधीयतां प्रदीयतां दाशरथाय मैथिली} %6-25-33


॥इत्यार्षे श्रीमद्रामायणे वाल्मीकीये आदिकाव्ये युद्धकाण्डे शुकसारणप्रषेणादिकम् नाम पञ्चविंशः सर्गः ॥६-२५॥
