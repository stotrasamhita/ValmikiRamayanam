\sect{शततमः सर्गः — रामरावणास्त्रपरम्परा}

\twolineshloka
{महोदरमहापार्श्वौ हतौ दृष्ट्वा स रावणः}
{तस्मिंश्च निहते वीरे विरूपाक्षे महाबले} %6-100-1

\twolineshloka
{आविवेश महान् क्रोधो रावणं तु महामृधे}
{सूतं संचोदयामास वाक्यं चेदमुवाच ह} %6-100-2

\twolineshloka
{निहतानाममात्यानां रुद्धस्य नगरस्य च}
{दुःखमेवापनेष्यामि हत्वा तौ रामलक्ष्मणौ} %6-100-3

\twolineshloka
{रामवृक्षं रणे हन्मि सीतापुष्पफलप्रदम्}
{प्रशाखा यस्य सुग्रीवो जाम्बवान् कुमुदो नलः} %6-100-4

\twolineshloka
{द्विविदश्चैव मैन्दश्च अङ्गदो गन्धमादनः}
{हनूमांश्च सुषेणश्च सर्वे च हरियूथपाः} %6-100-5

\twolineshloka
{स दिशो दश घोषेण रथस्यातिरथो महान्}
{नादयन् प्रययौ तूर्णं राघवं चाभ्यधावत} %6-100-6

\twolineshloka
{पूरिता तेन शब्देन सनदीगिरिकानना}
{संचचाल मही सर्वा त्रस्तसिंहमृगद्विजा} %6-100-7

\twolineshloka
{तामसं सुमहाघोरं चकारास्त्रं सुदारुणम्}
{निर्ददाह कपीन् सर्वांस्ते प्रपेतुः समन्ततः} %6-100-8

\twolineshloka
{उत्पपात रजो भूमौ तैर्भग्नैः सम्प्रधावितैः}
{नहि तत् सहितुं शेकुर्ब्रह्मणा निर्मितं स्वयम्} %6-100-9

\twolineshloka
{तान्यनीकान्यनेकानि रावणस्य शरोत्तमैः}
{दृष्ट्वा भग्नानि शतशो राघवः पर्यवस्थितः} %6-100-10

\twolineshloka
{ततो राक्षसशार्दूलो विद्राव्य हरिवाहिनीम्}
{स ददर्श ततो रामं तिष्ठन्तमपराजितम्} %6-100-11

\twolineshloka
{लक्ष्मणेन सह भ्रात्रा विष्णुना वासवं यथा}
{आलिखन्तमिवाकाशमवष्टभ्य महद् धनुः} %6-100-12

\twolineshloka
{पद्मपत्रविशालाक्षं दीर्घबाहुमरिंदमम्}
{ततो रामो महातेजाः सौमित्रिसहितो बली} %6-100-13

\twolineshloka
{वानरांश्च रणे भग्नानापतन्तं च रावणम्}
{समीक्ष्य राघवो हृष्टो मध्ये जग्राह कार्मुकम्} %6-100-14

\twolineshloka
{विस्फारयितुमारेभे ततः स धनुरुत्तमम्}
{महावेगं महानादं निर्भिन्दन्निव मेदिनीम्} %6-100-15

\twolineshloka
{रावणस्य च बाणौघै रामविस्फारितेन च}
{शब्देन राक्षसास्तेन पेतुश्च शतशस्तदा} %6-100-16

\twolineshloka
{तयोः शरपथं प्राप्य रावणो राजपुत्रयोः}
{स बभौ च यथा राहुः समीपे शशिसूर्ययोः} %6-100-17

\twolineshloka
{तमिच्छन् प्रथमं योद्धुं लक्ष्मणो निशितैः शरैः}
{मुमोच धनुरायम्य शरानग्निशिखोपमान्} %6-100-18

\twolineshloka
{तान् मुक्तमात्रानाकाशे लक्ष्मणेन धनुष्मता}
{बाणान् बाणैर्महातेजा रावणः प्रत्यवारयत्} %6-100-19

\twolineshloka
{एकमेकेन बाणेन त्रिभिस्त्रीन् दशभिर्दश}
{लक्ष्मणस्य प्रचिच्छेद दर्शयन् पाणिलाघवम्} %6-100-20

\twolineshloka
{अभ्यतिक्रम्य सौमित्रिं रावणः समितिंजयः}
{आससाद रणे रामं स्थितं शैलमिवापरम्} %6-100-21

\twolineshloka
{स राघवं समासाद्य क्रोधसंरक्तलोचनः}
{व्यसृजच्छरवर्षाणि रावणो राक्षसेश्वरः} %6-100-22

\twolineshloka
{शरधारास्ततो रामो रावणस्य धनुश्च्युताः}
{दृष्ट्वैवापतिताः शीघ्रं भल्लाञ्जग्राह सत्वरम्} %6-100-23

\twolineshloka
{ताञ्छरौघांस्ततो भल्लैस्तीक्ष्णैश्चिच्छेद राघवः}
{दीप्यमानान् महाघोराञ्छरानाशीविषोपमान्} %6-100-24

\twolineshloka
{राघवो रावणं तूर्णं रावणो राघवं तथा}
{अन्योन्यं विविधैस्तीक्ष्णैः शरवर्षैर्ववर्षतुः} %6-100-25

\twolineshloka
{चेरतुश्च चिरं चित्रं मण्डलं सव्यदक्षिणम्}
{बाणवेगात् समुत्क्षिप्तावन्योन्यमपराजितौ} %6-100-26

\twolineshloka
{तयोर्भूतानि वित्रेसुर्युगपत् सम्प्रयुध्यतोः}
{रौद्रयोः सायकमुचोर्यमान्तकनिकाशयोः} %6-100-27

\twolineshloka
{सततं विविधैर्बाणैर्बभूव गगनं तदा}
{घनैरिवातपापाये विद्युन्मालासमाकुलैः} %6-100-28

\twolineshloka
{गवाक्षितमिवाकाशं बभूव शरवृष्टिभिः}
{महावेगैः सुतीक्ष्णाग्रैर्गृध्रपत्रैः सुवाजितैः} %6-100-29

\twolineshloka
{शरान्धकारमाकाशं चक्रतुः परमं तदा}
{गतेऽस्तं तपने चापि महामेघाविवोत्थितौ} %6-100-30

\twolineshloka
{तयोरभून्महायुद्धमन्योन्यवधकांक्षिणोः}
{अनासाद्यमचिन्त्यं च वृत्रवासवयोरिव} %6-100-31

\twolineshloka
{उभौ हि परमेष्वासावुभौ युद्धविशारदौ}
{उभावस्त्रविदां मुख्यावुभौ युद्धे विचेरतुः} %6-100-32

\twolineshloka
{उभौ हि येन व्रजतस्तेन तेन शरोर्मयः}
{ऊर्मयो वायुना विद्धा जग्मुः सागरयोरिव} %6-100-33

\twolineshloka
{ततः संसक्तहस्तस्तु रावणो लोकरावणः}
{नाराचमालां रामस्य ललाटे प्रत्यमुञ्चत} %6-100-34

\twolineshloka
{रौद्रचापप्रयुक्तां तां नीलोत्पलदलप्रभाम्}
{शिरसाधारयद् रामो न व्यथामभ्यपद्यत} %6-100-35

\twolineshloka
{अथ मन्त्रानपि जपन् रौद्रमस्त्रमुदीरयन्}
{शरान् भूयः समादाय रामः क्रोधसमन्वितः} %6-100-36

\twolineshloka
{मुमोच च महातेजाश्चापमायम्य वीर्यवान्}
{तान् शरान् राक्षसेन्द्राय चिक्षेपाच्छिन्नसायकः} %6-100-37

\twolineshloka
{ते महामेघसंकाशे कवचे पतिताः शराः}
{अवध्ये राक्षसेन्द्रस्य न व्यथां जनयंस्तदा} %6-100-38

\twolineshloka
{पुनरेवाथ तं रामो रथस्थं राक्षसाधिपम्}
{ललाटे परमास्त्रेण सर्वास्त्रकुशलोऽभिनत्} %6-100-39

\twolineshloka
{ते भित्त्वा बाणरूपाणि पञ्चशीर्षा इवोरगाः}
{श्वसन्तो विविशुर्भूमिं रावणप्रतिकूलिताः} %6-100-40

\twolineshloka
{निहत्य राघवस्यास्त्रं रावणः क्रोधमूर्च्छितः}
{आसुरं सुमहाघोरमस्त्रं प्रादुश्चकार सः} %6-100-41

\twolineshloka
{सिंहव्याघ्रमुखांश्चापि कङ्ककोकमुखानपि}
{गृध्रश्येनमुखांश्चापि शृगालवदनांस्तथा} %6-100-42

\twolineshloka
{ईहामृगमुखांश्चापि व्यादितास्यान् भयावहान्}
{पञ्चास्याँल्लेलिहानांश्च ससर्ज निशितान् शरान्} %6-100-43

\twolineshloka
{शरान् खरमुखांश्चान्यान् वराहमुखसंश्रितान्}
{श्वानकुक्कुटवक्त्रांश्च मकराशीविषाननान्} %6-100-44

\twolineshloka
{एतांश्चान्यांश्च मायाभिः ससर्ज निशिताञ्छरान्}
{रामं प्रति महातेजाः क्रुद्धः सर्प इव श्वसन्} %6-100-45

\twolineshloka
{आसुरेण समाविष्टः सोऽस्त्रेण रघुपुङ्गवः}
{ससर्जास्त्रं महोत्साहं पावकं पावकोपमः} %6-100-46

\threelineshloka
{अग्निदीप्तमुखान् बाणांस्तत्र सूर्यमुखानपि}
{चन्द्रार्धचन्द्रवक्त्रांश्च धूमकेतुमुखानपि}
{ग्रहनक्षत्रवर्णांश्च महोल्कामुखसंस्थितान्} %6-100-47

\twolineshloka
{विद्युज्जिह्वोपमांश्चापि ससर्ज विविधाञ्छरान्}
{ते रावणशरा घोरा राघवास्त्रसमाहताः} %6-100-48

\twolineshloka
{विलयं जग्मुराकाशे जघ्नुश्चैव सहस्रशः}
{तदस्त्रं निहतं दृष्ट्वा रामेणाक्लिष्टकर्मणा} %6-100-49

\twolineshloka
{हृष्टा नेदुस्ततः सर्वे कपयः कामरूपिणः}
{सुग्रीवाभिमुखा वीराः सम्परिक्षिप्य राघवम्} %6-100-50

\twolineshloka
{ततस्तदस्त्रं विनिहत्य राघवः प्रसह्य तद् रावणबाहुनिःसृतम्}
{मुदान्वितो दाशरथिर्महात्मा विनेदुरुच्चैर्मुदिताः कपीश्वराः} %6-100-51


॥इत्यार्षे श्रीमद्रामायणे वाल्मीकीये आदिकाव्ये युद्धकाण्डे रामरावणास्त्रपरम्परा नाम शततमः सर्गः ॥६-१००॥
