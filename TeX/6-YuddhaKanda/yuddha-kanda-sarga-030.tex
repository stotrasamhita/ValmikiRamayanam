\sect{त्रिंशः सर्गः — वानरबलसङ्ख्यानम्}

\twolineshloka
{ततस्तमक्षोभ्यबलं लङ्काधिपतये चराः}
{सुवेले राघवं शैले निविष्टं प्रत्यवेदयन्} %6-30-1

\twolineshloka
{चाराणां रावणः श्रुत्वा प्राप्तं रामं महाबलम्}
{जातोद्वेगोऽभवत् किञ्चिच्छार्दूलं वाक्यमब्रवीत्} %6-30-2

\twolineshloka
{अयथावच्च ते वर्णो दीनश्चासि निशाचर}
{नासि कच्चिदमित्राणां क्रुद्धानां वशमागतः} %6-30-3

\twolineshloka
{इति तेनानुशिष्टस्तु वाचं मन्दमुदीरयन्}
{तदा राक्षसशार्दूलं शार्दूलो भयविक्लवः} %6-30-4

\twolineshloka
{न ते चारयितुं शक्या राजन् वानरपुङ्गवाः}
{विक्रान्ता बलवन्तश्च राघवेण च रक्षिताः} %6-30-5

\twolineshloka
{नापि सम्भाषितुं शक्याः सम्प्रश्नोऽत्र न लभ्यते}
{सर्वतो रक्ष्यते पन्था वानरैः पर्वतोपमैः} %6-30-6

\twolineshloka
{प्रविष्टमात्रे ज्ञातोऽहं बले तस्मिन् विचारिते}
{बलाद् गृहीतो रक्षोभिर्बहुधास्मि विचारितः} %6-30-7

\twolineshloka
{जानुभिर्मुष्टिभिर्दन्तैस्तलैश्चाभिहतो भृशम्}
{परिणीतोऽस्मि हरिभिर्बलमध्ये अमर्षणैः} %6-30-8

\twolineshloka
{परिणीय च सर्वत्र नीतोऽहं रामसंसदि}
{रुधिरस्राविदीनाङ्गो विह्वलश्चलितेन्द्रियः} %6-30-9

\twolineshloka
{हरिभिर्वध्यमानश्च याचमानः कृताञ्जलिः}
{राघवेण परित्रातो मा मेति च यदृच्छया} %6-30-10

\twolineshloka
{एष शैलशिलाभिस्तु पूरयित्वा महार्णवम्}
{द्वारमाश्रित्य लङ्काया रामस्तिष्ठति सायुधः} %6-30-11

\twolineshloka
{गरुडव्यूहमास्थाय सर्वतो हरिभिर्वृतः}
{मां विसृज्य महातेजा लङ्कामेवातिवर्तते} %6-30-12

\twolineshloka
{पुरा प्राकारमायाति क्षिप्रमेकतरं कुरु}
{सीतां वापि प्रयच्छाशु युद्धं वापि प्रदीयताम्} %6-30-13

\twolineshloka
{मनसा तत् तदा प्रेक्ष्य तच्छ्रुत्वा राक्षसाधिपः}
{शार्दूलं सुमहद्वाक्यमथोवाच स रावणः} %6-30-14

\twolineshloka
{यदि मां प्रतियुध्यन्ते देवगन्धर्वदानवाः}
{नैव सीतां प्रदास्यामि सर्वलोकभयादपि} %6-30-15

\twolineshloka
{एवमुक्त्वा महातेजा रावणः पुनरब्रवीत्}
{चरिता भवता सेना केऽत्र शूराः प्लवङ्गमाः} %6-30-16

\twolineshloka
{किम्प्रभाः कीदृशाः सौम्य वानरा ये दुरासदाः}
{कस्य पुत्राश्च पौत्राश्च तत्त्वमाख्याहि राक्षस} %6-30-17

\twolineshloka
{तथात्र प्रतिपत्स्यामि ज्ञात्वा तेषां बलाबलम्}
{अवश्यं खलु सङ्ख्यानं कर्तव्यं युद्धमिच्छता} %6-30-18

\twolineshloka
{अथैवमुक्तः शार्दूलो रावणेनोत्तमश्चरः}
{इदं वचनमारेभे वक्तुं रावणसन्निधौ} %6-30-19

\twolineshloka
{अथर्क्षरजसः पुत्रो युधि राजन् सुदुर्जयः}
{गद्गदस्याथ पुत्रोऽत्र जाम्बवानिति विश्रुतः} %6-30-20

\twolineshloka
{गद्गदस्याथ पुत्रोऽन्यो गुरुपुत्रः शतक्रतोः}
{कदनं यस्य पुत्रेण कृतमेकेन रक्षसाम्} %6-30-21

\twolineshloka
{सुषेणश्चात्र धर्मात्मा पुत्रो धर्मस्य वीर्यवान्}
{सौम्यः सोमात्मजश्चात्र राजन् दधिमुखः कपिः} %6-30-22

\twolineshloka
{सुमुखो दुर्मुखश्चात्र वेगदर्शी च वानरः}
{मृत्युर्वानररूपेण नूनं सृष्टः स्वयम्भुवा} %6-30-23

\twolineshloka
{पुत्रो हुतवहस्यात्र नीलः सेनापतिः स्वयम्}
{अनिलस्य तु पुत्रोऽत्र हनूमानिति विश्रुतः} %6-30-24

\twolineshloka
{नप्ता शक्रस्य दुर्धर्षो बलवानङ्गदो युवा}
{मैन्दश्च द्विविदश्चोभौ बलिनावश्विसम्भवौ} %6-30-25

\twolineshloka
{पुत्रा वैवस्वतस्याथ पञ्च कालान्तकोपमाः}
{गजो गवाक्षो गवयः शरभो गन्धमादनः} %6-30-26

\twolineshloka
{दश वानरकोट्यश्च शूराणां युद्धकाङ्क्षिणाम्}
{श्रीमतां देवपुत्राणां शेषं नाख्यातुमुत्सहे} %6-30-27

\twolineshloka
{पुत्रो दशरथस्यैष सिंहसंहननो युवा}
{दूषणो निहतो येन खरश्च त्रिशिरास्तथा} %6-30-28

\twolineshloka
{नास्ति रामस्य सदृशे विक्रमे भुवि कश्चन}
{विराधो निहतो येन कबन्धश्चान्तकोपमः} %6-30-29

\twolineshloka
{वक्तुं न शक्तो रामस्य गुणान् कश्चिन्नरः क्षितौ}
{जनस्थानगता येन तावन्तो राक्षसा हताः} %6-30-30

\twolineshloka
{लक्ष्मणश्चात्र धर्मात्मा मातङ्गानामिवर्षभः}
{यस्य बाणपथं प्राप्य न जीवेदपि वासवः} %6-30-31

\twolineshloka
{श्वेतो ज्योतिर्मुखश्चात्र भास्करस्यात्मसम्भवौ}
{वरुणस्याथ पुत्रोऽथ हेमकूटः प्लवङ्गमः} %6-30-32

\twolineshloka
{विश्वकर्मसुतो वीरो नलः प्लवगसत्तमः}
{विक्रान्तो वेगवानत्र वसुपुत्रः स दुर्धरः} %6-30-33

\twolineshloka
{राक्षसानां वरिष्ठश्च तव भ्राता विभीषणः}
{प्रतिगृह्य पुरीं लङ्कां राघवस्य हिते रतः} %6-30-34

\twolineshloka
{इति सर्वं समाख्यातं तथा वै वानरं बलम्}
{सुवेलेऽधिष्ठितं शैले शेषकार्ये भवान् गतिः} %6-30-35


॥इत्यार्षे श्रीमद्रामायणे वाल्मीकीये आदिकाव्ये युद्धकाण्डे वानरबलसङ्ख्यानम् नाम त्रिंशः सर्गः ॥६-३०॥
