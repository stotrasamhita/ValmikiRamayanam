\sect{एकोनशततमः सर्गः — महापार्श्ववधः}

\twolineshloka
{महोदरे तु निहते महापार्श्वो महाबलः}
{सुग्रीवेण समीक्ष्याथ क्रोधात् संरक्तलोचनः} %6-99-1

\twolineshloka
{अङ्गदस्य चमूं भीमां क्षोभयामास मार्गणैः}
{स वानराणां मुख्यानामुत्तमाङ्गानि राक्षसः} %6-99-2

\twolineshloka
{पातयामास कायेभ्यः फलं वृन्तादिवानिलः}
{केषांचिदिषुभिर्बाहूंश्चिच्छेदाथ स राक्षसः} %6-99-3

\twolineshloka
{वानराणां सुसंरब्धः पार्श्वं केषांचिदाक्षिपत्}
{तेऽर्दिता बाणवर्षेण महापार्श्वेन वानराः} %6-99-4

\twolineshloka
{विषादविमुखाः सर्वे बभूवुर्गतचेतसः}
{निशम्य बलमुद्विग्नमङ्गदो राक्षसार्दितम्} %6-99-5

\twolineshloka
{वेगं चक्रे महावेगः समुद्र इव पर्वसु}
{आयसं परिघं गृह्य सूर्यरश्मिसमप्रभम्} %6-99-6

\twolineshloka
{समरे वानरश्रेष्ठो महापार्श्वे न्यपातयत्}
{स तु तेन प्रहारेण महापार्श्वो विचेतनः} %6-99-7

\twolineshloka
{ससूतः स्यन्दनात् तस्माद् विसंज्ञश्चापतद् भुवि}
{तस्यर्क्षराजस्तेजस्वी नीलाञ्जनचयोपमः} %6-99-8

\twolineshloka
{निष्पत्य सुमहावीर्यः स्वयूथान्मेघसंनिभात्}
{प्रगृह्य गिरिशृङ्गाभां क्रुद्धः स विपुलां शिलाम्} %6-99-9

\twolineshloka
{अश्वाञ्जघान तरसा बभञ्ज स्यन्दनं च तम्}
{मुहूर्ताल्लब्धसंज्ञस्तु महापार्श्वो महाबलः} %6-99-10

\twolineshloka
{अङ्गदं बहुभिर्बाणैर्भूयस्तं प्रत्यविध्यत}
{जाम्बवन्तं त्रिभिर्बाणैराजघान स्तनान्तरे} %6-99-11

\twolineshloka
{ऋक्षराजं गवाक्षं च जघान बहुभिः शरैः}
{गवाक्षं जाम्बवन्तं च स दृष्ट्वा शरपीडितौ} %6-99-12

\twolineshloka
{जग्राह परिघं घोरमङ्गदः क्रोधमूर्च्छितः}
{तस्याङ्गदः सरोषाक्षो राक्षसस्य तमायसम्} %6-99-13

\twolineshloka
{दूरस्थितस्य परिघं रविरश्मिसमप्रभम्}
{द्वाभ्यां भुजाभ्यां संगृह्य भ्रामयित्वा च वेगवत्} %6-99-14

\twolineshloka
{महापार्श्वस्य चिक्षेप वधार्थं वालिनः सुतः}
{स तु क्षिप्तो बलवता परिघस्तस्य रक्षसः} %6-99-15

\twolineshloka
{धनुश्च सशरं हस्ताच्छिरस्त्राणं च पातयत्}
{तं समासाद्य वेगेन वालिपुत्रः प्रतापवान्} %6-99-16

\twolineshloka
{तलेनाभ्यहनत् क्रुद्धः कर्णमूले सकुण्डले}
{स तु क्रुद्धो महावेगो महापार्श्वो महाद्युतिः} %6-99-17

\twolineshloka
{करेणैकेन जग्राह सुमहान्तं परश्वधम्}
{तं तैलधौतं विमलं शैलसारमयं दृढम्} %6-99-18

\twolineshloka
{राक्षसः परमक्रुद्धो वालिपुत्रे न्यपातयत्}
{तेन वामांसफलके भृशं प्रत्यवपातितम्} %6-99-19

\twolineshloka
{अङ्गदो मोक्षयामास सरोषः स परश्वधम्}
{स वीरो वज्रसंकाशमङ्गदो मुष्टिमात्मनः} %6-99-20

\twolineshloka
{संवर्तयत् सुसंक्रुद्धः पितुस्तुल्यपराक्रमः}
{राक्षसस्य स्तनाभ्याशे मर्मज्ञो हृदयं प्रति} %6-99-21

\twolineshloka
{इन्द्राशनिसमस्पर्शं स मुष्टिं विन्यपातयत्}
{तेन तस्य निपातेन राक्षसस्य महामृधे} %6-99-22

\twolineshloka
{पफाल हृदयं चास्य स पपात हतो भुवि}
{तस्मिन् विनिहते भूमौ तत् सैन्यं सम्प्रचुक्षुभे} %6-99-23

\twolineshloka
{अभवच्च महान् क्रोधः समरे रावणस्य तु}
{वानराणां प्रहृष्टानां सिंहनादः सुपुष्कलः} %6-99-24

\twolineshloka
{स्फोटयन्निव शब्देन लङ्कां साट्टालगोपुराम्}
{सहेन्द्रेणेव देवानां नादः समभवन्महान्} %6-99-25

\twolineshloka
{अथेन्द्रशत्रुस्त्रिदशालयानां वनौकसां चैव महाप्रणादम्}
{श्रुत्वा सरोषं युधि राक्षसेन्द्रः पुनश्च युद्धाभिमुखोऽवतस्थे} %6-99-26


॥इत्यार्षे श्रीमद्रामायणे वाल्मीकीये आदिकाव्ये युद्धकाण्डे महापार्श्ववधः नाम एकोनशततमः सर्गः ॥६-९९॥
