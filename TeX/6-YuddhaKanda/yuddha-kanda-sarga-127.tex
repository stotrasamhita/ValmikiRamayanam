\sect{सप्तविंशत्यधिकशततमः सर्गः — भरद्वाजामन्त्रणम्}

\twolineshloka
{पूर्णे चतुर्दशे वर्षे पञ्चम्यां लक्ष्मणाग्रजः}
{भरद्वाजाश्रमं प्राप्य ववन्दे नियतो मुनिम्} %6-127-1

\threelineshloka
{सोऽपृच्छदभिवाद्यैनं भरद्वाजं तपोधनम्}
{शृणोषि कच्चिद् भगवन् सुभिक्षानामयं पुरे}
{कच्चित् स युक्तो भरतो जीवन्त्यपि च मातरः} %6-127-2

\twolineshloka
{एवमुक्तस्तु रामेण भरद्वाजो महामुनिः}
{प्रत्युवाच रघुश्रेष्ठं स्मितपूर्वं प्रहृष्टवत्} %6-127-3

\twolineshloka
{आज्ञावशत्वे भरतो जटिलस्त्वां प्रतीक्षते}
{पादुके ते पुरस्कृत्य सर्वं च कुशलं गृहे} %6-127-4

\twolineshloka
{त्वां पुरा चीरवसनं प्रविशन्तं महावनम्}
{स्त्रीतृतीयं च्युतं राज्याद् धर्मकामं च केवलम्} %6-127-5

\twolineshloka
{पदातिं त्यक्तसर्वस्वं पितृनिर्देशकारिणम्}
{सर्वभोगैः परित्यक्तं स्वर्गच्युतमिवामरम्} %6-127-6

\twolineshloka
{दृष्ट्वा तु करुणापूर्वं ममासीत् समितिंजय}
{कैकेयीवचने युक्तं वन्यमूलफलाशिनम्} %6-127-7

\twolineshloka
{साम्प्रतं तु समृद्धार्थं समित्रगणबान्धवम्}
{समीक्ष्य विजितारिं च ममाभूत् प्रीतिरुत्तमा} %6-127-8

\twolineshloka
{सर्वं च सुखदुःखं ते विदितं मम राघव}
{यत् त्वया विपुलं प्राप्तं जनस्थाननिवासिना} %6-127-9

\twolineshloka
{ब्राह्मणार्थे नियुक्तस्य रक्षतः सर्वतापसान्}
{रावणेन हृता भार्या बभूवेयमनिन्दिता} %6-127-10

\twolineshloka
{मारीचदर्शनं चैव सीतोन्मथनमेव च}
{कबन्धदर्शनं चैव पम्पाभिगमनं तथा} %6-127-11

\twolineshloka
{सुग्रीवेण च ते सख्यं यत्र वाली हतस्त्वया}
{मार्गणं चैव वैदेह्याः कर्म वातात्मजस्य च} %6-127-12

\twolineshloka
{विदितायां च वैदेह्यां नलसेतुर्यथा कृतः}
{यथा चादीपिता लङ्का प्रहृष्टैर्हरियूथपैः} %6-127-13

\twolineshloka
{सपुत्रबान्धवामात्यः सबलः सहवाहनः}
{यथा च निहतः संख्ये रावणो बलदर्पितः} %6-127-14

\twolineshloka
{यथा च निहते तस्मिन् रावणे देवकण्टके}
{समागमश्च त्रिदशैर्यथा दत्तश्च ते वरः} %6-127-15

\twolineshloka
{सर्वं ममैतद् विदितं तपसा धर्मवत्सल}
{सम्पतन्ति च मे शिष्याः प्रवृत्त्याख्याः पुरीमितः} %6-127-16

\twolineshloka
{अहमप्यत्र ते दद्मि वरं शस्त्रभृतां वर}
{अर्घ्यं प्रतिगृहाणेदमयोध्यां श्वो गमिष्यसि} %6-127-17

\twolineshloka
{तस्य तच्छिरसा वाक्यं प्रतिगृह्य नृपात्मजः}
{बाढमित्येव संहृष्टः श्रीमान् वरमयाचत} %6-127-18

\twolineshloka
{अकालफलिनो वृक्षाः सर्वे चापि मधुस्रवाः}
{फलान्यमृतगन्धीनि बहूनि विविधानि च} %6-127-19

\twolineshloka
{भवन्तु मार्गे भगवन्नयोध्यां प्रति गच्छतः}
{तथेति च प्रतिज्ञाते वचनात् समनन्तरम्} %6-127-20

\twolineshloka
{अभवन् पादपास्तत्र स्वर्गपादपसंनिभाः}
{निष्फलाः फलिनश्चासन् विपुष्पाः पुष्पशालिनः} %6-127-21

\twolineshloka
{शुष्काः समग्रपत्रास्ते नगाश्चैव मधुस्रवाः}
{सर्वतो योजनास्तिस्रो गच्छतामभवंस्तदा} %6-127-22

\twolineshloka
{ततः प्रहृष्टाः प्लवगर्षभास्ते बहूनि दिव्यानि फलानि चैव}
{कामादुपाश्नन्ति सहस्रशस्ते मुदान्विताः स्वर्गजितो यथैव} %6-127-23


॥इत्यार्षे श्रीमद्रामायणे वाल्मीकीये आदिकाव्ये युद्धकाण्डे भरद्वाजामन्त्रणम् नाम सप्तविंशत्यधिकशततमः सर्गः ॥६-१२७॥
