\sect{एकषष्ठितमः सर्गः — कुम्भकर्णवृत्तकथनम्}

\twolineshloka
{ततो रामो महातेजा धनुरादाय वीर्यवान्}
{किरीटिनं महाकायं कुम्भकर्णं ददर्श ह} %6-61-1

\twolineshloka
{तं दृष्ट्वा राक्षसश्रेष्ठं पर्वताकारदर्शनम्}
{क्रममाणमिवाकाशं पुरा नारायणं यथा} %6-61-2

\twolineshloka
{सतोयाम्बुदसङ्काशं काञ्चनाङ्गदभूषणम्}
{दृष्ट्वा पुनः प्रदुद्राव वानराणां महाचमूः} %6-61-3

\twolineshloka
{विद्रुतां वाहिनीं दृष्ट्वा वर्धमानं च राक्षसम्}
{सविस्मितमिदं रामो विभीषणमुवाच ह} %6-61-4

\twolineshloka
{कोऽसौ पर्वतसङ्काशः किरीटी हरिलोचनः}
{लङ्कायां दृश्यते वीरः सविद्युदिव तोयदः} %6-61-5

\twolineshloka
{पृथिव्यां केतुभूतोऽसौ महानेकोऽत्र दृश्यते}
{यं दृष्ट्वा वानराः सर्वे विद्रवन्ति ततस्ततः} %6-61-6

\twolineshloka
{आचक्ष्व सुमहान् कोऽसौ रक्षो वा यदि वासुरः}
{न मयैवंविधं भूतं दृष्टपूर्वं कदाचन} %6-61-7

\twolineshloka
{सम्पृष्टो राजपुत्रेण रामेणाक्लिष्टकर्मणा}
{विभीषणो महाप्राज्ञः काकुत्स्थमिदमब्रवीत्} %6-61-8

\threelineshloka
{येन वैवस्वतो युद्धे वासवश्च पराजितः}
{सैष विश्रवसः पुत्रः कुम्भकर्णः प्रतापवान्}
{अस्य प्रमाणसदृशो राक्षसोऽन्यो न विद्यते} %6-61-9

\twolineshloka
{एतेन देवा युधि दानवाश्च यक्षा भुजङ्गाः पिशिताशनाश्च}
{गन्धर्वविद्याधरकिन्नराश्च सहस्रशो राघव सम्प्रभग्नाः} %6-61-10

\twolineshloka
{शूलपाणिं विरूपाक्षं कुम्भकर्णं महाबलम्}
{हन्तुं न शेकुस्त्रिदशाः कालोऽयमिति मोहिताः} %6-61-11

\twolineshloka
{प्रकृत्या ह्येष तेजस्वी कुम्भकर्णो महाबलः}
{अन्येषां राक्षसेन्द्राणां वरदानकृतं बलम्} %6-61-12

\twolineshloka
{बालेन जातमात्रेण क्षुधार्तेन महात्मना}
{भक्षितानि सहस्राणि प्रजानां सुबहून्यपि} %6-61-13

\twolineshloka
{तेषु सम्भक्ष्यमाणेषु प्रजा भयनिपीडिताः}
{यान्त स्म शरणं शक्रं तमप्यर्थं न्यवेदयन्} %6-61-14

\twolineshloka
{स कुम्भकर्णं कुपितो महेन्द्रो जघान वज्रेण शितेन वज्री}
{स शक्रवज्राभिहतो महात्मा चचाल कोपाच्च भृशं ननाद} %6-61-15

\twolineshloka
{तस्य नानद्यमानस्य कुम्भकर्णस्य रक्षसः}
{श्रुत्वा निनादं वित्रस्ताः प्रजा भूयो वितत्रसुः} %6-61-16

\twolineshloka
{ततः क्रुद्धो महेन्द्रस्य कुम्भकर्णो महाबलः}
{निष्कृष्यैरावताद् दन्तं जघानोरसि वासवम्} %6-61-17

\twolineshloka
{कुम्भकर्णप्रहारार्तो विजज्वाल स वासवः}
{ततो विषेदुः सहसा देवा ब्रह्मर्षिदानवाः} %6-61-18

\twolineshloka
{प्रजाभिः सह शक्रश्च ययौ स्थानं स्वयम्भुवः}
{कुम्भकर्णस्य दौरात्म्यं शशंसुस्ते प्रजापतेः} %6-61-19

\twolineshloka
{प्रजानां भक्षणं चापि देवानां चापि धर्षणम्}
{आश्रमध्वंसनं चापि परस्त्रीहरणं भृशम्} %6-61-20

\twolineshloka
{एवं प्रजा यदि त्वेष भक्षयिष्यति नित्यशः}
{अचिरेणैव कालेन शून्यो लोको भविष्यति} %6-61-21

\twolineshloka
{वासवस्य वचः श्रुत्वा सर्वलोकपितामहः}
{रक्षांस्यावाहयामास कुम्भकर्णं ददर्श ह} %6-61-22

\twolineshloka
{कुम्भकर्णं समीक्ष्यैव वितत्रास प्रजापतिः}
{कुम्भकर्णमथाश्वास्तः स्वयम्भूरिदमब्रवीत्} %6-61-23

\twolineshloka
{ध्रुवं लोकविनाशाय पौलस्त्येनासि निर्मितः}
{तस्मात् त्वमद्यप्रभृति मृतकल्पः शयिष्यसे} %6-61-24

\twolineshloka
{ब्रह्मशापाभिभूतोऽथ निपपाताग्रतः प्रभोः}
{ततः परमसम्भ्रान्तो रावणो वाक्यमब्रवीत्} %6-61-25

\twolineshloka
{प्रवृद्धः काञ्चनो वृक्षः फलकाले निकृत्यते}
{न नप्तारं स्वकं न्याय्यं शप्तुमेवं प्रजापते} %6-61-26

\twolineshloka
{न मिथ्यावचनश्च त्वं स्वप्स्यत्येव न संशयः}
{कालस्तु क्रियतामस्य शयने जागरे तथा} %6-61-27

\twolineshloka
{रावणस्य वचः श्रुत्वा स्वयम्भूरिदमब्रवीत्}
{शयिता ह्येष षण्मासमेकाहं जागरिष्यति} %6-61-28

\twolineshloka
{एकेनाह्ना त्वसौ वीरश्चरन् भूमिं बुभुक्षितः}
{व्यात्तास्यो भक्षयेल्लोकान् संवृद्ध इव पावकः} %6-61-29

\twolineshloka
{सोऽसौ व्यसनमापन्नः कुम्भकर्णमबोधयत्}
{त्वत्पराक्रमभीतश्च राजा सम्प्रति रावणः} %6-61-30

\twolineshloka
{स एष निर्गतो वीरः शिबिराद् भीमविक्रमः}
{वानरान् भृशसङ्क्रुद्धो भक्षयन् परिधावति} %6-61-31

\twolineshloka
{कुम्भकर्णं समीक्ष्यैव हरयोऽद्य प्रदुद्रुवुः}
{कथमेनं रणे क्रुद्धं वारयिष्यन्ति वानराः} %6-61-32

\twolineshloka
{उच्यन्तां वानराः सर्वे यन्त्रमेतत् समुच्छ्रितम्}
{इति विज्ञाय हरयो भविष्यन्तीह निर्भयाः} %6-61-33

\twolineshloka
{विभीषणवचः श्रुत्वा हेतुमत् सुमुखोद्गतम्}
{उवाच राघवो वाक्यं नीलं सेनापतिं तदा} %6-61-34

\twolineshloka
{गच्छ सैन्यानि सर्वाणि व्यूह्य तिष्ठस्व पावके}
{द्वाराण्यादाय लङ्कायाश्चर्याश्चास्याथ सङ्क्रमान्} %6-61-35

\twolineshloka
{शैलशृङ्गाणि वृक्षांश्च शिलाश्चाप्युपसंहरन्}
{भवन्तः सायुधाः सर्वे वानराः शैलपाणयः} %6-61-36

\twolineshloka
{राघवेण समादिष्टो नीलो हरिचमूपतिः}
{शशास वानरानीकं यथावत् कपिकुञ्जरः} %6-61-37

\twolineshloka
{ततो गवाक्षः शरभो हनूमानङ्गदस्तथा}
{शैलशृङ्गाणि शैलाभा गृहीत्वा द्वारमभ्ययुः} %6-61-38

\twolineshloka
{रामवाक्यमुपश्रुत्य हरयो जितकाशिनः}
{पादपैरर्दयन् वीरा वानराः परवाहिनीम्} %6-61-39

\twolineshloka
{ततो हरीणां तदनीकमुग्रं रराज शैलोद्यतवृक्षहस्तम्}
{गिरेः समीपानुगतं यथैव महन्महाम्भोधरजालमुग्रम्} %6-61-40


॥इत्यार्षे श्रीमद्रामायणे वाल्मीकीये आदिकाव्ये युद्धकाण्डे कुम्भकर्णवृत्तकथनम् नाम एकषष्ठितमः सर्गः ॥६-६१॥
