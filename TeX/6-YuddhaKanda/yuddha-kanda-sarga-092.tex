\sect{द्विनवतितमः सर्गः — रावणिशस्त्रहतचिकित्सा}

\twolineshloka
{रुधिरक्लिन्नगात्रस्तु लक्ष्मणः शुभलक्षणः}
{बभूव हृष्टस्तं हत्वा शत्रुजेतारमाहवे} %6-92-1

\twolineshloka
{ततः स जाम्बवन्तं च हनूमन्तं च वीर्यवान्}
{संनिपत्य महातेजास्तांश्च सर्वान् वनौकसः} %6-92-2

\twolineshloka
{आजगाम ततः शीघ्रं यत्र सुग्रीवराघवौ}
{विभीषणमवष्टभ्य हनूमन्तं च लक्ष्मणः} %6-92-3

\twolineshloka
{ततो राममभिक्रम्य सौमित्रिरभिवाद्य च}
{तस्थौ भ्रातृसमीपस्थः शक्रस्येन्द्रानुजो यथा} %6-92-4

\twolineshloka
{निष्टनन्निव चागत्य राघवाय महात्मने}
{आचचक्षे तदा वीरो घोरमिन्द्रजितो वधम्} %6-92-5

\twolineshloka
{रावणेस्तु शिरश्छिन्नं लक्ष्मणेन महात्मना}
{न्यवेदयत रामाय तदा हृष्टो विभीषणः} %6-92-6

\twolineshloka
{श्रुत्वैव तु महावीर्यो लक्ष्मणेनेन्द्रजिद्वधम्}
{प्रहर्षमतुलं लेभे वाक्यं चेदमुवाच ह} %6-92-7

\twolineshloka
{साधु लक्ष्मण तुष्टोऽस्मि कर्म चासुकरं कृतम्}
{रावणेर्हि विनाशेन जितमित्युपधारय} %6-92-8

\twolineshloka
{स तं शिरस्युपाघ्राय लक्ष्मणं कीर्तिवर्धनम्}
{लज्जमानं बलात् स्नेहादङ्कमारोप्य वीर्यवान्} %6-92-9

\twolineshloka
{उपवेश्य तमुत्सङ्गे परिष्वज्यावपीडितम्}
{भ्रातरं लक्ष्मणं स्निग्धं पुनः पुनरुदैक्षत} %6-92-10

\twolineshloka
{शल्यसम्पीडितं शस्तं निःश्वसन्तं तु लक्ष्मणम्}
{रामस्तु दुःखसंतप्तं तं तु निःश्वासपीडितम्} %6-92-11

\twolineshloka
{मूर्ध्नि चैनमुपाघ्राय भूयः संस्पृश्य च त्वरन्}
{उवाच लक्ष्मणं वाक्यमाश्वास्य पुरुषर्षभः} %6-92-12

\twolineshloka
{कृतं परमकल्याणं कर्म दुष्करकर्मणा}
{अद्य मन्ये हते पुत्रे रावणं निहतं युधि} %6-92-13

\twolineshloka
{अद्याहं विजयी शत्रौ हते तस्मिन् दुरात्मनि}
{रावणस्य नृशंसस्य दिष्ट्या वीर त्वया रणे} %6-92-14

\twolineshloka
{छिन्नो हि दक्षिणो बाहुः स हि तस्य व्यपाश्रयः}
{विभीषणहनूमद्भ्यां कृतं कर्म महद् रणे} %6-92-15

\twolineshloka
{अहोरात्रैस्त्रिभिर्वीरः कथंचिद् विनिपातितः}
{निरमित्रः कृतोऽस्म्यद्य निर्यास्यति हि रावणः} %6-92-16

\twolineshloka
{बलव्यूहेन महता निर्यास्यति हि रावणः}
{बलव्यूहेन महता श्रुत्वा पुत्रं निपातितम्} %6-92-17

\twolineshloka
{तं पुत्रवधसंतप्तं निर्यान्तं राक्षसाधिपम्}
{बलेनावृत्य महता निहनिष्यामि दुर्जयम्} %6-92-18

\twolineshloka
{त्वया लक्ष्मण नाथेन सीता च पृथिवी च मे}
{न दुष्प्रापा हते तस्मिन् शक्रजेतरि चाहवे} %6-92-19

\twolineshloka
{स तं भ्रातरमाश्वास्य परिष्वज्य च राघवः}
{रामः सुषेणं मुदितः समाभाष्येदमब्रवीत्} %6-92-20

\twolineshloka
{विशल्योऽयं महाप्राज्ञ सौमित्रिर्मित्रवत्सलः}
{यथा भवति सुस्वस्थस्तथा त्वं समुपाचर} %6-92-21

\twolineshloka
{विशल्यः क्रियतां क्षिप्रं सौमित्रिः सविभीषणः}
{ऋक्षवानरसैन्यानां शूराणां द्रुमयोधिनाम्} %6-92-22

\twolineshloka
{ये चाप्यन्येऽत्र युध्यन्ति सशल्या व्रणिनस्तथा}
{तेऽपि सर्वे प्रयत्नेन क्रियन्ते सुखिनस्त्वया} %6-92-23

\twolineshloka
{एवमुक्तः स रामेण महात्मा हरियूथपः}
{लक्ष्मणाय ददौ नस्तः सुषेणः परमौषधम्} %6-92-24

\twolineshloka
{स तस्य गन्धमाघ्राय विशल्यः समपद्यत}
{तदा निर्वेदनश्चैव संरूढव्रण एव च} %6-92-25

\twolineshloka
{विभीषणमुखानां च सुहृदां राघवाज्ञया}
{सर्ववानरमुख्यानां चिकित्सामकरोत् तदा} %6-92-26

\twolineshloka
{ततः प्रकृतिमापन्नो हृतशल्यो गतक्लमः}
{सौमित्रिर्मुमुदे तत्र क्षणेन विगतज्वरः} %6-92-27

\twolineshloka
{तदैव रामः प्लवगाधिपस्तथा विभीषणश्चर्क्षपतिश्च वीर्यवान्}
{अवेक्ष्य सौमित्रिमरोगमुत्थितं मुदा ससैन्याः सुचिरं जहर्षिरे} %6-92-28

\twolineshloka
{अपूजयत् कर्म स लक्ष्मणस्य सुदुष्करं दाशरथिर्महात्मा}
{बभूव हृष्टो युधि वानरेन्द्रो निशम्य तं शक्रजितं निपातितम्} %6-92-29


॥इत्यार्षे श्रीमद्रामायणे वाल्मीकीये आदिकाव्ये युद्धकाण्डे रावणिशस्त्रहतचिकित्सा नाम द्विनवतितमः सर्गः ॥६-९२॥
