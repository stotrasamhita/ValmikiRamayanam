\sect{चतुःसप्ततितमः सर्गः — ओषधिपर्वतानयनम्}

\twolineshloka
{तयोस्तदासादितयो रणाग्रे मुमोह सैन्यं हरियूथपानाम्}
{सुग्रीवनीलाङ्गदजाम्बवन्तो न चापि किञ्चित् प्रतिपेदिरे ते} %6-74-1

\twolineshloka
{ततो विषण्णं समवेक्ष्य सर्वं विभीषणो बुद्धिमतां वरिष्ठः}
{उवाच शाखामृगराजवीरानाश्वासयन्नप्रतिमैर्वचोभिः} %6-74-2

\twolineshloka
{मा भैष्ट नास्त्यत्र विषादकालो यदार्यपुत्रौ ह्यवशौ विषण्णौ}
{स्वयम्भुवो वाक्यमथोद्वहन्तौ यत्सादिताविन्द्रजितास्त्रजालैः} %6-74-3

\twolineshloka
{तस्मै तु दत्तं परमास्त्रमेतत् स्वयम्भुवा ब्राह्मममोघवीर्यम्}
{तन्मानयन्तौ युधि राजपुत्रौ निपातितौ कोऽत्र विषादकालः} %6-74-4

\twolineshloka
{ब्राह्ममस्त्रं ततो धीमान् मानयित्वा तु मारुतिः}
{विभीषणवचः श्रुत्वा हनूमानिदमब्रवीत्} %6-74-5

\twolineshloka
{अस्मिन्नस्त्रहते सैन्ये वानराणां तरस्विनाम्}
{यो यो धारयते प्राणांस्तं तमाश्वासयावहे} %6-74-6

\twolineshloka
{तावुभौ युगपद् वीरौ हनूमद्राक्षसोत्तमौ}
{उल्काहस्तौ तदा रात्रौ रणशीर्षे विचेरतुः} %6-74-7

\twolineshloka
{भिन्नलाङ्गूलहस्तोरुपादाङ्गुलिशिरोधरैः}
{स्रवद्भिः क्षतजं गात्रैः प्रस्रवद्भिः समन्ततः} %6-74-8

\twolineshloka
{पतितैः पर्वताकारैर्वानरैरभिसंवृताम्}
{शस्त्रैश्च पतितैर्दीप्तैर्ददृशाते वसुन्धराम्} %6-74-9

\twolineshloka
{सुग्रीवमङ्गदं नीलं शरभं गन्धमादनम्}
{जाम्बवन्तं सुषेणं च वेगदर्शिनमेव च} %6-74-10

\twolineshloka
{मैन्दं नलं ज्योतिर्मुखं द्विविदं चापि वानरम्}
{विभीषणो हनूमांश्च ददृशाते हतान् रणे} %6-74-11

\twolineshloka
{सप्तषष्टिर्हताः कोट्यो वानराणां तरस्विनाम्}
{अह्नः पञ्चमशेषेण वल्लभेन स्वयम्भुवः} %6-74-12

\twolineshloka
{सागरौघनिभं भीमं दृष्ट्वा बाणार्दितं बलम्}
{मार्गते जाम्बवन्तं च हनूमान् सविभीषणः} %6-74-13

\twolineshloka
{स्वभावजरया युक्तं वृद्धं शरशतैश्चितम्}
{प्रजापतिसुतं वीरं शाम्यन्तमिव पावकम्} %6-74-14

\twolineshloka
{दृष्ट्वा समभिसङ्क्रम्य पौलस्त्यो वाक्यमब्रवीत्}
{कच्चिदार्य शरैस्तीक्ष्णैर्न प्राणा ध्वंसितास्तव} %6-74-15

\twolineshloka
{विभीषणवचः श्रुत्वा जाम्बवानृक्षपुङ्गवः}
{कृच्छ्रादभ्युद्गिरन् वाक्यमिदं वचनमब्रवीत्} %6-74-16

\twolineshloka
{नैर्ऋतेन्द्र महावीर्य स्वरेण त्वाभिलक्षये}
{विद्धगात्रः शितैर्बाणैर्न त्वां पश्यामि चक्षुषा} %6-74-17

\twolineshloka
{अञ्जना सुप्रजा येन मातरिश्वा च सुव्रत}
{हनूमान् वानरश्रेष्ठः प्राणान् धारयते क्वचित्} %6-74-18

\twolineshloka
{श्रुत्वा जाम्बवतो वाक्यमुवाचेदं विभीषणः}
{आर्यपुत्रावतिक्रम्य कस्मात् पृच्छसि मारुतिम्} %6-74-19

\twolineshloka
{नैव राजनि सुग्रीवे नाङ्गदे नापि राघवे}
{आर्य सन्दर्शितः स्नेहो यथा वायुसुते परः} %6-74-20

\twolineshloka
{विभीषणवचः श्रुत्वा जाम्बवान् वाक्यमब्रवीत्}
{शृणु नैर्ऋतशार्दूल यस्मात् पृच्छामि मारुतिम्} %6-74-21

\twolineshloka
{अस्मिञ्जीवति वीरे तु हतमप्यहतं बलम्}
{हनूमत्युज्झितप्राणे जीवन्तोऽपि मृता वयम्} %6-74-22

\twolineshloka
{धरते मारुतिस्तात मारुतप्रतिमो यदि}
{वैश्वानरसमो वीर्ये जीविताशा ततो भवेत्} %6-74-23

\twolineshloka
{ततो वृद्धमुपागम्य विनयेनाभ्यवादयत्}
{गृह्य जाम्बवतः पादौ हनूमान् मारुतात्मजः} %6-74-24

\twolineshloka
{श्रुत्वा हनूमतो वाक्यं तदा विव्यथितेन्द्रियः}
{पुनर्जातमिवात्मानं मन्यते स्मर्क्षपुङ्गवः} %6-74-25

\twolineshloka
{ततोऽब्रवीन्महातेजा हनूमन्तं स जाम्बवान्}
{आगच्छ हरिशार्दूल वानरांस्त्रातुमर्हसि} %6-74-26

\twolineshloka
{नान्यो विक्रमपर्याप्तस्त्वमेषां परमः सखा}
{त्वत्पराक्रमकालोऽयं नान्यं पश्यामि कञ्चन} %6-74-27

\twolineshloka
{ऋक्षवानरवीराणामनीकानि प्रहर्षय}
{विशल्यौ कुरु चाप्येतौ सादितौ रामलक्ष्मणौ} %6-74-28

\twolineshloka
{गत्वा परममध्वानमुपर्युपरि सागरम्}
{हिमवन्तं नगश्रेष्ठं हनूमन् गन्तुमर्हसि} %6-74-29

\twolineshloka
{ततः काञ्चनमत्युच्चमृषभं पर्वतोत्तमम्}
{कैलासशिखरं चात्र द्रक्ष्यस्यरिनिषूदन} %6-74-30

\twolineshloka
{तयोः शिखरयोर्मध्ये प्रदीप्तमतुलप्रभम्}
{सर्वौषधियुतं वीर द्रक्ष्यस्योषधिपर्वतम्} %6-74-31

\twolineshloka
{तस्य वानरशार्दूल चतस्रो मूर्ध्नि सम्भवाः}
{द्रक्ष्यस्योषधयो दीप्ता दीपयन्तीर्दिशो दश} %6-74-32

\twolineshloka
{मृतसञ्जीवनीं चैव विशल्यकरणीमपि}
{सुवर्णकरणीं चैव सन्धानीं च महौषधीम्} %6-74-33

\twolineshloka
{ताः सर्वा हनुमन् गृह्य क्षिप्रमागन्तुमर्हसि}
{आश्वासय हरीन् प्राणैर्योज्य गन्धवहात्मज} %6-74-34

\twolineshloka
{श्रुत्वा जाम्बवतो वाक्यं हनूमान् मारुतात्मजः}
{आपूर्यत बलोद्धर्षैर्वायुवेगैरिवार्णवः} %6-74-35

\twolineshloka
{स पर्वततटाग्रस्थः पीडयन् पर्वतोत्तमम्}
{हनूमान् दृश्यते वीरो द्वितीय इव पर्वतः} %6-74-36

\twolineshloka
{हरिपादविनिर्भग्नो निषसाद स पर्वतः}
{न शशाक तदात्मानं वोढुं भृशनिपीडितः} %6-74-37

\twolineshloka
{तस्य पेतुर्नगा भूमौ हरिवेगाच्च जज्वलुः}
{शृङ्गाणि च व्यकीर्यन्त पीडितस्य हनूमता} %6-74-38

\twolineshloka
{तस्मिन् सम्पीड्यमाने तु भग्नद्रुमशिलातले}
{न शेकुर्वानराः स्थातुं घूर्णमाने नगोत्तमे} %6-74-39

\twolineshloka
{सा घूर्णितमहाद्वारा प्रभग्नगृहगोपुरा}
{लङ्का त्रासाकुला रात्रौ प्रनृत्तेवाभवत् तदा} %6-74-40

\twolineshloka
{पृथिवीधरसङ्काशो निपीड्य पृथिवीधरम्}
{पृथिवीं क्षोभयामास सार्णवां मारुतात्मजः} %6-74-41

\twolineshloka
{आरुरोह तदा तस्माद्धरिर्मलयपर्वतम्}
{मेरुमन्दरसङ्काशं नानाप्रस्रवणाकुलम्} %6-74-42

\twolineshloka
{नानाद्रुमलताकीर्णं विकासिकमलोत्पलम्}
{सेवितं देवगन्धर्वैः षष्टियोजनमुच्छ्रितम्} %6-74-43

\twolineshloka
{विद्याधरैर्मुनिगणैरप्सरोभिर्निषेवितम्}
{नानामृगगणाकीर्णं बहुकन्दरशोभितम्} %6-74-44

\twolineshloka
{सर्वानाकुलयंस्तत्र यक्षगन्धर्वकिन्नरान्}
{हनूमान् मेघसङ्काशो ववृधे मारुतात्मजः} %6-74-45

\twolineshloka
{पद्भ्यां तु शैलमापीड्य वडवामुखवन्मुखम्}
{विवृत्योग्रं ननादोच्चैस्त्रासयन् रजनीचरान्} %6-74-46

\twolineshloka
{तस्य नानद्यमानस्य श्रुत्वा निनदमुत्तमम्}
{लङ्कास्था राक्षसव्याघ्रा न शेकुः स्पन्दितुं क्वचित्} %6-74-47

\twolineshloka
{नमस्कृत्वा समुद्राय मारुतिर्भीमविक्रमः}
{राघवार्थे परं कर्म समीहत परन्तपः} %6-74-48

\twolineshloka
{स पुच्छमुद्यम्य भुजङ्गकल्पं विनम्य पृष्ठं श्रवणे निकुच्य}
{विवृत्य वक्त्रं वडवामुखाभमापुप्लुवे व्योम्नि स चण्डवेगः} %6-74-49

\twolineshloka
{स वृक्षखण्डांस्तरसा जहार शैलान् शिलाः प्राकृतवानरांश्च}
{बाहूरुवेगोद्गतसम्प्रणुन्नास्ते क्षीणवेगाः सलिले निपेतुः} %6-74-50

\twolineshloka
{स तौ प्रसार्योरगभोगकल्पौ भुजौ भुजङ्गारिनिकाशवीर्यः}
{जगाम शैलं नगराजमग्र्यं दिशः प्रकर्षन्निव वायुसूनुः} %6-74-51

\twolineshloka
{स सागरं घूर्णितवीचिमालं तदम्भसा भ्रामितसर्वसत्त्वम्}
{समीक्षमाणः सहसा जगाम चक्रं यथा विष्णुकराग्रमुक्तम्} %6-74-52

\twolineshloka
{स पर्वतान् पक्षिगणान् सरांसि नदीस्तटाकानि पुरोत्तमानि}
{स्फीताञ्जनांस्तानपि सम्प्रवीक्ष्य जगाम वेगात् पितृतुल्यवेगः} %6-74-53

\twolineshloka
{आदित्यपथमाश्रित्य जगाम स गतश्रमः}
{हनूमांस्त्वरितो वीरः पितुस्तुल्यपराक्रमः} %6-74-54

\twolineshloka
{जवेन महता युक्तो मारुतिर्वातरंहसा}
{जगाम हरिशार्दूलो दिशः शब्देन नादयन्} %6-74-55

\twolineshloka
{स्मरञ्जाम्बवतो वाक्यं मारुतिर्भीमविक्रमः}
{ददर्श सहसा चापि हिमवन्तं महाकपिः} %6-74-56

\threelineshloka
{नानाप्रस्रवणोपेतं बहुकन्दरनिर्झरम्}
{श्वेताभ्रचयसङ्काशैः शिखरैश्चारुदर्शनैः}
{शोभितं विविधैर्वृक्षैरगमत् पर्वतोत्तमम्} %6-74-57

\twolineshloka
{स तं समासाद्य महानगेन्द्रमतिप्रवृद्धोत्तमहेमशृङ्गम्}
{ददर्श पुण्यानि महाश्रमाणि सुरर्षिसङ्घोत्तमसेवितानि} %6-74-58

\twolineshloka
{स ब्रह्मकोशं रजतालयं च शक्रालयं रुद्रशरप्रमोक्षम्}
{हयाननं ब्रह्मशिरश्च दीप्तं ददर्श वैवस्वतकिङ्करांश्च} %6-74-59

\twolineshloka
{वह्न्यालयं वैश्रवणालयं च सूर्यप्रभं सूर्यनिबन्धनं च}
{ब्रह्मालयं शङ्करकार्मुकं च ददर्श नाभिं च वसुन्धरायाः} %6-74-60

\twolineshloka
{कैलासमग्र्यं हिमवच्छिलां च तं वै वृषं काञ्चनशैलमग्र्यम्}
{प्रदीप्तसर्वौषधिसम्प्रदीप्तं ददर्श सर्वौषधिपर्वतेन्द्रम्} %6-74-61

\twolineshloka
{स तं समीक्ष्यानलराशिदीप्तं विसिस्मिये वासवदूतसूनुः}
{आप्लुत्य तं चौषधिपर्वतेन्द्रं तत्रौषधीनां विचयं चकार} %6-74-62

\twolineshloka
{स योजनसहस्राणि समतीत्य महाकपिः}
{दिव्यौषधिधरं शैलं व्यचरन्मारुतात्मजः} %6-74-63

\twolineshloka
{महौषध्यस्ततः सर्वास्तस्मिन् पर्वतसत्तमे}
{विज्ञायार्थिनमायान्तं ततो जग्मुरदर्शनम्} %6-74-64

\twolineshloka
{स ता महात्मा हनुमानपश्यंश्चुकोप रोषाच्च भृशं ननाद}
{अमृष्यमाणोऽग्निसमानचक्षुर्महीधरेन्द्रं तमुवाच वाक्यम्} %6-74-65

\twolineshloka
{किमेतदेवं सुविनिश्चितं ते यद् राघवे नासि कृतानुकम्पः}
{पश्याद्य मद्बाहुबलाभिभूतो विकीर्णमात्मानमथो नगेन्द्र} %6-74-66

\twolineshloka
{स तस्य शृङ्गं सनगं सनागं सकाञ्चनं धातुसहस्रजुष्टम्}
{विकीर्णकूटं ज्वलिताग्रसानुं प्रगृह्य वेगात् सहसोन्ममाथ} %6-74-67

\twolineshloka
{स तं समुत्पाट्य खमुत्पपात वित्रास्य लोकान् ससुरासुरेन्द्रान्}
{संस्तूयमानः खचरैरनेकैर्जगाम वेगाद् गरुडोग्रवेगः} %6-74-68

\twolineshloka
{स भास्कराध्वानमनुप्रपन्नस्तं भास्कराभं शिखरं प्रगृह्य}
{बभौ तदा भास्करसन्निकाशो रवेः समीपे प्रतिभास्कराभः} %6-74-69

\twolineshloka
{स तेन शैलेन भृशं रराज शैलोपमो गन्धवहात्मजस्तु}
{सहस्रधारेण सपावकेन चक्रेण खे विष्णुरिवार्पितेन} %6-74-70

\twolineshloka
{तं वानराः प्रेक्ष्य तदा विनेदुः स तानपि प्रेक्ष्य मुदा ननाद}
{तेषां समुत्कृष्टरवं निशम्य लङ्कालया भीमतरं विनेदुः} %6-74-71

\twolineshloka
{ततो महात्मा निपपात तस्मिन् शैलोत्तमे वानरसैन्यमध्ये}
{हर्युत्तमेभ्यः शिरसाभिवाद्य विभीषणं तत्र च सस्वजे सः} %6-74-72

\twolineshloka
{तावप्युभौ मानुषराजपुत्रौ तं गन्धमाघ्राय महौषधीनाम्}
{बभूवतुस्तत्र तदा विशल्यावुत्तस्थुरन्ये च हरिप्रवीराः} %6-74-73

\twolineshloka
{सर्वे विशल्या विरुजाः क्षणेन हरिप्रवीराश्च हताश्च ये स्युः}
{गन्धेन तासां प्रवरौषधीनां सुप्ता निशान्तेष्विव सम्प्रबुद्धाः} %6-74-74

\twolineshloka
{यदाप्रभृति लङ्कायां युध्यन्ते हरिराक्षसाः}
{तदाप्रभृति मानार्थमाज्ञया रावणस्य च} %6-74-75

\twolineshloka
{ये हन्यन्ते रणे तत्र राक्षसाः कपिकुञ्जरैः}
{हता हतास्तु क्षिप्यन्ते सर्व एव तु सागरे} %6-74-76

\twolineshloka
{ततो हरिर्गन्धवहात्मजस्तु तमोषधीशैलमुदग्रवेगः}
{निनाय वेगाद्धिमवन्तमेव पुनश्च रामेण समाजगाम} %6-74-77


॥इत्यार्षे श्रीमद्रामायणे वाल्मीकीये आदिकाव्ये युद्धकाण्डे ओषधिपर्वतानयनम् नाम चतुःसप्ततितमः सर्गः ॥६-७४॥
