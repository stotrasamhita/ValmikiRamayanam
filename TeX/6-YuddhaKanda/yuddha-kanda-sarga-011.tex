\sect{एकादशः सर्गः — द्वितीयमन्त्राधिवेषः}

\twolineshloka
{स बभूव कृशो राजा मैथिलीकाममोहितः}
{असन्मानाच्च सुहृदां पापः पापेन कर्मणा} %6-11-1

\threelineshloka
{अतीव कामसम्पन्नो वैदेहीमनुचिन्तयन्}
{अतीतसमये काले तस्मिन् वै युधि रावणः}
{अमात्यैश्च सुहृद्भिश्च प्राप्तकालममन्यत} %6-11-2

\twolineshloka
{स हेमजालविततं मणिविद्रुमभूषितम्}
{उपगम्य विनीताश्वमारुरोह महारथम्} %6-11-3

\twolineshloka
{तमास्थाय रथश्रेष्ठं महामेघसमस्वनम्}
{प्रययौ रक्षसां श्रेष्ठो दशग्रीवः सभां प्रति} %6-11-4

\twolineshloka
{असिचर्मधरा योधाः सर्वायुधधरास्ततः}
{राक्षसा राक्षसेन्द्रस्य पुरस्तात् सम्प्रतस्थिरे} %6-11-5

\twolineshloka
{नानाविकृतवेषाश्च नानाभूषणभूषिताः}
{पार्श्वतः पृष्ठतश्चैनं परिवार्य ययुस्तदा} %6-11-6

\twolineshloka
{रथैश्चातिरथाः शीघ्रं मत्तैश्च वरवारणैः}
{अनूत्पेतुर्दशग्रीवमाक्रीडद्भिश्च वाजिभिः} %6-11-7

\threelineshloka
{गदापरिघहस्ताश्च शक्तितोमरपाणयः}
{परश्वधधराश्चान्ये तथान्ये शूलपाणयः}
{ततस्तूर्यसहस्राणं सञ्जज्ञे निःस्वनो महान्} %6-11-8

\twolineshloka
{तुमुलः शङ्खशब्दश्च सभां गच्छति रावणे}
{स नेमिघोषेण महान् सहसाभिनिनादयन्} %6-11-9

\twolineshloka
{राजमार्गं श्रिया जुष्टं प्रतिपेदे महारथः}
{विमलं चातपत्रं च प्रगृहीतमशोभत} %6-11-10

\twolineshloka
{पाण्डुरं राक्षसेन्द्रस्य पूर्णस्ताराधिपो यथा}
{हेममञ्जरिगर्भे च शुद्धस्फटिकविग्रहे} %6-11-11

\twolineshloka
{चामरव्यजने तस्य रेजतुः सव्यदक्षिणे}
{ते कृताञ्जलयः सर्वे रथस्थं पृथिवीस्थिताः} %6-11-12

\twolineshloka
{राक्षसा राक्षसश्रेष्ठं शिरोभिस्तं ववन्दिरे}
{राक्षसैः स्तूयमानः सञ्जयाशीर्भिररिन्दमः} %6-11-13

\twolineshloka
{आससाद महातेजाः सभां विरचितां तदा}
{सुवर्णरजतास्तीर्णां विशुद्धस्फटिकान्तराम्} %6-11-14

\twolineshloka
{विराजमानो वपुषा रुक्मपट्टोत्तरच्छदाम्}
{तां पिशाचशतैः षड्भिरभिगुप्तां सदाप्रभाम्} %6-11-15

\twolineshloka
{प्रविवेश महातेजाः सुकृतां विश्वकर्मणा}
{तस्यां तु वैदूर्यमयं प्रियकाजिनसंवृतम्} %6-11-16

\twolineshloka
{महत्सोपाश्रयं भेजे रावणः परमासनम्}
{ततः शशासेश्वरवद्दूताँल्लघुपराक्रमान्} %6-11-17

\twolineshloka
{समानयत मे क्षिप्रमिहैतान् राक्षसानिति}
{कृत्यमस्ति महज्जाने कर्तव्यमिति शत्रुभिः} %6-11-18

\threelineshloka
{राक्षसास्तद्वचः श्रुत्वा लङ्कायां परिचक्रमुः}
{अनुगेहमवस्थाय विहारशयनेषु च}
{उद्यानेषु च रक्षांसि चोदयन्तो ह्यभीतवत्} %6-11-19

\twolineshloka
{ते रथान्तचरा एके दृप्तानेके दृढान् हयान्}
{नागानेकेऽधिरुरुहुर्जग्मुश्चैके पदातयः} %6-11-20

\twolineshloka
{सा पुरी परमाकीर्णा रथकुञ्जरवाजिभिः}
{सम्पतद्भिर्विरुरुचे गरुत्मद्भिरिवाम्बरम्} %6-11-21

\twolineshloka
{ते वाहनान्यवस्थाय यानानि विविधानि च}
{सभां पद्भिः प्रविविशुः सिंहा गिरिगुहामिव} %6-11-22

\twolineshloka
{राज्ञः पादौ गृहीत्वा तु राज्ञा ते प्रतिपूजिताः}
{पीठेष्वन्ये बृसीष्वन्ये भूमौ केचिदुपाविशन्} %6-11-23

\twolineshloka
{ते समेत्य सभायां वै राक्षसा राजशासनात्}
{यथार्हमुपतस्थुस्ते रावणं राक्षसाधिपम्} %6-11-24

\twolineshloka
{मन्त्रिणश्च यथामुख्या निश्चितार्थेषु पण्डिताः}
{अमात्याश्च गुणोपेताः सर्वज्ञा बुद्धिदर्शनाः} %6-11-25

\twolineshloka
{समीयुस्तत्र शतशः शूराश्च बहवस्तथा}
{सभायां हेमवर्णायां सर्वार्थस्य सुखाय वै} %6-11-26

\twolineshloka
{ततो महात्मा विपुलं सुयुग्यं रथं वरं हेमविचित्रिताङ्गम्}
{शुभं समास्थाय ययौ यशस्वी विभीषणः संसदमग्रजस्य} %6-11-27

\twolineshloka
{स पूर्वजायावरजः शशंस नामाथ पश्चाच्चरणौ ववन्दे}
{शुकः प्रहस्तश्च तथैव तेभ्यो ददौ यथार्हं पृथगासनानि} %6-11-28

\twolineshloka
{सुवर्णनानामणिभूषणानां सुवाससां संसदि राक्षसानाम्}
{तेषां परार्घ्यागुरुचन्दनानां स्रजां च गन्धाः प्रववुः समन्तात्} %6-11-29

\twolineshloka
{न चुक्रुशुर्नानृतमाह कश्चित् सभासदो नापि जजल्पुरुच्चैः}
{संसिद्धार्थाः सर्व एवोग्रवीर्या भर्तुः सर्वे ददृशुश्चाननं ते} %6-11-30

\twolineshloka
{स रावणः शस्त्रभृतां मनस्विनां महाबलानां समितौ मनस्वी}
{तस्यां सभायां प्रभया चकाशे मध्ये वसूनामिव वज्रहस्तः} %6-11-31


॥इत्यार्षे श्रीमद्रामायणे वाल्मीकीये आदिकाव्ये युद्धकाण्डे द्वितीयमन्त्राधिवेषः नाम एकादशः सर्गः ॥६-११॥
