\sect{त्रिषष्ठितमः सर्गः — कुम्भकर्णानुशोकः}

\twolineshloka
{तस्य राक्षसराजस्य निशम्य परिदेवितम्}
{कुम्भकर्णो बभाषेदं वचनं प्रजहास च} %6-63-1

\twolineshloka
{दृष्टो दोषो हि योऽस्माभिः पुरा मन्त्रविनिर्णये}
{हितेष्वनभियुक्तेन सोऽयमासादितस्त्वया} %6-63-2

\twolineshloka
{शीघ्रं खल्वभ्युपेतं त्वां फलं पापस्य कर्मणः}
{निरयेष्वेव पतनं यथा दुष्कृतकर्मणः} %6-63-3

\twolineshloka
{प्रथमं वै महाराज कृत्यमेतदचिन्तितम्}
{केवलं वीर्यदर्पेण नानुबन्धो विचारितः} %6-63-4

\twolineshloka
{यः पश्चात्पूर्वकार्याणि कुर्यादैश्वर्यमास्थितः}
{पूर्वं चोत्तरकार्याणि न स वेद नयानयौ} %6-63-5

\twolineshloka
{देशकालविहीनानि कर्माणि विपरीतवत्}
{क्रियमाणानि दुष्यन्ति हवींष्यप्रयतेष्विव} %6-63-6

\twolineshloka
{त्रयाणां पञ्चधा योगं कर्मणां यः प्रपद्यते}
{सचिवैः समयं कृत्वा स सम्यग् वर्तते पथि} %6-63-7

\twolineshloka
{यथागमं च यो राजा समयं च चिकीर्षति}
{बुध्यते सचिवैर्बुद्ध्या सुहृदश्चानुपश्यति} %6-63-8

\twolineshloka
{धर्ममर्थं हि कामं वा सर्वान् वा रक्षसां पते}
{भजेत पुरुषः काले त्रीणि द्वन्द्वानि वा पुनः} %6-63-9

\twolineshloka
{त्रिषु चैतेषु यच्छ्रेष्ठं श्रुत्वा तन्नावबुध्यते}
{राजा वा राजमात्रो वा व्यर्थं तस्य बहुश्रुतम्} %6-63-10

\twolineshloka
{उपप्रदानं सान्त्वं च भेदं काले च विक्रमम्}
{योगं च रक्षसां श्रेष्ठ तावुभौ च नयानयौ} %6-63-11

\twolineshloka
{काले धर्मार्थकामान् यः सम्मन्त्र्य सचिवैः सह}
{निषेवेतात्मवाँल्लोके न स व्यसनमाप्नुयात्} %6-63-12

\twolineshloka
{हितानुबन्धमालोक्य कुर्यात् कार्यमिहात्मनः}
{राजा सहार्थतत्त्वज्ञैः सचिवैर्बुद्धिजीविभिः} %6-63-13

\twolineshloka
{अनभिज्ञाय शास्त्रार्थान् पुरुषाः पशुबुद्धयः}
{प्रागल्भ्याद् वक्तुमिच्छन्ति मन्त्रिष्वभ्यन्तरीकृताः} %6-63-14

\twolineshloka
{अशास्त्रविदुषां तेषां कार्यं नाभिहितं वचः}
{अर्थशास्त्रानभिज्ञानां विपुलां श्रियमिच्छताम्} %6-63-15

\twolineshloka
{अहितं च हिताकारं धार्ष्ट्याज्जल्पन्ति ये नराः}
{अवश्यं मन्त्रबाह्यास्ते कर्तव्याः कृत्यदूषकाः} %6-63-16

\twolineshloka
{विनाशयन्तो भर्तारं सहिताः शत्रुभिर्बुधैः}
{विपरीतानि कृत्यानि कारयन्तीह मन्त्रिणः} %6-63-17

\twolineshloka
{तान् भर्ता मित्रसंकाशानमित्रान् मन्त्रनिर्णये}
{व्यवहारेण जानीयात् सचिवानुपसंहितान्} %6-63-18

\twolineshloka
{चपलस्येह कृत्यानि सहसानुप्रधावतः}
{छिद्रमन्ये प्रपद्यन्ते क्रौञ्चस्य खमिव द्विजाः} %6-63-19

\twolineshloka
{यो हि शत्रुमवज्ञाय आत्मानं नाभिरक्षति}
{अवाप्नोति हि सोऽनर्थान् स्थानाच्च व्यवरोप्यते} %6-63-20

\twolineshloka
{यदुक्तमिह ते पूर्वं प्रियया मेऽनुजेन च}
{तदेव नो हितं वाक्यं यथेच्छसि तथा कुरु} %6-63-21

\twolineshloka
{तत् तु श्रुत्वा दशग्रीवः कुम्भकर्णस्य भाषितम्}
{भ्रुकुटिं चैव संचक्रे क्रुद्धश्चैनमभाषत} %6-63-22

\twolineshloka
{मान्यो गुरुरिवाचार्यः किं मां त्वमनुशाससे}
{किमेवं वाक्श्रमं कृत्वा यद् युक्तं तद् विधीयताम्} %6-63-23

\twolineshloka
{विभ्रमाच्चित्तमोहाद् वा बलवीर्याश्रयेण वा}
{नाभिपन्नमिदानीं यद् व्यर्था तस्य पुनः कथा} %6-63-24

\twolineshloka
{अस्मिन् काले तु यद् युक्तं तदिदानीं विचिन्त्यताम्}
{गतं तु नानुशोचन्ति गतं तु गतमेव हि} %6-63-25

\twolineshloka
{ममापनयजं दोषं विक्रमेण समीकुरु}
{यदि खल्वस्ति मे स्नेहो विक्रमं वाधिगच्छसि} %6-63-26

\twolineshloka
{यदि कार्यं ममैतत्ते हृदि कार्यतमं मतम्}
{स सुहृद् यो विपन्नार्थं दीनमभ्युपपद्यते} %6-63-27

\twolineshloka
{स बन्धुर्योऽपनीतेषु साहाय्यायोपकल्पते}
{तमथैवं ब्रुवाणं स वचनं धीरदारुणम्} %6-63-28

\twolineshloka
{रुष्टोऽयमिति विज्ञाय शनैः श्लक्ष्णमुवाच ह}
{अतीव हि समालक्ष्य भ्रातरं क्षुभितेन्द्रियम्} %6-63-29

\twolineshloka
{कुम्भकर्णः शनैर्वाक्यं बभाषे परिसान्त्वयन्}
{शृणु राजन्नवहितो मम वाक्यमरिंदम} %6-63-30

\twolineshloka
{अलं राक्षसराजेन्द्र संतापमुपपद्य ते}
{रोषं च सम्परित्यज्य स्वस्थो भवितुमर्हसि} %6-63-31

\twolineshloka
{नैतन्मनसि कर्तव्यं मयि जीवति पार्थिव}
{तमहं नाशयिष्यामि यत् कृते परितप्यते} %6-63-32

\twolineshloka
{अवश्यं तु हितं वाच्यं सर्वावस्थं मया तव}
{बन्धुभावादभिहितं भ्रातृस्नेहाच्च पार्थिव} %6-63-33

\twolineshloka
{सदृशं यच्च कालेऽस्मिन् कर्तुं स्नेहेन बन्धुना}
{शत्रूणां कदनं पश्य क्रियमाणं मया रणे} %6-63-34

\twolineshloka
{अद्य पश्य महाबाहो मया समरमूर्धनि}
{हते रामे सह भ्रात्रा द्रवन्तीं हरिवाहिनीम्} %6-63-35

\twolineshloka
{अद्य रामस्य तद् दृष्ट्वा मयाऽऽनीतं रणाच्छिरः}
{सुखी भव महाबाहो सीता भवतु दुःखिता} %6-63-36

\twolineshloka
{अद्य रामस्य पश्यन्तु निधनं सुमहत् प्रियम्}
{लङ्कायां राक्षसाः सर्वे ये ते निहतबान्धवाः} %6-63-37

\twolineshloka
{अद्य शोकपरीतानां स्वबन्धुवधशोचिनाम्}
{शत्रोर्युधि विनाशेन करोम्यश्रुप्रमार्जनम्} %6-63-38

\twolineshloka
{अद्य पर्वतसंकाशं ससूर्यमिव तोयदम्}
{विकीर्णं पश्य समरे सुग्रीवं प्लवगेश्वरम्} %6-63-39

\twolineshloka
{कथं च राक्षसैरेभिर्मया च परिसान्त्वितः}
{जिघांसुभिर्दाशरथिं व्यथसे त्वं सदानघ} %6-63-40

\twolineshloka
{मां निहत्य किल त्वां हि निहनिष्यति राघवः}
{नाहमात्मनि संतापं गच्छेयं राक्षसाधिप} %6-63-41

\twolineshloka
{कामं त्विदानीमपि मां व्यादिश त्वं परंतप}
{न परः प्रेक्षणीयस्ते युद्धायातुलविक्रम} %6-63-42

\twolineshloka
{अहमुत्सादयिष्यामि शत्रूंस्तव महाबलान्}
{यदि शक्रो यदि यमो यदि पावकमारुतौ} %6-63-43

\twolineshloka
{तानहं योधयिष्यामि कुबेरवरुणावपि}
{गिरिमात्रशरीरस्य शितशूलधरस्य मे} %6-63-44

\twolineshloka
{नर्दतस्तीक्ष्णदंष्ट्रस्य बिभीयाद् वै पुरंदरः}
{अथ वा त्यक्तशस्त्रस्य मृद्नतस्तरसा रिपून्} %6-63-45

\twolineshloka
{न मे प्रतिमुखः कश्चित् स्थातुं शक्तो जिजीविषुः}
{नैव शक्त्या न गदया नासिना निशितैः शरैः} %6-63-46

\twolineshloka
{हस्ताभ्यामेव संरभ्य हनिष्यामि सवज्रिणम्}
{यदि मे मुष्टिवेगं स राघवोऽद्य सहिष्यति} %6-63-47

\twolineshloka
{ततः पास्यन्ति बाणौघा रुधिरं राघवस्य मे}
{चिन्तया तप्यसे राजन् किमर्थं मयि तिष्ठति} %6-63-48

\twolineshloka
{सोऽहं शत्रुविनाशाय तव निर्यातुमुद्यतः}
{मुञ्च रामाद् भयं घोरं निहनिष्यामि संयुगे} %6-63-49

\twolineshloka
{राघवं लक्ष्मणं चैव सुग्रीवं च महाबलम्}
{हनूमन्तं च रक्षोघ्नं येन लङ्का प्रदीपिता} %6-63-50

\twolineshloka
{हरींश्च भक्षयिष्यामि संयुगे समुपस्थिते}
{असाधारणमिच्छामि तव दातुं महद् यशः} %6-63-51

\twolineshloka
{यदि चेन्द्राद् भयं राजन् यदि चापि स्वयंभुवः}
{ततोऽहं नाशयिष्यामि नैशं तम इवांशुमान्} %6-63-52

\twolineshloka
{अपि देवाः शयिष्यन्ते मयि क्रुद्धे महीतले}
{यमं च शमयिष्यामि भक्षयिष्यामि पावकम्} %6-63-53

\twolineshloka
{आदित्यं पातयिष्यामि सनक्षत्रं महीतले}
{शतक्रतुं वधिष्यामि पास्यामि वरुणालयम्} %6-63-54

\twolineshloka
{पर्वतांश्चूर्णयिष्यामि दारयिष्यामि मेदिनीम्}
{दीर्घकालं प्रसुप्तस्य कुम्भकर्णस्य विक्रमम्} %6-63-55

\twolineshloka
{अद्य पश्यन्तु भूतानि भक्ष्यमाणानि सर्वशः}
{न त्विदं त्रिदिवं सर्वमाहारो मम पूर्यते} %6-63-56

\twolineshloka
{वधेन ते दाशरथेः सुखावहं सुखं समाहर्तुमहं व्रजामि}
{निहत्य रामं सह लक्ष्मणेन खादामि सर्वान् हरियूथमुख्यान्} %6-63-57

\twolineshloka
{रमस्व राजन् पिब चाद्य वारुणीं कुरुष्व कृत्यानि विनीय दुःखम्}
{मयाद्य रामे गमिते यमक्षयं चिराय सीता वशगा भविष्यति} %6-63-58


॥इत्यार्षे श्रीमद्रामायणे वाल्मीकीये आदिकाव्ये युद्धकाण्डे कुम्भकर्णानुशोकः नाम त्रिषष्ठितमः सर्गः ॥६-६३॥
