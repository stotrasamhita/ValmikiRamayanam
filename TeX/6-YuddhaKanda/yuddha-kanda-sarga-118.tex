\sect{अष्टादशाधिकशततमः सर्गः — सीताप्रत्यादेशः}

\twolineshloka
{तां तु पार्श्वे स्थितां प्रह्वां रामः सम्प्रेक्ष्य मैथिलीम्}
{हृदयान्तर्गतं भावं व्याहर्तुमुपचक्रमे} %6-118-1

\twolineshloka
{एषासि निर्जिता भद्रे शत्रुं जित्वा रणाजिरे}
{पौरुषाद् यदनुष्ठेयं मयैतदुपपादितम्} %6-118-2

\twolineshloka
{गतोऽस्म्यन्तममर्षस्य धर्षणा सम्प्रमार्जिता}
{अवमानश्च शत्रुश्च युगपन्निहतौ मया} %6-118-3

\twolineshloka
{अद्य मे पौरुषं दृष्टमद्य मे सफलः श्रमः}
{अद्य तीर्णप्रतिज्ञोऽहं प्रभवाम्यद्य चात्मनः} %6-118-4

\twolineshloka
{या त्वं विरहिता नीता चलचित्तेन रक्षसा}
{दैवसम्पादितो दोषो मानुषेण मया जितः} %6-118-5

\twolineshloka
{सम्प्राप्तमवमानं यस्तेजसा न प्रमार्जति}
{कस्तस्य पौरुषेणार्थो महताप्यल्पचेतसः} %6-118-6

\twolineshloka
{लङ्घनं च समुद्रस्य लङ्कायाश्चापि मर्दनम्}
{सफलं तस्य च श्लाघ्यमद्य कर्म हनूमतः} %6-118-7

\twolineshloka
{युद्धे विक्रमतश्चैव हितं मन्त्रयतस्तथा}
{सुग्रीवस्य ससैन्यस्य सफलोऽद्य परिश्रमः} %6-118-8

\twolineshloka
{विभीषणस्य च तथा सफलोऽद्य परिश्रमः}
{विगुणं भ्रातरं त्यक्त्वा यो मां स्वयमुपस्थितः} %6-118-9

\twolineshloka
{इत्येवं वदतः श्रुत्वा सीता रामस्य तद् वचः}
{मृगीवोत्फुल्लनयना बभूवाश्रुपरिप्लुता} %6-118-10

\twolineshloka
{पश्यतस्तां तु रामस्य समीपे हृदयप्रियाम्}
{जनवादभयाद् राज्ञो बभूव हृदयं द्विधा} %6-118-11

\twolineshloka
{सीतामुत्पलपत्राक्षीं नीलकुञ्चितमूर्धजाम्}
{अवदद् वै वरारोहां मध्ये वानररक्षसाम्} %6-118-12

\twolineshloka
{यत् कर्तव्यं मनुष्येण धर्षणां प्रतिमार्जता}
{तत् कृतं रावणं हत्वा मयेदं मानकाङ्क्षिणा} %6-118-13

\twolineshloka
{निर्जिता जीवलोकस्य तपसा भावितात्मना}
{अगस्त्येन दुराधर्षा मुनिना दक्षिणेव दिक्} %6-118-14

\twolineshloka
{विदितश्चास्तु भद्रं ते योऽयं रणपरिश्रमः}
{सुतीर्णः सुहृदां वीर्यान्न त्वदर्थं मया कृतः} %6-118-15

\twolineshloka
{रक्षता तु मया वृत्तमपवादं च सर्वतः}
{प्रख्यातस्यात्मवंशस्य न्यङ्गं च परिमार्जता} %6-118-16

\twolineshloka
{प्राप्तचारित्रसन्देहा मम प्रतिमुखे स्थिता}
{दीपो नेत्रातुरस्येव प्रतिकूलासि मे दृढा} %6-118-17

\twolineshloka
{तद् गच्छ त्वानुजानेऽद्य यथेष्टं जनकात्मजे}
{एता दश दिशो भद्रे कार्यमस्ति न मे त्वया} %6-118-18

\twolineshloka
{कः पुमांस्तु कुले जातः स्त्रियं परगृहोषिताम्}
{तेजस्वी पुनरादद्यात् सुहृल्लोभेन चेतसा} %6-118-19

\twolineshloka
{रावणाङ्कपरिक्लिष्टां दृष्टां दुष्टेन चक्षुषा}
{कथं त्वां पुनरादद्यां कुलं व्यपदिशन्महत्} %6-118-20

\twolineshloka
{यदर्थं निर्जिता मे त्वं सोऽयमासादितो मया}
{नास्ति मे त्वय्यभिष्वङ्गो यथेष्टं गम्यतामिति} %6-118-21

\twolineshloka
{तदद्य व्याहृतं भद्रे मयैतत् कृतबुद्धिना}
{लक्ष्मणे वाथ भरते कुरु बुद्धिं यथासुखम्} %6-118-22

\twolineshloka
{शत्रुघ्ने वाथ सुग्रीवे राक्षसे वा विभीषणे}
{निवेशय मनः सीते यथा वा सुखमात्मना} %6-118-23

\twolineshloka
{नहि त्वां रावणो दृष्ट्वा दिव्यरूपां मनोरमाम्}
{मर्षयेत चिरं सीते स्वगृहे पर्यवस्थिताम्} %6-118-24

\twolineshloka
{ततः प्रियार्हश्रवणा तदप्रियं प्रियादुपश्रुत्य चिरस्य मानिनी}
{मुमोच बाष्पं रुदती तदा भृशं गजेन्द्रहस्ताभिहतेव वल्लरी} %6-118-25


॥इत्यार्षे श्रीमद्रामायणे वाल्मीकीये आदिकाव्ये युद्धकाण्डे सीताप्रत्यादेशः नाम अष्टादशाधिकशततमः सर्गः ॥६-११८॥
