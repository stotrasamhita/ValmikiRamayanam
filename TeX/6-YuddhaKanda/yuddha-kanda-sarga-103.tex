\sect{त्र्यधिकशततमः सर्गः — ऐन्द्ररथकेतुपातनम्}

\twolineshloka
{लक्ष्मणेन तु तद् वाक्यमुक्तं श्रुत्वा स राघवः}
{संदधे परवीरघ्नो धनुरादाय वीर्यवान्} %6-103-1

\twolineshloka
{रावणाय शरान् घोरान् विससर्ज चमूमुखे}
{अथान्यं रथमास्थाय रावणो राक्षसाधिपः} %6-103-2

\threelineshloka
{अभ्यधावत काकुत्स्थं स्वर्भानुरिव भास्करम्}
{दशग्रीवो रथस्थस्तु रामं वज्रोपमैः शरैः}
{आजघान महाशैलं धाराभिरिव तोयदः} %6-103-3

\twolineshloka
{दीप्तपावकसंकाशैः शरैः काञ्चनभूषणैः}
{अभ्यवर्षद् रणे रामो दशग्रीवं समाहितः} %6-103-4

\twolineshloka
{भूमौ स्थितस्य रामस्य रथस्थस्य स रक्षसः}
{न समं युद्धमित्याहुर्देवगन्धर्वकिंनराः} %6-103-5

\twolineshloka
{ततो देववरः श्रीमान् श्रुत्वा तेषां वचोऽमृतम्}
{आहूय मातलिं शक्रो वचनं चेदमब्रवीत्} %6-103-6

\twolineshloka
{रथेन मम भूमिष्ठं शीघ्रं याहि रघूत्तमम्}
{आहूय भूतलं यात कुरु देवहितं महत्} %6-103-7

\twolineshloka
{इत्युक्तो देवराजेन मातलिर्देवसारथिः}
{प्रणम्य शिरसा देवं ततो वचनमब्रवीत्} %6-103-8

\twolineshloka
{शीघ्रं यास्यामि देवेन्द्र सारथ्यं च करोम्यहम्}
{ततो हयैश्च संयोज्य हरितैः स्यन्दनोत्तमम्} %6-103-9

\threelineshloka
{ततः काञ्चनचित्राङ्गः किङ्किणीशतभूषितः}
{तरुणादित्यसंकाशो वैदूर्यमयकूबरः}
{सदश्वैः काञ्चनापीडैर्युक्तः श्वेतप्रकीर्णकैः} %6-103-10

\twolineshloka
{हरिभिः सूर्यसंकाशैर्हेमजालविभूषितैः}
{रुक्मवेणुध्वजः श्रीमान् देवराजरथो वरः} %6-103-11

\twolineshloka
{देवराजेन संदिष्टो रथमारुह्य मातलिः}
{अभ्यवर्तत काकुत्स्थमवतीर्य त्रिविष्टपात्} %6-103-12

\twolineshloka
{अब्रवीच्च तदा रामं सप्रतोदो रथे स्थितः}
{प्राञ्जलिर्मातलिर्वाक्यं सहस्राक्षस्य सारथिः} %6-103-13

\twolineshloka
{सहस्राक्षेण काकुत्स्थ रथोऽयं विजयाय ते}
{दत्तस्तव महासत्त्व श्रीमन् शत्रुनिबर्हण} %6-103-14

\twolineshloka
{इदमैन्द्रं महच्चापं कवचं चाग्निसंनिभम्}
{शराश्चादित्यसंकाशाः शक्तिश्च विमला शिवा} %6-103-15

\twolineshloka
{आरुह्येमं रथं वीर राक्षसं जहि रावणम्}
{मया सारथिना देव महेन्द्र इव दानवान्} %6-103-16

\twolineshloka
{इत्युक्तः सम्परिक्रम्य रथं तमभिवाद्य च}
{आरुरोह तदा रामो लोकाँल्लक्ष्म्या विराजयन्} %6-103-17

\twolineshloka
{तद् बभौ चाद्भुतं युद्धं द्वैरथं रोमहर्षणम्}
{रामस्य च महाबाहो रावणस्य च रक्षसः} %6-103-18

\twolineshloka
{स गान्धर्वेण गान्धर्वं दैवं दैवेन राघवः}
{अस्त्रं राक्षसराजस्य जघान परमास्त्रवित्} %6-103-19

\twolineshloka
{अस्त्रं तु परमं घोरं राक्षसं राक्षसाधिपः}
{ससर्ज परमक्रुद्धः पुनरेव निशाचरः} %6-103-20

\twolineshloka
{ते रावणधनुर्मुक्ताः शराः काञ्चनभूषणाः}
{अभ्यवर्तन्त काकुत्स्थं सर्पा भूत्वा महाविषाः} %6-103-21

\twolineshloka
{ते दीप्तवदना दीप्तं वमन्तो ज्वलनं मुखैः}
{राममेवाभ्यवर्तन्त व्यादितास्या भयानकाः} %6-103-22

\twolineshloka
{तैर्वासुकिसमस्पर्शैर्दीप्तभोगैर्महाविषैः}
{दिशश्च संतताः सर्वा विदिशश्च समावृताः} %6-103-23

\twolineshloka
{तान् दृष्ट्वा पन्नगान् रामः समापतत आहवे}
{अस्त्रं गारुत्मतं घोरं प्रादुश्चक्रे भयावहम्} %6-103-24

\twolineshloka
{ते राघवधनुर्मुक्ता रुक्मपुङ्खाः शिखिप्रभाः}
{सुपर्णाः काञ्चना भूत्वा विचेरुः सर्पशत्रवः} %6-103-25

\twolineshloka
{ते तान् सर्वान् शराञ्जघ्नुः सर्परूपान् महाजवान्}
{सुपर्णरूपा रामस्य विशिखाः कामरूपिणः} %6-103-26

\twolineshloka
{अस्त्रे प्रतिहते क्रुद्धो रावणो राक्षसाधिपः}
{अभ्यवर्षत् तदा रामं घोराभिः शरवृष्टिभिः} %6-103-27

\twolineshloka
{ततः शरसहस्रेण राममक्लिष्टकारिणम्}
{अर्दयित्वा शरौघेण मातलिं प्रत्यविध्यत} %6-103-28

\twolineshloka
{चिच्छेद केतुमुद्दिश्य शरेणैकेन रावणः}
{पातयित्वा रथोपस्थे रथात् केतुं च काञ्चनम्} %6-103-29

\twolineshloka
{ऐन्द्रानपि जघानाश्वान् शरजालेन रावणः}
{विषेदुर्देवगन्धर्वचारणा दानवैः सह} %6-103-30

\twolineshloka
{राममार्तं तदा दृष्ट्वा सिद्धाश्च परमर्षयः}
{व्यथिता वानरेन्द्राश्च बभूवुः सविभीषणाः} %6-103-31

\twolineshloka
{रामचन्द्रमसं दृष्ट्वा ग्रस्तं रावणराहुणा}
{प्राजापत्यं च नक्षत्रं रोहिणीं शशिनः प्रियाम्} %6-103-32

\twolineshloka
{समाक्रम्य बुधस्तस्थौ प्रजानामहितावहः}
{सधूमपरिवृर्त्तोमिः प्रज्वलन्निव सागरः} %6-103-33

\twolineshloka
{उत्पपात तदा क्रुद्धः स्पृशन्निव दिवाकरम्}
{शस्त्रवर्णः सुपरुषो मन्दरश्मिर्दिवाकरः} %6-103-34

\twolineshloka
{अदृश्यत कबन्धाङ्कः संसक्तो धूमकेतुना}
{कोसलानां च नक्षत्रं व्यक्तमिन्द्राग्निदैवतम्} %6-103-35

\twolineshloka
{आहत्याङ्गारकस्तस्थौ विशाखमपि चाम्बरे}
{दशास्यो विंशतिभुजः प्रगृहीतशरासनः} %6-103-36

\twolineshloka
{अदृश्यत दशग्रीवो मैनाक इव पर्वतः}
{निरस्यमानो रामस्तु दशग्रीवेण रक्षसा} %6-103-37

\twolineshloka
{नाशक्नोदभिसंधातुं सायकान् रणमूर्धनि}
{स कृत्वा भ्रुकुटिं क्रुद्धः किंचित् संरक्तलोचनः} %6-103-38


॥इत्यार्षे श्रीमद्रामायणे वाल्मीकीये आदिकाव्ये युद्धकाण्डे ऐन्द्ररथकेतुपातनम् नाम त्र्यधिकशततमः सर्गः ॥६-१०३॥
