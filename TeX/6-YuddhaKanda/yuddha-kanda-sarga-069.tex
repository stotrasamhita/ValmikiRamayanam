\sect{एकोनसप्ततितमः सर्गः — नरान्तकवधः}

\twolineshloka
{एवं विलपमानस्य रावणस्य दुरात्मनः}
{श्रुत्वा शोकाभिभूतस्य त्रिशिरा वाक्यमब्रवीत्} %6-69-1

\twolineshloka
{एवमेव महावीर्यो हतो नस्तातमध्यमः}
{न तु सत्पुरुषा राजन् विलपन्ति यथा भवान्} %6-69-2

\twolineshloka
{नूनं त्रिभुवनस्यापि पर्याप्तस्त्वमसि प्रभो}
{स कस्मात् प्राकृत इव शोचस्यात्मानमीदृशम्} %6-69-3

\twolineshloka
{ब्रह्मदत्तास्ति ते शक्तिः कवचं सायको धनुः}
{सहस्रखरसंयुक्तो रथो मेघसमस्वनः} %6-69-4

\twolineshloka
{त्वयासकृद्धि शस्त्रेण विशस्ता देवदानवाः}
{स सर्वायुधसम्पन्नो राघवं शास्तुमर्हसि} %6-69-5

\twolineshloka
{कामं तिष्ठ महाराज निर्गमिष्याम्यहं रणे}
{उद्धरिष्यामि ते शत्रून् गरुडः पन्नगानिव} %6-69-6

\twolineshloka
{शम्बरो देवराजेन नरको विष्णुना यथा}
{तथाद्य शयिता रामो मया युधि निपातितः} %6-69-7

\twolineshloka
{श्रुत्वा त्रिशिरसो वाक्यं रावणो राक्षसाधिपः}
{पुनर्जातमिवात्मानं मन्यते कालचोदितः} %6-69-8

\twolineshloka
{श्रुत्वा त्रिशिरसो वाक्यं देवान्तकनरान्तकौ}
{अतिकायश्च तेजस्वी बभूवुर्युद्धहर्षिताः} %6-69-9

\twolineshloka
{ततोऽहमहमित्येवं गर्जन्तो नैर्ऋतर्षभाः}
{रावणस्य सुता वीराः शक्रतुल्यपराक्रमाः} %6-69-10

\twolineshloka
{अन्तरिक्षगताः सर्वे सर्वे मायाविशारदाः}
{सर्वे त्रिदशदर्पघ्नाः सर्वे समरदुर्मदाः} %6-69-11

\twolineshloka
{सर्वे सुबलसम्पन्नाः सर्वे विस्तीर्णकीर्तयः}
{सर्वे समरमासाद्य न श्रूयन्ते स्म निर्जिताः} %6-69-12

\threelineshloka
{देवैरपि सगन्धर्वैः सकिंनरमहोरगैः}
{सर्वेऽस्त्रविदुषो वीराः सर्वे युद्धविशारदाः}
{सर्वे प्रवरविज्ञानाः सर्वे लब्धवरास्तथा} %6-69-13

\twolineshloka
{स तैस्तथा भास्करतुल्यवर्चसैः सुतैर्वृतः शत्रुबलश्रियार्दनैः}
{रराज राजा मघवान् यथामरैर्वृतो महादानवदर्पनाशनैः} %6-69-14

\twolineshloka
{स पुत्रान् सम्परिष्वज्य भूषयित्वा च भूषणैः}
{आशीर्भिश्च प्रशस्ताभिः प्रेषयामास वै रणे} %6-69-15

\twolineshloka
{युद्धोन्मत्तं च मत्तं च भ्रातरौ चापि रावणः}
{रक्षणार्थं कुमाराणां प्रेषयामास संयुगे} %6-69-16

\twolineshloka
{तेऽभिवाद्य महात्मानं रावणं लोकरावणम्}
{कृत्वा प्रदक्षिणं चैव महाकायाः प्रतस्थिरे} %6-69-17

\twolineshloka
{सर्वौषधीभिर्गन्धैश्च समालभ्य महाबलाः}
{निर्जग्मुर्नैर्ऋतश्रेष्ठाः षडेते युद्धकाङ्क्षिणः} %6-69-18

\twolineshloka
{त्रिशिराश्चातिकायश्च देवान्तकनरान्तकौ}
{महोदरमहापार्श्वौ निर्जग्मुः कालचोदिताः} %6-69-19

\twolineshloka
{ततः सुदर्शनं नागं नीलजीमूतसंनिभम्}
{ऐरावतकुले जातमारुरोह महोदरः} %6-69-20

\twolineshloka
{सर्वायुधसमायुक्तस्तूणीभिश्चाप्यलंकृतः}
{रराज गजमास्थाय सवितेवास्तमूर्धनि} %6-69-21

\twolineshloka
{हयोत्तमसमायुक्तं सर्वायुधसमाकुलम्}
{आरुरोह रथश्रेष्ठं त्रिशिरा रावणात्मजः} %6-69-22

\twolineshloka
{त्रिशिरा रथमास्थाय विरराज धनुर्धरः}
{सविद्युदुल्कः सज्वालः सेन्द्रचाप इवाम्बुदः} %6-69-23

\twolineshloka
{त्रिभिः किरीटैस्त्रिशिराः शुशुभे स रथोत्तमे}
{हिमवानिव शैलेन्द्रस्त्रिभिः काञ्चनपर्वतैः} %6-69-24

\twolineshloka
{अतिकायोऽतितेजस्वी राक्षसेन्द्रसुतस्तदा}
{आरुरोह रथश्रेष्ठं श्रेष्ठः सर्वधनुष्मताम्} %6-69-25

\twolineshloka
{सुचक्राक्षं सुसंयुक्तं स्वनुकर्षं सुकूबरम्}
{तूणीबाणासनैर्दीप्तं प्रासासिपरिघाकुलम्} %6-69-26

\twolineshloka
{स काञ्चनविचित्रेण किरीटेन विराजता}
{भूषणैश्च बभौ मेरुः प्रभाभिरिव भासयन्} %6-69-27

\twolineshloka
{स रराज रथे तस्मिन् राजसूनुर्महाबलः}
{वृतो नैर्ऋतशार्दूलैर्वज्रपाणिरिवामरैः} %6-69-28

\twolineshloka
{हयमुच्चैःश्रवःप्रख्यं श्वेतं कनकभूषणम्}
{मनोजवं महाकायमारुरोह नरान्तकः} %6-69-29

\twolineshloka
{गृहीत्वा प्रासमुल्काभं विरराज नरान्तकः}
{शक्तिमादाय तेजस्वी गुहः शिखिगतो यथा} %6-69-30

\twolineshloka
{देवान्तकः समादाय परिघं हेमभूषणम्}
{परिगृह्य गिरिं दोर्भ्यां वपुर्विष्णोर्विडम्बयन्} %6-69-31

\twolineshloka
{महापार्श्वो महातेजा गदामादाय वीर्यवान्}
{विरराज गदापाणिः कुबेर इव संयुगे} %6-69-32

\twolineshloka
{ते प्रतस्थुर्महात्मानोऽमरावत्याः सुरा इव}
{तान् गजैश्च तुरङ्गैश्च रथैश्चाम्बुदनिःस्वनैः} %6-69-33

\twolineshloka
{अनूत्पेतुर्महात्मानो राक्षसाः प्रवरायुधाः}
{ते विरेजुर्महात्मानः कुमाराः सूर्यवर्चसः} %6-69-34

\twolineshloka
{किरीटिनः श्रिया जुष्टा ग्रहा दीप्ता इवाम्बरे}
{प्रगृहीता बभौ तेषां शस्त्राणामवलिः सिता} %6-69-35

\twolineshloka
{शरदभ्रप्रतीकाशा हंसावलिरिवाम्बरे}
{मरणं वापि निश्चित्य शत्रूणां वा पराजयम्} %6-69-36

\twolineshloka
{इति कृत्वा मतिं वीराः संजग्मुः संयुगार्थिनः}
{जगर्जुश्च प्रणेदुश्च चिक्षिपुश्चापि सायकान्} %6-69-37

\twolineshloka
{जगृहुश्च महात्मानो निर्यान्तो युद्धदुर्मदाः}
{क्ष्वेडितास्फोटितानां वै संचचालेव मेदिनी} %6-69-38

\twolineshloka
{रक्षसां सिंहनादैश्च संस्फोटितमिवाम्बरम्}
{तेऽभिनिष्क्रम्य मुदिता राक्षसेन्द्रा महाबलाः} %6-69-39

\twolineshloka
{ददृशुर्वानरानीकं समुद्यतशिलानगम्}
{हरयोऽपि महात्मानो ददृशू राक्षसं बलम्} %6-69-40

\twolineshloka
{हस्त्यश्वरथसम्बाधं किङ्किणीशतनादितम्}
{नीलजीमूतसंकाशं समुद्यतमहायुधम्} %6-69-41

\twolineshloka
{दीप्तानलरविप्रख्यैर्नैर्ऋतैः सर्वतो वृतम्}
{तद् दृष्ट्वा बलमायातं लब्धलक्षाः प्लवङ्गमाः} %6-69-42

\twolineshloka
{समुद्यतमहाशैलाः सम्प्रणेदुर्मुहुर्मुहुः}
{अमृष्यमाणा रक्षांसि प्रतिनर्दन्त वानराः} %6-69-43

\twolineshloka
{ततः समुत्कृष्टरवं निशम्य रक्षोगणा वानरयूथपानाम्}
{अमृष्यमाणाः परहर्षमुग्रं महाबला भीमतरं प्रणेदुः} %6-69-44

\twolineshloka
{ते राक्षसबलं घोरं प्रविश्य हरियूथपाः}
{विचेरुरुद्यतैः शैलैर्नगाः शिखरिणो यथा} %6-69-45

\twolineshloka
{केचिदाकाशमाविश्य केचिदुर्व्यां प्लवङ्गमाः}
{रक्षःसैन्येषु संक्रुद्धाः केचिद् द्रुमशिलायुधाः} %6-69-46

\twolineshloka
{द्रुमांश्च विपुलस्कन्धान् गृह्य वानरपुङ्गवाः}
{तद् युद्धमभवद् घोरं रक्षोवानरसंकुलम्} %6-69-47

\twolineshloka
{ते पादपशिलाशैलैश्चक्रुर्वृष्टिमनूपमाम्}
{बाणौघैर्वार्यमाणाश्च हरयो भीमविक्रमाः} %6-69-48

\twolineshloka
{सिंहनादान् विनेदुश्च रणे राक्षसवानराः}
{शिलाभिश्चूर्णयामासुर्यातुधानान् प्लवङ्गमाः} %6-69-49

\twolineshloka
{निर्जघ्नुः संयुगे क्रुद्धाः कवचाभरणावृतान्}
{केचिद् रथगतान् वीरान् गजवाजिगतानपि} %6-69-50

\twolineshloka
{निर्जघ्नुः सहसाऽऽप्लुत्य यातुधानान् प्लवङ्गमाः}
{शैलशृङ्गान्विताङ्गास्ते मुष्टिभिर्वान्तलोचनाः} %6-69-51

\twolineshloka
{चेलुः पेतुश्च नेदुश्च तत्र राक्षसपुङ्गवाः}
{राक्षसाश्च शरैस्तीक्ष्णैर्बिभिदुः कपिकुञ्जरान्} %6-69-52

\twolineshloka
{शूलमुद्गरखड्गैश्च जघ्नुः प्रासैश्च शक्तिभिः}
{अन्योन्यं पातयामासुः परस्परजयैषिणः} %6-69-53

\twolineshloka
{रिपुशोणितदिग्धाङ्गास्तत्र वानरराक्षसाः}
{ततः शैलैश्च खड्गैश्च विसृष्टैर्हरिराक्षसैः} %6-69-54

\threelineshloka
{मुहूर्तेनावृता भूमिरभवच्छोणितोक्षिता}
{विकीर्णैः पर्वताकारै रक्षोभिरभिमर्दितैः}
{आसीद् वसुमती पूर्णा तदा युद्धमदान्वितैः} %6-69-55

\twolineshloka
{आक्षिप्ताः क्षिप्यमाणाश्च भग्नशैलाश्च वानराः}
{पुनरङ्गैस्तदा चक्रुरासन्ना युद्धमद्भुतम्} %6-69-56

\twolineshloka
{वानरान् वानरैरेव जघ्नुस्ते नैर्ऋतर्षभाः}
{राक्षसान् राक्षसैरेव जघ्नुस्ते वानरा अपि} %6-69-57

\twolineshloka
{आक्षिप्य च शिलाः शैलाञ्जघ्नुस्ते राक्षसास्तदा}
{तेषां चाच्छिद्य शस्त्राणि जघ्नू रक्षांसि वानराः} %6-69-58

\twolineshloka
{निर्जघ्नुः शैलशृङ्गैश्च बिभिदुश्च परस्परम्}
{सिंहनादान् विनेदुश्च रणे राक्षसवानराः} %6-69-59

\twolineshloka
{छिन्नवर्मतनुत्राणा राक्षसा वानरैर्हताः}
{रुधिरं प्रसृतास्तत्र रससारमिव द्रुमाः} %6-69-60

\twolineshloka
{रथेन च रथं चापि वारणेनापि वारणम्}
{हयेन च हयं केचिन्निर्जघ्नुर्वानरा रणे} %6-69-61

\twolineshloka
{क्षुरप्रैरर्धचन्द्रैश्च भल्लैश्च निशितैः शरैः}
{राक्षसा वानरेन्द्राणां बिभिदुः पादपान् शिलाः} %6-69-62

\twolineshloka
{विकीर्णाः पर्वतास्तैश्च द्रुमच्छिन्नैश्च संयुगे}
{हतैश्च कपिरक्षोभिर्दुर्गमा वसुधाभवत्} %6-69-63

\twolineshloka
{ते वानरा गर्वितहृष्टचेष्टाः संग्राममासाद्य भयं विमुच्य}
{युद्धं स्म सर्वे सह राक्षसैस्ते नानायुधाश्चक्रुरदीनसत्त्वाः} %6-69-64

\twolineshloka
{तस्मिन् प्रवृत्ते तुमुले विमर्दे प्रहृष्यमाणेषु वलीमुखेषु}
{निपात्यमानेषु च राक्षसेषु महर्षयो देवगणाश्च नेदुः} %6-69-65

\twolineshloka
{ततो हयं मारुततुल्यवेगमारुह्य शक्तिं निशितां प्रगृह्य}
{नरान्तको वानरसैन्यमुग्रं महार्णवं मीन इवाविवेश} %6-69-66

\twolineshloka
{स वानरान् सप्त शतानि वीरः प्रासेन दीप्तेन विनिर्बिभेद}
{एकः क्षणेनेन्द्ररिपुर्महात्मा जघान सैन्यं हरिपुङ्गवानाम्} %6-69-67

\twolineshloka
{ददृशुश्च महात्मानं हयपृष्ठप्रतिष्ठितम्}
{चरन्तं हरिसैन्येषु विद्याधरमहर्षयः} %6-69-68

\twolineshloka
{स तस्य ददृशे मार्गो मांसशोणितकर्दमः}
{पतितैः पर्वताकारैर्वानरैरभिसंवृतः} %6-69-69

\twolineshloka
{यावद् विक्रमितुं बुद्धिं चक्रुः प्लवगपुङ्गवाः}
{तावदेतानतिक्रम्य निर्बिभेद नरान्तकः} %6-69-70

\twolineshloka
{ज्वलन्तं प्रासमुद्यम्य संग्रामाग्रे नरान्तकः}
{ददाह हरिसैन्यानि वनानीव विभावसुः} %6-69-71

\twolineshloka
{यावदुत्पाटयामासुर्वृक्षान् शैलान् वनौकसः}
{तावत् प्रासहताः पेतुर्वज्रकृत्ता इवाचलाः} %6-69-72

\twolineshloka
{दिक्षु सर्वासु बलवान् विचचार नरान्तकः}
{प्रमृद्नन् सर्वतो युद्धे प्रावृट्काले यथानिलः} %6-69-73

\twolineshloka
{न शेकुर्धावितुं वीरा न स्थातुं स्पन्दितुं भयात्}
{उत्पतन्तं स्थितं यान्तं सर्वान् विव्याध वीर्यवान्} %6-69-74

\twolineshloka
{एकेनान्तककल्पेन प्रासेनादित्यतेजसा}
{भग्नानि हरिसैन्यानि निपेतुर्धरणीतले} %6-69-75

\twolineshloka
{वज्रनिष्पेषसदृशं प्रासस्याभिनिपातनम्}
{न शेकुर्वानराः सोढुं ते विनेदुर्महास्वनम्} %6-69-76

\twolineshloka
{पततां हरिवीराणां रूपाणि प्रचकाशिरे}
{वज्रभिन्नाग्रकूटानां शैलानां पततामिव} %6-69-77

\twolineshloka
{ये तु पूर्वं महात्मानः कुम्भकर्णेन पातिताः}
{ते स्वस्था वानरश्रेष्ठाः सुग्रीवमुपतस्थिरे} %6-69-78

\twolineshloka
{प्रेक्षमाणः स सुग्रीवो ददृशे हरिवाहिनीम्}
{नरान्तकभयत्रस्तां विद्रवन्तीं यतस्ततः} %6-69-79

\twolineshloka
{विद्रुतां वाहिनीं दृष्ट्वा स ददर्श नरान्तकम्}
{गृहीतप्रासमायान्तं हयपृष्ठप्रतिष्ठितम्} %6-69-80

\twolineshloka
{दृष्ट्वोवाच महातेजाः सुग्रीवो वानराधिपः}
{कुमारमङ्गदं वीरं शक्रतुल्यपराक्रमम्} %6-69-81

\twolineshloka
{गच्छैनं राक्षसं वीरं योऽसौ तुरगमास्थितः}
{क्षोभयन्तं हरिबलं क्षिप्रं प्राणैर्वियोजय} %6-69-82

\twolineshloka
{स भर्तुर्वचनं श्रुत्वा निष्पपाताङ्गदस्तदा}
{अनीकान्मेघसंकाशादंशुमानिव वीर्यवान्} %6-69-83

\twolineshloka
{शैलसंघातसंकाशो हरीणामुत्तमोऽङ्गदः}
{रराजाङ्गदसंनद्धः सधातुरिव पर्वतः} %6-69-84

\twolineshloka
{निरायुधो महातेजाः केवलं नखदंष्ट्रवान्}
{नरान्तकमभिक्रम्य वालिपुत्रोऽब्रवीद् वचः} %6-69-85

\twolineshloka
{तिष्ठ किं प्राकृतैरेभिर्हरिभिस्त्वं करिष्यसि}
{अस्मिन् वज्रसमस्पर्शं प्रासं क्षिप्र ममोरसि} %6-69-86

\threelineshloka
{अङ्गदस्य वचः श्रुत्वा प्रचुक्रोध नरान्तकः}
{संदश्य दशनैरोष्ठं निःश्वस्य च भुजंगवत्}
{अभिगम्याङ्गदं क्रुद्धो वालिपुत्रं नरान्तकः} %6-69-87

\twolineshloka
{स प्रासमाविध्य तदाङ्गदाय समुज्ज्वलन्तं सहसोत्ससर्ज}
{स वालिपुत्रोरसि वज्रकल्पे बभूव भग्नो न्यपतच्च भूमौ} %6-69-88

\twolineshloka
{तं प्रासमालोक्य तदा विभग्नं सुपर्णकृत्तोरगभोगकल्पम्}
{तलं समुद्यम्य स वालिपुत्रस्तुरंगमस्याभिजघान मूर्ध्नि} %6-69-89

\twolineshloka
{निमग्नपादः स्फुटिताक्षितारो निष्क्रान्तजिह्वोऽचलसंनिकाशः}
{स तस्य वाजी निपपात भूमौ तलप्रहारेण विकीर्णमूर्धा} %6-69-90

\twolineshloka
{नरान्तकः क्रोधवशं जगाम हतं तुरंगं पतितं समीक्ष्य}
{स मुष्टिमुद्यम्य महाप्रभावो जघान शीर्षे युधि वालिपुत्रम्} %6-69-91

\twolineshloka
{अथाङ्गदो मुष्टिविशीर्णमूर्धा सुस्राव तीव्रं रुधिरं भृशोष्णम्}
{मुहुर्विजज्वाल मुमोह चापि संज्ञां समासाद्य विसिस्मिये च} %6-69-92

\twolineshloka
{अथाङ्गदो मृत्युसमानवेगं संवर्त्य मुष्टिं गिरिशृङ्गकल्पम्}
{निपातयामास तदा महात्मा नरान्तकस्योरसि वालिपुत्रः} %6-69-93

\twolineshloka
{स मुष्टिनिर्भिन्ननिमग्नवक्षा ज्वाला वमन् शोणितदिग्धगात्रः}
{नरान्तको भूमितले पपात यथाचलो वज्रनिपातभग्नः} %6-69-94

\twolineshloka
{तदान्तरिक्षे त्रिदशोत्तमानां वनौकसां चैव महाप्रणादः}
{बभूव तस्मिन् निहतेऽग्र्यवीर्ये नरान्तके वालिसुतेन संख्ये} %6-69-95

\twolineshloka
{अथाङ्गदो राममनःप्रहर्षणं सुदुष्करं तं कृतवान् हि विक्रमम्}
{विसिस्मिये सोऽप्यथ भीमकर्मा पुनश्च युद्धे स बभूव हर्षितः} %6-69-96


॥इत्यार्षे श्रीमद्रामायणे वाल्मीकीये आदिकाव्ये युद्धकाण्डे नरान्तकवधः नाम एकोनसप्ततितमः सर्गः ॥६-६९॥
