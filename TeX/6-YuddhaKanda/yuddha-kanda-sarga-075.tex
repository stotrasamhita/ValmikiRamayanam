\sect{पञ्चसप्ततितमः सर्गः — लङ्कादाहः}

\twolineshloka
{ततोऽब्रवीन्महातेजाः सुग्रीवो वानरेश्वरः}
{अर्थ्यं विज्ञापयंश्चापि हनूमन्तमिदं वचः} %6-75-1

\twolineshloka
{यतो हतः कुम्भकर्णः कुमाराश्च निषूदिताः}
{नेदानीमुपनिर्हारं रावणो दातुमर्हति} %6-75-2

\twolineshloka
{ये ये महाबलाः सन्ति लघवश्च प्लवंगमाः}
{लङ्कामभिपतन्त्वाशु गृह्योल्काः प्लवगर्षभाः} %6-75-3

\twolineshloka
{ततोऽस्तं गत आदित्ये रौद्रे तस्मिन् निशामुखे}
{लङ्कामभिमुखाः सोल्का जग्मुस्ते प्लवगर्षभाः} %6-75-4

\twolineshloka
{उल्काहस्तैर्हरिगणैः सर्वतः समभिद्रुताः}
{आरक्षस्था विरूपाक्षाः सहसा विप्रदुद्रुवुः} %6-75-5

\twolineshloka
{गोपुराट्टप्रतोलीषु चर्यासु विविधासु च}
{प्रासादेषु च संहृष्टाः ससृजुस्ते हुताशनम्} %6-75-6

\twolineshloka
{तेषां गृहसहस्राणि ददाह हुतभुक् तदा}
{प्रासादाः पर्वताकाराः पतन्ति धरणीतले} %6-75-7

\twolineshloka
{अगुरुर्दह्यते तत्र परं चैव सुचन्दनम्}
{मौक्तिका मणयः स्निग्धा वज्रं चापि प्रवालकम्} %6-75-8

\twolineshloka
{क्षौमं च दह्यते तत्र कौशेयं चापि शोभनम्}
{आविकं विविधं चौर्णं काञ्चनं भाण्डमायुधम्} %6-75-9

\twolineshloka
{नानाविकृतसंस्थानं वाजिभाण्डपरिच्छदम्}
{गजग्रैवेयकक्ष्याश्च रथभाण्डांश्च संस्कृतान्} %6-75-10

\twolineshloka
{तनुत्राणि च योधानां हस्त्यश्वानां च वर्म च}
{खड्गा धनूंषि ज्याबाणास्तोमराङ्कुशशक्तयः} %6-75-11

\twolineshloka
{रोमजं वालजं चर्म व्याघ्रजं चाण्डजं बहु}
{मुक्तामणिविचित्रांश्च प्रासादांश्च समन्ततः} %6-75-12

\twolineshloka
{विविधानस्त्रसंघातानग्निर्दहति तत्र वै}
{नानाविधान् गृहांश्चित्रान् ददाह हुतभुक् तदा} %6-75-13

\twolineshloka
{आवासान् राक्षसानां च सर्वेषां गृहगृध्नुनाम्}
{हेमचित्रतनुत्राणां स्रग्भाण्डाम्बरधारिणाम्} %6-75-14

\twolineshloka
{सीधुपानचलाक्षाणां मदविह्वलगामिनाम्}
{कान्तालम्बितवस्त्राणां शत्रुसंजातमन्युनाम्} %6-75-15

\twolineshloka
{गदाशूलासिहस्तानां खादतां पिबतामपि}
{शयनेषु महार्हेषु प्रसुप्तानां प्रियैः सह} %6-75-16

\twolineshloka
{त्रस्तानां गच्छतां तूर्णं पुत्रानादाय सर्वतः}
{तेषां शतसहस्राणि तदा लङ्कानिवासिनाम्} %6-75-17

\twolineshloka
{अदहत् पावकस्तत्र जज्वाल च पुनः पुनः}
{सारवन्ति महार्हाणि गम्भीरगुणवन्ति च} %6-75-18

\twolineshloka
{हेमचन्द्रार्धचन्द्राणि चन्द्रशालोन्नतानि च}
{तत्र चित्रगवाक्षाणि साधिष्ठानानि सर्वशः} %6-75-19

\twolineshloka
{मणिविद्रुमचित्राणि स्पृशन्तीव दिवाकरम्}
{क्रौञ्चबर्हिणवीणानां भूषणानां च निःस्वनैः} %6-75-20

\twolineshloka
{नादितान्यचलाभानि वेश्मान्यग्निर्ददाह सः}
{ज्वलनेन परीतानि तोरणानि चकाशिरे} %6-75-21

\twolineshloka
{विद्युद्भिरिव नद्धानि मेघजालानि घर्मगे}
{ज्वलनेन परीतानि गृहाणि प्रचकाशिरे} %6-75-22

\twolineshloka
{दावाग्निदीप्तानि यथा शिखराणि महागिरेः}
{विमानेषु प्रसुप्ताश्च दह्यमाना वराङ्गनाः} %6-75-23

\twolineshloka
{त्यक्ताभरणसंयोगा हाहेत्युच्चैर्विचुक्रुशुः}
{तत्र चाग्निपरीतानि निपेतुर्भवनान्यपि} %6-75-24

\twolineshloka
{वज्रिवज्रहतानीव शिखराणि महागिरेः}
{तानि निर्दह्यमानानि दूरतः प्रचकाशिरे} %6-75-25

\twolineshloka
{हिमवच्छिखराणीव दह्यमानानि सर्वशः}
{हर्म्याग्रैर्दह्यमानैश्च ज्वालाप्रज्वलितैरपि} %6-75-26

\threelineshloka
{रात्रौ सा दृश्यते लङ्का पुष्पितैरिव किंशुकैः}
{हस्त्यध्यक्षैर्गजैर्मुक्तैर्मुक्तैश्च तुरगैरपि}
{बभूव लङ्का लोकान्ते भ्रान्तग्राह इवार्णवः} %6-75-27

\twolineshloka
{अश्वं मुक्तं गजो दृष्ट्वा क्वचिद् भीतोऽपसर्पति}
{भीतो भीतं गजं दृष्ट्वा क्वचिदश्वो निवर्तते} %6-75-28

\twolineshloka
{लङ्कायां दह्यमानायां शुशुभे च महोदधिः}
{छायासंसक्तसलिलो लोहितोद इवार्णवः} %6-75-29

\twolineshloka
{सा बभूव मुहूर्तेन हरिभिर्दीपिता पुरी}
{लोकस्यास्य क्षये घोरे प्रदीप्तेव वसुन्धरा} %6-75-30

\twolineshloka
{नारीजनस्य धूमेन व्याप्तस्योच्चैर्विनेदुषः}
{स्वनो ज्वलनतप्तस्य शुश्रुवे शतयोजनम्} %6-75-31

\twolineshloka
{प्रदग्धकायानपरान् राक्षसान् निर्गतान् बहिः}
{सहसा ह्युत्पतन्ति स्म हरयोऽथ युयुत्सवः} %6-75-32

\twolineshloka
{उद्घुष्टं वानराणां च राक्षसानां च निःस्वनम्}
{दिशो दश समुद्रं च पृथिवीं च व्यनादयत्} %6-75-33

\twolineshloka
{विशल्यौ च महात्मानौ तावुभौ रामलक्ष्मणौ}
{असम्भ्रान्तौ जगृहतुस्ते उभे धनुषी वरे} %6-75-34

\twolineshloka
{ततो विस्फारयामास रामश्च धनुरुत्तमम्}
{बभूव तुमुलः शब्दो राक्षसानां भयावहः} %6-75-35

\twolineshloka
{अशोभत तदा रामो धनुर्विस्फारयन् महत्}
{भगवानिव संक्रुद्धो भवो वेदमयं धनुः} %6-75-36

\twolineshloka
{उद्घुष्टं वानराणां च राक्षसानां च निःस्वनम्}
{ज्याशब्दस्तावुभौ शब्दावति रामस्य शुश्रुवे} %6-75-37

\twolineshloka
{वानरोद्घुष्टघोषश्च राक्षसानां च निःस्वनः}
{ज्याशब्दश्चापि रामस्य त्रयं व्याप दिशो दश} %6-75-38

\twolineshloka
{तस्य कार्मुकनिर्मुक्तैः शरैस्तत्पुरगोपुरम्}
{कैलासशृङ्गप्रतिमं विकीर्णमभवद् भुवि} %6-75-39

\twolineshloka
{ततो रामशरान् दृष्ट्वा विमानेषु गृहेषु च}
{संनाहो राक्षसेन्द्राणां तुमुलः समपद्यत} %6-75-40

\twolineshloka
{तेषां संनह्यमानानां सिंहनादं च कुर्वताम्}
{शर्वरी राक्षसेन्द्राणां रौद्रीव समपद्यत} %6-75-41

\twolineshloka
{आदिष्टा वानरेन्द्रास्ते सुग्रीवेण महात्मना}
{आसन्नं द्वारमासाद्य युध्यध्वं च प्लवंगमाः} %6-75-42

\twolineshloka
{यश्च वो वितथं कुर्यात् तत्र तत्राप्युपस्थितः}
{स हन्तव्योऽभिसम्प्लुत्य राजशासनदूषकः} %6-75-43

\twolineshloka
{तेषु वानरमुख्येषु दीप्तोल्कोज्ज्वलपाणिषु}
{स्थितेषु द्वारमाश्रित्य रावणं क्रोध आविशत्} %6-75-44

\twolineshloka
{तस्य जृम्भितविक्षेपाद् व्यामिश्रा वै दिशो दश}
{रूपवानिव रुद्रस्य मन्युर्गात्रेष्वदृश्यत} %6-75-45

\twolineshloka
{स कुम्भं च निकुम्भं च कुम्भकर्णात्मजावुभौ}
{प्रेषयामास संक्रुद्धो राक्षसैर्बहुभिः सह} %6-75-46

\twolineshloka
{यूपाक्षः शोणिताक्षश्च प्रजङ्घः कम्पनस्तथा}
{निर्ययुः कौम्भकर्णिभ्यां सह रावणशासनात्} %6-75-47

\twolineshloka
{शशास चैव तान् सर्वान् राक्षसान् स महाबलान्}
{राक्षसा गच्छताद्यैव सिंहनादं च नादयन्} %6-75-48

\twolineshloka
{ततस्तु चोदितास्तेन राक्षसा ज्वलितायुधाः}
{लङ्काया निर्ययुर्वीराः प्रणदन्तः पुनः पुनः} %6-75-49

\twolineshloka
{रक्षसां भूषणस्थाभिर्भाभिः स्वाभिश्च सर्वशः}
{चक्रुस्ते सप्रभं व्योम हरयश्चाग्निभिः सह} %6-75-50

\twolineshloka
{तत्र ताराधिपस्याभा ताराणां भा तथैव च}
{तयोराभरणाभा च ज्वलिता द्यामभासयत्} %6-75-51

\twolineshloka
{चन्द्राभा भूषणाभा च ग्रहाणां ज्वलतां च भा}
{हरिराक्षससैन्यानि भ्राजयामास सर्वतः} %6-75-52

\twolineshloka
{तत्र चार्धप्रदीप्तानां गृहाणां सागरः पुनः}
{भाभिः संसक्तसलिलश्चलोर्मिः शुशुभेऽधिकम्} %6-75-53

\twolineshloka
{पताकाध्वजसंयुक्तमुत्तमासिपरश्वधम्}
{भीमाश्वरथमातङ्गं नानापत्तिसमाकुलम्} %6-75-54

\twolineshloka
{दीप्तशूलगदाखड्गप्रासतोमरकार्मुकम्}
{तद् राक्षसबलं भीमं घोरविक्रमपौरुषम्} %6-75-55

\twolineshloka
{ददृशे ज्वलितप्रासं किङ्किणीशतनादितम्}
{हेमजालाचितभुजं व्यावेष्टितपरश्वधम्} %6-75-56

\twolineshloka
{व्याघूर्णितमहाशस्त्रं बाणसंसक्तकार्मुकम्}
{गन्धमाल्यमधूत्सेकसम्मोदितमहानिलम्} %6-75-57

\twolineshloka
{घोरं शूरजनाकीर्णं महाम्बुधरनिःस्वनम्}
{तद् दृष्ट्वा बलमायातं राक्षसानां दुरासदम्} %6-75-58

\twolineshloka
{संचचाल प्लवंगानां बलमुच्चैर्ननाद च}
{जवेनाप्लुत्य च पुनस्तद् बलं रक्षसां महत्} %6-75-59

\twolineshloka
{अभ्ययात् प्रत्यरिबलं पतंगा इव पावकम्}
{तेषां भुजपरामर्शव्यामृष्टपरिघाशनि} %6-75-60

\twolineshloka
{राक्षसानां बलं श्रेष्ठं भूयः परमशोभत}
{तत्रोन्मत्ता इवोत्पेतुर्हरयोऽथ युयुत्सवः} %6-75-61

\twolineshloka
{तरुशैलैरभिघ्नन्तो मुष्टिभिश्च निशाचरान्}
{तथैवापततां तेषां हरीणां निशितैः शरैः} %6-75-62

\threelineshloka
{शिरांसि सहसा जह्रू राक्षसा भीमविक्रमाः}
{दशनैर्हतकर्णाश्च मुष्टिभिर्भिन्नमस्तकाः}
{शिलाप्रहारभग्नाङ्गा विचेरुस्तत्र राक्षसाः} %6-75-63

\twolineshloka
{तथैवाप्यपरे तेषां कपीनामसिभिः शितैः}
{प्रवरानभितो जघ्नुर्घोररूपा निशाचराः} %6-75-64

\twolineshloka
{घ्नन्तमन्यं जघानान्यः पातयन्तमपातयत्}
{गर्हमाणं जगर्हान्यो दशन्तमपरोऽदशत्} %6-75-65

\twolineshloka
{देहीत्यन्यो ददात्यन्यो ददामीत्यपरः पुनः}
{किं क्लेशयसि तिष्ठेति तत्रान्योन्यं बभाषिरे} %6-75-66

\twolineshloka
{विप्रलम्भितशस्त्रं च विमुक्तकवचायुधम्}
{समुद्यतमहाप्रासं मुष्टिशूलासिकुन्तलम्} %6-75-67

\twolineshloka
{प्रावर्तत महारौद्रं युद्धं वानररक्षसाम्}
{वानरान् दश सप्तेति राक्षसा जघ्नुराहवे} %6-75-68

\threelineshloka
{राक्षसान् दश सप्तेति वानराश्चाभ्यपातयन्}
{विप्रलम्भितवस्त्रं च विमुक्तकवचध्वजम्}
{बलं राक्षसमालम्ब्य वानराः पर्यवारयन्} %6-75-69


॥इत्यार्षे श्रीमद्रामायणे वाल्मीकीये आदिकाव्ये युद्धकाण्डे लङ्कादाहः नाम पञ्चसप्ततितमः सर्गः ॥६-७५॥
