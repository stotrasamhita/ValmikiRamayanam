\sect{पञ्चचत्वारिंशः सर्गः — नागपाशबन्धः}

\twolineshloka
{स तस्य गतिमन्विच्छन् राजपुत्रः प्रतापवान्}
{दिदेशातिबलो रामो दश वानरयूथपान्} %6-45-1

\twolineshloka
{द्वौ सुषेणस्य दायादौ नीलं च प्लवगाधिपम्}
{अङ्गदं वालिपुत्रं च शरभं च तरस्विनम्} %6-45-2

\twolineshloka
{द्विविदं च हनूमन्तं सानुप्रस्थं महाबलम्}
{ऋषभं चर्षभस्कन्धमादिदेश परंतपः} %6-45-3

\twolineshloka
{ते सम्प्रहृष्टा हरयो भीमानुद्यम्य पादपान्}
{आकाशं विविशुः सर्वे मार्गमाणा दिशो दश} %6-45-4

\twolineshloka
{तेषां वेगवतां वेगमिषुभिर्वेगवत्तरैः}
{अस्त्रवित् परमास्त्रस्तु वारयामास रावणिः} %6-45-5

\twolineshloka
{तं भीमवेगा हरयो नाराचैः क्षतविक्षताः}
{अन्धकारे न ददृशुर्मेघैः सूर्यमिवावृतम्} %6-45-6

\twolineshloka
{रामलक्ष्मणयोरेव सर्वदेहभिदः शरान्}
{भृशमावेशयामास रावणिः समितिंजयः} %6-45-7

\twolineshloka
{निरन्तरशरीरौ तु तावुभौ रामलक्ष्मणौ}
{क्रुद्धेनेन्द्रजिता वीरौ पन्नगैः शरतां गतैः} %6-45-8

\twolineshloka
{तयोः क्षतजमार्गेण सुस्राव रुधिरं बहु}
{तावुभौ च प्रकाशेते पुष्पिताविव किंशुकौ} %6-45-9

\twolineshloka
{ततः पर्यन्तरक्ताक्षो भिन्नाञ्जनचयोपमः}
{रावणिर्भ्रातरौ वाक्यमन्तर्धानगतोऽब्रवीत्} %6-45-10

\twolineshloka
{युध्यमानमनालक्ष्यं शक्रोऽपि त्रिदशेश्वरः}
{द्रष्टुमासादितुं वापि न शक्तः किं पुनर्युवाम्} %6-45-11

\twolineshloka
{प्रापिताविषुजालेन राघवौ कङ्कपत्रिणा}
{एष रोषपरीतात्मा नयामि यमसादनम्} %6-45-12

\twolineshloka
{एवमुक्त्वा तु धर्मज्ञौ भ्रातरौ रामलक्ष्मणौ}
{निर्बिभेद शितैर्बाणैः प्रजहर्ष ननाद च} %6-45-13

\twolineshloka
{भिन्नाञ्जनचयश्यामो विस्फार्य विपुलं धनुः}
{भूय एव शरान् घोरान् विससर्ज महामृधे} %6-45-14

\twolineshloka
{ततो मर्मसु मर्मज्ञो मज्जयन् निशितान् शरान्}
{रामलक्ष्मणयोर्वीरो ननाद च मुहुर्मुहुः} %6-45-15

\twolineshloka
{बद्धौ तु शरबन्धेन तावुभौ रणमूर्धनि}
{निमेषान्तरमात्रेण न शेकतुरवेक्षितुम्} %6-45-16

\twolineshloka
{ततो विभिन्नसर्वाङ्गौ शरशल्याचितौ कृतौ}
{ध्वजाविव महेन्द्रस्य रज्जुमुक्तौ प्रकम्पितौ} %6-45-17

\twolineshloka
{तौ सम्प्रचलितौ वीरौ मर्मभेदेन कर्शितौ}
{निपेततुर्महेष्वासौ जगत्यां जगतीपती} %6-45-18

\twolineshloka
{तौ वीरशयने वीरौ शयानौ रुधिरोक्षितौ}
{शरवेष्टितसर्वाङ्गावार्तौ परमपीडितौ} %6-45-19

\twolineshloka
{नह्यविद्धं तयोर्गात्रे बभूवाङ्गुलमन्तरम्}
{नानिर्विण्णं न चाध्वस्तमाकराग्रादजिह्मगैः} %6-45-20

\twolineshloka
{तौ तु क्रूरेण निहतौ रक्षसा कामरूपिणा}
{असृक् सुस्रुवतुस्तीव्रं जलं प्रस्रवणाविव} %6-45-21

\twolineshloka
{पपात प्रथमं रामो विद्धो मर्मसु मार्गणैः}
{क्रोधादिन्द्रजिता येन पुरा शक्रो विनिर्जितः} %6-45-22

\threelineshloka
{रुक्मपुङ्खैः प्रसन्नाग्रै रजोगतिभिराशुगैः}
{नाराचैरर्धनाराचैर्भल्लैरञ्जलिकैरपि}
{विव्याध वत्सदन्तैश्च सिंहदंष्ट्रैः क्षुरैस्तथा} %6-45-23

\twolineshloka
{स वीरशयने शिश्येऽविज्यमाविध्य कार्मुकम्}
{भिन्नमुष्टिपरीणाहं त्रिनतं रुक्मभूषितम्} %6-45-24

\twolineshloka
{बाणपातान्तरे रामं पतितं पुरुषर्षभम्}
{स तत्र लक्ष्मणो दृष्ट्वा निराशो जीवितेऽभवत्} %6-45-25

\twolineshloka
{रामं कमलपत्राक्षं शरण्यं रणतोषिणम्}
{शुशोच भ्रातरं दृष्ट्वा पतितं धरणीतले} %6-45-26

\twolineshloka
{हरयश्चापि तं दृष्ट्वा संतापं परमं गताः}
{शोकार्ताश्चुक्रुशुर्घोरमश्रुपूरितलोचनाः} %6-45-27

\twolineshloka
{बद्धौ तु तौ वीरशये शयानौ ते वानराः सम्परिवार्य तस्थुः}
{समागता वायुसुतप्रमुख्या विषादमार्ताः परमं च जग्मुः} %6-45-28


॥इत्यार्षे श्रीमद्रामायणे वाल्मीकीये आदिकाव्ये युद्धकाण्डे नागपाशबन्धः नाम पञ्चचत्वारिंशः सर्गः ॥६-४५॥
