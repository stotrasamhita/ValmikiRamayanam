\sect{अष्टषष्ठितमः सर्गः — रावणानुशोकः}

\twolineshloka
{कुम्भकर्णं हतं दृष्ट्वा राघवेण महात्मना}
{राक्षसा राक्षसेन्द्राय रावणाय न्यवेदयन्} %6-68-1

\twolineshloka
{राजन् स कालसङ्काशः संयुक्तः कालकर्मणा}
{विद्राव्य वानरीं सेनां भक्षयित्वा च वानरान्} %6-68-2

\twolineshloka
{प्रतपित्वा मुहूर्तं तु प्रशान्तो रामतेजसा}
{कायेनार्धप्रविष्टेन समुद्रं भीमदर्शनम्} %6-68-3

\twolineshloka
{निकृत्तनासाकर्णेन विक्षरद्रुधिरेण च}
{रुद्ध्वा द्वारं शरीरेण लङ्कायाः पर्वतोपमः} %6-68-4

\twolineshloka
{कुम्भकर्णस्तव भ्राता काकुत्स्थशरपीडितः}
{अगण्डभूतो विवृतो दावदग्ध इव द्रुमः} %6-68-5

\twolineshloka
{श्रुत्वा विनिहतं सङ्ख्ये कुम्भकर्णं महाबलम्}
{रावणः शोकसन्तप्तो मुमोह च पपात च} %6-68-6

\twolineshloka
{पितृव्यं निहतं श्रुत्वा देवान्तकनरान्तकौ}
{त्रिशिराश्चातिकायश्च रुरुदुः शोकपीडिताः} %6-68-7

\twolineshloka
{भ्रातरं निहतं श्रुत्वा रामेणाक्लिष्टकर्मणा}
{महोदरमहापार्श्वौ शोकाक्रान्तौ बभूवतुः} %6-68-8

\twolineshloka
{ततः कृच्छ्रात् समासाद्य संज्ञां राक्षसपुङ्गवः}
{कुम्भकर्णवधाद् दीनो विललापाकुलेन्द्रियः} %6-68-9

\twolineshloka
{हा वीर रिपुदर्पघ्न कुम्भकर्ण महाबल}
{त्वं मां विहाय वै दैवाद् यातोऽसि यमसादनम्} %6-68-10

\twolineshloka
{मम शल्यमनुद्धृत्य बान्धवानां महाबल}
{शत्रुसैन्यं प्रताप्यैकः क्व मां सन्त्यज्य गच्छसि} %6-68-11

\twolineshloka
{इदानीं खल्वहं नास्मि यस्य मे पतितो भुजः}
{दक्षिणोऽयं समाश्रित्य न बिभेमि सुरासुरात्} %6-68-12

\twolineshloka
{कथमेवंविधो वीरो देवदानवदर्पहा}
{कालाग्निप्रतिमो ह्यद्य राघवेण रणे हतः} %6-68-13

\twolineshloka
{यस्य ते वज्रनिष्पेषो न कुर्याद् व्यसनं सदा}
{स कथं रामबाणार्तः प्रसुप्तोऽसि महीतले} %6-68-14

\twolineshloka
{एते देवगणाः सार्धमृषिभिर्गगने स्थिताः}
{निहतं त्वां रणे दृष्ट्वा निनदन्ति प्रहर्षिताः} %6-68-15

\twolineshloka
{ध्रुवमद्यैव संहृष्टा लब्धलक्षाः प्लवङ्गमाः}
{आरोक्ष्यन्तीह दुर्गाणि लङ्काद्वाराणि सर्वशः} %6-68-16

\twolineshloka
{राज्येन नास्ति मे कार्यं किं करिष्यामि सीतया}
{कुम्भकर्णविहीनस्य जीविते नास्ति मे मतिः} %6-68-17

\twolineshloka
{यद्यहं भ्रातृहन्तारं न हन्मि युधि राघवम्}
{ननु मे मरणं श्रेयो न चेदं व्यर्थजीवितम्} %6-68-18

\twolineshloka
{अद्यैव तं गमिष्यामि देशं यत्रानुजो मम}
{नहि भ्रातॄन् समुत्सृज्य क्षणं जीवितुमुत्सहे} %6-68-19

\twolineshloka
{देवा हि मां हसिष्यन्ति दृष्ट्वा पूर्वापकारिणम्}
{कथमिन्द्रं जयिष्यामि कुम्भकर्ण हते त्वयि} %6-68-20

\twolineshloka
{तदिदं मामनुप्राप्तं विभीषणवचः शुभम्}
{यदज्ञानान्मया तस्य न गृहीतं महात्मनः} %6-68-21

\twolineshloka
{विभीषणवचस्तावत् कुम्भकर्णप्रहस्तयोः}
{विनाशोऽयं समुत्पन्नो मां व्रीडयति दारुणः} %6-68-22

\twolineshloka
{तस्यायं कर्मणः प्राप्तो विपाको मम शोकदः}
{यन्मया धार्मिकः श्रीमान् स निरस्तो विभीषणः} %6-68-23

\twolineshloka
{इति बहुविधमाकुलान्तरात्मा कृपणमतीव विलप्य कुम्भकर्णम्}
{न्यपतदपि दशाननो भृशार्तस्तमनुजमिन्द्ररिपुं हतं विदित्वा} %6-68-24


॥इत्यार्षे श्रीमद्रामायणे वाल्मीकीये आदिकाव्ये युद्धकाण्डे रावणानुशोकः नाम अष्टषष्ठितमः सर्गः ॥६-६८॥
