\sect{त्रिचत्वारिशः सर्गः — द्वन्द्वयुद्धम्}

\twolineshloka
{युध्यतां तु ततस्तेषां वानराणां महात्मनाम्}
{रक्षसां सम्बभूवाथ बलरोषः सुदारुणः} %6-43-1

\twolineshloka
{ते हयैः काञ्चनापीडैर्गजैश्चाग्निशिखोपमैः}
{रथैश्चादित्यसङ्काशैः कवचैश्च मनोरमैः} %6-43-2

\twolineshloka
{निर्ययू राक्षसा वीरा नादयन्तो दिशो दश}
{राक्षसा भीमकर्माणो रावणस्य जयैषिणः} %6-43-3

\twolineshloka
{वानराणामपि चमूर्बृहती जयमिच्छताम्}
{अभ्यधावत तां सेनां रक्षसां घोरकर्मणाम्} %6-43-4

\twolineshloka
{एतस्मिन्नन्तरे तेषामन्योन्यमभिधावताम्}
{रक्षसां वानराणां च द्वन्द्वयुद्धमवर्तत} %6-43-5

\twolineshloka
{अङ्गदेनेन्द्रजित्सार्धं वालिपुत्रेण राक्षसः}
{अयुध्यत महातेजास्त्र्यम्बकेण यथान्धकः} %6-43-6

\twolineshloka
{प्रजङ्घेन च सम्पातिर्नित्यं दुर्धर्षणो रणे}
{जम्बुमालिनमारब्धो हनूमानपि वानरः} %6-43-7

\twolineshloka
{सङ्गतस्तु महाक्रोधो राक्षसो रावणानुजः}
{समरे तीक्ष्णवेगेन शत्रुघ्नेन विभीषणः} %6-43-8

\twolineshloka
{तपनेन गजः सार्धं राक्षसेन महाबलः}
{निकुम्भेन महातेजा नीलोऽपि समयुध्यत} %6-43-9

\twolineshloka
{वानरेन्द्रस्तु सुग्रीवः प्रघसेन सुसङ्गतः}
{सङ्गतः समरे श्रीमान् विरूपाक्षेण लक्ष्मणः} %6-43-10

\twolineshloka
{अग्निकेतुः सुदुर्धर्षो रश्मिकेतुश्च राक्षसः}
{सुप्तघ्नो यज्ञकोपश्च रामेण सह सङ्गताः} %6-43-11

\twolineshloka
{वज्रमुष्टिश्च मैन्देन द्विविदेनाशनिप्रभः}
{राक्षसाभ्यां सुघोराभ्यां कपिमुख्यौ समागतौ} %6-43-12

\twolineshloka
{वीरः प्रतपनो घोरो राक्षसो रणदुर्धरः}
{समरे तीक्ष्णवेगेन नलेन समयुध्यत} %6-43-13

\twolineshloka
{धर्मस्य पुत्रो बलवान् सुषेण इति विश्रुतः}
{स विद्युन्मालिना सार्धमयुध्यत महाकपिः} %6-43-14

\twolineshloka
{वानराश्चापरे घोरा राक्षसैरपरैः सह}
{द्वन्द्वं समीयुः सहसा युद्ध्वा च बहुभिः सह} %6-43-15

\twolineshloka
{तत्रासीत् सुमहद् युद्धं तुमुलं रोमहर्षणम्}
{रक्षसां वानराणां च वीराणां जयमिच्छताम्} %6-43-16

\twolineshloka
{हरिराक्षसदेहेभ्यः प्रभूताः केशशाद्वलाः}
{शरीरसङ्घाटवहाः प्रसुस्रुः शोणतापगाः} %6-43-17

\twolineshloka
{आजघानेन्द्रजित् क्रुद्धो वज्रेणेव शतक्रतुः}
{अङ्गदं गदया वीरं शत्रुसैन्यविदारणम्} %6-43-18

\twolineshloka
{तस्य काञ्चनचित्राङ्गं रथं साश्वं ससारथिम्}
{जघान गदया श्रीमानङ्गदो वेगवान् हरिः} %6-43-19

\twolineshloka
{सम्पातिस्तु प्रजङ्घेन त्रिभिर्बाणैः समाहतः}
{निजघानाश्वकर्णेन प्रजङ्घं रणमूर्धनि} %6-43-20

\twolineshloka
{जम्बुमाली रथस्थस्तु रथशक्त्या महाबलः}
{बिभेद समरे क्रुद्धो हनूमन्तं स्तनान्तरे} %6-43-21

\twolineshloka
{तस्य तं रथमास्थाय हनूमान् मारुतात्मजः}
{प्रममाथ तलेनाशु सह तेनैव रक्षसा} %6-43-22

\twolineshloka
{नदन् प्रतपनो घोरो नलं सोऽभ्यनुधावत}
{नलः प्रतपनस्याशु पातयामास चक्षुषी} %6-43-23

\twolineshloka
{भिन्नगात्रः शरैस्तीक्ष्णैः क्षिप्रहस्तेन रक्षसा}
{ग्रसन्तमिव सैन्यानि प्रघसं वानराधिपः} %6-43-24

\twolineshloka
{सुग्रीवः सप्तपर्णेन निजघान जवेन च}
{प्रपीड्य शरवर्षेण राक्षसं भीमदर्शनम्} %6-43-25

\threelineshloka
{निजघान विरूपाक्षं शरेणैकेन लक्ष्मणः}
{अग्निकेतुश्च दुर्धर्षो रश्मिकेतुश्च राक्षसः}
{सुप्तघ्नो यज्ञकोपश्च रामं निर्बिभिदुः शरैः} %6-43-26

\twolineshloka
{तेषां चतुर्णां रामस्तु शिरांसि समरे शरैः}
{क्रुद्धश्चतुर्भिश्चिच्छेद घोरैरग्निशिखोपमैः} %6-43-27

\twolineshloka
{वज्रमुष्टिस्तु मैन्देन मुष्टिना निहतो रणे}
{पपात सरथः साश्वः पुराट्ट इव भूतले} %6-43-28

\twolineshloka
{निकुम्भस्तु रणे नीलं नीलाञ्जनचयप्रभम्}
{निर्बिभेद शरैस्तीक्ष्णैः करैर्मेघमिवांशुमान्} %6-43-29

\twolineshloka
{पुनः शरशतेनाथ क्षिप्रहस्तो निशाचरः}
{बिभेद समरे नीलं निकुम्भः प्रजहास च} %6-43-30

\twolineshloka
{तस्यैव रथचक्रेण नीलो विष्णुरिवाहवे}
{शिरश्चिच्छेद समरे निकुम्भस्य च सारथेः} %6-43-31

\twolineshloka
{वज्राशनिसमस्पर्शो द्विविदोऽप्यशनिप्रभम्}
{जघान गिरिशृङ्गेण मिषतां सर्वरक्षसाम्} %6-43-32

\twolineshloka
{द्विविदं वानरेन्द्रं तु द्रुमयोधिनमाहवे}
{शरैरशनिसङ्काशैः स विव्याधाशनिप्रभः} %6-43-33

\twolineshloka
{स शरैरभिविद्धाङ्गो द्विविदः क्रोधर्मूच्छितः}
{सालेन सरथं साश्वं निजघानाशनिप्रभम्} %6-43-34

\twolineshloka
{विद्युन्माली रथस्थस्तु शरैः काञ्चनभूषणैः}
{सुषेणं ताडयामास ननाद च मुहुर्मुहुः} %6-43-35

\twolineshloka
{तं रथस्थमथो दृष्ट्वा सुषेणो वानरोत्तमः}
{गिरिशृङ्गेण महता रथमाशु न्यपातयत्} %6-43-36

\twolineshloka
{लाघवेन तु संयुक्तो विद्युन्माली निशाचरः}
{अपक्रम्य रथात् तूर्णं गदापाणिः क्षितौ स्थितः} %6-43-37

\twolineshloka
{ततः क्रोधसमाविष्टः सुषेणो हरिपुङ्गवः}
{शिलां सुमहतीं गृह्य निशाचरमभिद्रवत्} %6-43-38

\twolineshloka
{तमापतन्तं गदया विद्युन्माली निशाचरः}
{वक्षस्यभिजघानाशु सुषेणं हरिपुङ्गवम्} %6-43-39

\twolineshloka
{गदाप्रहारं तं घोरमचिन्त्य प्लवगोत्तमः}
{तां तूष्णीं पातयामास तस्योरसि महामृधे} %6-43-40

\twolineshloka
{शिलाप्रहाराभिहतो विद्युन्माली निशाचरः}
{निष्पिष्टहृदयो भूमौ गतासुर्निपपात ह} %6-43-41

\twolineshloka
{एवं तैर्वानरैः शूरैः शूरास्ते रजनीचराः}
{द्वन्द्वे विमथितास्तत्र दैत्या इव दिवौकसैः} %6-43-42

\twolineshloka
{भल्लैश्चान्यैर्गदाभिश्च शक्तितोमरसायकैः}
{अपविद्धैश्चापि रथैस्तथा साङ्ग्रामिकैर्हयैः} %6-43-43

\twolineshloka
{निहतैः कुञ्जरैर्मत्तैस्तथा वानरराक्षसैः}
{चक्राक्षयुगदण्डैश्च भग्नैर्धरणिसंश्रितैः} %6-43-44

\threelineshloka
{बभूवायोधनं घोरं गोमायुगणसेवितम्}
{कबन्धानि समुत्पेतुर्दिक्षु वानररक्षसाम्}
{विमर्दे तुमुले तस्मिन् देवासुररणोपमे} %6-43-45

\twolineshloka
{निहन्यमाना हरिपुङ्गवैस्तदा निशाचराः शोणितगन्धमूर्च्छिताः}
{पुनः सुयुद्धं तरसा समाश्रिता दिवाकरस्यास्तमयाभिकाङ्क्षिणः} %6-43-46


॥इत्यार्षे श्रीमद्रामायणे वाल्मीकीये आदिकाव्ये युद्धकाण्डे द्वन्द्वयुद्धम् नाम त्रिचत्वारिशः सर्गः ॥६-४३॥
