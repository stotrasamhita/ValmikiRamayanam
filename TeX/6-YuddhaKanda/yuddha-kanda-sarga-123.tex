\sect{त्रयोविंशत्यधिकशततमः सर्गः — इन्द्रवरदानम्}

\twolineshloka
{प्रतिप्रयाते काकुत्स्थे महेन्द्रः पाकशासनः}
{अब्रवीत् परमप्रीतो राघवं प्राञ्जलिं स्थितम्} %6-123-1

\twolineshloka
{अमोघं दर्शनं राम तवास्माकं नरर्षभ}
{प्रीतियुक्ताः स्म तेन त्वं ब्रूहि यन्मनसेप्सितम्} %6-123-2

\twolineshloka
{एवमुक्तो महेन्द्रेण प्रसन्नेन महात्मना}
{सुप्रसन्नमना हृष्टो वचनं प्राह राघवः} %6-123-3

\twolineshloka
{यदि प्रीतिः समुत्पन्ना मयि ते विबुधेश्वर}
{वक्ष्यामि कुरु मे सत्यं वचनं वदतां वर} %6-123-4

\twolineshloka
{मम हेतोः पराक्रान्ता ये गता यमसादनम्}
{ते सर्वे जीवितं प्राप्य समुत्तिष्ठन्तु वानराः} %6-123-5

\twolineshloka
{मत्कृते विप्रयुक्ता ये पुत्रैर्दारैश्च वानराः}
{तान् प्रीतमनसः सर्वान् द्रष्टुमिच्छामि मानद} %6-123-6

\twolineshloka
{विक्रान्ताश्चापि शूराश्च न मृत्युं गणयन्ति च}
{कृतयत्ना विपन्नाश्च जीवयैतान् पुरंदर} %6-123-7

\twolineshloka
{मत्प्रियेष्वभिरक्ताश्च न मृत्युं गणयन्ति ये}
{त्वत्प्रसादात् समेयुस्ते वरमेतमहं वृणे} %6-123-8

\twolineshloka
{नीरुजो निर्व्रणांश्चैव सम्पन्नबलपौरुषान्}
{गोलाङ्गूलांस्तथर्क्षांश्च द्रष्टुमिच्छामि मानद} %6-123-9

\twolineshloka
{अकाले चापि पुष्पाणि मूलानि च फलानि च}
{नद्यश्च विमलास्तत्र तिष्ठेयुर्यत्र वानराः} %6-123-10

\twolineshloka
{श्रुत्वा तु वचनं तस्य राघवस्य महात्मनः}
{महेन्द्रः प्रत्युवाचेदं वचनं प्रीतिसंयुतम्} %6-123-11

\twolineshloka
{महानयं वरस्तात यस्त्वयोक्तो रघूत्तम}
{द्विर्मया नोक्तपूर्वं च तस्मादेतद् भविष्यति} %6-123-12

\twolineshloka
{समुत्तिष्ठन्तु ते सर्वे हता ये युधि राक्षसैः}
{ऋक्षाश्च सह गोपुच्छैर्निकृत्ताननबाहवः} %6-123-13

\twolineshloka
{नीरुजो निर्व्रणाश्चैव सम्पन्नबलपौरुषाः}
{समुत्थास्यन्ति हरयः सुप्ता निद्राक्षये यथा} %6-123-14

\twolineshloka
{सुहृद्भिर्बान्धवैश्चैव ज्ञातिभिः स्वजनेन च}
{सर्व एव समेष्यन्ति संयुक्ताः परया मुदा} %6-123-15

\twolineshloka
{अकाले पुष्पशबलाः फलवन्तश्च पादपाः}
{भविष्यन्ति महेष्वास नद्यश्च सलिलायुताः} %6-123-16

\twolineshloka
{सव्रणैः प्रथमं गात्रैरिदानीं निर्व्रणैः समैः}
{ततः समुत्थिताः सर्वे सुप्त्वेव हरिसत्तमाः} %6-123-17

\twolineshloka
{बभूवुर्वानराः सर्वे किं त्वेतदिति विस्मिताः}
{काकुत्स्थं परिपूर्णार्थं दृष्ट्वा सर्वे सुरोत्तमाः} %6-123-18

\twolineshloka
{अब्रुवन् परमप्रीताः स्तुत्वा रामं सलक्ष्मणम्}
{गच्छायोध्यामितो राजन् विसर्जय च वानरान्} %6-123-19

\twolineshloka
{मैथिलीं सान्त्वयस्वैनामनुरक्तां यशस्विनीम्}
{भ्रातरं भरतं पश्य त्वच्छोकाद् व्रतचारिणम्} %6-123-20

\twolineshloka
{शत्रुघ्नं च महात्मानं मातॄः सर्वाः परंतप}
{अभिषेचय चात्मानं पौरान् गत्वा प्रहर्षय} %6-123-21

\twolineshloka
{एवमुक्त्वा सहस्राक्षो रामं सौमित्रिणा सह}
{विमानैः सूर्यसंकाशैर्ययौ हृष्टः सुरैः सह} %6-123-22

\twolineshloka
{अभिवाद्य च काकुत्स्थः सर्वांस्तांस्त्रिदशोत्तमान्}
{लक्ष्मणेन सह भ्रात्रा वासमाज्ञापयत् तदा} %6-123-23

\twolineshloka
{ततस्तु सा लक्ष्मणरामपालिता महाचमूर्हृष्टजना यशस्विनी}
{श्रिया ज्वलन्ती विरराज सर्वतो निशा प्रणीतेव हि शीतरश्मिना} %6-123-24


॥इत्यार्षे श्रीमद्रामायणे वाल्मीकीये आदिकाव्ये युद्धकाण्डे इन्द्रवरदानम् नाम त्रयोविंशत्यधिकशततमः सर्गः ॥६-१२३॥
