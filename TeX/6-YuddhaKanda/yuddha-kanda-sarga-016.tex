\sect{षोडशः सर्गः — विभीषणाक्रोशः}

\twolineshloka
{सुनिविष्टं हितं वाक्यमुक्तवन्तं विभीषणम्}
{अब्रवीत् परुषं वाक्यं रावणः कालचोदितः} %6-16-1

\twolineshloka
{वसेत् सह सपत्नेन क्रुद्धेनाशीविषेण च}
{न तु मित्रप्रवादेन संवसेच्छत्रुसेविना} %6-16-2

\twolineshloka
{जानामि शीलं ज्ञातीनां सर्वलोकेषु राक्षस}
{हृष्यन्ति व्यसनेष्वेते ज्ञातीनां ज्ञातयः सदा} %6-16-3

\twolineshloka
{प्रधानं साधकं वैद्यं धर्मशीलं च राक्षस}
{ज्ञातयोऽप्यवमन्यन्ते शूरं परिभवन्ति च} %6-16-4

\twolineshloka
{नित्यमन्योन्यसंहृष्टा व्यसनेष्वाततायिनः}
{प्रच्छन्नहृदया घोरा ज्ञातयस्तु भयावहाः} %6-16-5

\twolineshloka
{श्रूयन्ते हस्तिभिर्गीताः श्लोकाः पद्मवने पुरा}
{पाशहस्तान् नरान् दृष्ट्वा शृणुष्व गदतो मम} %6-16-6

\twolineshloka
{नाग्निर्नान्यानि शस्त्राणि न नः पाशा भयावहाः}
{घोराः स्वार्थप्रयुक्तास्तु ज्ञातयो नो भयावहाः} %6-16-7

\twolineshloka
{उपायमेते वक्ष्यन्ति ग्रहणे नात्र संशयः}
{कृत्स्नाद् भयाज्ज्ञातिभयं कुकष्टं विहितं च नः} %6-16-8

\twolineshloka
{विद्यते गोषु सम्पन्नं विद्यते ज्ञातितो भयम्}
{विद्यते स्त्रीषु चापल्यं विद्यते ब्राह्मणे तपः} %6-16-9

\twolineshloka
{ततो नेष्टमिदं सौम्य यदहं लोकसत्कृतः}
{ऐश्वर्यमभिजातश्च रिपूणां मूर्ध्नि च स्थितः} %6-16-10

\twolineshloka
{यथा पुष्करपत्रेषु पतितास्तोयबिन्दवः}
{न श्लेषमभिगच्छन्ति तथानार्येषु सौहृदम्} %6-16-11

\twolineshloka
{यथा शरदि मेघानां सिञ्चतामपि गर्जताम्}
{न भवत्यम्बुसंक्लेदस्तथानार्येषु सौहृदम्} %6-16-12

\twolineshloka
{यथा मधुकरस्तर्षाद् रसं विन्दन्न तिष्ठति}
{तथा त्वमपि तत्रैव तथानार्येषु सौहृदम्} %6-16-13

\twolineshloka
{यथा मधुकरस्तर्षात् काशपुष्पं पिबन्नपि}
{रसमत्र न विन्देत तथानार्येषु सौहृदम्} %6-16-14

\twolineshloka
{यथा पूर्वं गजः स्नात्वा गृह्य हस्तेन वै रजः}
{दूषयत्यात्मनो देहं तथानार्येषु सौहृदम्} %6-16-15

\twolineshloka
{योऽन्यस्त्वेवंविधं ब्रूयाद् वाक्यमेतन्निशाचर}
{अस्मिन् मुहूर्ते न भवेत् त्वां तु धिक् कुलपांसन} %6-16-16

\twolineshloka
{इत्युक्तः परुषं वाक्यं न्यायवादी विभीषणः}
{उत्पपात गदापाणिश्चतुर्भिः सह राक्षसैः} %6-16-17

\twolineshloka
{अब्रवीच्च तदा वाक्यं जातक्रोधो विभीषणः}
{अन्तरिक्षगतः श्रीमान् भ्राता वै राक्षसाधिपम्} %6-16-18

\threelineshloka
{स त्वं भ्रान्तोऽसि मे राजन् ब्रूहि मां यद् यदिच्छसि}
{ज्येष्ठो मान्यः पितृसमो न च धर्मपथे स्थितः}
{इदं हि परुषं वाक्यं न क्षमाम्यग्रजस्य ते} %6-16-19

\twolineshloka
{सुनीतं हितकामेन वाक्यमुक्तं दशानन}
{न गृह्णन्त्यकृतात्मानः कालस्य वशमागताः} %6-16-20

\twolineshloka
{सुलभाः पुरुषा राजन् सततं प्रियवादिनः}
{अप्रियस्य च पथ्यस्य वक्ता श्रोता च दुर्लभः} %6-16-21

\twolineshloka
{बद्धं कालस्य पाशेन सर्वभूतापहारिणः}
{न नश्यन्तमुपेक्षे त्वां प्रदीप्तं शरणं यथा} %6-16-22

\twolineshloka
{दीप्तपावकसंकाशैः शितैः काञ्चनभूषणैः}
{न त्वामिच्छाम्यहं द्रष्टुं रामेण निहतं शरैः} %6-16-23

\twolineshloka
{शूराश्च बलवन्तश्च कृतास्त्राश्च नरा रणे}
{कालाभिपन्नाः सीदन्ति यथा वालुकसेतवः} %6-16-24

\threelineshloka
{तन्मर्षयतु यच्चोक्तं गुरुत्वाद्धितमिच्छता}
{आत्मानं सर्वथा रक्ष पुरीं चेमां सराक्षसाम्}
{स्वस्ति तेऽस्तु गमिष्यामि सुखी भव मया विना} %6-16-25

\twolineshloka
{निवार्यमाणस्य मया हितैषिणा न रोचते ते वचनं निशाचर}
{परान्तकाले हि गतायुषो नरा हितं न गृह्णन्ति सुहृद्भिरीरितम्} %6-16-26


॥इत्यार्षे श्रीमद्रामायणे वाल्मीकीये आदिकाव्ये युद्धकाण्डे विभीषणाक्रोशः नाम षोडशः सर्गः ॥६-१६॥
