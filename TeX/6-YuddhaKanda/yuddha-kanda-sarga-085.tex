\sect{पञ्चाशीतितमः सर्गः — निकुम्भिलाभियानम्}

\twolineshloka
{तस्य तद् वचनं श्रुत्वा राघवः शोककर्शितः}
{नोपधारयते व्यक्तं यदुक्तं तेन रक्षसा} %6-85-1

\twolineshloka
{ततो धैर्यमवष्टभ्य रामः परपुरञ्जयः}
{विभीषणमुपासीनमुवाच कपिसन्निधौ} %6-85-2

\twolineshloka
{नैर्ऋताधिपते वाक्यं यदुक्तं ते विभीषण}
{भूयस्तच्छ्रोतुमिच्छामि ब्रूहि यत्ते विवक्षितम्} %6-85-3

\twolineshloka
{राघवस्य वचः श्रुत्वा वाक्यं वाक्यविशारदः}
{यत् तत् पुनरिदं वाक्यं बभाषेऽथ विभीषणः} %6-85-4

\twolineshloka
{यथाऽऽज्ञप्तं महाबाहो त्वया गुल्मनिवेशनम्}
{तत् तथानुष्ठितं वीर त्वद्वाक्यसमनन्तरम्} %6-85-5

\twolineshloka
{तान्यनीकानि सर्वाणि विभक्तानि समन्ततः}
{विन्यस्ता यूथपाश्चैव यथान्यायं विभागशः} %6-85-6

\twolineshloka
{भूयस्तु मम विज्ञाप्यं तच्छृणुष्व महाप्रभो}
{त्वय्यकारणसन्तप्ते सन्तप्तहृदया वयम्} %6-85-7

\twolineshloka
{त्यज राजन्निमं शोकं मिथ्या सन्तापमागतम्}
{यदियं त्यज्यतां चिन्ता शत्रुहर्षविवर्धिनी} %6-85-8

\twolineshloka
{उद्यमः क्रियतां वीर हर्षः समुपसेव्यताम्}
{प्राप्तव्या यदि ते सीता हन्तव्याश्च निशाचराः} %6-85-9

\twolineshloka
{रघुनन्दन वक्ष्यामि श्रूयतां मे हितं वचः}
{साध्वयं यातु सौमित्रिर्बलेन महता वृतः} %6-85-10

\twolineshloka
{निकुम्भिलायां सम्प्राप्तं हन्तुं रावणिमाहवे}
{धनुर्मण्डलनिर्मुक्तैराशीविषविषोपमैः} %6-85-11

\threelineshloka
{शरैर्हन्तुं महेष्वासो रावणिं समितिञ्जयः}
{तेन वीरेण तपसा वरदानात् स्वयम्भुवः}
{अस्त्रं ब्रह्मशिरः प्राप्तं कामगाश्च तुरङ्गमाः} %6-85-12

\twolineshloka
{स एष किल सैन्येन प्राप्तः किल निकुम्भिलाम्}
{यद्युत्तिष्ठेत् कृतं कर्म हतान् सर्वांश्च विद्धि नः} %6-85-13

\twolineshloka
{निकुम्भिलामसम्प्राप्तमकृताग्निं च यो रिपुः}
{त्वामाततायिनं हन्यादिन्द्रशत्रो स ते वधः} %6-85-14

\twolineshloka
{वरो दत्तो महाबाहो सर्वलोकेश्वरेण वै}
{इत्येवं विहितो राजन् वधस्तस्यैष धीमतः} %6-85-15

\twolineshloka
{वधायेन्द्रजितो राम सन्दिशस्व महाबलम्}
{हते तस्मिन् हतं विद्धि रावणं ससुहृद्गणम्} %6-85-16

\twolineshloka
{विभीषणवचः श्रुत्वा रामो वाक्यमथाब्रवीत्}
{जानामि तस्य रौद्रस्य मायां सत्यपराक्रम} %6-85-17

\twolineshloka
{स हि ब्रह्मास्त्रवित् प्राज्ञो महामायो महाबलः}
{करोत्यसंज्ञान् सङ्ग्रामे देवान् सवरुणानपि} %6-85-18

\twolineshloka
{तस्यान्तरिक्षे चरतः सरथस्य महायशः}
{न गतिर्ज्ञायते वीर सूर्यस्येवाभ्रसम्प्लवे} %6-85-19

\twolineshloka
{राघवस्तु रिपोर्ज्ञात्वा मायावीर्यं दुरात्मनः}
{लक्ष्मणं कीर्तिसम्पन्नमिदं वचनमब्रवीत्} %6-85-20

\twolineshloka
{यद् वानरेन्द्रस्य बलं तेन सर्वेण संवृतः}
{हनूमत्प्रमुखैश्चैव यूथपैः सह लक्ष्मण} %6-85-21

\twolineshloka
{जाम्बवेनर्क्षपतिना सह सैन्येन संवृतः}
{जहि तं राक्षससुतं मायाबलसमन्वितम्} %6-85-22

\twolineshloka
{अयं त्वां सचिवैः सार्धं महात्मा रजनीचरः}
{अभिज्ञस्तस्य मायानां पृष्ठतोऽनुगमिष्यति} %6-85-23

\twolineshloka
{राघवस्य वचः श्रुत्वा लक्ष्मणः सविभीषणः}
{जग्राह कार्मुकश्रेष्ठमन्यद् भीमपराक्रमः} %6-85-24

\twolineshloka
{सन्नद्धः कवची खड्गी सशरी वामचापभृत्}
{रामपादावुपस्पृश्य हृष्टः सौमित्रिरब्रवीत्} %6-85-25

\twolineshloka
{अद्य मत्कार्मुकोन्मुक्ताः शरा निर्भिद्य रावणिम्}
{लङ्कामभिपतिष्यन्ति हंसाः पुष्करिणीमिव} %6-85-26

\twolineshloka
{अद्यैव तस्य रौद्रस्य शरीरं मामकाः शराः}
{विधमिष्यन्ति भित्त्वा तं महाचापगुणच्युताः} %6-85-27

\twolineshloka
{एवमुक्त्वा तु वचनं द्युतिमान् भ्रातुरग्रतः}
{स रावणिवधाकाङ्क्षी लक्ष्मणस्त्वरितं ययौ} %6-85-28

\twolineshloka
{सोऽभिवाद्य गुरोः पादौ कृत्वा चापि प्रदक्षिणम्}
{निकुम्भिलामभिययौ चैत्यं रावणिपालितम्} %6-85-29

\twolineshloka
{विभीषणेन सहितो राजपुत्रः प्रतापवान्}
{कृतस्वस्त्ययनो भ्रात्रा लक्ष्मणस्त्वरितो ययौ} %6-85-30

\twolineshloka
{वानराणां सहस्रैस्तु हनूमान् बहुभिर्वृतः}
{विभीषणश्च सामात्यो लक्ष्मणं त्वरितं ययौ} %6-85-31

\twolineshloka
{महता हरिसैन्येन सवेगमभिसंवृतः}
{ऋक्षराजबलं चैव ददर्श पथि विष्ठितम्} %6-85-32

\twolineshloka
{स गत्वा दूरमध्वानं सौमित्रिर्मित्रनन्दनः}
{राक्षसेन्द्रबलं दूरादपश्यद् व्यूहमाश्रितम्} %6-85-33

\twolineshloka
{स सम्प्राप्य धनुष्पाणिर्मायायोगमरिन्दमः}
{तस्थौ ब्रह्मविधानेन विजेतुं रघुनन्दनः} %6-85-34

\twolineshloka
{विभीषणेन सहितो राजपुत्रः प्रतापवान्}
{अङ्गदेन च वीरेण तथानिलसुतेन च} %6-85-35

\twolineshloka
{विविधममलशस्त्रभास्वरं तद् ध्वजगहनं गहनं महारथैश्च}
{प्रतिभयतममप्रमेयवेगं तिमिरमिव द्विषतां बलं विवेश} %6-85-36


॥इत्यार्षे श्रीमद्रामायणे वाल्मीकीये आदिकाव्ये युद्धकाण्डे निकुम्भिलाभियानम् नाम पञ्चाशीतितमः सर्गः ॥६-८५॥
