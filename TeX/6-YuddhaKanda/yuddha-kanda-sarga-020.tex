\sect{विंशः सर्गः — सुग्रीवभेदनोपायः}

\twolineshloka
{ततो निविष्टां ध्वजिनीं सुग्रीवेणाभिपालिताम्}
{ददर्श राक्षसोऽभ्येत्य शार्दूलो नाम वीर्यवान्} %6-20-1

\twolineshloka
{चारो राक्षसराजस्य रावणस्य दुरात्मनः}
{तां दृष्ट्वा सर्वतोऽव्यग्रां प्रतिगम्य स राक्षसः} %6-20-2

\twolineshloka
{आविश्य लङ्कां वेगेन राजानमिदमब्रवीत्}
{एष वै वानरर्क्षौघो लङ्कां समभिवर्तते} %6-20-3

\twolineshloka
{अगाधश्चाप्रमेयश्च द्वितीय इव सागरः}
{पुत्रौ दशरथस्येमौ भ्रातरौ रामलक्ष्मणौ} %6-20-4

\twolineshloka
{उत्तमौ रूपसम्पन्नौ सीतायाः पदमागतौ}
{एतौ सागरमासाद्य संनिविष्टौ महाद्युते} %6-20-5

\twolineshloka
{बलं चाकाशमावृत्य सर्वतो दशयोजनम्}
{तत्त्वभूतं महाराज क्षिप्रं वेदितुमर्हसि} %6-20-6

\twolineshloka
{तव दूता महाराज क्षिप्रमर्हन्ति वेदितुम्}
{उपप्रदानं सान्त्वं वा भेदो वात्र प्रयुज्यताम्} %6-20-7

\threelineshloka
{शार्दूलस्य वचः श्रुत्वा रावणो राक्षसेश्वरः}
{उवाच सहसा व्यग्रः सम्प्रधार्यार्थमात्मनः}
{शुकं साधु तदा रक्षो वाक्यमर्थविदां वरम्} %6-20-8

\twolineshloka
{सुग्रीवं ब्रूहि गत्वाऽऽशु राजानं वचनान्मम}
{यथासंदेशमक्लीबं श्लक्ष्णया परया गिरा} %6-20-9

\twolineshloka
{त्वं वै महाराजकुलप्रसूतो महाबलश्चर्क्षरजःसुतश्च}
{न कश्चनार्थस्तव नास्त्यनर्थस्तथापि मे भ्रातृसमो हरीश} %6-20-10

\twolineshloka
{अहं यद्यहरं भार्यां राजपुत्रस्य धीमतः}
{किं तत्र तव सुग्रीव किष्किन्धां प्रति गम्यताम्} %6-20-11

\twolineshloka
{नहीयं हरिभिर्लङ्का प्राप्तुं शक्या कथंचन}
{देवैरपि सगन्धर्वैः किं पुनर्नरवानरैः} %6-20-12

\twolineshloka
{स तदा राक्षसेन्द्रेण संदिष्टो रजनीचरः}
{शुको विहंगमो भूत्वा तूर्णमाप्लुत्य चाम्बरम्} %6-20-13

\twolineshloka
{स गत्वा दूरमध्वानमुपर्युपरि सागरम्}
{संस्थितो ह्यम्बरे वाक्यं सुग्रीवमिदमब्रवीत्} %6-20-14

\twolineshloka
{सर्वमुक्तं यथाऽऽदिष्टं रावणेन दुरात्मना}
{तत् प्रापयन्तं वचनं तूर्णमाप्लुत्य वानराः} %6-20-15

\twolineshloka
{प्रापद्यन्त तदा क्षिप्रं लोप्तुं हन्तुं च मुष्टिभिः}
{सर्वैः प्लवंगैः प्रसभं निगृहीतो निशाचरः} %6-20-16

\twolineshloka
{गगनाद् भूतले चाशु प्रतिगृह्यावतारितः}
{वानरैः पीड्यमानस्तु शुको वचनमब्रवीत्} %6-20-17

\threelineshloka
{न दूतान् घ्नन्ति काकुत्स्थ वार्यन्तां साधु वानराः}
{यस्तु हित्वा मतं भर्तुः स्वमतं सम्प्रधारयेत्}
{अनुक्तवादी दूतः सन् स दूतो वधमर्हति} %6-20-18

\twolineshloka
{शुकस्य वचनं रामः श्रुत्वा तु परिदेवितम्}
{उवाच मावधिष्टेति घ्नतः शाखामृगर्षभान्} %6-20-19

\twolineshloka
{स च पत्रलघुर्भूत्वा हरिभिर्दर्शितेऽभये}
{अन्तरिक्षे स्थितो भूत्वा पुनर्वचनमब्रवीत्} %6-20-20

\twolineshloka
{सुग्रीव सत्त्वसम्पन्न महाबलपराक्रम}
{किं मया खलु वक्तव्यो रावणो लोकरावणः} %6-20-21

\twolineshloka
{स एवमुक्तः प्लवगाधिपस्तदा प्लवंगमानामृषभो महाबलः}
{उवाच वाक्यं रजनीचरस्य चारं शुकं शुद्धमदीनसत्त्वः} %6-20-22

\twolineshloka
{न मेऽसि मित्रं न तथानुकम्प्यो न चोपकर्तासि न मे प्रियोऽसि}
{अरिश्च रामस्य सहानुबन्धस्ततोऽसि वालीव वधार्ह वध्यः} %6-20-23

\twolineshloka
{निहन्म्यहं त्वां ससुतं सबन्धुं सज्ञातिवर्गं रजनीचरेश}
{लङ्कां च सर्वां महता बलेन सर्वैः करिष्यामि समेत्य भस्म} %6-20-24

\threelineshloka
{न मोक्ष्यसे रावण राघवस्य सुरैः सहेन्द्रैरपि मूढ गुप्तः}
{अन्तर्हितः सूर्यपथं गतोऽपि तथैव पातालमनुप्रविष्टः}
{गिरीशपादाम्बुजसंगतो वा हतोऽसि रामेण सहानुजस्त्वम्} %6-20-25

\twolineshloka
{तस्य ते त्रिषु लोकेषु न पिशाचं न राक्षसम्}
{त्रातारं नानुपश्यामि न गन्धर्वं न चासुरम्} %6-20-26

\threelineshloka
{अवधीस्त्वं जरावृद्धं गृध्रराजं जटायुषम्}
{किं नु ते रामसांनिध्ये सकाशे लक्ष्मणस्य च}
{हृता सीता विशालाक्षी यां त्वं गृह्य न बुध्यसे} %6-20-27

\twolineshloka
{महाबलं महात्मानं दुराधर्षं सुरैरपि}
{न बुध्यसे रघुश्रेष्ठं यस्ते प्राणान् हरिष्यति} %6-20-28

\twolineshloka
{ततोऽब्रवीद् वालिसुतोऽप्यङ्गदो हरिसत्तमः}
{नायं दूतो महाराज चारकः प्रतिभाति मे} %6-20-29

\twolineshloka
{तुलितं हि बलं सर्वमनेन तव तिष्ठता}
{गृह्यतां मागमल्लङ्कामेतद्धि मम रोचते} %6-20-30

\twolineshloka
{ततो राज्ञा समादिष्टाः समुत्पत्य वलीमुखाः}
{जगृहुश्च बबन्धुश्च विलपन्तमनाथवत्} %6-20-31

\threelineshloka
{शुकस्तु वानरैश्चण्डैस्तत्र तैः सम्प्रपीडितः}
{व्याचुक्रोश महात्मानं रामं दशरथात्मजम्}
{लुप्येते मे बलात् पक्षौ भिद्येते मे तथाक्षिणी} %6-20-32

\threelineshloka
{यां च रात्रिं मरिष्यामि जाये रात्रिं च यामहम्}
{एतस्मिन्नन्तरे काले यन्मया ह्यशुभं कृतम्}
{सर्वं तदुपपद्येथा जह्यां चेद् यदि जीवितम्} %6-20-33

\twolineshloka
{नाघातयत् तदा रामः श्रुत्वा तत्परिदेवितम्}
{वानरानब्रवीद् रामो मुच्यतां दूत आगतः} %6-20-34


॥इत्यार्षे श्रीमद्रामायणे वाल्मीकीये आदिकाव्ये युद्धकाण्डे सुग्रीवभेदनोपायः नाम विंशः सर्गः ॥६-२०॥
