\sect{चतुस्त्रिंशः सर्गः — रावणनिश्चयकथनम्}

\twolineshloka
{अथ तां जातसन्तापां तेन वाक्येन मोहिताम्}
{सरमा ह्लादयामास महीं दग्धामिवाम्भसा} %6-34-1

\twolineshloka
{ततस्तस्या हितं सख्याश्चिकीर्षन्ती सखी वचः}
{उवाच काले कालज्ञा स्मितपूर्वाभिभाषिणी} %6-34-2

\twolineshloka
{उत्सहेयमहं गत्वा त्वद्वाक्यमसितेक्षणे}
{निवेद्य कुशलं रामे प्रतिच्छन्ना निवर्तितुम्} %6-34-3

\twolineshloka
{नहि मे क्रममाणाया निरालम्बे विहायसि}
{समर्थो गतिमन्वेतुं पवनो गरुडोऽपि वा} %6-34-4

\twolineshloka
{एवं ब्रुवाणां तां सीता सरमामिदमब्रवीत्}
{मधुरं श्लक्ष्णया वाचा पूर्वशोकाभिपन्नया} %6-34-5

\twolineshloka
{समर्था गगनं गन्तुमपि च त्वं रसातलम्}
{अवगच्छाद्य कर्तव्यं कर्तव्यं ते मदन्तरे} %6-34-6

\twolineshloka
{मत्प्रियं यदि कर्तव्यं यदि बुद्धिः स्थिरा तव}
{ज्ञातुमिच्छामि तं गत्वा किं करोतीति रावणः} %6-34-7

\twolineshloka
{स हि मायाबलः क्रूरो रावणः शत्रुरावणः}
{मां मोहयति दुष्टात्मा पीतमात्रेव वारुणी} %6-34-8

\twolineshloka
{तर्जापयति मां नित्यं भर्त्सापयति चासकृत्}
{राक्षसीभिः सुघोराभिर्यो मां रक्षति नित्यशः} %6-34-9

\twolineshloka
{उद्विग्ना शङ्किता चास्मि न स्वस्थं च मनो मम}
{तद्भयाच्चाहमुद्विग्ना अशोकवनिकां गता} %6-34-10

\twolineshloka
{यदि नाम कथा तस्य निश्चितं वापि यद्भवेत्}
{निवेदयेथाः सर्वं तद् वरो मे स्यादनुग्रहः} %6-34-11

\twolineshloka
{साप्येवं ब्रुवतीं सीतां सरमा मृदुभाषिणी}
{उवाच वदनं तस्याः स्पृशन्ती बाष्पविक्लवम्} %6-34-12

\twolineshloka
{एष ते यद्यभिप्रायस्तस्माद् गच्छामि जानकि}
{गृह्य शत्रोरभिप्रायमुपावर्तामि मैथिलि} %6-34-13

\twolineshloka
{एवमुक्त्वा ततो गत्वा समीपं तस्य रक्षसः}
{शुश्राव कथितं तस्य रावणस्य समन्त्रिणः} %6-34-14

\twolineshloka
{सा श्रुत्वा निश्चयं तस्य निश्चयज्ञा दुरात्मनः}
{पुनरेवागमत् क्षिप्रमशोकवनिकां शुभाम्} %6-34-15

\twolineshloka
{सा प्रविष्टा ततस्तत्र ददर्श जनकात्मजाम्}
{प्रतीक्षमाणां स्वामेव भ्रष्टपद्मामिव श्रियम्} %6-34-16

\twolineshloka
{तां तु सीता पुनः प्राप्तां सरमां प्रियभाषिणीम्}
{परिष्वज्य च सुस्निग्धं ददौ च स्वयमासनम्} %6-34-17

\twolineshloka
{इहासीना सुखं सर्वमाख्याहि मम तत्त्वतः}
{क्रूरस्य निश्चयं तस्य रावणस्य दुरात्मनः} %6-34-18

\twolineshloka
{एवमुक्ता तु सरमा सीतया वेपमानया}
{कथितं सर्वमाचष्ट रावणस्य समन्त्रिणः} %6-34-19

\twolineshloka
{जनन्या राक्षसेन्द्रो वै त्वन्मोक्षार्थं बृहद्वचः}
{अतिस्निग्धेन वैदेहि मन्त्रिवृद्धेन चोदितः} %6-34-20

\twolineshloka
{दीयतामभिसत्कृत्य मनुजेन्द्राय मैथिली}
{निदर्शनं ते पर्याप्तं जनस्थाने यदद्भुतम्} %6-34-21

\twolineshloka
{लङ्घनं च समुद्रस्य दर्शनं च हनूमतः}
{वधं च रक्षसां युद्धे कः कुर्यान्मानुषो युधि} %6-34-22

\twolineshloka
{एवं स मन्त्रवृद्धैश्च मात्रा च बहुबोधितः}
{न त्वामुत्सहते मोक्तुमर्थमर्थपरो यथा} %6-34-23

\twolineshloka
{नोत्सहत्यमृतो मोक्तुं युद्धे त्वामिति मैथिलि}
{सामात्यस्य नृशंसस्य निश्चयो ह्येष वर्तते} %6-34-24

\twolineshloka
{तदेषा सुस्थिरा बुद्धिर्मृत्युलोभादुपस्थिता}
{भयान्न शक्तस्त्वां मोक्तुमनिरस्तः स संयुगे} %6-34-25

\threelineshloka
{राक्षसानां च सर्वेषामात्मनश्च वधेन हि}
{निहत्य रावणं सङ्ख्ये सर्वथा निशितैः शरैः}
{प्रतिनेष्यति रामस्त्वामयोध्यामसितेक्षणे} %6-34-26

\twolineshloka
{एतस्मिन्नन्तरे शब्दो भेरीशङ्खसमाकुलः}
{श्रुतो वै सर्वसैन्यानां कम्पयन् धरणीतलम्} %6-34-27

\twolineshloka
{श्रुत्वा तु तं वानरसैन्यनादं लङ्कागता राक्षसराजभृत्याः}
{हतौजसो दैन्यपरीतचेष्टाः श्रेयो न पश्यन्ति नृपस्य दोषात्} %6-34-28


॥इत्यार्षे श्रीमद्रामायणे वाल्मीकीये आदिकाव्ये युद्धकाण्डे रावणनिश्चयकथनम् नाम चतुस्त्रिंशः सर्गः ॥६-३४॥
