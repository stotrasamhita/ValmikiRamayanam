\sect{त्रयस्त्रिंशः सर्गः — सरमासमाश्वासनम्}

\twolineshloka
{सीतां तु मोहितां दृष्ट्वा सरमा नाम राक्षसी}
{आससादाथ वैदेहीं प्रियां प्रणयिनी सखीम्} %6-33-1

\twolineshloka
{मोहितां राक्षसेन्द्रेण सीतां परमदुःखिताम्}
{आश्वासयामास तदा सरमा मृदुभाषिणी} %6-33-2

\twolineshloka
{सा हि तत्र कृता मित्रं सीतया रक्ष्यमाणया}
{रक्षन्ती रावणादिष्टा सानुक्रोशा दृढव्रता} %6-33-3

\twolineshloka
{सा ददर्श सखी सीतां सरमा नष्टचेतनाम्}
{उपावृत्योत्थितां ध्वस्तां वडवामिव पांसुषु} %6-33-4

\threelineshloka
{तां समाश्वासयामास सखीस्नेहेन सुव्रताम्}
{समाश्वसिहि वैदेहि मा भूत् ते मनसो व्यथा}
{उक्ता यद् रावणेन त्वं प्रयुक्तश्च स्वयं त्वया} %6-33-5

\threelineshloka
{सखीस्नेहेन तद् भीरु मया सर्वं प्रतिश्रुतम्}
{लीनया गहने शून्ये भयमुत्सृज्य रावणात्}
{तव हेतोर्विशालाक्षि नहि मे रावणाद् भयम्} %6-33-6

\twolineshloka
{स सम्भ्रान्तश्च निष्क्रान्तो यत्कृते राक्षसेश्वरः}
{तत्र मे विदितं सर्वमभिनिष्क्रम्य मैथिलि} %6-33-7

\twolineshloka
{न शक्यं सौप्तिकं कर्तुं रामस्य विदितात्मनः}
{वधश्च पुरुषव्याघ्रे तस्मिन् नैवोपपद्यते} %6-33-8

\twolineshloka
{न त्वेवं वानरा हन्तुं शक्याः पादपयोधिनः}
{सुरा देवर्षभेणेव रामेण हि सुरक्षिताः} %6-33-9

\twolineshloka
{दीर्घवृत्तभुजः श्रीमान् महोरस्कः प्रतापवान्}
{धन्वी सन्नहनोपेतो धर्मात्मा भुवि विश्रुतः} %6-33-10

\twolineshloka
{विक्रान्तो रक्षिता नित्यमात्मनश्च परस्य च}
{लक्ष्मणेन सह भ्रात्रा कुलीनो नयशास्त्रवित्} %6-33-11

\twolineshloka
{हन्ता परबलौघानामचिन्त्यबलपौरुषः}
{न हतो राघवः श्रीमान् सीते शत्रुनिबर्हणः} %6-33-12

\twolineshloka
{अयुक्तबुद्धिकृत्येन सर्वभूतविरोधिना}
{एवं प्रयुक्ता रौद्रेण माया मायाविना त्वयि} %6-33-13

\twolineshloka
{शोकस्ते विगतः सर्वकल्याणं त्वामुपस्थितम्}
{ध्रुवं त्वां भजते लक्ष्मीः प्रियं ते भवति शृणु} %6-33-14

\twolineshloka
{उत्तीर्य सागरं रामः सह वानरसेनया}
{सन्निविष्टः समुद्रस्य तीरमासाद्य दक्षिणम्} %6-33-15

\twolineshloka
{दृष्टो मे परिपूर्णार्थः काकुत्स्थः सहलक्ष्मणः}
{सहितैः सागरान्तस्थैर्बलैस्तिष्ठति रक्षितः} %6-33-16

\twolineshloka
{अनेन प्रेषिता ये च राक्षसा लघुविक्रमाः}
{राघवस्तीर्ण इत्येवं प्रवृत्तिस्तैरिहाहृता} %6-33-17

\twolineshloka
{स तां श्रुत्वा विशालाक्षि प्रवृत्तिं राक्षसाधिपः}
{एष मन्त्रयते सर्वैः सचिवैः सह रावणः} %6-33-18

\twolineshloka
{इति ब्रुवाणा सरमा राक्षसी सीतया सह}
{सर्वोद्योगेन सैन्यानां शब्दं शुश्राव भैरवम्} %6-33-19

\twolineshloka
{दण्डनिर्घातवादिन्याः श्रुत्वा भेर्या महास्वनम्}
{उवाच सरमा सीतामिदं मधुरभाषिणी} %6-33-20

\twolineshloka
{सन्नाहजननी ह्येषा भैरवा भीरु भेरिका}
{भेरीनादं च गम्भीरं शृणु तोयदनिःस्वनम्} %6-33-21

\twolineshloka
{कल्प्यन्ते मत्तमातङ्गा युज्यन्ते रथवाजिनः}
{दृश्यन्ते तुरगारूढाः प्रासहस्ताः सहस्रशः} %6-33-22

\twolineshloka
{तत्र तत्र च सन्नद्धाः सम्पतन्ति सहस्रशः}
{आपूर्यन्ते राजमार्गाः सैन्यैरद्भुतदर्शनैः} %6-33-23

\twolineshloka
{वेगवद्भिर्नदद्भिश्च तोयौघैरिव सागरः}
{शस्त्राणां च प्रसन्नानां चर्मणां वर्मणां तथा} %6-33-24

\twolineshloka
{रथवाजिगजानां च राक्षसेन्द्रानुयायिनाम्}
{सम्भ्रमो रक्षसामेष हृषितानां तरस्विनाम्} %6-33-25

\twolineshloka
{प्रभां विसृजतां पश्य नानावर्णसमुत्थिताम्}
{वनं निर्दहतो घर्मे यथा रूपं विभावसोः} %6-33-26

\twolineshloka
{घण्टानां शृणु निर्घोषं रथानां शृणु निःस्वनम्}
{हयानां हेषमाणानां शृणु तूर्यध्वनिं तथा} %6-33-27

\twolineshloka
{उद्यतायुधहस्तानां राक्षसेन्द्रानुयायिनाम्}
{सम्भ्रमो रक्षसामेष तुमुलो लोमहर्षणम्} %6-33-28

\twolineshloka
{श्रीस्त्वां भजति शोकघ्नी रक्षसां भयमागतम्}
{रामः कमलपत्राक्षो दैत्यानामिव वासवः} %6-33-29

\twolineshloka
{अवजित्य जितक्रोधस्तमचिन्त्यपराक्रमः}
{रावणं समरे हत्वा भर्ता त्वाधिगमिष्यति} %6-33-30

\twolineshloka
{विक्रमिष्यति रक्षःसु भर्ता ते सहलक्ष्मणः}
{यथा शत्रुषु शत्रुघ्नो विष्णुना सह वासवः} %6-33-31

\twolineshloka
{आगतस्य हि रामस्य क्षिप्रमङ्कागतां सतीम्}
{अहं द्रक्ष्यामि सिद्धार्थां त्वां शत्रौ विनिपातिते} %6-33-32

\twolineshloka
{अस्राण्यानन्दजानि त्वं वर्तयिष्यसि जानकि}
{समागम्य परिष्वक्ता तस्योरसि महोरसः} %6-33-33

\twolineshloka
{अचिरान्मोक्ष्यते सीते देवि ते जघनं गताम्}
{धृतामेकां बहून् मासान् वेणीं रामो महाबलः} %6-33-34

\twolineshloka
{तस्य दृष्ट्वा मुखं देवि पूर्णचन्द्रमिवोदितम्}
{मोक्ष्यसे शोकजं वारि निर्मोकमिव पन्नगी} %6-33-35

\twolineshloka
{रावणं समरे हत्वा नचिरादेव मैथिलि}
{त्वया समग्रः प्रियया सुखार्हो लप्स्यते सुखम्} %6-33-36

\twolineshloka
{सभाजिता त्वं रामेण मोदिष्यसि महात्मना}
{सुवर्षेण समायुक्ता यथा सस्येन मेदिनी} %6-33-37

\twolineshloka
{गिरिवरमभितो विवर्तमानो हय इव मण्डलमाशु यः करोति}
{तमिह शरणमभ्युपैहि देवि दिवसकरं प्रभवो ह्ययं प्रजानाम्} %6-33-38


॥इत्यार्षे श्रीमद्रामायणे वाल्मीकीये आदिकाव्ये युद्धकाण्डे सरमासमाश्वासनम् नाम त्रयस्त्रिंशः सर्गः ॥६-३३॥
