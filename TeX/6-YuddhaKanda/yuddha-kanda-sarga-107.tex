\sect{सप्ताधिकशततमः सर्गः — आदित्यहृदयम्}

\twolineshloka
{ततो युद्धपरिश्रान्तं समरे चिन्तया स्थितम्}
{रावणं चाग्रतो दृष्ट्वा युद्धाय समुपस्थितम्} %6-107-1

\twolineshloka
{दैवतैश्च समागम्य द्रष्टुमभ्यागतो रणम्}
{उपगम्याब्रवीद् राममगस्त्यो भगवांस्तदा} %6-107-2

\twolineshloka
{राम राम महाबाहो शृणु गुह्यं सनातनम्}
{येन सर्वानरीन् वत्स समरे विजयिष्यसे} %6-107-3

\twolineshloka
{आदित्यहृदयं पुण्यं सर्वशत्रुविनाशनम्}
{जयावहं जपं नित्यमक्षयं परमं शिवम्} %6-107-4

\twolineshloka
{सर्वमङ्गलमाङ्गल्यं सर्वपापप्रणाशनम्}
{चिन्ताशोकप्रशमनमायुर्वर्धनमुत्तमम्} %6-107-5

\twolineshloka
{रश्मिमन्तं समुद्यन्तं देवासुरनमस्कृतम्}
{पूजयस्व विवस्वन्तं भास्करं भुवनेश्वरम्} %6-107-6

\twolineshloka
{सर्वदेवात्मको ह्येष तेजस्वी रश्मिभावनः}
{एष देवासुरगणाँल्लोकान् पाति गभस्तिभिः} %6-107-7

\twolineshloka
{एष ब्रह्मा च विष्णुश्च शिवः स्कन्दः प्रजापतिः}
{महेन्द्रो धनदः कालो यमः सोमो ह्यपां पतिः} %6-107-8

\twolineshloka
{पितरो वसवः साध्या अश्विनौ मरुतो मनुः}
{वायुर्वह्निः प्रजाः प्राण ऋतुकर्ता प्रभाकरः} %6-107-9

\twolineshloka
{आदित्यः सविता सूर्यः खगः पूषा गभस्तिमान्}
{सुवर्णसदृशो भानुर्हिरण्यरेता दिवाकरः} %6-107-10

\twolineshloka
{हरिदश्वः सहस्रार्चिः सप्तसप्तिर्मरीचिमान्}
{तिमिरोन्मथनः शम्भुस्त्वष्टा मार्तण्डकोंऽशुमान्} %6-107-11

\twolineshloka
{हिरण्यगर्भः शिशिरस्तपनोऽहस्करो रविः}
{अग्निगर्भोऽदितेः पुत्रः शङ्खः शिशिरनाशनः} %6-107-12

\twolineshloka
{व्योमनाथस्तमोभेदी ऋग्यजुःसामपारगः}
{घनवृष्टिरपां मित्रो विन्ध्यवीथीप्लवङ्गमः} %6-107-13

\twolineshloka
{आतपी मण्डली मृत्युः पिङ्गलः सर्वतापनः}
{कविर्विश्वो महातेजा रक्तः सर्वभवोद्भवः} %6-107-14

\twolineshloka
{नक्षत्रग्रहताराणामधिपो विश्वभावनः}
{तेजसामपि तेजस्वी द्वादशात्मन् नमोऽस्तु ते} %6-107-15

\twolineshloka
{नमः पूर्वाय गिरये पश्चिमायाद्रये नमः}
{ज्योतिर्गणानां पतये दिनाधिपतये नमः} %6-107-16

\twolineshloka
{जयाय जयभद्राय हर्यश्वाय नमो नमः}
{नमो नमः सहस्रांशो आदित्याय नमो नमः} %6-107-17

\twolineshloka
{नम उग्राय वीराय सारङ्गाय नमो नमः}
{नमः पद्मप्रबोधाय प्रचण्डाय नमोऽस्तु ते} %6-107-18

\twolineshloka
{ब्रह्मेशानाच्युतेशाय सूरायादित्यवर्चसे}
{भास्वते सर्वभक्षाय रौद्राय वपुषे नमः} %6-107-19

\twolineshloka
{तमोघ्नाय हिमघ्नाय शत्रुघ्नायामितात्मने}
{कृतघ्नघ्नाय देवाय ज्योतिषां पतये नमः} %6-107-20

\twolineshloka
{तप्तचामीकराभाय हरये विश्वकर्मणे}
{नमस्तमोऽभिनिघ्नाय रुचये लोकसाक्षिणे} %6-107-21

\twolineshloka
{नाशयत्येष वै भूतं तमेव सृजति प्रभुः}
{पायत्येष तपत्येष वर्षत्येष गभस्तिभिः} %6-107-22

\twolineshloka
{एष सुप्तेषु जागर्ति भूतेषु परिनिष्ठितः}
{एष चैवाग्निहोत्रं च फलं चैवाग्निहोत्रिणाम्} %6-107-23

\twolineshloka
{देवाश्च क्रतवश्चैव क्रतूनां फलमेव च}
{यानि कृत्यानि लोकेषु सर्वेषु परमप्रभुः} %6-107-24

\twolineshloka
{एनमापत्सु कृच्छ्रेषु कान्तारेषु भयेषु च}
{कीर्तयन् पुरुषः कश्चिन्नावसीदति राघव} %6-107-25

\twolineshloka
{पूजयस्वैनमेकाग्रो देवदेवं जगत्पतिम्}
{एतत् त्रिगुणितं जप्त्वा युद्धेषु विजयिष्यति} %6-107-26

\twolineshloka
{अस्मिन् क्षणे महाबाहो रावणं त्वं जहिष्यसि}
{एवमुक्त्वा ततोऽगस्त्यो जगाम स यथागतम्} %6-107-27

\twolineshloka
{एतच्छ्रुत्वा महातेजा नष्टशोकोऽभवत् तदा}
{धारयामास सुप्रीतो राघवः प्रयतात्मवान्} %6-107-28

\twolineshloka
{आदित्यं प्रेक्ष्य जप्त्वेदं परं हर्षमवाप्तवान्}
{त्रिराचम्य शुचिर्भूत्वा धनुरादाय वीर्यवान्} %6-107-29

\twolineshloka
{रावणं प्रेक्ष्य हृष्टात्मा जयार्थं समुपागमत्}
{सर्वयत्नेन महता वृतस्तस्य वधेऽभवत्} %6-107-30

\twolineshloka
{अथ रविरवदन्निरीक्ष्य रामं मुदितमनाः परमं प्रहृष्यमाणः}
{निशिचरपतिसङ्क्षयं विदित्वा सुरगणमध्यगतो वचस्त्वरेति} %6-107-31


॥इत्यार्षे श्रीमद्रामायणे वाल्मीकीये आदिकाव्ये युद्धकाण्डे आदित्यहृदयम् नाम सप्ताधिकशततमः सर्गः ॥६-१०७॥
