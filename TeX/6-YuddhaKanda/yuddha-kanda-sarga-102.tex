\sect{द्व्यधिकशततमः सर्गः — लक्ष्मणसंजीवनम्}

\twolineshloka
{शक्त्या निपातितं दृष्ट्वा रावणेन बलीयसा}
{लक्ष्मणं समरे शूरं शोणितौघपरिप्लुतम्} %6-102-1

\twolineshloka
{स दत्त्वा तुमुलं युद्धं रावणस्य दुरात्मनः}
{विसृजन्नेव बाणौघान् सुषेणमिदमब्रवीत्} %6-102-2

\twolineshloka
{एष रावणवीर्येण लक्ष्मणः पतितो भुवि}
{सर्पवच्चेष्टते वीरो मम शोकमुदीरयन्} %6-102-3

\twolineshloka
{शोणितार्द्रमिमं वीरं प्राणैः प्रियतरं मम}
{पश्यतो मम का शक्तिर्योद्धुं पर्याकुलात्मनः} %6-102-4

\twolineshloka
{अयं स समरश्लाघी भ्राता मे शुभलक्षणः}
{यदि पञ्चत्वमापन्नः प्राणैर्मे किं सुखेन वा} %6-102-5

\twolineshloka
{लज्जतीव हि मे वीर्यं भ्रश्यतीव कराद् धनुः}
{सायका व्यवसीदन्ति दृष्टिर्बाष्पवशं गता} %6-102-6

\twolineshloka
{अवसीदन्ति गात्राणि स्वप्नयाने नृणामिव}
{चिन्ता मे वर्धते तीव्रा मुमूर्षापि च जायते} %6-102-7

\twolineshloka
{भ्रातरं निहतं दृष्ट्वा रावणेन दुरात्मना}
{विष्टनन्तं तु दुःखार्तं मर्मण्यभिहतं भृशम्} %6-102-8

\twolineshloka
{राघवो भ्रातरं दृष्ट्वा प्रियं प्राणं बहिश्चरम्}
{दुःखेन महताविष्टो ध्यानशोकपरायणः} %6-102-9

\twolineshloka
{परं विषादमापन्नो विललापाकुलेन्द्रियः}
{भ्रातरं निहतं दृष्ट्वा लक्ष्मणं रणपांसुषु} %6-102-10

\twolineshloka
{विजयोऽपि हि मे शूर न प्रियायोपकल्पते}
{अचक्षुर्विषयश्चन्द्रः कां प्रीतिं जनयिष्यति} %6-102-11

\twolineshloka
{किं मे युद्धेन किं प्राणैर्युद्धकार्यं न विद्यते}
{यत्रायं निहतः शेते रणमूर्धनि लक्ष्मणः} %6-102-12

\twolineshloka
{यथैव मां वनं यान्तमनुयाति महाद्युतिः}
{अहमप्यनुयास्यामि तथैवैनं यमक्षयम्} %6-102-13

\twolineshloka
{इष्टबन्धुजनो नित्यं मां स नित्यमनुव्रतः}
{इमामवस्थां गमितो राक्षसैः कूटयोधिभिः} %6-102-14

\twolineshloka
{देशे देशे कलत्राणि देशे देशे च बान्धवाः}
{तं तु देशं न पश्यामि यत्र भ्राता सहोदरः} %6-102-15

\twolineshloka
{किं नु राज्येन दुर्धर्षलक्ष्मणेन विना मम}
{कथं वक्ष्याम्यहं त्वम्बां सुमित्रां पुत्रवत्सलाम्} %6-102-16

\twolineshloka
{उपालम्भं न शक्ष्यामि सोढुं दत्तं सुमित्रया}
{किं नु वक्ष्यामि कौसल्यां मातरं किं नु कैकयीम्} %6-102-17

\twolineshloka
{भरतं किं नु वक्ष्यामि शत्रुघ्नं च महाबलम्}
{सह तेन वनं यातो विना तेनागतः कथम्} %6-102-18

\twolineshloka
{इहैव मरणं श्रेयो न तु बन्धुविगर्हणम्}
{किं मया दुष्कृतं कर्म कृतमन्यत्र जन्मनि} %6-102-19

\twolineshloka
{येन मे धार्मिको भ्राता निहतश्चाग्रतः स्थितः}
{हा भ्रातर्मनुजश्रेष्ठ शूराणां प्रवर प्रभो} %6-102-20

\twolineshloka
{एकाकी किं नु मां त्यक्त्वा परलोकाय गच्छसि}
{विलपन्तं च मां भ्रातः किमर्थं नावभाषसे} %6-102-21

\twolineshloka
{उत्तिष्ठ पश्य किं शेषे दीनं मां पश्य चक्षुषा}
{शोकार्तस्य प्रमत्तस्य पर्वतेषु वनेषु च} %6-102-22

\twolineshloka
{विषण्णस्य महाबाहो समाश्वासयिता मम}
{राममेवं ब्रुवाणं तु शोकव्याकुलितेन्द्रियम्} %6-102-23

\twolineshloka
{आश्वासयन्नुवाचेदं सुषेणः परमं वचः}
{त्यजेमां नरशार्दूल बुद्धिं वैक्लव्यकारिणीम्} %6-102-24

\twolineshloka
{शोकसंजननीं चिन्तां तुल्यां बाणैश्चमूमुखे}
{नैव पञ्चत्वमापन्नो लक्ष्मणो लक्ष्मिवर्धनः} %6-102-25

\twolineshloka
{नह्यस्य विकृतं वक्त्रं न च श्यामत्वमागतम्}
{सुप्रभं च प्रसन्नं च मुखमस्य निरीक्ष्यताम्} %6-102-26

\twolineshloka
{पद्मपत्रतलौ हस्तौ सुप्रसन्ने च लोचने}
{नेदृशं दृश्यते रूपं गतासूनां विशां पते} %6-102-27

\twolineshloka
{विषादं मा कृथा वीर सप्राणोऽयमरिंदम}
{आख्याति तु प्रसुप्तस्य स्रस्तगात्रस्य भूतले} %6-102-28

\twolineshloka
{सोच्छ्वासं हृदयं वीर कम्पमानं मुहुर्मुहुः}
{एवमुक्त्वा महाप्राज्ञः सुषेणो राघवं वचः} %6-102-29

\twolineshloka
{समीपस्थमुवाचेदं हनूमन्तं महाकपिम्}
{सौम्य शीघ्रमितो गत्वा पर्वतं हि महोदयम्} %6-102-30

\twolineshloka
{पूर्वं तु कथितो योऽसौ वीर जाम्बवता तव}
{दक्षिणे शिखरे जातां महौषधिमिहानय} %6-102-31

\twolineshloka
{विशल्यकरणीं नाम्ना सावर्ण्यकरणीं तथा}
{संजीवकरणीं वीर संधानीं च महौषधीम्} %6-102-32

\threelineshloka
{संजीवनार्थं वीरस्य लक्ष्मणस्य त्वमानय}
{इत्येवमुक्तो हनुमान् गत्वा चौषधिपर्वतम्}
{चिन्तामभ्यगमच्छ्रीमानजानंस्ता महौषधीः} %6-102-33

\twolineshloka
{तस्य बुद्धिः समुत्पन्ना मारुतेरमितौजसः}
{इदमेव गमिष्यामि गृहीत्वा शिखरं गिरेः} %6-102-34

\twolineshloka
{अस्मिंस्तु शिखरे जातामोषधीं तां सुखावहाम्}
{प्रतर्केणावगच्छामि सुषेणो ह्येवमब्रवीत्} %6-102-35

\twolineshloka
{अगृह्य यदि गच्छामि विशल्यकरणीमहम्}
{कालात्ययेन दोषः स्याद् वैक्लव्यं च महद्भवेत्} %6-102-36

\twolineshloka
{इति संचिन्त्य हनुमान् गत्वा क्षिप्रं महाबलः}
{आसाद्य पर्वतश्रेष्ठं त्रिः प्रकम्प्य गिरेः शिरः} %6-102-37

\twolineshloka
{फुल्लनानातरुगणं समुत्पाट्य महाबलः}
{गृहीत्वा हरिशार्दूलो हस्ताभ्यां समतोलयत्} %6-102-38

\twolineshloka
{स नीलमिव जीमूतं तोयपूर्णं नभस्तलात्}
{उत्पपात गृहीत्वा तु हनूमान् शिखरं गिरेः} %6-102-39

\twolineshloka
{समागम्य महावेगः संन्यस्य शिखरं गिरेः}
{विश्रम्य किंचिद्धनुमान् सुषेणमिदमब्रवीत्} %6-102-40

\twolineshloka
{औषधीर्नावगच्छामि ता अहं हरिपुङ्गव}
{तदिदं शिखरं कृत्स्नं गिरेस्तस्याहृतं मया} %6-102-41

\twolineshloka
{एवं कथयमानं तु प्रशस्य पवनात्मजम्}
{सुषेणो वानरश्रेष्ठो जग्राहोत्पाट्य चौषधीः} %6-102-42

\twolineshloka
{विस्मितास्तु बभूवुस्ते सर्वे वानरपुङ्गवाः}
{दृष्ट्वा तु हनुमत्कर्म सुरैरपि सुदुष्करम्} %6-102-43

\twolineshloka
{ततः संक्षोदयित्वा तामोषधीं वानरोत्तमः}
{लक्ष्मणस्य ददौ नस्तः सुषेणः सुमहाद्युतिः} %6-102-44

\twolineshloka
{सशल्यः स समाघ्राय लक्ष्मणः परवीरहा}
{विशल्यो विरुजः शीघ्रमुदतिष्ठन्महीतलात्} %6-102-45

\twolineshloka
{तमुत्थितं तु हरयो भूतलात् प्रेक्ष्य लक्ष्मणम्}
{साधुसाध्विति सुप्रीता लक्ष्मणं प्रत्यपूजयन्} %6-102-46

\twolineshloka
{एह्येहीत्यब्रवीद् रामो लक्ष्मणं परवीरहा}
{सस्वजे गाढमालिङ्गय बाष्पपर्याकुलेक्षणः} %6-102-47

\twolineshloka
{अब्रवीच्च परिष्वज्य सौमित्रिं राघवस्तदा}
{दिष्ट्या त्वां वीर पश्यामि मरणात् पुनरागतम्} %6-102-48

\twolineshloka
{नहि मे जीवितेनार्थः सीतया च जयेन वा}
{को हि मे जीवितेनार्थस्त्वयि पञ्चत्वमागते} %6-102-49

\twolineshloka
{इत्येवं ब्रुवतस्तस्य राघवस्य महात्मनः}
{खिन्नः शिथिलया वाचा लक्ष्मणो वाक्यमब्रवीत्} %6-102-50

\twolineshloka
{तां प्रतिज्ञां प्रतिज्ञाय पुरा सत्यपराक्रम}
{लघुः कश्चिदिवासत्त्वो नैवं त्वं वक्तुमर्हसि} %6-102-51

\twolineshloka
{नहि प्रतिज्ञां कुर्वन्ति वितथां सत्यवादिनः}
{लक्षणं हि महत्त्वस्य प्रतिज्ञापरिपालनम्} %6-102-52

\twolineshloka
{नैराश्यमुपगन्तुं च नालं ते मत्कृतेऽनघ}
{वधेन रावणस्याद्य प्रतिज्ञामनुपालय} %6-102-53

\twolineshloka
{न जीवन् यास्यते शत्रुस्तव बाणपथं गतः}
{नर्दतस्तीक्ष्णदंष्ट्रस्य सिंहस्येव महागजः} %6-102-54

\twolineshloka
{अहं तु वधमच्छामि शीघ्रमस्य दुरात्मनः}
{यावदस्तं न यात्येष कृतकर्मा दिवाकरः} %6-102-55

\twolineshloka
{यदि वधमिच्छसि रावणस्य संख्ये यदि च कृतां हि तवेच्छसि प्रतिज्ञाम्}
{यदि तव राजसुताभिलाष आर्य कुरु च वचो मम शीघ्रमद्य वीर} %6-102-56


॥इत्यार्षे श्रीमद्रामायणे वाल्मीकीये आदिकाव्ये युद्धकाण्डे लक्ष्मणसंजीवनम् नाम द्व्यधिकशततमः सर्गः ॥६-१०२॥
