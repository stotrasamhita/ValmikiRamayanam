\sect{एकोनविंशः सर्गः — शरतल्पसंवेशः}

\twolineshloka
{राघवेणाभये दत्ते संनतो रावणानुजः}
{विभीषणो महाप्राज्ञो भूमिं समवलोकयत्} %6-19-1

\twolineshloka
{खात् पपातावनिं हृष्टो भक्तैरनुचरैः सह}
{स तु रामस्य धर्मात्मा निपपात विभीषणः} %6-19-2

\twolineshloka
{पादयोर्निपपाताथ चतुर्भिः सह राक्षसैः}
{अब्रवीच्च तदा वाक्यं रामं प्रति विभीषणः} %6-19-3

\twolineshloka
{धर्मयुक्तं च युक्तं च साम्प्रतं सम्प्रहर्षणम्}
{अनुजो रावणस्याहं तेन चास्म्यवमानितः} %6-19-4

\twolineshloka
{भवन्तं सर्वभूतानां शरण्यं शरणं गतः}
{परित्यक्ता मया लङ्का मित्राणि च धनानि च} %6-19-5

\twolineshloka
{भवद्गतं हि मे राज्यं जीवितं च सुखानि च}
{तस्य तद् वचनं श्रुत्वा रामो वचनमब्रवीत्} %6-19-6

\twolineshloka
{वचसा सान्त्वयित्वैनं लोचनाभ्यां पिबन्निव}
{आख्याहि मम तत्त्वेन राक्षसानां बलाबलम्} %6-19-7

\twolineshloka
{एवमुक्तं तदा रक्षो रामेणाक्लिष्टकर्मणा}
{रावणस्य बलं सर्वमाख्यातुमुपचक्रमे} %6-19-8

\twolineshloka
{अवध्यः सर्वभूतानां गन्धर्वोरगपक्षिणाम्}
{राजपुत्र दशग्रीवो वरदानात् स्वयम्भुवः} %6-19-9

\twolineshloka
{रावणानन्तरो भ्राता मम ज्येष्ठश्च वीर्यवान्}
{कुम्भकर्णो महातेजाः शक्रप्रतिबलो युधि} %6-19-10

\twolineshloka
{राम सेनापतिस्तस्य प्रहस्तो यदि ते श्रुतः}
{कैलासे येन समरे मणिभद्रः पराजितः} %6-19-11

\twolineshloka
{बद्धगोधाङ्गुलित्राणस्त्ववध्यकवचो युधि}
{धनुरादाय यस्तिष्ठन्नदृश्यो भवतीन्द्रजित्} %6-19-12

\twolineshloka
{संग्रामे सुमहद्व्यूहे तर्पयित्वा हुताशनम्}
{अन्तर्धानगतः श्रीमानिन्द्रजिद्धन्ति राघव} %6-19-13

\twolineshloka
{महोदरमहापार्श्वौ राक्षसश्चाप्यकम्पनः}
{अनीकपास्तु तस्यैते लोकपालसमा युधि} %6-19-14

\twolineshloka
{दशकोटिसहस्राणि रक्षसां कामरूपिणाम्}
{मांसशोणितभक्ष्याणां लङ्कापुरनिवासिनाम्} %6-19-15

\twolineshloka
{स तैस्तु सहितो राजा लोकपालानयोधयत्}
{सह देवैस्तु ते भग्ना रावणेन दुरात्मना} %6-19-16

\twolineshloka
{विभीषणस्य तु वचस्तच्छ्रुत्वा रघुसत्तमः}
{अन्वीक्ष्य मनसा सर्वमिदं वचनमब्रवीत्} %6-19-17

\twolineshloka
{यानि कर्मापदानानि रावणस्य विभीषण}
{आख्यातानि च तत्त्वेन ह्यवगच्छामि तान्यहम्} %6-19-18

\twolineshloka
{अहं हत्वा दशग्रीवं सप्रहस्तं सहात्मजम्}
{राजानं त्वां करिष्यामि सत्यमेतच्छृणोतु मे} %6-19-19

\twolineshloka
{रसातलं वा प्रविशेत् पातालं वापि रावणः}
{पितामहसकाशं वा न मे जीवन् विमोक्ष्यते} %6-19-20

\twolineshloka
{अहत्वा रावणं संख्ये सपुत्रजनबान्धवम्}
{अयोध्यां न प्रवेक्ष्यामि त्रिभिस्तैर्भ्रातृभिः शपे} %6-19-21

\twolineshloka
{श्रुत्वा तु वचनं तस्य रामस्याक्लिष्टकर्मणः}
{शिरसाऽऽवन्द्य धर्मात्मा वक्तुमेवं प्रचक्रमे} %6-19-22

\twolineshloka
{राक्षसानां वधे साह्यं लङ्कायाश्च प्रधर्षणे}
{करिष्यामि यथाप्राणं प्रवेक्ष्यामि च वाहिनीम्} %6-19-23

\twolineshloka
{इति ब्रुवाणं रामस्तु परिष्वज्य विभीषणम्}
{अब्रवील्लक्ष्मणं प्रीतः समुद्राज्जलमानय} %6-19-24

\twolineshloka
{तेन चेमं महाप्राज्ञमभिषिञ्च विभीषणम्}
{राजानं रक्षसां क्षिप्रं प्रसन्ने मयि मानद} %6-19-25

\twolineshloka
{एवमुक्तस्तु सौमित्रिरभ्यषिञ्चद् विभीषणम्}
{मध्ये वानरमुख्यानां राजानं राजशासनात्} %6-19-26

\twolineshloka
{तं प्रसादं तु रामस्य दृष्ट्वा सद्यः प्लवङ्गमाः}
{प्रचुक्रुशुर्महात्मानं साधुसाध्विति चाब्रुवन्} %6-19-27

\threelineshloka
{अब्रवीच्च हनूमांश्च सुग्रीवश्च विभीषणम्}
{कथं सागरमक्षोभ्यं तराम वरुणालयम्}
{सैन्यैः परिवृताः सर्वे वानराणां महौजसाम्} %6-19-28

\twolineshloka
{उपायैरभिगच्छाम यथा नदनदीपतिम्}
{तराम तरसा सर्वे ससैन्या वरुणालयम्} %6-19-29

\twolineshloka
{एवमुक्तस्तु धर्मात्मा प्रत्युवाच विभीषणः}
{समुद्रं राघवो राजा शरणं गन्तुमर्हति} %6-19-30

\twolineshloka
{खानितः सगरेणायमप्रमेयो महोदधिः}
{कर्तुमर्हति रामस्य ज्ञातेः कार्यं महोदधिः} %6-19-31

\twolineshloka
{एवं विभीषणेनोक्तो राक्षसेन विपश्चिता}
{आजगामाथ सुग्रीवो यत्र रामः सलक्ष्मणः} %6-19-32

\twolineshloka
{ततश्चाख्यातुमारेभे विभीषणवचः शुभम्}
{सुग्रीवो विपुलग्रीवः सागरस्योपवेशनम्} %6-19-33

\twolineshloka
{प्रकृत्या धर्मशीलस्य रामस्यास्याप्यरोचत}
{सलक्ष्मणं महातेजाः सुग्रीवं च हरीश्वरम्} %6-19-34

\twolineshloka
{सत्क्रियार्थं क्रियादक्षं स्मितपूर्वमभाषत}
{विभीषणस्य मन्त्रोऽयं मम लक्ष्मण रोचते} %6-19-35

\twolineshloka
{सुग्रीवः पण्डितो नित्यं भवान् मन्त्रविचक्षणः}
{उभाभ्यां सम्प्रधार्यार्थं रोचते यत् तदुच्यताम्} %6-19-36

\twolineshloka
{एवमुक्तौ ततो वीरावुभौ सुग्रीवलक्ष्मणौ}
{समुदाचारसंयुक्तमिदं वचनमूचतुः} %6-19-37

\twolineshloka
{किमर्थं नौ नरव्याघ्र न रोचिष्यति राघव}
{विभीषणेन यत् तूक्तमस्मिन् काले सुखावहम्} %6-19-38

\twolineshloka
{अबद्ध्वा सागरे सेतुं घोरेऽस्मिन् वरुणालये}
{लङ्का नासादितुं शक्या सेन्द्रैरपि सुरासुरैः} %6-19-39

\threelineshloka
{विभीषणस्य शूरस्य यथार्थं क्रियतां वचः}
{अलं कालात्ययं कृत्वा सागरोऽयं नियुज्यताम्}
{यथा सैन्येन गच्छाम पुरीं रावणपालिताम्} %6-19-40

\twolineshloka
{एवमुक्तः कुशास्तीर्णे तीरे नदनदीपतेः}
{संविवेश तदा रामो वेद्यामिव हुताशनः} %6-19-41


॥इत्यार्षे श्रीमद्रामायणे वाल्मीकीये आदिकाव्ये युद्धकाण्डे शरतल्पसंवेशः नाम एकोनविंशः सर्गः ॥६-१९॥
