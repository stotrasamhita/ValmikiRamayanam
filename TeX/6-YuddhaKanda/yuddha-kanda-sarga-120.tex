\sect{विंशत्यधिकशततमः सर्गः — ब्रह्मकृतरामस्तवः}

\twolineshloka
{ततो हि दुर्मना रामः श्रुत्वैवं वदतां गिरः}
{दध्यौ मुहूर्तं धर्मात्मा बाष्पव्याकुललोचनः} %6-120-1

\twolineshloka
{ततो वैश्रवणो राजा यमश्च पितृभिः सह}
{सहस्राक्षश्च देवेशो वरुणश्च जलेश्वरः} %6-120-2

\twolineshloka
{षडर्धनयनः श्रीमान् महादेवो वृषध्वजः}
{कर्ता सर्वस्य लोकस्य ब्रह्मा ब्रह्मविदां वरः} %6-120-3

\twolineshloka
{एते सर्वे समागम्य विमानैः सूर्यसंनिभैः}
{आगम्य नगरीं लङ्कामभिजग्मुश्च राघवम्} %6-120-4

\twolineshloka
{ततः सहस्ताभरणान् प्रगृह्य विपुलान् भुजान्}
{अब्रुवंस्त्रिदशश्रेष्ठा राघवं प्राञ्जलिं स्थितम्} %6-120-5

\threelineshloka
{कर्ता सर्वस्य लोकस्य श्रेष्ठो ज्ञानविदां विभुः}
{उपेक्षसे कथं सीतां पतन्तीं हव्यवाहने}
{कथं देवगणश्रेष्ठमात्मानं नावबुद्ध््यसे} %6-120-6

\twolineshloka
{ऋतधामा वसुः पूर्वं वसूनां च प्रजापतिः}
{त्रयाणामपि लोकानामादिकर्ता स्वयंप्रभुः} %6-120-7

\twolineshloka
{रुद्राणामष्टमो रुद्रः साध्यानामपि पञ्चमः}
{अश्विनौ चापि कर्णौ ते सूर्याचन्द्रमसौ दृशौ} %6-120-8

\twolineshloka
{अन्ते चादौ च मध्ये च दृश्यसे च परंतप}
{उपेक्षसे च वैदेहीं मानुषः प्राकृतो यथा} %6-120-9

\twolineshloka
{इत्युक्तो लोकपालैस्तैः स्वामी लोकस्य राघवः}
{अब्रवीत् त्रिदशश्रेष्ठान् रामो धर्मभृतां वरः} %6-120-10

\twolineshloka
{आत्मानं मानुषं मन्ये रामं दशरथात्मजम्}
{सोऽहं यश्च यतश्चाहं भगवांस्तद् ब्रवीतु मे} %6-120-11

\twolineshloka
{इति ब्रुवाणं काकुत्स्थं ब्रह्मा ब्रह्मविदां वरः}
{अब्रवीच्छृणु मे वाक्यं सत्यं सत्यपराक्रम} %6-120-12

\twolineshloka
{भवान् नारायणो देवः श्रीमांश्चक्रायुधः प्रभुः}
{एकशृङ्गो वराहस्त्वं भूतभव्यसपत्नजित्} %6-120-13

\twolineshloka
{अक्षरं ब्रह्म सत्यं च मध्ये चान्ते च राघव}
{लोकानां त्वं परो धर्मो विष्वक्सेनश्चतुर्भुजः} %6-120-14

\twolineshloka
{शार्ङ्गधन्वा हृषीकेशः पुरुषः पुरुषोत्तमः}
{अजितः खड्गधृग् विष्णुः कृष्णश्चैव बृहद्बलः} %6-120-15

\twolineshloka
{सेनानीर्ग्रामणीश्च त्वं बुद्धिः सत्त्वं क्षमा दमः}
{प्रभवश्चाप्ययश्च त्वमुपेन्द्रो मधुसूदनः} %6-120-16

\twolineshloka
{इन्द्रकर्मा महेन्द्रस्त्वं पद्मनाभो रणान्तकृत्}
{शरण्यं शरणं च त्वामाहुर्दिव्या महर्षयः} %6-120-17

\twolineshloka
{सहस्रशृङ्गो वेदात्मा शतशीर्षो महर्षभः}
{त्वं त्रयाणां हि लोकानामादिकर्ता स्वयंप्रभुः} %6-120-18

\twolineshloka
{सिद्धानामपि साध्यानामाश्रयश्चासि पूर्वजः}
{त्वं यज्ञस्त्वं वषट्कारस्त्वमोंकारः परात्परः} %6-120-19

\twolineshloka
{प्रभवं निधनं चापि नो विदुः को भवानिति}
{दृश्यसे सर्वभूतेषु गोषु च ब्राह्मणेषु च} %6-120-20

\twolineshloka
{दिक्षु सर्वासु गगने पर्वतेषु नदीषु च}
{सहस्रचरणः श्रीमान् शतशीर्षः सहस्रदृक्} %6-120-21

\twolineshloka
{त्वं धारयसि भूतानि पृथिवीं सर्वपर्वतान्}
{अन्ते पृथिव्याः सलिले दृश्यसे त्वं महोरगः} %6-120-22

\twolineshloka
{त्रीँल्लोकान् धारयन् राम देवगन्धर्वदानवान्}
{अहं ते हृदयं राम जिह्वा देवी सरस्वती} %6-120-23

\twolineshloka
{देवा रोमाणि गात्रेषु ब्रह्मणा निर्मिताः प्रभो}
{निमेषस्ते स्मृता रात्रिरुन्मेषो दिवसस्तथा} %6-120-24

\twolineshloka
{संस्कारास्त्वभवन् वेदा नैतदस्ति त्वया विना}
{जगत् सर्वं शरीरं ते स्थैर्यं ते वसुधातलम्} %6-120-25

\twolineshloka
{अग्निः कोपः प्रसादस्ते सोमः श्रीवत्सलक्षणः}
{त्वया लोकास्त्रयः क्रान्ताः पुरा स्वैर्विक्रमैस्त्रिभिः} %6-120-26

\twolineshloka
{महेन्द्रश्च कृतो राजा बलिं बद्ध्वा सुदारुणम्}
{सीता लक्ष्मीर्भवान् विष्णुर्देवः कृष्णः प्रजापतिः} %6-120-27

\twolineshloka
{वधार्थं रावणस्येह प्रविष्टो मानुषीं तनुम्}
{तदिदं नस्त्वया कार्यं कृतं धर्मभृतां वर} %6-120-28

\twolineshloka
{निहतो रावणो राम प्रहृष्टो दिवमाक्रम}
{अमोघं देव वीर्यं ते न ते मोघाः पराक्रमाः} %6-120-29

\twolineshloka
{अमोघं दर्शनं राम अमोघस्तव संस्तवः}
{अमोघास्ते भविष्यन्ति भक्तिमन्तो नरा भुवि} %6-120-30

\twolineshloka
{ये त्वां देवं ध्रुवं भक्ताः पुराणं पुरुषोत्तमम्}
{प्राप्नुवन्ति तथा कामानिह लोके परत्र च} %6-120-31

\twolineshloka
{इममार्षं स्तवं दिव्यमितिहासं पुरातनम्}
{ये नराः कीर्तयिष्यन्ति नास्ति तेषां पराभवः} %6-120-32


॥इत्यार्षे श्रीमद्रामायणे वाल्मीकीये आदिकाव्ये युद्धकाण्डे ब्रह्मकृतरामस्तवः नाम विंशत्यधिकशततमः सर्गः ॥६-१२०॥
