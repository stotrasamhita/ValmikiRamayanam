\sect{एकचत्वारिंशः सर्गः — अङ्गददूत्यम्}

\twolineshloka
{अथ तस्मिन् निमित्तानि दृष्ट्वा लक्ष्मणपूर्वजः}
{सुग्रीवं सम्परिष्वज्य रामो वचनमब्रवीत्} %6-41-1

\twolineshloka
{असम्मन्त्र्य मया सार्धं तदिदं साहसं कृतम्}
{एवं साहसयुक्तानि न कुर्वन्ति जनेश्वराः} %6-41-2

\twolineshloka
{संशये स्थाप्य मां चेदं बलं चेमं विभीषणम्}
{कष्टं कृतमिदं वीर साहसं साहसप्रिय} %6-41-3

\twolineshloka
{इदानीं मा कृथा वीर एवंविधमरिन्दम}
{त्वयि किञ्चित्समापन्ने किं कार्यं सीतया मम} %6-41-4

\twolineshloka
{भरतेन महाबाहो लक्ष्मणेन यवीयसा}
{शत्रुघ्नेन च शत्रुघ्न स्वशरीरेण वा पुनः} %6-41-5

\twolineshloka
{त्वयि चानागते पूर्वमिति मे निश्चिता मतिः}
{जानतश्चापि ते वीर्यं महेन्द्रवरुणोपम} %6-41-6

\twolineshloka
{हत्वाहं रावणं युद्धे सपुत्रबलवाहनम्}
{अभिषिच्य च लङ्कायां विभीषणमथापि च} %6-41-7

\twolineshloka
{भरते राज्यमारोप्य त्यक्ष्ये देहं महाबल}
{तमेवं वादिनं रामं सुग्रीवः प्रत्यभाषत} %6-41-8

\twolineshloka
{तव भार्यापहर्तारं दृष्ट्वा राघव रावणम्}
{मर्षयामि कथं वीर जानन् विक्रममात्मनः} %6-41-9

\twolineshloka
{इत्येवं वादिनं वीरमभिनन्द्य च राघवः}
{लक्ष्मणं लक्ष्मिसम्पन्नमिदं वचनमब्रवीत्} %6-41-10

\twolineshloka
{परिगृह्योदकं शीतं वनानि फलवन्ति च}
{बलौघं संविभज्येमं व्यूह्य तिष्ठाम लक्ष्मण} %6-41-11

\twolineshloka
{लोकक्षयकरं भीमं भयं पश्याम्युपस्थितम्}
{निबर्हणं प्रवीराणामृक्षवानररक्षसाम्} %6-41-12

\twolineshloka
{वाता हि परुषं वान्ति कम्पते च वसुन्धरा}
{पर्वताग्राणि वेपन्ते नदन्ति धरणीधराः} %6-41-13

\twolineshloka
{मेघाः क्रव्यादसङ्काशाः परुषाः परुषस्वराः}
{क्रूराः क्रूरं प्रवर्षन्ते मिश्रं शोणितबिन्दुभिः} %6-41-14

\twolineshloka
{रक्तचन्दनसङ्काशा सन्ध्या परमदारुणा}
{ज्वलच्च निपतत्येतदादित्यादग्निमण्डलम्} %6-41-15

\twolineshloka
{आदित्यमभिवाश्यन्ति जनयन्तो महद्भयम्}
{दीना दीनस्वरा घोरा अप्रशस्ता मृगद्विजाः} %6-41-16

\twolineshloka
{रजन्यामप्रकाशश्च सन्तापयति चन्द्रमाः}
{कृष्णरक्तांशुपर्यन्तो यथा लोकस्य सङ्क्षये} %6-41-17

\twolineshloka
{ह्रस्वो रूक्षोऽप्रशस्तश्च परिवेषः सुलोहितः}
{आदित्यमण्डले नीलं लक्ष्म लक्ष्मण दृश्यते} %6-41-18

\twolineshloka
{दृश्यन्ते न यथावच्च नक्षत्राण्यभिवर्तते}
{युगान्तमिव लोकस्य पश्य लक्ष्मण शंसति} %6-41-19

\twolineshloka
{काकाः श्येनास्तथा गृध्रा नीचैः परिपतन्ति च}
{शिवाश्चाप्यशुभा वाचः प्रवदन्ति महास्वनाः} %6-41-20

\twolineshloka
{शैलैः शूलैश्च खड्गैश्च विमुक्तैः कपिराक्षसैः}
{भविष्यत्यावृता भूमिर्मांसशोणितकर्दमा} %6-41-21

\twolineshloka
{क्षिप्रमद्य दुराधर्षां पुरीं रावणपालिताम्}
{अभियाम जवेनैव सर्वतो हरिभिर्वृताः} %6-41-22

\twolineshloka
{इत्येवं तु वदन् वीरो लक्ष्मणं लक्ष्मणाग्रजः}
{तस्मादवातरच्छीघ्रं पर्वताग्रान्महाबलः} %6-41-23

\twolineshloka
{अवतीर्य तु धर्मात्मा तस्माच्छैलात् स राघवः}
{परैः परमदुर्धर्षं ददर्श बलमात्मनः} %6-41-24

\twolineshloka
{सन्नह्य तु ससुग्रीवः कपिराजबलं महत्}
{कालज्ञो राघवः काले संयुगायाभ्यचोदयत्} %6-41-25

\twolineshloka
{ततः काले महाबाहुर्बलेन महता वृतः}
{प्रस्थितः पुरतो धन्वी लङ्कामभिमुखः पुरीम्} %6-41-26

\twolineshloka
{तं विभीषणसुग्रीवौ हनूमाञ्जाम्बवान् नलः}
{ऋक्षराजस्तथा नीलो लक्ष्मणश्चान्वयुस्तदा} %6-41-27

\twolineshloka
{ततः पश्चात् सुमहती पृतनर्क्षवनौकसाम्}
{प्रच्छाद्य महतीं भूमिमनुयाति स्म राघवम्} %6-41-28

\twolineshloka
{शैलशृङ्गाणि शतशः प्रवृद्धांश्च महीरुहान्}
{जगृहुः कुञ्जरप्रख्या वानराः परवारणाः} %6-41-29

\twolineshloka
{तौ त्वदीर्घेण कालेन भ्रातरौ रामलक्ष्मणौ}
{रावणस्य पुरीं लङ्कामासेदतुररिन्दमौ} %6-41-30

\twolineshloka
{पताकामालिनीं रम्यामुद्यानवनशोभिताम्}
{चित्रवप्रां सुदुष्प्रापामुच्चैः प्राकारतोरणाम्} %6-41-31

\twolineshloka
{तां सुरैरपि दुर्धर्षां रामवाक्यप्रचोदिताः}
{यथानिदेशं सम्पीड्य न्यविशन्त वनौकसः} %6-41-32

\twolineshloka
{लङ्कायास्तूत्तरद्वारं शैलशृङ्गमिवोन्नतम्}
{रामः सहानुजो धन्वी जुगोप च रुरोध च} %6-41-33

\twolineshloka
{लङ्कामुपनिविष्टस्तु रामो दशरथात्मजः}
{लक्ष्मणानुचरो वीरः पुरीं रावणपालिताम्} %6-41-34

\twolineshloka
{उत्तरद्वारमासाद्य यत्र तिष्ठति रावणः}
{नान्यो रामाद्धि तद् द्वारं समर्थः परिरक्षितुम्} %6-41-35

\twolineshloka
{रावणाधिष्ठितं भीमं वरुणेनेव सागरम्}
{सायुधै राक्षसैर्भीमैरभिगुप्तं समन्ततः} %6-41-36

\twolineshloka
{लघूनां त्रासजननं पातालमिव दानवैः}
{विन्यस्तानि च योधानां बहूनि विविधानि च} %6-41-37

\twolineshloka
{ददर्शायुधजालानि तथैव कवचानि च}
{पूर्वं तु द्वारमासाद्य नीलो हरिचमूपतिः} %6-41-38

\twolineshloka
{अतिष्ठत् सह मैन्देन द्विविदेन च वीर्यवान्}
{अङ्गदो दक्षिणद्वारं जग्राह सुमहाबलः} %6-41-39

\twolineshloka
{ऋषभेण गवाक्षेण गजेन गवयेन च}
{हनूमान् पश्चिमद्वारं ररक्ष बलवान् कपिः} %6-41-40

\twolineshloka
{प्रमाथिप्रघसाभ्यां च वीरैरन्यैश्च सङ्गतः}
{मध्यमे च स्वयं गुल्मे सुग्रीवः समतिष्ठत} %6-41-41

\twolineshloka
{सह सर्वैर्हरिश्रेष्ठैः सुपर्णपवनोपमैः}
{वानराणां तु षट्त्रिंशत्कोट्यः प्रख्यातयूथपाः} %6-41-42

\twolineshloka
{निपीड्योपनिविष्टाश्च सुग्रीवो यत्र वानरः}
{शासनेन तु रामस्य लक्ष्मणः सविभीषणः} %6-41-43

\twolineshloka
{द्वारे द्वारे हरीणां तु कोटिं कोटीर्न्यवेशयत्}
{पश्चिमेन तु रामस्य सुषेणः सहजाम्बवान्} %6-41-44

\threelineshloka
{अदूरान्मध्यमे गुल्मे तस्थौ बहुबलानुगः}
{ते तु वानरशार्दूलाः शार्दूला इव दंष्ट्रिणः}
{गृहीत्वा द्रुमशैलाग्रान् हृष्टा युद्धाय तस्थिरे} %6-41-45

\twolineshloka
{सर्वे विकृतलाङ्गूलाः सर्वे दंष्ट्रानखायुधाः}
{सर्वे विकृतचित्राङ्गाः सर्वे च विकृताननाः} %6-41-46

\twolineshloka
{दशनागबलाः केचित् केचिद् दशगुणोत्तराः}
{केचिन्नागसहस्रस्य बभूवुस्तुल्यविक्रमाः} %6-41-47

\twolineshloka
{सन्ति चौघबलाः केचित् केचिच्छतगुणोत्तराः}
{अप्रमेयबलाश्चान्ये तत्रासन् हरियूथपाः} %6-41-48

\twolineshloka
{अद्भुतश्च विचित्रश्च तेषामासीत् समागमः}
{तत्र वानरसैन्यानां शलभानामिवोद्गमः} %6-41-49

\twolineshloka
{परिपूर्णमिवाकाशं सम्पूर्णेव च मेदिनी}
{लङ्कामुपनिविष्टैश्च सम्पतद्भिश्च वानरैः} %6-41-50

\twolineshloka
{शतं शतसहस्राणां पृतनर्क्षवनौकसाम्}
{लङ्काद्वाराण्युपाजग्मुरन्ये योद्धुं समन्ततः} %6-41-51

\twolineshloka
{आवृतः स गिरिः सर्वैस्तैः समन्तात् प्लवङ्गमैः}
{अयुतानां सहस्रं च पुरीं तामभ्यवर्तत} %6-41-52

\twolineshloka
{वानरैर्बलवद्भिश्च बभूव द्रुमपाणिभिः}
{सर्वतः संवृता लङ्का दुष्प्रवेशापि वायुना} %6-41-53

\twolineshloka
{राक्षसा विस्मयं जग्मुः सहसाभिनिपीडिताः}
{वानरैर्मेघसङ्काशैः शक्रतुल्यपराक्रमैः} %6-41-54

\twolineshloka
{महाञ्छब्दोऽभवत् तत्र बलौघस्याभिवर्ततः}
{सागरस्येव भिन्नस्य यथा स्यात् सलिलस्वनः} %6-41-55

\twolineshloka
{तेन शब्देन महता सप्राकारा सतोरणा}
{लङ्का प्रचलिता सर्वा सशैलवनकानना} %6-41-56

\twolineshloka
{रामलक्ष्मणगुप्ता सा सुग्रीवेण च वाहिनी}
{बभूव दुर्धर्षतरा सर्वैरपि सुरासुरैः} %6-41-57

\twolineshloka
{राघवः सन्निवेश्यैवं स्वसैन्यं रक्षसां वधे}
{सम्मन्त्र्य मन्त्रिभिः सार्धं निश्चित्य च पुनः पुनः} %6-41-58

\twolineshloka
{आनन्तर्यमभिप्रेप्सुः क्रमयोगार्थतत्त्ववित्}
{विभीषणस्यानुमते राजधर्ममनुस्मरन्} %6-41-59

\twolineshloka
{अङ्गदं वालितनयं समाहूयेदमब्रवीत्}
{गत्वा सौम्य दशग्रीवं ब्रूहि मद्वचनात् कपे} %6-41-60

\twolineshloka
{लङ्घयित्वा पुरीं लङ्कां भयं त्यक्त्वा गतव्यथः}
{भ्रष्टश्रीकं गतैश्वर्यं मुमूर्षानष्टचेतनम्} %6-41-61

\twolineshloka
{ऋषीणां देवतानां च गन्धर्वाप्सरसां तथा}
{नागानामथ यक्षाणां राज्ञां च रजनीचर} %6-41-62

\threelineshloka
{यच्च पापं कृतं मोहादवलिप्तेन राक्षस}
{नूनं ते विगतो दर्पः स्वयम्भूवरदानजः}
{तस्य पापस्य सम्प्राप्ता व्युष्टिरद्य दुरासदा} %6-41-63

\twolineshloka
{यस्य दण्डधरस्तेऽहं दाराहरणकर्शितः}
{दण्डं धारयमाणस्तु लङ्काद्वारे व्यवस्थितः} %6-41-64

\twolineshloka
{पदवीं देवतानां च महर्षीणां च राक्षस}
{राजर्षीणां च सर्वेषां गमिष्यसि युधि स्थिरः} %6-41-65

\twolineshloka
{बलेन येन वै सीतां मायया राक्षसाधम}
{मामतिक्रमयित्वा त्वं हृतवांस्तन्निदर्शय} %6-41-66

\twolineshloka
{अराक्षसमिमं लोकं कर्तास्मि निशितैः शरैः}
{न चेच्छरणमभ्येषि तामादाय तु मैथिलीम्} %6-41-67

\twolineshloka
{धर्मात्मा राक्षसश्रेष्ठः सम्प्राप्तोऽयं विभीषणः}
{लङ्कैश्वर्यमिदं श्रीमान् ध्रुवं प्राप्नोत्यकण्टकम्} %6-41-68

\twolineshloka
{नहि राज्यमधर्मेण भोक्तुं क्षणमपि त्वया}
{शक्यं मूर्खसहायेन पापेनाविदितात्मना} %6-41-69

\twolineshloka
{युध्यस्व मा धृतिं कृत्वा शौर्यमालम्ब्य राक्षस}
{मच्छरैस्त्वं रणे शान्तस्ततः पूतो भविष्यसि} %6-41-70

\twolineshloka
{यद्याविशसि लोकांस्त्रीन् पक्षीभूतो निशाचर}
{मम चक्षुःपथं प्राप्य न जीवन् प्रतियास्यसि} %6-41-71

\twolineshloka
{ब्रवीमि त्वां हितं वाक्यं क्रियतामौर्ध्वदेहिकम्}
{सुदृष्टा क्रियतां लङ्का जीवितं ते मयि स्थितम्} %6-41-72

\twolineshloka
{इत्युक्तः स तु तारेयो रामेणाक्लिष्टकर्मणा}
{जगामाकाशमाविश्य मूर्तिमानिव हव्यवाट्} %6-41-73

\twolineshloka
{सोऽतिपत्य मुहूर्तेन श्रीमान् रावणमन्दिरम्}
{ददर्शासीनमव्यग्रं रावणं सचिवैः सह} %6-41-74

\twolineshloka
{ततस्तस्याविदूरेण निपत्य हरिपुङ्गवः}
{दीप्ताग्निसदृशस्तस्थावङ्गदः कनकाङ्गदः} %6-41-75

\twolineshloka
{तद् रामवचनं सर्वमन्यूनाधिकमुत्तमम्}
{सामात्यं श्रावयामास निवेद्यात्मानमात्मना} %6-41-76

\twolineshloka
{दूतोऽहं कोसलेन्द्रस्य रामस्याक्लिष्टकर्मणः}
{वालिपुत्रोऽङ्गदो नाम यदि ते श्रोत्रमागतः} %6-41-77

\twolineshloka
{आह त्वां राघवो रामः कौसल्यानन्दवर्धनः}
{निष्पत्य प्रतियुध्यस्व नृशंस पुरुषो भव} %6-41-78

\twolineshloka
{हन्तास्मि त्वां सहामात्यं सपुत्रज्ञातिबान्धवम्}
{निरुद्विग्नास्त्रयो लोका भविष्यन्ति हते त्वयि} %6-41-79

\twolineshloka
{देवदानवयक्षाणां गन्धर्वोरगरक्षसाम्}
{शत्रुमद्योद्धरिष्यामि त्वामृषीणां च कण्टकम्} %6-41-80

\twolineshloka
{विभीषणस्य चैश्वर्यं भविष्यति हते त्वयि}
{न चेत् सत्कृत्य वैदेहीं प्रणिपत्य प्रदास्यसि} %6-41-81

\twolineshloka
{इत्येवं परुषं वाक्यं ब्रुवाणे हरिपुङ्गवे}
{अमर्षवशमापन्नो निशाचरगणेश्वरः} %6-41-82

\twolineshloka
{ततः स रोषमापन्नः शशास सचिवांस्तदा}
{गृह्यतामिति दुर्मेधा वध्यतामिति चासकृत्} %6-41-83

\twolineshloka
{रावणस्य वचः श्रुत्वा दीप्ताग्निमिव तेजसा}
{जगृहुस्तं ततो घोराश्चत्वारो रजनीचराः} %6-41-84

\twolineshloka
{ग्राहयामास तारेयः स्वयमात्मानमात्मवान्}
{बलं दर्शयितुं वीरो यातुधानगणे तदा} %6-41-85

\twolineshloka
{स तान् बाहुद्वयासक्तानादाय पतगानिव}
{प्रासादं शैलसङ्काशमुत्पपाताङ्गदस्तदा} %6-41-86

\twolineshloka
{तस्योत्पतनवेगेन निर्धूतास्तत्र राक्षसाः}
{भूमौ निपतिताः सर्वे राक्षसेन्द्रस्य पश्यतः} %6-41-87

\twolineshloka
{ततः प्रासादशिखरं शैलशृङ्गमिवोन्नतम्}
{चक्राम राक्षसेन्द्रस्य वालिपुत्रः प्रतापवान्} %6-41-88

\twolineshloka
{पफाल च तदाक्रान्तं दशग्रीवस्य पश्यतः}
{पुरा हिमवतः शृङ्गं वज्रेणेव विदारितम्} %6-41-89

\twolineshloka
{भङ्क्त्वा प्रासादशिखरं नाम विश्राव्य चात्मनः}
{विनद्य सुमहानादमुत्पपात विहायसा} %6-41-90

\twolineshloka
{व्यथयन् राक्षसान् सर्वान् हर्षयंश्चापि वानरान्}
{स वानराणां मध्ये तु रामपार्श्वमुपागतः} %6-41-91

\twolineshloka
{रावणस्तु परं चक्रे क्रोधं प्रासादधर्षणात्}
{विनाशं चात्मनः पश्यन् निःश्वासपरमोऽभवत्} %6-41-92

\twolineshloka
{रामस्तु बहुभिर्हृष्टैर्विनदद्भिः प्लवङ्गमैः}
{वृतो रिपुवधाकाङ्क्षी युद्धायैवाभ्यवर्तत} %6-41-93

\twolineshloka
{सुषेणस्तु महावीर्यो गिरिकूटोपमो हरिः}
{बहुभिः संवृतस्तत्र वानरैः कामरूपिभिः} %6-41-94

\twolineshloka
{स तु द्वाराणि संयम्य सुग्रीववचनात् कपिः}
{पर्यक्रामत दुर्धर्षो नक्षत्राणीव चन्द्रमाः} %6-41-95

\twolineshloka
{तेषामक्षौहिणिशतं समवेक्ष्य वनौकसाम्}
{लङ्कामुपनिविष्टानां सागरं चाभिवर्तताम्} %6-41-96

\twolineshloka
{राक्षसा विस्मयं जग्मुस्त्रासं जग्मुस्तथापरे}
{अपरे समरे हर्षाद्धर्षमेवोपपेदिरे} %6-41-97

\threelineshloka
{कृत्स्नं हि कपिभिर्व्याप्तं प्राकारपरिखान्तरम्}
{ददृशू राक्षसा दीनाः प्राकारं वानरीकृतम्}
{हाहाकारमकुर्वन्त राक्षसा भयमागताः} %6-41-98

\twolineshloka
{तस्मिन् महाभीषणके प्रवृत्ते कोलाहले राक्षसराजयोधाः}
{प्रगृह्य रक्षांसि महायुधानि युगान्तवाता इव संविचेरुः} %6-41-99


॥इत्यार्षे श्रीमद्रामायणे वाल्मीकीये आदिकाव्ये युद्धकाण्डे अङ्गददूत्यम् नाम एकचत्वारिंशः सर्गः ॥६-४१॥
