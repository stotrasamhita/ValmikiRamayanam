\sect{एकोनविंशत्यधिकशततमः सर्गः — हुताशनप्रवेशः}

\twolineshloka
{एवमुक्ता तु वैदेही परुषं रोमहर्षणम्}
{राघवेण सरोषेण श्रुत्वा प्रव्यथिताभवत्} %6-119-1

\twolineshloka
{सा तदाश्रुतपूर्वं हि जने महति मैथिली}
{श्रुत्वा भर्तुर्वचो घोरं लज्जयावनताभवत्} %6-119-2

\twolineshloka
{प्रविशन्तीव गात्राणि स्वानि सा जनकात्मजा}
{वाक्शरैस्तैः सशल्येव भृशमश्रूण्यवर्तयत्} %6-119-3

\twolineshloka
{ततो बाष्पपरिक्लिन्नं प्रमार्जन्ती स्वमाननम्}
{शनैर्गद्गदया वाचा भर्तारमिदमब्रवीत्} %6-119-4

\twolineshloka
{किं मामसदृशं वाक्यमीदृशं श्रोत्रदारुणम्}
{रूक्षं श्रावयसे वीर प्राकृतः प्राकृतामिव} %6-119-5

\twolineshloka
{न तथास्मि महाबाहो यथा मामवगच्छसि}
{प्रत्ययं गच्छ मे स्वेन चारित्रेणैव ते शपे} %6-119-6

\twolineshloka
{पृथक्स्त्रीणां प्रचारेण जातिं त्वं परिशङ्कसे}
{परित्यजैनां शङ्कां तु यदि तेऽहं परीक्षिता} %6-119-7

\twolineshloka
{यदहं गात्रसंस्पर्शं गतास्मि विवशा प्रभो}
{कामकारो न मे तत्र दैवं तत्रापराध्यति} %6-119-8

\twolineshloka
{मदधीनं तु यत् तन्मे हृदयं त्वयि वर्तते}
{पराधीनेषु गात्रेषु किं करिष्याम्यनीश्वरी} %6-119-9

\twolineshloka
{सह संवृद्धभावेन संसर्गेण च मानद}
{यदि तेऽहं न विज्ञाता हता तेनास्मि शाश्वतम्} %6-119-10

\twolineshloka
{प्रेषितस्ते महावीरो हनुमानवलोककः}
{लङ्कास्थाहं त्वया राजन् किं तदा न विसर्जिता} %6-119-11

\twolineshloka
{प्रत्यक्षं वानरस्यास्य तद्वाक्यसमनन्तरम्}
{त्वया संत्यक्तया वीर त्यक्तं स्याज्जीवितं मया} %6-119-12

\twolineshloka
{न वृथा ते श्रमोऽयं स्यात् संशये न्यस्य जीवितम्}
{सुहृज्जनपरिक्लेशो न चायं विफलस्तव} %6-119-13

\twolineshloka
{त्वया तु नृपशार्दूल रोषमेवानुवर्तता}
{लघुनेव मनुष्येण स्त्रीत्वमेव पुरस्कृतम्} %6-119-14

\twolineshloka
{अपदेशेन जनकान्नोत्पत्तिर्वसुधातलात्}
{मम वृत्तं च वृत्तज्ञ बहु ते न पुरस्कृतम्} %6-119-15

\twolineshloka
{न प्रमाणीकृतः पाणिर्बाल्ये मम निपीडितः}
{मम भक्तिश्च शीलं च सर्वं ते पृष्ठतः कृतम्} %6-119-16

\twolineshloka
{इति ब्रुवन्ती रुदती बाष्पगद्गदभाषिणी}
{उवाच लक्ष्मणं सीता दीनं ध्यानपरायणम्} %6-119-17

\twolineshloka
{चितां मे कुरु सौमित्रे व्यसनस्यास्य भेषजम्}
{मिथ्यापवादोपहता नाहं जीवितुमुत्सहे} %6-119-18

\twolineshloka
{अप्रीतेन गुणैर्भर्त्रा त्यक्ताया जनसंसदि}
{या क्षमा मे गतिर्गन्तुं प्रवेक्ष्ये हव्यवाहनम्} %6-119-19

\twolineshloka
{एवमुक्तस्तु वैदेह्या लक्ष्मणः परवीरहा}
{अमर्षवशमापन्नो राघवं समुदैक्षत} %6-119-20

\twolineshloka
{स विज्ञाय मनश्छन्दं रामस्याकारसूचितम्}
{चितां चकार सौमित्रिर्मते रामस्य वीर्यवान्} %6-119-21

\twolineshloka
{नहि रामं तदा कश्चित् कालान्तकयमोपमम्}
{अनुनेतुमथो वक्तुं द्रष्टुं वाप्यशकत् सुहृत्} %6-119-22

\twolineshloka
{अधोमुखं स्थितं रामं ततः कृत्वा प्रदक्षिणम्}
{उपावर्तत वैदेही दीप्यमानं हुताशनम्} %6-119-23

\twolineshloka
{प्रणम्य दैवतेभ्यश्च ब्राह्मणेभ्यश्च मैथिली}
{बद्धाञ्जलिपुटा चेदमुवाचाग्निसमीपतः} %6-119-24

\twolineshloka
{यथा मे हृदयं नित्यं नापसर्पति राघवात्}
{तथा लोकस्य साक्षी मां सर्वतः पातु पावकः} %6-119-25

\twolineshloka
{यथा मां शुद्धचारित्रां दुष्टां जानाति राघवः}
{तथा लोकस्य साक्षी मां सर्वतः पातु पावकः} %6-119-26

\twolineshloka
{कर्मणा मनसा वाचा यथा नातिचराम्यहम्}
{राघवं सर्वधर्मज्ञं तथा मां पातु पावकः} %6-119-27

\threelineshloka
{आदित्यो भगवान् वायुर्दिशश्चन्द्रस्तथैव च}
{अहश्चापि तथा संध्ये रात्रिश्च पृथिवी तथा}
{यथान्येऽपि विजानन्ति तथा चारित्रसंयुताम्} %6-119-28

\twolineshloka
{एवमुक्त्वा तु वैदेही परिक्रम्य हुताशनम्}
{विवेश ज्वलनं दीप्तं निःशङ्केनान्तरात्मना} %6-119-29

\twolineshloka
{जनश्च सुमहांस्तत्र बालवृद्धसमाकुलः}
{ददर्श मैथिलीं दीप्तां प्रविशन्तीं हुताशनम्} %6-119-30

\twolineshloka
{सा तप्तनवहेमाभा तप्तकाञ्चनभूषणा}
{पपात ज्वलनं दीप्तं सर्वलोकस्य संनिधौ} %6-119-31

\twolineshloka
{ददृशुस्तां विशालाक्षीं पतन्तीं हव्यवाहनम्}
{सीतां सर्वाणि रूपाणि रुक्मवेदिनिभां तदा} %6-119-32

\twolineshloka
{ददृशुस्तां महाभागां प्रविशन्तीं हुताशनम्}
{ऋषयो देवगन्धर्वा यज्ञे पूर्णाहुतीमिव} %6-119-33

\twolineshloka
{प्रचुक्रुशुः स्त्रियः सर्वास्तां दृष्ट्वा हव्यवाहने}
{पतन्तीं संस्कृतां मन्त्रैर्वसोर्धारामिवाध्वरे} %6-119-34

\twolineshloka
{ददृशुस्तां त्रयो लोका देवगन्धर्वदानवाः}
{शप्तां पतन्तीं निरये त्रिदिवाद् देवतामिव} %6-119-35

\twolineshloka
{तस्यामग्निं विशन्त्यां तु हाहेति विपुलः स्वनः}
{रक्षसां वानराणां च सम्बभूवाद्भुतोपमः} %6-119-36


॥इत्यार्षे श्रीमद्रामायणे वाल्मीकीये आदिकाव्ये युद्धकाण्डे हुताशनप्रवेशः नाम एकोनविंशत्यधिकशततमः सर्गः ॥६-११९॥
