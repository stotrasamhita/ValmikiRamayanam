\sect{चतुर्नवतितमः सर्गः — गन्धर्वास्त्रमोहनम्}

\twolineshloka
{स प्रविश्य सभां राजा दीनः परमदुःखितः}
{निषसादासने मुख्ये सिंहः क्रुद्ध इव श्वसन्} %6-94-1

\twolineshloka
{अब्रवीच्च स तान् सर्वान् बलमुख्यान् महाबलः}
{रावणः प्राञ्जलिर्वाक्यं पुत्रव्यसनकर्शितः} %6-94-2

\twolineshloka
{सर्वे भवन्तः सर्वेण हस्त्यश्वेन समावृताः}
{निर्यान्तु रथसङ्घैश्च पादातैश्चोपशोभिताः} %6-94-3

\twolineshloka
{एकं रामं परिक्षिप्य समरे हन्तुमर्हथ}
{वर्षन्तः शरवर्षाणि प्रावृट्काल इवाम्बुदाः} %6-94-4

\twolineshloka
{अथवाहं शरैस्तीक्ष्णैर्भिन्नगात्रं महाहवे}
{भवद्भिः श्वो निहन्तास्मि रामं लोकस्य पश्यतः} %6-94-5

\twolineshloka
{इत्येतद् वाक्यमादाय राक्षसेन्द्रस्य राक्षसाः}
{निर्ययुस्ते रथैः शीघ्रैर्नानानीकैश्च संयुताः} %6-94-6

\twolineshloka
{परिघान् पट्टिशांश्चैव शरखड्गपरश्वधान्}
{शरीरान्तकरान् सर्वे चिक्षिपुर्वानरान् प्रति} %6-94-7

\twolineshloka
{वानराश्च द्रुमान् शैलान् राक्षसान् प्रति चिक्षिपुः}
{स संग्रामो महाभीमः सूर्यस्योदयनं प्रति} %6-94-8

\twolineshloka
{रक्षसां वानराणां च तुमुलः समपद्यत}
{ते गदाभिश्च चित्राभिः प्रासैः खड्गैः परश्वधैः} %6-94-9

\twolineshloka
{अन्योन्यं समरे जघ्नुस्तदा वानरराक्षसाः}
{एवं प्रवृत्ते संग्रामे ह्यद्भुतं सुमहद्रजः} %6-94-10

\twolineshloka
{रक्षसां वानराणां च शान्तं शोणितविस्रवैः}
{मातंगरथकूलाश्च शरमत्स्या ध्वजद्रुमाः} %6-94-11

\twolineshloka
{शरीरसंघाटवहाः प्रसस्रुः शोणितापगाः}
{ततस्ते वानराः सर्वे शोणितौघपरिप्लुताः} %6-94-12

\twolineshloka
{ध्वजवर्मरथानश्वान् नानाप्रहरणानि च}
{आप्लुत्याप्लुत्य समरे वानरेन्द्रा बभञ्जिरे} %6-94-13

\twolineshloka
{केशान् कर्णललाटं च नासिकाश्च प्लवंगमाः}
{रक्षसां दशनैस्तीक्ष्णैर्नखैश्चापि व्यकर्तयन्} %6-94-14

\twolineshloka
{एकैकं राक्षसं संख्ये शतं वानरपुंगवाः}
{अभ्यधावन्त फलिनं वृक्षं शकुनयो यथा} %6-94-15

\twolineshloka
{तदा गदाभिर्गुर्वीभिः प्रासैः खड्गैः परश्वधैः}
{निर्जघ्नुर्वानरान् घोरान् राक्षसाः पर्वतोपमाः} %6-94-16

\twolineshloka
{राक्षसैर्वध्यमानानां वानराणां महाचमूः}
{शरण्यं शरणं याता रामं दशरथात्मजम्} %6-94-17

\twolineshloka
{ततो रामो महातेजा धनुरादाय वीर्यवान्}
{प्रविश्य राक्षसं सैन्यं शरवर्षं ववर्ष च} %6-94-18

\twolineshloka
{प्रविष्टं तु तदा रामं मेघाः सूर्यमिवाम्बरे}
{नाधिजग्मुर्महाघोरा निर्दहन्तं शराग्निना} %6-94-19

\twolineshloka
{कृतान्येव सुघोराणि रामेण रजनीचराः}
{रणे रामस्य ददृशुः कर्माण्यसुकराणि ते} %6-94-20

\twolineshloka
{चालयन्तं महासैन्यं विधमन्तं महारथान्}
{ददृशुस्ते न वै रामं वातं वनगतं यथा} %6-94-21

\twolineshloka
{छिन्नं भिन्नं शरैर्दग्धं प्रभग्नं शस्त्रपीडितम्}
{बलं रामेण ददृशुर्न रामं शीघ्रकारिणम्} %6-94-22

\twolineshloka
{प्रहरन्तं शरीरेषु न ते पश्यन्ति राघवम्}
{इन्द्रियार्थेषु तिष्ठन्तं भूतात्मानमिव प्रजाः} %6-94-23

\twolineshloka
{एष हन्ति गजानीकमेष हन्ति महारथान्}
{एष हन्ति शरैस्तीक्ष्णैः पदातीन् वाजिभिः सह} %6-94-24

\twolineshloka
{इति ते राक्षसाः सर्वे रामस्य सदृशान् रणे}
{अन्योन्यं कुपिता जघ्नुः सादृश्याद् राघवस्य तु} %6-94-25

\twolineshloka
{न ते ददृशिरे रामं दहन्तमपि वाहिनीम्}
{मोहिताः परमास्त्रेण गान्धर्वेण महात्मना} %6-94-26

\twolineshloka
{ते तु रामसहस्राणि रणे पश्यन्ति राक्षसाः}
{पुनः पश्यन्ति काकुत्स्थमेकमेव महाहवे} %6-94-27

\twolineshloka
{भ्रमन्तीं काञ्चनीं कोटिं कार्मुकस्य महात्मनः}
{अलातचक्रप्रतिमां ददृशुस्ते न राघवम्} %6-94-28

\twolineshloka
{शरीरनाभि सत्त्वार्चिः शरारं नेमिकार्मुकम्}
{ज्याघोषतलनिर्घोषं तेजोबुद्धिगुणप्रभम्} %6-94-29

\twolineshloka
{दिव्यास्त्रगुणपर्यन्तं निघ्नन्तं युधि राक्षसान्}
{ददृशू रामचक्रं तत् कालचक्रमिव प्रजाः} %6-94-30

\twolineshloka
{अनीकं दशसाहस्रं रथानां वातरंहसाम्}
{अष्टादश सहस्राणि कुञ्जराणां तरस्विनाम्} %6-94-31

\twolineshloka
{चतुर्दश सहस्राणि सारोहाणां च वाजिनाम्}
{पूर्णे शतसहस्रे द्वे राक्षसानां पदातिनाम्} %6-94-32

\twolineshloka
{दिवसस्याष्टभागेन शरैरग्निशिखोपमैः}
{हतान्येकेन रामेण रक्षसां कामरूपिणाम्} %6-94-33

\twolineshloka
{ते हताश्वा हतरथाः शान्ता विमथितध्वजाः}
{अभिपेतुः पुरीं लङ्कां हतशेषा निशाचराः} %6-94-34

\twolineshloka
{हतैर्गजपदात्यश्वैस्तद् बभूव रणाजिरम्}
{आक्रीडभूमिः क्रुद्धस्य रुद्रस्येव महात्मनः} %6-94-35

\twolineshloka
{ततो देवाः सगन्धर्वाः सिद्धाश्च परमर्षयः}
{साधु साध्विति रामस्य तत् कर्म समपूजयन्} %6-94-36

\twolineshloka
{अब्रवीच्च तदा रामः सुग्रीवं प्रत्यनन्तरम्}
{विभीषणं च धर्मात्मा हनूमन्तं च वानरम्} %6-94-37

\twolineshloka
{जाम्बवन्तं हरिश्रेष्ठं मैन्दं द्विविदमेव च}
{एतदस्त्रबलं दिव्यं मम वा त्र्यम्बकस्य वा} %6-94-38

\twolineshloka
{निहत्य तां राक्षसराजवाहिनीं रामस्तदा शक्रसमो महात्मा}
{अस्त्रेषु शस्त्रेषु जितक्लमश्च संस्तूयते देवगणैः प्रहृष्टैः} %6-94-39


॥इत्यार्षे श्रीमद्रामायणे वाल्मीकीये आदिकाव्ये युद्धकाण्डे गन्धर्वास्त्रमोहनम् नाम चतुर्नवतितमः सर्गः ॥६-९४॥
