\sect{अष्टादशः सर्गः — विभीषणसंग्रहनिर्णयः}

\twolineshloka
{अथ रामः प्रसन्नात्मा श्रुत्वा वायुसुतस्य ह}
{प्रत्यभाषत दुर्धर्षः श्रुतवानात्मनि स्थितम्} %6-18-1

\twolineshloka
{ममापि च विवक्षास्ति काचित् प्रति विभीषणम्}
{श्रोतुमिच्छामि तत् सर्वं भवद्भिः श्रेयसि स्थितैः} %6-18-2

\twolineshloka
{मित्रभावेन सम्प्राप्तं न त्यजेयं कथंचन}
{दोषो यद्यपि तस्य स्यात् सतामेतदगर्हितम्} %6-18-3

\twolineshloka
{सुग्रीवस्त्वथ तद्वाक्यमाभाष्य च विमृश्य च}
{ततः शुभतरं वाक्यमुवाच हरिपुङ्गवः} %6-18-4

\twolineshloka
{स दुष्टो वाप्यदुष्टो वा किमेष रजनीचरः}
{ईदृशं व्यसनं प्राप्तं भ्रातरं यः परित्यजेत्} %6-18-5

\twolineshloka
{को नाम स भवेत् तस्य यमेष न परित्यजेत्}
{वानराधिपतेर्वाक्यं श्रुत्वा सर्वानुदीक्ष्य तु} %6-18-6

\twolineshloka
{ईषदुत्स्मयमानस्तु लक्ष्मणं पुण्यलक्षणम्}
{इति होवाच काकुत्स्थो वाक्यं सत्यपराक्रमः} %6-18-7

\twolineshloka
{अनधीत्य च शास्त्राणि वृद्धाननुपसेव्य च}
{न शक्यमीदृशं वक्तुं यदुवाच हरीश्वरः} %6-18-8

\twolineshloka
{अस्ति सूक्ष्मतरं किंचिद् यथात्र प्रतिभाति मा}
{प्रत्यक्षं लौकिकं चापि वर्तते सर्वराजसु} %6-18-9

\twolineshloka
{अमित्रास्तत्कुलीनाश्च प्रातिदेश्याश्च कीर्तिताः}
{व्यसनेषु प्रहर्तारस्तस्मादयमिहागतः} %6-18-10

\twolineshloka
{अपापास्तत्कुलीनाश्च मानयन्ति स्वकान् हितान्}
{एष प्रायो नरेन्द्राणां शङ्कनीयस्तु शोभनः} %6-18-11

\twolineshloka
{यस्तु दोषस्त्वया प्रोक्तो ह्यादानेऽरिबलस्य च}
{तत्र ते कीर्तयिष्यामि यथाशास्त्रमिदं शृणु} %6-18-12

\twolineshloka
{न वयं तत्कुलीनाश्च राज्यकाङ्क्षी च राक्षसः}
{पण्डिता हि भविष्यन्ति तस्माद् ग्राह्यो विभीषणः} %6-18-13

\threelineshloka
{अव्यग्राश्च प्रहृष्टाश्च ते भविष्यन्ति संगताः}
{प्रणादश्च महानेषोऽन्योन्यस्य भयमागतम्}
{इति भेदं गमिष्यन्ति तस्माद् ग्राह्यो विभीषणः} %6-18-14

\twolineshloka
{न सर्वे भ्रातरस्तात भवन्ति भरतोपमाः}
{मद्विधा वा पितुः पुत्राः सुहृदो वा भवद्विधाः} %6-18-15

\twolineshloka
{एवमुक्तस्तु रामेण सुग्रीवः सहलक्ष्मणः}
{उत्थायेदं महाप्राज्ञः प्रणतो वाक्यमब्रवीत्} %6-18-16

\twolineshloka
{रावणेन प्रणिहितं तमवेहि निशाचरम्}
{तस्याहं निग्रहं मन्ये क्षमं क्षमवतां वर} %6-18-17

\twolineshloka
{राक्षसो जिह्मया बुद्ध्या संदिष्टोऽयमिहागतः}
{प्रहर्तुं त्वयि विश्वस्ते विश्वस्ते मयि वानघ} %6-18-18

\twolineshloka
{लक्ष्मणे वा महाबाहो स वध्यः सचिवैः सह}
{रावणस्य नृशंसस्य भ्राता ह्येष विभीषणः} %6-18-19

\twolineshloka
{एवमुक्त्वा रघुश्रेष्ठं सुग्रीवो वाहिनीपतिः}
{वाक्यज्ञो वाक्यकुशलं ततो मौनमुपागमत्} %6-18-20

\twolineshloka
{स सुग्रीवस्य तद् वाक्यं रामः श्रुत्वा विमृश्य च}
{ततः शुभतरं वाक्यमुवाच हरिपुङ्गवम्} %6-18-21

\twolineshloka
{स दुष्टो वाप्यदुष्टो वा किमेष रजनीचरः}
{सूक्ष्ममप्यहितं कर्तुं मम शक्तः कथंचन} %6-18-22

\twolineshloka
{पिशाचान् दानवान् यक्षान् पृथिव्यां चैव राक्षसान्}
{अङ्गुल्यग्रेण तान् हन्यामिच्छन् हरिगणेश्वर} %6-18-23

\twolineshloka
{श्रूयते हि कपोतेन शत्रुः शरणमागतः}
{अर्चितश्च यथान्यायं स्वैश्च मांसैर्निमन्त्रितः} %6-18-24

\twolineshloka
{स हि तं प्रतिजग्राह भार्याहर्तारमागतम्}
{कपोतो वानरश्रेष्ठ किं पुनर्मद्विधो जनः} %6-18-25

\twolineshloka
{ऋषेः कण्वस्य पुत्रोण कण्डुना परमर्षिणा}
{शृणु गाथा पुरा गीता धर्मिष्ठा सत्यवादिना} %6-18-26

\twolineshloka
{बद्धाञ्जलिपुटं दीनं याचन्तं शरणागतम्}
{न हन्यादानृशंस्यार्थमपि शत्रुं परंतप} %6-18-27

\twolineshloka
{आर्तो वा यदि वा दृप्तः परेषां शरणं गतः}
{अरिः प्राणान् परित्यज्य रक्षितव्यः कृतात्मना} %6-18-28

\twolineshloka
{स चेद् भयाद् वा मोहाद् वा कामाद् वापि न रक्षति}
{स्वया शक्त्या यथान्यायं तत् पापं लोकगर्हितम्} %6-18-29

\twolineshloka
{विनष्टः पश्यतस्तस्य रक्षिणः शरणं गतः}
{आनाय सुकृतं तस्य सर्वं गच्छेदरक्षितः} %6-18-30

\twolineshloka
{एवं दोषो महानत्र प्रपन्नानामरक्षणे}
{अस्वर्ग्यं चायशस्यं च बलवीर्यविनाशनम्} %6-18-31

\twolineshloka
{करिष्यामि यथार्थं तु कण्डोर्वचनमुत्तमम्}
{धर्मिष्ठं च यशस्यं च स्वर्ग्यं स्यात् तु फलोदये} %6-18-32

\twolineshloka
{सकृदेव प्रपन्नाय तवास्मीति च याचते}
{अभयं सर्वभूतेभ्यो ददाम्येतद् व्रतं मम} %6-18-33

\twolineshloka
{आनयैनं हरिश्रेष्ठ दत्तमस्याभयं मया}
{विभीषणो वा सुग्रीव यदि वा रावणः स्वयम्} %6-18-34

\twolineshloka
{रामस्य तु वचः श्रुत्वा सुग्रीवः प्लवगेश्वरः}
{प्रत्यभाषत काकुत्स्थं सौहार्देनाभिपूरितः} %6-18-35

\twolineshloka
{किमत्र चित्रं धर्मज्ञ लोकनाथशिखामणे}
{यत् त्वमार्यं प्रभाषेथाः सत्त्ववान् सत्पथे स्थितः} %6-18-36

\twolineshloka
{मम चाप्यन्तरात्मायं शुद्धं वेत्ति विभीषणम्}
{अनुमानाच्च भावाच्च सर्वतः सुपरीक्षितः} %6-18-37

\twolineshloka
{तस्मात् क्षिप्रं सहास्माभिस्तुल्यो भवतु राघव}
{विभीषणो महाप्राज्ञः सखित्वं चाभ्युपैतु नः} %6-18-38

\twolineshloka
{ततस्तु सुग्रीववचो निशम्य तद्धरीश्वरेणाभिहितं नरेश्वरः}
{विभीषणेनाशु जगाम संगमं पतत्त्रिराजेन यथा पुरंदरः} %6-18-39


॥इत्यार्षे श्रीमद्रामायणे वाल्मीकीये आदिकाव्ये युद्धकाण्डे विभीषणसंग्रहनिर्णयः नाम अष्टादशः सर्गः ॥६-१८॥
