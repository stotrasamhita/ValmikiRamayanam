\sect{सप्तचत्वारिंशः सर्गः — नागबद्धरामलक्ष्मणप्रदर्शनम्}

\twolineshloka
{तस्मिन् प्रविष्टे लङ्कायां कृतार्थे रावणात्मजे}
{राघवं परिवार्याथ ररक्षुर्वानरर्षभाः} %6-47-1

\twolineshloka
{हनुमानङ्गदो नीलः सुषेणः कुमुदो नलः}
{गजो गवाक्षो गवयः शरभो गन्धमादनः} %6-47-2

\twolineshloka
{जाम्बवानृषभः स्कन्धो रम्भः शतबलिः पृथुः}
{व्यूढानीकाश्च यत्ताश्च द्रुमानादाय सर्वतः} %6-47-3

\twolineshloka
{वीक्षमाणा दिशः सर्वास्तिर्यगूर्ध्वं च वानराः}
{तृणेष्वपि च चेष्टत्सु राक्षसा इति मेनिरे} %6-47-4

\twolineshloka
{रावणश्चापि संहृष्टो विसृज्येन्द्रजितं सुतम्}
{आजुहाव ततः सीतारक्षणी राक्षसीस्तदा} %6-47-5

\twolineshloka
{राक्षस्यस्त्रिजटा चापि शासनात् तमुपस्थिताः}
{ता उवाच ततो हृष्टो राक्षसी राक्षसाधिपः} %6-47-6

\twolineshloka
{हताविन्द्रजिताख्यात वैदेह्या रामलक्ष्मणौ}
{पुष्पकं तत्समारोप्य दर्शयध्वं रणे हतौ} %6-47-7

\twolineshloka
{यदाश्रयादवष्टब्धा नेयं मामुपतिष्ठते}
{सोऽस्या भर्ता सह भ्रात्रा निहतो रणमूर्धनि} %6-47-8

\twolineshloka
{निर्विशङ्का निरुद्विग्ना निरपेक्षा च मैथिली}
{मामुपस्थास्यते सीता सर्वाभरणभूषिता} %6-47-9

\twolineshloka
{अद्य कालवशं प्राप्तं रणे रामं सलक्ष्मणम्}
{अवेक्ष्य विनिवृत्ता सा चान्यां गतिमपश्यती} %6-47-10

\twolineshloka
{अनपेक्षा विशालाक्षी मामुपस्थास्यते स्वयम्}
{तस्य तद् वचनं श्रुत्वा रावणस्य दुरात्मनः} %6-47-11

\twolineshloka
{राक्षस्यस्तास्तथेत्युक्त्वा जग्मुर्वै यत्र पुष्पकम्}
{ततः पुष्पकमादाय राक्षस्यो रावणाज्ञया} %6-47-12

\twolineshloka
{अशोकवनिकास्थां तां मैथिलीं समुपानयन्}
{तामादाय तु राक्षस्यो भर्तृशोकपराजिताम्} %6-47-13

\twolineshloka
{सीतामारोपयामासुर्विमानं पुष्पकं तदा}
{ततः पुष्पकमारोप्य सीतां त्रिजटया सह} %6-47-14

\twolineshloka
{जग्मुर्दर्शयितुं तस्यै राक्षस्यो रामलक्ष्मणौ}
{रावणश्चारयामास पताकाध्वजमालिनीम्} %6-47-15

\twolineshloka
{प्राघोषयत हृष्टश्च लङ्कायां राक्षसेश्वरः}
{राघवो लक्ष्मणश्चैव हताविन्द्रजिता रणे} %6-47-16

\twolineshloka
{विमानेनापि गत्वा तु सीता त्रिजटया सह}
{ददर्श वानराणां तु सर्वं सैन्यं निपातितम्} %6-47-17

\twolineshloka
{प्रहृष्टमनसश्चापि ददर्श पिशिताशनान्}
{वानरांश्चातिदुःखार्तान् रामलक्ष्मणपार्श्वतः} %6-47-18

\twolineshloka
{ततः सीता ददर्शोभौ शयानौ शरतल्पगौ}
{लक्ष्मणं चैव रामं च विसंज्ञौ शरपीडितौ} %6-47-19

\twolineshloka
{विध्वस्तकवचौ वीरौ विप्रविद्धशरासनौ}
{सायकैश्छिन्नसर्वाङ्गौ शरस्तम्बमयौ क्षितौ} %6-47-20

\twolineshloka
{तौ दृष्ट्वा भ्रातरौ तत्र प्रवीरौ पुरुषर्षभौ}
{शयानौ पुण्डरीकाक्षौ कुमाराविव पावकी} %6-47-21

\twolineshloka
{शरतल्पगतौ वीरौ तथाभूतौ नरर्षभौ}
{दुःखार्ता करुणं सीता सुभृशं विललाप ह} %6-47-22

\twolineshloka
{भर्तारमनवद्याङ्गी लक्ष्मणं चासितेक्षणा}
{प्रेक्ष्य पांसुषु चेष्टन्तौ रुरोद जनकात्मजा} %6-47-23

\twolineshloka
{सबाष्पशोकाभिहता समीक्ष्य तौ भ्रातरौ देवसुतप्रभावौ}
{वितर्कयन्ती निधनं तयोः सा दुःखान्विता वाक्यमिदं जगाद} %6-47-24


॥इत्यार्षे श्रीमद्रामायणे वाल्मीकीये आदिकाव्ये युद्धकाण्डे नागबद्धरामलक्ष्मणप्रदर्शनम् नाम सप्तचत्वारिंशः सर्गः ॥६-४७॥
