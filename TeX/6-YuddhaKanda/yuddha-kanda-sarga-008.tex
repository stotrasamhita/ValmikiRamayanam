\sect{अष्टमः सर्गः — प्रहस्तादिवचनम्}

\twolineshloka
{ततो नीलाम्बुदप्रख्यः प्रहस्तो नाम राक्षसः}
{अब्रवीत् प्राञ्जलिर्वाक्यं शूरः सेनापतिस्तदा} %6-8-1

\twolineshloka
{देवदानवगन्धर्वाः पिशाचपतगोरगाः}
{सर्वे धर्षयितुं शक्याः किं पुनर्मानवौ रणे} %6-8-2

\twolineshloka
{सर्वे प्रमत्ता विश्वस्ता वञ्चिताः स्म हनूमता}
{नहि मे जीवतो गच्छेज्जीवन् स वनगोचरः} %6-8-3

\twolineshloka
{सर्वां सागरपर्यन्तां सशैलवनकाननाम्}
{करोम्यवानरां भूमिमाज्ञापयतु मां भवान्} %6-8-4

\twolineshloka
{रक्षां चैव विधास्यामि वानराद् रजनीचर}
{नागमिष्यति ते दुःखं किञ्चिदात्मापराधजम्} %6-8-5

\twolineshloka
{अब्रवीत् तु सुसङ्क्रुद्धो दुर्मुखो नाम राक्षसः}
{इदं न क्षमणीयं हि सर्वेषां नः प्रधर्षणम्} %6-8-6

\twolineshloka
{अयं परिभवो भूयः पुरस्यान्तःपुरस्य च}
{श्रीमतो राक्षसेन्द्रस्य वानरेण प्रधर्षणम्} %6-8-7

\twolineshloka
{अस्मिन् मुहूर्ते गत्वैको निवर्तिष्यामि वानरान्}
{प्रविष्टान् सागरं भीममम्बरं वा रसातलम्} %6-8-8

\twolineshloka
{ततोऽब्रवीत् सुसङ्क्रुद्धो वज्रदंष्ट्रो महाबलः}
{प्रगृह्य परिघं घोरं मांसशोणितरूषितम्} %6-8-9

\twolineshloka
{किं नो हनूमता कार्यं कृपणेन तपस्विना}
{रामे तिष्ठति दुर्धर्षे सुग्रीवेऽपि सलक्ष्मणे} %6-8-10

\twolineshloka
{अद्य रामं ससुग्रीवं परिघेण सलक्ष्मणम्}
{आगमिष्यामि हत्वैको विक्षोभ्य हरिवाहिनीम्} %6-8-11

\twolineshloka
{इदं ममापरं वाक्यं शृणु राजन् यदिच्छसि}
{उपायकुशलो ह्येव जयेच्छत्रूनतन्द्रितः} %6-8-12

\twolineshloka
{कामरूपधराः शूराः सुभीमा भीमदर्शनाः}
{राक्षसा वा सहस्राणि राक्षसाधिप निश्चिताः} %6-8-13

\twolineshloka
{काकुत्स्थमुपसङ्गम्य बिभ्रतो मानुषं वपुः}
{सर्वे ह्यसम्भ्रमा भूत्वा ब्रुवन्तु रघुसत्तमम्} %6-8-14

\twolineshloka
{प्रेषिता भरतेनैव भ्रात्रा तव यवीयसा}
{स हि सेनां समुत्थाप्य क्षिप्रमेवोपयास्यति} %6-8-15

\twolineshloka
{ततो वयमितस्तूर्णं शूलशक्तिगदाधराः}
{चापबाणासिहस्ताश्च त्वरितास्तत्र यामहे} %6-8-16

\twolineshloka
{आकाशे गणशः स्थित्वा हत्वा तां हरिवाहिनीम्}
{अश्मशस्त्रमहावृष्ट्या प्रापयाम यमक्षयम्} %6-8-17

\twolineshloka
{एवं चेदुपसर्पेतामनयं रामलक्ष्मणौ}
{अवश्यमपनीतेन जहतामेव जीवितम्} %6-8-18

\twolineshloka
{कौम्भकर्णिस्ततो वीरो निकुम्भो नाम वीर्यवान्}
{अब्रवीत् परमक्रुद्धो रावणं लोकरावणम्} %6-8-19

\twolineshloka
{सर्वे भवन्तस्तिष्ठन्तु महाराजेन सङ्गताः}
{अहमेको हनिष्यामि राघवं सहलक्ष्मणम्} %6-8-20

\twolineshloka
{सुग्रीवं सहनूमन्तं सर्वांश्चैवात्र वानरान्}
{ततो वज्रहनुर्नाम राक्षसः पर्वतोपमः} %6-8-21

\twolineshloka
{क्रुद्धः परिलिहन् सृक्कां जिह्वया वाक्यमब्रवीत्}
{स्वैरं कुर्वन्तु कार्याणि भवन्तो विगतज्वराः} %6-8-22

\twolineshloka
{एकोऽहं भक्षयिष्यामि तां सर्वां हरिवाहिनीम्}
{स्वस्थाः क्रीडन्तु निश्चिन्ताः पिबन्तु मधु वारुणीम्} %6-8-23

\twolineshloka
{अहमेको वधिष्यामि सुग्रीवं सहलक्ष्मणम्}
{साङ्गदं च हनूमन्तं सर्वांश्चैवात्र वानरान्} %6-8-24


॥इत्यार्षे श्रीमद्रामायणे वाल्मीकीये आदिकाव्ये युद्धकाण्डे प्रहस्तादिवचनम् नाम अष्टमः सर्गः ॥६-८॥
