\sect{अष्टपञ्चाशः सर्गः — प्रहस्तवधः}

\twolineshloka
{ततः प्रहस्तं निर्यान्तं दृष्ट्वा रणकृतोद्यमम्}
{उवाच सस्मितं रामो विभीषणमरिन्दमः} %6-58-1

\twolineshloka
{क एष सुमहाकायो बलेन महता वृतः}
{आगच्छति महावेगः किंरूपबलपौरुषः} %6-58-2

\twolineshloka
{आचक्ष्व मे महाबाहो वीर्यवन्तं निशाचरम्}
{राघवस्य वचः श्रुत्वा प्रत्युवाच विभीषणः} %6-58-3

\threelineshloka
{एष सेनापतिस्तस्य प्रहस्तो नाम राक्षसः}
{लङ्कायां राक्षसेन्द्रस्य त्रिभागबलसंवृतः}
{वीर्यवानस्त्रविच्छूरः सुप्रख्यातपराक्रमः} %6-58-4

\twolineshloka
{ततः प्रहस्तं निर्यान्तं भीमं भीमपराक्रमम्}
{गर्जन्तं सुमहाकायं राक्षसैरभिसंवृतम्} %6-58-5

\twolineshloka
{ददर्श महती सेना वानराणां बलीयसाम्}
{अभिसञ्जातघोषाणां प्रहस्तमभिगर्जताम्} %6-58-6

\twolineshloka
{खड्गशक्त्यृष्टिशूलाश्च बाणानि मुसलानि च}
{गदाश्च परिघाः प्रासा विविधाश्च परश्वधाः} %6-58-7

\twolineshloka
{धनूंषि च विचित्राणि राक्षसानां जयैषिणाम्}
{प्रगृहीतान्यराजन्त वानरानभिधावताम्} %6-58-8

\twolineshloka
{जगृहुः पादपांश्चापि पुष्पितांस्तु गिरींस्तथा}
{शिलाश्च विपुला दीर्घा योद्धुकामाः प्लवङ्गमाः} %6-58-9

\twolineshloka
{तेषामन्योन्यमासाद्य सङ्ग्रामः सुमहानभूत्}
{बहूनामश्मवृष्टिं च शरवर्षं च वर्षताम्} %6-58-10

\twolineshloka
{बहवो राक्षसा युद्धे बहून् वानरपुङ्गवान्}
{वानरा राक्षसांश्चापि निजघ्नुर्बहवो बहून्} %6-58-11

\twolineshloka
{शूलैः प्रमथिताः केचित् केचित् तु परमायुधैः}
{परिघैराहताः केचित् केचिच्छिन्नाः परश्वधैः} %6-58-12

\twolineshloka
{निरुच्छ्वासाः पुनः केचित् पतिता जगतीतले}
{विभिन्नहृदयाः केचिदिषुसन्धानसाधिताः} %6-58-13

\twolineshloka
{केचिद् द्विधा कृताः खड्गैः स्फुरन्तः पतिता भुवि}
{वानरा राक्षसैः शूरैः पार्श्वतश्च विदारिताः} %6-58-14

\twolineshloka
{वानरैश्चापि सङ्क्रुद्धै राक्षसौघाः समन्ततः}
{पादपैर्गिरिशृङ्गैश्च सम्पिष्टा वसुधातले} %6-58-15

\twolineshloka
{वज्रस्पर्शतलैर्हस्तैर्मुष्टिभिश्च हता भृशम्}
{वमन् शोणितमास्येभ्यो विशीर्णदशनेक्षणाः} %6-58-16

\twolineshloka
{आर्तस्वनं च स्वनतां सिंहनादं च नर्दताम्}
{बभूव तुमुलः शब्दो हरीणां रक्षसामपि} %6-58-17

\twolineshloka
{वानरा राक्षसाः क्रुद्धा वीरमार्गमनुव्रताः}
{विवृत्तवदनाः क्रूराश्चक्रुः कर्माण्यभीतवत्} %6-58-18

\twolineshloka
{नरान्तकः कुम्भहनुर्महानादः समुन्नतः}
{एते प्रहस्तसचिवाः सर्वे जघ्नुर्वनौकसः} %6-58-19

\twolineshloka
{तेषां निपततां शीघ्रं निघ्नतां चापि वानरान्}
{द्विविदो गिरिशृङ्गेण जघानैकं नरान्तकम्} %6-58-20

\twolineshloka
{दुर्मुखः पुनरुत्थाय कपिः सविपुलद्रुमम्}
{राक्षसं क्षिप्रहस्तं तु समुन्नतमपोथयत्} %6-58-21

\twolineshloka
{जाम्बवांस्तु सुसङ्क्रुद्धः प्रगृह्य महतीं शिलाम्}
{पातयामास तेजस्वी महानादस्य वक्षसि} %6-58-22

\twolineshloka
{अथ कुम्भहनुस्तत्र तारेणासाद्य वीर्यवान्}
{वृक्षेण महता सद्यः प्राणान् सन्त्याजयद् रणे} %6-58-23

\twolineshloka
{अमृष्यमाणस्तत्कर्म प्रहस्तो रथमास्थितः}
{चकार कदनं घोरं धनुष्पाणिर्वनौकसाम्} %6-58-24

\twolineshloka
{आवर्त इव सञ्जज्ञे सेनयोरुभयोस्तदा}
{क्षुभितस्याप्रमेयस्य सागरस्येव निःस्वनः} %6-58-25

\twolineshloka
{महता हि शरौघेण राक्षसो रणदुर्मदः}
{अर्दयामास सङ्क्रुद्धो वानरान् परमाहवे} %6-58-26

\twolineshloka
{वानराणां शरीरैस्तु राक्षसानां च मेदिनी}
{बभूवातिचिता घोरैः पर्वतैरिव संवृता} %6-58-27

\twolineshloka
{सा मही रुधिरौघेण प्रच्छन्ना सम्प्रकाशते}
{सञ्छन्ना माधवे मासि पलाशैरिव पुष्पितैः} %6-58-28

\twolineshloka
{हतवीरौघवप्रां तु भग्नायुधमहाद्रुमाम्}
{शोणितौघमहातोयां यमसागरगामिनीम्} %6-58-29

\twolineshloka
{यकृत् प्लीहमहापङ्कां विनिकीर्णान्त्रशैवलाम्}
{भिन्नकायशिरोमीनामङ्गावयवशाद्वलाम्} %6-58-30

\twolineshloka
{गृध्रहंसवराकीर्णां कङ्कसारससेविताम्}
{मेदःफेनसमाकीर्णामार्तस्तनितनिःस्वनाम्} %6-58-31

\twolineshloka
{तां कापुरुषदुस्तारां युद्धभूमिमयीं नदीम्}
{नदीमिव घनापाये हंससारससेविताम्} %6-58-32

\twolineshloka
{राक्षसाः कपिमुख्यास्ते तेरुस्तां दुस्तरां नदीम्}
{यथा पद्मरजोध्वस्तां नलिनीं गजयूथपाः} %6-58-33

\twolineshloka
{ततः सृजन्तं बाणौघान् प्रहस्तं स्यन्दने स्थितम्}
{ददर्श तरसा नीलो विधमन्तं प्लवङ्गमान्} %6-58-34

\twolineshloka
{उद्धूत इव वायुः खे महदभ्रबलं बलात्}
{समीक्ष्याभिद्रुतं युद्धे प्रहस्तो वाहिनीपतिः} %6-58-35

\twolineshloka
{रथेनादित्यवर्णेन नीलमेवाभिदुद्रुवे}
{स धनुर्धन्विनां श्रेष्ठो विकृष्य परमाहवे} %6-58-36

\twolineshloka
{नीलाय व्यसृजद् बाणान् प्रहस्तो वाहिनीपतिः}
{ते प्राप्य विशिखा नीलं विनिर्भिद्य समाहिताः} %6-58-37

\twolineshloka
{महीं जग्मुर्महावेगा रोषिता इव पन्नगाः}
{नीलः शरैरभिहतो निशितैर्ज्वलनोपमैः} %6-58-38

\twolineshloka
{स तं परमदुर्धर्षमापतन्तं महाकपिः}
{प्रहस्तं ताडयामास वृक्षमुत्पाट्य वीर्यवान्} %6-58-39

\twolineshloka
{स तेनाभिहतः क्रुद्धो नर्दन् राक्षसपुङ्गवः}
{ववर्ष शरवर्षाणि प्लवङ्गानां चमूपतौ} %6-58-40

\threelineshloka
{तस्य बाणगणानेव राक्षसस्य दुरात्मनः}
{अपारयन् वारयितुं प्रत्यगृह्णान्निमीलितः}
{यथैव गोवृषो वर्षं शारदं शीघ्रमागतम्} %6-58-41

\twolineshloka
{एवमेव प्रहस्तस्य शरवर्षान् दुरासदान्}
{निमीलिताक्षः सहसा नीलः सेहे दुरासदान्} %6-58-42

\twolineshloka
{रोषितः शरवर्षेण सालेन महता महान्}
{प्रजघान हयान् नीलः प्रहस्तस्य महाबलः} %6-58-43

\twolineshloka
{ततो रोषपरीतात्मा धनुस्तस्य दुरात्मनः}
{बभञ्ज तरसा नीलो ननाद च पुनः पुनः} %6-58-44

\twolineshloka
{विधनुः स कृतस्तेन प्रहस्तो वाहिनीपतिः}
{प्रगृह्य मुसलं घोरं स्यन्दनादवपुप्लुवे} %6-58-45

\twolineshloka
{तावुभौ वाहिनीमुख्यौ जातवैरौ तरस्विनौ}
{स्थितौ क्षतजसिक्ताङ्गौ प्रभिन्नाविव कुञ्जरौ} %6-58-46

\twolineshloka
{उल्लिखन्तौ सुतीक्ष्णाभिर्दंष्ट्राभिरितरेतरम्}
{सिंहशार्दूलसदृशौ सिंहशार्दूलचेष्टितौ} %6-58-47

\twolineshloka
{विक्रान्तविजयौ वीरौ समरेष्वनिवर्तिनौ}
{काङ्क्षमाणौ यशः प्राप्तुं वृत्रवासवयोरिव} %6-58-48

\twolineshloka
{आजघान तदा नीलं ललाटे मुसलेन सः}
{प्रहस्तः परमायत्तस्ततः सुस्राव शोणितम्} %6-58-49

\twolineshloka
{ततः शोणितदिग्धाङ्गः प्रगृह्य च महातरुम्}
{प्रहस्तस्योरसि क्रुद्धो विससर्ज महाकपिः} %6-58-50

\twolineshloka
{तमचिन्त्यप्रहारं स प्रगृह्य मुसलं महत्}
{अभिदुद्राव बलिनं बलान्नीलं प्लवङ्गमम्} %6-58-51

\twolineshloka
{तमुग्रवेगं संरब्धमापतन्तं महाकपिः}
{ततः सम्प्रेक्ष्य जग्राह महावेगो महाशिलाम्} %6-58-52

\twolineshloka
{तस्य युद्धाभिकामस्य मृधे मुसलयोधिनः}
{प्रहस्तस्य शिलां नीलो मूर्ध्नि तूर्णमपातयत्} %6-58-53

\twolineshloka
{नीलेन कपिमुख्येन विमुक्ता महती शिला}
{बिभेद बहुधा घोरा प्रहस्तस्य शिरस्तदा} %6-58-54

\twolineshloka
{स गतासुर्गतश्रीको गतसत्त्वो गतेन्द्रियः}
{पपात सहसा भूमौ छिन्नमूल इव द्रुमः} %6-58-55

\twolineshloka
{विभिन्नशिरसस्तस्य बहु सुस्राव शोणितम्}
{शरीरादपि सुस्राव गिरेः प्रस्रवणं यथा} %6-58-56

\twolineshloka
{हते प्रहस्ते नीलेन तदकम्प्यं महाबलम्}
{राक्षसानामहृष्टानां लङ्कामभिजगाम ह} %6-58-57

\twolineshloka
{न शेकुः समवस्थातुं निहते वाहिनीपतौ}
{सेतुबन्धं समासाद्य विशीर्णं सलिलं यथा} %6-58-58

\twolineshloka
{हते तस्मिंश्चमूमुख्ये राक्षसास्ते निरुद्यमाः}
{रक्षःपतिगृहं गत्वा ध्यानमूकत्वमागताः} %6-58-59

\onelineshloka
{प्राप्ताः शोकार्णवं तीव्रं विसंज्ञा इव तेऽभवन्} %6-58-60

\twolineshloka
{ततस्तु नीलो विजयी महाबलः प्रशस्यमानः सुकृतेन कर्मणा}
{समेत्य रामेण सलक्ष्मणेन प्रहृष्टरूपस्तु बभूव यूथपः} %6-58-61


॥इत्यार्षे श्रीमद्रामायणे वाल्मीकीये आदिकाव्ये युद्धकाण्डे प्रहस्तवधः नाम अष्टपञ्चाशः सर्गः ॥६-५८॥
