\sect{त्रिंशदधिकशततमः सर्गः — भरतसमागमः}

\twolineshloka
{श्रुत्वा तु परमानन्दं भरतः सत्यविक्रमः}
{हृष्टमाज्ञापयामास शत्रुघ्नं परवीरहा} %6-130-1

\twolineshloka
{दैवतानि च सर्वाणि चैत्यानि नगरस्य च}
{सुगन्धमाल्यैर्वादित्रैरर्चन्तु शुचयो नराः} %6-130-2

\twolineshloka
{सूताः स्तुतिपुराणज्ञाः सर्वे वैतालिकास्तथा}
{सर्वे वादित्रकुशला गणिकाश्चैव सर्वशः} %6-130-3

\twolineshloka
{राजदारास्तथामात्याः सैन्याः सेनाङ्गनागणाः}
{ब्राह्मणाश्च सराजन्याः श्रेणीमुख्यास्तथा गणाः} %6-130-4

\twolineshloka
{अभिनिर्यान्तु रामस्य द्रष्टुं शशिनिभं मुखम्}
{भरतस्य वचः श्रुत्वा शत्रुघ्नः परवीरहा} %6-130-5

\twolineshloka
{विष्टीरनेकसाहस्रीश्चोदयामास भागशः}
{समीकुरुत निम्नानि विषमाणि समानि च} %6-130-6

\twolineshloka
{स्थानानि च निरस्यन्तां नन्दिग्रामादितः परम्}
{सिञ्चन्तु पृथिवीं कृत्स्नां हिमशीतेन वारिणा} %6-130-7

\twolineshloka
{ततोऽभ्यवकिरन्त्वन्ये लाजैः पुष्पैश्च सर्वतः}
{समुच्छ्रितपताकास्तु रथ्याः पुरवरोत्तमे} %6-130-8

\twolineshloka
{शोभयन्तु च वेश्मानि सूर्यस्योदयनं प्रति}
{स्रग्दाममुक्तपुष्पैश्च सुवर्णैः पञ्चवर्णकैः} %6-130-9

\twolineshloka
{राजमार्गमसम्बाधं किरन्तु शतशो नराः}
{ततस्तच्छासनं श्रुत्वा शत्रुघ्नस्य मुदान्विताः} %6-130-10

\twolineshloka
{धृष्टिर्जयन्तो विजयः सिद्धार्थश्चार्थसाधकः}
{अशोको मन्त्रपालश्च सुमन्त्रश्चापि निर्ययुः} %6-130-11

\twolineshloka
{मत्तैर्नागसहस्रैश्च सध्वजैः सुविभूषितैः}
{अपरे हेमकक्षाभिः सगजाभिः करेणुभिः} %6-130-12

\twolineshloka
{निर्ययुस्तुरगाक्रान्ता रथैश्च सुमहारथाः}
{शक्त्यृष्टिपाशहस्तानां सध्वजानां पताकिनाम्} %6-130-13

\twolineshloka
{तुरगाणां सहस्रैश्च मुख्यैर्मुख्यतरान्वितैः}
{पदातीनां सहस्रैश्च वीराः परिवृता ययुः} %6-130-14

\twolineshloka
{ततो यानान्युपारूढाः सर्वा दशरथस्त्रियः}
{कौसल्यां प्रमुखे कृत्वा सुमित्रां चापि निर्ययुः} %6-130-15

\onelineshloka
{कैकेय्या सहिताः सर्वा नन्दिग्राममुपागमन्} %6-130-16

\twolineshloka
{द्विजातिमुख्यैर्धर्मात्मा श्रेणीमुख्यैः सनैगमैः}
{माल्यमोदकहस्तैश्च मन्त्रिभिर्भरतो वृतः} %6-130-17

\twolineshloka
{शङ्खभेरीनिनादैश्च बन्दिभिश्चाभिनन्दितः}
{आर्यपादौ गृहीत्वा तु शिरसा धर्मकोविदः} %6-130-18

\twolineshloka
{पाण्डुरं छत्रमादाय शुक्लमाल्योपशोभितम्}
{शुक्ले च वालव्यजने राजार्हे हेमभूषिते} %6-130-19

\twolineshloka
{उपवासकृशो दीनश्चीरकृष्णाजिनाम्बरः}
{भ्रातुरागमनं श्रुत्वा तत्पूर्वं हर्षमागतः} %6-130-20

\twolineshloka
{प्रत्युद्ययौ यदा रामं महात्मा सचिवैः सह}
{अश्वानां खुरशब्दैश्च रथनेमिस्वनेन च} %6-130-21

\twolineshloka
{शङ्खदुन्दुभिनादेन सञ्चचालेव मेदिनी}
{गजानां बृंहितैश्चापि शङ्खदुन्दुभिनिःस्वनैः} %6-130-22

\twolineshloka
{कृत्स्नं तु नगरं तत् तु नन्दिग्राममुपागमत्}
{समीक्ष्य भरतो वाक्यमुवाच पवनात्मजम्} %6-130-23

\twolineshloka
{कच्चिन्न खलु कापेयी सेव्यते चलचित्तता}
{नहि पश्यामि काकुत्स्थं राममार्यं परन्तपम्} %6-130-24

\twolineshloka
{कच्चिन्न चानुदृश्यन्ते कपयः कामरूपिणः}
{अथैवमुक्ते वचने हनूमानिदमब्रवीत्} %6-130-25

\twolineshloka
{अर्थ्यं विज्ञापयन्नेव भरतं सत्यविक्रमम्}
{सदाफलान् कुसुमितान् वृक्षान् प्राप्य मधुस्रवान्} %6-130-26

\twolineshloka
{भरद्वाजप्रसादेन मत्तभ्रमरनादितान्}
{तस्य चैव वरो दत्तो वासवेन परन्तप} %6-130-27

\twolineshloka
{ससैन्यस्य तदातिथ्यं कृतं सर्वगुणान्वितम्}
{निःस्वनः श्रूयते भीमः प्रहृष्टानां वनौकसाम्} %6-130-28

\twolineshloka
{मन्ये वानरसेना सा नदीं तरति गोमतीम्}
{रजोवर्षं समुद्भूतं पश्य सालवनं प्रति} %6-130-29

\twolineshloka
{मन्ये सालवनं रम्यं लोलयन्ति प्लवङ्गमाः}
{तदेतद् दृश्यते दूराद् विमानं चन्द्रसन्निभम्} %6-130-30

\twolineshloka
{विमानं पुष्पकं दिव्यं मनसा ब्रह्मनिर्मितम्}
{रावणं बान्धवैः सार्धं हत्वा लब्धं महात्मना} %6-130-31

\twolineshloka
{तरुणादित्यसङ्काशं विमानं रामवाहनम्}
{धनदस्य प्रसादेन दिव्यमेतन्मनोजवम्} %6-130-32

\twolineshloka
{एतस्मिन् भ्रातरौ वीरौ वैदेह्या सह राघवौ}
{सुग्रीवश्च महातेजा राक्षसश्च विभीषणः} %6-130-33

\twolineshloka
{ततो हर्षसमुद्भूतो निःस्वनो दिवमस्पृशत्}
{स्त्रीबालयुववृद्धानां रामोऽयमिति कीर्तिते} %6-130-34

\twolineshloka
{रथकुञ्जरवाजिभ्यस्तेऽवतीर्य महीं गताः}
{ददृशुस्तं विमानस्थं नराः सोममिवाम्बरे} %6-130-35

\twolineshloka
{प्राञ्जलिर्भरतो भूत्वा प्रहृष्टो राघवोन्मुखः}
{यथार्थेनार्घ्यपाद्याद्यैस्ततो राममपूजयत्} %6-130-36

\twolineshloka
{मनसा ब्रह्मणा सृष्टे विमाने भरताग्रजः}
{रराज पृथुदीर्घाक्षो वज्रपाणिरिवामरः} %6-130-37

\twolineshloka
{ततो विमानाग्रगतं भरतो भ्रातरं तदा}
{ववन्दे प्रणतो रामं मेरुस्थमिव भास्करम्} %6-130-38

\twolineshloka
{ततो रामाभ्यनुज्ञातं तद् विमानमनुत्तमम्}
{हंसयुक्तं महावेगं निपपात महीतलम्} %6-130-39

\twolineshloka
{आरोपितो विमानं तद् भरतः सत्यविक्रमः}
{राममासाद्य मुदितः पुनरेवाभ्यवादयत्} %6-130-40

\twolineshloka
{तं समुत्थाय काकुत्स्थश्चिरस्याक्षिपथं गतम्}
{अङ्के भरतमारोप्य मुदितः परिषस्वजे} %6-130-41

\twolineshloka
{ततो लक्ष्मणमासाद्य वैदेहीं च परन्तपः}
{अथाभ्यवादयत् प्रीतो भरतो नाम चाब्रवीत्} %6-130-42

\twolineshloka
{सुग्रीवं केकयीपुत्रो जाम्बवन्तमथाङ्गदम्}
{मैन्दं च द्विविदं नीलमृषभं चैव सस्वजे} %6-130-43

\twolineshloka
{सुषेणं च नलं चैव गवाक्षं गन्धमादनम्}
{शरभं पनसं चैव परितः परिषस्वजे} %6-130-44

\twolineshloka
{ते कृत्वा मानुषं रूपं वानराः कामरूपिणः}
{कुशलं पर्यपृच्छंस्ते प्रहृष्टा भरतं तदा} %6-130-45

\twolineshloka
{अथाब्रवीद् राजपुत्रः सुग्रीवं वानरर्षभम्}
{परिष्वज्य महातेजा भरतो धर्मिणां वरः} %6-130-46

\twolineshloka
{त्वमस्माकं चतुर्णां वै भ्राता सुग्रीव पञ्चमः}
{सौहृदाज्जायते मित्रमपकारोऽरिलक्षणम्} %6-130-47

\twolineshloka
{विभीषणं च भरतः सान्त्ववाक्यमथाब्रवीत्}
{दिष्ट्या त्वया सहायेन कृतं कर्म सुदुष्करम्} %6-130-48

\twolineshloka
{शत्रुघ्नश्च तदा राममभिवाद्य सलक्ष्मणम्}
{सीतायाश्चरणौ वीरो विनयादभ्यवादयत्} %6-130-49

\twolineshloka
{रामो मातरमासाद्य विवर्णां शोककर्शिताम्}
{जग्राह प्रणतः पादौ मनो मातुः प्रहर्षयन्} %6-130-50

\twolineshloka
{अभिवाद्य सुमित्रां च कैकेयीं च यशस्विनीम्}
{स मातॄश्च ततः सर्वाः पुरोहितमुपागमत्} %6-130-51

\twolineshloka
{स्वागतं ते महाबाहो कौसल्यानन्दवर्धन}
{इति प्राञ्जलयः सर्वे नागरा राममब्रुवन्} %6-130-52

\twolineshloka
{तान्यञ्जलिसहस्राणि प्रगृहीतानि नागरैः}
{व्याकोशानीव पद्मानि ददर्श भरताग्रजः} %6-130-53

\twolineshloka
{पादुके ते तु रामस्य गृहीत्वा भरतः स्वयम्}
{चरणाभ्यां नरेन्द्रस्य योजयामास धर्मवित्} %6-130-54

\twolineshloka
{अब्रवीच्च तदा रामं भरतः स कृताञ्जलिः}
{एतत् ते सकलं राज्यं न्यासं निर्यातितं मया} %6-130-55

\twolineshloka
{अद्य जन्म कृतार्थं मे संवृत्तश्च मनोरथः}
{यत् त्वां पश्यामि राजानमयोध्यां पुनरागतम्} %6-130-56

\twolineshloka
{अवेक्षतां भवान् कोशं कोष्ठागारं गृहं बलम्}
{भवतस्तेजसा सर्वं कृतं दशगुणं मया} %6-130-57

\twolineshloka
{तथा ब्रुवाणं भरतं दृष्ट्वा तं भ्रातृवत्सलम्}
{मुमुचुर्वानरा बाष्पं राक्षसश्च विभीषणः} %6-130-58

\twolineshloka
{ततः प्रहर्षाद् भरतमङ्कमारोप्य राघवः}
{ययौ तेन विमानेन ससैन्यो भरताश्रमम्} %6-130-59

\twolineshloka
{भरताश्रममासाद्य ससैन्यो राघवस्तदा}
{अवतीर्य विमानाग्रादवतस्थे महीतले} %6-130-60

\twolineshloka
{अब्रवीत् तु तदा रामस्तद् विमानमनुत्तमम्}
{वह वैश्रवणं देवमनुजानामि गम्यताम्} %6-130-61

\twolineshloka
{ततो रामाभ्यनुज्ञातं तद् विमानमनुत्तमम्}
{उत्तरां दिशमुद्दिश्य जगाम धनदालयम्} %6-130-62

\twolineshloka
{विमानं पुष्पकं दिव्यं सङ्गृहीतं तु रक्षसा}
{अगमद् धनदं वेगाद् रामवाक्यप्रचोदितम्} %6-130-63

\twolineshloka
{पुरोहितस्यात्मसखस्य राघवो बृहस्पतेः शक्र इवामराधिपः}
{निपीड्य पादौ पृथगासने शुभे सहैव तेनोपविवेश वीर्यवान्} %6-130-64


॥इत्यार्षे श्रीमद्रामायणे वाल्मीकीये आदिकाव्ये युद्धकाण्डे भरतसमागमः नाम त्रिंशदधिकशततमः सर्गः ॥६-१३०॥
