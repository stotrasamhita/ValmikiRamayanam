\sect{चतुर्दशाधिकशततमः सर्गः — मन्दोदरीविलापः}

\twolineshloka
{तासां विलपमानानां तदा राक्षसयोषिताम्}
{ज्येष्ठपत्नी प्रिया दीना भर्तारं समुदैक्षत} %6-114-1

\twolineshloka
{दशग्रीवं हतं दृष्ट्वा रामेणाचिन्त्यकर्मणा}
{पतिं मन्दोदरी तत्र कृपणा पर्यदेवयत्} %6-114-2

\twolineshloka
{ननु नाम महाबाहो तव वैश्रवणानुज}
{क्रुद्धस्य प्रमुखे स्थातुं त्रस्यत्यपि पुरंदरः} %6-114-3

\twolineshloka
{ऋषयश्च महान्तोऽपि गन्धर्वाश्च यशस्विनः}
{ननु नाम तवोद्वेगाच्चारणाश्च दिशो गताः} %6-114-4

\twolineshloka
{स त्वं मानुषमात्रेण रामेण युधि निर्जितः}
{न व्यपत्रपसे राजन् किमिदं राक्षसेश्वर} %6-114-5

\twolineshloka
{कथं त्रैलोक्यमाक्रम्य श्रिया वीर्येण चान्वितम्}
{अविषह्यं जघान त्वां मानुषो वनगोचरः} %6-114-6

\twolineshloka
{मानुषाणामविषये चरतः कामरूपिणः}
{विनाशस्तव रामेण संयुगे नोपपद्यते} %6-114-7

\twolineshloka
{न चैतत् कर्म रामस्य श्रद्दधामि चमूमुखे}
{सर्वतः समुपेतस्य तव तेनाभिमर्षणम्} %6-114-8

\twolineshloka
{अथवा रामरूपेण कृतान्तः स्वयमागतः}
{मायां तव विनाशाय विधायाप्रतितर्किताम्} %6-114-9

\twolineshloka
{अथवा वासवेन त्वं धर्षितोऽसि महाबल}
{वासवस्य तु का शक्तिस्त्वां द्रष्टुमपि संयुगे} %6-114-10

\twolineshloka
{महाबलं महावीर्यं देवशत्रुं महौजसम्}
{व्यक्तमेष महायोगी परमात्मा सनातनः} %6-114-11

\twolineshloka
{अनादिमध्यनिधनो महतः परमो महान्}
{तमसः परमो धाता शङ्खचक्रगदाधरः} %6-114-12

\twolineshloka
{श्रीवत्सवक्षा नित्यश्रीरजय्यः शाश्वतो ध्रुवः}
{मानुषं रूपमास्थाय विष्णुः सत्यपराक्रमः} %6-114-13

\twolineshloka
{सर्वैः परिवृतो देवैर्वानरत्वमुपागतैः}
{सर्वलोकेश्वरः श्रीमाँल्लोकानां हितकाम्यया} %6-114-14

\twolineshloka
{स राक्षसपरीवारं देवशत्रुं भयावहम्}
{इन्द्रियाणि पुरा जित्वा जितं त्रिभुवनं त्वया} %6-114-15

\twolineshloka
{स्मरद्भिरिव तद् वैरमिन्द्रियैरेव निर्जितः}
{यदैव हि जनस्थाने राक्षसैर्बहुभिर्वृतः} %6-114-16

\twolineshloka
{खरस्तु निहतो भ्राता तदा रामो न मानुषः}
{यदैव नगरीं लङ्कां दुष्प्रवेशां सुरैरपि} %6-114-17

\twolineshloka
{प्रविष्टो हनुमान् वीर्यात् तदैव व्यथिता वयम्}
{क्रियतामविरोधश्च राघवेणेति यन्मया} %6-114-18

\twolineshloka
{उच्यमानो न गृह्णासि तस्येयं व्युष्टिरागता}
{अकस्माच्चाभिकामोऽसि सीतां राक्षसपुङ्गव} %6-114-19

\twolineshloka
{ऐश्वर्यस्य विनाशाय देहस्य स्वजनस्य च}
{अरुन्धत्या विशिष्टां तां रोहिण्याश्चापि दुर्मते} %6-114-20

\twolineshloka
{सीतां धर्षयता मान्यां त्वया ह्यसदृशं कृतम्}
{वसुधाया हि वसुधां श्रियाः श्रीं भर्तृवत्सलाम्} %6-114-21

\twolineshloka
{सीतां सर्वानवद्याङ्गीमरण्ये विजने शुभाम्}
{आनयित्वा तु तां दीनां छद्मनाऽऽत्मस्वदूषणम्} %6-114-22

\twolineshloka
{अप्राप्य तं चैव कामं मैथिलीसंगमे कृतम्}
{पतिव्रतायास्तपसा नूनं दग्धोऽसि मे प्रभो} %6-114-23

\twolineshloka
{तदैव यन्न दग्धस्त्वं धर्षयंस्तनुमध्यमाम्}
{देवा बिभ्यति ते सर्वे सेन्द्राः साग्निपुरोगमाः} %6-114-24

\twolineshloka
{अवश्यमेव लभते फलं पापस्य कर्मणः}
{भर्तः पर्यागते काले कर्ता नास्त्यत्र संशयः} %6-114-25

\twolineshloka
{शुभकृच्छुभमाप्नोति पापकृत् पापमश्नुते}
{विभीषणः सुखं प्राप्तस्त्वं प्राप्तः पापमीदृशम्} %6-114-26

\twolineshloka
{सन्त्यन्याः प्रमदास्तुभ्यं रूपेणाभ्यधिकास्ततः}
{अनङ्गवशमापन्नस्त्वं तु मोहान्न बुद्ध्यसे} %6-114-27

\twolineshloka
{न कुलेन न रूपेण न दाक्षिण्येन मैथिली}
{मयाधिका वा तुल्या वा तत् तु मोहान्न बुद्ध्यसे} %6-114-28

\twolineshloka
{सर्वदा सर्वभूतानां नास्ति मृत्युरलक्षणः}
{तव तद्वदयं मृत्युर्मैथिलीकृतलक्षणः} %6-114-29

\twolineshloka
{सीतानिमित्तजो मृत्युस्त्वया दूरादुपाहृतः}
{मैथिली सह रामेण विशोका विहरिष्यति} %6-114-30

\twolineshloka
{अल्पपुण्या त्वहं घोरे पतिता शोकसागरे}
{कैलासे मन्दरे मेरौ तथा चैत्ररथे वने} %6-114-31

\twolineshloka
{देवोद्यानेषु सर्वेषु विहृत्य सहिता त्वया}
{विमानेनानुरूपेण या याम्यतुलया श्रिया} %6-114-32

\twolineshloka
{पश्यन्ती विविधान् देशांस्तांस्तांश्चित्रस्रगम्बरा}
{भ्रंशिता कामभोगेभ्यः सास्मि वीर वधात् तव} %6-114-33

\twolineshloka
{सैवान्येवास्मि संवृत्ता धिग्राज्ञां चञ्चलां श्रियम्}
{हा राजन् सुकुमारं ते सुभ्रु सुत्वक्समुन्नसम्} %6-114-34

\twolineshloka
{कान्तिश्रीद्युतिभिस्तुल्यमिन्दुपद्मदिवाकरैः}
{किरीटकूटोज्ज्वलितं ताम्रास्यं दीप्तकुण्डलम्} %6-114-35

\twolineshloka
{मदव्याकुललोलाक्षं भूत्वा यत्पानभूमिषु}
{विविधस्रग्धरं चारु वल्गुस्मितकथं शुभम्} %6-114-36

\twolineshloka
{तदेवाद्य तवैवं हि वक्त्रं न भ्राजते प्रभो}
{रामसायकनिर्भिन्नं रक्तं रुधिरविस्रवैः} %6-114-37

\twolineshloka
{विशीर्णमेदोमस्तिष्कं रूक्षं स्यन्दनरेणुभिः}
{हा पश्चिमा मे सम्प्राप्ता दशा वैधव्यदायिनी} %6-114-38

\twolineshloka
{या मयाऽऽसीन्न सम्बुद्धा कदाचिदपि मन्दया}
{पिता दानवराजो मे भर्ता मे राक्षसेश्वरः} %6-114-39

\twolineshloka
{पुत्रो मे शक्रनिर्जेता इत्यहं गर्विता भृशम्}
{दृप्तारिमथनाः क्रूराः प्रख्यातबलपौरुषाः} %6-114-40

\twolineshloka
{अकुतश्चिद्भया नाथा ममेत्यासीन्मतिर्ध्रुवा}
{तेषामेवंप्रभावाणां युष्माकं राक्षसर्षभाः} %6-114-41

\twolineshloka
{कथं भयमसम्बुद्धं मानुषादिदमागतम्}
{स्निग्धेन्द्रनीलनीलं तु प्रांशुशैलोपमं महत्} %6-114-42

\twolineshloka
{केयूराङ्गदवैदूर्यमुक्ताहारस्रगुज्ज्वलम्}
{कान्तं विहारेष्वधिकं दीप्तं संग्रामभूमिषु} %6-114-43

\twolineshloka
{भात्याभरणभाभिर्यद् विद्युद्भिरिव तोयदः}
{तदेवाद्य शरीरं ते तीक्ष्णैर्नैकशरैश्चितम्} %6-114-44

\twolineshloka
{पुनर्दुर्लभसंस्पर्शं परिष्वक्तुं न शक्यते}
{श्वाविधः शललैर्यद्वद् बाणैर्लग्नैर्निरन्तरम्} %6-114-45

\twolineshloka
{स्वर्पितैर्मर्मसु भृशं संछिन्नस्नायुबन्धनम्}
{क्षितौ निपतितं राजन् श्यामं वै रुधिरच्छवि} %6-114-46

\twolineshloka
{वज्रप्रहाराभिहतो विकीर्ण इव पर्वतः}
{हा स्वप्नः सत्यमेवेदं त्वं रामेण कथं हतः} %6-114-47

\twolineshloka
{त्वं मृत्योरपि मृत्युः स्याः कथं मृत्युवशं गतः}
{त्रैलोक्यवसुभोक्तारं त्रैलोक्योद्वेगदं महत्} %6-114-48

\twolineshloka
{जेतारं लोकपालानां क्षेप्तारं शंकरस्य च}
{दृप्तानां निग्रहीतारमाविष्कृतपराक्रमम्} %6-114-49

\twolineshloka
{लोकक्षोभयितारं च साधुभूतविदारणम्}
{ओजसा दृप्तवाक्यानां वक्तारं रिपुसंनिधौ} %6-114-50

\twolineshloka
{स्वयूथभृत्यगोप्तारं हन्तारं भीमकर्मणाम्}
{हन्तारं दानवेन्द्राणां यक्षाणां च सहस्रशः} %6-114-51

\twolineshloka
{निवातकवचानां तु निग्रहीतारमाहवे}
{नैकयज्ञविलोप्तारं त्रातारं स्वजनस्य च} %6-114-52

\twolineshloka
{धर्मव्यवस्थाभेत्तारं मायास्रष्टारमाहवे}
{देवासुरनृकन्यानामाहर्तारं ततस्ततः} %6-114-53

\twolineshloka
{शत्रुस्त्रीशोकदातारं नेतारं स्वजनस्य च}
{लङ्काद्वीपस्य गोप्तारं कर्तारं भीमकर्मणाम्} %6-114-54

\twolineshloka
{अस्माकं कामभोगानां दातारं रथिनां वरम्}
{एवंप्रभावं भर्तारं दृष्ट्वा रामेण पातितम्} %6-114-55

\twolineshloka
{स्थिरास्मि या देहमिमं धारयामि हतप्रिया}
{शयनेषु महार्हेषु शयित्वा राक्षसेश्वर} %6-114-56

\twolineshloka
{इह कस्मात् प्रसुप्तोऽसि धरण्यां रेणुगुण्ठितः}
{यदा मे तनयः शस्तो लक्ष्मणेनेन्द्रजिद् युधि} %6-114-57

\twolineshloka
{तदा त्वभिहता तीव्रमद्य त्वस्मिन् निपातिता}
{साहं बन्धुजनैर्हीना हीना नाथेन च त्वया} %6-114-58

\twolineshloka
{विहीना कामभोगैश्च शोचिष्ये शाश्वतीः समाः}
{प्रपन्नो दीर्घमध्वानं राजन्नद्य सुदुर्गमम्} %6-114-59

\twolineshloka
{नय मामपि दुःखार्तां न वर्तिष्ये त्वया विना}
{कस्मात् त्वं मां विहायेह कृपणां गन्तुमिच्छसि} %6-114-60

\twolineshloka
{दीनां विलपतीं मन्दां किं च मां नाभिभाषसे}
{दृष्ट्वा न खल्वभिक्रुद्धो मामिहानवगुण्ठिताम्} %6-114-61

\twolineshloka
{निर्गतां नगरद्वारात् पद्भ्यामेवागतां प्रभो}
{पश्येष्टदार दारांस्ते भ्रष्टलज्जावगुण्ठनान्} %6-114-62

\twolineshloka
{बहिर्निष्पतितान् सर्वान् कथं दृष्ट्वा न कुप्यसि}
{अयं क्रीडासहायस्तेऽनाथो लालप्यते जनः} %6-114-63

\twolineshloka
{न चैनमाश्वासयसि किं वा न बहुमन्यसे}
{यास्त्वया विधवा राजन् कृता नैकाः कुलस्त्रियः} %6-114-64

\twolineshloka
{पतिव्रता धर्मरता गुरुशुश्रूषणे रताः}
{ताभिः शोकाभितप्ताभिः शप्तः परवशं गतः} %6-114-65

\twolineshloka
{त्वया विप्रकृताभिश्च तदा शप्तस्तदागतम्}
{प्रवादः सत्यमेवायं त्वां प्रति प्रायशो नृप} %6-114-66

\twolineshloka
{पतिव्रतानां नाकस्मात् पतन्त्यश्रूणि भूतले}
{कथं च नाम ते राजँल्लोकानाक्रम्य तेजसा} %6-114-67

\twolineshloka
{नारीचौर्यमिदं क्षुद्रं कृतं शौण्डीर्यमानिना}
{अपनीयाश्रमाद् रामं यन्मृगच्छद्मना त्वया} %6-114-68

\twolineshloka
{आनीता रामपत्नी सा अपनीय च लक्ष्मणम्}
{कातर्यं च न ते युद्धे कदाचित् संस्मराम्यहम्} %6-114-69

\twolineshloka
{तत् तु भाग्यविपर्यासान्नूनं ते पक्वलक्षणम्}
{अतीतानागतार्थज्ञो वर्तमानविचक्षणः} %6-114-70

\twolineshloka
{मैथिलीमाहृतां दृष्ट्वा ध्यात्वा निःश्वस्य चायतम्}
{सत्यवाक् स महाबाहो देवरो मे यदब्रवीत्} %6-114-71

\twolineshloka
{अयं राक्षसमुख्यानां विनाशः प्रत्युपस्थितः}
{कामक्रोधसमुत्थेन व्यसनेन प्रसङ्गिना} %6-114-72

\twolineshloka
{निवृत्तस्त्वत्कृतेनार्थः सोऽयं मूलहरो महान्}
{त्वया कृतमिदं सर्वमनाथं राक्षसं कुलम्} %6-114-73

\twolineshloka
{नहि त्वं शोचितव्यो मे प्रख्यातबलपौरुषः}
{स्त्रीस्वभावात् तु मे बुद्धिः कारुण्ये परिवर्तते} %6-114-74

\twolineshloka
{सुकृतं दुष्कृतं च त्वं गृहीत्वा स्वां गतिं गतः}
{आत्मानमनुशोचामि त्वद्विनाशेन दुःखिताम्} %6-114-75

\twolineshloka
{सुहृदां हितकामानां न श्रुतं वचनं त्वया}
{भ्रातॄणां चैव कात्स्र्न्येन हितमुक्तं दशानन} %6-114-76

\twolineshloka
{हेत्वर्थयुक्तं विधिवच्छ्रेयस्करमदारुणम्}
{विभीषणेनाभिहितं न कृतं हेतुमत् त्वया} %6-114-77

\twolineshloka
{मारीचकुम्भकर्णाभ्यां वाक्यं मम पितुस्तथा}
{न कृतं वीर्यमत्तेन तस्येदं फलमीदृशम्} %6-114-78

\twolineshloka
{नीलजीमूतसंकाश पीताम्बर शुभाङ्गद}
{स्वगात्राणि विनिक्षिप्य किं शेषे रुधिरावृतः} %6-114-79

\twolineshloka
{प्रसुप्त इव शोकार्तां किं मां न प्रतिभाषसे}
{महावीर्यस्य दक्षस्य संयुगेष्वपलायिनः} %6-114-80

\twolineshloka
{यातुधानस्य दौहित्रीं किं मां न प्रतिभाषसे}
{उत्तिष्ठोत्तिष्ठ किं शेषे नवे परिभवे कृते} %6-114-81

\twolineshloka
{अद्य वै निर्भया लङ्कां प्रविष्टाः सूर्यरश्मयः}
{येन सूदयसे शत्रून् समरे सूर्यवर्चसा} %6-114-82

\twolineshloka
{वज्रं वज्रधरस्येव सोऽयं ते सततार्चितः}
{रणे बहुप्रहरणो हेमजालपरिष्कृतः} %6-114-83

\twolineshloka
{परिघो व्यवकीर्णस्ते बाणैश्छिन्नः सहस्रधा}
{प्रियामिवोपसंगृह्य किं शेषे रणमेदिनीम्} %6-114-84

\twolineshloka
{अप्रियामिव कस्माच्च मां नेच्छस्यभिभाषितुम्}
{धिगस्तु हृदयं यस्या ममेदं न सहस्रधा} %6-114-85

\twolineshloka
{त्वयि पञ्चत्वमापन्ने फलते शोकपीडितम्}
{इत्येवं विलपन्ती सा बाष्पपर्याकुलेक्षणा} %6-114-86

\twolineshloka
{स्नेहोपस्कन्नहृदया तदा मोहमुपागमत्}
{कश्मलाभिहता सन्ना बभौ सा रावणोरसि} %6-114-87

\twolineshloka
{संध्यानुरक्ते जलदे दीप्ता विद्युदिवोज्ज्वला}
{तथागतां समुत्थाप्य सपत्न्यस्तां भृशातुराः} %6-114-88

\twolineshloka
{पर्यवस्थापयामासू रुदत्यो रुदतीं भृशम्}
{किं ते न विदिता देवि लोकानां स्थितिरध्रुवा} %6-114-89

\twolineshloka
{दशाविभागपर्याये राज्ञां वै चञ्चलाः श्रियः}
{इत्येवमुच्यमाना सा सशब्दं प्ररुरोद ह} %6-114-90

\twolineshloka
{स्नपयन्ती तदास्रेण स्तनौ वक्त्रं सुनिर्मलम्}
{एतस्मिन्नन्तरे रामो विभीषणमुवाच ह} %6-114-91

\twolineshloka
{संस्कारः क्रियतां भ्रातुः स्त्रीगणः परिसान्त्व्यताम्}
{तमुवाच ततो धीमान् विभीषण इदं वचः} %6-114-92

\twolineshloka
{विमृश्य बुद्ध्या प्रश्रितं धर्मार्थसहितं हितम्}
{त्यक्तधर्मव्रतं क्रूरं नृशंसमनृतं तथा} %6-114-93

\twolineshloka
{नाहमर्हामि संस्कर्तुं परदाराभिमर्शनम्}
{भ्रातृरूपो हि मे शत्रुरेष सर्वाहिते रतः} %6-114-94

\twolineshloka
{रावणो नार्हते पूजां पूज्योऽपि गुरुगौरवात्}
{नृशंस इति मां राम वक्ष्यन्ति मनुजा भुवि} %6-114-95

\twolineshloka
{श्रुत्वा तस्यागुणान् सर्वे वक्ष्यन्ति सुकृतं पुनः}
{तच्छ्रुत्वा परमप्रीतो रामो धर्मभृतां वरः} %6-114-96

\twolineshloka
{विभीषणमुवाचेदं वाक्यज्ञं वाक्यकोविदः}
{तवापि मे प्रियं कार्यं त्वत्प्रभावान्मया जितम्} %6-114-97

\twolineshloka
{अवश्यं तु क्षमं वाच्यो मया त्वं राक्षसेश्वर}
{अधर्मानृतसंयुक्तः कामं त्वेष निशाचरः} %6-114-98

\twolineshloka
{तेजस्वी बलवाञ्छूरः संग्रामेषु च नित्यशः}
{शतक्रतुमुखैर्देवैः श्रूयते न पराजितः} %6-114-99

\twolineshloka
{महात्मा बलसम्पन्नो रावणो लोकरावणः}
{मरणान्तानि वैराणि निर्वृत्तं नः प्रयोजनम्} %6-114-100

\twolineshloka
{क्रियतामस्य संस्कारो ममाप्येष यथा तव}
{त्वत्सकाशान्महाबाहो संस्कारं विधिपूर्वकम्} %6-114-101

\twolineshloka
{क्षिप्रमर्हति धर्मेण त्वं यशोभाग् भविष्यसि}
{राघवस्य वचः श्रुत्वा त्वरमाणो विभीषणः} %6-114-102

\twolineshloka
{संस्कारयितुमारेभे भ्रातरं रावणं हतम्}
{स प्रविश्य पुरीं लङ्कां राक्षसेन्द्रो विभीषणः} %6-114-103

\twolineshloka
{रावणस्याग्निहोत्रं तु निर्यापयति सत्वरम्}
{शकटान् दारुरूपाणि अग्नीन् वै याजकांस्तथा} %6-114-104

\twolineshloka
{तथा चन्दनकाष्ठानि काष्ठानि विविधानि च}
{अगरूणि सुगन्धीनि गन्धांश्च सुरभींस्तथा} %6-114-105

\twolineshloka
{मणिमुक्ताप्रवालानि निर्यापयति राक्षसः}
{आजगाम मुहूर्तेन राक्षसैः परिवारितः} %6-114-106

\twolineshloka
{ततो माल्यवता सार्धं क्रियामेव चकार सः}
{सौवर्णीं शिबिकां दिव्यामारोप्य क्षौमवाससम्} %6-114-107

\twolineshloka
{रावणं राक्षसाधीशमश्रुवर्णमुखा द्विजाः}
{तूर्यघोषैश्च विविधैः स्तुवद्भिश्चाभिनन्दितम्} %6-114-108

\twolineshloka
{पताकाभिश्च चित्राभिः सुमनोभिश्च चित्रिताम्}
{उत्क्षिप्य शिबिकां तां तु विभीषणपुरोगमाः} %6-114-109

\twolineshloka
{दक्षिणाभिमुखाः सर्वे गृह्य काष्ठानि भेजिरे}
{अग्नयो दीप्यमानास्ते तदाध्वर्युसमीरिताः} %6-114-110

\twolineshloka
{शरणाभिगताः सर्वे पुरस्तात् तस्य ते ययुः}
{अन्तःपुराणि सर्वाणि रुदमानानि सत्वरम्} %6-114-111

\twolineshloka
{पृष्ठतोऽनुययुस्तानि प्लवमानानि सर्वतः}
{रावणं प्रयते देशे स्थाप्य ते भृशदुःखिताः} %6-114-112

\twolineshloka
{चितां चन्दनकाष्ठैश्च पद्मकोशीरचन्दनैः}
{ब्राह्म्या संवर्तयामासू राङ्कवास्तरणावृताम्} %6-114-113

\twolineshloka
{प्रचक्रू राक्षसेन्द्रस्य पितृमेधमनुत्तमम्}
{वेदिं च दक्षिणाप्राचीं यथास्थानं च पावकम्} %6-114-114

\twolineshloka
{पृषदाज्येन सम्पूर्णं स्रुवं स्कन्धे प्रचिक्षिपुः}
{पादयोः शकटं प्रापुरूर्वोश्चोलूखलं तदा} %6-114-115

\twolineshloka
{दारुपात्राणि सर्वाणि अरणिं चोत्तरारणिम्}
{दत्त्वा तु मुसलं चान्यं यथास्थानं विचक्रमुः} %6-114-116

\twolineshloka
{शास्त्रदृष्टेन विधिना महर्षिविहितेन च}
{तत्र मेध्यं पशुं हत्वा राक्षसेन्द्रस्य राक्षसाः} %6-114-117

\twolineshloka
{परिस्तरणिकां राज्ञो घृताक्तां समवेशयन्}
{गन्धैर्माल्यैरलंकृत्य रावणं दीनमानसाः} %6-114-118

\twolineshloka
{विभीषणसहायास्ते वस्त्रैश्च विविधैरपि}
{लाजैरवकिरन्ति स्म बाष्पपूर्णमुखास्तथा} %6-114-119

\twolineshloka
{स ददौ पावकं तस्य विधियुक्तं विभीषणः}
{स्नात्वा चैवार्द्रवस्त्रेण तिलान् दर्भविमिश्रितान्} %6-114-120

\twolineshloka
{उदकेन च सम्मिश्रान् प्रदाय विधिपूर्वकम्}
{ताः स्त्रियोऽनुनयामास सान्त्वयित्वा पुनः पुनः} %6-114-121

\threelineshloka
{गम्यतामिति ताः सर्वा विविशुर्नगरं ततः}
{प्रविष्टासु पुरीं स्त्रीषु राक्षसेन्द्रो विभीषणः}
{रामपार्श्वमुपागम्य समतिष्ठद् विनीतवत्} %6-114-122

\twolineshloka
{रामोऽपि सह सैन्येन ससुग्रीवः सलक्ष्मणः}
{हर्षं लेभे रिपुं हत्वा वृत्रं वज्रधरो यथा} %6-114-123

\twolineshloka
{ततो विमुक्त्वा सशरं शरासनं महेन्द्रदत्तं कवचं स तन्महत्}
{विमुच्य रोषं रिपुनिग्रहात् ततो रामः स सौम्यत्वमुपागतोऽरिहा} %6-114-124


॥इत्यार्षे श्रीमद्रामायणे वाल्मीकीये आदिकाव्ये युद्धकाण्डे मन्दोदरीविलापः नाम चतुर्दशाधिकशततमः सर्गः ॥६-११४॥
