\sect{दशाधिकशततमः सर्गः — रावणैकशतशिरश्छेदनम्}

\twolineshloka
{तौ तथा युध्यमानौ तु समरे रामरावणौ}
{ददृशुः सर्वभूतानि विस्मितेनान्तरात्मना} %6-110-1

\twolineshloka
{अर्दयन्तौ तु समरे तयोस्तौ स्यन्दनोत्तमौ}
{परस्परमभिक्रुद्धौ परस्परमभिद्रुतौ} %6-110-2

\twolineshloka
{परस्परवधे युक्तौ घोररूपौ बभूवतुः}
{मण्डलानि च वीथीश्च गतप्रत्यागतानि च} %6-110-3

\twolineshloka
{दर्शयन्तौ बहुविधां सूतौ सारथ्यजां गतिम्}
{अर्दयन् रावणं रामो राघवं चापि रावणः} %6-110-4

\twolineshloka
{गतिवेगं समापन्नौ प्रवर्तननिवर्तने}
{क्षिपतोः शरजालानि तयोस्तौ स्यन्दनोत्तमौ} %6-110-5

\twolineshloka
{चेरतुः संयुगमहीं सासारौ जलदाविव}
{दर्शयित्वा तदा तौ तु गतिं बहुविधां रणे} %6-110-6

\twolineshloka
{परस्परस्याभिमुखौ पुनरेव च तस्थतुः}
{धुरं धुरेण रथयोर्वक्त्रं वक्त्रेण वाजिनाम्} %6-110-7

\twolineshloka
{पताकाश्च पताकाभिः समीयुः स्थितयोस्तदा}
{रावणस्य ततो रामो धनुर्मुक्तैः शितैः शरैः} %6-110-8

\twolineshloka
{चतुर्भिश्चतुरो दीप्तान् हयान् प्रत्यपसर्पयत्}
{स क्रोधवशमापन्नो हयानामपसर्पणे} %6-110-9

\twolineshloka
{मुमोच निशितान् बाणान् राघवाय दशाननः}
{सोऽतिविद्धो बलवता दशग्रीवेण राघवः} %6-110-10

\twolineshloka
{जगाम न विकारं च न चापि व्यथितोऽभवत्}
{चिक्षेप च पुनर्बाणान् वज्रसारसमस्वनान्} %6-110-11

\twolineshloka
{सारथिं वज्रहस्तस्य समुद्दिश्य दशाननः}
{मातलेस्तु महावेगाः शरीरे पतिताः शराः} %6-110-12

\twolineshloka
{न सूक्ष्ममपि सम्मोहं व्यथां वा प्रददुर्युधि}
{तया धर्षणया क्रुद्धो मातलेर्न तथाऽऽत्मनः} %6-110-13

\twolineshloka
{चकार शरजालेन राघवो विमुखं रिपुम्}
{विंशतिं त्रिंशतिं षष्टिं शतशोऽथ सहस्रशः} %6-110-14

\twolineshloka
{मुमोच राघवो वीरः सायकान् स्यन्दने रिपोः}
{रावणोऽपि ततः क्रुद्धो रथस्थो राक्षसेश्वरः} %6-110-15

\twolineshloka
{गदामुसलवर्षेण रामं प्रत्यर्दयद् रणे}
{तत् प्रवृत्तं पुनर्युद्धं तुमुलं रोमहर्षणम्} %6-110-16

\twolineshloka
{गदानां मुसलानां च परिघाणां च निःस्वनैः}
{शराणां पुङ्खवातैश्च क्षुभिताः सप्त सागराः} %6-110-17

\twolineshloka
{क्षुब्धानां सागराणां च पातालतलवासिनः}
{व्यथिता दानवाः सर्वे पन्नगाश्च सहस्रशः} %6-110-18

\twolineshloka
{चकम्पे मेदिनी कृत्स्ना सशैलवनकानना}
{भास्करो निष्प्रभश्चासीन्न ववौ चापि मारुतः} %6-110-19

\twolineshloka
{ततो देवाः सगन्धर्वाः सिद्धाश्च परमर्षयः}
{चिन्तामापेदिरे सर्वे सकिन्नरमहोरगाः} %6-110-20

\twolineshloka
{स्वस्ति गोब्राह्मणेभ्यस्तु लोकास्तिष्ठन्तु शाश्वताः}
{जयतां राघवः सङ्ख्ये रावणं राक्षसेश्वरम्} %6-110-21

\twolineshloka
{एवं जपन्तोऽपश्यंस्ते देवाः सर्षिगणास्तदा}
{रामरावणयोर्युद्धं सुघोरं रोमहर्षणम्} %6-110-22

\twolineshloka
{गन्धर्वाप्सरसां सङ्घा दृष्ट्वा युद्धमनूपमम्}
{गगनं गगनाकारं सागरः सागरोपमः} %6-110-23

\twolineshloka
{रामरावणयोर्युद्धं रामरावणयोरिव}
{एवं ब्रुवन्तो ददृशुस्तद् युद्धं रामरावणम्} %6-110-24

\twolineshloka
{ततः क्रोधान्महाबाहू रघूणां कीर्तिवर्धनः}
{सन्धाय धनुषा रामः शरमाशीविषोपमम्} %6-110-25

\twolineshloka
{रावणस्य शिरोऽच्छिन्दच्छ्रीमज्ज्वलितकुण्डलम्}
{तच्छिरः पतितं भूमौ दृष्टं लोकैस्त्रिभिस्तदा} %6-110-26

\twolineshloka
{तस्यैव सदृशं चान्यद् रावणस्योत्थितं शिरः}
{तत् क्षिप्तं क्षिप्रहस्तेन रामेण क्षिप्रकारिणा} %6-110-27

\twolineshloka
{द्वितीयं रावणशिरश्छिन्नं संयति सायकैः}
{छिन्नमात्रं च तच्छीर्षं पुनरेव प्रदृश्यते} %6-110-28

\twolineshloka
{तदप्यशनिसङ्काशैश्छिन्नं रामस्य सायकैः}
{एवमेव शतं छिन्नं शिरसां तुल्यवर्चसाम्} %6-110-29

\twolineshloka
{न चैव रावणस्यान्तो दृश्यते जीवितक्षये}
{ततः सर्वास्त्रविद् वीरः कौसल्यानन्दवर्धनः} %6-110-30

\twolineshloka
{मार्गणैर्बहुभिर्युक्तश्चिन्तयामास राघवः}
{मारीचो निहतो यैस्तु खरो यैस्तु सदूषणः} %6-110-31

\twolineshloka
{क्रौञ्चावटे विराधस्तु कबन्धो दण्डकावने}
{यैः साला गिरयो भग्ना वाली च क्षुभितोऽम्बुधिः} %6-110-32

\twolineshloka
{त इमे सायकाः सर्वे युद्धे प्रात्ययिका मम}
{किं नु तत् कारणं येन रावणे मन्दतेजसः} %6-110-33

\twolineshloka
{इति चिन्तापरश्चासीदप्रमत्तश्च संयुगे}
{ववर्ष शरवर्षाणि राघवो रावणोरसि} %6-110-34

\twolineshloka
{रावणोऽपि ततः क्रुद्धो रथस्थो राक्षसेश्वरः}
{गदामुसलवर्षेण रामं प्रत्यर्दयद् रणे} %6-110-35

\twolineshloka
{तत् प्रवृत्तं महद् युद्धं तुमुलं रोमहर्षणम्}
{अन्तरिक्षे च भूमौ च पुनश्च गिरिमूर्धनि} %6-110-36

\twolineshloka
{देवदानवयक्षाणां पिशाचोरगरक्षसाम्}
{पश्यतां तन्महद् युद्धं सर्वरात्रमवर्तत} %6-110-37

\twolineshloka
{नैव रात्रिं न दिवसं न मुहूर्तं न च क्षणम्}
{रामरावणयोर्युद्धं विराममुपगच्छति} %6-110-38

\twolineshloka
{दशरथसुतराक्षसेन्द्रयोस्तयोर्जयमनवेक्ष्य रणे स राघवस्य}
{सुरवररथसारथिर्महात्मा रणरतराममुवाच वाक्यमाशु} %6-110-39


॥इत्यार्षे श्रीमद्रामायणे वाल्मीकीये आदिकाव्ये युद्धकाण्डे रावणैकशतशिरश्छेदनम् नाम दशाधिकशततमः सर्गः ॥६-११०॥
