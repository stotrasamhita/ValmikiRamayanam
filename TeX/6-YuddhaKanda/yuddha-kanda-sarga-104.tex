\sect{चतुरधिकशततमः सर्गः — रावणशूलभङ्गः}

\threelineshloka
{जगाम सुमहाक्रोधं निर्दहन्निव राक्षसान्}
{तस्य क्रुद्धस्य वदनं दृष्ट्वा रामस्य धीमतः}
{सर्वभूतानि वित्रेसुः प्राकम्पत च मेदिनी} %6-104-1

\twolineshloka
{सिंहशार्दूलवाञ्छैलः संचचाल चलद् द्रुमः}
{बभूव चापि क्षुभितः समुद्रः सरितां पतिः} %6-104-2

\twolineshloka
{खराश्च खरनिर्घोषा गगने परुषा घनाः}
{औत्पातिकाश्च नर्दन्तः समन्तात् परिचक्रमुः} %6-104-3

\twolineshloka
{रामं दृष्ट्वा सुसंक्रुद्धमुत्पातांश्चैव दारुणान्}
{वित्रेसुः सर्वभूतानि रावणस्याभवद् भयम्} %6-104-4

\twolineshloka
{विमानस्थास्तदा देवा गन्धर्वाश्च महोरगाः}
{ऋषिदानवदैत्याश्च गरुत्मन्तश्च खेचराः} %6-104-5

\twolineshloka
{ददृशुस्ते तदा युद्धं लोकसंवर्तसंस्थितम्}
{नानाप्रहरणैर्भीमैः शूरयोः सम्प्रयुध्यतोः} %6-104-6

\twolineshloka
{ऊचुः सुरासुराः सर्वे तदा विग्रहमागताः}
{प्रेक्षमाणा महायुद्धं वाक्यं भक्त्या प्रहृष्टवत्} %6-104-7

\twolineshloka
{दशग्रीवं जयेत्याहुरसुराः समवस्थिताः}
{देवा राममथोचुस्ते त्वं जयेति पुनः पुनः} %6-104-8

\twolineshloka
{एतस्मिन्नन्तरे क्रोधाद् राघवस्य च रावणः}
{प्रहर्तुकामो दुष्टात्मा स्पृशन् प्रहरणं महत्} %6-104-9

\twolineshloka
{वज्रसारं महानादं सर्वशत्रुनिबर्हणम्}
{शैलशृङ्गनिभैः कूटैश्चित्तदृष्टिभयावहम्} %6-104-10

\twolineshloka
{सधूममिव तीक्ष्णाग्रं युगान्ताग्निचयोपमम्}
{अतिरौद्रमनासाद्यं कालेनापि दुरासदम्} %6-104-11

\twolineshloka
{त्रासनं सर्वभूतानां दारणं भेदनं तथा}
{प्रदीप्त इव रोषेण शूलं जग्राह रावणः} %6-104-12

\twolineshloka
{तच्छूलं परमक्रुद्धो जग्राह युधि वीर्यवान्}
{अनीकैः समरे शूरै राक्षसैः परिवारितः} %6-104-13

\twolineshloka
{समुद्यम्य महाकायो ननाद युधि भैरवम्}
{संरक्तनयनो रोषात् स्वसैन्यमभिहर्षयन्} %6-104-14

\twolineshloka
{पृथिवीं चान्तरिक्षं च दिशश्च प्रदिशस्तथा}
{प्राकम्पयत् तदा शब्दो राक्षसेन्द्रस्य दारुणः} %6-104-15

\twolineshloka
{अतिकायस्य नादेन तेन तस्य दुरात्मनः}
{सर्वभूतानि वित्रेसुः सागरश्च प्रचुक्षुभे} %6-104-16

\twolineshloka
{स गृहीत्वा महावीर्यः शूलं तद् रावणो महत्}
{विनद्य सुमहानादं रामं परुषमब्रवीत्} %6-104-17

\twolineshloka
{शूलोऽयं वज्रसारस्ते राम रोषान्मयोद्यतः}
{तव भ्रातृसहायस्य सद्यः प्राणान् हरिष्यति} %6-104-18

\twolineshloka
{रक्षसामद्य शूराणां निहतानां चमूमुखे}
{त्वां निहत्य रणश्लाघिन् करोमि तरसा समम्} %6-104-19

\twolineshloka
{तिष्ठेदानीं निहन्मि त्वामेष शूलेन राघव}
{एवमुक्त्वा स चिक्षेप तच्छूलं राक्षसाधिपः} %6-104-20

\twolineshloka
{तद् रावणकरान्मुक्तं विद्युन्मालासमावृतम्}
{अष्टघण्टं महानादं वियद्गतमशोभत} %6-104-21

\twolineshloka
{तच्छूलं राघवो दृष्ट्वा ज्वलन्तं घोरदर्शनम्}
{ससर्ज विशिखान् रामश्चापमायम्य वीर्यवान्} %6-104-22

\twolineshloka
{आपतन्तं शरौघेण वारयामास राघवः}
{उत्पतन्तं युगान्ताग्निं जलौघैरिव वासवः} %6-104-23

\twolineshloka
{निर्ददाह स तान् बाणान् रामकार्मुकनिःसृतान्}
{रावणस्य महान् शूलः पतङ्गानिव पावकः} %6-104-24

\twolineshloka
{तान् दृष्ट्वा भस्मसाद्भूतान् शूलसंस्पर्शचूर्णितान्}
{सायकानन्तरिक्षस्थान् राघवः क्रोधमाहरत्} %6-104-25

\twolineshloka
{स तां मातलिना नीतां शक्तिं वासवसम्मताम्}
{जग्राह परमक्रुद्धो राघवो रघुनन्दनः} %6-104-26

\twolineshloka
{सा तोलिता बलवता शक्तिर्घण्टाकृतस्वना}
{नभः प्रज्वालयामास युगान्तोल्केव सप्रभा} %6-104-27

\twolineshloka
{सा क्षिप्ता राक्षसेन्द्रस्य तस्मिञ्छूले पपात ह}
{भिन्नः शक्त्या महान् शूलो निपपात गतद्युतिः} %6-104-28

\twolineshloka
{निर्बिभेद ततो बाणैर्हयानस्य महाजवान्}
{रामस्तीक्ष्णैर्महावेगैर्वज्रकल्पैरजिह्मगैः} %6-104-29

\twolineshloka
{निर्बिभेदोरसि तदा रावणं निशितैः शरैः}
{राघवः परमायत्तो ललाटे पत्त्रिभिस्त्रिभिः} %6-104-30

\twolineshloka
{स शरैर्भिन्नसर्वाङ्गो गात्रप्रस्रुतशोणितः}
{राक्षसेन्द्रः समूहस्थः फुल्लाशोक इवाबभौ} %6-104-31

\twolineshloka
{स रामबाणैरतिविद्धगात्रो निशाचरेन्द्रः क्षतजार्द्रगात्रः}
{जगाम खेदं च समाजमध्ये क्रोधं च चक्रे सुभृशं तदानीम्} %6-104-32


॥इत्यार्षे श्रीमद्रामायणे वाल्मीकीये आदिकाव्ये युद्धकाण्डे रावणशूलभङ्गः नाम चतुरधिकशततमः सर्गः ॥६-१०४॥
