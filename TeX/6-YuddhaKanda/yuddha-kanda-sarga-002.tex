\sect{द्वितीयः सर्गः — रामप्रोत्साहनम्}

\twolineshloka
{तं तु शोकपरिद्यूनं रामं दशरथात्मजम्}
{उवाच वचनं श्रीमान् सुग्रीवः शोकनाशनम्} %6-2-1

\twolineshloka
{किं त्वया तप्यते वीर यथान्यः प्राकृतस्तथा}
{मैवं भूस्त्यज संतापं कृतघ्न इव सौहृदम्} %6-2-2

\twolineshloka
{संतापस्य च ते स्थानं नहि पश्यामि राघव}
{प्रवृत्तावुपलब्धायां ज्ञाते च निलये रिपोः} %6-2-3

\twolineshloka
{मतिमान् शास्त्रवित् प्राज्ञः पण्डितश्चासि राघव}
{त्यजेमां प्राकृतां बुद्धिं कृतात्मेवार्थदूषिणीम्} %6-2-4

\twolineshloka
{समुद्रं लङ्घयित्वा तु महानक्रसमाकुलम्}
{लङ्कामारोहयिष्यामो हनिष्यामश्च ते रिपुम्} %6-2-5

\twolineshloka
{निरुत्साहस्य दीनस्य शोकपर्याकुलात्मनः}
{सर्वार्था व्यवसीदन्ति व्यसनं चाधिगच्छति} %6-2-6

\threelineshloka
{इमे शूराः समर्थाश्च सर्वतो हरियूथपाः}
{त्वत्प्रियार्थं कृतोत्साहाः प्रवेष्टुमपि पावकम्}
{एषां हर्षेण जानामि तर्कश्चापि दृढो मम} %6-2-7

\twolineshloka
{विक्रमेण समानेष्ये सीतां हत्वा यथा रिपुम्}
{रावणं पापकर्माणं तथा त्वं कर्तुमर्हसि} %6-2-8

\twolineshloka
{सेतुरत्र यथा बद्ध्येद् यथा पश्येम तां पुरीम्}
{तस्य राक्षसराजस्य तथा त्वं कुरु राघव} %6-2-9

\twolineshloka
{दृष्ट्वा तां हि पुरीं लङ्कां त्रिकूटशिखरे स्थिताम्}
{हतं च रावणं युद्धे दर्शनादवधारय} %6-2-10

\twolineshloka
{अबद्ध्वा सागरे सेतुं घोरे च वरुणालये}
{लङ्कां न मर्दितुं शक्या सेन्द्रैरपि सुरासुरैः} %6-2-11

\threelineshloka
{सेतुबन्धः समुद्रे च यावल्लङ्कासमीपतः}
{सर्वं तीर्णं च मे सैन्यं जितमित्युपधारय}
{इमे हि समरे वीरा हरयः कामरूपिणः} %6-2-12

\twolineshloka
{तदलं विक्लवां बुद्धिं राजन् सर्वार्थनाशिनीम्}
{पुरुषस्य हि लोकेऽस्मिन् शोकः शौर्यापकर्षणः} %6-2-13

\twolineshloka
{यत् तु कार्यं मनुष्येण शौटीर्यमवलम्ब्यताम्}
{तदलंकरणायैव कर्तुर्भवति सत्वरम्} %6-2-14

\threelineshloka
{अस्मिन् काले महाप्राज्ञ सत्त्वमातिष्ठ तेजसा}
{शूराणां हि मनुष्याणां त्वद्विधानां महात्मनाम्}
{विनष्टे वा प्रणष्टे वा शोकः सर्वार्थनाशनः} %6-2-15

\twolineshloka
{तत्त्वं बुद्धिमतां श्रेष्ठः सर्वशास्त्रार्थकोविदः}
{मद्विधैः सचिवैः सार्धमरिं जेतुं समर्हसि} %6-2-16

\twolineshloka
{नहि पश्याम्यहं कंचित् त्रिषु लोकेषु राघव}
{गृहीतधनुषो यस्ते तिष्ठेदभिमुखो रणे} %6-2-17

\twolineshloka
{वानरेषु समासक्तं न ते कार्यं विपत्स्यते}
{अचिराद् द्रक्ष्यसे सीतां तीर्त्वा सागरमक्षयम्} %6-2-18

\twolineshloka
{तदलं शोकमालम्ब्य क्रोधमालम्ब भूपते}
{निश्चेष्टाः क्षत्रिया मन्दाः सर्वे चण्डस्य बिभ्यति} %6-2-19

\twolineshloka
{लङ्घनार्थं च घोरस्य समुद्रस्य नदीपतेः}
{सहास्माभिरिहोपेतः सूक्ष्मबुद्धिर्विचारय} %6-2-20

\twolineshloka
{लङ्घिते तत्र तैः सैन्यैर्जितमित्येव निश्चिनु}
{सर्वं तीर्णं च मे सैन्यं जितमित्यवधार्यताम्} %6-2-21

\twolineshloka
{इमे हि हरयः शूराः समरे कामरूपिणः}
{तानरीन् विधमिष्यन्ति शिलापादपवृष्टिभिः} %6-2-22

\twolineshloka
{कथंचित् परिपश्यामि लङ्घितं वरुणालयम्}
{हतमित्येव तं मन्ये युद्धे शत्रुनिबर्हण} %6-2-23

\twolineshloka
{किमुक्त्वा बहुधा चापि सर्वथा विजयी भवान्}
{निमित्तानि च पश्यामि मनो मे सम्प्रहृष्यति} %6-2-24


॥इत्यार्षे श्रीमद्रामायणे वाल्मीकीये आदिकाव्ये युद्धकाण्डे रामप्रोत्साहनम् नाम द्वितीयः सर्गः ॥६-२॥
