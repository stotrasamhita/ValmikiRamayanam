\sect{एकपञ्चाशः सर्गः — धूम्राक्षाभिषेणनम्}

\twolineshloka
{तेषां तु तुमुलं शब्दं वानराणां महौजसाम्}
{नर्दतां राक्षसैः सार्धं तदा शुश्राव रावणः} %6-51-1

\twolineshloka
{स्निग्धगम्भीरनिर्घोषं श्रुत्वा तं निनदं भृशम्}
{सचिवानां ततस्तेषां मध्ये वचनमब्रवीत्} %6-51-2

\twolineshloka
{यथासौ सम्प्रहृष्टानां वानराणामुपस्थितः}
{बहूनां सुमहान् नादो मेघानामिव गर्जताम्} %6-51-3

\twolineshloka
{सुव्यक्तं महती प्रीतिरेतेषां नात्र संशयः}
{तथाहि विपुलैर्नादैश्चुक्षुभे लवणार्णवः} %6-51-4

\twolineshloka
{तौ तु बद्धौ शरैस्तीक्ष्णैर्भ्रातरौ रामलक्ष्मणौ}
{अयं च सुमहान् नादः शङ्कां जनयतीव मे} %6-51-5

\twolineshloka
{एवं च वचनं चोक्त्वा मन्त्रिणो राक्षसेश्वरः}
{उवाच नैर्ऋतांस्तत्र समीपपरिवर्तिनः} %6-51-6

\twolineshloka
{ज्ञायतां तूर्णमेतेषां सर्वेषां च वनौकसाम्}
{शोककाले समुत्पन्ने हर्षकारणमुत्थितम्} %6-51-7

\twolineshloka
{तथोक्तास्ते सुसम्भ्रान्ताः प्राकारमधिरुह्य च}
{ददृशुः पालितां सेनां सुग्रीवेण महात्मना} %6-51-8

\twolineshloka
{तौ च मुक्तौ सुघोरेण शरबन्धेन राघवौ}
{समुत्थितौ महाभागौ विषेदुः सर्वराक्षसाः} %6-51-9

\twolineshloka
{सन्त्रस्तहृदयाः सर्वे प्राकारादवरुह्य ते}
{विवर्णा राक्षसा घोरा राक्षसेन्द्रमुपस्थिताः} %6-51-10

\twolineshloka
{तदप्रियं दीनमुखा रावणस्य च राक्षसाः}
{कृत्स्नं निवेदयामासुर्यथावद् वाक्यकोविदाः} %6-51-11

\twolineshloka
{यौ ताविन्द्रजिता युद्धे भ्रातरौ रामलक्ष्मणौ}
{निबद्धौ शरबन्धेन निष्प्रकम्पभुजौ कृतौ} %6-51-12

\twolineshloka
{विमुक्तौ शरबन्धेन दृश्येते तौ रणाजिरे}
{पाशानिव गजौ छित्त्वा गजेन्द्रसमविक्रमौ} %6-51-13

\twolineshloka
{तच्छ्रुत्वा वचनं तेषां राक्षसेन्द्रो महाबलः}
{चिन्ताशोकसमाक्रान्तो विवर्णवदनोऽभवत्} %6-51-14

\twolineshloka
{घोरैर्दत्तवरैर्बद्धौ शरैराशीविषोपमैः}
{अमोघैः सूर्यसङ्काशैः प्रमथ्येन्द्रजिता युधि} %6-51-15

\twolineshloka
{तदस्त्रबन्धमासाद्य यदि मुक्तौ रिपू मम}
{संशयस्थमिदं सर्वमनुपश्याम्यहं बलम्} %6-51-16

\twolineshloka
{निष्फलाः खलु संवृत्ताः शराः पावकतेजसः}
{आदत्तं यैस्तु सङ्ग्रामे रिपूणां जीवितं मम} %6-51-17

\twolineshloka
{एवमुक्त्वा तु सङ्क्रुद्धो निःश्वसन्नुरगो यथा}
{अब्रवीद् रक्षसां मध्ये धूम्राक्षं नाम राक्षसम्} %6-51-18

\twolineshloka
{बलेन महता युक्तो रक्षसां भीमविक्रम}
{त्वं वधायाशु निर्याहि रामस्य सह वानरैः} %6-51-19

\twolineshloka
{एवमुक्तस्तु धूम्राक्षो राक्षसेन्द्रेण धीमता}
{परिक्रम्य ततः शीघ्रं निर्जगाम नृपालयात्} %6-51-20

\twolineshloka
{अभिनिष्क्रम्य तद् द्वारं बलाध्यक्षमुवाच ह}
{त्वरयस्व बलं शीघ्रं किं चिरेण युयुत्सतः} %6-51-21

\twolineshloka
{धूम्राक्षवचनं श्रुत्वा बलाध्यक्षो बलानुगः}
{बलमुद्योजयामास रावणस्याज्ञया भृशम्} %6-51-22

\twolineshloka
{ते बद्धघण्टा बलिनो घोररूपा निशाचराः}
{विनद्यमानाः संहृष्टा धूम्राक्षं पर्यवारयन्} %6-51-23

\twolineshloka
{विविधायुधहस्ताश्च शूलमुद्गरपाणयः}
{गदाभिः पट्टिशैर्दण्डैरायसैर्मुसलैरपि} %6-51-24

\twolineshloka
{परिघैर्भिन्दिपालैश्च भल्लैः पाशैः परश्वधैः}
{निर्ययू राक्षसा घोरा नर्दन्तो जलदा यथा} %6-51-25

\twolineshloka
{रथैः कवचिनस्त्वन्ये ध्वजैश्च समलङ्कृतैः}
{सुवर्णजालविहितैः खरैश्च विविधाननैः} %6-51-26

\twolineshloka
{हयैः परमशीघ्रैश्च गजैश्चैव मदोत्कटैः}
{निर्ययुर्नैर्ऋतव्याघ्रा व्याघ्रा इव दुरासदाः} %6-51-27

\twolineshloka
{वृकसिंहमुखैर्युक्तं खरैः कनकभूषितैः}
{आरुरोह रथं दिव्यं धूम्राक्षः खरनिःस्वनः} %6-51-28

\twolineshloka
{स निर्यातो महावीर्यो धूम्राक्षो राक्षसैर्वृतः}
{हसन् वै पश्चिमद्वाराद्धनूमान् यत्र तिष्ठति} %6-51-29

\twolineshloka
{रथप्रवरमास्थाय खरयुक्तं खरस्वनम्}
{प्रयान्तं तु महाघोरं राक्षसं भीमदर्शनम्} %6-51-30

\twolineshloka
{अन्तरिक्षगताः क्रूराः शकुनाः प्रत्यषेधयन्}
{रथशीर्षे महाभीमो गृध्रश्च निपपात ह} %6-51-31

\twolineshloka
{ध्वजाग्रे ग्रथिताश्चैव निपेतुः कुणपाशनाः}
{रुधिरार्द्रो महान् श्वेतः कबन्धः पतितो भुवि} %6-51-32

\twolineshloka
{विस्वरं चोत्सृजन्नादान् धूम्राक्षस्य निपातितः}
{ववर्ष रुधिरं देवः सञ्चचाल च मेदिनी} %6-51-33

\twolineshloka
{प्रतिलोमं ववौ वायुर्निर्घातसमनिःस्वनः}
{तिमिरौघावृतास्तत्र दिशश्च न चकाशिरे} %6-51-34

\threelineshloka
{स तूत्पातांस्ततो दृष्ट्वा राक्षसानां भयावहान्}
{प्रादुर्भूतान् सुघोरांश्च धूम्राक्षो व्यथितोऽभवत्}
{मुमुहू राक्षसाः सर्वे धूम्राक्षस्य पुरःसराः} %6-51-35

\twolineshloka
{ततः सुभीमो बहुभिर्निशाचरैर्वृतोऽभिनिष्क्रम्य रणोत्सुको बली}
{ददर्श तां राघवबाहुपालितां महौघकल्पां बहु वानरीं चमूम्} %6-51-36


॥इत्यार्षे श्रीमद्रामायणे वाल्मीकीये आदिकाव्ये युद्धकाण्डे धूम्राक्षाभिषेणनम् नाम एकपञ्चाशः सर्गः ॥६-५१॥
