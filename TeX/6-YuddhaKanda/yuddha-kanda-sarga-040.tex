\sect{चत्वारिंशः सर्गः — रावणसुग्रीवनियुद्धम्}

\twolineshloka
{ततो रामः सुवेलाग्रं योजनद्वयमण्डलम्}
{उपारोहत् ससुग्रीवो हरियूथैः समन्वितः} %6-40-1

\twolineshloka
{स्थित्वा मुहूर्तं तत्रैव दिशो दश विलोकयन्}
{त्रिकूटशिखरे रम्ये निर्मितां विश्वकर्मणा} %6-40-2

\twolineshloka
{ददर्श लङ्कां सुन्यस्तां रम्यकाननशोभिताम्}
{तस्य गोपुरशृङ्गस्थं राक्षसेन्द्रं दुरासदम्} %6-40-3

\twolineshloka
{श्वेतचामरपर्यन्तं विजयच्छत्रशोभितम्}
{रक्तचन्दनसंलिप्तं रत्नाभरणभूषितम्} %6-40-4

\twolineshloka
{नीलजीमूतसङ्काशं हेमसञ्छादिताम्बरम्}
{ऐरावतविषाणाग्रैरुत्कृष्टकिणवक्षसम्} %6-40-5

\twolineshloka
{शशलोहितरागेण संवीतं रक्तवाससा}
{सन्ध्यातपेन सञ्छन्नं मेघराशिमिवाम्बरे} %6-40-6

\twolineshloka
{पश्यतां वानरेन्द्राणां राघवस्यापि पश्यतः}
{दर्शनाद् राक्षसेन्द्रस्य सुग्रीवः सहसोत्थितः} %6-40-7

\twolineshloka
{क्रोधवेगेन संयुक्तः सत्त्वेन च बलेन च}
{अचलाग्रादथोत्थाय पुप्लुवे गोपुरस्थले} %6-40-8

\twolineshloka
{स्थित्वा मुहूर्तं सम्प्रेक्ष्य निर्भयेनान्तरात्मना}
{तृणीकृत्य च तद् रक्षः सोऽब्रवीत् परुषं वचः} %6-40-9

\twolineshloka
{लोकनाथस्य रामस्य सखा दासोऽस्मि राक्षस}
{न मया मोक्ष्यसेऽद्य त्वं पार्थिवेन्द्रस्य तेजसा} %6-40-10

\twolineshloka
{इत्युक्त्वा सहसोत्पत्य पुप्लुवे तस्य चोपरि}
{आकृष्य मुकुटं चित्रं पातयामास तद् भुवि} %6-40-11

\twolineshloka
{समीक्ष्य तूर्णमायान्तं बभाषे तं निशाचरः}
{सुग्रीवस्त्वं परोक्षं मे हीनग्रीवो भविष्यसि} %6-40-12

\twolineshloka
{इत्युक्त्वोत्थाय तं क्षिप्रं बाहुभ्यामाक्षिपत् तले}
{कन्दुवत् स समुत्थाय बाहुभ्यामाक्षिपद्धरिः} %6-40-13

\twolineshloka
{परस्परं स्वेदविदिग्धगात्रौ परस्परं शोणितरक्तदेहौ}
{परस्परं श्लिष्टनिरुद्धचेष्टौ परस्परं शाल्मलिकिंशुकाविव} %6-40-14

\twolineshloka
{मुष्टिप्रहारैश्च तलप्रहारैररत्निघातैश्च कराग्रघातैः}
{तौ चक्रतुर्युद्धमसह्यरूपं महाबलौ राक्षसवानरेन्द्रौ} %6-40-15

\twolineshloka
{कृत्वा नियुद्धं भृशमुग्रवेगौ कालं चिरं गोपुरवेदिमध्ये}
{उत्क्षिप्य चोत्क्षिप्य विनम्य देहौ पादक्रमाद् गोपुरवेदिलग्नौ} %6-40-16

\twolineshloka
{अन्योन्यमापीड्य विलग्नदेहौ तौ पेततुः सालनिखातमध्ये}
{उत्पेततुर्भूमितलं स्पृशन्तौ स्थित्वा मुहूर्तं त्वभिनिःश्वसन्तौ} %6-40-17

\twolineshloka
{आलिङ्ग्य चालिङ्ग्य च बाहुयोक्त्रैः संयोजयामासतुराहवे तौ}
{संरम्भशिक्षाबलसम्प्रयुक्तौ सुचेरतुः सम्प्रति युद्धमार्गैः} %6-40-18

\twolineshloka
{शार्दूलसिंहाविव जातदंष्ट्रौ गजेन्द्रपोताविव सम्प्रयुक्तौ}
{संहत्य संवेद्य च तौ कराभ्यां तौ पेततुर्वै युगपद् धरायाम्} %6-40-19

\twolineshloka
{उद्यम्य चान्योन्यमधिक्षिपन्तौ सञ्चक्रमाते बहु युद्धमार्गे}
{व्यायामशिक्षाबलसम्प्रयुक्तौ क्लमं न तौ जग्मतुराशु वीरौ} %6-40-20

\twolineshloka
{बाहूत्तमैर्वारणवारणाभैर्निवारयन्तौ परवारणाभौ}
{चिरेण कालेन भृशं प्रयुद्धौ सञ्चेरतुर्मण्डलमार्गमाशु} %6-40-21

\twolineshloka
{तौ परस्परमासाद्य यत्तावन्योन्यसूदने}
{मार्जाराविव भक्षार्थेऽवतस्थाते मुहुर्मुहुः} %6-40-22

\twolineshloka
{मण्डलानि विचित्राणि स्थानानि विविधानि च}
{गोमूत्रकाणि चित्राणि गतप्रत्यागतानि च} %6-40-23

\twolineshloka
{तिरश्चीनगतान्येव तथा वक्रगतानि च}
{परिमोक्षं प्रहाराणां वर्जनं परिधावनम्} %6-40-24

\twolineshloka
{अभिद्रवणमाप्लावमवस्थानं सविग्रहम्}
{परावृत्तमपावृत्तमपद्रुतमवप्लुतम्} %6-40-25

\twolineshloka
{उपन्यस्तमपन्यस्तं युद्धमार्गविशारदौ}
{तौ विचेरतुरन्योन्यं वानरेन्द्रश्च रावणः} %6-40-26

\twolineshloka
{एतस्मिन्नन्तरे रक्षो मायाबलमथात्मनः}
{आरब्धुमुपसम्पेदे ज्ञात्वा तं वानराधिपः} %6-40-27

\twolineshloka
{उत्पपात तदाऽऽकाशं जितकाशी जितक्लमः}
{रावणः स्थित एवात्र हरिराजेन वञ्चितः} %6-40-28

\twolineshloka
{अथ हरिवरनाथः प्राप्तसङ्ग्रामकीर्तिर्निशिचरपतिमाजौ योजयित्वा श्रमेण}
{गगनमतिविशालं लङ्घयित्वार्कसूनुर्हरिगणबलमध्ये रामपार्श्वं जगाम} %6-40-29

\twolineshloka
{इति स सवितृसूनुस्तत्र तत् कर्म कृत्वा पवनगतिरनीकं प्राविशत् सम्प्रहृष्टः}
{रघुवरनृपसूनोर्वर्धयन् युद्धहर्षं तरुमृगगणमुख्यैः पूज्यमानो हरीन्द्रः} %6-40-30


॥इत्यार्षे श्रीमद्रामायणे वाल्मीकीये आदिकाव्ये युद्धकाण्डे रावणसुग्रीवनियुद्धम् नाम चत्वारिंशः सर्गः ॥६-४०॥
