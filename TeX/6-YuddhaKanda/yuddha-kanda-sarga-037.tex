\sect{सप्तत्रिंशः सर्गः — रामगुल्मविभागः}

\twolineshloka
{नरवानरराजानौ स तु वायुसुतः कपिः}
{जाम्बवानृक्षराजश्च राक्षसश्च विभीषणः} %6-37-1

\twolineshloka
{अङ्गदो वालिपुत्रश्च सौमित्रिः शरभः कपिः}
{सुषेणः सहदायादो मैन्दो द्विविद एव च} %6-37-2

\twolineshloka
{गजो गवाक्षः कुमुदो नलोऽथ पनसस्तथा}
{अमित्रविषयं प्राप्ताः समवेताः समर्थयन्} %6-37-3

\twolineshloka
{इयं सा लक्ष्यते लङ्का पुरी रावणपालिता}
{सासुरोरगगन्धर्वैरमरैरपि दुर्जया} %6-37-4

\twolineshloka
{कार्यसिद्धिं पुरस्कृत्य मन्त्रयध्वं विनिर्णये}
{नित्यं सन्निहितो यत्र रावणो राक्षसाधिपः} %6-37-5

\twolineshloka
{अथ तेषु ब्रुवाणेषु रावणावरजोऽब्रवीत्}
{वाक्यमग्राम्यपदवत् पुष्कलार्थं विभीषणः} %6-37-6

\twolineshloka
{अनलः पनसश्चैव सम्पातिः प्रमतिस्तथा}
{गत्वा लङ्कां ममामात्याः पुरीं पुनरिहागताः} %6-37-7

\twolineshloka
{भूत्वा शकुनयः सर्वे प्रविष्टाश्च रिपोर्बलम्}
{विधानं विहितं यच्च तद् दृष्ट्वा समुपस्थिताः} %6-37-8

\twolineshloka
{संविधानं यथाहुस्ते रावणस्य दुरात्मनः}
{राम तद् ब्रुवतः सर्वं याथातथ्येन मे शृणु} %6-37-9

\twolineshloka
{पूर्वं प्रहस्तः सबलो द्वारमासाद्य तिष्ठति}
{दक्षिणं च महावीर्यौ महापार्श्वमहोदरौ} %6-37-10

\twolineshloka
{इन्द्रजित् पश्चिमं द्वारं राक्षसैर्बहुभिर्वृतः}
{पट्टिशासिधनुष्मद्भिः शूलमुद्गरपाणिभिः} %6-37-11

\twolineshloka
{नानाप्रहरणैः शूरैरावृतो रावणात्मजः}
{राक्षसानां सहस्रैस्तु बहुभिः शस्त्रपाणिभिः} %6-37-12

\twolineshloka
{युक्तः परमसंविग्नो राक्षसैः सह मन्त्रवित्}
{उत्तरं नगरद्वारं रावणः स्वयमास्थितः} %6-37-13

\twolineshloka
{विरूपाक्षस्तु महता शूलखड्गधनुष्मता}
{बलेन राक्षसैः सार्धं मध्यमं गुल्ममाश्रितः} %6-37-14

\twolineshloka
{एतानेवं विधान् गुल्माँल्लङ्कायां समुदीक्ष्य ते}
{मामका मन्त्रिणः सर्वे शीघ्रं पुनरिहागताः} %6-37-15

\twolineshloka
{गजानां दशसाहस्रं रथानामयुतं तथा}
{हयानामयुते द्वे च साग्रकोटिश्च रक्षसाम्} %6-37-16

\twolineshloka
{विक्रान्ता बलवन्तश्च संयुगेष्वाततायिनः}
{इष्टा राक्षसराजस्य नित्यमेते निशाचराः} %6-37-17

\twolineshloka
{एकैकस्यात्र युद्धार्थे राक्षसस्य विशाम्पते}
{परीवारः सहस्राणां सहस्रमुपतिष्ठते} %6-37-18

\twolineshloka
{एतां प्रवृत्तिं लङ्कायां मन्त्रिप्रोक्तां विभीषणः}
{एवमुक्त्वा महाबाहू राक्षसांस्तानदर्शयत्} %6-37-19

\twolineshloka
{लङ्कायां सचिवैः सर्वं रामाय प्रत्यवेदयत्}
{रामं कमलपत्राक्षमिदमुत्तरमब्रवीत्} %6-37-20

\twolineshloka
{रावणावरजः श्रीमान् रामप्रियचिकीर्षया}
{कुबेरं तु यदा राम रावणः प्रतियुद्ध्यति} %6-37-21

\threelineshloka
{षष्टिः शतसहस्राणि तदा निर्यान्ति राक्षसाः}
{पराक्रमेण वीर्येण तेजसा सत्त्वगौरवात्}
{सदृशा ह्यत्र दर्पेण रावणस्य दुरात्मनः} %6-37-22

\twolineshloka
{अत्र मन्युर्न कर्तव्यः कोपये त्वां न भीषये}
{समर्थो ह्यसि वीर्येण सुराणामपि निग्रहे} %6-37-23

\twolineshloka
{तद्भवांश्चतुरङ्गेण बलेन महता वृतम्}
{व्यूह्येदं वानरानीकं निर्मथिष्यसि रावणम्} %6-37-24

\twolineshloka
{रावणावरजे वाक्यमेवं ब्रुवति राघवः}
{शत्रूणां प्रतिघातार्थमिदं वचनमब्रवीत्} %6-37-25

\twolineshloka
{पूर्वद्वारं तु लङ्काया नीलो वानरपुङ्गवः}
{प्रहस्तं प्रतियोद्धा स्याद् वानरैर्बहुभिर्वृतः} %6-37-26

\twolineshloka
{अङ्गदो वालिपुत्रस्तु बलेन महता वृतः}
{दक्षिणे बाधतां द्वारे महापार्श्वमहोदरौ} %6-37-27

\twolineshloka
{हनूमान् पश्चिमद्वारं निष्पीड्य पवनात्मजः}
{प्रविशत्वप्रमेयात्मा बहुभिः कपिभिर्वृतः} %6-37-28

\twolineshloka
{दैत्यदानवसङ्घानामृषीणां च महात्मनाम्}
{विप्रकारप्रियः क्षुद्रो वरदानबलान्वितः} %6-37-29

\twolineshloka
{परिक्रमति यः सर्वान् लोकान् सन्तापयन् प्रजाः}
{तस्याहं राक्षसेन्द्रस्य स्वयमेव वधे धृतः} %6-37-30

\twolineshloka
{उत्तरं नगरद्वारमहं सौमित्रिणा सह}
{निपीड्याभिप्रवेक्ष्यामि सबलो यत्र रावणः} %6-37-31

\twolineshloka
{वानरेन्द्रश्च बलवानृक्षराजश्च वीर्यवान्}
{राक्षसेन्द्रानुजश्चैव गुल्मे भवतु मध्यमे} %6-37-32

\twolineshloka
{न चैव मानुषं रूपं कार्यं हरिभिराहवे}
{एषा भवतु नः संज्ञा युद्धेऽस्मिन् वानरे बले} %6-37-33

\twolineshloka
{वानरा एव नश्चिह्नं स्वजनेऽस्मिन् भविष्यति}
{वयं तु मानुषेणैव सप्त योत्स्यामहे परान्} %6-37-34

\twolineshloka
{अहमेव सह भ्रात्रा लक्ष्मणेन महौजसा}
{आत्मना पञ्चमश्चायं सखा मम विभीषणः} %6-37-35

\threelineshloka
{स रामः कृत्यसिद्ध्यर्थमेवमुक्त्वा विभीषणम्}
{सुवेलारोहणे बुद्धिं चकार मतिमान् प्रभुः}
{रमणीयतरं दृष्ट्वा सुवेलस्य गिरेस्तटम्} %6-37-36

\twolineshloka
{ततस्तु रामो महता बलेन प्रच्छाद्य सर्वां पृथिवीं महात्मा}
{प्रहृष्टरूपोऽभिजगाम लङ्कां कृत्वा मतिं सोऽरिवधे महात्मा} %6-37-37


॥इत्यार्षे श्रीमद्रामायणे वाल्मीकीये आदिकाव्ये युद्धकाण्डे रामगुल्मविभागः नाम सप्तत्रिंशः सर्गः ॥६-३७॥
