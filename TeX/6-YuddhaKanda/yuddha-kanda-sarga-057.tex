\sect{सप्तपञ्चाशः सर्गः — प्रहस्तयुद्धम्}

\twolineshloka
{अकम्पनवधं श्रुत्वा क्रुद्धो वै राक्षसेश्वरः}
{किंचिद् दीनमुखश्चापि सचिवांस्तानुदैक्षत} %6-57-1

\threelineshloka
{स तु ध्यात्वा मुहूर्तं तु मन्त्रिभिः संविचार्य च}
{ततस्तु रावणः पूर्वदिवसे राक्षसाधिपः}
{पुरीं परिययौ लङ्कां सर्वान् गुल्मानवेक्षितुम्} %6-57-2

\twolineshloka
{तां राक्षसगणैर्गुप्तां गुल्मैर्बहुभिरावृताम्}
{ददर्श नगरीं राजा पताकाध्वजमालिनीम्} %6-57-3

\twolineshloka
{रुद्धां तु नगरीं दृष्ट्वा रावणो राक्षसेश्वरः}
{उवाचात्महितं काले प्रहस्तं युद्धकोविदम्} %6-57-4

\twolineshloka
{पुरस्योपनिविष्टस्य सहसा पीडितस्य ह}
{नान्ययुद्धात् प्रपश्यामि मोक्षं युद्धविशारद} %6-57-5

\twolineshloka
{अहं वा कुम्भकर्णो वा त्वं वा सेनापतिर्मम}
{इन्द्रजिद् वा निकुम्भो वा वहेयुर्भारमीदृशम्} %6-57-6

\twolineshloka
{स त्वं बलमतः शीघ्रमादाय परिगृह्य च}
{विजयायाभिनिर्याहि यत्र सर्वे वनौकसः} %6-57-7

\twolineshloka
{निर्याणादेव तूर्णं च चलिता हरिवाहिनी}
{नर्दतां राक्षसेन्द्राणां श्रुत्वा नादं द्रविष्यति} %6-57-8

\twolineshloka
{चपला ह्यविनीताश्च चलचित्ताश्च वानराः}
{न सहिष्यन्ति ते नादं सिंहनादमिव द्विपाः} %6-57-9

\twolineshloka
{विद्रुते च बले तस्मिन् रामः सौमित्रिणा सह}
{अवशस्ते निरालम्बः प्रहस्त वशमेष्यति} %6-57-10

\twolineshloka
{आपत्संशयिता श्रेयो नात्र निःसंशयीकृता}
{प्रतिलोमानुलोमं वा यत् तु नो मन्यसे हितम्} %6-57-11

\twolineshloka
{रावणेनैवमुक्तस्तु प्रहस्तो वाहिनीपतिः}
{राक्षसेन्द्रमुवाचेदमसुरेन्द्रमिवोशना} %6-57-12

\twolineshloka
{राजन् मन्त्रितपूर्वं नः कुशलैः सह मन्त्रिभिः}
{विवादश्चापि नो वृत्तः समवेक्ष्य परस्परम्} %6-57-13

\twolineshloka
{प्रदानेन तु सीतायाः श्रेयो व्यवसितं मया}
{अप्रदाने पुनर्युद्धं दृष्टमेव तथैव नः} %6-57-14

\twolineshloka
{सोऽहं दानैश्च मानैश्च सततं पूजितस्त्वया}
{सान्त्वैश्च विविधैः काले किं न कुर्यां हितं तव} %6-57-15

\twolineshloka
{नहि मे जीवितं रक्ष्यं पुत्रदारधनानि च}
{त्वं पश्य मां जुहूषन्तं त्वदर्थे जीवितं युधि} %6-57-16

\twolineshloka
{एवमुक्त्वा तु भर्तारं रावणं वाहिनीपतिः}
{उवाचेदं बलाध्यक्षान् प्रहस्तः पुरतः स्थितान्} %6-57-17

\twolineshloka
{समानयत मे शीघ्रं राक्षसानां महाबलम्}
{मद्बाणानां तु वेगेन हतानां च रणाजिरे} %6-57-18

\twolineshloka
{अद्य तृप्यन्तु मांसादाः पक्षिणः काननौकसाम्}
{तस्य तद् वचनं श्रुत्वा बलाध्यक्षा महाबलाः} %6-57-19

\twolineshloka
{बलमुद्योजयामासुस्तस्मिन् राक्षसमन्दिरे}
{सा बभूव मुहूर्तेन भीमैर्नानाविधायुधैः} %6-57-20

\twolineshloka
{लङ्का राक्षसवीरैस्तैर्गजैरिव समाकुला}
{हुताशनं तर्पयतां ब्राह्मणांश्च नमस्यताम्} %6-57-21

\twolineshloka
{आज्यगन्धप्रतिवहः सुरभिर्मारुतो ववौ}
{स्रजश्च विविधाकारा जगृहुस्त्वभिमन्त्रिताः} %6-57-22

\twolineshloka
{संग्रामसज्जाः संहृष्टा धारयन् राक्षसास्तदा}
{सधनुष्काः कवचिनो वेगादाप्लुत्य राक्षसाः} %6-57-23

\twolineshloka
{रावणं प्रेक्ष्य राजानं प्रहस्तं पर्यवारयन्}
{अथामन्त्र्य तु राजानं भेरीमाहत्य भैरवाम्} %6-57-24

\twolineshloka
{आरुरोह रथं युक्तः प्रहस्तः सज्जकल्पितम्}
{हयैर्महाजवैर्युक्तं सम्यक्सूतं सुसंयतम्} %6-57-25

\twolineshloka
{महाजलदनिर्घोषं साक्षाच्चन्द्रार्कभास्वरम्}
{उरगध्वजदुर्धर्षं सुवरूथं स्वपस्करम्} %6-57-26

\twolineshloka
{सुवर्णजालसंयुक्तं प्रहसन्तमिव श्रिया}
{ततस्तं रथमास्थाय रावणार्पितशासनः} %6-57-27

\threelineshloka
{लङ्काया निर्ययौ तूर्णं बलेन महता वृतः}
{ततो दुन्दुभिनिर्घोषः पर्जन्यनिनदोपमः}
{वादित्राणां च निनदः पूरयन्निव मेदिनीम्} %6-57-28

\twolineshloka
{शुश्रुवे शङ्खशब्दश्च प्रयाते वाहिनीपतौ}
{निनदन्तः स्वरान् घोरान् राक्षसा जग्मुरग्रतः} %6-57-29

\threelineshloka
{भीमरूपा महाकायाः प्रहस्तस्य पुरःसराः}
{नरान्तकः कुम्भहनुर्महानादः समुन्नतः}
{प्रहस्तसचिवा ह्येते निर्ययुः परिवार्य तम्} %6-57-30

\twolineshloka
{व्यूढेनैव सुघोरेण पूर्वद्वारात् स निर्ययौ}
{गजयूथनिकाशेन बलेन महता वृतः} %6-57-31

\twolineshloka
{सागरप्रतिमौघेन वृतस्तेन बलेन सः}
{प्रहस्तो निर्ययौ क्रुद्धः कालान्तकयमोपमः} %6-57-32

\twolineshloka
{तस्य निर्याणघोषेण राक्षसानां च नर्दताम्}
{लङ्कायां सर्वभूतानि विनेदुर्विकृतैः स्वरैः} %6-57-33

\twolineshloka
{व्यभ्रमाकाशमाविश्य मांसशोणितभोजनाः}
{मण्डलान्यपसव्यानि खगाश्चक्रू रथं प्रति} %6-57-34

\twolineshloka
{वमन्त्यः पावकज्वालाः शिवा घोरा ववाशिरे}
{अन्तरिक्षात् पपातोल्का वायुश्च परुषं ववौ} %6-57-35

\twolineshloka
{अन्योन्यमभिसंरब्धा ग्रहाश्च न चकाशिरे}
{मेघाश्च खरनिर्घोषा रथस्योपरि रक्षसः} %6-57-36

\twolineshloka
{ववर्षू रुधिरं चास्य सिषिचुश्च पुरःसरान्}
{केतुमूर्धनि गृध्रस्तु विलीनो दक्षिणामुखः} %6-57-37

\twolineshloka
{नदन्नुभयतः पार्श्वं समग्रां श्रियमाहरत्}
{सारथेर्बहुशश्चास्य संग्राममवगाहतः} %6-57-38

\twolineshloka
{प्रतोदो न्यपतद्धस्तात् सूतस्य हयसादिनः}
{निर्याणश्रीश्च या च स्याद् भास्वरा च सुदुर्लभा} %6-57-39

\threelineshloka
{सा ननाश मुहूर्तेन समे च स्खलिता हयाः}
{प्रहस्तं तं हि निर्यान्तं प्रख्यातगुणपौरुषम्}
{युधि नानाप्रहरणा कपिसेनाभ्यवर्तत} %6-57-40

\twolineshloka
{अथ घोषः सुतुमुलो हरीणां समजायत}
{वृक्षानारुजतां चैव गुर्वीर्वै गृह्णतां शिलाः} %6-57-41

\twolineshloka
{नदतां राक्षसानां च वानराणां च गर्जताम्}
{उभे प्रमुदिते सैन्ये रक्षोगणवनौकसाम्} %6-57-42

\twolineshloka
{वेगितानां समर्थानामन्योन्यवधकाङ्क्षिणाम्}
{परस्परं चाह्वयतां निनादः श्रूयते महान्} %6-57-43

\twolineshloka
{ततः प्रहस्तः कपिराजवाहिनीमभिप्रतस्थे विजयाय दुर्मतिः}
{विवृद्धवेगां च विवेश तां चमूं यथा मुमूर्षुः शलभो विभावसुम्} %6-57-44


॥इत्यार्षे श्रीमद्रामायणे वाल्मीकीये आदिकाव्ये युद्धकाण्डे प्रहस्तयुद्धम् नाम सप्तपञ्चाशः सर्गः ॥६-५७॥
