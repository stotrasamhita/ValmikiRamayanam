\sect{सप्तदशः सर्गः — विभीषणशरणागतिनिवेदनम्}

\twolineshloka
{इत्युक्त्वा परुषं वाक्यं रावणं रावणानुजः}
{आजगाम मुहूर्तेन यत्र रामः सलक्ष्मणः} %6-17-1

\twolineshloka
{तं मेरुशिखराकारं दीप्तामिव शतह्रदाम्}
{गगनस्थं महीस्थास्ते ददृशुर्वानराधिपाः} %6-17-2

\twolineshloka
{ते चाप्यनुचरास्तस्य चत्वारो भीमविक्रमाः}
{तेऽपि वर्मायुधोपेता भूषणोत्तमभूषिताः} %6-17-3

\twolineshloka
{स च मेघाचलप्रख्यो वज्रायुधसमप्रभः}
{वरायुधधरो वीरो दिव्याभरणभूषितः} %6-17-4

\twolineshloka
{तमात्मपञ्चमं दृष्ट्वा सुग्रीवो वानराधिपः}
{वानरैः सह दुर्धर्षश्चिन्तयामास बुद्धिमान्} %6-17-5

\twolineshloka
{चिन्तयित्वा मुहूर्तं तु वानरांस्तानुवाच ह}
{हनुमत्प्रमुखान् सर्वानिदं वचनमुत्तमम्} %6-17-6

\twolineshloka
{एष सर्वायुधोपेतश्चतुर्भिः सह राक्षसैः}
{राक्षसोभ्येति पश्यध्वमस्मान् हन्तुं न संशयः} %6-17-7

\twolineshloka
{सुग्रीवस्य वचः श्रुत्वा सर्वे ते वानरोत्तमाः}
{सालानुद्यम्य शैलांश्च इदं वचनमब्रुवन्} %6-17-8

\twolineshloka
{शीघ्रं व्यादिश नो राजन् वधायैषां दुरात्मनाम्}
{निपतन्ति हता यावद् धरण्यामल्पचेतनाः} %6-17-9

\twolineshloka
{तेषां सम्भाषमाणानामन्योन्यं स विभीषणः}
{उत्तरं तीरमासाद्य खस्थ एव व्यतिष्ठत} %6-17-10

\twolineshloka
{स उवाच महाप्राज्ञः स्वरेण महता महान्}
{सुग्रीवं तांश्च सम्प्रेक्ष्य खस्थ एव विभीषणः} %6-17-11

\twolineshloka
{रावणो नाम दुर्वृत्तो राक्षसो राक्षसेश्वरः}
{तस्याहमनुजो भ्राता विभीषण इति श्रुतः} %6-17-12

\twolineshloka
{तेन सीता जनस्थानाद् हृता हत्वा जटायुषम्}
{रुद्धा च विवशा दीना राक्षसीभिः सुरक्षिता} %6-17-13

\twolineshloka
{तमहं हेतुभिर्वाक्यैर्विविधैश्च न्यदर्शयम्}
{साधु निर्यात्यतां सीता रामायेति पुनः पुनः} %6-17-14

\twolineshloka
{स च न प्रतिजग्राह रावणः कालचोदितः}
{उच्यमानं हितं वाक्यं विपरीत इवौषधम्} %6-17-15

\twolineshloka
{सोऽहं परुषितस्तेन दासवच्चावमानितः}
{त्यक्त्वा पुत्रांश्च दारांश्च राघवं शरणं गतः} %6-17-16

\twolineshloka
{निवेदयत मां क्षिप्रं राघवाय महात्मने}
{सर्वलोकशरण्याय विभीषणमुपस्थितम्} %6-17-17

\twolineshloka
{एतत्तु वचनं श्रुत्वा सुग्रीवो लघुविक्रमः}
{लक्ष्मणस्याग्रतो रामं संरब्धमिदमब्रवीत्} %6-17-18

\twolineshloka
{प्रविष्टः शत्रुसैन्यं हि प्राप्तः शत्रुरतर्कितः}
{निहन्यादन्तरं लब्ध्वा उलूको वायसानिव} %6-17-19

\twolineshloka
{मन्त्रे व्यूहे नये चारे युक्तो भवितुमर्हसि}
{वानराणां च भद्रं ते परेषां च परन्तप} %6-17-20

\twolineshloka
{अन्तर्धानगता ह्येते राक्षसाः कामरूपिणः}
{शूराश्च निकृतिज्ञाश्च तेषां जातु न विश्वसेत्} %6-17-21

\twolineshloka
{प्रणिधी राक्षसेन्द्रस्य रावणस्य भवेदयम्}
{अनुप्रविश्य सोऽस्मासु भेदं कुर्यान्न संशयः} %6-17-22

\twolineshloka
{अथ वा स्वयमेवैष छिद्रमासाद्य बुद्धिमान्}
{अनुप्रविश्य विश्वस्ते कदाचित् प्रहरेदपि} %6-17-23

\twolineshloka
{मित्राटविबलं चैव मौलभृत्यबलं तथा}
{सर्वमेतद् बलं ग्राह्यं वर्जयित्वा द्विषद्बलम्} %6-17-24

\twolineshloka
{प्रकृत्या राक्षसो ह्येष भ्रातामित्रस्य वै प्रभो}
{आगतश्च रिपुः साक्षात् कथमस्मिंश्च विश्वसेत्} %6-17-25

\twolineshloka
{रावणस्यानुजो भ्राता विभीषण इति श्रुतः}
{चतुर्भिः सह रक्षोभिर्भवन्तं शरणं गतः} %6-17-26

\twolineshloka
{रावणेन प्रणीतं हि तमवेहि विभीषणम्}
{तस्याहं निग्रहं मन्ये क्षमं क्षमवतां वर} %6-17-27

\twolineshloka
{राक्षसो जिह्मया बुद्ध्या सन्दिष्टोऽयमिहागतः}
{प्रहर्तुं मायया छन्नो विश्वस्ते त्वयि चानघ} %6-17-28

\twolineshloka
{वध्यतामेष तीव्रेण दण्डेन सचिवैः सह}
{रावणस्य नृशंसस्य भ्राता ह्येष विभीषणः} %6-17-29

\twolineshloka
{एवमुक्त्वा तु तं रामं संरब्धो वाहिनीपतिः}
{वाक्यज्ञो वाक्यकुशलं ततो मौनमुपागमत्} %6-17-30

\twolineshloka
{सुग्रीवस्य तु तद् वाक्यं श्रुत्वा रामो महाबलः}
{समीपस्थानुवाचेदं हनुमत्प्रमुखान् कपीन्} %6-17-31

\twolineshloka
{यदुक्तं कपिराजेन रावणावरजं प्रति}
{वाक्यं हेतुमदत्यर्थं भवद्भिरपि च श्रुतम्} %6-17-32

\twolineshloka
{सुहृदामर्थकृच्छ्रेषु युक्तं बुद्धिमता सदा}
{समर्थेनोपसन्देष्टुं शाश्वतीं भूतिमिच्छता} %6-17-33

\twolineshloka
{इत्येवं परिपृष्टास्ते स्वं स्वं मतमतन्द्रिताः}
{सोपचारं तदा राममूचुः प्रियचिकीर्षवः} %6-17-34

\twolineshloka
{अज्ञातं नास्ति ते किञ्चित् त्रिषु लोकेषु राघव}
{आत्मानं पूजयन् राम पृच्छस्यस्मान् सुहृत्तया} %6-17-35

\twolineshloka
{त्वं हि सत्यव्रतः शूरो धार्मिको दृढविक्रमः}
{परीक्ष्यकारी स्मृतिमान् निसृष्टात्मा सुहृत्सु च} %6-17-36

\twolineshloka
{तस्मादेकैकशस्तावद् ब्रुवन्तु सचिवास्तव}
{हेतुतो मतिसम्पन्नाः समर्थाश्च पुनः पुनः} %6-17-37

\twolineshloka
{इत्युक्ते राघवायाथ मतिमानङ्गदोऽग्रतः}
{विभीषणपरीक्षार्थमुवाच वचनं हरिः} %6-17-38

\twolineshloka
{शत्रोः सकाशात् सम्प्राप्तः सर्वथा तर्क्य एव हि}
{विश्वासनीयः सहसा न कर्तव्यो विभीषणः} %6-17-39

\twolineshloka
{छादयित्वाऽऽत्मभावं हि चरन्ति शठबुद्धयः}
{प्रहरन्ति च रन्ध्रेषु सोऽनर्थः सुमहान् भवेत्} %6-17-40

\twolineshloka
{अर्थानर्थौ विनिश्चित्य व्यवसायं भजेत ह}
{गुणतः सङ्ग्रहं कुर्याद् दोषतस्तु विसर्जयेत्} %6-17-41

\twolineshloka
{यदि दोषो महांस्तस्मिंस्त्यज्यतामविशङ्कितम्}
{गुणान् वापि बहून् ज्ञात्वा सङ्ग्रहः क्रियतां नृप} %6-17-42

\twolineshloka
{शरभस्त्वथ निश्चित्य सार्थं वचनमब्रवीत्}
{क्षिप्रमस्मिन् नरव्याघ्र चारः प्रतिविधीयताम्} %6-17-43

\twolineshloka
{प्रणिधाय हि चारेण यथावत् सूक्ष्मबुद्धिना}
{परीक्ष्य च ततः कार्यो यथान्यायं परिग्रहः} %6-17-44

\twolineshloka
{जाम्बवांस्त्वथ सम्प्रेक्ष्य शास्त्रबुद्ध्या विचक्षणः}
{वाक्यं विज्ञापयामास गुणवद् दोषवर्जितम्} %6-17-45

\twolineshloka
{बद्धवैराच्च पापाच्च राक्षसेन्द्राद् विभीषणः}
{अदेशकाले सम्प्राप्तः सर्वथा शङ्क्यतामयम्} %6-17-46

\twolineshloka
{ततो मैन्दस्तु सम्प्रेक्ष्य नयापनयकोविदः}
{वाक्यं वचनसम्पन्नो बभाषे हेतुमत्तरम्} %6-17-47

\twolineshloka
{अनुजो नाम तस्यैष रावणस्य विभीषणः}
{पृच्छ्यतां मधुरेणायं शनैर्नरपतीश्वर} %6-17-48

\twolineshloka
{भावमस्य तु विज्ञाय तत्त्वतस्तं करिष्यसि}
{यदि दुष्टो न दुष्टो वा बुद्धिपूर्वं नरर्षभ} %6-17-49

\twolineshloka
{अथ संस्कारसम्पन्नो हनूमान् सचिवोत्तमः}
{उवाच वचनं श्लक्ष्णमर्थवन्मधुरं लघु} %6-17-50

\twolineshloka
{न भवन्तं मतिश्रेष्ठं समर्थं वदतां वरम्}
{अतिशाययितुं शक्तो बृहस्पतिरपि ब्रुवन्} %6-17-51

\twolineshloka
{न वादान्नापि सङ्घर्षान्नाधिक्यान्न च कामतः}
{वक्ष्यामि वचनं राजन् यथार्थं राम गौरवात्} %6-17-52

\twolineshloka
{अर्थानर्थनिमित्तं हि यदुक्तं सचिवैस्तव}
{तत्र दोषं प्रपश्यामि क्रिया नह्युपपद्यते} %6-17-53

\twolineshloka
{ऋते नियोगात् सामर्थ्यमवबोद्धुं न शक्यते}
{सहसा विनियोगोऽपि दोषवान् प्रतिभाति मे} %6-17-54

\twolineshloka
{चारप्रणिहितं युक्तं यदुक्तं सचिवैस्तव}
{अर्थस्यासम्भवात् तत्र कारणं नोपपद्यते} %6-17-55

\twolineshloka
{अदेशकाले सम्प्राप्त इत्ययं यद् विभीषणः}
{विवक्षा तत्र मेऽस्तीयं तां निबोध यथामति} %6-17-56

\twolineshloka
{एष देशश्च कालश्च भवतीह यथा तथा}
{पुरुषात् पुरुषं प्राप्य तथा दोषगुणावपि} %6-17-57

\twolineshloka
{दौरात्म्यं रावणे दृष्ट्वा विक्रमं च तथा त्वयि}
{युक्तमागमनं ह्यत्र सदृशं तस्य बुद्धितः} %6-17-58

\twolineshloka
{अज्ञातरूपैः पुरुषैः स राजन् पृच्छ्यतामिति}
{यदुक्तमत्र मे प्रेक्षा काचिदस्ति समीक्षिता} %6-17-59

\twolineshloka
{पृच्छ्यमानो विशङ्केत सहसा बुद्धिमान् वचः}
{तत्र मित्रं प्रदुष्येत मिथ्या पृष्टं सुखागतम्} %6-17-60

\twolineshloka
{अशक्यं सहसा राजन् भावो बोद्धुं परस्य वै}
{अन्तरेण स्वरैर्भिन्नैर्नैपुण्यं पश्यतां भृशम्} %6-17-61

\twolineshloka
{न त्वस्य ब्रुवतो जातु लक्ष्यते दुष्टभावता}
{प्रसन्नं वदनं चापि तस्मान्मे नास्ति संशयः} %6-17-62

\twolineshloka
{अशङ्कितमतिः स्वस्थो न शठः परिसर्पति}
{न चास्य दुष्टवागस्ति तस्मान्मे नास्ति संशयः} %6-17-63

\twolineshloka
{आकारश्छाद्यमानोऽपि न शक्यो विनिगूहितुम्}
{बलाद्धि विवृणोत्येव भावमन्तर्गतं नृणाम्} %6-17-64

\twolineshloka
{देशकालोपपन्नं च कार्यं कार्यविदां वर}
{सफलं कुरुते क्षिप्रं प्रयोगेणाभिसंहितम्} %6-17-65

\twolineshloka
{उद्योगं तव सम्प्रेक्ष्य मिथ्यावृत्तं च रावणम्}
{वालिनं च हतं श्रुत्वा सुग्रीवं चाभिषेचितम्} %6-17-66

\twolineshloka
{राज्यं प्रार्थयमानस्तु बुद्धिपूर्वमिहागतः}
{एतावत् तु पुरस्कृत्य युज्यते तस्य सङ्ग्रहः} %6-17-67

\twolineshloka
{यथाशक्ति मयोक्तं तु राक्षसस्यार्जवं प्रति}
{प्रमाणं त्वं हि शेषस्य श्रुत्वा बुद्धिमतां वर} %6-17-68


॥इत्यार्षे श्रीमद्रामायणे वाल्मीकीये आदिकाव्ये युद्धकाण्डे विभीषणशरणागतिनिवेदनम् नाम सप्तदशः सर्गः ॥६-१७॥
