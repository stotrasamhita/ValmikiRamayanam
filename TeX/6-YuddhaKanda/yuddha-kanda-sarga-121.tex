\sect{एकविंशत्यधिकशततमः सर्गः — सीताप्रतिग्रहः}

\twolineshloka
{एतच्छ्रुत्वा शुभं वाक्यं पितामहसमीरितम्}
{अङ्केनादाय वैदेहीमुत्पपात विभावसुः} %6-121-1

\twolineshloka
{विधूयाथ चितां तां तु वैदेहीं हव्यवाहनः}
{उत्तस्थौ मूर्तिमानाशु गृहीत्वा जनकात्मजाम्} %6-121-2

\twolineshloka
{तरुणादित्यसङ्काशां तप्तकाञ्चनभूषणाम्}
{रक्ताम्बरधरां बालां नीलकुञ्चितमूर्धजाम्} %6-121-3

\twolineshloka
{अक्लिष्टमाल्याभरणां तथारूपामनिन्दिताम्}
{ददौ रामाय वैदेहीमङ्के कृत्वा विभावसुः} %6-121-4

\twolineshloka
{अब्रवीत् तु तदा रामं साक्षी लोकस्य पावकः}
{एषा ते राम वैदेही पापमस्यां न विद्यते} %6-121-5

\twolineshloka
{नैव वाचा न मनसा नैव बुद्ध्या न चक्षुषा}
{सुवृत्ता वृत्तशौटीर्यं न त्वामत्यचरच्छुभा} %6-121-6

\twolineshloka
{रावणेनापनीतैषा वीर्योत्सिक्तेन रक्षसा}
{त्वया विरहिता दीना विवशा निर्जने सती} %6-121-7

\twolineshloka
{रुद्धा चान्तःपुरे गुप्ता त्वच्चित्ता त्वत्परायणा}
{रक्षिता राक्षसीभिश्च घोराभिर्घोरबुद्धिभिः} %6-121-8

\twolineshloka
{प्रलोभ्यमाना विविधं तर्ज्यमाना च मैथिली}
{नाचिन्तयत तद्रक्षस्त्वद्गतेनान्तरात्मना} %6-121-9

\twolineshloka
{विशुद्धभावां निष्पापां प्रतिगृह्णीष्व मैथिलीम्}
{न किञ्चिदभिधातव्या अहमाज्ञापयामि ते} %6-121-10

\twolineshloka
{ततः प्रीतमना रामः श्रुत्वैवं वदतां वरः}
{दध्यौ मुहूर्तं धर्मात्मा हर्षव्याकुललोचनः} %6-121-11

\twolineshloka
{एवमुक्तो महातेजा धृतिमानुरुविक्रमः}
{उवाच त्रिदशश्रेष्ठं रामो धर्मभृतां वरः} %6-121-12

\twolineshloka
{अवश्यं चापि लोकेषु सीता पावनमर्हति}
{दीर्घकालोषिता हीयं रावणान्तःपुरे शुभा} %6-121-13

\twolineshloka
{बालिशो बत कामात्मा रामो दशरथात्मजः}
{इति वक्ष्यति मां लोको जानकीमविशोध्य हि} %6-121-14

\twolineshloka
{अनन्यहृदयां सीतां मच्चित्तपरिरक्षिणीम्}
{अहमप्यवगच्छामि मैथिलीं जनकात्मजाम्} %6-121-15

\twolineshloka
{इमामपि विशालाक्षीं रक्षितां स्वेन तेजसा}
{रावणो नातिवर्तेत वेलामिव महोदधिः} %6-121-16

\twolineshloka
{प्रत्ययार्थं तु लोकानां त्रयाणां सत्यसंश्रयः}
{उपेक्षे चापि वैदेहीं प्रविशन्तीं हुताशनम्} %6-121-17

\twolineshloka
{न हि शक्तः सुदुष्टात्मा मनसापि हि मैथिलीम्}
{प्रधर्षयितुमप्राप्यां दीप्तामग्निशिखामिव} %6-121-18

\twolineshloka
{नेयमर्हति वैक्लव्यं रावणान्तःपुरे सती}
{अनन्या हि मया सीता भास्करस्य प्रभा यथा} %6-121-19

\twolineshloka
{विशुद्धा त्रिषु लोकेषु मैथिली जनकात्मजा}
{न विहातुं मया शक्या कीर्तिरात्मवता यथा} %6-121-20

\twolineshloka
{अवश्यं च मया कार्यं सर्वेषां वो वचो हितम्}
{स्निग्धानां लोकनाथानामेवं च वदतां हितम्} %6-121-21

\twolineshloka
{इत्येवमुक्त्वा विजयी महाबलः प्रशस्यमानः स्वकृतेन कर्मणा}
{समेत्य रामः प्रियया महायशाः सुखं सुखार्होऽनुबभूव राघवः} %6-121-22


॥इत्यार्षे श्रीमद्रामायणे वाल्मीकीये आदिकाव्ये युद्धकाण्डे सीताप्रतिग्रहः नाम एकविंशत्यधिकशततमः सर्गः ॥६-१२१॥
