\sect{त्रयोविंशः सर्गः — लङ्काभिषेणनम्}

\twolineshloka
{निमित्तानि निमित्तज्ञो दृष्ट्वा लक्ष्मणपूर्वजः}
{सौमित्रिं सम्परिष्वज्य इदं वचनमब्रवीत्} %6-23-1

\twolineshloka
{परिगृह्योदकं शीतं वनानि फलवन्ति च}
{बलौघं संविभज्येमं व्यूह्य तिष्ठेम लक्ष्मण} %6-23-2

\twolineshloka
{लोकक्षयकरं भीमं भयं पश्याम्युपस्थितम्}
{प्रबर्हणं प्रवीराणामृक्षवानररक्षसाम्} %6-23-3

\twolineshloka
{वाताश्च कलुषा वान्ति कम्पते च वसुंधरा}
{पर्वताग्राणि वेपन्ते पतन्ति च महीरुहाः} %6-23-4

\twolineshloka
{मेघाः क्रव्यादसंकाशाः परुषाः परुषस्वनाः}
{क्रूराः क्रूरं प्रवर्षन्ति मिश्रं शोणितबिन्दुभिः} %6-23-5

\twolineshloka
{रक्तचन्दनसंकाशा संध्या परमदारुणा}
{ज्वलतः प्रपतत्येतदादित्यादग्निमण्डलम्} %6-23-6

\twolineshloka
{दीना दीनस्वराः क्रूराः सर्वतो मृगपक्षिणः}
{प्रत्यादित्यं विनर्दन्ति जनयन्तो महद्भयम्} %6-23-7

\twolineshloka
{रजन्यामप्रकाशस्तु संतापयति चन्द्रमाः}
{कृष्णरक्तांशुपर्यन्तो लोकक्षय इवोदितः} %6-23-8

\twolineshloka
{ह्रस्वो रूक्षोऽप्रशस्तश्च परिवेषस्तु लोहितः}
{आदित्ये विमले नीलं लक्ष्म लक्ष्मण दृश्यते} %6-23-9

\twolineshloka
{रजसा महता चापि नक्षत्राणि हतानि च}
{युगान्तमिव लोकानां पश्य शंसन्ति लक्ष्मण} %6-23-10

\twolineshloka
{काकाः श्येनास्तथा नीचा गृध्राः परिपतन्ति च}
{शिवाश्चाप्यशुभान् नादान् नदन्ति सुमहाभयान्} %6-23-11

\twolineshloka
{शैलैः शूलैश्च खड्गैश्च विमुक्तैः कपिराक्षसैः}
{भविष्यत्यावृता भूमिर्मांसशोणितकर्दमा} %6-23-12

\twolineshloka
{क्षिप्रमद्यैव दुर्धर्षां पुरीं रावणपालिताम्}
{अभियाम जवेनैव सर्वैर्हरिभिरावृताः} %6-23-13

\twolineshloka
{इत्येवमुक्त्वा धन्वी स रामः संग्रामधर्षणः}
{प्रतस्थे पुरतो रामो लङ्कामभिमुखो विभुः} %6-23-14

\twolineshloka
{सविभीषणसुग्रीवाः सर्वे ते वानरर्षभाः}
{प्रतस्थिरे विनर्दन्तो धृतानां द्विषतां वधे} %6-23-15

\twolineshloka
{राघवस्य प्रियार्थं तु सुतरां वीर्यशालिनाम्}
{हरीणां कर्मचेष्टाभिस्तुतोष रघुनन्दनः} %6-23-16


॥इत्यार्षे श्रीमद्रामायणे वाल्मीकीये आदिकाव्ये युद्धकाण्डे लङ्काभिषेणनम् नाम त्रयोविंशः सर्गः ॥६-२३॥
