\sect{पञ्चाधिकशततमः सर्गः — दशग्रीवविघूर्णनम्}

\twolineshloka
{स तु तेन तदा क्रोधात् काकुत्स्थेनार्दितो भृशम्}
{रावणः समरश्लाघी महाक्रोधमुपागमत्} %6-105-1

\twolineshloka
{स दीप्तनयनोऽमर्षाच्चापमुद्यम्य वीर्यवान्}
{अभ्यर्दयत् सुसंक्रुद्धो राघवं परमाहवे} %6-105-2

\twolineshloka
{बाणधारासहस्रैस्तैः स तोयद इवाम्बरात्}
{राघवं रावणो बाणैस्तटाकमिव पूरयन्} %6-105-3

\twolineshloka
{पूरितः शरजालेन धनुर्मुक्तेन संयुगे}
{महागिरिरिवाकम्प्यः काकुत्स्थो न प्रकम्पते} %6-105-4

\twolineshloka
{स शरैः शरजालानि वारयन् समरे स्थितः}
{गभस्तीनिव सूर्यस्य प्रतिजग्राह वीर्यवान्} %6-105-5

\twolineshloka
{ततः शरसहस्राणि क्षिप्रहस्तो निशाचरः}
{निजघानोरसि क्रुद्धो राघवस्य महात्मनः} %6-105-6

\twolineshloka
{स शोणितसमादिग्धः समरे लक्ष्मणाग्रजः}
{दृष्टः फुल्ल इवारण्ये सुमहान् किंशुकद्रुमः} %6-105-7

\twolineshloka
{शराभिघातसंरब्धः सोऽभिजग्राह सायकान्}
{काकुत्स्थः सुमहातेजा युगान्तादित्यवर्चसः} %6-105-8

\twolineshloka
{ततोऽन्योन्यं सुसंरब्धौ तावुभौ रामरावणौ}
{शरान्धकारे समरे नोपलक्षयतां तदा} %6-105-9

\twolineshloka
{ततः क्रोधसमाविष्टो रामो दशरथात्मजः}
{उवाच रावणं वीरः प्रहस्य परुषं वचः} %6-105-10

\twolineshloka
{मम भार्या जनस्थानादज्ञानाद् राक्षसाधम}
{हृता ते विवशा यस्मात् तस्मात् त्वं नासि वीर्यवान्} %6-105-11

\twolineshloka
{मया विरहितां दीनां वर्तमानां महावने}
{वैदेहीं प्रसभं हृत्वा शूरोऽहमिति मन्यसे} %6-105-12

\twolineshloka
{स्त्रीषु शूर विनाथासु परदाराभिमर्शनम्}
{कृत्वा कापुरुषं कर्म शूरोऽहमिति मन्यसे} %6-105-13

\twolineshloka
{भिन्नमर्याद निर्लज्ज चारित्रेष्वनवस्थित}
{दर्पान्मृत्युमुपादाय शूरोऽहमिति मन्यसे} %6-105-14

\twolineshloka
{शूरेण धनदभ्रात्रा बलैः समुदितेन च}
{श्लाघनीयं महत्कर्म यशस्यं च कृतं त्वया} %6-105-15

\twolineshloka
{उत्सेकेनाभिपन्नस्य गर्हितस्याहितस्य च}
{कर्मणः प्राप्नुहीदानीं तस्याद्य सुमहत् फलम्} %6-105-16

\twolineshloka
{शूरोऽहमिति चात्मानमवगच्छसि दुर्मते}
{नैव लज्जास्ति ते सीतां चौरवद् व्यपकर्षतः} %6-105-17

\twolineshloka
{यदि मत्संनिधौ सीता धर्षिता स्यात् त्वया बलात्}
{भ्रातरं तु खरं पश्येस्तदा मत्सायकैर्हतः} %6-105-18

\twolineshloka
{दिष्ट्यासि मम मन्दात्मंश्चक्षुर्विषयमागतः}
{अद्य त्वां सायकैस्तीक्ष्णैर्नयामि यमसादनम्} %6-105-19

\twolineshloka
{अद्य ते मच्छरैश्छिन्नं शिरो ज्वलितकुण्डलम्}
{क्रव्यादा व्यपकर्षन्तु विकीर्णं रणपांसुषु} %6-105-20

\twolineshloka
{निपत्योरसि गृध्रास्ते क्षितौ क्षिप्तस्य रावण}
{पिबन्तु रुधिरं तर्षाद् बाणशल्यान्तरोत्थितम्} %6-105-21

\twolineshloka
{अद्य मद्बाणभिन्नस्य गतासोः पतितस्य ते}
{कर्षन् त्वन्त्राणि पतगा गरुत्मन्त इवोरगान्} %6-105-22

\twolineshloka
{इत्येवं स वदन् वीरो रामः शत्रुनिबर्हणः}
{राक्षसेन्द्रं समीपस्थं शरवर्षैरवाकिरत्} %6-105-23

\twolineshloka
{बभूव द्विगुणं वीर्यं बलं हर्षश्च संयुगे}
{रामस्यास्त्रबलं चैव शत्रोर्निधनकांक्षिणः} %6-105-24

\twolineshloka
{प्रादुर्बभूवुरस्त्राणि सर्वाणि विदितात्मनः}
{प्रहर्षाच्च महातेजाः शीघ्रहस्ततरोऽभवत्} %6-105-25

\twolineshloka
{शुभान्येतानि चिह्नानि विज्ञायात्मगतानि सः}
{भूय एवार्दयद् रामो रावणं राक्षसान्तकृत्} %6-105-26

\twolineshloka
{हरीणां चाश्मनिकरैः शरवर्षैश्च राघवात्}
{हन्यमानो दशग्रीवो विघूर्णहृदयोऽभवत्} %6-105-27

\twolineshloka
{यदा च शस्त्रं नारेभे न चकर्ष शरासनम्}
{नास्य प्रत्यकरोद् वीर्यं विक्लवेनान्तरात्मना} %6-105-28

\twolineshloka
{क्षिप्ताश्चाशु शरास्तेन शस्त्राणि विविधानि च}
{मरणार्थाय वर्तन्ते मृत्युकालोऽभ्यवर्तत} %6-105-29

\twolineshloka
{सूतस्तु रथनेतास्य तदवस्थं निरीक्ष्य तम्}
{शनैर्युद्धादसम्भ्रान्तो रथं तस्यापवाहयत्} %6-105-30

\twolineshloka
{रथं च तस्याथ जवेन सारथिर्निवार्य भीमं जलदस्वनं तदा}
{जगाम भीत्या समरान्महीपतिं निरस्तवीर्यं पतितं समीक्ष्य} %6-105-31


॥इत्यार्षे श्रीमद्रामायणे वाल्मीकीये आदिकाव्ये युद्धकाण्डे दशग्रीवविघूर्णनम् नाम पञ्चाधिकशततमः सर्गः ॥६-१०५॥
