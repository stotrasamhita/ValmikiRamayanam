\sect{पञ्चत्रिंशः सर्गः — माल्यवदुपदेशः}

\twolineshloka
{तेन शङ्खविमिश्रेण भेरीशब्देन नादिना}
{उपयाति महाबाहू रामः परपुरञ्जयः} %6-35-1

\twolineshloka
{तं निनादं निशम्याथ रावणो राक्षसेश्वरः}
{मुहूर्तं ध्यानमास्थाय सचिवानभ्युदैक्षत} %6-35-2

\twolineshloka
{अथ तान् सचिवांस्तत्र सर्वानाभाष्य रावणः}
{सभां सन्नादयन् सर्वामित्युवाच महाबलः} %6-35-3

\twolineshloka
{जगत्सन्तापनः क्रूरोऽगर्हयन् राक्षसेश्वरः}
{तरणं सागरस्यास्य विक्रमं बलपौरुषम्} %6-35-4

\threelineshloka
{यदुक्तवन्तो रामस्य भवन्तस्तन्मया श्रुतम्}
{भवतश्चाप्यहं वेद्मि युद्धे सत्यपराक्रमान्}
{तूष्णीकानीक्षतोऽन्योन्यं विदित्वा रामविक्रमम्} %6-35-5

\twolineshloka
{ततस्तु सुमहाप्राज्ञो माल्यवान् नाम राक्षसः}
{रावणस्य वचः श्रुत्वा इति मातामहोऽब्रवीत्} %6-35-6

\twolineshloka
{विद्यास्वभिविनीतो यो राजा राजन् नयानुगः}
{स शास्ति चिरमैश्वर्यमरींश्च कुरुते वशे} %6-35-7

\twolineshloka
{सन्दधानो हि कालेन विगृह्णंश्चारिभिः सह}
{स्वपक्षे वर्धनं कुर्वन्महदैश्वर्यमश्नुते} %6-35-8

\twolineshloka
{हीयमानेन कर्तव्यो राज्ञा सन्धिः समेन च}
{न शत्रुमवमन्येत ज्यायान् कुर्वीत विग्रहम्} %6-35-9

\twolineshloka
{तन्मह्यं रोचते सन्धिः सह रामेण रावण}
{यदर्थमभियुक्तोऽसि सीता तस्मै प्रदीयताम्} %6-35-10

\twolineshloka
{तस्य देवर्षयः सर्वे गन्धर्वाश्च जयैषिणः}
{विरोधं मा गमस्तेन सन्धिस्ते तेन रोचताम्} %6-35-11

\twolineshloka
{असृजद् भगवान् पक्षौ द्वावेव हि पितामहः}
{सुराणामसुराणां च धर्माधर्मौ तदाश्रयौ} %6-35-12

\twolineshloka
{धर्मो हि श्रूयते पक्ष अमराणां महात्मनाम्}
{अधर्मो रक्षसां पक्षो ह्यसुराणां च राक्षस} %6-35-13

\twolineshloka
{धर्मो वै ग्रसतेऽधर्मं यदा कृतमभूद् युगम्}
{अधर्मो ग्रसते धर्मं यदा तिष्यः प्रवर्तते} %6-35-14

\twolineshloka
{तत् त्वया चरता लोकान् धर्मोऽपि निहतो महान्}
{अधर्मः प्रगृहीतश्च तेनास्मद् बलिनः परे} %6-35-15

\twolineshloka
{स प्रमादात् प्रवृद्धस्तेऽधर्मोऽहिर्ग्रसते हि नः}
{विवर्धयति पक्षं च सुराणां सुरभावनः} %6-35-16

\twolineshloka
{विषयेषु प्रसक्तेन यत्किञ्चित्कारिणा त्वया}
{ऋषीणामग्निकल्पानामुद्वेगो जनितो महान्} %6-35-17

\twolineshloka
{तेषां प्रभावो दुर्धर्षः प्रदीप्त इव पावकः}
{तपसा भावितात्मानो धर्मस्यानुग्रहे रताः} %6-35-18

\twolineshloka
{मुख्यैर्यज्ञैर्यजन्त्येते तैस्तैर्यत्ते द्विजातयः}
{जुह्वत्यग्नींश्च विधिवद् वेदांश्चोच्चैरधीयते} %6-35-19

\twolineshloka
{अभिभूय च रक्षांसि ब्रह्मघोषानुदीरयन्}
{दिशो विप्रद्रुताः सर्वाः स्तनयित्नुरिवोष्णगे} %6-35-20

\twolineshloka
{ऋषीणामग्निकल्पानामग्निहोत्रसमुत्थितः}
{आदत्ते रक्षसां तेजो धूमो व्याप्य दिशो दश} %6-35-21

\twolineshloka
{तेषु तेषु च देशेषु पुण्येष्वेव दृढव्रतैः}
{चर्यमाणं तपस्तीव्रं सन्तापयति राक्षसान्} %6-35-22

\threelineshloka
{देवदानवयक्षेभ्यो गृहीतश्च वरस्त्वया}
{मनुष्या वानरा ऋक्षा गोलाङ्गूला महाबलाः}
{बलवन्त इहागम्य गर्जन्ति दृढविक्रमाः} %6-35-23

\twolineshloka
{उत्पातान् विविधान् दृष्ट्वा घोरान् बहुविधान् बहून्}
{विनाशमनुपश्यामि सर्वेषां रक्षसामहम्} %6-35-24

\twolineshloka
{खराभिस्तनिता घोरा मेघाः प्रतिभयङ्कराः}
{शोणितेनाभिवर्षन्ति लङ्कामुष्णेन सर्वतः} %6-35-25

\twolineshloka
{रुदतां वाहनानां च प्रपतन्त्यश्रुबिन्दवः}
{रजोध्वस्ता विवर्णाश्च न प्रभान्ति यथापुरम्} %6-35-26

\twolineshloka
{व्याला गोमायवो गृध्रा वाश्यन्ति च सुभैरवम्}
{प्रविश्य लङ्कामारामे समवायांश्च कुर्वते} %6-35-27

\twolineshloka
{कालिकाः पाण्डुरैर्दन्तैः प्रहसन्त्यग्रतः स्थिताः}
{स्त्रियः स्वप्नेषु मुष्णन्त्यो गृहाणि प्रतिभाष्य च} %6-35-28

\twolineshloka
{गृहाणां बलिकर्माणि श्वानः पर्युपभुञ्जते}
{खरा गोषु प्रजायन्ते मूषका नकुलेषु च} %6-35-29

\twolineshloka
{मार्जारा द्वीपिभिः सार्धं सूकराः शुनकैः सह}
{किन्नरा राक्षसैश्चापि समेयुर्मानुषैः सह} %6-35-30

\twolineshloka
{पाण्डुरा रक्तपादाश्च विहगाः कालचोदिताः}
{राक्षसानां विनाशाय कपोता विचरन्ति च} %6-35-31

\twolineshloka
{चीचीकूचीति वाशन्त्यः शारिका वेश्मसु स्थिताः}
{पतन्ति ग्रथिताश्चापि निर्जिताः कलहैषिभिः} %6-35-32

\twolineshloka
{पक्षिणश्च मृगाः सर्वे प्रत्यादित्यं रुदन्ति ते}
{करालो विकटो मुण्डः पुरुषः कृष्णपिङ्गलः} %6-35-33

\twolineshloka
{कालो गृहाणि सर्वेषां काले कालेऽन्ववेक्षते}
{एतान्यन्यानि दुष्टानि निमित्तान्युत्पतन्ति च} %6-35-34

\twolineshloka
{विष्णुं मन्यामहे रामं मानुषं रूपमास्थितम्}
{नहि मानुषमात्रोऽसौ राघवो दृढविक्रमः} %6-35-35

\threelineshloka
{येन बद्धः समुद्रे च सेतुः स परमाद्भुतः}
{कुरुष्व नरराजेन सन्धिं रामेण रावण}
{ज्ञात्वावधार्य कर्माणि क्रियतामायतिक्षमम्} %6-35-36

\twolineshloka
{इदं वचस्तस्य निगद्य माल्यवान् परीक्ष्य रक्षोधिपतेर्मनः पुनः}
{अनुत्तमेषूत्तमपौरुषो बली बभूव तूष्णीं समवेक्ष्य रावणम्} %6-35-37


॥इत्यार्षे श्रीमद्रामायणे वाल्मीकीये आदिकाव्ये युद्धकाण्डे माल्यवदुपदेशः नाम पञ्चत्रिंशः सर्गः ॥६-३५॥
