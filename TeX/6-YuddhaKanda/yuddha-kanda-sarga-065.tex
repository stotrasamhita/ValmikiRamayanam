\sect{पञ्चषष्ठितमः सर्गः — कुम्भकर्णाभिषेणनम्}

\twolineshloka
{स तथोक्तस्तु निर्भर्त्स्य कुम्भकर्णो महोदरम्}
{अब्रवीद् राक्षसश्रेष्ठं भ्रातरं रावणं ततः} %6-65-1

\twolineshloka
{सोऽहं तव भयं घोरं वधात् तस्य दुरात्मनः}
{रामस्याद्य प्रमार्जामि निर्वैरो हि सुखी भव} %6-65-2

\twolineshloka
{गर्जन्ति न वृथा शूरा निर्जला इव तोयदाः}
{पश्य सम्पद्यमानं तु गर्जितं युधि कर्मणा} %6-65-3

\twolineshloka
{न मर्षयन्ति चात्मानं सम्भावयितुमात्मना}
{अदर्शयित्वा शूरास्तु कर्म कुर्वन्ति दुष्करम्} %6-65-4

\twolineshloka
{विक्लवानां ह्यबुद्धीनां राज्ञां पण्डितमानिनाम्}
{रोचते त्वद्वचो नित्यं कथ्यमानं महोदर} %6-65-5

\twolineshloka
{युद्धे कापुरुषैर्नित्यं भवद्भिः प्रियवादिभिः}
{राजानमनुगच्छद्भिः सर्वं कृत्यं विनाशितम्} %6-65-6

\twolineshloka
{राजशेषा कृता लङ्का क्षीणः कोशो बलं हतम्}
{राजानमिममासाद्य सुहृच्चिह्नममित्रकम्} %6-65-7

\twolineshloka
{एष निर्याम्यहं युद्धमुद्यतः शत्रुनिर्जये}
{दुर्नयं भवतामद्य समीकर्तुं महाहवे} %6-65-8

\twolineshloka
{एवमुक्तवतो वाक्यं कुम्भकर्णस्य धीमतः}
{प्रत्युवाच ततो वाक्यं प्रहसन् राक्षसाधिपः} %6-65-9

\twolineshloka
{महोदरोऽयं रामात् तु परित्रस्तो न संशयः}
{न हि रोचयते तात युद्धं युद्धविशारद} %6-65-10

\twolineshloka
{कश्चिन्मे त्वत्समो नास्ति सौहृदेन बलेन च}
{गच्छ शत्रुवधाय त्वं कुम्भकर्ण जयाय च} %6-65-11

\twolineshloka
{शयानः शत्रुनाशार्थं भवान् सम्बोधितो मया}
{अयं हि कालः सुमहान् राक्षसानामरिन्दम} %6-65-12

\twolineshloka
{सङ्गच्छ शूलमादाय पाशहस्त इवान्तकः}
{वानरान् राजपुत्रौ च भक्षयादित्यतेजसौ} %6-65-13

\twolineshloka
{समालोक्य तु ते रूपं विद्रविष्यन्ति वानराः}
{रामलक्ष्मणयोश्चापि हृदये प्रस्फुटिष्यतः} %6-65-14

\twolineshloka
{एवमुक्त्वा महातेजाः कुम्भकर्णं महाबलम्}
{पुनर्जातमिवात्मानं मेने राक्षसपुङ्गवः} %6-65-15

\twolineshloka
{कुम्भकर्णबलाभिज्ञो जानंस्तस्य पराक्रमम्}
{बभूव मुदितो राजा शशाङ्क इव निर्मलः} %6-65-16

\twolineshloka
{इत्येवमुक्तः संहृष्टो निर्जगाम महाबलः}
{राज्ञस्तु वचनं श्रुत्वा योद्धुमुद्युक्तवांस्तदा} %6-65-17

\twolineshloka
{आददे निशितं शूलं वेगाच्छत्रुनिबर्हणः}
{सर्वं कालायसं दीप्तं तप्तकाञ्चनभूषणम्} %6-65-18

\twolineshloka
{इन्द्राशनिसमप्रख्यं वज्रप्रतिमगौरवम्}
{देवदानवगन्धर्वयक्षपन्नगसूदनम्} %6-65-19

\twolineshloka
{रक्तमाल्यमहादामं स्वतश्चोद्गतपावकम्}
{आदाय विपुलं शूलं शत्रुशोणितरञ्जितम्} %6-65-20

\twolineshloka
{कुम्भकर्णो महातेजा रावणं वाक्यमब्रवीत्}
{गमिष्याम्यहमेकाकी तिष्ठत्विह बलं मम} %6-65-21

\twolineshloka
{अद्य तान् क्षुधितः क्रुद्धो भक्षयिष्यामि वानरान्}
{कुम्भकर्णवचः श्रुत्वा रावणो वाक्यमब्रवीत्} %6-65-22

\twolineshloka
{सैन्यैः परिवृतो गच्छ शूलमुद्गरपाणिभिः}
{वानरा हि महात्मानः शूराः सुव्यवसायिनः} %6-65-23

\threelineshloka
{एकाकिनं प्रमत्तं वा नयेयुर्दशनैः क्षयम्}
{तस्मात् परमदुर्धर्षः सैन्यैः परिवृतो व्रज}
{रक्षसामहितं सर्वं शत्रुपक्षं निषूदय} %6-65-24

\twolineshloka
{अथासनात् समुत्पत्य स्रजं मणिकृतान्तराम्}
{आबबन्ध महातेजाः कुम्भकर्णस्य रावणः} %6-65-25

\twolineshloka
{अङ्गदान्यङ्गुलीवेष्टान् वराण्याभरणानि च}
{हारं च शशिसङ्काशमाबबन्ध महात्मनः} %6-65-26

\twolineshloka
{दिव्यानि च सुगन्धीनि माल्यदामानि रावणः}
{गात्रेषु सज्जयामास श्रोत्रयोश्चास्य कुण्डले} %6-65-27

\twolineshloka
{काञ्चनाङ्गदकेयूरनिष्काभरणभूषितः}
{कुम्भकर्णो बृहत्कर्णः सुहुतोऽग्निरिवाबभौ} %6-65-28

\twolineshloka
{श्रोणीसूत्रेण महता मेचकेन व्यराजत}
{अमृतोत्पादने नद्धो भुजङ्गेनेव मन्दरः} %6-65-29

\twolineshloka
{स काञ्चनं भारसहं निवातं विद्युत्प्रभं दीप्तमिवात्मभासा}
{आबध्यमानः कवचं रराज सन्ध्याभ्रसंवीत इवाद्रिराजः} %6-65-30

\twolineshloka
{सर्वाभरणसर्वाङ्गः शूलपाणिः स राक्षसः}
{त्रिविक्रमकृतोत्साहो नारायण इवाबभौ} %6-65-31

\twolineshloka
{भ्रातरं सम्परिष्वज्य कृत्वा चापि प्रदक्षिणम्}
{प्रणम्य शिरसा तस्मै प्रतस्थे स महाबलः} %6-65-32

\twolineshloka
{तमाशीर्भिः प्रशस्ताभिः प्रेषयामास रावणः}
{शङ्खदुन्दुभिनिर्घोषैः सैन्यैश्चापि वरायुधैः} %6-65-33

\twolineshloka
{तं गजैश्च तुरङ्गैश्च स्यन्दनैश्चाम्बुदस्वनैः}
{अनुजग्मुर्महात्मानो रथिनो रथिनां वरम्} %6-65-34

\twolineshloka
{सर्पैरुष्ट्रैः खरैश्चैव सिंहद्विपमृगद्विजैः}
{अनुजग्मुश्च तं घोरं कुम्भकर्णं महाबलम्} %6-65-35

\twolineshloka
{स पुष्पवर्षैरवकीर्यमाणो धृतातपत्रः शितशूलपाणिः}
{मदोत्कटः शोणितगन्धमत्तो विनिर्ययौ दानवदेवशत्रुः} %6-65-36

\twolineshloka
{पदातयश्च बहवो महानादा महाबलाः}
{अन्वयू राक्षसा भीमा भीमाक्षाः शस्त्रपाणयः} %6-65-37

\twolineshloka
{रक्ताक्षाः सुबहुव्यामा नीलाञ्जनचयोपमाः}
{शूलानुद्यम्य खड्गांश्च निशितांश्च परश्वधान्} %6-65-38

\twolineshloka
{भिन्दिपालांश्च परिघान् गदाश्च मुसलानि च}
{तालस्कन्धांश्च विपुलान् क्षेपणीयान् दुरासदान्} %6-65-39

\twolineshloka
{अथान्यद्वपुरादाय दारुणं घोरदर्शनम्}
{निष्पपात महातेजाः कुम्भकर्णो महाबलः} %6-65-40

\twolineshloka
{धनुःशतपरीणाहः स षट्शतसमुच्छ्रितः}
{रौद्रः शकटचक्राक्षो महापर्वतसन्निभः} %6-65-41

\twolineshloka
{सन्निपत्य च रक्षांसि दग्धशैलोपमो महान्}
{कुम्भकर्णो महावक्त्रः प्रहसन्निदमब्रवीत्} %6-65-42

\twolineshloka
{अद्य वानरमुख्यानां तानि यूथानि भागशः}
{निर्दहिष्यामि सङ्क्रुद्धः पतङ्गानिव पावकः} %6-65-43

\twolineshloka
{नापराध्यन्ति मे कामं वानरा वनचारिणः}
{जातिरस्मद्विधानां सा पुरोद्यानविभूषणम्} %6-65-44

\twolineshloka
{पुररोधस्य मूलं तु राघवः सहलक्ष्मणः}
{हते तस्मिन् हतं सर्वं तं वधिष्यामि संयुगे} %6-65-45

\twolineshloka
{एवं तस्य ब्रुवाणस्य कुम्भकर्णस्य राक्षसाः}
{नादं चक्रुर्महाघोरं कम्पयन्त इवार्णवम्} %6-65-46

\twolineshloka
{तस्य निष्पततस्तूर्णं कुम्भकर्णस्य धीमतः}
{बभूवुर्घोररूपाणि निमित्तानि समन्ततः} %6-65-47

\twolineshloka
{उल्काशनियुता मेघा बभूवुर्गर्दभारुणाः}
{ससागरवना चैव वसुधा समकम्पत} %6-65-48

\twolineshloka
{घोररूपाः शिवा नेदुः सज्वालकवलैर्मुखैः}
{मण्डलान्यपसव्यानि बबन्धुश्च विहङ्गमाः} %6-65-49

\twolineshloka
{निष्पपात च गृध्रोऽस्य शूले वै पथि गच्छतः}
{प्रास्फुरन्नयनं चास्य सव्यो बाहुरकम्पत} %6-65-50

\twolineshloka
{निष्पपात तदा चोल्का ज्वलन्ती भीमनिःस्वना}
{आदित्यो निष्प्रभश्चासीन्न वाति च सुखोऽनिलः} %6-65-51

\twolineshloka
{अचिन्तयन् महोत्पातानुदितान् रोमहर्षणान्}
{निर्ययौ कुम्भकर्णस्तु कृतान्तबलचोदितः} %6-65-52

\twolineshloka
{स लङ्घयित्वा प्राकारं पद्भ्यां पर्वतसन्निभः}
{ददर्शाभ्रघनप्रख्यं वानरानीकमद्भुतम्} %6-65-53

\twolineshloka
{ते दृष्ट्वा राक्षसश्रेष्ठं वानराः पर्वतोपमम्}
{वायुनुन्ना इव घना ययुः सर्वा दिशस्तदा} %6-65-54

\twolineshloka
{तद् वानरानीकमतिप्रचण्डं दिशो द्रवद्भिन्नमिवाभ्रजालम्}
{स कुम्भकर्णः समवेक्ष्य हर्षान्ननाद भूयो घनवद्घनाभः} %6-65-55

\twolineshloka
{ते तस्य घोरं निनदं निशम्य यथा निनादं दिवि वारिदस्य}
{पेतुर्धरण्यां बहवः प्लवङ्गा निकृत्तमूला इव शालवृक्षाः} %6-65-56

\twolineshloka
{विपुलपरिघवान् स कुम्भकर्णो रिपुनिधनाय विनिःसृतो महात्मा}
{कपिगणभयमाददत् सुभीमं प्रभुरिव किङ्करदण्डवान् युगान्ते} %6-65-57


॥इत्यार्षे श्रीमद्रामायणे वाल्मीकीये आदिकाव्ये युद्धकाण्डे कुम्भकर्णाभिषेणनम् नाम पञ्चषष्ठितमः सर्गः ॥६-६५॥
