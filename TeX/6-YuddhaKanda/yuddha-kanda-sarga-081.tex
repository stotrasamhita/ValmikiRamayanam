\sect{एकाशीतितमः सर्गः — मायासीतावधः}

\twolineshloka
{विज्ञाय तु मनस्तस्य राघवस्य महात्मनः}
{स निवृत्याहवात् तस्मात् प्रविवेश पुरं ततः} %6-81-1

\twolineshloka
{सोऽनुस्मृत्य वधं तेषां राक्षसानां तरस्विनाम्}
{क्रोधताम्रेक्षणः शूरो निर्जगामाथ रावणिः} %6-81-2

\twolineshloka
{स पश्चिमेन द्वारेण निर्ययौ राक्षसैर्वृतः}
{इन्द्रजित् सुमहावीर्यः पौलस्त्यो देवकण्टकः} %6-81-3

\twolineshloka
{इन्द्रजित्तु ततो दृष्ट्वा भ्रातरौ रामलक्ष्मणौ}
{रणायाभ्युुद्यतौ वीरौ मायां प्रादुष्करोत् तदा} %6-81-4

\twolineshloka
{इन्द्रजित्तु रथे स्थाप्य सीतां मायामयीं तदा}
{बलेन महतावृत्य तस्या वधमरोचयत्} %6-81-5

\twolineshloka
{मोहनार्थं तु सर्वेषां बुद्धिं कृत्वा सुदुर्मतिः}
{हन्तुं सीतां व्यवसितो वानराभिमुखो ययौ} %6-81-6

\twolineshloka
{तं दृष्ट्वा त्वभिनिर्यान्तं सर्वे ते काननौकसः}
{उत्पेतुरभिसङ्क्रुद्धाः शिलाहस्ता युयुत्सवः} %6-81-7

\twolineshloka
{हनूमान् पुरतस्तेषां जगाम कपिकुञ्जरः}
{प्रगृह्य सुमहच्छृङ्गं पर्वतस्य दुरासदम्} %6-81-8

\twolineshloka
{स ददर्श हतानन्दां सीतामिन्द्रजितो रथे}
{एकवेणीधरां दीनामुपवासकृशाननाम्} %6-81-9

\twolineshloka
{परिक्लिष्टैकवसनाममृजां राघवप्रियाम्}
{रजोमलाभ्यामालिप्तैः सर्वगात्रैर्वरस्त्रियम्} %6-81-10

\twolineshloka
{तां निरीक्ष्य मुहूर्तं तु मैथिलीमध्यवस्य च}
{बभूवाचिरदृष्टा हि तेन सा जनकात्मजा} %6-81-11

\twolineshloka
{अब्रवीत् तां तु शोकार्तां निरानन्दां तपस्विनीम्}
{दृष्ट्वा रथस्थितां दीनां राक्षसेन्द्रसुतश्रिताम्} %6-81-12

\twolineshloka
{किं समर्थितमस्येति चिन्तयन् स महाकपिः}
{सह तैर्वानरश्रेष्ठैरभ्यधावत रावणिम्} %6-81-13

\twolineshloka
{तद् वानरबलं दृष्ट्वा रावणिः क्रोधमूर्च्छितः}
{कृत्वा विकोशं निस्त्रिंशं मूर्ध्नि सीतामकर्षयत्} %6-81-14

\twolineshloka
{तां स्त्रियं पश्यतां तेषां ताडयामास राक्षसः}
{क्रोशन्तीं राम रामेति मायया योजितां रथे} %6-81-15

\twolineshloka
{गृहीतमूर्धजां दृष्ट्वा हनूमान् दैन्यमागतः}
{दुःखजं वारि नेत्राभ्यामुत्सृजन् मारुतात्मजः} %6-81-16

\twolineshloka
{तां दृष्ट्वा चारुसर्वाङ्गीं रामस्य महिषीं प्रियाम्}
{अब्रवीत् परुषं वाक्यं क्रोधाद् रक्षोधिपात्मजम्} %6-81-17

\twolineshloka
{दुरात्मन्नात्मनाशाय केशपक्षे परामृशः}
{ब्रह्मर्षीणां कुले जातो राक्षसीं योनिमाश्रितः} %6-81-18

\threelineshloka
{धिक् त्वां पापसमाचारं यस्य ते मतिरीदृशी}
{नृशंसानार्य दुर्वृत्त क्षुद्र पापपराक्रम}
{अनार्यस्येदृशं कर्म घृणा ते नास्ति निर्घृण} %6-81-19

\twolineshloka
{च्युता गृहाच्च राज्याच्च रामहस्ताच्च मैथिली}
{किं तवैषापराद्धा हि यदेनां हंसि निर्दय} %6-81-20

\twolineshloka
{सीतां हत्वा तु न चिरं जीविष्यसि कथञ्चन}
{वधार्ह कर्मणा तेन मम हस्तगतो ह्यसि} %6-81-21

\twolineshloka
{ये च स्त्रीघातिनां लोका लोकवध्यैश्च कुत्सिताः}
{इह जीवितमुत्सृज्य प्रेत्य तान् प्रति लप्स्यसे} %6-81-22

\twolineshloka
{इति ब्रुवाणो हनुमान् सायुधैर्हरिभिर्वृतः}
{अभ्यधावत् सुसङ्क्रुद्धो राक्षसेन्द्रसुतं प्रति} %6-81-23

\twolineshloka
{आपतन्तं महावीर्यं तदनीकं वनौकसाम्}
{रक्षसां भीमकोपानामनीकेन न्यवारयत्} %6-81-24

\twolineshloka
{स तां बाणसहस्रेण विक्षोभ्य हरिवाहिनीम्}
{हनूमन्तं हरिश्रेष्ठमिन्द्रजित् प्रत्युवाच ह} %6-81-25

\twolineshloka
{सुग्रीवस्त्वं च रामश्च यन्निमित्तमिहागताः}
{तां वधिष्यामि वैदेहीमद्यैव तव पश्यतः} %6-81-26

\twolineshloka
{इमां हत्वा ततो रामं लक्ष्मणं त्वां च वानर}
{सुग्रीवं च वधिष्यामि तं चानार्यं विभीषणम्} %6-81-27

\twolineshloka
{न हन्तव्याः स्त्रियश्चेति यद् ब्रवीषि प्लवङ्गम}
{पीडाकरममित्राणां यच्च कर्तव्यमेव तत्} %6-81-28

\twolineshloka
{तमेवमुक्त्वा रुदतीं सीतां मायामयीं च ताम्}
{शितधारेण खड्गेन निजघानेन्द्रजित् स्वयम्} %6-81-29

\twolineshloka
{यज्ञोपवीतमार्गेण छिन्ना तेन तपस्विनी}
{सा पृथिव्यां पृथुश्रोणी पपात प्रियदर्शना} %6-81-30

\threelineshloka
{तामिन्द्रजित् स्त्रियं हत्वा हनूमन्तमुवाच ह}
{मया रामस्य पश्येमां प्रियां शस्त्रनिषूदिताम्}
{एषा विशस्ता वैदेही निष्फलो वः परिश्रमः} %6-81-31

\twolineshloka
{ततः खड्गेन महता हत्वा तामिन्द्रजित्स्वयम्}
{हृष्टः स रथमास्थाय ननाद च महास्वनम्} %6-81-32

\twolineshloka
{वानराः शुश्रुवुः शब्दमदूरे प्रत्यवस्थिताः}
{व्यादितास्यस्य नदतस्तद्दुर्गं संश्रितस्य तु} %6-81-33

\twolineshloka
{तथा तु सीतां विनिहत्य दुर्मतिः प्रहृष्टचेताः स बभूव रावणिः}
{तं हृष्टरूपं समुदीक्ष्य वानरा विषण्णरूपाः समभिप्रदुद्रुवुः} %6-81-34


॥इत्यार्षे श्रीमद्रामायणे वाल्मीकीये आदिकाव्ये युद्धकाण्डे मायासीतावधः नाम एकाशीतितमः सर्गः ॥६-८१॥
