\sect{चतुःपञ्चाशः सर्गः — वज्रदंष्ट्रवधः}

\twolineshloka
{स्वबलस्य च घातेन अङ्गदस्य बलेन च}
{राक्षसः क्रोधमाविष्टो वज्रदंष्ट्रो महाबलः} %6-54-1

\twolineshloka
{विस्फार्य च धनुर्घोरं शक्राशनिसमप्रभम्}
{वानराणामनीकानि प्राकिरच्छरवृष्टिभिः} %6-54-2

\twolineshloka
{राक्षसाश्चापि मुख्यास्ते रथेषु समवस्थिताः}
{नानाप्रहरणाः शूराः प्रायुध्यन्त तदा रणे} %6-54-3

\twolineshloka
{वानराणां च शूरास्तु ते सर्वे प्लवगर्षभाः}
{अयुध्यन्त शिलाहस्ताः समवेताः समन्ततः} %6-54-4

\twolineshloka
{तत्रायुधसहस्राणि तस्मिन्नायोधने भृशम्}
{राक्षसाः कपिमुख्येषु पातयाञ्चक्रिरे तदा} %6-54-5

\twolineshloka
{वानराश्चैव रक्षःसु गिरिवृक्षान् महाशिलाः}
{प्रवीराः पातयामासुर्मत्तवारणसन्निभाः} %6-54-6

\twolineshloka
{शूराणां युध्यमानानां समरेष्वनिवर्तिनाम्}
{तद् राक्षसगणानां च सुयुद्धं समवर्तत} %6-54-7

\twolineshloka
{प्रभिन्नशिरसः केचिच्छिन्नैः पादैश्च बाहुभिः}
{शस्त्रैरर्दितदेहास्तु रुधिरेण समुक्षिताः} %6-54-8

\twolineshloka
{हरयो राक्षसाश्चैव शेरते गां समाश्रिताः}
{कङ्कगृध्रबलाढ्याश्च गोमायुकुलसङ्कुलाः} %6-54-9

\twolineshloka
{कबन्धानि समुत्पेतुर्भीरूणां भीषणानि वै}
{भुजपाणिशिरश्छिन्नाश्छिन्नकायाश्च भूतले} %6-54-10

\twolineshloka
{वानरा राक्षसाश्चापि निपेतुस्तत्र भूतले}
{ततो वानरसैन्येन हन्यमानं निशाचरम्} %6-54-11

\twolineshloka
{प्राभज्यत बलं सर्वं वज्रदंष्ट्रस्य पश्यतः}
{राक्षसान् भयवित्रस्तान् हन्यमानान् प्लवङ्गमैः} %6-54-12

\twolineshloka
{दृष्ट्वा स रोषताम्राक्षो वज्रदंष्ट्रः प्रतापवान्}
{प्रविवेश धनुष्पाणिस्त्रासयन् हरिवाहिनीम्} %6-54-13

\twolineshloka
{शरैर्विदारयामास कङ्कपत्रैरजिह्मगैः}
{बिभेद वानरांस्तत्र सप्ताष्टौ नव पञ्च च} %6-54-14

\threelineshloka
{विव्याध परमक्रुद्धो वज्रदंष्ट्रः प्रतापवान्}
{त्रस्ताः सर्वे हरिगणाः शरैः सङ्कृत्तदेहिनः}
{अङ्गदं सम्प्रधावन्ति प्रजापतिमिव प्रजाः} %6-54-15

\twolineshloka
{ततो हरिगणान् भग्नान् दृष्ट्वा वालिसुतस्तदा}
{क्रोधेन वज्रदंष्ट्रं तमुदीक्षन्तमुदैक्षत} %6-54-16

\twolineshloka
{वज्रदंष्ट्रोऽङ्गदश्चोभौ योयुध्येते परस्परम्}
{चेरतुः परमक्रुद्धौ हरिमत्तगजाविव} %6-54-17

\twolineshloka
{ततः शतसहस्रेण हरिपुत्रं महाबलम्}
{जघान मर्मदेशेषु शरैरग्निशिखोपमैः} %6-54-18

\twolineshloka
{रुधिरोक्षितसर्वाङ्गो वालिसूनुर्महाबलः}
{चिक्षेप वज्रदंष्ट्राय वृक्षं भीमपराक्रमः} %6-54-19

\twolineshloka
{दृष्ट्वा पतन्तं तं वृक्षमसम्भ्रान्तश्च राक्षसः}
{चिच्छेद बहुधा सोऽपि मथितः प्रापतद् भुवि} %6-54-20

\twolineshloka
{तं दृष्ट्वा वज्रदंष्ट्रस्य विक्रमं प्लवगर्षभः}
{प्रगृह्य विपुलं शैलं चिक्षेप च ननाद च} %6-54-21

\twolineshloka
{तमापतन्तं दृष्ट्वा स रथादाप्लुत्य वीर्यवान्}
{गदापाणिरसम्भ्रान्तः पृथिव्यां समतिष्ठत} %6-54-22

\twolineshloka
{अङ्गदेन शिला क्षिप्ता गत्वा तु रणमूर्धनि}
{सचक्रकूबरं साश्वं प्रममाथ रथं तदा} %6-54-23

\twolineshloka
{ततोऽन्यच्छिखरं गृह्य विपुलं द्रुमभूषितम्}
{वज्रदंष्ट्रस्य शिरसि पातयामास वानरः} %6-54-24

\twolineshloka
{अभवच्छोणितोद्गारी वज्रदंष्ट्रः सुमूर्च्छितः}
{मुहूर्तमभवन्मूढो गदामालिङ्ग्य निःश्वसन्} %6-54-25

\twolineshloka
{स लब्धसंज्ञो गदया वालिपुत्रमवस्थितम्}
{जघान परमक्रुद्धो वक्षोदेशे निशाचरः} %6-54-26

\twolineshloka
{गदां त्यक्त्वा ततस्तत्र मुष्टियुद्धमकुर्वत}
{अन्योन्यं जघ्नतुस्तत्र तावुभौ हरिराक्षसौ} %6-54-27

\twolineshloka
{रुधिरोद्गारिणौ तौ तु प्रहारैर्जनितश्रमौ}
{बभूवतुः सुविक्रान्तावङ्गारकबुधाविव} %6-54-28

\twolineshloka
{ततः परमतेजस्वी अङ्गदः प्लवगर्षभः}
{उत्पाट्य वृक्षं स्थितवानासीत् पुष्पफलैर्युतः} %6-54-29

\twolineshloka
{जग्राह चार्षभं चर्म खड्गं च विपुलं शुभम्}
{किङ्किणीजालसञ्छन्नं चर्मणा च परिष्कृतम्} %6-54-30

\twolineshloka
{चित्रांश्च रुचिरान् मार्गांश्चेरतुः कपिराक्षसौ}
{जघ्नतुश्च तदान्योन्यं नर्दन्तौ जयकाङ्क्षिणौ} %6-54-31

\twolineshloka
{व्रणैः सास्रैरशोभेतां पुष्पिताविव किंशुकौ}
{युध्यमानौ परिश्रान्तौ जानुभ्यामवनीं गतौ} %6-54-32

\twolineshloka
{निमेषान्तरमात्रेण अङ्गदः कपिकुञ्जरः}
{उदतिष्ठत दीप्ताक्षो दण्डाहत इवोरगः} %6-54-33

\twolineshloka
{निर्मलेन सुधौतेन खड्गेनास्य महच्छिरः}
{जघान वज्रदंष्ट्रस्य वालिसूनुर्महाबलः} %6-54-34

\twolineshloka
{रुधिरोक्षितगात्रस्य बभूव पतितं द्विधा}
{तच्च तस्य परीताक्षं शुभं खड्गहतं शिरः} %6-54-35

\threelineshloka
{वज्रदंष्ट्रं हतं दृष्ट्वा राक्षसा भयमोहिताः}
{त्रस्ता ह्यभ्यद्रवँल्लङ्कां वध्यमानाः प्लवङ्गमैः}
{विषण्णवदना दीना ह्रिया किञ्चिदवाङ्मुखाः} %6-54-36

\twolineshloka
{निहत्य तं वज्रधरः प्रतापवान् स वालिसूनुः कपिसैन्यमध्ये}
{जगाम हर्षं महितो महाबलः सहस्रनेत्रस्त्रिदशैरिवावृतः} %6-54-37


॥इत्यार्षे श्रीमद्रामायणे वाल्मीकीये आदिकाव्ये युद्धकाण्डे वज्रदंष्ट्रवधः नाम चतुःपञ्चाशः सर्गः ॥६-५४॥
