\sect{एकोनचत्वारिंशः सर्गः — लङ्कादर्शनम्}

\twolineshloka
{तां रात्रिमुषितास्तत्र सुवेले हरियूथपाः}
{लङ्कायां ददृशुर्वीरा वनान्युपवनानि च} %6-39-1

\twolineshloka
{समसौम्यानि रम्याणि विशालान्यायतानि च}
{दृष्टिरम्याणि ते दृष्ट्वा बभूवुर्जातविस्मयाः} %6-39-2

\twolineshloka
{चम्पकाशोकबकुलशालतालसमाकुला}
{तमालवनसञ्छन्ना नागमालासमावृता} %6-39-3

\twolineshloka
{हिन्तालैरर्जुनैर्नीपैः सप्तपर्णैः सुपुष्पितैः}
{तिलकैः कर्णिकारैश्च पाटलैश्च समन्ततः} %6-39-4

\twolineshloka
{शुशुभे पुष्पिताग्रैश्च लतापरिगतैर्द्रुमैः}
{लङ्का बहुविधैर्दिव्यैर्यथेन्द्रस्यामरावती} %6-39-5

\twolineshloka
{विचित्रकुसुमोपेतै रक्तकोमलपल्लवैः}
{शाद्वलैश्च तथा नीलैश्चित्राभिर्वनराजिभिः} %6-39-6

\twolineshloka
{गन्धाढ्यान्यतिरम्याणि पुष्पाणि च फलानि च}
{धारयन्त्यगमास्तत्र भूषणानीव मानवाः} %6-39-7

\twolineshloka
{तच्चैत्ररथसङ्काशं मनोज्ञं नन्दनोपमम्}
{वनं सर्वर्तुकं रम्यं शुशुभे षट्पदायुतम्} %6-39-8

\twolineshloka
{दात्यूहकोयष्टिबकैर्नृत्यमानैश्च बर्हिणैः}
{रुतं परभृतानां च शुश्रुवे वननिर्झरे} %6-39-9

\twolineshloka
{नित्यमत्तविहङ्गानि भ्रमराचरितानि च}
{कोकिलाकुलखण्डानि विहङ्गाभिरुतानि च} %6-39-10

\threelineshloka
{भृङ्गराजाधिगीतानि कुररस्वनितानि च}
{कोणालकविघुष्टानि सारसाभिरुतानि च}
{विविशुस्ते ततस्तानि वनान्युपवनानि च} %6-39-11

\twolineshloka
{हृष्टाः प्रमुदिता वीरा हरयः कामरूपिणः}
{तेषां प्रविशतां तत्र वानराणां महौजसाम्} %6-39-12

\threelineshloka
{पुष्पसंसर्गसुरभिर्ववौ घ्राणसुखोऽनिलः}
{अन्ये तु हरिवीराणां यूथान्निष्क्रम्य यूथपाः}
{सुग्रीवेणाभ्यनुज्ञाता लङ्कां जग्मुः पताकिनीम्} %6-39-13

\twolineshloka
{वित्रासयन्तो विहगान् ग्लापयन्तो मृगद्विपान्}
{कम्पयन्तश्च तां लङ्कां नादैः स्वैर्नदतां वराः} %6-39-14

\twolineshloka
{कुर्वन्तस्ते महावेगा महीं चरणपीडिताम्}
{रजश्च सहसैवोर्ध्वं जगाम चरणोत्थितम्} %6-39-15

\twolineshloka
{ऋक्षाः सिंहाश्च महिषा वारणाश्च मृगाः खगाः}
{तेन शब्देन वित्रस्ता जग्मुर्भीता दिशो दश} %6-39-16

\twolineshloka
{शिखरं तु त्रिकूटस्य प्रांशु चैकं दिविस्पृशम्}
{समन्तात् पुष्पसञ्छन्नं महारजतसन्निभम्} %6-39-17

\twolineshloka
{शतयोजनविस्तीर्णं विमलं चारुदर्शनम्}
{श्लक्ष्णं श्रीमन्महच्चैव दुष्प्रापं शकुनैरपि} %6-39-18

\twolineshloka
{मनसापि दुरारोहं किं पुनः कर्मणा जनैः}
{निविष्टा तस्य शिखरे लङ्का रावणपालिता} %6-39-19

\threelineshloka
{दशयोजनविस्तीर्णा विंशद्योजनमायता}
{सा पुरी गोपुरैरुच्चैः पाण्डुराम्बुदसन्निभैः}
{काञ्चनेन च शालेन राजतेन च शोभते} %6-39-20

\twolineshloka
{प्रासादैश्च विमानैश्च लङ्का परमभूषिता}
{घनैरिवातपापाये मध्यमं वैष्णवं पदम्} %6-39-21

\twolineshloka
{यस्यां स्तम्भसहस्रेण प्रासादः समलङ्कृतः}
{कैलासशिखराकारो दृश्यते खमिवोल्लिखन्} %6-39-22

\twolineshloka
{चैत्यः स राक्षसेन्द्रस्य बभूव पुरभूषणम्}
{शतेन रक्षसां नित्यं यः समग्रेण रक्ष्यते} %6-39-23

\twolineshloka
{मनोज्ञां काञ्चनवतीं पर्वतैरुपशोभिताम्}
{नानाधातुविचित्रैश्च उद्यानैरुपशोभिताम्} %6-39-24

\twolineshloka
{नानाविहगसङ्घुष्टां नानामृगनिषेविताम्}
{नानाकुसुमसम्पन्नां नानाराक्षससेविताम्} %6-39-25

\twolineshloka
{तां समृद्धां समृद्धार्थां लक्ष्मीवाँल्लक्ष्मणाग्रजः}
{रावणस्य पुरीं रामो ददर्श सह वानरैः} %6-39-26

\twolineshloka
{तां महागृहसम्बाधां दृष्ट्वा लक्ष्मणपूर्वजः}
{नगरीं त्रिदिवप्रख्यां विस्मयं प्राप वीर्यवान्} %6-39-27

\twolineshloka
{तां रत्नपूर्णां बहुसंविधानां प्रासादमालाभिरलङ्कृतां च}
{पुरीं महायन्त्रकवाटमुख्यां ददर्श रामो महता बलेन} %6-39-28


॥इत्यार्षे श्रीमद्रामायणे वाल्मीकीये आदिकाव्ये युद्धकाण्डे लङ्कादर्शनम् नाम एकोनचत्वारिंशः सर्गः ॥६-३९॥
