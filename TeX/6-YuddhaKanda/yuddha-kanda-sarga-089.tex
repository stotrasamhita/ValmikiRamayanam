\sect{एकोननवतितमः सर्गः — सौमित्रिसन्धुक्षणम्}

\twolineshloka
{ततः शरान् दाशरथिः सन्धायामित्रकर्षणः}
{ससर्ज राक्षसेन्द्राय क्रुद्धः सर्प इव श्वसन्} %6-89-1

\twolineshloka
{तस्य ज्यातलनिर्घोषं स श्रुत्वा राक्षसाधिपः}
{विवर्णवदनो भूत्वा लक्ष्मणं समुदैक्षत} %6-89-2

\twolineshloka
{विवर्णवदनं दृष्ट्वा राक्षसं रावणात्मजम्}
{सौमित्रिं युद्धसंयुक्तं प्रत्युवाच विभीषणः} %6-89-3

\twolineshloka
{निमित्तान्युपपश्यामि यान्यस्मिन् रावणात्मजे}
{त्वर तेन महाबाहो भग्न एष न संशयः} %6-89-4

\twolineshloka
{ततः सन्धाय सौमित्रिः शरानाशीविषोपमान्}
{मुमोच विशिखांस्तस्मिन् सर्पानिव विषोल्बणान्} %6-89-5

\twolineshloka
{शक्राशनिसमस्पर्शैर्लक्ष्मणेनाहतः शरैः}
{मुहूर्तमभवन्मूढः सर्वसङ्क्षुभितेन्द्रियः} %6-89-6

\threelineshloka
{उपलभ्य मुहूर्तेन संज्ञां प्रत्यागतेन्द्रियः}
{ददर्शावस्थितं वीरमाजौ दशरथात्मजम्}
{सोऽभिचक्राम सौमित्रिं रोषात् संरक्तलोचनः} %6-89-7

\threelineshloka
{अब्रवीच्चैनमासाद्य पुनः स परुषं वचः}
{किं न स्मरसि तद् युद्धे प्रथमे मत्पराक्रमम्}
{निबद्धस्त्वं सह भ्रात्रा यदा युधि विचेष्टसे} %6-89-8

\twolineshloka
{युवां खलु महायुद्धे वज्राशनिसमैः शरैः}
{शायितौ प्रथमं भूमौ विसंज्ञौ सपुरःसरौ} %6-89-9

\twolineshloka
{स्मृतिर्वा नास्ति ते मन्ये व्यक्तं वा यमसादनम्}
{गन्तुमिच्छसि यन्मां त्वमाधर्षयितुमिच्छसि} %6-89-10

\twolineshloka
{यदि ते प्रथमे युद्धे न दृष्टो मत्पराक्रमः}
{अद्य त्वां दर्शयिष्यामि तिष्ठेदानीं व्यवस्थितः} %6-89-11

\twolineshloka
{इत्युक्त्वा सप्तभिर्बाणैरभिविव्याध लक्ष्मणम्}
{दशभिस्तु हनूमन्तं तीक्ष्णधारैः शरोत्तमैः} %6-89-12

\twolineshloka
{ततः शरशतेनैव सुप्रयुक्तेन वीर्यवान्}
{क्रोधाद् द्विगुणसंरब्धो निर्बिभेद विभीषणम्} %6-89-13

\twolineshloka
{तद् दृष्ट्वेन्द्रजिता कर्म कृतं रामानुजस्तदा}
{अचिन्तयित्वा प्रहसन्नैतत् किञ्चिदिति ब्रुवन्} %6-89-14

\twolineshloka
{मुमोच च शरान् घोरान् सङ्गृह्य नरपुङ्गवः}
{अभीतवदनः क्रुद्धो रावणिं लक्ष्मणो युधि} %6-89-15

\twolineshloka
{नैवं रणगताः शूराः प्रहरन्ति निशाचर}
{लघवश्चाल्पवीर्याश्च शरा हीमे सुखास्तव} %6-89-16

\twolineshloka
{नैवं शूरास्तु युध्यन्ते समरे युद्धकाङ्क्षिणः}
{इत्येवं तं ब्रुवन् धन्वी शरैरभिववर्ष ह} %6-89-17

\twolineshloka
{तस्य बाणैः सुविध्वस्तं कवचं काञ्चनं महत्}
{व्यशीर्यत रथोपस्थे ताराजालमिवाम्बरात्} %6-89-18

\twolineshloka
{विधूतवर्मा नाराचैर्बभूव स कृतव्रणः}
{इन्द्रजित् समरे वीरः प्रत्यूषे भानुमानिव} %6-89-19

\twolineshloka
{ततः शरसहस्रेण सङ्क्रुद्धो रावणात्मजः}
{बिभेद समरे वीरो लक्ष्मणं भीमविक्रमः} %6-89-20

\twolineshloka
{व्यशीर्यत महद्दिव्यं कवचं लक्ष्मणस्य तु}
{कृतप्रतिकृतान्योन्यं बभूवतुररिन्दमौ} %6-89-21

\twolineshloka
{अभीक्ष्णं निःश्वसन्तौ तौ युध्येतां तुमुलं युधि}
{शरसङ्कृत्तसर्वाङ्गौ सर्वतो रुधिरोक्षितौ} %6-89-22

\threelineshloka
{सुदीर्घकालं तौ वीरावन्योन्यं निशितैः शरैः}
{ततक्षतुर्महात्मानौ रणकर्मविशारदौ}
{बभूवतुश्चात्मजये यत्तौ भीमपराक्रमौ} %6-89-23

\twolineshloka
{तौ शरौघैस्तथाकीर्णौ निकृत्तकवचध्वजौ}
{सृजन्तौ रुधिरं चोष्णं जलं प्रस्रवणाविव} %6-89-24

\twolineshloka
{शरवर्षं ततो घोरं मुञ्चतोर्भीमनिःस्वनम्}
{सासारयोरिवाकाशे नीलयोः कालमेघयोः} %6-89-25

\twolineshloka
{तयोरथ महान् कालो व्यतीयाद् युध्यमानयोः}
{न च तौ युद्धवैमुख्यं क्लमं चाप्युपजग्मतुः} %6-89-26

\twolineshloka
{अस्त्राण्यस्त्रविदां श्रेष्ठौ दर्शयन्तौ पुनः पुनः}
{शरानुच्चावचाकारानन्तरिक्षे बबन्धतुः} %6-89-27

\twolineshloka
{व्यपेतदोषमस्यन्तौ लघु चित्रं च सुष्ठु च}
{उभौ तु तुमुलं घोरं चक्रतुर्नरराक्षसौ} %6-89-28

\twolineshloka
{तयोः पृथक् पृथग् भीमः शुश्रुवे तलनिस्वनः}
{स कम्पं जनयामास निर्घात इव दारुणः} %6-89-29

\twolineshloka
{तयोः स भ्राजते शब्दस्तथा समरमत्तयोः}
{सुघोरयोर्निष्टनतोर्गगने मेघयोरिव} %6-89-30

\twolineshloka
{सुवर्णपुङ्खैर्नाराचैर्बलवन्तौ कृतव्रणौ}
{प्रसुस्रुवाते रुधिरं कीर्तिमन्तौ जये धृतौ} %6-89-31

\twolineshloka
{ते गात्रयोर्निपतिता रुक्मपुङ्खाः शरा युधि}
{असृग्दिग्धा विनिष्पेतुर्विविशुर्धरणीतलम्} %6-89-32

\twolineshloka
{अन्ये सुनिशितैः शस्त्रैराकाशे सञ्जघट्टिरे}
{बभञ्जुश्चिच्छिदुश्चैव तयोर्बाणाः सहस्रशः} %6-89-33

\twolineshloka
{स बभूव रणो घोरस्तयोर्बाणमयश्चयः}
{अग्निभ्यामिव दीप्ताभ्यां सत्रे कुशमयश्चयः} %6-89-34

\twolineshloka
{तयोः कृतव्रणौ देहौ शुशुभाते महात्मनोः}
{सुपुष्पाविव निष्पत्रौ वने किंशुकशाल्मली} %6-89-35

\twolineshloka
{चक्रतुस्तुमुलं घोरं सन्निपातं मुहुर्मुहुः}
{इन्द्रजिल्लक्ष्मणश्चैव परस्परजयैषिणौ} %6-89-36

\twolineshloka
{लक्ष्मणो रावणिं युद्धे रावणिश्चापि लक्ष्मणम्}
{अन्योन्यं तावभिघ्नन्तौ न श्रमं प्रतिपद्यताम्} %6-89-37

\twolineshloka
{बाणजालैः शरीरस्थैरवगाढैस्तरस्विनौ}
{शुशुभाते महावीर्यौ प्ररूढाविव पर्वतौ} %6-89-38

\twolineshloka
{तयो रुधिरसिक्तानि संवृतानि शरैर्भृशम्}
{बभ्राजुः सर्वगात्राणि ज्वलन्त इव पावकाः} %6-89-39

\twolineshloka
{तयोरथ महान् कालो व्यतीयाद् युध्यमानयोः}
{न च तौ युद्धवैमुख्यं श्रमं चाप्यभिजग्मतुः} %6-89-40

\twolineshloka
{अथ समरपरिश्रमं निहन्तुं समरमुखेष्वजितस्य लक्ष्मणस्य}
{प्रियहितमुपपादयन् महात्मा समरमुपेत्य विभीषणोऽवतस्थे} %6-89-41


॥इत्यार्षे श्रीमद्रामायणे वाल्मीकीये आदिकाव्ये युद्धकाण्डे सौमित्रिसन्धुक्षणम् नाम एकोननवतितमः सर्गः ॥६-८९॥
