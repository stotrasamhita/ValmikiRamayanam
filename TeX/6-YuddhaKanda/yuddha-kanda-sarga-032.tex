\sect{द्वात्रिंशः सर्गः — सीताविलापः}

\twolineshloka
{सा सीता तच्छिरो दृष्ट्वा तच्च कार्मुकमुत्तमम्}
{सुग्रीवप्रतिसंसर्गमाख्यातं च हनूमता} %6-32-1

\twolineshloka
{नयने मुखवर्णं च भर्तुस्तत्सदृशं मुखम्}
{केशान् केशान्तदेशं च तं च चूडामणिं शुभम्} %6-32-2

\twolineshloka
{एतैः सर्वैरभिज्ञानैरभिज्ञाय सुदुःखिता}
{विजगर्हेऽत्र कैकेयीं क्रोशन्ती कुररी यथा} %6-32-3

\twolineshloka
{सकामा भव कैकेयि हतोऽयं कुलनन्दनः}
{कुलमुत्सादितं सर्वं त्वया कलहशीलया} %6-32-4

\twolineshloka
{आर्येण किं नु कैकेय्याः कृतं रामेण विप्रियम्}
{यन्मया चीरवसनं दत्त्वा प्रव्राजितो वनम्} %6-32-5

\twolineshloka
{एवमुक्त्वा तु वैदेही वेपमाना तपस्विनी}
{जगाम जगतीं बाला छिन्ना तु कदली यथा} %6-32-6

\twolineshloka
{सा मुहूर्तात् समाश्वस्य परिलभ्याथ चेतनाम्}
{तच्छिरः समुपास्थाय विललापायतेक्षणा} %6-32-7

\twolineshloka
{हा हतास्मि महाबाहो वीरव्रतमनुव्रत}
{इमां ते पश्चिमावस्थां गतास्मि विधवा कृता} %6-32-8

\twolineshloka
{प्रथमं मरणं नार्या भर्तुर्वैगुण्यमुच्यते}
{सुवृत्तः साधुवृत्तायाः संवृत्तस्त्वं ममाग्रतः} %6-32-9

\twolineshloka
{महद् दुःखं प्रपन्नाया मग्नायाः शोकसागरे}
{यो हि मामुद्यतस्त्रातुं सोऽपि त्वं विनिपातितः} %6-32-10

\twolineshloka
{सा श्वश्रूर्मम कौसल्या त्वया पुत्रेण राघव}
{वत्सेनेव यथा धेनुर्विवत्सा वत्सला कृता} %6-32-11

\twolineshloka
{उद्दिष्टं दीर्घमायुस्ते दैवज्ञैरपि राघव}
{अनृतं वचनं तेषामल्पायुरसि राघव} %6-32-12

\twolineshloka
{अथवा नश्यति प्रज्ञा प्राज्ञस्यापि सतस्तव}
{पचत्येनं तथा कालो भूतानां प्रभवो ह्ययम्} %6-32-13

\twolineshloka
{अदृष्टं मृत्युमापन्नः कस्मात् त्वं नयशास्त्रवित्}
{व्यसनानामुपायज्ञः कुशलो ह्यसि वर्जने} %6-32-14

\twolineshloka
{तथा त्वं सम्परिष्वज्य रौद्रयातिनृशंसया}
{कालरात्र्या ममाच्छिद्य हृतः कमललोचन} %6-32-15

\twolineshloka
{इह शेषे महाबाहो मां विहाय तपस्विनीम्}
{प्रियामिव यथा नारीं पृथिवीं पुरुषर्षभ} %6-32-16

\twolineshloka
{अर्चितं सततं यत्नाद् गन्धमाल्यैर्मया तव}
{इदं ते मत्प्रियं वीर धनुः काञ्चनभूषितम्} %6-32-17

\twolineshloka
{पित्रा दशरथेन त्वं श्वशुरेण ममानघ}
{सर्वैश्च पितृभिः सार्धं नूनं स्वर्गे समागतः} %6-32-18

\twolineshloka
{दिवि नक्षत्रभूतं च महत्कर्मकृतं तथा}
{पुण्यं राजर्षिवंशं त्वमात्मनः समुपेक्षसे} %6-32-19

\twolineshloka
{किं मां न प्रेक्षसे राजन् किं वा न प्रतिभाषसे}
{बालां बालेन सम्प्राप्तां भार्यां मां सहचारिणीम्} %6-32-20

\twolineshloka
{संश्रुतं गृह्णता पाणिं चरिष्यामीति यत् त्वया}
{स्मर तन्नाम काकुत्स्थ नय मामपि दुःखिताम्} %6-32-21

\twolineshloka
{कस्मान्मामपहाय त्वं गतो गतिमतां वर}
{अस्माल्लोकादमुं लोकं त्यक्त्वा मामपि दुःखिताम्} %6-32-22

\twolineshloka
{कल्याणै रुचिरं गात्रं परिष्वक्तं मयैव तु}
{क्रव्यादैस्तच्छरीरं ते नूनं विपरिकृष्यते} %6-32-23

\twolineshloka
{अग्निष्टोमादिभिर्यज्ञैरिष्टवानाप्तदक्षिणैः}
{अग्निहोत्रेण संस्कारं केन त्वं न तु लप्स्यसे} %6-32-24

\twolineshloka
{प्रव्रज्यामुपपन्नानां त्रयाणामेकमागतम्}
{परिप्रेक्ष्यति कौसल्या लक्ष्मणं शोकलालसा} %6-32-25

\twolineshloka
{स तस्याः परिपृच्छन्त्या वधं मित्रबलस्य ते}
{तव चाख्यास्यते नूनं निशायां राक्षसैर्वधम्} %6-32-26

\twolineshloka
{सा त्वां सुप्तं हतं ज्ञात्वा मां च रक्षोगृहं गताम्}
{हृदयेनावदीर्णेन न भविष्यति राघव} %6-32-27

\twolineshloka
{मम हेतोरनार्याया अनघः पार्थिवात्मजः}
{रामः सागरमुत्तीर्य वीर्यवान् गोष्पदे हतः} %6-32-28

\twolineshloka
{अहं दाशरथेनोढा मोहात् स्वकुलपांसनी}
{आर्यपुत्रस्य रामस्य भार्या मृत्युरजायत} %6-32-29

\twolineshloka
{नूनमन्यां मया जातिं वारितं दानमुत्तमम्}
{याहमद्यैव शोचामि भार्या सर्वातिथेरिह} %6-32-30

\twolineshloka
{साधु घातय मां क्षिप्रं रामस्योपरि रावण}
{समानय पतिं पत्न्या कुरु कल्याणमुत्तमम्} %6-32-31

\twolineshloka
{शिरसा मे शिरश्चास्य कायं कायेन योजय}
{रावणानुगमिष्यामि गतिं भर्तुर्महात्मनः} %6-32-32

\twolineshloka
{इतीव दुःखसन्तप्ता विललापायतेक्षणा}
{भर्तुः शिरो धनुश्चैव ददर्श जनकात्मजा} %6-32-33

\twolineshloka
{एवं लालप्यमानायां सीतायां तत्र राक्षसः}
{अभिचक्राम भर्तारमनीकस्थः कृताञ्जलिः} %6-32-34

\twolineshloka
{विजयस्वार्यपुत्रेति सोऽभिवाद्य प्रसाद्य च}
{न्यवेदयदनुप्राप्तं प्रहस्तं वाहिनीपतिम्} %6-32-35

\twolineshloka
{अमात्यैः सहितः सर्वैः प्रहस्तस्त्वामुपस्थितः}
{तेन दर्शनकामेन अहं प्रस्थापितः प्रभो} %6-32-36

\twolineshloka
{नूनमस्ति महाराज राजभावात् क्षमान्वित}
{किञ्चिदात्ययिकं कार्यं तेषां त्वं दर्शनं कुरु} %6-32-37

\twolineshloka
{एतच्छ्रुत्वा दशग्रीवो राक्षसप्रतिवेदितम्}
{अशोकवनिकां त्यक्त्वा मन्त्रिणां दर्शनं ययौ} %6-32-38

\twolineshloka
{स तु सर्वं समर्थ्यैव मन्त्रिभिः कृत्यमात्मनः}
{सभां प्रविश्य विदधे विदित्वा रामविक्रमम्} %6-32-39

\twolineshloka
{अन्तर्धानं तु तच्छीर्षं तच्च कार्मुकमुत्तमम्}
{जगाम रावणस्यैव निर्याणसमनन्तरम्} %6-32-40

\twolineshloka
{राक्षसेन्द्रस्तु तैः सार्धं मन्त्रिभिर्भीमविक्रमैः}
{समर्थयामास तदा रामकार्यविनिश्चयम्} %6-32-41

\twolineshloka
{अविदूरस्थितान् सर्वान् बलाध्यक्षान् हितैषिणः}
{अब्रवीत् कालसदृशं रावणो राक्षसाधिपः} %6-32-42

\twolineshloka
{शीघ्रं भेरीनिनादेन स्फुटं कोणाहतेन मे}
{समानयध्वं सैन्यानि वक्तव्यं च न कारणम्} %6-32-43

\twolineshloka
{ततस्तथेति प्रतिगृह्य तद्वचस्तदैव दूताः सहसा महद् बलम्}
{समानयंश्चैव समागतं च न्यवेदयन् भर्तरि युद्धकाङ्क्षिणि} %6-32-44


॥इत्यार्षे श्रीमद्रामायणे वाल्मीकीये आदिकाव्ये युद्धकाण्डे सीताविलापः नाम द्वात्रिंशः सर्गः ॥६-३२॥
