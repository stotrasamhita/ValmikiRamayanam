\sect{षडधिकशततमः सर्गः — सारथिविज्ञेयम्}

\twolineshloka
{स तु मोहात् सुसंक्रुद्धः कृतान्तबलचोदितः}
{क्रोधसंरक्तनयनो रावणः सूतमब्रवीत्} %6-106-1

\twolineshloka
{हीनवीर्यमिवाशक्तं पौरुषेण विवर्जितम्}
{भीरुं लघुमिवासत्त्वं विहीनमिव तेजसा} %6-106-2

\twolineshloka
{विमुक्तमिव मायाभिरस्त्रैरिव बहिष्कृतम्}
{मामवज्ञाय दुर्बुद्धे स्वया बुद्ध्या विचेष्टसे} %6-106-3

\twolineshloka
{किमर्थं मामवज्ञाय मच्छन्दमनवेक्ष्य च}
{त्वया शत्रुसमक्षं मे रथोऽयमपवाहितः} %6-106-4

\twolineshloka
{त्वयाद्य हि ममानार्य चिरकालमुपार्जितम्}
{यशो वीर्यं च तेजश्च प्रत्ययश्च विनाशितः} %6-106-5

\twolineshloka
{शत्रोः प्रख्यातवीर्यस्य रञ्जनीयस्य विक्रमैः}
{पश्यतो युद्धलुब्धोऽहं कृतः कापुरुषस्त्वया} %6-106-6

\twolineshloka
{यत् त्वं कथमिदं मोहान्न चेद् वहसि दुर्मते}
{सत्योऽयं प्रतितर्को मे परेण त्वमुपस्कृतः} %6-106-7

\twolineshloka
{नहि तद् विद्यते कर्म सुहृदो हितकांक्षिणः}
{रिपूणां सदृशं त्वेतद् यत् त्वयैतदनुष्ठितम्} %6-106-8

\twolineshloka
{निवर्तय रथं शीघ्रं यावन्नापैति मे रिपुः}
{यदि वाध्युषितोऽसि त्वं स्मर्यते यदि मे गुणः} %6-106-9

\twolineshloka
{एवं परुषमुक्तस्तु हितबुद्धिरबुद्धिना}
{अब्रवीद् रावणं सूतो हितं सानुनयं वचः} %6-106-10

\twolineshloka
{न भीतोऽस्मि न मूढोऽस्मि नोपजप्तोऽस्मि शत्रुभिः}
{न प्रमत्तो न निःस्नेहो विस्मृता न च सत्क्रिया} %6-106-11

\twolineshloka
{मया तु हितकामेन यशश्च परिरक्षता}
{स्नेहप्रस्कन्नमनसा हितमित्यप्रियं कृतम्} %6-106-12

\twolineshloka
{नास्मिन्नर्थे महाराज त्वं मां प्रियहिते रतम्}
{कश्चिल्लघुरिवानार्यो दोषतो गन्तुमर्हसि} %6-106-13

\twolineshloka
{श्रूयतां प्रति दास्यामि यन्निमित्तं मया रथः}
{नदीवेग इवाम्भोभिः संयुगे विनिवर्तितः} %6-106-14

\twolineshloka
{श्रमं तवावगच्छामि महता रणकर्मणा}
{नहि ते वीर्यसौमुख्यं प्रकर्षं नोपधारये} %6-106-15

\twolineshloka
{रथोद्वहनखिन्नाश्च भग्ना मे रथवाजिनः}
{दीना घर्मपरिश्रान्ता गावो वर्षहता इव} %6-106-16

\twolineshloka
{निमित्तानि च भूयिष्ठं यानि प्रादुर्भवन्ति नः}
{तेषु तेष्वभिपन्नेषु लक्षयाम्यप्रदक्षिणम्} %6-106-17

\twolineshloka
{देशकालौ च विज्ञेयौ लक्षणानीङ्गितानि च}
{दैन्यं हर्षश्च खेदश्च रथिनश्च बलाबलम्} %6-106-18

\twolineshloka
{स्थलनिम्नानि भूमेश्च समानि विषमाणि च}
{युद्धकालश्च विज्ञेयः परस्यान्तरदर्शनम्} %6-106-19

\twolineshloka
{उपयानापयाने च स्थानं प्रत्यपसर्पणम्}
{सर्वमेतद् रथस्थेन ज्ञेयं रथकुटुम्बिना} %6-106-20

\twolineshloka
{तव विश्रामहेतोस्तु तथैषां रथवाजिनाम्}
{रौद्रं वर्जयता खेदं क्षमं कृतमिदं मया} %6-106-21

\twolineshloka
{स्वेच्छया न मया वीर रथोऽयमपवाहितः}
{भर्तुः स्नेहपरीतेन मयेदं यत् कृतं प्रभो} %6-106-22

\twolineshloka
{आज्ञापय यथातत्त्वं वक्ष्यस्यरिनिषूदन}
{तत् करिष्याम्यहं वीर गतानृण्येन चेतसा} %6-106-23

\twolineshloka
{संतुष्टस्तेन वाक्येन रावणस्तस्य सारथेः}
{प्रशस्यैनं बहुविधं युद्धलुब्धोऽब्रवीदिदम्} %6-106-24

\twolineshloka
{रथं शीघ्रमिमं सूत राघवाभिमुखं नय}
{नाहत्वा समरे शत्रून् निवर्तिष्यति रावणः} %6-106-25

\threelineshloka
{एवमुक्त्वा रथस्थस्य रावणो राक्षसेश्वरः}
{ददौ तस्य शुभं ह्येकं हस्ताभरणमुत्तमम्}
{श्रुत्वा रावणवाक्यानि सारथिः संन्यवर्तत} %6-106-26

\twolineshloka
{ततो द्रुतं रावणवाक्यचोदितः प्रचोदयामास हयान् स सारथिः}
{स राक्षसेन्द्रस्य ततो महारथः क्षणेन रामस्य रणाग्रतोऽभवत्} %6-106-27


॥इत्यार्षे श्रीमद्रामायणे वाल्मीकीये आदिकाव्ये युद्धकाण्डे सारथिविज्ञेयम् नाम षडधिकशततमः सर्गः ॥६-१०६॥
