\sect{नवमः सर्गः — विभीषणसमालोचनम्}

\twolineshloka
{ततो निकुम्भो रभसः सूर्यशत्रुर्महाबलः}
{सुप्तघ्नो यज्ञकोपश्च महापार्श्वमहोदरौ} %6-9-1

\twolineshloka
{अग्निकेतुश्च दुर्धर्षो रश्मिकेतुश्च राक्षसः}
{इन्द्रजिच्च महातेजा बलवान् रावणात्मजः} %6-9-2

\twolineshloka
{प्रहस्तोऽथ विरूपाक्षो वज्रदंष्ट्रो महाबलः}
{धूम्राक्षश्चातिकायश्च दुर्मुखश्चैव राक्षसः} %6-9-3

\twolineshloka
{परिघान् पट्टिशान् शूलान् प्रासान् शक्तिपरश्वधान्}
{चापानि च सुबाणानि खड्गांश्च विपुलाम्बुभान्} %6-9-4

\twolineshloka
{प्रगृह्य परमक्रुद्धाः समुत्पत्य च राक्षसाः}
{अब्रुवन् रावणं सर्वे प्रदीप्ता इव तेजसा} %6-9-5

\twolineshloka
{अद्य रामं वधिष्यामः सुग्रीवं च सलक्ष्मणम्}
{कृपणं च हनूमन्तं लङ्का येन प्रधर्षिता} %6-9-6

\twolineshloka
{तान् गृहीतायुधान् सर्वान् वारयित्वा विभीषणः}
{अब्रवीत् प्राञ्जलिर्वाक्यं पुनः प्रत्युपवेश्य तान्} %6-9-7

\twolineshloka
{अप्युपायैस्त्रिभिस्तात योऽर्थः प्राप्तुं न शक्यते}
{तस्य विक्रमकालांस्तान् युक्तानाहुर्मनीषिणः} %6-9-8

\twolineshloka
{प्रमत्तेष्वभियुक्तेषु दैवेन प्रहतेषु च}
{विक्रमास्तात सिद्ध्यन्ति परीक्ष्य विधिना कृताः} %6-9-9

\twolineshloka
{अप्रमत्तं कथं तं तु विजिगीषुं बले स्थितम्}
{जितरोषं दुराधर्षं तं धर्षयितुमिच्छथ} %6-9-10

\twolineshloka
{समुद्रं लङ्घयित्वा तु घोरं नदनदीपतिम्}
{गतिं हनूमतो लोके को विद्यात् तर्कयेत वा} %6-9-11

\twolineshloka
{बलान्यपरिमेयानि वीर्याणि च निशाचराः}
{परेषां सहसावज्ञा न कर्तव्या कथंचन} %6-9-12

\twolineshloka
{किं च राक्षसराजस्य रामेणापकृतं पुरा}
{आजहार जनस्थानाद् यस्य भार्यां यशस्विनः} %6-9-13

\twolineshloka
{खरो यद्यतिवृत्तस्तु स रामेण हतो रणे}
{अवश्यं प्राणिनां प्राणा रक्षितव्या यथाबलम्} %6-9-14

\twolineshloka
{एतन्निमित्तं वैदेही भयं नः सुमहद् भवेत्}
{आहृता सा परित्याज्या कलहार्थे कृते नु किम्} %6-9-15

\twolineshloka
{न तु क्षमं वीर्यवता तेन धर्मानुवर्तिना}
{वैरं निरर्थकं कर्तुं दीयतामस्य मैथिली} %6-9-16

\twolineshloka
{यावन्न सगजां साश्वां बहुरत्नसमाकुलाम्}
{पुरीं दारयते बाणैर्दीयतामस्य मैथिली} %6-9-17

\twolineshloka
{यावत् सुघोरा महती दुर्धर्षा हरिवाहिनी}
{नावस्कन्दति नो लङ्कां तावत् सीता प्रदीयताम्} %6-9-18

\twolineshloka
{विनश्येद्धि पुरी लङ्का शूराः सर्वे च राक्षसाः}
{रामस्य दयिता पत्नी न स्वयं यदि दीयते} %6-9-19

\twolineshloka
{प्रसादये त्वां बन्धुत्वात् कुरुष्व वचनं मम}
{हितं तथ्यं त्वहं ब्रूमि दीयतामस्य मैथिली} %6-9-20

\twolineshloka
{पुरा शरत्सूर्यमरीचिसंनिभान् नवाग्रपुङ्खान् सुदृढान् नृपात्मजः}
{सृजत्यमोघान् विशिखान् वधाय ते प्रदीयतां दाशरथाय मैथिली} %6-9-21

\twolineshloka
{त्यजाशु कोपं सुखधर्मनाशनं भजस्व धर्मं रतिकीर्तिवर्धनम्}
{प्रसीद जीवेम सपुत्रबान्धवाः प्रदीयतां दाशरथाय मैथिली} %6-9-22

\twolineshloka
{विभीषणवचः श्रुत्वा रावणो राक्षसेश्वरः}
{विसर्जयित्वा तान् सर्वान् प्रविवेश स्वकं गृहम्} %6-9-23


॥इत्यार्षे श्रीमद्रामायणे वाल्मीकीये आदिकाव्ये युद्धकाण्डे विभीषणसमालोचनम् नाम नवमः सर्गः ॥६-९॥
