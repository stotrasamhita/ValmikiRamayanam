\sect{पञ्चमः सर्गः — रामविप्रलम्भः}

\twolineshloka
{सा तु नीलेन विधिवत्स्वारक्षा सुसमाहिता}
{सागरस्योत्तरे तीरे साधु सा विनिवेशिता} %6-5-1

\twolineshloka
{मैन्दश्च द्विविदश्चोभौ तत्र वानरपुङ्गवौ}
{विचेरतुश्च तां सेनां रक्षार्थं सर्वतोदिशम्} %6-5-2

\twolineshloka
{निविष्टायां तु सेनायां तीरे नदनदीपतेः}
{पार्श्वस्थं लक्ष्मणं दृष्ट्वा रामो वचनमब्रवीत्} %6-5-3

\twolineshloka
{शोकश्च किल कालेन गच्छता ह्यपगच्छति}
{मम चापश्यतः कान्तामहन्यहनि वर्धते} %6-5-4

\twolineshloka
{न मे दुःखं प्रिया दूरे न मे दुःखं हृतेति च}
{एतदेवानुशोचामि वयोऽस्या ह्यतिवर्तते} %6-5-5

\twolineshloka
{वाहि वात यतः कान्ता तां स्पृष्ट्वा मामपि स्पृश}
{त्वयि मे गात्रसंस्पर्शश्चन्द्रे दृष्टिसमागमः} %6-5-6

\twolineshloka
{तन्मे दहति गात्राणि विषं पीतमिवाशये}
{हा नाथेति प्रिया सा मां ह्रियमाणा यदब्रवीत्} %6-5-7

\twolineshloka
{तद्वियोगेन्धनवता तच्चिन्ताविमलार्चिषा}
{रात्रिन्दिवं शरीरं मे दह्यते मदनाग्निना} %6-5-8

\twolineshloka
{अवगाह्यार्णवं स्वप्स्ये सौमित्रे भवता विना}
{एवं च प्रज्वलन् कामो न मा सुप्तं जले दहेत्} %6-5-9

\twolineshloka
{बह्वेतत् कामयानस्य शक्यमेतेन जीवितुम्}
{यदहं सा च वामोरुरेकां धरणिमाश्रितौ} %6-5-10

\twolineshloka
{केदारस्येवाकेदारः सोदकस्य निरूदकः}
{उपस्नेहेन जीवामि जीवन्तीं यच्छृणोमि ताम्} %6-5-11

\twolineshloka
{कदा नु खलु सुश्रोणीं शतपत्रायतेक्षणाम्}
{विजित्य शत्रून् द्रक्ष्यामि सीतां स्फीतामिव श्रियम्} %6-5-12

\twolineshloka
{कदा सुचारुदन्तोष्ठं तस्याः पद्ममिवाननम्}
{ईषदुन्नाम्य पास्यामि रसायनमिवातुरः} %6-5-13

\twolineshloka
{तौ तस्याः सहितौ पीनौ स्तनौ तालफलोपमौ}
{कदा न खलु सोत्कम्पौ श्लिष्यन्त्या मां भजिष्यतः} %6-5-14

\twolineshloka
{सा नूनमसितापाङ्गी रक्षोमध्यगता सती}
{मन्नाथा नाथहीनेव त्रातारं नाधिगच्छति} %6-5-15

\twolineshloka
{कथं जनकराजस्य दुहिता मम च प्रिया}
{राक्षसीमध्यगा शेते स्नुषा दशरथस्य च} %6-5-16

\twolineshloka
{अविक्षोभ्याणि रक्षांसि सा विधूयोत्पतिष्यति}
{विधूय जलदान् नीलान् शशिलेखा शरत्स्विव} %6-5-17

\twolineshloka
{स्वभावतनुका नूनं शोकेनानशनेन च}
{भूयस्तनुतरा सीता देशकालविपर्ययात्} %6-5-18

\twolineshloka
{कदा नु राक्षसेन्द्रस्य निधायोरसि सायकान्}
{शोकं प्रत्याहरिष्यामि शोकमुत्सृज्य मानसम्} %6-5-19

\twolineshloka
{कदा नु खलु मे साध्वी सीतामरसुतोपमा}
{सोत्कण्ठा कण्ठमालम्ब्य मोक्ष्यत्यानन्दजं जलम्} %6-5-20

\twolineshloka
{कदा शोकमिमं घोरं मैथिलीविप्रयोगजम्}
{सहसा विप्रमोक्ष्यामि वासः शुक्लेतरं यथा} %6-5-21

\twolineshloka
{एवं विलपतस्तस्य तत्र रामस्य धीमतः}
{दिनक्षयान्मन्दवपुर्भास्करोऽस्तमुपागमत्} %6-5-22

\twolineshloka
{आश्वासितो लक्ष्मणेन रामः सन्ध्यामुपासत}
{स्मरन् कमलपत्राक्षीं सीतां शोकाकुलीकृतः} %6-5-23


॥इत्यार्षे श्रीमद्रामायणे वाल्मीकीये आदिकाव्ये युद्धकाण्डे रामविप्रलम्भः नाम पञ्चमः सर्गः ॥६-५॥
