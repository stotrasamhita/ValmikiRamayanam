\sect{षट्सप्ततितमः सर्गः — कम्पनादिवधः}

\twolineshloka
{प्रवृत्ते संकुले तस्मिन् घोरे वीरजनक्षये}
{अङ्गदः कम्पनं वीरमाससाद रणोत्सुकः} %6-76-1

\twolineshloka
{आहूय सोऽङ्गदं कोपात् ताडयामास वेगितः}
{गदया कम्पनः पूर्वं स चचाल भृशाहतः} %6-76-2

\twolineshloka
{स संज्ञां प्राप्य तेजस्वी चिक्षेप शिखरं गिरेः}
{अर्दितश्च प्रहारेण कम्पनः पतितो भुवि} %6-76-3

\twolineshloka
{ततस्तु कम्पनं दृष्ट्वा शोणिताक्षो हतं रणे}
{रथेनाभ्यपतत् क्षिप्रं तत्राङ्गदमभीतवत्} %6-76-4

\twolineshloka
{सोऽङ्गदं निशितैर्बाणैस्तदा विव्याध वेगितः}
{शरीरदारणैस्तीक्ष्णैः कालाग्निसमविग्रहैः} %6-76-5

\twolineshloka
{क्षुरक्षुरप्रनाराचैर्वत्सदन्तैः शिलीमुखैः}
{कर्णिशल्यविपाठैश्च बहुभिर्निशितैः शरैः} %6-76-6

\twolineshloka
{अङ्गदः प्रतिविद्धाङ्गो वालिपुत्रः प्रतापवान्}
{धनुरुग्रं रथं बाणान् ममर्द तरसा बली} %6-76-7

\twolineshloka
{शोणिताक्षस्ततः क्षिप्रमसिचर्म समाददे}
{उत्पपात तदा क्रुद्धो वेगवानविचारयन्} %6-76-8

\twolineshloka
{तं क्षिप्रतरमाप्लुत्य परामृश्याङ्गदो बली}
{करेण तस्य तं खड्गं समाच्छिद्य ननाद च} %6-76-9

\twolineshloka
{तस्यांसफलके खड्गं निजघान ततोऽङ्गदः}
{यज्ञोपवीतवच्चैनं चिच्छेद कपिकुञ्जरः} %6-76-10

\twolineshloka
{तं प्रगृह्य महाखड्गं विनद्य च पुनः पुनः}
{वालिपुत्रोऽभिदुद्राव रणशीर्षे परानरीन्} %6-76-11

\twolineshloka
{प्रजङ्घसहितो वीरो यूपाक्षस्तु ततो बली}
{रथेनाभिययौ क्रुद्धो वालिपुत्रं महाबलम्} %6-76-12

\twolineshloka
{आयसीं तु गदां गृह्य स वीरः कनकाङ्गदः}
{शोणिताक्षः समाश्वस्य तमेवानुपपात ह} %6-76-13

\twolineshloka
{प्रजङ्घस्तु महावीरो यूपाक्षसहितो बली}
{गदयाभिययौ क्रुद्धो वालिपुत्रं महाबलम्} %6-76-14

\twolineshloka
{तयोर्मध्ये कपिश्रेष्ठः शोणिताक्षप्रजङ्घयोः}
{विशाखयोर्मध्यगतः पूर्णचन्द्र इवाबभौ} %6-76-15

\twolineshloka
{अङ्गदं परिरक्षन्तौ मैन्दो द्विविद एव च}
{तस्य तस्थतुरभ्याशे परस्परदिदृक्षया} %6-76-16

\twolineshloka
{अभिपेतुर्महाकायाः प्रतियत्ता महाबलाः}
{राक्षसा वानरान् रोषादसिबाणगदाधराः} %6-76-17

\twolineshloka
{त्रयाणां वानरेन्द्राणां त्रिभी राक्षसपुंगवैः}
{संसक्तानां महद् युद्धमभवद् रोमहर्षणम्} %6-76-18

\twolineshloka
{ते तु वृक्षान् समादाय सम्प्रचिक्षिपुराहवे}
{खड्गेन प्रतिचिक्षेप तान् प्रजङ्घो महाबलः} %6-76-19

\twolineshloka
{रथानश्वान् द्रुमाञ्छैलान् प्रतिचिक्षिपुराहवे}
{शरौघैः प्रतिचिच्छेद तान् यूपाक्षो महाबलः} %6-76-20

\twolineshloka
{सृष्टान् द्विविदमैन्दाभ्यां द्रुमानुत्पाट्य वीर्यवान्}
{बभञ्ज गदया मध्ये शोणिताक्षः प्रतापवान्} %6-76-21

\twolineshloka
{उद्यम्य विपुलं खड्गं परमर्मविदारणम्}
{प्रजङ्घो वालिपुत्राय अभिदुद्राव वेगितः} %6-76-22

\twolineshloka
{तमभ्याशगतं दृष्ट्वा वानरेन्द्रो महाबलः}
{आजघानाश्वकर्णेन द्रुमेणातिबलस्तदा} %6-76-23

\twolineshloka
{बाहुं चास्य सनिस्त्रिंशमाजघान स मुष्टिना}
{वालिपुत्रस्य घातेन स पपात क्षितावसिः} %6-76-24

\twolineshloka
{तं दृष्ट्वा पतितं भूमौ खड्गं मुसलसंनिभम्}
{मुष्टिं संवर्तयामास वज्रकल्पं महाबलः} %6-76-25

\twolineshloka
{स ललाटे महावीर्यमङ्गदं वानरर्षभम्}
{आजघान महातेजाः स मुहूर्तं चचाल ह} %6-76-26

\twolineshloka
{स संज्ञां प्राप्य तेजस्वी वालिपुत्रः प्रतापवान्}
{प्रजङ्घस्य शिरः कायात् पातयामास मुष्टिना} %6-76-27

\twolineshloka
{स यूपाक्षोऽश्रुपूर्णाक्षः पितृव्ये निहते रणे}
{अवरुह्य रथात् क्षिप्रं क्षीणेषुः खड्गमाददे} %6-76-28

\twolineshloka
{तमापतन्तं सम्प्रेक्ष्य यूपाक्षं द्विविदस्त्वरन्}
{आजघानोरसि क्रुद्धो जग्राह च बलाद् बली} %6-76-29

\twolineshloka
{गृहीतं भ्रातरं दृष्ट्वा शोणिताक्षो महाबलः}
{आजघान महातेजा वक्षसि द्विविदं ततः} %6-76-30

\twolineshloka
{स ततोऽभिहतस्तेन चचाल च महाबलः}
{उद्यतां च पुनस्तस्य जहार द्विविदो गदाम्} %6-76-31

\twolineshloka
{एतस्मिन्नन्तरे मैन्दो द्विविदाभ्याशमागमत्}
{यूपाक्षं ताडयामास तलेनोरसि वीर्यवान्} %6-76-32

\twolineshloka
{तौ शोणिताक्षयूपाक्षौ प्लवंगाभ्यां तरस्विनौ}
{चक्रतुः समरे तीव्रमाकर्षोत्पाटनं भृशम्} %6-76-33

\twolineshloka
{द्विविदः शोणिताक्षं तु विददार नखैर्मुखे}
{निष्पिपेष स वीर्येण क्षितावाविध्य वीर्यवान्} %6-76-34

\twolineshloka
{यूपाक्षमभिसंक्रुद्धो मैन्दो वानरपुङ्गवः}
{पीडयामास बाहुभ्यां पपात स हतः क्षितौ} %6-76-35

\twolineshloka
{हतप्रवीरा व्यथिता राक्षसेन्द्रचमूस्तथा}
{जगामाभिमुखी सा तु कुम्भकर्णात्मजो यतः} %6-76-36

\twolineshloka
{आपतन्तीं च वेगेन कुम्भस्तां सान्त्वयच्चमूम्}
{अथोत्कृष्टं महावीर्यैर्लब्धलक्षैः प्लवंगमैः} %6-76-37

\twolineshloka
{निपातितमहावीरां दृष्ट्वा रक्षश्चमूं तदा}
{कुम्भः प्रचक्रे तेजस्वी रणे कर्म सुदुष्करम्} %6-76-38

\twolineshloka
{स धनुर्धन्विनां श्रेष्ठः प्रगृह्य सुसमाहितः}
{मुमोचाशीविषप्रख्याञ्छरान् देहविदारणान्} %6-76-39

\twolineshloka
{तस्य तच्छुशुभे भूयः सशरं धनुरुत्तमम्}
{विद्युदैरावतार्चिष्मद्द्वितीयेन्द्रधनुर्यथा} %6-76-40

\twolineshloka
{आकर्णकृष्टमुक्तेन जघान द्विविदं तदा}
{तेन हाटकपुङ्खेन पत्रिणा पत्रवाससा} %6-76-41

\twolineshloka
{सहसाभिहतस्तेन विप्रमुक्तपदः स्फुरन्}
{निपपात त्रिकूटाभो विह्वलन् प्लवगोत्तमः} %6-76-42

\twolineshloka
{मैन्दस्तु भ्रातरं तत्र भग्नं दृष्ट्वा महाहवे}
{अभिदुद्राव वेगेन प्रगृह्य विपुलां शिलाम्} %6-76-43

\twolineshloka
{तां शिलां तु प्रचिक्षेप राक्षसाय महाबलः}
{बिभेद तां शिलां कुम्भः प्रसन्नैः पञ्चभिः शरैः} %6-76-44

\twolineshloka
{संधाय चान्यं सुमुखं शरमाशीविषोपमम्}
{आजघान महातेजा वक्षसि द्विविदाग्रजम्} %6-76-45

\twolineshloka
{स तु तेन प्रहारेण मैन्दो वानरयूथपः}
{मर्मण्यभिहतस्तेन पपात भुवि मूर्च्छितः} %6-76-46

\twolineshloka
{अङ्गदो मातुलौ दृष्ट्वा मथितौ तु महाबलौ}
{अभिदुद्राव वेगेन कुम्भमुद्यतकार्मुकम्} %6-76-47

\threelineshloka
{तमापतन्तं विव्याध कुम्भः पञ्चभिरायसैः}
{त्रिभिश्चान्यैः शितैर्बाणैर्मातंगमिव तोमरैः}
{सोऽङ्गदं बहुभिर्बाणैः कुम्भो विव्याध वीर्यवान्} %6-76-48

\twolineshloka
{अकुण्ठधारैर्निशितैस्तीक्ष्णैः कनकभूषणैः}
{अङ्गदः प्रतिविद्धाङ्गो वालिपुत्रो न कम्पते} %6-76-49

\twolineshloka
{शिलापादपवर्षाणि तस्य मूर्ध्नि ववर्ष ह}
{स प्रचिच्छेद तान् सर्वान् बिभेद च पुनः शिलाः} %6-76-50

\twolineshloka
{कुम्भकर्णात्मजः श्रीमान् वालिपुत्रसमीरितान्}
{आपतन्तं च सम्प्रेक्ष्य कुम्भो वानरयूथपम्} %6-76-51

\twolineshloka
{भ्रुवौ विव्याध बाणाभ्यामुल्काभ्यामिव कुञ्जरम्}
{तस्य सुस्राव रुधिरं पिहिते चास्य लोचने} %6-76-52

\twolineshloka
{अङ्गदः पाणिना नेत्रे पिधाय रुधिरोक्षिते}
{सालमासन्नमेकेन परिजग्राह पाणिना} %6-76-53

\twolineshloka
{सम्पीड्योरसि सस्कन्धं करेणाभिनिवेश्य च}
{किंचिदभ्यवनम्यैनमुन्ममाथ महारणे} %6-76-54

\twolineshloka
{तमिन्द्रकेतुप्रतिमं वृक्षं मन्दरसंनिभम्}
{समुत्सृजत वेगेन मिषतां सर्वरक्षसाम्} %6-76-55

\twolineshloka
{स चिच्छेद शितैर्बाणैः सप्तभिः कायभेदनैः}
{अङ्गदो विव्यथेऽभीक्ष्णं स पपात मुमोह च} %6-76-56

\twolineshloka
{अङ्गदं पतितं दृष्ट्वा सीदन्तमिव सागरे}
{दुरासदं हरिश्रेष्ठा राघवाय न्यवेदयन्} %6-76-57

\twolineshloka
{रामस्तु व्यथितं श्रुत्वा वालिपुत्रं महाहवे}
{व्यादिदेश हरिश्रेष्ठाञ्जाम्बवत्प्रमुखांस्ततः} %6-76-58

\twolineshloka
{ते तु वानरशार्दूलाः श्रुत्वा रामस्य शासनम्}
{अभिपेतुः सुसंक्रुद्धाः कुम्भमुद्यतकार्मुकम्} %6-76-59

\twolineshloka
{ततो द्रुमशिलाहस्ताः कोपसंरक्तलोचनाः}
{रिरक्षिषन्तोऽभ्यपतन्नङ्गदं वानरर्षभाः} %6-76-60

\twolineshloka
{जाम्बवांश्च सुषेणश्च वेगदर्शी च वानरः}
{कुम्भकर्णात्मजं वीरं क्रुद्धाः समभिदुद्रुवुः} %6-76-61

\twolineshloka
{समीक्ष्यापततस्तांस्तु वानरेन्द्रान् महाबलान्}
{आववार शरौघेण नगेनेव जलाशयम्} %6-76-62

\twolineshloka
{तस्य बाणपथं प्राप्य न शेकुरपि वीक्षितुम्}
{वानरेन्द्रा महात्मानो वेलामिव महोदधिः} %6-76-63

\twolineshloka
{तांस्तु दृष्ट्वा हरिगणान् शरवृष्टिभिरर्दितान्}
{अङ्गदं पृष्ठतः कृत्वा भ्रातृजं प्लवगेश्वरः} %6-76-64

\twolineshloka
{अभिदुद्राव सुग्रीवः कुम्भकर्णात्मजं रणे}
{शैलसानुचरं नागं वेगवानिव केसरी} %6-76-65

\twolineshloka
{उत्पाट्य च महावृक्षानश्वकर्णादिकान् बहून्}
{अन्यांश्च विविधान् वृक्षांश्चिक्षेप स महाकपिः} %6-76-66

\twolineshloka
{तां छादयन्तीमाकाशं वृक्षवृष्टिं दुरासदाम्}
{कुम्भकर्णात्मजः श्रीमांश्चिच्छेद स्वशरैः शितैः} %6-76-67

\threelineshloka
{अभिलक्ष्येण तीव्रेण कुम्भेन निशितैः शरैः}
{आचितास्ते द्रुमा रेजुर्यथा घोराः शतघ्नयः}
{द्रुमवर्षं तु तद् भिन्नं दृष्ट्वा कुम्भेन वीर्यवान्} %6-76-68

\twolineshloka
{वानराधिपतिः श्रीमान् महासत्त्वो न विव्यथे}
{स विध्यमानः सहसा सहमानस्तु ताञ्छरान्} %6-76-69

\twolineshloka
{कुम्भस्य धनुराक्षिप्य बभञ्जेन्द्रधनुःप्रभम्}
{अवप्लुत्य ततः शीघ्रं कृत्वा कर्म सुदुष्करम्} %6-76-70

\twolineshloka
{अब्रवीत् कुपितः कुम्भं भग्नशृङ्गमिव द्विपम्}
{निकुम्भाग्रज वीर्यं ते बाणवेगं तदद्भुतम्} %6-76-71

\twolineshloka
{संनतिश्च प्रभावश्च तव वा रावणस्य वा}
{प्रह्लादबलिवृत्रघ्नकुबेरवरुणोपम} %6-76-72

\twolineshloka
{एकस्त्वमनुजातोऽसि पितरं बलवत्तरम्}
{त्वामेवैकं महाबाहुं शूलहस्तमरिंदमम्} %6-76-73

\twolineshloka
{त्रिदशा नातिवर्तन्ते जितेन्द्रियमिवाधयः}
{विक्रमस्व महाबुद्धे कर्माणि मम पश्य च} %6-76-74

\twolineshloka
{वरदानात् पितृव्यस्ते सहते देवदानवान्}
{कुम्भकर्णस्तु वीर्येण सहते च सुरासुरान्} %6-76-75

\twolineshloka
{धनुषीन्द्रजितस्तुल्यः प्रतापे रावणस्य च}
{त्वमद्य रक्षसां लोके श्रेष्ठोऽसि बलवीर्यतः} %6-76-76

\twolineshloka
{महाविमर्दं समरे मया सह तवाद्भुतम्}
{अद्य भूतानि पश्यन्तु शक्रशम्बरयोरिव} %6-76-77

\twolineshloka
{कृतमप्रतिमं कर्म दर्शितं चास्त्रकौशलम्}
{पतिता हरिवीराश्च त्वयैते भीमविक्रमाः} %6-76-78

\twolineshloka
{उपालम्भभयाच्चैव नासि वीर मया हतः}
{कृतकर्मपरिश्रान्तो विश्रान्तः पश्य मे बलम्} %6-76-79

\twolineshloka
{तेन सुग्रीववाक्येन सावमानेन मानितः}
{अग्नेराज्यहुतस्येव तेजस्तस्याभ्यवर्धत} %6-76-80

\twolineshloka
{ततः कुम्भस्तु सुग्रीवं बाहुभ्यां जगृहे तदा}
{गजाविवातीतमदौ निःश्वसन्तौ मुहुर्मुहुः} %6-76-81

\twolineshloka
{अन्योन्यगात्रग्रथितौ घर्षन्तावितरेतरम्}
{सधूमां मुखतो ज्वालां विसृजन्तौ परिश्रमात्} %6-76-82

\twolineshloka
{तयोः पादाभिघाताच्च निमग्ना चाभवन्मही}
{व्याघूर्णिततरङ्गश्च चुक्षुभे वरुणालयः} %6-76-83

\twolineshloka
{ततः कुम्भं समुत्क्षिप्य सुग्रीवो लवणाम्भसि}
{पातयामास वेगेन दर्शयन्नुदधेस्तलम्} %6-76-84

\twolineshloka
{ततः कुम्भनिपातेन जलराशिः समुत्थितः}
{विन्ध्यमन्दरसंकाशो विससर्प समन्ततः} %6-76-85

\twolineshloka
{ततः कुम्भः समुत्पत्य सुग्रीवमभिपात्य च}
{आजघानोरसि क्रुद्धो वज्रकल्पेन मुष्टिना} %6-76-86

\twolineshloka
{तस्य वर्म च पुस्फोट संजज्ञे चापि शोणितम्}
{तस्य मुष्टिर्महावेगः प्रतिजघ्नेऽस्थिमण्डले} %6-76-87

\twolineshloka
{तस्य वेगेन तत्रासीत् तेजः प्रज्वलितं महत्}
{वज्रनिष्पेषसंजाता ज्वाला मेरोर्यथा गिरेः} %6-76-88

\twolineshloka
{स तत्राभिहतस्तेन सुग्रीवो वानरर्षभः}
{मुष्टिं संवर्तयामास वज्रकल्पं महाबलः} %6-76-89

\twolineshloka
{अर्चिःसहस्रविकचरविमण्डलवर्चसम्}
{स मुष्टिं पातयामास कुम्भस्योरसि वीर्यवान्} %6-76-90

\twolineshloka
{स तु तेन प्रहारेण विह्वलो भृशपीडितः}
{निपपात तदा कुम्भो गतार्चिरिव पावकः} %6-76-91

\twolineshloka
{मुष्टिनाभिहतस्तेन निपपाताशु राक्षसः}
{लोहिताङ्ग इवाकाशाद् दीप्तरश्मिर्यदृच्छया} %6-76-92

\twolineshloka
{कुम्भस्य पततो रूपं भग्नस्योरसि मुष्टिना}
{बभौ रुद्राभिपन्नस्य यथा रूपं गवां पतेः} %6-76-93

\twolineshloka
{तस्मिन् हते भीमपराक्रमेण प्लवंगमानामृषभेण युद्धे}
{मही सशैला सवना चचाल भयं च रक्षांस्यधिकं विवेश} %6-76-94


॥इत्यार्षे श्रीमद्रामायणे वाल्मीकीये आदिकाव्ये युद्धकाण्डे कम्पनादिवधः नाम षट्सप्ततितमः सर्गः ॥६-७६॥
