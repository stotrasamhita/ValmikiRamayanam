\sect{अष्टाविंशः सर्गः — मैन्दादिपराक्रमाख्यानम्}

\twolineshloka
{सारणस्य वचः श्रुत्वा रावणं राक्षसाधिपम्}
{बलमादिश्य तत् सर्वं शुको वाक्यमथाब्रवीत्} %6-28-1

\twolineshloka
{स्थितान् पश्यसि यानेतान् मत्तानिव महाद्विपान्}
{न्यग्रोधानिव गाङ्गेयान् सालान् हैमवतानिव} %6-28-2

\twolineshloka
{एते दुष्प्रसहा राजन् बलिनः कामरूपिणः}
{दैत्यदानवसङ्काशा युद्धे देवपराक्रमाः} %6-28-3

\twolineshloka
{एषां कोटिसहस्राणि नव पञ्च च सप्त च}
{तथा शङ्कुसहस्राणि तथा वृन्दशतानि च} %6-28-4

\twolineshloka
{एते सुग्रीवसचिवाः किष्किन्धानिलयाः सदा}
{हरयो देवगन्धर्वैरुत्पन्नाः कामरूपिणः} %6-28-5

\twolineshloka
{यौ तौ पश्यसि तिष्ठन्तौ कुमारौ देवरूपिणौ}
{मैन्दश्च द्विविदश्चैव ताभ्यां नास्ति समो युधि} %6-28-6

\twolineshloka
{ब्रह्मणा समनुज्ञातावमृतप्राशिनावुभौ}
{आशंसेते यथा लङ्कामेतौ मर्दितुमोजसा} %6-28-7

\twolineshloka
{यं तु पश्यसि तिष्ठन्तं प्रभिन्नमिव कुञ्जरम्}
{यो बलात् क्षोभयेत् क्रुद्धः समुद्रमपि वानरः} %6-28-8

\twolineshloka
{एषोऽभिगन्ता लङ्कायां वैदेह्यास्तव च प्रभो}
{एनं पश्य पुरा दृष्टं वानरं पुनरागतम्} %6-28-9

\twolineshloka
{ज्येष्ठः केसरिणः पुत्रो वातात्मज इति श्रुतः}
{हनूमानिति विख्यातो लङ्घितो येन सागरः} %6-28-10

\twolineshloka
{कामरूपो हरिश्रेष्ठो बलरूपसमन्वितः}
{अनिवार्यगतिश्चैव यथा सततगः प्रभुः} %6-28-11

\twolineshloka
{उद्यन्तं भास्करं दृष्ट्वा बालः किल बुभुक्षितः}
{त्रियोजनसहस्रं तु अध्वानमवतीर्य हि} %6-28-12

\twolineshloka
{आदित्यमाहरिष्यामि न मे क्षुत् प्रतियास्यति}
{इति निश्चित्य मनसा पुप्लुवे बलदर्पितः} %6-28-13

\twolineshloka
{अनाधृष्यतमं देवमपि देवर्षिराक्षसैः}
{अनासाद्यैव पतितो भास्करोदयने गिरौ} %6-28-14

\twolineshloka
{पतितस्य कपेरस्य हनुरेका शिलातले}
{किञ्चिद् भिन्ना दृढहनुर्हनूमानेष तेन वै} %6-28-15

\twolineshloka
{सत्यमागमयोगेन ममैष विदितो हरिः}
{नास्य शक्यं बलं रूपं प्रभावो वानुभाषितुम्} %6-28-16

\threelineshloka
{एष आशंसते लङ्कामेको मथितुमोजसा}
{येन जाज्वल्यतेऽसौ वै धूमकेतुस्तवाद्य वै}
{लङ्कायां निहितश्चापि कथं विस्मरसे कपिम्} %6-28-17

\twolineshloka
{यश्चैषोऽनन्तरः शूरः श्यामः पद्मनिभेक्षणः}
{इक्ष्वाकूणामतिरथो लोके विश्रुतपौरुषः} %6-28-18

\twolineshloka
{यस्मिन् न चलते धर्मो यो धर्मं नातिवर्तते}
{यो ब्राह्ममस्त्रं वेदांश्च वेद वेदविदां वरः} %6-28-19

\twolineshloka
{यो भिन्द्याद् गगनं बाणैर्मेदिनीं वापि दारयेत्}
{यस्य मृत्योरिव क्रोधः शक्रस्येव पराक्रमः} %6-28-20

\twolineshloka
{यस्य भार्या जनस्थानात् सीता चापि हृता त्वया}
{स एष रामस्त्वां राजन् योद्धुं समभिवर्तते} %6-28-21

\twolineshloka
{यस्यैष दक्षिणे पार्श्वे शुद्धजाम्बूनदप्रभः}
{विशालवक्षास्ताम्राक्षो नीलकुञ्चितमूर्धजः} %6-28-22

\twolineshloka
{एषो हि लक्ष्मणो नाम भ्रातुः प्रियहिते रतः}
{नये युद्धे च कुशलः सर्वशस्त्रभृतां वरः} %6-28-23

\twolineshloka
{अमर्षी दुर्जयो जेता विक्रान्तश्च जयी बली}
{रामस्य दक्षिणो बाहुर्नित्यं प्राणो बहिश्चरः} %6-28-24

\twolineshloka
{नह्येष राघवस्यार्थे जीवितं परिरक्षति}
{एषैवाशंसते युद्धे निहन्तुं सर्वराक्षसान्} %6-28-25

\twolineshloka
{यस्तु सव्यमसौ पक्षं रामस्याश्रित्य तिष्ठति}
{रक्षोगणपरिक्षिप्तो राजा ह्येष विभीषणः} %6-28-26

\twolineshloka
{श्रीमता राजराजेन लङ्कायामभिषेचितः}
{त्वामसौ प्रतिसंरब्धो युद्धायैषोऽभिवर्तते} %6-28-27

\twolineshloka
{यं तु पश्यसि तिष्ठन्तं मध्ये गिरिमिवाचलम्}
{सर्वशाखामृगेन्द्राणां भर्तारममितौजसम्} %6-28-28

\twolineshloka
{तेजसा यशसा बुद्ध्या बलेनाभिजनेन च}
{यः कपीनतिबभ्राज हिमवानिव पर्वतः} %6-28-29

\twolineshloka
{किष्किन्धां यः समध्यास्ते गुहां सगहनद्रुमाम्}
{दुर्गां पर्वतदुर्गम्यां प्रधानैः सह यूथपैः} %6-28-30

\twolineshloka
{यस्यैषा काञ्चनी माला शोभते शतपुष्करा}
{कान्ता देवमनुष्याणां यस्यां लक्ष्मीः प्रतिष्ठिता} %6-28-31

\twolineshloka
{एतां मालां च तारां च कपिराज्यं च शाश्वतम्}
{सुग्रीवो वालिनं हत्वा रामेण प्रतिपादितः} %6-28-32

\twolineshloka
{शतं शतसहस्राणां कोटिमाहुर्मनीषिणः}
{शतं कोटिसहस्राणां शङ्कुरित्यभिधीयते} %6-28-33

\twolineshloka
{शतं शङ्कुसहस्राणां महाशङ्कुरिति स्मृतः}
{महाशङ्कुसहस्राणां शतं वृन्दमिहोच्यते} %6-28-34

\twolineshloka
{शतं वृन्दसहस्राणां महावृन्दमिति स्मृतम्}
{महावृन्दसहस्राणां शतं पद्ममिहोच्यते} %6-28-35

\twolineshloka
{शतं पद्मसहस्राणां महापद्ममिति स्मृतम्}
{महापद्मसहस्राणां शतं खर्वमिहोच्यते} %6-28-36

\threelineshloka
{शतं खर्वसहस्राणां महाखर्वमिति स्मृतम्}
{महाखर्वसहस्राणां समुद्रमभिधीयते}
{शतं समुद्रसाहस्रमोघ इत्यभिधीयते} %6-28-37

\threelineshloka
{शतमोघसहस्राणां महौघा इति विश्रुतः}
{एवं कोटिसहस्रेण शङ्कूनां च शतेन च}
{महाशङ्कुसहस्रेण तथा वृन्दशतेन च} %6-28-38

\twolineshloka
{महावृन्दसहस्रेण तथा पद्मशतेन च}
{महापद्मसहस्रेण तथा खर्वशतेन च} %6-28-39

\onelineshloka
{समुद्रेण च तेनैव महौघेन तथैव च एष कोटिमहौघेन समुद्रसदृशेन च} %6-28-40

\threelineshloka
{विभीषणेन वीरेण सचिवैः परिवारितः}
{सुग्रीवो वानरेन्द्रस्त्वां युद्धार्थमनुवर्तते}
{महाबलवृतो नित्यं महाबलपराक्रमः} %6-28-41

\twolineshloka
{इमां महाराज समीक्ष्य वाहिनीमुपस्थितां प्रज्वलितग्रहोपमाम्}
{ततः प्रयत्नः परमो विधीयतां यथा जयः स्यान्न परैः पराभवः} %6-28-42


॥इत्यार्षे श्रीमद्रामायणे वाल्मीकीये आदिकाव्ये युद्धकाण्डे मैन्दादिपराक्रमाख्यानम् नाम अष्टाविंशः सर्गः ॥६-२८॥
