\sect{एकविंशः सर्गः — समुद्रसंक्षोभः}

\twolineshloka
{ततः सागरवेलायां दर्भानास्तीर्य राघवः}
{अञ्जलिं प्राङ्मुखः कृत्वा प्रतिशिश्ये महोदधेः} %6-21-1

\twolineshloka
{बाहुं भुजङ्गभोगाभमुपधायारिसूदनः}
{जातरूपमयैश्चैव भूषणैर्भूषितं पुरा} %6-21-2

\twolineshloka
{मणिकाञ्चनकेयूरमुक्ताप्रवरभूषणैः}
{भुजैः परमनारीणामभिमृष्टमनेकधा} %6-21-3

\twolineshloka
{चन्दनागुरुभिश्चैव पुरस्तादभिसेवितम्}
{बालसूर्यप्रकाशैश्च चन्दनैरुपशोभितम्} %6-21-4

\twolineshloka
{शयने चोत्तमाङ्गेन सीतायाः शोभितं पुरा}
{तक्षकस्येव सम्भोगं गङ्गाजलनिषेवितम्} %6-21-5

\twolineshloka
{संयुगे युगसंकाशं शत्रूणां शोकवर्धनम्}
{सुहृदां नन्दनं दीर्घं सागरान्तव्यपाश्रयम्} %6-21-6

\twolineshloka
{अस्यता च पुनः सव्यं ज्याघातविहतत्वचम्}
{दक्षिणो दक्षिणं बाहुं महापरिघसंनिभम्} %6-21-7

\twolineshloka
{गोसहस्रप्रदातारं ह्युपधाय भुजं महत्}
{अद्य मे तरणं वाथ मरणं सागरस्य वा} %6-21-8

\twolineshloka
{इति रामो धृतिं कृत्वा महाबाहुर्महोदधिम्}
{अधिशिष्ये च विधिवत् प्रयतो नियतो मुनिः} %6-21-9

\twolineshloka
{तस्य रामस्य सुप्तस्य कुशास्तीर्णे महीतले}
{नियमादप्रमत्तस्य निशास्तिस्रोऽभिजग्मतुः} %6-21-10

\twolineshloka
{स त्रिरात्रोषितस्तत्र नयज्ञो धर्मवत्सलः}
{उपासत तदा रामः सागरं सरितां पतिम्} %6-21-11

\twolineshloka
{न च दर्शयते रूपं मन्दो रामस्य सागरः}
{प्रयतेनापि रामेण यथार्हमभिपूजितः} %6-21-12

\twolineshloka
{समुद्रस्य ततः क्रुद्धो रामो रक्तान्तलोचनः}
{समीपस्थमुवाचेदं लक्ष्मणं शुभलक्षणम्} %6-21-13

\twolineshloka
{अवलेपः समुद्रस्य न दर्शयति यः स्वयम्}
{प्रशमश्च क्षमा चैव आर्जवं प्रियवादिता} %6-21-14

\twolineshloka
{असामर्थ्यफला ह्येते निर्गुणेषु सतां गुणाः}
{आत्मप्रशंसिनं दुष्टं धृष्टं विपरिधावकम्} %6-21-15

\twolineshloka
{सर्वत्रोत्सृष्टदण्डं च लोकः सत्कुरुते नरम्}
{न साम्ना शक्यते कीर्तिर्न साम्ना शक्यते यशः} %6-21-16

\twolineshloka
{प्राप्तुं लक्ष्मण लोकेऽस्मिञ्जयो वा रणमूर्धनि}
{अद्य मद्बाणनिर्भग्नैर्मकरैर्मकरालयम्} %6-21-17

\twolineshloka
{निरुद्धतोयं सौमित्रे प्लवद्भिः पश्य सर्वतः}
{भोगिनां पश्य भोगानि मया भिन्नानि लक्ष्मण} %6-21-18

\twolineshloka
{महाभोगानि मत्स्यानां करिणां च करानिह}
{सशङ्खशुक्तिकाजालं समीनमकरं तथा} %6-21-19

\twolineshloka
{अद्य युद्धेन महता समुद्रं परिशोषये}
{क्षमया हि समायुक्तं मामयं मकरालयः} %6-21-20

\twolineshloka
{असमर्थं विजानाति धिक् क्षमामीदृशे जने}
{न दर्शयति साम्ना मे सागरो रूपमात्मनः} %6-21-21

\twolineshloka
{चापमानय सौमित्रे शरांश्चाशीविषोपमान्}
{समुद्रं शोषयिष्यामि पद्भ्यां यान्तु प्लवंगमाः} %6-21-22

\twolineshloka
{अद्याक्षोभ्यमपि क्रुद्धः क्षोभयिष्यामि सागरम्}
{वेलासु कृतमर्यादं सहस्रोर्मिसमाकुलम्} %6-21-23

\twolineshloka
{निर्मर्यादं करिष्यामि सायकैर्वरुणालयम्}
{महार्णवं क्षोभयिष्ये महादानवसंकुलम्} %6-21-24

\twolineshloka
{एवमुक्त्वा धनुष्पाणिः क्रोधविस्फारितेक्षणः}
{बभूव रामो दुर्धर्षो युगान्ताग्निरिव ज्वलन्} %6-21-25

\twolineshloka
{सम्पीड्य च धनुर्घोरं कम्पयित्वा शरैर्जगत्}
{मुमोच विशिखानुग्रान् वज्रानिव शतक्रतुः} %6-21-26

\twolineshloka
{ते ज्वलन्तो महावेगास्तेजसा सायकोत्तमाः}
{प्रविशन्ति समुद्रस्य जलं वित्रस्तपन्नगम्} %6-21-27

\twolineshloka
{तोयवेगः समुद्रस्य समीनमकरो महान्}
{स बभूव महाघोरः समारुतरवस्तथा} %6-21-28

\twolineshloka
{महोर्मिमालाविततः शङ्खशुक्तिसमावृतः}
{सधूमः परिवृत्तोर्मिः सहसासीन्महोदधिः} %6-21-29

\twolineshloka
{व्यथिताः पन्नगाश्चासन् दीप्तास्या दीप्तलोचनाः}
{दानवाश्च महावीर्याः पातालतलवासिनः} %6-21-30

\twolineshloka
{ऊर्मयः सिन्धुराजस्य सनक्रमकरास्तथा}
{विन्ध्यमन्दरसंकाशाः समुत्पेतुः सहस्रशः} %6-21-31

\twolineshloka
{आघूर्णिततरङ्गौघः सम्भ्रान्तोरगराक्षसः}
{उद्वर्तितमहाग्राहः सघोषो वरुणालयः} %6-21-32

\twolineshloka
{ततस्तु तं राघवमुग्रवेगं प्रकर्षमाणं धनुरप्रमेयम्}
{सौमित्रिरुत्पत्य विनिःश्वसन्तं मामेति चोक्त्वा धनुराललम्बे} %6-21-33

\twolineshloka
{एतद्विनापि ह्युदधेस्तवाद्य सम्पत्स्यते वीरतमस्य कार्यम्}
{भवद्विधाः क्रोधवशं न यान्ति दीर्घं भवान् पश्यतु साधुवृत्तम्} %6-21-34

\twolineshloka
{अन्तर्हितैश्चापि तथान्तरिक्षे ब्रह्मर्षिभिश्चैव सुरर्षिभिश्च}
{शब्दः कृतः कष्टमिति ब्रुवद्भिर्मामेति चोक्त्वा महता स्वरेण} %6-21-35


॥इत्यार्षे श्रीमद्रामायणे वाल्मीकीये आदिकाव्ये युद्धकाण्डे समुद्रसंक्षोभः नाम एकविंशः सर्गः ॥६-२१॥
