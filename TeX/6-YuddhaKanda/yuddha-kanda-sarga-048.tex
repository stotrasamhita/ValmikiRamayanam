\sect{अष्टचत्वारिंशः सर्गः — सीताश्वासनम्}

\twolineshloka
{भर्तारं निहतं दृष्ट्वा लक्ष्मणं च महाबलम्}
{विललाप भृशं सीता करुणं शोककर्शिता} %6-48-1

\twolineshloka
{ऊचुर्लाक्षणिका ये मां पुत्रिण्यविधवेति च}
{तेऽद्य सर्वे हते रामे ज्ञानिनोऽनृतवादिनः} %6-48-2

\twolineshloka
{यज्वनो महिषीं ये मामूचुः पत्नीं च सत्रिणः}
{तेऽद्य सर्वे हते रामे ज्ञानिनोऽनृतवादिनः} %6-48-3

\twolineshloka
{वीरपार्थिवपत्नीनां ये विदुर्भर्तृपूजिताम्}
{तेऽद्य सर्वे हते रामे ज्ञानिनोऽनृतवादिनः} %6-48-4

\twolineshloka
{ऊचुः संश्रवणे ये मां द्विजाः कार्तान्तिकाः शुभाम्}
{तेऽद्य सर्वे हते रामे ज्ञानिनोऽनृतवादिनः} %6-48-5

\twolineshloka
{इमानि खलु पद्मानि पादयोर्वै कुलस्त्रियः}
{आधिराज्येऽभिषिच्यन्ते नरेन्द्रैः पतिभिः सह} %6-48-6

\twolineshloka
{वैधव्यं यान्ति यैर्नार्योऽलक्षणैर्भाग्यदुर्लभाः}
{नात्मनस्तानि पश्यामि पश्यन्ती हतलक्षणा} %6-48-7

\twolineshloka
{सत्यनामानि पद्मानि स्त्रीणामुक्तानि लक्षणैः}
{तान्यद्य निहते रामे वितथानि भवन्ति मे} %6-48-8

\twolineshloka
{केशाः सूक्ष्माः समा नीला भ्रुवौ चासंहते मम}
{वृत्ते चारोमके जङ्घे दन्ताश्चाविरला मम} %6-48-9

\twolineshloka
{शङ्खे नेत्रे करौ पादौ गुल्फावूरू समौ चितौ}
{अनुवृत्तनखाः स्निग्धाः समाश्चाङ्गुलयो मम} %6-48-10

\twolineshloka
{स्तनौ चाविरलौ पीनौ मामकौ मग्नचूचुकौ}
{मग्ना चोत्सेधनी नाभिः पार्श्वोरस्कं च मे चितम्} %6-48-11

\twolineshloka
{मम वर्णो मणिनिभो मृदून्यङ्गरुहाणि च}
{प्रतिष्ठितां द्वादशभिर्मामूचुः शुभलक्षणाम्} %6-48-12

\twolineshloka
{समग्रयवमच्छिद्रं पाणिपादं च वर्णवत्}
{मन्दस्मितेत्येव च मां कन्यालाक्षणिका विदुः} %6-48-13

\twolineshloka
{आधिराज्येऽभिषेको मे ब्राह्मणैः पतिना सह}
{कृतान्तकुशलैरुक्तं तत् सर्वं वितथीकृतम्} %6-48-14

\twolineshloka
{शोधयित्वा जनस्थानं प्रवृत्तिमुपलभ्य च}
{तीर्त्वा सागरमक्षोभ्यं भ्रातरौ गोष्पदे हतौ} %6-48-15

\twolineshloka
{ननु वारुणमाग्नेयमैन्द्रं वायव्यमेव च}
{अस्त्रं ब्रह्मशिरश्चैव राघवौ प्रत्यपद्यत} %6-48-16

\twolineshloka
{अदृश्यमानेन रणे मायया वासवोपमौ}
{मम नाथावनाथाया निहतौ रामलक्ष्मणौ} %6-48-17

\twolineshloka
{नहि दृष्टिपथं प्राप्य राघवस्य रणे रिपुः}
{जीवन् प्रतिनिवर्तेत यद्यपि स्यान्मनोजवः} %6-48-18

\twolineshloka
{न कालस्यातिभारोऽस्ति कृतान्तश्च सुदुर्जयः}
{यत्र रामः सह भ्रात्रा शेते युधि निपातितः} %6-48-19

\twolineshloka
{न शोचामि तथा रामं लक्ष्मणं च महारथम्}
{नात्मानं जननीं चापि यथा श्वश्रूं तपस्विनीम्} %6-48-20

\twolineshloka
{सा तु चिन्तयते नित्यं समाप्तव्रतमागतम्}
{कदा द्रक्ष्यामि सीतां च लक्ष्मणं च सराघवम्} %6-48-21

\twolineshloka
{परिदेवयमानां तां राक्षसी त्रिजटाब्रवीत्}
{मा विषादं कृथा देवि भर्तायं तव जीवति} %6-48-22

\twolineshloka
{कारणनि च वक्ष्यामि महान्ति सदृशानि च}
{यथेमौ जीवतो देवि भ्रातरौ रामलक्ष्मणौ} %6-48-23

\twolineshloka
{नहि कोपपरीतानि हर्षपर्युत्सुकानि च}
{भवन्ति युधि योधानां मुखानि निहते पतौ} %6-48-24

\twolineshloka
{इदं विमानं वैदेहि पुष्पकं नाम नामतः}
{दिव्यं त्वां धारयेन् नेदं यद्येतौ गतजीवितौ} %6-48-25

\twolineshloka
{हतवीरप्रधाना हि गतोत्साहा निरुद्यमा}
{सेना भ्रमति सङ्ख्येषु हतकर्णेव नौर्जले} %6-48-26

\twolineshloka
{इयं पुनरसम्भ्रान्ता निरुद्विग्ना तपस्विनि}
{सेना रक्षति काकुत्स्थौ मया प्रीत्या निवेदितौ} %6-48-27

\twolineshloka
{सा त्वं भव सुविस्रब्धा अनुमानैः सुखोदयैः}
{अहतौ पश्य काकुत्स्थौ स्नेहादेतद् ब्रवीमि ते} %6-48-28

\twolineshloka
{अनृतं नोक्तपूर्वं मे न च वक्ष्यामि मैथिलि}
{चारित्रसुखशीलत्वात् प्रविष्टासि मनो मम} %6-48-29

\twolineshloka
{नेमौ शक्यौ रणे जेतुं सेन्द्रैरपि सुरासुरैः}
{तादृशं दर्शनं दृष्ट्वा मया चोदीरितं तव} %6-48-30

\twolineshloka
{इदं तु सुमहच्चित्रं शरैः पश्यस्व मैथिलि}
{विसंज्ञौ पतितावेतौ नैव लक्ष्मीर्विमुञ्चति} %6-48-31

\twolineshloka
{प्रायेण गतसत्त्वानां पुरुषाणां गतायुषाम्}
{दृश्यमानेषु वक्त्रेषु परं भवति वैकृतम्} %6-48-32

\twolineshloka
{त्यज शोकं च दुःखं च मोहं च जनकात्मजे}
{रामलक्ष्मणयोरर्थे नाद्य शक्यमजीवितुम्} %6-48-33

\twolineshloka
{श्रुत्वा तु वचनं तस्याः सीता सुरसुतोपमा}
{कृताञ्जलिरुवाचेमामेवमस्त्विति मैथिली} %6-48-34

\twolineshloka
{विमानं पुष्पकं तत्तु सन्निवर्त्य मनोजवम्}
{दीना त्रिजटया सीता लङ्कामेव प्रवेशिता} %6-48-35

\twolineshloka
{ततस्त्रिजटया सार्धं पुष्पकादवरुह्य सा}
{अशोकवनिकामेव राक्षसीभिः प्रवेशिता} %6-48-36

\twolineshloka
{प्रविश्य सीता बहुवृक्षखण्डां तां राक्षसेन्द्रस्य विहारभूमिम्}
{सम्प्रेक्ष्य सञ्चिन्त्य च राजपुत्रौ परं विषादं समुपाजगाम} %6-48-37


॥इत्यार्षे श्रीमद्रामायणे वाल्मीकीये आदिकाव्ये युद्धकाण्डे सीताश्वासनम् नाम अष्टचत्वारिंशः सर्गः ॥६-४८॥
