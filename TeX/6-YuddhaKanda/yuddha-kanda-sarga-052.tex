\sect{द्विपञ्चाशः सर्गः — धूम्राक्षवधः}

\twolineshloka
{धूम्राक्षं प्रेक्ष्य निर्यान्तं राक्षसं भीमविक्रमम्}
{विनेदुर्वानराः सर्वे प्रहृष्टा युद्धकाङ्क्षिणः} %6-52-1

\twolineshloka
{तेषां सुतुमुलं युद्धं सञ्जज्ञे कपिरक्षसाम्}
{अन्योन्यं पादपैर्घोरैर्निघ्नतां शूलमुद्गरैः} %6-52-2

\twolineshloka
{राक्षसैर्वानरा घोरा विनिकृत्ताः समन्ततः}
{वानरै राक्षसाश्चापि द्रुमैर्भूमिसमीकृताः} %6-52-3

\twolineshloka
{राक्षसास्त्वभिसङ्क्रुद्धा वानरान् निशितैः शरैः}
{विव्यधुर्घोरसङ्काशैः कङ्कपत्रैरजिह्मगैः} %6-52-4

\twolineshloka
{ते गदाभिश्च भीमाभिः पट्टिशैः कूटमुद्गरैः}
{घोरैश्च परिघैश्चित्रैस्त्रिशूलैश्चापि संश्रितैः} %6-52-5

\twolineshloka
{विदार्यमाणा रक्षोभिर्वानरास्ते महाबलाः}
{अमर्षजनितोद्धर्षाश्चक्रुः कर्माण्यभीतवत्} %6-52-6

\twolineshloka
{शरनिर्भिन्नगात्रास्ते शूलनिर्भिन्नदेहिनः}
{जगृहुस्ते द्रुमांस्तत्र शिलाश्च हरियूथपाः} %6-52-7

\twolineshloka
{ते भीमवेगा हरयो नर्दमानास्ततस्ततः}
{ममन्थू राक्षसान् वीरान् नामानि च बभाषिरे} %6-52-8

\twolineshloka
{तद् बभूवाद्भुतं घोरं युद्धं वानररक्षसाम्}
{शिलाभिर्विविधाभिश्च बहुशाखैश्च पादपैः} %6-52-9

\twolineshloka
{राक्षसा मथिताः केचिद् वानरैर्जितकाशिभिः}
{प्रवेमू रुधिरं केचिन्मुखै रुधिरभोजनाः} %6-52-10

\twolineshloka
{पार्श्वेषु दारिताः केचित् केचिद् राशीकृता द्रुमैः}
{शिलाभिश्चूर्णिताः केचित् केचिद् दन्तैर्विदारिताः} %6-52-11

\twolineshloka
{ध्वजैर्विमथितैर्भग्नैः खड्गैश्च विनिपातितैः}
{रथैर्विध्वंसितैः केचिद् व्यथिता रजनीचराः} %6-52-12

\twolineshloka
{गजेन्द्रैः पर्वताकारैः पर्वताग्रैर्वनौकसाम्}
{मथितैर्वाजिभिः कीर्णं सारोहैर्वसुधातलम्} %6-52-13

\twolineshloka
{वानरैर्भीमविक्रान्तैराप्लुत्योत्प्लुत्य वेगितैः}
{राक्षसाः करजैस्तीक्ष्णैर्मुखेषु विनिदारिताः} %6-52-14

\twolineshloka
{विषण्णवदना भूयो विप्रकीर्णशिरोरुहाः}
{मूढाः शोणितगन्धेन निपेतुर्धरणीतले} %6-52-15

\twolineshloka
{अन्ये तु परमक्रुद्धा राक्षसा भीमविक्रमाः}
{तलैरेवाभिधावन्ति वज्रस्पर्शसमैर्हरीन्} %6-52-16

\twolineshloka
{वानरैः पातयन्तस्ते वेगिता वेगवत्तरैः}
{मुष्टिभिश्चरणैर्दन्तैः पादपैश्चावपोथिताः} %6-52-17

\twolineshloka
{सैन्यं तु विद्रुतं दृष्ट्वा धूम्राक्षो राक्षसर्षभः}
{रोषेण कदनं चक्रे वानराणां युयुत्सताम्} %6-52-18

\twolineshloka
{प्रासैः प्रमथिताः केचिद् वानराः शोणितस्रवाः}
{मुद्गरैराहताः केचित् पतिता धरणीतले} %6-52-19

\twolineshloka
{परिघैर्मथिताः केचिद् भिन्दिपालैश्च दारिताः}
{पट्टिशैर्मथिताः केचिद् विह्वलन्तो गतासवः} %6-52-20

\twolineshloka
{केचिद् विनिहता भूमौ रुधिरार्द्रा वनौकसः}
{केचिद् विद्राविता नष्टाः सङ्क्रुद्धै राक्षसैर्युधि} %6-52-21

\twolineshloka
{विभिन्नहृदयाः केचिदेकपार्श्वेन शायिताः}
{विदारितास्त्रिशूलैश्च केचिदान्त्रैर्विनिःसृताः} %6-52-22

\twolineshloka
{तत् सुभीमं महद्युद्धं हरिराक्षससङ्कुलम्}
{प्रबभौ शस्त्रबहुलं शिलापादपसङ्कुलम्} %6-52-23

\twolineshloka
{धनुर्ज्यातन्त्रिमधुरं हिक्कातालसमन्वितम्}
{मन्दस्तनितगीतं तद् युद्धगान्धर्वमाबभौ} %6-52-24

\twolineshloka
{धूम्राक्षस्तु धनुष्पाणिर्वानरान् रणमूर्धनि}
{हसन् विद्रावयामास दिशस्ताञ्छरवृष्टिभिः} %6-52-25

\twolineshloka
{धूम्राक्षेणार्दितं सैन्यं व्यथितं प्रेक्ष्य मारुतिः}
{अभ्यवर्तत सङ्क्रुद्धः प्रगृह्य विपुलां शिलाम्} %6-52-26

\twolineshloka
{क्रोधाद् द्विगुणताम्राक्षः पितुस्तुल्यपराक्रमः}
{शिलां तां पातयामास धूम्राक्षस्य रथं प्रति} %6-52-27

\twolineshloka
{आपतन्तीं शिलां दृष्ट्वा गदामुद्यम्य सम्भ्रमात्}
{रथादाप्लुत्य वेगेन वसुधायां व्यतिष्ठत} %6-52-28

\twolineshloka
{सा प्रमथ्य रथं तस्य निपपात शिला भुवि}
{सचक्रकूबरं साश्वं सध्वजं सशरासनम्} %6-52-29

\twolineshloka
{स भङ्क्त्वा तु रथं तस्य हनूमान् मारुतात्मजः}
{रक्षसां कदनं चक्रे सस्कन्धविटपैर्द्रुमैः} %6-52-30

\twolineshloka
{विभिन्नशिरसो भूत्वा राक्षसा रुधिरोक्षिताः}
{द्रुमैः प्रमथिताश्चान्ये निपेतुर्धरणीतले} %6-52-31

\twolineshloka
{विद्राव्य राक्षसं सैन्यं हनूमान् मारुतात्मजः}
{गिरेः शिखरमादाय धूम्राक्षमभिदुद्रुवे} %6-52-32

\twolineshloka
{तमापतन्तं धूम्राक्षो गदामुद्यम्य वीर्यवान्}
{विनर्दमानः सहसा हनूमन्तमभिद्रवत्} %6-52-33

\twolineshloka
{तस्य क्रुद्धस्य रोषेण गदां तां बहुकण्टकाम्}
{पातयामास धूम्राक्षो मस्तकेऽथ हनूमतः} %6-52-34

\twolineshloka
{ताडितः स तया तत्र गदया भीमवेगया}
{स कपिर्मारुतबलस्तं प्रहारमचिन्तयन्} %6-52-35

\twolineshloka
{धूम्राक्षस्य शिरोमध्ये गिरिशृङ्गमपातयत्}
{स विस्फारितसर्वाङ्गो गिरिशृङ्गेण ताडितः} %6-52-36

\threelineshloka
{पपात सहसा भूमौ विकीर्ण इव पर्वतः}
{धूम्राक्षं निहतं दृष्ट्वा हतशेषा निशाचराः}
{त्रस्ताः प्रविविशुर्लङ्कां वध्यमानाः प्लवङ्गमैः} %6-52-37

\twolineshloka
{स तु पवनसुतो निहत्य शत्रून् क्षतजवहाः सरितश्च संविकीर्य}
{रिपुवधजनितश्रमो महात्मा मुदमगमत् कपिभिः सुपूज्यमानः} %6-52-38


॥इत्यार्षे श्रीमद्रामायणे वाल्मीकीये आदिकाव्ये युद्धकाण्डे धूम्राक्षवधः नाम द्विपञ्चाशः सर्गः ॥६-५२॥
