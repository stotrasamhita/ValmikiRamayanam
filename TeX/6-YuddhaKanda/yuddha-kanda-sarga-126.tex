\sect{षड्विंशत्यधिकशततमः सर्गः — प्रत्यवृत्तिपथवर्णनम्}

\twolineshloka
{अनुज्ञातं तु रामेण तद् विमानमनुत्तमम्}
{हंसयुक्तं महानादमुत्पपात विहायसम्} %6-126-1

\twolineshloka
{पातयित्वा ततश्चक्षुः सर्वतो रघुनन्दनः}
{अब्रवीन्मैथिलीं सीतां रामः शशिनिभाननाम्} %6-126-2

\twolineshloka
{कैलासशिखराकारे त्रिकूटशिखरे स्थिताम्}
{लङ्कामीक्षस्व वैदेहि निर्मितां विश्वकर्मणा} %6-126-3

\twolineshloka
{एतदायोधनं पश्य मांसशोणितकर्दमम्}
{हरीणां राक्षसानां च सीते विशसनं महत्} %6-126-4

\twolineshloka
{एष दत्तवरः शेते प्रमाथी राक्षसेश्वरः}
{तव हेतोर्विशालाक्षि निहतो रावणो मया} %6-126-5

\twolineshloka
{कुम्भकर्णोऽत्र निहतः प्रहस्तश्च निशाचरः}
{धूम्राक्षश्चात्र निहतो वानरेण हनूमता} %6-126-6

\twolineshloka
{विद्युन्माली हतश्चात्र सुषेणेन महात्मना}
{लक्ष्मणेनेन्द्रजिच्चात्र रावणिर्निहतो रणे} %6-126-7

\twolineshloka
{अङ्गदेनात्र निहतो विकटो नाम राक्षसः}
{विरूपाक्षश्च दुष्प्रेक्षो महापार्श्वमहोदरौ} %6-126-8

\twolineshloka
{अकम्पनश्च निहतो बलिनोऽन्ये च राक्षसाः}
{त्रिशिराश्चातिकायश्च देवान्तकनरान्तकौ} %6-126-9

\twolineshloka
{युद्धोन्मत्तश्च मत्तश्च राक्षसप्रवरावुभौ}
{निकुम्भश्चैव कुम्भश्च कुम्भकर्णात्मजौ बली} %6-126-10

\twolineshloka
{वज्रदंष्ट्रश्च दंष्ट्रश्च बहवो राक्षसा हताः}
{मकराक्षश्च दुर्धर्षो मया युधि निपातितः} %6-126-11

\twolineshloka
{अकम्पनश्च निहतः शोणिताक्षश्च वीर्यवान्}
{यूपाक्षश्च प्रजङ्घश्च निहतौ तु महाहवे} %6-126-12

\twolineshloka
{विद्युज्जिह्वोऽत्र निहतो राक्षसो भीमदर्शनः}
{यज्ञशत्रुश्च निहतः सुप्तघ्नश्च महाबलः} %6-126-13

\twolineshloka
{सूर्यशत्रुश्च निहतो ब्रह्मशत्रुस्तथापरः}
{अत्र मन्दोदरी नाम भार्या तं पर्यदेवयत्} %6-126-14

\twolineshloka
{सपत्नीनां सहस्रेण साग्रेण परिवारिता}
{एतत् तु दृश्यते तीर्थं समुद्रस्य वरानने} %6-126-15

\twolineshloka
{यत्र सागरमुत्तीर्य तां रात्रिमुषिता वयम्}
{एष सेतुर्मया बद्धः सागरे लवणार्णवे} %6-126-16

\twolineshloka
{तव हेतोर्विशालाक्षि नलसेतुः सुदुष्करः}
{पश्य सागरमक्षोभ्यं वैदेहि वरुणालयम्} %6-126-17

\twolineshloka
{अपारमिव गर्जन्तं शङ्खशुक्तिसमाकुलम्}
{हिरण्यनाभं शैलेन्द्रं काञ्चनं पश्य मैथिलि} %6-126-18

\twolineshloka
{विश्रमार्थं हनुमतो भित्त्वा सागरमुत्थितम्}
{एतत् कुक्षौ समुद्रस्य स्कन्धावारनिवेशनम्} %6-126-19

\twolineshloka
{अत्र पूर्वं महादेवः प्रसादमकरोद् विभुः}
{एतत् तु दृश्यते तीर्थं सागरस्य महात्मनः} %6-126-20

\twolineshloka
{सेतुबन्ध इति ख्यातं त्रैलोक्येन च पूजितम्}
{एतत् पवित्रं परमं महापातकनाशनम्} %6-126-21

\twolineshloka
{अत्र राक्षसराजोऽयमाजगाम विभीषणः}
{एषा सा दृश्यते सीते किष्किन्धा चित्रकानना} %6-126-22

\twolineshloka
{सुग्रीवस्य पुरी रम्या यत्र वाली मया हतः}
{अथ दृष्ट्वा पुरीं सीता किष्किन्धां वालिपालिताम्} %6-126-23

\twolineshloka
{अब्रवीत् प्रश्रितं वाक्यं रामं प्रणयसाध्वसा}
{सुग्रीवप्रियभार्याभिस्ताराप्रमुखतो नृप} %6-126-24

\twolineshloka
{अन्येषां वानरेन्द्राणां स्त्रीभिः परिवृता ह्यहम्}
{गन्तुमिच्छे सहायोध्यां राजधानीं त्वया सह} %6-126-25

\twolineshloka
{एवमुक्तोऽथ वैदेह्या राघवः प्रत्युवाच ताम्}
{एवमस्त्विति किष्किन्धां प्राप्य संस्थाप्य राघवः} %6-126-26

\twolineshloka
{विमानं प्रेक्ष्य सुग्रीवं वाक्यमेतदुवाच ह}
{ब्रूहि वानरशार्दूल सर्वान् वानरपुङ्गवान्} %6-126-27

\twolineshloka
{स्त्रीभिः परिवृताः सर्वे ह्ययोध्यां यान्तु सीतया}
{तथा त्वमपि सर्वाभिः स्त्रीभिः सह महाबल} %6-126-28

\twolineshloka
{अभित्वरय सुग्रीव गच्छामः प्लवगाधिप}
{एवमुक्तस्तु सुग्रीवो रामेणामिततेजसा} %6-126-29

\twolineshloka
{वानराधिपतिः श्रीमांस्तैश्च सर्वैः समावृतः}
{प्रविश्यान्तःपुरं शीघ्रं तारामुद्वीक्ष्य सोऽब्रवीत्} %6-126-30

\twolineshloka
{प्रिये त्वं सह नारीभिर्वानराणां महात्मनाम्}
{राघवेणाभ्यनुज्ञाता मैथिलीप्रियकाम्यया} %6-126-31

\twolineshloka
{त्वर त्वमभिगच्छामो गृह्य वानरयोषितः}
{अयोध्यां दर्शयिष्यामः सर्वा दशरथस्त्रियः} %6-126-32

\twolineshloka
{सुग्रीवस्य वचः श्रुत्वा तारा सर्वाङ्गशोभना}
{आहूय चाब्रवीत् सर्वा वानराणां तु योषितः} %6-126-33

\twolineshloka
{सुग्रीवेणाभ्यनुज्ञाता गन्तुं सर्वैश्च वानरैः}
{मम चापि प्रियं कार्यमयोध्यादर्शनेन च} %6-126-34

\twolineshloka
{प्रवेशं चैव रामस्य पौरजानपदैः सह}
{विभूतिं चैव सर्वासां स्त्रीणां दशरथस्य च} %6-126-35

\twolineshloka
{तारया चाभ्यनुज्ञाताः सर्वा वानरयोषितः}
{नेपथ्यविधिपूर्वं तु कृत्वा चापि प्रदक्षिणम्} %6-126-36

\twolineshloka
{अध्यारोहन् विमानं तत् सीतादर्शनकाङ्क्षया}
{ताभिः सहोत्थितं शीघ्रं विमानं प्रेक्ष्य राघवः} %6-126-37

\twolineshloka
{ऋष्यमूकसमीपे तु वैदेहीं पुनरब्रवीत्}
{दृश्यतेऽसौ महान् सीते सविद्युदिव तोयदः} %6-126-38

\twolineshloka
{ऋष्यमूको गिरिवरः काञ्चनैर्धातुभिर्वृतः}
{अत्राहं वानरेन्द्रेण सुग्रीवेण समागतः} %6-126-39

\twolineshloka
{समयश्च कृतः सीते वधार्थं वालिनो मया}
{एषा सा दृश्यते पम्पा नलिनी चित्रकानना} %6-126-40

\twolineshloka
{त्वया विहीनो यत्राहं विललाप सुदुःखितः}
{अस्यास्तीरे मया दृष्टा शबरी धर्मचारिणी} %6-126-41

\twolineshloka
{अत्र योजनबाहुश्च कबन्धो निहतो मया}
{दृश्यतेऽसौ जनस्थाने श्रीमान् सीते वनस्पतिः} %6-126-42

\twolineshloka
{जटायुश्च महातेजास्तव हेतोर्विलासिनि}
{रावणेन हतो यत्र पक्षिणां प्रवरो बली} %6-126-43

\twolineshloka
{खरश्च निहतो यत्र दूषणश्च निपातितः}
{त्रिशिराश्च महावीर्यो मया बाणैरजिह्मगैः} %6-126-44

\twolineshloka
{एतत् तदाश्रमपदमस्माकं वरवर्णिनि}
{पर्णशाला तथा चित्रा दृश्यते शुभदर्शने} %6-126-45

\twolineshloka
{यत्र त्वं राक्षसेन्द्रेण रावणेन हृता बलात्}
{एषा गोदावरी रम्या प्रसन्नसलिला शुभा} %6-126-46

\twolineshloka
{अगस्त्यस्याश्रमश्चैव दृश्यते कदलीवृतः}
{दीप्तश्चैवाश्रमे ह्येष सुतीक्ष्णस्य महात्मनः} %6-126-47

\twolineshloka
{दृश्यते चैव वैदेहि शरभङ्गाश्रमो महान्}
{उपयातः सहस्राक्षो यत्र शक्रः पुरंदरः} %6-126-48

\twolineshloka
{अस्मिन् देशे महाकायो विराधो निहतो मया}
{एते ते तापसा देवि दृश्यन्ते तनुमध्यमे} %6-126-49

\twolineshloka
{अत्रिः कुलपतिर्यत्र सूर्यवैश्वानरोपमः}
{अत्र सीते त्वया दृष्टा तापसी धर्मचारिणी} %6-126-50

\twolineshloka
{असौ सुतनु शैलेन्द्रश्चित्रकूटः प्रकाशते}
{अत्र मां कैकयीपुत्रः प्रसादयितुमागतः} %6-126-51

\twolineshloka
{एषा सा यमुना रम्या दृश्यते चित्रकानना}
{भरद्वाजाश्रमः श्रीमान् दृश्यते चैष मैथिलि} %6-126-52

\twolineshloka
{इयं च दृश्यते गङ्गा पुण्या त्रिपथगा नदी}
{नानाद्विजगणाकीर्णा सम्प्रपुष्पितकानना} %6-126-53

\twolineshloka
{शृङ्गवेरपुरं चैतद् गुहो यत्र सखा मम}
{एषा सा दृश्यते सीते सरयूर्यूपमालिनी} %6-126-54

\twolineshloka
{एषा सा दृश्यते सीते राजधानी पितुर्मम}
{अयोध्यां कुरु वैदेहि प्रणामं पुनरागता} %6-126-55

\twolineshloka
{ततस्ते वानराः सर्वे राक्षसाः सविभीषणाः}
{उत्पत्योत्पत्य संहृष्टास्तां पुरीं ददृशुस्तदा} %6-126-56

\twolineshloka
{ततस्तु तां पाण्डुरहर्म्यमालिनीं विशालकक्ष्यां गजवाजिभिर्वृताम्}
{पुरीमपश्यन् प्लवगाः सराक्षसाः पुरीं महेन्द्रस्य यथामरावतीम्} %6-126-57


॥इत्यार्षे श्रीमद्रामायणे वाल्मीकीये आदिकाव्ये युद्धकाण्डे प्रत्यवृत्तिपथवर्णनम् नाम षड्विंशत्यधिकशततमः सर्गः ॥६-१२६॥
