\sect{एकसप्ततितमः सर्गः — अतिकायवधः}

\twolineshloka
{स्वबलं व्यथितं दृष्ट्वा तुमुलं लोमहर्षणम्}
{भ्रातॄंश्च निहतान् दृष्ट्वा शक्रतुल्यपराक्रमान्} %6-71-1

\twolineshloka
{पितृव्यौ चापि संदृश्य समरे संनिपातितौ}
{युद्धोन्मत्तं च मत्तं च भ्रातरौ राक्षसोत्तमौ} %6-71-2

\twolineshloka
{चुकोप च महातेजा ब्रह्मदत्तवरो युधि}
{अतिकायोऽद्रिसंकाशो देवदानवदर्पहा} %6-71-3

\twolineshloka
{स भास्करसहस्रस्य संघातमिव भास्वरम्}
{रथमारुह्य शक्रारिरभिदुद्राव वानरान्} %6-71-4

\twolineshloka
{स विस्फार्य तदा चापं किरीटी मृष्टकुण्डलः}
{नाम संश्रावयामास ननाद च महास्वनम्} %6-71-5

\twolineshloka
{तेन सिंहप्रणादेन नामविश्रावणेन च}
{ज्याशब्देन च भीमेन त्रासयामास वानरान्} %6-71-6

\twolineshloka
{ते दृष्ट्वा देहमाहात्म्यं कुम्भकर्णोऽयमुत्थितः}
{भयार्ता वानराः सर्वे संश्रयन्ते परस्परम्} %6-71-7

\twolineshloka
{ते तस्य रूपमालोक्य यथा विष्णोस्त्रिविक्रमे}
{भयाद् वानरयोधास्ते विद्रवन्ति ततस्ततः} %6-71-8

\twolineshloka
{तेऽतिकायं समासाद्य वानरा मूढचेतसः}
{शरण्यं शरणं जग्मुर्लक्ष्मणाग्रजमाहवे} %6-71-9

\twolineshloka
{ततोऽतिकायं काकुत्स्थो रथस्थं पर्वतोपमम्}
{ददर्श धन्विनं दूराद् गर्जन्तं कालमेघवत्} %6-71-10

\twolineshloka
{स तं दृष्ट्वा महाकायं राघवस्तु सुविस्मितः}
{वानरान् सान्त्वयित्वा च विभीषणमुवाच ह} %6-71-11

\twolineshloka
{कोऽसौ पर्वतसंकाशो धनुष्मान् हरिलोचनः}
{युक्ते हयसहस्रेण विशाले स्यन्दने स्थितः} %6-71-12

\twolineshloka
{य एष निशितैः शूलैः सुतीक्ष्णैः प्रासतोमरैः}
{अर्चिष्मद्भिर्वृतो भाति भूतैरिव महेश्वरः} %6-71-13

\twolineshloka
{कालजिह्वाप्रकाशाभिर्य एषोऽभिविराजते}
{आवृतो रथशक्तीभिर्विद्युद्भिरिव तोयदः} %6-71-14

\twolineshloka
{धनूंषि चास्य सज्जानि हेमपृष्ठानि सर्वशः}
{शोभयन्ति रथश्रेष्ठं शक्रचापमिवाम्बरम्} %6-71-15

\twolineshloka
{य एष रक्षःशार्दूलो रणभूमिं विराजयन्}
{अभ्येति रथिनां श्रेष्ठो रथेनादित्यवर्चसा} %6-71-16

\twolineshloka
{ध्वजशृङ्गप्रतिष्ठेन राहुणाभिविराजते}
{सूर्यरश्मिप्रभैर्बाणैर्दिशो दश विराजयन्} %6-71-17

\twolineshloka
{त्रिनतं मेघनिर्ह्रादं हेमपृष्ठमलंकृतम्}
{शतक्रतुधनुःप्रख्यं धनुश्चास्य विराजते} %6-71-18

\twolineshloka
{सध्वजः सपताकश्च सानुकर्षो महारथः}
{चतुःसादिसमायुक्तो मेघस्तनितनिःस्वनः} %6-71-19

\twolineshloka
{विंशतिर्दश चाष्टौ च तूणास्य रथमास्थिताः}
{कार्मुकाणि च भीमानि ज्याश्च काञ्चनपिङ्गलाः} %6-71-20

\twolineshloka
{द्वौ च खड्गौ च पार्श्वस्थौ प्रदीप्तौ पार्श्वशोभितौ}
{चतुर्हस्तत्सरुयुतौ व्यक्तहस्तदशायतौ} %6-71-21

\twolineshloka
{रक्तकण्ठगुणो धीरो महापर्वतसंनिभः}
{कालः कालमहावक्त्रो मेघस्थ इव भास्करः} %6-71-22

\twolineshloka
{काञ्चनाङ्गदनद्धाभ्यां भुजाभ्यामेष शोभते}
{शृङ्गाभ्यामिव तुङ्गाभ्यां हिमवान् पर्वतोत्तमः} %6-71-23

\twolineshloka
{कुण्डलाभ्यामुभाभ्यां च भाति वक्त्रं सुभीषणम्}
{पुनर्वस्वन्तरगतं परिपूर्णो निशाकरः} %6-71-24

\twolineshloka
{आचक्ष्व मे महाबाहो त्वमेनं राक्षसोत्तमम्}
{यं दृष्ट्वा वानराः सर्वे भयार्ता विद्रुता दिशः} %6-71-25

\twolineshloka
{स पृष्टो राजपुत्रेण रामेणामिततेजसा}
{आचचक्षे महातेजा राघवाय विभीषणः} %6-71-26

\twolineshloka
{दशग्रीवो महातेजा राजा वैश्रवणानुजः}
{भीमकर्मा महात्मा हि रावणो राक्षसेश्वरः} %6-71-27

\twolineshloka
{तस्यासीद् वीर्यवान् पुत्रो रावणप्रतिमो बले}
{वृद्धसेवी श्रुतिधरः सर्वास्त्रविदुषां वरः} %6-71-28

\twolineshloka
{अश्वपृष्ठे नागपृष्ठे खड्गे धनुषि कर्षणे}
{भेदे सान्त्वे च दाने च नये मन्त्रे च सम्मतः} %6-71-29

\twolineshloka
{यस्य बाहुं समाश्रित्य लङ्का भवति निर्भया}
{तनयं धान्यमालिन्या अतिकायमिमं विदुः} %6-71-30

\twolineshloka
{एतेनाराधितो ब्रह्मा तपसा भावितात्मना}
{अस्त्राणि चाप्यवाप्तानि रिपवश्च पराजिताः} %6-71-31

\twolineshloka
{सुरासुरैरवध्यत्वं दत्तमस्मै स्वयंभुवा}
{एतच्च कवचं दिव्यं रथश्च रविभास्वरः} %6-71-32

\twolineshloka
{एतेन शतशो देवा दानवाश्च पराजिताः}
{रक्षितानि च रक्षांसि यक्षाश्चापि निषूदिताः} %6-71-33

\twolineshloka
{वज्रं विष्टम्भितं येन बाणैरिन्द्रस्य धीमता}
{पाशः सलिलराजस्य युद्धे प्रतिहतस्तथा} %6-71-34

\twolineshloka
{एषोऽतिकायो बलवान् राक्षसानामथर्षभः}
{स रावणसुतो धीमान् देवदानवदर्पहा} %6-71-35

\twolineshloka
{तदस्मिन् क्रियतां यत्नः क्षिप्रं पुरुषपुङ्गव}
{पुरा वानरसैन्यानि क्षयं नयति सायकैः} %6-71-36

\twolineshloka
{ततोऽतिकायो बलवान् प्रविश्य हरिवाहिनीम्}
{विस्फारयामास धनुर्ननाद च पुनः पुनः} %6-71-37

\twolineshloka
{तं भीमवपुषं दृष्ट्वा रथस्थं रथिनां वरम्}
{अभिपेतुर्महात्मानः प्रधाना ये वनौकसः} %6-71-38

\twolineshloka
{कुमुदो द्विविदो मैन्दो नीलः शरभ एव च}
{पादपैर्गिरिशृङ्गैश्च युगपत् समभिद्रवन्} %6-71-39

\twolineshloka
{तेषां वृक्षांश्च शैलांश्च शरैः कनकभूषणैः}
{अतिकायो महातेजाश्चिच्छेदास्त्रविदां वरः} %6-71-40

\twolineshloka
{तांश्चैव सर्वान् स हरीन् शरैः सर्वायसैर्बली}
{विव्याधाभिमुखान् संख्ये भीमकायो निशाचरः} %6-71-41

\twolineshloka
{तेऽर्दिता बाणवर्षेण भिन्नगात्राः पराजिताः}
{न शेकुरतिकायस्य प्रतिकर्तुं महाहवे} %6-71-42

\twolineshloka
{तत् सैन्यं हरिवीराणां त्रासयामास राक्षसः}
{मृगयूथमिव क्रुद्धो हरिर्यौवनदर्पितः} %6-71-43

\twolineshloka
{स राक्षसेन्द्रो हरियूथमध्ये नायुध्यमानं निजघान कंचित्}
{उत्पत्य रामं स धनुःकलापी सगर्वितं वाक्यमिदं बभाषे} %6-71-44

\twolineshloka
{रथे स्थितोऽहं शरचापपाणिर्न प्राकृतं कंचन योधयामि}
{यस्यास्ति शक्तिर्व्यवसाययुक्तो ददातु मे शीघ्रमिहाद्य युद्धम्} %6-71-45

\twolineshloka
{तत् तस्य वाक्यं ब्रुवतो निशम्य चुकोप सौमित्रिरमित्रहन्ता}
{अमृष्यमाणश्च समुत्पपात जग्राह चापं च ततः स्मयित्वा} %6-71-46

\twolineshloka
{क्रुद्धः सौमित्रिरुत्पत्य तूणादाक्षिप्य सायकम्}
{पुरस्तादतिकायस्य विचकर्ष महद्धनुः} %6-71-47

\twolineshloka
{पूरयन् स महीं सर्वामाकाशं सागरं दिशः}
{ज्याशब्दो लक्ष्मणस्योग्रस्त्रासयन् रजनीचरान्} %6-71-48

\twolineshloka
{सौमित्रेश्चापनिर्घोषं श्रुत्वा प्रतिभयं तदा}
{विसिस्मिये महातेजा राक्षसेन्द्रात्मजो बली} %6-71-49

\twolineshloka
{तदातिकायः कुपितो दृष्ट्वा लक्ष्मणमुत्थितम्}
{आदाय निशितं बाणमिदं वचनमब्रवीत्} %6-71-50

\twolineshloka
{बालस्त्वमसि सौमित्रे विक्रमेष्वविचक्षणः}
{गच्छ किं कालसंकाशं मां योधयितुमिच्छसि} %6-71-51

\twolineshloka
{नहि मद्बाहुसृष्टानां बाणानां हिमवानपि}
{सोढुमुत्सहते वेगमन्तरिक्षमथो मही} %6-71-52

\twolineshloka
{सुखप्रसुप्तं कालाग्निं विबोधयितुमिच्छसि}
{न्यस्य चापं निवर्तस्व प्राणान्न जहि मद्गतः} %6-71-53

\twolineshloka
{अथवा त्वं प्रतिस्तब्धो न निवर्तितुमिच्छसि}
{तिष्ठ प्राणान् परित्यज्य गमिष्यसि यमक्षयम्} %6-71-54

\twolineshloka
{पश्य मे निशितान् बाणान् रिपुदर्पनिषूदनान्}
{ईश्वरायुधसंकाशांस्तप्तकाञ्चनभूषणान्} %6-71-55

\threelineshloka
{एष ते सर्पसंकाशो बाणः पास्यति शोणितम्}
{मृगराज इव क्रुद्धो नागराजस्य शोणितम्}
{इत्येवमुक्त्वा संक्रुद्धः शरं धनुषि संदधे} %6-71-56

\twolineshloka
{श्रुत्वातिकायस्य वचः सरोषं सगर्वितं संयति राजपुत्रः}
{स संचुकोपातिबलो मनस्वी उवाच वाक्यं च ततो महार्थम्} %6-71-57

\twolineshloka
{न वाक्यमात्रेण भवान् प्रधानो न कत्थनात् सत्पुरुषा भवन्ति}
{मयि स्थिते धन्विनि बाणपाणौ निदर्शयस्वात्मबलं दुरात्मन्} %6-71-58

\twolineshloka
{कर्मणा सूचयात्मानं न विकत्थितुमर्हसि}
{पौरुषेण तु यो युक्तः स तु शूर इति स्मृतः} %6-71-59

\twolineshloka
{सर्वायुधसमायुक्तो धन्वी त्वं रथमास्थितः}
{शरैर्वा यदि वाप्यस्त्रैर्दर्शयस्व पराक्रमम्} %6-71-60

\twolineshloka
{ततः शिरस्ते निशितैः पातयिष्याम्यहं शरैः}
{मारुतः कालसम्पक्वं वृन्तात् तालफलं यथा} %6-71-61

\twolineshloka
{अद्य ते मामका बाणास्तप्तकाञ्चनभूषणाः}
{पास्यन्ति रुधिरं गात्राद् बाणशल्यान्तरोत्थितम्} %6-71-62

\twolineshloka
{बालोऽयमिति विज्ञाय न चावज्ञातुमर्हसि}
{बालो वा यदि वा वृद्धो मृत्युं जानीहि संयुगे} %6-71-63

\threelineshloka
{बालेन विष्णुना लोकास्त्रयः क्रान्तास्त्रिविक्रमैः}
{लक्ष्मणस्य वचः श्रुत्वा हेतुमत् परमार्थवत्}
{अतिकायः प्रचुक्रोध बाणं चोत्तममाददे} %6-71-64

\twolineshloka
{ततो विद्याधरा भूता देवा दैत्या महर्षयः}
{गुह्यकाश्च महात्मानस्तद् युद्धं द्रष्टुमागमन्} %6-71-65

\twolineshloka
{ततोऽतिकायः कुपितश्चापमारोप्य सायकम्}
{लक्ष्मणाय प्रचिक्षेप संक्षिपन्निव चाम्बरम्} %6-71-66

\twolineshloka
{तमापतन्तं निशितं शरमाशीविषोपमम्}
{अर्धचन्द्रेण चिच्छेद लक्ष्मणः परवीरहा} %6-71-67

\twolineshloka
{तं निकृत्तं शरं दृष्ट्वा कृत्तभोगमिवोरगम्}
{अतिकायो भृशं क्रुद्धः पञ्च बाणान् समादधे} %6-71-68

\twolineshloka
{तान् शरान् सम्प्रचिक्षेप लक्ष्मणाय निशाचरः}
{तानप्राप्तान् शितैर्बाणैश्चिच्छेद भरतानुजः} %6-71-69

\twolineshloka
{स तान् छित्त्वा शितैर्बाणैर्लक्ष्मणः परवीरहा}
{आददे निशितं बाणं ज्वलन्तमिव तेजसा} %6-71-70

\twolineshloka
{तमादाय धनुःश्रेष्ठे योजयामास लक्ष्मणः}
{विचकर्ष च वेगेन विससर्ज च सायकम्} %6-71-71

\twolineshloka
{पूर्णायतविसृष्टेन शरेण नतपर्वणा}
{ललाटे राक्षसश्रेष्ठमाजघान स वीर्यवान्} %6-71-72

\twolineshloka
{स ललाटे शरो मग्नस्तस्य भीमस्य रक्षसः}
{ददृशे शोणितेनाक्तः पन्नगेन्द्र इवाचले} %6-71-73

\twolineshloka
{राक्षसः प्रचकम्पेऽथ लक्ष्मणेषु प्रपीडितः}
{रुद्रबाणहतं घोरं यथा त्रिपुरगोपुरम्} %6-71-74

\twolineshloka
{चिन्तयामास चाश्वास्य विमृश्य च महाबलः}
{साधु बाणनिपातेन श्लाघनीयोऽसि मे रिपुः} %6-71-75

\twolineshloka
{विधायैवं विदार्यास्यं नियम्य च महाभुजौ}
{स रथोपस्थमास्थाय रथेन प्रचचार ह} %6-71-76

\twolineshloka
{एवं त्रीन् पञ्च सप्तेति सायकान् राक्षसर्षभः}
{आददे संदधे चापि विचकर्षोत्ससर्ज च} %6-71-77

\twolineshloka
{ते बाणाः कालसंकाशा राक्षसेन्द्रधनुश्च्युताः}
{हेमपुङ्खा रविप्रख्याश्चक्रुर्दीप्तमिवाम्बरम्} %6-71-78

\twolineshloka
{ततस्तान् राक्षसोत्सृष्टान् शरौघान् राघवानुजः}
{असम्भ्रान्तः प्रचिच्छेद निशितैर्बहुभिः शरैः} %6-71-79

\twolineshloka
{तान् शरान् युधि सम्प्रेक्ष्य निकृत्तान् रावणात्मजः}
{चुकोप त्रिदशेन्द्रारिर्जग्राह निशितं शरम्} %6-71-80

\twolineshloka
{स संधाय महातेजास्तं बाणं सहसोत्सृजत्}
{तेन सौमित्रिमायान्तमाजघान स्तनान्तरे} %6-71-81

\twolineshloka
{अतिकायेन सौमित्रिस्ताडितो युधि वक्षसि}
{सुस्राव रुधिरं तीव्रं मदं मत्त इव द्विपः} %6-71-82

\twolineshloka
{स चकार तदात्मानं विशल्यं सहसा विभुः}
{जग्राह च शरं तीक्ष्णमस्त्रेणापि समाददे} %6-71-83

\twolineshloka
{आग्नेयेन तदास्त्रेण योजयामास सायकम्}
{स जज्वाल तदा बाणो धनुष्यस्य महात्मनः} %6-71-84

\twolineshloka
{अतिकायोऽतितेजस्वी रौद्रमस्त्रं समाददे}
{तेन बाणं भुजङ्गाभं हेमपुङ्खमयोजयत्} %6-71-85

\twolineshloka
{तदस्त्रं ज्वलितं घोरं लक्ष्मणः शरमाहितम्}
{अतिकायाय चिक्षेप कालदण्डमिवान्तकः} %6-71-86

\twolineshloka
{आग्नेयास्त्राभिसंयुक्तं दृष्ट्वा बाणं निशाचरः}
{उत्ससर्ज तदा बाणं रौद्रं सूर्यास्त्रयोजितम्} %6-71-87

\twolineshloka
{तावुभावम्बरे बाणावन्योन्यमभिजघ्नतुः}
{तेजसा सम्प्रदीप्ताग्रौ क्रुद्धाविव भुजङ्गमौ} %6-71-88

\onelineshloka
{तावन्योन्यं विनिर्दह्य पेततुः पृथिवीतले} %6-71-89

\twolineshloka
{निरर्चिषौ भस्मकृतौ न भ्राजेते शरोत्तमौ}
{तावुभौ दीप्यमानौ स्म न भ्राजेते महीतले} %6-71-90

\twolineshloka
{ततोऽतिकायः संक्रुद्धस्त्वाष्ट्रमैषीकमुत्सृजत्}
{ततश्चिच्छेद सौमित्रिरस्त्रमैन्द्रेण वीर्यवान्} %6-71-91

\twolineshloka
{ऐषीकं निहतं दृष्ट्वा कुमारो रावणात्मजः}
{याम्येनास्त्रेण संक्रुद्धो योजयामास सायकम्} %6-71-92

\twolineshloka
{ततस्तदस्त्रं चिक्षेप लक्ष्मणाय निशाचरः}
{वायव्येन तदस्त्रेण निजघान स लक्ष्मणः} %6-71-93

\twolineshloka
{अथैनं शरधाराभिर्धाराभिरिव तोयदः}
{अभ्यवर्षत संक्रुद्धो लक्ष्मणो रावणात्मजम्} %6-71-94

\twolineshloka
{तेऽतिकायं समासाद्य कवचे वज्रभूषिते}
{भग्नाग्रशल्याः सहसा पेतुर्बाणा महीतले} %6-71-95

\twolineshloka
{तान्मोघानभिसम्प्रेक्ष्य लक्ष्मणः परवीरहा}
{अभ्यवर्षत बाणानां सहस्रेण महायशाः} %6-71-96

\twolineshloka
{स वृष्यमाणो बाणौघैरतिकायो महाबलः}
{अवध्यकवचः संख्ये राक्षसो नैव विव्यथे} %6-71-97

\twolineshloka
{शरं चाशीविषाकारं लक्ष्मणाय व्यपासृजत्}
{स तेन विद्धः सौमित्रिर्मर्मदेशे शरेण ह} %6-71-98

\twolineshloka
{मुहूर्तमात्रं निःसंज्ञो ह्यभवच्छत्रुतापनः}
{ततः संज्ञामुपालभ्य चतुर्भिः सायकोत्तमैः} %6-71-99

\twolineshloka
{निजघान हयान् संख्ये सारथिं च महाबलः}
{ध्वजस्योन्मथनं कृत्वा शरवर्षैररिंदमः} %6-71-100

\twolineshloka
{असम्भ्रान्तः स सौमित्रिस्तान् शरानभिलक्षितान्}
{मुमोच लक्ष्मणो बाणान् वधार्थं तस्य रक्षसः} %6-71-101

\twolineshloka
{न शशाक रुजं कर्तुं युधि तस्य नरोत्तमः}
{अथैनमभ्युपागम्य वायुर्वाक्यमुवाच ह} %6-71-102

\threelineshloka
{ब्रह्मदत्तवरो ह्येष अवध्यकवचावृतः}
{ब्राह्मेणास्त्रेण भिन्ध्येनमेष वध्यो हि नान्यथा}
{अवध्य एष ह्यन्येषामस्त्राणां कवची बली} %6-71-103

\twolineshloka
{ततस्तु वायोर्वचनं निशम्य सौमित्रिरिन्द्रप्रतिमानवीर्यः}
{समादधे बाणमथोग्रवेगं तद्ब्राह्ममस्त्रं सहसा नियुज्य} %6-71-104

\twolineshloka
{तस्मिन् वरास्त्रे तु नियुज्यमाने सौमित्रिणा बाणवरे शिताग्रे}
{दिशश्च चन्द्रार्कमहाग्रहाश्च नभश्च तत्रास ररास चोर्वी} %6-71-105

\twolineshloka
{तं ब्रह्मणोऽस्त्रेण नियुज्य चापे शरं सपुङ्खं यमदूतकल्पम्}
{सौमित्रिरिन्द्रारिसुतस्य तस्य ससर्ज बाणं युधि वज्रकल्पम्} %6-71-106

\twolineshloka
{तं लक्ष्मणोत्सृष्टविवृद्धवेगं समापतन्तं श्वसनोग्रवेगम्}
{सुपर्णवज्रोत्तमचित्रपुङ्खं तदातिकायः समरे ददर्श} %6-71-107

\twolineshloka
{तं प्रेक्षमाणः सहसातिकायो जघान बाणैर्निशितैरनेकैः}
{स सायकस्तस्य सुपर्णवेगस्तथातिवेगेन जगाम पार्श्वम्} %6-71-108

\twolineshloka
{तमागतं प्रेक्ष्य तदातिकायो बाणं प्रदीप्तान्तककालकल्पम्}
{जघान शक्त्यृष्टिगदाकुठारैः शूलैः शरैश्चाप्यविपन्नचेष्टः} %6-71-109

\twolineshloka
{तान्यायुधान्यद्भुतविग्रहाणि मोघानि कृत्वा स शरोऽग्निदीप्तः}
{प्रगृह्य तस्यैव किरीटजुष्टं तदातिकायस्य शिरो जहार} %6-71-110

\twolineshloka
{तच्छिरः सशिरस्त्राणं लक्ष्मणेषुप्रमर्दितम्}
{पपात सहसा भूमौ शृङ्गं हिमवतो यथा} %6-71-111

\twolineshloka
{तं भूमौ पतितं दृष्ट्वा विक्षिप्ताम्बरभूषणम्}
{बभूवुर्व्यथिताः सर्वे हतशेषा निशाचराः} %6-71-112

\twolineshloka
{ते विषण्णमुखा दीनाः प्रहारजनितश्रमाः}
{विनेदुरुच्चैर्बहवः सहसा विस्वरैः स्वरैः} %6-71-113

\twolineshloka
{ततस्तत्परितं याता निरपेक्षा निशाचराः}
{पुरीमभिमुखा भीता द्रवन्तो नायके हते} %6-71-114

\twolineshloka
{प्रहर्षयुक्ता बहवस्तु वानराः प्रफुल्लपद्मप्रतिमाननास्तदा}
{अपूजयँल्लक्ष्मणमिष्टभागिनं हते रिपौ भीमबले दुरासदे} %6-71-115

\twolineshloka
{अतिबलमतिकायमभ्रकल्पं युधि विनिपात्य स लक्ष्मणः प्रहृष्टः}
{त्वरितमथ तदा स रामपार्श्वं कपिनिवहैश्च सुपूजितो जगाम} %6-71-116


॥इत्यार्षे श्रीमद्रामायणे वाल्मीकीये आदिकाव्ये युद्धकाण्डे अतिकायवधः नाम एकसप्ततितमः सर्गः ॥६-७१॥
