\sect{सप्ततितमः सर्गः — देवान्तकादिवधः}

\twolineshloka
{नरान्तकं हतं दृष्ट्वा चुक्रुशुर्नैर्ऋतर्षभाः}
{देवान्तकस्त्रिमूर्धा च पौलस्त्यश्च महोदरः} %6-70-1

\twolineshloka
{आरूढो मेघसंकाशं वारणेन्द्रं महोदरः}
{वालिपुत्रं महावीर्यमभिदुद्राव वेगवान्} %6-70-2

\twolineshloka
{भ्रातृव्यसनसंतप्तस्तदा देवान्तको बली}
{आदाय परिघं घोरमङ्गदं समभिद्रवत्} %6-70-3

\twolineshloka
{रथमादित्यसंकाशं युक्तं परमवाजिभिः}
{आस्थाय त्रिशिरा वीरो वालिपुत्रमथाभ्यगात्} %6-70-4

\twolineshloka
{स त्रिभिर्देवदर्पघ्नै राक्षसेन्द्रैरभिद्रुतः}
{वृक्षमुत्पाटयामास महाविटपमङ्गदः} %6-70-5

\twolineshloka
{देवान्तकाय तं वीरश्चिक्षेप सहसाङ्गदः}
{महावृक्षं महाशाखं शक्रो दीप्तामिवाशनिम्} %6-70-6

\twolineshloka
{त्रिशिरास्तं प्रचिच्छेद शरैराशीविषोपमैः}
{स वृक्षं कृत्तमालोक्य उत्पपात तदाङ्गदः} %6-70-7

\twolineshloka
{स ववर्ष ततो वृक्षान् शिलाश्च कपिकुञ्जरः}
{तान् प्रचिच्छेद संक्रुद्धस्त्रिशिरा निशितैः शरैः} %6-70-8

\twolineshloka
{परिघाग्रेण तान् वृक्षान् बभञ्ज स महोदरः}
{त्रिशिराश्चाङ्गदं वीरमभिदुद्राव सायकैः} %6-70-9

\twolineshloka
{गजेन समभिद्रुत्य वालिपुत्रं महोदरः}
{जघानोरसि संक्रुद्धस्तोमरैर्वज्रसंनिभैः} %6-70-10

\twolineshloka
{देवान्तकश्च संक्रुद्धः परिघेण तदाङ्गदम्}
{उपगम्याभिहत्याशु व्यपचक्राम वेगवान्} %6-70-11

\twolineshloka
{स त्रिभिर्नैर्ऋतश्रेष्ठैर्युगपत् समभिद्रुतः}
{न विव्यथे महातेजा वालिपुत्रः प्रतापवान्} %6-70-12

\twolineshloka
{स वेगवान् महावेगं कृत्वा परमदुर्जयः}
{तलेन समभिद्रुत्य जघानास्य महागजम्} %6-70-13

\twolineshloka
{तस्य तेन प्रहारेण नागराजस्य संयुगे}
{पेततुर्नयने तस्य विननाश स कुञ्जरः} %6-70-14

\twolineshloka
{विषाणं चास्य निष्कृष्य वालिपुत्रो महाबलः}
{देवान्तकमभिद्रुत्य ताडयामास संयुगे} %6-70-15

\twolineshloka
{स विह्वलस्तु तेजस्वी वातोद्धूत इव द्रुमः}
{लाक्षारससवर्णं च सुस्राव रुधिरं महत्} %6-70-16

\twolineshloka
{अथाश्वास्य महातेजाः कृच्छ्राद् देवान्तको बली}
{आविध्य परिघं वेगादाजघान तदाङ्गदम्} %6-70-17

\twolineshloka
{परिघाभिहतश्चापि वानरेन्द्रात्मजस्तदा}
{जानुभ्यां पतितो भूमौ पुनरेवोत्पपात ह} %6-70-18

\twolineshloka
{तमुत्पतन्तं त्रिशिरास्त्रिभिर्बाणैरजिह्मगैः}
{घोरैर्हरिपतेः पुत्रं ललाटेऽभिजघान ह} %6-70-19

\twolineshloka
{ततोऽङ्गदं परिक्षिप्तं त्रिभिर्नैर्ऋतपुङ्गवैः}
{हनूमानथ विज्ञाय नीलश्चापि प्रतस्थतुः} %6-70-20

\twolineshloka
{ततश्चिक्षेप शैलाग्रं नीलस्त्रिशिरसे तदा}
{तद् रावणसुतो धीमान् बिभेद निशितैः शरैः} %6-70-21

\twolineshloka
{तद्बाणशतनिर्भिन्नं विदारितशिलातलम्}
{सविस्फुलिङ्गं सज्वालं निपपात गिरेः शिरः} %6-70-22

\twolineshloka
{स विजृम्भितमालोक्य हर्षाद् देवान्तको बली}
{परिघेणाभिदुद्राव मारुतात्मजमाहवे} %6-70-23

\twolineshloka
{तमापतन्तमुत्पत्य हनूमान् कपिकुञ्जरः}
{आजघान तदा मूर्ध्नि वज्रकल्पेन मुष्टिना} %6-70-24

\twolineshloka
{शिरसि प्राहरद् वीरस्तदा वायुसुतो बली}
{नादेनाकम्पयच्चैव राक्षसान् स महाकपिः} %6-70-25

\twolineshloka
{स मुष्टिनिष्पिष्टविभिन्नमूर्धा निर्वान्तदन्ताक्षिविलम्बिजिह्वः}
{देवान्तको राक्षसराजसूनुर्गतासुरुर्व्यां सहसा पपात} %6-70-26

\twolineshloka
{तस्मिन् हते राक्षसयोधमुख्ये महाबले संयति देवशत्रौ}
{क्रुद्धस्त्रिशीर्षा निशितास्त्रमुग्रं ववर्ष नीलोरसि बाणवर्षम्} %6-70-27

\twolineshloka
{महोदरस्तु संक्रुद्धः कुञ्जरं पर्वतोपमम्}
{भूयः समधिरुह्याशु मन्दरं रश्मिवानिव} %6-70-28

\twolineshloka
{ततो बाणमयं वर्षं नीलस्योपर्यपातयत्}
{गिरौ वर्षं तडिच्चक्रचापवानिव तोयदः} %6-70-29

\twolineshloka
{ततः शरौघैरभिवृष्यमाणो विभिन्नगात्रः कपिसैन्यपालः}
{नीलो बभूवाथ विसृष्टगात्रो विष्टम्भितस्तेन महाबलेन} %6-70-30

\twolineshloka
{ततस्तु नीलः प्रतिलब्धसंज्ञः शैलं समुत्पाट्य सवृक्षखण्डम्}
{ततः समुत्पत्य महोग्रवेगो महोदरं तेन जघान मूर्ध्नि} %6-70-31

\twolineshloka
{ततः स शैलाभिनिपातभग्नो महोदरस्तेन महाद्विपेन}
{व्यामोहितो भूमितले गतासुः पपात वज्राभिहतो यथाद्रिः} %6-70-32

\twolineshloka
{पितृव्यं निहतं दृष्ट्वा त्रिशिराश्चापमाददे}
{हनूमन्तं च संक्रुद्धो विव्याध निशितैः शरैः} %6-70-33

\twolineshloka
{स वायुसूनुः कुपितश्चिक्षेप शिखरं गिरेः}
{त्रिशिरास्तच्छरैस्तीक्ष्णैर्बिभेद बहुधा बली} %6-70-34

\twolineshloka
{तद् व्यर्थं शिखरं दृष्ट्वा द्रुमवर्षं तदा कपिः}
{विससर्ज रणे तस्मिन् रावणस्य सुतं प्रति} %6-70-35

\twolineshloka
{तमापतन्तमाकाशे द्रुमवर्षं प्रतापवान्}
{त्रिशिरा निशितैर्बाणैश्चिच्छेद च ननाद च} %6-70-36

\twolineshloka
{हनूमांस्तु समुत्पत्य हयं त्रिशिरसस्तदा}
{विददार नखैः क्रुद्धो नागेन्द्रं मृगराडिव} %6-70-37

\twolineshloka
{अथ शक्तिं समासाद्य कालरात्रिमिवान्तकः}
{चिक्षेपानिलपुत्राय त्रिशिरा रावणात्मजः} %6-70-38

\twolineshloka
{दिवः क्षिप्तामिवोल्कां तां शक्तिं क्षिप्तामसङ्गताम्}
{गृहीत्वा हरिशार्दूलो बभञ्ज च ननाद च} %6-70-39

\twolineshloka
{तां दृष्ट्वा घोरसंकाशां शक्तिं भग्नां हनूमता}
{प्रहृष्टा वानरगणा विनेदुर्जलदा यथा} %6-70-40

\twolineshloka
{ततः खड्गं समुद्यम्य त्रिशिरा राक्षसोत्तमः}
{निचखान तदा खड्गं वानरेन्द्रस्य वक्षसि} %6-70-41

\twolineshloka
{खड्गप्रहाराभिहतो हनूमान् मारुतात्मजः}
{आजघान त्रिमूर्धानं तलेनोरसि वीर्यवान्} %6-70-42

\twolineshloka
{स तलाभिहतस्तेन स्रस्तहस्तायुधो भुवि}
{निपपात महातेजास्त्रिशिरास्त्यक्तचेतनः} %6-70-43

\twolineshloka
{स तस्य पततः खड्गं तमाच्छिद्य महाकपिः}
{ननाद गिरिसंकाशस्त्रासयन् सर्वराक्षसान्} %6-70-44

\twolineshloka
{अमृष्यमाणस्तं घोषमुत्पपात निशाचरः}
{उत्पत्य च हनूमन्तं ताडयामास मुष्टिना} %6-70-45

\twolineshloka
{तेन मुष्टिप्रहारेण संचुकोप महाकपिः}
{कुपितश्च निजग्राह किरीटे राक्षसर्षभम्} %6-70-46

\twolineshloka
{स तस्य शीर्षाण्यसिना शितेन किरीटजुष्टानि सकुण्डलानि}
{क्रुद्धः प्रचिच्छेद सुतोऽनिलस्य त्वष्टुः सुतस्येव शिरांसि शक्रः} %6-70-47

\twolineshloka
{तान्यायताक्षाण्यगसंनिभानि प्रदीप्तवैश्वानरलोचनानि}
{पेतुः शिरांसीन्द्ररिपोः पृथिव्यां ज्योतींषि मुक्तानि यथार्कमार्गात्} %6-70-48

\twolineshloka
{तस्मिन् हते देवरिपौ त्रिशीर्षे हनूमता शक्रपराक्रमेण}
{नेदुः प्लवंगाः प्रचचाल भूमी रक्षांस्यथो दुद्रुविरे समन्तात्} %6-70-49

\twolineshloka
{हतं त्रिशिरसं दृष्ट्वा तथैव च महोदरम्}
{हतौ प्रेक्ष्य दुराधर्षौ देवान्तकनरान्तकौ} %6-70-50

\twolineshloka
{चुकोप परमामर्षी मत्तो राक्षसपुङ्गवः}
{जग्राहार्चिष्मतीं चापि गदां सर्वायसीं तदा} %6-70-51

\twolineshloka
{हेमपट्टपरिक्षिप्तां मांसशोणितफेनिलाम्}
{विराजमानां विपुलां शत्रुशोणिततर्पिताम्} %6-70-52

\twolineshloka
{तेजसा सम्प्रदीप्ताग्रां रक्तमाल्यविभूषिताम्}
{ऐरावतमहापद्मसार्वभौमभयावहाम्} %6-70-53

\twolineshloka
{गदामादाय संक्रुद्धो मत्तो राक्षसपुङ्गवः}
{हरीन् समभिदुद्राव युगान्ताग्निरिव ज्वलन्} %6-70-54

\twolineshloka
{अथर्षभः समुत्पत्य वानरो रावणानुजम्}
{मत्तानीकमुपागम्य तस्थौ तस्याग्रतो बली} %6-70-55

\twolineshloka
{तं पुरस्तात् स्थितं दृष्ट्वा वानरं पर्वतोपमम्}
{आजघानोरसि क्रुद्धो गदया वज्रकल्पया} %6-70-56

\twolineshloka
{स तयाभिहतस्तेन गदया वानरर्षभः}
{भिन्नवक्षाः समाधूतः सुस्राव रुधिरं बहु} %6-70-57

\twolineshloka
{स सम्प्राप्य चिरात् संज्ञामृषभो वानरेश्वरः}
{क्रुद्धो विस्फुरमाणौष्ठो महापार्श्वमुदैक्षत} %6-70-58

\twolineshloka
{स वेगवान् वेगवदभ्युपेत्य तं राक्षसं वानरवीरमुख्यः}
{संवर्त्य मुष्टिं सहसा जघान बाह्वन्तरे शैलनिकाशरूपः} %6-70-59

\twolineshloka
{स कृत्तमूलः सहसेव वृक्षः क्षितौ पपात क्षतजोक्षिताङ्गः}
{तां चास्य घोरां यमदण्डकल्पां गदां प्रगृह्याशु तदा ननाद} %6-70-60

\twolineshloka
{मुहूर्तमासीत् स गतासुकल्पः प्रत्यागतात्मा सहसा सुरारिः}
{उत्पत्य संध्याभ्रसमानवर्णस्तं वारिराजात्मजमाजघान} %6-70-61

\twolineshloka
{स मूर्च्छितो भूमितले पपात मुहूर्तमुत्पत्य पुनः ससंज्ञः}
{तामेव तस्याद्रिवराद्रिकल्पां गदां समाविध्य जघान संख्ये} %6-70-62

\twolineshloka
{सा तस्य रौद्रा समुपेत्य देहं रौद्रस्य देवाध्वरविप्रशत्रोः}
{बिभेद वक्षः क्षतजं च भूरि सुस्राव धात्वम्भ इवाद्रिराजः} %6-70-63

\twolineshloka
{अभिदुद्राव वेगेन गदां तस्य महात्मनः}
{तां गृहीत्वा गदां भीमामाविध्य च पुनः पुनः} %6-70-64

\twolineshloka
{मत्तानीकं महात्मा स जघान रणमूर्धनि}
{स स्वया गदया भग्नो विशीर्णदशनेक्षणः} %6-70-65

\threelineshloka
{निपपात तदा मत्तो वज्राहत इवाचलः}
{विशीर्णनयने भूमौ गतसत्त्वे गतायुषि}
{पतिते राक्षसे तस्मिन् विद्रुतं राक्षसं बलम्} %6-70-66

\twolineshloka
{तस्मिन् हते भ्रातरि रावणस्य तन्नैर्ऋतानां बलमर्णवाभम्}
{त्यक्तायुधं केवलजीवितार्थं दुद्राव भिन्नार्णवसंनिकाशम्} %6-70-67


॥इत्यार्षे श्रीमद्रामायणे वाल्मीकीये आदिकाव्ये युद्धकाण्डे देवान्तकादिवधः नाम सप्ततितमः सर्गः ॥६-७०॥
