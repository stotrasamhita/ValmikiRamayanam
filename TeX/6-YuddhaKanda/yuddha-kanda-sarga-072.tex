\sect{द्विसप्ततितमः सर्गः — रावणमन्युशल्याविष्कारः}

\twolineshloka
{अतिकायं हतं श्रुत्वा लक्ष्मणेन महात्मना}
{उद्वेगमगमद् राजा वचनं चेदमब्रवीत्} %6-72-1

\twolineshloka
{धूम्राक्षः परमामर्षी सर्वशस्त्रभृतां वरः}
{अकम्पनः प्रहस्तश्च कुम्भकर्णस्तथैव च} %6-72-2

\twolineshloka
{एते महाबला वीरा राक्षसा युद्धकाङ्क्षिणः}
{जेतारः परसैन्यानां परैर्नित्यापराजिताः} %6-72-3

\twolineshloka
{ससैन्यास्ते हता वीरा रामेणाक्लिष्टकर्मणा}
{राक्षसाः सुमहाकाया नानाशस्त्रविशारदाः} %6-72-4

\twolineshloka
{अन्ये च बहवः शूरा महात्मानो निपातिताः}
{प्रख्यातबलवीर्येण पुत्रेणेन्द्रजिता मम} %6-72-5

\twolineshloka
{तौ भ्रातरौ तदा बद्धौ घोरैर्दत्तवरैः शरैः}
{यन्न शक्यं सुरैः सर्वैरसुरैर्वा महाबलैः} %6-72-6

\twolineshloka
{मोक्तुं तद्बन्धनं घोरं यक्षगन्धर्वपन्नगैः}
{तन्न जाने प्रभावैर्वा मायया मोहनेन वा} %6-72-7

\twolineshloka
{शरबन्धाद् विमुक्तौ तौ भ्रातरौ रामलक्ष्मणौ}
{ये योधा निर्गताः शूरा राक्षसा मम शासनात्} %6-72-8

\twolineshloka
{ते सर्वे निहता युद्धे वानरैः सुमहाबलैः}
{तं न पश्याम्यहं युद्धे योऽद्य रामं सलक्ष्मणम्} %6-72-9

\twolineshloka
{नाशयेत् सबलं वीरं ससुग्रीवं विभीषणम्}
{अहो सुबलवान् रामो महदस्त्रबलं च वै} %6-72-10

\twolineshloka
{यस्य विक्रममासाद्य राक्षसा निधनं गताः}
{तं मन्ये राघवं वीरं नारायणमनामयम्} %6-72-11

\twolineshloka
{तद्भयाद्धि पुरी लङ्का पिहितद्वारतोरणा}
{अप्रमत्तैश्च सर्वत्र गुल्मे रक्ष्या पुरी त्वियम्} %6-72-12

\twolineshloka
{अशोकवनिका चैव यत्र सीताभिरक्ष्यते}
{निष्क्रमो वा प्रवेशो वा ज्ञातव्यः सर्वदैव नः} %6-72-13

\twolineshloka
{यत्र यत्र भवेद् गुल्मस्तत्र तत्र पुनः पुनः}
{सर्वतश्चापि तिष्ठध्वं स्वैः स्वैः परिवृता बलैः} %6-72-14

\twolineshloka
{द्रष्टव्यं च पदं तेषां वानराणां निशाचराः}
{प्रदोषे वार्धरात्रे वा प्रत्यूषे वापि सर्वशः} %6-72-15

\twolineshloka
{नावज्ञा तत्र कर्तव्या वानरेषु कदाचन}
{द्विषतां बलमुद्युक्तमापतत् किं स्थितं यथा} %6-72-16

\twolineshloka
{ततस्ते राक्षसाः सर्वे श्रुत्वा लङ्काधिपस्य तत्}
{वचनं सर्वमातिष्ठन् यथावत् तु महाबलाः} %6-72-17

\twolineshloka
{तान् सर्वान् हि समादिश्य रावणो राक्षसाधिपः}
{मन्युशल्यं वहन् दीनः प्रविवेश स्वमालयम्} %6-72-18

\twolineshloka
{ततः स सन्दीपितकोपवह्निर्निशाचराणामधिपो महाबलः}
{तदेव पुत्रव्यसनं विचिन्तयन् मुहुर्मुहुश्चैव तदा विनिःश्वसन्} %6-72-19


॥इत्यार्षे श्रीमद्रामायणे वाल्मीकीये आदिकाव्ये युद्धकाण्डे रावणमन्युशल्याविष्कारः नाम द्विसप्ततितमः सर्गः ॥६-७२॥
