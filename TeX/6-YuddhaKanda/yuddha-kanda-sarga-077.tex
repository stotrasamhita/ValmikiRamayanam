\sect{सप्तसप्ततितमः सर्गः — निकुम्भवधः}

\twolineshloka
{निकुम्भो भ्रातरं दृष्ट्वा सुग्रीवेण निपातितम्}
{प्रदहन्निव कोपेन वानरेन्द्रमुदैक्षत} %6-77-1

\twolineshloka
{ततः स्रग्दामसंनद्धं दत्तपञ्चाङ्गुलं शुभम्}
{आददे परिघं धीरो महेन्द्रशिखरोपमम्} %6-77-2

\twolineshloka
{हेमपट्टपरिक्षिप्तं वज्रविद्रुमभूषितम्}
{यमदण्डोपमं भीमं रक्षसां भयनाशनम्} %6-77-3

\twolineshloka
{तमाविध्य महातेजाः शक्रध्वजसमौजसम्}
{निननाद विवृत्तास्यो निकुम्भो भीमविक्रमः} %6-77-4

\twolineshloka
{उरोगतेन निष्केण भुजस्थैरङ्गदैरपि}
{कुण्डलाभ्यां च चित्राभ्यां मालया च सचित्रया} %6-77-5

\twolineshloka
{निकुम्भो भूषणैर्भाति तेन स्म परिघेण च}
{यथेन्द्रधनुषा मेघः सविद्युत्स्तनयित्नुमान्} %6-77-6

\twolineshloka
{परिघाग्रेण पुस्फोट वातग्रन्थिर्महात्मनः}
{प्रजज्वाल सघोषश्च विधूम इव पावकः} %6-77-7

\threelineshloka
{नगर्या विटपावत्या गन्धर्वभवनोत्तमैः}
{सतारागणनक्षत्रं सचन्द्रसमहाग्रहम्}
{निकुम्भपरिघाघूर्णं भ्रमतीव नभस्थलम्} %6-77-8

\twolineshloka
{दुरासदश्च संजज्ञे परिघाभरणप्रभः}
{क्रोधेन्धनो निकुम्भाग्निर्युगान्ताग्निरिवोत्थितः} %6-77-9

\twolineshloka
{राक्षसा वानराश्चापि न शेकुः स्पन्दितुं भयात्}
{हनुमांस्तु विवृत्योरस्तस्थौ प्रमुखतो बली} %6-77-10

\twolineshloka
{परिघोपमबाहुस्तु परिघं भास्करप्रभम्}
{बली बलवतस्तस्य पातयामास वक्षसि} %6-77-11

\twolineshloka
{स्थिरे तस्योरसि व्यूढे परिघः शतधा कृतः}
{विकीर्यमाणः सहसा उल्काशतमिवाम्बरे} %6-77-12

\twolineshloka
{स तु तेन प्रहारेण न चचाल महाकपिः}
{परिघेण समाधूतो यथा भूमिचलेऽचलः} %6-77-13

\twolineshloka
{स तथाभिहतस्तेन हनूमान् प्लवगोत्तमः}
{मुष्टिं संवर्तयामास बलेनातिमहाबलः} %6-77-14

\twolineshloka
{तमुद्यम्य महातेजा निकुम्भोरसि वीर्यवान्}
{अभिचिक्षेप वेगेन वेगवान् वायुविक्रमः} %6-77-15

\twolineshloka
{तत्र पुस्फोट वर्मास्य प्रसुस्राव च शोणितम्}
{मुष्टिना तेन संजज्ञे मेघे विद्युदिवोत्थिता} %6-77-16

\twolineshloka
{स तु तेन प्रहारेण निकुम्भो विचचाल च}
{स्वस्थश्चापि निजग्राह हनूमन्तं महाबलम्} %6-77-17

\twolineshloka
{चुक्रुशुश्च तदा संख्ये भीमं लङ्कानिवासिनः}
{निकुम्भेनोद्यतं दृष्ट्वा हनूमन्तं महाबलम्} %6-77-18

\twolineshloka
{स तथा ह्रियमाणोऽपि हनूमांस्तेन रक्षसा}
{आजघानानिलसुतो वज्रकल्पेन मुष्टिना} %6-77-19

\twolineshloka
{आत्मानं मोक्षयित्वाथ क्षितावभ्यवपद्यत}
{हनूमानुन्ममाथाशु निकुम्भं मारुतात्मजः} %6-77-20

\twolineshloka
{निक्षिप्य परमायत्तो निकुम्भं निष्पिपेष च}
{उत्पत्य चास्य वेगेन पपातोरसि वेगवान्} %6-77-21

\twolineshloka
{परिगृह्य च बाहुभ्यां परिवृत्य शिरोधराम्}
{उत्पाटयामास शिरो भैरवं नदतो महत्} %6-77-22

\twolineshloka
{अथ निनदति सादिते निकुम्भे पवनसुतेन रणे बभूव युद्धम्}
{दशरथसुतराक्षसेन्द्रसून्वोर्भृशतरमागतरोषयोः सुभीमम्} %6-77-23

\twolineshloka
{व्यपेते तु जीवे निकुम्भस्य हृष्टा विनेदुः प्लवंगा दिशः सस्वनुश्च}
{चचालेव चोर्वी पपातेव सा द्यौर्बलं राक्षसानां भयं चाविवेश} %6-77-24


॥इत्यार्षे श्रीमद्रामायणे वाल्मीकीये आदिकाव्ये युद्धकाण्डे निकुम्भवधः नाम सप्तसप्ततितमः सर्गः ॥६-७७॥
