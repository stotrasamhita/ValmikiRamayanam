\sect{नवतितमः सर्गः — सौमित्रिरावणियुद्धम्}

\twolineshloka
{युध्यमानौ ततो दृष्ट्वा प्रसक्तौ नरराक्षसौ}
{प्रभिन्नाविव मातङ्गौ परस्परजयैषिणौ} %6-90-1

\twolineshloka
{तयोर्युद्धं द्रष्टुकामो वरचापधरो बली}
{शूरः स रावणभ्राता तस्थौ संग्राममूर्धनि} %6-90-2

\twolineshloka
{ततो विस्फारयामास महद् धनुरवस्थितः}
{उत्ससर्ज च तीक्ष्णाग्रान् राक्षसेषु महाशरान्} %6-90-3

\twolineshloka
{ते शराः शिखिसंस्पर्शा निपतन्तः समाहिताः}
{राक्षसान् द्रावयामासुर्वज्राणीव महागिरीन्} %6-90-4

\twolineshloka
{विभीषणस्यानुचरास्तेऽपि शूलासिपट्टिशैः}
{चिच्छिदुः समरे वीरान् राक्षसान् राक्षसोत्तमाः} %6-90-5

\twolineshloka
{राक्षसैस्तैः परिवृतः स तदा तु विभीषणः}
{बभौ मध्ये प्रधृष्टानां कलभानामिव द्विपः} %6-90-6

\twolineshloka
{ततः संचोदमानो वै हरीन् रक्षोवधप्रियान्}
{उवाच वचनं काले कालज्ञो रक्षसां वरः} %6-90-7

\twolineshloka
{एकोऽयं राक्षसेन्द्रस्य परायणमवस्थितः}
{एतच्छेषं बलं तस्य किं तिष्ठत हरीश्वराः} %6-90-8

\twolineshloka
{अस्मिंश्च निहते पापे राक्षसे रणमूर्धनि}
{रावणं वर्जयित्वा तु शेषमस्य बलं हतम्} %6-90-9

\twolineshloka
{प्रहस्तो निहतो वीरो निकुम्भश्च महाबलः}
{कुम्भकर्णश्च कुम्भश्च धूम्राक्षश्च निशाचरः} %6-90-10

\twolineshloka
{जम्बुमाली महामाली तीक्ष्णवेगोऽशनिप्रभः}
{सुप्तघ्नो यज्ञकोपश्च वज्रदंष्ट्रश्च राक्षसः} %6-90-11

\twolineshloka
{संह्रादी विकटोऽरिघ्नस्तपनो मन्द एव च}
{प्रघासः प्रघसश्चैव प्रजङ्घो जङ्घ एव च} %6-90-12

\twolineshloka
{अग्निकेतुश्च दुर्धर्षो रश्मिकेतुश्च वीर्यवान्}
{विद्युज्जिह्वो द्विजिह्वश्च सूर्यशत्रुश्च राक्षसः} %6-90-13

\twolineshloka
{अकम्पनः सुपार्श्वश्च चक्रमाली च राक्षसः}
{कम्पनः सत्त्ववन्तौ तौ देवान्तकनरान्तकौ} %6-90-14

\twolineshloka
{एतान् निहत्यातिबलान् बहून् राक्षससत्तमान्}
{बाहुभ्यां सागरं तीर्त्वा लङ्घ्यतां गोष्पदं लघु} %6-90-15

\twolineshloka
{एतावदेव शेषं वो जेतव्यमिति वानराः}
{हताः सर्वे समागम्य राक्षसा बलदर्पिताः} %6-90-16

\twolineshloka
{अयुक्तं निधनं कर्तुं पुत्रस्य जनितुर्मम}
{घृणामपास्य रामार्थे निहन्यां भ्रातुरात्मजम्} %6-90-17

\twolineshloka
{हन्तुकामस्य मे बाष्पं चक्षुश्चैव निरुध्यति}
{तमेवैष महाबाहुर्लक्ष्मणः शमयिष्यति} %6-90-18

\twolineshloka
{वानरा घ्नत सम्भूय भृत्यानस्य समीपगान्}
{इति तेनातियशसा राक्षसेनाभिचोदिताः} %6-90-19

\threelineshloka
{वानरेन्द्रा जहृषिरे लाङ्गूलानि च विव्यधुः}
{ततस्तु कपिशार्दूलाः क्ष्वेडन्तश्च पुनः पुनः}
{मुमुचुर्विविधान् नादान् मेघान् दृष्ट्वेव बर्हिणः} %6-90-20

\twolineshloka
{जाम्बवानपि तैः सर्वैः स्वयूथ्यैरभिसंवृतः}
{तेऽश्मभिस्ताडयामासुर्नखैर्दन्तैश्च राक्षसान्} %6-90-21

\twolineshloka
{निघ्नन्तमृक्षाधिपतिं राक्षसास्ते महाबलाः}
{परिवव्रुर्भयं त्यक्त्वा तमनेकविधायुधाः} %6-90-22

\twolineshloka
{शरैः परशुभिस्तीक्ष्णैः पट्टिशैर्यष्टितोमरैः}
{जाम्बवन्तं मृधे जघ्नुर्निघ्नन्तं राक्षसीं चमूम्} %6-90-23

\twolineshloka
{स सम्प्रहारस्तुमुलः संजज्ञे कपिरक्षसाम्}
{देवासुराणां क्रुद्धानां यथा भीमो महास्वनः} %6-90-24

\twolineshloka
{हनूमानपि संक्रुद्धः सालमुत्पाट्य पर्वतात्}
{स लक्ष्मणं स्वयं पृष्ठादवरोप्य महामनाः} %6-90-25

\twolineshloka
{रक्षसां कदनं चक्रे दुरासादः सहस्रशः}
{स दत्त्वा तुमुलं युद्धं पितृव्यस्येन्द्रजिद् बली} %6-90-26

\twolineshloka
{लक्ष्मणं परवीरघ्नः पुनरेवाभ्यधावत}
{तौ प्रयुद्धौ तदा वीरौ मृधे लक्ष्मणराक्षसौ} %6-90-27

\twolineshloka
{शरौघानभिवर्षन्तौ जघ्नतुस्तौ परस्परम्}
{अभीक्ष्णमन्तर्दधतुः शरजालैर्महाबलौ} %6-90-28

\twolineshloka
{चन्द्रादित्याविवोष्णान्ते यथा मेघैस्तरस्विनौ}
{नह्यादानं न संधानं धनुषो वा परिग्रहः} %6-90-29

\twolineshloka
{न विप्रमोक्षो बाणानां न विकर्षो न विग्रहः}
{न मुष्टिप्रतिसंधानं न लक्ष्यप्रतिपादनम्} %6-90-30

\twolineshloka
{अदृश्यत तयोस्तत्र युध्यतोः पाणिलाघवात्}
{चापवेगप्रयुक्तैश्च बाणजालैः समन्ततः} %6-90-31

\twolineshloka
{अन्तरिक्षेऽभिसम्पन्ने न रूपाणि चकाशिरे}
{लक्ष्मणो रावणिं प्राप्य रावणिश्चापि लक्ष्मणम्} %6-90-32

\twolineshloka
{अव्यवस्था भवत्युग्रा ताभ्यामन्योन्यविग्रहे}
{ताभ्यामुभाभ्यां तरसा प्रसृष्टैर्विशिखैः शितैः} %6-90-33

\twolineshloka
{निरन्तरमिवाकाशं बभूव तमसा वृतम्}
{तैः पतद्भिश्च बहुभिस्तयोः शरशतैः शितैः} %6-90-34

\twolineshloka
{दिशश्च प्रदिशश्चैव बभूवुः शरसंकुलाः}
{तमसा पिहितं सर्वमासीत् प्रतिभयं महत्} %6-90-35

\twolineshloka
{अस्तं गते सहस्रांशौ संवृते तमसा च वै}
{रुधिरौघा महानद्यः प्रावर्तन्त सहस्रशः} %6-90-36

\twolineshloka
{क्रव्यादा दारुणा वाग्भिश्चिक्षिपुर्भीमनिःस्वनान्}
{न तदानीं ववौ वायुर्न च जज्वाल पावकः} %6-90-37

\twolineshloka
{स्वस्त्यस्तु लोकेभ्य इति जजल्पुस्ते महर्षयः}
{सम्पेतुश्चात्र संतप्ता गन्धर्वाः सह चारणैः} %6-90-38

\twolineshloka
{अथ राक्षससिंहस्य कृष्णान् कनकभूषणान्}
{शरैश्चतुर्भिः सौमित्रिर्विव्याध चतुरो हयान्} %6-90-39

\twolineshloka
{ततोऽपरेण भल्लेन पीतेन निशितेन च}
{सम्पूर्णायतमुक्तेन सुपत्रेण सुवर्चसा} %6-90-40

\twolineshloka
{महेन्द्राशनिकल्पेन सूतस्य विचरिष्यतः}
{स तेन बाणाशनिना तलशब्दानुनादिना} %6-90-41

\twolineshloka
{लाघवाद् राघवः श्रीमान् शिरः कायादपाहरत्}
{स यन्तरि महातेजा हते मन्दोदरीसुतः} %6-90-42

\twolineshloka
{स्वयं सारथ्यमकरोत् पुनश्च धनुरस्पृशत्}
{तदद्भुतमभूत् तत्र सारथ्यं पश्यतां युधि} %6-90-43

\twolineshloka
{हयेषु व्यग्रहस्तं तं विव्याध निशितैः शरैः}
{धनुष्यथ पुनर्व्यग्रं हयेषु मुमुचे शरान्} %6-90-44

\twolineshloka
{छिद्रेषु तेषु बाणौघैर्विचरन्तमभीतवत्}
{अर्दयामास समरे सौमित्रिः शीघ्रकृत्तमः} %6-90-45

\twolineshloka
{निहतं सारथिं दृष्ट्वा समरे रावणात्मजः}
{प्रजहौ समरोद्धर्षं विषण्णः स बभूव ह} %6-90-46

\twolineshloka
{विषण्णवदनं दृष्ट्वा राक्षसं हरियूथपाः}
{ततः परमसंहृष्टा लक्ष्मणं चाभ्यपूजयन्} %6-90-47

\twolineshloka
{ततः प्रमाथी रभसः शरभो गन्धमादनः}
{अमृष्यमाणाश्चत्वारश्चक्रुर्वेगं हरीश्वराः} %6-90-48

\twolineshloka
{ते चास्य हयमुख्येषु तूर्णमुत्पत्य वानराः}
{चतुर्षु सुमहावीर्या निपेतुर्भीमविक्रमाः} %6-90-49

\twolineshloka
{तेषामधिष्ठितानां तैर्वानरैः पर्वतोपमैः}
{मुखेभ्यो रुधिरं व्यक्तं हयानां समवर्तत} %6-90-50

\threelineshloka
{ते हया मथिता भग्ना व्यसवो धरणीं गताः}
{ते निहत्य हयांस्तस्य प्रमथ्य च महारथम्}
{पुनरुत्पत्य वेगेन तस्थुर्लक्ष्मणपार्श्वतः} %6-90-51

\twolineshloka
{स हताश्वादवप्लुत्य रथान्मथितसारथिः}
{शरवर्षेण सौमित्रिमभ्यधावत रावणिः} %6-90-52

\twolineshloka
{ततो महेन्द्रप्रतिमः स लक्ष्मणः पदातिनं तं निहतैर्हयोत्तमैः}
{सृजन्तमाजौ निशितान् शरोत्तमान् भृशं तदा बाणगणैर्व्यदारयत्} %6-90-53


॥इत्यार्षे श्रीमद्रामायणे वाल्मीकीये आदिकाव्ये युद्धकाण्डे सौमित्रिरावणियुद्धम् नाम नवतितमः सर्गः ॥६-९०॥
