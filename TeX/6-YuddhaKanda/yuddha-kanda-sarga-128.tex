\sect{अष्टाविंशत्यधिकशततमः सर्गः — भरतप्रियाख्यानम्}

\twolineshloka
{अयोध्यां तु समालोक्य चिन्तयामास राघवः}
{प्रियकामः प्रियं रामस्ततस्त्वरितविक्रमः} %6-128-1

\twolineshloka
{चिन्तयित्वा ततो दृष्टिं वानरेषु न्यपातयत्}
{उवाच धीमांस्तेजस्वी हनूमन्तं प्लवंगमम्} %6-128-2

\twolineshloka
{अयोध्यां त्वरितो गत्वा शीघ्रं प्लवगसत्तम}
{जानीहि कच्चित् कुशली जनो नृपतिमन्दिरे} %6-128-3

\twolineshloka
{शृङ्गवेरपुरं प्राप्य गुहं गहनगोचरम्}
{निषादाधिपतिं ब्रूहि कुशलं वचनान्मम} %6-128-4

\twolineshloka
{श्रुत्वा तु मां कुशलिनमरोगं विगतज्वरम्}
{भविष्यति गुहः प्रीतः स ममात्मसमः सखा} %6-128-5

\twolineshloka
{अयोध्यायाश्च ते मार्गं प्रवृत्तिं भरतस्य च}
{निवेदयिष्यति प्रीतो निषादाधिपतिर्गुहः} %6-128-6

\twolineshloka
{भरतस्तु त्वया वाच्यः कुशलं वचनान्मम}
{सिद्धार्थं शंस मां तस्मै सभार्यं सहलक्ष्मणम्} %6-128-7

\twolineshloka
{हरणं चापि वैदेह्या रावणेन बलीयसा}
{सुग्रीवेण च संवादं वालिनश्च वधं रणे} %6-128-8

\twolineshloka
{मैथिल्यन्वेषणं चैव यथा चाधिगता त्वया}
{लङ्घयित्वा महातोयमापगापतिमव्ययम्} %6-128-9

\twolineshloka
{उपयानं समुद्रस्य सागरस्य च दर्शनम्}
{यथा च कारितः सेतू रावणश्च यथा हतः} %6-128-10

\twolineshloka
{वरदानं महेन्द्रेण ब्रह्मणा वरुणेन च}
{महादेवप्रसादाच्च पित्रा मम समागमम्} %6-128-11

\twolineshloka
{उपयातं च मां सौम्य भरताय निवेदय}
{सह राक्षसराजेन हरीणामीश्वरेण च} %6-128-12

\twolineshloka
{जित्वा शत्रुगणान् रामः प्राप्य चानुत्तमं यशः}
{उपायाति समृद्धार्थः सह मित्रैर्महाबलैः} %6-128-13

\twolineshloka
{एतच्छ्रुत्वा यमाकारं भजते भरतस्ततः}
{स च ते वेदितव्यः स्यात् सर्वं यच्चापि मां प्रति} %6-128-14

\twolineshloka
{ज्ञेयाः सर्वे च वृत्तान्ता भरतस्येङ्गितानि च}
{तत्त्वेन मुखवर्णेन दृष्ट्या व्याभाषितेन च} %6-128-15

\twolineshloka
{सर्वकामसमृद्धं हि हस्त्यश्वरथसंकुलम्}
{पितृपैतामहं राज्यं कस्य नावर्तयेन्मनः} %6-128-16

\twolineshloka
{संगत्या भरतः श्रीमान् राज्येनार्थी स्वयं भवेत्}
{प्रशास्तु वसुधां सर्वामखिलां रघुनन्दनः} %6-128-17

\twolineshloka
{तस्य बुद्धिं च विज्ञाय व्यवसायं च वानर}
{यावन्न दूरं याताः स्मः क्षिप्रमागन्तुमर्हसि} %6-128-18

\twolineshloka
{इति प्रतिसमादिष्टो हनूमान् मारुतात्मजः}
{मानुषं धारयन् रूपमयोध्यां त्वरितो ययौ} %6-128-19

\twolineshloka
{अथोत्पपात वेगेन हनूमान् मारुतात्मजः}
{गरुत्मानिव वेगेन जिघृक्षन्नुरगोत्तमम्} %6-128-20

\twolineshloka
{लङ्घयित्वा पितृपथं विहगेन्द्रालयं शुभम्}
{गङ्गायमुनयोर्भीमं समतीत्य समागमम्} %6-128-21

\twolineshloka
{शृङ्गवेरपुरं प्राप्य गुहमासाद्य वीर्यवान्}
{स वाचा शुभया हृष्टो हनूमानिदमब्रवीत्} %6-128-22

\twolineshloka
{सखा तु तव काकुत्स्थो रामः सत्यपराक्रमः}
{ससीतः सह सौमित्रिः स त्वां कुशलमब्रवीत्} %6-128-23

\twolineshloka
{पञ्चमीमद्य रजनीमुषित्वा वचनान्मुनेः}
{भरद्वाजाभ्यनुज्ञातं द्रक्ष्यस्यत्रैव राघवम्} %6-128-24

\twolineshloka
{एवमुक्त्वा महातेजाः सम्प्रहृष्टतनूरुहः}
{उत्पपात महावेगाद् वेगवानविचारयन्} %6-128-25

\twolineshloka
{सोऽपश्यद् रामतीर्थं च नदीं वालुकिनीं तथा}
{वरूथीं गोमतीं चैव भीमं शालवनं तथा} %6-128-26

\twolineshloka
{प्रजाश्च बहुसाहस्रीः स्फीताञ्जनपदानपि}
{स गत्वा दूरमध्वानं त्वरितः कपिकुञ्जरः} %6-128-27

\twolineshloka
{आससाद द्रुमान् फुल्लान् नन्दिग्रामसमीपगान्}
{सुराधिपस्योपवने यथा चैत्ररथे द्रुमान्} %6-128-28

\twolineshloka
{स्त्रीभिः सपुत्रैः पौत्रैश्च रममाणैः स्वलंकृतैः}
{क्रोशमात्रे त्वयोध्यायाश्चीरकृष्णाजिनाम्बरम्} %6-128-29

\twolineshloka
{ददर्श भरतं दीनं कृशमाश्रमवासिनम्}
{जटिलं मलदिग्धाङ्गं भ्रातृव्यसनकर्शितम्} %6-128-30

\twolineshloka
{फलमूलाशिनं दान्तं तापसं धर्मचारिणम्}
{समुन्नतजटाभारं वल्कलाजिनवाससम्} %6-128-31

\twolineshloka
{नियतं भावितात्मानं ब्रह्मर्षिसमतेजसम्}
{पादुके ते पुरस्कृत्य प्रशासन्तं वसुंधराम्} %6-128-32

\twolineshloka
{चातुर्वर्ण्यस्य लोकस्य त्रातारं सर्वतो भयात्}
{उपस्थितममात्यैश्च शुचिभिश्च पुरोहितैः} %6-128-33

\twolineshloka
{बलमुख्यैश्च युक्तैश्च काषायाम्बरधारिभिः}
{नहि ते राजपुत्रं तं चीरकृष्णाजिनाम्बरम्} %6-128-34

\twolineshloka
{परिभोक्तुं व्यवस्यन्ति पौरा वै धर्मवत्सलाः}
{तं धर्ममिव धर्मज्ञं देहबन्धमिवापरम्} %6-128-35

\twolineshloka
{उवाच प्राञ्जलिर्वाक्यं हनूमान् मारुतात्मजः}
{वसन्तं दण्डकारण्ये यं त्वं चीरजटाधरम्} %6-128-36

\twolineshloka
{अनुशोचसि काकुत्स्थं स त्वां कौशलमब्रवीत्}
{प्रियमाख्यामि ते देव शोकं त्यज सुदारुणम्} %6-128-37

\twolineshloka
{अस्मिन् मुहूर्ते भ्रात्रा त्वं रामेण सह संगतः}
{निहत्य रावणं रामः प्रतिलभ्य च मैथिलीम्} %6-128-38

\threelineshloka
{उपयाति समृद्धार्थः सह मित्रैर्महाबलैः}
{लक्ष्मणश्च महातेजा वैदेही च यशस्विनी}
{सीता समग्रा रामेण महेन्द्रेण शची यथा} %6-128-39

\twolineshloka
{एवमुक्तो हनुमता भरतः कैकयीसुतः}
{पपात सहसा हृष्टो हर्षान्मोहमुपागमत्} %6-128-40

\twolineshloka
{ततो मुहूर्तादुत्थाय प्रत्याश्वस्य च राघवः}
{हनूमन्तमुवाचेदं भरतः प्रियवादिनम्} %6-128-41

\twolineshloka
{अशोकजैः प्रीतिमयैः कपिमालिङ्ग्य सम्भ्रमात्}
{सिषेच भरतः श्रीमान् विपुलैरश्रुबिन्दुभिः} %6-128-42

\twolineshloka
{देवो वा मानुषो वा त्वमनुक्रोशादिहागतः}
{प्रियाख्यानस्य ते सौम्य ददामि ब्रुवतः प्रियम्} %6-128-43

\twolineshloka
{गवां शतसहस्रं च ग्रामाणां च शतं परम्}
{सकुण्डलाः शुभाचारा भार्याः कन्यास्तु षोडश} %6-128-44

\twolineshloka
{हेमवर्णाः सुनासोरूः शशिसौम्याननाः स्त्रियः}
{सर्वाभरणसम्पन्नाः सम्पन्नाः कुलजातिभिः} %6-128-45

\twolineshloka
{निशम्य रामागमनं नृपात्मजः कपिप्रवीरस्य तदाद्भुतोपमम्}
{प्रहर्षितो रामदिदृक्षयाभवत् पुनश्च हर्षादिदमब्रवीद् वचः} %6-128-46


॥इत्यार्षे श्रीमद्रामायणे वाल्मीकीये आदिकाव्ये युद्धकाण्डे भरतप्रियाख्यानम् नाम अष्टाविंशत्यधिकशततमः सर्गः ॥६-१२८॥
