\sect{चतुर्थः सर्गः — रामाभिषेणनम्}

\twolineshloka
{श्रुत्वा हनूमतो वाक्यं यथावदनुपूर्वशः}
{ततोऽब्रवीन्महातेजा रामः सत्यपराक्रमः} %6-4-1

\twolineshloka
{यन्निवेदयसे लङ्कां पुरीं भीमस्य रक्षसः}
{क्षिप्रमेनां वधिष्यामि सत्यमेतद् ब्रवीमि ते} %6-4-2

\twolineshloka
{अस्मिन् मुहूर्ते सुग्रीव प्रयाणमभिरोचय}
{युक्तो मुहूर्ते विजये प्राप्तो मध्यं दिवाकरः} %6-4-3

\threelineshloka
{सीतां हृत्वा तु तद् यातु क्वासौ यास्यति जीवितः}
{सीता श्रुत्वाभियानं मे आशामेष्यति जीविते}
{जीवितान्तेऽमृतं स्पृष्ट्वा पीत्वामृतमिवातुरः} %6-4-4

\twolineshloka
{उत्तराफाल्गुनी ह्यद्य श्वस्तु हस्तेन योक्ष्यते}
{अभिप्रयाम सुग्रीव सर्वानीकसमावृताः} %6-4-5

\twolineshloka
{निमित्तानि च पश्यामि यानि प्रादुर्भवन्ति वै}
{निहत्य रावणं सीतामानयिष्यामि जानकीम्} %6-4-6

\twolineshloka
{उपरिष्टाद्धि नयनं स्फुरमाणमिमं मम}
{विजयं समनुप्राप्तं शंसतीव मनोरथम्} %6-4-7

\twolineshloka
{ततो वानरराजेन लक्ष्मणेन सुपूजितः}
{उवाच रामो धर्मात्मा पुनरप्यर्थकोविदः} %6-4-8

\twolineshloka
{अग्रे यातु बलस्यास्य नीलो मार्गमवेक्षितुम्}
{वृतः शतसहस्रेण वानराणां तरस्विनाम्} %6-4-9

\twolineshloka
{फलमूलवता नील शीतकाननवारिणा}
{पथा मधुमता चाशु सेनां सेनापते नय} %6-4-10

\twolineshloka
{दूषयेयुर्दुरात्मानः पथि मूलफलोदकम्}
{राक्षसाः पथि रक्षेथास्तेभ्यस्त्वं नित्यमुद्यतः} %6-4-11

\twolineshloka
{निम्नेषु वनदुर्गेषु वनेषु च वनौकसः}
{अभिप्लुत्याभिपश्येयुः परेषां निहितं बलम्} %6-4-12

\twolineshloka
{यत्तु फल्गु बलं किञ्चित् तदत्रैवोपपद्यताम्}
{एतद्धि कृत्यं घोरं नो विक्रमेण प्रयुज्यताम्} %6-4-13

\twolineshloka
{सागरौघनिभं भीममग्रानीकं महाबलाः}
{कपिसिंहाः प्रकर्षन्तु शतशोऽथ सहस्रशः} %6-4-14

\twolineshloka
{गजश्च गिरिसङ्काशो गवयश्च महाबलः}
{गवाक्षश्चाग्रतो यातु गवां दृप्त इवर्षभः} %6-4-15

\twolineshloka
{यातु वानरवाहिन्या वानरः प्लवतां पतिः}
{पालयन् दक्षिणं पार्श्वमृषभो वानरर्षभः} %6-4-16

\twolineshloka
{गन्धहस्तीव दुर्धर्षस्तरस्वी गन्धमादनः}
{यातु वानरवाहिन्याः सव्यं पार्श्वमधिष्ठितः} %6-4-17

\twolineshloka
{यास्यामि बलमध्येऽहं बलौघमभिहर्षयन्}
{अधिरुह्य हनूमन्तमैरावतमिवेश्वरः} %6-4-18

\twolineshloka
{अङ्गदेनैष संयातु लक्ष्मणश्चान्तकोपमः}
{सार्वभौमेन भूतेशो द्रविणाधिपतिर्यथा} %6-4-19

\twolineshloka
{जाम्बवांश्च सुषेणश्च वेगदर्शी च वानरः}
{ऋक्षराजो महाबाहुः कुक्षिं रक्षन्तु ते त्रयः} %6-4-20

\twolineshloka
{राघवस्य वचः श्रुत्वा सुग्रीवो वाहिनीपतिः}
{व्यादिदेश महावीर्यो वानरान् वानरर्षभः} %6-4-21

\twolineshloka
{ते वानरगणाः सर्वे समुत्पत्य महौजसः}
{गुहाभ्यः शिखरेभ्यश्च आशु पुप्लुविरे तदा} %6-4-22

\twolineshloka
{ततो वानरराजेन लक्ष्मणेन च पूजितः}
{जगाम रामो धर्मात्मा ससैन्यो दक्षिणां दिशम्} %6-4-23

\twolineshloka
{शतैः शतसहस्रैश्च कोटिभिश्चायुतैरपि}
{वारणाभैश्च हरिभिर्ययौ परिवृतस्तदा} %6-4-24

\twolineshloka
{तं यान्तमनुयान्ती सा महती हरिवाहिनी}
{हृष्टाः प्रमुदिताः सर्वे सुग्रीवेणापि पालिताः} %6-4-25

\twolineshloka
{आप्लवन्तः प्लवन्तश्च गर्जन्तश्च प्लवङ्गमाः}
{क्ष्वेलन्तो निनदन्तश्च जग्मुर्वै दक्षिणां दिशम्} %6-4-26

\twolineshloka
{भक्षयन्तः सुगन्धीनि मधूनि च फलानि च}
{उद्वहन्तो महावृक्षान् मञ्जरीपुञ्जधारिणः} %6-4-27

\twolineshloka
{अन्योन्यं सहसा दृप्ता निर्वहन्ति क्षिपन्ति च}
{पतन्तश्चोत्पतन्त्यन्ये पातयन्त्यपरे परान्} %6-4-28

\twolineshloka
{रावणो नो निहन्तव्यः सर्वे च रजनीचराः}
{इति गर्जन्ति हरयो राघवस्य समीपतः} %6-4-29

\twolineshloka
{पुरस्तादृषभो नीलो वीरः कुमुद एव च}
{पन्थानं शोधयन्ति स्म वानरैर्बहुभिः सह} %6-4-30

\twolineshloka
{मध्ये तु राजा सुग्रीवो रामो लक्ष्मण एव च}
{बलिभिर्बहुभिर्भीमैर्वृतः शत्रुनिबर्हणः} %6-4-31

\twolineshloka
{हरिः शतबलिर्वीरः कोटिभिर्दशभिर्वृतः}
{सर्वामेको ह्यवष्टभ्य ररक्ष हरिवाहिनीम्} %6-4-32

\twolineshloka
{कोटीशतपरीवारः केसरी पनसो गजः}
{अर्कश्च बहुभिः पार्श्वमेकं तस्याभिरक्षति} %6-4-33

\twolineshloka
{सुषेणो जाम्बवांश्चैव ऋक्षैर्बहुभिरावृतौ}
{सुग्रीवं पुरतः कृत्वा जघनं संररक्षतुः} %6-4-34

\twolineshloka
{तेषां सेनापतिर्वीरो नीलो वानरपुङ्गवः}
{सम्पतन् प्लवतां श्रेष्ठस्तद् बलं पर्यवारयत्} %6-4-35

\twolineshloka
{दरीमुखः प्रजङ्घश्च जम्भोऽथ रभसः कपिः}
{सर्वतश्च ययुर्वीरास्त्वरयन्तः प्लवङ्गमान्} %6-4-36

\twolineshloka
{एवं ते हरिशार्दूला गच्छन्ति बलदर्पिताः}
{अपश्यन्त गिरिश्रेष्ठं सह्यं गिरिशतायुतम्} %6-4-37

\twolineshloka
{सरांसि च सुफुल्लानि तटाकानि वराणि च}
{रामस्य शासनं ज्ञात्वा भीमकोपस्य भीतवत्} %6-4-38

\twolineshloka
{वर्जयन् नागराभ्याशांस्तथा जनपदानपि}
{सागरौघनिभं भीमं तद् वानरबलं महत्} %6-4-39

\twolineshloka
{निःससर्प महाघोरं भीमघोषमिवार्णवम्}
{तस्य दाशरथेः पार्श्वे शूरास्ते कपिकुञ्जराः} %6-4-40

\twolineshloka
{तूर्णमापुप्लुवुः सर्वे सदश्वा इव चोदिताः}
{कपिभ्यामुह्यमानौ तौ शुशुभाते नरर्षभौ} %6-4-41

\twolineshloka
{महद्भ्यामिव संस्पृष्टौ ग्रहाभ्यां चन्द्रभास्करौ}
{ततो वानरराजेन लक्ष्मणेन सुपूजितः} %6-4-42

\twolineshloka
{जगाम रामो धर्मात्मा ससैन्यो दक्षिणां दिशम्}
{तमङ्गदगतो रामं लक्ष्मणः शुभया गिरा} %6-4-43

\twolineshloka
{उवाच परिपूर्णार्थं पूर्णार्थप्रतिभानवान्}
{हृतामवाप्य वैदेहीं क्षिप्रं हत्वा च रावणम्} %6-4-44

\twolineshloka
{समृद्धार्थः समृद्धार्थामयोध्यां प्रतियास्यसि}
{महान्ति च निमित्तानि दिवि भूमौ च राघव} %6-4-45

\twolineshloka
{शुभानि तव पश्यामि सर्वाण्येवार्थसिद्धये}
{अनुवाति शिवो वायुः सेनां मृदुहितः सुखः} %6-4-46

\twolineshloka
{पूर्णवल्गुस्वराश्चेमे प्रवदन्ति मृगद्विजाः}
{प्रसन्नाश्च दिशः सर्वा विमलश्च दिवाकरः} %6-4-47

\threelineshloka
{उशना च प्रसन्नार्चिरनु त्वां भार्गवो गतः}
{ब्रह्मराशिर्विशुद्धश्च शुद्धाश्च परमर्षयः}
{अर्चिष्मन्तः प्रकाशन्ते ध्रुवं सर्वे प्रदक्षिणम्} %6-4-48

\twolineshloka
{त्रिशङ्कुर्विमलो भाति राजर्षिः सपुरोहितः}
{पितामहः पुरोऽस्माकमिक्ष्वाकूणां महात्मनाम्} %6-4-49

\twolineshloka
{विमले च प्रकाशेते विशाखे निरुपद्रवे}
{नक्षत्रं परमस्माकमिक्ष्वाकूणां महात्मनाम्} %6-4-50

\twolineshloka
{नैर्ऋतं नैर्ऋतानां च नक्षत्रमतिपीड्यते}
{मूलो मूलवता स्पृष्टो धूप्यते धूमकेतुना} %6-4-51

\twolineshloka
{सर्वं चैतद् विनाशाय राक्षसानामुपस्थितम्}
{काले कालगृहीतानां नक्षत्रं ग्रहपीडितम्} %6-4-52

\twolineshloka
{प्रसन्नाः सुरसाश्चापो वनानि फलवन्ति च}
{प्रवान्ति नाधिका गन्धा यथर्तुकुसुमा द्रुमाः} %6-4-53

\threelineshloka
{व्यूढानि कपिसैन्यानि प्रकाशन्तेऽधिकं प्रभो}
{देवानामिव सैन्यानि सङ्ग्रामे तारकामये}
{एवमार्य समीक्ष्यैतत् प्रीतो भवितुमर्हसि} %6-4-54

\twolineshloka
{इति भ्रातरमाश्वास्य हृष्टः सौमित्रिरब्रवीत्}
{अथावृत्य महीं कृत्स्नां जगाम हरिवाहिनी} %6-4-55

\twolineshloka
{ऋक्षवानरशार्दूलैर्नखद्रंष्ट्रायुधैरपि}
{कराग्रैश्चरणाग्रैश्च वानरैरुद्धतं रजः} %6-4-56

\twolineshloka
{भीममन्तर्दधे लोकं निवार्य सवितुः प्रभाम्}
{सपर्वतवनाकाशं दक्षिणां हरिवाहिनी} %6-4-57

\twolineshloka
{छादयन्ती ययौ भीमा द्यामिवाम्बुदसन्ततिः}
{उत्तरन्त्याश्च सेनायाः सततं बहुयोजनम्} %6-4-58

\twolineshloka
{नदीस्रोतांसि सर्वाणि सस्यन्दुर्विपरीतवत्}
{सरांसि विमलाम्भांसि द्रुमाकीर्णांश्च पर्वतान्} %6-4-59

\twolineshloka
{समान् भूमिप्रदेशांश्च वनानि फलवन्ति च}
{मध्येन च समन्ताच्च तिर्यक् चाधश्च साविशत्} %6-4-60

\twolineshloka
{समावृत्य महीं कृत्स्नां जगाम महती चमूः}
{ते हृष्टवदनाः सर्वे जग्मुर्मारुतरंहसः} %6-4-61

\twolineshloka
{हरयो राघवस्यार्थे समारोपितविक्रमाः}
{हर्षं वीर्यं बलोद्रेकान् दर्शयन्तः परस्परम्} %6-4-62

\twolineshloka
{यौवनोत्सेकजाद् दर्पाद् विविधांश्चक्रुरध्वनि}
{तत्र केचिद् द्रुतं जग्मुरुत्पेतुश्च तथापरे} %6-4-63

\twolineshloka
{केचित् किलकिलां चक्रुर्वानरा वनगोचराः}
{प्रास्फोटयंश्च पुच्छानि सन्निजघ्नुः पदान्यपि} %6-4-64

\twolineshloka
{भुजान् विक्षिप्य शैलांश्च द्रुमानन्ये बभञ्जिरे}
{आरोहन्तश्च शृङ्गाणि गिरीणां गिरिगोचराः} %6-4-65

\twolineshloka
{महानादान् प्रमुञ्चन्ति क्ष्वेडामन्ये प्रचक्रिरे}
{ऊरुवेगैश्च ममृदुर्लताजालान्यनेकशः} %6-4-66

\twolineshloka
{जृम्भमाणाश्च विक्रान्ता विचिक्रीडुः शिलाद्रुमैः}
{ततः शतसहस्रैश्च कोटिभिश्च सहस्रशः} %6-4-67

\twolineshloka
{वानराणां सुघोराणां श्रीमत्परिवृता मही}
{सा स्म याति दिवारात्रं महती हरिवाहिनी} %6-4-68

\threelineshloka
{प्रहृष्टमुदिताः सर्वे सुग्रीवेणाभिपालिताः}
{वानरास्त्वरिता यान्ति सर्वे युद्धाभिनन्दिनः}
{प्रमोक्षयिषवः सीतां मुहूर्तं क्वापि नावसन्} %6-4-69

\twolineshloka
{ततः पादपसम्बाधं नानावनसमायुतम्}
{सह्यपर्वतमासाद्य वानरास्ते समारुहन्} %6-4-70

\twolineshloka
{काननानि विचित्राणि नदीप्रस्रवणानि च}
{पश्यन्नपि ययौ रामः सह्यस्य मलयस्य च} %6-4-71

\twolineshloka
{चम्पकांस्तिलकांश्चूतानशोकान् सिन्दुवारकान्}
{तिनिशान् करवीरांश्च भञ्जन्ति स्म प्लवङ्गमाः} %6-4-72

\twolineshloka
{अङ्कोलांश्च करञ्जांश्च प्लक्षन्यग्रोधपादपान्}
{जम्बूकामलकान् नीपान् भञ्जन्ति स्म प्लवङ्गमाः} %6-4-73

\twolineshloka
{प्रस्तरेषु च रम्येषु विविधाः काननद्रुमाः}
{वायुवेगप्रचलिताः पुष्पैरवकिरन्ति तान्} %6-4-74

\twolineshloka
{मारुतः सुखसंस्पर्शो वाति चन्दनशीतलः}
{षट्पदैरनुकूजद्भिर्वनेषु मधुगन्धिषु} %6-4-75

\twolineshloka
{अधिकं शैलराजस्तु धातुभिस्तु विभूषितः}
{धातुभ्यः प्रसृतो रेणुर्वायुवेगेन घट्टितः} %6-4-76

\twolineshloka
{सुमहद्वानरानीकं छादयामास सर्वतः}
{गिरिप्रस्थेषु रम्येषु सर्वतः सम्प्रपुष्पिताः} %6-4-77

\twolineshloka
{केतक्यः सिन्दुवाराश्च वासन्त्यश्च मनोरमाः}
{माधव्यो गन्धपूर्णाश्च कुन्दगुल्माश्च पुष्पिताः} %6-4-78

\twolineshloka
{चिरिबिल्वा मधूकाश्च वञ्जुला बकुलास्तथा}
{रञ्जकास्तिलकाश्चैव नागवृक्षाश्च पुष्पिताः} %6-4-79

\twolineshloka
{चूताः पाटलिकाश्चैव कोविदाराश्च पुष्पिताः}
{मुचुलिन्दार्जुनाश्चैव शिंशपाः कुटजास्तथा} %6-4-80

\twolineshloka
{हिन्तालास्तिनिशाश्चैव चूर्णका नीपकास्तथा}
{नीलाशोकाश्च सरला अङ्कोलाः पद्मकास्तथा} %6-4-81

\twolineshloka
{प्रीयमाणैः प्लवङ्गैस्तु सर्वे पर्याकुलीकृताः}
{वाप्यस्तस्मिन् गिरौ रम्याः पल्वलानि तथैव च} %6-4-82

\twolineshloka
{चक्रवाकानुचरिताः कारण्डवनिषेविताः}
{प्लवैः क्रौञ्चैश्च सङ्कीर्णा वराहमृगसेविताः} %6-4-83

\twolineshloka
{ऋक्षैस्तरक्षुभिः सिंहैः शार्दूलैश्च भयावहैः}
{व्यालैश्च बहुभिर्भीमैः सेव्यमानाः समन्ततः} %6-4-84

\twolineshloka
{पद्मैः सौगन्धिकैः फुल्लैः कुमुदैश्चोत्पलैस्तथा}
{वारिजैर्विविधैः पुष्पै रम्यास्तत्र जलाशयाः} %6-4-85

\twolineshloka
{तस्य सानुषु कूजन्ति नानाद्विजगणास्तथा}
{स्नात्वा पीत्वोदकान्यत्र जले क्रीडन्ति वानराः} %6-4-86

\twolineshloka
{अन्योन्यं प्लावयन्ति स्म शैलमारुह्य वानराः}
{फलान्यमृतगन्धीनि मूलानि कुसुमानि च} %6-4-87

\twolineshloka
{बभञ्जुर्वानरास्तत्र पादपानां मदोत्कटाः}
{द्रोणमात्रप्रमाणानि लम्बमानानि वानराः} %6-4-88

\twolineshloka
{ययुः पिबन्तः स्वस्थास्ते मधूनि मधुपिङ्गलाः}
{पादपानवभञ्जन्तो विकर्षन्तस्तथा लताः} %6-4-89

\twolineshloka
{विधमन्तो गिरिवरान् प्रययुः प्लवगर्षभाः}
{वृक्षेभ्योऽन्ये तु कपयो नदन्तो मधु दर्पिताः} %6-4-90

\threelineshloka
{अन्ये वृक्षान् प्रपद्यन्ते प्रपिबन्त्यपि चापरे}
{बभूव वसुधा तैस्तु सम्पूर्णा हरिपुङ्गवैः}
{यथा कमलकेदारैः पक्वैरिव वसुन्धरा} %6-4-91

\twolineshloka
{महेन्द्रमथ सम्प्राप्य रामो राजीवलोचनः}
{आरुरोह महाबाहुः शिखरं द्रुमभूषितम्} %6-4-92

\twolineshloka
{ततः शिखरमारुह्य रामो दशरथात्मजः}
{कूर्ममीनसमाकीर्णमपश्यत् सलिलाशयम्} %6-4-93

\twolineshloka
{ते सह्यं समतिक्रम्य मलयं च महागिरिम्}
{आसेदुरानुपूर्व्येण समुद्रं भीमनिःस्वनम्} %6-4-94

\twolineshloka
{अवरुह्य जगामाशु वेलावनमनुत्तमम्}
{रामो रमयतां श्रेष्ठः ससुग्रीवः सलक्ष्मणः} %6-4-95

\twolineshloka
{अथ धौतोपलतलां तोयौघैः सहसोत्थितैः}
{वेलामासाद्य विपुलां रामो वचनमब्रवीत्} %6-4-96

\twolineshloka
{एते वयमनुप्राप्ताः सुग्रीव वरुणालयम्}
{इहेदानीं विचिन्ता सा या नः पूर्वमुपस्थिता} %6-4-97

\twolineshloka
{अतः परमतीरोऽयं सागरः सरितां पतिः}
{न चायमनुपायेन शक्यस्तरितुमर्णवः} %6-4-98

\twolineshloka
{तदिहैव निवेशोऽस्तु मन्त्रः प्रस्तूयतामिह}
{यथेदं वानरबलं परं पारमवाप्नुयात्} %6-4-99

\twolineshloka
{इतीव स महाबाहुः सीताहरणकर्शितः}
{रामः सागरमासाद्य वासमाज्ञापयत् तदा} %6-4-100

\twolineshloka
{सर्वाः सेना निवेश्यन्तां वेलायां हरिपुङ्गव}
{सम्प्राप्तो मन्त्रकालो नः सागरस्येह लङ्घने} %6-4-101

\twolineshloka
{स्वां स्वां सेनां समुत्सृज्य मा च कश्चित् कुतो व्रजेत्}
{गच्छन्तु वानराः शूरा ज्ञेयं छन्नं भयं च नः} %6-4-102

\twolineshloka
{रामस्य वचनं श्रुत्वा सुग्रीवः सहलक्ष्मणः}
{सेनां निवेशयत् तीरे सागरस्य द्रुमायुते} %6-4-103

\twolineshloka
{विरराज समीपस्थं सागरस्य च तद् बलम्}
{मधुपाण्डुजलः श्रीमान् द्वितीय इव सागरः} %6-4-104

\twolineshloka
{वेलावनमुपागम्य ततस्ते हरिपुङ्गवाः}
{निविष्टाश्च परं पारं काङ्क्षमाणा महोदधेः} %6-4-105

\twolineshloka
{तेषां निविशमानानां सैन्यसन्नाहनिःस्वनः}
{अन्तर्धाय महानादमर्णवस्य प्रशुश्रुवे} %6-4-106

\twolineshloka
{सा वानराणां ध्वजिनी सुग्रीवेणाभिपालिता}
{त्रिधा निविष्टा महती रामस्यार्थपराभवत्} %6-4-107

\twolineshloka
{सा महार्णवमासाद्य हृष्टा वानरवाहिनी}
{वायुवेगसमाधूतं पश्यमाना महार्णवम्} %6-4-108

\twolineshloka
{दूरपारमसम्बाधं रक्षोगणनिषेवितम्}
{पश्यन्तो वरुणावासं निषेदुर्हरियूथपाः} %6-4-109

\twolineshloka
{चण्डनक्रग्राहघोरं क्षपादौ दिवसक्षये}
{हसन्तमिव फेनौघैर्नृत्यन्तमिव चोर्मिभिः} %6-4-110

\twolineshloka
{चन्द्रोदये समुद्भूतं प्रतिचन्द्रसमाकुलम्}
{चण्डानिलमहाग्राहैः कीर्णं तिमितिमिङ्गिलैः} %6-4-111

\twolineshloka
{दीप्तभोगैरिवाकीर्णं भुजङ्गैर्वरुणालयम्}
{अवगाढं महासत्त्वैर्नानाशैलसमाकुलम्} %6-4-112

\threelineshloka
{सुदुर्गं दुर्गमार्गं तमगाधमसुरालयम्}
{मकरैर्नागभोगैश्च विगाढा वातलोलिताः}
{उत्पेतुश्च निपेतुश्च प्रहृष्टा जलराशयः} %6-4-113

\twolineshloka
{अग्निचूर्णमिवाविद्धं भास्वराम्बुमहोरगम्}
{सुरारिनिलयं घोरं पातालविषयं सदा} %6-4-114

\twolineshloka
{सागरं चाम्बरप्रख्यमम्बरं सागरोपमम्}
{सागरं चाम्बरं चेति निर्विशेषमदृश्यत} %6-4-115

\twolineshloka
{सम्पृक्तं नभसाप्यम्भः सम्पृक्तं च नभोऽम्भसा}
{तादृग्रूपे स्म दृश्येते तारारत्नसमाकुले} %6-4-116

\twolineshloka
{समुत्पतितमेघस्य वीचिमालाकुलस्य च}
{विशेषो न द्वयोरासीत् सागरस्याम्बरस्य च} %6-4-117

\twolineshloka
{अन्योन्यैरहताः सक्ताः सस्वनुर्भीमनिःस्वनाः}
{ऊर्मयः सिन्धुराजस्य महाभेर्य इवाम्बरे} %6-4-118

\twolineshloka
{रत्नौघजलसन्नादं विषक्तमिव वायुना}
{उत्पतन्तमिव क्रुद्धं यादोगणसमाकुलम्} %6-4-119

\twolineshloka
{ददृशुस्ते महात्मानो वाताहतजलाशयम्}
{अनिलोद्धूतमाकाशे प्रवलान्तमिवोर्मिभिः} %6-4-120

\twolineshloka
{ततो विस्मयमापन्ना हरयो ददृशुः स्थिताः}
{भ्रान्तोर्मिजालसन्नादं प्रलोलमिव सागरम्} %6-4-121


॥इत्यार्षे श्रीमद्रामायणे वाल्मीकीये आदिकाव्ये युद्धकाण्डे रामाभिषेणनम् नाम चतुर्थः सर्गः ॥६-४॥
