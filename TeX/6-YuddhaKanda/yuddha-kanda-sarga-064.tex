\sect{चतुःषष्ठितमः सर्गः — सीताप्रलोभनोपायः}

\twolineshloka
{तदुक्तमतिकायस्य बलिनो बाहुशालिनः}
{कुम्भकर्णस्य वचनं श्रुत्वोवाच महोदरः} %6-64-1

\twolineshloka
{कुम्भकर्ण कुले जातो धृष्टः प्राकृतदर्शनः}
{अवलिप्तो न शक्नोषि कृत्यं सर्वत्र वेदितुम्} %6-64-2

\twolineshloka
{नहि राजा न जानीते कुम्भकर्ण नयानयौ}
{त्वं तु कैशोरकाद् धृष्टः केवलं वक्तुमिच्छसि} %6-64-3

\twolineshloka
{स्थानं वृद्धिं च हानिं च देशकालविधानवित्}
{आत्मनश्च परेषां च बुध्यते राक्षसर्षभः} %6-64-4

\twolineshloka
{यत् त्वशक्यं बलवता वक्तुं प्राकृतबुद्धिना}
{अनुपासितवृद्धेन कः कुर्यात् तादृशं बुधः} %6-64-5

\twolineshloka
{यांस्तु धर्मार्थकामांस्त्वं ब्रवीषि पृथगाश्रयान्}
{अवबोद्धुं स्वभावेन नहि लक्षणमस्ति तान्} %6-64-6

\twolineshloka
{कर्म चैव हि सर्वेषां कारणानां प्रयोजनम्}
{श्रेयः पापीयसां चात्र फलं भवति कर्मणाम्} %6-64-7

\twolineshloka
{निःश्रेयसफलावेव धर्मार्थावितरावपि}
{अधर्मानर्थयोः प्राप्तं फलं च प्रत्यवायिकम्} %6-64-8

\twolineshloka
{ऐहलौकिकपारक्यं कर्म पुम्भिर्निषेव्यते}
{कर्माण्यपि तु कल्याणि लभते काममास्थितः} %6-64-9

\twolineshloka
{तत्र क्लृप्तमिदं राज्ञा हृदि कार्यं मतं च नः}
{शत्रौ हि साहसं यत् तत् किमिवात्रापनीयते} %6-64-10

\twolineshloka
{एकस्यैवाभियाने तु हेतुर्यः प्राहृतस्त्वया}
{तत्राप्यनुपपन्नं ते वक्ष्यामि यदसाधु च} %6-64-11

\twolineshloka
{येन पूर्वं जनस्थाने बहवोऽतिबला हताः}
{राक्षसा राघवं तं त्वं कथमेको जयिष्यसि} %6-64-12

\twolineshloka
{ये पूर्वं निर्जितास्तेन जनस्थाने महौजसः}
{राक्षसांस्तान् पुरे सर्वान् भीतानद्य न पश्यसि} %6-64-13

\twolineshloka
{तं सिंहमिव सङ्क्रुद्धं रामं दशरथात्मजम्}
{सर्पं सुप्तमहो बुद्ध्वा प्रबोधयितुमिच्छसि} %6-64-14

\twolineshloka
{ज्वलन्तं तेजसा नित्यं क्रोधेन च दुरासदम्}
{कस्तं मृत्युमिवासह्यमासादयितुमर्हति} %6-64-15

\twolineshloka
{संशयस्थमिदं सर्वं शत्रोः प्रतिसमासने}
{एकस्य गमनं तात नहि मे रोचते भृशम्} %6-64-16

\twolineshloka
{हीनार्थस्तु समृद्धार्थं को रिपुं प्राकृतं यथा}
{निश्चितं जीवितत्यागे वशमानेतुमिच्छति} %6-64-17

\twolineshloka
{यस्य नास्ति मनुष्येषु सदृशो राक्षसोत्तम}
{कथमाशंससे योद्धुं तुल्येनेन्द्रविवस्वतोः} %6-64-18

\twolineshloka
{एवमुक्त्वा तु संरब्धं कुम्भकर्णं महोदरः}
{उवाच रक्षसां मध्ये रावणं लोकरावणम्} %6-64-19

\twolineshloka
{लब्ध्वा पुरस्ताद् वैदेहीं किमर्थं त्वं विलम्बसे}
{यदीच्छसि तदा सीता वशगा ते भविष्यति} %6-64-20

\twolineshloka
{दृष्टः कश्चिदुपायो मे सीतोपस्थानकारकः}
{रुचितश्चेत् स्वया बुद्ध्या राक्षसेन्द्र ततः शृणु} %6-64-21

\twolineshloka
{अहं द्विजिह्वः संह्रादी कुम्भकर्णो वितर्दनः}
{पञ्च रामवधायैते निर्यान्तीत्यवघोषय} %6-64-22

\twolineshloka
{ततो गत्वा वयं युद्धं दास्यामस्तस्य यत्नतः}
{जेष्यामो यदि ते शत्रून् नोपायैः कार्यमस्ति नः} %6-64-23

\twolineshloka
{अथ जीवति नः शत्रुर्वयं च कृतसंयुगाः}
{ततः समभिपत्स्यामो मनसा यत् समीक्षितम्} %6-64-24

\twolineshloka
{वयं युद्धादिहैष्यामो रुधिरेण समुक्षिताः}
{विदार्य स्वतनुं बाणै रामनामाङ्कितैः शरैः} %6-64-25

\twolineshloka
{भक्षितो राघवोऽस्माभिर्लक्ष्मणश्चेति वादिनः}
{ततः पादौ ग्रहीष्यामस्त्वं नः कामं प्रपूरय} %6-64-26

\twolineshloka
{ततोऽवघोषय पुरे गजस्कन्धेन पार्थिव}
{हतो रामः सह भ्रात्रा ससैन्य इति सर्वतः} %6-64-27

\twolineshloka
{प्रीतो नाम ततो भूत्वा भृत्यानां त्वमरिन्दम}
{भोगांश्च परिवारांश्च कामान् वसु च दापय} %6-64-28

\twolineshloka
{ततो माल्यानि वासांसि वीराणामनुलेपनम्}
{पेयं च बहु योधेभ्यः स्वयं च मुदितः पिब} %6-64-29

\twolineshloka
{ततोऽस्मिन् बहुलीभूते कौलीने सर्वतो गते}
{भक्षितः ससुहृद् रामो राक्षसैरिति विश्रुते} %6-64-30

\twolineshloka
{प्रविश्याश्वास्य चापि त्वं सीतां रहसि सान्त्वयन्}
{धनधान्यैश्च कामैश्च रत्नैश्चैनां प्रलोभय} %6-64-31

\twolineshloka
{अनयोपधया राजन् भूयः शोकानुबन्धया}
{अकामा त्वद्वशं सीता नष्टनाथा गमिष्यति} %6-64-32

\twolineshloka
{रमणीयं हि भर्तारं विनष्टमधिगम्य सा}
{नैराश्यात् स्त्रीलघुत्वाच्च त्वद्वशं प्रतिपत्स्यते} %6-64-33

\twolineshloka
{सा पुरा सुखसंवृद्धा सुखार्हा दुःखकर्शिता}
{त्वय्यधीनं सुखं ज्ञात्वा सर्वथैव गमिष्यति} %6-64-34

\twolineshloka
{एतत् सुनीतं मम दर्शनेन रामं हि दृष्ट्वैव भवेदनर्थः}
{इहैव ते सेत्स्यति मोत्सुको भूर्महानयुद्धेन सुखस्य लाभः} %6-64-35

\twolineshloka
{अनष्टसैन्यो ह्यनवाप्तसंशयो रिपुं त्वयुद्धेन जयञ्जनाधिपः}
{यशश्च पुण्यं च महान्महीपते श्रियं च कीर्तिं च चिरं समश्नुते} %6-64-36


॥इत्यार्षे श्रीमद्रामायणे वाल्मीकीये आदिकाव्ये युद्धकाण्डे सीताप्रलोभनोपायः नाम चतुःषष्ठितमः सर्गः ॥६-६४॥
