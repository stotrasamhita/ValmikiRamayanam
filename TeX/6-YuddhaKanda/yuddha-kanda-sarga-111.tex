\sect{एकादशाधिकशततमः सर्गः — पौलस्त्यवधः}

\twolineshloka
{अथ संस्मारयामास मातली राघवं तदा}
{अजानन्निव किं वीर त्वमेनमनुवर्तसे} %6-111-1

\twolineshloka
{विसृजास्मै वधाय त्वमस्त्रं पैतामहं प्रभो}
{विनाशकालः कथितो यः सुरैः सोऽद्य वर्तते} %6-111-2

\twolineshloka
{ततः संस्मारितो रामस्तेन वाक्येन मातलेः}
{जग्राह स शरं दीप्तं निःश्वसन्तमिवोरगम्} %6-111-3

\twolineshloka
{यं तस्मै प्रथमं प्रादादगस्त्यो भगवानृषिः}
{ब्रह्मदत्तं महद् बाणममोघं युधि वीर्यवान्} %6-111-4

\twolineshloka
{ब्रह्मणा निर्मितं पूर्वमिन्द्रार्थममितौजसा}
{दत्तं सुरपतेः पूर्वं त्रिलोकजयकाङ्क्षिणः} %6-111-5

\twolineshloka
{यस्य वाजेषु पवनः फले पावकभास्करौ}
{शरीरमाकाशमयं गौरवे मेरुमन्दरौ} %6-111-6

\twolineshloka
{जाज्वल्यमानं वपुषा सुपुङ्खं हेमभूषितम्}
{तेजसा सर्वभूतानां कृतं भास्करवर्चसम्} %6-111-7

\twolineshloka
{सधूममिव कालाग्निं दीप्तमाशीविषोपमम्}
{नरनागाश्ववृन्दानां भेदनं क्षिप्रकारिणम्} %6-111-8

\twolineshloka
{द्वाराणां परिघाणां च गिरीणां चापि भेदनम्}
{नानारुधिरदिग्धाङ्गं मेदोदिग्धं सुदारुणम्} %6-111-9

\twolineshloka
{वज्रसारं महानादं नानासमितिदारुणम्}
{सर्ववित्रासनं भीमं श्वसन्तमिव पन्नगम्} %6-111-10

\twolineshloka
{कङ्कगृध्रबकानां च गोमायुगणरक्षसाम्}
{नित्यभक्षप्रदं युद्धे यमरूपं भयावहम्} %6-111-11

\twolineshloka
{नन्दनं वानरेन्द्राणां रक्षसामवसादनम्}
{वाजितं विविधैर्वाजैश्चारुचित्रैर्गरुत्मतः} %6-111-12

\twolineshloka
{तमुत्तमेषुं लोकानामिक्ष्वाकुभयनाशनम्}
{द्विषतां कीर्तिहरणं प्रहर्षकरमात्मनः} %6-111-13

\twolineshloka
{अभिमन्त्र्य ततो रामस्तं महेषुं महाबलः}
{वेदप्रोक्तेन विधिना सन्दधे कार्मुके बली} %6-111-14

\twolineshloka
{तस्मिन् सन्धीयमाने तु राघवेण शरोत्तमे}
{सर्वभूतानि सन्त्रेसुश्चचाल च वसुन्धरा} %6-111-15

\twolineshloka
{स रावणाय सङ्क्रुद्धो भृशमायम्य कार्मुकम्}
{चिक्षेप परमायत्तः शरं मर्मविदारणम्} %6-111-16

\twolineshloka
{स वज्र इव दुर्धर्षो वज्रिबाहुविसर्जितः}
{कृतान्त इव चावार्यो न्यपतद् रावणोरसि} %6-111-17

\twolineshloka
{स विसृष्टो महावेगः शरीरान्तकरः परः}
{बिभेद हृदयं तस्य रावणस्य दुरात्मनः} %6-111-18

\twolineshloka
{रुधिराक्तः स वेगेन शरीरान्तकरः शरः}
{रावणस्य हरन् प्राणान् विवेश धरणीतलम्} %6-111-19

\twolineshloka
{स शरो रावणं हत्वा रुधिरार्द्रकृतच्छविः}
{कृतकर्मा निभृतवत् स तूणीं पुनराविशत्} %6-111-20

\twolineshloka
{तस्य हस्ताद्धतस्याशु कार्मुकं तत् ससायकम्}
{निपपात सह प्राणैर्भ्रश्यमानस्य जीवितात्} %6-111-21

\twolineshloka
{गतासुर्भीमवेगस्तु नैर्ऋतेन्द्रो महाद्युतिः}
{पपात स्यन्दनाद् भूमौ वृत्रो वज्रहतो यथा} %6-111-22

\twolineshloka
{तं दृष्ट्वा पतितं भूमौ हतशेषा निशाचराः}
{हतनाथा भयत्रस्ताः सर्वतः सम्प्रदुद्रुवुः} %6-111-23

\twolineshloka
{नर्दन्तश्चाभिपेतुस्तान् वानरा द्रुमयोधिनः}
{दशग्रीववधं दृष्ट्वा वानरा जितकाशिनः} %6-111-24

\twolineshloka
{अर्दिता वानरैर्हृष्टैर्लङ्कामभ्यपतन् भयात्}
{हताश्रयत्वात् करुणैर्बाष्पप्रस्रवणैर्मुखैः} %6-111-25

\twolineshloka
{ततो विनेदुः संहृष्टा वानरा जितकाशिनः}
{वदन्तो राघवजयं रावणस्य च तद्वधम्} %6-111-26

\twolineshloka
{अथान्तरिक्षे व्यनदत् सौम्यस्त्रिदशदुन्दुभिः}
{दिव्यगन्धवहस्तत्र मारुतः सुसुखो ववौ} %6-111-27

\twolineshloka
{निपपातान्तरिक्षाच्च पुष्पवृष्टिस्तदा भुवि}
{किरन्ती राघवरथं दुरावापा मनोहरा} %6-111-28

\twolineshloka
{राघवस्तवसंयुक्ता गगने च विशुश्रुवे}
{साधुसाध्विति वागग्र्या देवतानां महात्मनाम्} %6-111-29

\twolineshloka
{आविवेश महान् हर्षो देवानां चारणैः सह}
{रावणे निहते रौद्रे सर्वलोकभयङ्करे} %6-111-30

\twolineshloka
{ततः सकामं सुग्रीवमङ्गदं च विभीषणम्}
{चकार राघवः प्रीतो हत्वा राक्षसपुङ्गवम्} %6-111-31

\twolineshloka
{ततः प्रजग्मुः प्रशमं मरुद्गणा दिशः प्रसेदुर्विमलं नभोऽभवत्}
{मही चकम्पे न च मारुतो ववौ स्थिरप्रभश्चाप्यभवद् दिवाकरः} %6-111-32

\twolineshloka
{ततस्तु सुग्रीवविभीषणाङ्गदाः सुहृद्विशिष्टाः सहलक्ष्मणस्तदा}
{समेत्य हृष्टा विजयेन राघवं रणेऽभिरामं विधिनाभ्यपूजयन्} %6-111-33

\twolineshloka
{स तु निहतरिपुः स्थिरप्रतिज्ञः स्वजनबलाभिवृतो रणे बभूव}
{रघुकुलनृपनन्दनो महौजास्त्रिदशगणैरभिसंवृतो महेन्द्रः} %6-111-34


॥इत्यार्षे श्रीमद्रामायणे वाल्मीकीये आदिकाव्ये युद्धकाण्डे पौलस्त्यवधः नाम एकादशाधिकशततमः सर्गः ॥६-१११॥
