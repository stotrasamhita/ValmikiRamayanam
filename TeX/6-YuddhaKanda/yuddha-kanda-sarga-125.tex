\sect{पञ्चविंशत्यधिकशततमः सर्गः — पुष्पकोत्पतनम्}

\twolineshloka
{उपस्थितं तु तं कृत्वा पुष्पकं पुष्पभूषितम्}
{अविदूरे स्थितो राममित्युवाच विभीषणः} %6-125-1

\twolineshloka
{स तु बद्धाञ्जलिपुटो विनीतो राक्षसेश्वरः}
{अब्रवीत् त्वरयोपेतः किं करोमीति राघवम्} %6-125-2

\twolineshloka
{तमब्रवीन्महातेजा लक्ष्मणस्योपशृण्वतः}
{विमृश्य राघवो वाक्यमिदं स्नेहपुरस्कृतम्} %6-125-3

\twolineshloka
{कृतप्रयत्नकर्माणः सर्व एव वनौकसः}
{रत्नैरर्थैश्च विविधैः सम्पूज्यन्तां विभीषण} %6-125-4

\twolineshloka
{सहामीभिस्त्वया लङ्का निर्जिता राक्षसेश्वर}
{हृष्टैः प्राणभयं त्यक्त्वा सङ्ग्रामेष्वनिवर्तिभिः} %6-125-5

\twolineshloka
{त इमे कृतकर्माणः सर्व एव वनौकसः}
{धनरत्नप्रदानैश्च कर्मैषां सफलं कुरु} %6-125-6

\twolineshloka
{एवं सम्मानिताश्चैते नन्द्यमाना यथा त्वया}
{भविष्यन्ति कृतज्ञेन निर्वृता हरियूथपाः} %6-125-7

\twolineshloka
{त्यागिनं सङ्ग्रहीतारं सानुक्रोशं जितेन्द्रियम्}
{सर्वे त्वामभिगच्छन्ति ततः सम्बोधयामि ते} %6-125-8

\twolineshloka
{हीनं रतिगुणैः सर्वैरभिहन्तारमाहवे}
{सेना त्यजति संविग्ना नृपतिं तं नरेश्वर} %6-125-9

\twolineshloka
{एवमुक्तस्तु रामेण वानरांस्तान् विभीषणः}
{रत्नार्थसंविभागेन सर्वानेवाभ्यपूजयत्} %6-125-10

\twolineshloka
{ततस्तान् पूजितान् दृष्ट्वा रत्नार्थैर्हरियूथपान्}
{आरुरोह तदा रामस्तद् विमानमनुत्तमम्} %6-125-11

\twolineshloka
{अङ्केनादाय वैदेहीं लज्जमानां मनस्विनीम्}
{लक्ष्मणेन सह भ्रात्रा विक्रान्तेन धनुष्मता} %6-125-12

\twolineshloka
{अब्रवीत् स विमानस्थः पूजयन् सर्ववानरान्}
{सुग्रीवं च महावीर्यं काकुत्स्थः सविभीषणम्} %6-125-13

\twolineshloka
{मित्रकार्यं कृतमिदं भवद्भिर्वानरर्षभाः}
{अनुज्ञाता मया सर्वे यथेष्टं प्रतिगच्छत} %6-125-14

\twolineshloka
{यत् तु कार्यं वयस्येन स्निग्धेन च हितेन च}
{कृतं सुग्रीव तत् सर्वं भवताधर्मभीरुणा} %6-125-15

\threelineshloka
{किष्किन्धां प्रति याह्याशु स्वसैन्येनाभिसंवृतः}
{स्वराज्ये वस लङ्कायां मया दत्ते विभीषण}
{न त्वां धर्षयितुं शक्ताः सेन्द्रा अपि दिवौकसः} %6-125-16

\twolineshloka
{अयोध्यां प्रति यास्यामि राजधानीं पितुर्मम}
{अभ्यनुज्ञातुमिच्छामि सर्वानामन्त्रयामि वः} %6-125-17

\twolineshloka
{एवमुक्तास्तु रामेण हरीन्द्रा हरयस्तथा}
{ऊचुः प्राञ्जलयः सर्वे राक्षसश्च विभीषणः} %6-125-18

\twolineshloka
{अयोध्यां गन्तुमिच्छामः सर्वान् नयतु नो भवान्}
{मुद्युक्ता विचरिष्यामो वनान्युपवनानि च} %6-125-19

\twolineshloka
{दृष्ट्वा त्वामभिषेकार्द्रं कौसल्यामभिवाद्य च}
{अचिरादागमिष्यामः स्वगृहान् नृपसत्तम} %6-125-20

\twolineshloka
{एवमुक्तस्तु धर्मात्मा वानरैः सविभीषणैः}
{अब्रवीद् वानरान् रामः ससुग्रीवविभीषणान्} %6-125-21

\twolineshloka
{प्रियात् प्रियतरं लब्धं यदहं ससुहृज्जनः}
{सर्वैर्भवद्भिः सहितः प्रीतिं लप्स्ये पुरीं गतः} %6-125-22

\twolineshloka
{क्षिप्रमारोह सुग्रीव विमानं सह वानरैः}
{त्वमप्यारोह सामात्यो राक्षसेन्द्र विभीषण} %6-125-23

\twolineshloka
{ततः स पुष्पकं दिव्यं सुग्रीवः सह वानरैः}
{आरुरोह मुदा युक्तः सामात्यश्च विभीषणः} %6-125-24

\twolineshloka
{तेष्वारूढेषु सर्वेषु कौबेरं परमासनम्}
{राघवेणाभ्यनुज्ञातमुत्पपात विहायसम्} %6-125-25

\twolineshloka
{खगतेन विमानेन हंसयुक्तेन भास्वता}
{प्रहृष्टश्च प्रतीतश्च बभौ रामः कुबेरवत्} %6-125-26

\twolineshloka
{ते सर्वे वानरर्क्षाश्च राक्षसाश्च महाबलाः}
{यथासुखमसम्बाधं दिव्ये तस्मिन्नुपाविशन्} %6-125-27


॥इत्यार्षे श्रीमद्रामायणे वाल्मीकीये आदिकाव्ये युद्धकाण्डे पुष्पकोत्पतनम् नाम पञ्चविंशत्यधिकशततमः सर्गः ॥६-१२५॥
