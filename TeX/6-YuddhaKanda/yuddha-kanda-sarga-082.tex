\sect{द्व्यशीतितमः सर्गः — हनूमदादिनिर्वेदः}

\twolineshloka
{श्रुत्वा तु भीमनिर्ह्रादं शक्राशनिसमस्वनम्}
{वीक्ष्यमाणा दिशः सर्वा दुद्रुवुर्वानरा भृशम्} %6-82-1

\twolineshloka
{तानुवाच ततः सर्वान् हनूमान् मारुतात्मजः}
{विषण्णवदनान् दीनांस्त्रस्तान् विद्रवतः पृथक्} %6-82-2

\twolineshloka
{कस्माद् विषण्णवदना विद्रवध्वं प्लवङ्गमाः}
{त्यक्तयुद्धसमुत्साहाः शूरत्वं क्व नु वो गतम्} %6-82-3

\twolineshloka
{पृष्ठतोऽनुव्रजध्वं मामग्रतो यान्तमाहवे}
{शूरैरभिजनोपेतैरयुक्तं हि निवर्तितुम्} %6-82-4

\twolineshloka
{एवमुक्ताः सुसङ्क्रुद्धा वायुपुत्रेण धीमता}
{शैलशृङ्गान् द्रुमांश्चैव जगृहुर्हृष्टमानसाः} %6-82-5

\twolineshloka
{अभिपेतुश्च गर्जन्तो राक्षसान् वानरर्षभाः}
{परिवार्य हनूमन्तमन्वयुश्च महाहवे} %6-82-6

\twolineshloka
{स तैर्वानरमुख्यैस्तु हनूमान् सर्वतो वृतः}
{हुताशन इवार्चिष्मानदहच्छत्रुवाहिनीम्} %6-82-7

\twolineshloka
{स राक्षसानां कदनं चकार सुमहाकपिः}
{वृतो वानरसैन्येन कालान्तकयमोपमः} %6-82-8

\twolineshloka
{स तु शोकेन चाविष्टः कोपेन महता कपिः}
{हनूमान् रावणिरथे महतीं पातयच्छिलाम्} %6-82-9

\twolineshloka
{तामापतन्तीं दृष्ट्वैव रथः सारथिना तदा}
{विधेयाश्वसमायुक्तः विदूरमपवाहितः} %6-82-10

\twolineshloka
{तमिन्द्रजितमप्राप्य रथस्थं सहसारथिम्}
{विवेश धरणीं भित्त्वा सा शिला व्यर्थमुद्यता} %6-82-11

\twolineshloka
{पतितायां शिलायां तु व्यथिता रक्षसां चमूः}
{निपतन्त्या च शिलया राक्षसा मथिता भृशम्} %6-82-12

\twolineshloka
{तमभ्यधावन् शतशो नदन्तः काननौकसः}
{ते द्रुमांश्च महाकाया गिरिशृङ्गाणि चोद्यताः} %6-82-13

\twolineshloka
{क्षिपन्तीन्द्रजितं सङ्ख्ये वानरा भीमविक्रमाः}
{वृक्षशैलमहावर्षं विसृजन्तः प्लवङ्गमाः} %6-82-14

\twolineshloka
{शत्रूणां कदनं चक्रुर्नेदुश्च विविधैः स्वनैः}
{वानरैस्तैर्महाभीमैर्घोररूपा निशाचराः} %6-82-15

\twolineshloka
{वीर्यादभिहता वृक्षैर्व्यचेष्टन्त रणक्षितौ}
{स सैन्यमभिवीक्ष्याथ वानरार्दितमिन्द्रजित्} %6-82-16

\twolineshloka
{प्रगृहीतायुधः क्रुद्धः परानभिमुखो ययौ}
{स शरौघानवसृजन् स्वसैन्येनाभिसंवृतः} %6-82-17

\twolineshloka
{जघान कपिशार्दूलान् सुबहून् दृढविक्रमः}
{शूलैरशनिभिः खड्गैः पट्टिशैः शूलमुद्गरैः} %6-82-18

\twolineshloka
{ते चाप्यनुचरांस्तस्य वानरा जघ्नुराहवे}
{सुस्कन्धविटपैः शैलैः शिलाभिश्च महाबलः} %6-82-19

\twolineshloka
{हनूमान् कदनं चक्रे रक्षसां भीमकर्मणाम्}
{सन्निवार्य परानीकमब्रवीत् तान् वनौकसः} %6-82-20

\twolineshloka
{हनूमान् सन्निवर्तध्वं न नः साध्यमिदं बलम्}
{त्यक्त्वा प्राणान् विचेष्टन्तो रामप्रियचिकीर्षवः} %6-82-21

\twolineshloka
{यन्निमित्तं हि युध्यामो हता सा जनकात्मजा}
{इममर्थं हि विज्ञाप्य रामं सुग्रीवमेव च} %6-82-22

\twolineshloka
{तौ यत् प्रतिविधास्येते तत् करिष्यामहे वयम्}
{इत्युक्त्वा वानरश्रेष्ठो वारयन् सर्ववानरान्} %6-82-23

\twolineshloka
{शनैः शनैरसन्त्रस्तः सबलः सन्न्यवर्तत}
{ततः प्रेक्ष्य हनूमन्तं व्रजन्तं यत्र राघवः} %6-82-24

\twolineshloka
{स होतुकामो दुष्टात्मा गतश्चैत्यं निकुम्भिलाम्}
{निकुम्भिलामधिष्ठाय पावकं जुहवेन्द्रजित्} %6-82-25

\twolineshloka
{यज्ञभूम्यां ततो गत्वा पावकस्तेन रक्षसा}
{हूयमानः प्रजज्वाल होमशोणितभुक् तदा} %6-82-26

\twolineshloka
{सार्चिःपिनद्धो ददृशे होमशोणिततर्पितः}
{सन्ध्यागत इवादित्यः सुतीव्रोऽग्निः समुत्थितः} %6-82-27

\twolineshloka
{अथेन्द्रजिद् राक्षसभूतये तु जुहाव हव्यं विधिना विधानवित्}
{दृष्ट्वा व्यतिष्ठन्त च राक्षसास्ते महासमूहेषु नयानयज्ञाः} %6-82-28


॥इत्यार्षे श्रीमद्रामायणे वाल्मीकीये आदिकाव्ये युद्धकाण्डे हनूमदादिनिर्वेदः नाम द्व्यशीतितमः सर्गः ॥६-८२॥
