\sect{पञ्चदशः सर्गः — इन्द्रजिद्विभीषणविवादः}

\twolineshloka
{बृहस्पतेस्तुल्यमतेर्वचस्तन्निशम्य यत्नेन विभीषणस्य}
{ततो महात्मा वचनं बभाषे तत्रेन्द्रजिन्नैर्ऋतयूथमुख्यः} %6-15-1

\twolineshloka
{किं नाम ते तात कनिष्ठ वाक्यमनर्थकं वै बहुभीतवच्च}
{अस्मिन् कुले योऽपि भवेन्न जातः सोऽपीदृशं नैव वदेन्न कुर्यात्} %6-15-2

\twolineshloka
{सत्त्वेन वीर्येण पराक्रमेण धैर्येण शौर्येण च तेजसा च}
{एकः कुलेऽस्मिन् पुरुषो विमुक्तो विभीषणस्तात कनिष्ठ एषः} %6-15-3

\twolineshloka
{किं नाम तौ मानुषराजपुत्रावस्माकमेकेन हि राक्षसेन}
{सुप्राकृतेनापि निहन्तुमेतौ शक्यौ कुतो भीषयसे स्म भीरो} %6-15-4

\twolineshloka
{त्रिलोकनाथो ननु देवराजः शक्रो मया भूमितले निविष्टः}
{भयार्पिताश्चापि दिशः प्रपन्नाः सर्वे तदा देवगणाः समग्राः} %6-15-5

\twolineshloka
{ऐरावतो निःस्वनमुन्नदन् स निपातितो भूमितले मया तु}
{विकृष्य दन्तौ तु मया प्रसह्य वित्रासिता देवगणाः समग्राः} %6-15-6

\twolineshloka
{सोऽहं सुराणामपि दर्पहन्ता दैत्योत्तमानामपि शोककर्ता}
{कथं नरेन्द्रात्मजयोर्न शक्तो मनुष्ययोः प्राकृतयोः सुवीर्यः} %6-15-7

\twolineshloka
{अथेन्द्रकल्पस्य दुरासदस्य महौजसस्तद् वचनं निशम्य}
{ततो महार्थं वचनं बभाषे विभीषणः शस्त्रभृतां वरिष्ठः} %6-15-8

\twolineshloka
{न तात मन्त्रे तव निश्चयोऽस्ति बालस्त्वमद्याप्यविपक्वबुद्धिः}
{तस्मात् त्वयाप्यात्मविनाशनाय वचोऽर्थहीनं बहु विप्रलप्तम्} %6-15-9

\twolineshloka
{पुत्रप्रवादेन तु रावणस्य त्वमिन्द्रजिन्मित्रमुखोऽसि शत्रुः}
{यस्येदृशं राघवतो विनाशं निशम्य मोहादनुमन्यसे त्वम्} %6-15-10

\twolineshloka
{त्वमेव वध्यश्च सुदुर्मतिश्च स चापि वध्यो य इहानयत् त्वाम्}
{बालं दृढं साहसिकं च योऽद्य प्रावेशयन्मन्त्रकृतां समीपम्} %6-15-11

\twolineshloka
{मूढोऽप्रगल्भोऽविनयोपपन्नस्तीक्ष्णस्वभावोऽल्पमतिर्दुरात्मा}
{मूर्खस्त्वमत्यन्तसुदुर्मतिश्च त्वमिन्द्रजिद् बालतया ब्रवीषि} %6-15-12

\twolineshloka
{को ब्रह्मदण्डप्रतिमप्रकाशानर्चिष्मतः कालनिकाशरूपान्}
{सहेत बाणान् यमदण्डकल्पान् समक्षमुक्तान् युधि राघवेण} %6-15-13

\twolineshloka
{धनानि रत्नानि सुभूषणानि वासांसि दिव्यानि मणींश्च चित्रान्}
{सीतां च रामाय निवेद्य देवीं वसेम राजन्निह वीतशोकाः} %6-15-14


॥इत्यार्षे श्रीमद्रामायणे वाल्मीकीये आदिकाव्ये युद्धकाण्डे इन्द्रजिद्विभीषणविवादः नाम पञ्चदशः सर्गः ॥६-१५॥
