\sect{षण्णवतितमः सर्गः — रावणाभिषेणनम्}

\twolineshloka
{आर्तानां राक्षसीनां तु लङ्कायां वै कुले कुले}
{रावणः करुणं शब्दं शुश्राव परिदेवितम्} %6-96-1

\twolineshloka
{स तु दीर्घं विनिःश्वस्य मुहूर्तं ध्यानमास्थितः}
{बभूव परमक्रुद्धो रावणो भीमदर्शनः} %6-96-2

\twolineshloka
{सन्दश्य दशनैरोष्ठं क्रोधसंरक्तलोचनः}
{राक्षसैरपि दुर्दर्शः कालाग्निरिव मूर्तिमान्} %6-96-3

\twolineshloka
{उवाच च समीपस्थान् राक्षसान् राक्षसेश्वरः}
{क्रोधाव्यक्तकथस्तत्र निर्दहन्निव चक्षुषा} %6-96-4

\twolineshloka
{महोदरं महापार्श्वं विरूपाक्षं च राक्षसम्}
{शीघ्रं वदत सैन्यानि निर्यातेति ममाज्ञया} %6-96-5

\twolineshloka
{तस्य तद् वचनं श्रुत्वा राक्षसास्ते भयार्दिताः}
{चोदयामासुरव्यग्रान् राक्षसांस्तान् नृपाज्ञया} %6-96-6

\twolineshloka
{ते तु सर्वे तथेत्युक्त्वा राक्षसा भीमदर्शनाः}
{कृतस्वस्त्ययनाः सर्वे ते रणाभिमुखा ययुः} %6-96-7

\twolineshloka
{प्रतिपूज्य यथान्यायं रावणं ते महारथाः}
{तस्थुः प्राञ्जलयः सर्वे भर्तुर्विजयकाङ्क्षिणः} %6-96-8

\twolineshloka
{ततोवाच प्रहस्यैतान् रावणः क्रोधमूर्च्छितः}
{महोदरमहापार्श्वौ विरूपाक्षं च राक्षसम्} %6-96-9

\twolineshloka
{अद्य बाणैर्धनुर्मुक्तैर्युगान्तादित्यसन्निभैः}
{राघवं लक्ष्मणं चैव नेष्यामि यमसादनम्} %6-96-10

\twolineshloka
{खरस्य कुम्भकर्णस्य प्रहस्तेन्द्रजितोस्तथा}
{करिष्यामि प्रतीकारमद्य शत्रुवधादहम्} %6-96-11

\twolineshloka
{नैवान्तरिक्षं न दिशो न च द्यौर्नापि सागराः}
{प्रकाशत्वं गमिष्यन्ति मद्बाणजलदावृताः} %6-96-12

\twolineshloka
{अद्य वानरमुख्यानां तानि यूथानि भागशः}
{धनुषा शरजालेन वधिष्यामि पतत्त्रिणा} %6-96-13

\twolineshloka
{अद्य वानरसैन्यानि रथेन पवनौजसा}
{धनुःसमुद्रादुद्भूतैर्मथिष्यामि शरोर्मिभिः} %6-96-14

\twolineshloka
{व्याकोशपद्मवक्त्राणि पद्मकेसरवर्चसाम्}
{अद्य यूथतटाकानि गजवत् प्रमथाम्यहम्} %6-96-15

\twolineshloka
{सशरैरद्य वदनैः सङ्ख्ये वानरयूथपाः}
{मण्डयिष्यन्ति वसुधां सनालैरिव पङ्कजैः} %6-96-16

\twolineshloka
{अद्य यूथप्रचण्डानां हरीणां द्रुमयोधिनाम्}
{मुक्तेनैकेषुणा युद्धे भेत्स्यामि च शतं शतम्} %6-96-17

\twolineshloka
{हतो भ्राता च येषां वै येषां च तनयो हतः}
{वधेनाद्य रिपोस्तेषां करोम्यश्रुप्रमार्जनम्} %6-96-18

\twolineshloka
{अद्य मद्बाणनिर्भिन्नैः प्रस्तीर्णैर्गतचेतनैः}
{करोमि वानरैर्युद्धे यत्नावेक्ष्यतलां महीम्} %6-96-19

\twolineshloka
{अद्य काकाश्च गृध्राश्च ये च मांसाशिनोऽपरे}
{सर्वांस्तांस्तर्पयिष्यामि शत्रुमांसैः शराहतैः} %6-96-20

\twolineshloka
{कल्प्यतां मे रथः शीघ्रं क्षिप्रमानीयतां धनुः}
{अनुप्रयान्तु मां युद्धे येऽत्र शिष्टा निशाचराः} %6-96-21

\twolineshloka
{तस्य तद् वचनं श्रुत्वा महापार्श्वोऽब्रवीद् वचः}
{बलाध्यक्षान् स्थितांस्तत्र बलं सन्त्वर्यतामिति} %6-96-22

\twolineshloka
{बलाध्यक्षास्तु संयुक्ता राक्षसांस्तान् गृहे गृहे}
{चोदयन्तः परिययुर्लङ्कां लघुपराक्रमाः} %6-96-23

\twolineshloka
{ततो मुहूर्तान्निष्पेतू राक्षसा भीमदर्शनाः}
{नदन्तो भीमवदना नानाप्रहरणैर्भुजैः} %6-96-24

\twolineshloka
{असिभिः पट्टिशैः शूलैर्गदाभिर्मुसलैर्हलैः}
{शक्तिभिस्तीक्ष्णधाराभिर्महद्भिः कूटमुद्गरैः} %6-96-25

\twolineshloka
{यष्टिभिर्विविधैश्चक्रैर्निशितैश्च परश्वधैः}
{भिन्दिपालैः शतघ्नीभिरन्यैश्चापि वरायुधैः} %6-96-26

\twolineshloka
{अथानयन् बलाध्यक्षाश्चत्वारो रावणाज्ञया}
{रथानां नियुतं साग्रं नागानां नियुतत्रयम्} %6-96-27

\twolineshloka
{अश्वानां षष्टिकोट्यस्तु खरोष्ट्राणां तथैव च}
{पदातयस्त्वसङ्ख्याता जग्मुस्ते राजशासनात्} %6-96-28

\twolineshloka
{बलाध्यक्षाश्च संस्थाप्य राज्ञः सेनां पुरःस्थिताम्}
{एतस्मिन्नन्तरे सूतः स्थापयामास तं रथम्} %6-96-29

\twolineshloka
{दिव्यास्त्रवरसम्पन्नं नानालङ्कारभूषितम्}
{नानायुधसमाकीर्णं किङ्किणीजालसंयुतम्} %6-96-30

\twolineshloka
{नानारत्नपरिक्षिप्तं रत्नस्तम्भैर्विराजितम्}
{जाम्बूनदमयैश्चैव सहस्रकलशैर्वृतम्} %6-96-31

\twolineshloka
{तं दृष्ट्वा राक्षसाः सर्वे विस्मयं परमं गताः}
{तं दृष्ट्वा सहसोत्थाय रावणो राक्षसेश्वरः} %6-96-32

\threelineshloka
{कोटिसूर्यप्रतीकाशं ज्वलन्तमिव पावकम्}
{द्रुतं सूतसमायुक्तं युक्ताष्टतुरगं रथम्}
{आरुरोह तदा भीमं दीप्यमानं स्वतेजसा} %6-96-33

\twolineshloka
{ततः प्रयातः सहसा राक्षसैर्बहुभिर्वृतः}
{रावणः सत्त्वगाम्भीर्याद् दारयन्निव मेदिनीम्} %6-96-34

\twolineshloka
{ततश्चासीन्महानादस्तूर्याणां च ततस्ततः}
{मृदङ्गैः पटहैः शङ्खैः कलहैः सह रक्षसाम्} %6-96-35

\threelineshloka
{आगतो रक्षसां राजा छत्रचामरसंयुतः}
{सीतापहारी दुर्वृत्तो ब्रह्मघ्नो देवकण्टकः}
{योद्धुं रघुवरेणेति शुश्रुवे कलहध्वनिः} %6-96-36

\twolineshloka
{तेन नादेन महता पृथिवी समकम्पत}
{तं शब्दं सहसा श्रुत्वा वानरा दुद्रुवुर्भयात्} %6-96-37

\twolineshloka
{रावणस्तु महाबाहुः सचिवैः परिवारितः}
{आजगाम महातेजा जयाय विजयं प्रति} %6-96-38

\twolineshloka
{रावणेनाभ्यनुज्ञातौ महापार्श्वमहोदरौ}
{विरूपाक्षश्च दुर्धर्षो रथानारुरुहुस्तदा} %6-96-39

\twolineshloka
{ते तु हृष्टाभिनर्दन्तो भिन्दन्त इव मेदिनीम्}
{नादं घोरं विमुञ्चन्तो निर्ययुर्जयकाङ्क्षिणः} %6-96-40

\twolineshloka
{ततो युद्धाय तेजस्वी रक्षोगणबलैर्वृतः}
{निर्ययावुद्यतधनुः कालान्तकयमोपमः} %6-96-41

\twolineshloka
{ततः प्रजविताश्वेन रथेन स महारथः}
{द्वारेण निर्ययौ तेन यत्र तौ रामलक्ष्मणौ} %6-96-42

\twolineshloka
{ततो नष्टप्रभः सूर्यो दिशश्च तिमिरावृताः}
{द्विजाश्च नेदुर्घोराश्च सञ्चचाल च मेदिनी} %6-96-43

\twolineshloka
{ववर्ष रुधिरं देवश्चस्खलुश्च तुरङ्गमाः}
{ध्वजाग्रे न्यपतद् गृध्रो विनेदुश्चाशिवं शिवाः} %6-96-44

\twolineshloka
{नयनं चास्फुरद् वामं वामो बाहुरकम्पत}
{विवर्णवदनश्चासीत् किञ्चिदभ्रश्यत स्वनः} %6-96-45

\twolineshloka
{ततो निष्पततो युद्धे दशग्रीवस्य रक्षसः}
{रणे निधनशंसीनि रूपाण्येतानि जज्ञिरे} %6-96-46

\twolineshloka
{अन्तरिक्षात् पपातोल्का निर्घातसमनिःस्वना}
{विनेदुरशिवा गृध्रा वायसैरभिमिश्रिताः} %6-96-47

\twolineshloka
{एतानचिन्तयन् घोरानुत्पातान् समवस्थितान्}
{निर्ययौ रावणो मोहाद् वधार्थं कालचोदितः} %6-96-48

\twolineshloka
{तेषां तु रथघोषेण राक्षसानां महात्मनाम्}
{वानराणामपि चमूर्युद्धायैवाभ्यवर्तत} %6-96-49

\twolineshloka
{तेषां तु तुमुलं युद्धं बभूव कपिरक्षसाम्}
{अन्योन्यमाह्वयानानां क्रुद्धानां जयमिच्छताम्} %6-96-50

\twolineshloka
{ततः क्रुद्धो दशग्रीवः शरैः काञ्चनभूषणैः}
{वानराणामनीकेषु चकार कदनं महत्} %6-96-51

\twolineshloka
{निकृत्तशिरसः केचिद् रावणेन वलीमुखाः}
{केचिद् विच्छिन्नहृदयाः केचिच्छ्रोत्रविवर्जिताः} %6-96-52

\twolineshloka
{निरुच्छ्वासा हताः केचित् केचित् पार्श्वेषु दारिताः}
{केचिद् विभिन्नशिरसः केचिच्चक्षुर्विनाकृताः} %6-96-53

\twolineshloka
{दशाननः क्रोधविवृत्तनेत्रो यतो यतोऽभ्येति रथेन सङ्ख्ये}
{ततस्ततस्तस्य शरप्रवेगं सोढुं न शेकुर्हरियूथपास्ते} %6-96-54


॥इत्यार्षे श्रीमद्रामायणे वाल्मीकीये आदिकाव्ये युद्धकाण्डे रावणाभिषेणनम् नाम षण्णवतितमः सर्गः ॥६-९६॥
