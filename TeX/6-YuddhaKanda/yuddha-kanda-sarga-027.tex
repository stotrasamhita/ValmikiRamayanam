\sect{सप्तविंशः सर्गः — हरादिवानरपराक्रमाख्यानम्}

\twolineshloka
{तांस्तु ते सम्प्रवक्ष्यामि प्रेक्षमाणस्य यूथपान्}
{राघवार्थे पराक्रान्ता ये न रक्षन्ति जीवितम्} %6-27-1

\twolineshloka
{स्निग्धा यस्य बहुव्यामा दीर्घलाङ्गूलमाश्रिताः}
{ताम्राः पीताः सिताः श्वेताः प्रकीर्णा घोरकर्मणः} %6-27-2

\twolineshloka
{प्रगृहीताः प्रकाशन्ते सूर्यस्येव मरीचयः}
{पृथिव्यां चानुकृष्यन्ते हरो नामैष वानरः} %6-27-3

\twolineshloka
{यं पृष्ठतोऽनुगच्छन्ति शतशोऽथ सहस्रशः}
{वृक्षानुद्यम्य सहसा लङ्कारोहणतत्पराः} %6-27-4

\twolineshloka
{यूथपा हरिराजस्य किंकराः समुपस्थिताः}
{नीलानिव महामेघांस्तिष्ठतो यांस्तु पश्यसि} %6-27-5

\twolineshloka
{असिताञ्जनसंकाशान् युद्धे सत्यपराक्रमान्}
{असंख्येयाननिर्देशान् परं पारमिवोदधेः} %6-27-6

\twolineshloka
{पर्वतेषु च ये केचिद् विषयेषु नदीषु च}
{एते त्वामभिवर्तन्ते राजन्नृक्षाः सुदारुणाः} %6-27-7

\twolineshloka
{एषां मध्ये स्थितो राजन् भीमाक्षो भीमदर्शनः}
{पर्जन्य इव जीमूतैः समन्तात् परिवारितः} %6-27-8

\twolineshloka
{ऋक्षवन्तं गिरिश्रेष्ठमध्यास्ते नर्मदां पिबन्}
{सर्वर्क्षाणामधिपतिर्धूम्रो नामैष यूथपः} %6-27-9

\twolineshloka
{यवीयानस्य तु भ्राता पश्यैनं पर्वतोपमम्}
{भ्रात्रा समानो रूपेण विशिष्टस्तु पराक्रमे} %6-27-10

\twolineshloka
{स एष जाम्बवान् नाम महायूथपयूथपः}
{प्रशान्तो गुरुवर्ती च सम्प्रहारेष्वमर्षणः} %6-27-11

\twolineshloka
{एतेन साह्यं तु महत् कृतं शक्रस्य धीमता}
{दैवासुरे जाम्बवता लब्धाश्च बहवो वराः} %6-27-12

\twolineshloka
{आरुह्य पर्वताग्रेभ्यो महाभ्रविपुलाः शिलाः}
{मुञ्चन्ति विपुलाकारा न मृत्योरुद्विजन्ति च} %6-27-13

\twolineshloka
{राक्षसानां च सदृशाः पिशाचानां च रोमशाः}
{एतस्य सैन्या बहवो विचरन्त्यमितौजसः} %6-27-14

\twolineshloka
{य एनमभिसंरब्धं प्लवमानमवस्थितम्}
{प्रेक्षन्ते वानराः सर्वे स्थिता यूथपयूथपम्} %6-27-15

\twolineshloka
{एष राजन् सहस्राक्षं पर्युपास्ते हरीश्वरः}
{बलेन बलसंयुक्तो दम्भो नामैष यूथपः} %6-27-16

\twolineshloka
{यः स्थितं योजने शैलं गच्छन् पार्श्वेन सेवते}
{ऊर्ध्वं तथैव कायेन गतः प्राप्नोति योजनम्} %6-27-17

\twolineshloka
{यस्मात् तु परमं रूपं चतुष्पात्सु न विद्यते}
{श्रुतः संनादनो नाम वानराणां पितामहः} %6-27-18

\twolineshloka
{येन युद्धं तदा दत्तं रणे शक्रस्य धीमता}
{पराजयश्च न प्राप्तः सोऽयं यूथपयूथपः} %6-27-19

\twolineshloka
{यस्य विक्रममाणस्य शक्रस्येव पराक्रमः}
{एष गन्धर्वकन्यायामुत्पन्नः कृष्णवर्त्मना} %6-27-20

\twolineshloka
{तदा देवासुरे युद्धे साह्यार्थं त्रिदिवौकसाम्}
{यत्र वैश्रवणो राजा जम्बूमुपनिषेवते} %6-27-21

\twolineshloka
{यो राजा पर्वतेन्द्राणां बहुकिंनरसेविनाम्}
{विहारसुखदो नित्यं भ्रातुस्ते राक्षसाधिप} %6-27-22

\twolineshloka
{तत्रैष रमते श्रीमान् बलवान् वानरोत्तमः}
{युद्धेष्वकत्थनो नित्यं क्रथनो नाम यूथपः} %6-27-23

\twolineshloka
{वृतः कोटिसहस्रेण हरीणां समवस्थितः}
{एषैवाशंसते लङ्कां स्वेनानीकेन मर्दितुम्} %6-27-24

\twolineshloka
{यो गङ्गामनुपर्येति त्रासयन् गजयूथपान्}
{हस्तिनां वानराणां च पूर्ववैरमनुस्मरन्} %6-27-25

\twolineshloka
{एष यूथपतिर्नेता गर्जन् गिरिगुहाशयः}
{गजान् रोधयते वन्यानारुजंश्च महीरुहान्} %6-27-26

\twolineshloka
{हरीणां वाहिनीमुख्यो नदीं हैमवतीमनु}
{उशीरबीजमाश्रित्य मन्दरं पर्वतोत्तमम्} %6-27-27

\twolineshloka
{रमते वानरश्रेष्ठो दिवि शक्र इव स्वयम्}
{एनं शतसहस्राणां सहस्रमभिवर्तते} %6-27-28

\twolineshloka
{वीर्यविक्रमदृप्तानां नर्दतां बाहुशालिनाम्}
{स एष नेता चैतेषां वानराणां महात्मनाम्} %6-27-29

\twolineshloka
{स एष दुर्धरो राजन् प्रमाथी नाम यूथपः}
{वातेनेवोद्धतं मेघं यमेनमनुपश्यसि} %6-27-30

\twolineshloka
{अनीकमपि संरब्धं वानराणां तरस्विनाम्}
{उद्धूतमरुणाभासं पवनेन समन्ततः} %6-27-31

\twolineshloka
{विवर्तमानं बहुशो यत्रैतद्बहुलं रजः}
{एतेऽसितमुखा घोरा गोलाङ्गूला महाबलाः} %6-27-32

\twolineshloka
{शतं शतसहस्राणि दृष्ट्वा वै सेतुबन्धनम्}
{गोलाङ्गूलं महाराज गवाक्षं नाम यूथपम्} %6-27-33

\twolineshloka
{परिवार्याभिनर्दन्ते लङ्कां मर्दितुमोजसा}
{भ्रमराचरिता यत्र सर्वकालफलद्रुमाः} %6-27-34

\twolineshloka
{यं सूर्यस्तुल्यवर्णाभमनुपर्येति पर्वतम्}
{यस्य भासा सदा भान्ति तद्वर्णा मृगपक्षिणः} %6-27-35

\twolineshloka
{यस्य प्रस्थं महात्मानो न त्यजन्ति महर्षयः}
{सर्वकामफला वृक्षाः सदा फलसमन्विताः} %6-27-36

\twolineshloka
{मधूनि च महार्हाणि यस्मिन् पर्वतसत्तमे}
{तत्रैष रमते राजन् रम्ये काञ्चनपर्वते} %6-27-37

\twolineshloka
{मुख्यो वानरमुख्यानां केसरी नाम यूथपः}
{षष्टिर्गिरिसहस्राणि रम्याः काञ्चनपर्वताः} %6-27-38

\twolineshloka
{तेषां मध्ये गिरिवरस्त्वमिवानघ रक्षसाम्}
{तत्रैके कपिलाः श्वेतास्ताम्रास्या मधुपिङ्गलाः} %6-27-39

\twolineshloka
{निवसन्त्यन्तिमगिरौ तीक्ष्णदंष्ट्रा नखायुधाः}
{सिंहा इव चतुर्दंष्ट्रा व्याघ्रा इव दुरासदाः} %6-27-40

\twolineshloka
{सर्वे वैश्वानरसमा ज्वलदाशीविषोपमाः}
{सुदीर्घाञ्चितलाङ्गूला मत्तमातङ्गसंनिभाः} %6-27-41

\twolineshloka
{महापर्वतसंकाशा महाजीमूतनिःस्वनाः}
{वृत्तपिङ्गलनेत्रा हि महाभीमगतिस्वनाः} %6-27-42

\twolineshloka
{मर्दयन्तीव ते सर्वे तस्थुर्लङ्कां समीक्ष्य ते}
{एष चैषामधिपतिर्मध्ये तिष्ठति वीर्यवान्} %6-27-43

\twolineshloka
{जयार्थी नित्यमादित्यमुपतिष्ठति वीर्यवान्}
{नाम्ना पृथिव्यां विख्यातो राजन् शतबलीति यः} %6-27-44

\twolineshloka
{एषैवाशंसते लङ्कां स्वेनानीकेन मर्दितुम्}
{विक्रान्तो बलवान् शूरः पौरुषे स्वे व्यवस्थितः} %6-27-45

\twolineshloka
{रामप्रियार्थं प्राणानां दयां न कुरुते हरिः}
{गजो गवाक्षो गवयो नलो नीलश्च वानरः} %6-27-46

\threelineshloka
{एकैकमेव योधानां कोटिभिर्दशभिर्वृतः}
{तथान्ये वानरश्रेष्ठा विन्ध्यपर्वतवासिनः}
{न शक्यन्ते बहुत्वात् तु संख्यातुं लघुविक्रमाः} %6-27-47

\twolineshloka
{सर्वे महाराज महाप्रभावाः सर्वे महाशैलनिकाशकायाः}
{सर्वे समर्थाः पृथिवीं क्षणेन कर्तुं प्रविध्वस्तविकीर्णशैलाम्} %6-27-48


॥इत्यार्षे श्रीमद्रामायणे वाल्मीकीये आदिकाव्ये युद्धकाण्डे हरादिवानरपराक्रमाख्यानम् नाम सप्तविंशः सर्गः ॥६-२७॥
