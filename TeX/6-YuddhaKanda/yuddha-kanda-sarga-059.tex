\sect{एकोनषष्ठितमः सर्गः — रावणाभिषेणनम्}

\twolineshloka
{तस्मिन् हते राक्षससैन्यपाले प्लवङ्गमानामृषभेण युद्धे}
{भीमायुधं सागरवेगतुल्यं विदुद्रुवे राक्षसराजसैन्यम्} %6-59-1

\twolineshloka
{गत्वा तु रक्षोधिपतेः शशंसुः सेनापतिं पावकसूनुशस्तम्}
{तच्चापि तेषां वचनं निशम्य रक्षोधिपः क्रोधवशं जगाम} %6-59-2

\twolineshloka
{सङ्ख्ये प्रहस्तं निहतं निशम्य क्रोधार्दितः शोकपरीतचेताः}
{उवाच तान् राक्षसयूथमुख्यानिन्द्रो यथा निर्जरयूथमुख्यान्} %6-59-3

\twolineshloka
{नावज्ञा रिपवे कार्या यैरिन्द्रबलसादनः}
{सूदितः सैन्यपालो मे सानुयात्रः सकुञ्जरः} %6-59-4

\twolineshloka
{सोऽहं रिपुविनाशाय विजयायाविचारयन्}
{स्वयमेव गमिष्यामि रणशीर्षं तदद्भुतम्} %6-59-5

\threelineshloka
{अद्य तद् वानरानीकं रामं च सहलक्ष्मणम्}
{निर्दहिष्यामि बाणौघैर्वनं दीप्तैरिवाग्निभिः}
{अद्य सन्तर्पयिष्यामि पृथिवीं कपिशोणितैः} %6-59-6

\twolineshloka
{स एवमुक्त्वा ज्वलनप्रकाशं रथं तुरङ्गोत्तमराजियुक्तम्}
{प्रकाशमानं वपुषा ज्वलन्तं समारुरोहामरराजशत्रुः} %6-59-7

\twolineshloka
{स शङ्खभेरीपणवप्रणादैरास्फोटितक्ष्वेडितसिंहनादैः}
{पुण्यैः स्तवैश्चापि सुपूज्यमानस्तदा ययौ राक्षसराजमुख्यः} %6-59-8

\twolineshloka
{स शैलजीमूतनिकाशरूपैर्मांसाशनैः पावकदीप्तनेत्रैः}
{बभौ वृतो राक्षसराजमुख्यो भूतैर्वृतो रुद्र इवामरेशः} %6-59-9

\twolineshloka
{ततो नगर्याः सहसा महौजा निष्क्रम्य तद् वानरसैन्यमुग्रम्}
{महार्णवाभ्रस्तनितं ददर्श समुद्यतं पादपशैलहस्तम्} %6-59-10

\twolineshloka
{तद् राक्षसानीकमतिप्रचण्डमालोक्य रामो भुजगेन्द्रबाहुः}
{विभीषणं शस्त्रभृतां वरिष्ठमुवाच सेनानुगतः पृथुश्रीः} %6-59-11

\twolineshloka
{नानापताकाध्वजछत्रजुष्टं प्रासासिशूलायुधशस्त्रजुष्टम्}
{कस्येदमक्षोभ्यमभीरुजुष्टं सैन्यं महेन्द्रोपमनागजुष्टम्} %6-59-12

\twolineshloka
{ततस्तु रामस्य निशम्य वाक्यं विभीषणः शक्रसमानवीर्यः}
{शशंस रामस्य बलप्रवेकं महात्मनां राक्षसपुङ्गवानाम्} %6-59-13

\twolineshloka
{योऽसौ गजस्कन्धगतो महात्मा नवोदितार्कोपमताम्रवक्त्रः}
{सङ्कम्पयन्नागशिरोऽभ्युपैति ह्यकम्पनं त्वेनमवेहि राजन्} %6-59-14

\twolineshloka
{योऽसौ रथस्थो मृगराजकेतुर्धुन्वन् धनुः शक्रधनुःप्रकाशम्}
{करीव भात्युग्रविवृत्तदंष्ट्रः स इन्द्रजिन्नाम वरप्रधानः} %6-59-15

\twolineshloka
{यश्चैष विन्ध्यास्तमहेन्द्रकल्पो धन्वी रथस्थोऽतिरथोऽतिवीरः}
{विस्फारयंश्चापमतुल्यमानं नाम्नातिकायोऽतिविवृद्धकायः} %6-59-16

\twolineshloka
{योऽसौ नवार्कोदितताम्रचक्षुरारुह्य घण्टानिनदप्रणादम्}
{गजं खरं गर्जति वै महात्मा महोदरो नाम स एष वीरः} %6-59-17

\twolineshloka
{योऽसौ हयं काञ्चनचित्रभाण्डमारुह्य सन्ध्याभ्रगिरिप्रकाशम्}
{प्रासं समुद्यम्य मरीचिनद्धं पिशाच एषोऽशनितुल्यवेगः} %6-59-18

\twolineshloka
{यश्चैष शूलं निशितं प्रगृह्य विद्युत्प्रभं किङ्करवज्रवेगम्}
{वृषेन्द्रमास्थाय शशिप्रकाशमायाति योऽसौ त्रिशिरा यशस्वी} %6-59-19

\twolineshloka
{असौ च जीमूतनिकाशरूपः कुम्भः पृथुव्यूढसुजातवक्षाः}
{समाहितः पन्नगराजकेतुर्विस्फारयन् याति धनुर्विधुन्वन्} %6-59-20

\twolineshloka
{यश्चैष जाम्बूनदवज्रजुष्टं दीप्तं सधूमं परिघं प्रगृह्य}
{आयाति रक्षोबलकेतुभूतो योऽसौ निकुम्भोऽद्भुतघोरकर्मा} %6-59-21

\twolineshloka
{यश्चैष चापासिशरौघजुष्टं पताकिनं पावकदीप्तरूपम्}
{रथं समास्थाय विभात्युदग्रो नरान्तकोऽसौ नगशृङ्गयोधी} %6-59-22

\twolineshloka
{यश्चैष नानाविधघोररूपैर्व्याघ्रोष्ट्रनागेन्द्रमृगाश्ववक्त्रैः}
{भूतैर्वृतो भाति विवृत्तनेत्रैर्योऽसौ सुराणामपि दर्पहन्ता} %6-59-23

\twolineshloka
{यत्रैतदिन्दुप्रतिमं विभाति छत्रं सितं सूक्ष्मशलाकमग्र्यम्}
{अत्रैष रक्षोधिपतिर्महात्मा भूतैर्वृतो रुद्र इवावभाति} %6-59-24

\twolineshloka
{असौ किरीटी चलकुण्डलास्यो नगेन्द्रविन्ध्योपमभीमकायः}
{महेन्द्रवैवस्वतदर्पहन्ता रक्षोधिपः सूर्य इवावभाति} %6-59-25

\twolineshloka
{प्रत्युवाच ततो रामो विभीषणमरिन्दमः}
{अहो दीप्तमहातेजा रावणो राक्षसेश्वरः} %6-59-26

\twolineshloka
{आदित्य इव दुष्प्रेक्ष्यो रश्मिभिर्भाति रावणः}
{न व्यक्तं लक्षये ह्यस्य रूपं तेजःसमावृतम्} %6-59-27

\twolineshloka
{देवदानववीराणां वपुर्नैवंविधं भवेत्}
{यादृशं राक्षसेन्द्रस्य वपुरेतद् विराजते} %6-59-28

\twolineshloka
{सर्वे पर्वतसङ्काशाः सर्वे पर्वतयोधिनः}
{सर्वे दीप्तायुधधरा योधास्तस्य महात्मनः} %6-59-29

\twolineshloka
{विभाति रक्षोराजोऽसौ प्रदीप्तैर्भीमदर्शनैः}
{भूतैः परिवृतस्तीक्ष्णैर्देहवद्भिरिवान्तकः} %6-59-30

\twolineshloka
{दिष्ट्यायमद्य पापात्मा मम दृष्टिपथं गतः}
{अद्य क्रोधं विमोक्ष्यामि सीताहरणसम्भवम्} %6-59-31

\twolineshloka
{एवमुक्त्वा ततो रामो धनुरादाय वीर्यवान्}
{लक्ष्मणानुचरस्तस्थौ समुद्धृत्य शरोत्तमम्} %6-59-32

\twolineshloka
{ततः स रक्षोधिपतिर्महात्मा रक्षांसि तान्याह महाबलानि}
{द्वारेषु चर्यागृहगोपुरेषु सुनिर्वृतास्तिष्ठत निर्विशङ्काः} %6-59-33

\twolineshloka
{इहागतं मां सहितं भवद्भिर्वनौकसश्छिद्रमिदं विदित्वा}
{शून्यां पुरीं दुष्प्रसहां प्रमथ्य प्रधर्षयेयुः सहसा समेताः} %6-59-34

\twolineshloka
{विसर्जयित्वा सचिवांस्ततस्तान् गतेषु रक्षःसु यथानियोगम्}
{व्यदारयद् वानरसागरौघं महाझषः पूर्णमिवार्णवौघम्} %6-59-35

\twolineshloka
{तमापतन्तं सहसा समीक्ष्य दीप्तेषुचापं युधि राक्षसेन्द्रम्}
{महत् समुत्पाट्य महीधराग्रं दुद्राव रक्षोधिपतिं हरीशः} %6-59-36

\twolineshloka
{तच्छैलशृङ्गं बहुवृक्षसानुं प्रगृह्य चिक्षेप निशाचराय}
{तमापतन्तं सहसा समीक्ष्य चिच्छेद बाणैस्तपनीयपुङ्खैः} %6-59-37

\twolineshloka
{तस्मिन् प्रवृद्धोत्तमसानुवृक्षे शृङ्गे विदीर्णे पतिते पृथिव्याम्}
{महाहिकल्पं शरमन्तकाभं समादधे राक्षसलोकनाथः} %6-59-38

\twolineshloka
{स तं गृहीत्वानिलतुल्यवेगं सविस्फुलिङ्गज्वलनप्रकाशम्}
{बाणं महेन्द्राशनितुल्यवेगं चिक्षेप सुग्रीववधाय रुष्टः} %6-59-39

\twolineshloka
{स सायको रावणबाहुमुक्तः शक्राशनिप्रख्यवपुःप्रकाशम्}
{सुग्रीवमासाद्य बिभेद वेगाद् गुहेरिता क्रौञ्चमिवोग्रशक्तिः} %6-59-40

\twolineshloka
{स सायकार्तो विपरीतचेताः कूजन् पृथिव्यां निपपात वीरः}
{तं वीक्ष्य भूमौ पतितं विसंज्ञं नेदुः प्रहृष्टा युधि यातुधानाः} %6-59-41

\twolineshloka
{ततो गवाक्षो गवयः सुषेणस्त्वथर्षभो ज्योतिमुखो नलश्च}
{शैलान् समुत्पाट्य विवृद्धकायाः प्रदुद्रुवुस्तं प्रति राक्षसेन्द्रम्} %6-59-42

\twolineshloka
{तेषां प्रहारान् स चकार मोघान् रक्षोधिपो बाणशतैः शिताग्रैः}
{तान् वानरेन्द्रानपि बाणजालैर्बिभेद जाम्बूनदचित्रपुङ्खैः} %6-59-43

\twolineshloka
{ते वानरेन्द्रास्त्रिदशारिबाणैर्भिन्ना निपेतुर्भुवि भीमकायाः}
{ततस्तु तद् वानरसैन्यमुग्रं प्रच्छादयामास स बाणजालैः} %6-59-44

\twolineshloka
{ते वध्यमानाः पतिताश्च वीरा नानद्यमाना भयशल्यविद्धाः}
{शाखामृगा रावणसायकार्ता जग्मुः शरण्यं शरणं स्म रामम्} %6-59-45

\twolineshloka
{ततो महात्मा स धनुर्धनुष्मानादाय रामः सहसा जगाम}
{तं लक्ष्मणः प्राञ्जलिरभ्युपेत्य उवाच रामं परमार्थयुक्तम्} %6-59-46

\twolineshloka
{काममार्य सुपर्याप्तो वधायास्य दुरात्मनः}
{विधमिष्याम्यहं चैतमनुजानीहि मां विभो} %6-59-47

\twolineshloka
{तमब्रवीन्महातेजा रामः सत्यपराक्रमः}
{गच्छ यत्नपरश्चापि भव लक्ष्मण संयुगे} %6-59-48

\twolineshloka
{रावणो हि महावीर्यो रणेऽद्भुतपराक्रमः}
{त्रैलोक्येनापि सङ्क्रुद्धो दुष्प्रसह्यो न संशयः} %6-59-49

\twolineshloka
{तस्यच्छिद्राणि मार्गस्व स्वच्छिद्राणि च लक्षय}
{चक्षुषा धनुषाऽऽत्मानं गोपायस्व समाहितः} %6-59-50

\twolineshloka
{राघवस्य वचः श्रुत्वा सम्परिष्वज्य पूज्य च}
{अभिवाद्य च रामाय ययौ सौमित्रिराहवे} %6-59-51

\twolineshloka
{स रावणं वारणहस्तबाहुं ददर्श भीमोद्यतदीप्तचापम्}
{प्रच्छादयन्तं शरवृष्टिजालैस्तान् वानरान् भिन्नविकीर्णदेहान्} %6-59-52

\twolineshloka
{तमालोक्य महातेजा हनूमान् मारुतात्मजः}
{निवार्य शरजालानि विदुद्राव स रावणम्} %6-59-53

\twolineshloka
{रथं तस्य समासाद्य बाहुमुद्यम्य दक्षिणम्}
{त्रासयन् रावणं धीमान् हनूमान् वाक्यमब्रवीत्} %6-59-54

\twolineshloka
{देवदानवगन्धर्वैर्यक्षैश्च सह राक्षसैः}
{अवध्यत्वं त्वया प्राप्तं वानरेभ्यस्तु ते भयम्} %6-59-55

\twolineshloka
{एष मे दक्षिणो बाहुः पञ्चशाखः समुद्यतः}
{विधमिष्यति ते देहे भूतात्मानं चिरोषितम्} %6-59-56

\twolineshloka
{श्रुत्वा हनूमतो वाक्यं रावणो भीमविक्रमः}
{संरक्तनयनः क्रोधादिदं वचनमब्रवीत्} %6-59-57

\twolineshloka
{क्षिप्रं प्रहर निःशङ्कं स्थिरां कीर्तिमवाप्नुहि}
{ततस्त्वां ज्ञातविक्रान्तं नाशयिष्यामि वानर} %6-59-58

\twolineshloka
{रावणस्य वचः श्रुत्वा वायुसूनुर्वचोऽब्रवीत्}
{प्रहतं हि मया पूर्वमक्षं तव सुतं स्मर} %6-59-59

\twolineshloka
{एवमुक्तो महातेजा रावणो राक्षसेश्वरः}
{आजघानानिलसुतं तलेनोरसि वीर्यवान्} %6-59-60

\twolineshloka
{स तलाभिहतस्तेन चचाल च मुहुर्मुहुः}
{स्थितो मुहूर्तं तेजस्वी स्थैर्यं कृत्वा महामतिः} %6-59-61

\twolineshloka
{आजघान च सङ्क्रुद्धस्तलेनैवामरद्विषम्}
{ततः स तेनाभिहतो वानरेण महात्मना} %6-59-62

\twolineshloka
{दशग्रीवः समाधूतो यथा भूमितलेऽचलः}
{सङ्ग्रामे तं तथा दृष्ट्वा रावणं तलताडितम्} %6-59-63

\twolineshloka
{ऋषयो वानराः सिद्धा नेदुर्देवाः सहासुरैः}
{अथाश्वस्य महातेजा रावणो वाक्यमब्रवीत्} %6-59-64

\twolineshloka
{साधु वानर वीर्येण श्लाघनीयोऽसि मे रिपुः}
{रावणेनैवमुक्तस्तु मारुतिर्वाक्यमब्रवीत्} %6-59-65

\twolineshloka
{धिगस्तु मम वीर्यस्य यत् त्वं जीवसि रावण}
{सकृत् तु प्रहरेदानीं दुर्बुद्धे किं विकत्थसे} %6-59-66

\twolineshloka
{ततस्त्वां मामको मुष्टिर्नयिष्यति यमक्षयम्}
{ततो मारुतिवाक्येन कोपस्तस्य प्रजज्वले} %6-59-67

\twolineshloka
{संरक्तनयनो यत्नान्मुष्टिमावृत्य दक्षिणम्}
{पातयामास वेगेन वानरोरसि वीर्यवान्} %6-59-68

\twolineshloka
{हनूमान् वक्षसि व्यूढे सञ्चचाल पुनः पुनः}
{विह्वलं तु तदा दृष्ट्वा हनूमन्तं महाबलम्} %6-59-69

\twolineshloka
{रथेनातिरथः शीघ्रं नीलं प्रति समभ्यगात्}
{राक्षसानामधिपतिर्दशग्रीवः प्रतापवान्} %6-59-70

\twolineshloka
{पन्नगप्रतिमैर्भीमैः परमर्माभिभेदनैः}
{शरैरादीपयामास नीलं हरिचमूपतिम्} %6-59-71

\twolineshloka
{स शरौघसमायस्तो नीलो हरिचमूपतिः}
{करेणैकेन शैलाग्रं रक्षोधिपतयेऽसृजत्} %6-59-72

\twolineshloka
{हनूमानपि तेजस्वी समाश्वस्तो महामनाः}
{विप्रेक्षमाणो युद्धेप्सुः सरोषमिदमब्रवीत्} %6-59-73

\twolineshloka
{नीलेन सह संयुक्तं रावणं राक्षसेश्वरम्}
{अन्येन युध्यमानस्य न युक्तमभिधावनम्} %6-59-74

\twolineshloka
{रावणोऽथ महातेजास्तं शृङ्गं सप्तभिः शरैः}
{आजघान सुतीक्ष्णाग्रैस्तद् विकीर्णं पपात ह} %6-59-75

\twolineshloka
{तद् विकीर्णं गिरेः शृङ्गं दृष्ट्वा हरिचमूपतिः}
{कालाग्निरिव जज्वाल कोपेन परवीरहा} %6-59-76

\twolineshloka
{सोऽश्वकर्णद्रुमान् शालांश्चूतांश्चापि सुपुष्पितान्}
{अन्यांश्च विविधान् वृक्षान् नीलश्चिक्षेप संयुगे} %6-59-77

\twolineshloka
{स तान् वृक्षान् समासाद्य प्रतिचिच्छेद रावणः}
{अभ्यवर्षच्च घोरेण शरवर्षेण पावकिम्} %6-59-78

\twolineshloka
{अभिवृष्टः शरौघेण मेघेनेव महाचलः}
{ह्रस्वं कृत्वा ततो रूपं ध्वजाग्रे निपपात ह} %6-59-79

\twolineshloka
{पावकात्मजमालोक्य ध्वजाग्रे समवस्थितम्}
{जज्वाल रावणः क्रोधात् ततो नीलो ननाद च} %6-59-80

\twolineshloka
{ध्वजाग्रे धनुषश्चाग्रे किरीटाग्रे च तं हरिम्}
{लक्ष्मणोऽथ हनूमांश्च रामश्चापि सुविस्मिताः} %6-59-81

\twolineshloka
{रावणोऽपि महातेजाः कपिलाघवविस्मितः}
{अस्त्रमाहारयामास दीप्तमाग्नेयमद्भुतम्} %6-59-82

\twolineshloka
{ततस्ते चुक्रुशुर्हृष्टा लब्धलक्षाः प्लवङ्गमाः}
{नीललाघवसम्भ्रान्तं दृष्ट्वा रावणमाहवे} %6-59-83

\twolineshloka
{वानराणां च नादेन संरब्धो रावणस्तदा}
{सम्भ्रमाविष्टहृदयो न किञ्चित् प्रत्यपद्यत} %6-59-84

\twolineshloka
{आग्नेयेनापि संयुक्तं गृहीत्वा रावणः शरम्}
{ध्वजशीर्षस्थितं नीलमुदैक्षत निशाचरः} %6-59-85

\twolineshloka
{ततोऽब्रवीन्महातेजा रावणो राक्षसेश्वरः}
{कपे लाघवयुक्तोऽसि मायया परया सह} %6-59-86

\twolineshloka
{जीवितं खलु रक्षस्व यदि शक्तोऽसि वानर}
{तानि तान्यात्मरूपाणि सृजसि त्वमनेकशः} %6-59-87

\twolineshloka
{तथापि त्वां मया मुक्तः सायकोऽस्त्रप्रयोजितः}
{जीवितं परिरक्षन्तं जीविताद् भ्रंशयिष्यति} %6-59-88

\twolineshloka
{एवमुक्त्वा महाबाहू रावणो राक्षसेश्वरः}
{सन्धाय बाणमस्त्रेण चमूपतिमताडयत्} %6-59-89

\twolineshloka
{सोऽस्त्रमुक्तेन बाणेन नीलो वक्षसि ताडितः}
{निर्दह्यमानः सहसा स पपात महीतले} %6-59-90

\twolineshloka
{पितृमाहात्म्यसंयोगादात्मनश्चापि तेजसा}
{जानुभ्यामपतद् भूमौ न तु प्राणैर्वियुज्यत} %6-59-91

\twolineshloka
{विसंज्ञं वानरं दृष्ट्वा दशग्रीवो रणोत्सुकः}
{रथेनाम्बुदनादेन सौमित्रिमभिदुद्रुवे} %6-59-92

\twolineshloka
{आसाद्य रणमध्ये तं वारयित्वा स्थितो ज्वलन्}
{धनुर्विस्फारयामास राक्षसेन्द्रः प्रतापवान्} %6-59-93

\twolineshloka
{तमाह सौमित्रिरदीनसत्त्वो विस्फारयन्तं धनुरप्रमेयम्}
{अवेहि मामद्य निशाचरेन्द्र न वानरांस्त्वं प्रतियोद्धुमर्हसि} %6-59-94

\twolineshloka
{स तस्य वाक्यं प्रतिपूर्णघोषं ज्याशब्दमुग्रं च निशम्य राजा}
{आसाद्य सौमित्रिमुपस्थितं तं रोषान्वितं वाचमुवाच रक्षः} %6-59-95

\twolineshloka
{दिष्ट्यासि मे राघव दृष्टिमार्गं प्राप्तोऽन्तगामी विपरीतबुद्धिः}
{अस्मिन् क्षणे यास्यसि मृत्युलोकं संसाद्यमानो मम बाणजालैः} %6-59-96

\twolineshloka
{तमाह सौमित्रिरविस्मयानो गर्जन्तमुद्वृत्तशिताग्रदंष्ट्रम्}
{राजन् न गर्जन्ति महाप्रभावा विकत्थसे पापकृतां वरिष्ठ} %6-59-97

\twolineshloka
{जानामि वीर्यं तव राक्षसेन्द्र बलं प्रतापं च पराक्रमं च}
{अवस्थितोऽहं शरचापपाणिरागच्छ किं मोघविकत्थनेन} %6-59-98

\twolineshloka
{स एवमुक्तः कुपितः ससर्ज रक्षोधिपः सप्त शरान् सुपुङ्खान्}
{ताँल्लक्ष्मणः काञ्चनचित्रपुङ्खैश्चिच्छेद बाणैर्निशिताग्रधारैः} %6-59-99

\twolineshloka
{तान् प्रेक्षमाणः सहसा निकृत्तान् निकृत्तभोगानिव पन्नगेन्द्रान्}
{लङ्केश्वरः क्रोधवशं जगाम ससर्ज चान्यान् निशितान् पृषत्कान्} %6-59-100

\twolineshloka
{स बाणवर्षं तु ववर्ष तीव्रं रामानुजः कार्मुकसम्प्रयुक्तम्}
{क्षुरार्धचन्द्रोत्तमकर्णिभल्लैः शरांश्च चिच्छेद न चुक्षुभे च} %6-59-101

\twolineshloka
{स बाणजालान्यपि तानि तानि मोघानि पश्यंस्त्रिदशारिराजः}
{विसिस्मिये लक्ष्मणलाघवेन पुनश्च बाणान् निशितान् मुमोच} %6-59-102

\twolineshloka
{स लक्ष्मणश्चापि शिताञ् शिताग्रान् महेन्द्रतुल्योऽशनिभीमवेगान्}
{सन्धाय चापे ज्वलनप्रकाशान् ससर्ज रक्षोधिपतेर्वधाय} %6-59-103

\twolineshloka
{स तान् प्रचिच्छेद हि राक्षसेन्द्रः शिताञ् शराल्ँ लक्ष्मणमाजघान}
{शरेण कालाग्निसमप्रभेण स्वयम्भुदत्तेन ललाटदेशे} %6-59-104

\twolineshloka
{स लक्ष्मणो रावणसायकार्तश्चचाल चापं शिथिलं प्रगृह्य}
{पुनश्च संज्ञां प्रतिलभ्य कृच्छ्राच्चिच्छेद चापं त्रिदशेन्द्रशत्रोः} %6-59-105

\twolineshloka
{निकृत्तचापं त्रिभिराजघान बाणैस्तदा दाशरथिः शिताग्रैः}
{स सायकार्तो विचचाल राजा कृच्छ्राच्च संज्ञां पुनराससाद} %6-59-106

\twolineshloka
{स कृत्तचापः शरताडितश्च मेदार्द्रगात्रो रुधिरावसिक्तः}
{जग्राह शक्तिं स्वयमुग्रशक्तिः स्वयम्भुदत्तां युधि देवशत्रुः} %6-59-107

\twolineshloka
{स तां सधूमानलसन्निकाशां वित्रासनां संयति वानराणाम्}
{चिक्षेप शक्तिं तरसा ज्वलन्तीं सौमित्रये राक्षसराष्ट्रनाथः} %6-59-108

\twolineshloka
{तामापतन्तीं भरतानुजोऽस्त्रैर्जघान बाणैश्च हुताग्निकल्पैः}
{तथापि सा तस्य विवेश शक्तिर्भुजान्तरं दाशरथेर्विशालम्} %6-59-109

\twolineshloka
{स शक्तिमाञ् शक्तिसमाहतः सन् जज्वाल भूमौ स रघुप्रवीरः}
{तं विह्वलन्तं सहसाभ्युपेत्य जग्राह राजा तरसा भुजाभ्याम्} %6-59-110

\twolineshloka
{हिमवान् मन्दरो मेरुस्त्रैलोक्यं वा सहामरैः}
{शक्यं भुजाभ्यामुद्धर्तुं न शक्यो भरतानुजः} %6-59-111

\twolineshloka
{शक्त्या ब्राह्म्या तु सौमित्रिस्ताडितोऽपि स्तनान्तरे}
{विष्णोरमीमांस्यभागमात्मानं प्रत्यनुस्मरत्} %6-59-112

\twolineshloka
{ततो दानवदर्पघ्नं सौमित्रिं देवकण्टकः}
{तं पीडयित्वा बाहुभ्यां न प्रभुर्लङ्घनेऽभवत्} %6-59-113

\twolineshloka
{ततः क्रुद्धो वायुसुतो रावणं समभिद्रवत्}
{आजघानोरसि क्रुद्धो वज्रकल्पेन मुष्टिना} %6-59-114

\twolineshloka
{तेन मुष्टिप्रहारेण रावणो राक्षसेश्वरः}
{जानुभ्यामगमद् भूमौ चचाल च पपात च} %6-59-115

\twolineshloka
{आस्यैश्च नेत्रैः श्रवणैः पपात रुधिरं बहु}
{विघूर्णमानो निश्चेष्टो रथोपस्थ उपाविशत्} %6-59-116

\twolineshloka
{विसंज्ञो मूर्च्छितश्चासीन्न च स्थानं समालभत्}
{विसंज्ञं रावणं दृष्ट्वा समरे भीमविक्रमम्} %6-59-117

\twolineshloka
{ऋषयो वानराश्चैव नेदुर्देवाश्च सासुराः}
{हनूमानथ तेजस्वी लक्ष्मणं रावणार्दितम्} %6-59-118

\threelineshloka
{आनयद् राघवाभ्याशं बाहुभ्यां परिगृह्य तम्}
{वायुसूनोः सुहृत्त्वेन भक्त्या परमया च सः}
{शत्रूणामप्यकम्प्योऽपि लघुत्वमगमत् कपेः} %6-59-119

\twolineshloka
{तं समुत्सृज्य सा शक्तिः सौमित्रिं युधि निर्जितम्}
{रावणस्य रथे तस्मिन् स्थानं पुनरुपागमत्} %6-59-120

\twolineshloka
{रावणोऽपि महातेजाः प्राप्य संज्ञां महाहवे}
{आददे निशितान् बाणाञ्जग्राह च महद्धनुः} %6-59-121

\twolineshloka
{आश्वस्तश्च विशल्यश्च लक्ष्मणः शत्रुसूदनः}
{विष्णोर्भागममीमांस्यमात्मानं प्रत्यनुस्मरन्} %6-59-122

\twolineshloka
{निपातितमहावीरां वानराणां महाचमूम्}
{राघवस्तु रणे दृष्ट्वा रावणं समभिद्रवत्} %6-59-123

\twolineshloka
{अथैनमनुसङ्क्रम्य हनूमान् वाक्यमब्रवीत्}
{मम पृष्ठं समारुह्य राक्षसं शास्तुमर्हसि} %6-59-124

\twolineshloka
{विष्णुर्यथा गरुत्मन्तमारुह्यामरवैरिणम्}
{तच्छ्रुत्वा राघवो वाक्यं वायुपुत्रेण भाषितम्} %6-59-125

\twolineshloka
{अथारुरोह सहसा हनूमन्तं महाकपिम्}
{रथस्थं रावणं सङ्ख्ये ददर्श मनुजाधिपः} %6-59-126

\twolineshloka
{तमालोक्य महातेजाः प्रदुद्राव स रावणम्}
{वैरोचनमिव क्रुद्धो विष्णुरभ्युद्यतायुधः} %6-59-127

\twolineshloka
{ज्याशब्दमकरोत् तीव्रं वज्रनिष्पेषनिष्ठुरम्}
{गिरा गम्भीरया रामो राक्षसेन्द्रमुवाच ह} %6-59-128

\twolineshloka
{तिष्ठ तिष्ठ मम त्वं हि कृत्वा विप्रियमीदृशम्}
{क्व नु राक्षसशार्दूल गत्वा मोक्षमवाप्स्यसि} %6-59-129

\twolineshloka
{यदीन्द्रवैवस्वतभास्करान् वा स्वयम्भुवैश्वानरशङ्करान् वा}
{गमिष्यसि त्वं दशधा दिशो वा तथापि मे नाद्य गतो विमोक्ष्यसे} %6-59-130

\twolineshloka
{यश्चैष शक्त्या निहतस्त्वयाद्य गच्छन् विषादं सहसाभ्युपेत्य}
{स एष रक्षोगणराज मृत्युः सपुत्रपौत्रस्य तवाद्य युद्धे} %6-59-131

\twolineshloka
{एतेन चात्यद्भुतदर्शनानि शरैर्जनस्थानकृतालयानि}
{चतुर्दशान्यात्तवरायुधानि रक्षःसहस्राणि निषूदितानि} %6-59-132

\twolineshloka
{राघवस्य वचः श्रुत्वा राक्षसेन्द्रो महाबलः}
{वायुपुत्रं महावेगं वहन्तं राघवं रणे} %6-59-133

\twolineshloka
{रोषेण महताऽऽविष्टः पूर्ववैरमनुस्मरन्}
{आजघान शरैर्दीप्तैः कालानलशिखोपमैः} %6-59-134

\twolineshloka
{राक्षसेनाहवे तस्य ताडितस्यापि सायकैः}
{स्वभावतेजोयुक्तस्य भूयस्तेजोऽभ्यवर्धत} %6-59-135

\twolineshloka
{ततो रामो महातेजा रावणेन कृतव्रणम्}
{दृष्ट्वा प्लवगशार्दूलं क्रोधस्य वशमेयिवान्} %6-59-136

\twolineshloka
{तस्याभिसङ्क्रम्य रथं सचक्रं साश्वध्वजच्छत्रमहापताकम्}
{ससारथिं साशनिशूलखड्गं रामः प्रचिच्छेद शितैः शराग्रैः} %6-59-137

\twolineshloka
{अथेन्द्रशत्रुं तरसा जघान बाणेन वज्राशनिसन्निभेन}
{भुजान्तरे व्यूढसुजातरूपे वज्रेण मेरुं भगवानिवेन्द्रः} %6-59-138

\twolineshloka
{यो वज्रपाताशनिसन्निपातान्न चुक्षुभे नापि चचाल राजा}
{स रामबाणाभिहतो भृशार्तश्चचाल चापं च मुमोच वीरः} %6-59-139

\twolineshloka
{तं विह्वलन्तं प्रसमीक्ष्य रामः समाददे दीप्तमथार्धचन्द्रम्}
{तेनार्कवर्णं सहसा किरीटं चिच्छेद रक्षोधिपतेर्महात्मा} %6-59-140

\twolineshloka
{तं निर्विषाशीविषसन्निकाशं शान्तार्चिषं सूर्यमिवाप्रकाशम्}
{गतश्रियं कृत्तकिरीटकूटमुवाच रामो युधि राक्षसेन्द्रम्} %6-59-141

\twolineshloka
{कृतं त्वया कर्म महत् सुभीमं हतप्रवीरश्च कृतस्त्वयाहम्}
{तस्मात् परिश्रान्त इति व्यवस्य न त्वां शरैर्मृत्युवशं नयामि} %6-59-142

\twolineshloka
{प्रयाहि जानामि रणार्दितस्त्वं प्रविश्य रात्रिञ्चरराज लङ्काम्}
{आश्वस्य निर्याहि रथी च धन्वी तदा बलं प्रेक्ष्यसि मे रथस्थः} %6-59-143

\twolineshloka
{स एवमुक्तो हतदर्पहर्षो निकृत्तचापः स हताश्वसूतः}
{शरार्दितो भग्नमहाकिरीटो विवेश लङ्कां सहसा स्म राजा} %6-59-144

\twolineshloka
{तस्मिन् प्रविष्टे रजनीचरेन्द्रे महाबले दानवदेवशत्रौ}
{हरीन् विशल्यान् सह लक्ष्मणेन चकार रामः परमाहवाग्रे} %6-59-145

\twolineshloka
{तस्मिन् प्रभग्ने त्रिदशेन्द्रशत्रौ सुरासुरा भूतगणा दिशश्च}
{ससागराः सर्षिमहोरगाश्च तथैव भूम्यम्बुचराः प्रहृष्टाः} %6-59-146


॥इत्यार्षे श्रीमद्रामायणे वाल्मीकीये आदिकाव्ये युद्धकाण्डे रावणाभिषेणनम् नाम एकोनषष्ठितमः सर्गः ॥६-५९॥
