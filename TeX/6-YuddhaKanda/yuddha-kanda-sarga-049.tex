\sect{एकोनपञ्चाशः सर्गः — रामनिर्वेदः}

\twolineshloka
{घोरेण शरबन्धेन बद्धौ दशरथात्मजौ}
{निःश्वसन्तौ यथा नागौ शयानौ रुधिरोक्षितौ} %6-49-1

\twolineshloka
{सर्वे ते वानरश्रेष्ठाः ससुग्रीवमहाबलाः}
{परिवार्य महात्मानौ तस्थुः शोकपरिप्लुताः} %6-49-2

\twolineshloka
{एतस्मिन्नन्तरे रामः प्रत्यबुध्यत वीर्यवान्}
{स्थिरत्वात् सत्त्वयोगाच्च शरैः सन्दानितोऽपि सन्} %6-49-3

\twolineshloka
{ततो दृष्ट्वा सरुधिरं निषण्णं गाढमर्पितम्}
{भ्रातरं दीनवदनं पर्यदेवयदातुरः} %6-49-4

\twolineshloka
{किं नु मे सीतया कार्यं लब्धया जीवितेन वा}
{शयानं योऽद्य पश्यामि भ्रातरं युधि निर्जितम्} %6-49-5

\twolineshloka
{शक्या सीतासमा नारी मर्त्यलोके विचिन्वता}
{न लक्ष्मणसमो भ्राता सचिवः साम्परायिकः} %6-49-6

\twolineshloka
{परित्यक्ष्याम्यहं प्राणान् वानराणां तु पश्यताम्}
{यदि पञ्चत्वमापन्नः सुमित्रानन्दवर्धनः} %6-49-7

\twolineshloka
{किं नु वक्ष्यामि कौसल्यां मातरं किं नु कैकयीम्}
{कथमम्बां सुमित्रां च पुत्रदर्शनलालसाम्} %6-49-8

\twolineshloka
{विवत्सां वेपमानां च वेपन्तीं कुररीमिव}
{कथमाश्वासयिष्यामि यदि यास्यामि तं विना} %6-49-9

\twolineshloka
{कथं वक्ष्यामि शत्रुघ्नं भरतं च यशस्विनम्}
{मया सह वनं यातो विना तेनाहमागतः} %6-49-10

\twolineshloka
{उपालम्भं न शक्ष्यामि सोढुमम्बासुमित्रया}
{इहैव देहं त्यक्ष्यामि नहि जीवितुमुत्सहे} %6-49-11

\twolineshloka
{धिङ्मां दुष्कृतकर्माणमनार्यं यत्कृते ह्यसौ}
{लक्ष्मणः पतितः शेते शरतल्पे गतासुवत्} %6-49-12

\twolineshloka
{त्वं नित्यं सुविषण्णं मामाश्वासयसि लक्ष्मण}
{गतासुर्नाद्य शक्तोऽसि मामार्तमभिभाषितुम्} %6-49-13

\twolineshloka
{येनाद्य बहवो युद्धे निहता राक्षसाः क्षितौ}
{तस्यामेवाद्य शूरस्त्वं शेषे विनिहतः शनैः} %6-49-14

\twolineshloka
{शयानः शरतल्पेऽस्मिन् सशोणितपरिस्रुतः}
{शरभूतस्ततो भासि भास्करोऽस्तमिव व्रजन्} %6-49-15

\twolineshloka
{बाणाभिहतमर्मत्वान्न शक्नोषीह भाषितुम्}
{रुजा चाब्रुवतो यस्य दृष्टिरागेण सूच्यते} %6-49-16

\twolineshloka
{यथैव मां वनं यान्तमनुयातो महाद्युतिः}
{अहमप्यनुयास्यामि तथैवैनं यमक्षयम्} %6-49-17

\twolineshloka
{इष्टबन्धुजनो नित्यं मां च नित्यमनुव्रतः}
{इमामद्य गतोऽवस्थां ममानार्यस्य दुर्नयैः} %6-49-18

\twolineshloka
{सुरुष्टेनापि वीरेण लक्ष्मणेन न संस्मरे}
{परुषं विप्रियं चापि श्रावितं तु कदाचन} %6-49-19

\twolineshloka
{विससर्जैकवेगेन पञ्चबाणशतानि यः}
{इष्वस्त्रेष्वधिकस्तस्मात् कार्तवीर्याच्च लक्ष्मणः} %6-49-20

\twolineshloka
{अस्त्रैरस्त्राणि यो हन्याच्छक्रस्यापि महात्मनः}
{सोऽयमुर्व्यां हतः शेते महार्हशयनोचितः} %6-49-21

\twolineshloka
{तत्तु मिथ्या प्रलप्तं मां प्रधक्ष्यति न संशयः}
{यन्मया न कृतो राजा राक्षसानां विभीषणः} %6-49-22

\twolineshloka
{अस्मिन् मुहूर्ते सुग्रीव प्रतियातुमितोऽर्हसि}
{मत्वा हीनं मया राजन् रावणोऽभिभविष्यति} %6-49-23

\twolineshloka
{अङ्गदं तु पुरस्कृत्य ससैन्यं सपरिच्छदम्}
{सागरं तर सुग्रीव नीलेन च नलेन च} %6-49-24

\twolineshloka
{कृतं हि सुमहत्कर्म यदन्यैर्दुष्करं रणे}
{ऋक्षराजेन तुष्यामि गोलाङ्गूलाधिपेन च} %6-49-25

\twolineshloka
{अङ्गदेन कृतं कर्म मैन्देन द्विविदेन च}
{युद्धं केसरिणा सङ्ख्ये घोरं सम्पातिना कृतम्} %6-49-26

\twolineshloka
{गवयेन गवाक्षेण शरभेण गजेन च}
{अन्यैश्च हरिभिर्युद्धं मदर्थे त्यक्तजीवितैः} %6-49-27

\twolineshloka
{न चातिक्रमितुं शक्यं दैवं सुग्रीव मानुषैः}
{यत्तु शक्यं वयस्येन सुहृदा वा परं मम} %6-49-28

\twolineshloka
{कृतं सुग्रीव तत् सर्वं भवता धर्मभीरुणा}
{मित्रकार्यं कृतमिदं भवद्भिर्वानरर्षभाः} %6-49-29

\twolineshloka
{अनुज्ञाता मया सर्वे यथेष्टं गन्तुमर्हथ}
{शुश्रुवुस्तस्य ये सर्वे वानराः परिदेवितम्} %6-49-30

\onelineshloka
{वर्तयाञ्चक्रिरेऽश्रूणि नेत्रैः कृष्णेतरेक्षणाः} %6-49-31

\twolineshloka
{ततः सर्वाण्यनीकानि स्थापयित्वा विभीषणः}
{आजगाम गदापाणिस्त्वरितं यत्र राघवः} %6-49-32

\twolineshloka
{तं दृष्ट्वा त्वरितं यान्तं नीलाञ्जनचयोपमम्}
{वानरा दुद्रुवुः सर्वे मन्यमानास्तु रावणिम्} %6-49-33


॥इत्यार्षे श्रीमद्रामायणे वाल्मीकीये आदिकाव्ये युद्धकाण्डे रामनिर्वेदः नाम एकोनपञ्चाशः सर्गः ॥६-४९॥
