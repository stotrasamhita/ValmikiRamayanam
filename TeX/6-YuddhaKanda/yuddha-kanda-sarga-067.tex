\sect{सप्तषष्ठितमः सर्गः — कुम्भकर्णवधः}

\twolineshloka
{ते निवृत्ता महाकायाः श्रुत्वाङ्गदवचस्तदा}
{नैष्ठिकीं बुद्धिमास्थाय सर्वे संग्रामकाङ्क्षिणः} %6-67-1

\twolineshloka
{समुदीरितवीर्यास्ते समारोपितविक्रमाः}
{पर्यवस्थापिता वाक्यैरङ्गदेन बलीयसा} %6-67-2

\twolineshloka
{प्रयाताश्च गता हर्षं मरणे कृतनिश्चयाः}
{चक्रुः सुतुमुलं युद्धं वानरास्त्यक्तजीविताः} %6-67-3

\twolineshloka
{अथ वृक्षान् महाकायाः सानूनि सुमहान्ति च}
{वानरास्तूर्णमुद्यम्य कुम्भकर्णमभिद्रवन्} %6-67-4

\twolineshloka
{कुम्भकर्णः सुसंक्रुद्धो गदामुद्यम्य वीर्यवान्}
{धर्षयन् स महाकायः समन्ताद् व्यक्षिपद् रिपून्} %6-67-5

\twolineshloka
{शतानि सप्त चाष्टौ च सहस्राणि च वानराः}
{प्रकीर्णाः शेरते भूमौ कुम्भकर्णेन ताडिताः} %6-67-6

\threelineshloka
{षोडशाष्टौ च दश च विंशत्त्रिंशत्तथैव च}
{परिक्षिप्य च बाहुभ्यां खादन् स परिधावति}
{भक्षयन् भृशसंक्रुद्धो गरुडः पन्नगानिव} %6-67-7

\twolineshloka
{कृच्छ्रेण च समाश्वस्ताः संगम्य च ततस्ततः}
{वृक्षाद्रिहस्ता हरयस्तस्थुः संग्राममूर्धनि} %6-67-8

\twolineshloka
{ततः पर्वतमुत्पाट्य द्विविदः प्लवगर्षभः}
{दुद्राव गिरिशृङ्गाभं विलम्ब इव तोयदः} %6-67-9

\twolineshloka
{तं समुत्पाट्य चिक्षेप कुम्भकर्णाय वानरः}
{तमप्राप्य महाकायं तस्य सैन्येऽपतत् ततः} %6-67-10

\twolineshloka
{ममर्दाश्वान् गजांश्चापि रथांश्चापि गजोत्तमान्}
{तानि चान्यानि रक्षांसि एवं चान्यद्गिरेः शिरः} %6-67-11

\twolineshloka
{तच्छैलवेगाभिहतं हताश्वं हतसारथिम्}
{रक्षसां रुधिरक्लिन्नं बभूवायोधनं महत्} %6-67-12

\twolineshloka
{रथिनो वानरेन्द्राणां शरैः कालान्तकोपमैः}
{शिरांसि नर्दतां जह्रुः सहसा भीमनिःस्वनाः} %6-67-13

\twolineshloka
{वानराश्च महात्मानः समुत्पाट्य महाद्रुमान्}
{रथानश्वान् गजानुष्ट्रान् राक्षसानभ्यसूदयन्} %6-67-14

\twolineshloka
{हनूमान् शैलशृङ्गाणि शिलाश्च विविधान् द्रुमान्}
{ववर्ष कुम्भकर्णस्य शिरस्यम्बरमास्थितः} %6-67-15

\twolineshloka
{तानि पर्वतशृङ्गाणि शूलेन स बिभेद ह}
{बभञ्ज वृक्षवर्षं च कुम्भकर्णो महाबलः} %6-67-16

\twolineshloka
{ततो हरीणां तदनीकमुग्रं दुद्राव शूलं निशितं प्रगृह्य}
{तस्थौ स तस्यापततः परस्तान्महीधराग्रं हनुमान् प्रगृह्य} %6-67-17

\twolineshloka
{स कुम्भकर्णं कुपितो जघान वेगेन शैलोत्तमभीमकायम्}
{संचुक्षुभे तेन तदाभिभूतो मेदार्द्रगात्रो रुधिरावसिक्तः} %6-67-18

\twolineshloka
{स शूलमाविध्य तडित्प्रकाशं गिरिं यथा प्रज्वलिताग्निशृङ्गम्}
{बाह्वन्तरे मारुतिमाजघान गुहोऽचलं क्रौञ्चमिवोग्रशक्त्या} %6-67-19

\twolineshloka
{स शूलनिर्भिन्नमहाभुजान्तरः प्रविह्वलः शोणितमुद्वमन् मुखात्}
{ननाद भीमं हनुमान् महाहवे युगान्तमेघस्तनितस्वनोपमम्} %6-67-20

\twolineshloka
{ततो विनेदुः सहसा प्रहृष्टा रक्षोगणास्तं व्यथितं समीक्ष्य}
{प्लवंगमास्तु व्यथिता भयार्ताः प्रदुद्रुवुः संयति कुम्भकर्णात्} %6-67-21

\twolineshloka
{ततस्तु नीलो बलवान् पर्यवस्थापयन् बलम्}
{प्रविचिक्षेप शैलाग्रं कुम्भकर्णाय धीमते} %6-67-22

\threelineshloka
{तदापतन्तं सम्प्रेक्ष्य मुष्टिनाभिजघान ह}
{मुष्टिप्रहाराभिहतं तच्छैलाग्रं व्यशीर्यत}
{सविस्फुलिङ्गं सज्वालं निपपात महीतले} %6-67-23

\twolineshloka
{ऋषभः शरभो नीलो गवाक्षो गन्धमादनः}
{पञ्च वानरशार्दूलाः कुम्भकर्णमुपाद्रवन्} %6-67-24

\twolineshloka
{शैलैर्वृक्षैस्तलैः पादैर्मुष्टिभिश्च महाबलाः}
{कुम्भकर्णं महाकायं निजघ्नुः सर्वतो युधि} %6-67-25

\twolineshloka
{स्पर्शानिव प्रहारांस्तान् वेदयानो न विव्यथे}
{ऋषभं तु महावेगं बाहुभ्यां परिषस्वजे} %6-67-26

\twolineshloka
{कुम्भकर्णभुजाभ्यां तु पीडितो वानरर्षभः}
{निपपातर्षभो भीमः प्रमुखागतशोणितः} %6-67-27

\threelineshloka
{मुष्टिना शरभं हत्वा जानुना नीलमाहवे}
{आजघान गवाक्षं तु तलेनेन्द्ररिपुस्तदा}
{पादेनाभ्यहनत् क्रुद्धस्तरसा गन्धमादनम्} %6-67-28

\twolineshloka
{दत्तप्रहारव्यथिता मुमुहुः शोणितोक्षिताः}
{निपेतुस्ते तु मेदिन्यां निकृत्ता इव किंशुकाः} %6-67-29

\twolineshloka
{तेषु वानरमुख्येषु पातितेषु महात्मसु}
{वानराणां सहस्राणि कुम्भकर्णं प्रदुद्रुवुः} %6-67-30

\twolineshloka
{तं शैलमिव शैलाभाः सर्वे तु प्लवगर्षभाः}
{समारुह्य समुत्पत्य ददंशुश्च महाबलाः} %6-67-31

\twolineshloka
{तं नखैर्दशनैश्चापि मुष्टिभिर्बाहुभिस्तथा}
{कुम्भकर्णं महाबाहुं निजघ्नुः प्लवगर्षभाः} %6-67-32

\twolineshloka
{स वानरसहस्रैस्तु विचितः पर्वतोपमः}
{रराज राक्षसव्याघ्रो गिरिरात्मरुहैरिव} %6-67-33

\twolineshloka
{बाहुभ्यां वानरान् सर्वान् प्रगृह्य स महाबलः}
{भक्षयामास संक्रुद्धो गरुडः पन्नगानिव} %6-67-34

\twolineshloka
{प्रक्षिप्ताः कुम्भकर्णेन वक्त्रे पातालसंनिभे}
{नासापुटाभ्यां संजग्मुः कर्णाभ्यां चैव वानराः} %6-67-35

\twolineshloka
{भक्षयन् भृशसंक्रुद्धो हरीन् पर्वतसंनिभः}
{बभञ्ज वानरान् सर्वान् संक्रुद्धो राक्षसोत्तमः} %6-67-36

\twolineshloka
{मांसशोणितसंक्लेदां कुर्वन् भूमिं स राक्षसः}
{चचार हरिसैन्येषु कालाग्निरिव मूर्च्छितः} %6-67-37

\twolineshloka
{वज्रहस्तो यथा शक्रः पाशहस्त इवान्तकः}
{शूलहस्तो बभौ युद्धे कुम्भकर्णो महाबलः} %6-67-38

\twolineshloka
{यथा शुष्काण्यरण्यानि ग्रीष्मे दहति पावकः}
{तथा वानरसैन्यानि कुम्भकर्णो ददाह सः} %6-67-39

\twolineshloka
{ततस्ते वध्यमानास्तु हतयूथाः प्लवंगमाः}
{वानरा भयसंविग्ना विनेदुर्विकृतैः स्वरैः} %6-67-40

\twolineshloka
{अनेकशो वध्यमानाः कुम्भकर्णेन वानराः}
{राघवं शरणं जग्मुर्व्यथिता भिन्नचेतसः} %6-67-41

\twolineshloka
{प्रभग्नान् वानरान् दृष्ट्वा वज्रहस्तात्मजात्मजः}
{अभ्यधावत वेगेन कुम्भकर्णं महाहवे} %6-67-42

\twolineshloka
{शैलशृङ्गं महद् गृह्य विनदन् स मुहुर्मुहुः}
{त्रासयन् राक्षसान् सर्वान् कुम्भकर्णपदानुगान्} %6-67-43

\twolineshloka
{चिक्षेप शैलशिखरं कुम्भकर्णस्य मूर्धनि}
{स तेनाभिहतो मूर्ध्नि शैलेनेन्द्ररिपुस्तदा} %6-67-44

\twolineshloka
{कुम्भकर्णः प्रजज्वाल क्रोधेन महता तदा}
{सोऽभ्यधावत वेगेन वालिपुत्रममर्षणः} %6-67-45

\twolineshloka
{कुम्भकर्णो महानादस्त्रासयन् सर्ववानरान्}
{शूलं ससर्ज वै रोषादङ्गदे तु महाबलः} %6-67-46

\twolineshloka
{तदापतन्तं बलवान् युद्धमार्गविशारदः}
{लाघवान्मोक्षयामास बलवान् वानरर्षभः} %6-67-47

\twolineshloka
{उत्पत्य चैनं तरसा तलेनोरस्यताडयत्}
{स तेनाभिहतः कोपात् प्रमुमोहाचलोपमः} %6-67-48

\twolineshloka
{स लब्धसंज्ञोऽतिबलो मुष्टिं संगृह्य राक्षसः}
{अपहस्तेन चिक्षेप विसंज्ञः स पपात ह} %6-67-49

\twolineshloka
{तस्मिन् प्लवगशार्दूले विसंज्ञे पतिते भुवि}
{तच्छूलं समुपादाय सुग्रीवमभिदुद्रुवे} %6-67-50

\twolineshloka
{तमापतन्तं सम्प्रेक्ष्य कुम्भकर्णं महाबलम्}
{उत्पपात तदा वीरः सुग्रीवो वानराधिपः} %6-67-51

\twolineshloka
{स पर्वताग्रमुत्क्षिप्य समाविध्य महाकपिः}
{अभिदुद्राव वेगेन कुम्भकर्णं महाबलम्} %6-67-52

\twolineshloka
{तमापतन्तं सम्प्रेक्ष्य कुम्भकर्णः प्लवंगमम्}
{तस्थौ विवृत्तसर्वाङ्गो वानरेन्द्रस्य सम्मुखः} %6-67-53

\twolineshloka
{कपिशोणितदिग्धाङ्गं भक्षयन्तं महाकपीन्}
{कुम्भकर्णं स्थितं दृष्ट्वा सुग्रीवो वाक्यमब्रवीत्} %6-67-54

\twolineshloka
{पातिताश्च त्वया वीराः कृतं कर्म सुदुष्करम्}
{भक्षितानि च सैन्यानि प्राप्तं ते परमं यशः} %6-67-55

\twolineshloka
{त्यज तद् वानरानीकं प्राकृतैः किं करिष्यसि}
{सहस्वैकं निपातं मे पर्वतस्यास्य राक्षस} %6-67-56

\twolineshloka
{तद् वाक्यं हरिराजस्य सत्त्वधैर्यसमन्वितम्}
{श्रुत्वा राक्षसशार्दूलः कुम्भकर्णोऽब्रवीद् वचः} %6-67-57

\twolineshloka
{प्रजापतेस्तु पौत्रस्त्वं तथैवर्क्षरजःसुतः}
{धृतिपौरुषसम्पन्नस्तस्माद् गर्जसि वानर} %6-67-58

\twolineshloka
{स कुम्भकर्णस्य वचो निशम्य व्याविध्य शैलं सहसा मुमोच}
{तेनाजघानोरसि कुम्भकर्णं शैलेन वज्राशनिसंनिभेन} %6-67-59

\twolineshloka
{तच्छैलशृङ्गं सहसा विभिन्नं भुजान्तरे तस्य तदा विशाले}
{ततो विषेदुः सहसा प्लवंगा रक्षोगणाश्चापि मुदा विनेदुः} %6-67-60

\twolineshloka
{स शैलशृङ्गाभिहतश्चुकोप ननाद रोषाच्च विवृत्य वक्त्रम्}
{व्याविध्य शूलं स तडित्प्रकाशं चिक्षेप हर्यृक्षपतेर्वधाय} %6-67-61

\twolineshloka
{तत् कुम्भकर्णस्य भुजप्रणुन्नं शूलं शितं काञ्चनधामयष्टिम्}
{क्षिप्रं समुत्पत्य निगृह्य दोर्भ्यां बभञ्ज वेगेन सुतोऽनिलस्य} %6-67-62

\twolineshloka
{कृतं भारसहस्रस्य शूलं कालायसं महत्}
{बभञ्ज जानुमारोप्य तदा हृष्टः प्लवंगमः} %6-67-63

\twolineshloka
{शूलं भग्नं हनुमता दृष्ट्वा वानरवाहिनी}
{हृष्टा ननाद बहुशः सर्वतश्चापि दुद्रुवे} %6-67-64

\threelineshloka
{बभूवाथ परित्रस्तो राक्षसो विमुखोऽभवत्}
{सिंहनादं च ते चक्रुः प्रहृष्टा वनगोचराः}
{मारुतिं पूजयांचक्रुर्दृष्ट्वा शूलं तथागतम्} %6-67-65

\twolineshloka
{स तत् तथा भग्नमवेक्ष्य शूलं चुकोप रक्षोधिपतिर्महात्मा}
{उत्पाट्य लङ्कामलयात् स शृङ्गं जघान सुग्रीवमुपेत्य तेन} %6-67-66

\twolineshloka
{स शैलशृङ्गाभिहतो विसंज्ञः पपात भूमौ युधि वानरेन्द्रः}
{तं वीक्ष्य भूमौ पतितं विसंज्ञं नेदुः प्रहृष्टा युधि यातुधानाः} %6-67-67

\twolineshloka
{समभ्युपेत्याद्भुतघोरवीर्यं स कुम्भकर्णो युधि वानरेन्द्रम्}
{जहार सुग्रीवमभिप्रगृह्य यथानिलो मेघमिव प्रचण्डः} %6-67-68

\twolineshloka
{स तं महामेघनिकाशरूपमुत्पाट्य गच्छन् युधि कुम्भकर्णः}
{रराज मेरुप्रतिमानरूपो मेरुर्यथा व्युच्छ्रितघोरशृङ्गः} %6-67-69

\twolineshloka
{ततस्तमादाय जगाम वीरः संस्तूयमानो युधि राक्षसेन्द्रः}
{शृण्वन् निनादं त्रिदिवालयानां प्लवङ्गराजग्रहविस्मितानाम्} %6-67-70

\twolineshloka
{ततस्तमादाय तदा स मेने हरीन्द्रमिन्द्रोपममिन्द्रवीर्यः}
{अस्मिन् हते सर्वमिदं हतं स्यात् सराघवं सैन्यमितीन्द्रशत्रुः} %6-67-71

\twolineshloka
{विद्रुतां वाहिनीं दृष्ट्वा वानराणामितस्ततः}
{कुम्भकर्णेन सुग्रीवं गृहीतं चापि वानरम्} %6-67-72

\twolineshloka
{हनूमांश्चिन्तयामास मतिमान् मारुतात्मजः}
{एवं गृहीते सुग्रीवे किं कर्तव्यं मया भवेत्} %6-67-73

\twolineshloka
{यद्धि न्याय्यं मया कर्तुं तत् करिष्याम्यसंशयम्}
{भूत्वा पर्वतसंकाशो नाशयिष्यामि राक्षसम्} %6-67-74

\twolineshloka
{मया हते संयति कुम्भकर्णे महाबले मुष्टिविशीर्णदेहे}
{विमोचिते वानरपार्थिवे च भवन्तु हृष्टाः प्लवगाः समग्राः} %6-67-75

\twolineshloka
{अथवा स्वयमप्येष मोक्षं प्राप्स्यति वानरः}
{गृहीतोऽयं यदि भवेत् त्रिदशैः सासुरोरगैः} %6-67-76

\twolineshloka
{मन्ये न तावदात्मानं बुध्यते वानराधिपः}
{शैलप्रहाराभिहतः कुम्भकर्णेन संयुगे} %6-67-77

\twolineshloka
{अयं मुहूर्तात् सुग्रीवो लब्धसंज्ञो महाहवे}
{आत्मनो वानराणां च यत् पथ्यं तत् करिष्यति} %6-67-78

\twolineshloka
{मया तु मोक्षितस्यास्य सुग्रीवस्य महात्मनः}
{अप्रीतिश्च भवेत् कष्टा कीर्तिनाशश्च शाश्वतः} %6-67-79

\twolineshloka
{तस्मान्मुहूर्तं कांक्षिष्ये विक्रमं मोक्षितस्य तु}
{भिन्नं च वानरानीकं तावदाश्वासयाम्यहम्} %6-67-80

\twolineshloka
{इत्येवं चिन्तयित्वाथ हनूमान् मारुतात्मजः}
{भूयः संस्तम्भयामास वानराणां महाचमूम्} %6-67-81

\twolineshloka
{स कुम्भकर्णोऽथ विवेश लङ्कां स्फुरन्तमादाय महाहरिं तम्}
{विमानचर्यागृहगोपुरस्थैः पुष्पाग्र्यवर्षैरभिपूज्यमानः} %6-67-82

\twolineshloka
{लाजगन्धोदवर्षैस्तु सेच्यमानः शनैः शनैः}
{राजवीथ्यास्तु शीतत्वात् संज्ञां प्राप महाबलः} %6-67-83

\twolineshloka
{ततः स संज्ञामुपलभ्य कृच्छ्राद् बलीयसस्तस्य भुजान्तरस्थः}
{अवेक्षमाणः पुरराजमार्गं विचिन्तयामास मुहुर्महात्मा} %6-67-84

\twolineshloka
{एवं गृहीतेन कथं नु नाम शक्यं मया सम्प्रतिकर्तुमद्य}
{तथा करिष्यामि यथा हरीणां भविष्यतीष्टं च हितं च कार्यम्} %6-67-85

\twolineshloka
{ततः कराग्रैः सहसा समेत्य राजा हरीणाममरेन्द्रशत्रोः}
{खरैश्च कर्णौ दशनैश्च नासां ददंश पादैर्विददार पार्श्वौ} %6-67-86

\twolineshloka
{स कुम्भकर्णो हृतकर्णनासो विदारितस्तेन रदैर्नखैश्च}
{रोषाभिभूतः क्षतजार्द्रगात्रः सुग्रीवमाविध्य पिपेष भूमौ} %6-67-87

\twolineshloka
{स भूतले भीमबलाभिपिष्टः सुरारिभिस्तैरभिहन्यमानः}
{जगाम खं कन्दुकवज्जवेन पुनश्च रामेण समाजगाम} %6-67-88

\twolineshloka
{कर्णनासाविहीनस्तु कुम्भकर्णो महाबलः}
{रराज शोणितोत्सिक्तो गिरिः प्रस्रवणैरिव} %6-67-89

\twolineshloka
{शोणितार्द्रो महाकायो राक्षसो भीमदर्शनः}
{युद्धायाभिमुखो भूयो मनश्चक्रे निशाचरः} %6-67-90

\twolineshloka
{अमर्षाच्छोणितोद्गारी शुशुभे रावणानुजः}
{नीलाञ्जनचयप्रख्यः ससंध्य इव तोयदः} %6-67-91

\twolineshloka
{गते च तस्मिन् सुरराजशत्रुः क्रोधात् प्रदुद्राव रणाय भूयः}
{अनायुधोऽस्मीति विचिन्त्य रौद्रो घोरं तदा मुद्गरमाससाद} %6-67-92

\twolineshloka
{ततः स पुर्याः सहसा महौजा निष्क्रम्य तद् वानरसैन्यमुग्रम्}
{बभक्ष रक्षो युधि कुम्भकर्णः प्रजा युगान्ताग्निरिव प्रवृद्धः} %6-67-93

\threelineshloka
{बुभुक्षितः शोणितमांसगृध्नुः प्रविश्य तद् वानरसैन्यमुग्रम्}
{चखाद रक्षांसि हरीन् पिशाचान्नृक्षांश्च मोहाद् युधि कुम्भकर्णः}
{यथैव मृत्युर्हरते युगान्ते स भक्षयामास हरींश्च मुख्यान्} %6-67-94

\twolineshloka
{एकं द्वौ त्रीन् बहून् क्रुद्धो वानरान् सह राक्षसैः}
{समादायैकहस्तेन प्रचिक्षेप त्वरन् मुखे} %6-67-95

\twolineshloka
{सम्प्रस्रवंस्तदा मेदः शोणितं च महाबलः}
{वध्यमानो नगेन्द्राग्रैर्भक्षयामास वानरान्} %6-67-96

\twolineshloka
{ते भक्ष्यमाणा हरयो रामं जग्मुस्तदा गतिम्}
{कुम्भकर्णो भृशं क्रुद्धः कपीन् खादन् प्रधावति} %6-67-97

\twolineshloka
{शतानि सप्त चाष्टौ च विंशत् त्रिंशत् तथैव च}
{सम्परिष्वज्य बाहुभ्यां खादन् विपरिधावति} %6-67-98

\twolineshloka
{मेदोवसाशोणितदिग्धगात्रः कर्णावसक्तग्रथितान्त्रमालः}
{ववर्ष शूलानि सुतीक्ष्णदंष्ट्रः कालो युगान्तस्थ इव प्रवृद्धः} %6-67-99

\twolineshloka
{तस्मिन् काले सुमित्रायाः पुत्रः परबलार्दनः}
{चकार लक्ष्मणः क्रुद्धो युद्धं परपुरंजयः} %6-67-100

\twolineshloka
{स कुम्भकर्णस्य शरान् शरीरे सप्त वीर्यवान्}
{निचखानाददे चान्यान् विससर्ज च लक्ष्मणः} %6-67-101

\twolineshloka
{पीड्यमानस्तदस्त्रं तु विशेषं तत् स राक्षसः}
{ततश्चुकोप बलवान् सुमित्रानन्दवर्धनः} %6-67-102

\twolineshloka
{अथास्य कवचं शुभ्रं जाम्बूनदमयं शुभम्}
{प्रच्छादयामास शरैः संध्याभ्रमिव मारुतः} %6-67-103

\twolineshloka
{नीलाञ्जनचयप्रख्यः शरैः काञ्चनभूषणैः}
{आपीड्यमानः शुशुभे मेघैः सूर्य इवांशुमान्} %6-67-104

\twolineshloka
{ततः स राक्षसो भीमः सुमित्रानन्दवर्धनम्}
{सावज्ञमेव प्रोवाच वाक्यं मेघौघनिःस्वनः} %6-67-105

\twolineshloka
{अन्तकस्याप्यकष्टेन युधि जेतारमाहवे}
{युध्यता मामभीतेन ख्यापिता वीरता त्वया} %6-67-106

\twolineshloka
{प्रगृहीतायुधस्येह मृत्योरिव महामृधे}
{तिष्ठन्नप्यग्रतः पूज्यः किमु युद्धप्रदायकः} %6-67-107

\twolineshloka
{ऐरावतं समारूढो वृतः सर्वामरैः प्रभुः}
{नैव शक्रोऽपि समरे स्थितपूर्वः कदाचन} %6-67-108

\twolineshloka
{अद्य त्वयाहं सौमित्रे बालेनापि पराक्रमैः}
{तोषितो गन्तुमिच्छामि त्वामनुज्ञाप्य राघवम्} %6-67-109

\twolineshloka
{यत् तु वीर्यबलोत्साहैस्तोषितोऽहं रणे त्वया}
{राममेवैकमिच्छामि हन्तुं यस्मिन् हते हतम्} %6-67-110

\twolineshloka
{रामे मयात्र निहते येऽन्ये स्थास्यन्ति संयुगे}
{तानहं योधयिष्यामि स्वबलेन प्रमाथिना} %6-67-111

\twolineshloka
{इत्युक्तवाक्यं तद् रक्षः प्रोवाच स्तुतिसंहितम्}
{मृधे घोरतरं वाक्यं सौमित्रिः प्रहसन्निव} %6-67-112

\twolineshloka
{यस्त्वं शक्रादिभिर्देवैरसह्यः प्राप्य पौरुषम्}
{तत् सत्यं नान्यथा वीर दृष्टस्तेऽद्य पराक्रमः} %6-67-113

\twolineshloka
{एष दाशरथी रामस्तिष्ठत्यद्रिरिवाचलः}
{इति श्रुत्वा ह्यनादृत्य लक्ष्मणं स निशाचरः} %6-67-114

\twolineshloka
{अतिक्रम्य च सौमित्रिं कुम्भकर्णो महाबलः}
{राममेवाभिदुद्राव कम्पयन्निव मेदिनीम्} %6-67-115

\twolineshloka
{अथ दाशरथी रामो रौद्रमस्त्रं प्रयोजयन्}
{कुम्भकर्णस्य हृदये ससर्ज निशितान् शरान्} %6-67-116

\twolineshloka
{तस्य रामेण विद्धस्य सहसाभिप्रधावतः}
{अङ्गारमिश्राः क्रुद्धस्य मुखान्निश्चेरुरर्चिषः} %6-67-117

\twolineshloka
{रामास्त्रविद्धो घोरं वै नर्दन् राक्षसपुङ्गवः}
{अभ्यधावत संक्रुद्धो हरीन् विद्रावयन् रणे} %6-67-118

\twolineshloka
{तस्योरसि निमग्नास्ते शरा बर्हिणवाससः}
{हस्ताच्चास्य परिभ्रष्टा गदा चोर्व्यां पपात ह} %6-67-119

\twolineshloka
{आयुधानि च सर्वाणि विप्रकीर्यन्त भूतले}
{स निरायुधमात्मानं यदा मेने महाबलः} %6-67-120

\threelineshloka
{मुष्टिभ्यां च कराभ्यां च चकार कदनं महत्}
{स बाणैरतिविद्धाङ्गः क्षतजेन समुक्षितः}
{रुधिरं परिसुस्राव गिरिः प्रस्रवणं यथा} %6-67-121

\twolineshloka
{स तीव्रेण च कोपेन रुधिरेण च मूर्च्छितः}
{वानरान् राक्षसानृक्षान् खादन् स परिधावति} %6-67-122

\twolineshloka
{अथ शृङ्गं समाविध्य भीमं भीमपराक्रमः}
{चिक्षेप राममुद्दिश्य बलवानन्तकोपमः} %6-67-123

\twolineshloka
{अप्राप्तमन्तरा रामः सप्तभिस्तमजिह्मगैः}
{चिच्छेद गिरिशृङ्गं तं पुनः संधाय कार्मुकम्} %6-67-124

\twolineshloka
{ततस्तु रामो धर्मात्मा तस्य शृङ्गं महत् तदा}
{शरैः काञ्चनचित्राङ्गैश्चिच्छेद भरताग्रजः} %6-67-125

\twolineshloka
{तन्मेरुशिखराकारं द्योतमानमिव श्रिया}
{द्वे शते वानराणां च पतमानमपातयत्} %6-67-126

\twolineshloka
{तस्मिन् काले स धर्मात्मा लक्ष्मणो राममब्रवीत्}
{कुम्भकर्णवधे युक्तो योगान् परिमृशन् बहून्} %6-67-127

\twolineshloka
{नैवायं वानरान् राजन् न विजानाति राक्षसान्}
{मत्तः शोणितगन्धेन स्वान् परांश्चैव खादति} %6-67-128

\twolineshloka
{साध्वेनमधिरोहन्तु सर्वतो वानरर्षभाः}
{यूथपाश्च यथा मुख्यास्तिष्ठन्त्वस्मिन् समन्ततः} %6-67-129

\twolineshloka
{अद्यायं दुर्मतिः काले गुरुभारप्रपीडितः}
{प्रचरन् राक्षसो भूमौ नान्यान् हन्यात् प्लवंगमान्} %6-67-130

\twolineshloka
{तस्य तद् वचनं श्रुत्वा राजपुत्रस्य धीमतः}
{ते समारुरुहुर्हृष्टाः कुम्भकर्णं महाबलाः} %6-67-131

\twolineshloka
{कुम्भकर्णस्तु संक्रुद्धः समारूढः प्लवंगमैः}
{व्यधूनयत् तान् वेगेन दुष्टहस्तीव हस्तिपान्} %6-67-132

\twolineshloka
{तान् दृष्ट्वा निर्धुतान् रामो रुष्टोऽयमिति राक्षसम्}
{समुत्पपात वेगेन धनुरुत्तममाददे} %6-67-133

\threelineshloka
{क्रोधरक्तेक्षणो धीरो निर्दहन्निव चक्षुषा}
{राघवो राक्षसं वेगादभिदुद्राव वेगितः}
{यूथपान् हर्षयन् सर्वान् कुम्भकर्णबलार्दितान्} %6-67-134

\twolineshloka
{स चापमादाय भुजंगकल्पं दृढज्यमुग्रं तपनीयचित्रम्}
{हरीन् समाश्वास्य समुत्पपात रामो निबद्धोत्तमतूणबाणः} %6-67-135

\twolineshloka
{स वानरगणैस्तैस्तु वृतः परमदुर्जयैः}
{लक्ष्मणानुचरो वीरः सम्प्रतस्थे महाबलः} %6-67-136

\twolineshloka
{स ददर्श महात्मानं किरीटिनमरिंदमम्}
{शोणिताप्लुतरक्ताक्षं कुम्भकर्णं महाबलः} %6-67-137

\twolineshloka
{सर्वान् समभिधावन्तं यथा रुष्टं दिशागजम्}
{मार्गमाणं हरीन् क्रुद्धं राक्षसैः परिवारितम्} %6-67-138

\twolineshloka
{विन्ध्यमन्दरसंकाशं काञ्चनाङ्गदभूषणम्}
{स्रवन्तं रुधिरं वक्त्राद् वर्षमेघमिवोत्थितम्} %6-67-139

\twolineshloka
{जिह्वया परिलिह्यन्तं सृक्किणी शोणितोक्षिते}
{मृद्नन्तं वानरानीकं कालान्तकयमोपमम्} %6-67-140

\twolineshloka
{तं दृष्ट्वा राक्षसश्रेष्ठं प्रदीप्तानलवर्चसम्}
{विस्फारयामास तदा कार्मुकं पुरुषर्षभः} %6-67-141

\twolineshloka
{स तस्य चापनिर्घोषात् कुपितो राक्षसर्षभः}
{अमृष्यमाणस्तं घोषमभिदुद्राव राघवम्} %6-67-142

\twolineshloka
{ततस्तु वातोद्धतमेघकल्पं भुजंगराजोत्तमभोगबाहुः}
{तमापतन्तं धरणीधराभमुवाच रामो युधि कुम्भकर्णम्} %6-67-143

\twolineshloka
{आगच्छ रक्षोऽधिप मा विषादमवस्थितोऽहं प्रगृहीतचापः}
{अवेहि मां राक्षसवंशनाशनं यस्त्वं मुहूर्ताद् भविता विचेताः} %6-67-144

\twolineshloka
{रामोऽयमिति विज्ञाय जहास विकृतस्वनम्}
{अभ्यधावत संक्रुद्धो हरीन् विद्रावयन् रणे} %6-67-145

\twolineshloka
{दारयन्निव सर्वेषां हृदयानि वनौकसाम्}
{प्रहस्य विकृतं भीमं स मेघस्तनितोपमम्} %6-67-146

\threelineshloka
{कुम्भकर्णो महातेजा राघवं वाक्यमब्रवीत्}
{नाहं विराधो विज्ञेयो न कबन्धः खरो न च}
{न वाली न च मारीचः कुम्भकर्णः समागतः} %6-67-147

\twolineshloka
{पश्य मे मुद्गरं भीमं सर्वं कालायसं महत्}
{अनेन निर्जिता देवा दानवाश्च पुरा मया} %6-67-148

\twolineshloka
{विकर्णनास इति मां नावज्ञातुं त्वमर्हसि}
{स्वल्पापि हि न मे पीडा कर्णनासाविनाशनात्} %6-67-149

\twolineshloka
{दर्शयेक्ष्वाकुशार्दूल वीर्यं गात्रेषु मेऽनघ}
{ततस्त्वां भक्षयिष्यामि दृष्टपौरुषविक्रमम्} %6-67-150

\twolineshloka
{स कुम्भकर्णस्य वचो निशम्य रामः सुपुङ्खान् विससर्ज बाणान्}
{तैराहतो वज्रसमप्रवेगैर्न चुक्षुभे न व्यथते सुरारिः} %6-67-151

\twolineshloka
{यैः सायकैः सालवरा निकृत्ता वाली हतो वानरपुङ्गवश्च}
{ते कुम्भकर्णस्य तदा शरीरं वज्रोपमा न व्यथयाम्प्रचक्रुः} %6-67-152

\twolineshloka
{स वारिधारा इव सायकांस्तान् पिबन् शरीरेण महेन्द्रशत्रुः}
{जघान रामस्य शरप्रवेगं व्याविध्य तं मुद्गरमुग्रवेगम्} %6-67-153

\twolineshloka
{ततस्तु रक्षः क्षतजानुलिप्तं वित्रासनं देवमहाचमूनाम्}
{व्याविध्य तं मुद्गरमुग्रवेगं विद्रावयामास चमूं हरीणाम्} %6-67-154

\twolineshloka
{वायव्यमादाय ततोऽपरास्त्रं रामः प्रचिक्षेप निशाचराय}
{समुद्गरं तेन जहार बाहुं स कृत्तबाहुस्तुमुलं ननाद} %6-67-155

\twolineshloka
{स तस्य बाहुर्गिरिशृङ्गकल्पः समुद्गरो राघवबाणकृत्तः}
{पपात तस्मिन् हरिराजसैन्ये जघान तां वानरवाहिनीं च} %6-67-156

\twolineshloka
{ते वानरा भग्नहतावशेषाः पर्यन्तमाश्रित्य तदा विषण्णाः}
{प्रपीडिताङ्गा ददृशुः सुघोरं नरेन्द्ररक्षोऽधिपसंनिपातम्} %6-67-157

\twolineshloka
{स कुम्भकर्णोऽस्त्रनिकृत्तबाहुर्महासिकृत्ताग्र इवाचलेन्द्रः}
{उत्पाटयामास करेण वृक्षं ततोऽभिदुद्राव रणे नरेन्द्रम्} %6-67-158

\twolineshloka
{तं तस्य बाहुं सहतालवृक्षं समुद्यतं पन्नगभोगकल्पम्}
{ऐन्द्रास्त्रयुक्तेन जघान रामो बाणेन जाम्बूनदचित्रितेन} %6-67-159

\twolineshloka
{स कुम्भकर्णस्य भुजो निकृत्तः पपात भूमौ गिरिसंनिकाशः}
{विचेष्टमानो निजघान वृक्षान् शैलान् शिलावानरराक्षसांश्च} %6-67-160

\twolineshloka
{तं छिन्नबाहुं समवेक्ष्य रामः समापतन्तं सहसा नदन्तम्}
{द्वावर्धचन्द्रौ निशितौ प्रगृह्य चिच्छेद पादौ युधि राक्षसस्य} %6-67-161

\twolineshloka
{तौ तस्य पादौ प्रदिशो दिशश्च गिरेर्गुहाश्चैव महार्णवं च}
{लङ्कां च सेनां कपिराक्षसानां विनादयन्तौ विनिपेततुश्च} %6-67-162

\twolineshloka
{निकृत्तबाहुर्विनिकृत्तपादो विदार्य वक्त्रं वडवामुखाभम्}
{दुद्राव रामं सहसाभिगर्जन् राहुर्यथा चन्द्रमिवान्तरिक्षे} %6-67-163

\twolineshloka
{अपूरयत् तस्य मुखं शिताग्रै रामः शरैर्हेमपिनद्धपुङ्खैः}
{सम्पूर्णवक्त्रो न शशाक वक्तुं चुकूज कृच्छ्रेण मुमूर्च्छ चापि} %6-67-164

\twolineshloka
{अथाददे सूर्यमरीचिकल्पं स ब्रह्मदण्डान्तककालकल्पम्}
{अरिष्टमैन्द्रं निशितं सुपुङ्खं रामः शरं मारुततुल्यवेगम्} %6-67-165

\twolineshloka
{तं वज्रजाम्बूनदचारुपुङ्खं प्रदीप्तसूर्यज्वलनप्रकाशम्}
{महेन्द्रवज्राशनितुल्यवेगं रामः प्रचिक्षेप निशाचराय} %6-67-166

\twolineshloka
{स सायको राघवबाहुचोदितो दिशःस्वभासा दश सम्प्रकाशयन्}
{विधूमवैश्वानरभीमदर्शनो जगाम शक्राशनिभीमविक्रमः} %6-67-167

\twolineshloka
{स तन्महापर्वतकूटसंनिभं सुवृत्तदंष्ट्रं चलचारुकुण्डलम्}
{चकर्त रक्षोऽधिपतेः शिरस्तदा यथैव वृत्रस्य पुरा पुरंदरः} %6-67-168

\twolineshloka
{कुम्भकर्णशिरो भाति कुण्डलालंकृतं महत्}
{आदित्येऽभ्युदिते रात्रौ मध्यस्थ इव चन्द्रमाः} %6-67-169

\twolineshloka
{तद् रामबाणाभिहतं पपात रक्षःशिरः पर्वतसंनिकाशम्}
{बभञ्ज चर्यागृहगोपुराणि प्राकारमुच्चं तमपातयच्च} %6-67-170

\twolineshloka
{तच्चातिकायं हिमवत् प्रकाशं रक्षस्तदा तोयनिधौ पपात}
{ग्राहान् परान् मीनवरान् भुजंगमान् ममर्द भूमिं च तथा विवेश} %6-67-171

\twolineshloka
{तस्मिन् हते ब्राह्मणदेवशत्रौ महाबले संयति कुम्भकर्णे}
{चचाल भूर्भूमिधराश्च सर्वे हर्षाच्च देवास्तुमुलं प्रणेदुः} %6-67-172

\twolineshloka
{ततस्तु देवर्षिमहर्षिपन्नगाः सुराश्च भूतानि सुपर्णगुह्यकाः}
{सयक्षगन्धर्वगणा नभोगताः प्रहर्षिता रामपराक्रमेण} %6-67-173

\twolineshloka
{ततस्तु ते तस्य वधेन भूरिणा मनस्विनो नैर्ऋतराजबान्धवाः}
{विनेदुरुच्चैर्व्यथिता रघूत्तमं हरिं समीक्ष्यैव यथा मतंगजाः} %6-67-174

\twolineshloka
{स देवलोकस्य तमो निहत्य सूर्यो यथा राहुमुखाद् विमुक्तः}
{तथा व्यभासीद्धरिसैन्यमध्ये निहत्य रामो युधि कुम्भकर्णम्} %6-67-175

\twolineshloka
{प्रहर्षमीयुर्बहवश्च वानराः प्रबुद्धपद्मप्रतिमैरिवाननैः}
{अपूजयन् राघवमिष्टभागिनं हते रिपौ भीमबले नृपात्मजम्} %6-67-176

\twolineshloka
{स कुम्भकर्णं सुरसैन्यमर्दनं महत्सु युद्धेषु कदाचनाजितम्}
{ननन्द हत्वा भरताग्रजो रणे महासुरं वृत्रमिवामराधिपः} %6-67-177


॥इत्यार्षे श्रीमद्रामायणे वाल्मीकीये आदिकाव्ये युद्धकाण्डे कुम्भकर्णवधः नाम सप्तषष्ठितमः सर्गः ॥६-६७॥
