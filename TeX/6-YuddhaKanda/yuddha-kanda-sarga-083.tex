\sect{त्र्यशीतितमः सर्गः — रामाश्वासनम्}

\twolineshloka
{राघवश्चापि विपुलं तं राक्षसवनौकसाम्}
{श्रुत्वा सङ्ग्रामनिर्घोषं जाम्बवन्तमुवाच ह} %6-83-1

\twolineshloka
{सौम्य नूनं हनुमता कृतं कर्म सुदुष्करम्}
{श्रूयते च यथा भीमः सुमहानायुधस्वनः} %6-83-2

\twolineshloka
{तद् गच्छ कुरु साहाय्यं स्वबलेनाभिसंवृतः}
{क्षिप्रमृक्षपते तस्य कपिश्रेष्ठस्य युध्यतः} %6-83-3

\twolineshloka
{ऋक्षराजस्तथेत्युक्त्वा स्वेनानीकेन संवृतः}
{आगच्छत् पश्चिमं द्वारं हनूमान् यत्र वानरः} %6-83-4

\twolineshloka
{अथायान्तं हनूमन्तं ददर्शर्क्षपतिस्तदा}
{वानरैः कृतसङ्ग्रामैः श्वसद्भिरभिसंवृतम्} %6-83-5

\twolineshloka
{दृष्ट्वा पथि हनूमांश्च तदृक्षबलमुद्यतम्}
{नीलमेघनिभं भीमं सन्निवार्य न्यवर्तत} %6-83-6

\twolineshloka
{स तेन सह सैन्येन सन्निकर्षं महायशाः}
{शीघ्रमागम्य रामाय दुःखितो वाक्यमब्रवीत्} %6-83-7

\twolineshloka
{समरे युध्यमानानामस्माकं प्रेक्षतां च सः}
{जघान रुदतीं सीतामिन्द्रजिद् रावणात्मजः} %6-83-8

\twolineshloka
{उद्भ्रान्तचित्तस्तां दृष्ट्वा विषण्णोऽहमरिन्दम}
{तदहं भवतो वृत्तं विज्ञापयितुमागतः} %6-83-9

\twolineshloka
{तस्य तद् वचनं श्रुत्वा राघवः शोकमूर्च्छितः}
{निपपात तदा भूमौ छिन्नमूल इव द्रुमः} %6-83-10

\twolineshloka
{तं भूमौ देवसङ्काशं पतितं दृश्य राघवम्}
{अभिपेतुः समुत्पत्य सर्वतः कपिसत्तमाः} %6-83-11

\twolineshloka
{आसिञ्चन् सलिलैश्चैनं पद्मोत्पलसुगन्धिभिः}
{प्रदहन्तमसंहार्यं सहसाग्निमिवोत्थितम्} %6-83-12

\twolineshloka
{तं लक्ष्मणोऽथ बाहुभ्यां परिष्वज्य सुदुःखितः}
{उवाच राममस्वस्थं वाक्यं हेत्वर्थसंयुतम्} %6-83-13

\twolineshloka
{शुभे वर्त्मनि तिष्ठन्तं त्वामार्य विजितेन्द्रियम्}
{अनर्थेभ्यो न शक्नोति त्रातुं धर्मो निरर्थकः} %6-83-14

\twolineshloka
{भूतानां स्थावराणां च जङ्गमानां च दर्शनम्}
{यथास्ति न तथा धर्मस्तेन नास्तीति मे मतिः} %6-83-15

\twolineshloka
{यथैव स्थावरं व्यक्तं जङ्गमं च तथाविधम्}
{नायमर्थस्तथा युक्तस्त्वद्विधो न विपद्यते} %6-83-16

\twolineshloka
{यद्यधर्मो भवेद् भूतो रावणो नरकं व्रजेत्}
{भवांश्च धर्मसंयुक्तो नैव व्यसनमाप्नुयात्} %6-83-17

\twolineshloka
{तस्य च व्यसनाभावाद् व्यसनं चागते त्वयि}
{धर्मो भवत्यधर्मश्च परस्परविरोधिनौ} %6-83-18

\twolineshloka
{धर्मेणोपलभेद् धर्ममधर्मं चाप्यधर्मतः}
{यद्यधर्मेण युज्येयुर्येष्वधर्मः प्रतिष्ठितः} %6-83-19

\twolineshloka
{न धर्मेण वियुज्येरन्नाधर्मरुचयो जनाः}
{धर्मेणाचरतां तेषां तथा धर्मफलं भवेत्} %6-83-20

\twolineshloka
{यस्मादर्था विवर्धन्ते येष्वधर्मः प्रतिष्ठितः}
{क्लिश्यन्ते धर्मशीलाश्च तस्मादेतौ निरर्थकौ} %6-83-21

\twolineshloka
{वध्यन्ते पापकर्माणो यद्यधर्मेण राघव}
{वधकर्महतोऽधर्मः स हतः कं वधिष्यति} %6-83-22

\twolineshloka
{अथवा विहितेनायं हन्यते हन्ति चापरम्}
{विधिः स लिप्यते तेन न स पापेन कर्मणा} %6-83-23

\twolineshloka
{अदृष्टप्रतिकारेण अव्यक्तेनासता सता}
{कथं शक्यं परं प्राप्तुं धर्मेणारिविकर्षण} %6-83-24

\twolineshloka
{यदि सत् स्यात् सतां मुख्य नासत् स्यात् तव किञ्चन}
{त्वया यदीदृशं प्राप्तं तस्मात् तन्नोपपद्यते} %6-83-25

\twolineshloka
{अथवा दुर्बलः क्लीबो बलं धर्मोऽनुवर्तते}
{दुर्बलो हृतमर्यादो न सेव्य इति मे मतिः} %6-83-26

\twolineshloka
{बलस्य यदि चेद् धर्मो गुणभूतः पराक्रमैः}
{धर्ममुत्सृज्य वर्तस्व यथा धर्मे तथा बले} %6-83-27

\twolineshloka
{अथ चेत् सत्यवचनं धर्मः किल परन्तप}
{अनृतं त्वय्यकरणे किं न बद्धस्त्वया विना} %6-83-28

\twolineshloka
{यदि धर्मो भवेद् भूत अधर्मो वा परन्तप}
{न स्म हत्वा मुनिं वज्री कुर्यादिज्यां शतक्रतुः} %6-83-29

\twolineshloka
{अधर्मसंश्रितो धर्मो विनाशयति राघव}
{सर्वमेतद् यथाकामं काकुत्स्थ कुरुते नरः} %6-83-30

\twolineshloka
{मम चेदं मतं तात धर्मोऽयमिति राघव}
{धर्ममूलं त्वया छिन्नं राज्यमुत्सृजता तदा} %6-83-31

\twolineshloka
{अर्थेभ्योऽथ प्रवृद्धेभ्यः संवृत्तेभ्यस्ततस्ततः}
{क्रियाः सर्वाः प्रवर्तन्ते पर्वतेभ्य इवापगाः} %6-83-32

\twolineshloka
{अर्थेन हि विमुक्तस्य पुरुषस्याल्पचेतसः}
{विच्छिद्यन्ते क्रियाः सर्वा ग्रीष्मे कुसरितो यथा} %6-83-33

\twolineshloka
{सोऽयमर्थं परित्यज्य सुखकामः सुखैधितः}
{पापमाचरते कर्तुं तदा दोषः प्रवर्तते} %6-83-34

\twolineshloka
{यस्यार्थास्तस्य मित्राणि यस्यार्थास्तस्य बान्धवाः}
{यस्यार्थाः स पुमाँल्लोके यस्यार्थाः स च पण्डितः} %6-83-35

\twolineshloka
{यस्यार्थाः स च विक्रान्तो यस्यार्थाः स च बुद्धिमान्}
{यस्यार्थाः स महाभागो यस्यार्थाः स गुणाधिकः} %6-83-36

\twolineshloka
{अर्थस्यैते परित्यागे दोषाः प्रव्याहृता मया}
{राज्यमुत्सृजता धीर येन बुद्धिस्त्वया कृता} %6-83-37

\twolineshloka
{यस्यार्था धर्मकामार्थास्तस्य सर्वं प्रदक्षिणम्}
{अधनेनार्थकामेन नार्थः शक्यो विचिन्वता} %6-83-38

\twolineshloka
{हर्षः कामश्च दर्पश्च धर्मः क्रोधः शमो दमः}
{अर्थादेतानि सर्वाणि प्रवर्तन्ते नराधिप} %6-83-39

\twolineshloka
{येषां नश्यत्ययं लोकश्चरतां धर्मचारिणाम्}
{तेऽर्थास्त्वयि न दृश्यन्ते दुर्दिनेषु यथा ग्रहाः} %6-83-40

\twolineshloka
{त्वयि प्रव्रजिते वीर गुरोश्च वचने स्थिते}
{रक्षसापहृता भार्या प्राणौः प्रियतरा तव} %6-83-41

\twolineshloka
{तदद्य विपुलं वीर दुःखमिन्द्रजिता कृतम्}
{कर्मणा व्यपनेष्यामि तस्मादुत्तिष्ठ राघव} %6-83-42

\twolineshloka
{उत्तिष्ठ नरशार्दूल दीर्घबाहो धृतव्रत}
{किमात्मानं महात्मानमात्मानं नावबुध्यसे} %6-83-43

\twolineshloka
{अयमनघ तवोदितः प्रियार्थं जनकसुतानिधनं निरीक्ष्य रुष्टः}
{सरथगजहयां सराक्षसेन्द्रां भृशमिषुभिर्विनिपातयामि लङ्काम्} %6-83-44


॥इत्यार्षे श्रीमद्रामायणे वाल्मीकीये आदिकाव्ये युद्धकाण्डे रामाश्वासनम् नाम त्र्यशीतितमः सर्गः ॥६-८३॥
