\sect{अष्टनवतितमः सर्गः — महोदरवधः}

\twolineshloka
{हन्यमाने बले तूर्णमन्योन्यं ते महामृधे}
{सरसीव महाघर्मे सूपक्षीणे बभूवतुः} %6-98-1

\twolineshloka
{स्वबलस्य तु घातेन विरूपाक्षवधेन च}
{बभूव द्विगुणं क्रुद्धो रावणो राक्षसाधिपः} %6-98-2

\twolineshloka
{प्रक्षीणं स्वबलं दृष्ट्वा वध्यमानं वलीमुखैः}
{बभूवास्य व्यथा युद्धे दृष्ट्वा दैवविपर्ययम्} %6-98-3

\twolineshloka
{उवाच च समीपस्थं महोदरमनन्तरम्}
{अस्मिन् काले महाबाहो जयाशा त्वयि मे स्थिता} %6-98-4

\twolineshloka
{जहि शत्रुचमूं वीर दर्शयाद्य पराक्रमम्}
{भर्तृपिण्डस्य कालोऽयं निर्वेष्टुं साधु युध्यताम्} %6-98-5

\twolineshloka
{एवमुक्तस्तथेत्युक्त्वा राक्षसेन्द्रो महोदरः}
{प्रविवेशारिसेनां स पतङ्ग इव पावकम्} %6-98-6

\twolineshloka
{ततः स कदनं चक्रे वानराणां महाबलः}
{भर्तृवाक्येन तेजस्वी स्वेन वीर्येण चोदितः} %6-98-7

\twolineshloka
{वानराश्च महासत्त्वाः प्रगृह्य विपुलाः शिलाः}
{प्रविश्यारिबलं भीमं जघ्नुस्ते सर्वराक्षसान्} %6-98-8

\twolineshloka
{महोदरः सुसङ्क्रुद्धः शरैः काञ्चनभूषणैः}
{चिच्छेद पाणिपादोरु वानराणां महाहवे} %6-98-9

\twolineshloka
{ततस्ते वानराः सर्वे राक्षसैरर्दिता भृशम्}
{दिशो दश द्रुताः केचित् केचित् सुग्रीवमाश्रिताः} %6-98-10

\twolineshloka
{प्रभग्नं समरे दृष्ट्वा वानराणां महाबलम्}
{अभिदुद्राव सुग्रीवो महोदरमनन्तरम्} %6-98-11

\twolineshloka
{प्रगृह्य विपुलां घोरां महीधरसमां शिलाम्}
{चिक्षेप च महातेजास्तद्वधाय हरीश्वरः} %6-98-12

\twolineshloka
{तामापतन्तीं सहसा शिलां दृष्ट्वा महोदरः}
{असम्भ्रान्तस्ततो बाणैर्निर्बिभेद दुरासदाम्} %6-98-13

\twolineshloka
{रक्षसा तेन बाणौघैर्निकृत्ता सा सहस्रधा}
{निपपात तदा भूमौ गृध्रचक्रमिवाकुलम्} %6-98-14

\twolineshloka
{तां तु भिन्नां शिलां दृष्ट्वा सुग्रीवः क्रोधमूर्च्छितः}
{सालमुत्पाट्य चिक्षेप तं स चिच्छेद नैकधा} %6-98-15

\twolineshloka
{शरैश्च विददारैनं शूरः परबलार्दनः}
{स ददर्श ततः क्रुद्धः परिघं पतितं भुवि} %6-98-16

\twolineshloka
{आविध्य तु स तं दीप्तं परिघं तस्य दर्शयन्}
{परिघेणोग्रवेगेन जघानास्य हयोत्तमान्} %6-98-17

\twolineshloka
{तस्माद्धतहयाद् वीरः सोऽवप्लुत्य महारथात्}
{गदां जग्राह सङ्क्रुद्धो राक्षसोऽथ महोदरः} %6-98-18

\twolineshloka
{गदापरिघहस्तौ तौ युधि वीरौ समीयतुः}
{नर्दन्तौ गोवृषप्रख्यौ घनाविव सविद्युतौ} %6-98-19

\twolineshloka
{ततः क्रुद्धो गदां तस्मै चिक्षेप रजनीचरः}
{ज्वलन्तीं भास्कराभासां सुग्रीवाय महोदरः} %6-98-20

\twolineshloka
{गदां तां सुमहाघोरामापतन्तीं महाबलः}
{सुग्रीवो रोषताम्राक्षः समुद्यम्य महाहवे} %6-98-21

\twolineshloka
{आजघान गदां तस्य परिघेण हरीश्वरः}
{पपात तरसा भिन्नः परिघस्तस्य भूतले} %6-98-22

\twolineshloka
{ततो जग्राह तेजस्वी सुग्रीवो वसुधातलात्}
{आयसं मुसलं घोरं सर्वतो हेमभूषितम्} %6-98-23

\twolineshloka
{स तमुद्यम्य चिक्षेप सोऽप्यस्य प्राक्षिपद् गदाम्}
{भिन्नावन्योन्यमासाद्य पेततुस्तौ महीतले} %6-98-24

\twolineshloka
{ततो भिन्नप्रहरणौ मुष्टिभ्यां तौ समीयतुः}
{तेजोबलसमाविष्टौ दीप्ताविव हुताशनौ} %6-98-25

\twolineshloka
{जघ्नतुस्तौ तदान्योन्यं नदन्तौ च पुनः पुनः}
{तलैश्चान्योन्यमासाद्य पेततुश्च महीतले} %6-98-26

\twolineshloka
{उत्पेततुस्तदा तूर्णं जघ्नतुश्च परस्परम्}
{भुजैश्चिक्षिपतुर्वीरावन्योन्यमपराजितौ} %6-98-27

\twolineshloka
{जग्मतुस्तौ श्रमं वीरौ बाहुयुद्धे परन्तपौ}
{आजहार तदा खड्गमदूरपरिवर्तिनम्} %6-98-28

\threelineshloka
{राक्षसश्चर्मणा सार्धं महावेगो महोदरः}
{तथैव च महाखड्गं चर्मणा पतितं सह}
{जग्राह वानरश्रेष्ठः सुग्रीवो वेगवत्तरः} %6-98-29

\twolineshloka
{ततो रोषपरीताङ्गौ नदन्तावभ्यधावताम्}
{उद्यतासी रणे हृष्टौ युधि शस्त्रविशारदौ} %6-98-30

\twolineshloka
{दक्षिणं मण्डलं चोभौ सुतूर्णं सम्परीयतुः}
{अन्योन्यमभिसङ्क्रुद्धौ जये प्रणिहितावुभौ} %6-98-31

\twolineshloka
{स तु शूरो महावेगो वीर्यश्लाघी महोदरः}
{महावर्मणि तं खड्गं पातयामास दुर्मतिः} %6-98-32

\twolineshloka
{लग्नमुत्कर्षतः खड्गं खड्गेन कपिकुञ्जरः}
{जहार सशिरस्त्राणं कुण्डलोपगतं शिरः} %6-98-33

\twolineshloka
{निकृत्तशिरसस्तस्य पतितस्य महीतले}
{तद् बलं राक्षसेन्द्रस्य दृष्ट्वा तत्र न दृश्यते} %6-98-34

\twolineshloka
{हत्वा तं वानरैः सार्धं ननाद मुदितो हरिः}
{चुक्रोध च दशग्रीवो बभौ हृष्टश्च राघवः} %6-98-35

\twolineshloka
{विषण्णवदनाः सर्वे राक्षसा दीनचेतसः}
{विद्रवन्ति ततः सर्वे भयवित्रस्तचेतसः} %6-98-36

\twolineshloka
{महोदरं तं विनिपात्य भूमौ महागिरेः कीर्णमिवैकदेशम्}
{सूर्यात्मजस्तत्र रराज लक्ष्म्या सूर्यः स्वतेजोभिरिवाप्रधृष्यः} %6-98-37

\twolineshloka
{अथ विजयमवाप्य वानरेन्द्रः समरमुखे सुरसिद्धयक्षसङ्घैः}
{अवनितलगतैश्च भूतसङ्घैर्हरुषसमाकुलितैर्निरीक्ष्यमाणः} %6-98-38


॥इत्यार्षे श्रीमद्रामायणे वाल्मीकीये आदिकाव्ये युद्धकाण्डे महोदरवधः नाम अष्टनवतितमः सर्गः ॥६-९८॥
