\sect{द्वाविंशत्यधिकशततमः सर्गः — दशरथप्रतिसमादेशः}

\twolineshloka
{एतच्छ्रुत्वा शुभं वाक्यं राघवेणानुभाषितम्}
{ततः शुभतरं वाक्यं व्याजहार महेश्वरः} %6-122-1

\twolineshloka
{पुष्कराक्ष महाबाहो महावक्षः परंतप}
{दिष्ट्या कृतमिदं कर्म त्वया धर्मभृतां वर} %6-122-2

\twolineshloka
{दिष्ट्या सर्वस्य लोकस्य प्रवृद्धं दारुणं तमः}
{अपवृत्तं त्वया संख्ये राम रावणजं भयम्} %6-122-3

\twolineshloka
{आश्वास्य भरतं दीनं कौसल्यां च यशस्विनीम्}
{कैकेयीं च सुमित्रां च दृष्ट्वा लक्ष्मणमातरम्} %6-122-4

\twolineshloka
{प्राप्य राज्यमयोध्यायां नन्दयित्वा सुहृज्जनम्}
{इक्ष्वाकूणां कुले वंशं स्थापयित्वा महाबल} %6-122-5

\twolineshloka
{इष्ट्वा तुरगमेधेन प्राप्य चानुत्तमं यशः}
{ब्राह्मणेभ्यो धनं दत्त्वा त्रिदिवं गन्तुमर्हसि} %6-122-6

\twolineshloka
{एष राजा दशरथो विमानस्थः पिता तव}
{काकुत्स्थ मानुषे लोके गुरुस्तव महायशाः} %6-122-7

\twolineshloka
{इन्द्रलोकं गतः श्रीमांस्त्वया पुत्रेण तारितः}
{लक्ष्मणेन सह भ्रात्रा त्वमेनमभिवादय} %6-122-8

\twolineshloka
{महादेववचः श्रुत्वा राघवः सहलक्ष्मणः}
{विमानशिखरस्थस्य प्रणाममकरोत् पितुः} %6-122-9

\twolineshloka
{दीप्यमानं स्वया लक्ष्म्या विरजोऽम्बरधारिणम्}
{लक्ष्मणेन सह भ्रात्रा ददर्श पितरं प्रभुः} %6-122-10

\twolineshloka
{हर्षेण महताऽऽविष्टो विमानस्थो महीपतिः}
{प्राणैः प्रियतरं दृष्ट्वा पुत्रं दशरथस्तदा} %6-122-11

\twolineshloka
{आरोप्याङ्के महाबाहुर्वरासनगतः प्रभुः}
{बाहुभ्यां सम्परिष्वज्य ततो वाक्यं समाददे} %6-122-12

\twolineshloka
{न मे स्वर्गो बहु मतः सम्मानश्च सुरर्षभैः}
{त्वया राम विहीनस्य सत्यं प्रतिशृणोमि ते} %6-122-13

\twolineshloka
{अद्य त्वां निहतामित्रं दृष्ट्वा सम्पूर्णमानसम्}
{निस्तीर्णवनवासं च प्रीतिरासीत् परा मम} %6-122-14

\twolineshloka
{कैकेय्या यानि चोक्तानि वाक्यानि वदतां वर}
{तव प्रव्राजनार्थानि स्थितानि हृदये मम} %6-122-15

\twolineshloka
{त्वां तु दृष्ट्वा कुशलिनं परिष्वज्य सलक्ष्मणम्}
{अद्य दुःखाद् विमुक्तोऽस्मि नीहारादिव भास्करः} %6-122-16

\twolineshloka
{तारितोऽहं त्वया पुत्र सुपुत्रेण महात्मना}
{अष्टावक्रेण धर्मात्मा कहोलो ब्राह्मणो यथा} %6-122-17

\twolineshloka
{इदानीं च विजानामि यथा सौम्य सुरेश्वरैः}
{वधार्थं रावणस्येह विहितं पुरुषोत्तमम्} %6-122-18

\twolineshloka
{सिद्धार्था खलु कौसल्या या त्वां राम गृहं गतम्}
{वनान्निवृत्तं संहृष्टा द्रक्ष्यते शत्रुसूदनम्} %6-122-19

\twolineshloka
{सिद्धार्थाः खलु ते राम नरा ये त्वां पुरीं गतम्}
{राज्ये चैवाभिषिक्तं च द्रक्ष्यन्ते वसुधाधिपम्} %6-122-20

\twolineshloka
{अनुरक्तेन बलिना शुचिना धर्मचारिणा}
{इच्छेयं त्वामहं द्रष्टुं भरतेन समागतम्} %6-122-21

\twolineshloka
{चतुर्दश समाः सौम्य वने निर्यातितास्त्वया}
{वसता सीतया सार्धं मत्प्रीत्या लक्ष्मणेन च} %6-122-22

\twolineshloka
{निवृत्तवनवासोऽसि प्रतिज्ञा पूरिता त्वया}
{रावणं च रणे हत्वा देवताः परितोषिताः} %6-122-23

\twolineshloka
{कृतं कर्म यशः श्लाघ्यं प्राप्तं ते शत्रुसूदन}
{भ्रातृभिः सह राज्यस्थो दीर्घमायुरवाप्नुहि} %6-122-24

\twolineshloka
{इति ब्रुवाणं राजानं रामः प्राञ्जलिरब्रवीत्}
{कुरु प्रसादं धर्मज्ञ कैकय्या भरतस्य च} %6-122-25

\twolineshloka
{सपुत्रां त्वां त्यजामीति यदुक्ता कैकयी त्वया}
{स शापः कैकयीं घोरः सपुत्रां न स्पृशेत् प्रभो} %6-122-26

\twolineshloka
{तथेति स महाराजो राममुक्त्वा कृताञ्जलिम्}
{लक्ष्मणं च परिष्वज्य पुनर्वाक्यमुवाच ह} %6-122-27

\twolineshloka
{रामं शुश्रूषता भक्त्या वैदेह्या सह सीतया}
{कृता मम महाप्रीतिः प्राप्तं धर्मफलं च ते} %6-122-28

\twolineshloka
{धर्मं प्राप्स्यसि धर्मज्ञ यशश्च विपुलं भुवि}
{रामे प्रसन्ने स्वर्गं च महिमानं तथोत्तमम्} %6-122-29

\twolineshloka
{रामं शुश्रूष भद्रं ते सुमित्रानन्दवर्धन}
{रामः सर्वस्य लोकस्य हितेष्वभिरतः सदा} %6-122-30

\twolineshloka
{एते सेन्द्रास्त्रयो लोकाः सिद्धाश्च परमर्षयः}
{अभिवाद्य महात्मानमर्चन्ति पुरुषोत्तमम्} %6-122-31

\twolineshloka
{एतत् तदुक्तमव्यक्तमक्षरं ब्रह्मसम्मितम्}
{देवानां हृदयं सौम्य गुह्यं रामः परंतपः} %6-122-32

\twolineshloka
{अवाप्तधर्माचरणं यशश्च विपुलं त्वया}
{एवं शुश्रूषताव्यग्रं वैदेह्या सह सीतया} %6-122-33

\twolineshloka
{इत्युक्त्वा लक्ष्मणं राजा स्नुषां बद्धाञ्जलिं स्थिताम्}
{पुत्रीत्याभाष्य मधुरं शनैरेनामुवाच ह} %6-122-34

\twolineshloka
{कर्तव्यो न तु वैदेहि मन्युस्त्यागमिमं प्रति}
{रामेणेदं विशुद्ध्यर्थं कृतं वै त्वद्धितैषिणा} %6-122-35

\twolineshloka
{सुदुष्करमिदं पुत्रि तव चारित्रलक्षणम्}
{कृतं यत् तेऽन्यनारीणां यशो ह्यभिभविष्यति} %6-122-36

\twolineshloka
{न त्वं कामं समाधेया भर्तृशुश्रूषणं प्रति}
{अवश्यं तु मया वाच्यमेष ते दैवतं परम्} %6-122-37

\twolineshloka
{इति प्रतिसमादिश्य पुत्रौ सीतां च राघवः}
{इन्द्रलोकं विमानेन ययौ दशरथो नृपः} %6-122-38

\twolineshloka
{विमानमास्थाय महानुभावः श्रिया च संहृष्टतनुर्नृपोत्तमः}
{आमन्त्र्य पुत्रौ सह सीतया च जगाम देवप्रवरस्य लोकम्} %6-122-39


॥इत्यार्षे श्रीमद्रामायणे वाल्मीकीये आदिकाव्ये युद्धकाण्डे दशरथप्रतिसमादेशः नाम द्वाविंशत्यधिकशततमः सर्गः ॥६-१२२॥
