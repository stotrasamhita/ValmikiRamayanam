\sect{सप्तनवतितमः सर्गः — विरूपाक्षवधः}

\twolineshloka
{तथा तैः कृत्तगात्रैस्तु दशग्रीवेण मार्गणैः}
{बभूव वसुधा तत्र प्रकीर्णा हरिभिस्तदा} %6-97-1

\twolineshloka
{रावणस्याप्रसह्यं तं शरसम्पातमेकतः}
{न शेकुः सहितुं दीप्तं पतङ्गा ज्वलनं यथा} %6-97-2

\twolineshloka
{तेऽर्दिता निशितैर्बाणैः क्रोशन्तो विप्रदुद्रुवुः}
{पावकार्चिः समाविष्टा दह्यमाना यथा गजाः} %6-97-3

\twolineshloka
{प्लवंगानामनीकानि महाभ्राणीव मारुतः}
{संययौ समरे तस्मिन् विधमन् रावणः शरैः} %6-97-4

\twolineshloka
{कदनं तरसा कृत्वा राक्षसेन्द्रो वनौकसाम्}
{आससाद ततो युद्धे त्वरितं राघवं रणे} %6-97-5

\twolineshloka
{सुग्रीवस्तान् कपीन् दृष्ट्वा भग्नान् विद्रावितान् रणे}
{गुल्मे सुषेणं निक्षिप्य चक्रे युद्धे द्रुतं मनः} %6-97-6

\twolineshloka
{आत्मनः सदृशं वीरं स तं निक्षिप्य वानरम्}
{सुग्रीवोऽभिमुखं शत्रुं प्रतस्थे पादपायुधः} %6-97-7

\twolineshloka
{पार्श्वतः पृष्ठतश्चास्य सर्वे वानरयूथपाः}
{अनुजग्मुर्महाशैलान् विविधांश्च वनस्पतीन्} %6-97-8

\twolineshloka
{ननर्द युधि सुग्रीवः स्वरेण महता महान्}
{पोथयन् विविधांश्चान्यान् ममन्थोत्तमराक्षसान्} %6-97-9

\twolineshloka
{ममर्द च महाकायो राक्षसान् वानरेश्वरः}
{युगान्तसमये वायुः प्रवृद्धानगमानिव} %6-97-10

\twolineshloka
{राक्षसानामनीकेषु शैलवर्षं ववर्ष ह}
{अश्मवर्षं यथा मेघः पक्षिसङ्घेषु कानने} %6-97-11

\twolineshloka
{कपिराजविमुक्तैस्तैः शैलवर्षैस्तु राक्षसाः}
{विकीर्णशिरसः पेतुर्विकीर्णा इव पर्वताः} %6-97-12

\twolineshloka
{अथ संक्षीयमाणेषु राक्षसेषु समन्ततः}
{सुग्रीवेण प्रभग्नेषु नदत्सु च पतत्सु च} %6-97-13

\twolineshloka
{विरूपाक्षः स्वकं नाम धन्वी विश्राव्य राक्षसः}
{रथादाप्लुत्य दुर्धर्षो गजस्कन्धमुपारुहत्} %6-97-14

\twolineshloka
{स तं द्विपमथारुह्य विरूपाक्षो महाबलः}
{ननर्द भीमनिर्ह्रादं वानरानभ्यधावत} %6-97-15

\twolineshloka
{सुग्रीवे स शरान् घोरान् विससर्ज चमूमुखे}
{स्थापयामास चोद्विग्नान् राक्षसान् सम्प्रहर्षयन्} %6-97-16

\twolineshloka
{सोऽतिविद्धः शितैर्बाणैः कपीन्द्रस्तेन रक्षसा}
{चुक्रोश च महाक्रोधो वधे चास्य मनो दधे} %6-97-17

\twolineshloka
{ततः पादपमुद्धृत्य शूरः सम्प्रधनो हरिः}
{अभिपत्य जघानास्य प्रमुखे तं महागजम्} %6-97-18

\twolineshloka
{स तु प्रहाराभिहतः सुग्रीवेण महागजः}
{अपासर्पद् धनुर्मात्रं निषसाद ननाद च} %6-97-19

\twolineshloka
{गजात् तु मथितात् तूर्णमपक्रम्य स वीर्यवान्}
{राक्षसोऽभिमुखः शत्रुं प्रत्युद्गम्य ततः कपिम्} %6-97-20

\twolineshloka
{आर्षभं चर्म खड्गं च प्रगृह्य लघुविक्रमः}
{भर्त्सयन्निव सुग्रीवमाससाद व्यवस्थितम्} %6-97-21

\twolineshloka
{स हि तस्याभिसंक्रुद्धः प्रगृह्य विपुलां शिलाम्}
{विरूपाक्षस्य चिक्षेप सुग्रीवो जलदोपमाम्} %6-97-22

\twolineshloka
{स तां शिलामापतन्तीं दृष्ट्वा राक्षसपुंगवः}
{अपक्रम्य सुविक्रान्तः खड्गेन प्राहरत् तदा} %6-97-23

\twolineshloka
{तेन खड्गप्रहारेण रक्षसा बलिना हतः}
{मुहूर्तमभवद् भूमौ विसंज्ञ इव वानरः} %6-97-24

\twolineshloka
{सहसा स तदोत्पत्य राक्षसस्य महाहवे}
{मुष्टिं संवर्त्य वेगेन पातयामास वक्षसि} %6-97-25

\twolineshloka
{मुष्टिप्रहाराभिहतो विरूपाक्षो निशाचरः}
{तेन खड्गेन संक्रुद्धः सुग्रीवस्य चमूमुखे} %6-97-26

\twolineshloka
{कवचं पातयामास पद्भ्यामभिहतोऽपतत्}
{स समुत्थाय पतितः कपिस्तस्य व्यसर्जयत्} %6-97-27

\twolineshloka
{तलप्रहारमशनेः समानं भीमनिःस्वनम्}
{तलप्रहारं तद् रक्षः सुग्रीवेण समुद्यतम्} %6-97-28

\twolineshloka
{नैपुण्यान्मोचयित्वैनं मुष्टिनोरसि ताडयत्}
{ततस्तु संक्रुद्धतरः सुग्रीवो वानरेश्वरः} %6-97-29

\twolineshloka
{मोक्षितं चात्मनो दृष्ट्वा प्रहारं तेन रक्षसा}
{स ददर्शान्तरं तस्य विरूपाक्षस्य वानरः} %6-97-30

\twolineshloka
{ततोऽन्यं पातयत् क्रोधाच्छङ्खदेशे महातलम्}
{महेन्द्राशनिकल्पेन तलेनाभिहतः क्षितौ} %6-97-31

\twolineshloka
{पपात रुधिरक्लिन्नः शोणितं हि समुद्गिरन्}
{स्रोतोभ्यस्तु विरूपाक्षो जलं प्रस्रवणादिव} %6-97-32

\twolineshloka
{विवृत्तनयनं क्रोधात् सफेनं रुधिराप्लुतम्}
{ददृशुस्ते विरूपाक्षं विरूपाक्षतरं कृतम्} %6-97-33

\twolineshloka
{स्फुरन्तं परिवर्तन्तं पार्श्वेन रुधिरोक्षितम्}
{करुणं च विनर्दन्तं ददृशुः कपयो रिपुम्} %6-97-34

\twolineshloka
{तथा तु तौ संयति सम्प्रयुक्तौ तरस्विनौ वानरराक्षसानाम्}
{बलार्णवौ सस्वनतुश्च भीमौ महार्णवौ द्वाविव भिन्नसेतू} %6-97-35

\twolineshloka
{विनाशितं प्रेक्ष्य विरूपनेत्रं महाबलं तं हरिपार्थिवेन}
{बलं समेतं कपिराक्षसानामुद्वृत्तगङ्गाप्रतिमं बभूव} %6-97-36


॥इत्यार्षे श्रीमद्रामायणे वाल्मीकीये आदिकाव्ये युद्धकाण्डे विरूपाक्षवधः नाम सप्तनवतितमः सर्गः ॥६-९७॥
