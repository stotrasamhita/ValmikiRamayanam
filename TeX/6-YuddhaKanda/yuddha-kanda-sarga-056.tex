\sect{षट्पञ्चाशः सर्गः — अकम्पनवधः}

\twolineshloka
{तद् दृष्ट्वा सुमहत् कर्म कृतं वानरसत्तमैः}
{क्रोधमाहारयामास युधि तीव्रमकम्पनः} %6-56-1

\twolineshloka
{क्रोधमूिर्च्छत तरूपस्तु धुन्वन् परमकार्मुकम्}
{दृष्ट्वा तु कर्म शत्रूणां सारथिं वाक्यमब्रवीत्} %6-56-2

\twolineshloka
{तत्रैव तावत् त्वरितो रथं प्रापय सारथे}
{एते च बलिनो घ्नन्ति सुबहून् राक्षसान् रणे} %6-56-3

\twolineshloka
{एते च बलवन्तो वा भीमकोपाश्च वानराः}
{द्रुमशैलप्रहरणास्तिष्ठन्ति प्रमुखे मम} %6-56-4

\twolineshloka
{एतान् निहन्तुमिच्छामि समरश्लाघिनो ह्यहम्}
{एतैः प्रमथितं सर्वं रक्षसां दृश्यते बलम्} %6-56-5

\twolineshloka
{ततः प्रचलिताश्वेन रथेन रथिनां वरः}
{हरीनभ्यपतद् दूराच्छरजालैरकम्पनः} %6-56-6

\twolineshloka
{न स्थातुं वानराः शेकुः किं पुनर्योद्धुमाहवे}
{अकम्पनशरैर्भग्नाः सर्व एवाभिदुद्रुवुः} %6-56-7

\twolineshloka
{तान् मृत्युवशमापन्नानकम्पनशरानुगान्}
{समीक्ष्य हनुमान् ज्ञातीनुपतस्थे महाबलः} %6-56-8

\twolineshloka
{तं महाप्लवगं दृष्ट्वा सर्वे ते प्लवगर्षभाः}
{समेत्य समरे वीराः संहृष्टाः पर्यवारयन्} %6-56-9

\twolineshloka
{व्यवस्थितं हनूमन्तं ते दृष्ट्वा प्लवगर्षभाः}
{बभूवुर्बलवन्तो हि बलवन्तमुपाश्रिताः} %6-56-10

\twolineshloka
{अकम्पनस्तु शैलाभं हनूमन्तमवस्थितम्}
{महेन्द्र इव धाराभिः शरैरभिववर्ष ह} %6-56-11

\twolineshloka
{अचिन्तयित्वा बाणौघान् शरीरे पातितान् कपिः}
{अकम्पनवधार्थाय मनो दध्रे महाबलः} %6-56-12

\twolineshloka
{स प्रहस्य महातेजा हनूमान् मारुतात्मजः}
{अभिदुद्राव तद्रक्षः कम्पयन्निव मेदिनीम्} %6-56-13

\twolineshloka
{तस्याथ नर्दमानस्य दीप्यमानस्य तेजसा}
{बभूव रूपं दुर्धर्षं दीप्तस्येव विभावसोः} %6-56-14

\twolineshloka
{आत्मानं त्वप्रहरणं ज्ञात्वा क्रोधसमन्वितः}
{शैलमुत्पाटयामास वेगेन हरिपुङ्गवः} %6-56-15

\twolineshloka
{गृहीत्वा सुमहाशैलं पाणिनैकेन मारुतिः}
{स विनद्य महानादं भ्रामयामास वीर्यवान्} %6-56-16

\twolineshloka
{ततस्तमभिदुद्राव राक्षसेन्द्रमकम्पनम्}
{पुरा हि नमुचिं संख्ये वज्रेणेव पुरंदरः} %6-56-17

\twolineshloka
{अकम्पनस्तु तद् दृष्ट्वा गिरिशृङ्गं समुद्यतम्}
{दूरादेव महाबाणैरर्धचन्द्रैर्व्यदारयत्} %6-56-18

\twolineshloka
{तं पर्वताग्रमाकाशे रक्षोबाणविदारितम्}
{विकीर्णं पतितं दृष्ट्वा हनूमान् क्रोधमूिर्च्छतः} %6-56-19

\twolineshloka
{सोऽश्वकर्णं समासाद्य रोषदर्पान्वितो हरिः}
{तूर्णमुत्पाटयामास महागिरिमिवोच्छ्रितम्} %6-56-20

\twolineshloka
{तं गृहीत्वा महास्कन्धं सोऽश्वकर्णं महाद्युतिः}
{प्रगृह्य परया प्रीत्या भ्रामयामास संयुगे} %6-56-21

\twolineshloka
{प्रधावन्नुरुवेगेन बभञ्ज तरसा द्रुमान्}
{हनूमान् परमक्रुद्धश्चरणैर्दारयन् महीम्} %6-56-22

\twolineshloka
{गजांश्च सगजारोहान् सरथान् रथिनस्तथा}
{जघान हनुमान् धीमान् राक्षसांश्च पदातिगान्} %6-56-23

\twolineshloka
{तमन्तकमिव क्रुद्धं सद्रुमं प्राणहारिणम्}
{हनूमन्तमभिप्रेक्ष्य राक्षसा विप्रदुद्रुवुः} %6-56-24

\twolineshloka
{तमापतन्तं संक्रुद्धं राक्षसानां भयावहम्}
{ददर्शाकम्पनो वीरश्चुक्षोभ च ननाद च} %6-56-25

\twolineshloka
{स चतुर्दशभिर्बाणैर्निशितैर्देहदारणैः}
{निर्बिभेद महावीर्यं हनूमन्तमकम्पनः} %6-56-26

\twolineshloka
{स तथा विप्रकीर्णस्तु नाराचैः शितशक्तिभिः}
{हनूमान् ददृशे वीरः प्ररूढ इव सानुमान्} %6-56-27

\twolineshloka
{विरराज महावीर्यो महाकायो महाबलः}
{पुष्पिताशोकसंकाशो विधूम इव पावकः} %6-56-28

\twolineshloka
{ततोऽन्यं वृक्षमुत्पाट्य कृत्वा वेगमनुत्तमम्}
{शिरस्याभिजघानाशु राक्षसेन्द्रमकम्पनम्} %6-56-29

\twolineshloka
{स वृक्षेण हतस्तेन सक्रोधेन महात्मना}
{राक्षसो वानरेन्द्रेण पपात च ममार च} %6-56-30

\twolineshloka
{तं दृष्ट्वा निहतं भूमौ राक्षसेन्द्रमकम्पनम्}
{व्यथिता राक्षसाः सर्वे क्षितिकम्प इव द्रुमाः} %6-56-31

\twolineshloka
{त्यक्तप्रहरणाः सर्वे राक्षसास्ते पराजिताः}
{लङ्कामभिययुस्त्रासाद् वानरैस्तैरभिद्रुताः} %6-56-32

\twolineshloka
{ते मुक्तकेशाः सम्भ्रान्ता भग्नमानाः पराजिताः}
{भयाच्छ्रमजलैरङ्गैः प्रस्रवद्भिर्विदुद्रुवुः} %6-56-33

\twolineshloka
{अन्योन्यं ये प्रमथ्नन्तो विविशुर्नगरं भयात्}
{पृष्ठतस्ते तु सम्मूढाः प्रेक्षमाणा मुहुर्मुहुः} %6-56-34

\twolineshloka
{तेषु लङ्कां प्रविष्टेषु राक्षसेषु महाबलाः}
{समेत्य हरयः सर्वे हनूमन्तमपूजयन्} %6-56-35

\twolineshloka
{सोऽपि प्रवृद्धस्तान् सर्वान् हरीन् सम्प्रत्यपूजयत्}
{हनूमान् सत्त्वसम्पन्नो यथार्हमनुकूलतः} %6-56-36

\twolineshloka
{विनेदुश्च यथाप्राणं हरयो जितकाशिनः}
{चकृषुश्च पुनस्तत्र सप्राणानेव राक्षसान्} %6-56-37

\twolineshloka
{स वीरशोभामभजन्महाकपिः समेत्य रक्षांसि निहत्य मारुतिः}
{महासुरं भीमममित्रनाशनं विष्णुर्यथैवोरुबलं चमूमुखे} %6-56-38

\twolineshloka
{अपूजयन् देवगणास्तदाकपिं स्वयं च रामोऽतिबलश्च लक्ष्मणः}
{तथैव सुग्रीवमुखाः प्लवंगमा विभीषणश्चैव महाबलस्तदा} %6-56-39


॥इत्यार्षे श्रीमद्रामायणे वाल्मीकीये आदिकाव्ये युद्धकाण्डे अकम्पनवधः नाम षट्पञ्चाशः सर्गः ॥६-५६॥
