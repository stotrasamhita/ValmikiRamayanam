\sect{पञ्चदशाधिकशततमः सर्गः — विभीषणाभिषेकः}

\twolineshloka
{ते रावणवधं दृष्ट्वा देवगन्धर्वदानवाः}
{जग्मुः स्वैः स्वैर्विमानैस्ते कथयन्तः शुभाः कथाः} %6-115-1

\twolineshloka
{रावणस्य वधं घोरं राघवस्य पराक्रमम्}
{सुयुद्धं वानराणां च सुग्रीवस्य च मन्त्रितम्} %6-115-2

\twolineshloka
{अनुरागं च वीर्यं च मारुतेर्लक्ष्मणस्य च}
{पतिव्रतात्वं सीताया हनूमति पराक्रमम्} %6-115-3

\twolineshloka
{कथयन्तो महाभागा जग्मुर्हृष्टा यथागतम्}
{राघवस्तु रथं दिव्यमिन्द्रदत्तं शिखिप्रभम्} %6-115-4

\twolineshloka
{अनुज्ञाप्य महाबाहुर्मातलिं प्रत्यपूजयत्}
{राघवेणाभ्यनुज्ञातो मातलिः शक्रसारथिः} %6-115-5

\twolineshloka
{दिव्यं तं रथमास्थाय दिवमेवोत्पपात ह}
{तस्मिंस्तु दिवमारूढे सरथे रथिनां वरः} %6-115-6

\twolineshloka
{राघवः परमप्रीतः सुग्रीवं परिषस्वजे}
{परिष्वज्य च सुग्रीवं लक्ष्मणेनाभिवादितः} %6-115-7

\twolineshloka
{पूज्यमानो हरिगणैराजगाम बलालयम्}
{अथोवाच स काकुत्स्थः समीपपरिवर्तिनम्} %6-115-8

\twolineshloka
{सौमित्रिं सत्त्वसम्पन्नं लक्ष्मणं दीप्ततेजसम्}
{विभीषणमिमं सौम्य लङ्कायामभिषेचय} %6-115-9

\twolineshloka
{अनुरक्तं च भक्तं च तथा पूर्वोपकारिणम्}
{एष मे परमः कामो यदिमं रावणानुजम्} %6-115-10

\twolineshloka
{लङ्कायां सौम्य पश्येयमभिषिक्तं विभीषणम्}
{एवमुक्तस्तु सौमित्री राघवेण महात्मना} %6-115-11

\twolineshloka
{तथेत्युक्त्वा सुसंहृष्टः सौवर्णं घटमाददे}
{तं घटं वानरेन्द्राणां हस्ते दत्त्वा मनोजवान्} %6-115-12

\twolineshloka
{व्यादिदेश महासत्त्वान् समुद्रसलिलं तदा}
{अतिशीघ्रं ततो गत्वा वानरास्ते मनोजवाः} %6-115-13

\twolineshloka
{आगतास्तु जलं गृह्य समुद्राद् वानरोत्तमाः}
{ततस्त्वेकं घटं गृह्य संस्थाप्य परमासने} %6-115-14

\twolineshloka
{घटेन तेन सौमित्रिरभ्यषिञ्चद् विभीषणम्}
{लङ्कायां रक्षसां मध्ये राजानं रामशासनात्} %6-115-15

\twolineshloka
{विधिना मन्त्रदृष्टेन सुहृद्गणसमावृतम्}
{अभ्यषिञ्चस्तदा सर्वे राक्षसा वानरास्तथा} %6-115-16

\twolineshloka
{प्रहर्षमतुलं गत्वा तुष्टुवू राममेव हि}
{तस्यामात्या जहृषिरे भक्ता ये चास्य राक्षसाः} %6-115-17

\twolineshloka
{दृष्ट्वाभिषिक्तं लङ्कायां राक्षसेन्द्रं विभीषणम्}
{राघवः परमां प्रीतिं जगाम सहलक्ष्मणः} %6-115-18

\twolineshloka
{स तद् राज्यं महत् प्राप्य रामदत्तं विभीषणः}
{सान्त्वयित्वा प्रकृतयस्ततो राममुपागमत्} %6-115-19

\twolineshloka
{दध्यक्षतान् मोदकांश्च लाजाः सुमनसस्तथा}
{आजह्रुरथ संहृष्टाः पौरास्तस्मै निशाचराः} %6-115-20

\twolineshloka
{स तान् गृहीत्वा दुर्धर्षो राघवाय न्यवेदयत्}
{मङ्गल्यं मङ्गलं सर्वं लक्ष्मणाय च वीर्यवान्} %6-115-21

\twolineshloka
{कृतकार्यं समृद्धार्थं दृष्ट्वा रामो विभीषणम्}
{प्रतिजग्राह तत् सर्वं तस्यैव प्रतिकाम्यया} %6-115-22

\twolineshloka
{ततः शैलोपमं वीरं प्राञ्जलिं प्रणतं स्थितम्}
{उवाचेदं वचो रामो हनूमन्तं प्लवङ्गमम्} %6-115-23

\twolineshloka
{अनुज्ञाप्य महाराजमिमं सौम्य विभीषणम्}
{प्रविश्य नगरीं लङ्कां कौशलं ब्रूहि मैथिलीम्} %6-115-24

\twolineshloka
{वैदेह्यै मां च कुशलं सुग्रीवं च सलक्ष्मणम्}
{आचक्ष्व वदतां श्रेष्ठ रावणं च हतं रणे} %6-115-25

\twolineshloka
{प्रियमेतदिहाख्याहि वैदेह्यास्त्वं हरीश्वर}
{प्रतिगृह्य तु संदेशमुपावर्तितुमर्हसि} %6-115-26


॥इत्यार्षे श्रीमद्रामायणे वाल्मीकीये आदिकाव्ये युद्धकाण्डे विभीषणाभिषेकः नाम पञ्चदशाधिकशततमः सर्गः ॥६-११५॥
