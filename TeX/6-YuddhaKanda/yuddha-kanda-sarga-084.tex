\sect{चतुरशीतितमः सर्गः — इन्द्रजिन्मायाविवरणम्}

\twolineshloka
{राममाश्वासमाने तु लक्ष्मणे भ्रातृवत्सले}
{निक्षिप्य गुल्मान् स्वस्थाने तत्रागच्छद् विभीषणः} %6-84-1

\twolineshloka
{नानाप्रहरणैर्वीरैश्चतुर्भिरभिसंवृतः}
{नीलाञ्जनचयाकारैर्मातंगैरिव यूथपैः} %6-84-2

\twolineshloka
{सोऽभिगम्य महात्मानं राघवं शोकलालसम्}
{वानरांश्चापि ददृशे बाष्पपर्याकुलेक्षणान्} %6-84-3

\twolineshloka
{राघवं च महात्मानमिक्ष्वाकुकुलनन्दनम्}
{ददर्श मोहमापन्नं लक्ष्मणस्याङ्कमाश्रितम्} %6-84-4

\twolineshloka
{व्रीडितं शोकसंतप्तं दृष्ट्वा रामं विभीषणः}
{अन्तर्दुःखेन दीनात्मा किमेतदिति सोऽब्रवीत्} %6-84-5

\twolineshloka
{विभीषणमुखं दृष्ट्वा सुग्रीवं तांश्च वानरान्}
{लक्ष्मणोवाच मन्दार्थमिदं बाष्पपरिप्लुतः} %6-84-6

\twolineshloka
{हता इन्द्रजिता सीता इति श्रुत्वैव राघवः}
{हनूमद्वचनात् सौम्य ततो मोहमुपाश्रितः} %6-84-7

\twolineshloka
{कथयन्तं तु सौमित्रिं संनिवार्य विभीषणः}
{पुष्कलार्थमिदं वाक्यं विसंज्ञं राममब्रवीत्} %6-84-8

\twolineshloka
{मनुजेन्द्रार्तरूपेण यदुक्तस्त्वं हनूमता}
{तदयुक्तमहं मन्ये सागरस्येव शोषणम्} %6-84-9

\twolineshloka
{अभिप्रायं तु जानामि रावणस्य दुरात्मनः}
{सीतां प्रति महाबाहो न च घातं करिष्यति} %6-84-10

\twolineshloka
{याच्यमानः सुबहुशो मया हितचिकीर्षुणा}
{वैदेहीमुत्सृजस्वेति न च तत् कृतवान् वचः} %6-84-11

\twolineshloka
{नैव साम्ना न दानेन न भेदेन कुतो युधा}
{सा द्रष्टुमपि शक्येत नैव चान्येन केनचित्} %6-84-12

\twolineshloka
{वानरान् मोहयित्वा तु प्रतियातः स राक्षसः}
{मायामयीं महाबाहो तां विद्धि जनकात्मजाम्} %6-84-13

\twolineshloka
{चैत्यं निकुम्भिलामद्य प्राप्य होमं करिष्यति}
{हुतवानुपयातो हि देवैरपि सवासवैः} %6-84-14

\twolineshloka
{दुराधर्षो भवत्येष संग्रामे रावणात्मजः}
{तेन मोहयता नूनमेषा माया प्रयोजिता} %6-84-15

\twolineshloka
{विघ्नमन्विच्छता तत्र वानराणां पराक्रमे}
{ससैन्यास्तत्र गच्छामो यावत्तन्न समाप्यते} %6-84-16

\twolineshloka
{त्यजैनं नरशार्दूल मिथ्या संतापमागतम्}
{सीदते हि बलं सर्वं दृष्ट्वा त्वां शोककर्शितम्} %6-84-17

\twolineshloka
{इह त्वं स्वस्थहृदयस्तिष्ठ सत्त्वसमुच्छ्रितः}
{लक्ष्मणं प्रेषयास्माभिः सह सैन्यानुकर्षिभिः} %6-84-18

\twolineshloka
{एष तं नरशार्दूलो रावणिं निशितैः शरैः}
{त्याजयिष्यति तत्कर्म ततो वध्यो भविष्यति} %6-84-19

\twolineshloka
{तस्यैते निशितास्तीक्ष्णाः पत्रिपत्राङ्गवाजिनः}
{पतत्त्रिण इवासौम्याः शराः पास्यन्ति शोणितम्} %6-84-20

\twolineshloka
{तत् संदिश महाबाहो लक्ष्मणं शुभलक्षणम्}
{राक्षसस्य विनाशाय वज्रं वज्रधरो यथा} %6-84-21

\twolineshloka
{मनुजवर न कालविप्रकर्षो रिपुनिधनं प्रति यत्क्षमोऽद्य कर्तुम्}
{त्वमतिसृज रिपोर्वधाय वज्रं दिविजरिपोर्मथने यथा महेन्द्रः} %6-84-22

\twolineshloka
{समाप्तकर्मा हि स राक्षसर्षभो भवत्यदृश्यः समरे सुरासुरैः}
{युयुत्सता तेन समाप्तकर्मणा भवेत् सुराणामपि संशयो महान्} %6-84-23


॥इत्यार्षे श्रीमद्रामायणे वाल्मीकीये आदिकाव्ये युद्धकाण्डे इन्द्रजिन्मायाविवरणम् नाम चतुरशीतितमः सर्गः ॥६-८४॥
