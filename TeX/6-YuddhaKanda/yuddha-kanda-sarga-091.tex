\sect{एकनवतितमः सर्गः — रावणिवधः}

\twolineshloka
{स हताश्वो महातेजा भूमौ तिष्ठन् निशाचरः}
{इन्द्रजित् परमक्रुद्धः सम्प्रजज्वाल तेजसा} %6-91-1

\twolineshloka
{तौ धन्विनौ जिघांसन्तावन्योन्यमिषुभिर्भृशम्}
{विजयेनाभिनिष्क्रान्तौ वने गजवृषाविव} %6-91-2

\twolineshloka
{निबर्हयन्तश्चान्योन्यं ते राक्षसवनौकसः}
{भर्तारं न जहुर्युद्धे सम्पतन्तस्ततस्ततः} %6-91-3

\twolineshloka
{ततस्तान् राक्षसान् सर्वान् हर्षयन् रावणात्मजः}
{स्तुन्वानो हर्षमाणश्च इदं वचनमब्रवीत्} %6-91-4

\twolineshloka
{तमसा बहुलेनेमाः संसक्ताः सर्वतो दिशः}
{नेह विज्ञायते स्वो वा परो वा राक्षसोत्तमाः} %6-91-5

\twolineshloka
{धृष्टं भवन्तो युध्यन्तु हरीणां मोहनाय वै}
{अहं तु रथमास्थाय आगमिष्यामि संयुगे} %6-91-6

\twolineshloka
{तथा भवन्तः कुर्वन्तु यथेमे हि वनौकसः}
{न युध्येयुर्महात्मानः प्रविष्टे नगरं मयि} %6-91-7

\twolineshloka
{इत्युक्त्वा रावणसुतो वञ्चयित्वा वनौकसः}
{प्रविवेश पुरीं लङ्कां रथहेतोरमित्रहा} %6-91-8

\twolineshloka
{स रथं भूषयित्वाथ रुचिरं हेमभूषितम्}
{प्रासासिशरसंयुक्तं युक्तं परमवाजिभिः} %6-91-9

\twolineshloka
{अधिष्ठितं हयज्ञेन सूतेनाप्तोपदेशिना}
{आरुरोह महातेजा रावणिः समितिञ्जयः} %6-91-10

\twolineshloka
{स राक्षसगणैर्मुख्यैर्वृतो मन्दोदरीसुतः}
{निर्ययौ नगराद् वीरः कृतान्तबलचोदितः} %6-91-11

\twolineshloka
{सोऽभिनिष्क्रम्य नगरादिन्द्रजित् परमौजसा}
{अभ्ययाज्जवनैरश्वैर्लक्ष्मणं सविभीषणम्} %6-91-12

\twolineshloka
{ततो रथस्थमालोक्य सौमित्री रावणात्मजम्}
{वानराश्च महावीर्या राक्षसश्च विभीषणः} %6-91-13

\twolineshloka
{विस्मयं परमं जग्मुर्लाघवात् तस्य धीमतः}
{रावणिश्चापि सङ्क्रुद्धो रणे वानरयूथपान्} %6-91-14

\twolineshloka
{पातयामास बाणौघैः शतशोऽथ सहस्रशः}
{स मण्डलीकृतधनू रावणिः समितिञ्जयः} %6-91-15

\twolineshloka
{हरीनभ्यहनत् क्रुद्धः परं लाघवमास्थितः}
{ते वध्यमाना हरयो नाराचैर्भीमविक्रमाः} %6-91-16

\threelineshloka
{सौमित्रिं शरणं प्राप्ताः प्रजापतिमिव प्रजाः}
{ततः समरकोपेन ज्वलितो रघुनन्दनः}
{चिच्छेद कार्मुकं तस्य दर्शयन् पाणिलाघवम्} %6-91-17

\twolineshloka
{सोऽन्यत्कार्मुकमादाय सज्यं चक्रे त्वरन्निव}
{तदप्यस्य त्रिभिर्बाणैर्लक्ष्मणो निरकृन्तत} %6-91-18

\twolineshloka
{अथैनं छिन्नधन्वानमाशीविषविषोपमैः}
{विव्याधोरसि सौमित्री रावणिं पञ्चभिः शरैः} %6-91-19

\twolineshloka
{ते तस्य कायं निर्भिद्य महाकार्मुकनिःसृताः}
{निपेतुर्धरणीं बाणा रक्ता इव महोरगाः} %6-91-20

\twolineshloka
{स च्छिन्नधन्वा रुधिरं वमन् वक्त्रेण रावणिः}
{जग्राह कार्मुकश्रेष्ठं दृढज्यं बलवत्तरम्} %6-91-21

\twolineshloka
{स लक्ष्मणं समुद्दिश्य परं लाघवमास्थितः}
{ववर्ष शरवर्षाणि वर्षाणीव पुरन्दरः} %6-91-22

\twolineshloka
{मुक्तमिन्द्रजिता तत्तु शरवर्षमरिन्दमः}
{आवारयदसम्भ्रान्तो लक्ष्मणः सुदुरासदम्} %6-91-23

\twolineshloka
{सन्दर्शयामास तदा रावणिं रघुनन्दनः}
{असम्भ्रान्तो महातेजास्तदद्भुतमिवाभवत्} %6-91-24

\threelineshloka
{ततस्तान् राक्षसान् सर्वांस्त्रिभिरेकैकमाहवे}
{अविध्यत् परमक्रुद्धः शीघ्रास्त्रं सम्प्रदर्शयन्}
{राक्षसेन्द्रसुतं चापि बाणौघैः समताडयत्} %6-91-25

\twolineshloka
{सोऽतिविद्धो बलवता शत्रुणा शत्रुघातिना}
{असक्तं प्रेषयामास लक्ष्मणाय बहून् शरान्} %6-91-26

\twolineshloka
{तानप्राप्तान् शितैर्बाणैश्चिच्छेद परवीरहा}
{सारथेरस्य च रणे रथिनो रथसत्तमः} %6-91-27

\twolineshloka
{शिरो जहार धर्मात्मा भल्लेनानतपर्वणा}
{असूतास्ते हयास्तत्र रथमूहुरविक्लवाः} %6-91-28

\twolineshloka
{मण्डलान्यभिधावन्ति तदद्भुतमिवाभवत्}
{अमर्षवशमापन्नः सौमित्रिर्दृढविक्रमः} %6-91-29

\twolineshloka
{प्रत्यविध्यद्धयांस्तस्य शरैर्वित्रासयन् रणे}
{अमर्षमाणस्तत्कर्म रावणस्य सुतो रणे} %6-91-30

\threelineshloka
{विव्याध दशभिर्बाणैः सौमित्रिं तममर्षणम्}
{ते तस्य वज्रप्रतिमाः शराः सर्पविषोपमाः}
{विलयं जग्मुरागत्य कवचं काञ्चनप्रभम्} %6-91-31

\twolineshloka
{अभेद्यकवचं मत्वा लक्ष्मणं रावणात्मजः}
{ललाटे लक्ष्मणं बाणैः सुपुङ्खैस्त्रिभिरिन्द्रजित्} %6-91-32

\twolineshloka
{अविध्यत् परमक्रुद्धः शीघ्रमस्त्रं प्रदर्शयन्}
{तैः पृषत्कैर्ललाटस्थैः शुशुभे रघुनन्दनः} %6-91-33

\twolineshloka
{रणाग्रे समरश्लाघी त्रिशृङ्ग इव पर्वतः}
{स तथाप्यर्दितो बाणै राक्षसेन तदा मृधे} %6-91-34

\twolineshloka
{तमाशु प्रतिविव्याध लक्ष्मणः पञ्चभिः शरैः}
{विकृष्येन्द्रजितो युद्धे वदने शुभकुण्डले} %6-91-35

\twolineshloka
{लक्ष्मणेन्द्रजितौ वीरौ महाबलशरासनौ}
{अन्योन्यं जघ्नतुर्वीरौ विशिखैर्भीमविक्रमौ} %6-91-36

\twolineshloka
{ततः शोणितदिग्धाङ्गौ लक्ष्मणेन्द्रजितावुभौ}
{रणे तौ रेजतुर्वीरौ पुष्पिताविव किंशुकौ} %6-91-37

\twolineshloka
{तौ परस्परमभ्येत्य सर्वगात्रेषु धन्विनौ}
{घोरैर्विव्यधतुर्बाणैः कृतभावावुभौ जये} %6-91-38

\twolineshloka
{ततः समरकोपेन संयुतो रावणात्मजः}
{विभीषणं त्रिभिर्बाणैर्विव्याध वदने शुभे} %6-91-39

\twolineshloka
{अयोमुखैस्त्रिभिर्विद्ध्वा राक्षसेन्द्रं विभीषणम्}
{एकैकेनाभिविव्याध तान् सर्वान् हरियूथपान्} %6-91-40

\twolineshloka
{तस्मै दृढतरं क्रुद्धो जघान गदया हयान्}
{विभीषणो महातेजा रावणेः स दुरात्मनः} %6-91-41

\twolineshloka
{स हताश्वादवप्लुत्य रथान्निहतसारथेः}
{अथ शक्तिं महातेजाः पितृव्याय मुमोच ह} %6-91-42

\twolineshloka
{तामापतन्तीं सम्प्रेक्ष्य सुमित्रानन्दवर्धनः}
{चिच्छेद निशितैर्बाणैर्दशधापातयद् भुवि} %6-91-43

\twolineshloka
{तस्मै दृढधनुः क्रुद्धो हताश्वाय विभीषणः}
{वज्रस्पर्शसमान् पञ्च ससर्जोरसि मार्गणान्} %6-91-44

\twolineshloka
{ते तस्य कायं भित्त्वा तु रुक्मपुङ्खा निमित्तगाः}
{बभूवुर्लोहितादिग्धा रक्ता इव महोरगाः} %6-91-45

\twolineshloka
{स पितृव्यस्य सङ्क्रुद्ध इन्द्रजिच्छरमाददे}
{उत्तमं रक्षसां मध्ये यमदत्तं महाबलः} %6-91-46

\twolineshloka
{तं समीक्ष्य महातेजा महेषुं तेन संहितम्}
{लक्ष्मणोऽप्याददे बाणमन्यद् भीमपराक्रमः} %6-91-47

\twolineshloka
{कुबेरेण स्वयं स्वप्ने यद् दत्तममितात्मना}
{दुर्जयं दुर्विषह्यं च सेन्द्रैरपि सुरासुरैः} %6-91-48

\twolineshloka
{तयोस्तु धनुषी श्रेष्ठे बाहुभिः परिघोपमैः}
{विकृष्यमाणे बलवत् क्रौञ्चाविव चुकूजतुः} %6-91-49

\twolineshloka
{ताभ्यां तु धनुषि श्रेष्ठे संहितौ सायकोत्तमौ}
{विकृष्यमाणौ वीराभ्यां भृशं जज्वलतुः श्रिया} %6-91-50

\twolineshloka
{तौ भासयन्तावाकाशं धनुर्भ्यां विशिखौ च्युतौ}
{मुखेन मुखमाहत्य सन्निपेततुरोजसा} %6-91-51

\twolineshloka
{सन्निपातस्तयोश्चासीच्छरयोर्घोररूपयोः}
{सधूमविस्फुलिङ्गश्च तज्जोऽग्निर्दारुणोऽभवत्} %6-91-52

\twolineshloka
{तौ महाग्रहसङ्काशावन्योन्यं सन्निपत्य च}
{सङ्ग्रामे शतधा यातौ मेदिन्यां चैव पेततुः} %6-91-53

\twolineshloka
{शरौ प्रतिहतौ दृष्ट्वा तावुभौ रणमूर्धनि}
{व्रीडितौ जातरोषौ च लक्ष्मणेन्द्रजितौ तदा} %6-91-54

\twolineshloka
{सुसंरब्धस्तु सौमित्रिरस्त्रं वारुणमाददे}
{रौद्रं महेन्द्रजिद् युद्धेऽप्यसृजद् युधि निष्ठितः} %6-91-55

\threelineshloka
{तेन तद्विहितं शस्त्रं वारुणं परमाद्भुतम्}
{ततः क्रुद्धो महातेजा इन्द्रजित् समितिञ्जयः}
{आग्नेयं सन्दधे दीप्तं स लोकं सङ्क्षिपन्निव} %6-91-56

\twolineshloka
{सौरेणास्त्रेण तद् वीरो लक्ष्मणः पर्यवारयत्}
{अस्त्रं निवारितं दृष्ट्वा रावणिः क्रोधमूर्च्छितः} %6-91-57

\twolineshloka
{आददे निशितं बाणमासुरं शत्रुदारणम्}
{तस्माच्चापाद् विनिष्पेतुर्भास्वराः कूटमुद्गराः} %6-91-58

\twolineshloka
{शूलानि च भुशुण्ड्यश्च गदाः खड्गाः परश्वधाः}
{तद् दृष्ट्वा लक्ष्मणः सङ्ख्ये घोरमस्त्रमथासुरम्} %6-91-59

\twolineshloka
{अवार्यं सर्वभूतानां सर्वशस्त्रविदारणम्}
{माहेश्वरेण द्युतिमांस्तदस्त्रं प्रत्यवारयत्} %6-91-60

\twolineshloka
{तयोः समभवद् युद्धमद्भुतं रोमहर्षणम्}
{गगनस्थानि भूतानि लक्ष्मणं पर्यवारयन्} %6-91-61

\twolineshloka
{भैरवाभिरुते भीमे युद्धे वानररक्षसाम्}
{भूतैर्बहुभिराकाशं विस्मितैरावृतं बभौ} %6-91-62

\twolineshloka
{ऋषयः पितरो देवा गन्धर्वगरुडोरगाः}
{शतक्रतुं पुरस्कृत्य ररक्षुर्लक्ष्मणं रणे} %6-91-63

\twolineshloka
{अथान्यं मार्गणश्रेष्ठं सन्दधे राघवानुजः}
{हुताशनसमस्पर्शं रावणात्मजदारणम्} %6-91-64

\twolineshloka
{सुपत्रमनुवृत्ताङ्गं सुपर्वाणं सुसंस्थितम्}
{सुवर्णविकृतं वीरः शरीरान्तकरं शरम्} %6-91-65

\twolineshloka
{दुरावारं दुर्विषहं राक्षसानां भयावहम्}
{आशीविषविषप्रख्यं देवसङ्घैः समर्चितम्} %6-91-66

\twolineshloka
{येन शक्रो महातेजा दानवानजयत् प्रभुः}
{पुरा देवासुरे युद्धे वीर्यवान् हरिवाहनः} %6-91-67

\twolineshloka
{अथैन्द्रमस्त्रं सौमित्रिः संयुगेष्वपराजितम्}
{शरश्रेष्ठं धनुश्रेष्ठे विकर्षन्निदमब्रवीत्} %6-91-68

\threelineshloka
{लक्ष्मीवाँल्लक्ष्मणो वाक्यमर्थसाधकमात्मनः}
{धर्मात्मा सत्यसन्धश्च रामो दाशरथिर्यदि}
{पौरुषे चाप्रतिद्वन्द्वस्तदैनं जहि रावणिम्} %6-91-69

\threelineshloka
{इत्युक्त्वा बाणमाकर्णं विकृष्य तमजिह्मगम्}
{लक्ष्मणः समरे वीरः ससर्जेन्द्रजितं प्रति}
{ऐन्द्रास्त्रेण समायुज्य लक्ष्मणः परवीरहा} %6-91-70

\twolineshloka
{तच्छिरः सशिरस्त्राणं श्रीमज्ज्वलितकुण्डलम्}
{प्रमथ्येन्द्रजितः कायात् पातयामास भूतले} %6-91-71

\twolineshloka
{तद् राक्षसतनूजस्य भिन्नस्कन्धं शिरो महत्}
{तपनीयनिभं भूमौ ददृशे रुधिरोक्षितम्} %6-91-72

\twolineshloka
{हतः स निपपाताथ धरण्यां रावणात्मजः}
{कवची सशिरस्त्राणो विप्रविद्धशरासनः} %6-91-73

\twolineshloka
{चुक्रुशुस्ते ततः सर्वे वानराः सविभीषणाः}
{हृष्यन्ते निहते तस्मिन् देवा वृत्रवधे यथा} %6-91-74

\twolineshloka
{अथान्तरिक्षे देवानामृषीणां च महात्मनाम्}
{जज्ञेऽथ जयसन्नादो गन्धर्वाप्सरसामपि} %6-91-75

\twolineshloka
{पतितं समभिज्ञाय राक्षसी सा महाचमूः}
{वध्यमाना दिशो भेजे हरिभिर्जितकाशिभिः} %6-91-76

\twolineshloka
{वानरैर्वध्यमानास्ते शस्त्राण्युत्सृज्य राक्षसाः}
{लङ्कामभिमुखाः सस्रुर्भ्रष्टसंज्ञाः प्रधाविताः} %6-91-77

\twolineshloka
{दुद्रुवुर्बहुधा भीता राक्षसाः शतशो दिशः}
{त्यक्त्वा प्रहरणान् सर्वे पट्टिशासिपरश्वधान्} %6-91-78

\twolineshloka
{केचिल्लङ्कां परित्रस्ताः प्रविष्टा वानरार्दिताः}
{समुद्रे पतिताः केचित् केचित् पर्वतमाश्रिताः} %6-91-79

\twolineshloka
{हतमिन्द्रजितं दृष्ट्वा शयानं च रणक्षितौ}
{राक्षसानां सहस्रेषु न कश्चित् प्रत्यदृश्यत} %6-91-80

\twolineshloka
{यथास्तं गत आदित्ये नावतिष्ठन्ति रश्मयः}
{तथा तस्मिन् निपतिते राक्षसास्ते गता दिशः} %6-91-81

\twolineshloka
{शान्तरश्मिरिवादित्यो निर्वाण इव पावकः}
{बभूव स महाबाहुर्व्यपास्तगतजीवितः} %6-91-82

\twolineshloka
{प्रशान्तपीडाबहुलो विनष्टारिः प्रहर्षवान्}
{बभूव लोकः पतिते राक्षसेन्द्रसुते तदा} %6-91-83

\twolineshloka
{हर्षं च शक्रो भगवान् सह सर्वैर्महर्षिभिः}
{जगाम निहते तस्मिन् राक्षसे पापकर्मणि} %6-91-84

\twolineshloka
{आकाशे चापि देवानां शुश्रुवे दुन्दुभिस्वनः}
{नृत्यद्भिरप्सरोभिश्च गन्धर्वैश्च महात्मभिः} %6-91-85

\twolineshloka
{ववर्षुः पुष्पवर्षाणि तदद्भुतमिवाभवत्}
{प्रशशाम हते तस्मिन् राक्षसे क्रूरकर्मणि} %6-91-86

\twolineshloka
{शुद्धा आपो नभश्चैव जहृषुर्देवदानवाः}
{आजग्मुः पतिते तस्मिन् सर्वलोकभयावहे} %6-91-87

\twolineshloka
{ऊचुश्च सहितास्तुष्टा देवगन्धर्वदानवाः}
{विज्वराः शान्तकलुषा ब्राह्मणा विचरन्त्विति} %6-91-88

\twolineshloka
{ततोऽभ्यनन्दन् संहृष्टाः समरे हरियूथपाः}
{तमप्रतिबलं दृष्ट्वा हतं नैर्ऋतपुङ्गवम्} %6-91-89

\twolineshloka
{विभीषणो हनूमांश्च जाम्बवांश्चर्क्षयूथपः}
{विजयेनाभिनन्दन्तस्तुष्टुवुश्चापि लक्ष्मणम्} %6-91-90

\twolineshloka
{क्ष्वेडन्तश्च प्लवन्तश्च गर्जन्तश्च प्लवङ्गमाः}
{लब्धलक्षा रघुसुतं परिवार्योपतस्थिरे} %6-91-91

\twolineshloka
{लाङ्गूलानि प्रविध्यन्तः स्फोटयन्तश्च वानराः}
{लक्ष्मणो जयतीत्येव वाक्यं विश्रावयंस्तदा} %6-91-92

\twolineshloka
{अन्योन्यं च समाश्लिष्य हरयो हृष्टमानसाः}
{चक्रुरुच्चावचगुणा राघवाश्रयसत्कथाः} %6-91-93

\twolineshloka
{तदसुकरमथाभिवीक्ष्य हृष्टाः प्रियसुहृदो युधि लक्ष्मणस्य कर्म}
{परममुपलभन्मनःप्रहर्षं विनिहतमिन्द्ररिपुं निशम्य देवाः} %6-91-94


॥इत्यार्षे श्रीमद्रामायणे वाल्मीकीये आदिकाव्ये युद्धकाण्डे रावणिवधः नाम एकनवतितमः सर्गः ॥६-९१॥
