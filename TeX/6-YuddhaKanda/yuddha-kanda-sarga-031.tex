\sect{एकत्रिंशः सर्गः — विद्युद्जिह्वमायाप्रयोगः}

\twolineshloka
{ततस्तमक्षोभ्यबलं लङ्कायां नृपतेश्चराः}
{सुवेले राघवं शैले निविष्टं प्रत्यवेदयन्} %6-31-1

\twolineshloka
{चाराणां रावणः श्रुत्वा प्राप्तं रामं महाबलम्}
{जातोद्वेगोऽभवत् किञ्चित् सचिवानिदमब्रवीत्} %6-31-2

\twolineshloka
{मन्त्रिणः शीघ्रमायान्तु सर्वे वै सुसमाहिताः}
{अयं नो मन्त्रकालो हि सम्प्राप्त इति राक्षसाः} %6-31-3

\twolineshloka
{तस्य तच्छासनं श्रुत्वा मन्त्रिणोऽभ्यागमन् द्रुतम्}
{ततः स मन्त्रयामास राक्षसैः सचिवैः सह} %6-31-4

\twolineshloka
{मन्त्रयित्वा तु दुर्धर्षः क्षमं यत् तदनन्तरम्}
{विसर्जयित्वा सचिवान् प्रविवेश स्वमालयम्} %6-31-5

\twolineshloka
{ततो राक्षसमादाय विद्युज्जिह्वं महाबलम्}
{मायाविनं महामायं प्राविशद् यत्र मैथिली} %6-31-6

\twolineshloka
{विद्युज्जिह्वं च मायाज्ञमब्रवीद् राक्षसाधिपः}
{मोहयिष्यावहे सीतां मायया जनकात्मजाम्} %6-31-7

\twolineshloka
{शिरो मायामयं गृह्य राघवस्य निशाचर}
{मां त्वं समुपतिष्ठस्व महच्च सशरं धनुः} %6-31-8

\twolineshloka
{एवमुक्तस्तथेत्याह विद्युज्जिह्वो निशाचरः}
{दर्शयामास तां मायां सुप्रयुक्तां स रावणे} %6-31-9

\twolineshloka
{तस्य तुष्टोऽभवद् राजा प्रददौ च विभूषणम्}
{अशोकवनिकायां च सीतादर्शनलालसः} %6-31-10

\twolineshloka
{नैर्ऋतानामधिपतिः संविवेश महाबलः}
{ततो दीनामदैन्यार्हां ददर्श धनदानुजः} %6-31-11

\twolineshloka
{अधोमुखीं शोकपरामुपविष्टां महीतले}
{भर्तारं समनुध्यान्तीमशोकवनिकां गताम्} %6-31-12

\twolineshloka
{उपास्यमानां घोराभी राक्षसीभिरदूरतः}
{उपसृत्य ततः सीतां प्रहर्षं नाम कीर्तयन्} %6-31-13

\twolineshloka
{इदं च वचनं धृष्टमुवाच जनकात्मजाम्}
{सान्त्व्यमाना मया भद्रे यमाश्रित्य विमन्यसे} %6-31-14

\twolineshloka
{खरहन्ता स ते भर्ता राघवः समरे हतः}
{छिन्नं ते सर्वथा मूलं दर्पश्च निहतो मया} %6-31-15

\twolineshloka
{व्यसनेनात्मनः सीते मम भार्या भविष्यसि}
{विसृजैतां मतिं मूढे किं मृतेन करिष्यसि} %6-31-16

\threelineshloka
{भवस्व भद्रे भार्याणां सर्वासामीश्वरी मम}
{अल्पपुण्ये निवृत्तार्थे मूढे पण्डितमानिनि}
{शृणु भर्तृवधं सीते घोरं वृत्रवधं यथा} %6-31-17

\twolineshloka
{समायातः समुद्रान्तं हन्तुं मां किल राघवः}
{वानरेन्द्रप्रणीतेन बलेन महता वृतः} %6-31-18

\twolineshloka
{सन्निविष्टः समुद्रस्य पीड्य तीरमथोत्तरम्}
{बलेन महता रामो व्रजत्यस्तं दिवाकरे} %6-31-19

\twolineshloka
{अथाध्वनि परिश्रान्तमर्धरात्रे स्थितं बलम्}
{सुखसुप्तं समासाद्य चरितं प्रथमं चरैः} %6-31-20

\twolineshloka
{तत्प्रहस्तप्रणीतेन बलेन महता मम}
{बलमस्य हतं रात्रौ यत्र रामः सलक्ष्मणः} %6-31-21

\twolineshloka
{पट्टिशान् परिघांश्चक्रानृष्टीन् दण्डान् महायुधान्}
{बाणजालानि शूलानि भास्वरान् कूटमुद्गरान्} %6-31-22

\twolineshloka
{यष्टीश्च तोमरान् प्रासांश्चक्राणि मुसलानि च}
{उद्यम्योद्यम्य रक्षोभिर्वानरेषु निपातिताः} %6-31-23

\twolineshloka
{अथ सुप्तस्य रामस्य प्रहस्तेन प्रमाथिना}
{असक्तं कृतहस्तेन शिरश्छिन्नं महासिना} %6-31-24

\twolineshloka
{विभीषणः समुत्पत्य निगृहीतो यदृच्छया}
{दिशः प्रव्राजितः सैन्यैर्लक्ष्मणः प्लवगैः सह} %6-31-25

\twolineshloka
{सुग्रीवो ग्रीवया सीते भग्नया प्लवगाधिपः}
{निरस्तहनुकः सीते हनूमान् राक्षसैर्हतः} %6-31-26

\twolineshloka
{जाम्बवानथ जानुभ्यामुत्पतन् निहतो युधि}
{पट्टिशैर्बहुभिश्छिन्नो निकृत्तः पादपो यथा} %6-31-27

\twolineshloka
{मैन्दश्च द्विविदश्चोभौ तौ वानरवरर्षभौ}
{निःश्वसन्तौ रुदन्तौ च रुधिरेण परिप्लुतौ} %6-31-28

\twolineshloka
{असिना व्यायतौ छिन्नौ मध्ये ह्यरिनिषूदनौ}
{अनुश्वसिति मेदिन्यां पनसः पनसो यथा} %6-31-29

\twolineshloka
{नाराचैर्बहुभिश्छिन्नः शेते दर्यां दरीमुखः}
{कुमुदस्तु महातेजा निष्कूजन् सायकैर्हतः} %6-31-30

\twolineshloka
{अङ्गदो बहुभिश्छिन्नः शरैरासाद्य राक्षसैः}
{परितो रुधिरोद्गारी क्षितौ निपतितोऽङ्गदः} %6-31-31

\twolineshloka
{हरयो मथिता नागै रथजालैस्तथापरे}
{शयाना मृदितास्तत्र वायुवेगैरिवाम्बुदाः} %6-31-32

\twolineshloka
{प्रसृताश्च परे त्रस्ता हन्यमाना जघन्यतः}
{अनुद्रुतास्तु रक्षोभिः सिंहैरिव महाद्विपाः} %6-31-33

\twolineshloka
{सागरे पतिताः केचित् केचिद् गगनमाश्रिताः}
{ऋक्षा वृक्षानुपारूढा वानरीं वृत्तिमाश्रिताः} %6-31-34

\twolineshloka
{सागरस्य च तीरेषु शैलेषु च वनेषु च}
{पिङ्गलास्ते विरूपाक्षै राक्षसैर्बहवो हताः} %6-31-35

\twolineshloka
{एवं तव हतो भर्ता ससैन्यो मम सेनया}
{क्षतजार्द्रं रजोध्वस्तमिदं चास्याहृतं शिरः} %6-31-36

\twolineshloka
{ततः परमदुर्धर्षो रावणो राक्षसेश्वरः}
{सीतायामुपशृण्वन्त्यां राक्षसीमिदमब्रवीत्} %6-31-37

\twolineshloka
{राक्षसं क्रूरकर्माणं विद्युज्जिह्वं समानय}
{येन तद्राघवशिरः सङ्ग्रामात् स्वयमाहृतम्} %6-31-38

\twolineshloka
{विद्युज्जिह्वस्तदा गृह्य शिरस्तत्सशरासनम्}
{प्रणामं शिरसा कृत्वा रावणस्याग्रतः स्थितः} %6-31-39

\twolineshloka
{तमब्रवीत् ततो राजा रावणो राक्षसं स्थितम्}
{विद्युज्जिह्वं महाजिह्वं समीपपरिवर्तिनम्} %6-31-40

\twolineshloka
{अग्रतः कुरु सीतायाः शीघ्रं दाशरथेः शिरः}
{अवस्थां पश्चिमां भर्तुः कृपणा साधु पश्यतु} %6-31-41

\twolineshloka
{एवमुक्तं तु तद् रक्षः शिरस्तत् प्रियदर्शनम्}
{उपनिक्षिप्य सीतायाः क्षिप्रमन्तरधीयत} %6-31-42

\twolineshloka
{रावणश्चापि चिक्षेप भास्वरं कार्मुकं महत्}
{त्रिषु लोकेषु विख्यातं रामस्यैतदिति ब्रुवन्} %6-31-43

\twolineshloka
{इदं तत् तव रामस्य कार्मुकं ज्यासमावृतम्}
{इह प्रहस्तेनानीतं तं हत्वा निशि मानुषम्} %6-31-44

\twolineshloka
{स विद्युज्जिह्वेन सहैव तच्छिरो धनुश्च भूमौ विनिकीर्यमाणः}
{विदेहराजस्य सुतां यशस्विनीं ततोऽब्रवीत् तां भव मे वशानुगा} %6-31-45


॥इत्यार्षे श्रीमद्रामायणे वाल्मीकीये आदिकाव्ये युद्धकाण्डे विद्युद्जिह्वमायाप्रयोगः नाम एकत्रिंशः सर्गः ॥६-३१॥
