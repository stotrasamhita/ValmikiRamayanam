\sect{षट्षष्ठितमः सर्गः — वानरपर्यवस्थापनम्}

\twolineshloka
{स लङ्घयित्वा प्राकारं गिरिकूटोपमो महान्}
{निर्ययौ नगरात् तूर्णं कुम्भकर्णो महाबलः} %6-66-1

\twolineshloka
{ननाद च महानादं समुद्रमभिनादयन्}
{विजयन्निव निर्घातान् विधमन्निव पर्वतान्} %6-66-2

\twolineshloka
{तमवध्यं मघवता यमेन वरुणेन वा}
{प्रेक्ष्य भीमाक्षमायान्तं वानरा विप्रदुद्रुवुः} %6-66-3

\twolineshloka
{तांस्तु विप्रद्रुतान् दृष्ट्वा राजपुत्रोऽङ्गदोऽब्रवीत्}
{नलं नीलं गवाक्षं च कुमुदं च महाबलम्} %6-66-4

\twolineshloka
{आत्मनस्तानि विस्मृत्य वीर्याण्यभिजनानि च}
{क्व गच्छत भयत्रस्ताः प्राकृता हरयो यथा} %6-66-5

\twolineshloka
{साधु सौम्या निवर्तध्वं किं प्राणान् परिरक्षथ}
{नालं युद्धाय वै रक्षो महतीयं विभीषिका} %6-66-6

\twolineshloka
{महतीमुत्थितामेनां राक्षसानां विभीषिकाम्}
{विक्रमाद् विधमिष्यामो निवर्तध्वं प्लवङ्गमाः} %6-66-7

\twolineshloka
{कृच्छ्रेण तु समाश्वस्य सङ्गम्य च ततस्ततः}
{वृक्षान् गृहीत्वा हरयः सम्प्रतस्थू रणाजिरे} %6-66-8

\twolineshloka
{ते निवर्त्य तु संरब्धाः कुम्भकर्णं वनौकसः}
{निजघ्नुः परमक्रुद्धाः समदा इव कुञ्जराः} %6-66-9

\twolineshloka
{प्रांशुभिर्गिरिशृङ्गैश्च शिलाभिश्च महाबलाः}
{पादपैः पुष्पिताग्रैश्च हन्यमानो न कम्पते} %6-66-10

\twolineshloka
{तस्य गात्रेषु पतिता भिद्यन्ते बहवः शिलाः}
{पादपाः पुष्पिताग्राश्च भग्नाः पेतुर्महीतले} %6-66-11

\twolineshloka
{सोऽपि सैन्यानि सङ्क्रुद्धौ वानराणां महौजसाम्}
{ममन्थ परमायत्तो वनान्यग्निरिवोत्थितः} %6-66-12

\twolineshloka
{लोहितार्द्रास्तु बहवः शेरते वानरर्षभाः}
{निरस्ताः पतिता भूमौ ताम्रपुष्पा इव द्रुमाः} %6-66-13

\twolineshloka
{लङ्घयन्तः प्रधावन्तो वानरा नावलोकयन्}
{केचित् समुद्रे पतिताः केचिद् गगनमास्थिताः} %6-66-14

\twolineshloka
{वध्यमानास्तु ते वीरा राक्षसेन च लीलया}
{सागरं येन ते तीर्णाः पथा तेनैव दुद्रुवुः} %6-66-15

\twolineshloka
{ते स्थलानि तदा निम्नं विवर्णवदना भयात्}
{ऋक्षा वृक्षान् समारूढाः केचित् पर्वतमाश्रिताः} %6-66-16

\threelineshloka
{ममज्जुरर्णवे केचिद् गुहाः केचित् समाश्रिताः}
{निपेतुः केचिदपरे केचिन्नैवावतस्थिरे}
{केचिद् भूमौ निपतिताः केचित् सुप्ता मृता इव} %6-66-17

\twolineshloka
{तान् समीक्ष्याङ्गदो भग्नान् वानरानिदमब्रवीत्}
{अवतिष्ठत युध्यामो निवर्तध्वं प्लवङ्गमाः} %6-66-18

\twolineshloka
{भग्नानां वो न पश्यामि परिक्रम्य महीमिमाम्}
{स्थानं सर्वे निवर्तध्वं किं प्राणान् परिरक्षथ} %6-66-19

\twolineshloka
{निरायुधानां क्रमतामसङ्गगतिपौरुषाः}
{दारा ह्युपहसिष्यन्ति स वै घातः सुजीवताम्} %6-66-20

\threelineshloka
{कुलेषु जाताः सर्वेऽस्मिन् विस्तीर्णेषु महत्सु च}
{क्व गच्छत भयत्रस्ताः प्राकृता हरयो यथा}
{अनार्याः खलु यद्भीतास्त्यक्त्वा वीर्यं प्रधावत} %6-66-21

\twolineshloka
{विकत्थनानि वो यानि भवद्भिर्जनसंसदि}
{तानि वः क्व नु यातानि सोदग्राणि हितानि च} %6-66-22

\twolineshloka
{भीरोः प्रवादाः श्रूयन्ते यस्तु जीवति धिक्कृतः}
{मार्गः सत्पुरुषैर्जुष्टः सेव्यतां त्यज्यतां भयम्} %6-66-23

\twolineshloka
{शयामहे वा निहताः पृथिव्यामल्पजीविताः}
{प्राप्नुयामो ब्रह्मलोकं दुष्प्रापं च कुयोधिभिः} %6-66-24

\twolineshloka
{अवाप्नुयामः कीर्तिं वा निहत्वा शत्रुमाहवे}
{निहता वीरलोकस्य भोक्ष्यामो वसु वानराः} %6-66-25

\twolineshloka
{न कुम्भकर्णः काकुत्स्थं दृष्ट्वा जीवन् गमिष्यति}
{दीप्यमानमिवासाद्य पतङ्गो ज्वलनं यथा} %6-66-26

\twolineshloka
{पलायनेन चोद्दिष्टाः प्राणान् रक्षामहे वयम्}
{एकेन बहवो भग्ना यशो नाशं गमिष्यति} %6-66-27

\twolineshloka
{एवं ब्रुवाणं तं शूरमङ्गदं कनकाङ्गदम्}
{द्रवमाणास्ततो वाक्यमूचुः शूरविगर्हितम्} %6-66-28

\twolineshloka
{कृतं नः कदनं घोरं कुम्भकर्णेन रक्षसा}
{न स्थानकालो गच्छामो दयितं जीवितं हि नः} %6-66-29

\twolineshloka
{एतावदुक्त्वा वचनं सर्वे ते भेजिरे दिशः}
{भीमं भीमाक्षमायान्तं दृष्ट्वा वानरयूथपाः} %6-66-30

\twolineshloka
{द्रवमाणास्तु ते वीरा अङ्गदेन बलीमुखाः}
{सान्त्वनैश्चानुमानैश्च ततः सर्वे निवर्तिताः} %6-66-31

\twolineshloka
{प्रहर्षमुपनीताश्च वालिपुत्रेण धीमता}
{आज्ञाप्रतीक्षास्तस्थुश्च सर्वे वानरयूथपाः} %6-66-32

\twolineshloka
{ऋषभशरभमैन्दधूम्रनीलाः कुमुदसुषेणगवाक्षरम्भताराः}
{द्विविदपनसवायुपुत्रमुख्यास्त्वरिततराभिमुखं रणं प्रयाताः} %6-66-33


॥इत्यार्षे श्रीमद्रामायणे वाल्मीकीये आदिकाव्ये युद्धकाण्डे वानरपर्यवस्थापनम् नाम षट्षष्ठितमः सर्गः ॥६-६६॥
