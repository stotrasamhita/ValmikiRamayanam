\sect{द्विषष्ठितमः सर्गः — रावणाभ्यर्थना}

\twolineshloka
{स तु राक्षसशार्दूलो निद्रामदसमाकुलः}
{राजमार्गं श्रिया जुष्टं ययौ विपुलविक्रमः} %6-62-1

\twolineshloka
{राक्षसानां सहस्रैश्च वृतः परमदुर्जयः}
{गृहेभ्यः पुष्पवर्षेण कीर्यमाणस्तदा ययौ} %6-62-2

\twolineshloka
{स हेमजालविततं भानुभास्वरदर्शनम्}
{ददर्श विपुलं रम्यं राक्षसेन्द्रनिवेशनम्} %6-62-3

\twolineshloka
{स तत्तदा सूर्य इवाभ्रजालं प्रविश्य रक्षोधिपतेर्निवेशनम्}
{ददर्श दूरेऽग्रजमासनस्थं स्वयम्भुवं शक्र इवासनस्थम्} %6-62-4

\twolineshloka
{भ्रातुः स भवनं गच्छन् रक्षोगणसमन्वितः}
{कुम्भकर्णः पदन्यासैरकम्पयत मेदिनीम्} %6-62-5

\twolineshloka
{सोऽभिगम्य गृहं भ्रातुः कक्ष्यामभिविगाह्य च}
{ददर्शोद्विग्नमासीनं विमाने पुष्पके गुरुम्} %6-62-6

\twolineshloka
{अथ दृष्ट्वा दशग्रीवः कुम्भकर्णमुपस्थितम्}
{तूर्णमुत्थाय संहृष्टः सन्निकर्षमुपानयत्} %6-62-7

\twolineshloka
{अथासीनस्य पर्यङ्के कुम्भकर्णो महाबलः}
{भ्रातुर्ववन्दे चरणौ किं कृत्यमिति चाब्रवीत्} %6-62-8

\twolineshloka
{उत्पत्य चैनं मुदितो रावणः परिषस्वजे}
{स भ्रात्रा सम्परिष्वक्तो यथावच्चाभिनन्दितः} %6-62-9

\twolineshloka
{कुम्भकर्णः शुभं दिव्यं प्रतिपेदे वरासनम्}
{स तदासनमाश्रित्य कुम्भकर्णो महाबलः} %6-62-10

\twolineshloka
{संरक्तनयनः क्रोधाद् रावणं वाक्यमब्रवीत्}
{किमर्थमहमादृत्य त्वया राजन् प्रबोधितः} %6-62-11

\twolineshloka
{शंस कस्माद् भयं तेऽत्र को वा प्रेतो भविष्यति}
{भ्रातरं रावणः क्रुद्धं कुम्भकर्णमवस्थितम्} %6-62-12

\twolineshloka
{रोषेण परिवृत्ताभ्यां नेत्राभ्यां वाक्यमब्रवीत्}
{अद्य ते सुमहान् कालः शयानस्य महाबल} %6-62-13

\twolineshloka
{सुषुप्तस्त्वं न जानीषे मम रामकृतं भयम्}
{एष दाशरथिः श्रीमान् सुग्रीवसहितो बली} %6-62-14

\twolineshloka
{समुद्रं लङ्घयित्वा तु मूलं नः परिकृन्तति}
{हन्त पश्यस्व लङ्कायां वनान्युपवनानि च} %6-62-15

\twolineshloka
{सेतुना सुखमागत्य वानरैकार्णवं कृतम्}
{ये राक्षसा मुख्यतमा हतास्ते वानरैर्युधि} %6-62-16

\twolineshloka
{वानराणां क्षयं युद्धे न पश्यामि कथञ्चन}
{न चापि वानरा युद्धे जितपूर्वाः कदाचन} %6-62-17

\twolineshloka
{तदेतद् भयमुत्पन्नं त्रायस्वेह महाबल}
{नाशय त्वमिमानद्य तदर्थं बोधितो भवान्} %6-62-18

\twolineshloka
{सर्वक्षपितकोशं च स त्वमभ्युपपद्य माम्}
{त्रायस्वेमां पुरीं लङ्कां बालवृद्धावशेषिताम्} %6-62-19

\twolineshloka
{भ्रातुरर्थे महाबाहो कुरु कर्म सुदुष्करम्}
{मयैवं नोक्तपूर्वो हि भ्राता कश्चित् परन्तप} %6-62-20

\twolineshloka
{त्वय्यस्ति मम च स्नेहः परा सम्भावना च मे}
{देवासुरेषु युद्धेषु बहुशो राक्षसर्षभ} %6-62-21

\onelineshloka
{त्वया देवाः प्रतिव्यूह्य निर्जिताश्चासुरा युधि} %6-62-22

\twolineshloka
{तदेतत् सर्वमातिष्ठ वीर्यं भीमपराक्रम}
{नहि ते सर्वभूतेषु दृश्यते सदृशो बली} %6-62-23

\twolineshloka
{कुरुष्व मे प्रियहितमेतदुत्तमं यथाप्रियं प्रियरण बान्धवप्रिय}
{स्वतेजसा व्यथय सपत्नवाहिनीं शरद्घनं पवन इवोद्यतो महान्} %6-62-24


॥इत्यार्षे श्रीमद्रामायणे वाल्मीकीये आदिकाव्ये युद्धकाण्डे रावणाभ्यर्थना नाम द्विषष्ठितमः सर्गः ॥६-६२॥
