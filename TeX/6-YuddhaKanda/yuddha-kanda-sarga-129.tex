\sect{नवविंशत्यधिकशततमः सर्गः — हनूमद्भरतसंभाषणम्}

\twolineshloka
{बहूनि नाम वर्षाणि गतस्य सुमहद्वनम्}
{शृणोम्यहं प्रीतिकरं मम नाथस्य कीर्तनम्} %6-129-1

\twolineshloka
{कल्याणी बत गाथेयं लौकिकी प्रतिभाति माम्}
{एति जीवन्तमानन्दो नरं वर्षशतादपि} %6-129-2

\twolineshloka
{राघवस्य हरीणां च कथमासीत् समागमः}
{कस्मिन् देशे किमाश्रित्य तत्त्वमाख्याहि पृच्छतः} %6-129-3

\twolineshloka
{स पृष्टो राजपुत्रेण बृस्यां समुपवेशितः}
{आचचक्षे ततः सर्वं रामस्य चरितं वने} %6-129-4

\twolineshloka
{यथा प्रव्राजितो रामो मातुर्दत्तौ वरौ तव}
{यथा च पुत्रशोकेन राजा दशरथो मृतः} %6-129-5

\twolineshloka
{यथा दूतैस्त्वमानीतस्तूर्णं राजगृहात् प्रभो}
{त्वयायोध्यां प्रविष्टेन यथा राज्यं न चेप्सितम्} %6-129-6

\twolineshloka
{चित्रकूटगिरिं गत्वा राज्येनामित्रकर्शनः}
{निमन्त्रितस्त्वया भ्राता धर्ममाचरता सताम्} %6-129-7

\twolineshloka
{स्थितेन राज्ञो वचने यथा राज्यं विसर्जितम्}
{आर्यस्य पादुके गृह्य यथासि पुनरागतः} %6-129-8

\twolineshloka
{सर्वमेतन्महाबाहो यथावद् विदितं तव}
{त्वयि प्रतिप्रयाते तु यद् वृत्तं तन्निबोध मे} %6-129-9

\twolineshloka
{अपयाते त्वयि तदा समुद्भ्रान्तमृगद्विजम्}
{परिद्यूनमिवात्यर्थं तद् वनं समपद्यत} %6-129-10

\twolineshloka
{तद्धस्तिमृदितं घोरं सिंहव्याघ्रमृगाकुलम्}
{प्रविवेशाथ विजनं स महद् दण्डकावनम्} %6-129-11

\twolineshloka
{तेषां पुरस्ताद् बलवान् गच्छतां गहने वने}
{विनदन् सुमहानादं विराधः प्रत्यदृश्यत} %6-129-12

\twolineshloka
{तमुत्क्षिप्य महानादमूर्ध्वबाहुमधोमुखम्}
{निखाते प्रक्षिपन्ति स्म नदन्तमिव कुञ्जरम्} %6-129-13

\twolineshloka
{तत् कृत्वा दुष्करं कर्म भ्रातरौ रामलक्ष्मणौ}
{सायाह्ने शरभङ्गस्य रम्यमाश्रममीयतुः} %6-129-14

\twolineshloka
{शरभङ्गे दिवं प्राप्ते रामः सत्यपराक्रमः}
{अभिवाद्य मुनीन् सर्वाञ्जनस्थानमुपागमत्} %6-129-15

\twolineshloka
{पश्चाच्छूर्पणखा नाम रामपार्श्वमुपागता}
{ततो रामेण संदिष्टो लक्ष्मणः सहसोत्थितः} %6-129-16

\twolineshloka
{प्रगृह्य खड्गं चिच्छेद कर्णनासं महाबलः}
{चतुर्दश सहस्राणि रक्षसां भीमकर्मणाम्} %6-129-17

\twolineshloka
{हतानि वसता तत्र राघवेण महात्मना}
{एकेन सह संगम्य रामेण रणमूर्धनि} %6-129-18

\twolineshloka
{अह्नश्चतुर्थभागेन निःशेषा राक्षसाः कृताः}
{महाबला महावीर्यास्तपसो विघ्नकारिणः} %6-129-19

\twolineshloka
{निहता राघवेणाजौ दण्डकारण्यवासिनः}
{राक्षसाश्च विनिष्पिष्टाः खरश्च निहतो रणे} %6-129-20

\twolineshloka
{दूषणं चाग्रतो हत्वा त्रिशिरास्तदनन्तरम्}
{ततस्तेनार्दिता बाला रावणं समुपागता} %6-129-21

\twolineshloka
{रावणानुचरो घोरो मारीचो नाम राक्षसः}
{लोभयामास वैदेहीं भूत्वा रत्नमयो मृगः} %6-129-22

\twolineshloka
{सा राममब्रवीद् दृष्ट्वा वैदेही गृह्यतामिति}
{अयं मनोहरः कान्त आश्रमो नो भविष्यति} %6-129-23

\twolineshloka
{ततो रामो धनुष्पाणिर्मृगं तमनुधावति}
{स तं जघान धावन्तं शरेणानतपर्वणा} %6-129-24

\twolineshloka
{अथ सौम्य दशग्रीवो मृगं याति तु राघवे}
{लक्ष्मणे चापि निष्क्रान्ते प्रविवेशाश्रमं तदा} %6-129-25

\twolineshloka
{जग्राह तरसा सीतां ग्रहः खे रोहिणीमिव}
{त्रातुकामं ततो युद्धे हत्वा गृध्रं जटायुषम्} %6-129-26

\twolineshloka
{प्रगृह्य सहसा सीतां जगामाशु स राक्षसः}
{ततस्त्वद्भुतसंकाशाः स्थिताः पर्वतमूर्धनि} %6-129-27

\twolineshloka
{सीतां गृहीत्वा गच्छन्तं वानराः पर्वतोपमाः}
{ददृशुर्विस्मिताकारा रावणं राक्षसाधिपम्} %6-129-28

\twolineshloka
{ततः शीघ्रतरं गत्वा तद् विमानं मनोजवम्}
{आरुह्य सह वैदेह्या पुष्पकं स महाबलः} %6-129-29

\twolineshloka
{प्रविवेश तदा लङ्कां रावणो राक्षसेश्वरः}
{तां सुवर्णपरिष्कारे शुभे महति वेश्मनि} %6-129-30

\twolineshloka
{प्रवेश्य मैथिलीं वाक्यैः सान्त्वयामास रावणः}
{तृणवद् भाषितं तस्य तं च नैर्ऋतपुङ्गवम्} %6-129-31

\twolineshloka
{अचिन्तयन्ती वैदेही ह्यशोकवनिकां गता}
{न्यवर्तत तदा रामो मृगं हत्वा तदा वने} %6-129-32

\twolineshloka
{निवर्तमानः काकुत्स्थो दृष्ट्वा गृध्रं स विव्यथे}
{गृध्रं हतं तदा दृष्ट्वा रामः प्रियतरं पितुः} %6-129-33

\twolineshloka
{मार्गमाणस्तु वैदेहीं राघवः सहलक्ष्मणः}
{गोदावरीमनुचरन् वनोद्देशांश्च पुष्पितान्} %6-129-34

\twolineshloka
{आसेदतुर्महारण्ये कबन्धं नाम राक्षसम्}
{ततः कबन्धवचनाद् रामः सत्यपराक्रमः} %6-129-35

\twolineshloka
{ऋष्यमूकगिरिं गत्वा सुग्रीवेण समागतः}
{तयोः समागमः पूर्वं प्रीत्या हार्दो व्यजायत} %6-129-36

\twolineshloka
{भ्रात्रा निरस्तः क्रुद्धेन सुग्रीवो वालिना पुरा}
{इतरेतरसंवादात् प्रगाढः प्रणयस्तयोः} %6-129-37

\twolineshloka
{रामः स्वबाहुवीर्येण स्वराज्यं प्रत्यपादयत्}
{वालिनं समरे हत्वा महाकायं महाबलम्} %6-129-38

\twolineshloka
{सुग्रीवः स्थापितो राज्ये सहितः सर्ववानरैः}
{रामाय प्रतिजानीते राजपुत्र्यास्तु मार्गणम्} %6-129-39

\twolineshloka
{आदिष्टा वानरेन्द्रेण सुग्रीवेण महात्मना}
{दश कोट्यः प्लवङ्गानां सर्वाः प्रस्थापिता दिशः} %6-129-40

\twolineshloka
{तेषां नो विप्रकृष्टानां विन्ध्ये पर्वतसत्तमे}
{भृशं शोकाभितप्तानां महान् कालोऽत्यवर्तत} %6-129-41

\twolineshloka
{भ्राता तु गृध्रराजस्य सम्पातिर्नाम वीर्यवान्}
{समाख्याति स्म वसतीं सीतां रावणमन्दिरे} %6-129-42

\threelineshloka
{सोऽहं दुःखपरीतानां दुःखं तज्ज्ञातिनां नुदन्}
{आत्मवीर्यं समास्थाय योजनानां शतं प्लुतः}
{तत्राहमेकामद्राक्षमशोकवनिकां गताम्} %6-129-43

\twolineshloka
{कौशेयवस्त्रां मलिनां निरानन्दां दृढव्रताम्}
{तया समेत्य विधिवत् पृष्ट्वा सर्वमनिन्दिताम्} %6-129-44

\twolineshloka
{अभिज्ञानं मया दत्तं रामनामाङ्गुलीयकम्}
{अभिज्ञानं मणिं लब्ध्वा चरितार्थोऽहमागतः} %6-129-45

\twolineshloka
{मया च पुनरागम्य रामस्याक्लिष्टकर्मणः}
{अभिज्ञानं मया दत्तमर्चिष्मान् स महामणिः} %6-129-46

\twolineshloka
{श्रुत्वा तां मैथिलीं रामस्त्वाशशंसे च जीवितम्}
{जीवितान्तमनुप्राप्तः पीत्वामृतमिवातुरः} %6-129-47

\twolineshloka
{उद्योजयिष्यन्नुद्योगं दध्रे लङ्कावधे मनः}
{जिघांसुरिव लोकान्ते सर्वाँल्लोकान् विभावसुः} %6-129-48

\twolineshloka
{ततः समुद्रमासाद्य नलं सेतुमकारयत्}
{अतरत् कपिवीराणां वाहिनी तेन सेतुना} %6-129-49

\twolineshloka
{प्रहस्तमवधीन्नीलः कुम्भकर्णं तु राघवः}
{लक्ष्मणो रावणसुतं स्वयं रामस्तु रावणम्} %6-129-50

\twolineshloka
{स शक्रेण समागम्य यमेन वरुणेन च}
{महेश्वरस्वयंभूभ्यां तथा दशरथेन च} %6-129-51

\twolineshloka
{तैश्च दत्तवरः श्रीमानृषिभिश्च समागतैः}
{सुरर्षिभिश्च काकुत्स्थो वराँल्लेभे परंतपः} %6-129-52

\twolineshloka
{स तु दत्तवरः प्रीत्या वानरैश्च समागतैः}
{पुष्पकेण विमानेन किष्किन्धामभ्युपागमत्} %6-129-53

\twolineshloka
{तां गङ्गां पुनरासाद्य वसन्तं मुनिसंनिधौ}
{अविघ्नं पुष्ययोगेन श्वो रामं द्रष्टुमर्हसि} %6-129-54

\twolineshloka
{ततः स वाक्यैर्मधुरैर्हनूमतो निशम्य हृष्टो भरतः कृताञ्जलिः}
{उवाच वाणीं मनसः प्रहर्षिणीं चिरस्य पूर्णः खलु मे मनोरथः} %6-129-55


॥इत्यार्षे श्रीमद्रामायणे वाल्मीकीये आदिकाव्ये युद्धकाण्डे हनूमद्भरतसंभाषणम् नाम नवविंशत्यधिकशततमः सर्गः ॥६-१२९॥
