\sect{पञ्चपञ्चाशः सर्गः — अकम्पनयुद्धम्}

\twolineshloka
{वज्रदंष्ट्रं हतं श्रुत्वा वालिपुत्रेण रावणः}
{बलाध्यक्षमुवाचेदं कृताञ्जलिमुपस्थितम्} %6-55-1

\twolineshloka
{शीघ्रं निर्यान्तुदुर्धर्षा राक्षसा भीमविक्रमाः}
{अकम्पनं पुरस्कृत्य सर्वशस्त्रास्त्रकोविदम्} %6-55-2

\twolineshloka
{एष शास्ता च गोप्ता च नेता च युधि सत्तमः}
{भूतिकामश्च मे नित्यं नित्यं च समरप्रियः} %6-55-3

\twolineshloka
{एष जेष्यति काकुत्स्थौ सुग्रीवं च महाबलम्}
{वानरांश्चापरान् घोरान् हनिष्यति न संशयः} %6-55-4

\twolineshloka
{परिगृह्य स तामाज्ञां रावणस्य महाबलः}
{बलं सम्प्रेरयामास तदा लघुपराक्रमः} %6-55-5

\twolineshloka
{ततो नानाप्रहरणा भीमाक्षा भीमदर्शनाः}
{निष्पेतू राक्षसा मुख्या बलाध्यक्षप्रचोदिताः} %6-55-6

\twolineshloka
{रथमास्थाय विपुलं तप्तकाञ्चनभूषणम्}
{मेघाभो मेघवर्णश्च मेघस्वनमहास्वनः} %6-55-7

\twolineshloka
{राक्षसैः संवृतो घोरैस्तदा निर्यात्यकम्पनः}
{नहि कम्पयितुं शक्यः सुरैरपि महामृधे} %6-55-8

\twolineshloka
{अकम्पनस्ततस्तेषामादित्य इव तेजसा}
{तस्य निर्धावमानस्य संरब्धस्य युयुत्सया} %6-55-9

\twolineshloka
{अकस्माद् दैन्यमागच्छद्धयानां रथवाहिनाम्}
{व्यस्फुरन्नयनं चास्य सव्यं युद्धाभिनन्दिनः} %6-55-10

\twolineshloka
{विवर्णो मुखवर्णश्च गद्गदश्चाभवत् स्वनः}
{अभवत् सुदिने काले दुर्दिनं रूक्षमारुतम्} %6-55-11

\twolineshloka
{ऊचुः खगमृगाः सर्वे वाचः क्रूरा भयावहाः}
{स सिंहोपचितस्कन्धः शार्दूलसमविक्रमः} %6-55-12

\twolineshloka
{तानुत्पातानचिन्त्यैव निर्जगाम रणाजिरम्}
{तथा निर्गच्छतस्तस्य रक्षसः सह राक्षसैः} %6-55-13

\twolineshloka
{बभूव सुमहान् नादः क्षोभयन्निव सागरम्}
{तेन शब्देन वित्रस्ता वानराणां महाचमूः} %6-55-14

\twolineshloka
{द्रुमशैलप्रहाराणां योद्धुं समुपतिष्ठताम्}
{तेषां युद्धं महारौद्रं संजज्ञे कपिरक्षसाम्} %6-55-15

\twolineshloka
{रामरावणयोरर्थे समभित्यक्तदेहिनः}
{सर्वे ह्यतिबलाः शूराः सर्वे पर्वतसंनिभाः} %6-55-16

\twolineshloka
{हरयो राक्षसाश्चैव परस्परजिघांसया}
{तेषां विनर्दतां शब्दः संयुगेऽतितरस्विनाम्} %6-55-17

\twolineshloka
{शुश्रुवे सुमहान् कोपादन्योन्यमभिगर्जताम्}
{रजश्चारुणवर्णाभं सुभीममभवद् भृशम्} %6-55-18

\twolineshloka
{उद्धृतं हरिरक्षोभिः संरुरोध दिशो दश}
{अन्योन्यं रजसा तेन कौशेयोद्धतपाण्डुना} %6-55-19

\twolineshloka
{संवृतानि च भूतानि ददृशुर्न रणाजिरे}
{न ध्वजो न पताका वा चर्म वा तुरगोऽपि वा} %6-55-20

\twolineshloka
{आयुधं स्यन्दनो वापि ददृशे तेन रेणुना}
{शब्दश्च सुमहांस्तेषां नर्दतामभिधावताम्} %6-55-21

\twolineshloka
{श्रूयते तुमुलो युद्धे न रूपाणि चकाशिरे}
{हरीनेव सुसंरुष्टा हरयो जघ्नुराहवे} %6-55-22

\twolineshloka
{राक्षसा राक्षसांश्चापि निजघ्नुस्तिमिरे तदा}
{ते परांश्च विनिघ्नन्तः स्वांश्च वानरराक्षसाः} %6-55-23

\twolineshloka
{रुधिरार्द्रां तदा चक्रुर्महीं पङ्कानुलेपनाम्}
{ततस्तु रुधिरौघेण सिक्तं ह्यपगतं रजः} %6-55-24

\twolineshloka
{शरीरशवसंकीर्णा बभूव च वसुंधरा}
{द्रुमशक्तिगदाप्रासैः शिलापरिघतोमरैः} %6-55-25

\twolineshloka
{राक्षसा हरयस्तूर्णं जघ्नुरन्योन्यमोजसा}
{बाहुभिः परिघाकारैर्युध्यन्तः पर्वतोपमान्} %6-55-26

\twolineshloka
{हरयो भीमकर्माणो राक्षसाञ्जघ्नुराहवे}
{राक्षसास्त्वभिसंक्रुद्धाः प्रासतोमरपाणयः} %6-55-27

\twolineshloka
{कपीन् निजघ्निरे तत्र शस्त्रैः परमदारुणैः}
{अकम्पनः सुसंक्रुद्धो राक्षसानां चमूपतिः} %6-55-28

\twolineshloka
{संहर्षयति तान् सर्वान् राक्षसान् भीमविक्रमान्}
{हरयस्त्वपि रक्षांसि महाद्रुममहाश्मभिः} %6-55-29

\twolineshloka
{विदारयन्त्यभिक्रम्य शस्त्राण्याच्छिद्य वीर्यतः}
{एतस्मिन्नन्तरे वीरा हरयः कुमुदो नलः} %6-55-30

\twolineshloka
{मैन्दश्च द्विविदः क्रुद्धाश्चक्रुर्वेगमनुत्तमम्}
{ते तु वृक्षैर्महावीरा राक्षसानां चमूमुखे} %6-55-31

\twolineshloka
{कदनं सुमहच्चक्रुर्लीलया हरिपुंगवाः}
{ममन्थू राक्षसान् सर्वे नानाप्रहरणैर्भृशम्} %6-55-32


॥इत्यार्षे श्रीमद्रामायणे वाल्मीकीये आदिकाव्ये युद्धकाण्डे अकम्पनयुद्धम् नाम पञ्चपञ्चाशः सर्गः ॥६-५५॥
