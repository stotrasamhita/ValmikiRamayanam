\sect{अष्टाशीतितमः सर्गः — सौमित्रिरावणियुद्धम्}

\twolineshloka
{विभीषणवचः श्रुत्वा रावणिः क्रोधमूर्च्छितः}
{अब्रवीत् परुषं वाक्यं क्रोधेनाभ्युत्पपात च} %6-88-1

\twolineshloka
{उद्यतायुधनिस्त्रिंशो रथे सुसमलंकृते}
{कालाश्वयुक्ते महति स्थितः कालान्तकोपमः} %6-88-2

\twolineshloka
{महाप्रमाणमुद्यम्य विपुलं वेगवद् दृढम्}
{धनुर्भीमबलो भीमं शरांश्चामित्रनाशनान्} %6-88-3

\twolineshloka
{तं ददर्श महेष्वासो रथस्थः समलंकृतः}
{अलंकृतममित्रघ्नो रावणस्यात्मजो बली} %6-88-4

\twolineshloka
{हनूमत्पृष्ठमारूढमुदयस्थरविप्रभम्}
{उवाचैनं सुसंरब्धः सौमित्रिं सविभीषणम्} %6-88-5

\twolineshloka
{तांश्च वानरशार्दूलान् पश्यध्वं मे पराक्रमम्}
{अद्य मत्कार्मुकोत्सृष्टं शरवर्षं दुरासदम्} %6-88-6

\threelineshloka
{मुक्तवर्षमिवाकाशे धारयिष्यथ संयुगे}
{अद्य वो मामका बाणा महाकार्मुकनिःसृताः}
{विधमिष्यन्ति गात्राणि तूलराशिमिवानलः} %6-88-7

\twolineshloka
{तीक्ष्णसायकनिर्भिन्नान् शूलशक्त्यृष्टितोमरैः}
{अद्य वो गमयिष्यामि सर्वानेव यमक्षयम्} %6-88-8

\twolineshloka
{सृजतः शरवर्षाणि क्षिप्रहस्तस्य संयुगे}
{जीमूतस्येव नदतः कः स्थास्यति ममाग्रतः} %6-88-9

\twolineshloka
{रात्रियुद्धे तदा पूर्वं वज्राशनिसमैः शरैः}
{शायितौ तौ मया भूयो विसंज्ञौ सपुरःसरौ} %6-88-10

\twolineshloka
{स्मृतिर्न तेऽस्ति वा मन्ये व्यक्तं यातो यमक्षयम्}
{आशीविषसमं क्रुद्धं यन्मां योद्धुमुपस्थितः} %6-88-11

\twolineshloka
{तच्छ्रुत्वा राक्षसेन्द्रस्य गर्जितं राघवस्तदा}
{अभीतवदनः क्रुद्धो रावणिं वाक्यमब्रवीत्} %6-88-12

\twolineshloka
{उक्तश्च दुर्गमः पारः कार्याणां राक्षस त्वया}
{कार्याणां कर्मणा पारं यो गच्छति स बुद्धिमान्} %6-88-13

\twolineshloka
{स त्वमर्थस्य हीनार्थो दुरवापस्य केनचित्}
{वाचा व्याहृत्य जानीषे कृतार्थोऽस्मीति दुर्मते} %6-88-14

\twolineshloka
{अन्तर्धानगतेनाजौ यत्त्वया चरितस्तदा}
{तस्कराचरितो मार्गो नैष वीरनिषेवितः} %6-88-15

\twolineshloka
{यथा बाणपथं प्राप्य स्थितोऽस्मि तव राक्षस}
{दर्शयस्वाद्य तत्तेजो वाचा त्वं किं विकत्थसे} %6-88-16

\twolineshloka
{एवमुक्तो धनुर्भीमं परामृश्य महाबलः}
{ससर्ज निशितान् बाणानिन्द्रजित् समितिंजयः} %6-88-17

\twolineshloka
{तेन सृष्टा महावेगाः शराः सर्पविषोपमाः}
{सम्प्राप्य लक्ष्मणं पेतुः श्वसन्त इव पन्नगाः} %6-88-18

\twolineshloka
{शरैरतिमहावेगैर्वेगवान् रावणात्मजः}
{सौमित्रिमिन्द्रजिद् युद्धे विव्याध शुभलक्षणम्} %6-88-19

\twolineshloka
{स शरैरतिविद्धाङ्गो रुधिरेण समुक्षितः}
{शुशुभे लक्ष्मणः श्रीमान् विधूम इव पावकः} %6-88-20

\twolineshloka
{इन्द्रजित् त्वात्मनः कर्म प्रसमीक्ष्याभिगम्य च}
{विनद्य सुमहानादमिदं वचनमब्रवीत्} %6-88-21

\twolineshloka
{पत्रिणः शितधारास्ते शरा मत्कार्मुकच्युताः}
{आदास्यन्तेऽद्य सौमित्रे जीवितं जीवितान्तकाः} %6-88-22

\twolineshloka
{अद्य गोमायुसङ्घाश्च श्येनसङ्घाश्च लक्ष्मण}
{गृध्राश्च निपतन्तु त्वां गतासुं निहतं मया} %6-88-23

\twolineshloka
{क्षत्रबन्धुं सदानार्यं रामः परमदुर्मतिः}
{भक्तं भ्रातरमद्यैव त्वां द्रक्ष्यति हतं मया} %6-88-24

\twolineshloka
{विस्रस्तकवचं भूमौ व्यपविद्धशरासनम्}
{हृतोत्तमाङ्गं सौमित्रे त्वामद्य निहतं मया} %6-88-25

\twolineshloka
{इति ब्रुवाणं संक्रुद्धः परुषं रावणात्मजम्}
{हेतुमद् वाक्यमर्थज्ञो लक्ष्मणः प्रत्युवाच ह} %6-88-26

\twolineshloka
{वाग्बलं त्यज दुर्बुद्धे क्रूरकर्मन् हि राक्षस}
{अथ कस्माद् वदस्येतत् सम्पादय सुकर्मणा} %6-88-27

\twolineshloka
{अकृत्वा कत्थसे कर्म किमर्थमिह राक्षस}
{कुरु तत् कर्म येनाहं श्रद्धेयं तव कत्थनम्} %6-88-28

\twolineshloka
{अनुक्त्वा परुषं वाक्यं किंचिदप्यनवक्षिपन्}
{अविकत्थन् वधिष्यामि त्वां पश्य पुरुषादन} %6-88-29

\twolineshloka
{इत्युक्त्वा पञ्च नाराचानाकर्णापूरितान् शरान्}
{विजघान महावेगाल्लक्ष्मणो राक्षसोरसि} %6-88-30

\twolineshloka
{सुपत्रवाजिता बाणा ज्वलिता इव पन्नगाः}
{नैर्ऋतोरस्यभासन्त सवितू रश्मयो यथा} %6-88-31

\twolineshloka
{स शरैराहतस्तेन सरोषो रावणात्मजः}
{सुप्रयुक्तैस्त्रिभिर्बाणैः प्रतिविव्याध लक्ष्मणम्} %6-88-32

\twolineshloka
{स बभूव महाभीमो नरराक्षससिंहयोः}
{विमर्दस्तुमुलो युद्धे परस्परजयैषिणोः} %6-88-33

\twolineshloka
{विक्रान्तौ बलसम्पन्नावुभौ विक्रमशालिनौ}
{उभौ परमदुर्जेयावतुल्यबलतेजसौ} %6-88-34

\twolineshloka
{युयुधाते तदा वीरौ ग्रहाविव नभोगतौ}
{बलवृत्राविव हि तौ युधि वै दुष्प्रधर्षणौ} %6-88-35

\threelineshloka
{युयुधाते महात्मानौ तदा केसरिणाविव}
{बहूनवसृजन्तौ हि मार्गणौघानवस्थितौ}
{नरराक्षसमुख्यौ तौ प्रहृष्टावभ्ययुध्यताम्} %6-88-36


॥इत्यार्षे श्रीमद्रामायणे वाल्मीकीये आदिकाव्ये युद्धकाण्डे सौमित्रिरावणियुद्धम् नाम अष्टाशीतितमः सर्गः ॥६-८८॥
