\sect{त्रिपञ्चाशः सर्गः — वज्रदंष्ट्रयुद्धम्}

\twolineshloka
{धूम्राक्षं निहतं श्रुत्वा रावणो राक्षसेश्वरः}
{क्रोधेन महताऽऽविष्टो निःश्वसन्नुरगो यथा} %6-53-1

\twolineshloka
{दीर्घमुष्णं विनिःश्वस्य क्रोधेन कलुषीकृतः}
{अब्रवीद् राक्षसं क्रूरं वज्रदंष्ट्रं महाबलम्} %6-53-2

\twolineshloka
{गच्छ त्वं वीर निर्याहि राक्षसैः परिवारितः}
{जहि दाशरथिं रामं सुग्रीवं वानरैः सह} %6-53-3

\twolineshloka
{तथेत्युक्त्वा द्रुततरं मायावी राक्षसेश्वरः}
{निर्जगाम बलैः सार्धं बहुभिः परिवारितः} %6-53-4

\twolineshloka
{नागैरश्वैः खरैरुष्ट्रैः संयुक्तः सुसमाहितः}
{पताकाध्वजचित्रैश्च बहुभिः समलंकृतः} %6-53-5

\twolineshloka
{ततो विचित्रकेयूरमुकुटेन विभूषितः}
{तनुत्रं स समावृत्य सधनुर्निर्ययौ द्रुतम्} %6-53-6

\twolineshloka
{पताकालंकृतं दीप्तं तप्तकाञ्चनभूषितम्}
{रथं प्रदक्षिणं कृत्वा समारोहच्चमूपतिः} %6-53-7

\twolineshloka
{ऋष्टिभिस्तोमरैश्चित्रैः श्लक्ष्णैश्च मुसलैरपि}
{भिन्दिपालैश्च चापैश्च शक्तिभिः पट्टिशैरपि} %6-53-8

\twolineshloka
{खड्गैश्चक्रैर्गदाभिश्च निशितैश्च परश्वधैः}
{पदातयश्च निर्यान्ति विविधाः शस्त्रपाणयः} %6-53-9

\twolineshloka
{विचित्रवाससः सर्वे दीप्ता राक्षसपुङ्गवाः}
{गजा महोत्कटाः शूराश्चलन्त इव पर्वताः} %6-53-10

\twolineshloka
{ते युद्धकुशला रूढास्तोमराङ्कुशपाणिभिः}
{अन्ये लक्षणसंयुक्ताः शूरारूढा महाबलाः} %6-53-11

\twolineshloka
{तद् राक्षसबलं सर्वं विप्रस्थितमशोभत}
{प्रावृट्काले यथा मेघा नर्दमानाः सविद्युतः} %6-53-12

\twolineshloka
{निःसृता दक्षिणद्वारादङ्गदो यत्र यूथपः}
{तेषां निष्क्रममाणानामशुभं समजायत} %6-53-13

\twolineshloka
{आकाशाद् विघनात् तीव्रा उल्काश्चाभ्यपतंस्तदा}
{वमन्तः पावकज्वालाः शिवा घोरा ववाशिरे} %6-53-14

\twolineshloka
{व्याहरन्त मृगा घोरा रक्षसां निधनं तदा}
{समापतन्तो योधास्तु प्रास्खलंस्तत्र दारुणम्} %6-53-15

\twolineshloka
{एतानौत्पातिकान् दृष्ट्वा वज्रदंष्ट्रो महाबलः}
{धैर्यमालम्ब्य तेजस्वी निर्जगाम रणोत्सुकः} %6-53-16

\twolineshloka
{तांस्तु विद्रवतो दृष्ट्वा वानरा जितकाशिनः}
{प्रणेदुः सुमहानादान् दिशः शब्देन पूरयन्} %6-53-17

\twolineshloka
{ततः प्रवृत्तं तुमुलं हरीणां राक्षसैः सह}
{घोराणां भीमरूपाणामन्योन्यवधकाङ्क्षिणाम्} %6-53-18

\twolineshloka
{निष्पतन्तो महोत्साहा भिन्नदेहशिरोधराः}
{रुधिरोक्षितसर्वाङ्गा न्यपतन् धरणीतले} %6-53-19

\twolineshloka
{केचिदन्योन्यमासाद्य शूराः परिघबाहवः}
{चिक्षिपुर्विविधान् शस्त्रान् समरेष्वनिवर्तिनः} %6-53-20

\twolineshloka
{द्रुमाणां च शिलानां च शस्त्राणां चापि निःस्वनः}
{श्रूयते सुमहांस्तत्र घोरो हृदयभेदनः} %6-53-21

\twolineshloka
{रथनेमिस्वनस्तत्र धनुषश्चापि घोरवत्}
{शङ्खभेरीमृदङ्गानां बभूव तुमुलः स्वनः} %6-53-22

\onelineshloka
{केचिदस्त्राणि संत्यज्य बाहुयुद्धमकुर्वत} %6-53-23

\threelineshloka
{तलैश्च चरणैश्चापि मुष्टिभिश्च द्रुमैरपि}
{जानुभिश्च हताः केचिद् भग्नदेहाश्च राक्षसाः}
{शिलाभिश्चूर्णिताः केचिद् वानरैर्युद्धदुर्मदैः} %6-53-24

\twolineshloka
{वज्रदंष्ट्रो भृशं बाणै रणे वित्रासयन् हरीन्}
{चचार लोकसंहारे पाशहस्त इवान्तकः} %6-53-25

\twolineshloka
{बलवन्तोऽस्त्रविदुषो नानाप्रहरणा रणे}
{जघ्नुर्वानरसैन्यानि राक्षसाः क्रोधर्मूच्छिताः} %6-53-26

\twolineshloka
{जघ्ने तान् राक्षसान् सर्वान् धृष्टो वालिसुतो रणे}
{क्रोधेन द्विगुणाविष्टः संवर्तक इवानलः} %6-53-27

\twolineshloka
{तान् राक्षसगणान् सर्वान् वृक्षमुद्यम्य वीर्यवान्}
{अङ्गदः क्रोधताम्राक्षः सिंहः क्षुद्रमृगानिव} %6-53-28

\twolineshloka
{चकार कदनं घोरं शक्रतुल्यपराक्रमः}
{अङ्गदाभिहतास्तत्र राक्षसा भीमविक्रमाः} %6-53-29

\twolineshloka
{विभिन्नशिरसः पेतुर्निकृत्ता इव पादपाः}
{रथैश्चित्रैर्ध्वजैरश्वैः शरीरैर्हरिरक्षसाम्} %6-53-30

\twolineshloka
{रुधिरौघेण संछन्ना भूमिर्भयकरी तदा}
{हारकेयूरवस्त्रैश्च शस्त्रैश्च समलंकृता} %6-53-31

\threelineshloka
{भूमिर्भाति रणे तत्र शारदीव यथा निशा}
{अङ्गदस्य च वेगेन तद् राक्षसबलं महत्}
{प्राकम्पत तदा तत्र पवनेनाम्बुदो यथा} %6-53-32


॥इत्यार्षे श्रीमद्रामायणे वाल्मीकीये आदिकाव्ये युद्धकाण्डे वज्रदंष्ट्रयुद्धम् नाम त्रिपञ्चाशः सर्गः ॥६-५३॥
