\sect{पञ्चनवतितमः सर्गः — राक्षसीविलापः}

\twolineshloka
{तानि नागसहस्राणि सारोहाणि च वाजिनाम्}
{रथानां त्वग्निवर्णानां सध्वजानां सहस्रशः} %6-95-1

\twolineshloka
{राक्षसानां सहस्राणि गदापरिघयोधिनाम्}
{काञ्चनध्वजचित्राणां शूराणां कामरूपिणाम्} %6-95-2

\twolineshloka
{निहतानि शरैर्दीप्तैस्तप्तकाञ्चनभूषणैः}
{रावणेन प्रयुक्तानि रामेणाक्लिष्टकर्मणा} %6-95-3

\twolineshloka
{दृष्ट्वा श्रुत्वा च सम्भ्रान्ता हतशेषा निशाचराः}
{राक्षस्यश्च समागम्य दीनाश्चिन्तापरिप्लुताः} %6-95-4

\twolineshloka
{विधवा हतपुत्राश्च क्रोशन्त्यो हतबान्धवाः}
{राक्षस्यः सह संगम्य दुःखार्ताः पर्यदेवयन्} %6-95-5

\twolineshloka
{कथं शूर्पणखा वृद्धा कराला निर्णतोदरी}
{आससाद वने रामं कंदर्पसमरूपिणम्} %6-95-6

\twolineshloka
{सुकुमारं महासत्त्वं सर्वभूतहिते रतम्}
{तं दृष्ट्वा लोकवध्या सा हीनरूपा प्रकामिता} %6-95-7

\twolineshloka
{कथं सर्वगुणैर्हीना गुणवन्तं महौजसम्}
{सुमुखं दुर्मुखी रामं कामयामास राक्षसी} %6-95-8

\twolineshloka
{जनस्यास्याल्पभाग्यत्वाद् वलिनी श्वेतमूर्धजा}
{अकार्यमपहास्यं च सर्वलोकविगर्हितम्} %6-95-9

\twolineshloka
{राक्षसानां विनाशाय दूषणस्य खरस्य च}
{चकाराप्रतिरूपा सा राघवस्य प्रधर्षणम्} %6-95-10

\twolineshloka
{तन्निमित्तमिदं वैरं रावणेन कृतं महत्}
{वधाय सीता साऽऽनीता दशग्रीवेण रक्षसा} %6-95-11

\twolineshloka
{न च सीतां दशग्रीवः प्राप्नोति जनकात्मजाम्}
{बद्धं बलवता वैरमक्षयं राघवेण च} %6-95-12

\twolineshloka
{वैदेहीं प्रार्थयानं तं विराधं प्रेक्ष्य राक्षसम्}
{हतमेकेन रामेण पर्याप्तं तन्निदर्शनम्} %6-95-13

\twolineshloka
{चतुर्दश सहस्राणि रक्षसां भीमकर्मणाम्}
{निहतानि जनस्थाने शरैरग्निशिखोपमैः} %6-95-14

\twolineshloka
{खरश्च निहतः संख्ये दूषणस्त्रिशिरास्तथा}
{शरैरादित्यसंकाशैः पर्याप्तं तन्निदर्शनम्} %6-95-15

\twolineshloka
{हतो योजनबाहुश्च कबन्धो रुधिराशनः}
{क्रोधान्नादं नदन् सोऽथ पर्याप्तं तन्निदर्शनम्} %6-95-16

\twolineshloka
{जघान बलिनं रामः सहस्रनयनात्मजम्}
{वालिनं मेरुसंकाशं पर्याप्तं तन्निदर्शनम्} %6-95-17

\twolineshloka
{ऋष्यमूके वसंश्चैव दीनो भग्नमनोरथः}
{सुग्रीवः प्रापितो राज्यं पर्याप्तं तन्निदर्शनम्} %6-95-18

\twolineshloka
{धर्मार्थसहितं वाक्यं सर्वेषां रक्षसां हितम्}
{युक्तं विभीषणेनोक्तं मोहात् तस्य न रोचते} %6-95-19

\twolineshloka
{विभीषणवचः कुर्याद् यदि स्म धनदानुजः}
{श्मशानभूता दुःखार्ता नेयं लङ्का भविष्यति} %6-95-20

\threelineshloka
{कुम्भकर्णं हतं श्रुत्वा राघवेण महाबलम्}
{अतिकायं च दुर्मर्षं लक्ष्मणेन हतं तदा}
{प्रियं चेन्द्रजितं पुत्रं रावणो नावबुध्यते} %6-95-21

\twolineshloka
{मम पुत्रो मम भ्राता मम भर्ता रणे हतः}
{इत्येष श्रूयते शब्दो राक्षसीनां कुले कुले} %6-95-22

\twolineshloka
{रथाश्वनागाश्च हतास्तत्र तत्र सहस्रशः}
{रणे रामेण शूरेण हताश्चापि पदातयः} %6-95-23

\twolineshloka
{रुद्रो वा यदि वा विष्णुर्महेन्द्रो वा शतक्रतुः}
{हन्ति नो रामरूपेण यदि वा स्वयमन्तकः} %6-95-24

\twolineshloka
{हतप्रवीरा रामेण निराशा जीविते वयम्}
{अपश्यन्त्यो भयस्यान्तमनाथा विलपामहे} %6-95-25

\twolineshloka
{रामहस्ताद् दशग्रीवः शूरो दत्तमहावरः}
{इदं भयं महाघोरं समुत्पन्नं न बुद्ध्यते} %6-95-26

\twolineshloka
{तं न देवा न गन्धर्वा न पिशाचा न राक्षसाः}
{उपसृष्टं परित्रातुं शक्ता रामेण संयुगे} %6-95-27

\twolineshloka
{उत्पाताश्चापि दृश्यन्ते रावणस्य रणे रणे}
{कथयन्ति हि रामेण रावणस्य निबर्हणम्} %6-95-28

\twolineshloka
{पितामहेन प्रीतेन देवदानवराक्षसैः}
{रावणस्याभयं दत्तं मनुष्येभ्यो न याचितम्} %6-95-29

\twolineshloka
{तदिदं मानुषं मन्ये प्राप्तं निःसंशयं भयम्}
{जीवितान्तकरं घोरं रक्षसां रावणस्य च} %6-95-30

\twolineshloka
{पीड्यमानास्तु बलिना वरदानेन रक्षसा}
{दीप्तैस्तपोभिर्विबुधाः पितामहमपूजयन्} %6-95-31

\twolineshloka
{देवतानां हितार्थाय महात्मा वै पितामहः}
{उवाच देवतास्तुष्ट इदं सर्वा महद्वचः} %6-95-32

\twolineshloka
{अद्यप्रभृति लोकांस्त्रीन् सर्वे दानवराक्षसाः}
{भयेन प्रभृता नित्यं विचरिष्यन्ति शाश्वतम्} %6-95-33

\twolineshloka
{दैवतैस्तु समागम्य सर्वैश्चेन्द्रपुरोगमैः}
{वृषध्वजस्त्रिपुरहा महादेवः प्रतोषितः} %6-95-34

\twolineshloka
{प्रसन्नस्तु महादेवो देवानेतद् वचोऽब्रवीत्}
{उत्पत्स्यति हितार्थं वो नारी रक्षःक्षयावहा} %6-95-35

\twolineshloka
{एषा देवैः प्रयुक्ता तु क्षुद् यथा दानवान् पुरा}
{भक्षयिष्यति नः सर्वान् राक्षसघ्नी सरावणान्} %6-95-36

\twolineshloka
{रावणस्यापनीतेन दुर्विनीतस्य दुर्मतेः}
{अयं निष्टानको घोरः शोकेन समभिप्लुतः} %6-95-37

\twolineshloka
{तं न पश्यामहे लोके यो नः शरणदो भवेत्}
{राघवेणोपसृष्टानां कालेनेव युगक्षये} %6-95-38

\twolineshloka
{नास्ति नः शरणं किंचिद् भये महति तिष्ठताम्}
{दावाग्निवेष्टितानां हि करेणूनां यथा वने} %6-95-39

\twolineshloka
{प्राप्तकालं कृतं तेन पौलस्त्येन महात्मना}
{यत एव भयं दृष्टं तमेव शरणं गतः} %6-95-40

\twolineshloka
{इतीव सर्वा रजनीचरस्त्रियः परस्परं सम्परिरभ्य बाहुभिः}
{विषेदुरार्तातिभयाभिपीडिता विनेदुरुच्चैश्च तदा सुदारुणम्} %6-95-41


॥इत्यार्षे श्रीमद्रामायणे वाल्मीकीये आदिकाव्ये युद्धकाण्डे राक्षसीविलापः नाम पञ्चनवतितमः सर्गः ॥६-९५॥
