\sect{अष्टाधिकशततमः सर्गः — शुभाशुभनिमित्तदर्शनम्}

\twolineshloka
{सारथिः स रथं हृष्टः परसैन्यप्रधर्षणम्}
{गन्धर्वनगराकारं समुच्छ्रितपताकिनम्} %6-108-1

\twolineshloka
{युक्तं परमसम्पन्नैर्वाजिभिर्हेममालिभिः}
{युद्धोपकरणैः पूर्णं पताकाध्वजमालिनम्} %6-108-2

\twolineshloka
{ग्रसन्तमिव चाकाशं नादयन्तं वसुन्धराम्}
{प्रणाशं परसैन्यानां स्वसैन्यस्य प्रहर्षणम्} %6-108-3

\twolineshloka
{रावणस्य रथं क्षिप्रं चोदयामास सारथिः}
{तमापतन्तं सहसा स्वनवन्तं महाध्वजम्} %6-108-4

\twolineshloka
{रथं राक्षसराजस्य नरराजो ददर्श ह}
{कृष्णवाजिसमायुक्तं युक्तं रौद्रेण वर्चसा} %6-108-5

\twolineshloka
{दीप्यमानमिवाकाशे विमानं सूर्यवर्चसम्}
{तडित्पताकागहनं दर्शितेन्द्रायुधप्रभम्} %6-108-6

\twolineshloka
{शरधारा विमुञ्चन्तं धाराधरमिवाम्बुदम्}
{स दृष्ट्वा मेघसङ्काशमापतन्तं रथं रिपोः} %6-108-7

\twolineshloka
{गिरेर्वज्राभिमृष्टस्य दीर्यतः सदृशस्वनम्}
{विस्फारयन् वै वेगेन बालचन्द्रानतं धनुः} %6-108-8

\twolineshloka
{उवाच मातलिं रामः सहस्राक्षस्य सारथिम्}
{मातले पश्य संरब्धमापतन्तं रथं रिपोः} %6-108-9

\twolineshloka
{यथापसव्यं पतता वेगेन महता पुनः}
{समरे हन्तुमात्मानं तथानेन कृता मतिः} %6-108-10

\twolineshloka
{तदप्रमादमातिष्ठ प्रत्युद्गच्छ रथं रिपोः}
{विध्वंसयितुमिच्छामि वायुर्मेघमिवोत्थितम्} %6-108-11

\twolineshloka
{अविक्लवमसम्भ्रान्तमव्यग्रहृदयेक्षणम्}
{रश्मिसञ्चारनियतं प्रचोदय रथं द्रुतम्} %6-108-12

\twolineshloka
{कामं न त्वं समाधेयः पुरन्दररथोचितः}
{युयुत्सुरहमेकाग्रः स्मारये त्वां न शिक्षये} %6-108-13

\twolineshloka
{परितुष्टः स रामस्य तेन वाक्येन मातलिः}
{प्रचोदयामास रथं सुरसारथिरुत्तमः} %6-108-14

\twolineshloka
{अपसव्यं ततः कुर्वन् रावणस्य महारथम्}
{चक्रसम्भूतरजसा रावणं व्यवधूनयत्} %6-108-15

\twolineshloka
{ततः क्रुद्धो दशग्रीवस्ताम्रविस्फारितेक्षणः}
{रथप्रतिमुखं रामं सायकैरवधूनयत्} %6-108-16

\twolineshloka
{धर्षणामर्षितो रामो धैर्यं रोषेण लम्भयन्}
{जग्राह सुमहावेगमैन्द्रं युधि शरासनम्} %6-108-17

\threelineshloka
{शरांश्च सुमहावेगान् सूर्यरश्मिसमप्रभान्}
{तदुपोढं महद् युद्धमन्योन्यवधकाङ्क्षिणोः}
{परस्पराभिमुखयोर्दृप्तयोरिव सिंहयोः} %6-108-18

\twolineshloka
{ततो देवाः सगन्धर्वाः सिद्धाश्च परमर्षयः}
{समीयुर्द्वैरथं द्रष्टुं रावणक्षयकाङ्क्षिणः} %6-108-19

\twolineshloka
{समुत्पेतुरथोत्पाता दारुणा रोमहर्षणाः}
{रावणस्य विनाशाय राघवस्योदयाय च} %6-108-20

\twolineshloka
{ववर्ष रुधिरं देवो रावणस्य रथोपरि}
{वाता मण्डलिनस्तीव्रा व्यपसव्यं प्रचक्रमुः} %6-108-21

\twolineshloka
{महद्गृध्रकुलं चास्य भ्रममाणं नभस्थले}
{येन येन रथो याति तेन तेन प्रधावति} %6-108-22

\twolineshloka
{सन्ध्यया चावृता लङ्का जपापुष्पनिकाशया}
{दृश्यते सम्प्रदीप्तेव दिवसेऽपि वसुन्धरा} %6-108-23

\twolineshloka
{सनिर्घाता महोल्काश्च सम्प्रपेतुर्महास्वनाः}
{विषादयंस्ते रक्षांसि रावणस्य तदाहिताः} %6-108-24

\twolineshloka
{रावणश्च यतस्तत्र प्रचचाल वसुन्धरा}
{रक्षसां च प्रहरतां गृहीता इव बाहवः} %6-108-25

\twolineshloka
{ताम्राः पीताः सिताः श्वेताः पतिताः सूर्यरश्मयः}
{दृश्यन्ते रावणस्याग्रे पर्वतस्येव धातवः} %6-108-26

\twolineshloka
{गृध्रैरनुगताश्चास्य वमन्त्यो ज्वलनं मुखैः}
{प्रणेदुर्मुखमीक्षन्त्यः संरब्धमशिवं शिवाः} %6-108-27

\twolineshloka
{प्रतिकूलं ववौ वायू रणे पांसून् समुत्किरन्}
{तस्य राक्षसराजस्य कुर्वन् दृष्टिविलोपनम्} %6-108-28

\twolineshloka
{निपेतुरिन्द्राशनयः सैन्ये चास्य समन्ततः}
{दुर्विषह्यस्वरा घोरा विना जलधरोदयम्} %6-108-29

\twolineshloka
{दिशश्च प्रदिशः सर्वा बभूवुस्तिमिरावृताः}
{पांसुवर्षेण महता दुर्दर्शं च नभोऽभवत्} %6-108-30

\twolineshloka
{कुर्वन्त्यः कलहं घोरं सारिकास्तद्रथं प्रति}
{निपेतुः शतशस्तत्र दारुणा दारुणारुताः} %6-108-31

\twolineshloka
{जघनेभ्यः स्फुलिङ्गाश्च नेत्रेभ्योऽश्रूणि सन्ततम्}
{मुमुचुस्तस्य तुरगास्तुल्यमग्निं च वारि च} %6-108-32

\twolineshloka
{एवम्प्रकारा बहवः समुत्पाता भयावहाः}
{रावणस्य विनाशाय दारुणाः सम्प्रजज्ञिरे} %6-108-33

\twolineshloka
{रामस्यापि निमित्तानि सौम्यानि च शिवानि च}
{बभूवुर्जयशंसीनि प्रादुर्भूतानि सर्वशः} %6-108-34

\twolineshloka
{निमित्तानीह सौम्यानि राघवः स्वजयाय वै}
{दृष्ट्वा परमसंहृष्टो हतं मेने च रावणम्} %6-108-35

\twolineshloka
{ततो निरीक्ष्यात्मगतानि राघवो रणे निमित्तानि निमित्तकोविदः}
{जगाम हर्षं च परां च निर्वृतिं चकार युद्धे ह्यधिकं च विक्रमम्} %6-108-36


॥इत्यार्षे श्रीमद्रामायणे वाल्मीकीये आदिकाव्ये युद्धकाण्डे शुभाशुभनिमित्तदर्शनम् नाम अष्टाधिकशततमः सर्गः ॥६-१०८॥
