\sect{षडशीतितमः सर्गः — रावणिबलकदनम्}

\twolineshloka
{अथ तस्यामवस्थायां लक्ष्मणं रावणानुजः}
{परेषामहितं वाक्यमर्थसाधकमब्रवीत्} %6-86-1

\twolineshloka
{यदेतद् राक्षसानीकं मेघश्यामं विलोक्यते}
{एतदायोध्यतां शीघ्रं कपिभिश्च शिलायुधैः} %6-86-2

\twolineshloka
{तस्यानीकस्य महतो भेदने यत लक्ष्मण}
{राक्षसेन्द्रसुतोऽप्यत्र भिन्ने दृश्यो भविष्यति} %6-86-3

\twolineshloka
{स त्वमिन्द्राशनिप्रख्यैः शरैरवकिरन् परान्}
{अभिद्रवाशु यावद् वै नैतत् कर्म समाप्यते} %6-86-4

\twolineshloka
{जहि वीर दुरात्मानं मायापरमधार्मिकम्}
{रावणिं क्रूरकर्माणं सर्वलोकभयावहम्} %6-86-5

\twolineshloka
{विभीषणवचः श्रुत्वा लक्ष्मणः शुभलक्षणः}
{ववर्ष शरवर्षेण राक्षसेन्द्रसुतं प्रति} %6-86-6

\twolineshloka
{ऋक्षाः शाखामृगाश्चैव द्रुमप्रवरयोधिनः}
{अभ्यधावन्त सहितास्तदनीकमवस्थितम्} %6-86-7

\twolineshloka
{राक्षसाश्च शितैर्बाणैरसिभिः शक्तितोमरैः}
{अभ्यवर्तन्त समरे कपिसैन्यजिघांसवः} %6-86-8

\twolineshloka
{स सम्प्रहारस्तुमुलः संजज्ञे कपिरक्षसाम्}
{शब्देन महता लङ्कां नादयन् वै समन्ततः} %6-86-9

\twolineshloka
{शस्त्रैश्च विविधाकारैः शितैर्बाणैश्च पादपैः}
{उद्यतैर्गिरिशृङ्गैश्च घोरैराकाशमावृतम्} %6-86-10

\twolineshloka
{राक्षसा वानरेन्द्रेषु विकृताननबाहवः}
{निवेशयन्तः शस्त्राणि चक्रुस्ते सुमहद्भयम्} %6-86-11

\twolineshloka
{तथैव सकलैर्वृक्षैर्गिरिशृङ्गैश्च वानराः}
{अभिजघ्नुर्निजघ्नुश्च समरे सर्वराक्षसान्} %6-86-12

\twolineshloka
{ऋक्षवानरमुख्यैश्च महाकायैर्महाबलैः}
{रक्षसां युध्यमानानां महद्भयमजायत} %6-86-13

\twolineshloka
{स्वमनीकं विषण्णं तु श्रुत्वा शत्रुभिरर्दितम्}
{उदतिष्ठत दुर्धर्षः स कर्मण्यननुष्ठिते} %6-86-14

\twolineshloka
{वृक्षान्धकारान्निर्गत्य जातक्रोधः स रावणिः}
{आरुरोह रथं सज्जं पूर्वयुक्तं सुसंयतम्} %6-86-15

\twolineshloka
{स भीमकार्मुकशरः कृष्णाञ्जनचयोपमः}
{रक्तास्यनयनो भीमो बभौ मृत्युरिवान्तकः} %6-86-16

\twolineshloka
{दृष्ट्वैव तु रथस्थं तं पर्यवर्तत तद् बलम्}
{रक्षसां भीमवेगानां लक्ष्मणेन युयुत्सताम्} %6-86-17

\twolineshloka
{तस्मिंस्तु काले हनुमानरुजत् स दुरासदम्}
{धरणीधरसंकाशो महावृक्षमरिंदमः} %6-86-18

\twolineshloka
{स राक्षसानां तत् सैन्यं कालाग्निरिव निर्दहन्}
{चकार बहुभिर्वृक्षैर्निःसंज्ञं युधि वानरः} %6-86-19

\twolineshloka
{विध्वंसयन्तं तरसा दृष्ट्वैव पवनात्मजम्}
{राक्षसानां सहस्राणि हनूमन्तमवाकिरन्} %6-86-20

\twolineshloka
{शितशूलधराः शूलैरसिभिश्चासिपाणयः}
{शक्तिहस्ताश्च शक्तीभिः पट्टिशैः पट्टिशायुधाः} %6-86-21

\twolineshloka
{परिघैश्च गदाभिश्च कुन्तैश्च शुभदर्शनैः}
{शतशश्च शतघ्नीभिरायसैरपि मुद्गरैः} %6-86-22

\twolineshloka
{घोरैः परशुभिश्चैव भिन्दिपालैश्च राक्षसाः}
{मुष्टिभिर्वज्रकल्पैश्च तलैरशनिसंनिभैः} %6-86-23

\twolineshloka
{अभिजघ्नुः समासाद्य समन्तात् पर्वतोपमम्}
{तेषामपि च संक्रुद्धश्चकार कदनं महत्} %6-86-24

\twolineshloka
{स ददर्श कपिश्रेष्ठमचलोपममिन्द्रजित्}
{सूदमानमसंत्रस्तममित्रान् पवनात्मजम्} %6-86-25

\twolineshloka
{स सारथिमुवाचेदं याहि यत्रैष वानरः}
{क्षयमेव हि नः कुर्याद् राक्षसानामुपेक्षितः} %6-86-26

\twolineshloka
{इत्युक्तः सारथिस्तेन ययौ यत्र स मारुतिः}
{वहन् परमदुर्धर्षं स्थितमिन्द्रजितं रथे} %6-86-27

\twolineshloka
{सोऽभ्युपेत्य शरान् खड्गान् पट्टिशांश्च परश्वधान्}
{अभ्यवर्षत दुर्धर्षः कपिमूर्धनि राक्षसः} %6-86-28

\twolineshloka
{तानि शस्त्राणि घोराणि प्रतिगृह्य स मारुतिः}
{रोषेण महताविष्टो वाक्यं चेदमुवाच ह} %6-86-29

\twolineshloka
{युध्यस्व यदि शूरोऽसि रावणात्मज दुर्मते}
{वायुपुत्रं समासाद्य न जीवन् प्रतियास्यसि} %6-86-30

\twolineshloka
{बाहुभ्यां सम्प्रयुध्यस्व यदि मे द्वन्द्वमाहवे}
{वेगं सहस्व दुर्बुद्धे ततस्त्वं रक्षसां वरः} %6-86-31

\twolineshloka
{हनूमन्तं जिघांसन्तं समुद्यतशरासनम्}
{रावणात्मजमाचष्टे लक्ष्मणाय विभीषणः} %6-86-32

\twolineshloka
{यः स वासवनिर्जेता रावणस्यात्मसम्भवः}
{स एष रथमास्थाय हनूमन्तं जिघांसति} %6-86-33

\twolineshloka
{तमप्रतिमसंस्थानैः शरैः शत्रुनिवारणैः}
{जीवितान्तकरैर्घोरैः सौमित्रे रावणिं जहि} %6-86-34

\twolineshloka
{इत्येवमुक्तस्तु तदा महात्मा विभीषणेनारिविभीषणेन}
{ददर्श तं पर्वतसंनिकाशं रथस्थितं भीमबलं दुरासदम्} %6-86-35


॥इत्यार्षे श्रीमद्रामायणे वाल्मीकीये आदिकाव्ये युद्धकाण्डे रावणिबलकदनम् नाम षडशीतितमः सर्गः ॥६-८६॥
