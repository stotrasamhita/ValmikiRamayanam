\sect{अशीतितमः सर्गः — तिरोहितरावणियुद्धम्}

\twolineshloka
{मकराक्षं हतं श्रुत्वा रावणः समितिंजयः}
{रोषेण महताविष्टो दन्तान् कटकटाय्य च} %6-80-1

\twolineshloka
{कुपितश्च तदा तत्र किं कार्यमिति चिन्तयन्}
{आदिदेशाथ संक्रुद्धो रणायेन्द्रजितं सुतम्} %6-80-2

\twolineshloka
{जहि वीर महावीर्यौ भ्रातरौ रामलक्ष्मणौ}
{अदृश्यो दृश्यमानो वा सर्वथा त्वं बलाधिकः} %6-80-3

\twolineshloka
{त्वमप्रतिमकर्माणमिन्द्रं जयसि संयुगे}
{किं पुनर्मानुषौ दृष्ट्वा न वधिष्यसि संयुगे} %6-80-4

\twolineshloka
{तथोक्तो राक्षसेन्द्रेण प्रतिगृह्य पितुर्वचः}
{यज्ञभूमौ स विधिवत् पावकं जुहुवेन्द्रजित्} %6-80-5

\twolineshloka
{जुह्वतश्चापि तत्राग्निं रक्तोष्णीषधराः स्त्रियः}
{आजग्मुस्तत्र सम्भ्रान्ता राक्षस्यो यत्र रावणिः} %6-80-6

\twolineshloka
{शस्त्राणि शरपत्राणि समिधोऽथ बिभीतकाः}
{लोहितानि च वासांसि स्रुवं कार्ष्णायसं तथा} %6-80-7

\twolineshloka
{सर्वतोऽग्निं समास्तीर्य शरपत्रैः सतोमरैः}
{छागस्य सर्वकृष्णस्य गलं जग्राह जीवतः} %6-80-8

\twolineshloka
{सकृद्धोमसमिद्धस्य विधूमस्य महार्चिषः}
{बभूवुस्तानि लिङ्गानि विजयं दर्शयन्ति च} %6-80-9

\twolineshloka
{प्रदक्षिणावर्तशिखस्तप्तहाटकसंनिभः}
{हविस्तत् प्रतिजग्राह पावकः स्वयमुत्थितः} %6-80-10

\twolineshloka
{हुत्वाग्निं तर्पयित्वाथ देवदानवराक्षसान्}
{आरुरोह रथश्रेष्ठमन्तर्धानगतं शुभम्} %6-80-11

\twolineshloka
{स वाजिभिश्चतुर्भिस्तु बाणैस्तु निशितैर्युतः}
{आरोपितमहाचापः शुशुभे स्यन्दनोत्तमः} %6-80-12

\twolineshloka
{जाज्वल्यमानो वपुषा तपनीयपरिच्छदः}
{मृगैश्चन्द्रार्धचन्द्रैश्च स रथः समलंकृतः} %6-80-13

\twolineshloka
{जाम्बूनदमहाकम्बुर्दीप्तपावकसंनिभः}
{बभूवेन्द्रजितः केतुर्वैदूर्यसमलंकृतः} %6-80-14

\twolineshloka
{तेन चादित्यकल्पेन ब्रह्मास्त्रेण च पालितः}
{स बभूव दुराधर्षो रावणिः सुमहाबलः} %6-80-15

\twolineshloka
{सोऽभिनिर्याय नगरादिन्द्रजित् समितिंजयः}
{हुत्वाग्निं राक्षसैर्मन्त्रैरन्तर्धानगतोऽब्रवीत्} %6-80-16

\twolineshloka
{अद्य हत्वा रणे यौ तौ मिथ्या प्रव्रजितौ वने}
{जयं पित्रे प्रदास्यामि रावणाय रणेऽधिकम्} %6-80-17

\twolineshloka
{अद्य निर्वानरामुर्वीं हत्वा रामं च लक्ष्मणम्}
{करिष्ये परमां प्रीतिमित्युक्त्वान्तरधीयत} %6-80-18

\twolineshloka
{आपपाताथ संक्रुद्धो दशग्रीवेण चोदितः}
{तीक्ष्णकार्मुकनाराचैस्तीक्ष्णस्त्विन्द्ररिपू रणे} %6-80-19

\twolineshloka
{स ददर्श महावीर्यौ नागौ त्रिशिरसाविव}
{सृजन्ताविषुजालानि वीरौ वानरमध्यगौ} %6-80-20

\twolineshloka
{इमौ ताविति संचिन्त्य सज्यं कृत्वा च कार्मुकम्}
{संततानेषुधाराभिः पर्जन्य इव वृष्टिमान्} %6-80-21

\twolineshloka
{स तु वैहायसरथो युधि तौ रामलक्ष्मणौ}
{अचक्षुर्विषये तिष्ठन् विव्याध निशितैः शरैः} %6-80-22

\twolineshloka
{तौ तस्य शरवेगेन परीतौ रामलक्ष्मणौ}
{धनुषी सशरे कृत्वा दिव्यमस्त्रं प्रचक्रतुः} %6-80-23

\twolineshloka
{प्रच्छादयन्तौ गगनं शरजालैर्महाबलौ}
{तमस्त्रैः सूर्यसंकाशैर्नैव पस्पर्शतुः शरैः} %6-80-24

\twolineshloka
{स हि धूमान्धकारं च चक्रे प्रच्छादयन्नभः}
{दिशश्चान्तर्दधे श्रीमान् नीहारतमसा वृताः} %6-80-25

\twolineshloka
{नैव ज्यातलनिर्घोषो न च नेमिखुरस्वनः}
{शुश्रुवे चरतस्तस्य न च रूपं प्रकाशते} %6-80-26

\twolineshloka
{घनान्धकारे तिमिरे शिलावर्षमिवाद्भुतम्}
{स ववर्ष महाबाहुर्नाराचशरवृष्टिभिः} %6-80-27

\twolineshloka
{स रामं सूर्यसंकाशैः शरैर्दत्तवरैर्भृशम्}
{विव्याध समरे क्रुद्धः सर्वगात्रेषु रावणिः} %6-80-28

\twolineshloka
{तौ हन्यमानौ नाराचैर्धाराभिरिव पर्वतौ}
{हेमपुङ्खान् नरव्याघ्रौ तिग्मान् मुमुचतुः शरान्} %6-80-29

\twolineshloka
{अन्तरिक्षे समासाद्य रावणिं कङ्कपत्रिणः}
{निकृत्य पतगा भूमौ पेतुस्ते शोणिताप्लुताः} %6-80-30

\twolineshloka
{अतिमात्रं शरौघेण दीप्यमानौ नरोत्तमौ}
{तानिषून् पततो भल्लैरनेकैर्विचकर्ततुः} %6-80-31

\twolineshloka
{यतो हि ददृशाते तौ शरान् निपतिताञ्छितान्}
{ततस्तु तौ दाशरथी ससृजातेऽस्त्रमुत्तमम्} %6-80-32

\twolineshloka
{रावणिस्तु दिशः सर्वा रथेनातिरथोऽपतत्}
{विव्याध तौ दाशरथी लघ्वस्त्रो निशितैः शरैः} %6-80-33

\twolineshloka
{तेनातिविद्धौ तौ वीरौ रुक्मपुङ्खैः सुसंहतैः}
{बभूवतुर्दाशरथी पुष्पिताविव किंशुकौ} %6-80-34

\twolineshloka
{नास्य वेगगतिं कश्चिन्न च रूपं धनुः शरान्}
{न चास्य विदितं किंचित् सूर्यस्येवाभ्रसम्प्लवे} %6-80-35

\twolineshloka
{तेन विद्धाश्च हरयो निहताश्च गतासवः}
{बभूवुः शतशस्तत्र पतिता धरणीतले} %6-80-36

\twolineshloka
{लक्ष्मणस्तु ततः क्रुद्धो भ्रातरं वाक्यमब्रवीत्}
{ब्राह्ममस्त्रं प्रयोक्ष्यामि वधार्थं सर्वरक्षसाम्} %6-80-37

\twolineshloka
{तमुवाच ततो रामो लक्ष्मणं शुभलक्षणम्}
{नैकस्य हेतो रक्षांसि पृथिव्यां हन्तुमर्हसि} %6-80-38

\twolineshloka
{अयुध्यमानं प्रच्छन्नं प्राञ्जलिं शरणागतम्}
{पलायमानं मत्तं वा न हन्तुं त्वमिहार्हसि} %6-80-39

\twolineshloka
{तस्यैव तु वधे यत्नं करिष्यामि महाभुज}
{आदेक्ष्यावो महावेगानस्त्रानाशीविषोपमान्} %6-80-40

\twolineshloka
{तमेनं मायिनं क्षुद्रमन्तर्हितरथं बलात्}
{राक्षसं निहनिष्यन्ति दृष्ट्वा वानरयूथपाः} %6-80-41

\twolineshloka
{यद्येष भूमिं विशते दिवं वा रसातलं वापि नभस्तलं वा}
{एवं विगूढोऽपि ममास्त्रदग्धः पतिष्यते भूमितले गतासुः} %6-80-42

\twolineshloka
{इत्येवमुक्त्वा वचनं महार्थं रघुप्रवीरः प्लवगर्षभैर्वृतः}
{वधाय रौद्रस्य नृशंसकर्मणस्तदा महात्मा त्वरितं निरीक्षते} %6-80-43


॥इत्यार्षे श्रीमद्रामायणे वाल्मीकीये आदिकाव्ये युद्धकाण्डे तिरोहितरावणियुद्धम् नाम अशीतितमः सर्गः ॥६-८०॥
