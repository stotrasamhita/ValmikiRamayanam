\sect{चतुर्दशः सर्गः — प्रहस्तविभीषणविवादः}

\twolineshloka
{निशाचरेन्द्रस्य निशम्य वाक्यं स कुम्भकर्णस्य च गर्जितानि}
{विभीषणो राक्षसराजमुख्यमुवाच वाक्यं हितमर्थयुक्तम्} %6-14-1

\twolineshloka
{वृतो हि बाह्वन्तरभोगराशिश्चिन्ताविषः सुस्मिततीक्ष्णदंष्ट्रः}
{पञ्चाङ्गुलीपञ्चशिरोऽतिकायः सीतामहाहिस्तव केन राजन्} %6-14-2

\twolineshloka
{यावन्न लङ्कां समभिद्रवन्ति बलीमुखाः पर्वतकूटमात्राः}
{दंष्ट्रायुधाश्चैव नखायुधाश्च प्रदीयतां दाशरथाय मैथिली} %6-14-3

\twolineshloka
{यावन्न गृह्णन्ति शिरांसि बाणा रामेरिता राक्षसपुंगवानाम्}
{वज्रोपमा वायुसमानवेगाः प्रदीयतां दाशरथाय मैथिली} %6-14-4

\twolineshloka
{न कुम्भकर्णेन्द्रजितौ च राजंस्तथा महापार्श्वमहोदरौ वा}
{निकुम्भकुम्भौ च तथातिकायः स्थातुं समर्था युधि राघवस्य} %6-14-5

\twolineshloka
{जीवंस्तु रामस्य न मोक्ष्यसे त्वं गुप्तः सवित्राप्यथवा मरुद्भिः}
{न वासवस्याङ्कगतो न मृत्योर्नभो न पातालमनुप्रविष्टः} %6-14-6

\twolineshloka
{निशम्य वाक्यं तु विभीषणस्य ततः प्रहस्तो वचनं बभाषे}
{न नो भयं विद्म न दैवतेभ्यो न दानवेभ्योऽप्यथवा कदाचित्} %6-14-7

\twolineshloka
{न यक्षगन्धर्वमहोरगेभ्यो भयं न संख्ये पतगोरगेभ्यः}
{कथं नु रामाद् भविता भयं नो नरेन्द्रपुत्रात् समरे कदाचित्} %6-14-8

\twolineshloka
{प्रहस्तवाक्यं त्वहितं निशम्य विभीषणो राजहितानुकाङ्क्षी}
{ततो महार्थं वचनं बभाषे धर्मार्थकामेषु निविष्टबुद्धिः} %6-14-9

\twolineshloka
{प्रहस्त राजा च महोदरश्च त्वं कुम्भकर्णश्च यथार्थजातम्}
{ब्रवीत रामं प्रति तन्न शक्यं यथा गतिः स्वर्गमधर्मबुद्धेः} %6-14-10

\twolineshloka
{वधस्तु रामस्य मया त्वया च प्रहस्त सर्वैरपि राक्षसैर्वा}
{कथं भवेदर्थविशारदस्य महार्णवं तर्तुमिवाप्लवस्य} %6-14-11

\twolineshloka
{धर्मप्रधानस्य महारथस्य इक्ष्वाकुवंशप्रभवस्य राज्ञः}
{पुरोऽस्य देवाश्च तथाविधस्य कृत्येषु शक्तस्य भवन्ति मूढाः} %6-14-12

\twolineshloka
{तीक्ष्णा न तावत् तव कङ्कपत्रा दुरासदा राघवविप्रमुक्ताः}
{भित्त्वा शरीरं प्रविशन्ति बाणाः प्रहस्त तेनैव विकत्थसे त्वम्} %6-14-13

\twolineshloka
{भित्त्वा न तावत् प्रविशन्ति कायं प्राणान्तिकास्तेऽशनितुल्यवेगाः}
{शिताः शरा राघवविप्रमुक्ताः प्रहस्त तेनैव विकत्थसे त्वम्} %6-14-14

\twolineshloka
{न रावणो नातिबलस्त्रिशीर्षो न कुम्भकर्णस्य सुतो निकुम्भः}
{न चेन्द्रजिद् दाशरथिं प्रवोढुं त्वं वा रणे शक्रसमं समर्थः} %6-14-15

\twolineshloka
{देवान्तको वापि नरान्तको वा तथातिकायोऽतिरथो महात्मा}
{अकम्पनश्चाद्रिसमानसारः स्थातुं न शक्ता युधि राघवस्य} %6-14-16

\twolineshloka
{अयं च राजा व्यसनाभिभूतो मित्रैरमित्रप्रतिमैर्भवद्भिः}
{अन्वास्यते राक्षसनाशनार्थे तीक्ष्णः प्रकृत्या ह्यसमीक्षकारी} %6-14-17

\twolineshloka
{अनन्तभोगेन सहस्रमूर्ध्ना नागेन भीमेन महाबलेन}
{बलात् परिक्षिप्तमिमं भवन्तो राजानमुत्क्षिप्य विमोचयन्तु} %6-14-18

\twolineshloka
{यावद्धि केशग्रहणात् सुहृद्भिः समेत्य सर्वैः परिपूर्णकामैः}
{निगृह्य राजा परिरक्षितव्यो भूतैर्यथा भीमबलैर्गृहीतः} %6-14-19

\twolineshloka
{सुवारिणा राघवसागरेण प्रच्छाद्यमानस्तरसा भवद्भिः}
{युक्तस्त्वयं तारयितुं समेत्य काकुत्स्थपातालमुखे पतन् सः} %6-14-20

\twolineshloka
{इदं पुरस्यास्य सराक्षसस्य राज्ञश्च पथ्यं ससुहृज्जनस्य}
{सम्यग्घि वाक्यं स्वमतं ब्रवीमि नरेन्द्रपुत्राय ददातु मैथिलीम्} %6-14-21

\twolineshloka
{परस्य वीर्यं स्वबलं च बुद्ध्वा स्थानं क्षयं चैव तथैव वृद्धिम्}
{तथा स्वपक्षेऽप्यनुमृश्य बुद्ध्या वदेत् क्षमं स्वामिहितं स मन्त्री} %6-14-22


॥इत्यार्षे श्रीमद्रामायणे वाल्मीकीये आदिकाव्ये युद्धकाण्डे प्रहस्तविभीषणविवादः नाम चतुर्दशः सर्गः ॥६-१४॥
