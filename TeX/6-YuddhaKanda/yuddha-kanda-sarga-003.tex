\sect{तृतीयः सर्गः — लङ्कादुर्गादिकथनम्}

\twolineshloka
{सुग्रीवस्य वचः श्रुत्वा हेतुमत् परमार्थवत्}
{प्रतिजग्राह काकुत्स्थो हनूमन्तमथाब्रवीत्} %6-3-1

\twolineshloka
{तपसा सेतुबन्धेन सागरोच्छोषणेन च}
{सर्वथापि समर्थोऽस्मि सागरस्यास्य लङ्घने} %6-3-2

\twolineshloka
{कति दुर्गाणि दुर्गाया लङ्कायास्तद् ब्रवीष्व मे}
{ज्ञातुमिच्छामि तत् सर्वं दर्शनादिव वानर} %6-3-3

\twolineshloka
{बलस्य परिमाणं च द्वारदुर्गक्रियामपि}
{गुप्तिकर्म च लङ्काया रक्षसां सदनानि च} %6-3-4

\twolineshloka
{यथासुखं यथावच्च लङ्कायामसि दृष्टवान्}
{सर्वमाचक्ष्व तत्त्वेन सर्वथा कुशलो ह्यसि} %6-3-5

\twolineshloka
{श्रुत्वा रामस्य वचनं हनूमान् मारुतात्मजः}
{वाक्यं वाक्यविदां श्रेष्ठो रामं पुनरथाब्रवीत्} %6-3-6

\twolineshloka
{श्रूयतां सर्वमाख्यास्ये दुर्गकर्म विधानतः}
{गुप्ता पुरी यथा लङ्का रक्षिता च यथा बलैः} %6-3-7

\twolineshloka
{राक्षसाश्च यथा स्निग्धा रावणस्य च तेजसा}
{परां समृद्धिं लङ्कायाः सागरस्य च भीमताम्} %6-3-8

\twolineshloka
{विभागं च बलौघस्य निर्देशं वाहनस्य च}
{एवमुक्त्वा कपिश्रेष्ठः कथयामास तत्त्वतः} %6-3-9

\twolineshloka
{हृष्टप्रमुदिता लङ्का मत्तद्विपसमाकुला}
{महती रथसम्पूर्णा रक्षोगणनिषेविता} %6-3-10

\twolineshloka
{दृढबद्धकपाटानि महापरिघवन्ति च}
{चत्वारि विपुलान्यस्या द्वाराणि सुमहान्ति च} %6-3-11

\twolineshloka
{तत्रेषूपलयन्त्राणि बलवन्ति महान्ति च}
{आगतं प्रतिसैन्यं तैस्तत्र प्रतिनिवार्यते} %6-3-12

\twolineshloka
{द्वारेषु संस्कृता भीमाः कालायसमयाः शिताः}
{शतशो रचिता वीरैः शतघ्न्यो रक्षसां गणैः} %6-3-13

\twolineshloka
{सौवर्णस्तु महांस्तस्याः प्राकारो दुष्प्रधर्षणः}
{मणिविद्रुमवैदूर्यमुक्ताविरचितान्तरः} %6-3-14

\twolineshloka
{सर्वतश्च महाभीमाः शीततोया महाशुभाः}
{अगाधा ग्राहवत्यश्च परिखा मीनसेविताः} %6-3-15

\twolineshloka
{द्वारेषु तासां चत्वारः संक्रमाः परमायताः}
{यन्त्रैरुपेता बहुभिर्महद्भिर्गृहपङ्क्तिभिः} %6-3-16

\twolineshloka
{त्रायन्ते संक्रमास्तत्र परसैन्यागते सति}
{यन्त्रैस्तैरवकीर्यन्ते परिखासु समन्ततः} %6-3-17

\twolineshloka
{एकस्त्वकम्प्यो बलवान् संक्रमः सुमहादृढः}
{काञ्चनैर्बहुभिः स्तम्भैर्वेदिकाभिश्च शोभितः} %6-3-18

\twolineshloka
{स्वयं प्रकृतिमापन्नो युयुत्सू राम रावणः}
{उत्थितश्चाप्रमत्तश्च बलानामनुदर्शने} %6-3-19

\twolineshloka
{लङ्का पुनर्निरालम्बा देवदुर्गा भयावहा}
{नादेयं पार्वतं वान्यं कृत्रिमं च चतुर्विधम्} %6-3-20

\twolineshloka
{स्थिता पारे समुद्रस्य दूरपारस्य राघव}
{नौपथश्चापि नास्त्यत्र निरुद्देशश्च सर्वतः} %6-3-21

\twolineshloka
{शैलाग्रे रचिता दुर्गा सा पूर्देवपुरोपमा}
{वाजिवारणसम्पूर्णा लङ्का परमदुर्जया} %6-3-22

\twolineshloka
{परिखाश्च शतघ्न्यश्च यन्त्राणि विविधानि च}
{शोभयन्ति पुरीं लङ्कां रावणस्य दुरात्मनः} %6-3-23

\twolineshloka
{अयुतं रक्षसामत्र पूर्वद्वारं समाश्रितम्}
{शूलहस्ता दुराधर्षाः सर्वे खड्गाग्रयोधिनः} %6-3-24

\twolineshloka
{नियुतं रक्षसामत्र दक्षिणद्वारमाश्रितम्}
{चतुरङ्गेण सैन्येन योधास्तत्राप्यनुत्तमाः} %6-3-25

\twolineshloka
{प्रयुतं रक्षसामत्र पश्चिमद्वारमाश्रितम्}
{चर्मखड्गधराः सर्वे तथा सर्वास्त्रकोविदाः} %6-3-26

\twolineshloka
{न्यर्बुदं रक्षसामत्र उत्तरद्वारमाश्रितम्}
{रथिनश्चाश्ववाहाश्च कुलपुत्राः सुपूजिताः} %6-3-27

\twolineshloka
{शतशोऽथ सहस्राणि मध्यमं स्कन्धमाश्रिताः}
{यातुधाना दुराधर्षाः साग्रकोटिश्च रक्षसाम्} %6-3-28

\threelineshloka
{ते मया संक्रमा भग्नाः परिखाश्चावपूरिताः}
{दग्धा च नगरी लङ्का प्राकाराश्चावसादिताः}
{बलैकदेशः क्षपितो राक्षसानां महात्मनाम्} %6-3-29

\twolineshloka
{येन केन तु मार्गेण तराम वरुणालयम्}
{हतेति नगरी लङ्का वानरैरुपधार्यताम्} %6-3-30

\twolineshloka
{अङ्गदो द्विविदो मैन्दो जाम्बवान् पनसो नलः}
{नीलः सेनापतिश्चैव बलशेषेण किं तव} %6-3-31

\threelineshloka
{प्लवमाना हि गत्वा त्वां रावणस्य महापुरीम्}
{सपर्वतवनां भित्त्वा सखातां च सतोरणाम्}
{सप्राकारां सभवनामानयिष्यन्ति राघव} %6-3-32

\twolineshloka
{एवमाज्ञापय क्षिप्रं बलानां सर्वसंग्रहम्}
{मुहूर्तेन तु युक्तेन प्रस्थानमभिरोचय} %6-3-33


॥इत्यार्षे श्रीमद्रामायणे वाल्मीकीये आदिकाव्ये युद्धकाण्डे लङ्कादुर्गादिकथनम् नाम तृतीयः सर्गः ॥६-३॥
