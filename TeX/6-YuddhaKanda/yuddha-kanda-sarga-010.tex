\sect{दशमः सर्गः — विभीषणपथ्योपदेशः}

\twolineshloka
{ततः प्रत्युषसि प्राप्ते प्राप्तधर्मार्थनिश्चयः}
{राक्षसाधिपतेर्वेश्म भीमकर्मा विभीषणः} %6-10-1

\twolineshloka
{शैलाग्रचयसङ्काशं शैलशृङ्गमिवोन्नतम्}
{सुविभक्तमहाकक्षं महाजनपरिग्रहम्} %6-10-2

\twolineshloka
{मतिमद्भिर्महामात्रैरनुरक्तैरधिष्ठितम्}
{राक्षसैराप्तपर्याप्तैः सर्वतः परिरक्षितम्} %6-10-3

\twolineshloka
{मत्तमातङ्गनिःश्वासैर्व्याकुलीकृतमारुतम्}
{शङ्खघोषमहाघोषं तूर्यसम्बाधनादितम्} %6-10-4

\twolineshloka
{प्रमदाजनसम्बाधं प्रजल्पितमहापथम्}
{तप्तकाञ्चननिर्यूहं भूषणोत्तमभूषितम्} %6-10-5

\twolineshloka
{गन्धर्वाणामिवावासमालयं मरुतामिव}
{रत्नसञ्चयसम्बाधं भवनं भोगिनामिव} %6-10-6

\twolineshloka
{तं महाभ्रमिवादित्यस्तेजोविस्तृतरश्मिवान्}
{अग्रजस्यालयं वीरः प्रविवेश महाद्युतिः} %6-10-7

\twolineshloka
{पुण्यान् पुण्याहघोषांश्च वेदविद्भिरुदाहृतान्}
{शुश्राव सुमहातेजा भ्रातुर्विजयसंश्रितान्} %6-10-8

\twolineshloka
{पूजितान् दधिपात्रैश्च सर्पिभिः सुमनोक्षतैः}
{मन्त्रवेदविदो विप्रान् ददर्श स महाबलः} %6-10-9

\twolineshloka
{स पूज्यमानो रक्षोभिर्दीप्यमानं स्वतेजसा}
{आसनस्थं महाबाहुर्ववन्दे धनदानुजम्} %6-10-10

\twolineshloka
{स राजदृष्टिसम्पन्नमासनं हेमभूषितम्}
{जगाम समुदाचारं प्रयुज्याचारकोविदः} %6-10-11

\twolineshloka
{स रावणं महात्मानं विजने मन्त्रिसन्निधौ}
{उवाच हितमत्यर्थं वचनं हेतुनिश्चितम्} %6-10-12

\twolineshloka
{प्रसाद्य भ्रातरं ज्येष्ठं सान्त्वेनोपस्थितक्रमः}
{देशकालार्थसंवादि दृष्टलोकपरावरः} %6-10-13

\twolineshloka
{यदाप्रभृति वैदेही सम्प्राप्तेह परन्तप}
{तदाप्रभृति दृश्यन्ते निमित्तान्यशुभानि नः} %6-10-14

\twolineshloka
{सस्फुलिङ्गः सधूमार्चिः सधूमकलुषोदयः}
{मन्त्रसन्धुक्षितोऽप्यग्निर्न सम्यगभिवर्धते} %6-10-15

\twolineshloka
{अग्निष्टेष्वग्निशालासु तथा ब्रह्मस्थलीषु च}
{सरीसृपाणि दृश्यन्ते हव्येषु च पिपीलिकाः} %6-10-16

\twolineshloka
{गवां पयांसि स्कन्नानि विमदा वरकुञ्जराः}
{दीनमश्वाः प्रहेषन्ते नवग्रासाभिनन्दिनः} %6-10-17

\twolineshloka
{खरोष्ट्राश्वतरा राजन् भिन्नरोमाः स्रवन्ति च}
{न स्वभावेऽवतिष्ठन्ते विधानैरपि चिन्तिताः} %6-10-18

\twolineshloka
{वायसाः सङ्घशः क्रूरा व्याहरन्ति समन्ततः}
{समवेताश्च दृश्यन्ते विमानाग्रेषु सङ्घशः} %6-10-19

\twolineshloka
{गृध्राश्च परिलीयन्ते पुरीमुपरि पिण्डिताः}
{उपपन्नाश्च सन्ध्ये द्वे व्याहरन्त्यशिवं शिवाः} %6-10-20

\twolineshloka
{क्रव्यादानां मृगाणां च पुरीद्वारेषु सङ्घशः}
{श्रूयन्ते विपुला घोषाः सविस्फूर्जितनिःस्वनाः} %6-10-21

\twolineshloka
{तदेवं प्रस्तुते कार्ये प्रायश्चित्तमिदं क्षमम्}
{रोचये वीर वैदेही राघवाय प्रदीयताम्} %6-10-22

\twolineshloka
{इदं च यदि वा मोहाल्लोभाद् वा व्याहृतं मया}
{तत्रापि च महाराज न दोषं कर्तुमर्हसि} %6-10-23

\twolineshloka
{अयं हि दोषः सर्वस्य जनस्यास्योपलक्ष्यते}
{रक्षसां राक्षसीनां च पुरस्यान्तःपुरस्य च} %6-10-24

\threelineshloka
{प्रापणे चास्य मन्त्रस्य निवृत्ताः सर्वमन्त्रिणः}
{अवश्यं च मया वाच्यं यद् दृष्टमथवा श्रुतम्}
{सम्प्रधार्य यथान्यायं तद् भवान् कर्तुमर्हति} %6-10-25

\twolineshloka
{इति स्वमन्त्रिणां मध्ये भ्राता भ्रातरमूचिवान्}
{रावणं रक्षसां श्रेष्ठं पथ्यमेतद् विभीषणः} %6-10-26

\twolineshloka
{हितं महार्थं मृदु हेतुसंहितं व्यतीतकालायतिसम्प्रतिक्षमम्}
{निशम्य तद्वाक्यमुपस्थितज्वरः प्रसङ्गवानुत्तरमेतदब्रवीत्} %6-10-27

\twolineshloka
{भयं न पश्यामि कुतश्चिदप्यहं न राघवः प्राप्स्यति जातु मैथिलीम्}
{सुरैः सहेन्द्रैरपि सङ्गरे कथं ममाग्रतः स्थास्यति लक्ष्मणाग्रजः} %6-10-28

\twolineshloka
{इत्येवमुक्त्वा सुरसैन्यनाशनो महाबलः संयति चण्डविक्रमः}
{दशाननो भ्रातरमाप्तवादिनं विसर्जयामास तदा विभीषणम्} %6-10-29


॥इत्यार्षे श्रीमद्रामायणे वाल्मीकीये आदिकाव्ये युद्धकाण्डे विभीषणपथ्योपदेशः नाम दशमः सर्गः ॥६-१०॥
