\sect{एकोनाशीतितमः सर्गः — मकराक्षवधः}

\twolineshloka
{निर्गतं मकराक्षं ते दृष्ट्वा वानरपुङ्गवाः}
{आप्लुत्य सहसा सर्वे योद्धुकामा व्यवस्थिताः} %6-79-1

\twolineshloka
{ततः प्रवृत्तं सुमहत् तद् युद्धं लोमहर्षणम्}
{निशाचरैः प्लवङ्गानां देवानां दानवैरिव} %6-79-2

\twolineshloka
{वृक्षशूलनिपातैश्च गदापरिघपातनैः}
{अन्योन्यं मर्दयन्ति स्म तदा कपिनिशाचराः} %6-79-3

\twolineshloka
{शक्तिखड्गगदाकुन्तैस्तोमरैश्च निशाचराः}
{पट्टिशैर्भिन्दिपालैश्च बाणपातैः समन्ततः} %6-79-4

\twolineshloka
{पाशमुद्गरदण्डैश्च निर्घातैश्चापरैस्तथा}
{कदनं कपिसिंहानां चक्रुस्ते रजनीचराः} %6-79-5

\twolineshloka
{बाणौघैरर्दिताश्चापि खरपुत्रेण वानराः}
{सम्भ्रान्तमनसः सर्वे दुद्रुवुर्भयपीडिताः} %6-79-6

\twolineshloka
{तान् दृष्ट्वा राक्षसाः सर्वे द्रवमाणान् वनौकसः}
{नेदुस्ते सिंहवद् दृप्ता राक्षसा जितकाशिनः} %6-79-7

\twolineshloka
{विद्रवत्सु तदा तेषु वानरेषु समन्ततः}
{रामस्तान् वारयामास शरवर्षेण राक्षसान्} %6-79-8

\twolineshloka
{वारितान् राक्षसान् दृष्ट्वा मकराक्षो निशाचरः}
{कोपानलसमाविष्टो वचनं चेदमब्रवीत्} %6-79-9

\twolineshloka
{तिष्ठ राम मया सार्धं द्वन्द्वयुद्धं भविष्यति}
{त्याजयिष्यामि ते प्राणान् धनुर्मुक्तैः शितैः शरैः} %6-79-10

\twolineshloka
{यत् तदा दण्डकारण्ये पितरं हतवान् मम}
{तदग्रतः स्वकर्मस्थं स्मृत्वा रोषोऽभिवर्धते} %6-79-11

\twolineshloka
{दह्यन्ते भृशमङ्गानि दुरात्मन् मम राघव}
{यन्मयासि न दृष्टस्त्वं तस्मिन् काले महावने} %6-79-12

\twolineshloka
{दिष्ट्यासि दर्शनं राम मम त्वं प्राप्तवानिह}
{काङ्क्षितोऽसि क्षुधार्तस्य सिंहस्येवेतरो मृगः} %6-79-13

\twolineshloka
{अद्य मद्बाणवेगेन प्रेतराड्विषयं गतः}
{ये त्वया निहताः शूराः सह तैश्च वसिष्यसि} %6-79-14

\twolineshloka
{बहुनात्र किमुक्तेन शृणु राम वचो मम}
{पश्यन्तु सकला लोकास्त्वां मां चैव रणाजिरे} %6-79-15

\twolineshloka
{अस्त्रैर्वा गदया वापि बाहुभ्यां वा रणाजिरे}
{अभ्यस्तं येन वा राम वर्ततां तेन वा मृधम्} %6-79-16

\twolineshloka
{मकराक्षवचः श्रुत्वा रामो दशरथात्मजः}
{अब्रवीत् प्रहसन् वाक्यमुत्तरोत्तरवादिनम्} %6-79-17

\twolineshloka
{कत्थसे किं वृथा रक्षो बहून्यसदृशानि ते}
{न रणे शक्यते जेतुं विना युद्धेन वाग्बलात्} %6-79-18

\twolineshloka
{चतुर्दश सहस्राणि रक्षसां त्वत्पिता च यः}
{त्रिशिरा दूषणश्चापि दण्डके निहतो मया} %6-79-19

\twolineshloka
{स्वाशिताश्चापि मांसेन गृध्रगोमायुवायसाः}
{भविष्यन्त्यद्य वै पाप तीक्ष्णतुण्डनखाङ्कुशाः} %6-79-20

\twolineshloka
{राघवेणैवमुक्तस्तु मकराक्षो महाबलः}
{बाणौघानमुचत् तस्मै राघवाय रणाजिरे} %6-79-21

\twolineshloka
{ताञ्छराञ्छरवर्षेण रामश्चिच्छेद नैकधा}
{निपेतुर्भुवि विच्छिन्ना रुक्मपुङ्खाः सहस्रशः} %6-79-22

\twolineshloka
{तद् युद्धमभवत् तत्र समेत्यान्योन्यमोजसा}
{खरराक्षसपुत्रस्य सूनोर्दशरथस्य च} %6-79-23

\twolineshloka
{जीमूतयोरिवाकाशे शब्दो ज्यातलयोरिव}
{धनुर्मुक्तः स्वनोऽन्योन्यं श्रूयते च रणाजिरे} %6-79-24

\twolineshloka
{देवदानवगन्धर्वाः किन्नराश्च महोरगाः}
{अन्तरिक्षगताः सर्वे द्रष्टुकामास्तदद्भुतम्} %6-79-25

\twolineshloka
{विद्धमन्योन्यगात्रेषु द्विगुणं वर्धते बलम्}
{कृतप्रतिकृतान्योन्यं कुरुतां तौ रणाजिरे} %6-79-26

\twolineshloka
{राममुक्तांस्तु बाणौघान् राक्षसस्त्वच्छिनद् रणे}
{रक्षोमुक्तांस्तु रामो वै नैकधा प्राच्छिनच्छरैः} %6-79-27

\twolineshloka
{बाणौघवितताः सर्वा दिशश्च प्रदिशस्तथा}
{सञ्छन्ना वसुधा चैव समन्तान्न प्रकाशते} %6-79-28

\twolineshloka
{ततः क्रुद्धो महाबाहुर्धनुश्चिच्छेद संयुगे}
{अष्टाभिरथ नाराचैः सूतं विव्याध राघवः} %6-79-29

\twolineshloka
{भित्त्वा रथं शरै रामो हत्वा अश्वानपातयत्}
{विरथो वसुधास्थः स मकराक्षो निशाचरः} %6-79-30

\twolineshloka
{तत्तिष्ठद् वसुधां रक्षः शूलं जग्राह पाणिना}
{त्रासनं सर्वभूतानां युगान्ताग्निसमप्रभम्} %6-79-31

\twolineshloka
{दुरवापं महच्छूलं रुद्रदत्तं भयङ्करम्}
{जाज्वल्यमानमाकाशे संहारास्त्रमिवापरम्} %6-79-32

\twolineshloka
{यं दृष्ट्वा देवताः सर्वा भयार्ता विद्रुता दिशः}
{विभ्राम्य च महच्छूलं प्रज्वलन्तं निशाचरः} %6-79-33

\twolineshloka
{स क्रोधात् प्राहिणोत् तस्मै राघवाय महाहवे}
{तमापतन्तं ज्वलितं खरपुत्रकराच्च्युतम्} %6-79-34

\threelineshloka
{बाणैश्चतुर्भिराकाशे शूलं चिच्छेद राघवः}
{स भिन्नो नैकधा शूलो दिव्यहाटकमण्डितः}
{व्यशीर्यत महोल्केव रामबाणार्दितो भुवि} %6-79-35

\twolineshloka
{तच्छूलं निहतं दृष्ट्वा रामेणाक्लिष्टकर्मणा}
{साधु साध्विति भूतानि व्याहरन्ति नभोगताः} %6-79-36

\twolineshloka
{तं दृष्ट्वा निहतं शूलं मकराक्षो निशाचरः}
{मुष्टिमुद्यम्य काकुत्स्थं तिष्ठ तिष्ठेति चाब्रवीत्} %6-79-37

\twolineshloka
{स तं दृष्ट्वा पतन्तं तु प्रहस्य रघुनन्दनः}
{पावकास्त्रं ततो रामः सन्दधे तु शरासने} %6-79-38

\twolineshloka
{तेनास्त्रेण हतं रक्षः काकुत्स्थेन तदा रणे}
{सञ्छिन्नहृदयं तत्र पपात च ममार च} %6-79-39

\twolineshloka
{दृष्ट्वा ते राक्षसाः सर्वे मकराक्षस्य पातनम्}
{लङ्कामेव प्रधावन्त रामबाणभयार्दिताः} %6-79-40

\twolineshloka
{दशरथनृपसूनुबाणवेगै रजनिचरं निहतं खरात्मजं तम्}
{प्रददृशुरथ देवताः प्रहृष्टा गिरिमिव वज्रहतं यथा विकीर्णम्} %6-79-41


॥इत्यार्षे श्रीमद्रामायणे वाल्मीकीये आदिकाव्ये युद्धकाण्डे मकराक्षवधः नाम एकोनाशीतितमः सर्गः ॥६-७९॥
