\sect{पञ्चाशः सर्गः — नागपाशविमोक्षणम्}

\twolineshloka
{अथोवाच महातेजा हरिराजो महाबलः}
{किमियं व्यथिता सेना मूढवातेव नौर्जले} %6-50-1

\twolineshloka
{सुग्रीवस्य वचः श्रुत्वा वालिपुत्रोऽङ्गदोऽब्रवीत्}
{न त्वं पश्यसि रामं च लक्ष्मणं च महारथम्} %6-50-2

\twolineshloka
{शरजालाचितौ वीरावुभौ दशरथात्मजौ}
{शरतल्पे महात्मानौ शयानौ रुधिरोक्षितौ} %6-50-3

\twolineshloka
{अथाब्रवीद् वानरेन्द्रः सुग्रीवः पुत्रमङ्गदम्}
{नानिमित्तमिदं मन्ये भवितव्यं भयेन तु} %6-50-4

\twolineshloka
{विषण्णवदना ह्येते त्यक्तप्रहरणा दिशः}
{पलायन्तेऽत्र हरयस्त्रासादुत्फुल्ललोचनाः} %6-50-5

\twolineshloka
{अन्योन्यस्य न लज्जन्ते न निरीक्षन्ति पृष्ठतः}
{विप्रकर्षन्ति चान्योन्यं पतितं लङ्घयन्ति च} %6-50-6

\twolineshloka
{एतस्मिन्नन्तरे वीरो गदापाणिर्विभीषणः}
{सुग्रीवं वर्धयामास राघवं च जयाशिषा} %6-50-7

\twolineshloka
{विभीषणं च सुग्रीवो दृष्ट्वा वानरभीषणम्}
{ऋक्षराजं महात्मानं समीपस्थमुवाच ह} %6-50-8

\twolineshloka
{विभीषणोऽयं सम्प्राप्तो यं दृष्ट्वा वानरर्षभाः}
{द्रवन्त्यायतसन्त्रासा रावणात्मजशङ्कया} %6-50-9

\twolineshloka
{शीघ्रमेतान् सुसन्त्रस्तान् बहुधा विप्रधावितान्}
{पर्यवस्थापयाख्याहि विभीषणमुपस्थितम्} %6-50-10

\twolineshloka
{सुग्रीवेणैवमुक्तस्तु जाम्बवानृक्षपार्थिवः}
{वानरान् सान्त्वयामास सन्निवर्त्य प्रधावतः} %6-50-11

\twolineshloka
{ते निवृत्ताः पुनः सर्वे वानरास्त्यक्तसाध्वसाः}
{ऋक्षराजवचः श्रुत्वा तं च दृष्ट्वा विभीषणम्} %6-50-12

\twolineshloka
{विभीषणस्तु रामस्य दृष्ट्वा गात्रं शरैश्चितम्}
{लक्ष्मणस्य तु धर्मात्मा बभूव व्यथितस्तदा} %6-50-13

\twolineshloka
{जलक्लिन्नेन हस्तेन तयोर्नेत्रे विमृज्य च}
{शोकसम्पीडितमना रुरोद विललाप च} %6-50-14

\twolineshloka
{इमौ तौ सत्त्वसम्पन्नौ विक्रान्तौ प्रियसंयुगौ}
{इमामवस्थां गमितौ राक्षसैः कूटयोधिभिः} %6-50-15

\twolineshloka
{भ्रातृपुत्रेण चैतेन दुष्पुत्रेण दुरात्मना}
{राक्षस्या जिह्मया बुद्ध्या वञ्चितावृजुविक्रमौ} %6-50-16

\twolineshloka
{शरैरिमावलं विद्धौ रुधिरेण समुक्षितौ}
{वसुधायामिमौ सुप्तौ दृश्येते शल्यकाविव} %6-50-17

\twolineshloka
{ययोर्वीर्यमुपाश्रित्य प्रतिष्ठा काङ्क्षिता मया}
{ताविमौ देहनाशाय प्रसुप्तौ पुरुषर्षभौ} %6-50-18

\twolineshloka
{जीवन्नद्य विपन्नोऽस्मि नष्टराज्यमनोरथः}
{प्राप्तप्रतिज्ञश्च रिपुः सकामो रावणः कृतः} %6-50-19

\twolineshloka
{एवं विलपमानं तं परिष्वज्य विभीषणम्}
{सुग्रीवः सत्त्वसम्पन्नो हरिराजोऽब्रवीदिदम्} %6-50-20

\twolineshloka
{राज्यं प्राप्स्यसि धर्मज्ञ लङ्कायां नेह संशयः}
{रावणः सह पुत्रेण स्वकामं नेह लप्स्यते} %6-50-21

\twolineshloka
{गरुडाधिष्ठितावेतावुभौ राघवलक्ष्मणौ}
{त्यक्त्वा मोहं वधिष्येते सगणं रावणं रणे} %6-50-22

\twolineshloka
{तमेवं सान्त्वयित्वा तु समाश्वास्य तु राक्षसम्}
{सुषेणं श्वशुरं पार्श्वे सुग्रीवस्तमुवाच ह} %6-50-23

\twolineshloka
{सह शूरैर्हरिगणैर्लब्धसंज्ञावरिन्दमौ}
{गच्छ त्वं भ्रातरौ गृह्य किष्किन्धां रामलक्ष्मणौ} %6-50-24

\twolineshloka
{अहं तु रावणं हत्वा सपुत्रं सहबान्धवम्}
{मैथिलीमानयिष्यामि शक्रो नष्टामिव श्रियम्} %6-50-25

\twolineshloka
{श्रुत्वैतद् वानरेन्द्रस्य सुषेणो वाक्यमब्रवीत्}
{देवासुरं महायुद्धमनुभूतं पुरातनम्} %6-50-26

\twolineshloka
{तदा स्म दानवा देवान् शरसंस्पर्शकोविदान्}
{निजघ्नुः शस्त्रविदुषश्छादयन्तो मुहुर्मुहुः} %6-50-27

\twolineshloka
{तानार्तान् नष्टसंज्ञांश्च गतासूंश्च बृहस्पतिः}
{विद्याभिर्मन्त्रयुक्ताभिरोषधीभिश्चिकित्सति} %6-50-28

\twolineshloka
{तान्यौषधान्यानयितुं क्षीरोदं यान्तु सागरम्}
{जवेन वानराः शीघ्रं सम्पातिपनसादयः} %6-50-29

\twolineshloka
{हरयस्तु विजानन्ति पार्वती ते महौषधी}
{सञ्जीवकरणीं दिव्यां विशल्यां देवनिर्मिताम्} %6-50-30

\twolineshloka
{चन्द्रश्च नाम द्रोणश्च क्षीरोदे सागरोत्तमे}
{अमृतं यत्र मथितं तत्र ते परमौषधी} %6-50-31

\twolineshloka
{तौ तत्र विहितौ देवैः पर्वतौ तौ महोदधौ}
{अयं वायुसुतो राजन् हनूमांस्तत्र गच्छतु} %6-50-32

\twolineshloka
{एतस्मिन्नन्तरे वायुर्मेघाश्चापि सविद्युतः}
{पर्यस्य सागरे तोयं कम्पयन्निव पर्वतान्} %6-50-33

\twolineshloka
{महता पक्षवातेन सर्वद्वीपमहाद्रुमाः}
{निपेतुर्भग्नविटपाः सलिले लवणाम्भसि} %6-50-34

\twolineshloka
{अभवन् पन्नगास्त्रस्ता भोगिनस्तत्रवासिनः}
{शीघ्रं सर्वाणि यादांसि जग्मुश्च लवणार्णवम्} %6-50-35

\twolineshloka
{ततो मुहूर्ताद् गरुडं वैनतेयं महाबलम्}
{वानरा ददृशुः सर्वे ज्वलन्तमिव पावकम्} %6-50-36

\twolineshloka
{तमागतमभिप्रेक्ष्य नागास्ते विप्रदुद्रुवुः}
{यैस्तु तौ पुरुषौ बद्धौ शरभूतैर्महाबलैः} %6-50-37

\twolineshloka
{ततः सुपर्णः काकुत्स्थौ स्पृष्ट्वा प्रत्यभिनन्द्य च}
{विममर्श च पाणिभ्यां मुखे चन्द्रसमप्रभे} %6-50-38

\twolineshloka
{वैनतेयेन संस्पृष्टास्तयोः संरुरुहुर्व्रणाः}
{सुवर्णे च तनू स्निग्धे तयोराशु बभूवतुः} %6-50-39

\twolineshloka
{तेजो वीर्यं बलं चौज उत्साहश्च महागुणाः}
{प्रदर्शनं च बुद्धिश्च स्मृतिश्च द्विगुणा तयोः} %6-50-40

\twolineshloka
{तावुत्थाप्य महातेजा गरुडो वासवोपमौ}
{उभौ च सस्वजे हृष्टो रामश्चैनमुवाच ह} %6-50-41

\twolineshloka
{भवत्प्रसादाद् व्यसनं रावणिप्रभवं महत्}
{उपायेन व्यतिक्रान्तौ शीघ्रं च बलिनौ कृतौ} %6-50-42

\twolineshloka
{यथा तातं दशरथं यथाजं च पितामहम्}
{तथा भवन्तमासाद्य हृदयं मे प्रसीदति} %6-50-43

\twolineshloka
{को भवान् रूपसम्पन्नो दिव्यस्रगनुलेपनः}
{वसानो विरजे वस्त्रे दिव्याभरणभूषितः} %6-50-44

\twolineshloka
{तमुवाच महातेजा वैनतेयो महाबलः}
{पतत्त्रिराजः प्रीतात्मा हर्षपर्याकुलेक्षणम्} %6-50-45

\twolineshloka
{अहं सखा ते काकुत्स्थ प्रियः प्राणो बहिश्चरः}
{गरुत्मानिह सम्प्राप्तो युवयोः साह्यकारणात्} %6-50-46

\twolineshloka
{असुरा वा महावीर्या दानवा वा महाबलाः}
{सुराश्चापि सगन्धर्वाः पुरस्कृत्य शतक्रतुम्} %6-50-47

\twolineshloka
{नेमं मोक्षयितुं शक्ताः शरबन्धं सुदारुणम्}
{मायाबलादिन्द्रजिता निर्मितं क्रूरकर्मणा} %6-50-48

\twolineshloka
{एते नागाः काद्रवेयास्तीक्ष्णदंष्ट्रा विषोल्बणाः}
{रक्षोमायाप्रभावेण शरभूतास्त्वदाश्रयाः} %6-50-49

\twolineshloka
{सभाग्यश्चासि धर्मज्ञ राम सत्यपराक्रम}
{लक्ष्मणेन सह भ्रात्रा समरे रिपुघातिना} %6-50-50

\twolineshloka
{इमं श्रुत्वा तु वृत्तान्तं त्वरमाणोऽहमागतः}
{सहसैवावयोः स्नेहात् सखित्वमनुपालयन्} %6-50-51

\twolineshloka
{मोक्षितौ च महाघोरादस्मात् सायकबन्धनात्}
{अप्रमादश्च कर्तव्यो युवाभ्यां नित्यमेव हि} %6-50-52

\twolineshloka
{प्रकृत्या राक्षसाः सर्वे सङ्ग्रामे कूटयोधिनः}
{शूराणां शुद्धभावानां भवतामार्जवं बलम्} %6-50-53

\twolineshloka
{तन्न विश्वसनीयं वो राक्षसानां रणाजिरे}
{एतेनैवोपमानेन नित्यं जिह्मा हि राक्षसाः} %6-50-54

\twolineshloka
{एवमुक्त्वा तदा रामं सुपर्णः स महाबलः}
{परिष्वज्य च सुस्निग्धमाप्रष्टुमुपचक्रमे} %6-50-55

\twolineshloka
{सखे राघव धर्मज्ञ रिपूणामपि वत्सल}
{अभ्यनुज्ञातुमिच्छामि गमिष्यामि यथासुखम्} %6-50-56

\twolineshloka
{न च कौतूहलं कार्यं सखित्वं प्रति राघव}
{कृतकर्मा रणे वीर सखित्वं प्रतिवेत्स्यसि} %6-50-57

\twolineshloka
{बालवृद्धावशेषां तु लङ्कां कृत्वा शरोर्मिभिः}
{रावणं तु रिपुं हत्वा सीतां त्वमुपलप्स्यसे} %6-50-58

\twolineshloka
{इत्येवमुक्त्वा वचनं सुपर्णः शीघ्रविक्रमः}
{रामं च नीरुजं कृत्वा मध्ये तेषां वनौकसाम्} %6-50-59

\twolineshloka
{प्रदक्षिणं ततः कृत्वा परिष्वज्य च वीर्यवान्}
{जगामाकाशमाविश्य सुपर्णः पवनो यथा} %6-50-60

\twolineshloka
{नीरुजौ राघवौ दृष्ट्वा ततो वानरयूथपाः}
{सिंहनादं तदा नेदुर्लाङ्गूलं दुधुवुश्च ते} %6-50-61

\twolineshloka
{ततो भेरीः समाजघ्नुर्मृदङ्गांश्चाप्यवादयन्}
{दध्मुः शङ्खान् सम्प्रहृष्टाः क्ष्वेलन्त्यपि यथापुरम्} %6-50-62

\twolineshloka
{अपरे स्फोट्य विक्रान्ता वानरा नगयोधिनः}
{द्रुमानुत्पाट्य विविधांस्तस्थुः शतसहस्रशः} %6-50-63

\twolineshloka
{विसृजन्तो महानादांस्त्रासयन्तो निशाचरान्}
{लङ्काद्वाराण्युपाजग्मुर्योद्धुकामाः प्लवङ्गमाः} %6-50-64

\twolineshloka
{तेषां सुभीमस्तुमुलो निनादो बभूव शाखामृगयूथपानाम्}
{क्षये निदाघस्य यथा घनानां नादः सुभीमो नदतां निशीथे} %6-50-65


॥इत्यार्षे श्रीमद्रामायणे वाल्मीकीये आदिकाव्ये युद्धकाण्डे नागपाशविमोक्षणम् नाम पञ्चाशः सर्गः ॥६-५०॥
