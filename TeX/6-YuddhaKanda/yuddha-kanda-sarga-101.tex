\sect{एकाधिकशततमः सर्गः — लक्ष्मणशक्तिक्षेपः}

\twolineshloka
{तस्मिन् प्रतिहतेऽस्त्रे तु रावणो राक्षसाधिपः}
{क्रोधं च द्विगुणं चक्रे क्रोधाच्चास्त्रमनन्तरम्} %6-101-1

\twolineshloka
{मयेन विहितं रौद्रमन्यदस्त्रं महाद्युतिः}
{उत्स्रष्टुं रावणो भीमं राघवाय प्रचक्रमे} %6-101-2

\twolineshloka
{ततः शूलानि निश्चेरुर्गदाश्च मुसलानि च}
{कार्मुकाद् दीप्यमानानि वज्रसाराणि सर्वशः} %6-101-3

\twolineshloka
{मुद्गराः कूटपाशाश्च दीप्ताश्चाशनयस्तथा}
{निष्पेतुर्विविधास्तीक्ष्णा वाता इव युगक्षये} %6-101-4

\twolineshloka
{तदस्त्रं राघवः श्रीमानुत्तमास्त्रविदां वरः}
{जघान परमास्त्रेण गान्धर्वेण महाद्युतिः} %6-101-5

\twolineshloka
{तस्मिन् प्रतिहतेऽस्त्रे तु राघवेण महात्मना}
{रावणः क्रोधताम्राक्षः सौरमस्त्रमुदीरयत्} %6-101-6

\twolineshloka
{ततश्चक्राणि निष्पेतुर्भास्वराणि महान्ति च}
{कार्मुकाद् भीमवेगस्य दशग्रीवस्य धीमतः} %6-101-7

\twolineshloka
{तैरासीद् गगनं दीप्तं सम्पतद्भिः समन्ततः}
{पतद्भिश्च दिशो दीप्ताश्चन्द्रसूर्यग्रहैरिव} %6-101-8

\twolineshloka
{तानि चिच्छेद बाणौघैश्चक्राणि तु स राघवः}
{आयुधानि च चित्राणि रावणस्य चमूमुखे} %6-101-9

\twolineshloka
{तदस्त्रं तु हतं दृष्ट्वा रावणो राक्षसाधिपः}
{विव्याध दशभिर्बाणै रामं सर्वेषु मर्मसु} %6-101-10

\twolineshloka
{स विद्धो दशभिर्बाणैर्महाकार्मुकनिःसृतैः}
{रावणेन महातेजा न प्राकम्पत राघवः} %6-101-11

\twolineshloka
{ततो विव्याध गात्रेषु सर्वेषु समितिंजयः}
{राघवस्तु सुसंक्रुद्धो रावणं बहुभिः शरैः} %6-101-12

\twolineshloka
{एतस्मिन्नन्तरे क्रुद्धो राघवस्यानुजो बली}
{लक्ष्मणः सायकान् सप्त जग्राह परवीरहा} %6-101-13

\twolineshloka
{तैः सायकैर्महावेगै रावणस्य महाद्युतिः}
{ध्वजं मनुष्यशीर्षं तु तस्य चिच्छेद नैकधा} %6-101-14

\twolineshloka
{सारथेश्चापि बाणेन शिरो ज्वलितकुण्डलम्}
{जहार लक्ष्मणः श्रीमान् नैर्ऋतस्य महाबलः} %6-101-15

\twolineshloka
{तस्य बाणैश्च चिच्छेद धनुर्गजकरोपमम्}
{लक्ष्मणो राक्षसेन्द्रस्य पञ्चभिर्निशितैस्तदा} %6-101-16

\twolineshloka
{नीलमेघनिभांश्चास्य सदश्वान् पर्वतोपमान्}
{जघानाप्लुत्य गदया रावणस्य विभीषणः} %6-101-17

\twolineshloka
{हताश्वात् तु तदा वेगादवप्लुत्य महारथात्}
{कोपमाहारयत् तीव्रं भ्रातरं प्रति रावणः} %6-101-18

\twolineshloka
{ततः शक्तिं महाशक्तिः प्रदीप्तामशनीमिव}
{विभीषणाय चिक्षेप राक्षसेन्द्रः प्रतापवान्} %6-101-19

\twolineshloka
{अप्राप्तामेव तां बाणैस्त्रिभिश्चिच्छेद लक्ष्मणः}
{अथोदतिष्ठत् संनादो वानराणां महारणे} %6-101-20

\twolineshloka
{सम्पपात त्रिधा छिन्ना शक्तिः काञ्चनमालिनी}
{सविस्फुलिङ्गा ज्वलिता महोल्केव दिवश्च्युता} %6-101-21

\twolineshloka
{ततः सम्भाविततरां कालेनापि दुरासदाम्}
{जग्राह विपुलां शक्तिं दीप्यमानां स्वतेजसा} %6-101-22

\twolineshloka
{सा वेगिता बलवता रावणेन दुरात्मना}
{जज्वाल सुमहातेजा दीप्ताशनिसमप्रभा} %6-101-23

\twolineshloka
{एतस्मिन्नन्तरे वीरो लक्ष्मणस्तं विभीषणम्}
{प्राणसंशयमापन्नं तूर्णमभ्यवपद्यत} %6-101-24

\twolineshloka
{तं विमोक्षयितुं वीरश्चापमायम्य लक्ष्मणः}
{रावणं शक्तिहस्तं वै शरवर्षैरवाकिरत्} %6-101-25

\twolineshloka
{कीर्यमाणः शरौघेण विसृष्टेन महात्मना}
{न प्रहर्तुं मनश्चक्रे विमुखीकृतविक्रमः} %6-101-26

\twolineshloka
{मोक्षितं भ्रातरं दृष्ट्वा लक्ष्मणेन स रावणः}
{लक्ष्मणाभिमुखस्तिष्ठन्निदं वचनमब्रवीत्} %6-101-27

\twolineshloka
{मोक्षितस्ते बलश्लाघिन् यस्मादेवं विभीषणः}
{विमुच्य राक्षसं शक्तिस्त्वयीयं विनिपात्यते} %6-101-28

\twolineshloka
{एषा ते हृदयं भित्त्वा शक्तिर्लोहितलक्षणा}
{मद्बाहुपरिघोत्सृष्टा प्राणानादाय यास्यति} %6-101-29

\twolineshloka
{इत्येवमुक्त्वा तां शक्तिमष्टघण्टां महास्वनाम्}
{मयेन मायाविहिताममोघां शत्रुघातिनीम्} %6-101-30

\twolineshloka
{लक्ष्मणाय समुद्दिश्य ज्वलन्तीमिव तेजसा}
{रावणः परमक्रुद्धश्चिक्षेप च ननाद च} %6-101-31

\twolineshloka
{सा क्षिप्ता भीमवेगेन वज्राशनिसमस्वना}
{शक्तिरभ्यपतद् वेगाल्लक्ष्मणं रणमूर्धनि} %6-101-32

\twolineshloka
{तामनुव्याहरच्छक्तिमापतन्तीं स राघवः}
{स्वस्त्यस्तु लक्ष्मणायेति मोघा भव हतोद्यमा} %6-101-33

\twolineshloka
{रावणेन रणे शक्तिः क्रुद्धेनाशीविषोपमा}
{मुक्ताऽऽशूरस्य भीतस्य लक्ष्मणस्य ममज्ज सा} %6-101-34

\twolineshloka
{न्यपतत् सा महावेगा लक्ष्मणस्य महोरसि}
{जिह्वेवोरगराजस्य दीप्यमाना महाद्युतिः} %6-101-35

\twolineshloka
{ततो रावणवेगेन सुदूरमवगाढया}
{शक्त्या विभिन्नहृदयः पपात भुवि लक्ष्मणः} %6-101-36

\twolineshloka
{तदवस्थं समीपस्थो लक्ष्मणं प्रेक्ष्य राघवः}
{भ्रातृस्नेहान्महातेजा विषण्णहृदयोऽभवत्} %6-101-37

\twolineshloka
{स मुहूर्तमिव ध्यात्वा बाष्पपर्याकुलेक्षणः}
{बभूव संरब्धतरो युगान्त इव पावकः} %6-101-38

\threelineshloka
{न विषादस्य कालोऽयमिति संचिन्त्य राघवः}
{चक्रे सुतुमलं युद्धं रावणस्य वधे धृतः}
{सर्वयत्नेन महता लक्ष्मणं परिवीक्ष्य च} %6-101-39

\twolineshloka
{स ददर्श ततो रामः शक्त्या भिन्नं महाहवे}
{लक्ष्मणं रुधिरादिग्धं सपन्नगमिवाचलम्} %6-101-40

\twolineshloka
{तामपि प्रहितां शक्तिं रावणेन बलीयसा}
{यत्नतस्ते हरिश्रेष्ठा न शेकुरवमर्दितुम्} %6-101-41

\twolineshloka
{अर्दिताश्चैव बाणौघैस्ते प्रवेकेण रक्षसाम्}
{सौमित्रेः सा विनिर्भिद्य प्रविष्टा धरणीतलम्} %6-101-42

\twolineshloka
{तां कराभ्यां परामृश्य रामः शक्तिं भयावहाम्}
{बभञ्ज समरे क्रुद्धो बलवान् विचकर्ष च} %6-101-43

\twolineshloka
{तस्य निष्कर्षतः शक्तिं रावणेन बलीयसा}
{शराः सर्वेषु गात्रेषु पातिता मर्मभेदिनः} %6-101-44

\twolineshloka
{अचिन्तयित्वा तान् बाणान् समाश्लिष्य च लक्ष्मणम्}
{अब्रवीच्च हनूमन्तं सुग्रीवं च महाकपिम्} %6-101-45

\twolineshloka
{लक्ष्मणं परिवार्यैवं तिष्ठध्वं वानरोत्तमाः}
{पराक्रमस्य कालोऽयं सम्प्राप्तो मे चिरेप्सितः} %6-101-46

\twolineshloka
{पापात्मायं दशग्रीवो वध्यतां पापनिश्चयः}
{कांक्षितं चातकस्येव घर्मान्ते मेघदर्शनम्} %6-101-47

\twolineshloka
{अस्मिन् मुहूर्ते नचिरात् सत्यं प्रतिशृणोमि वः}
{अरावणमरामं वा जगद् द्रक्ष्यथ वानराः} %6-101-48

\twolineshloka
{राज्यनाशं वने वासं दण्डके परिधावनम्}
{वैदेह्याश्च परामर्शो रक्षोभिश्च समागमम्} %6-101-49

\twolineshloka
{प्राप्तं दुःखं महाघोरं क्लेशश्च निरयोपमः}
{अद्य सर्वमहं त्यक्ष्ये निहत्वा रावणं रणे} %6-101-50

\threelineshloka
{यदर्थं वानरं सैन्यं समानीतमिदं मया}
{सुग्रीवश्च कृतो राज्ये निहत्वा वालिनं रणे}
{यदर्थं सागरः क्रान्तः सेतुर्बद्धश्च सागरे} %6-101-51

\twolineshloka
{सोऽयमद्य रणे पापश्चक्षुर्विषयमागतः}
{चक्षुर्विषयमागत्य नायं जीवितुमर्हति} %6-101-52

\twolineshloka
{दृष्टिं दृष्टिविषस्येव सर्पस्य मम रावणः}
{यथा वा वैनतेयस्य दृष्टिं प्राप्तो भुजंगमः} %6-101-53

\twolineshloka
{सुखं पश्यत दुर्धर्षा युद्धं वानरपुङ्गवाः}
{आसीनाः पर्वताग्रेषु ममेदं रावणस्य च} %6-101-54

\twolineshloka
{अद्य पश्यन्तु रामस्य रामत्वं मम संयुगे}
{त्रयो लोकाः सगन्धर्वाः सदेवाः सर्षिचारणाः} %6-101-55

\threelineshloka
{अद्य कर्म करिष्यामि यल्लोकाः सचराचराः}
{सदेवाः कथयिष्यन्ति यावद् भूमिर्धरिष्यति}
{समागम्य सदा लोके यथा युद्धं प्रवर्तितम्} %6-101-56

\twolineshloka
{एवमुक्त्वा शितैर्बाणैस्तप्तकाञ्चनभूषणैः}
{आजघान रणे रामो दशग्रीवं समाहितः} %6-101-57

\twolineshloka
{तथा प्रदीप्तैर्नाराचैर्मुसलैश्चापि रावणः}
{अभ्यवर्षत् तदा रामं धाराभिरिव तोयदः} %6-101-58

\twolineshloka
{रामरावणमुक्तानामन्योन्यमभिनिघ्नताम्}
{वराणां च शराणां च बभूव तुमुलः स्वनः} %6-101-59

\twolineshloka
{विच्छिन्नाश्च विकीर्णाश्च रामरावणयोः शराः}
{अन्तरिक्षात् प्रदीप्ताग्रा निपेतुर्धरणीतले} %6-101-60

\twolineshloka
{तयोर्ज्यातलनिर्घोषो रामरावणयोर्महान्}
{त्रासनः सर्वभूतानां सम्बभूवाद्भुतोपमः} %6-101-61

\twolineshloka
{स कीर्यमाणः शरजालवृष्टिभिर्महात्मना दीप्तधनुष्मतार्दितः}
{भयात् प्रदुद्राव समेत्य रावणो यथानिलेनाभिहतो बलाहकः} %6-101-62


॥इत्यार्षे श्रीमद्रामायणे वाल्मीकीये आदिकाव्ये युद्धकाण्डे लक्ष्मणशक्तिक्षेपः नाम एकाधिकशततमः सर्गः ॥६-१०१॥
