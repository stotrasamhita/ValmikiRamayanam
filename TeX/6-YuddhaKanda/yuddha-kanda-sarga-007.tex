\sect{सप्तमः सर्गः — सचिवोक्तिः}

\twolineshloka
{इत्युक्ता राक्षसेन्द्रेण राक्षसास्ते महाबलाः}
{ऊचुः प्राञ्जलयः सर्वे रावणं राक्षसेश्वरम्} %6-7-1

\twolineshloka
{द्विषत्पक्षमविज्ञाय नीतिबाह्यास्त्वबुद्धयः}
{राजन् परिघशक्त्यृष्टिशूलपट्टिशकुन्तलम्} %6-7-2

\twolineshloka
{सुमहन्नो बलं कस्माद् विषादं भजते भवान्}
{त्वया भोगवतीं गत्वा निर्जिताः पन्नगा युधि} %6-7-3

\twolineshloka
{कैलासशिखरावासी यक्षैर्बहुभिरावृतः}
{सुमहत्कदनं कृत्वा वश्यस्ते धनदः कृतः} %6-7-4

\twolineshloka
{स महेश्वरसख्येन श्लाघमानस्त्वया विभो}
{निर्जितः समरे रोषाल्लोकपालो महाबलः} %6-7-5

\twolineshloka
{विनिपात्य च यक्षौघान् विक्षोभ्य विनिगृह्य च}
{त्वया कैलासशिखराद् विमानमिदमाहृतम्} %6-7-6

\twolineshloka
{मयेन दानवेन्द्रेण त्वद्भयात् सख्यमिच्छता}
{दुहिता तव भार्यार्थे दत्ता राक्षसपुङ्गव} %6-7-7

\twolineshloka
{दानवेन्द्रो महाबाहो वीर्योत्सिक्तो दुरासदः}
{विगृह्य वशमानीतः कुम्भीनस्याः सुखावहः} %6-7-8

\twolineshloka
{निर्जितास्ते महाबाहो नागा गत्वा रसातलम्}
{वासुकिस्तक्षकः शङ्खो जटी च वशमाहृताः} %6-7-9

\twolineshloka
{अक्षया बलवन्तश्च शूरा लब्धवराः पुनः}
{त्वया संवत्सरं युद्ध्वा समरे दानवा विभो} %6-7-10

\twolineshloka
{स्वबलं समुपाश्रित्य नीता वशमरिन्दम}
{मायाश्चाधिगतास्तत्र बह्व्यो वै राक्षसाधिप} %6-7-11

\twolineshloka
{शूराश्च बलवन्तश्च वरुणस्य सुता रणे}
{निर्जितास्ते महाभाग चतुर्विधबलानुगाः} %6-7-12

\twolineshloka
{मृत्युदण्डमहाग्राहं शाल्मलीद्रुममण्डितम्}
{कालपाशमहावीचिं यमकिङ्करपन्नगम्} %6-7-13

\twolineshloka
{महाज्वरेण दुर्धर्षं यमलोकमहार्णवम्}
{अवगाह्य त्वया राजन् यमस्य बलसागरम्} %6-7-14

\twolineshloka
{जयश्च विपुलः प्राप्तो मृत्युश्च प्रतिषेधितः}
{सुयुद्धेन च ते सर्वे लोकस्तत्र सुतोषिताः} %6-7-15

\twolineshloka
{क्षत्रियैर्बहुभिर्वीरैः शक्रतुल्यपराक्रमैः}
{आसीद् वसुमती पूर्णा महद्भिरिव पादपैः} %6-7-16

\twolineshloka
{तेषां वीर्यगुणोत्साहैर्न समो राघवो रणे}
{प्रसह्य ते त्वया राजन् हताः समरदुर्जयाः} %6-7-17

\twolineshloka
{तिष्ठ वा किं महाराज श्रमेण तव वानरान्}
{अयमेको महाबाहुरिन्द्रजित् क्षपयिष्यति} %6-7-18

\twolineshloka
{अनेन च महाराज माहेश्वरमनुत्तमम्}
{इष्ट्वा यज्ञं वरो लब्धो लोके परमदुर्लभः} %6-7-19

\twolineshloka
{शक्तितोमरमीनं च विनिकीर्णान्त्रशैवलम्}
{गजकच्छपसम्बाधमश्वमण्डूकसङ्कुलम्} %6-7-20

\twolineshloka
{रुद्रादित्यमहाग्राहं मरुद्वसुमहोरगम्}
{रथाश्वगजतोयौघं पदातिपुलिनं महत्} %6-7-21

\twolineshloka
{अनेन हि समासाद्य देवानां बलसागरम्}
{गृहीतो दैवतपतिर्लङ्कां चापि प्रवेशितः} %6-7-22

\twolineshloka
{पितामहनियोगाच्च मुक्तः शम्बरवृत्रहा}
{गतस्त्रिविष्टपं राजन् सर्वदेवनमस्कृतः} %6-7-23

\twolineshloka
{तमेव त्वं महाराज विसृजेन्द्रजितं सुतम्}
{यावद् वानरसेनां तां सरामां नयति क्षयम्} %6-7-24

\twolineshloka
{राजन्नापदयुक्तेयमागता प्राकृताज्जनात्}
{हृदि नैव त्वया कार्या त्वं वधिष्यसि राघवम्} %6-7-25


॥इत्यार्षे श्रीमद्रामायणे वाल्मीकीये आदिकाव्ये युद्धकाण्डे सचिवोक्तिः नाम सप्तमः सर्गः ॥६-७॥
