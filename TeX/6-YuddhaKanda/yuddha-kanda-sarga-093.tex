\sect{त्रिनवतितमः सर्गः — सीताहननोद्यमनिवृत्तिः}

\twolineshloka
{ततः पौलस्त्यसचिवाः श्रुत्वा चेन्द्रजितो वधम्}
{आचचक्षुरभिज्ञाय दशग्रीवाय सत्वराः} %6-93-1

\twolineshloka
{युद्धे हतो महाराज लक्ष्मणेन तवात्मजः}
{विभीषणसहायेन मिषतां नो महाद्युतिः} %6-93-2

\twolineshloka
{शूरः शूरेण संगम्य संयुगेष्वपराजितः}
{लक्ष्मणेन हतः शूरः पुत्रस्ते विबुधेन्द्रजित्} %6-93-3

\twolineshloka
{गतः स परमाँल्लोकान् शरैः संतर्प्य लक्ष्मणम्}
{स तं प्रतिभयं श्रुत्वा वधं पुत्रस्य दारुणम्} %6-93-4

\twolineshloka
{घोरमिन्द्रजितः संख्ये कश्मलं प्राविशन्महत्}
{उपलभ्य चिरात् संज्ञां राजा राक्षसपुंगवः} %6-93-5

\twolineshloka
{पुत्रशोकाकुलो दीनो विललापाकुलेन्द्रियः}
{हा राक्षसचमूमुख्य मम वत्स महाबल} %6-93-6

\twolineshloka
{जित्वेन्द्रं कथमद्य त्वं लक्ष्मणस्य वशं गतः}
{ननु त्वमिषुभिः क्रुद्धो भिन्द्याः कालान्तकावपि} %6-93-7

\twolineshloka
{मन्दरस्यापि शृङ्गाणि किं पुनर्लक्ष्मणं युधि}
{अद्य वैवस्वतो राजा भूयो बहुमतो मम} %6-93-8

\threelineshloka
{येनाद्य त्वं महाबाहो संयुक्तः कालधर्मणा}
{एष पन्थाः सुयोधानां सर्वामरगणेष्वपि}
{यः कृते हन्यते भर्तुः स पुमान् स्वर्गमृच्छति} %6-93-9

\twolineshloka
{अद्य देवगणाः सर्वे लोकपाला महर्षयः}
{हतमिन्द्रजितं श्रुत्वा सुखं स्वप्स्यन्ति निर्भयाः} %6-93-10

\twolineshloka
{अद्य लोकास्त्रयः कृत्स्ना पृथिवी च सकानना}
{एकेनेन्द्रजिता हीना शून्येव प्रतिभाति मे} %6-93-11

\twolineshloka
{अद्य नैर्ऋतकन्यानां श्रोष्याम्यन्तःपुरे रवम्}
{करेणुसङ्घस्य यथा निनादं गिरिगह्वरे} %6-93-12

\twolineshloka
{यौवराज्यं च लङ्कां च रक्षांसि च परंतप}
{मातरं मां च भार्याश्च क्व गतोऽसि विहाय नः} %6-93-13

\twolineshloka
{मम नाम त्वया वीर गतस्य यमसादनम्}
{प्रेतकार्याणि कार्याणि विपरीते हि वर्तसे} %6-93-14

\twolineshloka
{स त्वं जीवति सुग्रीवे लक्ष्मणे च सराघवे}
{मम शल्यमनुद्धृत्य क्व गतोऽसि विहाय नः} %6-93-15

\twolineshloka
{एवमादिविलापार्तं रावणं राक्षसाधिपम्}
{आविवेश महान् कोपः पुत्रव्यसनसम्भवः} %6-93-16

\twolineshloka
{प्रकृत्या कोपनं ह्येनं पुत्रस्य पुनराधयः}
{दीप्तं संदीपयामासुर्घर्मेऽर्कमिव रश्मयः} %6-93-17

\twolineshloka
{ललाटे भ्रुकुटीभिश्च संगताभिर्व्यरोचत}
{युगान्ते सह नक्रैस्तु महोर्मिभिरिवोदधिः} %6-93-18

\twolineshloka
{कोपाद् विजृम्भमाणस्य वक्त्राद् व्यक्तमिव ज्वलन्}
{उत्पपात सधूमाग्निर्वृत्रस्य वदनादिव} %6-93-19

\twolineshloka
{स पुत्रवधसंतप्तः शूरः क्रोधवशं गतः}
{समीक्ष्य रावणो बुद्ध्या वैदेह्या रोचयद् वधम्} %6-93-20

\twolineshloka
{तस्य प्रकृत्या रक्ते च रक्ते क्रोधाग्निनापि च}
{रावणस्य महाघोरे दीप्ते नेत्रे बभूवतुः} %6-93-21

\twolineshloka
{घोरं प्रकृत्या रूपं तत् तस्य क्रोधाग्निमूर्च्छितम्}
{बभूव रूपं क्रुद्धस्य रुद्रस्येव दुरासदम्} %6-93-22

\twolineshloka
{तस्य क्रुद्धस्य नेत्राभ्यां प्रापतन्नश्रुबिन्दवः}
{दीपाभ्यामिव दीप्ताभ्यां सार्चिषः स्नेहबिन्दवः} %6-93-23

\twolineshloka
{दन्तान् विदशतस्तस्य श्रूयते दशनस्वनः}
{यन्त्रस्याकृष्यमाणस्य मथ्नतो दानवैरिव} %6-93-24

\twolineshloka
{कालाग्निरिव संक्रुद्धो यां यां दिशमवैक्षत}
{तस्यां तस्यां भयत्रस्ता राक्षसाः संविलिल्यिरे} %6-93-25

\twolineshloka
{तमन्तकमिव क्रुद्धं चराचरचिखादिषुम्}
{वीक्षमाणं दिशः सर्वा राक्षसा नोपचक्रमुः} %6-93-26

\twolineshloka
{ततः परमसंक्रुद्धो रावणो राक्षसाधिपः}
{अब्रवीद् रक्षसां मध्ये संस्तम्भयिषुराहवे} %6-93-27

\twolineshloka
{मया वर्षसहस्राणि चरित्वा परमं तपः}
{तेषु तेष्ववकाशेषु स्वयंभूः परितोषितः} %6-93-28

\twolineshloka
{तस्यैव तपसो व्युष्ट्या प्रसादाच्च स्वयंभुवः}
{नासुरेभ्यो न देवेभ्यो भयं मम कदाचन} %6-93-29

\twolineshloka
{कवचं ब्रह्मदत्तं मे यदादित्यसमप्रभम्}
{देवासुरविमर्देषु न च्छिन्नं वज्रमुष्टिभिः} %6-93-30

\twolineshloka
{तेन मामद्य संयुक्तं रथस्थमिह संयुगे}
{प्रतीयात् कोऽद्य मामाजौ साक्षादपि पुरंदरः} %6-93-31

\twolineshloka
{यत् तदाभिप्रसन्नेन सशरं कार्मुकं महत्}
{देवासुरविमर्देषु मम दत्तं स्वयंभुवा} %6-93-32

\twolineshloka
{अद्य तूर्यशतैर्भीमं धनुरुत्थाप्यतां मम}
{रामलक्ष्मणयोरेव वधाय परमाहवे} %6-93-33

\twolineshloka
{स पुत्रवधसंतप्तः क्रूरः क्रोधवशं गतः}
{समीक्ष्य रावणो बुद्ध्या सीतां हन्तुं व्यवस्यत} %6-93-34

\twolineshloka
{प्रत्यवेक्ष्य तु ताम्राक्षः सुघोरो घोरदर्शनः}
{दीनो दीनस्वरान् सर्वांस्तानुवाच निशाचरान्} %6-93-35

\twolineshloka
{मायया मम वत्सेन वञ्चनार्थं वनौकसाम्}
{किंचिदेव हतं तत्र सीतेयमिति दर्शितम्} %6-93-36

\twolineshloka
{तदिदं तथ्यमेवाहं करिष्ये प्रियमात्मनः}
{वैदेहीं नाशयिष्यामि क्षत्रबन्धुमनुव्रताम्} %6-93-37

\twolineshloka
{इत्येवमुक्त्वा सचिवान् खड्गमाशु परामृशत्}
{उद्धृत्य गुणसम्पन्नं विमलाम्बरवर्चसम्} %6-93-38

\twolineshloka
{निष्पपात स वेगेन सभार्यः सचिवैर्वृतः}
{रावणः पुत्रशोकेन भृशमाकुलचेतनः} %6-93-39

\twolineshloka
{संक्रुद्धः खड्गमादाय सहसा यत्र मैथिली}
{व्रजन्तं राक्षसं प्रेक्ष्य सिंहनादं विचुक्रुशुः} %6-93-40

\twolineshloka
{ऊचुश्चान्योन्यमालिङ्ग्य संक्रुद्धं प्रेक्ष्य राक्षसम्}
{अद्यैनं तावुभौ दृष्ट्वा भ्रातरौ प्रव्यथिष्यतः} %6-93-41

\twolineshloka
{लोकपाला हि चत्वारः क्रुद्धेनानेन निर्जिताः}
{बहवः शत्रवश्चान्ये संयुगेष्वभिपातिताः} %6-93-42

\twolineshloka
{त्रिषु लोकेषु रत्नानि भुङ्क्ते आहृत्य रावणः}
{विक्रमे च बले चैव नास्त्यस्य सदृशो भुवि} %6-93-43

\twolineshloka
{तेषां संजल्पमानानामशोकवनिकां गताम्}
{अभिदुद्राव वैदेहीं रावणः क्रोधमूर्च्छितः} %6-93-44

\twolineshloka
{वार्यमाणः सुसंक्रुद्धः सुहृद्भिर्हितबुद्धिभिः}
{अभ्यधावत संक्रुद्धः खे ग्रहो रोहिणीमिव} %6-93-45

\twolineshloka
{मैथिली रक्ष्यमाणा तु राक्षसीभिरनिन्दिता}
{ददर्श राक्षसं क्रुद्धं निस्त्रिंशवरधारिणम्} %6-93-46

\twolineshloka
{तं निशम्य सनिस्त्रिंशं व्यथिता जनकात्मजा}
{निवार्यमाणं बहुशः सुहृद्भिरनिवर्तिनम्} %6-93-47

\twolineshloka
{सीता दुःखसमाविष्टा विलपन्तीदमब्रवीत्}
{यथायं मामभिक्रुद्धः समभिद्रवति स्वयम्} %6-93-48

\twolineshloka
{वधिष्यति सनाथां मामनाथामिव दुर्मतिः}
{बहुशश्चोदयामास भर्तारं मामनुव्रताम्} %6-93-49

\twolineshloka
{भार्या मम भवस्वेति प्रत्याख्यातो ध्रुवं मया}
{सोऽयं मामनुपस्थाने व्यक्तं नैराश्यमागतः} %6-93-50

\twolineshloka
{क्रोधमोहसमाविष्टो व्यक्तं मां हन्तुमुद्यतः}
{अथवा तौ नरव्याघ्रौ भ्रातरौ रामलक्ष्मणौ} %6-93-51

\twolineshloka
{मन्निमित्तमनार्येण समरेऽद्य निपातितौ}
{भैरवो हि महान् नादो राक्षसानां श्रुतो मया} %6-93-52

\twolineshloka
{बहूनामिह हृष्टानां तथा विक्रोशतां प्रियम्}
{अहो धिङ्मन्निमित्तोऽयं विनाशो राजपुत्रयोः} %6-93-53

\twolineshloka
{अथवा पुत्रशोकेन अहत्वा रामलक्ष्मणौ}
{विधमिष्यति मां रौद्रो राक्षसः पापनिश्चयः} %6-93-54

\twolineshloka
{हनूमतस्तु तद् वाक्यं न कृतं क्षुद्रया मया}
{यद्यहं तस्य पृष्ठेन तदायासमनिर्जिता} %6-93-55

\twolineshloka
{नाद्यैवमनुशोचेयं भर्तुरङ्कगता सती}
{मन्ये तु हृदयं तस्याः कौसल्यायाः फलिष्यति} %6-93-56

\twolineshloka
{एकपुत्रा यदा पुत्रं विनष्टं श्रोष्यते युधि}
{सा हि जन्म च बाल्यं च यौवनं च महात्मनः} %6-93-57

\twolineshloka
{धर्मकार्याणि रूपं च रुदती संस्मरिष्यति}
{निराशा निहते पुत्रे दत्त्वा श्राद्धमचेतना} %6-93-58

\twolineshloka
{अग्निमावेक्ष्यते नूनमपो वापि प्रवेक्ष्यति}
{धिगस्तु कुब्जामसतीं मन्थरां पापनिश्चयाम्} %6-93-59

\twolineshloka
{यन्निमित्तमिमं शोकं कौसल्या प्रतिपत्स्यते}
{इत्येवं मैथिलीं दृष्ट्वा विलपन्तीं तपस्विनीम्} %6-93-60

\twolineshloka
{रोहिणीमिव चन्द्रेण बिना ग्रहवशं गताम्}
{एतस्मिन्नन्तरे तस्य अमात्यः शीलवान् शुचिः} %6-93-61

\twolineshloka
{सुपार्श्वो नाम मेधावी रावणं रक्षसां वरम्}
{निवार्यमाणः सचिवैरिदं वचनमब्रवीत्} %6-93-62

\twolineshloka
{कथं नाम दशग्रीव साक्षाद्वैश्रवणानुज}
{हन्तुमिच्छसि वैदेहीं क्रोधाद् धर्ममपास्य च} %6-93-63

\twolineshloka
{वेदविद्याव्रतस्नातः स्वकर्मनिरतस्तथा}
{स्त्रियः कस्माद् वधं वीर मन्यसे राक्षसेश्वर} %6-93-64

\twolineshloka
{मैथिलीं रूपसम्पन्नां प्रत्यवेक्षस्व पार्थिव}
{तस्मिन्नेव सहास्माभिराहवे क्रोधमुत्सृज} %6-93-65

\twolineshloka
{अभ्युत्थानं त्वमद्यैव कृष्णपक्षचतुर्दशी}
{कृत्वा निर्याह्यमावास्यां विजयाय बलैर्वृतः} %6-93-66

\twolineshloka
{शूरो धीमान् रथी खड्गी रथप्रवरमास्थितः}
{हत्वा दाशरथिं रामं भवान् प्राप्स्यति मैथिलीम्} %6-93-67

\twolineshloka
{स तद् दुरात्मा सुहृदा निवेदितं वचः सुधर्म्यं प्रतिगृह्य रावणः}
{गृहं जगामाथ ततश्च वीर्यवान् पुनः सभां च प्रययौ सुहृद्वृतः} %6-93-68


॥इत्यार्षे श्रीमद्रामायणे वाल्मीकीये आदिकाव्ये युद्धकाण्डे सीताहननोद्यमनिवृत्तिः नाम त्रिनवतितमः सर्गः ॥६-९३॥
