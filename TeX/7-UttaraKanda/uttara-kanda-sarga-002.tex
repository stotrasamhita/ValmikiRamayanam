\sect{द्वितीयः सर्गः — पौलस्योत्पत्तिः}

\twolineshloka
{तस्य तद्वचनं श्रुत्वा राघवस्य महात्मनः}
{कुम्भयोनिर्मिहातेजा राममेतदुवाच ह} %7-2-1

\twolineshloka
{शृणु राम कथावृत्तं तस्य तेजोबलं महत्}
{जघान शत्रून्येनासौ न च वध्यः स शत्रुभिः} %7-2-2

\twolineshloka
{तावत्ते रावणस्येदं कुलं जन्म च राघव}
{वरप्रदानं च तथा तस्मै दत्तं ब्रवीमि ते} %7-2-3

\twolineshloka
{पुरा कृतयुगे राम प्रजापतिसुतः प्रभुः}
{पुलस्त्यो नाम ब्रह्मर्षिः साक्षादिव पितामहः} %7-2-4

\twolineshloka
{नानुकीर्त्या गुणास्तस्य धर्मतः शीलतस्तथा}
{प्रजापतेः पुत्र इति वक्तुं शक्यं हि नामतः} %7-2-5

\twolineshloka
{प्रजापतिसुतत्वेन देवानां वल्लभो हि सः}
{हृष्टः सर्वस्य लोकस्य गुणैः शुभ्रैर्महामतिः} %7-2-6

\twolineshloka
{स तु धर्मप्रसङ्गेन मेरोः पार्श्वे महागिरेः}
{तृणविन्द्वाश्रमं गत्वा न्यवसन्मुनिपुङ्गवः} %7-2-7

\twolineshloka
{तपस्तेपे स धर्मात्मा स्वाध्यायनियतेन्द्रियः}
{गत्वाश्रमपदं तस्य विघ्नं कुर्वन्ति कन्यकाः} %7-2-8

\twolineshloka
{देवपन्नगकन्याश्च राजर्षितनयाश्च याः}
{क्रीडन्त्योऽप्सरसश्चैव तं देशमुपपेदिरे} %7-2-9

\twolineshloka
{सर्वर्तुषृपभोग्यत्वाद्रम्यत्वात्काननस्य च}
{नित्यशस्तास्तु तं देशं गत्वा क्रीडन्ति कन्यकाः} %7-2-10

\threelineshloka
{देशस्य रमणीयत्वात्पुलस्त्यो यत्र स द्विजः}
{गायन्त्यो वादयन्त्यश्च लासयन्त्यस्तथैव च}
{मुनेस्तपस्विनस्तस्य विघ्नं चक्रुरनिन्दिताः} %7-2-11

\twolineshloka
{अथ क्रुद्धो महातेजा व्याजहार महामुनिः}
{या मे दर्शनमागच्छेत्सा गर्भं धारयिष्यति} %7-2-12

\twolineshloka
{तास्तु सर्वाः प्रतिश्रुत्य तस्य वाक्यं महात्मनः}
{ब्रह्मशापभयाद्भीतास्तं देशं नोपचक्रमुः} %7-2-13

\onelineshloka
{तृणबिन्दोस्तु राजर्षेस्तनया न शृणोति तत्} %7-2-14

\twolineshloka
{गत्वाश्रमपदं तत्र विचचार सुनिर्भया}
{न सापश्यत्स्थिता तत्र काञ्चिदभ्यागतां सखीम्} %7-2-15

\twolineshloka
{तस्मिन्काले महातेजाः प्राजापत्यो महानृषिः}
{स्वाध्यायमकरोत्तत्र तपसा भावितः स्वयम्} %7-2-16

\twolineshloka
{सा तु वेदश्रुतिं श्रुत्वा दृष्ट्वा वै तपसो निधिम्}
{अभवत्पाण्डुदेहा सा सुव्यञ्जितशरीरजा} %7-2-17

\twolineshloka
{वभूव च समुद्विग्ना दृष्ट्वा तद्दोषमात्मनः}
{इदं मे किन्त्विति ज्ञात्वा पितुर्गत्वाऽऽश्रमे स्थिता} %7-2-18

\twolineshloka
{तां तु दृष्ट्वा तथाभूतां तृणविन्दुरथाब्रवीत्}
{किं त्वमेतत्त्वसदृशं धारयस्यात्मनो वपुः} %7-2-19

\twolineshloka
{सा तु कृत्वाञ्जलिं दीना कन्योवाच तपोधनम्}
{न जाने कारणं तात येन मे रूपमीदृशम्} %7-2-20

\twolineshloka
{किं तु पूर्वं गतास्म्येका महर्षेर्भावितात्मनः}
{पुलस्त्यस्याश्रमं दिव्यमन्वेष्टुं स्वसखीजनम्} %7-2-21

\twolineshloka
{न च पश्याम्यहं तत्र काञ्चिदभ्यागतां सखीम्}
{रूपस्य तु विपर्यासं दृष्ट्वा त्रासादिहागता} %7-2-22

\twolineshloka
{तृणबिन्दुस्तु राजर्षिस्तपसा द्योतितप्रभः}
{ध्यानं विवेश तच्चापि ह्यपश्यदृषिकर्मजम्} %7-2-23

\twolineshloka
{स तु विज्ञाय तं शापं महर्षेर्भावितात्मनः}
{गृहीत्वा तनयां गत्वा पुलस्त्यमिदमब्रवीत्} %7-2-24

\twolineshloka
{भगवंस्तनयां मे त्वं गुणैः स्वैरेव भूषिताम्}
{भिक्षां प्रतिगृहाणेमां महर्षे स्वयमुद्यताम्} %7-2-25

\twolineshloka
{तपश्चरणयुक्तस्य श्रम्यमाणेन्द्रियस्य ते}
{शुश्रूषणपरा नित्यं भविष्यति न संशयः} %7-2-26

\twolineshloka
{तं ब्रुवाणं तु तद्वाक्यं राजर्षिं धार्मिकं तदा}
{जिघृक्षुरब्रवीत्कन्यां बाढमित्येव स द्विजः} %7-2-27

\twolineshloka
{दत्त्वा स तु यथान्यायं स्वमाश्रमपदं गतः}
{सापि तत्रावसत्कन्या तोषयन्ती पतिं गुणैः} %7-2-28

\twolineshloka
{तस्यास्तु शीलवृत्ताभ्यां तुतोष मुनिपुङ्गवः}
{प्रीतः स तु महातेजा वाक्यमेतदुवाच ह} %7-2-29

\twolineshloka
{परितुष्टोऽस्मि सुश्रोणि गुणानां सम्पदा भृशम्}
{तस्माद्देवि ददाम्यद्य पुत्रमात्मसमं तव} %7-2-30

\twolineshloka
{उभयोर्वंशकर्तारं पौलस्त्य इति विश्रुतम्}
{यस्मात्तु विश्रुतो वेदस्त्वयैषोऽध्ययतो मम} %7-2-31

\twolineshloka
{तस्मात्स विश्रवा नाम भविष्यति न संशयः}
{एवमुक्ता तु सा देवी प्रहृष्टेनान्तरात्मना} %7-2-32

\twolineshloka
{अचिरेणैव कालेनासूत विश्रवसं सुतम्}
{त्रिषु लोकेषु विख्यातं यशोधर्मसमन्वितम्} %7-2-33

\twolineshloka
{श्रुतिमान्समदर्शी च व्रताचाररतस्तथा}
{पितेव तपसा युक्तो ह्यभवद्विश्रवा मुनिः} %7-2-34


॥इत्यार्षे श्रीमद्रामायणे वाल्मीकीये आदिकाव्ये उत्तरकाण्डे पौलस्योत्पत्तिः नाम द्वितीयः सर्गः ॥७-२॥
