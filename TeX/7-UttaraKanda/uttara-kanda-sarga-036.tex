\sect{षड्त्रिंशः सर्गः — हनूमद्वरप्राप्त्यादि}

\twolineshloka
{ततः पितामहं दृष्ट्वा वायुः पुत्रवधार्दितः}
{शिशुकं तं समादाय उत्तस्थौ धातुरग्रतः} %7-36-1

\twolineshloka
{चलकुण्डलमौलिस्रक्तपनीयविभूषणः}
{पादयोर्न्यपतद्वायुस्तिस्रोपस्थाय वेधसे} %7-36-2

\twolineshloka
{तं तु वेदविदा तेन लम्बाभरणशोभिना}
{वायुमुत्थाप्य हस्तेन शिशुं तं परिमृष्टवान्} %7-36-3

\twolineshloka
{स्पृष्टमात्रस्ततः सोऽथ सलीलं पद्मयोनिना}
{जलसिक्तं यथा सस्यं पुनर्जीवितमाप्तवान्} %7-36-4

\twolineshloka
{प्राणवन्तमिमं दृष्ट्वा प्राणो गन्धवहो मुदा}
{चचार सर्वभूतेषु सन्निरुद्धं यथा पुरा} %7-36-5

\twolineshloka
{मरुद्रोधाद्विनिर्मुक्तास्ताः प्रजा मुदिताऽभवन्}
{शीतदाहविनिर्मुक्ताः पद्मिन्य इव साम्बुजाः} %7-36-6

\twolineshloka
{ततस्त्रियुग्मस्त्रिककुत्ऺित्रधामा त्रिदशार्चितः}
{उवाच देवता ब्रह्मा मारुतप्रियकाम्यया} %7-36-7

\twolineshloka
{भो महेन्द्रेशवरुणप्रजेश्वरधनेश्वराः}
{जानतामपि वः सर्वं वक्ष्यामि श्रूयतां हितम्} %7-36-8

\twolineshloka
{अनेन शिशुना कार्यं कर्तव्यं वो भविष्यति}
{तद्वदध्वं वरान्सर्वे मारुतस्यास्य तुष्टये} %7-36-9

\twolineshloka
{ततः सहस्रनयनः प्रीतियुक्तः शुभाननः}
{कुशेशयमयीं मालामुत्क्षिप्येदं वचोऽब्रवीत्} %7-36-10

\twolineshloka
{मत्करोत्सृष्टवज्रेण हनुरस्य यथा हतः}
{नाम्ना वै कपिशार्दूलो भविता हनुमानिति} %7-36-11

\twolineshloka
{अहमस्य प्रदास्यामि परमं वरमद्भुतम्}
{इतः प्रभृति वज्रस्य ममावध्यो भविष्यति} %7-36-12

\twolineshloka
{मार्तण्डस्त्वब्रवीत्तत्र भगवांस्तिमिरापहः}
{तेजसोऽस्य मदीयस्य ददामि शतिकां कलाम्} %7-36-13

\twolineshloka
{यदा तु शास्त्राण्यध्येतुं शक्तिरस्य भविष्यति}
{तदास्य शास्त्रं दास्यामि येन वाग्ग्मी भविष्यति} %7-36-14

\threelineshloka
{नचास्य भविता कश्चित्सदृशः शास्त्रदर्शने}
{वरुणश्च वरं प्रादान्नास्य मृत्युर्भविष्यति}
{वर्षायुतशतेनापि मत्पाशादुदकादपि} %7-36-15

\twolineshloka
{यमो दण्डादवध्यत्वमरोगत्वं च नित्यशः}
{वरं ददामि सन्तुष्ट अविषादं च संयुगे} %7-36-16

\twolineshloka
{गदेयं मामिका नैनं संयुगेषु वधिष्यति}
{इत्येवं वरदः प्राह तदा ह्येकाक्षिपिङ्गलः} %7-36-17

\twolineshloka
{मत्तो मदायुधानां च न वध्योऽयं भविष्यति}
{इत्येवं शङ्करेणापि दत्तोऽस्य परमो वरः} %7-36-18

\twolineshloka
{सर्वेषां ब्रह्मदण्डानामवध्योऽयं भविष्यति}
{दीर्घायुश्च महात्मा च इति ब्रह्माब्रवीद्वचः} %7-36-19

\twolineshloka
{विश्वकर्मा च दृष्ट्वैनं बालसूर्योपमं शिशुम्}
{शिल्पिनां प्रवरः प्रादाद्वरमस्य महामतिः} %7-36-20

\twolineshloka
{मत्कृतानि च शस्त्राणि यानि दिव्यानि संयुगे}
{तैरवध्यत्वमापन्नश्चिरजीवी भविष्यति} %7-36-21

\twolineshloka
{ततः सुराणां तु वरैर्दृष्ट्वा ह्येनमलङ्कृतम्}
{चतुर्मुखस्तुष्टमना वायुमाह जगद्गुरुः} %7-36-22

\twolineshloka
{अमित्राणां भयकरो मित्राणामभयङ्करः}
{अजेयो भविता पुत्रस्तव मारुत मारुतिः} %7-36-23

\twolineshloka
{कामरूपः कामचारी कामगः प्लवतां वरः}
{भवत्यव्याहतगतिः कीर्तिमांश्च भविष्यति} %7-36-24

\twolineshloka
{रावणोत्सादनार्थानि रामप्रियकरणि च}
{रोमहर्षकराण्येष कर्ता कर्माणि संयुगे} %7-36-25

\twolineshloka
{एवमुक्त्वा तमामन्त्र्य मारुतं त्वमरैः सह}
{यथागतं ययुः सर्वे पितामहपुरोगमाः} %7-36-26

\twolineshloka
{सोऽपि गन्धवहः पुत्रं प्रगृह्य गृहमानयत्}
{अञ्जनायास्तमाचख्यौ वरदत्तं विनिर्गतः} %7-36-27

\twolineshloka
{प्राप्य राम वरानेष वरदानसमन्वितः}
{बलेनात्मनि संस्थेन सोऽपूर्यत यथाऽर्णवः} %7-36-28

\twolineshloka
{तरसा पूर्यमाणोऽपि तदा वानरपुङ्गवः}
{आश्रमेषु महर्षीणामपराध्यति निर्भयः} %7-36-29

\twolineshloka
{स्रुग्भाण्डान्यग्निहोत्रं च वल्कलाजिनसञ्चयान्}
{भग्नविच्छिन्नविध्वस्तान्संशान्तानां करोत्ययम्} %7-36-30

\twolineshloka
{एवंविधानि कर्माणि प्रावर्तत महाबलः}
{सर्वेषां ब्रह्मदण्डानामवध्यः शम्भुना कृतः} %7-36-31

\onelineshloka
{जानन्त ऋषयस्तं वै सहन्ते तस्य शक्तितः} %7-36-32

\twolineshloka
{यथा केसरिणा त्वेष वायुना सोऽञ्जनासुतः}
{प्रतिषिद्धोऽपि मर्यादां लङ्घयत्येव वानरः} %7-36-33

\twolineshloka
{ततो महर्षयः क्रुद्धा भृग्वङ्गिरसवंशजाः}
{शेपुरेनं रघुश्रेष्ठ नातिक्रुद्धातिमन्यवः} %7-36-34

\threelineshloka
{बाधसे यत्समाश्रित्य बलमस्मान्प्लवङ्गम}
{तद्दीर्घकालं वेत्तासि नास्माकं शापमोहितः}
{यदा ते स्मार्यते कीर्तिस्तदा ते वर्धते बलम्} %7-36-35

\twolineshloka
{ततस्स हृततेजौजा महर्षिवचनौजसा}
{एषोश्रमाणि तान्येव मृदुभावं गतोऽचरत्} %7-36-36

\twolineshloka
{अथर्क्षरजसो नाम वालिसुग्रीवयोः पिता}
{सर्ववानरराजाऽऽसीत्तेजसा भास्करप्रभः} %7-36-37

\twolineshloka
{स तु राज्यं चिरं कृत्वा वानराणां हरीश्वरः}
{स च ऋक्षरजा नाम कालधर्मेण सङ्गतः} %7-36-38

\twolineshloka
{तस्मिन्नस्तमिते चाथ मन्त्रिभिर्मन्त्रकोविदैः}
{पित्र्ये पदे कृतो वाली सुग्रीवो वालिनः पदे} %7-36-39

\twolineshloka
{सुग्रीवेण समं त्वस्य अद्वैधं छिद्रवर्जितम्}
{आबाल्यं सख्यमभवदनिलस्याग्निना यथा} %7-36-40

\twolineshloka
{एष शापवशादेव न वेद बलमात्मनः}
{वालिसुग्रीवयोर्वैरं यदा राम समुत्थितम्} %7-36-41

\onelineshloka
{न ह्येष राम सुग्रीवो भ्राम्यमाणोऽपि वालिना} %7-36-42

\twolineshloka
{देव जानाति न ह्येष बलमात्मनि मारुतिः}
{ऋषिशापाहृतबलस्तदैष कपिसत्तमः} %7-36-43

\onelineshloka
{सिंहः कुञ्जररुद्धो वा आस्थितः सहितो रणे} %7-36-44

\twolineshloka
{पराक्रमोत्साहमतिप्रतापसौशील्यमाधुर्यनयानयैश्च}
{गाम्भीर्यचातुर्यसुवीर्यधैर्यैर्हनूमतः कोऽभ्यधिकोऽस्ति लोके} %7-36-45

\twolineshloka
{असौ पुनर्व्याकरणं ग्रहीष्यन्सूर्योन्मुखः प्रष्टुमना कपीन्द्रः}
{उद्यद्गिरेरस्तगिरिं जगाम ग्रन्थं महद्धारयनप्रमेयः} %7-36-46

\twolineshloka
{ससूत्रवृत्त्यर्थपदं महार्थं ससङ्ग्रहं साद्ध्यति वै कपीन्द्रः}
{नह्यस्य कश्चित्सदृशोऽस्ति शास्त्रे वैशारदे च्छन्दगतौ तथैव} %7-36-47

\twolineshloka
{सर्वासु विद्यासु तपोविधाने प्रस्पर्धतेऽयो हि गुरुं सुराणाम्}
{सोऽयं नवव्याकरणार्थवेत्ता ब्रह्मा भविष्यत्यपि ते प्रसादात्} %7-36-48

\onelineshloka
{प्रवीविवक्षोरिव सागरस्य लोकान्दिधक्षोरिव पावकस्य युगक्षये ह्येव यथान्तकस्य हनूमतः स्थास्यति कः पुरस्तात्} %7-36-49

\twolineshloka
{एषेव चान्ये च महाकपीन्द्राः सुग्रीवमैन्दद्विविदाः सनीलाः}
{सतारतारेयनलाः सरम्भास्त्वत्कारणाद्राम सुरैर्हि सृष्टाः} %7-36-50

\twolineshloka
{तदेत्कथितं सर्वं यन्मां त्वं परिपृच्छसि}
{हनूमतो बालभावे कर्मैतत्कथितं मया} %7-36-51

\twolineshloka
{श्रुत्वाऽगस्त्यस्य कथितं रामः सौमित्रिरेव च}
{विस्मयं परमं जग्मुर्वानरा राक्षसैः सह} %7-36-52

\twolineshloka
{अगस्त्यस्त्वब्रवीद्रामं सर्वमेतछ्रुतं त्वया}
{दृष्टः सम्भाषितश्चासि राम गच्छामहे वयम्} %7-36-53

\twolineshloka
{श्रुत्वैतद्राघवो वाक्यमगस्त्यस्योग्रतेजसः}
{प्राञ्जलिः प्रणतश्चापि महर्षिमिदमब्रवीत्} %7-36-54

\twolineshloka
{अद्य मे देवता हृष्टाः पितरः प्रपितामहाः}
{युष्माकं दर्शनादेव नित्यं तुष्टाः सबान्धवाः} %7-36-55

\twolineshloka
{विज्ञाप्यं तु ममैतद्धि यद्वदाम्यागतस्पृहः}
{तद्भवद्भिर्मम कृते कर्तव्यमनुकम्पया} %7-36-56

\twolineshloka
{पौरजानपदान्स्थाप्य स्वकार्येष्वहमागतः}
{क्रतूनेव करिष्यामि प्रभावाद्भवतां सताम्} %7-36-57

\twolineshloka
{सदस्या मम यज्ञेषु भवन्तो नित्यमेव तत्}
{भविष्यथ महावीर्या ममानुग्रहकाङ्क्षिणः} %7-36-58

\twolineshloka
{अहं युष्मान्समाश्रित्य तपोनिर्धूतकल्मषान्}
{अनुग्रहीतः पितृभिर्भविष्यामि सुनिर्वृतः} %7-36-59

\twolineshloka
{तदागन्तव्यमनिशं भवद्भिरिह सङ्गतैः}
{अगस्त्याद्यास्तु तच्छ्रुत्वा ऋषयः संशितव्रताः} %7-36-60

\onelineshloka
{एवमस्त्विति तं चोक्त्वा प्रयातुमुपचक्रमुः} %7-36-61

\twolineshloka
{एवमुक्त्वा गताः सर्वे ऋषयस्ते यथागतम्}
{राघवश्च तमेवार्थं चिन्तयामास विस्मितः} %7-36-62

\threelineshloka
{ततोऽस्तं भास्करे याते विसृज्य नृपवानरान्}
{सन्ध्यामुपास्य विधिवत्तदा नरवरोत्तमः}
{प्रवृत्तायां रजन्यां तु सोऽन्तःपुरचरोऽभवत्} %7-36-63


॥इत्यार्षे श्रीमद्रामायणे वाल्मीकीये आदिकाव्ये उत्तरकाण्डे हनूमद्वरप्राप्त्यादि नाम षड्त्रिंशः सर्गः ॥७-३६॥
