\sect{एकोनसप्ततितमः सर्गः — लवणवधः}

\twolineshloka
{तच्छ्रुत्वा भाषितं तस्य शत्रुघ्नस्य महात्मनः}
{क्रोधामाहारयत्तीव्रं तिष्ठ तिष्ठेति चाब्रवीत्} %7-69-1

\twolineshloka
{पाणौ पाणिं विनिष्पिष्य दन्तान्कटकटाय्य च}
{लवणो रघुशार्दूलमाह्वयामास चासकृत्} %7-69-2

\twolineshloka
{तं ब्रुवाणं तथा वाक्यं लवणं घोरदर्शनम्}
{शत्रुघ्नो देवशत्रुघ्न इदं वचनमब्रवीत्} %7-69-3

\twolineshloka
{न शत्रुघ्नस्तथा जातो यथाऽन्ये निर्जितास्त्वया}
{तदद्य बाणाभिहतो व्रज त्वं यमसादनम्} %7-69-4

\twolineshloka
{ऋषयोऽप्यद्य पापात्मन्मया त्वां निहतं रणे}
{पश्यन्तु विप्रा विद्वांसस्त्रिदशा इव रावणम्} %7-69-5

\twolineshloka
{त्वयि मद्बाणनिर्दग्धे पतितेऽद्य निशाचर}
{पुरे जनपदे चापि क्षेममेव भविष्यति} %7-69-6

\twolineshloka
{अद्य मद्बाहुनिष्क्रान्तः शरो वज्रनिभाननः}
{प्रवेक्ष्यते ते हृदयं पद्ममंशुरिवार्कजः} %7-69-7

\twolineshloka
{एवमुक्तो महावृक्षं लवणः क्रोधमूर्च्छितः}
{शत्रुघ्नोरसि चिक्षेप स च तं शतधाच्छिनत्} %7-69-8

\twolineshloka
{तद्दृष्ट्वा विफलं कर्म राक्षसः पुनरेव तु}
{पादपान्सुबहून्गृह्य शत्रुघ्नायासृजद्बली} %7-69-9

\twolineshloka
{शत्रुघ्नश्चापि तेजस्वी वृक्षानापततो बहून्}
{त्रिभिश्चतुर्भिरेकैकं चिच्छेद नतपर्वभिः} %7-69-10

\twolineshloka
{ततो बाणमयं वर्षं व्यसृजद्राक्षसोरसि}
{शत्रुघ्नो वीर्यसम्पन्नो विव्यथे न स राक्षसः} %7-69-11

\twolineshloka
{ततः प्रहस्य लवणो वृक्षमुद्यम्य वीर्यवान्}
{शिरस्यभ्यहनच्छूरं स्रस्ताङ्गः स मुमोह वै} %7-69-12

\twolineshloka
{तस्मिन्निपतिते वीरे हाहाकारो महानभूत्}
{ऋषीणां देवसङ्घानां गन्धर्वाप्सरसां तथा} %7-69-13

\twolineshloka
{तमवज्ञाय तु हतं शत्रुघ्नं भुवि पातितम्}
{रक्षो लब्धान्तरमपि न विवेश स्वमालयम्} %7-69-14

\twolineshloka
{नापि शूलं प्रजग्राह तं दृष्ट्वा भुवि पातितम्}
{ततो हत इति ज्ञात्वा तान्भक्षान्समुदावहत्} %7-69-15

\twolineshloka
{मुहूर्ताल्लब्धसञ्ज्ञस्तु पुनस्तस्थौ धृतायुधः}
{शत्रुघ्नो वै पुरद्वारि ऋषिभिः सम्प्रपूजितः} %7-69-16

\twolineshloka
{ततो दिव्यममोघं तं जग्राह शरमुत्तमम्}
{ज्वलन्तं तेजसा घोरं पूरयन्तं दिशो दश} %7-69-17

\twolineshloka
{वज्राननं वज्रवेगं मेरुमन्दरसन्निभम्}
{नतं पर्वसु सर्वेषु संयुगेष्वपराजितम्} %7-69-18

\twolineshloka
{असृक्चन्दनदिग्धाङ्गं चारुपत्रं पतत्रिणम्}
{दानवेन्द्राचलेन्द्राणामसुराणां च दारणम्} %7-69-19

\twolineshloka
{तं दीप्तमिव कालाग्निं युगान्ते समुपस्थिते}
{दृष्ट्वा सर्वाणि भूतानि परित्रासमुपागमन्} %7-69-20

\twolineshloka
{सदेवासुरगन्धर्वं मुनिभिः साप्सरोगणम्}
{जगद्धि सर्वमस्वस्थं पितामहमुपस्थितम्} %7-69-21

\twolineshloka
{उवाच देवदेवेशं वरदं प्रपितामहम्}
{देवानां भयसंमोहो लोकानां सङ्क्षयं प्रति} %7-69-22

\twolineshloka
{नेदृशं दृष्टपूर्वं च न श्रुतं प्रपितामह}
{देवानां भयसम्मोहो लोकानां सङ्क्षयं प्रति} %7-69-23

\threelineshloka
{तेषां तद्वचनं श्रुत्वा ब्रह्मा लोकपितामहः}
{भयकारणमाचष्ट लोकानामभयङ्करः}
{उवाच मधुरां वाणीं शृणुध्वं सर्वदेवताः} %7-69-24

\twolineshloka
{वधाय लवणस्याजौ शरः शत्रुघ्नधारितः}
{तेजसा तस्य सम्मूढाः सर्वे स्मः सुरसत्तमाः} %7-69-25

\twolineshloka
{एष पूर्वस्य देवस्य लोककर्तुः सनातनः}
{शरस्तेजोमयो वत्सा येन वै भयमागतम्} %7-69-26

\twolineshloka
{एष वै कैटभस्यार्थे मधुनश्च महाशरः}
{सृष्टो महात्मना तेन वधार्थे दैत्ययोस्तयोः} %7-69-27

\twolineshloka
{एक एव प्रजानाति विष्णुस्तेजोमयं शरम्}
{एषा एव तनुः पूर्वा विष्णोस्तस्य महात्मनः} %7-69-28

\twolineshloka
{इतो गच्छत पश्यध्वं वध्यमानं महात्मना}
{रामानुजेन वीरेण लवणं राक्षसोत्तमम्} %7-69-29

\twolineshloka
{तस्य ते देवदेवस्य निशम्य वचनं सुराः}
{आजग्मुर्यत्र युध्येते शत्रुघ्नलवणावुभौ} %7-69-30

\twolineshloka
{तं शरं दिव्यसङ्काशं शत्रुघ्नकरधारितम्}
{ददृशुः सर्वभूतानि युगान्ताग्निमिवोत्थितम्} %7-69-31

\twolineshloka
{अकाशमावृतं दृष्ट्वा देवैर्हि रघुनन्दनः}
{सिंहनादं भृशं कृत्वा ददर्श लक्षणं पुनः} %7-69-32

\twolineshloka
{आहूतश्च पुनस्तेन शत्रुघ्नेन महात्मना}
{लवणः क्रोधसंयुक्तो युद्धाय समुपस्थितः} %7-69-33

\twolineshloka
{आकर्णात्स विकृष्याथ तद्धनुर्धन्विनां वरः}
{तं मुमोच महाबाणं लवणस्य महोरसि} %7-69-34

\onelineshloka
{उरस्तस्य विदार्याशु प्रविवेश रसातलम्} %7-69-35

\twolineshloka
{गत्वा रसातलं दिव्यः शरो विबुधपूजितः}
{पुनरेवागमत्तूर्णमिक्ष्वाकुकुलनन्दनम्} %7-69-36

\twolineshloka
{शत्रुघ्नशरनिर्भिन्नो लवणः स निशाचरः}
{पपात सहसा भूमौ वज्राहत इवाचलः} %7-69-37

\twolineshloka
{तच्च शूलं महत्तेन हते लवणराक्षसे}
{पश्यतां सर्वदेवानां रुद्रस्य वशमन्वगात्} %7-69-38

\twolineshloka
{एकेषुपातेन भृशं निपात्य लोकत्रयस्यापि रघुप्रवीरः}
{विनिर्बभावुत्तमचापबाणस्तमः प्रणुद्येव सहस्ररश्मिः} %7-69-39

\twolineshloka
{ततो हि देवा ऋषिपन्नगाश्च प्रपूजिरे ह्यप्सरसश्च सर्वाः}
{दिष्ट्या जयो दाशरथेरवाप्तस्त्यक्त्वा भयं सर्प इव प्रशान्तः} %7-69-40


॥इत्यार्षे श्रीमद्रामायणे वाल्मीकीये आदिकाव्ये उत्तरकाण्डे लवणवधः नाम एकोनसप्ततितमः सर्गः ॥७-६९॥
