\sect{सप्ताधिकशततमः सर्गः — कुशलवाभिषेकः}

\twolineshloka
{विसृज्य लक्ष्मणं रामो दुःखशोकसमन्वितः}
{पुरोधसं मन्त्रिणश्च नैगमांश्चेदमब्रवीत्} %7-107-1

\twolineshloka
{अद्य राज्येऽभिषेक्ष्यामि भरतं धर्मवत्सलम्}
{अयोध्यायाः पतिं वीरं ततो यास्याम्यहं वनम्} %7-107-2

\twolineshloka
{प्रवेशयत सम्भारान्माभूत्कालस्य पर्ययः}
{अद्यैवाहं गमिष्यामि लक्ष्मणेन गतां गतिम्} %7-107-3

\twolineshloka
{तच्छ्रुत्वा राघवेणोक्तं सर्वाः प्रकृतयो भृशम्}
{मूर्धभिः प्रणता भूमौ गतसत्त्वा इवाभवन्} %7-107-4

\twolineshloka
{भरतश्च विसञ्ज्ञोऽभूच्छ्रुत्वा रामस्य भाषितम्}
{राज्यं विगर्हयामास राघवं चेदमब्रवीत्} %7-107-5

\twolineshloka
{सत्येनाहं शपे राजन्स्वर्गलोके न चैव हि}
{न कामये यथा राज्यं त्वां विना रघुनन्दन} %7-107-6

\twolineshloka
{इमौ कुशलवौ राजन्नभिषिञ्च नराधिप}
{कोसलेषु कुशं वीरमुत्तरेषु तथा लवम्} %7-107-7

\twolineshloka
{शत्रुघ्नस्य तु गच्छन्तु दूतास्त्वरितविक्रमाः}
{इदं गमनमस्माकं स्वर्गायाख्यातु मा चिरम्} %7-107-8

\twolineshloka
{तच्छ्रुत्वा भरतेनोक्तं दृष्ट्वा चापि ह्यधोमुखान्}
{पौरान्दुःखेन सन्तप्तान्वसिष्ठो वाक्यमब्रवीत्} %7-107-9

\twolineshloka
{वत्स राम इमाः पश्य धरणीं प्रकृतीर्गताः}
{ज्ञात्वैषामीप्सितं कार्यं मा चैषां विप्रियं कृथाः} %7-107-10

\twolineshloka
{वसिष्ठस्य तु वाक्येन उत्थाप्य प्रकृतीजनम्}
{किं करोमीति काकुत्स्थः सर्वा वचनमब्रवीत्} %7-107-11

\twolineshloka
{ततः सर्वाः प्रकृतयो रामं वचनमब्रुवन्}
{गच्छन्तमनुगच्छामो यत्र राम गमिष्यसि} %7-107-12

\twolineshloka
{पौरेषु यदि ते प्रीतिर्यदि स्नेहो ह्यनुत्तमः}
{सपुत्रदाराः काकुत्स्थ समागच्छाम सत्पथम्} %7-107-13

\twolineshloka
{तपोवनं वा दुर्गं वा नदीमम्भोनिधिं तथा}
{वयं ते यदि न त्याज्याः सर्वान्नो नय ईश्वर} %7-107-14

\twolineshloka
{एषा नः परमा प्रीतिरेष नः परमो वरः}
{हृद्गता नः सदा प्रीतिस्तवानुगमने नृप} %7-107-15

\twolineshloka
{पौराणां दृढभक्तिं च बाढमित्येव सोऽब्रवीत्}
{स्वकृतान्तं चान्ववेक्ष्य तस्मिन्नहनि राघवः} %7-107-16

\twolineshloka
{कोसलेषु कुशं वीरमुत्तरेषु तथा लवम्}
{अभिषिच्य महात्मानावुभौ रामः कुशीलवौ} %7-107-17

\twolineshloka
{अभिषिक्तौ सुतावङ्के प्रतिष्ठाप्य पुरे ततः}
{परिष्वज्य महाबाहुर्मूर्ध्न्युपाघ्राय चासकृत्} %7-107-18

\twolineshloka
{रथानां तु सहस्राणि नागानामयुतानि च}
{दशायुतानि चाश्वानामेकैकस्य धनं ददौ} %7-107-19

\twolineshloka
{बहुरत्नौ बहुधनौ हृष्टपुष्टजनावृतौ}
{स्वे पुरे प्रेषयामास भ्रातरौ तु कुशीलवौ} %7-107-20

\twolineshloka
{अभिषिच्य सुतौ वीरौ प्रतिष्ठाप्य पुरे तदा}
{दूतान्सम्प्रेषयामास शत्रुघ्नाय महात्मने} %7-107-21


॥इत्यार्षे श्रीमद्रामायणे वाल्मीकीये आदिकाव्ये उत्तरकाण्डे कुशलवाभिषेकः नाम सप्ताधिकशततमः सर्गः ॥७-१०७॥
