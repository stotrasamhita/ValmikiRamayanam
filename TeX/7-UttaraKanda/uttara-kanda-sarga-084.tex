\sect{चतुरशीतितमः सर्गः — वृत्रतपोवर्णनम्}

\twolineshloka
{तथोक्तवति रामे तु भरते च महात्मनि}
{लक्ष्मणोऽथ शुभं वाक्यमुवाच रघुनन्दनम्} %7-84-1

\twolineshloka
{अश्वमेधो महायज्ञः पावनः सर्वपाप्मनाम्}
{पावनस्तव दुर्धर्षो रोचतां रघुनन्दन} %7-84-2

\twolineshloka
{श्रूयते हि पुरावृत्तं वासवे सुमहात्मनि}
{ब्रह्महत्यावृतः शक्रो हयमेधेन पावितः} %7-84-3

\twolineshloka
{पुरा किल महाबाहो देवासुरसमागमे}
{वृत्रो नाम महानासीद्दैतेयो लोकसम्मतः} %7-84-4

\twolineshloka
{विस्तीर्णो योजनशतमुच्छ्रितस्त्रिगुणं ततः}
{अनुरागेण लोकांस्त्रीन्स्नेहात्पश्यति सर्वतः} %7-84-5

\twolineshloka
{धर्मज्ञश्च कृतज्ञश्च बुद्ध्या च परिनिष्ठितः}
{शशास पृथिवीं स्फीतां धर्मेण सुसमाहितः} %7-84-6

\twolineshloka
{तस्मिन्प्रशासति तदा सर्वकामदुघा मही}
{रसवन्ति प्रभूनानि मूलानि च फलानि च} %7-84-7

\twolineshloka
{अकृष्टपच्या पृथिवी सुसम्पन्ना महात्मनः}
{स राज्यं तादृशं भुङ्क्ते स्फीतमद्भुतदर्शनम्} %7-84-8

\twolineshloka
{तस्य बुद्धिः समुप्तन्ना तपः कुर्यामनुत्तमम्}
{तपो हि परमं श्रेयः सम्मोहमितरत्सुखम्} %7-84-9

\twolineshloka
{स निक्षिप्य सुतं ज्येष्ठं पौरेषु मधुरेश्वरम्}
{तप उग्रं समातिष्ठत्तापयन्सर्वदेवताः} %7-84-10

\twolineshloka
{तपस्तप्यति वृत्रे तु वासवः परमार्तवत्}
{विष्णुं समुपसङ्क्रम्य वाक्यमेतदुवाच ह} %7-84-11

\twolineshloka
{तपस्यता महाबाहो लोकाः वृत्रेण निर्जिताः}
{बलवान्स हि धर्मात्मा नैनं शक्ष्यामि शासितुम्} %7-84-12

\twolineshloka
{यद्यसौ तप आतिष्ठेद्भूय एवासुरेश्वर}
{यावल्लोका धरिष्यन्ति तावदस्य वशानुगाः} %7-84-13

\twolineshloka
{तं चैनं परमोदारमुपेक्षसि महाबलम्}
{क्षणं हि न भवेद्वृत्रः क्रुद्धे त्वयि सुरेश्वर} %7-84-14

\twolineshloka
{यदा हि प्रीतिसंयोगं त्वया विष्णो समागतः}
{तदाप्रभृति लोकानां नाथत्वमुपलब्धवान्} %7-84-15

\twolineshloka
{स त्वं प्रसादं लोकानां कुरुष्व सुसमाहितः}
{त्वत्कृतेन हि सर्वं स्याप्रशान्तमरुजं जगत्} %7-84-16

\twolineshloka
{इमे हि सर्वे विष्णो त्वां निरीक्षन्ते दिवौकसः}
{वृत्रघातेन महता तेषां साह्यं कुरुष्व ह} %7-84-17

\twolineshloka
{त्वया हि नित्यशः साह्यं कृतमेषां महात्मनाम्}
{असह्यमिदमन्येषामगतीनां गतिर्भवान्} %7-84-18


॥इत्यार्षे श्रीमद्रामायणे वाल्मीकीये आदिकाव्ये उत्तरकाण्डे वृत्रतपोवर्णनम् नाम चतुरशीतितमः सर्गः ॥७-८४॥
