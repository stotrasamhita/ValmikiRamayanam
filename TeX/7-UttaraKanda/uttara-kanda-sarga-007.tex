\sect{सप्तमः सर्गः — मालिवधः}

\twolineshloka
{नारायणगिरिं ते तु गर्जन्तो राक्षसाम्बुदाः}
{ववर्षुः शरवर्षेण वर्षेणेवाद्रिमम्बुदाः} %7-7-1

\twolineshloka
{श्यामावदातस्तैर्विष्णुर्नीलैर्नक्तञ्चरोत्तमैः}
{वृतोऽञ्जनगिरीवासीत् वर्षमाणैः पयोधरैः} %7-7-2

\twolineshloka
{शलभा इव केदारं मशका इव पर्वतम्}
{यथामृतघटं दंशा मकरा इव चार्णवम्} %7-7-3

\twolineshloka
{तथा रक्षोधनुर्मुक्ता वज्रानिलमनोजवाः}
{हरिं विशन्ति स्म शरा लोका इव विपर्यये} %7-7-4

\twolineshloka
{स्यन्दनैः स्यन्दनगता गजैश्च गजपृष्ठगाः}
{अश्वारोहास्तथाश्वैश्च पादाताश्चाम्बरे स्थिताः} %7-7-5

\twolineshloka
{राक्षसेन्द्रा गिरिनिभाः शरैः शक्त्यृष्टितोमरैः}
{निरुछ्वासं हरिं चक्रुः प्राणायामा इव द्विजम्} %7-7-6

\twolineshloka
{निशाचरैस्ताड्यमानो मीनैरिव महोदधिः}
{शार्ङ्गमायम्य दुर्धर्षो राक्षसेभ्योऽसृजच्छरान्} %7-7-7

\twolineshloka
{शरैः पूर्णायतोत्सृष्टैर्वज्रवक्त्रैर्मनोजवैः}
{चिच्छेद विष्णुर्निशितैः शतशोऽथ सहस्रशः} %7-7-8

\twolineshloka
{विद्राव्य शरवर्षेण वर्षा वायुरिवोत्थितम्}
{पाञ्चजन्यं महाशङ्खं प्रदध्मौ पुरुषोत्तमः} %7-7-9

\twolineshloka
{सोऽम्बुजो हरिणा ध्मातः सर्वप्राणेन शङ्खराट्}
{ररास भीमनिर्हादस्त्रैलोक्यं व्यथयन्निव} %7-7-10

\twolineshloka
{शङ्खराजरवः सोऽथ त्रासयामास राक्षसान्}
{मृगराज इवारण्ये समदानिव कुञ्जरान्} %7-7-11

\twolineshloka
{न शेकुरश्वाः संस्थातुं विमदाः कुञ्जराभवन्}
{स्यन्दनेभ्यश्च्युता वीराः शङ्खरावितदुर्बलाः} %7-7-12

\twolineshloka
{शार्ङ्गचापविनिर्मुक्ता वज्रतुल्याननाः शराः}
{विदार्य तानि रक्षांसि सुपुङ्खा विविशुः क्षितिम्} %7-7-13

\twolineshloka
{भिद्यमानाः शरैः सङ्ख्ये नारायणकरच्युतैः}
{निपेतू राक्षसा भूमौ शैला वज्रहता इव} %7-7-14

\twolineshloka
{व्रणानि परगात्रेभ्यो विष्णुचक्रकृतानि वै}
{असृक्क्षरन्ति धाराभिः स्वर्णधारा इवाचलाः} %7-7-15

\twolineshloka
{शङ्खराजरवश्चापि शार्ङ्गचापरस्वस्तथा}
{राक्षसानां रवांश्चापि ग्रसते वैष्णवो रवः} %7-7-16

\twolineshloka
{तेषां शिरोधरान्धूताञ्छरध्वजधनूंषि च}
{रथान्पताकास्तूणीरांश्चिच्छेद स हरिः शरैः} %7-7-17

\twolineshloka
{सूर्यादिव करा घोरा ऊर्मयः सागरादिव}
{पर्वतादिव नागेन्द्रा धारौघा इव चाम्बुदात्} %7-7-18

\twolineshloka
{तथा शार्ङ्गविनिर्मुक्ताः शरा नारायणेरिताः}
{निर्धावन्तीषवस्तूर्णं शतशोथ सहस्रशः} %7-7-19

\twolineshloka
{शरभेण यथा सिंहाः सिंहेन द्विरदा यथा}
{द्विरदेन यथा व्याघ्रा व्याघ्रेण द्वीपिनो यथा} %7-7-20

\twolineshloka
{द्वीपिनेव यथा श्वानः शुना मार्जारका यथा}
{मार्जारेण यथा सर्पाः सर्पेण च यथाऽऽखवः} %7-7-21

\twolineshloka
{तथा ते राक्षसाः सर्वे विष्णुना प्रभविष्णुना}
{द्रवन्ति द्राविताश्चन्ये शायिताश्च महीतले} %7-7-22

\twolineshloka
{राक्षसानां सहस्राणि निहत्य मधुसूदनः}
{वारिजं पूरयामास तोयदं सुरराडिव} %7-7-23

\twolineshloka
{नारायणशरत्रस्तं शङ्खनादसुविह्वलम्}
{ययौ लङ्कामभिमुखं प्रभग्नं राक्षसं बलम्} %7-7-24

\twolineshloka
{प्रभग्ने राक्षसबले नारायणशराहते}
{सुमाली शरवर्षेण निववार रणे हरिम्} %7-7-25

\twolineshloka
{स तु तं छादयामास नीहार इव भास्करम्}
{राक्षसाः सत्त्वसम्पन्नाः पुनर्धैर्यं समादधुः} %7-7-26

\twolineshloka
{अथ सोऽभ्यपतद्रोषाद्राक्षसो बलदर्पितः}
{महानादं प्रकुर्वाणो राक्षसाञ्जीवयन्निव} %7-7-27

\twolineshloka
{उत्क्षिप्य लम्बाभरणं धुन्वन्करमिव द्विपः}
{ररास राक्षसो हर्षात्सतडित्तोयदो यथा} %7-7-28

\twolineshloka
{सुमालेर्नर्दतस्तस्य शिरो ज्वलितकुण्डलम्}
{चिच्छेद यन्तुरश्वाश्च भ्रान्तास्तस्य तु रक्षसः} %7-7-29

\twolineshloka
{तैरश्वैर्भ्राम्यते भ्रान्तैः सुमाली राक्षसेश्वरः}
{इन्द्रियाश्वैः परिभ्रान्तैर्धृतिहीनो यथा नरः} %7-7-30

\twolineshloka
{ततो विष्णुं महाबाहुं प्रपतन्तं रणाजिरे}
{हृते सुमालेरश्वैश्च रथे विष्णुरथं प्रति} %7-7-31

\threelineshloka
{माली चाभ्यद्रवद्युक्तः प्रगृह्य सशरं धनुः}
{मालेर्धनुश्च्युता बाणाः कार्तस्वरविभूषिताः}
{विविशुर्हरिमासाद्य क्रौञ्चं पत्ररथा इव} %7-7-32

\twolineshloka
{अर्द्यमानः शरैः सोऽथ मालिमुक्तैः सहस्रशः}
{चुक्षुभे न रणे विष्णुर्जितेन्द्रिय इवाधिभिः} %7-7-33

\twolineshloka
{अथ मौर्वीस्वनं कृत्वा भगवान्भूतभावनः}
{मालिनं प्रति बाणौघान्ससर्जारिनिषूदनः} %7-7-34

\twolineshloka
{ते मालिदेहमासाद्य वज्रविद्युत्प्रभाः शराः}
{पिबन्ति रुधिरं तस्य नागा इव सुधारसम्} %7-7-35

\twolineshloka
{मालिनं विमुखं कृत्वा शङ्खचक्रगदाधरः}
{मालिमौलिं ध्वजं चापं वाजिनश्चाप्यपातयत्} %7-7-36

\twolineshloka
{विरथस्तु गदां गृह्य माली नक्तञ्चरोत्तमः}
{आपुप्लुवे गदापाणिर्गिर्यग्रादिव केसरी} %7-7-37

\twolineshloka
{गदया गरुडेशानमीशानमिव चान्तकः}
{ललाटदेशेऽभ्यहनद्वज्रेणेन्द्रो यथाऽचलम्} %7-7-38

\twolineshloka
{गदयाभिहतस्तेन मालिना गरुडो भृशम्}
{रणात्पराङ्मुखं देवं कृतवान्वेदनातुरः} %7-7-39

\twolineshloka
{पराङ्मुखे कृते देवे मालिना गरुडेन वै}
{उदतिष्ठन्महाञ्छब्दो रक्षसामभिनर्दताम्} %7-7-40

\onelineshloka
{रक्षसां रवतां रावं श्रुत्वा हरिहयानुजः} %7-7-41

\twolineshloka
{तिर्यगास्थाय सङ्क्रुद्धः पक्षीशे भगवान्हरिः}
{पराङ्मुखोऽप्युत्ससर्ज मालेश्चक्रं जिघांसया} %7-7-42

\twolineshloka
{तत्सूर्यमण्डलाभासं स्वभासा भासयन्नभः}
{कालचक्रनिभं चक्रं मालेः शीर्षमपातयत्} %7-7-43

\twolineshloka
{तच्छिरो राक्षसेन्द्रस्य चक्रोत्कृत्तं बिभीषणम्}
{पपात रुधिरोद्गारि पुरा राहुशिरो यथा} %7-7-44

\twolineshloka
{ततः सुरैः संसहृष्टैः सर्वप्राणसमीरितः}
{सिंहनादरवोन्मुक्तः साधु देवेतिवादिभिः} %7-7-45

\twolineshloka
{मालिनं निहतं दृष्ट्वा सुमाली माल्यवानपि}
{सबलौ शोकसन्तप्तौ लङ्कामेव प्रधावितौ} %7-7-46

\twolineshloka
{गरुडस्तु समाश्वस्तः सन्निवृत्य यथा पुरा}
{राक्षसान्द्रावयामास पक्षवातेन कोपितः} %7-7-47

\twolineshloka
{चक्रकृत्तास्यकमला गदासञ्चूर्णितोरसः}
{लाङ्गलग्लपितग्रीवा मुसलैर्भिन्नमस्तकाः} %7-7-48

\twolineshloka
{केचिच्चैवासिना छिन्नास्तथान्ये शरपीडिताः}
{निपेतुरम्बरात्तूर्णं राक्षसाः सागराम्भसि} %7-7-49

\twolineshloka
{नारायणो बाणवराशनीभिर्विदारयामास धनुर्विमुक्तैः}
{नक्तञ्चरान्मुक्तविधूतकेशान्यथाऽशनीभिः सतडिन्महाभ्राः} %7-7-50

\twolineshloka
{भिन्नातपत्रं पतमानमस्त्रं शरैरपध्वस्तविनीतवेषम्}
{विनिस्सृतान्त्रं भयलोलनेत्रं बलं तदुन्मत्ततरं बभूव} %7-7-51

\twolineshloka
{सिंहार्दितानामिव कुञ्जराणां निशाचराणां सह कुञ्जराणाम्}
{रवाश्च वेगाश्च समं बभूवुः पुराणसिंहेन विमर्दितानाम्} %7-7-52

\twolineshloka
{ते वार्यमाणा हरिबाणजालैः सबाणजालानि समुत्सृजन्तः}
{धावन्ति नक्तञ्चरकालमेघा वायुप्रभिन्ना इव कालमेघाः} %7-7-53

\twolineshloka
{चक्रपहारैर्विनिकृत्तशीर्षाः सञ्चूर्णिताङ्गाश्च गदाप्रहारैः}
{अभिप्रहारैर्द्विविधा विभिन्नाः पतन्ति शैला इव राक्षसेन्द्राः} %7-7-54

\twolineshloka
{विलम्बमानैर्मणिहारकुण्डलैर्निशाचरैर्नीलबलाहकोपमैः}
{निपात्यमानैर्ददृशे निरन्तरं निपात्यमानैरिव नीलपर्वतैः} %7-7-55


॥इत्यार्षे श्रीमद्रामायणे वाल्मीकीये आदिकाव्ये उत्तरकाण्डे मालिवधः नाम सप्तमः सर्गः ॥७-७॥
