\sect{एकनवतितमः सर्गः — यज्ञसंविधानम्}

\twolineshloka
{एतदाख्याय काकुत्स्थो भ्रातृभ्याममितप्रभः}
{लक्ष्मणं पुनरेवाह धर्मयुक्तमिदं वचः} %7-91-1

\twolineshloka
{वसिष्ठं वामदेवं च जाबालिमथ कश्यपम्}
{द्विजांश्च सर्वप्रवरानश्वमेधपुरस्कृतान्} %7-91-2

\twolineshloka
{एतान्सर्वान्समानीय मन्त्रयित्वा च लक्ष्मण}
{हयं लक्षणसम्पन्नं विमोक्ष्यामि समाधिना} %7-91-3

\twolineshloka
{तद्वाक्यं राघवेणोक्तं श्रुत्वा त्वरितविक्रमः}
{द्विजान्सर्वान्समाहूय दर्शयामास राघवम्} %7-91-4

\twolineshloka
{ते दृष्ट्वा देवसङ्काशं कृतपादाभिवन्दनम्}
{राघवं सुदुराधर्षमाशीर्भिः समपूजयन्} %7-91-5

\twolineshloka
{प्राञ्जलिः स तदा भूत्वा राघवो द्विजसत्तमान्}
{आचचक्षेऽश्वमेधस्य ह्यभिप्रायं महायशाः} %7-91-6

\twolineshloka
{ते तु रामस्य तच्छ्रुत्वा नमस्कृत्वा वृषध्वजम्}
{अश्वमेधं द्विजाः सर्वे पूजयन्ति स्म सर्वशः} %7-91-7

\twolineshloka
{स तेषां द्विजमुख्यानां वाक्यमद्भुतदर्शनम्}
{अश्वमेधाश्रितं श्रुत्वा भृशं प्रीतोऽभवत्तदा} %7-91-8

\twolineshloka
{विज्ञाय कर्म तत्तेषां रामो लक्ष्मणमब्रवीत्}
{प्रेषयस्व द्रुतं दूतान् सुग्रीवाय महात्मने} %7-91-9

\twolineshloka
{यथा महद्भिर्हरिभिर्बहुभिश्च वनौकसाम्}
{सार्धमागच्छ भद्रं ते ह्यनुभोक्तुं महोत्सवम्} %7-91-10

\twolineshloka
{बिभीषणश्च रक्षोभिः कामगैर्बहुभिर्वृतः}
{अश्वमेधं महायज्ञमायात्वतुलविक्रमः} %7-91-11

\twolineshloka
{राजानश्च महाभागा ये मे प्रियचिकीर्षवः}
{सानुगाः क्षिप्रमायान्तु यज्ञं द्रष्टुमनुत्तमम्} %7-91-12

\twolineshloka
{देशान्तरगता ये च द्विजा धर्मसमाहिताः}
{आमन्त्रयस्व तान्सर्वानश्वमेधाय लक्ष्मण} %7-91-13

\threelineshloka
{ऋषयश्च महाबाहो आहूयन्तां तपोधनाः}
{देशान्तरगताः सर्वे सदाराश्च द्विजातयः}
{तथैव तालावचरास्तथैव नटनर्तकाः} %7-91-14

\twolineshloka
{यज्ञवाटश्च सुमहान्गोमत्या नैमिशे वने}
{आज्ञाप्यतां महाबाहो तद्धि पुण्यमनुत्तमम्} %7-91-15

\onelineshloka
{शान्तयश्च महाबाहो प्रवर्त्यन्तां समन्ततः} %7-91-16

\twolineshloka
{शतशश्चापि धर्मज्ञाः क्रतुमुख्यमनुत्तमम्}
{अनुभूय महायज्ञं नैमिशे रघुनन्दन} %7-91-17

\twolineshloka
{तुष्टः पुष्टश्च सर्वोऽसौ मानितश्च यथाविधि}
{प्रीतिं यास्यति धर्मज्ञ शीघ्रमामन्त्र्यतां जनः} %7-91-18

\twolineshloka
{शतं वाहसहस्राणां तण्डुलानां वपुष्मताम्}
{अयुतं तिलमुद्गानां प्रयात्वग्रे महाबल} %7-91-19

\twolineshloka
{चणकानां कुलित्थानां माषाणां लवणस्य च}
{अतोऽनुरूपं स्नेहं च गन्धं सङ्क्षिप्तमेव च} %7-91-20

\twolineshloka
{सुवर्णकोट्यो बहुला हिरण्यस्य शतोत्तराः}
{अग्रतो भरतः कृत्वा गच्छत्वग्रे समाधिना} %7-91-21

\twolineshloka
{अन्तरा पण्यवीथ्यश्च सर्वे च नटनर्तकाः}
{सूदा नार्यश्च बहवो नित्यं यौवनशालिनः} %7-91-22

\twolineshloka
{भरतेन तु सार्धं ते यान्तु सैन्यानि चाग्रतः}
{नैगमा बलवृद्धाश्च द्विजाश्च सुसमाहिताः} %7-91-23

\twolineshloka
{कर्मान्तिकान्वर्धकिनः कोशाध्यक्षांश्च नैगमान्}
{मम मातऽस्तथा सर्वाः कुमारान्तःपुराणि च} %7-91-24

\twolineshloka
{काञ्चनीं मम पत्नीं च दीक्षायां ज्ञांश्च कर्मणि}
{अग्रतो भरतः कृत्वा गच्छत्वग्रे महायशाः} %7-91-25

\twolineshloka
{उपकार्या महार्हाश्च पार्थिवानां महौजसाम्}
{सानुगानां नरश्रेष्ठो व्यादिदेश महाबलः} %7-91-26


॥इत्यार्षे श्रीमद्रामायणे वाल्मीकीये आदिकाव्ये उत्तरकाण्डे यज्ञसंविधानम् नाम एकनवतितमः सर्गः ॥७-९१॥
