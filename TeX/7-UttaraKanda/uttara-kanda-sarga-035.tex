\sect{पञ्चत्रिंशः सर्गः — हनूमदुत्पत्तिः}

\twolineshloka
{अपृच्छत तदा रामो दक्षिणाशाश्रमं मुनिम्}
{प्राञ्जलिर्विनयोपेत इदमाह वचोऽर्थवत्} %7-35-1

\twolineshloka
{अतुलं बलमेतद्वै वालिनो रावणस्य च}
{न त्वेताभ्यां हनुमता समं त्विति मतिर्मम} %7-35-2

\twolineshloka
{शौर्यं दाक्ष्यं बलं धैर्यं प्राज्ञता नयसाधनम्}
{विक्रमश्च प्रभावश्च हनूमति कृतालयाः} %7-35-3

\twolineshloka
{दृष्ट्वैव सागरं वीक्ष्य सीदन्तीं कपिवाहिनीम्}
{समाश्वास्य महाबाहुर्योजनानां शतं प्लुतः} %7-35-4

\twolineshloka
{धर्षयित्वा पुरीं लङ्कां रावणान्तःपुरं तदा}
{दृष्ट्वा सम्भाषिता चापि सीता ह्याश्वासिता तथा} %7-35-5

\twolineshloka
{सेनाग्रगा मन्त्रिसुताः किङ्करा रावणात्मजः}
{एते हनुमता तत्र ह्येकेन वितिपातिताः} %7-35-6

\twolineshloka
{भूयो बन्धाद्विमुक्तेन भाषयित्वा दशाननम्}
{लङ्का भस्मीकृता येन पावकेनेव मेदिनी} %7-35-7

\twolineshloka
{न कालस्य न शक्रस्य न विष्णोर्वित्तपस्य च}
{कर्माणि तानि श्रूयन्ते यानि युद्धे हनूमतः} %7-35-8

\twolineshloka
{एतस्य बाहुवीर्येण लङ्का सीता च लक्ष्मणः}
{प्राप्ता मया जयश्चैव राज्यं मित्राणि बान्धवाः} %7-35-9

\twolineshloka
{हनूमान्यदि मे नः स्याद्वानराधिपतेः सखा}
{प्रवृत्तिमपि को वेत्तुं जानक्याः शक्तिमान्भवेत्} %7-35-10

\twolineshloka
{किमर्थं वाल्यनेनैव सुग्रीवप्रियकाम्यया}
{तदा वैरे समुत्पन्ने न दग्धो वीरुधो यथा} %7-35-11

\twolineshloka
{नहि वेदितवान्मन्ये हनूमानात्मनो बलम्}
{यद्दृष्टवाञ्जीवितेष्टं क्लिश्यन्तं वानराधिपम्} %7-35-12

\twolineshloka
{एतन्मे भगवन्सर्वं हनूमति महामतौ}
{विस्तरेण यथातत्त्वं कथयामरपूजित} %7-35-13

\twolineshloka
{राघवस्य वचः श्रुत्वा हेतुयुक्तमृषिस्तदा}
{हनूमतः समक्षं तमिदं वचनमब्रवीत्} %7-35-14

\twolineshloka
{सत्यमेतद्रघुश्रेष्ठ यद्ब्रवीषि हनूमतः}
{न बले विद्यते तुल्यो न गतौ न मतौ परः} %7-35-15

\twolineshloka
{अमोघशापैः शप्तस्तु दत्तोऽस्य मुनिभिः पुरा}
{न वेत्ता हि बलं येन बली सन्निरिमर्दनः} %7-35-16

\twolineshloka
{बाल्येऽप्येतेन यत्कर्म कृतं राम महाबल}
{तन्न वर्णयितुं शक्यमिति बालतयाऽऽस्यते} %7-35-17

\twolineshloka
{यदि वाऽस्ति त्वभिप्रायस्तच्छ्रोतुं तव राघव}
{समाधाय मतिं राम निशामय वदाम्यहम्} %7-35-18

\twolineshloka
{सूर्यदत्तवरस्वर्णः सुमेरुर्नाम पर्वतः}
{यत्र राज्यं प्रशास्त्यस्य केसरी नाम वै पिता} %7-35-19

\twolineshloka
{तस्य भार्या बभूवैषा ह्यञ्जनेति परिश्रुता}
{जनयामास तस्यां वै वायुरात्मजमुत्तमम्} %7-35-20

\twolineshloka
{शालिशूकनिभाभासं प्रासूतामुं तदाऽञ्जना}
{फलान्याहर्तुकामा वै निष्क्रान्ता गहनेचरा} %7-35-21

\twolineshloka
{एष मातुर्वियोगाच्च क्षुधया च भृशार्दितः}
{रुरोद शिशुरत्यर्थं शिशुः शरवणे यथा} %7-35-22

\twolineshloka
{तदोद्यन्तं विवस्वन्तं जपापुष्पोत्करोपमम्}
{ददर्श फललोभाच्च ह्युत्पपात रविं प्रति} %7-35-23

\twolineshloka
{बालार्काभिमुखो बालो बालार्क इव मूर्तिमान्}
{ग्रहीतुकामो बालार्कं प्लवतेऽम्बरमध्यगः} %7-35-24

\twolineshloka
{एतस्मिन्प्लवमाने तु शिशुभावे हनूमति}
{देवदानवयक्षाणां विस्मयः सुमहानभूत्} %7-35-25

\twolineshloka
{नाप्येवं वेगवान्वायुर्गरुडो वा मनस्तथा}
{यथाऽयं वायुपुत्रस्तु क्रमतेऽम्बरमुत्तमम्} %7-35-26

\twolineshloka
{यदि तावच्छिशोरस्य त्वीदृशो गतिविक्रमः}
{यौवनं बलमासाद्य कथं वेगो भविष्यति} %7-35-27

\twolineshloka
{तमनुप्लवते वायुः प्लवन्तं पुत्रमात्मनः}
{सूर्यदाहभयाद्रक्षंस्तुषारचयशीतलः} %7-35-28

\twolineshloka
{बहुयोजनसाहस्रं क्रमत्येष गतोम्बरम्}
{पितुर्बलाच्च बाल्याच्च भास्कराभ्याशमागतः} %7-35-29

\twolineshloka
{शिशुरेष त्वदोषज्ञ इति मत्वा दिवाकरः}
{कार्यं चात्र समायत्तमित्येवं न ददाह सः} %7-35-30

\twolineshloka
{यमेव दिवसं ह्येष ग्रहीतुं भास्करं प्लुतः}
{तमेव दिवसं राहुर्जिघृक्षति दिवाकरम्} %7-35-31

\twolineshloka
{अनेन स परामृष्टो राम सूर्यरथोपरि}
{अपक्रान्तस्ततस्त्रस्तो राहुश्चन्द्रार्कमर्दनः} %7-35-32

\twolineshloka
{स इन्द्रभवनं गत्वा सरोषः सिंहिकासुतः}
{अब्रवीद्भ्रुकुटिं कृत्वा देवं देवगणैर्वृतम्} %7-35-33

\twolineshloka
{बुभुक्षापनयं दत्त्वा चन्द्रार्कौ मम वासव}
{किमिदं तत्त्वया दत्तमन्यस्य बलवृत्रहन्} %7-35-34

\twolineshloka
{अद्याहं पर्वकाले तु जिघृक्षुः सूर्यमागतः}
{अथान्यो राहुरासाद्य जग्राह सहसा रविम्} %7-35-35

\twolineshloka
{स राहोर्वचनं श्रुत्वा वासवः सम्भ्रमान्वितः}
{उत्पपातासनं हित्वा चोद्वहन्काञ्चनीं स्रजम्} %7-35-36

\twolineshloka
{ततः कैलासकूटाभं चतुर्दन्तं मदस्रवम्}
{शृङ्गारधारिणं प्रांशुं स्वर्णघण्टाट्टहासिनम्} %7-35-37

\twolineshloka
{इन्द्रः करीन्द्रमारुह्य राहुं कृत्वा पुरःसरम्}
{प्रायाद्यत्राभवत्सूर्यः सहानेन हनूमता} %7-35-38

\twolineshloka
{अथातिरभसेनागाद्राहुरुत्सृज्य वासवम्}
{अनेन च स वै दृष्टः प्रधावञ्छैलकूटवत्} %7-35-39

\twolineshloka
{ततः सूर्यं समुत्सृज्य राहुं फलमवेक्ष्य च}
{उत्पपात पुनर्व्योम ग्रहीतुं सिंहिकासुतम्} %7-35-40

\twolineshloka
{उत्सृज्यार्कमिमं राम प्रधावन्तं प्लवङ्गमम्}
{अवेक्ष्यैवं परावृत्त्य मुखशेषः पराङ्मुखः} %7-35-41

\twolineshloka
{इन्द्रमाशंसमानस्तु त्रातारं सिंहिकासुतः}
{इन्द्र इन्द्रेति संत्रासान्मुहुर्महुरभाषत} %7-35-42

\twolineshloka
{राहोर्विक्रोशमानस्य प्रागेवालक्षितं स्वरम्}
{श्रुत्वेन्द्रोवाच मा भैषीरहमेनं निषूदये} %7-35-43

\twolineshloka
{ऐरावतं ततो दृष्ट्वा महत्तदिदमित्यपि}
{फलं मत्वा हस्तिराजमभिदुद्राव मारुतिः} %7-35-44

\twolineshloka
{तथास्य धावतो रूपमैरावतजिघृक्षया}
{मुहूर्तमभवद्घोरमिन्द्राग्न्योरिव भास्वरम्} %7-35-45

\twolineshloka
{एवमाधावमानं तु नातिक्रुद्धः शचीपतिः}
{हस्तान्तादतिमुक्तेन कुलिशेनाभ्यताडयत्} %7-35-46

\twolineshloka
{ततो गिरौ पपातैष इन्द्रवज्राभिताडितः}
{पतमानस्य चैतस्य वामो हनुरभज्यत} %7-35-47

\twolineshloka
{तस्मिंस्तु पतिते बाले वज्रताडनविह्वले}
{चुक्रोधेन्द्राय पवनः प्रजानामहिताय सः} %7-35-48

\twolineshloka
{प्रचारं स तु सङ्गृह्य प्रजास्वन्तर्गतः प्रभुः}
{गुहां प्रविष्टः स्वसुतं शिशुमादाय मारुतः} %7-35-49

\twolineshloka
{विण्मूत्राशयमावृत्य प्रजानां परमार्तिकृत्}
{रुरोध सर्वभूतानि यथा वर्षाणि वासवः} %7-35-50

\twolineshloka
{वायुप्रकोपाद्भूतानि निरुच्छ्वासानि सर्वतः}
{सन्धिभिर्भिद्यमानैश्च काष्ठभूतानि जज्ञिरे} %7-35-51

\twolineshloka
{निःस्वाध्यायवषट्कारं निष्क्रियं धर्मवर्जितम्}
{वायुप्रकोपात्ऺत्रैलोक्यं निरयस्थमिवाभवत्} %7-35-52

\twolineshloka
{ततः प्रजाः सगन्धर्वाः सदेवासुरमानुषाः}
{प्रजापतिं समाधावन्दुःखिताश्च सुखेच्छया} %7-35-53

\twolineshloka
{ऊचुः प्राञ्जलयो देवा महोदरनिभोदराः}
{त्वया नु भगवन्सृष्टाः प्रजानाथ चतुर्विधाः} %7-35-54

\twolineshloka
{त्वया दत्तोऽयमस्माकमायुषः पवनः पतिः}
{सोऽस्मान्प्राणेश्वरो भूत्वा कस्मादेषोऽद्य सत्तम} %7-35-55

\twolineshloka
{रुरोध दुःखं जनयन्नन्तःपुर इव स्त्रियः}
{तस्मात्त्वां शरणं प्राप्ता वायुनोहता वयम्} %7-35-56

\twolineshloka
{एतत्प्रजानां श्रुत्वा तु प्रजानाथः प्रजापतिः}
{कारणादिति चोक्त्वासौ प्रजाः पुनरभाषत} %7-35-57

\twolineshloka
{यस्मिंश्च कारणे वायुश्चुक्रोध च रुरोध च}
{प्रजाः शृणुध्वं तत्सर्वं श्रोतव्यं चात्मनः क्षमम्} %7-35-58

\twolineshloka
{पुत्रस्तस्यामरेशेन इन्द्रेणाद्य निपातितः}
{राहोर्वचनमास्थाय ततः स कुपितोऽनिलः} %7-35-59

\twolineshloka
{अशरीरः शरीरेषु वायुश्चरति पालयन्}
{शरीरं हि विना वायुं समतां याति दारुभिः} %7-35-60

\twolineshloka
{वायुः प्राणः सुखं वायुर्वायुः सर्वमिदं जगत्}
{वायुना सम्परित्यक्तं न सुखं विन्दते जगत्} %7-35-61

\twolineshloka
{अद्यैव च परित्यक्तं वायुना जगदायुषा}
{अद्यैव ते निरुच्छ्वासाः काष्ठकुड्योपमाः स्थिताः} %7-35-62

\twolineshloka
{तद्यामस्तत्र यत्रास्ते मारुतो रुक्प्रदो हि नः}
{मा विनाशं गमिष्याम अप्रसाद्यादितेः सुताः} %7-35-63

\twolineshloka
{ततः प्रजाभिः सहितः प्रजापतिः सदेवगन्धर्वभुजङ्गगुह्यकैः}
{जगाम तत्रास्यति यत्र मारुतः सुतं सुरेन्द्राभिहतं प्रगृह्य सः} %7-35-64

\twolineshloka
{ततोऽर्कवैश्वानरकाञ्चनप्रभं सुतं तदोत्सङ्गगतं सदागतेः}
{चतुर्मुखो वीक्ष्य कृपामथाकरोत्सदेवगन्धर्वर्षियक्षराक्षसैः} %7-35-65


॥इत्यार्षे श्रीमद्रामायणे वाल्मीकीये आदिकाव्ये उत्तरकाण्डे हनूमदुत्पत्तिः नाम पञ्चत्रिंशः सर्गः ॥७-३५॥
