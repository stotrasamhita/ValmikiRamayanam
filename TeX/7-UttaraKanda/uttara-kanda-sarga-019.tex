\sect{एकोनविंशः सर्गः — अनरण्यशापः}

\twolineshloka
{अथ जित्वा मरुत्तं स प्रययौ राक्षसाधिपः}
{नगराणि नरेन्द्राणां युद्धकाङ्क्षी दशाननः} %7-19-1

\twolineshloka
{समासाद्य तु राजेन्द्रान्महेन्द्रवरुणोपमान्}
{अब्रवीद्राक्षसेन्द्रस्तु युद्धं मे दीयतामिति} %7-19-2

\twolineshloka
{निर्जिताः स्मेति वा ब्रूत एष मे हि सुनिश्चयः}
{अन्यथा कुर्वतामेवं मोक्षो नैवोपपद्यते} %7-19-3

\threelineshloka
{ततस्त्वभीरवः प्राज्ञाः पार्थिवा धर्मनिश्चयाः}
{मन्त्रयित्वा ततोऽन्योन्यं राजानः सुमहाबलाः}
{निर्जिताः स्मेत्यभाषन्त ज्ञात्वा वरबलं रिपोः} %7-19-4

\twolineshloka
{दुष्यन्तः सुरथो गाधिर्गयो राजा पुरूरवाः}
{एते सर्वेऽब्रुवंस्तात निर्जिताः स्मेति पार्थिवाः} %7-19-5

\onelineshloka
{अथायोध्यां समासाद्य रावणो राक्षसाधिपः} %7-19-6

\twolineshloka
{सुगुप्तामनरण्येन शक्रेणेवामरावतीम्}
{स तं पुरुषशार्दूलं पुरन्दरसमं बले} %7-19-7

\twolineshloka
{प्राह राजानमासाद्य युद्धं देहीति रावणः}
{निर्जितोऽस्मीति वा ब्रूहि त्वमेवं मम शासनम्} %7-19-8

\twolineshloka
{अयोध्याधिपतिस्तस्य श्रुत्वा पापात्मनो वचः}
{अनरण्यस्तु सङ्क्रुद्धो राक्षसेन्द्रमथाब्रवीत्} %7-19-9

\twolineshloka
{दीयते द्वन्द्वयुद्धं ते राक्षसाधिपते मया}
{सन्तिष्ठ क्षिप्रमायत्तो भव चैवं भवाम्यहम्} %7-19-10

\twolineshloka
{अथ पूर्वं श्रुतार्थेन निर्जितं सुमहद्बलम्}
{निष्क्रामत्तन्नरेन्द्रस्य बलं रक्षोवधोद्यतम्} %7-19-11

\twolineshloka
{नागानां दशसाहस्रं वाजिनां नियुतं तथा}
{रथानां बहुसाहस्रं पत्तीनां च नरोत्तम} %7-19-12

\twolineshloka
{महीं सञ्छाद्य निष्क्रान्तं सपदातिरथं रणे}
{ततः प्रवृत्तं सुमहद्युद्धं युद्धविशारद} %7-19-13

\twolineshloka
{अनरण्यस्य नृपते राक्षसेन्द्रस्य चाद्भुतम्}
{तद्रावणबलं प्राप्य बलं तस्य महीपतेः} %7-19-14

\twolineshloka
{प्राणश्यत तदा सर्वं हव्यं हुतमिवानले}
{युद्ध्वा च सुचिरं कालं कृत्वा विक्रममुत्तमम्} %7-19-15

\threelineshloka
{प्रज्वलन्तं तमासाद्य क्षिप्रमेवावशेषितम्}
{प्राविशत्सङ्कुलं तत्र शलभा इव पावकम्}
{नश्यति स्म बलं तत्र हव्यं हुतमिवानले} %7-19-16

\twolineshloka
{सोऽपश्यत्तन्नरेन्द्रस्तु नश्यमानं महाबलम्}
{महार्णवं समासाद्य वनापगशतं यथा} %7-19-17

\twolineshloka
{ततः शक्रधनुःप्रख्यं धनुर्विस्फारयन्स्वयम्}
{आससाद नरेन्द्रस्तं रावणं क्रोधमूर्च्छितः} %7-19-18

\twolineshloka
{अनरण्येन तेऽमात्या मारीचशुकसारणाः}
{प्रहस्तसहिता भग्ना व्यद्रवन्त मृगा इव} %7-19-19

\twolineshloka
{ततो बाणशतान्यष्टौ पातयामास मूर्धनि}
{तस्य राक्षसराजस्य इक्ष्वाकुकुलनन्दनः} %7-19-20

\twolineshloka
{तस्य बाणाः पतन्तस्ते चक्रिरे न क्षतं क्वचित्}
{वारिधारा इवाभ्रेभ्यः पतन्त्यो गिरिमूर्धनि} %7-19-21

\twolineshloka
{ततो राक्षसराजेन क्रुद्धेन नृपतिस्तदा}
{तलेनाभिहतो मूर्ध्नि स रथान्निपपात ह} %7-19-22

\twolineshloka
{स राजा पतितो भूमौ विह्वलाङ्गः प्रवेपितः}
{वज्रदग्ध इवारण्ये सालो निपतितो यथा} %7-19-23

\twolineshloka
{तं प्रहस्याब्रवीद्द्रक्ष इक्ष्वाकुं पृथिवीपतिम्}
{किमिदानीं फलं प्राप्तं त्वया मां प्रति युद्ध्यता} %7-19-24

\twolineshloka
{त्रैलोक्ये नास्ति यो द्वन्द्वं मम दद्यान्नराधिप}
{शङ्के प्रसक्तो भोगेषु न शृणोषि बलं मम} %7-19-25

\twolineshloka
{तस्यैवं ब्रुवतो राजा मन्दासुर्वाक्यमब्रवीत्}
{किं शक्यमिह कर्तुं वै कालो हि दुरतिक्रमः} %7-19-26

\twolineshloka
{नह्यहं निर्जितो रक्षस्त्वया चात्मप्रशंसिना}
{कालेनैव विपन्नोऽहं हेतुभूतस्तु मे भवान्} %7-19-27

\twolineshloka
{किं त्विदानीं मया शक्यं कर्तुं प्राणपरिक्षये}
{नह्यहं विमुखी रक्षो युध्यमानस्त्वया हतः} %7-19-28

\twolineshloka
{इक्ष्वाकुपरिभावित्वाद्वचो वक्ष्यामि राक्षस}
{यदि दत्तं यदि हुतं यदि मे सुकृतं तपः यदि गुप्ताः प्रजाः सम्यक्तदा सत्यं वचोऽस्तु मे} %7-19-29

\twolineshloka
{उत्पत्स्यते कुले ह्यस्मिन्निक्ष्वाकूणां महात्मनाम्}
{रामो दाशरथिर्नाम यस्ते प्राणान्हरिष्यति} %7-19-30

\twolineshloka
{ततो जलधरोदग्रस्ताडितो देवदुन्दुभिः}
{तस्मिन्नुदाहृते शापे पुष्पवृष्टिश्च खाच्च्युता} %7-19-31

\twolineshloka
{ततः स राजा राजेन्द्र गतः स्थानं त्रिविष्टपम्}
{स्वर्गते च नृपे तस्मिन्राक्षसः सोऽपसर्पत} %7-19-32


॥इत्यार्षे श्रीमद्रामायणे वाल्मीकीये आदिकाव्ये उत्तरकाण्डे अनरण्यशापः नाम एकोनविंशः सर्गः ॥७-१९॥
