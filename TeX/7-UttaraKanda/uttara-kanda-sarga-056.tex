\sect{षट्पञ्चाशः सर्गः — मैत्रावरुणित्वप्राप्तिः}

\twolineshloka
{रामस्य भाषितं श्रुत्वा लक्ष्मणः परवीरहा}
{उवाच प्राञ्जलिर्वाक्यं राघवं दीप्ततेजसम्} %7-56-1

\twolineshloka
{निक्षिप्तदेहौ काकुत्स्य कथं तौ द्विजपार्थिवौ}
{पुनर्देहेन संयोगं जग्मतुर्देवसम्मतौ} %7-56-2

\twolineshloka
{लक्ष्मणेनैवमुक्तस्तु रामश्चेक्ष्वाकुनन्दनः}
{प्रत्युवाच महातेजा लक्ष्मणं पुरुषर्षभः} %7-56-3

\twolineshloka
{तौ परस्परशापेन देहावुत्सृज्य धार्मिकौ}
{अभूतां नृपविप्रर्षी वायुभूतौ तपोधनौ} %7-56-4

\twolineshloka
{अशरीरः शरीरस्य कृतेऽन्यस्य महामुनिः}
{वसिष्ठः सुमहातेजा जगाम पितुरन्तिकम्} %7-56-5

\twolineshloka
{सोऽभिवाद्य ततः पादौ देवदेवस्य धर्मवित्}
{पितामहमथोवाच वायुभूत इदं वचः} %7-56-6

\twolineshloka
{भगवन्निमिशापेन विदेहत्वमुपागमम्}
{लोकनाथ महादेव अण्डजोऽपि त्वमब्जजः} %7-56-7

\threelineshloka
{सर्वेषां देहहीनानां महद्दुःखं भविष्यति}
{लुप्यन्ते सर्वकार्यणि हीनदेहस्य वै प्रभो}
{देहस्यान्यस्य सद्भावे प्रसादं कर्तुमर्हसि} %7-56-8

\twolineshloka
{तमुवाच ततो ब्रह्मा स्वयम्भूरमितप्रभः}
{मित्रावरुणजं तेज प्रविश त्वं महायशः} %7-56-9

\twolineshloka
{अयोनिजस्त्वं भविता तत्रापि द्विजसत्तम}
{धर्मेण महता युक्तः पुनरेष्यसि मे वशम्} %7-56-10

\twolineshloka
{एवमुक्तस्तु देवेन चाभिवाद्य प्रदक्षिणम्}
{कृत्वा पितामहं तूर्णं प्रययौ वरुणालयम्} %7-56-11

\twolineshloka
{तमेव कालं मित्रोऽपि वरुणत्वमकारयत्}
{क्षीरोदेन सहोपेतः पूज्यमानः सुरोत्तमैः} %7-56-12

\twolineshloka
{एतस्मिन्नेव काले तु उर्वशी परमाप्सराः}
{यदृच्छया तमुद्देशमाययौ सखिभिर्वृता} %7-56-13

\twolineshloka
{तां दृष्ट्वा रूपसम्पन्नां क्रीडन्तीं वरुणालये}
{आविशत्परमो हर्षो वरुणं चोर्वशीकृते} %7-56-14

\twolineshloka
{स तां पद्मपलाशाक्षीं पूर्णचन्द्रनिभाननाम्}
{वरुणो वरयामास मैथुनायाप्सरोवराम्} %7-56-15

\twolineshloka
{प्रत्युवाच ततः सा तु वरुणं प्राञ्जलिः स्थिता}
{मित्रेणाहं वृता साक्षात्पूर्वमेव सुरेश्वर} %7-56-16

\twolineshloka
{वरुणस्त्वब्रवीद्वाक्यं कन्दर्पशरपीडितः}
{इदं तेजः समुत्स्रक्ष्ये कुम्भेऽस्मिन्देवनिर्मिते} %7-56-17

\twolineshloka
{एवमुत्सृज्य सुश्रोणि त्वय्यहं वरवर्णिनि}
{कृतकामो भविष्यामि यदि नेच्छसि सङ्गमम्} %7-56-18

\twolineshloka
{तस्य तल्लोकपालस्य वरुणस्य सुभाषितम्}
{उर्वशी परमप्रीता श्रुत्वा वाक्यमुवाच ह} %7-56-19

\twolineshloka
{काममेतद्भवत्वेवं हृदयं मे त्वयि स्थितम्}
{भावश्चाप्यधिकस्तुभ्यं देहो मित्रस्य तु प्रभो} %7-56-20

\twolineshloka
{उर्वश्या एवमुक्तस्तु रेतस्तन्महदद्भुतम्}
{ज्वलदग्निशिखाप्रख्यं तस्मिन्कुम्भे ह्यपासृजत्} %7-56-21

\twolineshloka
{उर्वशी त्वगमत्तत्र मित्रो वै यत्र देवता}
{तां तु मित्रः सुसङ्क्रम्य उर्वशीमिदमब्रवीत्} %7-56-22

\twolineshloka
{मया निमन्त्रिता पूर्वं कस्मात्त्वमवसर्जिता}
{पतिमन्यं वृतवती तस्मात्त्वं दुष्टचारिणी} %7-56-23

\twolineshloka
{अनेन दुष्कृतेन त्वं मत्क्रोधकलुषीकृता}
{मनुष्यलोकमास्थाय कञ्चित्कालं निवत्स्यसि} %7-56-24

\twolineshloka
{बुधस्य पुत्रो राजर्षिः काशीराजः पुरूरवाः}
{तमद्य गच्छ दुर्बुद्धे स ते भर्ता भविष्यति} %7-56-25

\twolineshloka
{ततः सा शापदोषेण पुरूरवसमभ्यगात्}
{प्रतियाते पुरूरवं बुधस्यात्मजमौरसम्} %7-56-26

\twolineshloka
{तस्य जज्ञे ततः श्रीमानायुः पुत्रो महाबलः}
{नहुषो यस्य पुत्रस्तु बभूवेन्द्रसमद्युतिः} %7-56-27

\twolineshloka
{वज्रमुत्सृज्य वृत्राय भ्रान्तेऽथ त्रिदिवेश्वरे}
{शतं वर्षसहस्राणि येनेन्द्रत्वं प्रशासितम्} %7-56-28

\twolineshloka
{सा तेन शापेन जगाम भूमिं तदोर्वशी चारुदती सुनेत्रा}
{बहूनि वर्षाण्यवसच्च सुभ्रूः शापक्षयादिन्द्रसदो ययौ च} %7-56-29


॥इत्यार्षे श्रीमद्रामायणे वाल्मीकीये आदिकाव्ये उत्तरकाण्डे मैत्रावरुणित्वप्राप्तिः नाम षट्पञ्चाशः सर्गः ॥७-५६॥
