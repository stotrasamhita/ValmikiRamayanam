\sect{पञ्चदशः सर्गः — पुष्पकहरणम्}

\twolineshloka
{ततस्ताँल्लक्ष्य वित्रस्तान्यक्षेन्द्रांश्च सहस्रशः}
{धनाध्यक्षो महायक्षं माणिचारमथाब्रवीत्} %7-15-1

\twolineshloka
{रावणं जहि यक्षेन्द्र दुर्वृत्तं पापचेतसम्}
{शरणं भव वीराणां यक्षाणां युद्धशालिनाम्} %7-15-2

\twolineshloka
{एवमुक्तो महाबाहुर्माणिभद्रः सुदुर्जयः}
{वृतो यक्षसहस्रैस्तु चतुर्भिः समयोधयत्} %7-15-3

\twolineshloka
{ते गदामुसलप्रासैः शक्तितोमरमुद्गरैः}
{अभिघ्नन्तस्तदा यक्षा राक्षसान्समुपाद्रवन्} %7-15-4

\twolineshloka
{कुर्वन्तस्तुमुलं युद्धं चरन्तः श्येनवल्लघु}
{बाढं प्रयच्छन्नेच्छामि दीयतामिति भाषिणः} %7-15-5

\twolineshloka
{ततो देवाः सगन्धर्वा ऋषयो ब्रह्मवादिनः}
{दृष्ट्वा तत्तुमुलं युद्धं परं विस्मयमागमन्} %7-15-6

\twolineshloka
{यक्षाणां तु प्रहस्तेन सहस्रं निहतं रणे}
{महोदरेण चानिन्द्यं सहस्रमपरं हतम्} %7-15-7

\twolineshloka
{क्रुद्धेन च तदा राजन्मारीचेन युयुत्सुना}
{निमेषान्तरमात्रेण द्वे सहस्रे निपातिते} %7-15-8

\twolineshloka
{क्व च यक्षार्जवं युद्धं क्व च मायाबलाश्रयम्}
{रक्षसां पुरुषव्याघ्र तेन तेऽभ्यधिका युधि} %7-15-9

\twolineshloka
{धूम्राक्षेण समागम्य माणिभद्रो महारणे}
{मुसलेनोरसि क्रोधात्तडितो न च कम्पितः} %7-15-10

\twolineshloka
{ततो गदां समाविध्य माणिभद्रेण राक्षसः}
{धूम्राक्षस्ताडितो मूर्ध्नि विह्वलः स पपात ह} %7-15-11

\twolineshloka
{धूम्राक्षं ताडितं दृष्ट्वा पतितं शोणितोक्षितम्}
{अभ्यधावत सङ्ग्रामे माणिभद्रं दशाननः} %7-15-12

\twolineshloka
{तं क्रुद्धमभिधावन्तं माणिभद्रो दशाननम्}
{शक्तिभिस्ताडयामास तिसृभिर्यक्षपुङ्गवः} %7-15-13

\twolineshloka
{ताडितो माणिभद्रस्य मुकुटे प्राहरद्रणे}
{तस्य तेन प्रहारेम मुकुटं पार्श्वमागतम्} %7-15-14

\threelineshloka
{ततः संयुध्यमानेन विष्टब्धो न व्यकम्पत}
{तदाप्रभृति यक्षोऽसौ पार्श्वमौलिरिति स्मृतः}
{सन्नादः सुमहान्राजंस्तस्मिन्शैलेऽभ्यवर्तत} %7-15-15

\twolineshloka
{ततो दूरात्प्रददृशे धनाध्यक्षो गदाधरः}
{शुक्रप्रौष्ठपदाभ्यां च पद्मशङ्खसमावृतः} %7-15-16

\twolineshloka
{स दृष्ट्वा भ्रातरं सङ्ख्ये शापाद्विभ्रष्टगौरवम्}
{उवाच वचनं धीमान्युक्तं पैतामहे कुले} %7-15-17

\twolineshloka
{यन्मया वार्यमाणस्त्वं नावगच्छसि दुर्मते}
{पश्चादस्य फलं प्राप्य ज्ञास्यसे निरयं गतः} %7-15-18

\twolineshloka
{यो हि मोहाद्विषं पीत्वा नावगच्छति दुर्मतिः}
{स तस्य परिणामान्ते जानीते कर्मणः फलम्} %7-15-19

\twolineshloka
{देवता नाभिनन्दन्ति धर्मयुक्तेन केनचित्}
{येन त्वमीदृशं भावं नीतस्सन्नावबुद्ध्यसे} %7-15-20

\twolineshloka
{मातरं पितरं यो हि आचार्यं चावमन्यते}
{स पश्यति फलं तस्य प्रेतराजवशं गतः} %7-15-21

\twolineshloka
{अध्रुवे हि शरीरे यो न करोति तपोर्जनम्}
{स पश्चात्तप्यते मूढो मृतो दृष्ट्वाऽऽत्मनो गतिम्} %7-15-22

\twolineshloka
{धर्माद्राज्यं धनं सौख्यमधर्माद्दुःखमेव च}
{तस्माद्धर्मं सुखार्थाय कुर्यात्पापं विसर्जयेत्} %7-15-23

\twolineshloka
{पापस्य हि फलं दुःखं तद्भोक्तव्यमिहात्मना}
{तस्मादात्मापघातार्थं मूढः पापं करिष्यति} %7-15-24

\threelineshloka
{कस्य चिन्न हि दुर्बुद्धेश्छन्दतो जायते मतिः}
{देवं चेष्टयते सर्वं हतो दैवेन हन्यते}
{यादृशं कुरुते कर्म तादृशं फलमश्नुते} %7-15-25

\twolineshloka
{बुद्धिरूपं फलं पुत्राञ्छौर्यं धीरत्वमेव च}
{प्राप्नुवन्ति नरा लोके निर्जितं पुण्यकर्मभिः} %7-15-26

\twolineshloka
{एवं निरयगामी त्वं यस्य ते मतिरीदृशी}
{न त्वां समभिभाषिष्येऽसद्वृत्तेष्वेव निर्णयः} %7-15-27

\twolineshloka
{एवमुक्तास्ततस्तेन तस्यामात्याः समाहताः}
{मारीचप्रमुखाः सर्वे विमुखा विप्रदुद्रुवुः} %7-15-28

\twolineshloka
{ततस्तेन दशग्रीवो यक्षेन्द्रेण महात्मना}
{गदयाभिहतो मूर्ध्नि न च स्थानात्प्रकम्पितः} %7-15-29

\twolineshloka
{ततस्तौ राम निघ्नन्तौ तदान्योन्यं महामृधे}
{न विह्वलौ न च श्रान्तौ बभूवतुरमर्षणौ} %7-15-30

\twolineshloka
{आग्नेयमस्त्रं तस्मै स मुमोच धनदस्तदा}
{राक्षसेन्द्रो वारुणेन तदस्त्रं प्रत्यवारयत्} %7-15-31

\twolineshloka
{ततो मायां प्रविष्टोऽसौ राक्षसीं राक्षसेश्वरः}
{रूपाणां शतसाहस्रं विनाशाय चकार च} %7-15-32

\twolineshloka
{व्याघ्रो वराहो जीमूतः पर्वतः सागरो द्रुमः}
{यक्षो दैत्यस्वरूपी च सोऽदृश्यत दशाननः} %7-15-33

\twolineshloka
{बहूनि च करोति स्म दृश्यन्ते न त्वसौ ततः}
{प्रतिगृह्य ततो राम महदस्त्रं दशाननः} %7-15-34

\twolineshloka
{जघान मूर्ध्नि धनदं व्याविद्ध्य महतीं गदाम्}
{एवं स तेनाभिहतो विह्वलः शोणितोक्षितः} %7-15-35

\onelineshloka
{कृत्तमूल इवाशोको निपपात धनाधिपः} %7-15-36

\twolineshloka
{ततः पद्मादिभिस्तत्र निधिभिः स तदा वृतः}
{धनदोच्छ्वासितस्तैस्तु वनमानीय नन्दनम्} %7-15-37

\twolineshloka
{निर्जित्य राक्षसेन्द्रस्तं धनदं हृष्टमानसः}
{पुष्पकं तस्य जग्राह विमानं जयलक्षणम्} %7-15-38

\twolineshloka
{काञ्चनस्तम्भसंवीतं वैडूर्यमणितोरणम्}
{मुक्ताजालप्रतिच्छन्नं सर्वकामफलद्रुमम्} %7-15-39

\twolineshloka
{मनोजवं कामगमं कामरूपं विहङ्गमम्}
{मणिकाञ्चनसोपानं तप्तकाञ्चनवेदिकम्} %7-15-40

\twolineshloka
{देवोपवाह्यमक्षय्यं सदादृष्टिमनःसुखम्}
{बह्वाश्चर्यं भक्तिचित्रं ब्रह्मणा परिनिर्मितम्} %7-15-41

\twolineshloka
{निर्मितं सर्वकामैस्तु मनोहरमनुत्तमम्}
{न तु शीतं न चोष्णं च सर्वर्तुसुखदं शुभम्} %7-15-42

\twolineshloka
{स तं राजा समारुह्य कामगं वीर्यनिर्जितम्}
{जितं त्रिभुवनं मेने दर्पोत्सेकात्सुदुर्मतिः} %7-15-43

\onelineshloka
{जित्वा वैश्रवणं देवं कैलासात्समवातरत्} %7-15-44

\twolineshloka
{स्वतेजसा विपुलमवाप्य तं जयं प्रतापवान्विमलकिरीटहारवान्}
{रराज वै परमविमानमास्थितो निशाचरः सदसि गतो यथानलः} %7-15-45


॥इत्यार्षे श्रीमद्रामायणे वाल्मीकीये आदिकाव्ये उत्तरकाण्डे पुष्पकहरणम् नाम पञ्चदशः सर्गः ॥७-१५॥
