\sect{एकादशाधिकशततमः सर्गः — श्रीमद्रामायणफलश्रुतिः}

\twolineshloka
{एतावदेतदाख्यानं सोत्तरं ब्रह्मपूजितम्}
{रामायणमिति ख्यातं मुख्यं वाल्मीकीना कृतम्} %7-111-1

\twolineshloka
{ततः प्रतिष्ठितो विष्णुः स्वर्गलोके यथा पुरम्}
{येन व्याप्तमिदं सर्वं त्रैलोक्यं सचराचरम्} %7-111-2

\twolineshloka
{ततो देवाः सगन्धर्वाः सिद्धाश्च परमर्षयः}
{नित्यं शृण्वन्ति सन्तुष्टाः दिव्यं रामायणं दिवि} %7-111-3

\twolineshloka
{इदमाख्यानमायुष्यं सौभाग्यं पापनाशनम्}
{रामायणं वेदसमं श्राद्धेषु श्रावयेद्बुधः} %7-111-4

\twolineshloka
{अपुत्रो लभते पुत्रमधनो लभते धनम्}
{सर्वपापैः प्रमुच्येत पदमप्यस्य यः पठेत्} %7-111-5

\twolineshloka
{पापान्यपि च यः कुर्यादहन्यहनि मानवः}
{पठत्येकमपि श्लोकं पापात्स परिमुच्यते} %7-111-6

\twolineshloka
{वाचकाय च दातव्यं वस्त्रं धेनुं हिरण्यकम्}
{वाचके परितुष्टे तु तुष्टाः स्युः सर्वदेवताः} %7-111-7

\twolineshloka
{एतदाख्यानमायुष्यं पठन्रामायणं नरः}
{सपुत्रपौत्रो लोकेऽस्मिन्प्रेत्य चेह महीयते} %7-111-8

\twolineshloka
{अयोध्याऽपि पुरी रम्या शून्या वर्षगणान्बहून्}
{ऋषभं प्राप्य राजानं निवासमुपयास्यति} %7-111-9

\twolineshloka
{एतदाख्यानमायुष्यं सभविष्यं सहोत्तरम्}
{कृतवान्प्रचेतसः पुत्रस्तद्ब्रह्माप्यन्वमन्यत} %7-111-10

\twolineshloka
{अश्वमेधसहस्रस्य वाजपेयायुतस्य च}
{लभते श्रावणादेव सर्गस्यैकस्य मानवः} %7-111-11

\threelineshloka
{प्रयागादीनि तीर्थानि गङ्गाद्याः सरितस्तथा}
{नैमिशादीन्यरण्यानि कुरुक्षेत्रादिकान्यपि}
{गतानि तेन लोकेऽस्मिन्येन रामायणं श्रुतम्} %7-111-12

\twolineshloka
{हेमभारं कुरुक्षेत्रे ग्रस्ते भानौ प्रयच्छति}
{यश्च रामायणं लोके शृणोति सदृशावुभौ} %7-111-13

\twolineshloka
{सम्यक्छ्रद्धासमायुक्तः शृणुते राघवीं कथाम्}
{सर्वपापात्प्रमुच्येत विष्णुलोकं स गच्छति} %7-111-14

\twolineshloka
{आदिकाव्यमिदं त्वार्षं पुरा वाल्मीकिना कृतम्}
{यः शृणोति सदा भक्त्या स गच्छेद्वैष्णवीं तनुम्} %7-111-15

\twolineshloka
{पुत्रदाराश्च वर्धन्ते सम्पदः सन्ततिस्तथा}
{सत्यमेतद्विदित्वा तु श्रोतव्यं नियतात्मभिः} %7-111-16

\onelineshloka
{गायत्र्याश्च स्वरूपं तद्रामायणमनुत्तमम्} %7-111-17

\twolineshloka
{अपुत्रो लभते पुत्रमधनो लभते धनम्}
{सर्वपापैः प्रमुच्येत पदमप्यस्य यः पठेत्} %7-111-18

\twolineshloka
{यः पठेच्छृणुयान्नित्यं चरितं राघवस्य ह}
{भक्त्या निष्कल्मषो भूत्वा दीर्घमायुरवाप्नुयात्} %7-111-19

\twolineshloka
{चिन्तयेद्राघवं नित्यं श्रेयः प्राप्तुं य इच्छति}
{श्रावयेदिदमाख्यानं ब्राह्मणेभ्यो दिने दिने} %7-111-20

\twolineshloka
{यस्त्विदं रघुनाथस्य चरितं सकलं पठेत्}
{सोऽसुक्षये विष्णुलोकं गच्छत्येव न संशयः} %7-111-21

\twolineshloka
{पिता पितामहस्तस्य तथैव प्रपितामहः}
{तत्पिता तत्पिता चैव विष्णुं यान्ति न संशयः} %7-111-22

\twolineshloka
{चतुर्वर्गप्रदं नित्यं चरितं राघवस्य तु}
{तस्माद्यत्नवता नित्यं श्रोतव्यं परमं सदा} %7-111-23

\twolineshloka
{शृण्वन्रामायणं भक्त्या यः पादं पदमेव वा}
{स याति ब्रह्मणः स्थानं ब्रह्मणा पूज्यते सदा} %7-111-24

\twolineshloka
{एवमेतत्पुरावृत्तमाख्यानं भद्रमस्तु वः}
{प्रव्याहरत विस्रब्धं बलं विष्णोः प्रवर्धताम्} %7-111-25


॥इत्यार्षे श्रीमद्रामायणे वाल्मीकीये आदिकाव्ये उत्तरकाण्डे श्रीमद्रामायणफलश्रुतिः नाम एकादशाधिकशततमः सर्गः ॥७-१११॥
