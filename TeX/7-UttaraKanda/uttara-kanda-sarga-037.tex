\sect{सप्तत्रिंशः सर्गः — पौरोपस्थानम्
वालिसुग्रीवोत्पत्तिः
नारायणहतगतिकथनम्
रामावतारकथनम्
सीतारामकथाश्रवणफलम्
श्वेतद्वीपवासप्राप्त्युपायकथनम्}

\twolineshloka
{अभिषिक्ते तु काकुत्स्थे धर्मेण विदितात्मनि}
{व्यतीता या निशा पूर्वा पौराणां हर्षवर्धिनी} %7-37-1

\twolineshloka
{तस्यां रजन्यां व्युष्टायां प्रातर्नृपतिबोधकाः}
{वन्दिनः समुपातिष्ठन्सौम्या नृपतिवेश्मनि} %7-37-2

\twolineshloka
{ते रक्तकण्ठिनः सर्वे किन्नरा इव शिक्षिताः}
{तुष्टुवुर्नपतिं वीरं यथावत् सम्प्रहर्षिणः} %7-37-3

\twolineshloka
{वीर सौम्य प्रबुध्यस्व कौसल्याप्रीतिवर्धन}
{जगद्धि सर्वं स्वपिति त्वयि सुप्ते नराधिप} %7-37-4

\twolineshloka
{विक्रमस्ते यथा विष्णो रूपं चैवाश्विनोरिव}
{बुद्ध्या बृहस्पतेस्तुल्यः प्रजापतिसमो ह्यसि} %7-37-5

\twolineshloka
{क्षमा ते पृथिवीतुल्या तेजसा भास्करोपमः}
{वेगस्ते वायुना तुल्यो गाम्भीर्यमुदधेरिव} %7-37-6

\twolineshloka
{अप्रकम्प्यो यता स्थाणुश्चन्द्रे सौम्यत्वमीदृशम्}
{नेदृशाः पार्थिवाः पूर्वं भवितारो नराधिप} %7-37-7

\twolineshloka
{यथा त्वमतिदुर्धर्षो धर्मनित्यः प्रजाहितः}
{न त्वां जहाति कीर्तिश्च लक्ष्मीश्च पुरुषर्षभ} %7-37-8

\twolineshloka
{श्रीश्च धर्मश्च काकुत्स्थ त्वयि नित्यं प्रतिष्ठितौ}
{एताश्चान्याश्च मधुरा वन्दिभिः परिकीर्तिताः} %7-37-9

\twolineshloka
{सूताश्च संस्तवैर्दिव्यैर्बोधयन्ति स्म राघवम्}
{स्तुतिभिः स्तूयमानाभिः प्रत्यबुध्यत राघवः} %7-37-10

\twolineshloka
{स तद्विहाय शयनं पाण्डराच्छादनास्तृतम्}
{उत्तस्थौ नागशयनाद्धरिर्नारायणो यथा} %7-37-11

\twolineshloka
{तमुत्थितं महात्मानं प्रह्वाः प्राञ्जलयो नराः}
{सलिलं भाजनैः शुभ्रैरुपतस्थुः सहस्रशः} %7-37-12

\twolineshloka
{कृतोदकशुचिर्भूत्वा काले हुतहुताशनः}
{देवागारं जगामाशु पुण्यमिक्ष्वाकुसेवितम्} %7-37-13

\twolineshloka
{तत्र देवान्पितऽन्विप्रानर्चयित्वा यथाविधि}
{बाह्यकक्षान्तरं रामो निर्जगाम जनैर्वृतः} %7-37-14

\twolineshloka
{उपतस्थुर्महात्मानो मन्त्रिणः सपुरोहिताः}
{वसिष्ठप्रमुखाः सर्वे दीप्यमाना इवाग्नयः} %7-37-15

\twolineshloka
{क्षत्रियाश्च महात्मानो नानाजनपदेश्वराः}
{रामस्योपाविशन्पार्श्वे शक्रस्येव यथाऽमराः} %7-37-16

\twolineshloka
{भरतो लक्ष्मणश्चात्र शत्रुघ्नश्च महायशाः}
{उपासाञ्चक्रिरे हृष्टा वेदास्त्रय इवाध्वरम्} %7-37-17

\twolineshloka
{याताः प्राञ्जलयो भूत्वा किङ्करा मुदिताननाः}
{मुदिता नाम पार्श्वस्था बहवः समुपाविशन्} %7-37-18

\twolineshloka
{वानराश्च महावीर्या विंशतिः कामरूपिणः}
{सुग्रीवप्रमुखा राममुपासन्ते महौजसः} %7-37-19

\twolineshloka
{विभीषणश्च रक्षोभिश्चतुर्भिः परिवारितः}
{उपासते महात्मानं धनेशमिव गुह्यकाः} %7-37-20

\twolineshloka
{तथा निगमवृद्धाश्च कुलीना ये च मानवाः}
{शिरसाऽऽवन्द्य राजानमुपासन्ते विचक्षणाः} %7-37-21

\twolineshloka
{तथा परिवृतो राजा श्रीमद्भिर्ऋषिभिर्वृतः}
{राजभिश्च महावीर्यैर्वानरैश्च सराक्षसैः} %7-37-22

\twolineshloka
{यथा देवेश्वरो नित्यमृषिभिः समुपास्यते}
{अधिकस्तेन रूपेण सहस्राक्षाद्विरोचते} %7-37-23

\twolineshloka
{तेषां समुपविष्टानां तास्ताः सुमधुराः कथाः}
{कथ्यन्ते धर्मसंयुक्ताः पुराणज्ञैर्महात्मभिः} %7-37-24


॥इत्यार्षे श्रीमद्रामायणे वाल्मीकीये आदिकाव्ये उत्तरकाण्डे पौरोपस्थानम् नाम सप्तत्रिंशः सर्गः ॥७-३७॥
