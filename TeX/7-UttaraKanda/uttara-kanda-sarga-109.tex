\sect{नवाधिकशततमः सर्गः — श्रीराममहाप्रस्थानम्}

\twolineshloka
{प्रभातायां तु शर्वर्यां पृथुवक्षा महायशाः}
{रामः कमलपत्राक्षः पुरोधसमथाब्रवीत्} %7-109-1

\twolineshloka
{अग्निहोत्रं व्रजत्वग्रे दीप्यमानं सह द्विजैः}
{वाजपेयातपत्रं च शोभमानं महापथे} %7-109-2

\twolineshloka
{ततो वसिष्ठस्तेजस्वी सर्वं निरवशेषतः}
{चकार विधिवद्धर्मं महाप्रास्थानिकं विधिम्} %7-109-3

\twolineshloka
{ततः सूक्ष्माम्बरधरो ब्रह्ममावर्तयन्परम्}
{कुशान्गृहीत्वा पाणिभ्यां प्रसज्य प्रययावथ} %7-109-4

\twolineshloka
{अव्याहरन्क्वचित्किञ्चिन्निश्चेष्टो निःसुखः पथि}
{निर्जगाम गृहात्तस्माद्दीप्यमान इवांशुमान्} %7-109-5

\twolineshloka
{रामस्य दक्षिणे पार्श्वे सपद्मा श्रीरपाश्रिता}
{सव्ये तु ह्रीर्महादेवी व्यवसायस्तथाग्रतः} %7-109-6

\twolineshloka
{शरा नानाविधाश्चापि धनुरायतमुत्तमम्}
{तथाऽऽयुधानि ते सर्वे ययुः पुरुषविग्रहाः} %7-109-7

\twolineshloka
{वेदा ब्राह्मणरूपेण गायत्री सर्वरक्षिणी}
{ओङ्कारोऽथ वषट्कारः सर्वे राममनुव्रताः} %7-109-8

\twolineshloka
{ऋषयश्च महात्मानः सर्व एव महीसुराः}
{अन्वगच्छन्महात्मानं स्वर्गद्वारमपावृतम्} %7-109-9

\twolineshloka
{तं यान्तमनुगच्छन्ति ह्यन्तःपुरचराः स्त्रियः}
{सवृद्धबालदासीकाः सवर्षवरकिङ्कराः} %7-109-10

\twolineshloka
{सान्तःपुरश्च भरतः शत्रुघ्नसहितो ययौ}
{रामं गतिमुपागम्य साग्निहोत्रमनुव्रतः} %7-109-11

\twolineshloka
{ते च सर्वे महात्मानः साग्निहोत्राः समागताः}
{सपुत्रदाराः काकुत्स्थमनुजग्मुर्महामतिम्} %7-109-12

\twolineshloka
{मन्त्रिणो भृत्यवर्गाश्च सपुत्रपशुबान्धवाः}
{सर्वे सहानुगा राममन्वगच्छन्प्रहृष्टवत्} %7-109-13

\twolineshloka
{ततः सर्वाः प्रकृतयो हृष्टपुष्टजनावृताः}
{गच्छन्तमन्वगछंस्तं राघवं गुणरञ्जिताः} %7-109-14

\twolineshloka
{ततः सस्त्रीपुमांसस्ते सपक्षिपशुवाहनाः}
{राघवस्यानुगाः सर्वे हृष्टा विगतकल्मषाः} %7-109-15

\twolineshloka
{स्नाताः प्रमुदिताः सर्वे हृष्टाः पुष्टाश्च वानराः}
{दृढं किलकिलाशब्दैः सर्वं राममनुव्रतम्} %7-109-16

\twolineshloka
{न तत्र कश्चिद्दीनो वा व्रीडितो वाऽपि दुःखितः}
{हृष्टं समुदितं सर्वं बभूव परमाद्भुतम्} %7-109-17

\twolineshloka
{द्रष्टुकामोऽथ निर्यान्तं रामं जानपदो जनः}
{यः प्राप्तः सोऽपि दृष्ट्वैव स्वर्गायानुगतो मुदा} %7-109-18

\twolineshloka
{ऋक्षवानररक्षांसि जनाश्च पुरवासिनः}
{आगच्छन्परया भक्त्या पृष्ठतः सुसमाहिताः} %7-109-19

\twolineshloka
{यानि भूतानि नगरेऽप्यन्तर्धानगतानि च}
{राघवं तान्यनुययुः स्वर्गाय समुपस्थितम्} %7-109-20

\twolineshloka
{यानि पश्यन्ति काकुत्स्थं स्थावराणि चराणि च}
{सर्वाणि रामगमने ह्यनुजग्मुर्हि तान्यपि} %7-109-21

\twolineshloka
{नोच्छ्वसत्तदयोध्यायां सुसूक्ष्ममपि दृश्यते}
{तिर्यग्योनिगताश्चापि सर्वे राममनुव्रताः} %7-109-22


॥इत्यार्षे श्रीमद्रामायणे वाल्मीकीये आदिकाव्ये उत्तरकाण्डे श्रीराममहाप्रस्थानम् नाम नवाधिकशततमः सर्गः ॥७-१०९॥
