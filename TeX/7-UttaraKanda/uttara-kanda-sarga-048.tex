\sect{अष्टचत्वारिंशः सर्गः — सीतापरित्यागः}

\twolineshloka
{लक्ष्मणस्य वचः श्रुत्वा दारुणं जनकात्मजा}
{परं विषादमागम्य वैदेही निपपात ह} %7-48-1

\twolineshloka
{सा मुहूर्तमिवासञ्ज्ञा बाष्पपर्याकुलेक्षणा}
{लक्ष्मणं दीनया वाचा उवाच जनकात्मजा} %7-48-2

\twolineshloka
{मामिकेयं तनुर्नूनं सृष्टा दुःखाय लक्ष्मण}
{धात्रा यस्यास्तथा मेऽद्य दुःखमूर्तिः प्रदृश्यते} %7-48-3

\twolineshloka
{किं नु पापं कृतं पूर्वं को वा दारैर्वियोजितः}
{याऽहं शुद्धसमाचारा त्यक्ता नृपतिना सती} %7-48-4

\twolineshloka
{पुराऽहमाश्रमे वासं रामपादानुवर्तिनी}
{अनुरुध्यापि सौमित्रे दुःखे च परिवर्तिनी} %7-48-5

\twolineshloka
{सा कथं ह्याश्रमे सौम्य वत्स्यामि विजनीकृता}
{आख्यास्यामि च कस्याहं दुःखं दुःखपरायणा} %7-48-6

\twolineshloka
{किं नु वक्ष्यामि मुनिषु कर्म वा सत्कृतं च किम्}
{कस्मिंश्चित् कारणे त्यक्ता राघवेण महात्मना} %7-48-7

\twolineshloka
{न खल्वद्यैव सौमित्रे जीवितं जाह्नवीजले}
{त्यजेयं राजवंशस्तु भर्तुर्मे परिहास्यते} %7-48-8

\twolineshloka
{यथाज्ञं कुरु सोमित्रे त्यज मां दुःखभागिनीम्}
{निदेशे स्थीयतां राज्ञः शृणु चेदं वचो मम} %7-48-9

\twolineshloka
{श्वश्रूणामविशेषेण प्राञ्जलिप्रग्रहेण च}
{शिरसाऽऽवन्द्य चरणौ कुशलं ब्रूहि पार्थिवम्} %7-48-10

\twolineshloka
{शिरसाऽभिनतो ब्रूयाः सर्वासामेव लक्ष्मण}
{वक्तव्यश्चापि नृपतिर्धर्मेषु सुसमाहितः} %7-48-11

\twolineshloka
{जानासि च यथा शुद्धा सीता तत्त्वेन राघव}
{भक्त्या च परया युक्ता हिता च तव नित्यशः} %7-48-12

\threelineshloka
{अहं त्यक्ता त्वया वीर अयशोभीरुणा जने}
{यच्च ते वचनीयं स्यादपवादसमुत्थितम्}
{मया च परिहर्तव्यं त्वं हि मे परमा गतिः} %7-48-13

\twolineshloka
{वक्तव्यश्चेति नृपतिर्धमेण सुसमाहितः}
{यथा भ्रातृषु वर्तेथास्तथा पौरेषु नित्यदा} %7-48-14

\onelineshloka
{परमो ह्येष धर्मस्ते तस्मात्कीर्तिरनुत्तमा} %7-48-15

\twolineshloka
{यत्तु पौरजनो राजन्धर्मेण समवाप्नुयात्}
{अहं तु नानुशोचामि स्वशरीरं नरर्षभ} %7-48-16

\twolineshloka
{यथापवादं पौराणां तथैव रघुनन्दन}
{पतिर्हि देवता नार्याः पतिर्बन्धुः पतिर्गुरुः} %7-48-17

\threelineshloka
{प्राणैरपि प्रियं तस्माद्भर्तुः कार्यं विशेषतः}
{इति मद्वचनाद्रामो वक्तव्यो मम सङ्ग्रहः}
{निरीक्ष्य माऽद्य गच्छ त्वमृतुकालातिवर्तिनीम्} %7-48-18

\twolineshloka
{एवं ब्रुवन्त्यां सीतायां लक्ष्मणो दीनचेतनः}
{शिरसाऽऽवन्द्य धरणीं व्याहर्तुं न शशाक ह} %7-48-19

\twolineshloka
{प्रदक्षिणं च तां कृत्वा रुदन्नेव महास्वनः}
{ध्यात्वा मुहूर्तं तामाह किं मां वक्ष्यसि शोभने} %7-48-20

\twolineshloka
{दृष्टपूर्वं न ते रूपं पादौ दृष्टौ तवानघे}
{कथमत्र हि पश्यामि रामेण रहितां वने} %7-48-21

\twolineshloka
{इत्युक्त्वा तां नमस्कृत्य पुनर्नावमुपारुहत्}
{आरुह्य च पुनर्नावं नाविकं चाभ्यचोदयत्} %7-48-22

\twolineshloka
{स गत्वा चोत्तरं तीरं शोकभारसमन्वितः}
{सम्मूढ इव दुःखेन रथमध्यारुहद्द्रुतम्} %7-48-23

\twolineshloka
{मुहुर्मुहुः परावृत्य दृष्ट्वा सीतामनाथवत्}
{चेष्टन्तीं परतीरस्थां लक्ष्मणः प्रययावथ} %7-48-24

\twolineshloka
{दूरस्थं रथमालोक्य लक्ष्मणं च मुहूर्मुहुः}
{निरीक्षमाणां तूद्विग्नां सीतां शोकः समाविशत्} %7-48-25

\twolineshloka
{सा दुःखभारावनता यशस्विनी यशोधना नाथमपश्यती सती}
{रुरोद सा बर्हिणनादिते वने महास्वनं दुःखपरायणा सती} %7-48-26


॥इत्यार्षे श्रीमद्रामायणे वाल्मीकीये आदिकाव्ये उत्तरकाण्डे सीतापरित्यागः नाम अष्टचत्वारिंशः सर्गः ॥७-४८॥
