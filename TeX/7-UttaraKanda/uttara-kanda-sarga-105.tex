\sect{पञ्चाधिकशततमः सर्गः — दुर्वासोऽभिगमः}

\twolineshloka
{तथा तयोः संवदतोर्दुर्वासा भगवानृषिः}
{रामस्य दर्शनाकाङ्क्षी राजद्वारमुपागमत्} %7-105-1

\twolineshloka
{सोऽभिगम्य तु सौमित्रिमुवाच ऋषिसत्तमः}
{रामं दर्शय मे शीघ्रं पुरा मेऽर्थोऽतिवर्तते} %7-105-2

\twolineshloka
{मुनेस्तु भाषितं श्रुत्वा लक्ष्मणः परवीरहा}
{अभिवाद्य महात्मानं वाक्यमेतदुवाच ह} %7-105-3

\twolineshloka
{किं कार्यं ब्रूहि भगवन्को वाऽर्थः किं करोम्यहम्}
{व्यग्रो हि राघवो ब्रह्मन्मुहूर्तं प्रतिपाल्यताम्} %7-105-4

\twolineshloka
{तच्छ्रुत्वा ऋषिशार्दूलः क्रोधेन कलुषीकृतः}
{उवाच लक्ष्मणं वाक्यं निर्दहन्निव चक्षुषा} %7-105-5

\twolineshloka
{अस्मिन्क्षणे मां सौमित्रे रामाय प्रतिवेदय}
{अस्मिन्क्षणे मां सौमित्रे न निवेदयसे यदि} %7-105-6

\twolineshloka
{विषयं त्वां पुरं चैव शपिष्ये राघवं तथा}
{भरतं चैव सौमित्रे युष्माकं या च सन्ततिः} %7-105-7

\threelineshloka
{न हि शक्ष्याम्यहं भूयो मन्युं धारयितुं हृदि}
{तच्छ्रुत्वा घोरसङ्काशं वाक्यं तस्य महात्मनः}
{चिन्तयामास मनसा तस्य वाक्यस्य निश्चयम्} %7-105-8

\twolineshloka
{एकस्य मरणं मेऽस्तु मा भूत्सर्वविनाशनम्}
{इति बुद्ध्या विनिश्चित्य राघवाय न्यवेदयत्} %7-105-9

\twolineshloka
{लक्ष्मणस्य वचः श्रुत्वा रामः कालं विसृज्य च}
{निस्सृत्य त्वरितं राजा अत्रेः पुत्रं ददर्श ह} %7-105-10

\twolineshloka
{सोऽभिवाद्य महात्मानं ज्वलन्तमिव तेजसा}
{किं कार्यमिति काकुत्स्थः कृताञ्जलिरभाषत} %7-105-11

\twolineshloka
{तद्वाक्यं राघवेणोक्तं श्रुत्वा मुनिवरः प्रभुम्}
{प्रत्याह रामं दुर्वासाः श्रूयतां धर्मवत्सल} %7-105-12

\twolineshloka
{अद्य वर्षसहस्रस्य समाप्तिस्तपसो मम}
{सोऽहं भोजनमिच्छामि यथासिद्धं तवानघ} %7-105-13

\twolineshloka
{तच्छ्रुत्वा वचनं राजा राघवः प्रीतमानसः}
{भोजनं मुनिमुख्याय यथासिद्धमुपाहरत्} %7-105-14

\twolineshloka
{स तु भुक्त्वा मुनिश्रेष्ठस्तदन्नममृतोपमम्}
{साधु रामेति सम्भाष्य स्वमाश्रममुपागमत्} %7-105-15

\twolineshloka
{तस्मिन् गते मुनिवरे स्वाश्रमं लक्ष्मणाग्रजः}
{संस्मृत्य कालवाक्यानि ततो दुःखमुपागमत्} %7-105-16

\twolineshloka
{दुःखेन च सुसन्तप्तः स्मृत्वा तद्घोरदर्शनम्}
{अवाङ्मुखो दीनमना व्याहर्तुं न शशाक ह} %7-105-17

\twolineshloka
{ततो बुद्ध्या विनिश्चित्य कालवाक्यानि राघवः}
{नैतदस्तीति निश्चित्य तूष्णीमासीन्महायशाः} %7-105-18


॥इत्यार्षे श्रीमद्रामायणे वाल्मीकीये आदिकाव्ये उत्तरकाण्डे दुर्वासोऽभिगमः नाम पञ्चाधिकशततमः सर्गः ॥७-१०५॥
