\sect{चतुर्विंशः सर्गः — खरशूर्पनखादण्डकानिवासादेशः}

\twolineshloka
{निवर्तमानः संहृष्टो रावणस्सुदुरात्मवान्}
{जह्रे पथि नरेन्द्रर्षिदेवगन्धर्वकन्यकाः} %7-24-1

\twolineshloka
{दर्शनीयां हि रक्षः स कन्यां स्त्रीं वापि पश्यति}
{हत्वा बन्धुजनं तस्या विमाने तां रुरोध ह} %7-24-2

\twolineshloka
{एवं पन्नगकन्याश्च राक्षसासुरमानुषीः}
{यक्षदानवकन्याश्च विमाने सोऽध्यरोपयत्} %7-24-3

\twolineshloka
{ताश्च सर्वाः समं दुःखान्मुमुचुर्बाष्पजं जलम्}
{तुल्यमग्न्यर्चिषां तत्र शोकाग्निभयसम्भवम्} %7-24-4

\twolineshloka
{ताभिः सर्वानवद्याभिर्नदीभिरिव सागरः}
{आपूरितं विमानं तद्भयशोकाशिवाश्रुभिः} %7-24-5

\twolineshloka
{नागगन्धर्वकन्याश्च महर्षितनयाश्च याः}
{दैत्यदानवकन्याश्च विमाने शतशोऽरुदन्} %7-24-6

\twolineshloka
{दीर्घकेश्यः सुचार्वङ्ग्यः पूर्णचन्द्रनिभाननाः}
{पीनस्तन्यस्तथा वज्रवेदिमध्यसमप्रभाः} %7-24-7

\twolineshloka
{रथकूबरसङ्काशैः श्रोणिदेशैर्मनोहराः}
{स्त्रियः सुराङ्गनाप्रख्या निष्टप्तकनकप्रभाः} %7-24-8

\twolineshloka
{शोकदुःखभयत्रस्ता विह्वलाश्च सुमध्यमाः}
{तासां निश्वासवातेन सर्वतः सम्प्रदीपितम्} %7-24-9

\twolineshloka
{अग्निहोत्रमिवाभाति सन्निरुद्धाग्निपुष्पकम्}
{दशग्रीववशं प्राप्तास्तास्तु शोकाकुलाः स्त्रियः} %7-24-10

\twolineshloka
{दीनवक्रेक्षणाः श्यामा मृग्यः सिंहवशा इव}
{काचिच्चिन्तयती तत्र किं नु मां भक्षयिष्यति} %7-24-11

\threelineshloka
{काचिद्दध्यौ सुदुःखार्ता अपि मां मारयेदयम्}
{इति मातृपितऽन्स्मृत्वा भर्तऽन्भ्रातऽंस्तथैव च}
{दुःखशोकसमाविष्टा विलेपुः सहिताः स्त्रियः} %7-24-12

\twolineshloka
{कथं नु खलु मे पुत्रो भविष्यति मया विना}
{कथं माता कथं भ्राता निमग्नाः शोकसागरे} %7-24-13

\twolineshloka
{हा कथं नु करिष्यामि भर्तुस्तस्मादहं विना}
{मृत्यो प्रसादयामि त्वां नय मां दुःखभागिनीम्} %7-24-14

\onelineshloka
{किन्नु तद्दुष्कृतं कर्म पुरा देहान्तरे कृतम्} %7-24-15

\twolineshloka
{एवं स्म दुःखिताः सर्वाः पतिताः शोकसागरे}
{न खल्विदानीं पश्यामो दुःखस्यास्यान्तमात्मनः} %7-24-16

\twolineshloka
{अहो धिङ्मानुषं लोकं नास्ति खल्वधमः परः}
{यद्दुर्बला बलवता भर्तारो रावणेन नः} %7-24-17

\twolineshloka
{सूर्येणोदयता काले नक्षत्राणीव नाशिताः}
{अहो सुबलवद्रक्षो वधोपायेषु युज्यते} %7-24-18

\twolineshloka
{अहो दुर्वृत्तमास्थाय नात्मानं वैजुगुप्सते}
{सर्वथा सदृशस्तावद्विक्रमोऽस्य दुरात्मनः} %7-24-19

\twolineshloka
{इदं त्वसदृशं कर्म परदाराभिमर्शनम्}
{यस्मादेष परक्यासु रमते राक्षसाधमः} %7-24-20

\twolineshloka
{तस्माद्वै स्त्रीकृतेनैव प्राप्स्यते दुर्मतिर्वधम्}
{सतीभिर्वरनारीभिरेवं वाक्येऽभ्युदीरिते} %7-24-21

\twolineshloka
{नेदुर्दुन्दुभयः खस्थाः पुष्पवृष्टिः पपात च}
{शप्तः स्त्रीभिः स तु समं हतौजा इव निष्प्रभः} %7-24-22

\threelineshloka
{पतिव्रताभिः साध्वीभिर्बभूव विमना इव}
{एवं विलपितं तासां शृण्वन्राक्षसपुङ्गवः}
{प्रविवेश पुरीं लङ्कां पूज्यमानो निशाचरैः} %7-24-23

\twolineshloka
{एतस्मिन्नन्तरे घोरा राक्षसी कामरूपिणी}
{सहसा पतिता भूमौ भगिनी रावणस्य सा} %7-24-24

\twolineshloka
{तां स्वसारं समुत्थाप्य रावणः परिसान्त्वयन्}
{अब्रवीत्किमिदं भद्रे वक्तुकामाऽसि मे द्रुतम्} %7-24-25

\twolineshloka
{सा बाष्पपरिरुद्धाक्षी रक्ताक्षी वाक्यमब्रवीत्}
{कृताऽस्मि विधवा राजंस्त्वया बलवता बलात्} %7-24-26

\twolineshloka
{एते राजंस्त्वया वीरा दैत्या विनिहता रणे}
{कालकेया इति ख्याताः सहस्राणि चतुर्दश} %7-24-27

\onelineshloka
{प्राणेभ्योऽपि गरीयान्मे तत्र भर्ता महाबलः} %7-24-28

\twolineshloka
{सोऽपि त्वया हतस्तात रिपुणा भ्रातृगृध्नुना}
{त्वयाऽस्मि निहता राजन्स्वयमेव हि बन्धुना} %7-24-29

\threelineshloka
{राजन्वैधव्यशब्दं च भोक्ष्यामि त्वत्कृते ह्यहम्}
{ननु नाम त्वया रक्ष्यो जामाता समरेष्वपि}
{स त्वया निहतो युद्धे स्वयमेव न लज्जसे} %7-24-30

\twolineshloka
{एवमुक्तो दशग्रीवो भगिन्या क्रोशमानया}
{अब्रवीत्सान्त्वयित्वा तां सामपूर्वमिदं वचः} %7-24-31

\twolineshloka
{अलं वत्से रुदित्वा ते न भेतव्यं च सर्वशः}
{दानमानप्रसादैस्त्वां तोषयिष्यामि यत्नतः} %7-24-32

\twolineshloka
{युद्धप्रमत्तो व्याक्षिप्तो जयाकाङ्क्षी क्षिपञ्छरान्}
{नावगच्छामि युद्धेषु स्वान्परान्वाप्यहं शुभे} %7-24-33

\twolineshloka
{जामातरं न जाने स्म प्रहरन्युद्धदुर्मदः}
{तेनासौ निहतः सङ्ख्ये मया भर्ता तव स्वसः} %7-24-34

\twolineshloka
{अस्मिन्काले तु यत्प्राप्तं तत्करिष्यामि ते हितम्}
{भ्रातुरैश्वर्ययुक्तस्य खरस्य वस पार्श्वतः} %7-24-35

\twolineshloka
{चतुर्दशानां भ्राता ते सहस्राणां भविष्यति}
{प्रभुः प्रयाणे दाने च राक्षसानां महाबलः} %7-24-36

\twolineshloka
{तत्र मातृष्वसेयस्ते भ्राताऽयं वै खरः प्रभुः}
{भविष्यति तवादेशं सदा कुर्वन्निशाचरः} %7-24-37

\twolineshloka
{शीघ्रं गच्छत्वयं वीरो दण्डकान्परिरक्षितुम्}
{दूषणोऽस्य बलाध्यक्षो भविष्यति महाबलः} %7-24-38

\twolineshloka
{तत्र ते वचनं शूरः करिष्यति सदा खरः}
{रक्षसां कामरूपाणां प्रभुरेष भविष्यति} %7-24-39

\twolineshloka
{एवमुक्त्वा दशग्रीवः सैन्यमस्यादिदेश ह}
{चतुर्दश सहस्राणि रक्षसां वीर्यशालिनाम्} %7-24-40

\twolineshloka
{स तैः परिवृतस्सर्वै राक्षसैर्घोरदर्शनैः}
{आगच्छत खरः शीघ्रं दण्डकानकुतोभयः} %7-24-41

\twolineshloka
{स तत्र कारयामास राज्यं निहतकण्टकम्}
{सा च शूर्पणखा तत्र न्यवसद्दण्डकावने} %7-24-42


॥इत्यार्षे श्रीमद्रामायणे वाल्मीकीये आदिकाव्ये उत्तरकाण्डे खरशूर्पनखादण्डकानिवासादेशः नाम चतुर्विंशः सर्गः ॥७-२४॥
