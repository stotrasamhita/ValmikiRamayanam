\sect{षड्विंशः सर्गः — नलकूबरशापः}

\twolineshloka
{स तु तत्र दशग्रीवः सह सैन्येन वीर्यवान्}
{अस्तं प्राप्ते दिनकरे निवासं समरोचयत्} %7-26-1

\twolineshloka
{उदिते विमले चन्द्रे तुल्यपर्वतवर्चसि}
{प्रसुप्तं सुमहत्सैन्यं नानाप्रहरणायुधम्} %7-26-2

\twolineshloka
{रावणस्तु महावीर्यो निषण्णः शैलमूर्धनि}
{स ददर्श गुणांस्तत्र चन्द्रपादपशोभितान्} %7-26-3

\twolineshloka
{कर्णिकारवनैर्दीप्तैः कदम्बगहनैस्तथा}
{पद्मिनीभिश्च फुल्लाभिर्मन्दाकिन्या जलैरपि} %7-26-4

\threelineshloka
{चम्पकाशोकपुन्नागमन्दारतरुभिस्तथा}
{चूतपाटललोध्रैश्च प्रियङ्ग्वर्जुनकेतकैः}
{तगरैर्नारिकैलैश्च प्रियालपनसैस्तथा} %7-26-5

\threelineshloka
{आरग्वधैस्तमालैश्च प्रियालवकुलैरपि}
{एतैरन्यैश्च तरुभिरुद्भासितवनान्तरे}
{किन्नरा मदनेनार्ता रक्ता मधुरकण्ठिनः} %7-26-6

\onelineshloka
{समं सम्प्रजगुर्यत्र मनस्तुष्टिविवर्धनम्} %7-26-7

\threelineshloka
{भिरुद्भासितवनान्तरे}
{विद्याधरा मदक्षीबा मदरक्तान्तलोचनाः}
{योषिद्भिः सह सङ्क्रान्ताश्चिक्रीडुर्जहृषुश्च वै} %7-26-8

\twolineshloka
{घण्टानामिव सन्नादः शुश्रुवे मधुरस्वरः}
{अप्सरोगणसङ्घानां गायतां धनदालये} %7-26-9

\twolineshloka
{पुष्पवर्षाणि मुञ्चन्तो नगाः पवनताडिताः}
{शैलं तं वासयन्तीव मधुमाधवगन्धिनः} %7-26-10

\twolineshloka
{मधुपुष्परजःपृक्तं गन्धमादाय पुष्कलम्}
{प्रववौ वर्धयन्कामं रावणस्य सुखोऽनिलः} %7-26-11

\twolineshloka
{गेयात्पुष्पसमृद्ध्या च शैत्याद्वायोर्गिरेर्गुणात्}
{प्रवृत्तायां रजन्यां च चन्द्रस्योदयनेन च} %7-26-12

\twolineshloka
{रावणस्तु महावीर्यः कामस्य वशमागतः}
{विनिःश्वस्य विनिःश्वस्य शशिनं समवैक्षत} %7-26-13

\twolineshloka
{एतस्मिन्नन्तरे तत्र दिव्याभरणभूषिता}
{सर्वाप्सरोवरा रम्भा दिव्यपुष्पविभूषिता} %7-26-14

\twolineshloka
{दिव्यचन्दनलिप्ताङ्गी मन्दारकृतमूर्धजा}
{दिव्योत्सवकृतारम्भा पूर्णचन्द्रनिभानना} %7-26-15

\twolineshloka
{चक्षुर्मनोहरं पीनं मेखलादामभूषितम्}
{समुद्वहन्ती जघनं रतिप्राभृतमुत्तमम्} %7-26-16

\onelineshloka
{कृतैर्विशेषकैरार्द्रैः षडर्तुकुसुमोद्भवैः} %7-26-17

\twolineshloka
{बभावन्यतमेव श्रीकान्तिद्युतिमतिह्रियाम्}
{नीलं सतोयमेघाभं वस्त्रं समवकुण्ठिता} %7-26-18

\threelineshloka
{यस्या वक्त्रं शशिनिभं भ्रुवौ चापनिभे शुभे}
{ऊरू करिकराकारौ करौ पल्लवकोमलौ}
{सैन्यमध्येन गच्छन्ती रावणेनोपवीक्षिता} %7-26-19

\twolineshloka
{तां समुत्थाय गच्छन्तीं कामबाणवशं गतः}
{करे गृहीत्वा लज्जन्तीं स्मयमानोऽभ्यभाषत} %7-26-20

\twolineshloka
{क्व गच्छसि वरारोहे कां सिद्धिं भजसे स्वयम्}
{कस्याभ्युदयकालोऽयं यस्त्वां समुपभोक्ष्यते} %7-26-21

\twolineshloka
{त्वदाननरसस्याद्य पद्मोत्पलसुगन्धिनः}
{सुधामृतरसस्येव कोऽद्य तृप्तिं गमिष्यति} %7-26-22

\twolineshloka
{स्वर्णकुम्भनिभौ पीनौ शुभौ भीरु निरन्तरौ}
{कस्योरस्थलसंस्पर्शं दास्यतस्ते कुचाविमौ} %7-26-23

\twolineshloka
{सुवर्णचक्रप्रतिमं स्वर्णदामचितं पृथु}
{अध्यारोहति कस्तेऽद्य जघनं स्वर्गरूपिणम्} %7-26-24

\twolineshloka
{मद्विशिष्टः पुमान्कोऽद्य शक्रो विष्णुरथाश्विनौ}
{मामतीत्य हि यं च त्वं यासि भीरु न शोभनम्} %7-26-25

\onelineshloka
{विश्रम त्वं पृथुश्रोणि शिलातलमिदं शुभम्} %7-26-26

\onelineshloka
{त्रैलोक्ये यः प्रभुश्चैव मदन्यो नैव विद्यते} %7-26-27

\twolineshloka
{तदेवं प्राञ्जलिः प्रह्वो याचते त्वां दशाननः}
{भर्तुर्भर्ता विधाता च त्रैलोक्यस्य भजस्व माम्} %7-26-28

\twolineshloka
{एवमुक्ताऽब्रवीद्रम्भा वेपमाना कृताञ्जलिः}
{प्रसीद नार्हसे वक्तुमीदृशं त्वं हि मे गुरुः} %7-26-29

\twolineshloka
{अन्येभ्यो हि त्वया रक्ष्या प्राप्नुयां धर्षणं यदि}
{तद्धर्मतः स्नुषा तेऽहं तत्त्वमेव ब्रवीमि ते} %7-26-30

\twolineshloka
{अथाब्रवीद्दशग्रीवश्चरणाधोमुखीं स्थिताम्}
{रोमहर्षमनुप्राप्तां दृष्टमात्रेण तां तदा} %7-26-31

\twolineshloka
{सुतस्य यदि मे भार्या ततस्त्वं हि स्नुषा भवेः}
{बाढमित्येव सा रम्भा प्राह रावणमुत्तरम्} %7-26-32

\threelineshloka
{धर्मतस्ते सुतस्याहं भार्या राक्षसपुङ्गव}
{पुत्रः प्रियतरः प्राणैर्भ्रातुर्वैश्रवणस्य ते}
{विख्यातस्त्रिषु लोकेषु नलकूबर इत्ययम्} %7-26-33

\twolineshloka
{धर्मतो यो भवेद्विप्रः क्षत्रियो वीर्यतो भवेत्}
{क्रोधाद्यश्च भवेदग्निः क्षान्त्या च वसुधासमः} %7-26-34

\twolineshloka
{तस्यास्मि कृतसङ्केता लोकपालसुतस्य वै}
{तमुद्दिश्य तु मे सर्वं विभूषणमिदं कृतम्} %7-26-35

\onelineshloka
{तथा तस्य हि नान्यस्य भावो मां प्रति तिष्ठति} %7-26-36

\onelineshloka
{तेन सत्येन मां राजन्मोक्तुमर्हस्यरिन्दम} %7-26-37

\twolineshloka
{स हि तिष्ठति धर्मात्मा मां प्रतीक्ष्य समुत्सुकः}
{तत्र विघ्नं सुतस्येह कर्तुं नार्हसि मुञ्च माम्} %7-26-38

\twolineshloka
{सद्भिराचरितं मार्गं गच्छ राक्षसुपुङ्गव}
{माननीयो मम त्वं हि पालनीया तथाऽस्मि ते} %7-26-39

\threelineshloka
{एवमुक्तो दशग्रीवः प्रत्युवाच विनीतवत्}
{स्नुषाऽस्मि यदवोचस्त्वमेकपत्नीष्वयं क्रमः}
{देवलोकस्थितिरियं सुराणां शाश्वती मता} %7-26-40

\onelineshloka
{पतिरप्सरसां नास्ति न चैकस्त्रीपरिग्रहः} %7-26-41

\threelineshloka
{स्मि यदवोचस्त्वमेकपत्नीष्वयं क्रमः}
{एवमुक्त्वा स तां रक्षो निवेश्य च शिलातले}
{कामभोगाभिसंसक्तो मैथुनायोपचक्रमे} %7-26-42

\twolineshloka
{सा विमुक्ता ततो रम्भा भ्रष्टमाल्यविभूषणा}
{गजेन्द्राक्रीडमथिता नदीवाकुलतां गता} %7-26-43

\twolineshloka
{लुलिताकुलकेशान्ता करवेपितपल्लवा}
{पवनेनावधूतेव लता कुसुमशालिनी} %7-26-44

\twolineshloka
{सा वेपमाना लज्जन्ती भीता करकृताञ्जलिः}
{नलकूबरमासाद्य पादयोर्निपपात ह} %7-26-45

\twolineshloka
{तदवस्थां च तां दृष्ट्वा महात्मा नलकूबरः}
{अब्रवीत्किमिदं भद्रे पादयोः पतिताऽसि मे} %7-26-46

\twolineshloka
{सा वै निःश्वसमाना तु वेपमाना कृताञ्जलिः}
{तस्मै सर्वं यथातत्त्वमाख्यातुमुपचक्रमे} %7-26-47

\twolineshloka
{एष देव दशग्रीवः प्राप्तो गन्तुं त्रिविष्टपम्}
{तेन सैन्यसहायेन निशेयं परिणामिता} %7-26-48

\twolineshloka
{आयन्ती तेन दृष्टाऽस्मि त्वत्सकाशमरिन्दम}
{गृहीता तेन पृष्टाऽस्मि कस्य त्वमिति रक्षसा} %7-26-49

\twolineshloka
{मया तु सर्वं यत्सत्यं तस्मै सर्वं निवेदितम्}
{काममोहाभिभूतात्मा नाश्रौषीत्तद्वचो मम} %7-26-50

\twolineshloka
{याच्यमानो मया देव स्नुषा तेऽहमिति प्रभो}
{तत्सर्वं पृष्ठतः कृत्वा बलात्तेनास्मि धर्षिता} %7-26-51

\twolineshloka
{एवं त्वमपराधं मे क्षन्तुमर्हसि सुव्रत}
{न हि तुल्यं बलं सौम्य स्त्रियाश्च पुरुषस्य च} %7-26-52

\twolineshloka
{एतछुत्वा तु सङ्क्रुद्धस्तदा वैश्रवणात्मजः}
{धर्षणां तां परां श्रुत्वा ध्यानं सम्प्रविवेश ह} %7-26-53

\twolineshloka
{तस्य तत्कर्म विज्ञाय तदा वैश्रवणात्मजः}
{मुहूर्तात्क्रोधताम्राक्षस्तोयं जग्राह पाणिना} %7-26-54

\twolineshloka
{गृहीत्वा सलिलं सर्वमुपस्पृश्य यथाविधि}
{उत्ससर्ज यथा शापं राक्षसेन्द्राय दारुणम्} %7-26-55

\twolineshloka
{अकामा तेन यस्मात्त्वं बलाद्भद्रे प्रधर्षिता}
{तस्मात्स युवतीमन्यां नाकामामुपयास्यति} %7-26-56

\twolineshloka
{यदा ह्यकामां कामार्तो धर्षयिष्यति योषितम्}
{मूर्धा तु सप्तधा तस्य शकलीभविता तदा} %7-26-57

\threelineshloka
{तस्मिन्नुदाहृते शापे ज्वलिताग्निसमप्रभे}
{देवदुन्दुभयो नेदुः पुष्पवृष्टिश्च खाच्च्युता}
{पितामहमुखाश्चैव सर्वे देवाः प्रहर्षिताः} %7-26-58

\twolineshloka
{ज्ञात्वा लोकगतिं सर्वां तस्य मृत्युं च रक्षसः}
{ऋषयः पितरश्चैव प्रीतिमापुरनुत्तमाम्} %7-26-59

\twolineshloka
{श्रुत्वा तु स दशग्रीवस्तं शापं रोमहर्षणम्}
{नारीषु मैथुने भावं नाकामास्वभ्यरोचयत्} %7-26-60

\twolineshloka
{तेन नीताः स्त्रियः प्रीतिमापुः सर्वाः पतिव्रताः}
{नलकूबरनिर्मुक्तं शापं श्रुत्वा मनःप्रियम्} %7-26-61


॥इत्यार्षे श्रीमद्रामायणे वाल्मीकीये आदिकाव्ये उत्तरकाण्डे नलकूबरशापः नाम षड्विंशः सर्गः ॥७-२६॥
