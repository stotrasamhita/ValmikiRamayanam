\sect{अष्टाविंशः सर्गः — जयन्तापवाहनम्}

\twolineshloka
{सुमालिनं हतं दृष्ट्वा वसुना भस्मसात्कृतम्}
{स्वसैन्यं विद्रुतं चापि लक्षयित्वाऽर्दितं सुरैः} %7-28-1

\twolineshloka
{ततः स बलवान्क्रुद्धो रावणस्य सुतस्तदा}
{निवर्त्य राक्षसान्सर्वान्मेघनादो व्यवस्थितः} %7-28-2

\twolineshloka
{सुरथेनाग्निवर्णेन कामगेन महारथः}
{अभिदुद्राव सेनां तां वनान्यग्निरिव ज्वलन्} %7-28-3

\twolineshloka
{ततः प्रविशतस्तस्य विविधायुधधारिणः}
{विदुद्रुवुर्दिशः सर्वा दर्शनादेव देवताः} %7-28-4

\twolineshloka
{न बभूव तदा कश्चिद्युयुत्सोरस्य सम्मुखे}
{सर्वानाविद्ध्य वित्रस्तांस्ततः शक्रोऽब्रवीत्सुरान्} %7-28-5

\twolineshloka
{न भेतव्यं न गन्तव्यं निवर्तध्वं रणे सुराः}
{एष गच्छति पुत्रो मे युद्धार्थमपराजितः} %7-28-6

\twolineshloka
{ततः शक्रसुतो देवो जयन्त इति विश्रुतः}
{रथेनाद्भुतकल्पेन सङ्ग्रामे सोऽभ्यवर्तत} %7-28-7

\twolineshloka
{ततस्ते त्रिदशाः सर्वे परिवार्य शचीसुतम्}
{रावणस्य सुतं युद्धे समासाद्य प्रजघ्निरे} %7-28-8

\twolineshloka
{तेषां युद्धं समभवत्सदृशं देवरक्षसाम्}
{महेन्द्रस्य च पुत्रस्य राक्षसेन्द्रसुतस्य च} %7-28-9

\twolineshloka
{ततो मातलिपुत्रे तु गोमुखे राक्षसात्मजः}
{सारथौ पातयामास शरान्कनकभूषणान्} %7-28-10

\twolineshloka
{शचीसुतश्चापि तथा जयन्तस्तस्य सारथिम्}
{तं चापि रावणिः क्रुद्धः समन्तात्प्रत्यविध्यत} %7-28-11

\twolineshloka
{स हि क्रोधसमाविष्टो बली विस्फारितेक्षणः}
{रावणिः शक्रतनयं शरवर्षैरवाकिरत्} %7-28-12

\twolineshloka
{ततो नानाप्रहरणाञ्छितधारान्सहस्रशः}
{पातयामास सङ्क्रुद्धः सुरसैन्येषु रावणिः} %7-28-13

\twolineshloka
{शतघ्नीमुसलप्रासगदाखड्गपरश्वधान्}
{महान्ति गिरिशृङ्गाणि पातयामास रावणिः} %7-28-14

\twolineshloka
{ततः प्रव्यथिता लोकाः सञ्जज्ञे च तमोऽभवत्}
{तस्य रावणपुत्रस्य शत्रुसैन्यानि निघ्नतः} %7-28-15

\twolineshloka
{ततस्तद्दैवतबलं समन्तात्तं शचीसुतम्}
{बहुप्रकारमस्वस्थमभवच्छरपीडितम्} %7-28-16

\twolineshloka
{नाभ्यजानन्त चान्योन्यं रक्षो वा देवताऽथवा}
{तत्र तत्र विपर्यस्तं समन्तात्परिधावति} %7-28-17

\twolineshloka
{देवा देवान्निजघ्नुस्ते राक्षसान्राक्षसास्तथा}
{सम्मूढास्तमसाच्छन्ना व्यद्रवन्नपरे तथा} %7-28-18

\twolineshloka
{एतस्मिन्नन्तरे वीरः पुलोमा नाम वीर्यवान्}
{दैत्येन्द्रस्तेन सङ्गृह्य शचीपुत्रोऽपवाहितः} %7-28-19

\twolineshloka
{सङ्गृह्य तं तु दौहित्रं प्रविष्टः सागरं तदा}
{आर्यकः स हि तस्यासीत्पुलोमा येन सा शची} %7-28-20

\twolineshloka
{ज्ञात्वा प्रणाशं तु तदा जयन्तस्याथ देवताः}
{अप्रहृष्टास्ततः सर्वा व्यथिताः सम्प्रदुद्रुवुः} %7-28-21

\twolineshloka
{रावणिस्त्वथ सङ्क्रुद्धो बलैः परिवृतः स्वकैः}
{अभ्यधावत देवांस्तान्मुमोच च महास्वनम्} %7-28-22

\twolineshloka
{दृष्ट्वा प्रणाशं पुत्रस्य दैवतेषु च विद्रुतम्}
{मातलिं चाह देवेशो रथः समुपनीयताम्} %7-28-23

\twolineshloka
{स तु दिव्यो महाभीमः सज्ज एव महारथः}
{उपस्थितो मातलिना वाह्यमानो महाजवः} %7-28-24

\twolineshloka
{ततो मेघा रथे तस्मिंस्तडित्त्वन्तो महाबलाः}
{अग्रतो वायुचपला नेदुः परमनिःस्वनाः} %7-28-25

\twolineshloka
{नानावाद्यानि वाद्यन्त गन्धर्वाश्च समाहिताः}
{ननृतुश्चाप्सरःसङ्घा निर्याते त्रिदशेश्वरे} %7-28-26

\twolineshloka
{रुद्रैर्वसुभिरादित्यैस्साध्यैश्च समरुद्गणैः}
{वृतो नानाप्रहरणैर्निर्ययौ त्रिदशाधिपः} %7-28-27

\twolineshloka
{निर्गच्छतस्तु शक्रस्य परुषं पवनो ववौ}
{भास्करो निष्प्रभश्चासीन्महोल्काश्च प्रपेदिरे} %7-28-28

\twolineshloka
{एतस्मिन्नन्तरे शूरो दशग्रीवः प्रतापवान्}
{आरुरोह रथं दिव्यं निर्मितं विश्वकर्मणा} %7-28-29

\twolineshloka
{पन्नगैः सुमहाकार्यैर्वेष्टितं रोमहर्षणैः}
{तेषां निःश्वासवातेन प्रदीप्तमिव संयुगे} %7-28-30

\twolineshloka
{दैत्यैर्निशाचरैश्चैव स रथः परिवारितः}
{समराभिमुखो दैत्यो महेन्द्रं सोऽभ्यवर्तत} %7-28-31

\twolineshloka
{पुत्रं तं वारयित्वा तु स्वयमेव व्यवस्थितः}
{सोऽपि युद्धाद्विनिष्क्रम्य रावणिः समुपाविशत्} %7-28-32

\twolineshloka
{ततो युद्धं प्रवृत्तं तु सुराणां राक्षसैः सह}
{शस्त्राणि वर्षतां घोरं मेघानामिव संयुगे} %7-28-33

\twolineshloka
{कुम्भकर्णस्तु दुष्टात्मा नानाप्रहरणोद्यतः}
{नाज्ञायत तदा युद्धे सह केनाप्ययुध्यत} %7-28-34

\threelineshloka
{दन्तैः भुजाभ्यां पद्मां च शक्तितोमरसायकैः}
{येन केनैव संरब्धस्ताडयामास वै सुरान् ततो रुद्रैर्महाघोरैः सङ्गम्याथ निशाचरः}
{प्रयुद्धस्तैश्च सङ्ग्रामे क्षतः शस्त्रैर्निरन्तरम्} %7-28-35

\twolineshloka
{बभौ शस्त्राचिततनुः कुम्भकर्णः क्षरन्नसृक्}
{विद्युत्स्तनितनिर्घोषो धारवानिव तोयदः} %7-28-36

\twolineshloka
{ततस्तद्राक्षसं सैन्यं प्रयुद्धं समरुद्गणैः}
{रणे विद्रावितं सर्वं नानाप्रहरणैः शितैः} %7-28-37

\twolineshloka
{केचिद्विनिहताः कृत्ताश्चेष्टन्ति स्म महीतले}
{वाहनेष्ववसक्ताश्च स्थिता एवापरे रणे} %7-28-38

\twolineshloka
{रथान्नागान्खरानुष्ट्रान्पन्नगांस्तुरगान् रणे}
{शिंशुमारान्वराहांश्च पिशाचवदनान्तथा} %7-28-39

\twolineshloka
{तान्समालिङ्ग्य बाहुभ्यां विष्टब्धाः केचिदुच्छ्रुताः}
{देवैस्तु शस्त्रसम्भिन्ना मम्रिरे च निशाचराः} %7-28-40

\twolineshloka
{चित्रकर्म इवाभाति स तेषां रणसम्प्लवः}
{निहतानां प्रमत्तानां राक्षसानां महीतले} %7-28-41

\twolineshloka
{शोणितोदकनिष्पन्दकङ्कगृध्रसमाकुला}
{प्रवृत्ता संयुगमुखे शस्त्रग्राहवती नदी} %7-28-42

\twolineshloka
{एतस्मिन्नन्तरे क्रुद्धो दशग्रीवः प्रतापवान्}
{निरीक्ष्य तद्बलं कृत्स्नं दैवतैर्विनिपातितम्} %7-28-43

\twolineshloka
{स तं प्रति विगाह्याशु प्रवृद्धं सैन्यसागरम्}
{त्रिदशान्समरे निघ्नञ्छक्रमेवाभ्यवर्तत} %7-28-44

\twolineshloka
{आगाच्छक्रो महच्चापं विस्फार्य सुमहास्वनम्}
{यस्य विस्फारघोषेण स्तनन्ति स्म दिशो दश} %7-28-45

\twolineshloka
{तद्विकृष्य महच्चापमिन्द्रो रावणमूर्धनि}
{निपातयामास शरान्पावकादित्यवर्चसः} %7-28-46

\twolineshloka
{तथैव च महाबाहुर्दशग्रीवो व्यवस्थितः}
{शक्रं कार्मुकविभ्रष्टैः शरवर्षैरवाकिरत्} %7-28-47

\twolineshloka
{प्रयुध्यतोरथ तयोर्बाणवर्षैः समन्ततः}
{न ज्ञायते तदा किञ्चित्सर्वं हि तमसा वृतम्} %7-28-48


॥इत्यार्षे श्रीमद्रामायणे वाल्मीकीये आदिकाव्ये उत्तरकाण्डे जयन्तापवाहनम् नाम अष्टाविंशः सर्गः ॥७-२८॥
