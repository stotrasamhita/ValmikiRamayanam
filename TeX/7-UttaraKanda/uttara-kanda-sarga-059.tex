\sect{एकोनषष्ठितमः सर्गः — पूरुराज्याभिषेकः
सारमेयावस्थानम्
सर्वार्थसिद्धकौलपत्यदानम्
गृध्रोलूकविवादः}

\twolineshloka
{श्रुत्वा तूशनसं क्रुद्धं तदाऽऽर्तो नहुषात्मजः}
{जरां परमिकां प्राप्य यदुं वचनमब्रवीत्} %7-59-1

\twolineshloka
{यदो त्वमसि धर्मज्ञो मदर्थं प्रतिगृह्यताम्}
{जरां परमिकां पुत्र भोगै रंस्ये महायशः} %7-59-2

\twolineshloka
{न तावत्कृतकृत्योऽस्मि विषयेषु नरर्षभ}
{अनुभूय तदा कामं ततः प्राप्स्याम्यहं जराम्} %7-59-3

\twolineshloka
{यदुस्तद्वचनं श्रुत्वा प्रत्युवाच नरर्षभम्}
{पुत्रस्ते दयितः पूरुः प्रतिगृह्णातु वै जराम्} %7-59-4

\twolineshloka
{बहिष्कृतोऽहमर्थेषु सन्निकर्षाच्च पार्थिव}
{प्रतिगृह्णातु वै राजन् क सहाश्नाति भोजनम्} %7-59-5

\twolineshloka
{तस्य तद्वचनं श्रुत्वा राजा पूरुमथाब्रवीत्}
{इयं जरा महाबाहो मदर्थं प्रतिगृह्यताम्} %7-59-6

\twolineshloka
{नाहुषेणैवमुक्तस्तु पूरुः प्राञ्जलिरब्रवीत्}
{धन्योऽस्म्यनुगृहीतोऽस्मि शासनेऽस्मि तव स्थितः} %7-59-7

\twolineshloka
{पूरोर्वचनमाज्ञाय नाहुषः परया मुदा}
{प्रहर्षमतुलं लेभे जरां सङ्क्रामयच्च ताम्} %7-59-8

\twolineshloka
{ततः स राजा तरुणः प्राप्य यज्ञान्सहस्रशः}
{बहुवर्षसहस्राणि पालयामास मेदिनीम्} %7-59-9

\twolineshloka
{अथ दीर्घस्य कालस्य राजा पूरुमथाब्रवीत्}
{आनयस्व जरां पुत्र न्यासं निर्यातयस्व मे} %7-59-10

\twolineshloka
{न्यासभूता मया पुत्र त्वयि सङ्क्रामिता जरा}
{तस्मात्प्रतिग्रहीष्यामि तां जरां मा व्यथां कृथाः} %7-59-11

\twolineshloka
{प्रीतश्चास्मि महाबाहो शासनस्य प्रतिग्रहात्}
{त्वां चाहमभिषेक्ष्यामि प्रीतियुक्तो नराधिपम्} %7-59-12

\twolineshloka
{एवमुक्त्वा सुतं पूरुं ययातिर्नहुषात्मजः}
{देवयानीसुतं क्रुद्धो राजा वाक्यमुवाच ह} %7-59-13

\twolineshloka
{राक्षसस्त्वं मया जातः पुत्ररूपो दुरासदः}
{प्रतिहंसि ममाज्ञां यत् प्रजार्थे विफलो भव} %7-59-14

\twolineshloka
{पितरं गुरुभूतं मां यस्मात्त्वमवमन्यसे}
{राक्षसान्यातुधानांस्त्वं जनयिष्यसि दारुणान्} %7-59-15

\twolineshloka
{न तु सोमकुलोत्पन्ने वंशे स्थास्यति दुर्मतेः}
{वंशोऽपि भवतस्तुल्यो दुर्विनीतो भविष्यति} %7-59-16

\twolineshloka
{तमेवमुक्त्वा राजर्षिः पूरुं राज्यविवर्धनम्}
{अभिषेकेण सम्पूज्य आश्रमं प्रविवेश ह} %7-59-17

\twolineshloka
{ततः कालेन महता दिष्टान्तमुपजग्मिवान्}
{त्रिदिवं सङ्गतो राजा ययातिर्नहुषात्मजः} %7-59-18

\twolineshloka
{पूरुश्चकार तद्राज्यं धर्मेण महता वृतः}
{प्रतिष्ठाने पुरवरे काशिराज्ये महायशाः} %7-59-19

\twolineshloka
{यदुस्तु जनयामास यातुधानान्सहस्रशः}
{पुरे क्रौञ्चवने दुर्गे राजवंशबहिष्कृतः} %7-59-20

\twolineshloka
{एष तूशनसा मुक्तः शापोत्सर्गो ययातिना}
{धारितः क्षत्रधर्मेण यन्निमिश्च न चक्षमे} %7-59-21

\twolineshloka
{एतत्ते सर्वमाख्यातं दर्शनं सर्वकारिणाम्}
{अनुवर्तामहे सौम्य दोषो न स्याद्यथा नृगे} %7-59-22

\twolineshloka
{इति कथयति रामे चन्द्रतुल्यानने च प्रविरलतरतारं व्योम जज्ञे तदानीम्}
{अरुणकिरणरक्ता दिग्बभौ चैव पूर्वा कुसुमरसविमुक्तं वस्त्रमाकुण्ठितेव} %7-59-23


॥इत्यार्षे श्रीमद्रामायणे वाल्मीकीये आदिकाव्ये उत्तरकाण्डे पूरुराज्याभिषेकः नाम एकोनषष्ठितमः सर्गः ॥७-५९॥
