\sect{सप्तपञ्चाशः सर्गः — निमिनिमीषीकरणम्}

\twolineshloka
{तां श्रुत्वा दिव्यसङ्काशां कथामद्भुतदर्शनाम्}
{लक्ष्मणः परमप्रीतो राघवं वाक्यमब्रवीत्} %7-57-1

\twolineshloka
{निक्षिप्तदेहौ काकुत्स्थ कथं तौ द्विजपार्थिवौ}
{पुनर्देहेन संयोगं जग्मुर्देवसम्मतौ} %7-57-2

\twolineshloka
{तस्य तद्भाषितं श्रुत्वा रामः सत्यपराक्रमः}
{तां कथां कथयामास वसिष्ठस्य महात्मनः} %7-57-3

\twolineshloka
{यस्तु कुम्भो रघुश्रेष्ठ तेजःपूर्णो महात्मनोः}
{तस्मिंस्तेजोमयौ विप्रौ सम्भूतावृषिसत्तमौ} %7-57-4

\twolineshloka
{पूर्वं समभवत्तत्र ह्यगस्त्यो भगवानृषिः}
{नाहं सुतस्तवेत्युक्त्वा मित्रं तस्मादपाक्रमत्} %7-57-5

\twolineshloka
{तद्धि तेजस्तु मित्रस्य उर्वस्याः पूर्वमाहितम्}
{तस्मिन्समभवत्कुम्भे तत्तेजो यत्र वारुणम्} %7-57-6

\twolineshloka
{कस्यचित्त्वथ कालस्य मित्रावरुणसम्भवः}
{वसिष्ठस्तेजसा युक्तो जज्ञे इक्ष्वाकुदैवतम्} %7-57-7

\twolineshloka
{तमिक्ष्वाकुर्महातेजा जातमात्रमनिन्दितम्}
{वव्रे पुरोधसं सौम्य वंशस्यास्य भवाय नः} %7-57-8

\twolineshloka
{एवं त्वपूर्वदेहस्य वसिष्ठस्य महात्मनः}
{कथितो निर्गमः सौम्य निमेः शृणु यथाऽभवत्} %7-57-9

\twolineshloka
{दृष्ट्वा विदेहं राजानमृषयः सर्व एव ते}
{तं च ते योजयामासुर्यागदीक्षां मनीषिणः} %7-57-10

\twolineshloka
{तं च देहं नरेन्द्रस्य रक्षन्ति स्म द्विजोत्तमाः}
{गन्धैर्माल्यैश्च वस्त्रैश्च पौरभृत्यसमन्विताः} %7-57-11

\twolineshloka
{ततो यज्ञे समाप्ते तु भृगुस्तत्रेदमब्रवीत्}
{आनयिष्यामि ते चेतस्तुष्टोऽस्मि तव पार्थिव} %7-57-12

\twolineshloka
{सुप्रीताश्च सुराः सर्वे निमेश्चेतस्तथाब्रुवन्}
{वरं वरय राजर्षे क्व ते चेतो निरूप्यताम्} %7-57-13

\twolineshloka
{एवमुक्तः सुरैः सर्वैर्निमेश्चेतस्तदाब्रवीत्}
{नेत्रेषु सर्वभूतानां वसेयं सुरसत्तमाः} %7-57-14

\twolineshloka
{बाढमित्येव विबुधा निमेश्चेतस्तदाऽब्रुवन्}
{नेत्रेषु सर्वभूतानां वायुभूतश्चरिष्यसि} %7-57-15

\twolineshloka
{त्वत्कृते च निमिष्यन्ति चक्षूंषि पृथिवीपते}
{वायुभूतेन चरता विश्रमार्थं मुहुर्मुहुः} %7-57-16

\twolineshloka
{एवमुक्त्वा तु विबुधाः सर्वे जग्मुर्यथागतम्}
{ऋषयोऽपि महात्मानो निमेर्देहं समाहरन्} %7-57-17

\twolineshloka
{अरणिं तत्र निक्षिप्य मथनं चक्रुरोजसा}
{मन्त्रहोमैर्महात्मानः पुत्रहेतोर्निमेस्तदा} %7-57-18

\threelineshloka
{अरण्यां मथ्यामानायां प्रादुर्भूतो महातपाः}
{मथनान्मिथिरित्याहुर्जननाज्जनकोऽभवत्}
{यस्माद्विदेहात्सम्भूतो वैदेहस्तु ततः स्मृतः} %7-57-19

\twolineshloka
{एवं विदेहराजश्च जनकः पूर्वकोऽभवत्}
{मिथिर्नाम महातेजास्तेनायं मैथिलोऽभवत्} %7-57-20


॥इत्यार्षे श्रीमद्रामायणे वाल्मीकीये आदिकाव्ये उत्तरकाण्डे निमिनिमीषीकरणम् नाम सप्तपञ्चाशः सर्गः ॥७-५७॥
