\sect{द्व्यशीतितमः सर्गः — रामनिवर्तनम्}

\twolineshloka
{ऋषेर्वचनमाज्ञाय रामः सन्ध्यामुपासितुम्}
{उपाक्रमत्सरः पुण्यमप्सरोगणसेवितम्} %7-82-1

\twolineshloka
{तत्रोदकमुपस्पृश्य सन्ध्यामन्वास्य पश्चिमाम्}
{आश्रमं प्राविशद्रामः कुम्भयोनेर्महात्मनः} %7-82-2

\twolineshloka
{तस्यागस्त्यो बहुगुणं कन्दमूलं तथौषधिम्}
{शाल्यादीनि पवित्राणि भोजनार्थमकल्पयत्} %7-82-3

\twolineshloka
{स भुक्तवान्नरश्रेष्ठस्तदन्नममृतोपमम्}
{प्रीतश्च परितुष्टश्च तां रात्रिं समुपाविशत्} %7-82-4

\twolineshloka
{प्रभाते काल्यमुत्थाय कृत्वाऽऽह्निकमरिन्दमः}
{ऋषिं समुपचक्राम गमनाय रघूत्तमः} %7-82-5

\twolineshloka
{अभिवाद्याब्रवीद्रामो महर्षिं कुम्भसम्भवम्}
{आपृच्छे स्वां पुरीं गन्तुं मामनुज्ञातुमर्हसि} %7-82-6

\twolineshloka
{धन्योऽस्म्यनुगृहीतोऽस्मि दर्शनेन महात्मनः}
{द्रष्टुं चैवागमिष्यामि पावनार्थमिहात्मनः} %7-82-7

\twolineshloka
{तथा वदति काकुत्स्थे वाक्यमद्भुतदर्शनम्}
{उवाच परमप्रीतो धर्मनेत्रस्तपोधनः} %7-82-8

\twolineshloka
{अत्यद्भुतमिदं वाक्यं तव राम शुभाक्षरम्}
{पावनः सर्वभूतानां त्वमेव रघुनन्दन} %7-82-9

\twolineshloka
{मुहूर्तमपि राम त्वां ये च पश्यन्ति केचन}
{पाविताः स्वर्गभूताश्च पूज्यास्ते सर्वदैवतैः} %7-82-10

\twolineshloka
{ये च त्वां घोरचक्षुर्भिः पश्यन्ति प्राणिनो भुवि}
{हतास्ते यमदण्डेन सद्यो निरयगामिनः} %7-82-11

\twolineshloka
{ईदृशस्त्वं रघुश्रेष्ठ पावनः सर्वदेहिनाम्}
{भुवि त्वां कथयन्तोऽपि सिद्धिमेष्यन्ति राघव} %7-82-12

\twolineshloka
{गच्छ चारिष्टमव्याग्रः पन्थानमकुतोभयम्}
{प्रशाधि राज्यं धर्मेण गतिर्हि जगतो भवान्} %7-82-13

\twolineshloka
{एवमुक्तस्तु मुनिना प्राञ्जलिः प्रग्रहो नृपः}
{अभ्यवादयत प्राज्ञस्तमृषिं पुण्यशालिनम्} %7-82-14

\twolineshloka
{अभिवाद्य मुनिश्रेष्ठं तांश्च सर्वांस्तपोधनान्}
{अध्यारोहत्तदव्यग्रः पुष्पकं हेमभूषितम्} %7-82-15

\twolineshloka
{तं प्रयान्तं मुनिगणा ह्याशीर्वादैः समन्ततः}
{अपूजयन्महेन्द्राभं सहस्राक्षमिवामराः} %7-82-16

\twolineshloka
{स्वस्थः स ददृशे रामः पुष्पके हेमभूषिते}
{शशी मेघसमीपस्थो यथा जलधरागमे} %7-82-17

\twolineshloka
{ततोऽर्धदिवसे प्राप्ते पूज्यमानस्ततस्ततः}
{अयोध्यां प्राप्य काकुत्स्थो मध्यकक्ष्यामवारुहत्} %7-82-18

\twolineshloka
{ततो विसृज्य रुचिरं पुष्पकं कामगामि तत्}
{विसर्जयित्वा गच्छेति स्वस्ति तेऽस्त्विति च प्रभुः} %7-82-19

\twolineshloka
{कक्षान्तरविनिक्षिप्तं द्वास्स्थं रामोऽब्रवीद्वचः}
{लक्ष्मणं भरतं चैव गत्वा तौ लघुविक्रमौ ममागमनमाख्याय शब्दापयत मा चिरम्} %7-82-20


॥इत्यार्षे श्रीमद्रामायणे वाल्मीकीये आदिकाव्ये उत्तरकाण्डे रामनिवर्तनम् नाम द्व्यशीतितमः सर्गः ॥७-८२॥
