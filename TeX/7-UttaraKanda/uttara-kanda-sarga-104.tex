\sect{चतुरधिकशततमः सर्गः — पितामहवाक्यकथनम्}

\twolineshloka
{शृणु राजन्महासत्व यदर्थमहमागतः}
{पितामहेन देवेन प्रेषितोऽस्मि महाबल} %7-104-1

\twolineshloka
{तवाहं पूर्वसद्भावे पुत्रः परपुरञ्जय}
{मायासम्भावितो वीर कालः सर्वसमाहरः} %7-104-2

\twolineshloka
{पितामहश्च भगवानाह लोकपतिः प्रभुः}
{समयस्ते कृतः सौम्य लोकान्सम्परिरक्षितुम्} %7-104-3

\twolineshloka
{सङ्क्षिप्य हि पुरा लोकान्मायया स्वयमेव हि}
{महार्णवे शयानोऽप्सु मां त्वं पूर्वमजीजनः} %7-104-4

\twolineshloka
{भोगवन्तं ततो नागमनन्तमुदकेशयम्}
{मायया जनयित्वा त्वं द्वौ च सत्त्वौ महाबलौ} %7-104-5

\twolineshloka
{मधुं च कैटभं चैव ययोरस्थिचयैर्वृता}
{इयं पर्वतसम्बाधा मेदिनी चाभवन्मही} %7-104-6

\twolineshloka
{पद्मे दिव्येऽर्कसङ्काशे नाभ्यामुत्पाद्य मामपि}
{प्राजापत्यं त्वया कर्म मयि सर्वं निवेशितम्} %7-104-7

\twolineshloka
{सोऽहं सन्न्यस्तभारो हि त्वामुपासे जगत्पतिम्}
{रक्षां विधत्स्व भूतेषु मम तेजस्करो भवान्} %7-104-8

\twolineshloka
{ततस्त्वमपि दुर्धर्षात्तस्माद्भावात्सनातनात्}
{रक्षार्थं सर्वभूतानां विष्णुत्वमुपजग्मिवान्} %7-104-9

\twolineshloka
{अदित्यां वीर्यवान्पुत्रो भ्रातऽणां वीर्यवर्धनः}
{समुत्पन्नेषु कृत्येषु तेषां साह्याय कल्पसे} %7-104-10

\twolineshloka
{स त्वं वित्रास्यमानासु प्रजासु जगतां वर}
{रावणस्य वधाकाङ्क्षी मानुषेषु मनोऽदधाः} %7-104-11

\twolineshloka
{दश वर्षसहस्राणि दश वर्षशतानि च}
{कृत्वा वासस्य नियतिं स्वयमेवात्मना पुरा} %7-104-12

\twolineshloka
{स त्वं मनोमयः पुत्रः पूर्णायुर्मानुषेष्विह}
{कालोऽयं ते नरश्रेष्ठ समीपमुपवर्तितुम्} %7-104-13

\twolineshloka
{यदि भूयो महाराज प्रजा इच्छस्युपासितुम्}
{वस वा वीर भद्रं त एवमाह पितामहः} %7-104-14

\twolineshloka
{अथ वा विजिगीषा ते सुरलोकाय राघव}
{सनाथा विष्णुना देवा भवन्तु विगतज्वराः} %7-104-15

\twolineshloka
{श्रुत्वा पितामहेनोक्तं वाक्यं कालसमीरितम्}
{राघवः प्रहसन्वाक्यं सर्वसंहारमब्रवीत्} %7-104-16

\twolineshloka
{श्रुत्वा मे देवदेवस्य वाक्यं परममद्भुतम्}
{प्रीतिर्हि महती जाता तवागमनसम्भवा} %7-104-17

\twolineshloka
{त्रयाणामपि लोकानां कार्यार्थं मम सम्भवः}
{भद्रं तेऽस्तु गमिष्यामि यत एवाहमागतः} %7-104-18

\threelineshloka
{हद्गतो ह्यसि सम्प्राप्तो न मे तत्र विचारणा}
{मया हि सर्वकृत्येषु देवानां वशवर्तिनाम्}
{स्थातव्यं सर्वसंहार यथा ह्याह पितामहः} %7-104-19


॥इत्यार्षे श्रीमद्रामायणे वाल्मीकीये आदिकाव्ये उत्तरकाण्डे पितामहवाक्यकथनम् नाम चतुरधिकशततमः सर्गः ॥७-१०४॥
