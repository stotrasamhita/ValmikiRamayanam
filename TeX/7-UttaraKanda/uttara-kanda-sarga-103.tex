\sect{त्र्यधिकशततमः सर्गः — कालागमनम्}

\twolineshloka
{कस्यचित्त्वथ कालस्य रामे धर्मपरे स्थिते}
{कालस्तापसरूपेण राजद्वारमुपागमत्} %7-103-1

\twolineshloka
{सोऽब्रवील्लक्ष्मणं वाक्यं धृतिमन्तं यशश्विनम्}
{मां निवेदय रामाय सम्प्राप्तं कार्यगौरवात्} %7-103-2

\twolineshloka
{दूतोऽस्म्यतिबलस्याहं महर्षेरमितौजसः}
{रामं दिदृक्षुरायातः कार्येण हि महाबल} %7-103-3

\twolineshloka
{तस्य तद्वचनं श्रुत्वा सौमित्रिस्त्वरयाऽन्वितः}
{न्यवेदयत रामाय तापसं तं समागतम्} %7-103-4

\twolineshloka
{जयस्व राम धर्मेण उभौ लोकौ महाद्युते}
{दूतस्त्वां द्रष्टुमायातस्तपसा भास्करप्रभः} %7-103-5

\twolineshloka
{तद्वाक्यं लक्ष्मणेनोक्तं श्रुत्वा राम उवाच ह}
{प्रवेश्यतां मुनिस्तात महौजास्तस्य वाक्यधृत्} %7-103-6

\twolineshloka
{सौमित्रिस्तु तथेत्युक्त्वा प्रावेशयत तं मुनिम्}
{ज्वलन्तमेव तेजोभिः प्रदहन्तमिवांशुभिः} %7-103-7

\twolineshloka
{सोऽभिगम्य रघुश्रेष्ठं दीप्यपानं स्वतेजसा}
{ऋषिर्मधुरया वाचा वर्धस्वेत्याह राघवम्} %7-103-8

\twolineshloka
{तस्मै रामो महातेजाः पूजामर्घ्यपुरोगमाम्}
{ददौ कुशलमव्यग्रं प्रष्टुमेवोपचक्रमे} %7-103-9

\twolineshloka
{पृष्टश्च कुशलं तेन रामेण वदतां वरः}
{आसने काञ्चने दिव्ये निषसाद महायशाः} %7-103-10

\twolineshloka
{तमुवाच ततो रामः स्वागतं ते महामुने}
{प्रापयास्य च वाक्यानि यतो दूतस्त्वमागतः} %7-103-11

\twolineshloka
{चोदितो राजसिंहेन मुनिर्वाक्यमभाषत}
{द्वन्द्वमेतत्प्रवक्तव्यं हितं वै यद्यपेक्षसे} %7-103-12

\twolineshloka
{यः शृणोति निरीक्षेद्वा स वध्यो भविता तव}
{भवेद्वै मुनिमुख्यस्य वचनं यद्यपेक्षसे} %7-103-13

\twolineshloka
{स तथेति प्रतिज्ञाय रामो लक्ष्मणमब्रवीत्}
{द्वारि तिष्ठ महाबाहो प्रतिहारं विसर्जय} %7-103-14

\twolineshloka
{स मे वध्यः खलु भवेत् कथाद्वन्द्वं समीरितम्}
{ऋषेर्मम च सौमित्रे पश्येद्वा शृणुयाच्च यः} %7-103-15

\twolineshloka
{ततो निक्षिप्य काकुत्स्थो लक्ष्मणं द्वारि सङ्ग्रहम्}
{तमुवाच मुने वाक्यं कथयस्वेति राघवः} %7-103-16

\twolineshloka
{यत्ते मनीषितं वाक्यं येन वाऽसि समाहितः}
{कथयस्वाविशङ्कस्त्वं ममापि हृदि वर्तते} %7-103-17


॥इत्यार्षे श्रीमद्रामायणे वाल्मीकीये आदिकाव्ये उत्तरकाण्डे कालागमनम् नाम त्र्यधिकशततमः सर्गः ॥७-१०३॥
