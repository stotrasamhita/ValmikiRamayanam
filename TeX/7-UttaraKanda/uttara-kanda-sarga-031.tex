\sect{एकत्रिंशः सर्गः — रावणनर्मदावगाहः}

\twolineshloka
{ततो रामो महातेजा विस्मयात्पुनरेव हि}
{उवाच प्रणतो वाक्यमगस्त्यमृषिसत्तमम्} %7-31-1

\twolineshloka
{भगवन्राक्षसः क्रूरो यदाप्रभृति मेदिनीम्}
{पर्यटत्किं तदा लोकाः शून्या आसन्द्विजोत्तम} %7-31-2

\twolineshloka
{राजा वा राजमात्रो वा किं तदा नात्र कश्चन}
{धर्षणं यत्र न प्राप्तो रावणो राक्षसेश्वरः} %7-31-3

\twolineshloka
{उताहो हतवीर्यास्ते बभूवुः पृथिवीक्षितः}
{बहिष्कृता वरास्त्रैश्च बहवो निर्जिता नृपाः} %7-31-4

\twolineshloka
{राघवस्य वचः श्रुत्वा ह्यगस्त्यो भगवानृषिः}
{उवाच रामं प्रहसन्पितामह इवेश्वरम्} %7-31-5

\twolineshloka
{इत्येवं बाधमानस्तु पार्थिवान्पार्थिवर्षभ}
{चचार रावणो राम पृथिवीं पृथिवीपते} %7-31-6

\twolineshloka
{ततो माहिष्मतीं नाम पुरीं स्वर्गपुरीप्रभाम्}
{सम्प्राप्तो यत्र सान्निध्यं सदासीद्वसुरेतसः} %7-31-7

\twolineshloka
{तुल्य आसीन्नृपस्तस्य प्रभावाद्वसुरेतसः}
{अर्जुनो नाम यत्राग्निः शरकुण्डेशयः सदा} %7-31-8

\twolineshloka
{तमेव दिवसं सोऽथ हैहयाधिपतिर्बली}
{अर्जुनो नर्मदां रन्तुं गतः स्त्रीभिः सहेश्वरः} %7-31-9

\twolineshloka
{तमेव दिवसं सोऽथ रावणस्तत्र आगतः}
{रावणो राक्षसेन्द्रस्तु तस्यामात्यानपृच्छत} %7-31-10

\threelineshloka
{क्वार्जुनो नृपतिः शीघ्रं सम्यगाख्यातुमर्हथ}
{रावणोऽहमनुप्राप्तो युद्धेप्सुर्नृवरेण ह}
{ममागमनमप्यग्रे युष्माभिः सन्निवेद्यताम्} %7-31-11

\twolineshloka
{इत्येवं रावणेनोक्तास्तेऽमात्याः सुविपश्चितः}
{अब्रुवन्राक्षसपतिमसान्निध्यं महीपतेः} %7-31-12

\twolineshloka
{श्रुत्वा विश्रवसः पुत्रः पौराणामर्जुनं गतम्}
{अपसृत्यागतो विन्ध्यं हिमवत्सन्निभं गिरिम्} %7-31-13

\twolineshloka
{स तमभ्रमिवाविष्टमुद्भ्रन्तमिव मेदिनीम्}
{अपश्यद्रावणो विन्ध्यमालिखन्तमिवाम्बरम्} %7-31-14

\twolineshloka
{सहस्रशिखरोपेतं सिंहाध्युषितकन्दरम्}
{प्रपातपतितैस्तोयैः साट्टहासमिवाम्बुधिम्} %7-31-15

\twolineshloka
{देवदानवगन्धर्वैः साप्सरोगणकिन्नरैः}
{स्वस्त्रीभिः क्रीडमानैश्च स्वर्गभूतं महोच्छ्रयम्} %7-31-16

\twolineshloka
{नदीभिः स्यन्दमानाभिः स्फटिकप्रतिमं जलम्}
{फणाभिश्चलजिह्वाभिरनन्तमिव विष्ठितम्} %7-31-17

\twolineshloka
{उत्क्रामन्तं दरीवन्तं हिमवत्सन्निभं गिरिम्}
{पश्यमानस्ततो विन्ध्यं रावणो नर्मदां ययौ} %7-31-18

\onelineshloka
{चलोपलजलां पुण्यां पश्चिमोदधिगामिनीम्} %7-31-19

\twolineshloka
{महिषैः सृमरैः सिंहैः शार्दूलर्क्षगजोत्तमैः}
{उष्णाभितप्तैस्तृषितैः सङ्क्षोभितजलाशयाम्} %7-31-20

\twolineshloka
{चक्रवाकैः सकारण्डैः सहंसजलकुक्कुटैः}
{सारसैश्च सदा मत्तैः सुकूजद्भिः समावृताम्} %7-31-21

\twolineshloka
{फुल्लद्रुमकृतोत्तंसां चक्रवाकयुगस्तनीम्}
{विस्तीर्णपुलिनश्रोणीं हंसावलिसुमेखलाम्} %7-31-22

\threelineshloka
{पुष्परेण्वनुलिप्ताङ्गीं जलफेनामलांशुकाम्}
{जलावगाहसंस्पर्शां फुल्लोत्पलशुभेक्षणाम्}
{पुष्पकादवरुह्याशु नर्मदां सरितां वराम्} %7-31-23

\twolineshloka
{इष्टामिव वरां नारीमवगाह्य दशाननः}
{स तस्याः पुलिने रम्ये नानामुनिनिषेविते} %7-31-24

\twolineshloka
{उपोपविष्टैः सचिवैः सार्धं राक्षसपुङ्गवः}
{प्रख्याय नर्मदां सोऽथ गङ्गेयमिति रावणः} %7-31-25

\twolineshloka
{नर्मदादर्शजं हर्षमाप्तवान् राक्षसाधिपः}
{उवाच सचिवांस्तत्र सलीलं शुकसारणौ} %7-31-26

\twolineshloka
{एष रश्मिसहस्रेण जगत्कृत्वैव काञ्चनम्}
{तीक्ष्णतापकरः सूर्यो नभसोऽर्धं समाश्रितः} %7-31-27

\onelineshloka
{मामासीनं विदित्वैव चन्द्रायति दिवाकरः} %7-31-28

\twolineshloka
{नर्मदाजलशीतश्च सुगन्धिः श्रमनाशनः}
{मद्भयादनिलोऽप्यत्र वात्येष सुसमाहितः} %7-31-29

\twolineshloka
{इयं वापि सरिच्छ्रेष्ठा नर्मदा नर्मवर्धिनी}
{नक्रमीनविहङ्गोर्मिः सभयेवाङ्गना स्थिता} %7-31-30

\twolineshloka
{तद्भवन्तः क्षताः शस्त्रैर्नृपैरिन्द्रसमैर्युधि}
{चन्दनस्य रसेनेव रुधिरेण समुक्षिताः} %7-31-31

\threelineshloka
{ते यूयमवगाहध्वं नर्मदां शर्मदां शुभाम्}
{महापद्ममुखा मत्ता गङ्गामिव गहागजाः}
{अस्यां स्नात्वा महानद्यां पाप्मानं विप्रमोक्ष्यथ} %7-31-32

\twolineshloka
{अहमप्यद्य पुलिने शरदिन्दुसमप्रभे}
{पुष्पोपहारं शनकैः करिष्यामि कपर्दिनः} %7-31-33

\twolineshloka
{रावणेनैवमुक्तास्तु प्रहस्तशुकसारणाः}
{समहोदरधूम्राक्षा नर्मदां विजगाहिरे} %7-31-34

\twolineshloka
{राक्षसेन्द्रगजैस्तैस्तु क्षोभिता नर्मदा नदी}
{वामनाञ्जनपद्माद्यैर्गङ्गा इव महागजैः} %7-31-35

\twolineshloka
{ततस्ते राक्षसाः स्नात्वा नर्मदायां महाबलाः}
{उत्तीर्य पुष्पाण्याजह्रुर्बल्यर्थं रावणस्य तु} %7-31-36

\twolineshloka
{नर्मदापुलिने हृद्ये शुभ्राभ्रसदृशप्रभे}
{राक्षसैस्तु मुहूर्तेन कृतः पुष्पमयो गिरिः} %7-31-37

\twolineshloka
{पुष्पेषूपहृतेष्वेवं रावणो राक्षसेश्वरः}
{अवतीर्णो नदीं स्नातुं गङ्गामिव महागजः} %7-31-38

\twolineshloka
{तत्र स्नात्वा च विधिवज्जप्त्वा जप्यमनुत्तमम्}
{नर्मदासलिलात्तस्मादुत्ततार स रावणः} %7-31-39

\twolineshloka
{ततः क्लिन्नाम्बरं त्यक्त्वा शुक्लवस्त्रसमावृतम्}
{रावणं प्राञ्जलिं यान्तमन्वयुः सर्वराक्षसाः} %7-31-40

\threelineshloka
{तद्गतीवशमापन्ना मूर्तिमन्त इवाचलाः}
{यत्र यत्र च याति स्म रावणो राक्षसेश्वरः}
{जाम्बूनदमयं लिङ्गं तत्र तत्र स्म नीयते} %7-31-41

\twolineshloka
{वालुकावेदिमध्ये तु तल्लिङ्गं स्थाप्य रावणः}
{अर्चयामास गन्धैश्च पुष्पैश्चामृतगन्धिभिः} %7-31-42

\twolineshloka
{ततः सतामार्तिहरं परं वरं वरप्रदं चन्द्रमयूखभूषणम्}
{समर्चयित्वा स निशाचरो जगौ प्रसार्य हस्तान्प्रणनर्त चाग्रतः} %7-31-43


॥इत्यार्षे श्रीमद्रामायणे वाल्मीकीये आदिकाव्ये उत्तरकाण्डे रावणनर्मदावगाहः नाम एकत्रिंशः सर्गः ॥७-३१॥
