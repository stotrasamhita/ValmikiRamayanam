\sect{पञ्चसप्ततितमः सर्गः — शम्बूकविचयः}

\twolineshloka
{नारदस्य तु तद्वाक्यं श्रुत्वाऽमृतमयं तदा}
{प्रहर्षमतुलं लेभे लक्ष्मणं चेदमब्रवीत्} %7-75-1

\twolineshloka
{गच्छ सौम्य द्विजश्रेष्ठं समाश्वासय सुव्रतम्}
{बालस्य तु शरीरं तत्तैलद्रोण्यां निधापय} %7-75-2

\twolineshloka
{गन्धैश्च परमोदारैस्तैलैश्चापि सुगन्धिभिः}
{यथा न क्षीयते बालस्तथा सौम्य विधीयताम्} %7-75-3

\twolineshloka
{यथा शरीरो बालस्य गुप्तः सन् शिष्टकर्मणः}
{विपत्तिः परिभेदो वा न भवेच्च तथा कुरु} %7-75-4

\twolineshloka
{एवमादिश्य काकुत्स्थो लक्ष्मणं शुभलक्षणम्}
{मनसा पुष्पकं दध्यावागच्छेति महायशाः} %7-75-5

\twolineshloka
{इङ्गितं स तु विज्ञाय पुष्पको हेमभूषितः}
{आजगाम मुहूर्तेन समीपे राघवस्य वै} %7-75-6

\twolineshloka
{सोऽब्रवीत्प्रणतो भूत्वा अयमस्मि नराधिप}
{वश्यस्तव महाबाहो किङ्करः समुपस्थितः} %7-75-7

\twolineshloka
{भाषितं रुचिरं श्रुत्वा पुष्पकस्य नराधिपः}
{अभिवाद्य महर्षींस्तान् विमानं सोऽध्यरोहत} %7-75-8

\twolineshloka
{धनुर्गृहीत्वा तूणी च खड्गं च रुचिरप्रभम्}
{निक्षिप्य नगरे वीरौ सौमित्रिभरतावुभौ} %7-75-9

\twolineshloka
{प्रायात्प्रतीचीं हरितं विचिन्वंश्च ततस्ततः}
{उत्तरामगमच्छ्रीमान्दिशं हिमवता वृताम्} %7-75-10

\twolineshloka
{अपश्यमानस्तत्रापि स्वल्पमप्यथ दुष्कृतम्}
{पूर्वामपि दिशं सर्वामथोऽपश्यन्नराधिपः} %7-75-11

\twolineshloka
{प्रविशुद्धसमाचारामादर्शतलनिर्मलाम्}
{पुष्पकस्थो महाबाहुस्तदापश्यन्नराधिपः} %7-75-12

\twolineshloka
{दक्षिणां दिशमाक्रामत्ततो राजर्षिनन्दनः}
{शैवलस्योत्तरे पार्श्वे ददर्श सुमहत्सरः} %7-75-13

\twolineshloka
{तस्मिन्सरसि तप्यन्तं तापसं सुमहत्तपः}
{ददर्श राघवः श्रीमाँल्लम्बमानमधोमुखम्} %7-75-14

\twolineshloka
{राघवस्तमुपागम्य तप्यन्तं तप उत्तमम्}
{उवाच स तदा वाक्यं धन्यस्त्वमसि सुव्रत} %7-75-15

\twolineshloka
{कस्यां योन्यां तपोवृद्ध वर्तसे दृढविक्रम}
{कौतूहलात्त्वां पृच्छामि रामो दाशरथिर्ह्यहम्} %7-75-16

\threelineshloka
{कोऽर्थो मनीषितस्तुभ्यं स्वर्गलाभः परोऽथवा}
{वराश्रयो यदर्थं त्वं तपस्यसि सुदुष्करम्}
{यमाश्रित्य तपस्तप्तं श्रोतुमिच्छामि तापस} %7-75-17

\twolineshloka
{ब्राह्मणो वाऽसि भद्रं ते क्षत्रियो वाऽसि दुर्जयः}
{वैश्यस्तृतीयवर्णो वा शूद्रो वा सत्यवाग्भव} %7-75-18

\twolineshloka
{इत्येवमुक्तः स नराधिपेन ह्यवाक्छिरा दाशरथाय तस्मै}
{उवाच जातिं नृपपुङ्गवाय यत्कारणे चैव तपःप्रयत्नः} %7-75-19


॥इत्यार्षे श्रीमद्रामायणे वाल्मीकीये आदिकाव्ये उत्तरकाण्डे शम्बूकविचयः नाम पञ्चसप्ततितमः सर्गः ॥७-७५॥
