\sect{एकषष्ठितमः सर्गः — लवणत्राणप्रार्थना}

\twolineshloka
{एवं ब्रुवद्भिर्ऋषिभिः काकुत्स्थो वाक्यमब्रवीत्}
{किं कार्यं ब्रूत मुनयो भयं तावदपैतु वः} %7-61-1

\twolineshloka
{तथा ब्रुवति काकुत्स्थे भार्गवो वाक्यमब्रवीत्}
{भयानां शृणु यन्मूलं देशस्य च नरेश्वर} %7-61-2

\twolineshloka
{पूर्वं कृतयुगे राजन्दैतेयः सुमहाबलः}
{लोलापुत्रोऽभवज्ज्येष्ठो मधुर्नाम महासुरः} %7-61-3

\twolineshloka
{ब्रह्मण्यश्च शरण्यश्च बुद्ध्या च परिनिष्ठितः}
{सुरैश्च परमोदारैः प्रीतिस्तस्यातुलाऽभवत्} %7-61-4

\twolineshloka
{स मधुर्वीर्यसम्पन्नो धर्मे च सुसमाहितः}
{बहुवर्षसहस्राणि रुद्रप्रीत्याऽकरोत्तपः} %7-61-5

\twolineshloka
{रुद्रः प्रीतोऽभवत्तस्मै वरं दातुं ययौ च सः}
{बहुमानाच्च रुद्रेण दत्तस्तस्याद्भुतो वरः} %7-61-6

\twolineshloka
{शूलं शूलाद्विनिष्कृष्य महावीर्यं महाप्रभम्}
{ददौ महात्मा सुप्रीतो वाक्यं चैतदुवाच ह} %7-61-7

\twolineshloka
{त्वयायमतुलो धर्मो मत्प्रसादकरः कृतः}
{प्रीत्या परमया युक्तो ददाम्यायुधमुत्तमम्} %7-61-8

\twolineshloka
{यावत्सुरैश्च विप्रैश्च न विरुध्येर्महासुर}
{तावच्छूलं तवेदं स्यादन्यथा नाशमेप्यति} %7-61-9

\twolineshloka
{यश्च त्वामभियुञ्जीत युद्धाय विगतज्वरः}
{तं शूलो भस्मसात्कृत्वा पुनरेष्यति ते करम्} %7-61-10

\twolineshloka
{एवं रुद्राद्वरं लब्ध्वा भूय एव महासुरः}
{प्रणिपत्य महादेवं वाक्यमेतदुवाच ह} %7-61-11

\twolineshloka
{भगवन्मम वंशस्य शूलमेतदनुत्तमम्}
{भवेत्तु सततं देव सुराणामीश्वरो ह्यसि} %7-61-12

\twolineshloka
{तं ब्रुवाणं मधुं देवः सर्वभूतपतिः शिवः}
{प्रत्युवाच महातेजा नैतदेवं भविष्यति} %7-61-13

\twolineshloka
{मा भूत्ते विफला वाणी मत्प्रसादकृता शुभा}
{भवतः पुत्रमेकं तु शूलमेतद्भजिष्यते} %7-61-14

\twolineshloka
{यावत्करस्थः शूलोऽयं भविष्यति सुतस्य ते}
{अवध्यः सर्वभूतानां शूलहस्तो भविष्यति} %7-61-15

\twolineshloka
{एवं मधुर्वरं लब्ध्वा देवात्सुमहदद्भुतम्}
{भवनं सोऽसुरश्रेष्ठः कारयामास सुप्रभम्} %7-61-16

\twolineshloka
{तस्य पत्नी महाभागा प्रिया कुम्भीनसीति या}
{विश्वावसोरपत्यं सा ह्यनलायां महाप्रभा} %7-61-17

\twolineshloka
{तस्याः पुत्रो महावीर्यो लवणो नाम दारुणः}
{बाल्यात्प्रभृति दुष्टात्मा पापान्येव समाचरत्} %7-61-18

\twolineshloka
{तं पुत्रं दुर्विनीतं तु दृष्ट्वा क्रोधसमन्वितः}
{मधुः स शोकमापेदे न चैनं किञ्चिदब्रवीत्} %7-61-19

\twolineshloka
{स विहाय त्विमं लोकं प्रविष्टो वरुणालयम्}
{शूलं निवेश्य लवणे वरं तस्मै न्यवेदयत्} %7-61-20

\twolineshloka
{स प्रभावेण शूलस्य दौरात्म्येनात्मनस्तथा}
{सन्तापयति लोकांस्त्रीन्विशेषेण च तापसान्} %7-61-21

\twolineshloka
{एवंप्रभावो लवणः शूलं चैव तथाविधम्}
{श्रुत्वा प्रमाणं काकुत्स्थ त्वं हि नः परमा गतिः} %7-61-22

\twolineshloka
{बहवः पार्थिवा राम भयार्तैर्ऋषिभिः पुरा}
{अभयं याचिता वीर त्रातारं न च विद्महे} %7-61-23

\threelineshloka
{ते वयं रावणं श्रुत्वा हतं सबलवाहनम्}
{त्रातारं विद्महे तात नान्यं भुवि नराधिपम्}
{तत्परित्रातुमिच्छामो लवणाद्भयपीडितान्} %7-61-24

\twolineshloka
{इति राम निवेदितं तु ते भयजं कारणमुत्थितं च यत्}
{विनिवारयितुं भवान्क्षमः कुरु तं काममहीनविक्रम} %7-61-25


॥इत्यार्षे श्रीमद्रामायणे वाल्मीकीये आदिकाव्ये उत्तरकाण्डे लवणत्राणप्रार्थना नाम एकषष्ठितमः सर्गः ॥७-६१॥
