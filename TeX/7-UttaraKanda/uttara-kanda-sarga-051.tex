\sect{एकपञ्चाशः सर्गः — दुर्वासोवाक्यकथनम्}

\twolineshloka
{तथा सञ्चोदितः सूतो लक्ष्मणेन महात्मना}
{तद्वाक्यमृषिणा प्रोक्तं व्याहर्तुमुपचक्रमे} %7-51-1

\twolineshloka
{पुरा नाम्ना हि दुर्वासा अत्रेः पुत्रो महामुनिः}
{वसिष्ठस्याश्रमे पुण्ये वार्षिक्यं समुवास ह} %7-51-2

\twolineshloka
{तमाश्रमं महातेजाः पिता ते सुमहायशाः}
{पुरोहितं महात्मानं दिदृक्षुरगमस्त्वयम्} %7-51-3

\twolineshloka
{स दृष्ट्वा सूर्यसङ्काशं ज्वलन्तमिव तेजसा}
{उपविष्टं वसिष्ठस्य सव्यपार्श्वे महामुनिम्} %7-51-4

\twolineshloka
{तौ मुनी तापसश्रेष्ठौ विनीतो ह्यभिवादयत्}
{स ताभ्यां पूजितो राजा स्वागतेनासनेन च पाद्येन फलमूलैश्च उवास मुनिभिः सह} %7-51-5

\twolineshloka
{तेषां तत्रोपविष्टानां तास्ताः सुमधुराः कथाः}
{बभूवुः परमर्षीणां मध्यादित्यगतेऽहनि} %7-51-6

\twolineshloka
{ततः कथायां कस्याञ्चित्प्राञ्जलिः प्रग्रहो नृपः}
{उवाच तं महात्मानमत्रेः पुत्रं तपोधनम्} %7-51-7

\twolineshloka
{भगवन्किं प्रमाणेन मम वंशो भविष्यति}
{किमायुश्च हि मे रामः पुत्राश्चान्ये किमायुषः} %7-51-8

\twolineshloka
{रामस्य च सुता ये स्युस्तेषामायुः कियद्भवेत्}
{काम्यया भगवन्बूहि वंशस्यास्य गतिं मम} %7-51-9

\twolineshloka
{तच्छ्रुत्वा व्याहृतं वाक्यं राज्ञो दशरथस्य च}
{दुर्वासाः सुमहातेजा व्याहर्तुमुपचक्रमे} %7-51-10

\twolineshloka
{शृणु राजन्पुरावृत्तं तदा देवासुरे युधि}
{दैत्याः सुरैर्भर्त्स्यमाना भृगुपत्नीं समाश्रिताः} %7-51-11

\onelineshloka
{तया दत्ताभयास्तत्र न्यवसन्नभयास्तदा} %7-51-12

\twolineshloka
{तया परिगृहीतांस्तान्दृष्ट्वा क्रुद्धः सुरेश्वरः}
{चक्रेण शितधारेण भृगुपत्न्याः शिरोऽहरत्} %7-51-13

\twolineshloka
{ततस्तां निहतां दृष्ट्वा पत्नीं भृगुकुलोद्वहः}
{शशाप सहसा क्रुद्धो विष्णुं रिपुकुलार्दनम्} %7-51-14

\twolineshloka
{यस्मादवध्यां मे पत्नीमवधीः क्रोधमूर्च्छितः}
{तस्मात्त्वं मानुषे लोके जनिष्यसि जनार्दन} %7-51-15

\onelineshloka
{तत्र पत्नीवियोगं त्वं प्राप्स्यसे बहुवार्षिकम्} %7-51-16

\onelineshloka
{शापाभिहतचेतास्स स्वात्मना भावितोऽभवत्} %7-51-17

\threelineshloka
{अर्चयामास तं देवं भृगुः शापेन पीडितः}
{तपसाराधितो देवो ह्यब्रवीद्भक्तवत्सलः}
{लोकानां संहितार्थं तु तं शापं ग्राह्यमुक्तवान्} %7-51-18

\twolineshloka
{इति शप्तो महातेजा भृगुणा पूर्वजन्मनि}
{इहागतो हि पुत्रत्वं तव पार्थिवसत्तम} %7-51-19

\twolineshloka
{राम इत्यभिविख्यातस्त्रिषु लोकेषु मानद}
{तत्फलं प्राप्स्यते चापि भृगुशापकृतं महत्} %7-51-20

\twolineshloka
{अयोध्यायाः पती रामो दीर्घकालं भविष्यति}
{सुखिनश्च समृद्धाश्च भविष्यन्त्यस्य येऽनुगाः} %7-51-21

\twolineshloka
{दश वर्षसहस्राणि दश वर्षशतानि च}
{रामो राज्यमुपासित्वा ब्रह्मलोकं गमिष्यति} %7-51-22

\twolineshloka
{समृद्धैश्चाश्वमेधैश्च इष्ट्वा परमदुर्जयः}
{राजवंशांश्च बहुशो बहून्संस्थापयिष्यति} %7-51-23

\twolineshloka
{द्वौ पुत्रौ तु भविष्येते सीतायां राघवस्य तु}
{अन्यत्र न त्वयोध्यायां सत्यमेतन्नसंशयः} %7-51-24

\onelineshloka
{सीतायाश्च ततः पुत्रावभिषेक्ष्यति राघवः} %7-51-25

\twolineshloka
{स सर्वमखिलं राज्ञो वंशस्याह गतागतम्}
{आख्याय सुमहातेजास्तूष्णीमासीन्महामुनिः} %7-51-26

\twolineshloka
{तूष्णीम्भूते तदा तस्मिन्राजा दशरथो मुनौ}
{अभिवाद्य महात्मानौ पुनरायात्पुरोत्तमम्} %7-51-27

\twolineshloka
{एतद्वचो मया तत्र मुनिना व्याहृतं पुरा}
{श्रुतं हृदि च निक्षिप्तं नान्यथा तद्भविष्यति} %7-51-28

\twolineshloka
{एवं गते न सन्तापं कर्तुमर्हसि राघव}
{सीतार्थे राघवार्थे वा दृढो भव नरोत्तम} %7-51-29

\twolineshloka
{श्रुत्वा तु व्याहृतं वाक्यं सूतस्य परमाद्भुतम्}
{प्रहर्षमतुलं लेभे साधु साध्विति चाब्रवीत्} %7-51-30

\twolineshloka
{ततः संवदतोरेवं सूतलक्ष्मणयोः पथि}
{अस्तमर्के गते वासं केशिन्यां तावथोषतुः} %7-51-31


॥इत्यार्षे श्रीमद्रामायणे वाल्मीकीये आदिकाव्ये उत्तरकाण्डे दुर्वासोवाक्यकथनम् नाम एकपञ्चाशः सर्गः ॥७-५१॥
