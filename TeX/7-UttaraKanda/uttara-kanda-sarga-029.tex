\sect{एकोनत्रिंशः सर्गः — वासवग्रहणम्}

\twolineshloka
{ततस्तमसि सञ्जाते सर्वे ते देवराक्षसाः}
{अयुद्ध्यन्त बलोन्मत्ताः सूदयन्तः परस्परम्} %7-29-1

\twolineshloka
{ततस्तु देवसैन्येन राक्षसानां बृहद्बलम्}
{दशांशं स्थापितं युद्धे शेषं नीतं यमक्षयम्} %7-29-2

\twolineshloka
{तस्मिंस्तु तामसे युद्धे सर्वे ते देवराक्षसाः}
{अन्योन्यं नाभ्यजानन्त युध्यमानाः परस्परम्} %7-29-3

\twolineshloka
{इन्द्रश्च रावणश्चैव रावणिश्च महाबलः}
{तस्मिंस्तमोजालवृते मोहमीयुर्न ते त्रयः} %7-29-4

\twolineshloka
{स तु दृष्ट्वा बलं सर्वं रावणो निहतं क्षणात्}
{क्रोधमभ्यगमत्तीव्रं महानादं च मुक्तवान्} %7-29-5

\twolineshloka
{क्रोधात्सूतं च दुर्धर्षः स्यन्दनस्थमुवाच ह}
{परसैन्यस्य मध्येन यावदन्तो नयस्व माम्} %7-29-6

\twolineshloka
{अद्यैतान् त्रिदशान्सर्वान्विक्रमैः समरे स्वयम्}
{नानाशस्त्रैर्महाघोरैर्नयामि यमसादनम्} %7-29-7

\threelineshloka
{अहमिन्द्रं वधिष्यामि धनदं वरुणं यमम्}
{त्रिदशान्विनिहत्याथ स्वयं स्थास्याम्यथोपरि}
{विषादो न च कर्तव्यः शीघ्रं वाहय मे रथम्} %7-29-8

\onelineshloka
{द्विः खलु त्वां ब्रवीम्यद्य यावदन्तो नयस्व माम्} %7-29-9

\twolineshloka
{अयं स नन्दनोद्देशो यत्र वर्तामहे वयम्}
{नय मामद्य तत्र त्वमुदयो यत्र पर्वतः} %7-29-10

\twolineshloka
{तस्य तद्वचनं श्रुत्वा तुरगान्स मनोजवान्}
{आदिदेशाथ शत्रणां मध्येनैव च सारथिः} %7-29-11

\twolineshloka
{तस्य तं निश्चयं ज्ञात्वा शक्रो देवेश्वरस्तदा}
{रथस्थः समरस्थस्तान्देवान्वाक्यमथाब्रवीत्} %7-29-12

\twolineshloka
{सुराः शृणुत मद्वाक्यं यत्तावन्मम रोचते}
{जीवन्नेव दशग्रीवः साधु रक्षो निगृह्यताम्} %7-29-13

\twolineshloka
{एष ह्यतिबलः सैन्यै रथेन पवनौजसा}
{गमिष्यति प्रवृद्धोर्मिः समुद्र इव पर्वणि} %7-29-14

\twolineshloka
{नह्येष हन्तुं शक्योऽद्य वरदानात्सुनिर्भयः}
{तद्ग्रहीष्यामहे रक्षो यत्ता भवत संयुगे} %7-29-15

\twolineshloka
{यथा बलौ निरुद्धे च त्रैलोक्यं भुज्यते मया}
{एवमेतस्य पापस्य निरोधो मम रोचते} %7-29-16

\twolineshloka
{ततोऽन्यं देशमास्थाय शक्रः सन्त्यज्य रावणम्}
{अयुध्यत महाराज राक्षसांस्त्रासयन्रणे} %7-29-17

\twolineshloka
{उत्तरेण दशग्रीवः प्रविवेशानिवर्तकः}
{दक्षिणेन तु पार्श्वेन प्रविवेश शतक्रतुः} %7-29-18

\twolineshloka
{ततः स योजनशतं प्रविष्टो राक्षसाधिपः}
{देवतानां बलं सर्वं शरवर्षैरवाकिरत्} %7-29-19

\twolineshloka
{ततः शक्रो निरीक्ष्याथ प्रनष्टं तु स्वकं बलम्}
{न्यवर्तयदसम्भ्रान्तः समावृत्य दशाननम्} %7-29-20

\twolineshloka
{एतस्मिन्नन्तरे नादो मुक्तो दानवराक्षसैः}
{हा हताः स्म इति ग्रस्तं दृष्ट्वा शक्रेण रावणम्} %7-29-21

\twolineshloka
{ततो रथं समास्थाय रावणिः क्रोधमूर्च्छितः}
{तत्सैन्यमतिसङ्क्रुद्धः प्रविवेश सुदारुणम्} %7-29-22

\twolineshloka
{तां प्रविश्य महामायां प्राप्तां गोपतिना पुरा}
{प्रविवेश सुसंरब्धस्तत्सैन्यं समभिद्रवत्} %7-29-23

\twolineshloka
{स सर्वा देवतास्त्यक्त्वा शक्रमेवाभ्यधावत}
{महेन्द्रश्च महातेजा नापश्यच्च सुतं रिपोः} %7-29-24

\twolineshloka
{विमुक्तकवचस्तत्र वध्यमानोऽपि रावणिः}
{त्रिदशैः सुमहावीर्यैर्न चकार च किञ्चन} %7-29-25

\twolineshloka
{स मातलिं समायान्तं ताडयित्वा शरोत्तमैः}
{महेन्द्रं बाणवर्षेण भूय एवाभ्यवाकिरत्} %7-29-26

\twolineshloka
{ततस्त्यक्त्वा रथं शक्रो विससर्ज च सारथिम्}
{ऐरावतं समारुह्य मृगयामास रावणिम्} %7-29-27

\twolineshloka
{स तत्र मायाबलवानदृश्योऽथान्तरिक्षगः}
{इन्द्रं मायापरिक्षिप्तं कृत्वा स प्राद्रवच्छरैः} %7-29-28

\twolineshloka
{स तं यदा परिश्रान्तमिन्द्रं जज्ञेऽथ रावणिः}
{तदैनं मायया बद्ध्वा स्वसैन्यमभितोऽनयत्} %7-29-29

\twolineshloka
{तं तु दृष्ट्वा बलात्तेन नीयमानं महारणात्}
{महेन्द्रममराः सर्वे किं नु स्यादित्यचिन्तयन्} %7-29-30

\twolineshloka
{दृश्यते न स मायावी शक्रजित्समितिञ्जयः}
{विद्यावानपि येनेन्द्रो मायया नीयते बलात्} %7-29-31

\twolineshloka
{एतस्मिन्नन्तरे क्रुद्धाः सर्वे सुरगणास्तदा}
{रावणं विमुखीकृत्य शरवर्षैरवाकिरन्} %7-29-32

\twolineshloka
{रावणस्तु समासाद्य आदित्यांश्च वसूंस्तदा}
{न शशाक स सङ्ग्रामे योद्धुं शत्रुभिरर्दितः} %7-29-33

\twolineshloka
{स तं दृष्ट्वा परिम्लानं प्रहारैर्जर्जरीकृतम्}
{रावणिः पितरं युद्धे दर्शनस्थोऽब्रवीदिदम्} %7-29-34

\twolineshloka
{आगच्छ तात गच्छामो रणकर्म निवर्तताम्}
{जितं नो विदितं तेऽस्तु स्वस्थो भव गतज्वरः} %7-29-35

\twolineshloka
{अयं हि सुरसैन्यस्य त्रैलोक्यस्य च यः प्रभुः}
{स गृहीतो दैवबलाद्भग्नदर्पाः सुराः कृताः} %7-29-36

\twolineshloka
{यथेष्टं भुङ्क्ष्व लोकांस्त्रीन्निगृह्यारातिमोजसा}
{वृथा किं ते श्रमेणेह युद्धमद्य तु निष्फलम्} %7-29-37

\twolineshloka
{ततस्ते दैवतगणा निवृत्ता रणकर्मणः}
{तच्छ्रुत्वा रावणेर्वाक्यं स्वस्थचेताः बभूव ह} %7-29-38

\twolineshloka
{अथ स रणविगतज्वरः प्रभुर्विजयमवाप्य निशाचराधिपः}
{स्वभवनमभितो जगाम हृष्टः स्वसुतमवाप्य च वाक्यमब्रवीत्} %7-29-39

\twolineshloka
{अतिबलसदृशैः पराक्रमैस्तैर्मम कुलमानविवर्धनं कृतम्}
{यदयमतुल्यबलस्त्वयाद्य वै त्रिदशपतिस्त्रिदशाश्च निर्जिताः} %7-29-40

\twolineshloka
{नय रथमधिरोप्य वासवं नगरमितो व्रज सेनया वृतस्त्वम्}
{अहमपि तव गच्छतो द्रुतं सह सचिवैरनुयामि पृष्ठतः} %7-29-41

\twolineshloka
{अथ स बलवृतः सवाहनस्त्रिदशपतिं परिगृह्य रावणिः}
{स्वभवनमभिगम्य वीर्यवान्कृतसमरान्विससर्ज राक्षसान्} %7-29-42


॥इत्यार्षे श्रीमद्रामायणे वाल्मीकीये आदिकाव्ये उत्तरकाण्डे वासवग्रहणम् नाम एकोनत्रिंशः सर्गः ॥७-२९॥
