\sect{द्विपञ्चाशः सर्गः — रामसमाधानम्}

\twolineshloka
{तत्र तां रजनीमुष्य केशिन्यां रघुनन्दनः}
{प्रभाते पुनरुत्थाय लक्ष्मणः प्रययौ तदा} %7-52-1

\twolineshloka
{ततोऽर्धदिवसे प्राप्ते प्रविवेश महारथः}
{अयोध्यां रत्नसम्पूर्णां हृष्टपुष्टजनावृताम्} %7-52-2

\twolineshloka
{सौमित्रिस्तु परं दैन्यं जगाम सुमहामतिः}
{रामपादौ समासाद्य वक्ष्यामि किमहं गतः} %7-52-3

\twolineshloka
{तस्यैवं चिन्तयानस्य भवनं शशिसन्निभम्}
{रामस्य परमोदारं पुरस्तान्समदृश्यत} %7-52-4

\twolineshloka
{राज्ञस्तु भवनद्वारि सोऽवतीर्य रथोत्तमात्}
{अवाङ्मुखो दीनमनाः प्रविवेशानिवारितः} %7-52-5

\twolineshloka
{स दृष्ट्वा राघवं दीनमासीनं परमासने}
{नेत्राभ्यामश्रुपूर्णाभ्यां ददर्शाग्रजमग्रतः} %7-52-6

\twolineshloka
{जग्राह चरणौ तस्य लक्ष्मणो दीनचेतनः}
{उवाच दीनया वाचा प्राञ्जलिः सुसमाहितः} %7-52-7

\twolineshloka
{आर्यस्याज्ञां पुरस्कृत्य विसृज्य जनकात्मजाम्}
{गङ्गातीरे यथोद्दिष्टे वाल्मीकेराश्रमे शुचौ} %7-52-8

\twolineshloka
{तत्र तां च शुभाचारामाश्रमान्ते यशस्विनीम्}
{पुनरप्यागतो वीर पादमूलमुपासितुम्} %7-52-9

\twolineshloka
{मा शुचः पुरुषव्याघ्र कालस्य गतिरीदृशी}
{त्वद्विधा न हि शोचन्ति बुद्धिमन्तो मनस्विनः} %7-52-10

\twolineshloka
{सर्वे क्षयान्ता निचयाः पतनान्ताः समुच्छ्रयाः}
{संयोगा विप्रयोगान्ता मरणान्तं च जीवितम्} %7-52-11

\twolineshloka
{तस्मात्पुत्रेषु दारेषु मित्रेषु च धनेषु च}
{नातिप्रसङ्गः कर्तव्यो विप्रयोगो हि तैर्ध्रुवम्} %7-52-12

\twolineshloka
{शक्तस्त्वमात्मनाऽऽत्मानं विनेतुं मनसैव हि}
{लोकान्सर्वांश्च काकुत्स्थ किं पुनः शोकमात्मनः} %7-52-13

\twolineshloka
{नेदृशेषु विमुह्यन्ति त्वद्विधाः पुरुषर्षभाः}
{अपवादः स किल ते पुनरेष्यति राघव} %7-52-14

\twolineshloka
{यदर्थं मैथिली त्यक्ता अपवादभयान्नृप}
{सोऽपवादः पुरे राजन्भविष्यति न संशयः} %7-52-15

\twolineshloka
{स त्वं पुरुषशार्दूल धैर्येण सुसमाहितः}
{त्यजैनां दुर्बलां बुद्धिं सन्तापं मा कुरुष्व ह} %7-52-16

\twolineshloka
{एवमुक्तः स काकुत्स्थो लक्ष्मणेन महात्मना}
{उवाच परया प्रीत्या सौमित्रिं मित्रवत्सलः} %7-52-17

\twolineshloka
{एवमेतन्नरश्रेष्ठ यथा वदसि लक्ष्मण}
{परितोषश्च मे वीर मम कार्यानुशासने} %7-52-18

\twolineshloka
{निवृत्तिश्चागता सौम्य सन्तापश्च निराकृतः}
{भवद्वाक्यैः सुरुचिरैरनुनीतोऽस्मि लक्ष्मण} %7-52-19


॥इत्यार्षे श्रीमद्रामायणे वाल्मीकीये आदिकाव्ये उत्तरकाण्डे रामसमाधानम् नाम द्विपञ्चाशः सर्गः ॥७-५२॥
