\sect{एकाधिकशततमः सर्गः — तक्षपुष्कलनिवेशः}

\twolineshloka
{श्रुत्वा सेनापतिं प्राप्तं भरतं केकयाधिपः}
{युधाजिद्गार्ग्यसहितं परां प्रीतिमुपागमत्} %7-101-1

\twolineshloka
{स निर्ययौ जनौघेन महता केकयाधिपः}
{त्वरमाणोऽभिचक्राम गन्धर्वान्कामरूपिणः} %7-101-2

\twolineshloka
{भरतश्च युधाजिच्च समेतौ लघुविक्रमैः}
{गन्धर्वनगरं प्राप्तौ सबलौ सपदानुगौ} %7-101-3

\twolineshloka
{श्रुत्वा तु भरतं प्राप्तं गन्धर्वास्ते समागताः}
{योद्धुकामा महावीर्या व्यनदन् वै समन्ततः} %7-101-4

\twolineshloka
{ततः समभवद्युद्धं तुमुलं रोमहर्षणम्}
{सप्तरात्रं महाभीमं न चान्यतरयोर्जयः} %7-101-5

\twolineshloka
{खड्गशक्तिधनुर्ग्राहा नद्यः शोणितसंस्रवाः}
{नृकलेवरवाहिन्यः प्रवृत्ताः सर्वतोदिशम्} %7-101-6

\twolineshloka
{ततो रामानुजः क्रुद्धः कालस्यास्त्रं सुदारुणम्}
{संवर्तं नाम भरतो गन्धर्वेष्वभ्यचोदयत्} %7-101-7

\twolineshloka
{ते बद्धाः कालपाशेन संवर्तेन विदारिताः}
{क्षणेनाभिहतास्तेन तिस्रः कोट्यो महात्मनाम्} %7-101-8

\twolineshloka
{तं घातं घोरसङ्काशं न स्मरन्ति दिवौकसः}
{निमेषान्तरमात्रेण तादृशानां महात्मनाम्} %7-101-9

\twolineshloka
{हतेषु तेषु सर्वेषु भरतः केकयीसुतः}
{निवेशयामास तदा समृद्धे द्वे पुरोत्तमे} %7-101-10

\twolineshloka
{तक्षं तक्षशिलायां तु पुष्कलं पुष्कलावते}
{तक्षपुष्कलोपरि टिप्पणी गन्धर्वदेशे रुचिरे गान्धारविषये च सः} %7-101-11

\twolineshloka
{धनरत्नौघसङ्कीर्णे काननैरुपशोभिते}
{अन्योन्यसङ्घर्षकृते स्पर्धया गुणविस्तरैः} %7-101-12

\twolineshloka
{उभे सुरुचिरप्रख्ये व्यवहारैरकिल्बिषैः}
{उद्यानयानसम्पूर्णे सुविभक्तान्तरापणे} %7-101-13

\twolineshloka
{उभे पुरवरे रम्ये विस्तरैरुपशोभिते}
{गृहमुख्यैः सुरुचिरैर्विमानसमवर्णिभिः} %7-101-14

\twolineshloka
{शोभिते शोभनीयैश्च देवायतनविस्तरैः}
{तालैस्तमालैस्तिलकैर्वकुलैरुपशोभिते} %7-101-15

\twolineshloka
{निवेश्य पञ्चभिर्वर्षैर्भरतो राघवानुजः}
{पुनरायान्महाबाहुरयोध्यां केकयीसुतः} %7-101-16

\twolineshloka
{सोऽभिवाद्य महात्मानं साक्षाद्धर्ममिवापरम्}
{राघवं भरतः श्रीमान्ब्रह्माणमिव वासवः} %7-101-17

\twolineshloka
{शशंस च यथावृत्तं गन्धर्ववधमुत्तमम्}
{निवेशनं च देशस्य श्रुत्वा प्रीतोऽस्य राघवः} %7-101-18


॥इत्यार्षे श्रीमद्रामायणे वाल्मीकीये आदिकाव्ये उत्तरकाण्डे तक्षपुष्कलनिवेशः नाम एकाधिकशततमः सर्गः ॥७-१०१॥
