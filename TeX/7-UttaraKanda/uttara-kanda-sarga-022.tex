\sect{द्वाविंशः सर्गः — यमजयः}

\twolineshloka
{स तस्य तु महानादं श्रुत्वा वैवस्वतः प्रभुः}
{शत्रुं विजयिनं मेने स्वबलस्य च सङ्क्षयम्} %7-22-1

\twolineshloka
{स हि योधान्हतान्मत्वा क्रोधसंरक्तलोचनः}
{अब्रवीत्त्वरितं सूतं रथोऽयमुपनीयताम्} %7-22-2

\threelineshloka
{तस्य सूतस्तदा दिव्यमुपस्थाप्य महारथम्}
{स्थितः स च महातेजा ह्यध्यारोहत तं रथम्}
{प्राशमुद्गरहस्तश्च मृत्युस्तस्याग्रतः स्थितः} %7-22-3

\threelineshloka
{येन सङ्क्षिप्यते सर्वं त्रैलोक्यमिदमव्ययम्}
{कालदण्डस्तु पार्श्वस्थो मूर्तिमानस्य चाभवत्}
{यमप्रहरणं दिव्यं तेजसा ज्वलदग्निमत्} %7-22-4

\onelineshloka
{तस्य पार्श्वेषु निच्छिद्राः कालपाशाः प्रतिष्ठिताः} %7-22-5

\onelineshloka
{पावकस्पर्शसङ्काशः स्थितो मूर्तश्च मुद्गरः} %7-22-6

\twolineshloka
{ततो लोकत्रयं क्षुब्धमकम्पन्त दिवौकसः}
{कालं दृष्ट्वा तथा क्रुद्धं सर्वलोकभयावहम्} %7-22-7

\twolineshloka
{ततः प्रचोदयन्सूतस्तानश्वान्रुचिरप्रभान्}
{प्रययौ भीमसन्नादो यत्र रक्षःपतिः स्थितः} %7-22-8

\twolineshloka
{मुहूर्तेन यमं ते तु हया हरिहयोपमाः}
{प्रापयन्मनसस्तुल्या यत्र तत्प्रस्तुतं रणम्} %7-22-9

\twolineshloka
{दृष्ट्वा तथैव विकृतं रथं मृत्युसमन्वितम्}
{सचिवा राक्षसेन्द्रस्य सहसा विप्रदुद्रुवुः} %7-22-10

\twolineshloka
{लघुसत्त्वतया ते हि नष्टसञ्ज्ञा भयार्दिताः}
{नेह योद्धुं समर्थाः स्म इत्युक्त्वा प्रययुर्दिशः} %7-22-11

\twolineshloka
{स तु तं तादृशं दृष्ट्वा रथं लोकभयावहम्}
{नाक्षुभ्यत दशग्रीवो न चापि भयमाविशत्} %7-22-12

\twolineshloka
{स तु रावणमासाद्य व्यसृजच्छक्तितोमरान्}
{यमो मर्माणि सङ्क्रुद्धो रावणस्योपकृन्तत} %7-22-13

\twolineshloka
{रावणस्तु ततः स्वस्थः शरवर्षं मुमोच ह}
{तस्मिन्वैवस्वतरथे तोयवर्षमिवाम्बुदः} %7-22-14

\twolineshloka
{ततो महाशक्तिशरैः पात्यमानैर्महोरसि पात्यमानो}
{नाशक्नोत्प्रतिकर्तुं स राक्षसः शल्यपीडितः} %7-22-15

\twolineshloka
{एवं नानाप्रहरणैर्यमेनामित्रकर्षिणा}
{सप्तरात्रं कृतः सङ्ख्ये विसञ्ज्ञो विमुखो रिपुः} %7-22-16

\twolineshloka
{तदासीत्तुमुलं युद्धं यमराक्षसयोर्द्वयोः}
{जयमाकाङ्क्षतोर्वीर समरेष्वनिवर्तिनोः} %7-22-17

\twolineshloka
{ततो देवाः सगन्धर्वाः सिद्धाश्च परमर्षयः}
{प्रजापतिं पुरस्कृत्य समेतास्तद्रणाजिरम्} %7-22-18

\twolineshloka
{संवर्त इव लोकानां क्रुध्यतोरभवत्तदा}
{राक्षसानां च मुख्यस्य प्रेतानामीश्वरस्य च} %7-22-19

\twolineshloka
{राक्षसेन्द्रोऽपि विस्फार्य चापमिन्द्राशनिप्रभम्}
{निरन्तरमिवाकाशं कुर्वन्बाणांस्ततोऽसृजत्} %7-22-20

\twolineshloka
{मृत्युं चतुर्भिर्विशिखैः सूतं सप्तभिरार्दयत्}
{यमं शतसहस्रेण शीघ्रं मर्मस्वताडयत्} %7-22-21

\twolineshloka
{ततः क्रुद्धस्य वदनाद्यमस्य समजायत}
{ज्वालामाली सनिःश्वासः सधूमः कोपपावकः} %7-22-22

\twolineshloka
{तदाश्चर्यमथो दृष्ट्वा देवदानवसन्निधौ}
{प्रहर्षितौ सुसंरब्धौ मृत्युकालौ बभूवतुः} %7-22-23

\threelineshloka
{ततो मृत्युः क्रुद्धतरो वैवस्वतमभाषत}
{मुञ्च मां समरे यावद्धन्मीमं पापराक्षसम्}
{नैषा रक्षो भवेदद्य मर्यादा हि निसर्गतः} %7-22-24

\twolineshloka
{हिरण्यकशिपुः श्रीमान्नमुचिः शम्बरस्तथा}
{विसन्धिर्धूमकेतुश्च बलिर्वैरोचनोऽपि च} %7-22-25

\twolineshloka
{दम्भुर्दैत्यमहाराजो वृत्रो बाणस्तथैव च}
{राजर्षयः शास्त्रविदो गन्धर्वाः समहोरगाः} %7-22-26

\twolineshloka
{ऋषयः पन्नगा दैत्या यक्षाश्चाप्यप्सरोगणाः}
{युगान्तपरिवर्ते च पृथिवी समहार्णवा} %7-22-27

\twolineshloka
{क्षयं नीता महाराज सपर्वतसरिद्द्रुमा}
{एते चान्ये च बहवो बलवन्तो दुरासदाः} %7-22-28

\onelineshloka
{विनिपन्ना मया दृष्टाः किमुतायं निशाचरः} %7-22-29

\twolineshloka
{मुञ्चं मां साधु धर्मज्ञ यावदेनं निहन्म्यहम्}
{नहि कश्चिन्मया दृष्टो बलवानपि जीवति} %7-22-30

\twolineshloka
{बलं मम न खल्वेतन्मर्यादैषा निसर्गतः}
{स दृष्टो न मया कालं मुहुर्तमापि जीवति} %7-22-31

\twolineshloka
{तस्यैवं वचनं श्रुत्वा धर्मराजः प्रतापवान्}
{अब्रवीत्तत्र तं मुत्युं त्वं तिष्ठैनं निहन्म्यहम्} %7-22-32

\twolineshloka
{ततः संरक्तनयनः क्रुद्धो वैवस्वतः प्रभुः}
{कालदण्डममोघं तु तोलयामास पाणिना} %7-22-33

\twolineshloka
{यस्य पार्श्वेषु निखिला कालपाशाः प्रतिष्ठिताः}
{पावकाशनिसङ्काशो मुद्गरो मूर्तिमान्स्थितः} %7-22-34

\twolineshloka
{दर्शनादेव यः प्राणान्प्राणिनामपकर्षति}
{किं पुनः स्पृश्यमानस्य पात्यमानस्य वा पुनः} %7-22-35

\twolineshloka
{स ज्वालापरिवारस्तु निर्दहन्निव राक्षसम्}
{तेन स्पृष्टो बलवता महाप्रहरणोऽस्फुरत्} %7-22-36

\twolineshloka
{ततो विदुद्रुवुः सर्वे तस्मात्रस्ता रणाजिरे}
{सुराश्च क्षुभिताः सर्वे दृष्ट्वा दण्डोद्यतं यमम्} %7-22-37

\twolineshloka
{तस्मिन्प्रहर्तुकामे तु यमे दण्डेन रावणम्}
{यमं पितामहः साक्षाद्दर्शयित्वेदमब्रवीत्} %7-22-38

\twolineshloka
{वैवस्वत महाबाहो न खल्वमितविक्रम}
{न हन्तव्यस्त्वया तेन दण्डेनैव निशाचरः} %7-22-39

\twolineshloka
{वरः खलु मयैतस्मै दत्तस्त्रिदशपुङ्गव}
{स त्वया नानृतः कार्यो यन्मया व्याहृतं वचः} %7-22-40

\twolineshloka
{यो हि मामनृतं कुर्याद्देवो वा मानुषोऽपि वा}
{त्रैलोक्यमनृतं तेन कृतं स्यान्नात्र संशयः} %7-22-41

\twolineshloka
{क्रुद्धेन विप्रमुक्तोऽयं निर्विशेषं प्रियाप्रिये}
{प्रजाः संहरते रौद्रो लोकत्रयभयावहः} %7-22-42

\twolineshloka
{अमोघो ह्येष सर्वेषां प्राणिनाममितप्रभः}
{कालदण्डो मया सृष्टः पूर्वं मृत्युपुरस्कृतः} %7-22-43

\twolineshloka
{तन्न खल्वेष ते सौम्य पात्यो रावणमूर्धनि}
{नह्यस्मिन्पतिते कश्चिन्मुहूर्तमपि जीवति} %7-22-44

\twolineshloka
{यदि ह्यन्मिन्निपतिते न म्रियेतैष राक्षसः}
{म्रियते वा दशग्रीवस्तदाप्युभयतोऽनृतम्} %7-22-45

\twolineshloka
{तन्निवर्तय लङ्केशं द्दण्डमेतं समुद्यतम्}
{सत्यं च मां कुरुष्वाद्य लोकांस्त्वं यद्यवेक्षसे} %7-22-46

\twolineshloka
{एवमुक्तस्तु धर्मात्मा प्रत्युवाच यमस्तदा}
{एष व्यावर्तितो दण्डः प्रभविष्णुर्हि नो भवान्} %7-22-47

\twolineshloka
{किं न्विदानीं मया शक्यं कर्तुं रणगतेन हि}
{न मया यद्ययं शक्यो हन्तुं वरपुरस्कृतः} %7-22-48

\twolineshloka
{एष तस्मात्प्रणश्यामि दर्शनादस्य रक्षसः}
{इत्युक्त्वा सरथः साश्वस्तत्रैवान्तरधीयत} %7-22-49

\twolineshloka
{दशग्रीवस्तु तं जित्वा नाम विश्राव्य चात्मनः}
{आरुह्य पुष्पकं भूयो निष्क्रान्तो यमसादनात्} %7-22-50

\twolineshloka
{स तु वैवस्वतो देवैः सह ब्रह्मपुरोगमैः}
{जगाम त्रिदिवं हृष्टो नारदश्च महामुनिः} %7-22-51


॥इत्यार्षे श्रीमद्रामायणे वाल्मीकीये आदिकाव्ये उत्तरकाण्डे यमजयः नाम द्वाविंशः सर्गः ॥७-२२॥
