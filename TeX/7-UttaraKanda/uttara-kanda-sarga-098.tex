\sect{अष्टनवतितमः सर्गः — रामकोपोपशमः}

\twolineshloka
{रसातलं प्रविष्टायां वैदेह्यां सर्ववानराः}
{चुक्रुशुः साधु साध्वीति मुनयो रामसन्निधौ} %7-98-1

\twolineshloka
{दण्डकाष्ठमवष्टभ्य बाष्पव्याकुलितेक्षणः}
{अवाक्छिरा दीनमना रामो ह्यासीत्सुदुःखितः} %7-98-2

\twolineshloka
{स रुदित्वा चिरं कालं बहुशो बाष्पमुत्सृजन्}
{क्रोधशोकसमाविष्टो रामो वचनमब्रवीत्} %7-98-3

\twolineshloka
{अभूतपूर्वं शोकं मे मनः स्प्रष्टुमिवेच्छति}
{पश्यतो मे यथा नष्टा सीता श्रीरिव रूपिणी} %7-98-4

\twolineshloka
{साऽदर्शनं पुरा सीता लङ्कापारे महोदधेः}
{ततश्चापि मयानीता किं पुनर्वसुधातलात्} %7-98-5

\twolineshloka
{वसुधे देवि भवति सीता निर्यात्यतां मम}
{दर्शयिष्यामि वा रोषं यथा मामवगच्छसि} %7-98-6

\twolineshloka
{कामं श्वश्रूर्ममैव त्वं त्वत्सकाशाद्धि मैथिली}
{कर्षता हलहस्तेन जनकेनोद्धृता पुरा} %7-98-7

\twolineshloka
{तस्मान्निर्यात्यतां सीता विवरं वा प्रयच्छ मे}
{पाताले नाकपृष्ठे वा वसेयं सहितस्तया} %7-98-8

\twolineshloka
{आनय त्वं हि तां सीतां मत्तोऽहं मैथिलीकृते}
{न मे दास्यसि चेत्सीतां यथारूपां महीतले} %7-98-9

\twolineshloka
{सपर्वतवनां कृत्स्नां विधमिष्यामि ते स्थितम्}
{नाशयिष्याम्यहं भूमिं सर्वमापो भवत्विह} %7-98-10

\twolineshloka
{एवं ब्रुवाणे काकुत्स्थे क्रोधशोकसमन्विते}
{ब्रह्मा सुरगणैः सार्धमुवाच रघुनन्दनम्} %7-98-11

\twolineshloka
{राम राम न सन्तापं कर्तुमर्हसि सुव्रत}
{स्मर त्वं पूर्वकं भावं मन्त्रं चामित्रकर्शन} %7-98-12

\twolineshloka
{न खलु त्वां महाबाहो स्मारयेयमनुत्तमम्}
{इमं मुहूर्तं दुर्धर्ष स्मर त्वं जन्म वैष्णवम्} %7-98-13

\twolineshloka
{सीता हि विमला साध्वी तव पूर्वपरायणा}
{नागलोकं सुखं प्रायात्त्वदाश्रयतपोबलात्} %7-98-14

\twolineshloka
{स्वर्गे ते सङ्गमो भूयो भविष्यति न संशयः}
{अस्यास्तु परिषन्मध्ये यद्ब्रवीमि निबोध तत्} %7-98-15

\twolineshloka
{एतदेव हि काव्यं ते काव्यानामुत्तमं श्रुतम्}
{सर्वं विस्तरतो राम व्याख्यास्यति न संशयः} %7-98-16

\twolineshloka
{जन्मप्रभृति ते वीर सुखदुःखोपसेवनम्}
{भविष्यदुत्तरं चेह सर्वं वाल्मीकिना कृतम्} %7-98-17

\twolineshloka
{आदिकाव्यमिदं राम त्वयि सर्वं प्रतिष्ठितम्}
{नह्यन्योऽर्हति काव्यानां यशोभाग्राघवादृते} %7-98-18

\twolineshloka
{श्रुतं ते पूर्वमेतद्धि मया सर्वं सुरैः सह}
{दिव्यमद्भुतरूपं च सत्यवाक्यमनावृतम्} %7-98-19

\twolineshloka
{स त्वं पुरुषशार्दूल धर्मेण सुसमाहितः}
{शेषं भविष्यं काकुत्स्थ काव्यं रामायणं शृणु} %7-98-20

\twolineshloka
{उत्तरं नाम काव्यस्य शेषमत्र महायशः}
{तच्छृणुष्व महातेज ऋषिभिः सार्धमुत्तमम्} %7-98-21

\twolineshloka
{न खल्वन्येन काकुत्स्थ श्रोतव्यमिदमुत्तमम्}
{परमम् ऋषिणा वीर त्वयैव रघुनन्दन} %7-98-22

\twolineshloka
{एतावदुक्त्वा वचनं ब्रह्मा त्रिभुवनेश्वरः}
{जगाम त्रिदिवं देवो देवैः सह सबान्धवैः} %7-98-23

\threelineshloka
{ये च तत्र महात्मान ऋषयो ब्राह्मलौकिकाः}
{ब्रह्मणा समनुज्ञाता न्यवर्तन्त महौजसः}
{उत्तरं श्रोतुमनसो भविष्यं यच्च राघवे} %7-98-24

\twolineshloka
{ततो रामः शुभां वाणीं देवदेवस्य भाषिताम्}
{श्रुत्वा परमतेजस्वी वाल्मीकिमिदमब्रवीत्} %7-98-25

\twolineshloka
{भगवन् श्रोतुमनस ऋषयो ब्राह्मलौकिकाः}
{भविष्यदुत्तरं यन्मे श्वोभूते सम्प्रवर्तताम्} %7-98-26

\threelineshloka
{एवं विनिश्चयं कृत्वा सम्प्रगृह्य कुशीलवौ}
{तं जनौघं विसृज्याथ पर्णशालामुपागमत्}
{तामेव शोचतः सीतां सा व्यतीयाय शर्वरी} %7-98-27


॥इत्यार्षे श्रीमद्रामायणे वाल्मीकीये आदिकाव्ये उत्तरकाण्डे रामकोपोपशमः नाम अष्टनवतितमः सर्गः ॥७-९८॥
