\sect{एकोनशततमः सर्गः — कौसल्यादिकालधर्मः}

\twolineshloka
{रजन्यां तु प्रभातायां समानीय महामुनीन्}
{गीयतामविशङ्काभ्यां रामः पुत्रावुवाच ह} %7-99-1

\twolineshloka
{ततः समुपविष्टेषु ब्रह्मर्षिषु महात्मसु}
{भविष्यदुत्तरं काव्यं जगतुस्तौ कुशीलवौ} %7-99-2

\twolineshloka
{प्रविष्टायां तु सीतायां भूतलं सत्यसम्पदा}
{तस्यावसाने यज्ञस्य रामः परमदुर्मनाः} %7-99-3

\twolineshloka
{अपश्यमानो वैदेहीं मेने शून्यमिदं जगत्}
{शोकेन परमायस्तो न शान्तिं मनसाऽगमत्} %7-99-4

\twolineshloka
{विसृज्य पार्थिवान्सर्वानृक्षवानरराक्षसान्}
{जनौघं विप्रमुख्यानां वित्तपूर्वं विसृज्य च} %7-99-5

\twolineshloka
{एवं समाप्य यज्ञं तु विधिवत्स तु राघवः}
{ततो विसृज्य तान्सर्वान्रामो राजीवलोचनः} %7-99-6

\twolineshloka
{हृदि कृत्वा तदा सीतामयोध्यां प्रविवेश ह}
{इष्टयज्ञो नरपतिः पुत्रद्वयसमन्वितः} %7-99-7

\twolineshloka
{न सीतायाः परां भार्यां वव्रे स रघुनन्दनः}
{यज्ञे यज्ञे च पत्न्यर्थं जानकी काञ्चनी भवत्} %7-99-8

\twolineshloka
{दशवर्षसहस्राणि वाजिमेघानथाकरोत्}
{वाजपेयान्दशगुणांस्तथा बहुसुवर्णकान्} %7-99-9

\twolineshloka
{अग्निष्टोमातिरात्राभ्यां गोसवैश्च महाधनैः}
{ईजे क्रतुभिरन्यैश्च स श्रीमानाप्तदक्षिणैः} %7-99-10

\twolineshloka
{एवं स कालः सुमहान्राज्यस्थस्य महात्मनः}
{धर्मे प्रयतमानस्य व्यतीयाद्राघवस्य तु} %7-99-11

\twolineshloka
{अनुरञ्जन्ति राजानमहन्यहनि राघवम्}
{ऋक्षवानररक्षांसि स्थिता रामस्य शासने} %7-99-12

\twolineshloka
{काले वर्षति पर्जन्यः सुभिक्षं विमला दिशः}
{हृष्टपुष्टजनाकीर्णं पुरं जनपदास्तथा} %7-99-13

\twolineshloka
{नाकाले म्रियते कश्चिन्न व्याधिः प्राणिनां तथा}
{नानर्थो विद्यते कश्चिद्रामे राज्यं प्रशासति} %7-99-14

\twolineshloka
{अथ दीर्घस्य कालस्य राममाता यशस्विनी}
{पुत्रपौत्रैः परिवृता कालधर्ममुपागमत्} %7-99-15

\twolineshloka
{अन्वियाय सुमित्रा च कैकेयी च यशस्विनी}
{धर्मं कृत्वा बहुविधं त्रिदिवे पर्यवस्थिता} %7-99-16

\twolineshloka
{सर्वाः प्रमुदिताः स्वर्गे राज्ञा दशरथेन च}
{समागता महाभागाः सर्वधर्मं च लेभिरे} %7-99-17

\twolineshloka
{तासां रामो महादानं काले काले प्रयच्छति}
{मातऽणामविशेषेण ब्राह्मणेषु तपस्विषु} %7-99-18

\twolineshloka
{पित्र्याणि ब्रह्मरत्नानि यज्ञान्परमदुस्तरान्}
{चकार रामो धर्मात्मा पितऽन्देवान्विवर्धयन्} %7-99-19

\twolineshloka
{एवं वर्षसहस्राणि बहून्यथ ययुः सुखम्}
{यज्ञैर्बहुविधं धर्मं वर्धयानस्य सर्वदा} %7-99-20


॥इत्यार्षे श्रीमद्रामायणे वाल्मीकीये आदिकाव्ये उत्तरकाण्डे कौसल्यादिकालधर्मः नाम एकोनशततमः सर्गः ॥७-९९॥
