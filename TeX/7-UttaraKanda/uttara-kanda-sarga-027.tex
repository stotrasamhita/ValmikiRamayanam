\sect{सप्तविंशः सर्गः — सुमालिवधः}

\twolineshloka
{कैलासं लङ्घयित्वाथ दशग्रीवः स रावणः}
{आससाद महातेजा इन्द्रलोकं निशाचरः} %7-27-1

\twolineshloka
{तस्य राक्षससैन्यस्य समन्तादुपयास्यतः}
{देवलोकं ययौ शब्दो मथ्यमानार्णवोपमः} %7-27-2

\twolineshloka
{श्रुत्वा तु रावणं प्राप्तमिन्द्रश्चलित आसनात्}
{अब्रवीत्तत्र तान् देवान् सर्वानेव समागतान्} %7-27-3

\twolineshloka
{आदित्यान्सवसून्रुद्रान्विश्वान्साध्यान्मरुद्गणान्}
{सज्जीभवत युद्धार्थं रावणस्य दुरात्मनः} %7-27-4

\twolineshloka
{एवमुक्तास्तु शक्रेण देवाः शक्रसमा युधि}
{सन्नह्य सुमहासत्त्वा युद्धश्रद्धासमन्विताः} %7-27-5

\twolineshloka
{स तु दीनः परित्रस्तो महेन्द्रो रावणं प्रति}
{विष्णोः समीपमागत्य वाक्यमेतदुवाच ह} %7-27-6

\twolineshloka
{विष्णो कथं करिष्यामि महावीर्यपराक्रमः}
{असौ हि बलवद्रक्षो युद्धार्थमभिवर्तते} %7-27-7

\twolineshloka
{वरप्रदानाद्बलवान्न खल्वन्येन हेतुना}
{तत्तु सत्यवचः कार्यं यदुक्तं पद्मयोनिना} %7-27-8

\twolineshloka
{तद्यथा नमुचिर्वृत्रो बलिर्नरकशम्बरौ}
{त्वद्बलं समवष्टभ्य मया दग्धास्तथा कुरु} %7-27-9

\twolineshloka
{नह्यन्यो देवदेवेश त्वामृते मधुसूदन}
{गतिः परायणं नास्ति त्रैलोक्ये सचराचरे} %7-27-10

\twolineshloka
{त्वं हि नारायणः श्रीमान्पद्मनाभः सनातनः}
{त्वयेमे स्थापिता लोकाः शक्रश्चाहं सुरेश्वरः} %7-27-11

\twolineshloka
{त्वया सृष्टमिदं सर्वं त्रैलोक्यं सचराचरम्}
{त्वामेव भगवन्सर्वे प्रविशन्ति युगक्षये} %7-27-12

\twolineshloka
{तदाचक्ष्व यथा तत्त्वं देवदेव मम स्वयम्}
{अपि चक्रसहायस्त्वं योत्स्यसे रावणं प्रभो} %7-27-13

\twolineshloka
{एवमुक्तः स शक्रेण देवो नारायणः प्रभुः}
{अब्रवीन्न परित्रासः कर्तव्यः श्रूयतां च मे} %7-27-14

\twolineshloka
{न तावदेष दुष्टात्मा शक्यो जेतुं सुरासुरैः}
{हन्तुं चापि समासाद्य वरदानेन दुर्जयः} %7-27-15

\twolineshloka
{सर्वथा तु महत्कर्म करिष्यति बलोत्कटः}
{राक्षसः पुत्रसहितो दृष्टमेतन्निसर्गतः} %7-27-16

\twolineshloka
{यत्तु मां त्वमभाषिष्ठ युद्ध्यस्वेति सुरेश्वर}
{नाहं तं प्रतियोत्स्यामि रावणं राक्षसं युधि} %7-27-17

\twolineshloka
{नाहत्वा समरे शत्रुं विष्णुः प्रतिनिवर्तते}
{दुर्लभश्चैव कामोऽद्य वरगुप्ताद्धि रावणात्} %7-27-18

\twolineshloka
{प्रतिजाने च देवेन्द्र त्वत्समीपे शतक्रतो}
{भविताऽस्मि यथाऽस्याहं रक्षसो मृत्युकारणम्} %7-27-19

\threelineshloka
{अहमेव निहन्ताऽस्मि रावणं सपुरःसरम्}
{देवता नन्दयिष्यामि ज्ञात्वा कालमुपागतम्}
{एतत्ते कथितं तत्त्वं देवराज शचीपते} %7-27-20

\onelineshloka
{युद्ध्यस्व विगतत्रासः सर्वैः सार्धं महाबल} %7-27-21

\threelineshloka
{यामि ज्ञात्वा कालमुपागतम्}
{ततो रुद्राः सहादित्या वसवो मरुतोऽश्वनौ}
{सन्नद्धा निर्ययुस्तूर्णं राक्षसानभितः पुरात्} %7-27-22

\twolineshloka
{एतस्मिन्नन्तरे नादः शुश्रुवे रजनीक्षये}
{तस्य रावणसैन्यस्य प्रयुद्धस्य समन्ततः} %7-27-23

\twolineshloka
{ते प्रयुद्धा महावीर्या ह्यन्योन्यमभिवीक्ष्य वै}
{सङ्ग्राममेवाभिमुखा ह्यभवर्तन्त हृष्टवत्} %7-27-24

\twolineshloka
{ततो दैवतसैन्यनां सङ्क्षोभः समजायत}
{तदक्षयं महासैन्यं दृष्ट्वा समरमूर्धनि} %7-27-25

\twolineshloka
{ततो युद्धं समभवेद्देवदानवरक्षसाम्}
{घोरं तुमुलनिर्ह्रादं नानाप्रहरणोद्यतम्} %7-27-26

\twolineshloka
{एतस्मिन्नन्तरे शूरा राक्षसा घोरदर्शनाः}
{युद्धार्थं समवर्तन्त सचिवा रावणस्य ते} %7-27-27

\twolineshloka
{मारीचश्च प्रहस्तश्च महापार्श्वमहोदरौ}
{अकम्पनो निकुम्भश्च शुकः सारण एव च} %7-27-28

\twolineshloka
{संह्लादो धूमकेतुश्च महादंष्ट्रो घटोदरः}
{जम्बुमाली महाह्रादो विरूपाक्षश्च राक्षसः} %7-27-29

\twolineshloka
{सुप्तघ्नो यज्ञकोपश्च दुर्मुखो दूषणः खरः}
{त्रिशिराः करवीराक्षः सूर्यशत्रुश्च राक्षसः} %7-27-30

\twolineshloka
{महाकायोऽतिकायश्च देवान्तकनरान्तकौ}
{एतैः सर्वैः परिवृतो महावीर्यो महाबलः} %7-27-31

\threelineshloka
{रावणस्यार्यकः सैन्यं सुमाली प्रविवेश ह}
{स दैवतगणान्सर्वान्नानाप्रहरणैः शितैः}
{व्यध्वंसयत्सुसङ्क्रुद्धो वायुर्जलधरानिव} %7-27-32

\twolineshloka
{तद्दैवतबलं राम हन्यमानं निशाचरैः}
{प्रणुन्नं सर्वतो दिग्भ्यः सिंहनुन्ना मृगा इव} %7-27-33

\twolineshloka
{एतस्मिन्नन्तरे शूरो वसूनामष्टमो वसुः}
{सावित्र इति विख्यातः प्रविवेश रणाजिरम्} %7-27-34

\twolineshloka
{तथाऽऽदित्यौ महावीर्यौ त्वष्टा पूषा च दंशितौ}
{निंर्भयौ सह सैन्येन तदा प्राविशतां रणे} %7-27-35

\twolineshloka
{ततो युद्धं समभवत्सुराणां सह राक्षसैः}
{क्रुद्धानां रक्षसां कीर्तिं समरेष्वनिवर्तिनाम्} %7-27-36

\twolineshloka
{ततस्ते राक्षसाः सर्वे विबुधान्समरे स्थितान्}
{नानाप्रहरणैर्घोरैर्जघ्नुः शतसहस्रशः} %7-27-37

\twolineshloka
{देवाश्च राक्षसान्घोरान्महाबलपराक्रमान्}
{समरे विमलैः शस्त्रैरुपनिन्युर्यमक्षयम्} %7-27-38

\twolineshloka
{एतस्मिन्नन्तरे राम सुमाली नाम राक्षसः}
{नानाप्रहरणैः क्रुद्धस्तत्सैन्यं सोऽभ्यवर्तत} %7-27-39

\twolineshloka
{स दैवतबलं सर्वं नानाप्रहणैः शितैः}
{व्यध्वंसयत सङ्क्रुद्धो वायुर्जलधरं यथा} %7-27-40

\twolineshloka
{ते महाबाणवर्षैश्च शूलप्रासैः सुदारुणैः}
{हन्यमानाः सुराः सर्वे न व्यतिष्ठन्त संहताः} %7-27-41

\twolineshloka
{ततो विद्राव्यमाणेषु दैवतेषु सुमालिना}
{वसूनामष्टमः क्रुद्धः सावित्रो वै व्यवस्थितः} %7-27-42

\twolineshloka
{संवृतः स्वैरथानीकैः प्रहरन्तं निशाचरम्}
{विक्रमेण महातेजा वारयामास संयुगे} %7-27-43

\onelineshloka
{ततस्तयोर्महायुद्धमभवद्रोमहर्षणम्} %7-27-44

\threelineshloka
{सुमालिनो वसोश्चैव समरेष्वनिवर्तिनोः}
{ततस्तस्य महाबाणैर्वसुना सुमहात्मना}
{निहतः पन्नगरथः क्षणेन विनिपातितः} %7-27-45

\twolineshloka
{हत्वा तु संयुगे तस्य रथं बाणशतैश्चितम्}
{गदां तस्य वधार्थाय वसुर्जग्राह पाणिना} %7-27-46

\twolineshloka
{ततः प्रगृह्य दीप्ताग्रां कालदण्डोपमां गदाम्}
{तां मूर्ध्नि पातयामास सावित्रो वै सुमालिनः} %7-27-47

\twolineshloka
{सा तस्योपरि चोल्काभा पतन्तीव बभौ गदा}
{इन्द्रप्रमुक्ता गर्जन्ती गिराविव महाशनिः} %7-27-48

\twolineshloka
{तस्य नैवास्थि न शिरो न मांसं ददृशे तदा}
{गदया भस्मतां नीतं निहतस्य रणाजिरे} %7-27-49

\threelineshloka
{तं दृष्ट्वा निहतं सङ्ख्ये राक्षसास्ते समन्ततः}
{व्यद्रवन्सहिताः सर्वे क्रोशमानाः परस्परम्}
{विद्राव्यमाणा वसुना राक्षसा नावतस्थिरे} %7-27-50


॥इत्यार्षे श्रीमद्रामायणे वाल्मीकीये आदिकाव्ये उत्तरकाण्डे सुमालिवधः नाम सप्तविंशः सर्गः ॥७-२७॥
