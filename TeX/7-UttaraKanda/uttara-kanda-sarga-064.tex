\sect{चतुःषष्ठितमः सर्गः — शत्रुघ्नप्रस्थानम्}

\twolineshloka
{एवमुक्त्वा च काकुत्स्थं प्रशस्य च पुनः पुनः}
{पुनरेवापरं वाक्यमुवाच रघुनन्दनः} %7-64-1

\twolineshloka
{इमान्यश्वसहस्राणि चत्वारि पुरुषर्षभ}
{रथानां द्वे सहस्रे च गजानां शतमुत्तमम्} %7-64-2

\twolineshloka
{अन्तरा पणवीथ्यश्च नानापण्योपशोभिताः}
{अनुगच्छन्तु काकुत्स्थं तथैव नटनर्तकाः} %7-64-3

\twolineshloka
{हिरण्यस्य सुवर्णस्य नियुतं पुरुषर्षभ}
{आदाय गच्छ शत्रुघ्न पर्याप्तधनवाहनः} %7-64-4

\twolineshloka
{बलं च सुभृतं वीर हृष्टपुष्टमनुद्धतम्}
{सम्भाषासम्प्रदानेन रञ्जयस्व नगेत्तम} %7-64-5

\twolineshloka
{न ह्यर्थास्तत्र तिष्ठन्ति न दारा न च बान्धवाः}
{सुप्रीतो भृत्यवर्गश्च यत्र तिष्ठसि राघव} %7-64-6

\twolineshloka
{ततो हृष्टजनाकीर्णां प्रस्थाप्य महतीं चमूम्}
{एक एव धनुष्पाणिर्गच्छ त्वं मधुनो वनम्} %7-64-7

\twolineshloka
{यथा त्वां न प्रजानाति गच्छन्तं युद्धकाङ्क्षिणम्}
{लवणस्तु मधोः पुत्रस्तथा गच्छेरशङ्कितम्} %7-64-8

\twolineshloka
{न तस्य मूत्युरन्योऽस्ति कश्चिद्धि पुरुषर्षभ}
{दर्शनं योऽभिगच्छेत स वध्यो लवणेन हि} %7-64-9

\twolineshloka
{स हि ग्रीष्मोऽपयाते तु वर्षारात्र उपागते}
{हन्यास्त्वं लवणं सौम्य सहि कालोऽस्य दुर्मतेः} %7-64-10

\twolineshloka
{महर्षींस्तु पुरत्कृत्य प्रयान्तु तव सौनिकाः}
{यथा ग्रीष्मावशेषेण तरेयुर्जाह्नवीजलम्} %7-64-11

\twolineshloka
{तत्र स्थाप्य बलं सर्वं नदीतीरे समाहितः}
{अग्रतो धनुषा सार्धं गच्छ त्वं लघुविक्रमः} %7-64-12

\twolineshloka
{एवमुक्तस्तु रामेण शत्रुघ्नस्तान्महाबलान्}
{सेनामुख्यान्समानीय ततो वाक्यमुवाच ह} %7-64-13

\twolineshloka
{एते वो गणिता वासा यत्र तत्र निवत्स्यथ}
{स्थातव्यं चाविरोधेन यथा बाधा न कस्यचित्} %7-64-14

\twolineshloka
{तथा तांस्तु समाज्ञाप्य प्रस्थाप्य च महद्बलम्}
{कौसल्यां च सुमित्रां च कैकेयीं चाभ्यवादयत्} %7-64-15

\twolineshloka
{रामं प्रदक्षिणीकृत्य शिरसाऽभिप्रणम्य च}
{रामेण चाभ्यनुज्ञातः शत्रुघ्नः शत्रुतापनः} %7-64-16

\threelineshloka
{लक्ष्मणं भरतं चैव प्रणिपत्य कृताञ्जलिः}
{पुरोहितं वसिष्ठं च शत्रुघ्नः प्रयतात्मवान्}
{प्रदक्षिणमथो कृत्वा निर्जगाम महाबलः} %7-64-17

\twolineshloka
{प्रस्थाप्य सेनामथ सोऽग्रतस्तदा गजेन्द्रवाजिप्रवरौघसङ्कुलाम्}
{उपास मासं तु नरेन्द्रपार्श्वतस्त्वथ प्रयातो रघुवंशवर्धनः} %7-64-18


॥इत्यार्षे श्रीमद्रामायणे वाल्मीकीये आदिकाव्ये उत्तरकाण्डे शत्रुघ्नप्रस्थानम् नाम चतुःषष्ठितमः सर्गः ॥७-६४॥
