\sect{पञ्चपञ्चाशः सर्गः — निमिवसिष्ठशापः}

\twolineshloka
{एष ते नृगशापस्य विस्तरोऽभिहितो मया}
{यद्यस्ति श्रवणे श्रद्धा शृणुष्वेहापरां कथाम्} %7-55-1

\twolineshloka
{एवमुक्तस्तु रामेण सौमित्रिः पुनरब्रवीत्}
{तृप्तिराश्चर्यभूतानां कथानां नास्ति मे नृप} %7-55-2

\twolineshloka
{लक्ष्मणेनैवमुक्तस्तु राम इक्ष्वाकुनन्दनः}
{कथां परमधर्मिष्ठां व्याहर्तुमुपचक्रमे} %7-55-3

\twolineshloka
{आसीद्राजा निमिर्नाम इक्ष्वाकूणां महात्मनाम्}
{पुत्रो द्वादशमो वीर्ये धर्मे च परिनिष्ठितः} %7-55-4

\twolineshloka
{स राजा वीर्यसम्पन्नः पुरं देवपुरोपमम्}
{निवेशयामास तदा अभ्याशे गौतमस्य तु} %7-55-5

\twolineshloka
{पुरस्य सुकृतं नाम वैजयन्तमिति श्रुतम्}
{निवेशं यत्र राजर्षिर्निमिश्चक्रे महायशाः} %7-55-6

\twolineshloka
{तस्य बुद्धिः समुत्पन्ना निवेश्य सुमहापुरम्}
{यजेयं दीर्घसत्रेण पितुः प्रह्लादयन्मनः} %7-55-7

\twolineshloka
{ततः पितरमामन्त्र्य इक्ष्वाकुं हि मनोः सुतम्}
{वसिष्ठं वरयामास पूर्वं ब्रह्मर्षिसत्तमम्} %7-55-8

\twolineshloka
{अनन्तरं स राजर्षिर्निमिरिक्ष्वाकुनन्दनः}
{अत्रिमङ्गिरसं चैव भृगुं चैव तपोधनम्} %7-55-9

\twolineshloka
{तमुवाच वसिष्ठस्तु निमिं राजर्षिसत्तमम्}
{वृतोऽहं पूर्वमिन्द्रेण अन्तरं प्रतिपालय} %7-55-10

\twolineshloka
{अनन्तरं महाविप्रो गौतमः प्रत्यपूरयत्}
{वसिष्ठोऽपि महातेजा इन्द्रयज्ञमथाकरोत्} %7-55-11

\threelineshloka
{निमिस्तु राजा विप्रांस्तान्समानीय नराधिपः}
{अयजद्धिमवत्पार्श्वे स्वपुरस्य समीपतः}
{पञ्चवर्षसहस्राणि राजा दीक्षामुपागमत्} %7-55-12

\twolineshloka
{इन्द्रयज्ञावसाने तु वसिष्ठो भगवनृषिः}
{सकाशमागतो राज्ञो हौत्रं कर्तुमनिन्दितः} %7-55-13

\twolineshloka
{तदन्तरमथापश्यद्गौतमेनाभिपूरितम्}
{कोपेन महताविष्टो वसिष्ठो ब्रह्मणः सुतः} %7-55-14

\twolineshloka
{स राज्ञो दर्शनाकाङ्क्षी मुहूर्तं समुपाविशत्}
{तस्मिन्नहनि राजर्षिर्निद्रयाऽपहृतो भृशम्} %7-55-15

\twolineshloka
{ततो मन्युर्वसिष्ठस्य प्रादुरासीन्महात्मनः}
{अदर्शनेन राजर्षेर्व्याहर्तुमुपचक्रमे} %7-55-16

\twolineshloka
{यस्मात्त्वमन्यं वृतवान्मामवज्ञाय पार्थिव}
{चेतनेन विनाभूतो देहस्तव भविष्यति} %7-55-17

\twolineshloka
{ततः प्रबुद्धो राजर्षिः श्रुत्वा शापमुदाहृतम्}
{ब्रह्मयोनिमथोवाच संरम्भात् क्रोधमूर्च्छितः} %7-55-18

\twolineshloka
{अजानतः शयानस्य क्रोधेन कलुषीकृतः}
{मुक्तवान्मयि शापाग्निं यमदण्डमिवापरम्} %7-55-19

\twolineshloka
{तस्मात्तवापि ब्रह्मर्षे चेतनेन विना कृतः}
{देहः सुरुचिरप्रख्यो भविष्यति न संशयः} %7-55-20

\twolineshloka
{इति रोषवशादुभौ तदानीमन्योन्यं शपितौ नृपद्विजेन्द्रौ}
{सहसैव बभूवतुर्विदेहौ तत्तुल्याधिगतप्रभाववन्तौ} %7-55-21


॥इत्यार्षे श्रीमद्रामायणे वाल्मीकीये आदिकाव्ये उत्तरकाण्डे निमिवसिष्ठशापः नाम पञ्चपञ्चाशः सर्गः ॥७-५५॥
