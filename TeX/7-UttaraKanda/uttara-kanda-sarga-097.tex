\sect{सप्तनवतितमः सर्गः — सीतारसातलप्रवेशः}

\twolineshloka
{वाल्मीकिनैवमुक्तस्तु राघवः प्रत्यभाषत}
{प्राञ्जलिर्जगतो मध्ये दृष्ट्वा तां वरवर्णिनीम्} %7-97-1

\twolineshloka
{एवमेतन्महाभाग यथा वदसि धर्मवित्}
{प्रत्ययस्तु मम ब्रह्मंस्तव वाक्यैरकल्मषैः} %7-97-2

\threelineshloka
{प्रत्ययस्तु पुरा वृत्तो वैदेह्याः सुरसन्निधौ}
{शपथस्तु कृतस्तत्र तेन वेश्म प्रवेशिता}
{लोकापवादो बलवान्येन त्यक्ता हि मैथिली} %7-97-3

\twolineshloka
{सेयं लोकभयाद्ब्रह्मन्नपापेत्यभिजानता}
{परित्यक्ता मया सीता तद्भावन्क्षन्तुमर्हति} %7-97-4

\twolineshloka
{जानामि चेमौ पुत्रौ मे यमजातौ कुशीलवौ}
{शुद्धायां जगतो मध्ये मैथिल्यां प्रीतिरस्तु मे} %7-97-5

\twolineshloka
{अभिप्रायं तु विज्ञाय रामस्य सुरसत्तमाः}
{सीतायाः शपथे तस्मिन् महेन्द्राद्या महौजसः} %7-97-6

\onelineshloka
{पितामहं पुरस्कृत्य सर्व एव समागताः} %7-97-7

\twolineshloka
{आदित्या वसवो रुद्रा ह्यश्विनौ समरुद्गणाः}
{गन्धर्वाप्सरसश्चैव सर्व एव समागताः} %7-97-8

\threelineshloka
{साध्याश्च विश्वेदेवाश्च सर्वे च परमर्षयः}
{नागाः सुपर्णाः सिद्धाश्च ते सर्वे हृष्टमानसाः}
{सीताशपथसम्भ्रान्ताः सर्व एव समागताः} %7-97-9

\twolineshloka
{दृष्ट्वा देवानृषींश्चैव राघवः पुनरब्रवीत्}
{प्रत्ययो मे नरश्रेष्ठा ऋषिवाक्यैरकल्मषैः} %7-97-10

\onelineshloka
{शुद्धायां जगतो मध्ये वैदेह्यां प्रीतिरस्तु मे} %7-97-11

\twolineshloka
{ततो वायुः शुभः पुण्यो दिव्यगन्धो मनोरमः}
{तज्जनौघं सुरश्रेष्ठो ह्लादयामास सर्वतः} %7-97-12

\twolineshloka
{तदद्भुतमिवाचिन्त्यं निरैक्षन्त समागताः}
{मानवाः सर्वराष्ट्रेभ्यः पूर्वं कृतयुगे यथा} %7-97-13

\twolineshloka
{सर्वान्समागतान्दृष्ट्वा सीता काषायवासिनी}
{अब्रवीत्प्राञ्जलिर्वाक्यमधोदृष्टिरवाङ्मुखी} %7-97-14

\twolineshloka
{यथाऽहं राघवादन्यं मनसापि न चिन्तये}
{तथा मे माधवी देवी विवरं दातुमर्हति} %7-97-15

\twolineshloka
{मनसा कर्मणा वाचा यथा रामं समर्चये}
{तथा मे माधवी देवी विवरं दातुमर्हति} %7-97-16

\twolineshloka
{यथैतत्सत्यमुक्तं मे वेद्मि रामात्परं न च}
{तथा मे माधवी देवी विवरं दातुमर्हति} %7-97-17

\twolineshloka
{तथा शपन्त्यां वैदेह्यां प्रादुरासीत्तदद्भुतम्}
{भूतलादुत्थितं दिव्यं सिंहासनमनुत्तमम्} %7-97-18

\twolineshloka
{ध्रियमाणं शिरोभिस्तु नागैरमितविक्रमैः}
{दिव्यं दिव्येन वपुषा दिव्यरत्नविभूषितैः} %7-97-19

\twolineshloka
{तस्मिंस्तु धरणी देवी बाहुभ्यां गृह्य मैथिलीम्}
{स्वागतेनाभिनन्द्यैनामासने चोपवेशयत्} %7-97-20

\twolineshloka
{तामासनगतां दृष्ट्वा प्रविशन्तीं रसातलम्}
{पुष्पवृष्टिरविच्छिन्ना दिव्या सीतामवाकिरत्} %7-97-21

\twolineshloka
{साधुकारश्च सुमहान्देवानां सहसोत्थितः}
{साधु साध्विति वै सीते यस्यास्ते शीलमीदृशम्} %7-97-22

\twolineshloka
{एवं बहुविधा वाचो ह्यन्तरिक्षगताः सुराः}
{व्याजह्रुर्हृष्टमनसो दृष्ट्वा सीताप्रवेशनम्} %7-97-23

\twolineshloka
{यज्ञवाटगताश्चापि मुनयः सर्व एव ते}
{राजानश्च नरव्याघ्रा विस्मयान्नोपरेमिरे} %7-97-24

\twolineshloka
{अन्तरिक्षे च भूमौ च सर्वे स्थावरजङ्गमाः}
{दानवाश्च महाकायाः पाताले पन्नगाधिपाः} %7-97-25

\twolineshloka
{केचिद्विनेदुः संहृष्टाः केचिद्ध्यानपरायणाः}
{केचिद्रामं निरीक्षन्ते केचित्सीतामचेतनाः} %7-97-26

\twolineshloka
{सीताप्रवेशनं दृष्ट्वा तेषामासीत्समागमः}
{तन्मुहूर्तमिवात्यर्थं समं सम्मोहितं जगत्} %7-97-27


॥इत्यार्षे श्रीमद्रामायणे वाल्मीकीये आदिकाव्ये उत्तरकाण्डे सीतारसातलप्रवेशः नाम सप्तनवतितमः सर्गः ॥७-९७॥
