\sect{द्विसप्ततितमः सर्गः — शत्रुघ्नरामसमागमः}

\twolineshloka
{तं शयानं नरव्याघ्रं निद्रा नाभ्यागमत्तदा}
{चिन्तयन्तमनेकार्थं रामगीतमनुत्तमम्} %7-72-1

\twolineshloka
{तस्य शब्दं सुमधुरं तन्त्रीलयसमन्वितम्}
{श्रुत्वा रात्रिर्जगामाशु शत्रुघ्नस्य महात्मनः} %7-72-2

\twolineshloka
{तस्यां निशायां व्युष्टायां कृत्वा पौर्वाह्णिकं क्रमम्}
{उवाच प्राञ्जलिर्वाक्यं शत्रुघ्नो मुनिपुङ्गवम्} %7-72-3

\twolineshloka
{भगवन्द्रष्टुमिच्छामि राघवं रघुनन्दनम्}
{त्वयाऽनुज्ञातुमिच्छामि सहैभिः संशितव्रतैः} %7-72-4

\twolineshloka
{इत्येवंवादिनं तं तु शत्रुघ्नं शत्रुतापनम्}
{वाल्मीकिः सम्परिष्वज्य विससर्ज च राघवम्} %7-72-5

\twolineshloka
{सोऽभिवाद्य मुनिश्रेष्ठं रथमारुह्य सुप्रभम्}
{अयोध्यामगमत्तूर्णं राघवोत्सुकदर्शनः} %7-72-6

\twolineshloka
{स प्रविष्टः पुरीं रम्यां श्रीमानिक्ष्वाकुनन्दनः}
{प्रविवेश महाबाहुर्यत्र रामो महाद्युतिः} %7-72-7

\twolineshloka
{स रामं मन्त्रिमध्यस्थं पूर्णचन्द्रनिभाननम्}
{पश्यन्नमरमध्यस्थं सहस्रनयनं यथा} %7-72-8

\twolineshloka
{सोऽभिवाद्य महात्मानं ज्वलन्तमिव तेजसा}
{उवाच प्राञ्जलिर्वाक्यं रामं सत्यपराक्रमम्} %7-72-9

\twolineshloka
{यदाज्ञप्तं महाराज सर्वं तत्कृतवानहम्}
{हतः स लवणः पापः पुरी चास्य निवेशिता} %7-72-10

\twolineshloka
{द्वादशैते गता वर्षास्त्वां विना रघुनन्दन}
{नोत्सहेयमहं वस्तुं त्वया विरहितो नृप} %7-72-11

\twolineshloka
{स मे प्रसादं काकुत्स्थ कुरुष्वामितविक्रम}
{मातृहीनो यथा वत्सो न चिरं प्रवसाम्यहम्} %7-72-12

\twolineshloka
{एवं ब्रुवाणं शत्रुघ्नं परिष्वज्येदमब्रवीत्}
{मा विषादं कृथाः शूर नैतत्क्षत्रियचेष्टितम्} %7-72-13

\twolineshloka
{नावसीदन्ति राजानो विप्रवासेषु राघव}
{प्रजा नः परिपाल्या हि क्षत्रधर्मेण राघव} %7-72-14

\twolineshloka
{काले काले तु मां वीर अयोध्यामवलोकितुम्}
{आगच्छ त्वं नरश्रेष्ठ गन्तासि च पुरं तव} %7-72-15

\twolineshloka
{ममापि त्वं सुदयितः प्राणैरपि न संशयः}
{अवश्यं करणीयं च राज्यस्य परिपालनम्} %7-72-16

\twolineshloka
{तस्मात्त्वं वस काकुत्स्थ सप्तरात्रमिहावस}
{ऊर्ध्वं गन्तासि मधुरां सभृत्यबलवाहनः} %7-72-17

\twolineshloka
{रामस्यैतद्वचः श्रुत्वा धर्मयुक्तं मनोगतम्}
{शत्रुघ्नो दीनया वाचा बाढमित्येव चाब्रवीत्} %7-72-18

\twolineshloka
{सप्तरात्रं च काकुत्स्थो राघवस्य यथाज्ञया}
{उष्य तत्र महेष्वासो गमनायोपचक्रमे} %7-72-19

\twolineshloka
{आमन्त्र्य तु महात्मानं रामं सत्यपराक्रमम्}
{भरतं लक्ष्मणं चैव महारथमुपारुहत्} %7-72-20

\twolineshloka
{दूरमाभ्यामनुगतो लक्ष्मणेन महात्मना}
{भरतेन च शत्रुघ्नो जगामाशु पुरं ततः} %7-72-21


॥इत्यार्षे श्रीमद्रामायणे वाल्मीकीये आदिकाव्ये उत्तरकाण्डे शत्रुघ्नरामसमागमः नाम द्विसप्ततितमः सर्गः ॥७-७२॥
