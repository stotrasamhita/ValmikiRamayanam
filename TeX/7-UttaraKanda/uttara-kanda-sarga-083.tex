\sect{त्र्यशीतितमः सर्गः — राजसूयाजिहीर्षा}

\twolineshloka
{तच्छ्रुत्वा भाषितं तस्य रामस्याक्लिष्ट कर्मणः}
{द्वास्स्थः कुमारावाहूय राघवाय न्यवेदयत्} %7-83-1

\twolineshloka
{दृष्ट्वा तु राघवः प्राप्तौ प्रियौ भरतलक्ष्मणौ}
{परिष्वज्य तदा रामो वाक्यमेतदुवाच ह} %7-83-2

\twolineshloka
{कृतं मया यथा तथ्यं द्विजकार्यमनुत्तमम्}
{धर्मसेतुमथो भूयः कर्तुमिच्छामि राघवौ} %7-83-3

\twolineshloka
{अक्षय्यश्चाव्ययश्चैव धर्मसेतुर्मतो मम}
{धर्मप्रसाधकं ह्येतत् सर्वपापप्रणाशनम्} %7-83-4

\twolineshloka
{युवाभ्यामात्मभूताभ्यां राजसूयमनुत्तमम्}
{सहितो यष्टुमिच्छामि तत्र धर्मो हि शाश्वतः} %7-83-5

\twolineshloka
{इष्ट्वा तु राजसूयेन मित्रः शत्रुनिबर्हणः}
{सुहुतेन सुयज्ञेन वरुणत्वमुपागमत्} %7-83-6

\twolineshloka
{सोमश्च राजसूयेन इष्ट्वा धर्मेण धर्मवित्}
{प्राप्तश्च सर्वलोकेषु कीर्तिस्थानं च शाश्वतम्} %7-83-7

\twolineshloka
{अस्मिन्नहनि यच्छ्रेयश्चिन्त्यतां तन्मया सह}
{हितं चायतियुक्तं च प्रयतौ कर्तुमर्हथः} %7-83-8

\twolineshloka
{श्रुत्वा तु राघवस्यैतद्वाक्यं वाक्यविशारदः}
{भरतः प्राञ्जलिर्भूत्वा वाक्यमेतदुवाच ह} %7-83-9

\twolineshloka
{त्वयि धर्मः परः साधो त्वयि सर्वा वसुन्धरा}
{प्रतिष्ठिता महाबाहो यशश्चामितविक्रम} %7-83-10

\twolineshloka
{महीपालाश्च सर्वे त्वां प्रजापतिमिवामराः}
{निरीक्षन्ते महात्मानं लोकानाथं यथा वयम्} %7-83-11

\twolineshloka
{पुत्राश्च पितृवद्राजन्पश्यन्ति त्वां महाबल}
{पृथिव्या गतिभूतोऽसि प्राणिनामपि राघव} %7-83-12

\twolineshloka
{स त्वमेवंविधं यज्ञमाहर्तासि कथं नृप}
{पृथिव्यां राजवंशानां विनाशो यत्र दृश्यते} %7-83-13

\twolineshloka
{पृथिव्यां ये च पुरुषा राजन्पौरुषमागताः}
{सर्वेषां भविता तत्र सङ्क्षयः सर्वकोपजः} %7-83-14

\twolineshloka
{स त्वं पुरुषशार्दूल गुणैरतुलविक्रम}
{पृथिवीं नार्हसे हन्तुं वशे हि तव वर्तते} %7-83-15

\twolineshloka
{भरतस्य तु तद्वाक्यं श्रुत्वाऽमृतमयं तदा}
{प्रहर्षमतुलं लेभे रामः सत्यपराक्रमः} %7-83-16

\twolineshloka
{उवाच च शुभं वाक्यं कैकेय्यानन्दवर्धनम्}
{प्रीतोऽस्मि परितुष्टोऽस्मि तवाद्य वचनेऽनघ} %7-83-17

\twolineshloka
{इदं वचनमक्लीबं त्वया धर्मसमाहितम्}
{व्याहृतं पुरुषव्याघ्र पृथिव्याः परिपालनम्} %7-83-18

\twolineshloka
{एष्यदस्मदभिप्रायाद्राजसूयात्क्रतूत्तमात्}
{निवर्तयामि धर्मज्ञ तव सुव्याहृतेन च} %7-83-19

\threelineshloka
{लोकपीडाकरं कर्म न कर्तव्यं विचक्षणैः}
{बालानां तु शुभं वाक्यं ग्राह्य लक्ष्मणपूर्वज}
{तस्माच्छृणोमि ते वाक्यं साधु युक्तं महामते} %7-83-20


॥इत्यार्षे श्रीमद्रामायणे वाल्मीकीये आदिकाव्ये उत्तरकाण्डे राजसूयाजिहीर्षा नाम त्र्यशीतितमः सर्गः ॥७-८३॥
