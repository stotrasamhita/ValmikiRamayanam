\sect{एकचत्वारिंशः सर्गः — पुष्पकपुनरभ्यनुज्ञा}

\twolineshloka
{विसृज्य च महाबाहुर्ऋक्षवानरराक्षसान्}
{भ्रातृभिः सहितो रामः प्रमुमोद सुखं सुखी} %7-41-1

\twolineshloka
{अथापराह्णसमये भ्रातृभिः सह राघवः}
{शुश्राव मधुरां वाणीमन्तरिक्षात् प्रभाषिताम्} %7-41-2

\twolineshloka
{सौम्य राम निरीक्षस्व सौम्येन वदनेन माम्}
{कुबेरभवनात्प्राप्तं विद्धि मां पुष्पकं प्रभो} %7-41-3

\twolineshloka
{तव शासनमाज्ञाय गतोऽस्मि भवनं प्रति}
{उपस्थातुं नरश्रेष्ठ स च मां प्रत्यभाषत} %7-41-4

\twolineshloka
{निर्जितस्त्वं नरेन्द्रेण राघवेण महात्मना}
{निहत्य युधि दुर्धर्षं रावणं राक्षसेश्वरम्} %7-41-5

\twolineshloka
{ममापि परमा प्रीतर्हते तस्मिन्दुरात्मनि}
{रावणे सगणे चैव सपुत्रे सहबान्धवे} %7-41-6

\twolineshloka
{स त्वं रामेण लङ्कायां निर्जितः परमात्मना}
{वह सौम्य तमेव त्वमहमाज्ञापयामि ते} %7-41-7

\twolineshloka
{परमो ह्येष मे कामो यत्त्वं राघवनन्दनम्}
{वहेर्लोकस्य संयानं गच्छस्व विगतज्वरः} %7-41-8

\twolineshloka
{सोऽहं शासनमाज्ञाय धनदस्य महात्मनः}
{त्वत्सकाशमनुप्राप्तो निर्विशङ्कः प्रतीक्ष माम्} %7-41-9

\twolineshloka
{अदृष्यः सर्वभूतानां सर्वेषां धनदाज्ञया}
{चराम्यहं प्रभावेण तवाज्ञां परिपालयन्} %7-41-10

\twolineshloka
{एवमुक्तस्तदा रामः पुष्पकेण महाबलः}
{उवाच पुष्पकं दृष्ट्वा विमानं पुनरागतम्} %7-41-11

\twolineshloka
{यद्येवं स्वागतं तेऽस्तु विमानवर पुष्पक}
{आनुकूल्याद्धनेशस्य वृत्तदोषो न नो भवेत्} %7-41-12

\twolineshloka
{लाजैश्चैव तथा पुष्पैर्धूपैश्चैव सुगन्धिभिः}
{पूजयित्वा महाबाहू राघवः पुष्पकं तदा} %7-41-13

\threelineshloka
{गम्यतामिति चोवाच आगच्छ त्वं स्मरे यदा}
{सिद्धानां च गतौ सौम्य मा विषादेन योजय}
{प्रतिघातश्च ते मा भूद्यथेष्टं गच्छतो दिशः} %7-41-14

\twolineshloka
{एवमस्त्विति रामेण पूजयित्वा विसर्जितम्}
{अभिप्रेतां दिशं तस्मात्प्रायात्तत्पुष्पकं तदा} %7-41-15

\twolineshloka
{एवमन्तर्हिते तस्मिन्पुष्पके सुकृतात्मनि}
{भरतः प्राञ्जलिर्वाक्यमुवाच रघुनन्दनम्} %7-41-16

\twolineshloka
{विविधात्मनि दृश्यन्ते त्वयि वीर प्रशासति}
{अमानुषाणां सत्त्वानां व्याहृतानि मुहुर्मुहुः} %7-41-17

\twolineshloka
{अनामयश्च सत्वानां साग्रो मासो गतो ह्ययम्}
{जीर्णानामपि सत्त्वानां मृत्युर्नायाति राघव} %7-41-18

\twolineshloka
{अरोगप्रसवा नार्यो वपुष्मन्तो हि मानवाः}
{हर्षश्चाभ्यधिको राजञ्जनस्य पुरवासिनः} %7-41-19

\twolineshloka
{काले वर्षति पर्जन्यः पातयन्नमृतं पयः}
{वाताश्चापि प्रवान्त्येते स्पर्शयुक्ताः सुखाः शिवाः} %7-41-20

\twolineshloka
{ईदृशोऽनश्वरो राजा भवेदिति नरेश्वरः}
{कथयन्ति पुरे राजन्पौरजानपदास्तथा} %7-41-21

\twolineshloka
{एता वाचः सुमधुरा भरतेन समीरिताः}
{श्रुत्वा रामो मुदा युक्तो बभूव नृपसत्तमः} %7-41-22


॥इत्यार्षे श्रीमद्रामायणे वाल्मीकीये आदिकाव्ये उत्तरकाण्डे पुष्पकपुनरभ्यनुज्ञा नाम एकचत्वारिंशः सर्गः ॥७-४१॥
