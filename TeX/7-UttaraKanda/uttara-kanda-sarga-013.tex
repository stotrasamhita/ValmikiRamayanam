\sect{त्रयोदशः सर्गः — धनददूतहननम्}

\twolineshloka
{अथ लोकेश्वरोत्सृष्टा तत्र कालेन केनचित्}
{निद्रा समभवत्तीव्रा कुम्भकर्णस्य रूपिणी} %7-13-1

\twolineshloka
{ततो भ्रातरमासीनं कुम्भकर्णोऽब्रवीद्वचः}
{निद्रा मां बाधते राजन्कारयस्व ममालयम्} %7-13-2

\twolineshloka
{विनियुक्तास्ततो राज्ञा शिल्पिनो विश्वकर्मवत्}
{विस्तीर्णं योजनं शुभ्रं ततो द्विगुणमायतम्} %7-13-3

\twolineshloka
{दर्शनीयं निराबाधं कुम्भकर्णस्य चक्रिरे}
{स्फाटिकैः काञ्चनैश्चित्रैः स्तम्भैः सर्वत्र शोभितम्} %7-13-4

\twolineshloka
{वैडूर्यकृतसोपानं किङ्किणीजालकं तथा}
{दान्ततोरणविन्यस्तं वज्रस्फटिकवेदिकम्} %7-13-5

\twolineshloka
{मनोहरं सर्वसुखं कारयामास राक्षसः}
{सर्वत्र सुखदं नित्यं मेरोः पुण्यां गुहामिव} %7-13-6

\twolineshloka
{तत्र निद्रां समाविष्टः कुम्भकर्णो महाबलः}
{बहून्यब्दसहस्राणि शयानो न प्रबुद्ध्यते} %7-13-7

\twolineshloka
{निद्राभिभूते तु तदा कुम्भकर्णे दशाननः}
{देवर्षियक्षगन्धर्वान्सञ्जघ्ने हि निरङ्कुशः} %7-13-8

\twolineshloka
{उद्यानानि च चित्राणि नन्दनादीनि यानि च}
{तानि गत्वा सुसङक्रुद्धो भिनत्ति स्म दशाननः} %7-13-9

\twolineshloka
{नदीं गज इव क्रीडन्वृक्षान्वायुरिव क्षिपन्}
{नगान्व्रज्र इवोत्सृष्टो विध्वंसयति राक्षसः} %7-13-10

\twolineshloka
{तथावृत्तं तु विज्ञाय दशग्रीवं धनेश्वरः}
{कुलानुरूपं धर्मज्ञो वृत्तं संस्मृत्य चात्मनः} %7-13-11

\twolineshloka
{सौभ्रात्रदर्शनार्थं तु दूतं वैश्रवणस्तदा}
{लङ्कां सम्प्रेषयामास दशग्रीवस्य वै हितम्} %7-13-12

\twolineshloka
{स गत्वा नगरीं लङ्कामाससाद विभीषणम्}
{मानितस्तेन धर्मेण पृष्टश्चागमनं प्रति} %7-13-13

\twolineshloka
{पृष्ट्वा च कुशलं राज्ञो ज्ञातीनां च बिभीषणः}
{सभायां दर्शयामास तमासीनं दशाननम्} %7-13-14

\twolineshloka
{स दृष्ट्वा तत्र राजानं दीप्यमानं स्वतेजसा}
{जयेति वाचा सम्पूज्य तूष्णीं समभिवर्तत} %7-13-15

\twolineshloka
{तं तत्रोत्तमपर्यङ्के वरास्तरणशोभिते}
{उपविष्टं दशग्रीवं दूतो वाक्यमथाब्रवीत्} %7-13-16

\twolineshloka
{राजन्वदामि ते सर्वं भ्राता तव यदब्रवीत्}
{उभयोः सदृशं वीर वृत्तस्य च कुलस्य च} %7-13-17

\twolineshloka
{साधु पर्याप्तमेतावत्कृतश्चारित्रसङ्ग्रहः}
{साधु धर्मे व्यवस्थानं क्रियतां यदि शक्यते} %7-13-18

\twolineshloka
{दृष्टं मे नन्दनं भग्नमृषयो निहताः श्रुताः}
{देवतानां समुद्योगस्त्वत्तो राजन्मम श्रुतः} %7-13-19

\twolineshloka
{निराकृतश्च बहुशस्त्वयाऽहं राक्षसाधिप}
{अपराद्धा हि बाल्याच्च रमणीयाः स्वबान्धवाः} %7-13-20

\twolineshloka
{अहं तु हिमवत्पृष्ठं गतो धर्ममुपासितुम्}
{रौद्रं व्रत्तं समास्थाय नियतो नियतेन्द्रियः} %7-13-21

\twolineshloka
{तत्र देवो मया दृष्टः सह देव्या मया प्रभुः}
{सव्यं चक्षुर्मया दैवात्तत्र देव्यां निपातितम्} %7-13-22

\twolineshloka
{का न्वियं स्यादिति शुभा न खल्वन्येन हेतुना}
{रूपं ह्यनुपमं कृत्वा रुद्राणी तत्र तिष्ठति} %7-13-23

\twolineshloka
{देव्या दिव्यप्रभावेण दग्धं सव्यं ममेक्षणम्}
{रेणुध्वस्तमिव ज्योतिः पिङ्गुलत्वमुपागतम्} %7-13-24

\twolineshloka
{ततोऽहमन्यद्विस्तीर्णं गत्वा तस्य गिरेस्तटम्}
{तूष्णीं वर्षशतान्यष्टौ समाधारं महाव्रतम्} %7-13-25

\twolineshloka
{समाप्ते नियमे तस्मिंस्तत्र देवो महेश्वरः}
{प्रीतः प्रीतेन मनसा प्राह वाक्यमिदं प्रभुः} %7-13-26

\twolineshloka
{पैङ्गल्यं यदवाप्तं हि देव्या रूपनिरीक्षणात्}
{प्रीतोऽस्मि तव धर्मज्ञ तपसा नेन सुव्रत} %7-13-27

\twolineshloka
{मया चैतद्व्रतं चीर्णं त्वया चैव धनाधिप}
{तृतीयः पुरुषो नास्ति यश्चरेव्रतमीदृशम्} %7-13-28

\twolineshloka
{व्रतं सुनिश्चयं ह्येतन्मया ह्युत्पादितं पुरा}
{तत्सखित्वं मया सौम्य रोचयस्व धनेश्वर} %7-13-29

\twolineshloka
{तपसा निर्जितश्चैव सखा भव ममानघ}
{देव्या दग्धं प्रभावेण यच्च सव्यं तवेक्षणम्} %7-13-30

\twolineshloka
{एकाक्षिपिङ्गलेत्येव नाम स्थास्यति शाश्वतम्}
{एवं तेन सखित्वं च प्राप्यानुज्ञां च शङ्करात्} %7-13-31

\onelineshloka
{आगत्य च श्रुतोऽयं मे तव पापविनिश्चयः} %7-13-32

\twolineshloka
{तदधर्मिष्ठसंयोगान्निवर्त कुलदूषणात्}
{चिन्त्यते हि वधोपायः सर्षिसङ्घैः सुरैस्तव} %7-13-33

\twolineshloka
{एवमुक्तो दशग्रीवः क्रुद्धसंरक्तलोचनः}
{हस्तौ दन्तांश्च सम्पीड्य वाक्यमेतदुवाच ह} %7-13-34

\twolineshloka
{विज्ञातं ते मया दूत वाक्यं यस्य प्रभाषसे}
{नैतत्त्वमसि नैवासौ भ्रात्रा येनासि चोदितः} %7-13-35

\twolineshloka
{हितं नैष ममैतद्धि ब्रवीति धनरक्षकः}
{महेश्वरसखित्वं तु मूढ श्रावयते किल} %7-13-36

\twolineshloka
{न चेदं क्षमणीयं मे यदेतद्भाषितं त्वया}
{यदेतावन्मया कालं दूत तस्य तु मर्षितम्} %7-13-37

\twolineshloka
{न हन्तव्यो गुरुर्ज्येष्ठो मयाऽयमिति मन्यते}
{तस्य त्विदानीं श्रुत्वा मे वाक्यमेषा कृता मतिः} %7-13-38

\onelineshloka
{त्रीँल्लोकानपि जेष्यामि बाहुवीर्यमुपाश्रितः} %7-13-39

\twolineshloka
{एतन्मुहूर्तमेवाहं तस्यैकस्य तु वै कृते}
{चतुरो लोकपालांस्तान्नयिष्यामि यमक्षयम्} %7-13-40

\twolineshloka
{एवमुक्त्वा तु लङ्केशो दूतं खड्गेन जघ्निवान्}
{ददौ भक्षयितुं ह्येनं राक्षसानां दुरात्मनाम्} %7-13-41

\twolineshloka
{एवं कृतस्वस्त्ययनो रथमारुह्य रावणः}
{त्रैलोक्यविजयाकाङ्क्षी ययौ यत्र धनेश्वरः} %7-13-42


॥इत्यार्षे श्रीमद्रामायणे वाल्मीकीये आदिकाव्ये उत्तरकाण्डे धनददूतहननम् नाम त्रयोदशः सर्गः ॥७-१३॥
