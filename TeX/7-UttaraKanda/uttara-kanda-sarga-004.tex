\sect{चतुर्थः सर्गः — रावणादिपूर्वतनराक्षसोत्पत्तिकथनम्}

\twolineshloka
{श्रुत्वागस्त्येरितं वाक्यं रामो विस्मयमागतः}
{कथमासीत्तु लङ्कायां सम्भवो रक्षसां पुरा} %7-4-1

\twolineshloka
{ततः शिरः कम्पयित्वा त्रेताग्निसमविग्रहम्}
{तमगस्त्यं मुहुर्दृष्ट्वा स्मयमानोऽभ्यभाषत} %7-4-2

\twolineshloka
{भगवन्पूर्वमप्येषा लङ्कासीत्पिशिताशिनाम्}
{श्रुत्वेदं भगवद्वाक्यं जातो मे विस्मयः परः} %7-4-3

\twolineshloka
{पुलस्त्यवंशादुद्भूता राक्षसा इति नः श्रुतम्}
{इदानीमन्यतश्चापि सम्भवः कीर्तितस्त्वया} %7-4-4

\twolineshloka
{रावणात्कुम्भकर्णाच्च प्रहस्ताद्विकटादपि}
{रावणस्य च पुत्रेभ्यः किं नु ते बलवत्तराः} %7-4-5

\twolineshloka
{क एषां पूर्वको ब्रह्मन्किन्नामा च बलोत्कटः}
{अपराधं च कं प्राप्य विष्णुना द्राविताः कथम्} %7-4-6

\twolineshloka
{एतद्विस्तरतः सर्वं कथयस्व ममानघ}
{कुतूहलमिदं मह्यं नुद भानुर्यथा तमः} %7-4-7

\twolineshloka
{राघवस्य वचः श्रुत्वा संस्कारालङ्कृतं शुभम्}
{ईषद्विस्मयमानस्तमगस्त्यः प्राह राघवम्} %7-4-8

\twolineshloka
{प्रजापतिः पुरा सृष्ट्वा ह्यपः सलिलसम्भवः}
{तासां गोपायने सत्त्वानसृजत्पद्मसम्भवः} %7-4-9

\twolineshloka
{ते सत्त्वाः सत्त्वकर्तारं विनीतवदुपस्थिताः}
{किं कुर्म इति भाषन्तः क्षुत्पिपासाभयार्दिताः} %7-4-10

\twolineshloka
{प्रजापतिस्तु तान्याह सत्वानि प्रहसन्निव}
{आभाष्य वाचा यत्नेन रक्षध्वमिति मानदः} %7-4-11

\twolineshloka
{रक्षामेति च तत्रान्ये जक्षाम इति चापरे}
{भुक्षिताभुक्षितैरुक्तस्ततस्तानाह भूतकृत्} %7-4-12

\twolineshloka
{रक्षामेति च यैरुक्तं राक्षसास्ते भवन्तु वः}
{जक्षाम इति यैरुक्तं यक्षा एव भवन्तु वः} %7-4-13

\twolineshloka
{तत्र हेतिः प्रहेतिश्च भ्रातरौ राक्षसाधिपौ}
{मधुकैटभसङ्काशौ बभूवतुररिन्दमौ} %7-4-14

\twolineshloka
{प्रहेतिर्धार्मिकस्तत्र तपोवनगतस्तदा}
{हेतिर्दारक्रियार्थे तु परं यत्नमथाकरोत्} %7-4-15

\twolineshloka
{स कालभगिनीं कन्यां भयां नाम भयावहाम्}
{उदावहदमेयात्मा स्वयमेव महामतिः} %7-4-16

\twolineshloka
{स तस्यां जनयामास हेती राक्षसपुङ्गवः}
{पुत्रं पुत्रवतां श्रेष्ठो विद्युत्केश इति श्रुतम्} %7-4-17

\twolineshloka
{विद्युत्केशो हेतिपुत्रः स दीप्तार्कसमप्रभः}
{व्यवर्धत महातेजास्तोयमध्य इवाम्बुदः} %7-4-18

\twolineshloka
{स यदा यौवनं भद्रमनुप्राप्तो निशाचरः}
{ततो दारक्रियां तस्य कर्तुं व्यवसितः पिता} %7-4-19

\twolineshloka
{सन्ध्यायास्तनयां सोऽथ सन्ध्यातुल्यां प्रभावतः}
{वरयामास पुत्रार्थं हेती राक्षसपुङ्गवः} %7-4-20

\twolineshloka
{अवश्यमेव दातव्या परस्मै सेति सन्ध्यया}
{चिन्तयित्वा सुता दत्ता विद्युत्केशाय राघव} %7-4-21

\twolineshloka
{सन्ध्यायास्तनयां लब्ध्वा विद्युत्केशो निशाचरः}
{रमते स्म तया सार्धं पौलोम्या मघवानिव} %7-4-22

\twolineshloka
{केनचित्त्वथ कालेन राम सालकटङ्कटा}
{विद्युत्केशाद्गर्भमाप घनराजिरिवार्णवात्} %7-4-23

\twolineshloka
{ततः सा राक्षसी गर्भं घनगर्भसमप्रभम्}
{प्रसूता मन्दरं गत्वा गङ्गा गर्भमिवाग्निजम्} %7-4-24

\twolineshloka
{समुत्सृज्य तु सा गर्भं विद्युत्केशरतार्थिनी}
{रेमे तु सार्धं पतिना विस्मृत्य सुतमात्मजम्} %7-4-25

\threelineshloka
{उत्सृष्टस्तु तदा गर्भो घनशब्दसमस्वनः}
{तयोत्सृष्टः स तु शिशुः शरदर्कसमद्युतिः}
{निधायास्ये स्वयं मुष्टिं रुरोद शनकैस्तदा} %7-4-26

\twolineshloka
{ततो वृषभमास्थाय पार्वत्या सहितः शिवः}
{वायुमार्गेण गच्छन्वै शुश्राव रुदितस्वनम्} %7-4-27

\twolineshloka
{अपश्यदुमया सार्धं रुदन्तं राक्षसात्मजम्}
{कारुण्यभावात्पार्वत्या भवस्त्रिपुरसूदनः} %7-4-28

\twolineshloka
{तं राक्षसात्मजं चक्रे मातुरेव वयःसमम्}
{अमरं चैव तं कृत्वा महादेवोऽक्षरोऽव्ययः} %7-4-29

\twolineshloka
{पुरमाकाशगं प्रादात्पार्वत्याः प्रियकाम्यया}
{उमयापि वरो दत्तो राक्षसानां नृपात्मज} %7-4-30

\twolineshloka
{सद्योपलब्धिर्गर्भस्य प्रसूतिः सद्य एव च}
{सद्य एव वयःप्राप्तिर्मातुरेव वयस्समम्} %7-4-31

\twolineshloka
{ततः सुकेशो वरदानगर्वितः श्रियं प्रभोः प्राप्य हरस्य पार्श्वतः}
{चचार सर्वत्र महान्महामतिः खगं पुरं प्राप्य पुरन्दरो यथा} %7-4-32


॥इत्यार्षे श्रीमद्रामायणे वाल्मीकीये आदिकाव्ये उत्तरकाण्डे रावणादिपूर्वतनराक्षसोत्पत्तिकथनम् नाम चतुर्थः सर्गः ॥७-४॥
