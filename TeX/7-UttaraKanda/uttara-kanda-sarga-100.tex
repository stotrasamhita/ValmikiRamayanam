\sect{शततमः सर्गः — गन्धर्वविषयविजययात्रा}

\threelineshloka
{कस्यचित्त्वथ कालस्य युधाजित्केकयो नृपः}
{स्वगुरुं प्रेषयामास राघवाय महात्मने}
{गार्ग्यमङ्गिरसः पुत्रं ब्रह्मर्षिममितप्रभम्} %7-100-1

\onelineshloka
{दश चाश्वसहस्राणि प्रीतिदानमनुत्तमम्} %7-100-2

\twolineshloka
{कम्बलानि च रत्नानि चित्रवस्त्रमथोत्तमम्}
{रामाय प्रददौ राजा शुभान्याभरणानि च} %7-100-3

\twolineshloka
{श्रुत्वा तु राघवो धीमान्महर्षिं गार्ग्यमागतम्}
{मातुलस्याश्वपतिनः प्रहितं तन्महाधनम्} %7-100-4

\twolineshloka
{प्रत्युद्गम्य च काकुत्स्थः क्रोशमात्रं सहानुजः}
{गार्ग्यं सम्पूजयामास यथा शक्रो बृहस्पतिम्} %7-100-5

\threelineshloka
{तथा सम्पूज्य तमृषिं तद्धनं प्रतिगृह्य च}
{पृष्ट्वा प्रतिपदं सर्वं कुशलं मातुलस्य च}
{उपविष्टं महाभागं रामः प्रष्टुं प्रचक्रमे} %7-100-6

\twolineshloka
{किमाह मातुलो वाक्यं यदर्थं भगवानिह}
{प्राप्तो वाक्यविदां श्रेष्ठः साक्षादिव बृहस्पतिः} %7-100-7

\twolineshloka
{रामस्य भाषितं श्रुत्वा महर्षिः कार्यविस्तरम्}
{वक्तुमद्भुतसङ्काशं राघवायोपचक्रमे} %7-100-8

\twolineshloka
{मातुलस्ते महाबाहो वाक्यमाह नरर्षभः}
{युधाजित्प्रीतिसंयुक्तं श्रूयतां यदि रोचते} %7-100-9

\twolineshloka
{अयं गन्धर्वविषयः फलमूलोपशोभितः}
{सिन्धोरुभयतः पार्श्वे देशः परमशोभनः} %7-100-10

\twolineshloka
{तं च रक्षन्ति गन्धर्वाः सायुधा युद्धकोविदाः}
{शैलूषस्य सुता वीर त्रिकोट्यो वै महाबलाः} %7-100-11

\twolineshloka
{तान्विनिर्जित्य काकुत्स्थ गन्धर्वनगरं शुभम्}
{निवेशय महाबोहो स्वे पुरे सुसमाहिते} %7-100-12

\twolineshloka
{अन्यस्य न गतिस्तत्र देशः परमशोभनः}
{रोचतां ते महाबाहो नाहं त्वामहितं वदे} %7-100-13

\twolineshloka
{तच्छ्रुत्वा राघवः प्रीतो महर्षेर्मातुलस्य च}
{उवाच बाढमित्येव भरतं चान्ववैक्षत} %7-100-14

\twolineshloka
{सोऽब्रवीद्राघवः प्रीतः साञ्जलिप्रग्रहो द्विजम्}
{इमौ कुमारौ तं देशं ब्रह्मर्षे विचरिष्यतः} %7-100-15

\twolineshloka
{भरतस्यात्मजौ वीरौ तक्षः पुष्कल एव च}
{मातुलेन सुगुप्तौ तु धर्मेण सुसमाहितौ} %7-100-16

\twolineshloka
{भरतं चाग्रतः कृत्वा कुमारौ सबलानुगौ}
{निहत्य गन्धर्वसुतान्द्वे पुरे विभजिष्यतः} %7-100-17

\twolineshloka
{निवेश्य ते पुरवरे आत्मजौ सन्निवेश्य च}
{आगमिष्यति मे भूयः सकाशमतिधार्मिकः} %7-100-18

\twolineshloka
{ब्रह्मर्षिमेवमुक्त्वा तु भरतं सबलानुगम्}
{आज्ञापयामास तदा कुमारौ चाभ्यषेचयत्} %7-100-19

\twolineshloka
{नक्षत्रेण च सौम्येन पुरस्कृत्याङ्गिरस्सुतम्}
{भरतः सह सैन्येन कुमाराभ्यां विनिर्ययौ} %7-100-20

\twolineshloka
{सा सेना शक्रयुक्तेव नगरान्निर्ययावथ}
{राघवानुगता दूरं दुराधर्षा सुरैरपि} %7-100-21

\twolineshloka
{मांसादानि च सत्त्वानि रक्षांसि सुमहान्ति च}
{अनुजग्मुर्हि भरतं रुधिरस्य पिपासया} %7-100-22

\twolineshloka
{भूतग्रामाश्च बहवो मांसभक्षाः सुदारुणाः}
{गन्धर्वपुत्रमांसानि भोक्तुकामाः सहस्रशः} %7-100-23

\twolineshloka
{सिंहव्याघ्रवराहाणां खेचराणां च पक्षिणाम्}
{बहूनि वै सहस्राणि सेनाया ययुरग्रतः} %7-100-24

\twolineshloka
{अध्यर्धमासमुषिता पथि सेना निरामया}
{हृष्टपुष्टजनाकीर्णा केकयं समुपागमत्} %7-100-25


॥इत्यार्षे श्रीमद्रामायणे वाल्मीकीये आदिकाव्ये उत्तरकाण्डे गन्धर्वविषयविजययात्रा नाम शततमः सर्गः ॥७-१००॥
