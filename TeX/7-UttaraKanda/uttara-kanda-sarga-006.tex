\sect{षष्ठः सर्गः — विष्णुमाल्यवदादियुद्धम्}

\twolineshloka
{तैर्वध्यमाना देवाश्च ऋषयश्च तपोधनाः}
{भयार्ताः शरणं जग्मुर्देवदेवं महेश्वरम्} %7-6-1

\twolineshloka
{जगत्सृष्ट्यन्तकर्तारमजमव्यक्तरूपिणम्}
{आधारं सर्वलोकानामाराध्यं परमं गुरुम्} %7-6-2

\twolineshloka
{ते समेत्य तु कामारिं त्रिपुरारिं त्रिलोचनम्}
{ऊचुः प्राञ्जलयो देवा भयगद्गदभाषिणः} %7-6-3

\twolineshloka
{सुकेशपुत्रैर्भगवन्पितामहवरोद्धतैः}
{प्रजाध्यक्ष प्रजाः सर्वा बाध्यन्ते रिपुबाधनैः} %7-6-4

\twolineshloka
{शरण्यान्यशरण्यानि ह्याश्रमाणि कृतानि नः}
{स्वर्गाच्च देवान्प्रच्याव्य स्वर्गे क्रीडन्ति देववत्} %7-6-5

\twolineshloka
{अहं विष्णुरहं रुद्रो ब्रह्माहं देवराडहम्}
{अहं यमश्च वरुणश्चन्द्रोऽहं रविरप्यहम्} %7-6-6

\threelineshloka
{इति माली सुमाली च माल्यवांश्चैव राक्षसाः}
{बाधन्ते समरोद्धर्षा ये च तेषां पुरःसराः}
{तन्नो देव भयार्तानामभयं दातुमर्हसि} %7-6-7

\onelineshloka
{अशिवं वपुरास्थाय जहि वै देवकण्टकान्} %7-6-8

\threelineshloka
{धन्ते समरोद्धर्षा ये च तेषां पुरःसराः}
{इत्युक्तस्तु सुरैः सर्वैः कपर्दी नीललोहितः}
{सुकेशं प्रति सापेक्षः प्राह देवगणान्प्रभुः} %7-6-9

\twolineshloka
{अहं तान्न हनिष्यामि मयावध्या हि ते सुराः}
{किं तु मन्त्रं प्रदास्यामि यो वै तान्निहनिष्यति} %7-6-10

\twolineshloka
{एतमेव समुद्योगं पुरस्कृत्य महर्षयः}
{गच्छध्वं शरणं विष्णुं हनिष्यति स तान्प्रभुः} %7-6-11

\twolineshloka
{ततस्तु जयशब्देन प्रतिनन्द्य महेश्वरम्}
{विष्णोः समीपमाजग्मुर्निशाचरभयार्दिताः} %7-6-12

\twolineshloka
{शङ्खचक्रधरं देवं प्रणम्य बहुमान्य च}
{ऊचुः सम्भ्रान्तवद्वाक्यं सुकेशतनयान्प्रति} %7-6-13

\twolineshloka
{सुकेशतनयैर्देव त्रिभिस्त्रेताग्निसन्निभैः}
{आक्रम्य वरदानेन स्थानान्यपहृतानि नः} %7-6-14

\twolineshloka
{लङ्का नाम पुरी दुर्गा त्रिकूटशिखरे स्थिता}
{तत्र स्थिताः प्रबाधन्ते सर्वान्नः क्षणदाचराः} %7-6-15

\twolineshloka
{स त्वमस्मद्धितार्थाय जहि तान्मधुसूदन}
{शरणं त्वां वयं प्राप्ता गतिर्भव सुरेश्वर} %7-6-16

\twolineshloka
{चक्रकृत्तास्यकमलान्निवेदय यमाय वै}
{भयेष्वभयदोऽस्माकं नान्योऽस्ति भवता विना} %7-6-17

\twolineshloka
{राक्षसान्समरे दुष्टान्सानुबन्धान्मदोद्धतान्}
{नुदं त्वं नो भयं देव नीहारमिव भास्करः} %7-6-18

\twolineshloka
{इत्येवं दैवतैरुक्तो देवदेवो जनार्दनः}
{अभयं भयदोऽरीणां दत्त्वा देवानुवाच ह} %7-6-19

\twolineshloka
{सुकेशं राक्षसं जाने ईशानवरदर्पितम्}
{तांश्चास्य तनयाञ्जाने येषां ज्येष्ठः स माल्यवान्} %7-6-20

\twolineshloka
{तानहं समतिक्रान्तमर्यादान्राक्षसाधमान्}
{निहनिष्यामि सङ्क्रुद्धः सुरा भवत विज्वराः} %7-6-21

\twolineshloka
{इत्युक्तास्ते सुराः सर्वे विष्णुना प्रभविष्णुना}
{यथावासं ययुर्हृष्टाः प्रशंसन्तो जनार्दनम्} %7-6-22

\twolineshloka
{विबुधानां समुद्योगं माल्यवांस्तु निशाचरः}
{श्रुत्वा तौ भ्रातरौ वीराविदं वचनमब्रवीत्} %7-6-23

\twolineshloka
{अमरा ऋषयश्चैव सङ्गम्य किल शकरम्}
{अस्मद्वधं परीप्सन्त इदं वचनमब्रुवन्} %7-6-24

\twolineshloka
{सुकेशतनया देव वरदानबलोद्धताः}
{बाधन्तेऽस्मान्समुद्दृप्ता घोररूपाः पदे पदे} %7-6-25

\twolineshloka
{राक्षसैरभिभूताः स्म न शक्ताः स्म प्रजापते}
{स्वेषु सद्मसु संस्थातुं भयात्तेषां दुरात्मनाम्} %7-6-26

\twolineshloka
{तदस्माकं हितार्थाय जहि तांश्च त्रिलोचन}
{राक्षसान्हुङ्कृतेनैव दह प्रदहतां वर} %7-6-27

\twolineshloka
{इत्येवं त्रिदशैरुक्तो निशम्यान्धकसूदनः}
{शिरः करं च धुन्वान इदं वचनमब्रवीत्} %7-6-28

\twolineshloka
{अवध्या मम ते देवाः सुकेशतनया रणे}
{मन्त्रं तु वः प्रदास्यामि यस्तान्वै निहनिष्यति} %7-6-29

\twolineshloka
{योऽसौ चक्रगदापाणिः पीतवासा जनार्दनः}
{हरिर्नारायणः श्रीमान् शरणं तं प्रपद्यथ} %7-6-30

\twolineshloka
{हरादवाप्य ते मन्त्रं कामारिमभिवाद्य च}
{नारायणालयं प्राप्य तस्मै सर्वं न्यवेदयन्} %7-6-31

\twolineshloka
{ततो नारायणेनोक्ता देवा इन्द्रपुरोगमाः}
{सुरारींस्तान्हनिष्यामि सुरा भवत विज्वराः} %7-6-32

\twolineshloka
{देवानां भयभीतानां हरिणा राक्षसर्षभौ}
{प्रतिज्ञातो वधोऽस्माकं चिन्त्यतां यदिह क्षमम्} %7-6-33

\twolineshloka
{हिरण्यकशिपोर्मृत्युरन्येषां च सुरद्विषाम्}
{नमुचिः कालनेमिश्च संह्रादो वीरसत्तमः} %7-6-34

\twolineshloka
{राधेयो बहुमायी च लोकपालोऽथ धार्मिकः}
{यमलार्जुनौ च हार्दिक्यः शुम्भश्चैव निशुम्भकः} %7-6-35

\twolineshloka
{असुरा दानवाश्चैव सत्त्ववन्तो महाबलाः}
{सर्वे समरमासाद्य न श्रूयन्तेऽपराजिताः} %7-6-36

\twolineshloka
{सर्वैः क्रतुशतैरिष्टं सर्वे मायाविदस्तथा}
{सर्वे सर्वास्त्रकुशलाः सर्वे शत्रुभयङ्कराः} %7-6-37

\twolineshloka
{नारायणेन निहताः शतशोऽथ सहस्रशः}
{एतज्ज्ञात्वा तु सर्वेषां क्षमं कर्तुमिहार्हथ} %7-6-38

\twolineshloka
{ततः सुमाली माली च श्रुत्वा माल्यवतो वचः}
{ऊचतुर्भ्रातरं ज्येष्ठं भगांशाविव वासवम्} %7-6-39

\twolineshloka
{स्वधीतं दत्तमिष्टं चाप्यैश्वर्यं परिपालितम्}
{आयुर्निरामयं प्राप्तं सुधर्मः प्रापितः पथि} %7-6-40

\twolineshloka
{देवसागरमक्षोभ्यं शस्त्रैः समवगाह्य च}
{जिता द्विषो ह्यप्रतिमास्तन्नो मृत्युकृतं भयम्} %7-6-41

\twolineshloka
{नारायणश्च रुद्रश्च शक्रश्चापि यमस्तथा}
{अस्माकं प्रमुखे स्थातुं सर्वे बिभ्यति सर्वदा} %7-6-42

\twolineshloka
{विष्णोर्देवस्य नास्त्येव कारणं राक्षसेश्वर}
{देवानामेव दोषेण विष्णोः प्रचलितं मनः} %7-6-43

\twolineshloka
{तस्मादद्य समुद्युक्ताः सर्वसैन्यसमावृताः}
{देवानेव जिघांसाम एभ्यो दोषः समुत्थितः} %7-6-44

\twolineshloka
{एवं सम्मन्त्र्य बलिनः सर्वे सैन्यसमावृताः}
{उद्योगं घोषयित्वा तु सर्वे नैर्ऋतपुङ्गवाः युद्धाय निर्ययुः क्रुद्धा जम्भवृत्रबला इव} %7-6-45

\twolineshloka
{इति ते राम सम्मन्त्र्य सर्वोद्योगेन राक्षसाः}
{युद्धाय निर्ययुः सर्वे महाकाया महाबलाः} %7-6-46

\twolineshloka
{स्यन्दनैर्वारणैश्चैव हयैश्च गिरिसन्निभैः}
{खरैर्गोभी रथोष्ट्रैश्च शिंशुमारैर्भुजङ्गमैः} %7-6-47

\twolineshloka
{मकरैः कच्छपैर्मीनैर्विहङ्गैर्गरुडोपमैः}
{सिंहैर्व्याघ्रैर्वराहैश्च सृमरैश्चमरैरपि} %7-6-48

\twolineshloka
{त्यक्त्वा लङ्कां गताः सर्वे राक्षसा बलगर्विताः}
{प्रयाता देवलोकाय योद्धुं दैवतशत्रवः} %7-6-49

\twolineshloka
{लङ्काविपर्ययं दृष्ट्वा यानि लङ्कालयान्यथ}
{भूतानि भयदर्शीनि विमनस्कानि सर्वशः} %7-6-50

\onelineshloka
{रथोत्तमैरुह्यमानाः शतशोऽथ सहस्रशः} %7-6-51

\twolineshloka
{प्रयाता राक्षसास्तूर्णं देवलोकं प्रयत्नतः}
{रक्षसामेव मार्गेण दैवतान्यपचक्रमुः} %7-6-52

\twolineshloka
{भौमाश्चैवान्तरिक्षाश्च कालाज्ञप्ता भयावहाः}
{उत्पाता राक्षसेन्द्राणामभावाय समुत्थिताः} %7-6-53

\twolineshloka
{अस्थीनि मेघा ववृषुरुष्णं शोणितमेव च}
{वेलां समुद्राश्चोत्क्रान्ताश्चेलुश्चाप्यथ भूधराः} %7-6-54

\twolineshloka
{अट्टहासान्विमुञ्चन्तो घननादसमस्वनाः}
{वाश्यन्त्यश्च शिवास्तत्र दारुणं घोरदर्शनाः} %7-6-55

\twolineshloka
{सम्पतन्त्यथ भूतानि दृश्यन्ते च यथाक्रमम्}
{गृध्रचक्रं महच्चात्र ज्वलनोद्गारिभिर्मुखैः} %7-6-56

\twolineshloka
{राक्षसानामुपरि खे भ्रमतेऽलातचक्रवत्}
{कपोता रक्तपादाश्च शारिका विद्रुता ययुः काका वाश्यन्ति तत्रैव बिडाला वै द्विपादयः} %7-6-57

\twolineshloka
{उत्पातांस्ताननादृत्य राक्षसा बलगर्विताः}
{यान्त्येव न निवर्त्तन्ते मृत्युपाशावपाशिताः} %7-6-58

\twolineshloka
{माल्यवांश्च सुमाली च माली च सुमहाबलाः}
{आसन्पुरःसरास्तेषां क्रतूनामिव पावकाः} %7-6-59

\twolineshloka
{माल्यवन्तं च ते सर्वे माल्यवन्तमिवाचलम्}
{निशाचरा ह्याश्रयन्ति धातारमिव देवताः} %7-6-60

\twolineshloka
{तद्बलं राक्षसेन्द्राणां महाभ्रघननादितम्}
{जयेप्सया देवलोकं ययौ मालिवशे स्थितम्} %7-6-61

\twolineshloka
{राक्षसानां समुद्योगं तं तु नारायणः प्रभुः}
{देवदूतादुपश्रुत्य चक्रे युद्धे तदा मनः} %7-6-62

\twolineshloka
{स सज्जायुधतूणीरो वैनतेयोपरि स्थितः}
{आसज्ज्य कवचं दिव्यं सहस्रार्कसमद्युति} %7-6-63

\twolineshloka
{आबध्य शरसम्पूर्णे इषुधी विमले तदा}
{श्रोणिसूत्रं च खड्गं च विमलं कमलेक्षणः शङ्खचक्रगदाशार्ङ्गखड्गाख्यप्रवरायुधान्} %7-6-64

\twolineshloka
{सुपर्णं गिरिसङ्काशं वैनतेयमथास्थितः}
{राक्षसानामभावाय ययौ तूर्णतरं प्रभुः} %7-6-65

\twolineshloka
{सुपर्णपृष्ठे स बभौ श्यामः पीताम्बरो हरिः}
{काञ्चनस्य गिरेः शृङ्गे सतडित्तोयदो यथा} %7-6-66

\twolineshloka
{स सिद्धदेवर्षिमहोरगैश्च गन्धर्वयक्षैरुपगीयमानः}
{समाससादासुरसैन्यशत्रूंश्चक्रासिशार्ङ्गायुधशङ्खपाणिः} %7-6-67

\twolineshloka
{सुपर्णपक्षानिलनुन्नपक्षं भ्रमत्पताकं प्रविकीर्णशस्त्रम्}
{चचाल तद्राक्षसराजसैन्यं चलोपलं नील इवाचलेन्द्रः} %7-6-68

\twolineshloka
{ततः शरैः शोणितमांसरूषितैर्युगान्तवैश्वानरतुल्यविग्रहैः}
{निशाचराः सम्परिवार्य माधवं वरायुधैर्निर्बिभिदुः सहस्रशः} %7-6-69


॥इत्यार्षे श्रीमद्रामायणे वाल्मीकीये आदिकाव्ये उत्तरकाण्डे विष्णुमाल्यवदादियुद्धम् नाम षष्ठः सर्गः ॥७-६॥
