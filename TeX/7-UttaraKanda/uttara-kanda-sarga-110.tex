\sect{दशाधिकशततमः सर्गः — सहानुगश्रीरामस्वर्गारोहः}

\twolineshloka
{अध्यर्धयोजनं गत्वा नदीं पश्चान्मुखाश्रिताम्}
{सरयूं पुण्यसलिलां ददर्श रघुनन्दनः} %7-110-1

\twolineshloka
{तां नदीमाकुलावर्तां सर्वत्रानुसरन्नृपः}
{आगतः सप्रजो रामस्तं देशं रघुनन्दनः} %7-110-2

\twolineshloka
{अथ तस्मिन्मुहूर्ते तु ब्रह्मा लोकपितामहः}
{सर्वैः परिवृतो देवैर्ऋषिभिश्च महात्मभिः} %7-110-3

\twolineshloka
{आययौ यत्र काकुत्स्थः स्वर्गाय समुपस्थितः}
{विमानशतकोटीभिर्दिव्याभिरभिसंवृतः} %7-110-4

\twolineshloka
{दिव्यतेजोवृतं व्योम ज्योतिर्भूतमनुत्तमम्}
{स्वयम्प्रभैः स्वतेजोभिः स्वर्गिभिः पुण्यकर्मभिः} %7-110-5

\twolineshloka
{पुण्या वाता ववुश्चैव गन्धवन्तः सुखप्रदाः}
{पपात पुष्पवृष्टिश्च देवैर्मुक्ता महौघवत्} %7-110-6

\twolineshloka
{तस्मिंस्तूर्यशतैः कीर्णे गन्धर्वाप्सरसङ्कुले}
{सरयूसलिलं रामः पद्भ्यां समुपचक्रमे} %7-110-7

\twolineshloka
{ततः पितामहो वाणीमन्तरिक्षादभाषत}
{आगच्छ विष्णो भद्रं ते दिष्ट्या प्राप्तोऽसि राघव} %7-110-8

\threelineshloka
{भ्रातृभिः सह देवाभैः प्रविशस्व स्विकां तनुम्}
{यामिच्छसि महाबाहो तां तनुं प्रविश स्विकाम्}
{वैष्णवीं तां महातेजस्तद्वाकाशं सनातनम्} %7-110-9

\threelineshloka
{त्वं हि लोकगतिर्देव न त्वां केचित्प्रजानते}
{ऋते मायां विशालाक्षीं तव पूर्वपरिग्रहाम्}
{त्वामचिन्त्यं महद्भूतमक्षयं सर्वसङ्ग्रहम्} %7-110-10

\onelineshloka
{यामिच्छसि महातेजस्तां तनुं प्रविश स्वयम्} %7-110-11

\threelineshloka
{ां विशालाक्षीं तव पूर्वपरिग्रहाम्}
{पितामहवचः श्रुत्वा विनिश्चित्य महामतिः}
{विवेश वैष्णवं तेजः सशरीरः सहानुजः} %7-110-12

\twolineshloka
{ततो विष्णुमयं देवं पूजयन्ति स्म देवताः}
{साध्या मरुद्गणाश्चैव सेन्द्राः साग्निपुरोगमाः} %7-110-13

\twolineshloka
{ये च दिव्या ऋषिगणा गन्धर्वाप्सरसश्च याः}
{सुपर्णनागयक्षाश्च दैत्यदानवराक्षसाः} %7-110-14

\twolineshloka
{सर्वं पुष्टं प्रमुदितं सुसम्पूर्णमनोरथम्}
{साधु साध्विति तैर्देवैस्त्रिदिवं गतकल्मषम्} %7-110-15

\twolineshloka
{अथ विष्णुर्महातेजाः पितामहमुवाच ह}
{एषां लोकं जनौघानां दातुमर्हसि सुव्रत} %7-110-16

\twolineshloka
{इमे हि सर्वे स्नेहान्मामनुयाता यशस्विनः}
{भक्ता हि भजितव्याश्च त्यक्तात्मानश्च मत्कृते} %7-110-17

\twolineshloka
{तच्छ्रुत्वा विष्णुवचनं ब्रह्मा लोकगुरुः प्रभुः}
{लोकान्सान्तानिकान्नाम यास्यन्तीमे समागताः} %7-110-18

\twolineshloka
{यच्च तिर्यग्गतं किञ्चित्त्वामेवमनुचिन्तयत्}
{प्राणांस्त्यक्ष्यति भक्त्या वै तत्सन्ताने निवत्स्यति} %7-110-19

\onelineshloka
{सर्वैर्ब्रह्मगुणैर्युक्ते ब्रह्मलोकादनन्तरे} %7-110-20

\twolineshloka
{वानराश्च स्विकां योनिमृक्षाश्चैव तथा ययुः}
{येभ्यो विनिस्सृताः सर्वे सुरेभ्यः सुरसम्भवाः} %7-110-21

\twolineshloka
{तेषु प्रविविशे चैव सुग्रीवः सूर्यमण्डलम्}
{पश्यतां सर्वदेवानां स्वान्पितऽन्प्रतिपेदिरे} %7-110-22

\twolineshloka
{तथोक्तवति देवेशे गोप्रतारमुपागताः}
{भेजिरे सरयूं सर्वे हर्षपूर्णाश्रुविक्लवाः} %7-110-23

\twolineshloka
{अवगाह्य जलं यो यः प्राणी ह्यासीत् प्रहृष्टवत्}
{मानुषं देहमुत्सृज्य विमानं सोऽध्यरोहत} %7-110-24

\threelineshloka
{तिर्यग्योनिगतानां च शतानि सरयूजलम्}
{सम्प्राप्य त्रिदिवं जग्मुः प्रभासुरवपूंषि च}
{दिव्या दिव्येन वपुषा देवा दीप्ता इवाभवन्} %7-110-25

\twolineshloka
{गत्वा तु सरयूतोयं स्थावराणि चराणि च}
{प्राप्य तत्तोयविक्लेदं देवलोकमुपागमन्} %7-110-26

\twolineshloka
{तस्मिन्नपि समापन्ना ऋक्षवानरराक्षसाः}
{तेऽपि स्वर्गं प्रविविशुर्देहान्निक्षिप्य चाम्भसि} %7-110-27

\twolineshloka
{ततः समागतान्सर्वान्स्थाप्य लोकगुरुर्दिवि}
{जगाम त्रिदशैः सार्धं सदा हृष्टैर्दिवं महत्} %7-110-28


॥इत्यार्षे श्रीमद्रामायणे वाल्मीकीये आदिकाव्ये उत्तरकाण्डे सहानुगश्रीरामस्वर्गारोहः नाम दशाधिकशततमः सर्गः ॥७-११०॥
