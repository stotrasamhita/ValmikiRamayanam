\sect{अष्टाशीतितमः सर्गः — बुधसमागमः}

\twolineshloka
{तां कथामिलसम्बद्धां रामेण समुदीरिताम्}
{लक्ष्मणो भरतश्चैव श्रुत्वा परमविस्मितौ} %7-88-1

\twolineshloka
{तौ रामं प्राञ्जली भूत्वा तस्य राज्ञो महात्मनः}
{विस्तरं तस्य भावस्य तदा पप्रच्छतुः पुनः} %7-88-2

\twolineshloka
{कथं स राजा स्त्रीभूतो वर्तयामास दुर्गतिः}
{पुरुषः स यदा भूतः कां वृत्तिं वर्तयत्यसौ} %7-88-3

\twolineshloka
{तयोस्तद्भाषितं श्रुत्वा कौतूहलसमन्वितम्}
{कथयामास काकुत्स्थस्तस्य राज्ञो यथागतम्} %7-88-4

\twolineshloka
{तमेव प्रथमं मासं स्त्रीभूता लोकसुन्दरी}
{ताभिः परिवृता स्त्रीभिर्ये च पूर्वपदानुगाः} %7-88-5

\twolineshloka
{तत्काननं विगाह्वाशु विजह्रे लोकसुन्दरी}
{द्रुमगुल्मलताकीर्णं पद्भ्यां पद्मदलेक्षणा} %7-88-6

\twolineshloka
{वाहनानि च सर्वाणि सा त्यक्त्वा वै समन्ततः}
{पर्वताभोगविवरे तस्मिन्रेमे इला तदा} %7-88-7

\twolineshloka
{अथ तस्मिन्वनोद्देशे पर्वतस्याविदूरतः}
{सरः सुरुचिरप्रख्यं नानापक्षिगणैर्युतम्} %7-88-8

\twolineshloka
{ददर्श सा त्विला तस्मिन्बुधं सोमसुतं तदा}
{ज्वलन्तं स्वेन वपुषा पूर्णसोममिवोदितम्} %7-88-9

\twolineshloka
{तपन्तं च तपस्तीव्रमम्भोमध्ये दुरासदम्}
{यशस्करं कामकरं कारुण्ये पर्यवस्थितम्} %7-88-10

\twolineshloka
{सा तं जलाशयं सर्वं क्षोभयामास विस्मिता}
{सहगैः पूर्वपुरुषैः स्त्रीभूतै रघुनन्दन} %7-88-11

\twolineshloka
{बुधस्तु तां समीक्ष्यैव कामबाणवशं गतः}
{नोपलेभे तदाऽऽत्मानं सञ्चचाल तदाऽऽम्भसि} %7-88-12

\twolineshloka
{इलां निरीक्षमाणस्तु त्रैलोक्याभ्यधिकां शुभाम्}
{चिन्तां समभ्यतिक्रामत् का न्वियं देवताधिका} %7-88-13

\twolineshloka
{न देवीषु न नागीषु नासुरीष्वप्सरःसु च}
{दृष्टपूर्वा मया काचिद्रूपेणानेन शोभिता} %7-88-14

\twolineshloka
{सदृशीयं मम भवेद्यदि नान्यपरिग्रहः}
{इति बुद्धिं समास्थाय जलात्कूलमुपागमत्} %7-88-15

\twolineshloka
{आश्रमं समुपागम्य ततस्ताः प्रमदोत्तमाः}
{शब्दापयत धर्मात्मा ताश्चैनं च ववन्दिरे} %7-88-16

\twolineshloka
{स ताः पप्रच्छ धर्मात्मा कस्यैषा लोकसुन्दरी}
{किमर्थमागता चैव सर्वमाख्यात मा चिरम्} %7-88-17

\twolineshloka
{शुभं तु तस्य तद्वाक्यं मधुरं मधुराक्षरम्}
{श्रुत्वा स्त्रियश्च ताः सर्वा ऊचुर्मधुरया गिरा} %7-88-18

\twolineshloka
{अस्माकमेषा सुश्रोणी प्रभुत्वे वर्तते सदा}
{अपतिः काननान्तेषु सहास्माभिश्चरत्यसौ} %7-88-19

\twolineshloka
{तद्वाक्यमव्यक्तपदं तासां स्त्रीणां निशम्य च}
{विद्यामावर्तनीं पुण्यामावर्तयत स द्विजः} %7-88-20

\twolineshloka
{सोऽर्थं विदित्वा सकलं तस्य राज्ञो यथागतम्}
{सर्वा एव स्त्रियस्ताश्च बभाषे मुनिपुङ्गवः} %7-88-21

\twolineshloka
{अत्र किम्पुरुषीर्भूत्वा शैलरोधसि वत्स्यथ}
{आवासस्तु गिरावस्मिञ्छीघ्रमेव विधीयताम्} %7-88-22

\twolineshloka
{मूलपत्रफलैः सर्वा वर्तयिष्यथ नित्यदा}
{स्त्रियः किम्पुरुषान्नाम भर्तऽन्समुपलप्स्यथ} %7-88-23

\twolineshloka
{ताः श्रुत्वा सोमपुत्रस्य वाचं किम्पुरुषीकृताः}
{उपासाञ्चक्रिरे शैलं वध्वस्ता बहुलास्तदा} %7-88-24


॥इत्यार्षे श्रीमद्रामायणे वाल्मीकीये आदिकाव्ये उत्तरकाण्डे बुधसमागमः नाम अष्टाशीतितमः सर्गः ॥७-८८॥
