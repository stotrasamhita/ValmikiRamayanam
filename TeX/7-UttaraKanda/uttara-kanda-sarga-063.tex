\sect{त्रिषष्ठितमः सर्गः — लवणवधोपायकथनम्}

\twolineshloka
{एवमुक्तस्तु रामेण परां व्रीडामुपागमत्}
{शत्रुघ्नो वीर्यसम्पन्नो मन्दं मन्दमुवाच ह} %7-63-1

\twolineshloka
{अधर्मं विद्म काकुत्स्थ ह्यस्मिन्नर्थे नरेश्वर}
{कथं तिष्ठत्सु ज्येष्ठेषु कनीयानभिषिच्यते} %7-63-2

\twolineshloka
{अवश्यं करणीयं च शासनं पुरुषर्षभ}
{तव चैव महाभाग शासनं दुरतिक्रमम्} %7-63-3

\twolineshloka
{त्वत्तो मया श्रुतं वीर श्रुतिभ्यश्च मया श्रुतम्}
{नोत्तरं हि मया वाच्यं मध्यमे प्रतिजानति} %7-63-4

\twolineshloka
{व्याहृतं दुर्वचो घोरं हन्तास्मि लवणं मृधे}
{तस्यैवं मे दुरुक्तस्य दुर्गतिः पुरुषर्षभ} %7-63-5

\twolineshloka
{उत्तरं नहि वक्तव्यं ज्येष्ठेनाभिहिते पुनः}
{अधर्मसहितं चैव परलोकविवर्जितम्} %7-63-6

\twolineshloka
{सोऽहं द्वितीयं काकुत्स्थ न वक्ष्यामि तवोत्तरम्}
{मा द्वितीयेन दण्डो वै निपतेन्मयि मानद} %7-63-7

\twolineshloka
{कामकारो ह्यहं राजंस्तवास्मि पुरुषर्षभ}
{अधर्मं जहि काकुत्स्थ मत्कृते रघुनन्दन} %7-63-8

\twolineshloka
{एवमुक्ते तु शूरेण शत्रुघ्नेन महात्मना}
{उवाच रामः संहृष्टो भरतं लक्ष्मणं तथा} %7-63-9

\twolineshloka
{सम्भारानभिषेकस्य आनयध्वं समाहिताः}
{अद्यैव पुरुषव्याघ्रमभिषेक्ष्यामि राघवम्} %7-63-10

\twolineshloka
{पुरोहितं च काकुत्स्थं नैगमानृत्विजस्तथा}
{मन्त्रिणश्चैव तान्सर्वानानयध्वं ममाज्ञया} %7-63-11

\threelineshloka
{राज्ञः शासनमाज्ञाय तथाऽकुर्वन्महारथाः}
{अभिषेकसमारम्भं पुरस्कृत्य पुरोधसम्}
{प्रविष्टा राजभवनं राजानो ब्राह्मणास्तथा} %7-63-12

\twolineshloka
{तथोऽभिषेको ववृधे शत्रुघ्नस्य महात्मनः}
{सम्प्रहर्षकरः श्रीमान्राघवस्य पुरस्य च} %7-63-13

\twolineshloka
{अभिषिक्तस्तु शत्रुघ्नो बभौ चादित्यसन्निभः}
{अभिषिक्तः पुरा स्कन्दः सेन्द्रैरिव मरुद्गणैः} %7-63-14

\twolineshloka
{अभिषिक्ते तु शत्रुघ्ने रामेणाक्लिष्टकर्मणा}
{पौराः प्रमुदिताश्चासन्ब्राह्मणाश्च बहुश्रुताः} %7-63-15

\twolineshloka
{कौसल्या च सुमित्रा च मङ्गलं केकयी तथा}
{चक्रुस्ता राजभवने याश्चान्या राजयोषितः} %7-63-16

\twolineshloka
{ऋषयश्च महात्मानो यमुनातीरवासिनः}
{हतं लवणमाशंसुः शत्रुघ्नस्याभिषेचनात्} %7-63-17

\twolineshloka
{ततोऽभिषिक्तं शत्रुघ्नमङ्कमारोप्य राघवः}
{उवाच मधुरां वाणीं तेजस्तस्याभिपूरयन्} %7-63-18

\twolineshloka
{अयं शरस्त्वमोघस्ते दिव्यः परपुरञ्जयः}
{अनेन लवणं सौम्य हन्ताऽसि रघुनन्दन} %7-63-19

\onelineshloka
{सृष्टः शरोऽयं काकुत्स्थ यदा शेते महार्णवे} %7-63-20

\twolineshloka
{स्वयम्भूरजितो देवो यन्नापश्यन्सुरासुराः}
{अदृश्यः सर्वभूतानां तेनायं तु शरोत्तमः} %7-63-21

\threelineshloka
{सृष्टः क्रोधाभिभूतेन विनाशार्थं दुरात्मनोः}
{मधुकैटभयोर्वीर विघाते वर्तमानयोः}
{स्रष्टुकामेन लोकांस्त्रींस्तौ चानेन हतौ युधि} %7-63-22

\twolineshloka
{तौ हत्वा जनभोगार्थं कैटभं तु मधुं तथा}
{अनेन शरमुख्येन ततो लोकांश्चकार सः} %7-63-23

\twolineshloka
{नायं मया शरः पूर्वं रावणस्य वधार्थिना}
{मुक्तः शत्रुघ्न भूतानां महांस्त्रासो भवेदिति} %7-63-24

\twolineshloka
{यच्च तस्य महच्छूलं त्र्यम्बकेण महात्मना}
{दत्तं शत्रुविनाशाय मधोरायुधमुत्तमम्} %7-63-25

\twolineshloka
{स तं निक्षिप्य भवने पूज्यमान पुनः पुनः}
{दिशः सर्वाः समासाद्य प्राप्नोत्याहारमुत्तमम्} %7-63-26

\twolineshloka
{यदा तु युद्धमाकाङ्क्षन् कश्चिदेनं समाह्वयेत्}
{तदा शूलं गहीत्वा तं भस्म रक्षः करोति हि} %7-63-27

\twolineshloka
{स त्वं पुरुषशार्दूल तमायुधविनाकृतम्}
{अप्रविष्टं पुरं पूर्वं द्वारि तिष्ठ धृतायुधः} %7-63-28

\twolineshloka
{अप्रविष्टं च भवनं युद्धाय पुरुषर्षभ}
{आह्वयेथा महाबाहो ततो हन्तासि राक्षसम्} %7-63-29

\twolineshloka
{अन्यथा क्रियमाणे तु अवध्यः स भविष्यति}
{यदि त्वेवं कृते वीर विनाशमुपयास्यति} %7-63-30

\twolineshloka
{एतत्ते सर्वमाख्यातं शूलस्य च विपर्ययः}
{श्रीमतः शितिकण्ठस्य कृत्यं हि दुरतिक्रमम्} %7-63-31


॥इत्यार्षे श्रीमद्रामायणे वाल्मीकीये आदिकाव्ये उत्तरकाण्डे लवणवधोपायकथनम् नाम त्रिषष्ठितमः सर्गः ॥७-६३॥
