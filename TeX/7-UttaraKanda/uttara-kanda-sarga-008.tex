\sect{अष्टमः सर्गः — सुमाल्यादिनिग्रहः}

\twolineshloka
{हन्यमाने बले तस्मिन्पद्मनाभेन पृष्ठतः}
{माल्वान्सन्निवृत्तोऽथ वेलामेत्य इवार्णवः} %7-8-1

\twolineshloka
{संरक्तनयनः कोपाच्चलन्मौलिर्निशाचरः}
{पद्मनाभमिदं प्राह वचनं पुरुषोत्तमम्} %7-8-2

\twolineshloka
{नारायण न जानीषे क्षात्रधर्मं पुरातनम्}
{अयुद्धमनसो भीतानस्मान्हंसि यथेतरः} %7-8-3

\twolineshloka
{पराङ्मुखवधं पापं यः करोत्यसुरेतरः}
{स हन्ता न गतः स्वर्गं लभते पुण्यकर्मणाम्} %7-8-4

\twolineshloka
{युद्धश्रद्धाऽथवा तेऽस्ति शङ्खचक्रगदाधर}
{अहं स्थितोऽस्मि पश्यामि बलं दर्शय यत्तव} %7-8-5

\twolineshloka
{माल्यवन्तं स्थितं दृष्ट्वा माल्यवन्तमिवाचलम्}
{उवाच राक्षसेन्द्रं तं देवराजानुजो बली} %7-8-6

\twolineshloka
{युष्मत्तो भयभीतानां देवानां वै मयाऽभयम्}
{राक्षसोत्सादनं दत्तं तदेतदनुपाल्यते} %7-8-7

\twolineshloka
{प्राणैरपि प्रियं कार्यं देवानां हि सदा मया}
{सोऽहं वो निहनिष्यामि रसातलगतानपि} %7-8-8

\twolineshloka
{देवदेवं ब्रुवाणं तं रक्ताम्बुरुहलोचनम्}
{शक्त्या बिभेद सङ्क्रुद्धो राक्षसेन्द्रो भुजान्तरे} %7-8-9

\twolineshloka
{माल्यवद्भुजनिर्मुक्ता शक्तिर्घण्टाकृतस्वना}
{हरेरुरसि बभ्राज मेघस्थेव शतह्रदा} %7-8-10

\twolineshloka
{ततस्तामेव चोत्कृष्य शक्तिं शक्तिधरप्रियः}
{माल्यवन्तं समुद्दिश्य चिक्षेपाम्बुरुहेक्षणः} %7-8-11

\twolineshloka
{स्कन्दोत्सृष्टेव सा शक्तिर्गोविन्दकरनिस्सृता}
{काङ्क्षन्ती राक्षसं प्रायान्माहेन्द्रीवाञ्जनाचलम्} %7-8-12

\twolineshloka
{सा तस्योरसि विस्तीर्णे हारभाराऽवभासिते}
{अपतद्राक्षसेन्द्रस्य गिरिकूट इवाशनिः} %7-8-13

\twolineshloka
{तया भिन्नतनुत्राणः प्राविशद्विपुलं तमः}
{माल्यवान्पुनराश्वस्तस्तस्थौ गिरिरिवाचलः} %7-8-14

\twolineshloka
{ततः कार्ष्णायसं शूलं कण्टकैर्बहुभिर्वतम्}
{प्रगृह्याभ्यहनद्देवं स्तनयोरन्तरे दृढम्} %7-8-15

\twolineshloka
{तथैव रणरक्तस्तु मुष्टिना वासवानुजम्}
{ताडयित्वा धनुर्मात्रमपक्रान्तो निशाचरः} %7-8-16

\twolineshloka
{ततोऽम्बरे महाञ्छब्दः साधुसाध्विति चोदितः}
{आहत्य राक्षसो विष्णुं गरुडं चाप्यताडयत्} %7-8-17

\twolineshloka
{वैनतेयस्ततः क्रुद्धः पक्षवातेन राक्षसम्}
{व्यपोहद्बलवान्वायुः शुष्कपर्णचयं यथा} %7-8-18

\twolineshloka
{द्विजेन्द्रपक्षवातेन द्रावितं दृश्य पूर्वजम्}
{सुमाली स्वबलैः सार्धं लङ्कामभिमुखो ययौ} %7-8-19

\twolineshloka
{पक्षवातबलोद्धूतो माल्यवानपि राक्षसः}
{स्वबलेन समागम्य ययौ लङ्कां ह्रिया वृतः} %7-8-20

\twolineshloka
{एवं ते राक्षसा तेन हरिणा कमलेक्षण}
{बहुशः संयुगे भग्ना हतप्रवरनायकाः} %7-8-21

\twolineshloka
{अशक्नुवन्तस्ते विष्णुं प्रतियोद्धुं भयार्दिताः}
{त्यक्त्वा लङ्कां गता वस्तुं पातालं सहपत्नयः} %7-8-22

\twolineshloka
{सुमालिनं समासाद्य राक्षसं रघुसत्तम}
{स्थिताः प्रख्यातवीर्यास्ते वंशे सालकटङ्कटे} %7-8-23

\twolineshloka
{ये त्वया निहतास्ते तु पौलस्त्या नाम राक्षसाः}
{सुमाली माल्यवान्माली ये च तेषां पुरःसराः सर्वे तेभ्यो महाभागा रावणाद्बलवत्तराः} %7-8-24

\twolineshloka
{न चान्यो राक्षसान्हन्ता सुरारीन्देवकण्टकान्}
{ऋते नारायणं देवं शङ्खचक्रगदाधरम्} %7-8-25

\twolineshloka
{भवान्नारायणो देवश्चतुर्बाहुः सनातनः}
{राक्षसान्हन्तुमुत्पन्नो ह्यजेयः प्रभुरव्ययः} %7-8-26

\twolineshloka
{नष्टधर्मव्यवस्थाता काले काले प्रजाकरः}
{उत्पद्यते दस्युवधे शरणागतवत्सलः} %7-8-27

\twolineshloka
{एषा मया तव नराधिप राक्षसानामुत्पत्तिरद्य कथिता सकला यथावत्}
{भूयो निबोध रघुसत्तम रावणस्य जन्मप्रभावमतुलं ससुतस्य सर्वम्} %7-8-28

\twolineshloka
{चिरात्सुमाली व्यचरद्रसातलं स राक्षसो विष्णुभयार्दितस्तदा}
{पुत्रैश्च पौत्रैश्च समन्वितो बली ततस्तु लङ्कामवसद्धनेश्वरः} %7-8-29


॥इत्यार्षे श्रीमद्रामायणे वाल्मीकीये आदिकाव्ये उत्तरकाण्डे सुमाल्यादिनिग्रहः नाम अष्टमः सर्गः ॥७-८॥
