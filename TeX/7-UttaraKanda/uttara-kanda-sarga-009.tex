\sect{नवमः सर्गः — रावणाद्युत्पत्तिः}

\twolineshloka
{कस्यचित्त्वथ कालस्य सुमाली नाम राक्षसः}
{रसातलान्मर्त्यलोकं सर्वं वै विचचार ह} %7-9-1

\twolineshloka
{नीलजीमूतसङ्काशस्तप्तकाञ्चनकुण्डलः}
{कन्यां दुहितरं गृह्य विना पद्ममिव श्रियम्} %7-9-2

\twolineshloka
{राक्षसेन्द्रः स तु तदा विचरन्वै महीतलम्}
{तदापश्यत्स गच्छन्तं पुष्पकेण धनेश्वरम्} %7-9-3

\twolineshloka
{गच्छन्तं पितरं द्रष्टुं पुलस्त्यतनयं विभुम्}
{तं दृष्ट्वाऽमरसङ्काशं स्वच्छन्दं तपनोपमम्} %7-9-4

\twolineshloka
{रसातलं प्रविष्टः सन्मर्त्यलोकात्सविस्मयः}
{इत्येवं चिन्तयामास राक्षसानां महामतिः} %7-9-5

\twolineshloka
{किं कृतं श्रेय इत्येवं वर्धेमहि कथं वयम्}
{अथाब्रवीत्सुतां रक्षः कैकसीं नाम नामतः} %7-9-6

\twolineshloka
{पुत्रि प्रदानकालोऽयं यौवनं व्यतिवर्तते}
{प्रत्याख्यानाच्च भीतैस्त्वं न वरैः प्रतिगृह्यसे} %7-9-7

\threelineshloka
{त्वत्कृते च वयं सर्वे यन्त्रिता धर्मबुद्धयः}
{त्वं हि सर्वगुणोपेता श्रीः साक्षादिव पुत्रिके}
{कन्यापितृत्वं दुःखं हि सर्वेषां मानकाङ्क्षिणाम्} %7-9-8

\onelineshloka
{न ज्ञायते च कः कन्यां वरयेदिति कन्यके} %7-9-9

\threelineshloka
{त्वं हि सर्वगुणोपेता श्रीः साक्षादिव पुत्रिके}
{मातुः कुलं पितृकुलं यत्र चैव प्रदीयते}
{कुलत्रयं सदा कन्या संशये स्थाप्य तिष्ठति} %7-9-10

\twolineshloka
{सा त्वं मुनिवरं श्रेष्ठं प्रजापतिकुलोद्भवम्}
{भज विश्रवसं पुत्रि पौलस्त्यं वरय स्वयम्} %7-9-11

\twolineshloka
{ईदृशास्ते भविष्यन्ति पुत्राः पुत्रि न संशयः}
{तेजसा भास्करसमो यादृशोऽयं धनेश्वरः} %7-9-12

\twolineshloka
{सा तु तद्वचनं श्रुत्वा कन्यका पितृगौरवात्}
{तत्रोपागम्य सा तस्थौ विश्रवा यत्र तप्यते} %7-9-13

\twolineshloka
{एतस्मिन्नन्तरे राम पुलस्त्यतनयो द्विजः}
{अग्निहोत्रमुपातिष्ठच्चतुर्थ इव पावकः} %7-9-14

\threelineshloka
{अविचिन्त्य तु तां वेलां दारुणां पितृगौरवात्}
{उपसृत्याग्रतस्तस्य चरणाधोमुखी स्थिता}
{विलिखन्ती मुहुर्भूमिमङ्गुष्ठाग्रेण भामिनी} %7-9-15

\twolineshloka
{स तु तां वीक्ष्य सुश्रोणीं पूर्णचन्द्रनिभाननाम्}
{अब्रवीत्परमोदारो दीप्यमानां स्वतेजसा} %7-9-16

\twolineshloka
{भद्रे कस्यासि दुहिता कुतो वा त्वमिहागता}
{किं कार्यं कस्य वा हेतोस्तत्त्वतो ब्रूहि शोभने} %7-9-17

\twolineshloka
{एवमुक्ता तु सा कन्या कृताञ्जलिरथाब्रवीत्}
{आत्मप्रभावेण मुने ज्ञातुमर्हसि मे मतम्} %7-9-18

\twolineshloka
{किं तु मां विद्धि ब्रह्मर्षे शासनात्पितुरागताम्}
{कैकसी नाम नाम्नाहं शेषं त्वं ज्ञातुमर्हसि} %7-9-19

\twolineshloka
{स तु गत्वा मुनिर्ध्यानं वाक्यमेतदुवाच ह}
{विज्ञातं ते मया भद्रे कारणं यन्मनोगतम्} %7-9-20

\twolineshloka
{सुताभिलाषो मत्तस्ते मत्तमातङ्गगामिनि}
{दारुणायां तु वेलायां यस्मात्त्वं मामुपस्थिता} %7-9-21

\twolineshloka
{शृणु तस्मात्सुतान्भद्रे यादृशाञ्जनयिष्यसि}
{दारुणान्दारुणाकारान्दारुणाभिजनप्रियान्} %7-9-22

\twolineshloka
{प्रसविष्यसि सुश्रोणि राक्षसान्क्रूरकर्मणः}
{सा तु तद्वचनं श्रुत्वा प्रणिपत्याब्रवीद्वचः} %7-9-23

\twolineshloka
{भगवन्नीदृशान्पुत्रांस्त्वत्तोऽहं ब्रह्मवादिनः}
{नेच्छामि सुदुराचारान्प्रसादं कर्तुमर्हसि} %7-9-24

\twolineshloka
{कन्यया त्वेवमुक्तस्तु विश्रवा मुनिपुङ्गवः}
{उवाच कैकसीं भूयः पूर्णेन्दुरिव रोहिणीम्} %7-9-25

\twolineshloka
{पश्चिमो यस्तव सुतो भविष्यति शुभानने}
{मम वंशानुरूपः स धर्मात्मा च भविष्यति} %7-9-26

\twolineshloka
{एवमुक्ता तु सा कन्या राम कालेन केनचित्}
{जनयामास बीभत्सं रक्षोरूपं सुदारुणम्} %7-9-27

\twolineshloka
{दशग्रीवं महादंष्ट्रं नीलाञ्जनचयोपमम्}
{ताम्रोष्ठं विंशतिभुजं महास्यं दीप्तमूर्धजम्} %7-9-28

\twolineshloka
{तस्मिञ्जाते तु तत्काले सज्वालकवलाः शिवाः}
{क्रव्यादाश्चापसव्यानि मण्डलानि प्रचक्रमुः} %7-9-29

\twolineshloka
{ववर्ष रुधिरं देवो मेघाश्च खरनिःस्वनाः}
{प्रबभौ न च सूर्यो वै महोल्काश्चापतन्भुवि} %7-9-30

\twolineshloka
{चकम्पे जगती चैव ववुर्वाताः सुदारुणाः}
{अक्षोभ्यः क्षुभितश्चैव समुद्रः सरितां पतिः} %7-9-31

\twolineshloka
{अथ नामाकरोत्तस्य पितामहसमः पिता}
{दशग्रीवः प्रसूतोऽयं दशग्रीवो भविष्यति} %7-9-32

\twolineshloka
{तस्य त्वनन्तरं जातः कुम्भकर्णो महाबलः}
{प्रमाणाद्यस्य विपुलं प्रमाणं नेह विद्यते} %7-9-33

\twolineshloka
{ततः शूर्पणखा नाम सञ्जज्ञे विकृतानना}
{विभीषणश्च धर्मात्मा कैकस्याः पश्चिमः सुतः} %7-9-34

\onelineshloka
{तस्मिञ्जाते महासत्त्वे पुष्पवर्षं पपात ह} %7-9-35

\twolineshloka
{नभःस्थाने दुन्दुभयो देवानां प्राणदंस्तदा}
{वाक्यं चैवान्तरिक्षे च साधु साध्विति तत्तदा} %7-9-36

\twolineshloka
{तौ तु तत्र महारण्ये ववृधाते महौजसौ}
{कुम्भकर्णदशग्रीवौ लोकोद्वेगकरौ तदा} %7-9-37

\twolineshloka
{कुम्भकर्णः प्रमत्तस्तु महर्षीन्धर्मवत्सलान्}
{त्रैलोक्यं भक्षयन्नित्यासन्तुष्टो विचचार ह} %7-9-38

\twolineshloka
{विभीषणस्तु धर्मात्मा नित्यं धर्मे व्यवस्थितः}
{स्वाध्यायनियताहार उवास विजितेन्द्रियः} %7-9-39

\twolineshloka
{अथ वैश्रवणो देवस्तत्र कालेन केनचित्}
{आगतः पितरं द्रष्टुं पुष्पकेण धनेश्वरः} %7-9-40

\twolineshloka
{तं दृष्ट्वा कैकसी तत्र ज्वलन्तमिव तेजसा}
{आगम्य राक्षसी तत्र दशग्रीवमुवाच ह} %7-9-41

\twolineshloka
{पुत्र वैश्रवणं पश्य भ्रातरं तेजसा वृतम्}
{भ्रातृभावे समे चापि पश्यात्मानं त्वमीदृशम्} %7-9-42

\twolineshloka
{दशग्रीव तथा यत्नं कुरुष्वामितविक्रम}
{यथा त्वमसि मे पुत्र भव र्वैश्रवणोपमः} %7-9-43

\twolineshloka
{मातुस्तद्वचनं श्रुत्वा दशग्रीवः प्रतापवान्}
{अमर्षमतुलं लेभे प्रतिज्ञां चाकरोत्तदा} %7-9-44

\twolineshloka
{सत्यं ते प्रतिजानामि भ्रातृतुल्योऽधिकोऽपि वा}
{भविष्याम्योजसा चैव सन्तापं त्यज हृद्गतम्} %7-9-45

\twolineshloka
{ततस्तेनैव कोपेन दशग्रीवः सहानुजः}
{चिकीर्षुर्दुष्करं कर्म तपसे धृतमानसः} %7-9-46

\twolineshloka
{प्राप्स्यामि तपसा काममिति कृत्वाऽध्यवस्य च}
{आगच्छदात्मसिद्ध्यर्थं गोकर्णस्याश्रमं शुभम्} %7-9-47

\twolineshloka
{स राक्षसस्तत्र सहानुजस्तदा तपश्चकारातुलमुग्रविक्रमः}
{अतोषयच्चापि पितामहं विभुं ददौ स तुष्टश्च वराञ्जयावहान्} %7-9-48


॥इत्यार्षे श्रीमद्रामायणे वाल्मीकीये आदिकाव्ये उत्तरकाण्डे रावणाद्युत्पत्तिः नाम नवमः सर्गः ॥७-९॥
