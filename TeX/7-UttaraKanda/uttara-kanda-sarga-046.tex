\sect{षड्चत्वारिंशः सर्गः — सीतागङ्गातीरनयनम्}

\twolineshloka
{ततो रजन्यां व्युष्टायां लक्ष्मणो दीनचेतनः}
{सुमन्त्रमब्रवीद्वाक्यं मुखेन परिशुष्यता} %7-46-1

\twolineshloka
{सारथे तुरगाञ्छीघ्रं योजयस्व रथोत्तमे}
{स्वास्तीर्णं राजभवनात्सीतायाश्चासनं कुरु} %7-46-2

\twolineshloka
{सीता हि राजवचनादाश्रमं पुण्यकर्मणाम्}
{मया नेया महर्षीणां श्रीघ्रमानीयतां रथः} %7-46-3

\twolineshloka
{सुमन्त्रस्तु तथेत्युक्त्वा युक्तं परमवाजिभिः}
{रथं सुरुचिरप्रख्यं स्वास्तीर्णं सुखशय्यया} %7-46-4

\twolineshloka
{आनीयोवाच सौमित्रिं मित्राणां मानवर्धनम्}
{रथोऽयं समनुप्राप्तो यत्कार्यं क्रियतां प्रभो} %7-46-5

\twolineshloka
{एवमुक्तः सुमन्त्रेण राजवेश्मनि लक्ष्मणः}
{प्रविश्य सीतामासाद्य व्याजहार नरर्षभः} %7-46-6

\twolineshloka
{त्वया किलैष नृपतिर्वरं वै याचितः प्रभुः}
{नृपेण च प्रतिज्ञातमाज्ञप्तश्चाश्रमं प्रति} %7-46-7

\threelineshloka
{गङ्गीतीरे मया देवि ऋषीणामाश्रमाञ्छुभान्}
{शीघ्रं गत्वा तु वैदेहि शासनात्पार्थिवस्य नः}
{अरण्ये मुनिभिर्जुष्टे अपनेया भविष्यसि} %7-46-8

\twolineshloka
{एवमुक्ता तु वैदेही लक्ष्मणेन महात्मना}
{प्रहर्षमतुलं लेभे गमनं चाप्यरोचयत्} %7-46-9

\twolineshloka
{वासांसि च महार्हाणि रत्नानि विविधानि च}
{गृहीत्वा तानि वैदेही गमनायोपचक्रमे} %7-46-10

\twolineshloka
{इमानि मुनिपत्नीनां दास्याम्याभरणान्हम्}
{वस्त्राणि च महार्हाणि धनानि विविधानि च} %7-46-11

\twolineshloka
{सौमित्रिस्तु तथेत्युक्त्वा रथमारोप्य मैथिलीम्}
{प्रययौ शीघ्रतुरगै रामस्याज्ञामनुस्मरन्} %7-46-12

\twolineshloka
{अब्रवीच्च तदा सीता लक्ष्मणं लक्ष्मिवर्धनम्}
{अशुभानि बहून्येव पश्यामि रघुनन्दन} %7-46-13

\twolineshloka
{नयनं मे स्फुरत्यद्य गात्रोत्कम्पश्च जायते}
{हृदयं चैव सौमित्रे अस्वस्थमिव लक्षये} %7-46-14

\twolineshloka
{औत्सुक्यं परमं चापि अधृतिश्च परा मम}
{शून्यामेव च पश्यामि पृथिवीं पृथुलोचन} %7-46-15

\twolineshloka
{अपि स्वस्ति भवेत्तस्य भ्रातुस्ते भ्रातृवत्सल}
{श्वश्रूणां चैव मे वीर सर्वासामविशेषतः} %7-46-16

\twolineshloka
{पुरे जनपदे चैव कुशलं प्राणिनामपि}
{इत्यञ्जलिकृता सीता देवता अभ्ययाचत} %7-46-17

\twolineshloka
{लक्ष्मणोऽर्थं ततः श्रुत्वा शिरसा वन्द्य मैथिलीम्}
{शिवमित्यब्रवीद्धृष्टो हृदयेन विशुष्यता} %7-46-18

\twolineshloka
{ततो वासमुपागम्य गोमतीतीर आश्रमे}
{प्रभाते पुनरुत्थाय सौमित्रिः सूतमब्रवीत्} %7-46-19

\twolineshloka
{योजयस्व रथं शीघ्रमद्य भागीरथीजलम्}
{शिरसा धारयिष्यामि त्र्यम्बकः पर्वते यथा} %7-46-20

\twolineshloka
{सोऽश्वान् रज्वाथ चतुरो रथे युङ्क्त्वा मनोजवान्}
{आरोहस्वेति वैदेहीं सूतः प्राञ्जलिरब्रवीत्} %7-46-21

\twolineshloka
{सा तु सूतस्य वचनादारुरोह रथोत्तमम्}
{सीता सौमित्रिणा सार्धं सुमन्त्रेण च धीमता} %7-46-22

\twolineshloka
{आससाद विशालाक्षी गङ्गां पापविनाशिनीम्}
{अथार्धदिवसं गत्वा भागीरथ्या जलाशयम्} %7-46-23

\onelineshloka
{निरीक्ष्य लक्ष्मणो दीनः प्ररुरोद महास्वनः} %7-46-24

\twolineshloka
{सीता तु परमायत्ता दृष्ट्वा लक्ष्मणमातुरम्}
{उवाच वाक्यं धर्मज्ञा किमिदं रुद्यते त्वया} %7-46-25

\twolineshloka
{जाह्नवीतीरमासाद्य चिराभिलषितं मम}
{हर्षकाले किमर्थं मां विषादयसि लक्ष्मण} %7-46-26

\twolineshloka
{नित्यं त्वं रामपार्श्वेषु वर्तसे पुरुषर्षभ}
{कच्चिद्विनाकृतस्तेन द्विरात्रं शोकमागतः} %7-46-27

\twolineshloka
{ममापि दयितो रामो जीवितादपि लक्ष्मण}
{न चाहमेवं शोचामि मैवं त्वं बालिशो भव} %7-46-28

\twolineshloka
{तारयस्व च मां गङ्गां दर्शयस्व च तापसान्}
{ततो मुनिभ्यो दास्यामि वांसास्याभरणानि च} %7-46-29

\twolineshloka
{ततः कृत्वा महर्षीणां यथार्हमभिवादनम्}
{तत्र चैकां निशामुष्य यास्यामस्तां पुरीं पुनः} %7-46-30

\twolineshloka
{ममापि पद्मपत्राक्षं सिंहोरस्कं कृशोदरम्}
{त्वरते हि मनो द्रष्टुं रामं रमयतां वरम्} %7-46-31

\twolineshloka
{तस्यास्तद्वचनं श्रुत्वा प्रमृज्य नयने शुभे}
{नाविकानाह्वयामास लक्ष्मणः परवीरहा} %7-46-32

\onelineshloka
{इयं च सज्जा नौश्चेति दाशाः प्राञ्जलयोऽब्रुवन्} %7-46-33

\twolineshloka
{तितीर्षुर्लक्ष्मणो गङ्गां शुभां नावमुपारुहत्}
{गङ्गां सन्तारयामास लक्ष्मणस्तां समाहितः} %7-46-34


॥इत्यार्षे श्रीमद्रामायणे वाल्मीकीये आदिकाव्ये उत्तरकाण्डे सीतागङ्गातीरनयनम् नाम षड्चत्वारिंशः सर्गः ॥७-४६॥
