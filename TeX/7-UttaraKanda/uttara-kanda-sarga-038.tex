\sect{अष्टात्रिंशः सर्गः — जनकादिप्रतिप्रयाणम्}

\twolineshloka
{एवमास्ते महाबाहुरहन्यहनि राघवः}
{प्रशासत्सर्वकार्याणि पौरजानपदेषु च} %7-38-1

\twolineshloka
{ततः कतिपयाहःसु वैदेहं मिथिलाधिपम्}
{राघवः प्राञ्जलिर्भूत्वा वाक्यमेतदुवाच ह} %7-38-2

\twolineshloka
{भवान्हि गतिरव्यग्रा भवता पालिता वयम्}
{भवतस्तेजसोग्रेण रावणो निहतो मया} %7-38-3

\twolineshloka
{इक्ष्वाकूणां च सर्वेषां मैथिलानां च सर्वशः}
{अतुलाः प्रीतयो राजन्सम्बन्धकपुरोगमाः} %7-38-4

\twolineshloka
{तद्भवान्स्वपुरं यातु रत्नान्यादाय पार्थिव}
{भरतश्च सहायार्थं पृष्ठतस्तेऽनुयास्यति} %7-38-5

\twolineshloka
{स तथेति नृपः कृत्वा राघवं वाक्यमब्रवीत्}
{प्रीतोऽस्मि भवतो राजन्दर्शनेन नयेन च} %7-38-6

\twolineshloka
{यान्येतानि तु रत्नानि मदर्थं सञ्चितानि वै}
{दुहित्रे तानि वै राजन्सर्वाण्येव ददामि च} %7-38-7

\twolineshloka
{एवमुक्त्वा तु काकुत्स्थं जनको हृष्टमानसः}
{प्रययौ मिथिलां श्रीमांस्तमनुज्ञाय राघवम्} %7-38-8

\twolineshloka
{ततः प्रयाते जनके केकयं मातुलं प्रभुः}
{राघवः प्राञ्जलिर्भूत्वा वाक्यमेतदुवाच ह} %7-38-9

\twolineshloka
{इदं राज्यमहं चैव भरतश्च सलक्ष्मणः}
{आयत्तास्त्वं हि नो राजन्गतिश्च पुरुषर्षभ} %7-38-10

\twolineshloka
{राजा हि वृद्धः सन्तापं त्वदर्थमुपयास्यति}
{तस्माद्गमनमद्यैव रोचते तव पार्थिव} %7-38-11

\twolineshloka
{लक्ष्मणेनानुयात्रेण पृष्ठतोऽनुगमिष्यते}
{धनमादाय विपुलं रत्नानि विविधानि च} %7-38-12

\twolineshloka
{युधाजित्तु तथेत्याह गमनं प्रति राघवम्}
{रत्नानि च धनं चैव त्वय्येवाक्षय्यमस्त्विति} %7-38-13

\twolineshloka
{प्रदक्षिणं स राजानं कृत्वा केकयवर्धनः}
{रामेण हि कृतः पूर्वमभिवाद्य प्रदक्षिणम्} %7-38-14

\twolineshloka
{लक्ष्मणेन सहायेन प्रयातः केकयेश्वरः}
{हतेऽसुरे यथा वृत्रे विष्णुना सह वासवः} %7-38-15

\twolineshloka
{तं विसृज्य ततो रामो वयस्यमकुतोभयम्}
{प्रतर्दनं काशिपतिं परिष्वज्येदमब्रवीत्} %7-38-16

\twolineshloka
{दर्शिता भवता प्रीतिर्दर्शितं सौहृदं परम्}
{उद्योगश्च कृतो राजन्भरतेन त्वया सह} %7-38-17

\twolineshloka
{तद्भवानद्य काशेय पुरीं वाराणसीं व्रज}
{रमणीयां त्वया गुप्तां सुप्रकाशां सुतोरणाम्} %7-38-18

\twolineshloka
{एतावदुक्त्वा चोत्थाय काकुत्स्थः परमासनात्}
{पर्यष्वजत धर्मात्मा निरन्तरमुरोगतम्} %7-38-19

\threelineshloka
{विसर्जयामास तदा कौसल्यानन्दवर्धनः}
{राघवेणाभ्यनुज्ञातः काशीशोऽप्यकुतोभयः}
{वाराणसीं ययौ तूर्णं राघवेण विसर्जितः} %7-38-20

\twolineshloka
{विसृज्य तं काशिपतिं त्रिशतं पृथिवीपतीन्}
{प्रहसन्राघवो वाक्यमुवाच मधुराक्षरम्} %7-38-21

\twolineshloka
{भवतां प्रीतिरव्यग्रा तेजसा परिरक्षिता}
{धर्मश्च नियतो नित्यं सत्यं च भवतां सदा} %7-38-22

\twolineshloka
{युष्माकं चानुभावेन तेजसा च महात्मनाम्}
{हतो दुरात्मा दुर्बुद्धी रावणो राक्षसाधमः} %7-38-23

\twolineshloka
{हेतुमात्रमहं तत्र भवतां तेजसा हतः}
{रावणः सगणो युद्धे सपुत्रामात्यबान्धवः} %7-38-24

\twolineshloka
{भवन्तश्च समानीता भरतेन महात्मना}
{श्रुत्वा जनकराजस्य काननात्तनयां हृताम्} %7-38-25

\twolineshloka
{उद्युक्तानां च सर्वेषां पार्थिवानां महात्मनाम्}
{कालो व्यतीतः सुमहान्गमनं रोचयाम्यतः} %7-38-26

\twolineshloka
{प्रत्यूचुस्तं च राजानो हर्षेण महता वृताः}
{दिष्ट्यां त्वं विजयी राम स्वराज्येऽपि प्रतिष्ठितः} %7-38-27

\twolineshloka
{दिष्ट्या प्रत्याहृता सीता दिष्ट्या शत्रुः पराजितः}
{एष नः परमः काम एषा नः पीतिरुत्तमा} %7-38-28

\onelineshloka
{यत्त्वां विजयिनं राम पश्यामो हतशात्रवम्} %7-38-29

\twolineshloka
{एतत्त्वय्युपपन्नं च यदस्मांस्त्वं प्रशंससे}
{प्रशंसार्ह न जानीमः प्रशंसां वक्तुमीदृशीम्} %7-38-30

\threelineshloka
{आपृच्छामो गमिष्यामो हृदिस्थो नः सदा भवान्}
{वर्तामहे महाबाहो प्रीत्यात्र महता वृताः}
{भवेच्च ते महाराज प्रीतिरस्मासु नित्यदा} %7-38-31

\twolineshloka
{बाढमित्येव राजानो हर्षेण परमान्विताः}
{उचुः प्राञ्जलयः सर्वे राघवं गमनोत्सुकाः} %7-38-32

\onelineshloka
{पूजिताश्चैव रामेण जग्मुर्देशान्स्वकान्स्वकान्} %7-38-33


॥इत्यार्षे श्रीमद्रामायणे वाल्मीकीये आदिकाव्ये उत्तरकाण्डे जनकादिप्रतिप्रयाणम् नाम अष्टात्रिंशः सर्गः ॥७-३८॥
