\sect{षष्ठितमः सर्गः — भार्गवच्यवनाध्यागमनम्}

\twolineshloka
{तयोः संवदतोरेवं रामलक्ष्मणयोस्तदा}
{वासन्तिकी निशा प्राप्ता न शीता न च घर्मदा} %7-60-1

\twolineshloka
{ततः प्रभाते विमले कृतपौर्वाह्णिकक्रियः}
{अभिचक्राम काकुत्स्थो दर्शनं पौरकार्यवित्} %7-60-2

\twolineshloka
{ततः सुमन्त्रस्त्वागम्य राघवं वाक्यमब्रवीत्}
{एते प्रतिहता राजन्द्वारि तिष्ठन्ति तापसाः} %7-60-3

\threelineshloka
{भार्गवच्यवनं चैव पुरस्कृत्य महर्षयः}
{दर्शनं ते महाराज्ञश्चोदयन्ति कृतत्वराः}
{प्रीयमाणा नरव्याघ्र यमुनातीरवासिनः} %7-60-4

\twolineshloka
{तस्य तद्वचनं श्रुत्वा रामः प्रोवाच धर्मवित्}
{प्रवेश्यन्तां महाभागा भार्गवप्रमुखा द्विजाः} %7-60-5

\twolineshloka
{राज्ञस्त्वाज्ञां पुरस्कृत्य द्वाःस्थो मूर्ध्नि कृताञ्जलिः}
{प्रवेशयामास तदा तापसान्सुदुरासदान्} %7-60-6

\twolineshloka
{शतं समधिकं तत्र दीप्यमानं स्वतेजसा}
{प्रविष्टं राजभवनं तापसानां महात्मनाम्} %7-60-7

\twolineshloka
{ते द्विजाः पूर्णकलशैः सर्वतीर्थाम्बुसत्कृतैः}
{गृहीत्वा फलमूलं च रामस्याभ्याहरन्बहु} %7-60-8

\twolineshloka
{प्रतिगृह्य तु तत्सर्वं रामः प्रीतिपुरस्कृतः}
{तीर्थोदकानि सर्वाणि फलानि विविधानि च} %7-60-9

\twolineshloka
{उवाच च महाबाहुः सर्वानेव महामुनीन्}
{इमान्यासनमुख्यानि यथार्हमुपविश्यताम्} %7-60-10

\twolineshloka
{रामस्य भाषितं श्रुत्वा सर्व एव महर्षयः}
{बृसीषु रुचिराख्यासु निषेदुः काञ्चनीषु ते} %7-60-11

\twolineshloka
{उपविष्टानृषींस्तत्र दृष्ट्वा परपुरञ्जयः}
{प्रयतः प्राञ्जलिर्भूत्वा राघवो वाक्यमब्रवीत्} %7-60-12

\twolineshloka
{किमागमनकार्यं वः किं करोमि समाहितः}
{आज्ञाप्योऽहं महर्षीणां सर्वकामकरः सुखम्} %7-60-13

\twolineshloka
{इदं राज्यं च सकलं जीवितं च हृदि स्थितम्}
{सर्वमेतद्द्विदार्थं मे सत्यमेतद्ब्रवीमि वः} %7-60-14

\twolineshloka
{तस्य तद्वचनं श्रुत्वा साधुकारो महानभूत्}
{ऋषीणामुग्रतपसां यमुनातीरवासिनाम्} %7-60-15

\twolineshloka
{ऊचुश्च ते महात्मानो हर्षेण महता वृताः}
{उपपन्नं नरश्रेष्ठ तवैव भुवि नान्यतः} %7-60-16

\twolineshloka
{बहवः पार्थिवा राजन्नतिक्रान्ता महाबलाः}
{कार्यस्य गौरवं मत्वा प्रतिज्ञां नाभ्यरोचयन्} %7-60-17

\twolineshloka
{त्वया पुनर्ब्राह्मणगौरवादियं कृता प्रतिज्ञा ह्यनवेक्ष्य कारणम्}
{ततश्च कर्ता ह्यसि नात्र संशयो महाभयात्त्रातुमृषींस्त्वमर्हसि} %7-60-18


॥इत्यार्षे श्रीमद्रामायणे वाल्मीकीये आदिकाव्ये उत्तरकाण्डे भार्गवच्यवनाध्यागमनम् नाम षष्ठितमः सर्गः ॥७-६०॥
