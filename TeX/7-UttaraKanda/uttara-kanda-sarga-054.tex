\sect{चतुःपञ्चाशः सर्गः — नृगश्वभ्रप्रवेशः}

\twolineshloka
{रामस्य भाषितं श्रुत्वा लक्ष्मणः परमार्थवित्}
{उवाच प्राञ्जलिर्वाक्यं राघवं दीप्ततेजसम्} %7-54-1

\twolineshloka
{अल्पापराधे काकुत्स्थ द्विजाभ्यां शाप ईदृशः}
{महान्नृगस्य राजर्षेर्यमदण्ड इवापरः} %7-54-2

\twolineshloka
{श्रुत्वा तु पापसंयुक्तमात्मानं पुरषर्षभ}
{किमुवाच नृगो राजा द्विजौ क्रोधसमन्वितौ} %7-54-3

\twolineshloka
{लक्ष्मणेनैवमुक्तस्तु राघवः पुनरब्रवीत्}
{शृणु सौम्य यथा पूर्वं स राजा शापविक्षतः} %7-54-4

\twolineshloka
{अथाध्वनि गतौ विप्रौ विज्ञाय स नृगस्तदा}
{आहूय मन्त्रिणः सर्वान्नैगमान्सपुरोधसः} %7-54-5

\twolineshloka
{तानुवाच नृगो राजा सर्वाश्च प्रकृतीस्तथा}
{दुःखेन सुसमाविष्टः श्रूयतां मे समाहितैः} %7-54-6

\twolineshloka
{नारदः पर्वतश्चैव मम दत्त्वा महद्भयम्}
{गतौ त्रिभुवनं भद्रौ वायुभूतावनिन्दितौ} %7-54-7

\threelineshloka
{कुमारोऽयं वसुर्नाम स देवोऽद्याभिषिच्यताम्}
{श्वभ्रं च यत्सुखस्पर्शं क्रियतां शिल्पिभिर्मम}
{यत्राहं सङ्क्षयिष्यामि शापं ब्राह्मणनिःसृतम्} %7-54-8

\twolineshloka
{वर्षघ्नमेकं श्वभ्रं तु हिमघ्नमपरं तथा}
{ग्रीष्मघ्नं तु सुखस्पर्शमेकं कुर्वन्तु शिल्पिनः} %7-54-9

\twolineshloka
{फलवन्तश्च ये वृक्षाः पुष्पवत्यश्च या लताः}
{विरोप्यन्तां बहुविधाश्छायावन्तश्च गुल्मिनः} %7-54-10

\twolineshloka
{क्रियतां रमणीयं च श्वभ्राणां सर्वतोदिशम्}
{सुखमत्र वसिष्यामि यावत्कालस्य पर्ययः} %7-54-11

\twolineshloka
{पुष्पाणि च सुगन्धीनि क्रियतां तेषु नित्यशः}
{परिवार्य यथा मे स्युरध्यर्धं योजनं तथा} %7-54-12

\twolineshloka
{एवं कृत्वा विधानं स सन्दिदेश वसं तदा}
{धर्मनित्यः प्रजाः पुत्र क्षत्रधर्मेण पालय} %7-54-13

\twolineshloka
{प्रत्यक्षं ते यथा शापो द्विजाभ्यां मयि पातितः}
{नरश्रेष्ठ सरोषाभ्यामपराधेऽपि तादृशे} %7-54-14

\twolineshloka
{मा कृथास्तनुसन्तापं मत्कृतेऽपि नरर्षभ}
{कृतान्तः कुशलः पुत्र येनास्मि व्यसनीकृतः} %7-54-15

\twolineshloka
{प्राप्तव्यान्येव प्राप्नोति गन्तव्यान्येव गच्छति}
{लब्धव्यान्येव लभते दुःखानि च सुखानि च} %7-54-16

\onelineshloka
{पूर्वे जात्यन्तरे वत्स मा विषादं कुरुष्व ह} %7-54-17

\twolineshloka
{एवमुक्त्वा नृपस्तत्र सुतं राजा महायशाः}
{श्वभ्रं जगाम सुकृतं वासाय पुरुषर्षभ} %7-54-18

\twolineshloka
{एवं प्रविश्यैव नृपस्तदानीं श्वभ्रं महारत्नविभूषितं तत्}
{सम्पादयामास तदा महात्मा शापं द्विजाभ्यां हि रुषा विमुक्तम्} %7-54-19


॥इत्यार्षे श्रीमद्रामायणे वाल्मीकीये आदिकाव्ये उत्तरकाण्डे नृगश्वभ्रप्रवेशः नाम चतुःपञ्चाशः सर्गः ॥७-५४॥
