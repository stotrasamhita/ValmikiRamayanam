\sect{त्रिपञ्चाशः सर्गः — नृगशापकथनम्}

\twolineshloka
{लक्ष्मणस्य तु तद्वाक्यं निशम्य परमाद्भुतम्}
{सुप्रीतश्चाभवद्रामो वाक्यमेतदुवाच ह} %7-53-1

\twolineshloka
{दुर्लभस्त्वीदृशो बन्धुरस्मिन्काले विशेषतः}
{यादृशस्त्वं महाबुद्धेर्मम सौम्य मनोऽनुगः} %7-53-2

\twolineshloka
{यच्च मे हृदये किञ्चिद्वर्तते शुभलक्षण}
{तन्निशामय च श्रुत्वा कुरुष्व वचनं मम} %7-53-3

\twolineshloka
{चत्वारो दिवसाः सौम्य कार्यं पौरजनस्य वै}
{अकुर्वाणस्य सौमित्रे तन्मे मर्माणि कृन्तति} %7-53-4

\twolineshloka
{आहूयन्तां प्रकृतयः पुरोधा मन्त्रिणस्तथा}
{कार्यार्थिनश्च पुरुषाः स्त्रियश्च पुरुषर्षभ} %7-53-5

\twolineshloka
{पौरकार्याणि यो राजा न करोति दिने दिने}
{संवृते नरके घोरे पतितो नात्र संशयः} %7-53-6

\twolineshloka
{श्रूयते हि पुरा राजा नृगो नाम महायशाः}
{बभूव पृथिवीपालो ब्रह्मण्यः सत्यवाक् शुचिः} %7-53-7

\twolineshloka
{स कदाचिद्गवां कोटीः सवत्साः स्वर्णभूषिताः}
{नृदेवो भूमिदेवेभ्यः पुष्करेषु ददौ नृपः} %7-53-8

\twolineshloka
{ततः सङ्गाद्गता धेनुः सवत्सा स्पर्शिताऽनघ}
{ब्राह्मणस्याहिताग्नेश्च दरिद्रस्योञ्छवर्तिनः} %7-53-9

\twolineshloka
{स नष्टां गां क्षुधार्तो वै अन्विषंस्तत्र तत्र च}
{नापश्यत्सर्वराज्येषु संवत्सरगणान्बहून्} %7-53-10

\twolineshloka
{ततः कनखलं गत्वा जीर्णवत्सां निरामयाम्}
{ददर्श गां स्वकां धेनुं ब्राह्मणस्य निवेशने} %7-53-11

\twolineshloka
{अथ तां नामधेयेन स्वकेनोवाच स द्विजः}
{आगच्छ शबलेत्येवं सा तु शुश्राव गौः स्वरम्} %7-53-12

\twolineshloka
{तस्य तं स्वरमाज्ञाय क्षुधार्तस्य द्विजस्य वै}
{अन्वगात्पृष्ठतः सा गौर्गच्छन्तं पावकोपमम्} %7-53-13

\twolineshloka
{योऽपि पालयते विप्रः सोऽपि गामन्वगाद्द्रुतम्}
{गत्वा तमृषिमाचष्ट मम गौरिति सत्वरम्} %7-53-14

\twolineshloka
{स्पर्शिता राजसिंहेन मम दत्ता नृगेण ह}
{तयोर्ब्राह्मणयोर्वादो महानासीद्विपश्चितोः} %7-53-15

\twolineshloka
{विवदन्तौ ततोऽन्योन्यं दातारमभिजग्मतुः}
{तौ राजभवनद्वारि न प्राप्तौ नृगशासनम्} %7-53-16

\twolineshloka
{अहोरात्राण्यनेकानि वसन्तौ क्रोधमीयतुः}
{ऊचतुश्च महात्मानौ तावुभौ द्विजसत्तमौ क्रुद्धौ परमसम्प्राप्तौ वाक्यं घोराभिसंहितम्} %7-53-17

\twolineshloka
{अर्थिनां कार्यसिद्ध्यर्थं यस्मात्त्वं नैषि दर्शनम्}
{अदृश्यः सर्वभूतानां कृकलासो भविष्यसि} %7-53-18

\twolineshloka
{बहुवर्षसहस्राणि बहुवर्षशतानि च}
{श्वभ्रेऽस्मिन् कृकलासो वै दीर्घकालं वसिष्यसि} %7-53-19

\twolineshloka
{उत्पत्स्यते हि लोकेऽस्मिन्यदूनां कीर्तिवर्धनः}
{वासुदेव इति ख्यातो विष्णुः पुरुषविग्रहः} %7-53-20

\twolineshloka
{स ते मोक्षयिता राजंस्तस्माच्छापाद्भविष्यसि}
{कृता च तेन कालेन निष्कृतिस्ते भविष्यति} %7-53-21

\twolineshloka
{भारावतरणार्थं हि नरनारायणावुभौ}
{उत्पत्स्येते महावीर्यौ कलौ युग उपस्थिते} %7-53-22

\twolineshloka
{एवं तौ शापमुत्सृज्य ब्राह्मणौ विगतज्वरौ}
{तां गां हि दुर्बलां वृद्धां ददतुर्ब्राह्मणाय वै} %7-53-23

\onelineshloka
{एवं स राजा तं शापमुपभुङ्क्ते सुदारुणम्} %7-53-24

\twolineshloka
{कार्यार्थिनां विमर्दो हि राज्ञां दोषाय कल्पते}
{तच्छीघ्रं दर्शनं मह्यमभिवर्तन्तु कार्यिणः} %7-53-25

\twolineshloka
{सुकृतस्य हि कार्यस्य फलं नापैति पार्थिवः}
{तस्माद्गच्छ प्रतीक्षस्व सौमित्रे कार्यवाञ्जनः} %7-53-26


॥इत्यार्षे श्रीमद्रामायणे वाल्मीकीये आदिकाव्ये उत्तरकाण्डे नृगशापकथनम् नाम त्रिपञ्चाशः सर्गः ॥७-५३॥
