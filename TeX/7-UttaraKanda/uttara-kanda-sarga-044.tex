\sect{चतुश्चत्वारिंशः सर्गः — लक्ष्मणाद्यानयनम्}

\twolineshloka
{विसृज्य तु सुहृद्वर्गं बुद्ध्या निश्चित्य राघवः}
{समीपे द्वाःस्थमासीनमिदं वचनमब्रवीत्} %7-44-1

\twolineshloka
{शीघ्रमानय सौमित्रिं लक्ष्मणं शुभलक्षणम्}
{भरतं च महाभागं शत्रुघ्नमपराजितम्} %7-44-2

\twolineshloka
{रामस्य वचनं श्रुत्वा द्वाःस्थो मूर्ध्नि कृताञ्जलिः}
{लक्ष्मणस्य गृहं गत्वा प्रविवेशानिवारितः} %7-44-3

\twolineshloka
{उवाच सुमहात्मानं वर्धयित्वा कृताञ्जलिः}
{द्रष्टुमिच्छति राजा त्वां गम्यतां तत्र मा चिरम्} %7-44-4

\twolineshloka
{बाढमित्येव सौमित्रिः श्रुत्वा राघवशासनम्}
{प्राद्रवद्रथमारुह्य राघवस्य निवेशनम्} %7-44-5

\twolineshloka
{प्रयान्तं लक्ष्मणं दृष्ट्वा द्वाःस्थो भरतमन्तिकात्}
{उवाच भरतं तत्र वर्धयित्वा कृताञ्जलिः} %7-44-6

\twolineshloka
{विनयावनतो भूत्वा राजा त्वां द्रष्टुमिच्छति}
{भरतस्तु वचःश्रुत्वा द्वाःस्थाद्रामसमीरितम्} %7-44-7

\twolineshloka
{उत्पपातासनात्तूर्णं पद्भ्यामेव ययौ बली}
{दृष्ट्वा प्रयान्तं भरतं त्वरमाणः कृताञ्जलिः} %7-44-8

\twolineshloka
{शत्रुघ्नभवनं गत्वा ततो वाक्यमुवाच ह}
{एह्यागच्छ रघुश्रेष्ठ राजा त्वां द्रष्टुमिच्छति} %7-44-9

\twolineshloka
{गतो हि लक्ष्मणः पूर्वं भरतश्च महायशाः}
{श्रुत्वा तु वचनं तस्य शत्रुघ्नः परमासनात्} %7-44-10

\threelineshloka
{शिरसा धरणीं प्राप्य प्रययौ यत्र राघवः}
{द्वाःस्थस्त्वागम्य रामाय सर्वानेव कृताञ्जलिः}
{निवेदयामास तदा भ्रातऽन्स्वान्समुपस्थितान्} %7-44-11

\twolineshloka
{कुमारानागताञ्छ्रुत्वा चिन्ताव्याकुलितेन्द्रियः}
{अवाङ्मुखो दीनमना द्वाःस्थं वचनमब्रवीत्} %7-44-12

\twolineshloka
{प्रवेशय कुमारांस्त्वं मत्समीपं त्वरान्वितः}
{एतेषु जीवितं मह्यमेते प्राणाः प्रिया मम} %7-44-13

\twolineshloka
{आज्ञप्तास्तु नरेन्द्रेण कुमाराः शक्रतेजसः}
{प्रह्वाः प्राञ्जलयो भूत्वा विविशुस्ते समाहिताः} %7-44-14

\twolineshloka
{ते तु दृष्ट्वा मुखं तस्य सग्रहं शशिनं यथा}
{सन्ध्यागतमिवादित्यं प्रभया परिवर्जितम्} %7-44-15

\twolineshloka
{बाष्पपूर्णे च नयने दृष्ट्वा रामस्य धीमतः}
{हतशोभं यथा पद्मं मुखं वीक्ष्य च तस्य ते} %7-44-16

\twolineshloka
{ततोऽभिवाद्य त्वरिताः पादौ रामस्य मूर्धभिः}
{तस्थुः समाहिताः सर्वे रामस्त्वश्रूण्यवर्तयत्} %7-44-17

\twolineshloka
{तान्परिष्वज्य बाहुभ्यामुत्थाप्य च महाबलः}
{आसनेष्वासतेत्युक्त्वा ततो वाक्यं जगाद ह} %7-44-18

\twolineshloka
{भवन्तो मम सर्वस्वं भवन्तो जीवितं मम}
{भवद्भिश्च कृतं राज्यं पालयामि नरेश्वराः} %7-44-19

\twolineshloka
{भवन्तः कृतशास्त्रार्था बुद्ध्या च परिनिष्ठिताः}
{सम्भूय च मदर्थोऽयमन्वेष्टव्यो नरेश्वराः} %7-44-20

\twolineshloka
{तथा वदति काकुत्स्थे अवधानपरायणाः}
{उद्विग्नमनसः सर्वे किं नु राजाभिधास्यति} %7-44-21


॥इत्यार्षे श्रीमद्रामायणे वाल्मीकीये आदिकाव्ये उत्तरकाण्डे लक्ष्मणाद्यानयनम् नाम चतुश्चत्वारिंशः सर्गः ॥७-४४॥
