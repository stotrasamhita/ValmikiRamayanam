\sect{द्विचत्वारिंशः सर्गः — रामसीताविहारः}

\twolineshloka
{स विसृज्य ततो रामः पुष्पकं हेमभूषितम्}
{प्रविवेश महाबाहुरशोकवनिकां तदा} %7-42-1

\twolineshloka
{चन्दनागुरुचूतैश्च तुङ्गकालेयकैरपि}
{देवदारुवनैश्चापि समन्तादुपशोभिताम्} %7-42-2

\twolineshloka
{चम्पकाशोकपुन्नागमधूकपनसासनैः}
{शोभितां पारिजातैश्च विधूमज्वलनप्रभैः} %7-42-3

\twolineshloka
{लोध्रनीपार्जुनैर्नागैः सप्तपर्णातिमुक्तकैः}
{मन्दारकदलीगुल्मलताजालसमावृताम्} %7-42-4

\twolineshloka
{प्रियङ्गुभिः कदम्बैश्च तथा च वकुलैरपि}
{जम्बूभिर्दाडिमैश्चैव कोविदारैश्च शोभिताम्} %7-42-5

\twolineshloka
{सर्वदा कुसुमै रम्यैः फलवद्भिर्मनोरमैः}
{दिव्यगन्धरसोपेतैस्तरुणाङ्कुरपल्लवैः} %7-42-6

\twolineshloka
{तथैव तरुभिर्दिव्यैः शिल्पिभिः परिकल्पितैः}
{चारुपल्लवपुष्पाढ्यैर्मत्तभ्रमरसङ्कुलैः} %7-42-7

\twolineshloka
{कोकिलैर्भृङ्गराजैश्च नानावर्णैश्च पक्षिभिः}
{शोभितां शतशश्चित्रां चूतवृक्षावतंसकैः} %7-42-8

\twolineshloka
{शातकुम्भनिभाः केचित्केचिदग्निशिखोपमाः}
{नीलाञ्जननिभाश्चान्ये भान्ति तत्रत्यपादपाः} %7-42-9

\twolineshloka
{सुरभीणि च पुष्पाणि माल्यानि विविधानि च}
{दीर्घिका विविधाकाराः पूर्णाः परमवारिणा} %7-42-10

\twolineshloka
{माणिक्यकृतसोपानाः स्फाटिकान्तरकुट्टिमाः}
{फुल्लपद्मोत्पलवनाश्चक्रवाकोपशोभिताः} %7-42-11

\twolineshloka
{दात्यूहशुकसङ्घुष्टा हंससारसनादिताः}
{तरुभिः पुष्पवद्भिश्च तीरजैरुपशोभिताः} %7-42-12

\twolineshloka
{प्राकारैर्विविधाकारैः शोभिताश्च शिलातलैः}
{तत्रैव च वनोद्देशे वैडूर्यमणिसन्निभैः} %7-42-13

\twolineshloka
{शाद्वलैः परमोपेतां पुष्पितद्रुमकाननाम्}
{तत्र सङ्घर्षजातानां वृक्षाणां पुष्पशालिनाम्} %7-42-14

\threelineshloka
{प्रस्तराः पुष्पशबला नभस्तारागणैरिव}
{नन्दनं हि यथेन्द्रस्य ब्राह्मं चैत्ररथं यथा}
{तथाभूतं हि रामस्य काननं सन्निवेशनम्} %7-42-15

\twolineshloka
{बह्वासनगृहोपेतां लतागृहसमावृताम्}
{अशोकवनिकां स्फीतां प्रविश्य रघुनन्दनः} %7-42-16

\twolineshloka
{आसने च शुभाकारे पुष्पप्रकरभूषिते}
{कुशास्तरणसंस्तीर्णे रामः सन्निषसाद ह} %7-42-17

\twolineshloka
{सीतामादायं हस्तेन मधुमैरेयकं शुचि}
{पाययामास काकुत्स्थः शचीमिव पुरन्दरः} %7-42-18

\twolineshloka
{मांसानि च समृष्टानि फलानि विविधानि च}
{रामस्याभ्यवहारार्थं किङ्करास्तूर्णमाहरन्} %7-42-19

\twolineshloka
{उपानृत्यंश्च राजानं नृत्यगीतविशारदाः}
{बालाश्च रूपवत्यश्च स्त्रियः पानवशानुगाः} %7-42-20

\twolineshloka
{मनोभिरामा रामास्ता रामो रमयतां वरः}
{रमयामास धर्मात्मा नित्यं परमभूषितः} %7-42-21

\twolineshloka
{स तया सीतया सार्धमासीनो विरराज ह}
{अरुन्धत्या सहासीनो वसिष्ठ इव तेजसा} %7-42-22

\twolineshloka
{एवं रामो मुदा युक्तः सीतां सुरसुतोपमाम्}
{रमयामास वैदेहीमहन्यहनि देववत्} %7-42-23

\twolineshloka
{तथा तयोर्विहरतोः सीताराघवयोश्चिरम्}
{अत्यक्रामच्छुभः कालः शैशिरो भोगदः सदा} %7-42-24

\twolineshloka
{दशवर्षसहस्राणि गतानि सुमहात्मनोः}
{प्राप्तयोर्विविधान्भोगानतीतः शिशिरागमः} %7-42-25

\twolineshloka
{पूर्वाह्णे धर्मकार्याणि कृत्वा धर्मेण धर्मवित्}
{शेषं दिवसभागार्धमन्तःपुरगतोऽभवत्} %7-42-26

\twolineshloka
{सीतापि देवकार्याणि कृत्वा पौर्वाह्णिकानि वै}
{श्वश्रूणामकरोत्पूजां सर्वासामविशेषतः} %7-42-27

\twolineshloka
{अभ्यगच्छत्ततो रामं विचित्राभरणाम्बरा}
{त्रिविष्टपे सहस्राक्षमुपविष्टं यथा शची} %7-42-28

\twolineshloka
{दृष्ट्वा तु राघवः पत्नीं कल्याणेन समन्विताम्}
{प्रहर्षमतुलं लेभे साधु साध्विति चाब्रवीत्} %7-42-29

\twolineshloka
{अब्रवीच्च वरारोहां सीतां सुरसुतोपमाम्}
{अपत्यलाभो वैदेहि त्वयि मे समुपस्थितः} %7-42-30

\onelineshloka
{किमिच्छसि वरारोहे कामः किं क्रियतां तव} %7-42-31

\twolineshloka
{स्मितं कृत्वा तु वैदेही रामं वाक्यमथाब्रवीत्}
{तपोवनानि पुण्यानि द्रष्टुमिच्छामि राघव} %7-42-32

\twolineshloka
{गङ्गातीरोपविष्टानामृषीणामुग्रतेजसाम्}
{फलमूलाशिनां देव पादमूलेषु वर्तितुम्} %7-42-33

\twolineshloka
{एष मे परमः कामो यन्मूलफलभोजिनाम्}
{अप्येकरात्रं काकुत्स्थ निवसेयं तपोवने} %7-42-34

\twolineshloka
{तथेति च प्रतिज्ञातं रामेणाक्लिष्टकर्मणा}
{विस्रब्धा भव वैदेहि श्वो गमिष्यस्यसंशयम्} %7-42-35

\twolineshloka
{एवमुक्त्वा तु काकुत्स्थो मैथिलीं जनकात्मजाम्}
{मध्यकक्षान्तरं रामो निर्जगाम सुहृद्वृतः} %7-42-36


॥इत्यार्षे श्रीमद्रामायणे वाल्मीकीये आदिकाव्ये उत्तरकाण्डे रामसीताविहारः नाम द्विचत्वारिंशः सर्गः ॥७-४२॥
