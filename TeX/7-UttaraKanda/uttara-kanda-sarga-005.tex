\sect{पञ्चमः सर्गः — माल्यवदाद्यपत्योत्पत्तिः}

\twolineshloka
{सुकेशं धार्मिकं दृष्ट्वा वरलब्धं च राक्षसम्}
{ग्रामणीर्नाम गन्धर्वो विश्वावसुसमप्रभः} %7-5-1

\threelineshloka
{तस्य देववती नाम द्वितीया श्रीरिवात्मजा}
{त्रिषु लोकेषु विख्याता रूपयौवनशालिनी}
{तां सुकेशाय धर्मेण ददौ रक्षःश्रियं यथा} %7-5-2

\twolineshloka
{वरदानकृतैश्वर्यं सा तं प्राप्य पतिं प्रियम्}
{आसीद्देववती तुष्टा धनं प्राप्येव निर्धनः} %7-5-3

\twolineshloka
{स तया सह संयुक्तो रराज रजनीचरः}
{अञ्जनादभिनिष्क्रान्तः करेण्वेव महागजः} %7-5-4

\onelineshloka
{देववत्यां सुकेशस्तु जनयामास राघव} %7-5-5

\threelineshloka
{त्रीन् पुत्रान् जनयामास त्रेताग्निसमविग्रहान्}
{माल्यवन्तं सुमालिं च मालिं च बलिनां वरम्}
{त्रींस्त्रिनेत्रसमान्पुत्रान्राक्षसान्राक्षसाधिपः} %7-5-6

\twolineshloka
{त्रयो लोका इवाव्यग्राः स्थितास्त्रय इवाग्नयः}
{त्रयो मन्त्रा इवात्युग्रास्त्रयो घोरा इवामयाः} %7-5-7

\twolineshloka
{त्रयः सुकेशस्य सुतास्त्रेताग्निसमतेजसः}
{विवृद्धिमगमंस्तत्र व्याधयोपेक्षिता इव} %7-5-8

\twolineshloka
{वरप्राप्तिं पितुस्ते तु ज्ञात्वेश्वरतपोबलात्}
{तपस्तप्तुं गता मेरुं भ्रातरः कृतनिश्चयाः} %7-5-9

\twolineshloka
{प्रगृह्य नियमान्घोरान्राक्षसा नृपसत्तम}
{विचेरुस्ते तपो घोरं सर्वभूतभयावहम्} %7-5-10

\twolineshloka
{सत्यार्जवशमोपेतैस्तपोभिरतिदुष्करैः}
{सन्तापयन्तस्त्रील्लोँकान्सदेवासुरमानुषान्} %7-5-11

\twolineshloka
{ततो विभुश्चतुर्वक्त्रो विमानवरमास्थितः}
{सुकेशपुत्रानामन्त्र्य वरदोऽस्मीत्यभाषत} %7-5-12

\twolineshloka
{ब्रह्माणं वरदं ज्ञात्वा सेन्द्रैर्देवगणैर्वृतम्}
{ऊचुः प्राञ्जलयः सर्वे वेपमाना इव द्रुमाः} %7-5-13

\twolineshloka
{तपसाराधितो देव यदि नो दिशसे वरम्}
{अजेयाः शत्रुहन्तारस्तथैव चिरजीविनः} %7-5-14

\onelineshloka
{प्रभविष्ण्वो भवामेति परस्परमनुव्रताः} %7-5-15

\twolineshloka
{एवं भविष्यतीत्युक्त्वा सुकेशतनयान्विभुः}
{स ययौ ब्रह्मलोकाय ब्रह्मा ब्राह्मणवत्सलः} %7-5-16

\twolineshloka
{वरं लब्ध्वा तु ते सर्वे राम रात्रिञ्चरास्तदा}
{सुरासुरान्प्रबाधन्ते वरदानसुनिर्भयाः} %7-5-17

\twolineshloka
{तैर्वध्यमानास्त्रिदशाः सर्षिसङ्घाः सचारणाः}
{त्रातारं नाधिगच्छन्ति निरयस्था यथा नराः} %7-5-18

\twolineshloka
{अथ ते विश्वकर्माणं शिल्पिनां वरमव्ययम्}
{ऊचुः समेत्य संहृष्टा राक्षसा रघुसत्तम} %7-5-19

\twolineshloka
{ओजस्तेजोबलवतां महतामात्मतेजसा}
{गृहकर्ता भवानेव देवानां हृदयेप्सितम्} %7-5-20

\threelineshloka
{अस्माकमपि तावत्त्वं गृहं कुरु महामते}
{हिमवन्तमपाश्रित्य मेरुमन्दरमेव वा}
{महेश्वरगृहप्रख्यं गृहं नः क्रियतां महत्} %7-5-21

\twolineshloka
{विश्वकर्मा ततस्तेषां राक्षसानां महाभुजः}
{निवासं कथयामास शक्रस्येवामरावतीम्} %7-5-22

\twolineshloka
{दक्षिणस्योदधेस्तीरे त्रिकूटो नाम पर्वतः}
{सुवेल इति चाप्यन्यो द्वितीयस्तत्र सत्तमाः} %7-5-23

\twolineshloka
{शिखरे तस्य शैलस्य मध्यमेऽम्बुदसन्निभे}
{शकुनैरपि दुष्प्रापे टङ्कच्छिन्नचतुर्दिशि} %7-5-24

\twolineshloka
{त्रिंशद्योजनक्स्तीर्णा शतयोजनमायता}
{स्वर्णप्राकारसंवीता हेमतोरणसंवृता} %7-5-25

\twolineshloka
{मया लङ्केति नगरी शक्राज्ञप्तेन निर्मिता}
{तस्यां वसत दुर्धर्षा यूयं राक्षसपुङ्गवाः} %7-5-26

\twolineshloka
{अमरावतीं समासाद्य सेन्द्रा इव दिवौकसः}
{लङ्कादुर्गं समासाद्य राक्षसैर्बहुभिर्वृताः} %7-5-27

\onelineshloka
{भविष्यथ दुराधर्षाः शत्रूणां शत्रुसूदनाः} %7-5-28

\twolineshloka
{विश्वकर्मवचः श्रुत्वा ततस्ते राक्षसोत्तमाः}
{सहस्रानुचरा भूत्वा गत्वा तामवसन्पुरीम्} %7-5-29

\twolineshloka
{दृढप्राकारपरिखां हैमैर्गृहशतैर्वृताम्}
{लङ्कामवाप्य ते हृष्टा न्यवसन्रजनीचराः} %7-5-30

\twolineshloka
{एतस्मिन्नेव काले तु यथाकामं च राघव}
{नर्मदा नाम गन्धर्वी बभूव रघुनन्दन} %7-5-31

\twolineshloka
{तस्याः कान्यात्रयं ह्यासीत् धीश्रीकिर्तिसमद्युति}
{ज्येष्ठक्रमेण सा तेषां राक्षसानामराक्षसी} %7-5-32

\twolineshloka
{कन्यास्ताः प्रददौ हृष्टा पूर्णचन्द्रनिभाननाः}
{त्रयाणां राक्षसेन्द्राणां तिस्रो गन्धर्वकन्यकाः} %7-5-33

\twolineshloka
{दत्ता मात्रा महाभागा नक्षत्रे भगदैवते}
{कुतदारास्तु ते राम सुकेशतनयास्तदा} %7-5-34

\twolineshloka
{चिक्रीडुः सह भार्याभिरप्सरोभिरिवामराः}
{ततो माल्यवतो भार्या सुन्दरी नाम सुन्दरी} %7-5-35

\twolineshloka
{स तस्यां जनयामास यदपत्यं निबोध तत्}
{वज्रमुष्टिर्विरूपाक्षोदुर्मुखश्चैव राक्षसः} %7-5-36

\twolineshloka
{सुप्तघ्नो यज्ञकोपश्च मत्तोन्मत्तौ तथैव च}
{अनला चाभवत्कन्या सुन्दर्यां राम सुन्दरी} %7-5-37

\twolineshloka
{सुमालिनोऽपि भार्यासीत्पूर्णचद्रनिभानना}
{नाम्ना केतुमती राम प्राणेभ्योऽपि गरीयसी} %7-5-38

\twolineshloka
{सुमाली जनयामास यदपत्यं निशाचरः}
{केतुमत्यां महाराज तन्निबोधानुपूर्वशः} %7-5-39

\twolineshloka
{प्रहस्तोऽकम्पनश्चैव विकटः कालकार्मुखः}
{धूम्राक्षश्चैव दण्डश्च सुपार्श्वश्च महाबलः} %7-5-40

\twolineshloka
{संह्रादिः प्रघसश्चैव भासकर्णश्च राक्षसः}
{राका पुष्पोत्कटा चैव कैकसी च शुचिस्मिता कुम्भीनसी च इत्येते सुमालेः प्रसवाः स्मृताः} %7-5-41

\twolineshloka
{मालेस्तु वसुधा नाम गन्धर्वी रूपशालिनी}
{भार्यासीत्पद्मपत्राक्षी स्वक्षी यक्षीवरोपमा} %7-5-42

\twolineshloka
{सुमालेरनुजस्तस्यां जनयामास यत् प्रभो}
{अपत्यं कथ्यमानं तु मया त्वं शृणु राघव} %7-5-43

\twolineshloka
{अनिलश्चानलश्चैव हरः सम्पातिरेव च}
{एते विभीषणामात्या मालेयास्तु निशाचरः} %7-5-44

\twolineshloka
{ततस्तु ते राक्षसपुङ्गवास्त्रयो निशाचरैः पुत्रशतैश्च संवृताः}
{सुरान्सहेन्द्रानृषिनागयक्षान्बबाधिरे तान्बहुवीर्यदर्पिताः} %7-5-45

\twolineshloka
{जगद्भ्रमन्तोऽनिलवद्दुरासदा रणेषु मृत्युप्रतिमानतेजसः}
{वरप्रदानादतिगर्विता भृशं क्रतुक्रियाणां प्रशमङ्कराः सदा} %7-5-46


॥इत्यार्षे श्रीमद्रामायणे वाल्मीकीये आदिकाव्ये उत्तरकाण्डे माल्यवदाद्यपत्योत्पत्तिः नाम पञ्चमः सर्गः ॥७-५॥
