\sect{चतुर्नवतितमः सर्गः — रामायणगानम्}

\twolineshloka
{तौ रजन्यां प्रभातायां स्नातौ हुतहुताशनौ}
{यथोक्तमृषिणा पूर्वं सर्वं तत्रोपगायताम्} %7-94-1

\twolineshloka
{तां स शुश्राव काकुत्स्थः पूर्वाचार्यविजिर्मिताम्}
{अपूर्वां पाठ्यजातिं च गेयेन समलङ्कृताम्} %7-94-2

\twolineshloka
{प्रमाणैर्बहुभिर्बद्धां तन्त्रीलयसमन्विताम्}
{बालाभ्यां राघवः श्रुत्वा कौतूहलपरोऽभवत्} %7-94-3

\twolineshloka
{अथ कर्मान्तरे राजा समाहूय महामुनीन्}
{पार्थिवांश्च नरव्याघ्रः पण्डितान्नैगमांस्तथा} %7-94-4

\twolineshloka
{पौराणिकाञ्छब्दविदो ये च वृद्धा द्विजातयः}
{स्वराणां लक्षणज्ञांश्च उत्सुकान्द्विजसत्तमान्} %7-94-5

\twolineshloka
{लक्षणज्ञांश्च गान्धर्वान्नैगमांश्च विशेषतः}
{पादाक्षरसमासज्ञांश्छन्दःसु परिनिष्ठितान्} %7-94-6

\twolineshloka
{कलामात्राविशेषज्ञाञ्ज्योतिषे च परं गतान्}
{क्रियाकल्पविदश्चैव तथा कार्यविदो जनान्} %7-94-7

\twolineshloka
{भाषाज्ञानिङ्गितज्ञांश्च नैगमांश्चाप्यशेषतः}
{हेतूपचारकुशलान् वचने चापि हैतुकान्} %7-94-8

\twolineshloka
{छन्दोविदः पुराणज्ञान्वैदिकान्द्विजसत्तमान्}
{चित्रज्ञान्वृत्तसूत्रज्ञान्गीतनृत्यविशारदान्} %7-94-9

\twolineshloka
{शास्त्रज्ञान्नीतिनिपुणान्वेदान्तार्थप्रबोधकान्}
{एतान्सर्वान्समानीय गातारौ समवेशयत्} %7-94-10

\twolineshloka
{दृष्ट्वा मुनिगणाः सर्वे पार्थिवाश्च महौजसः}
{पिबन्त इव चक्षुर्भ्यां राजानं गायकौ च तौ} %7-94-11

\twolineshloka
{ऊचुः परस्परं चेदं सर्व एव समन्ततः}
{उभौ रामस्य सदृशौ बिम्बाद्बिम्बमिवोत्थितौ} %7-94-12

\twolineshloka
{जटिलौ यदि न स्यातां न वल्कलधरौ यदि}
{विशेषं नाधिगच्छामो गायतो राघवस्य च} %7-94-13

\twolineshloka
{तेषां संवदतामेवं श्रोतऽणां हर्षवर्धनम्}
{गेयं प्रचक्रतुस्तत्र तावुभौ मुनिदारकौ} %7-94-14

\twolineshloka
{ततः प्रवृत्तं मधुरं गान्धर्वमतिमानुषम्}
{न च तृप्तिं ययुः सर्वे श्रोतारो गानसम्पदा} %7-94-15

\twolineshloka
{प्रवृत्तमादितः पूर्वसर्गं नारददर्शितम्}
{ततः प्रभृति सर्गांश्च यावद्विंशत्यगायताम्} %7-94-16

\twolineshloka
{ततोऽपराह्णसमये राघवः समभाषत}
{श्रुत्वा विंशतिसर्गांस्तान्भ्रातरं भ्रातृवत्सलः} %7-94-17

\threelineshloka
{अष्टादशसहस्राणि सुवर्णस्य महात्मनोः}
{प्रयच्छ शीघ्रं काकुत्स्थ यदन्यदभिकाङ्क्षितम्}
{ददौ शीघ्रं स काकुत्स्थो बालयोर्वै पृथक्पृथक्} %7-94-18

\twolineshloka
{दीयमानं सुवर्णं तु नागृह्णीतां कुशीलवौ}
{ऊचतुश्च महात्मानौ किमनेनेति विस्मितौ} %7-94-19

\twolineshloka
{वन्येन फलमूलेन निरतौ वनवासिनौ}
{सुवर्णेन हिरण्येन किं करिष्यावहे वने} %7-94-20

\twolineshloka
{तथा तयोः प्रब्रुवतोः कौतूहलसमन्विताः}
{श्रोतारश्चैव रामश्च सर्व एव सुविस्मिताः} %7-94-21

\twolineshloka
{तस्य चैवागमं रामः काव्यस्य श्रोतुमुत्सुकः}
{पप्रच्छ तौ महातेजास्तावुभौ मुनिदारकौ} %7-94-22

\twolineshloka
{किम्प्रमाणमिदं काव्यं का प्रतिष्ठा महात्मनः}
{कर्ता काव्यस्य महतः क्व चासौ मुनिपुङ्गवः} %7-94-23

\onelineshloka
{पृच्छन्तं राघवं वाक्यमूचतुर्मुनिदारकौ} %7-94-24

\twolineshloka
{वाल्मीकिर्भगवान्कर्ता सम्प्राप्तो यज्ञसंविधम्}
{येनेदं चरितं तुभ्यमशेषं सम्प्रदर्शितम्} %7-94-25

\twolineshloka
{सन्निबद्धं हि श्लोकानां चतुर्विशत्सहस्रकम्}
{उपाख्यानशतं चैव भार्गवेण तपस्विना} %7-94-26

\threelineshloka
{आदिप्रभृति वै राजन्पञ्चसर्गशतानि च}
{काण्डानि षट् कृतानीह सोत्तराणि महात्मना}
{कृतानि गुरुणास्माकमृषिणा चरितं तव} %7-94-27

\onelineshloka
{प्रतिष्ठाऽऽजीवितं यावत्तावत्सर्वस्य वर्तते} %7-94-28

\threelineshloka
{सोत्तराणि महात्मना}
{यदि बुद्धिः कृता राजञ्छ्रवणाय महारथ}
{कर्मान्तरे क्षणीभूतस्तच्छृणुष्व सहानुजः} %7-94-29

\twolineshloka
{बाढमित्यब्रवीद्रामस्तौ चानुज्ञाप्य राघवम्}
{प्रहृष्टौ जग्मतुः स्थानं यत्रास्ते मुनिपुङ्गवः} %7-94-30

\twolineshloka
{रामोऽपि मुनिभिः सार्धं पार्थिवैश्च महात्मभिः}
{श्रुत्वा तद्गीतिमाधुर्यं कर्मशालामुपागमत्} %7-94-31

\twolineshloka
{शुश्राव तत्ताललयोपपन्नं सर्गान्वितं स स्वरशब्दयुक्तम्}
{तन्त्रीलयव्यञ्जनयोगयुक्तं कुशीलवाभ्यां परिगीयमानम्} %7-94-32


॥इत्यार्षे श्रीमद्रामायणे वाल्मीकीये आदिकाव्ये उत्तरकाण्डे रामायणगानम् नाम चतुर्नवतितमः सर्गः ॥७-९४॥
