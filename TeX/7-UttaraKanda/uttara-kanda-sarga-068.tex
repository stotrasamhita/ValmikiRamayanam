\sect{अष्टषष्ठितमः सर्गः — लवणशत्रुघ्नविवादः}

\twolineshloka
{कथां कथयतां तेषां जयं चाकाङ्क्षतां शुभम्}
{व्यतीता रजनी शीघ्रं शत्रुघ्नस्य महात्मनः} %7-68-1

\twolineshloka
{ततः प्रभाते विमले तस्मिन्काले स राक्षसः}
{निर्गतस्तु पुराद्धीरो भक्ष्याहारप्रचोदितः} %7-68-2

\twolineshloka
{एतस्मिन्नन्तरे वीरः शत्रुघ्नो यमुनां नदीम्}
{तीर्त्वा मधुपुरद्वारि धनुष्पाणिरतिष्ठत} %7-68-3

\twolineshloka
{ततोऽर्धदिवसे प्राप्ते क्रूरकर्मा स राक्षसः}
{आगच्छद्बहुसाहस्रं प्राणिनां भारमुद्वहन्} %7-68-4

\twolineshloka
{ततो ददर्श शत्रुघ्नं स्थितं द्वारि धृतायुधम्}
{तमुवाच ततो रक्षः किमनेन करिष्यसि} %7-68-5

\twolineshloka
{ईदृशानां सहस्राणि सायुधानां नराधम}
{भक्षितानि मया रोषात्कालमाकाङ्क्षसे नु किम्} %7-68-6

\twolineshloka
{आहारश्चास्य सम्पूर्णो ममायं पुरुषाधम}
{स्वयं प्रविष्टोऽद्य मुखं कथमासाद्य दुर्मते} %7-68-7

\twolineshloka
{तस्यैवं भाषमाणस्य हसतश्च मुहुर्मुहुः}
{शत्रुघ्नो वीर्यसम्पन्नो रोषादश्रूण्यवासृजत्} %7-68-8

\twolineshloka
{तस्य रोषाभिभूतस्य शत्रुघ्नस्य महात्मनः}
{तेजोमया मरीच्यस्तु सर्वगात्रैर्विनिष्पतन्} %7-68-9

\twolineshloka
{उवाच च सुसङ्क्रुद्धः शत्रुघ्नस्तं निशाचरम्}
{योद्धुमिच्छामि दुर्बुद्धे द्वन्द्वयुद्धं त्वया सह} %7-68-10

\twolineshloka
{पुत्रो दशरथस्याहं भ्राता रामस्य धीमतः}
{शत्रुघ्नो नित्यशत्रुघ्नो वधाकाङ्क्षी तवागतः} %7-68-11

\twolineshloka
{तस्य मे युद्धकामस्य द्वन्द्वयुद्धं प्रदीयताम्}
{शत्रुस्त्वं सर्वभूतानां न मे जीवन्गमिष्यसि} %7-68-12

\twolineshloka
{तस्मिंस्तथा ब्रुवाणे तु राक्षसः प्रहसन्निव}
{प्रत्युवाच नरश्रेष्ठं दिष्ट्या प्राप्तोऽसि दुर्मते} %7-68-13

\twolineshloka
{मम मातृष्वसुर्भ्राता रावणो राक्षसाधिपः}
{हतो रामेण दुर्बुद्धे स्त्रीहेतोः पुरुषाधम} %7-68-14

\twolineshloka
{तच्च सर्वं मया क्षान्तं रावणस्य कुलक्षयम्}
{अवज्ञां पुरतः कृत्वा मया यूयं विशेषतः} %7-68-15

\twolineshloka
{निहताश्च हि मे सर्वे परिभूतास्तृणं यथा}
{भूताश्चैव भविष्याश्च यूयं च पुरुषाधमाः} %7-68-16

\threelineshloka
{तस्य ते युद्धकामस्य युद्धं दास्यामि दुर्मते}
{तिष्ठं त्वं च मुहूर्तं तु यावदायुधमानये}
{ईप्सितं यादृशं तुभ्यं सज्जये यावदायुधम्} %7-68-17

\twolineshloka
{तमुवाचाशु शत्रुघ्नः क्व मे जीवन्गमिष्यसि}
{शत्रुर्यदृच्छया दृष्टो न मोक्तव्यः कृतात्मना} %7-68-18

\twolineshloka
{यो हि विक्लवया बुद्ध्या प्रसरं शत्रवे ददौ}
{स हतो मन्दबुद्धित्वाद्यथा कापुरुषस्तथा} %7-68-19

\twolineshloka
{तस्मात्सुदृष्टं कुरु जीवलोकं शरैः शितैस्त्वां विविधैर्नयामि}
{यमस्य गेहाभिमुखं हि पापं रिपुं त्रिलोकस्य च राघवस्य} %7-68-20


॥इत्यार्षे श्रीमद्रामायणे वाल्मीकीये आदिकाव्ये उत्तरकाण्डे लवणशत्रुघ्नविवादः नाम अष्टषष्ठितमः सर्गः ॥७-६८॥
