\sect{पञ्चाशीतितमः सर्गः — वृत्रवधः}

\twolineshloka
{लक्ष्मणस्य तु तद्वाक्यं श्रुत्वा शत्रुनिबर्हणः}
{वृत्रघातमशेषेण कथयेत्याह सुव्रत} %7-85-1

\twolineshloka
{राघवेणैवमुक्तस्तु सुमित्रानन्दवर्धनः}
{भूय एव कथां दिव्यां कथयामास सुव्रतः} %7-85-2

\twolineshloka
{सहस्राक्षवचः श्रुत्वा सर्वेषां च दिवौकसाम्}
{विष्णुर्देवानुवाचेदं सर्वानिन्द्रपुरोगमान्} %7-85-3

\twolineshloka
{पूर्वं सौहृदबद्धोऽस्मि वृत्रस्य ह महात्मनः}
{तेन युष्मत्प्रियार्थं हि नाहं हन्मि महासुरम्} %7-85-4

\twolineshloka
{अवश्यं करणीयं च भवतां सुखमुत्तमम्}
{तस्मादुपायमाख्यास्ये येन वृत्रो निहन्यते} %7-85-5

\twolineshloka
{त्रिधाभूतं करिष्यामि ह्यात्मानं सुरसत्तमाः}
{तेन वृत्रं सहस्राक्षो वधिष्यति न संशयः} %7-85-6

\twolineshloka
{एकांशो वासवं यातु द्वितीयो वज्रमेव तु}
{तृतीयो भूतलं शक्रस्तदा वृत्रं वधिष्यति} %7-85-7

\twolineshloka
{तथा ब्रुवति देवेशे देवा वाक्यमथाब्रुवन्}
{एवमेतन्न सन्देहो यथा वदसि दैत्यहन्} %7-85-8

\twolineshloka
{भद्रं तेऽस्तु गमिष्यामो वृत्रासुरवधैषिणः}
{भजस्व परमोदार वासवं स्वेन तेजसा} %7-85-9

\twolineshloka
{ततः सर्वे महात्मानः सहस्राक्षपुरोगमाः}
{तदारण्यमुपाक्रामन्यत्र वृत्रो महासुरः} %7-85-10

\twolineshloka
{ते पश्यंस्तेजसा भूतं तप्यन्तमसुरोत्तमम्}
{पिबन्तमिव लोकांस्त्रीन्निर्दहन्तमिवाम्बरम्} %7-85-11

\twolineshloka
{दृष्ट्वैव चासुरश्रेष्ठं देवास्त्रासमुपागमन्}
{कथमेनं वधिष्यामः कथं न स्यात्पराजयः} %7-85-12

\twolineshloka
{तेषां चिन्तयतां तत्र सहस्राक्षः पुरन्दरः}
{वज्रं प्रगृह्य पाणिभ्यां प्राहिणोद्वृत्रमूर्धनि} %7-85-13

\twolineshloka
{कालाग्निनेव घोरेण तप्तेनैव महार्चिषा}
{पतता वृत्रशिरसा जगत्ऺत्रासमुपागमत्} %7-85-14

\twolineshloka
{असम्भाव्यं वधं तस्य वृत्रस्य विबुधाधिपः}
{चिन्तयानो जगामाशु लोकस्यान्तं महायशाः} %7-85-15

\twolineshloka
{तमिन्द्रं ब्रह्महत्याऽऽशु गच्छन्तमनुगच्छति}
{अपतच्चास्य गात्रेषु तमिन्द्रं दुःखमाविशत्} %7-85-16

\twolineshloka
{हतारयः प्रनष्टेन्द्रा देवाः साग्निपुरोगमाः}
{विष्णुं त्रिभुवनेशानं मुहुर्मुहुरपूजयन्} %7-85-17

\twolineshloka
{त्वं गतिः परमेशान पूर्वजो जगतः पिता}
{रक्षार्थं सर्वभूतानां विष्णुत्वमुपजग्मिवान्} %7-85-18

\twolineshloka
{हतश्चायं त्वया वृत्रो ब्रह्महत्या च वासवम्}
{बाधते सुरशार्दूल मोक्षं तस्या विनिर्दिश} %7-85-19

\twolineshloka
{तेषां तद्वचनं श्रुत्वा देवानां विष्णुरब्रवीत्}
{मामेव यजतां शक्रः पावयिष्यामि वज्रिणम्} %7-85-20

\twolineshloka
{पुण्येन हयमेधेन मामिष्ट्वा पाकशासनः}
{पुनरेष्यति देवानामिन्द्रत्वमकुतोभयम्} %7-85-21

\twolineshloka
{एवं सन्दिश्य तां वाणीं देवानाममृतोपमाम्}
{जगाम विष्णुर्देवेशः स्तूयमानस्त्रिविष्टपम्} %7-85-22


॥इत्यार्षे श्रीमद्रामायणे वाल्मीकीये आदिकाव्ये उत्तरकाण्डे वृत्रवधः नाम पञ्चाशीतितमः सर्गः ॥७-८५॥
