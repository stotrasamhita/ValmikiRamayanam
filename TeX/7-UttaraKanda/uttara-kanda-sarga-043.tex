\sect{त्रिचत्वारिशः सर्गः — भद्रवाक्यश्रवणम्}

\twolineshloka
{तत्रोपविष्टं राजानमुपासन्ते विचक्षणाः}
{कथानां बहुरूपाणां हास्यकाराः समन्ततः} %7-43-1

\twolineshloka
{विजयो मधुमत्तश्च काश्यपो मङ्गलः कुटः}
{सुराजः कालियो भद्रो दन्तवक्रः सुमागधः} %7-43-2

\twolineshloka
{एते कथा बहुविधाः परिहाससमन्विताः}
{कथयन्ति स्म संहृष्टा राघवस्य महात्मनः} %7-43-3

\twolineshloka
{ततः कथायां कस्याञ्चिद्राघवः समभाषत}
{काः कथा नगरे भद्र वर्तन्ते विषयेषु च} %7-43-4

\twolineshloka
{मामाश्रितानि कान्याहुः पौरजानपदा जनाः}
{किं च सीतां समाश्रित्य भरतं किं च लक्ष्मणम्} %7-43-5

\twolineshloka
{किं नु शत्रुघ्नमुद्दिश्य कैकयीं किं नु मातरम्}
{वक्तव्यतां च राजानो वरे राज्ये व्रजन्ति च} %7-43-6

\twolineshloka
{एवमुक्ते तु रामेण भद्रः प्राञ्जलिरब्रवीत्}
{स्थिताः कथा शुभाः राजन्वर्तन्ते पुरवासिनाम्} %7-43-7

\twolineshloka
{अमुं तु विजयं सौम्य दशग्रीववधार्जितम्}
{भूयिष्ठं स्वपुरे पौरैः कथ्यन्ते पुरुषर्षभ} %7-43-8

\twolineshloka
{एवमुक्तस्तु भद्रेण राघवो वाक्यमब्रवीत्}
{कथयस्व यथातत्त्वं सर्वं निरवशेषतः} %7-43-9

\twolineshloka
{शुभाशुभानि वाक्यानि यान्याहुः पुरवासिनः}
{श्रुत्वेदानीं शुभं कुर्यां नकुर्यामशुभानि च} %7-43-10

\twolineshloka
{कथयस्व च विस्रब्धो निर्भयं विगतज्वरः}
{कथयन्ति यथा पौराः पापा जनपदेषु च} %7-43-11

\twolineshloka
{राघवेणैवमुक्तस्तु भद्रः सुरुचिरं वचः}
{प्रत्युवाच महाबाहुं प्राञ्जलिः सुसमाहितः} %7-43-12

\twolineshloka
{शृणु राजन्यथा पौराः कथयन्ति शुभाशुभम्}
{चत्वरापणरथ्यासु वनेषूपवनेषु च} %7-43-13

\twolineshloka
{दुष्करं कृतवान्रामः समुद्रे सेतुबन्धनम्}
{अश्रुतं पूर्वकैः कैश्चिद्देवैरपि सदानवैः} %7-43-14

\twolineshloka
{रावणश्च दुराधर्षो हतः सबलवाहनः}
{वानराश्च वशं नीता ऋक्षाश्च सह राक्षसैः} %7-43-15

\twolineshloka
{हत्वा च रावणं सङ्ख्ये सीतामाहृत्य राघवः}
{अमर्षं पृष्ठतः कृत्वा स्ववेश्म पुनरानयत्} %7-43-16

\twolineshloka
{कीदृशं हृदये तस्य सीतासम्भोगजं सुखम्}
{अङ्कमारोप्य तु पुरा रावणेन बलाद्धृताम्} %7-43-17

\twolineshloka
{लङ्कामपि पुरा नीतामशोकवनिकां गताम्}
{रक्षसां वशमापन्नां कथं रामो न कुत्सते} %7-43-18

\twolineshloka
{अस्माकमपि दारेषु सहनीयं भविष्यति}
{यथा हि कुरुते राजा प्रजा तमनुवर्तते} %7-43-19

\twolineshloka
{एवं बहुविधा वाचो वदन्ति पुरवासिनः}
{नगरेषु च सर्वेषु राजञ्जनपदेषु च} %7-43-20

\twolineshloka
{तस्यैवं भाषितं श्रुत्वा राघवः परमार्तवत्}
{उवाच सुहृदः सर्वान्कथमेतद्ब्रवीथ माम्} %7-43-21

\twolineshloka
{सर्वे तु शिरसा भूमावभिवाद्य प्रणम्य च}
{प्रत्यूचू राघवं दीनमेवमेतन्न संशयः} %7-43-22

\twolineshloka
{श्रुत्वा तु वाक्यं काकुत्स्थः सर्वेषां समुदीरितम्}
{विसर्जयामास तदा वयस्याञ्छत्रुसूदनः} %7-43-23


॥इत्यार्षे श्रीमद्रामायणे वाल्मीकीये आदिकाव्ये उत्तरकाण्डे भद्रवाक्यश्रवणम् नाम त्रिचत्वारिशः सर्गः ॥७-४३॥
