\sect{तृतीयः सर्गः — वैश्रवणलोकपालपदलङ्कादिप्राप्तिः}

\twolineshloka
{अथ पुत्रः पुलस्त्यस्य विश्रवा मुनिपुङ्गवः}
{अचिरेणैव कालेन पितेव तपसि स्थितः} %7-3-1

\twolineshloka
{सत्यवाञ्छीलवाञ्छान्तः स्वाध्यायनिरतः शुचिः}
{सर्वभोगेष्वसंसक्तो नित्यं धर्मपरायणः} %7-3-2

\twolineshloka
{ज्ञात्वा तस्य तु तद्वृत्तं भरद्वाजो महामुनिः}
{ददौ विश्रवसे भार्यां स्वसुतां देववर्णिनीम्} %7-3-3

\twolineshloka
{प्रतिगृह्य तु धर्मेण भरद्वाजसुतां तदा}
{प्रजान्वेक्षिकया बुद्ध्या श्रेयो ह्यस्य विचिन्तयन्} %7-3-4

\twolineshloka
{मुदा परमया युक्तो विश्रवा मुनिपुङ्गवः}
{स तस्यां वीर्यसम्पन्नमपत्यं परमाद्भुतम्} %7-3-5

\twolineshloka
{जनयामास धर्मज्ञः सर्वैर्ब्रह्मगुणैर्युतम्}
{तस्मिञ्जाते तु संहृष्टः सम्बभूव पितामहः} %7-3-6

\twolineshloka
{दृष्ट्वा श्रेयस्करीं बुद्धिं धनाध्यक्षो भविष्यति}
{नाम तस्याकरोत्प्रीतः सार्धं देवर्षिभिस्तदा} %7-3-7

\twolineshloka
{यस्माद्विश्रवसोऽपत्यं सादृश्याद्विश्रवा इव}
{तस्माद्वैश्रवणो नाम भविष्यत्वेष विश्रुतः} %7-3-8

\twolineshloka
{स तु वैश्रवणस्तत्र तपोवनगतस्तदा}
{अवर्धताहुतिहुतो महातेजा यथानलः} %7-3-9

\twolineshloka
{तस्याश्रमपदस्थस्य बुद्धिर्जज्ञे महात्मनः}
{चरिष्ये परमं धर्मं धर्मो हि परमा गतिः} %7-3-10

\twolineshloka
{स तु वर्षसहस्राणि तपस्तप्त्वा महावने}
{यन्त्रितो नियमैरुग्रैश्चकार सुमहत्तपः} %7-3-11

\twolineshloka
{पूर्णे वर्षसहस्रान्ते तं तं विधिमकल्पयत्}
{जलाशी मारुताहारो निराहारस्तथैव च} %7-3-12

\twolineshloka
{एवं वर्षसहस्राणि जग्मुस्तान्येकवर्षवत्}
{अथ प्रीतो महातेजाः सेन्द्रैः सुरगणैः सह} %7-3-13

\threelineshloka
{गत्वा तस्याश्रमपदं ब्रह्मेदं वाक्यमब्रवीत्}
{परितुष्टोऽस्मि ते वत्स कर्मणानेन सुव्रत}
{वरं वृणीष्व भद्रं ते वरार्हस्त्वं महामते} %7-3-14

\twolineshloka
{अथाब्रवीद्वैश्रवणः पितामहमुपस्थितम्}
{भगवँल्लोकपालत्वमिच्छेयं वित्तरक्षणम्} %7-3-15

\twolineshloka
{अथाब्रवीद्वैश्रवणं परितुष्टेन चेतसा}
{ब्रह्मा सुरगणैः सार्धं बाढमित्येव हृष्टवत्} %7-3-16

\onelineshloka
{अहं वै लोकपालानां चतुर्थं स्रष्टुमुद्यतः} %7-3-17

\threelineshloka
{यमेन्द्रवरुणानां च पदं यत्तव चेप्सितम्}
{तद्गच्छ त्वं हि धर्मज्ञ निधीशत्वमवाप्नुहि}
{शक्राम्बुपयमानां च चतुर्थस्त्वं भविष्यसि} %7-3-18

\twolineshloka
{एतच्च पुष्पकं नाम विमानं सूर्यसन्निभम्}
{प्रतिगृह्णीष्व यानार्थं त्रिदशैः समतां व्रज} %7-3-19

\twolineshloka
{स्वस्ति तेऽस्तु गमिष्यामः सर्व एव यथागतम्}
{कृतकृत्या वयं तात दत्त्वा तव वरद्वयम्} %7-3-20

\onelineshloka
{इत्युक्त्वा स गतो ब्रह्मा स्वस्थानं त्रिदशैः सह} %7-3-21

\threelineshloka
{गतेषु ब्रह्मपूर्वेषु देवेष्वथ नभस्थलम्}
{वने स पितरं प्राह प्राञ्जलिः प्रयतात्मवान्}
{निवासनं न मे देवो विदधे स प्रजापतिः} %7-3-22

\twolineshloka
{भगवँल्लब्धवानस्मि वरमिष्टं पितामहात्}
{तं पश्य भगवन्कञ्चिन्निवासं साधु मे प्रभो} %7-3-23

\twolineshloka
{न च पीडा भवेद्यत्र प्राणिनो यस्य कस्यचित्}
{एवमुक्तस्तु पुत्रेण विश्रवा मुनिपुङ्गवः} %7-3-24

\twolineshloka
{वचनं प्राह धर्मज्ञः श्रूयतामिति सत्तमः}
{दक्षिणस्योदधेस्तीरे त्रिकूटो नाम पर्वतः} %7-3-25

\twolineshloka
{तस्याग्रे तु विशाला सा महेन्द्रस्य पुरी यथा}
{लङ्का नाम पुरी रम्या निर्मिता विश्वकर्मणा} %7-3-26

\twolineshloka
{राक्षसानां निवासार्थं यथेन्द्रस्यामरावती}
{तत्र त्वं वस भद्रं ते लङ्कायां नात्र संशयः} %7-3-27

\twolineshloka
{हेमप्राकारपरिघा यन्त्रशस्त्रसमावृता}
{रमणीया पुरी सा हि रुक्मवैडूर्यतोरणा} %7-3-28

\twolineshloka
{राक्षसैः सा परित्यक्ता पुरा विष्णुभयार्दितैः}
{शून्या रक्षोगणैः सर्वै रसातलतलं गतैः} %7-3-29

\twolineshloka
{शून्या सम्प्रति लङ्का सा प्रभुस्तस्या न विद्यते}
{स त्वं तत्र निवासाय गच्छ पुत्र यथासुखम्} %7-3-30

\twolineshloka
{निर्दोषस्तत्र ते वासो न बाधास्तत्र कस्यचित्}
{एतच्छुत्वा स धर्मात्मा धर्मिष्ठं वचनं पितुः} %7-3-31

\twolineshloka
{निवासयामास तदा लङ्कां पर्वतमूर्धनि}
{नैर्ऋतानां सहस्रैस्तु हृष्टैः प्रमुदुतैः सह} %7-3-32

\onelineshloka
{अचिरेणैव कालेन सम्पूर्णा तस्य शासनात्} %7-3-33

\twolineshloka
{स तु तत्रावसत्प्रीतो धर्मात्मा नैर्ऋतर्षभः}
{समुद्रपरिघायां तु लङ्कायां विश्रवात्मजः} %7-3-34

\twolineshloka
{काले काले तु धर्मात्मा पुष्पकेण धनेश्वरः}
{अभ्यागच्छद्विनीतात्मा पितरं मातरं च हि} %7-3-35

\twolineshloka
{स देवगन्धर्वगणैरभिष्टुतस्तथाप्सरोनृत्यविभूषितालयः}
{गभस्तिभिः सूर्य इवावभासयन्पितुः समीपं प्रययौ स वित्तपः} %7-3-36


॥इत्यार्षे श्रीमद्रामायणे वाल्मीकीये आदिकाव्ये उत्तरकाण्डे वैश्रवणलोकपालपदलङ्कादिप्राप्तिः नाम तृतीयः सर्गः ॥७-३॥
