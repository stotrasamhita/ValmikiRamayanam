\sect{एकविंशः सर्गः — यमरावणयुद्धम्}

\twolineshloka
{एवं सञ्चिन्त्य विप्रेन्दो जगाम लघुविक्रमः}
{आख्यातुं तद्यथावृत्तं यमस्य सदनं प्रति} %7-21-1

\twolineshloka
{अपश्यत्स यमं तत्र देवमग्निपुरस्कृतम्}
{विधानमुपतिष्ठन्तं प्राणिनो यस्य यादृशम्} %7-21-2

\twolineshloka
{स तु दृष्ट्वा यमः प्राप्तं महर्षिं तत्र नारदम्}
{अब्रवीत्सुखमासीनमर्घ्यमावेद्य धर्मतः} %7-21-3

\twolineshloka
{कच्चित्क्षेमं तु देवर्षे कच्चिद्धर्मो न नश्यति}
{किमागमनकृत्यं ते देवगन्धर्वसेवित} %7-21-4

\twolineshloka
{अब्रवीत्तु तदा वाक्यं नारदो भगवानृषिः}
{श्रूयतामभिधास्यामि विधानं च विधीयताम्} %7-21-5

\twolineshloka
{एष नाम्नो दशग्रीवः पितृराज निशाचरः}
{उपयाति वशं नेतुं विक्रमैस्त्वां सुदुर्जयम्} %7-21-6

\twolineshloka
{एतेन कारणेनाहं त्वरितो ह्यागतः प्रभो}
{दण्डप्रहरणस्याद्य तव किं नु भविष्यति} %7-21-7

\twolineshloka
{एतस्मिन्नन्तरे दुरादंशुमन्तमिवोदितम्}
{ददृशुर्दीप्तमायान्तं विमानं तस्य रक्षसः} %7-21-8

\twolineshloka
{तं देशं प्रभया तस्य पुष्पकस्य महाबलः}
{कृत्वा वितिमिरं सर्वं समीपं सोऽभ्यवर्तत} %7-21-9

\twolineshloka
{सोऽपश्यत्स महाबाहुर्दशग्रीवस्ततस्ततः}
{प्राणिनः सुकृतं कर्म भुञ्जानांश्चैव दुष्कृतम्} %7-21-10

\twolineshloka
{अपश्यत्सैनिकांश्चास्य यमस्यानुचरैः सह}
{यमस्य पुरुषैरुग्रैर्घोररूपैर्भयानकैः} %7-21-11

\twolineshloka
{ददर्श वध्यमानांश्च क्लिश्यमानांश्च देहिनः}
{क्रोशतश्च महानादं तीव्रनिष्टनतत्परान्} %7-21-12

\onelineshloka
{कृमिभिर्भक्ष्यमाणांश्च सारमेयैश्च दारुणैः} %7-21-13

\twolineshloka
{क्षोत्रायासकरा वाचो वदतश्च भयावहाः}
{सन्तार्यमाणान्वैतरणीं बहुशः शोणितोदकाम्} %7-21-14

\twolineshloka
{वालुकासु च तप्तासु तप्यमानान्मुहूर्मुहुः}
{असिपत्रवने चैव भिद्यमानानधार्मिकान्} %7-21-15

\twolineshloka
{रौरवे क्षारनद्यां च क्षुरधारासु चैव हि}
{पानीयं याचमानांश्च तृषितान्क्षुधितानपि} %7-21-16

\twolineshloka
{शवभूतान्कृशान्दीनान्विवर्णान्मुक्तमूर्धजान्}
{मलपङ्कधरान्दीनान्रूक्षांश्च परिधावतः} %7-21-17

\threelineshloka
{ददर्श रावणो मार्गे शतशोऽथ सहस्रशः}
{कांश्चिच्च गृहमुख्येषु गीतवादित्रनिःस्वनैः}
{प्रमोदमानानद्राक्षीद्रावणः सुकृतैः स्वकैः} %7-21-18

\twolineshloka
{गौरसं गोप्रदातारो ह्यन्नं चैवान्नदायिनः}
{गृहांश्च गृहदातारः स्वकर्मफलमश्नतः} %7-21-19

\onelineshloka
{सुवर्णमणिमुक्ताभिः प्रमदाभिरलङ्कृतान्} %7-21-20

\twolineshloka
{धार्मिकानपरांस्तत्र दीप्यमानान्स्वतेजसा}
{ददर्श सुमहाबाहू रावणो राक्षसाधिपः} %7-21-21

\twolineshloka
{ततस्तान्भिद्यमानांश्च कर्मभिर्दुष्कृतैः स्वकैः}
{रावणो मोचयामास विक्रमेण बलाद्बली} %7-21-22

\twolineshloka
{प्राणिनो मोचितास्तेन दशग्रीवेण रक्षसा}
{सुखमापुर्मुहूर्तं ते ह्यतर्कितमचिन्तितम्} %7-21-23

\twolineshloka
{प्रेतेषु मुच्यमानेषु राक्षसेन महीयसा}
{प्रेतगोपाः सुसङ्क्रुद्धा राक्षसेन्द्रमभिद्रवन्} %7-21-24

\twolineshloka
{ततो हलहलाशब्दः सर्वदिग्भ्यः समुत्थितः}
{धर्मराजस्य योधानां शूराणां सम्प्रधावताम्} %7-21-25

\twolineshloka
{ते प्रासैः परिघैः शूलैर्मुसलैः शक्तितोमरैः}
{पुष्पकं समवर्षन्त शूराः शतसहस्रशः} %7-21-26

\twolineshloka
{तस्यासनानि प्रासादान्वेदिकास्तोरणानि च}
{पुष्पकस्य बभञ्जुस्ते शीघ्रं मधुकरा इव} %7-21-27

\twolineshloka
{देवनिष्ठानभूतं तद्विमानं पुष्पकं मृधे}
{भज्यमानं तथैवासीदक्षयं ब्रह्मतेजसा} %7-21-28

\twolineshloka
{असङ्ख्या सुमहत्यासीत्तस्य सेना महात्मनः}
{शूराणामग्रयातऽणां सहस्राणि शतानि च} %7-21-29

\threelineshloka
{ततो वृक्षैश्च शैलैश्च प्रासादानां शतैस्तथा}
{ततस्ते सचिवास्तस्य यथाकामं यथाबलम्}
{अयुध्यन्त महावीराः स च राजा दशाननः} %7-21-30

\twolineshloka
{ते तु शोणितदिग्धाङ्गाः सर्वशस्त्रसमाहताः}
{अमात्या राक्षसेन्द्रस्य चक्रुरायोधनं महत्} %7-21-31

\twolineshloka
{अन्योन्यं ते महाभागा जघ्नुः प्रहरणैर्भृशम्}
{यमस्य च महाबाहो रावणस्य च मन्त्रिणः} %7-21-32

\twolineshloka
{अमात्यांस्तांस्तु सन्त्यज्य यमयोधा महाबलाः}
{तमेव चाभ्यधावन्त शूलवर्षैर्दशाननम्} %7-21-33

\twolineshloka
{ततः शोणितदिग्धाङ्गः प्रहारैर्जर्जरीकृतः}
{फुल्लाशोक इवाभाति पुष्पके राक्षसाधिपः} %7-21-34

\twolineshloka
{स तु शूलगदापासाञ्छक्तितोमरसायकान्}
{मुसलानि शिलावृक्षान्मुमोचास्त्रबलाद्बली} %7-21-35

\twolineshloka
{तरूणां च शिलानां च शस्त्राणां चातिदारुणम्}
{यमसैन्येषु तद्वर्षं पपात धरणीतले} %7-21-36

\twolineshloka
{तांस्तु सर्वान्विनिर्भिद्य तदस्त्रमपहत्य च}
{जघ्नुस्ते राक्षसं घोरमेकं शतसहस्रशः} %7-21-37

\twolineshloka
{परिवार्य च तं सर्वे शैलं मोघोत्करा इव}
{भिन्दिपालैश्च शूलैश्च निरुच्छ्वासमपोथयन्} %7-21-38

\twolineshloka
{विमुक्तकवचः क्रुद्धः सिक्तः शोणितविस्रवैः}
{ततः स पुष्पकं त्यक्त्वा पृथिव्यामवतिष्ठत} %7-21-39

\twolineshloka
{ततः स कार्मुकी वाणी समरे चाभिवर्तत}
{लब्धसञ्ज्ञो मुहूर्तेन क्रुद्धस्तस्थौ यथान्तकः} %7-21-40

\twolineshloka
{ततः पाशुपतं दिव्यमस्त्रं सन्धाय कार्मुके}
{तिष्ठ तिष्ठेति तानुक्त्वा तच्चापं विचकर्ष स} %7-21-41

\twolineshloka
{आकर्णात्स विकृष्याथ चापमिन्द्रारिराहवे}
{मुमोच तं शरं क्रुद्धस्त्रिपुरे शङ्करो यथा} %7-21-42

\twolineshloka
{तस्य रूपं शरस्यासीत्विधूमज्वालमण्डलम्}
{वनं दहिष्यतो घर्मे दावाग्नेरिव मूर्च्छतः} %7-21-43

\twolineshloka
{ज्वालामाली स तु शरः क्रव्यादानुगतो रणे}
{मुक्तो गुल्मान्द्रुमांश्चापि भस्म कृत्वा प्रधावति} %7-21-44

\twolineshloka
{ते तस्य तेजसा दग्धाः सैन्या वैवस्वतस्य तु}
{रणे तस्मिन्निपतिता दावदग्धा नगा इव} %7-21-45

\twolineshloka
{ततस्तु सिचवैः सार्धं राक्षसो भीमविक्रमः}
{ननाद सुमहानादं कम्पयन्निव मेदिनीम्} %7-21-46


॥इत्यार्षे श्रीमद्रामायणे वाल्मीकीये आदिकाव्ये उत्तरकाण्डे यमरावणयुद्धम् नाम एकविंशः सर्गः ॥७-२१॥
