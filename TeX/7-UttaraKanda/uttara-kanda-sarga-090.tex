\sect{नवतितमः सर्गः — इलापुरुषत्वप्राप्तिः}

\twolineshloka
{तथोक्तवति रामे तु तस्य जन्म तदद्भुतम्}
{उवाच लक्ष्मणो भूयो भरतश्च महायशाः} %7-90-1

\twolineshloka
{इला सा सोमपुत्रस्य संवत्सरमथोषिता}
{अकरोत्किं नरश्रेष्ठ तत्त्वं शंसितुमर्हसि} %7-90-2

\twolineshloka
{तयोस्तद्वाक्यमाधुर्यं निशम्य परिपृच्छतोः}
{रामः पुनरुवाचेमां प्रजापतिसुते कथाम्} %7-90-3

\twolineshloka
{पुरुषत्वं गते शूरे बुधः परमबुद्धिमान्}
{संवर्तं परमोदारमाजुहाव महायशाः} %7-90-4

\twolineshloka
{च्यवनं भृगुपुत्रं च मुनिं चारिष्टनेमिनम्}
{प्रमोदनं मोदकरं ततो दुर्वाससं मुनिम्} %7-90-5

\twolineshloka
{एतान्सर्वान्समानीय वाक्यज्ञस्तत्त्वदर्शनः}
{उवाच सर्वान्सुहृदो धैर्येण सुसमाहितान्} %7-90-6

\twolineshloka
{अयं राजा महाबाहुः कर्दमस्य इलः सुतः}
{जानीतैनं यथाभूतं श्रेयो ह्यत्र विधीयताम्} %7-90-7

\twolineshloka
{तेषां संवदतामेव तमाश्रममुपागमत्}
{कर्दमस्तु महातेजा द्विजैः सह महात्मभिः} %7-90-8

\twolineshloka
{पुलस्त्यश्च क्रतुश्चैव वषट्कारस्तथैव च}
{ओङ्कारश्च महातेजास्तमाश्रममुपागमत्} %7-90-9

\twolineshloka
{ते सर्वे हृष्टमनसः परस्परसमागमे}
{हितैषिणो बाल्हिपतेः पृथग्वाक्यान्यथाब्रुवन्} %7-90-10

\twolineshloka
{कर्दमस्त्वब्रवीद्वाक्यं सुतार्थं परमं हितम्}
{द्विजाः शृणुत मद्वाक्यं यच्छ्रेयः पार्थिवस्य हि} %7-90-11

\threelineshloka
{नान्यं पश्यामि भैषज्यमन्तरा वृषभध्वजम्}
{नाश्वमेधात्परो यज्ञः प्रियश्चैव महात्मनः}
{तस्माद्यजामहे सर्वे पार्थिवार्थे दुरासदम्} %7-90-12

\twolineshloka
{कर्दमेनैवमुक्तास्तु सर्व एव द्विजर्षभाः}
{रोचयन्ति स्म तं यज्ञं रुद्रस्याराधनं प्रति} %7-90-13

\twolineshloka
{संवर्तस्य तु राजर्षेः शिष्यः पुरपुरञ्जयः}
{मरुत्त इति विख्यातस्तं यज्ञं समुपाहरत्} %7-90-14

\twolineshloka
{ततो यज्ञो महानासीद्बुधाश्रमसमीपतः}
{रुद्रश्च परमं तोषमाजगाम महायशाः} %7-90-15

\twolineshloka
{अथ यज्ञे समाप्ते तु प्रीतः परमया मुदा}
{उमापतिर्द्विजान्सर्वानुवाच इलसन्निधौ} %7-90-16

\twolineshloka
{प्रीतोऽस्मि हयमेधेन भक्त्या च द्विजसत्तमाः}
{अस्य बाह्लिपतेश्चैव किं करोमि प्रियं शुभम्} %7-90-17

\twolineshloka
{तथा वदति देवेशे द्विजास्ते सुसमाहिताः}
{प्रसादयन्ति देवेशं यथा स्यात्पुरुषस्त्विला} %7-90-18

\twolineshloka
{ततः प्रीतो महादेवः पुरुषत्वं ददौ पुनः}
{इलायै सुमहातेजा दत्त्वा चान्तरधीयत} %7-90-19

\twolineshloka
{निवृत्ते हयमेधे च गतश्चादर्शनं हरः}
{यथागतं द्विजाः सर्वे ह्यगच्छन्दीर्घदर्शिनः} %7-90-20

\twolineshloka
{राजा तु बाह्लिमुत्सृज्य मध्यदेशे ह्यनुत्तमम्}
{निवेशयामास पुरं प्रतिष्ठानं यशस्करम्} %7-90-21

\twolineshloka
{शशबिन्दुश्च राजर्षिर्बाह्लिं पुरपुरञ्जयः}
{प्रतिष्ठाने इलो राजा प्रजापतिसुतो बली} %7-90-22

\twolineshloka
{स काले प्राप्तवाँल्लोकमिलो ब्राह्ममनुत्तमम्}
{ऐलः पुरूरवा राजा प्रतिष्ठानमवाप्तवान्} %7-90-23

\twolineshloka
{ईदृशो ह्यश्वमेधस्य प्रभावः पुरुषर्षभौ}
{स्त्रीभूतः पौरुषं लेभे यच्चान्यदपि दुर्लभम्} %7-90-24


॥इत्यार्षे श्रीमद्रामायणे वाल्मीकीये आदिकाव्ये उत्तरकाण्डे इलापुरुषत्वप्राप्तिः नाम नवतितमः सर्गः ॥७-९०॥
