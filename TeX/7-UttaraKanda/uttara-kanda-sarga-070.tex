\sect{सप्ततितमः सर्गः — मधुपुरीनिवेशः}

\twolineshloka
{हते तु लवणे देवाः सेन्द्राः साग्निपुरोगमाः}
{ऊचुः सुमधुरां वाणीं शत्रुघ्नं शत्रुतापनम्} %7-70-1

\twolineshloka
{दिष्ट्या ते विजयो वत्स दिष्ट्या लवणराक्षसः}
{हतः पुरुषशार्दूल वरं वरय सुव्रत} %7-70-2

\twolineshloka
{वरदास्तु महाबाहो सर्व एव समागताः}
{विजयाकाङ्क्षिणस्तुभ्यममोघं दर्शनं हि नः} %7-70-3

\twolineshloka
{देवानां भाषितं श्रुत्वा शूरो मूर्ध्नि कृताञ्जलिः}
{प्रत्युवाच महाबाहुः शत्रुघ्नः प्रयतात्मवान्} %7-70-4

\twolineshloka
{इयं मधुपुरी रस्या मधुरा देवनिर्मिता}
{निवेशं प्राप्नुयाच्छीघ्रमेष मेऽस्तु वरः परः} %7-70-5

\twolineshloka
{तं देवाः प्रीतमनसो बाढमित्येव राघवम्}
{भविष्यति पुरी रम्या शूरसेना न संशयः} %7-70-6

\twolineshloka
{ते तथोक्त्वा महात्मानो दिवमारुरुहुस्तदा}
{शत्रुघ्नोऽपि महातेजास्तां सेनां समुपानयत्} %7-70-7

\twolineshloka
{सा सेना शीघ्रमागच्छच्छ्रुत्वा शत्रुघ्नशासनम्}
{निवेशनं च शत्रुघ्नः श्रावणेन समारभत्} %7-70-8

\twolineshloka
{सा पुरा दिव्यसङ्काशा वर्षे द्वादशमे शुभे}
{निविष्टा शूरसेनानां विषयश्चाकुतोभयः} %7-70-9

\twolineshloka
{क्षेत्राणि सस्ययुक्तानि काले वर्षति वासवः}
{अरोगवीरपुरुषा शत्रुघ्नभुजपालिता} %7-70-10

\threelineshloka
{अर्धचन्द्रप्रतीकाशा यमुनातीरशोभिता}
{शोभिता गृहमुख्यैश्च चत्वरापणवीथिकैः}
{चातुर्वर्ण्यसमायुक्ता नानावाणिज्यशोभिता} %7-70-11

\twolineshloka
{यच्च तेन पुरा शुभ्रं लवणेन कृतं महत्}
{तच्छोभयति शत्रुघ्नो नानावर्णोपशोभिताम्} %7-70-12

\twolineshloka
{आरामैश्च विहारैश्च शोभमानं समन्ततः}
{शोभितां शोभनीयैश्च तथाऽन्यैर्दैवमानुषैः} %7-70-13

\twolineshloka
{तां पुरीं दिव्यसङ्काशां नानापण्योपशोभिताम्}
{नानादेशागतैश्चापि वणिग्भिरुपशोभिताम्} %7-70-14

\twolineshloka
{तां समृद्धां समृद्धार्थः शत्रुघ्नो भरतानुजः}
{निरीक्ष्य परमप्रीतः परं हर्षमुपागमत्} %7-70-15

\twolineshloka
{तस्य बुद्धिः समुत्पन्ना निवेश्य मधुरां पुरीम्}
{रामपादौ निरीक्षेऽहं वर्षे द्वादश आगते} %7-70-16

\twolineshloka
{ततः स ताममरपुरोपमां पुरीं निवेश्य वै विविधजनाभिसंवृताम्}
{नराधिपो रघुपतिपाददर्शने दधे मतिं रघुकुलवंशवर्धनः} %7-70-17


॥इत्यार्षे श्रीमद्रामायणे वाल्मीकीये आदिकाव्ये उत्तरकाण्डे मधुपुरीनिवेशः नाम सप्ततितमः सर्गः ॥७-७०॥
