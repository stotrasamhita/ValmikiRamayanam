\sect{अशीतितमः सर्गः — अरजासङ्गमः}

\twolineshloka
{एतदाख्याय रामाय महर्षिः कुम्भसम्भवः}
{अस्यामेवापरं वाक्यं कथायामुपचक्रमे} %7-80-1

\twolineshloka
{ततः स दण्डः काकुत्स्थ बहुवर्षगणायुतम्}
{अकरोत्तत्र दान्तात्मा राज्यं निहतकण्टकम्} %7-80-2

\twolineshloka
{अथ काले तु कस्मिंश्चिद्राजा भार्गवमाश्रमम्}
{रमणीयमुपाक्रामच्चैत्रे मासि मनोरमे} %7-80-3

\twolineshloka
{तत्र भार्गवकन्यां स रूपेणाप्रतिमां भुवि}
{विचरन्तीं वनोद्देशे दण्डोऽपश्यदनुत्तमाम्} %7-80-4

\twolineshloka
{स दृष्ट्वा तां सुदुर्मेधा अनङ्गशरपीडितः}
{अभिगम्य सुसंविग्नां कन्यां वचनमब्रवीत्} %7-80-5

\twolineshloka
{कुतस्त्वमसि सुश्रोणि कस्य वाऽसि सुता शुभे}
{पीडितोऽहमनङ्गेन गच्छामि त्वां शुभानने} %7-80-6

\twolineshloka
{तस्य त्वेवं ब्रुवाणस्य मोहोन्मत्तस्य कामिनः}
{भार्गवी प्रत्युवाचेदं वचः सानुनयं नृपम्} %7-80-7

\twolineshloka
{भार्गवस्य सुतां विद्धि देवस्याक्लिष्टकर्मणः}
{अरजां नाम राजेन्द्र ज्येष्ठामाश्रमवासिनीम्} %7-80-8

\threelineshloka
{मा मां स्पृश बलाद्राजन्कन्या पितृवशा ह्यहम्}
{गुरुः पिता मे राजेन्द्र त्वं च शिष्यो महात्मनः}
{व्यसनं सुमहत्क्रुद्धः स ते दद्यान्महातपाः} %7-80-9

\twolineshloka
{यदि वान्यन्मया कार्यं धर्मदृष्टेन सत्पथा}
{वरयस्व नरश्रेष्ठ पितरं मे महाद्युतिम्} %7-80-10

\twolineshloka
{अन्यथा तु फलं तुभ्यं भवेद्घोराभिसंहितम्}
{क्रोधेन हि पिता मेऽसौ त्रैलोक्यमपि निर्दहेत्} %7-80-11

\onelineshloka
{दास्यते चानवद्याङ्ग तव मा याचितः पिता} %7-80-12

\twolineshloka
{एवं ब्रुवाणामरजां दण्डः कामवशं गतः}
{प्रत्युवाच मदोन्मत्तः शिरस्याधाय चाञ्जलिम्} %7-80-13

\twolineshloka
{प्रसादं कुरु सुश्रोणि न कालं क्षेप्तुमर्हसि}
{त्वत्कृते हि मम प्राणा विदीर्यन्ते वरानने} %7-80-14

\twolineshloka
{त्वां प्राप्य मे वधो वा स्याच्छापो वा यदि दारुणः}
{भक्तं भजस्व मां भीरु भजमानं सुविह्वलम्} %7-80-15

\twolineshloka
{एवमुक्त्वा तु तां कन्यां दोर्भ्यां गृह्य बलाद्बली}
{विस्फुरन्तीं यथाकामं मैथुनायोपचक्रमे} %7-80-16

\twolineshloka
{एतमर्थं महाघोरं दण्डः कृत्वा सुदारुणम्}
{नगरं प्रययावाशु मधुमन्तमनुत्तमम्} %7-80-17

\twolineshloka
{अरजाऽपि रुदन्ती सा आश्रमस्याविदूरतः}
{प्रतीक्षन्ति सुसंत्रस्ता पितरं देवसन्निभम्} %7-80-18


॥इत्यार्षे श्रीमद्रामायणे वाल्मीकीये आदिकाव्ये उत्तरकाण्डे अरजासङ्गमः नाम अशीतितमः सर्गः ॥७-८०॥
