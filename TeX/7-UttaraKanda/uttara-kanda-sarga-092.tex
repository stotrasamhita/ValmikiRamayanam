\sect{द्विनवतितमः सर्गः — हयचर्या}

\twolineshloka
{तत्सर्वमखिलेनाशु प्रस्थाप्य भरताग्रजः}
{हयं लक्षणसम्पन्नं कृष्णसारं मुमोच ह} %7-92-1

\twolineshloka
{ऋत्विग्भिर्लक्ष्मणं सार्धमश्वतन्त्रे नियोज्य च}
{ततोऽभ्यगच्छत्काकुत्स्थः सह सैन्येन नैमिशम्} %7-92-2

\twolineshloka
{यज्ञवाटं महाबाहुर्दृष्ट्वा परममद्भुतम्}
{प्रबर्षमतुलं लेभे श्रीमानिति वचोऽब्रवीत्} %7-92-3

\twolineshloka
{नैमिषे वसतस्तस्य सर्व एव नराधिपाः}
{आनिन्युरुपहारांश्च तान्रामः प्रत्यपूजयत्} %7-92-4

\twolineshloka
{उपकार्या महार्हाश्च पार्थिवानां महात्मनाम्}
{सानुगानां नरश्रेष्ठो व्यादिदेश महामतिः} %7-92-5

\twolineshloka
{अन्नपानानि वस्त्राणि सानुगानां महात्मनाम्}
{भरतः सन्ददावाशु शत्रुघ्नसहितस्तदा} %7-92-6

\twolineshloka
{वानराश्च महात्मानः सुग्रीवसहितास्तदा}
{विप्राणां प्रणताः सर्वे चक्रिरे परिवेषणम्} %7-92-7

\twolineshloka
{बिभीषणश्च रक्षोभिर्बहुभिः स्रग्विभिर्वृतः}
{ऋषीणामुग्रतपसां पूजां चक्रे महात्मनाम्} %7-92-8

\threelineshloka
{एवं सुविहितो यज्ञो हयमेधोऽभ्यवर्तत}
{लक्ष्मणेनागुप्ता च हयचर्या प्रवर्तत}
{ईदृशं राजसिंहस्य यज्ञप्रवरमुत्तमम्} %7-92-9

\twolineshloka
{नान्यः शब्दोऽभवत्तत्र हयमेधे महात्मनः}
{छन्दतो देहि विस्रब्धो यावत्तुष्यन्ति याचकाः} %7-92-10

\twolineshloka
{तावत्सर्वाणि दत्तानि क्रतुमध्ये महात्मनः}
{विविधानि च गौडानि खाण्डवानि तथैव च} %7-92-11

\twolineshloka
{न निस्सृतं भतत्योष्ठाद्वचनं यावदर्थिनाम्}
{तावद्वानररक्षोभिर्दत्तमेवाभ्यदृश्यत} %7-92-12

\twolineshloka
{न कश्चिन्मलिनस्तत्र दीनो वाऽप्यथ कृर्शतः}
{तस्मिन्यज्ञवरे राज्ञो हृष्टपुष्टजनावृते} %7-92-13

\twolineshloka
{ये च तत्र महात्मानो मुनयश्चिरजीविनः}
{नास्मरंस्तादृशं यज्ञं नाप्यासीत् स कदाचन} %7-92-14

\twolineshloka
{यः कृत्यवान्सुवर्णेन सुवर्णं लभते स्म सः}
{वित्तार्थी लभते वित्तं रत्नार्थी रत्नमेव च} %7-92-15

\twolineshloka
{रजतानां सुवर्णानामश्मनामथ वाससाम्}
{अनिशं दीयमानानां राशिः समुपदृश्यते} %7-92-16

\twolineshloka
{न शक्रस्य धनेशस्य यमस्य वरुणस्य वा}
{ईदृशो दृष्टपूर्वो न एवमूचुस्तपोधनाः} %7-92-17

\twolineshloka
{सर्वत्र वानरास्तस्थुः सर्वत्रैव च राक्षसाः}
{वासोधनान्नमर्थिभ्यः पूर्णहस्ता ददुर्भृशम्} %7-92-18

\twolineshloka
{ईदृशो राजसिंहस्य यज्ञः सर्वगुणान्वितः}
{संवत्सरमथो साग्रं वर्तते न च हीयते} %7-92-19


॥इत्यार्षे श्रीमद्रामायणे वाल्मीकीये आदिकाव्ये उत्तरकाण्डे हयचर्या नाम द्विनवतितमः सर्गः ॥७-९२॥
