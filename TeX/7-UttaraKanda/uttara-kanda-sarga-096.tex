\sect{षण्णवतितमः सर्गः — वाल्मीकिप्रत्ययदानम्}

\twolineshloka
{तस्यां रजन्यां व्युष्टायां यज्ञवाटगतो नृपः}
{ऋषीन्सर्वान्महातेजाः शब्दापयति राघवः} %7-96-1

\twolineshloka
{वसिष्ठो वामदेवश्च जाबालिरथ काश्यपः}
{विश्वामित्रो दीर्घतपा दुर्वासाश्च महातपाः} %7-96-2

\twolineshloka
{पुलस्त्योऽपि तथा शक्तिर्भार्गवश्चैव वामनः}
{मार्कण्डेयश्च दीर्घायुर्मौद्गल्यश्च महायशाः} %7-96-3

\twolineshloka
{गर्गश्च च्यवनश्चैव शतानन्दश्च धर्मवित्}
{भरद्वाजश्च तेजस्वी ह्यग्निपुत्रश्च सुप्रभः} %7-96-4

\twolineshloka
{नारदः पर्वतश्चैव गौतमश्च महायशाः}
{कात्यायनः सुयज्ञश्च ह्यगस्त्यस्तपसां निधिः} %7-96-5

\twolineshloka
{एते चान्ये च बहवो मुनयः संशितव्रताः}
{कौतूहलसमाविष्टाः सर्व एव समागताः} %7-96-6

\twolineshloka
{राक्षसाश्च महावीर्या वानराश्च महाबलाः}
{सर्व एव समाजग्मुर्महात्मानः कुतूहलात्} %7-96-7

\twolineshloka
{क्षत्रिया ये च शूद्राश्च वैश्याश्चैव सहस्रशः}
{नानादेशगताश्चैव ब्राह्मणाः संशितव्रताः} %7-96-8

\twolineshloka
{ज्ञाननिष्ठाः कर्मनिष्ठाः योगनिष्ठास्तथापरे}
{सीताशपथवीक्षार्थं सर्व एव समागताः} %7-96-9

\twolineshloka
{तदा समागतं सर्वमश्मभूतमिवाचलम्}
{श्रुत्वा मुनिवरस्तूर्णं ससीतः समुपागमत्} %7-96-10

\twolineshloka
{तमृषिं पृष्ठतः सीता त्वन्वगच्छदवाङ्मुखी}
{कृताञ्जलिर्बाष्पगला कृत्वा रामं मनोगतम्} %7-96-11

\twolineshloka
{दृष्ट्वा श्रुतिमिवायान्तीं ब्रह्माणमनुगामिनीम्}
{वाल्मीकेः पृष्ठतः सीतां साधुवादो महानभूत्} %7-96-12

\twolineshloka
{ततो हलहलाशब्दः सर्वेषामेवमाबभौ}
{दुःखजन्मविशालेन शोकेनाकुलितात्मनाम्} %7-96-13

\twolineshloka
{साधु रामेति केचित्तु साधु सीतेति चापरे}
{उभावेव च तत्रान्ये प्रेक्षकाः सम्प्रचुक्रुशुः} %7-96-14

\twolineshloka
{ततो मध्ये जनौघस्य प्रविश्य मुनिपुङ्गवः}
{सीतासहायो वाल्मीकिरिति होवाच राघवम्} %7-96-15

\twolineshloka
{इयं दाशरथे सीता सुव्रता धर्मचारिणी}
{अपवादैः परित्यक्ता ममाश्रमसमीपतः} %7-96-16

\twolineshloka
{लोकापवादभीतस्य तव राम महाव्रत}
{प्रत्ययं दास्यते सीता तदनुज्ञातुमर्हसि} %7-96-17

\twolineshloka
{इमौ तु जानकीपुत्रावुभौ च यमजातकौ}
{सुतौ तवैव दुर्धर्षौ सत्यमेतद्ब्रवीमि ते} %7-96-18

\twolineshloka
{प्रचेतसोऽहं दशमः पुत्रो राघवनन्दन}
{न स्मराम्यनृतं वाक्यमिमौ तु तव पुत्रकौ} %7-96-19

\twolineshloka
{बहुवर्षसहस्राणि तपश्चर्या मया कृता}
{नोपाश्नीयां फलं तस्या दुष्टेयं यदि मैथिली} %7-96-20

\twolineshloka
{मनसा कर्मणा वाचा भूतपूर्वं न किल्बिषम्}
{तस्याः फलमुपाश्नीयामपापा मैथिली यदि} %7-96-21

\twolineshloka
{अहं पञ्चसु भूतेषु मनष्षष्ठेषु राघव}
{विचिन्त्य सीतां शुद्धेति जग्राह वननिर्झरे} %7-96-22

\twolineshloka
{इयं शुद्धसमाचारा अपापा पतिदेवता}
{लोकापवादभीतस्य प्रत्ययं तव दास्यति} %7-96-23

\twolineshloka
{तस्मादियं नरवरात्मज शुद्धभावा दिव्येन दृष्टिविषयेण तदा प्रविष्टा}
{लोकापवादकलुषीकृतचेतसा या त्यक्ता त्वया प्रियतमा विदिताऽपि शुद्धा} %7-96-24


॥इत्यार्षे श्रीमद्रामायणे वाल्मीकीये आदिकाव्ये उत्तरकाण्डे वाल्मीकिप्रत्ययदानम् नाम षण्णवतितमः सर्गः ॥७-९६॥
