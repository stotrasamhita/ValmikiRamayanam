\sect{एकादशः सर्गः — रावणलङ्काप्राप्तिः}

\twolineshloka
{सुमाली वरलब्धांस्तु ज्ञात्वा चैतान्निशाचरान्}
{उदतिष्ठद्भयं त्यक्त्वा सानुगः स रसातलात्} %7-11-1

\twolineshloka
{मारीचश्च प्रहस्तश्च विरूपाक्षो महोदरः}
{उदतिष्ठन्सुसंरब्धाः सचिवास्तस्य रक्षसः} %7-11-2

\twolineshloka
{सुमाली सचिवैः सार्धं वृतो राक्षसपुङ्गवैः}
{अभिगम्य दशग्रीवं परिष्वज्येदमब्रवीत्} %7-11-3

\twolineshloka
{दिष्ट्या ते वत्स सम्प्राप्तश्चिन्तितोऽयं मनोरथः}
{यस्त्वं त्रिभुवनश्रेष्ठाल्लब्धवान्वरमुत्तमम्} %7-11-4

\twolineshloka
{यत्कृते च वयं लङ्कां त्यकत्वा याता रसातलम्}
{तद्गतं नो महाबाहो महद्विष्णुकृतं भयम्} %7-11-5

\twolineshloka
{असकृत्तद्भयाद्भीताः परित्यज्य स्वमालयम्}
{विद्रुताः सहिताः सर्वे प्रविष्टाः स्म रसातलम्} %7-11-6

\twolineshloka
{अस्मदीया च लङ्केयं नगरी राक्षसोचिता}
{निवेशिता तव भ्रात्रा धनाध्यक्षेण धीमता} %7-11-7

\twolineshloka
{यदि नामात्र शक्यं स्यात्साम्ना दानेन वानघ}
{तरसा वा महाबाहो प्रत्यानेतुं कृतं भवेत्} %7-11-8

\threelineshloka
{त्वं तु लङ्केश्वरस्तात भविष्यसि न संशयः}
{त्वया राक्षसवंशोऽयं निमग्नोऽपि समुद्धृतः}
{सर्वेषां नः प्रभुश्चैव भविष्यसि महाबल} %7-11-9

\twolineshloka
{अथाब्रवीद्दशग्रीवो मातामहमुपस्थितम्}
{वित्तेशो गुरुरस्माकं नार्हसे वक्तुमीदृशम्} %7-11-10

\twolineshloka
{साम्नापि राक्षसेन्द्रेण प्रत्याख्यातो गरीयसा}
{किञ्चिन्नाह तदा रक्षो ज्ञात्वा तस्य चकीर्षितम्} %7-11-11

\twolineshloka
{कस्यचित्त्वथ कालस्य वसन्तं रावणं ततः}
{उक्तवन्तं तथा वाक्यं दशग्रीवं निशाचरः} %7-11-12

\onelineshloka
{प्रहस्तः प्रश्रितं वाक्यमिदमाह स कारणम्} %7-11-13

\twolineshloka
{दशग्रीव महाबाहो नार्हस्त्वं वक्तुमीदृशम्}
{सौभ्रात्रं नास्ति शूराणां शृणु चेदं वचो मम} %7-11-14

\twolineshloka
{अदितिश्च दितिश्चैव भगिन्यौ सहिते हि ते}
{भार्ये परमरूपिण्यौ कश्यपस्य प्रजापतेः} %7-11-15

\twolineshloka
{अदितिर्जनयामास देवांस्त्रिभुवनेश्वरान्}
{दितिस्त्वजनयत्पुत्रान्कश्यपस्यात्मसम्भवान्} %7-11-16

\twolineshloka
{दैत्यानां किल धर्मज्ञ पुरीयं सवनार्णवा}
{सपर्वता मही वीर तेऽभवन्प्रभविष्णवः} %7-11-17

\twolineshloka
{निहत्य तांस्तु समरे विष्णुना प्रभविष्णुना}
{देवानां वशमानीतं त्रैलोक्यमिदमव्ययम्} %7-11-18

\twolineshloka
{नैतदेको भवानेव करिष्यति विपर्ययम्}
{सुरासुरैराचरितं तत्कुरुष्व वचो मम} %7-11-19

\twolineshloka
{एवमुक्तो दशग्रीवः प्रहृष्टेनान्तरात्मना}
{चिन्तयित्वा मुहूर्तं वै बाढमित्येव सोऽब्रवीत्} %7-11-20

\twolineshloka
{स तु तेनैव हर्षेण तस्मिन्नहनि वीर्यवान्}
{वनं गतो दशग्रीवः सह तैः क्षणदाचरैः} %7-11-21

\twolineshloka
{त्रिकूटस्थः स तु तदा दशग्रीवो निशाचरः}
{प्रेषयामास दौत्येन प्रहस्तं वाक्यकोविदम्} %7-11-22

\twolineshloka
{प्रहस्त शीघ्रं गच्छ त्वं ब्रूहि नैर्ऋतपुङ्गवम्}
{वचसा मम वित्तेशं सामपूर्वमिदं वचः} %7-11-23

\twolineshloka
{इयं लङ्का पुरी राजन्राक्षसानां महात्मनाम्}
{त्वया निवेशिता सौम्य नैतद्युक्तं तवानघ} %7-11-24

\twolineshloka
{तद्भवान्यदि नो ह्यद्य दद्यादतुलविक्रम}
{कृता भवेन्मम प्रीतिर्धुर्मश्चैवानुपालितः} %7-11-25

\twolineshloka
{स तु गत्वा पुरीं लङ्कां धनदेन सुरक्षिताम्}
{अब्रवीत्परमोदारं वित्तपालमिदं वचः} %7-11-26

\twolineshloka
{प्रेषितोऽहं तव भ्रात्रा दशग्रीवेण सुव्रत}
{त्वत्समीपं महाबाहो सर्वशस्त्रभृतां वर} %7-11-27

\twolineshloka
{तच्छ्रूयतां महाप्राज्ञ सर्वशास्त्रविशारद}
{वचनं मम वित्तेश यद्ब्रवीति दशाननः} %7-11-28

\twolineshloka
{इयं किल पुरी रम्या सुमालिप्रमुखैः पुरा}
{भुक्तपूर्वा विशालाक्ष राक्षसैर्भीमविक्रमैः} %7-11-29

\twolineshloka
{तेन विज्ञाप्यते सोऽयं साम्प्रतं विश्रवात्मज}
{तदेषा दीयतां तात याचतस्तस्य सामतः} %7-11-30

\twolineshloka
{प्रहस्तादभिसंश्रुत्य देवो वैश्रवणो वचः}
{प्रत्युवाच प्रहस्तं तं वाक्यं वाक्यविशारदः} %7-11-31

\twolineshloka
{दत्ता ममेयं पित्रा तु लङ्का शून्या निशाचरैः}
{निवासिते च मे यक्षैर्दानमानादिभिर्गुणैः} %7-11-32

\twolineshloka
{ब्रूहि गच्छ दशग्रीवं पुरं राज्यं च यन्मम}
{तवाप्येतन्महाबाहो भुङ्क्ष्व राज्यमकण्टकम्} %7-11-33

\onelineshloka
{एवमुक्त्वा धनाध्यक्षो जगाम पितुरन्तिकम्} %7-11-34

\twolineshloka
{अभिवाद्य गुरुं प्राह रावणस्य यदीप्सितम्}
{एष तात दशग्रीवो दूतं प्रेषितवान्मम} %7-11-35

\twolineshloka
{दीयतां नगरी लङ्का पूर्वं रक्षोगणोषिता}
{मयात्र यदनुष्ठेयं तन्ममाचक्ष्व सुव्रत} %7-11-36

\twolineshloka
{ब्रह्मर्षिस्त्वेवमुक्तोऽसौ विश्रवा मुनिपुङ्गवः}
{प्राञ्जलिं धनदं प्राह शृणु पुत्र वचो मम} %7-11-37

\twolineshloka
{दशग्रीवो महाबाहुरुक्तवान्मम सन्निधौ}
{मया निर्भर्त्सितश्चासीद्बहुशोक्तः सुदुर्मतिः} %7-11-38

\twolineshloka
{स क्रोधेन मया चोक्तो ध्वंसते च पुनः पुनः}
{श्रेयोऽभियुक्तं धर्म्यं च शृणु पुत्र वचो मम} %7-11-39

\twolineshloka
{वरप्रदानसम्मूढो मान्याऽमान्यान् स दुर्मतिः}
{न वेत्ति मम शापाच्च प्रकृतिं दारुणां गतः} %7-11-40

\twolineshloka
{तस्माद्गच्छ महाबाहो कैलासं धरणीधरम्}
{निवेशय निवासार्थं त्यक्त्वा लङ्कां सहानुगः} %7-11-41

\twolineshloka
{तत्र मन्दाकिनी रम्या नदीनामुत्तमा नदी}
{काञ्चनैः सूर्यसङ्काशैः पङ्कजैः संवृतोदका} %7-11-42

\onelineshloka
{कुमुदैरुत्पलैश्चैव तथाऽन्यैश्च सुगन्धिभिः} %7-11-43

\twolineshloka
{तत्र देवाः सगन्धर्वाः साप्सरोरगकिंनराः}
{विहारशीलाः सततं रमन्ते सर्वदाऽऽश्रिताः} %7-11-44

\twolineshloka
{न हि क्षमं तवानेन वैरं धनद रक्षसा}
{जानीषे हि यथाऽनेन लब्धः परमको वरः} %7-11-45

\twolineshloka
{एवमुक्तो गृहीत्वाऽऽशु तद्वचः पितृगौरवात्}
{सदारपुत्रः सामान्यः सवाहनधनो गतः} %7-11-46

\twolineshloka
{प्रहस्तोऽथ दशग्रीवं गत्वा वचनमब्रवीत्}
{प्रहृष्टात्मा महात्मानं सहामात्यं सहानुजम्} %7-11-47

\twolineshloka
{शून्या सा नगरी लङ्का त्यक्त्त्वैनां धनदो गतः}
{प्रविश्य तां सहास्माभिः स्वधर्मं प्रतिपालय} %7-11-48

\twolineshloka
{एवमुक्तो दशग्रीवः प्रहस्तेन महाबलः}
{विवेश नगरीं लङ्कां भ्रातृभिः सबलानुगैः} %7-11-49

\twolineshloka
{धनदेन परित्यक्तां सुविभक्तमहापथाम्}
{आरुरोह स देवारिः स्वर्गं देवाधिपो यथा} %7-11-50

\twolineshloka
{स चाभिषिक्तः क्षणदाचरैस्तदा निवेशयामास पुरीं दशाननः}
{निकामपूर्णा च बभूव सा पुरी निशाचरैर्नीलबलाहकोपमैः} %7-11-51

\twolineshloka
{धनेश्वरस्त्वथ पितृवाक्यगौरवान्न्यवेशयच्छशिविमले गिरौ पुरीम्}
{स्वलङ्कृतैर्भवनवरैर्विभूषितां पुरन्दरस्वेव तदाऽमरावतीम्} %7-11-52


॥इत्यार्षे श्रीमद्रामायणे वाल्मीकीये आदिकाव्ये उत्तरकाण्डे रावणलङ्काप्राप्तिः नाम एकादशः सर्गः ॥७-११॥
