\sect{षट्षष्ठितमः सर्गः — कुशलवजननम्}

\twolineshloka
{यामेव रात्रिं शत्रुघ्नः पर्णशालामुपाविशत्}
{तामेव रात्रिं सीतापि प्रसूता दारकद्वयम्} %7-66-1

\twolineshloka
{ततोऽर्धरात्रसमये बालका मुनिदारकाः}
{वाल्मीकेः प्रियमाचख्युः सीतायाः प्रसवं शुभम्} %7-66-2

\twolineshloka
{भगवन्रामपत्नी सा प्रसूता दारकद्वयम्}
{ततो रक्षां महातेजः कुरु भूतविनाशिनीम्} %7-66-3

\twolineshloka
{तेषां तद्वचनं श्रुत्वा महर्षिः समुपागमत्}
{बालचन्द्रप्रतीकाशौ देवपुत्रौ महौजसौ} %7-66-4

\twolineshloka
{जगाम तत्र हृष्टात्मा ददर्श च कुमारकौ}
{भूतघ्नीं चाकरोत्ताभ्यां रक्षां रक्षोविनाशिनीम्} %7-66-5

\twolineshloka
{कुशमुष्टिमुपादाय लवं चैव तु स द्विजः}
{वाल्मीकिः प्रददौ ताभ्यां रक्षां भूतविनाशिनीम्} %7-66-6

\twolineshloka
{यस्तयोः पूर्वजो जातः स कुशैर्मन्त्रसत्कृतैः}
{निर्मार्जनीयस्तु तदा कुश इत्यस्य नाम तत्} %7-66-7

\twolineshloka
{यश्चावरो भवेत्ताभ्यां लवेन स समाहितः}
{निर्मार्जनीयो वृद्धाभिर्लव इत्येव नामतः} %7-66-8

\twolineshloka
{एवं कुशलवौ नाम्ना तावुभौ यमजातकौ}
{मत्कृताभ्यां च नामभ्यां ख्यातियुक्तौ भविष्यतः} %7-66-9

\twolineshloka
{तां रक्षां जगृहुस्ताश्च मुनिहस्तात्समाहिताः}
{अकुर्वंश्च ततो रक्षां तयोर्विगतकल्मषाः} %7-66-10

\twolineshloka
{तथा तां क्रियमाणां च वृद्धाभिर्जन्म नाम च}
{सङ्कीर्तनं च रामस्य सीतायाः प्रसवौ शुभौ} %7-66-11

\twolineshloka
{अर्धरात्रे तु शत्रुघ्नः शुश्राव सुमहत्प्रियम्}
{पर्णशालां ततो गत्वा मातर्दिष्ट्येति चाब्रवीत्} %7-66-12

\twolineshloka
{तथा तस्य प्रहृष्टस्य शत्रुघ्नस्य महात्मनः}
{व्यतीता वार्षिकी रात्रिः श्रावणी लघुविक्रमः} %7-66-13

\twolineshloka
{प्रभाते सुमहावीर्यः कृत्वा पौर्वाह्णिकीं क्रियाम्}
{मुनिं प्राञ्जलिरामन्त्र्य ययौ पश्चान्मुखः पुनः} %7-66-14

\twolineshloka
{स गत्वा यमुनातीरं सप्तरात्रोषितः पथि}
{ऋषीणां पुण्यकीर्तीनामाश्रमे वासमभ्ययात्} %7-66-15

\twolineshloka
{स तत्र मुनिभिः सार्धं भार्गवप्रमुखैर्नृपः}
{कथाभिरभिरूपाभिर्वासं चक्रे महायशाः} %7-66-16

\twolineshloka
{स काञ्चनाद्यैर्मुनिभिः समेतो रघुप्रवीरो रजनीं तदानीम्}
{कथाप्रकारैर्बहुभिर्महात्मा विरामयामास नरेन्द्रसूनुः} %7-66-17


॥इत्यार्षे श्रीमद्रामायणे वाल्मीकीये आदिकाव्ये उत्तरकाण्डे कुशलवजननम् नाम षट्षष्ठितमः सर्गः ॥७-६६॥
