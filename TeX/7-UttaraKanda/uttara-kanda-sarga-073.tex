\sect{त्रिसप्ततितमः सर्गः — ब्राह्मणपरिदेवनम्}

\twolineshloka
{प्रस्थाप्य तु स शत्रुघ्नं भ्रातृभ्यां सह राघवः}
{प्रमुमोद सुखी राज्यं धर्मेण परिपालयन्} %7-73-1

\twolineshloka
{ततः कतिपयाहःसु वृद्धो जानपदो द्विजः}
{मृतं बालमुपादाय राजद्वारमुपागमत्} %7-73-2

\twolineshloka
{रुदन्बहुविधा वाचः स्नेहदुःखसमन्विताः}
{असकृत्पुत्र पुत्रेति वाक्यमेतदुवाच ह} %7-73-3

\twolineshloka
{किं नु मे दुष्कृतं कर्म पुरा देहान्तरे कृतम्}
{यदहं पुत्रमेकं त्वां पश्यामि निधनं गतम्} %7-73-4

\twolineshloka
{अप्राप्तयौवनं बालं पञ्चवर्षसहस्रकम्}
{अकाले कालमापन्नं मम दुःखाय पुत्रक} %7-73-5

\twolineshloka
{अल्पैरहोभिर्निधनं गमिष्यामि न संशयः}
{अहं च जननी चैव तव शोकेन पुत्रक} %7-73-6

\twolineshloka
{न स्मराम्यनृतं ह्युक्तं न च हिंसां स्मराम्यहम्}
{सर्वेषां प्राणिनां पापं कृतं नैव स्मराम्यहम्} %7-73-7

\twolineshloka
{केनाद्य दुष्कृतेनायं बाल एव ममात्मजः}
{अकृत्वा पितृकार्याणि गतो वैवस्वतक्षयम्} %7-73-8

\twolineshloka
{नेदृशं दृष्टपूर्वं मे श्रुतं वा घोरदर्शनम्}
{मृत्युरप्राप्तकालानां रामस्य विषये यथा} %7-73-9

\twolineshloka
{रामस्य दुष्कृतं किञ्चिन्महदस्ति न संशयः}
{यथा हि विषयस्थानां बालानां मृत्युरागतः} %7-73-10

\twolineshloka
{नह्यन्यविषयस्थानां बालानां मृत्युतो भयम्}
{त्वं राजञ्जीवयस्वैनं बालं मृत्युवशं गतम्} %7-73-11

\twolineshloka
{राजद्वारि मरिष्यामि पत्न्या सार्धमनाथवत्}
{ब्रह्महत्यां ततो राम समुपेत्य सुखी भव} %7-73-12

\twolineshloka
{भ्रातृभिः सहितो राजन्दीर्घमायुरवाप्स्यसि}
{उषिताः स्म सुखं राज्ये तवास्मिन्सुमहाबल} %7-73-13

\twolineshloka
{इदं तु पतितं तस्मात्तव राम वशे स्थिताः}
{कालस्य वशमापन्नाः स्वल्पं हि नहि नः सुखम्} %7-73-14

\twolineshloka
{सम्प्रत्यनाथो विषय इक्ष्वाकूणां महात्मनाम्}
{रामं नाथमिहासाद्य बालान्तकरणं ध्रुवम्} %7-73-15

\twolineshloka
{राजदोषैर्विपद्यन्ते प्रजा ह्यविधिपालिताः}
{असद्वृत्ते तु नृपतावकाले म्रियते जनः} %7-73-16

\twolineshloka
{यद्वा पुरेष्वयुक्तानि जना जनपदेषु च}
{कुर्वते न च रक्षाऽस्ति तदा कालकृतं भयम्} %7-73-17

\twolineshloka
{सुव्यक्तं राजदोषो हि भविष्यति न संशयः}
{पुरे जनपदे चापि ततो बालवधो ह्ययम्} %7-73-18

\twolineshloka
{एवं बहुविधैर्वाक्यैरुपरुध्य मुहुर्मुहुः}
{राजानं दुःखसन्तप्तः सुतं तमुपगूहते} %7-73-19


॥इत्यार्षे श्रीमद्रामायणे वाल्मीकीये आदिकाव्ये उत्तरकाण्डे ब्राह्मणपरिदेवनम् नाम त्रिसप्ततितमः सर्गः ॥७-७३॥
