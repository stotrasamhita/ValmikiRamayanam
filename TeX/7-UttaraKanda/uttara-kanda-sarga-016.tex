\sect{षोडशः सर्गः — रावणनामप्राप्तिः}

\twolineshloka
{स जित्वा धनदं राम भ्रातरं राक्षसाधिपः}
{महासेनप्रसूतिं तद्ययौ शरवणं महत्} %7-16-1

\twolineshloka
{अथापश्यदृशग्रीवो रौक्मं शरवणं महत्}
{गभस्तिजालसंवीतं द्वितीयमिव भास्करम्} %7-16-2

\twolineshloka
{स पर्वतं समारुह्य कञ्चिद्रम्यवनान्तरम्}
{अपश्यत् पुष्पकं तत्र रामविष्टम्भितं तदा} %7-16-3

\twolineshloka
{विष्टब्धं पुष्पकं दृष्ट्वा ह्यगमं कामागं कृतम्}
{अचिन्तयद्राक्षसेन्द्रः सचिवैस्तैः समावृतः} %7-16-4

\twolineshloka
{किन्निमित्तं चेच्छया मे नेदं गच्छति पुष्पकम्}
{पर्वतस्योपरिष्ठस्य कर्मेदं कस्यचिद्भवेत्} %7-16-5

\twolineshloka
{ततोऽब्रवीत्तदा राम मारीचो बुद्धिकोविदः}
{नेदं निष्कारणं राजन्पुष्पकं यन्न गच्छति} %7-16-6

\twolineshloka
{अथवा पुष्पकमिदं धनदान्नान्यवाहनम्}
{अतो निष्पन्दमभवद्धनाध्यक्षविनाकृतम्} %7-16-7

\twolineshloka
{इति वाक्यान्तरे तस्य करालः कृष्णपिङ्गलः}
{वामनो विकटो मुण्डी नन्दी प्रह्वभुजो बली} %7-16-8

\twolineshloka
{ततः पार्श्वमुपागम्य भवस्यानुचरोऽब्रवीत्}
{नन्दीश्वरो वचश्चेदं राक्षसेन्द्रमशङ्कितः} %7-16-9

\twolineshloka
{निवर्तस्व दशग्रीव शैले क्रीडति शङ्करः}
{सुपर्णनागयक्षाणां देवगन्धर्वरक्षसाम्} %7-16-10

\twolineshloka
{सर्वेषामेव भूतानामगम्यः पर्वतः कृतः}
{इति नन्दिवचः श्रुत्वा क्रोधात्कम्पितकुण्डलः} %7-16-11

\twolineshloka
{रोषात्तु ताम्रनयनः पुष्पकादवरुह्य सः}
{कोऽयं शङ्कर इत्युक्त्वा शैलमूलमुपागतः} %7-16-12

\twolineshloka
{सोऽपश्यन्नन्दिनं तत्र देवस्यादूरतः स्थितम्}
{दीप्तं शूलमवष्टभ्य द्वितीयमिव शङ्करम्} %7-16-13

\twolineshloka
{तं दृष्ट्वा वानरमुखमवज्ञाय स राक्षसः}
{प्रहासं मुमुचे तत्र सतोय इव तोयदः} %7-16-14

\twolineshloka
{तं क्रुद्धो भगवान्नन्दी शङ्करस्यापरा तनुः}
{अब्रवीत्तत्र तद्रक्षो दशाननमुपस्थितम्} %7-16-15

\twolineshloka
{यस्माद्वानररूपं मामवज्ञाय दशानन}
{अशनीपातसङ्काशमपहासं प्रमुक्तवान्} %7-16-16

\twolineshloka
{तस्मान्मद्रूपसम्पन्ना मद्वीर्यसमतेजसः}
{उत्पत्स्यन्ति वधार्थं हि कुलस्य तव वानराः} %7-16-17

\twolineshloka
{नखदंष्ट्रायुधाः क्रूरा मनःसम्पातरंहसः}
{युद्धोन्मत्ता बलोद्रिक्ताः शैला इव विसर्पिणः} %7-16-18

\twolineshloka
{ते तव प्रबलं दर्पमुत्सेधं च पृथग्विधम्}
{व्यपनेष्यन्ति सम्भूय सहामात्यसुतस्य च} %7-16-19

\twolineshloka
{किं त्विदानीं मया शक्यं हन्तुं त्वां हे निशाचर}
{न हन्तव्यो हतस्त्वं हि पूर्वमेव स्वकर्मभिः} %7-16-20

\twolineshloka
{इत्युदीरितवाक्ये तु देवे तस्मिन्महात्मनि}
{देवदुन्दुभयो नेदुः पुष्पवृष्टिश्च खाच्च्युता} %7-16-21

\twolineshloka
{अचिन्तयित्वा स तदा नन्दिवाक्यं महाबलः}
{पर्वतं तु समासाद्य वाक्यमाह दशाननः} %7-16-22

\twolineshloka
{पुष्पकस्य गतिश्छिन्ना यत्कृते मम गच्छतः}
{तमिमं शैलमुन्मूलं करोमि तव गोपते} %7-16-23

\twolineshloka
{केन प्रभावेण भवो नित्यं क्रीडति राजवत्}
{विज्ञातव्यं न जानीते भयस्थानमुपस्थितम्} %7-16-24

\twolineshloka
{एवमुक्त्वा ततो राम भुजान्विक्षिप्य पर्वते}
{तोलयामास तं शीघ्रं स शैलं समकम्पत} %7-16-25

\twolineshloka
{चालनात्पर्वतस्यैव गणा देवस्य कम्पिताः}
{चचाल पार्वती चापि तदाऽऽश्लिष्टा महेश्वरम्} %7-16-26

\twolineshloka
{ततो राम महादवो देवानां प्रवरो हरः}
{पादाङ्गुष्ठेन तं शैलं पीडयामास लीलया} %7-16-27

\twolineshloka
{पीडितास्तु ततस्तस्य शैलस्याधोगता भुजाः}
{विस्मिताश्चाभवंस्तत्र सचिवास्तस्य रक्षसः} %7-16-28

\twolineshloka
{रक्षसा तेन रोषाच्च भुजानां पीडनात्तदा}
{मुक्तो विरावः सहसा त्रैलोक्यं येन कम्पितम्} %7-16-29

\twolineshloka
{मेनिरे वज्रनिष्पेषं तस्यामात्या युगक्षये}
{तदा वर्त्मस्थचलिता देवा इन्द्रपुरोगमाः} %7-16-30

\twolineshloka
{समुद्राश्चापि सङ्क्षुब्धाश्चलिताश्चापि पर्वताः}
{यक्षा विद्याधराः सिद्धाः किमेतदिति चाब्रुवन्} %7-16-31

\twolineshloka
{अथ ते मन्त्रिणस्तस्य विक्रोशन्तमथाब्रुवन्}
{तोषयस्व महादेवं नीलकण्ठमुमापतिम्} %7-16-32

\twolineshloka
{तमृते शरणं नान्यं पश्यामोऽत्र दशानन}
{स्तुतिभिः प्रणतो भूत्वा तमेव शरणं व्रज} %7-16-33

\twolineshloka
{कृपालुः शङ्करस्तुष्टः प्रसादं ते विधास्यति}
{एवमुक्तस्तदामात्यैस्तुष्टाव वृषभध्वजम्} %7-16-34

\twolineshloka
{सामभिर्विविधैः स्तोत्रैः प्रणम्य स दशाननः}
{संवत्सरसहस्रं तु रुदतो रक्षसो गतम्} %7-16-35

\twolineshloka
{ततः प्रीतो महादेवः शैलाग्रे विष्ठितः प्रभुः}
{मुक्त्वा चास्य भुजान्राम प्राह वाक्यं दशाननम्} %7-16-36

\twolineshloka
{प्रीतोऽस्मि तव वीर्यस्य शौण्डार्याच्च दशानन}
{शैलाक्रान्तेन यो मुक्तस्त्वया रावः सुदारुणः} %7-16-37

\twolineshloka
{यस्माल्लोकत्रयं चैतद्रावितं भयमागतम्}
{तस्मात्त्वं रावणो नाम नाम्ना राजन्भविष्यसि} %7-16-38

\twolineshloka
{देवता मानुषा यक्षा ये चान्ये जगतीतले}
{एवं त्वामभिधास्यन्ति रावणं लोकरावणम्} %7-16-39

\twolineshloka
{गच्छ पौलस्त्य विस्रब्धं पथा येन त्वमिच्छसि}
{मया चैवाभ्यनुज्ञातो राक्षसाधिप गम्यताम्} %7-16-40

\twolineshloka
{एवमुक्तस्तु लङ्केशः शम्भुना स्वयमब्रवीत्}
{प्रीतो यदि महादेव वरं मे देहि याचतः} %7-16-41

\twolineshloka
{अवध्यत्वं मया प्राप्तं देवगन्धर्वदानवैः}
{राक्षसैर्गुह्यकैर्नागैर्ये चान्ये बलवत्तराः} %7-16-42

\twolineshloka
{मानुषान्न गणे देव स्वल्पास्ते मम सम्मताः}
{दीर्घमायुश्च मे प्राप्तं ब्रह्मणस्त्रिपुरान्तक} %7-16-43

\twolineshloka
{वाञ्छितं चायुषः शेषं शस्त्रं त्वं च प्रयच्छ मे}
{एवमुक्तस्ततस्तेन रावणेन स शङ्करः} %7-16-44

\twolineshloka
{ददौ खड्गं महादीप्तं चन्द्रहासमिति श्रुतम्}
{आयुषश्चावशेषं च स्थित्वा भूतपतिस्तदा} %7-16-45

\twolineshloka
{दत्त्वोवाच ततः शम्भुर्नावज्ञेयमिदं त्वया}
{अवज्ञातं यदि हि ते मामेवैष्यत्यसंशयः} %7-16-46

\twolineshloka
{एवं महेश्वरेणैव कृतनामा स रावणः}
{अभिवाद्य महादेवमारुरोहाथ पुष्पकम्} %7-16-47

\twolineshloka
{ततो महीतले राम पर्यक्रामत रावणः}
{क्षत्रियान्सुमहावीर्यान्बाधमान इतस्ततः} %7-16-48

\twolineshloka
{केचित्तेजस्विनः शूराः क्षत्रिया युद्धदुर्मदाः}
{तच्छासनमकुर्वन्तो विनेशुः सपरिच्छदाः} %7-16-49

\twolineshloka
{अपरे दुर्जयं रक्षो जानन्तः प्राज्ञसम्मताः}
{जिताः स्म इत्यभाषन्त राक्षसं बलदर्पितम्} %7-16-50


॥इत्यार्षे श्रीमद्रामायणे वाल्मीकीये आदिकाव्ये उत्तरकाण्डे रावणनामप्राप्तिः नाम षोडशः सर्गः ॥७-१६॥
