\sect{सप्तदशः सर्गः — वेदवतीशापः}

\twolineshloka
{अथ राजन्महाबाहुर्विचरन्स महीतले}
{हिमवद्वनमासाद्य परिचक्राम रावणः} %7-17-1

\twolineshloka
{तत्रापश्यत्स वै कन्यां कृष्णाजिनजटाधराम्}
{आर्षेण विधिना युक्तां दीप्यन्तीं देवतामिव} %7-17-2

\twolineshloka
{स दृष्ट्वा रूपसम्पन्नां कन्यां तां सुमहाव्रताम्}
{काममोहपरीतात्मा पप्रच्छ प्रहसन्निव} %7-17-3

\twolineshloka
{किमिदं वर्तसे भद्रे विरुद्धं यौवनस्य ते}
{नहि युक्ता तवैतस्य रूपस्यैव प्रति क्रिया} %7-17-4

\twolineshloka
{रूपं तेऽनुपमं भीरु कामोन्मादकरं नृणाम्}
{न युक्तं तपसि स्थातुं निर्गतो ह्येष निर्णयः} %7-17-5

\twolineshloka
{कस्यासि किमिदं भद्रे कश्च भर्ता वरानने}
{येन सम्भुज्यसे भीरु स नरः पुण्यभाग्भुवि} %7-17-6

\twolineshloka
{पृच्छतः शंस मे सर्वं कस्य हेतोः परिश्रमः}
{एवमुक्ता तु सा कन्या रावणेन यशस्विनी} %7-17-7

\threelineshloka
{अब्रवीद्विधिवत्कृत्वा तस्यातिथ्यं तपोधना}
{कुशध्वजो मम पिता ब्रह्मर्षिरमितप्रभः}
{बृहस्पतिसुतः श्रीमान्बुद्ध्या तुल्यो बृहस्पतेः} %7-17-8

\twolineshloka
{तस्याहं कुर्वतो नित्यं वेदाभ्यासं महात्मनः}
{सम्भूय वाङ्मयी कन्या नाम्ना वेदवती स्मृता} %7-17-9

\twolineshloka
{ततो देवाः सगन्धर्वा यक्षराक्षसपन्नगाः}
{तेऽपि गत्वा हि पितरं वरणं रोचयन्ति मे} %7-17-10

\twolineshloka
{न च मां स पिता तेभ्यो दत्तवान्राक्षसर्षभ}
{कारणं तद्वदिष्यामि निशाचर निशामय} %7-17-11

\twolineshloka
{पितुस्तु मम जामाता विष्णुः किल सुरेश्वरः}
{अभिप्रेतस्त्रिलोकेशस्तस्मान्नान्यस्य मे पिता} %7-17-12

\twolineshloka
{दातुमिच्छति तस्मै तु तच्छ्रुत्वा बलदर्पितः}
{दम्भुर्नाम ततो राजा दैत्यानां कुपितोऽभवत्} %7-17-13

\onelineshloka
{तेन रात्रौ शयानो मे पिता पापेन हिंसितः} %7-17-14

\twolineshloka
{ततो मे जननी दीना तच्छरीरं पितुर्मम}
{परिष्वज्य महाभागा प्रविष्टा हव्यवाहनम्} %7-17-15

\twolineshloka
{ततो मनोरथं सत्यं पितुर्नारायणं प्रति}
{करोमीति तमेवाहं हृदयेन समुद्वहे} %7-17-16

\twolineshloka
{इति प्रतिज्ञामारुह्य चरामि विपुलं तपः}
{एतत्ते सर्वमाख्यातं मया राक्षसपुङ्गव} %7-17-17

\threelineshloka
{नारायणो मम पतिर्न त्वन्यः पुरुषोत्तमात्}
{आश्रये नियमं घोरं नारायणपरीप्सया}
{विज्ञातस्त्वं हि मे राजन्गच्छ पौलस्त्यनन्दन} %7-17-18

\onelineshloka
{जानामि तपसा सर्वं त्रैलोक्ये यद्धि वर्तते} %7-17-19

\twolineshloka
{सोऽब्रवीद्रावणो भूयस्तां कन्यां सुमहाव्रताम्}
{अवरुह्य विमानाग्रात्कन्दर्पशरपीडितः} %7-17-20

\twolineshloka
{अवलिप्ताऽसि सुश्रोणि यस्यास्ते मतिरीदृशी}
{वृद्धानां मृगशावाक्षि भ्राजते पुण्यसञ्चयः} %7-17-21

\twolineshloka
{त्वं सर्वंगुणसम्पन्ना नार्हसे वक्तुमीदृशम्}
{त्रैलोक्यसुन्दरी भीरु यौवनं ते निवर्तते} %7-17-22

\twolineshloka
{अहं लङ्कापतिर्भद्रे दशग्रीव इति श्रुतः}
{तस्य मे भव भार्या त्वं भुङ्क्ष्व भोगान्यथासुखम्} %7-17-23

\twolineshloka
{कश्च तावदसौ यं त्वं विष्णुरित्यभिभाषसे}
{वीर्येण तपसा चैव भोगेन च बलेन च न मया स समो भद्रे यं त्वं कामयसेऽङ्गने} %7-17-24

\onelineshloka
{इत्युक्तवति तस्मिंस्तु वेदवत्यथ साऽब्रवीत्} %7-17-25

\threelineshloka
{ैव भोगेन च बलेन च मा मैवमिति सा कन्या तमुवाच निशाचरम्}
{त्रैलोक्याधिपति विष्णुं सर्वलोकनमस्कृतम्}
{त्वदृते राक्षसेन्द्रान्यः कोऽवमन्येत बुद्धिमान्} %7-17-26

\twolineshloka
{एवमुक्तस्तया तत्र वेदवत्या निशाचरः}
{मूर्धजेषु तदा कन्यां कराग्रेण परामृशत्} %7-17-27

\twolineshloka
{ततो वेदवती क्रुद्धा केशान्हस्तेन साऽच्छिनत्}
{असिर्भूत्वा करस्तस्याः केशांश्छिन्नांस्तदाऽकरोत्} %7-17-28

\twolineshloka
{सा ज्वलन्तीव रोषेण दहन्तीव निशाचरम्}
{उवाचाग्निं समाधाय मरणाय कृतत्वरा} %7-17-29

\twolineshloka
{धर्षितायास्त्वयानार्य न मे जीवितमिष्यते}
{रक्षस्तस्मात्प्रवेक्ष्यामि पश्यतस्ते हुताशनम्} %7-17-30

\twolineshloka
{यस्मात्तु धर्षिता चाहं त्वया पापात्मना वने}
{तस्मात्तव वधार्थं हि समुत्पत्स्ये ह्यहं पुनः} %7-17-31

\twolineshloka
{न हि शक्यं स्त्रिया हन्तुं पुरुषः पापनिश्चः}
{शापे त्वयि मयोत्सृष्टे तपसश्च व्ययो भवेत्} %7-17-32

\twolineshloka
{यदि त्वस्ति मया किञ्चित्कृतं दत्तं हुतं तथा}
{तस्मात्त्वयोनिजा साध्वी भवेयं धर्मिणः सुता} %7-17-33

\twolineshloka
{एवमुक्त्वा प्रविष्टा सा ज्वलितं जातवेदसम्}
{पपात च दिवो दिव्या पुष्पवृष्टिः समन्ततः} %7-17-34

\twolineshloka
{पुनरेव समुद्भूता पद्मे पद्मसमप्रभा}
{तस्मादपि पुनः प्राप्ता पूर्ववत्तेन रक्षसा} %7-17-35

\twolineshloka
{कन्यां कमलगर्भाभां प्रगृह्य स्वगृहं ययौ}
{प्रगृह्य रावणस्त्वेतां दर्शयामास मन्त्रिणे} %7-17-36

\twolineshloka
{लक्षणज्ञो निरीक्ष्यैव रावणं चैवमब्रवीत्}
{गृहस्थैषां हि सुश्रोणी त्वद्वधायैव दृश्यते} %7-17-37

\twolineshloka
{एतच्छ्रुत्वाऽर्णवे राम तां प्रचिक्षेप रावणः}
{सा चैव क्षितिमासाद्य यज्ञायतनमध्यगा} %7-17-38

\threelineshloka
{राज्ञो हलमुखोत्कृष्टा पुनरप्युत्थिता सती}
{सैषा जनकराजस्य प्रसूता तनया प्रभो}
{तव भार्या महाबाहो विष्णुस्त्वं हि सनातनः} %7-17-39

\twolineshloka
{पूर्वं क्रोधाहितः शत्रुर्ययाऽसौ निहतस्तथा}
{उपाश्रयित्वा शैलभस्तव वीर्यममानुषम्} %7-17-40

\twolineshloka
{एवमेषा महाभागा मर्त्येषूत्पत्स्यते पुनः}
{क्षेत्रे हलमुखोत्कृष्टे वेद्यामग्निशिखोपमा} %7-17-41

\twolineshloka
{एषा वेदवती नाम पूर्वमासीत्कृते युगे}
{त्रेतायुगमनुप्राप्य वधार्थं तस्य रक्षसः} %7-17-42

\twolineshloka
{उत्पन्ना मैथिलकुले जनकस्य महात्मनः}
{सीतोत्पन्ना तु सीतेति मानुषैः पुनरुच्यते} %7-17-43


॥इत्यार्षे श्रीमद्रामायणे वाल्मीकीये आदिकाव्ये उत्तरकाण्डे वेदवतीशापः नाम सप्तदशः सर्गः ॥७-१७॥
