\sect{एकोनपञ्चाशः सर्गः — वाल्मीक्याश्रमप्रवेशः}

\twolineshloka
{सीतां तु रुदतीं दृष्ट्वा ते तत्र मुनिदारकाः}
{प्राद्रवन्यत्र भगवानास्ते वाल्मीकिरुग्रधीः} %7-49-1

\twolineshloka
{अभिवाद्य मुनेः पादौ मुनिपुत्रा महर्षये}
{सर्वं निवेदयामासुस्तस्यास्तु रुदितस्वनम्} %7-49-2

\twolineshloka
{अदृष्टपूर्वा भगवन्कस्याप्येषा महात्मनः}
{पत्नी श्रीरिव सम्मोहाद्विरौति विकृतानना} %7-49-3

\twolineshloka
{भगवन्साधु पश्य त्वं देवतामिव खाच्च्युताम्}
{नद्यास्तु तीरे भगवन्वरस्त्री कापि दुःखिता} %7-49-4

\twolineshloka
{दृष्टास्माभिः प्ररुदिता दृढं शोकपरायणा}
{अनर्हा दुःखशोकाभ्यामेकां दीनां ह्यनाथवत्} %7-49-5

\twolineshloka
{न ह्येनां मानुषीं विद्मः सत्क्रियाऽस्याः प्रयुज्यताम्}
{आश्रमस्याविदूरे च त्वामियं शरणं गता} %7-49-6

\threelineshloka
{त्रातारमिच्छते साध्वी भगवंस्त्रातुमर्हसि}
{तेषां तु वचनं श्रुत्वा बुद्ध्या निश्चित्य धर्मवित्}
{तपसा लब्धचक्षुष्मान् प्राद्रवद्यत्र मैथिली} %7-49-7

\twolineshloka
{तं प्रयान्तमभिप्रेत्य शिष्या ह्येनं महामतिम्}
{तं तु देशमभि प्रेत्य किञ्चित्पद्भ्यां महामतिः} %7-49-8

\twolineshloka
{अर्घ्यमादाय रुचिरं जाह्नवीतीरमागमत्}
{ददर्श राघवस्येष्टां सीतां पत्नीमनाथवत्} %7-49-9

\twolineshloka
{तां सीतां शोकभारार्तां वाल्मीकिर्मुनिपुङ्गवः}
{उवाच मधुरां वाणीं ह्लादयन्निव तेजसा} %7-49-10

\twolineshloka
{स्नुषा दशरथस्य त्वं रामस्य महिषी प्रिया}
{जनकस्य सुता राज्ञः स्वागतं ते पतिव्रते} %7-49-11

\twolineshloka
{आयान्ती चासि विज्ञाता मया धर्मसमाधिना}
{कारणं चैव सर्वं मे हृदयेनोपलक्षितम्} %7-49-12

\twolineshloka
{तव चैव महाभागे विदितं मम तत्त्वतः}
{सर्वं च विदितं मह्यं त्रैलोक्ये यद्धि वर्तते} %7-49-13

\twolineshloka
{अपापां वेद्मि सीते त्वां तपोलब्धेन चक्षुषा}
{विस्रब्धा भव वैदेहि साम्प्रतं मयि वर्तसे} %7-49-14

\twolineshloka
{आश्रमस्याविदूरे मे तापस्यस्तपसि स्थिताः}
{तास्त्वां वत्से यथा वत्सं पालयिष्यन्ति नित्यशः} %7-49-15

\twolineshloka
{इदमर्ध्यं प्रतीच्छ त्वं विस्रब्धा विगतज्वरा}
{यथा स्वगृहमभ्येत्य विषादं चैव मा कृथाः} %7-49-16

\twolineshloka
{श्रुत्वा तु भाषितं सीता मुनेः परममद्भुतम्}
{शिरसाऽऽवन्द्य चरणौ तथेत्याह कृताञ्जलिः} %7-49-17

\onelineshloka
{तं प्रयान्तं मुनिं सीता प्राञ्जलिः पृष्ठतोऽन्वगात्} %7-49-18

\twolineshloka
{तं दृष्ट्वा मुनिमायान्तं वैदेह्या मुनिपत्नयः}
{उपाजग्मुर्मुदा युक्ता वचनं चेदमब्रुवन्} %7-49-19

\twolineshloka
{स्वागतं ते मुनिश्रेष्ठ चिरस्यागमनं च ते}
{अभिवादयामस्त्वां सर्वा उच्यतां कि च कुर्महे} %7-49-20

\twolineshloka
{तासां तद्वचनं श्रुत्वा वाल्मीकिरिदमब्रवीत्}
{सीतेयं समनुप्राप्ता पत्नी रामस्य धीमतः} %7-49-21

\twolineshloka
{स्नुषा दशरथस्यैषा जनकस्य सुता सती}
{अपापा पतिना त्यक्ता परिपाल्या मया सदा} %7-49-22

\twolineshloka
{इमां भवन्त्यः पश्यन्तु स्नेहेन परमेण हि}
{गौरवान्मम वाक्याच्च पूज्या वोऽस्तु विशेषतः} %7-49-23

\twolineshloka
{मुहुर्मुहुश्च वैदेहीं प्रणिधाय महायशाः}
{स्वमाश्रमं शिष्यवृतः पुनरायान्महातपाः} %7-49-24


॥इत्यार्षे श्रीमद्रामायणे वाल्मीकीये आदिकाव्ये उत्तरकाण्डे वाल्मीक्याश्रमप्रवेशः नाम एकोनपञ्चाशः सर्गः ॥७-४९॥
