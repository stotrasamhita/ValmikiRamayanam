\sect{चत्वारिंशः सर्गः — हनूमत्प्रार्थना}

\twolineshloka
{तथा स्म तेषां वसतामृक्षवानररक्षसाम्}
{राघवस्तु महातेजाः सुग्रीवमिदमब्रवीत्} %7-40-1

\twolineshloka
{गम्यतां सौम्य किष्किन्धां दुराधर्षां सुरासुरैः}
{पालयस्व सहामात्यो राज्यं निहतकण्टकम्} %7-40-2

\twolineshloka
{अङ्गदं च महाबाहो प्रीत्या परमया युतः}
{पश्य त्वं हनुमन्तं च नलं च सुमहाबलम्} %7-40-3

\twolineshloka
{सुषेणं श्वशुरं वीरं तारं च बलिनां वरम्}
{कुमुदं चैव दुर्धर्षं नीलं चैव महाबलम्} %7-40-4

\threelineshloka
{वीरं शतवलिं चैव मैन्दं द्विविदमेव च}
{गजं गवाक्षं गवयं शरभं च महाबलम्}
{पश्य प्रीतिसमायुक्तो गन्धमादनमेव च} %7-40-5

\twolineshloka
{ऋषभं च सुविक्रान्तं जाम्बवन्तं महाबलम्}
{ये चेमे सुमहात्मानो मदर्थे त्यक्तजीविताः} %7-40-6

\onelineshloka
{पश्य त्वं प्रीतिसंयुक्तो मा चैषां विप्रियं कृथाः} %7-40-7

\twolineshloka
{एवमुक्त्वा तु सुग्रीवमाश्लिष्य च पुनः पुनः}
{विभीषणमुवाचाथ रामो मधुरया गिरा} %7-40-8

\twolineshloka
{लङ्कां प्रशाधि धर्मेण धर्मज्ञस्त्वं मतो मम}
{पुरस्य राक्षसानां च स्वभ्रातुस्सम्मतो ह्यसि} %7-40-9

\twolineshloka
{मा च बुद्धिमधर्मे त्वं कुर्या राजन्कथञ्चन}
{बुद्धिमन्तो हि राजानो ध्रुवमश्नन्ति मेदिनीम्} %7-40-10

\twolineshloka
{अहं च नित्यशो राजन्सुग्रीवसहितस्त्वया}
{स्मर्तव्यः परया प्रीत्या गच्छ त्वं विगतज्वरः} %7-40-11

\twolineshloka
{रामस्य भाषितं श्रुत्वा ऋक्षवानरराक्षसाः}
{साधुसाध्विति काकुत्स्थं प्रशशंसुः पुनः पुनः} %7-40-12

\twolineshloka
{तव बुद्धिर्महाबाहो वीर्यमद्भुतमेव च}
{माधुर्यं परमं राम स्वयम्भोरिव नित्यदा} %7-40-13

\twolineshloka
{तेषामेवं ब्रुवाणानां वानराणां च राक्षसाम्}
{हनूमान्प्रवणः भूत्वा राघवं वाक्यमब्रवीत्} %7-40-14

\twolineshloka
{स्नेहो मे परमो राजंस्त्वयि तिष्ठतु नित्यदा}
{भक्तिश्च नियता वीर भावो नान्यत्र गच्छतु} %7-40-15

\twolineshloka
{वायद्रामकथा वीर चरिष्यति महीतले}
{तावच्छरीरे वत्स्यन्ति प्राणा मम न संशयः} %7-40-16

\twolineshloka
{यच्चैतच्चरितं दिव्यं कथा ते रघुनन्दन}
{तन्मयाप्सरसो नाम श्रावयेयुर्नरर्षभ} %7-40-17

\twolineshloka
{तच्छ्रुत्वाहं ततो वीर तव चर्यामृतं प्रभो}
{उत्कण्ठां तां हरिष्यामि मेघलेखामिवानिलः} %7-40-18

\twolineshloka
{एवं ब्रुवाणं रामस्तु हनूमन्तं वरासनात्}
{उत्थाय सस्वजे स्नेहाद्वाक्यमेतदुवाच ह} %7-40-19

\threelineshloka
{एवमेतत्कपिश्रेष्ठ भविता नात्र संशयः}
{चरिष्यति कथा यावदेषा लोके च मामिका}
{तावत्ते भविता कीर्तिः शरीरेऽप्यसवस्तथा} %7-40-20

\onelineshloka
{लोका हि यावत्स्थास्यन्ति तावत्स्थास्यति मे कथा} %7-40-21

\threelineshloka
{लोके च मामिका}
{एकैकस्योपकारस्य प्राणान्दास्यामि ते कपे}
{नरः प्रत्युपकाराणामापस्त्वायाति पात्रताम्} %7-40-22

\twolineshloka
{ततोऽस्य हारं चन्द्राभं मुच्य कण्ठात्स राघवः}
{वैडूर्यतरलं कण्ठे बबन्ध च हनूमतः} %7-40-23

\twolineshloka
{तेनोरसि निबद्धेन हारेण महता कपिः}
{रराज हेमशैलेन्द्रश्चन्द्रेणाक्रान्तमस्तकः} %7-40-24

\twolineshloka
{श्रुत्वा तु राघवस्यैतदुत्थायोत्थाय वानराः}
{प्रणम्य शिरसा पादौ निर्जग्मुस्ते महाबलाः} %7-40-25

\twolineshloka
{सुग्रीवः स च रामेण निरन्तरमुरोगतः}
{विभीषणश्च धर्मात्मा सर्वे ते बाष्पविक्लवाः} %7-40-26

\twolineshloka
{सर्वे च ते बाष्पकलाः साश्रुनेत्रा विचेतसः}
{सम्मूढा इव दुःखेन त्यजन्तो राघवं तदा} %7-40-27

\twolineshloka
{कृतप्रसादास्तेनैवं राघवेण महात्मना}
{जग्मुः स्वं स्वं गृहं सर्वे देही देहमिव त्यजन्} %7-40-28

\twolineshloka
{ततस्तु ते राक्षसऋक्षवानराः प्रणम्य रामं रघुवंशवर्धनम्}
{वियोगजाश्रुप्रतिपूर्णलोचनाः प्रतिप्रयातास्तु यथा निवासिनः} %7-40-29


॥इत्यार्षे श्रीमद्रामायणे वाल्मीकीये आदिकाव्ये उत्तरकाण्डे हनूमत्प्रार्थना नाम चत्वारिंशः सर्गः ॥७-४०॥
