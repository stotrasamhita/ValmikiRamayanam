\sect{त्रयस्त्रिंशः सर्गः — रावणविमोक्षः}

\twolineshloka
{रावणग्रहणं तत्तु वायुग्रहणसन्निभम्}
{ततः पुलस्त्यः शुश्राव कथितं दिवि दैवतैः} %7-33-1

\twolineshloka
{ततः पुत्रकृतस्नेहात्कम्प्यमानो महाधृतिः}
{माहिष्मतीपतिं द्रष्टुमाजगाम महानृषिः} %7-33-2

\twolineshloka
{स वायुमार्गमास्थाय वायुतुल्यगतिर्द्विजः}
{पुरीं माहिष्मतीं प्राप्तो मनःसम्पातविक्रमः} %7-33-3

\twolineshloka
{सोऽमरावतिसङ्काशां हृष्टपुष्टजनाकुलाम्}
{प्रविवेश पुरीं ब्रह्मा इन्द्रस्येवामरावतीम्} %7-33-4

\twolineshloka
{पादचारमिवादित्यं निष्पतन्तं सुदुर्दशम्}
{ततस्ते प्रत्यभिज्ञाय अर्जुनाय निवेदयन्} %7-33-5

\twolineshloka
{पुलस्त्य इति विज्ञाय वचनाद्धैहयाधिपः}
{शिरस्यञ्जलिमाधाय प्रत्युद्गच्छत्तपस्विनम्} %7-33-6

\twolineshloka
{पुरोहितोऽस्य गृह्यार्घ्यं मधुपर्कं तथैव च}
{पुरस्तात्प्रययौ राज्ञः शक्रस्येव बृहस्पतिः} %7-33-7

\twolineshloka
{ततस्तमृषिमायान्तमुद्यन्तमिव भास्करम्}
{अर्जुनो दृश्य सम्भ्रान्तो ववन्देन्द्र इवेश्वरम्} %7-33-8

\twolineshloka
{स तस्य मधुपर्कं गां पाद्यमर्घ्यं निवेद्य च}
{पुलस्त्यमाह राजेन्द्रो हर्षगद्गदया गिरा} %7-33-9

\twolineshloka
{अद्यैवममरावत्या तुल्या माहिष्मती कृता}
{अद्याहं तु द्विजेन्द्र त्वां यस्मात्पश्यामि दुर्दशम्} %7-33-10

\threelineshloka
{अद्य मे कुशलं देव अद्य मे कुशलं व्रतम्}
{अद्य मे सफलं जन्म अद्य मे सफलं तपः}
{यस्माद्देवगणैर्वन्द्यौ वन्देऽहं चरणौ तव} %7-33-11

\twolineshloka
{इदं राज्यमिमे पुत्रा इमे दारा इमे वयम्}
{ब्रह्मन्किं कुर्मि किं कार्यमाज्ञापयतु नो भवान्} %7-33-12

\twolineshloka
{तं धर्मेऽग्निषु पुत्रेषु शिवं पृष्ट्वा च पार्थिवम्}
{पुलस्त्योवाच राजानं हैहयानां तथाऽर्जुनम्} %7-33-13

\twolineshloka
{नरेन्द्राम्बुजपत्राक्ष पूर्णचन्द्रनिभानन}
{अतुलं ते बलं येन दशग्रीवस्त्वया जितः} %7-33-14

\twolineshloka
{भयाद्यस्योपतिष्ठेतां निष्पन्दौ सागरानिलौ}
{सोऽयं मृधे त्वया बद्धः पौत्रो मे रणदुर्जयः} %7-33-15

\twolineshloka
{पुत्रकस्य यशः पीतं नाम विश्रावितं त्वया}
{मद्वाक्याद्याच्यमानोऽद्य मुञ्च वत्सं दशाननम्} %7-33-16

\twolineshloka
{पुलस्त्याज्ञां प्रगृह्योचे न किञ्चन वचोऽर्जुनः}
{मुमोच वै पार्थिवेन्द्रो राक्षसेन्द्रं प्रहृष्टवत्} %7-33-17

\twolineshloka
{स तं प्रमुच्य त्रिदशारिमर्जुनः प्रपूज्य दिव्याभरणस्रगम्बरैः}
{अहिंसकं सख्यमुपेत्य साग्निकं प्रणम्य तं ब्रह्मसुतं गृहं ययौ} %7-33-18

\twolineshloka
{पुलस्त्येनापि सन्त्यक्तो राक्षसेन्द्रः प्रतापवान्}
{परिष्वक्तः कृतातिथ्यो लज्जमानो विनिर्जितः} %7-33-19

\twolineshloka
{पितामहसुतश्चापि पुलस्त्यो मुनिपुङ्गवः}
{मोचयित्वा दशग्रीवं ब्रह्मलोकं जगाम ह} %7-33-20

\twolineshloka
{एवं स रावणः प्राप्तः कार्तवीर्यात्प्रधर्षणम्}
{पुलस्त्यवचनाच्चापि पुनर्मुक्तो महाबलः} %7-33-21

\twolineshloka
{एवं बलिभ्यो बलिनः सन्ति राघवनन्दन}
{नावज्ञा हि परे कार्या य इच्छेत् प्रियमात्मनः} %7-33-22

\twolineshloka
{ततः स राजा पिशिताशनानां सहस्रबाहोरुपलभ्य मैत्रीम्}
{पुनर्नृपाणां कदनं चकार चचार सर्वां पृथिवीं च दर्पात्} %7-33-23


॥इत्यार्षे श्रीमद्रामायणे वाल्मीकीये आदिकाव्ये उत्तरकाण्डे रावणविमोक्षः नाम त्रयस्त्रिंशः सर्गः ॥७-३३॥
