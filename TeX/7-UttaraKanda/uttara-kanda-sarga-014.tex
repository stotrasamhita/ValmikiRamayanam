\sect{चतुर्दशः सर्गः — यक्षरक्षोयुद्धम्}

\twolineshloka
{ततस्तु सचिवैः सार्धं षड्भिर्नित्यं बलोद्धतैः}
{महोदरप्रहस्ताभ्यां मारीचशुकसारणैः} %7-14-1

\twolineshloka
{धूम्राक्षेण च वीरेण नित्यं समरगृध्नुना}
{वृतः संप्रययौ श्रीमान्क्रोधाल्लोकान्दहन्निव} %7-14-2

\twolineshloka
{पुराणि स नदीः शैलान्वनान्युपवनानि च}
{अतिक्रम्य मुहूर्तेन कैलासं गिरिमागमत्} %7-14-3

\twolineshloka
{सन्निविष्टं गिरौ तस्मिन्राक्षसेन्द्रं निशम्य तु}
{युद्धेऽत्यर्थकृतोत्साहं दुरात्मानं समन्त्रिणम्} %7-14-4

\twolineshloka
{यक्षा न शेकुः संस्थातुं प्रमुखे तस्य रक्षसः}
{राज्ञो भ्रातेति विज्ञाय गता यत्र धनेश्वरः} %7-14-5

\twolineshloka
{ते गत्वा सर्वमाचख्युर्भ्रातुस्तस्य चिकीर्षितम्}
{अनुज्ञाता ययुर्हष्टा युद्धाय धनदेन ते} %7-14-6

\twolineshloka
{ततो बलानां सङ्क्षोभो व्यवर्धत महोदधेः}
{तस्य नैर्ऋतराजस्य शैलं सञ्चालयन्निव} %7-14-7

\twolineshloka
{ततो युद्धं समभवद्यक्षराक्षससङ्कुलम्}
{व्यथिताश्चाभवंस्तत्र सचिवा राक्षसस्य ते} %7-14-8

\twolineshloka
{स दृष्ट्वा तादृशं सैन्यं दशग्रीवो निशाचरः}
{हर्षनादान्बहून्कृत्वा स क्रोधादभ्यधावत} %7-14-9

\twolineshloka
{ये तु ते राक्षसेन्द्रस्य सचिवा घोरविक्रमाः}
{तेषां सहस्रमेकैकं यक्षाणां समयोधयन्} %7-14-10

\twolineshloka
{ततो गदाभिर्मुसलैरसिभिः शक्तितोमरैः}
{हन्यमानो दशग्रीवस्तत्सैन्यं समगाहत} %7-14-11

\twolineshloka
{स निरुच्छ्वासवत्तत्र वध्यमानो दशाननः}
{वर्षद्भिरिव जीमूतैर्धाराभिरवरुध्यत} %7-14-12

\twolineshloka
{न चकार व्यथां चैव यक्षशस्त्रैः समाहतः}
{महीधर इवाम्भोदैर्धाराशतसमुक्षितः} %7-14-13

\twolineshloka
{स दुरात्मा समुद्यम्य कालदण्डोपमां गदाम्}
{प्रविवेश ततः सैन्यं नयन्यक्षान्यमक्षयम्} %7-14-14

\twolineshloka
{स कक्षमिव विस्तीर्णं शुष्केन्धनमिवाकुलम्}
{वातेनाग्निरिवायत्तो यक्षसैन्यं ददाह तत्} %7-14-15

\twolineshloka
{तैस्तु तत्र महामात्यैर्महोदरशुकादिभिः}
{अल्पावशेषास्ते यक्षाः कृता वातैरिवाम्बुदाः} %7-14-16

\twolineshloka
{केचित्समाहता भग्नाः पतिताः समरक्षितौ}
{ओष्ठांश्च दशनैस्तीक्ष्णैरदशन्कुपिता रणे} %7-14-17

\twolineshloka
{श्रान्ताश्चान्योन्यमालिङ्ग्य भ्रष्टशस्त्रा रणाजिरे}
{सीदन्ति च तदा यक्षाः कूला इव जलेन ह} %7-14-18

\twolineshloka
{हतानां गच्छतां स्वर्गं युध्यतां पृथिवीतले}
{प्रेक्षतामृषिसङ्घानां न बभूवान्तरं दिवि} %7-14-19

\twolineshloka
{भग्नांस्तु तान्समालक्ष्य यक्षेन्द्रांस्तु महाबलान्}
{धनाध्यक्षो महाबाहुः प्रेषयामास यक्षकान्} %7-14-20

\twolineshloka
{एतस्मिन्नन्तरे राम विस्तीर्णबलवाहनः}
{प्रेषितो न्यपतद्यक्षो नाम्ना संयोधकण्टकः} %7-14-21

\twolineshloka
{तेन चक्रेण मारीचो विष्णुनेव रणे हतः}
{पतितो भूतले शैलात्क्षीणपुण्य इव ग्रहः} %7-14-22

\twolineshloka
{ससञ्ज्ञस्तु मुहूर्तेन स विश्रम्य निशाचरः}
{तं यक्षं योधयामास स च भग्नः प्रदुद्रुवे} %7-14-23

\twolineshloka
{ततः काञ्चनचित्राङ्गं वैडूर्यरजसोक्षितम्}
{मर्यादां प्रतिहाराणां तोरणान्तरमाविशत्} %7-14-24

\twolineshloka
{तं तु राजन्दशग्रीवं प्रविशन्तं निशाचरम्}
{सूर्यभानुरिति ख्यातो द्वारपालो न्यवारयत्} %7-14-25

\twolineshloka
{स वार्यमाणो यक्षेण प्रविवेश निशाचरः}
{यदा तु वारितो राम न व्यतिष्ठत्स राक्षसः} %7-14-26

\twolineshloka
{ततस्तोरणमुत्पाट्य तेन यक्षेण ताडितः}
{रुधिरं प्रस्रवन्भाति शैलो धातुस्रवैरिव} %7-14-27

\twolineshloka
{स शैलशिखराभेण तोरणेन समाहतः}
{जगाम न क्षतिं वीरो वरदानात्स्वयम्भुवः} %7-14-28

\twolineshloka
{तेनैव तोरणेनाथ यक्षस्तेनाभिताडितः}
{नादृश्यत तदा यक्षो भस्मीकृततनुस्तदा} %7-14-29

\threelineshloka
{ततः प्रदुद्रुवुः सर्वे दृष्ट्वा रक्षःपराक्रमम्}
{ततो नदीर्गुहाश्चैव विविशुर्भयपीडिताः}
{त्यक्तप्रहरणाः श्रान्ता विवर्णवदनास्तदा} %7-14-30


॥इत्यार्षे श्रीमद्रामायणे वाल्मीकीये आदिकाव्ये उत्तरकाण्डे यक्षरक्षोयुद्धम् नाम चतुर्दशः सर्गः ॥७-१४॥
