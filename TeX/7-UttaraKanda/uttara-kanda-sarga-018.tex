\sect{अष्टादशः सर्गः — मरुत्तविजयः}

\twolineshloka
{प्रविष्टायां हुताशं तु वेदवत्यां स रावणः}
{पुष्पकं तु समारुह्य परिचक्राम मेदिनीम्} %7-18-1

\twolineshloka
{ततो मरुत्तं नृपतिं यजन्तं सह दैवतैः}
{उशीरबीजमासाद्य ददर्श स तु रावणः} %7-18-2

\twolineshloka
{संवर्तो नाम ब्रह्मर्षिः साक्षाद्भ्राता बृहस्पतेः}
{याजयामास धर्मज्ञः सर्वैर्देवगणैर्वृतः} %7-18-3

\twolineshloka
{दृष्ट्वा देवास्तु तद्रक्षो वरदानेन दुर्जयम्}
{तिर्यग्योनिं समाविष्टास्तस्य दर्शनभीरवः} %7-18-4

\twolineshloka
{इन्द्रो मयूरः संवृत्तो धर्मराजस्तु वायसः}
{कृकलासो धनाध्यक्षो हंसश्च वरुणोऽभवत्} %7-18-5

\twolineshloka
{अन्येष्वपि गतेष्वेवं देवेष्वरिनिषूदन}
{रावणः प्राविशद्यज्ञं सारमेय इवाशुचिः} %7-18-6

\twolineshloka
{तं च राजानमासाद्य रावणो राक्षसाधिपः}
{प्राह युद्धं प्रयच्छेति निर्जितोऽस्मीति वा वद} %7-18-7

\twolineshloka
{ततो मरुत्तो नृपतिः को भवानित्युवाच तम्}
{अपहासं ततो मुक्त्वा रावणो वाक्यमब्रवीत्} %7-18-8

\twolineshloka
{अकुतूहलभावेन प्रीतोऽस्मि तव पार्थिव}
{धनदस्यानुजं यो मां नावगच्छसि रावणम्} %7-18-9

\threelineshloka
{त्रिषु लोकेषु कोऽन्योऽस्ति यो न जानाति मे बलम्}
{भ्रातरं येन निर्जित्य विमानमिदमाहृतम्}
{ततो मरुत्तः स नृपस्तं रावणमथाब्रवीत्} %7-18-10

\twolineshloka
{धन्यः खलु भवान्येन ज्येष्ठो भ्राता रणे जितः}
{न त्वया सदृशः श्लाध्यस्त्रिषु लोकेषु विद्यते} %7-18-11

\twolineshloka
{नाधर्मसहितं श्लाध्यं तल्लोकं प्रतिसंहितम्}
{कर्म दौरात्म्यकं कृत्वा श्लाघ्यसे भ्रातृनिर्जयात्} %7-18-12

\twolineshloka
{क त्वं प्राक्केवलं धर्मं चरित्वा लब्धवान्वरम्}
{श्रूतपूर्वं हि न मया भाषसे यादृशं स्वयम्} %7-18-13

\twolineshloka
{तिष्ठेदानीं न मे जीवन्प्रतियास्यसि दुर्मते}
{अद्य त्वां निशितैर्बाणैः प्रेषयामि यमक्षयम्} %7-18-14

\twolineshloka
{ततः शरासनं गृह्य सायकांश्च नराधिपः}
{रणाय निर्ययौ क्रुद्धः संवर्तो मार्गमावृणोत्} %7-18-15

\twolineshloka
{सोऽब्रवीत्स्नेहसंयुक्तं मरुत्तं तं महानृषिः}
{श्रोतव्यं यदि मद्वाक्यं सम्प्रहारो न ते क्षमः} %7-18-16

\onelineshloka
{माहेश्वरमिदं सत्रमसमाप्तं कुलं दहेत्} %7-18-17

\twolineshloka
{दीक्षितस्य कुतो युद्धं क्रोधित्वं दीक्षिते कुतः}
{संशयश्च जये नित्यं राक्षसश्च सुदुर्जयः} %7-18-18

\twolineshloka
{स निवृत्तो गुरोर्वाक्यान्मरुत्तः पृथिवीपतिः}
{विसृज्य सशरं चापं स्वस्थो मखमुखोऽभवत्} %7-18-19

\twolineshloka
{ततस्तं निर्जितं मत्वा घोषयामास वै शुकः}
{रावणो जयतीत्युच्चैर्हर्षान्नादं विमुक्तवान्} %7-18-20

\twolineshloka
{तान्भक्षयित्वा तत्रस्थान्महर्षीन्यज्ञमागतान्}
{वितृप्तो रुधिरैस्तेषां पुनः सम्प्रययौ महीम्} %7-18-21

\twolineshloka
{रावणे तु गते देवाः सेन्द्राश्चैव दिवौकसः}
{ततः स्वां योनिमासाद्य तानि सत्त्वानि चाब्रुवन्} %7-18-22

\twolineshloka
{हर्षात्तदाब्रवीदिन्द्रो मयूरं नीलबर्हिणम्}
{प्रीतोऽस्मि तव धर्मज्ञ भुजङ्गाद्धि न ते भयम्} %7-18-23

\threelineshloka
{इदं नेत्रसहस्रं तु यत्त्वद्बर्हे भविष्यति}
{वर्षमाणे मयि मुदं प्राप्स्यसे प्रीतिलक्षणाम्}
{एवमिन्द्रो वरं प्रादान्मयूरस्य सुरेश्वरः} %7-18-24

\twolineshloka
{नीलाः किल पुरा बर्हा मयूराणां नराधिप}
{सुराधिपाद्वरं प्राप्य गताः सर्वेऽपि बर्हिणः} %7-18-25

\twolineshloka
{धर्मराजोऽब्रवीद्राम प्राग्वंशे वायसं स्थितम्}
{पक्षिंस्तवास्मि सुप्रीतः प्रीतस्य वचनं शृणु} %7-18-26

\twolineshloka
{यथान्ये विविधै रोगै पीड्यन्ते प्राणिनो मया}
{ते न ते प्रभविष्यन्ति मयि प्रीते न संशयः} %7-18-27

\twolineshloka
{मृत्युतस्ते भयं नास्ति वरान्मम विहङ्गम}
{यावत्त्वां न वधिष्यन्ति नरास्तावद्भविष्यसि} %7-18-28

\twolineshloka
{एते मद्विषयस्था वै मानवाः क्षुद्भयार्दिताः}
{त्वयि भुक्ते तु तृप्तास्ते भविष्यन्ति सबान्धवाः} %7-18-29

\twolineshloka
{वरुणस्त्वब्रवीद्धंसं गङ्गातोयविहारिणम्}
{श्रूयतां प्रीतिसंयुक्तं वचः पत्ररथेश्वर} %7-18-30

\twolineshloka
{वर्णो मनोहरः सौम्यश्चन्द्रमण्डलसन्निभः}
{भविष्यति तवोदग्रः शुद्धफेनसमप्रभः} %7-18-31

\twolineshloka
{मच्छरीरं समासाद्य कान्तो नित्यं भविष्यसि}
{प्राप्स्यसे चातुलां प्रीतिमेतन्मे प्रीतिलक्षणम्} %7-18-32

\twolineshloka
{हंसानां हि पुरा राम नीलवर्णः सपाण्डुरः}
{पक्षा नीलाग्रसंवीताः क्रोडाः शष्पाग्रनिर्मलाः} %7-18-33

\twolineshloka
{अथाब्रवीद्वैश्रवणः कृकलासं गिरौ स्थितम्}
{हैरण्यं सम्प्रयच्छामि वर्णं प्रीतस्तवाप्यहम्} %7-18-34

\twolineshloka
{सद्रव्यं च शिरो नित्यं भविष्यति तवाक्षयम्}
{एष काञ्चनको वर्णो मत्प्रीत्या ते भविष्यति} %7-18-35

\twolineshloka
{एवं दत्त्वा वरांस्तेभ्यस्तस्मिन्यज्ञोत्सवे सुराः}
{निवृत्ताः सह राज्ञा ते पुनः स्वभवनं गताः} %7-18-36


॥इत्यार्षे श्रीमद्रामायणे वाल्मीकीये आदिकाव्ये उत्तरकाण्डे मरुत्तविजयः नाम अष्टादशः सर्गः ॥७-१८॥
