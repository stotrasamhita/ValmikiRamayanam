\sect{षडशीतितमः सर्गः — ब्रह्महत्यातरणम्}

\twolineshloka
{तदा वृत्रवधं सर्वमखिलेन स लक्ष्मणः}
{कथयित्वा नरश्रेष्ठः कथाशेषं प्रचक्रमे} %7-86-1

\twolineshloka
{ततो हते महावीर्ये वृत्रे देवभयङ्करे}
{ब्रह्महत्यावृतः शक्रः सञ्ज्ञां लेभे न वृत्रहा} %7-86-2

\twolineshloka
{सोऽन्तमाश्रित्य लोकानां नष्टसञ्ज्ञो विचेतनः}
{कालं तत्रावसत्कञ्चिद्वेष्टमान इवोरगः} %7-86-3

\twolineshloka
{अथ नष्टे सहस्राक्षे उद्विग्नमभवज्जगत्}
{भूमिश्च ध्वस्तसङ्काशा निःस्नेहा शुष्ककानना} %7-86-4

\twolineshloka
{निःस्रोतसश्च ते सर्वे तु ह्रदाश्च सरितस्तथा}
{सङ्क्षोभश्चैव सत्त्वानामनावृष्टिकृतोऽभवत्} %7-86-5

\twolineshloka
{क्षीयमाणे तु लोकेऽस्मिन्सम्भ्रान्तमनसः सुराः}
{यदुक्तं विष्णुना पूर्वं तं यज्ञं समुपानयन्} %7-86-6

\twolineshloka
{ततः सर्वे सुरगणाः सोपाध्यायाः सहर्षिभिः}
{तं देशं समुपाजग्मुर्यत्रेन्द्रो भयमोहितः} %7-86-7

\twolineshloka
{ते तु दृष्ट्वा सहस्राक्षमावृतं ब्रह्महत्यया}
{तं पुरस्कृत्य देवेशमश्वमेधमुपाक्रमन्} %7-86-8

\twolineshloka
{ततोऽश्वमेधः सुमहान्महेन्द्रस्य महात्मनः}
{ववृधे ब्रह्महत्ययाः पावनार्थं नरेश्वर} %7-86-9

\twolineshloka
{ततो यज्ञे समाप्ते तु ब्रह्महत्या महात्मनः}
{अभिगम्याब्रवीद्वाक्यं क्व मे स्थानं विधास्यथ} %7-86-10

\twolineshloka
{ते तामूचुस्तदा देवास्तुष्टाः प्रीतिसमन्विताः}
{चतुर्धा विभजात्मानमात्मनैव दुरासदे} %7-86-11

\twolineshloka
{देवानां भाषितं श्रुत्वा ब्रह्महत्या महात्मनाम्}
{सन्निधौ स्थानमन्यत्र वरयामास दुर्वसा} %7-86-12

\twolineshloka
{एकेनांशेन वत्स्यामि पूर्णोदासु नदीषु वै}
{चतुरो वार्षिकान्मासान्दर्पघ्नी कामवारिणी} %7-86-13

\twolineshloka
{भूम्यामहं सर्वकालमेकेनांशेन सर्वदा}
{वसिष्यामि न सन्देहः सत्येनैतद्ब्रवीमि वः} %7-86-14

\twolineshloka
{योऽयमंशस्तृतीयो मे स्त्रीषु यौवनशालिषु}
{त्रिरात्रं दर्पपूर्णासु वशिष्ये दर्पघातिनी} %7-86-15

\twolineshloka
{हन्तारो ब्राह्मणान्ये तु मृषापूर्वमदूषकान्}
{तांश्चतुर्थेन भागेन संश्रयिष्ये सुरर्षभाः} %7-86-16

\twolineshloka
{प्रत्यूचुस्तां ततो देवा यथा वदसि दुर्वसे}
{तथा भवतु तत्सर्वं साधयस्व यदीप्सितम्} %7-86-17

\twolineshloka
{ततः प्रीत्याऽन्विता देवाः सहस्राक्षं ववन्दिरे}
{विज्वरः स च पूतात्मा वासवः समपद्यत} %7-86-18

\twolineshloka
{प्रशान्तं च जगत्सर्वं सहस्राक्षे प्रतिष्ठिते}
{यज्ञं चाद्भुतसङ्काशं तदा शक्रोऽभ्यपूजयत्} %7-86-19

\twolineshloka
{ईदृशो ह्यश्वमेधस्य प्रसादो रघुनन्दन}
{यजस्व सुमहाभाग हयमेधेन पार्थिव} %7-86-20

\twolineshloka
{इति लक्ष्मणवाक्यमुत्तमं नृपतिरतीव मनोहरं महात्मा}
{परितोषमवाप हृष्टचेता निशमय्येन्द्रसमानविक्रमौजाः} %7-86-21


॥इत्यार्षे श्रीमद्रामायणे वाल्मीकीये आदिकाव्ये उत्तरकाण्डे ब्रह्महत्यातरणम् नाम षडशीतितमः सर्गः ॥७-८६॥
