\sect{सप्तषष्ठितमः सर्गः — मान्धातृवधः}

\twolineshloka
{अथ रात्र्यां प्रवृत्तायां शत्रुघ्नो भृगुनन्दनम्}
{पप्रच्छ च्यवनं विप्रं लवणस्य यथा बलम्} %7-67-1

\twolineshloka
{शूलस्य च बलं ब्रह्मन्के च पूर्वं विनाशिताः}
{अनेन शूलमुख्येन द्वन्द्वयुद्धमुपागताः} %7-67-2

\twolineshloka
{तस्य तद्वचनं श्रुत्वा शत्रुघ्नस्य महात्मनः}
{प्रत्युवाच महातेजाश्च्यवनो रघुनन्दनम्} %7-67-3

\twolineshloka
{असङ्ख्येयानि कर्माणि यान्यस्य रघुनन्दन}
{इक्ष्वाकुवंशप्रभवे यद्वृत्तं तच्छृणष्व मे} %7-67-4

\twolineshloka
{अयोध्यायां पुरा राजा युवनाश्वसुतो बली}
{मान्धातेति स विख्यातस्त्रिषु लोकेषु वीर्यवान्} %7-67-5

\twolineshloka
{स कृत्वा पृथिवीं कृत्स्नां शासने पृथिवीपतिः}
{सुरलोकमितो जेतुमुद्योगमकरोन्नृपः} %7-67-6

\twolineshloka
{इन्द्रस्य च भयं तीव्रं सुराणां च महात्मनाम्}
{मान्धातरि कृतोद्योगे देवलोकजिगीषया} %7-67-7

\twolineshloka
{अर्धासनेन शक्रस्य राज्यार्धेन च पार्थिवः}
{वन्द्यमानः सुरगणैः प्रतिज्ञामध्यरोहत} %7-67-8

\twolineshloka
{तस्य पापमभिप्रायं विदित्वा पाकशासनः}
{सान्त्वपूर्वमिदं वाक्यमुवाच युवनाश्वजम्} %7-67-9

\twolineshloka
{राजा त्वं मानुषे लोके न तावत्पुरुषर्षभ}
{अकृत्वा पृथिवीं वश्यां देवराज्यमिहेच्छसि} %7-67-10

\twolineshloka
{यदि वीर समग्रा ते मेदिनी निखिला वशे}
{देवराज्यं कुरुष्वेह सभृत्यबलवाहनः} %7-67-11

\twolineshloka
{इन्द्रमेवं ब्रुवाणं तं मान्धाता वाक्यमब्रवीत्}
{क्व मे शक्र प्रतिहतं शासनं पृथिवीतले} %7-67-12

\twolineshloka
{तमुवाच सहस्राक्षो लवणो नाम राक्षसः}
{मधुपुत्रो मधुवने न तेऽऽज्ञां कुरुतेऽनघ} %7-67-13

\twolineshloka
{तच्छ्रुत्वा विप्रियं घोरं सहस्राक्षेण भाषितम्}
{व्रीडितोऽवाङ्मुखो राजा व्याहर्तुं न शशाक ह} %7-67-14

\twolineshloka
{आमन्त्र्य तु सहस्राक्षं ह्रिया किञ्चिदवाङ्मुखः}
{पुनरेवागमच्छ्रीमानिमं लोकं नरेश्वरः} %7-67-15

\twolineshloka
{स कृत्वा हृदयेऽमर्षं सभृत्यबलवाहनः}
{आजगाम मधोः पुत्रं वशे कर्तुमरिन्दमः} %7-67-16

\twolineshloka
{स काङ्क्षमाणो लवणं युद्धाय पुरुषर्षभः}
{दूतं सम्प्रेषयामास सकाशं लवणस्य हि} %7-67-17

\twolineshloka
{स गत्वा विप्रियाण्याह बहूनि मधुनः सुतम्}
{वदन्तमेवं तं दूतं भक्षयामास राक्षसः} %7-67-18

\twolineshloka
{चिरायमाणे दूते तु राजा क्रोधसमन्वितः}
{अर्दयामास तद्रक्षः शरवृष्ट्या समन्ततः} %7-67-19

\twolineshloka
{ततः प्रहस्य तद्रक्षः शूलं जग्राह पाणिना}
{वधाय सानुबन्धस्य मुमोचायुधमुत्तमम्} %7-67-20

\twolineshloka
{तच्छूलं दीप्यमानं तु सभृत्यबलवाहनम्}
{भस्मीकृत्वा नृपं भूयो लवणस्यागमत्करम्} %7-67-21

\twolineshloka
{एवं स राजा सुमहान्हतः सबलवाहनः}
{शूलस्य तु बलं सौम्य अप्रमेयमनुत्तमम्} %7-67-22

\twolineshloka
{श्वः प्रभाते तु लवणं वधिष्यसि न संशयः}
{अगृहीतायुधं क्षिप्रं ध्रुवो हि विजयस्तव} %7-67-23

\twolineshloka
{लोकानां स्वस्ति चैव स्यात्कृते कर्मणि च त्वया}
{एतत्ते सर्वमाख्यातं लवणस्य दुरात्मनः} %7-67-24

\twolineshloka
{शूलस्य च बलं घोरमप्रमेयं नरर्षभ}
{विनाशश्चैव मान्धातुर्यत्तेनाभूच्च पार्थिव} %7-67-25

\twolineshloka
{त्वं श्वः प्रभाते लवणं महात्मन्वधिष्यसे नात्र तु संशयो मे}
{शूलं विना निर्गतमामिषार्थे ध्रुवो जयस्ते भविता नरेन्द्र} %7-67-26


॥इत्यार्षे श्रीमद्रामायणे वाल्मीकीये आदिकाव्ये उत्तरकाण्डे मान्धातृवधः नाम सप्तषष्ठितमः सर्गः ॥७-६७॥
