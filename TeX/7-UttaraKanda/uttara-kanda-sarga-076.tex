\sect{षट्सप्ततितमः सर्गः — शम्बूकवधः}

\twolineshloka
{तस्य तद्वचनं श्रुत्वा रामस्याक्लिष्टकर्मणः}
{अवाक्छिरास्तथाभूत्वा वाक्यमेतदुवाच ह} %7-76-1

\twolineshloka
{शूद्रयोन्यां प्रसूतोऽस्मि शम्बूको नाम नामतः}
{देवत्वं पार्थये राम सशरीरो महायशः} %7-76-2

\twolineshloka
{न मिथ्याऽहं वदे राम देवलोकजिगीषया}
{शूद्रं मां विद्धि काकुत्स्थ तप उग्रं समास्थितम्} %7-76-3

\twolineshloka
{भाषतस्तस्य शूद्रस्य खड्गं सुरुचिरप्रभम्}
{निष्कृष्य कोशाद्विमलं शिरश्चिच्छेद राघवः} %7-76-4

\twolineshloka
{तस्मिञ्छूद्रे हते देवाः सेन्द्राः साग्निपुरोगमाः}
{साधु साध्विति काकुत्स्थं प्रशशंसुर्मुहुर्मुहुः} %7-76-5

\twolineshloka
{पुष्पवृष्टिर्महत्यासीद्दिव्यानां सुसुगन्धिनाम्}
{पुष्पाणां वायुमुक्तानां सर्वतः प्रपपात ह} %7-76-6

\twolineshloka
{सुप्रीताश्चाब्रुवन्रामं देवाः सत्यपराक्रमम्}
{सुरकार्यमिदं सौम्य सुकृतं ते महामते} %7-76-7

\twolineshloka
{गृहाण च वरं सौम्य यत्त्वमिच्छस्यरिन्दम}
{स्वर्गभाङ्नहि शूद्रोऽयं त्वत्कृते रघुनन्दन} %7-76-8

\twolineshloka
{देवानां भाषितं श्रुत्वा राघवः सुसमाहितः}
{उवाच प्राञ्जलिर्वाक्यं सहस्राक्षं पुरन्दरम्} %7-76-9

\twolineshloka
{यदि देवाः प्रसन्ना मे द्विजपुत्रः स जीवतु}
{दिशन्तु वरमेतं मे इप्सितं परमं मम} %7-76-10

\twolineshloka
{ममापचाराद्यातोऽसौ ब्राह्मणस्यैकपुत्रकः}
{अप्राप्तकालः कालेन नीतो वैवस्वतक्षयम्} %7-76-11

\twolineshloka
{तं जीवयथ भद्रं वो नानृतं कर्तुमर्हथ}
{द्विजस्य संश्रुतोऽर्थो मे जीवयिष्यामि ते सुतम्} %7-76-12

\twolineshloka
{राघवस्य तु तद्वाक्यं श्रुत्वा विबुधसत्तमाः}
{प्रत्युचू राघवं प्रीता देवाः प्रीतिसमन्वितम्} %7-76-13

\twolineshloka
{निर्वृतो भव काकुत्स्थ सोऽस्मिन्नहनि बालकः}
{जीवितं प्राप्तवान्भूयः समेतश्चापि बन्धुभिः} %7-76-14

\twolineshloka
{यस्मिन्मुहूर्ते काकुत्स्थ शूद्रोऽयं विनिपातितः}
{तस्मिन्मुहूर्ते बालोऽसौ जीवेन समयुज्यत} %7-76-15

\twolineshloka
{स्वस्ति प्राप्नुहि भद्रं ते साधु याम नरर्षभ}
{अगस्त्यस्याश्रमपदं द्रष्टुमिच्छाम राघव} %7-76-16

\twolineshloka
{तस्य दीक्षा समाप्ता हि ब्रह्मर्षेः सुमहाद्युतेः}
{द्वादशं हि गतं वर्षं जलशय्यां समासतः} %7-76-17

\twolineshloka
{काकुत्स्थ तद्गमिष्यामो मुनिं समभिनन्दितुम्}
{त्वं चाप्यागच्छ भद्रं ते द्रष्टुं तमृषिसत्तमम्} %7-76-18

\twolineshloka
{स तथेति प्रतिज्ञाय देवानां रघुनन्दनः}
{आरुरोह विमानं तं पुष्पकं हेमभूषितम्} %7-76-19

\twolineshloka
{ततो देवाः प्रयातास्ते विमानैर्बहुविस्तरैः}
{रामोऽप्यनुजगामाशु कुम्भयोनेस्तपोवनम्} %7-76-20

\twolineshloka
{दृष्ट्वा तु देवान्सम्प्राप्तानगस्त्यस्तपसां निधिः}
{अर्चयामास धर्मात्मा सर्वांस्तानविशेषतः} %7-76-21

\twolineshloka
{प्रतिगृह्य ततः पूजां सम्पूज्य च महामुनिम्}
{जग्मुस्ते त्रिदशा हृष्टा नाकपृष्ठं सहानुगैः} %7-76-22

\twolineshloka
{गतेषु तेषु काकुत्स्थः पुष्पकादवरुह्य च}
{ततोऽभिवादयामास ह्यगस्त्यमृषिसत्तमम्} %7-76-23

\twolineshloka
{सोऽभिवाद्य महात्मानं ज्वलन्तमिव तेजसा}
{आतिथ्यं परमं प्राप्य निषसाद नराधिपः} %7-76-24

\twolineshloka
{तमुवाच महातेजाः कुम्भयोनिर्महातपाः}
{स्वागतं ते नरश्रेष्ठ दिष्ट्या प्राप्तोऽसि राघव} %7-76-25

\twolineshloka
{त्वं मे बहुमतो राम गुणैर्बहुभिरुत्तमैः}
{अतिथिः पूजनीयश्च मम नित्यं हृदि स्थितः} %7-76-26

\twolineshloka
{सुरा हि कथयन्ति त्वामागतं शूद्रघातिनम्}
{ब्राह्मणस्य तु धर्मेण त्वया जीवापितः सुतः} %7-76-27

\twolineshloka
{उष्यतां चेह रजनी सकाशे मम राघव}
{प्रभाते पुष्पकेण त्वं गन्ताऽसि पुरमेव हि} %7-76-28

\twolineshloka
{त्वं हि नारायणः श्रीमांस्त्वयि सर्वं प्रतिष्ठितम्}
{त्वं प्रभुः सर्वदेवानां पुरुषस्त्वं सनातनः} %7-76-29

\threelineshloka
{इदं चाभरणं सौम्य निर्मितं विश्वकर्मणा}
{दिव्यं दिव्येन वपुषा दीप्यमानं स्वतेजसा}
{प्रतिगृह्णीष्व काकुत्स्थ मत्प्रियं कुरु राघव} %7-76-30

\twolineshloka
{दत्तस्य हि पुनर्दाने सुमहत्फलमुच्यते}
{भरणे हि भवान् शक्तः सेन्द्राणां मरुतामपि} %7-76-31

\twolineshloka
{त्वं हि शक्तस्तारयितुं सेन्द्रानपि दिवौकसः}
{तस्मात्प्रदास्ये विधिवत्तत्प्रतीच्छ नराधिप} %7-76-32

\onelineshloka
{दिव्यमाभरणं चित्रं प्रदीप्तमिव भास्करम्} %7-76-33

\twolineshloka
{अथोवाच महात्मानमिक्ष्वाकूणां महारथः}
{रामो मतिमतां श्रेष्ठः क्षत्रधर्ममनुस्मरन्} %7-76-34

\twolineshloka
{प्रतिग्रहोऽयं भगवन्ब्राह्मणस्याविगर्हितः}
{गृह्णीयां क्षत्रियोऽहं वै कथं ब्राह्मणपुङ्गव} %7-76-35

\twolineshloka
{ब्राह्मणेन विशेषेण दत्तं तद्वक्तुमर्हसि}
{एवमुक्तस्तु रामेण प्रत्युवाच महानृषिः} %7-76-36

\twolineshloka
{आसन्कृतयुगे राम ब्रह्मभूते पुरायुगे}
{अपार्थिवाः प्रजाः सर्वाः सुराणां तु शतक्रतुः} %7-76-37

\twolineshloka
{ताः प्रजा देवदेवेशं राजार्थं समुपाद्रवन्}
{सुराणां स्थापितो राजा त्वया देव शतक्रतुः} %7-76-38

\threelineshloka
{प्रयच्छ नो हि लोकेश पार्थिवं नरपुङ्गवम्}
{यस्मै पूजां प्रयुञ्जाना धूतपापाश्चरेमहि}
{न वसामो विना राज्ञा एष नो निश्चयः परः} %7-76-39

\onelineshloka
{प्रजानां वचनं श्रुत्वा निश्चयित्वाऽर्थमुत्तमम्} %7-76-40

\twolineshloka
{ततो ब्रह्मा सुरश्रेष्ठो लोकपालान्सवासवान्}
{समाहूयाब्रवीत्सर्वांस्तेजोभागान्प्रयच्छत} %7-76-41

\twolineshloka
{ततो ददुर्लोकपालाः सर्वे भागान्स्वतेजसः}
{अक्षुपच्च ततो ब्रह्मा यतो जातः क्षुपो नृपः} %7-76-42

\twolineshloka
{तं ब्रह्मा लोकपालानां सहांशैः समयोजयत्}
{ततो ददौ नृपं तासां प्रजानामीश्वरं क्षुपम्} %7-76-43

\twolineshloka
{तत्रैन्द्रेण च भागेन महीमाज्ञापयन्नृपः}
{वारुणेन तु भागेन वपुः पुष्यति पार्थिवः} %7-76-44

\twolineshloka
{कौबेरेण तु भागेन वित्तमासां ददौ तदा}
{यस्तु याम्योऽभवद्भागस्तेन शास्ति स्म स प्रजाः} %7-76-45

\twolineshloka
{तत्रैन्द्रेण नरश्रेष्ठ भागेन रघुनन्दन}
{प्रतिगृह्णीष्व भद्रं ते तारणार्थं मम प्रभो} %7-76-46

\twolineshloka
{तस्य तद्वचनं श्रुत्वा ऋषेः परमधार्मिकम्}
{तद्रामः प्रतिजग्राह मुनेराभरणं वरम्} %7-76-47

\twolineshloka
{प्रतिगृह्य ततो रामस्तदाभरणमुत्तमम्}
{आगमं तस्य दीप्तस्य प्रष्टुमेवोपचक्रमे} %7-76-48

\twolineshloka
{अत्यद्भुतमिदं दिव्यं वपुषा युक्तमुत्तमम्}
{कथं वा भगवता प्राप्तं कुतो वा केन वा हृतम्} %7-76-49

\twolineshloka
{कुतूहलितया ब्रह्मन्पृच्छामि त्वां महायशः}
{आश्चर्याणां बहूनां हि निधिः परमको भवान्} %7-76-50

\twolineshloka
{एवं ब्रुवति काकुत्स्थे मुनिर्वाक्यमथाब्रवीत्}
{शृणु राम यथावृत्तं पुरा त्रेतायुगे युगे} %7-76-51

\twolineshloka
{रमणीयप्रदेशेऽस्मिन् वने यद्दृष्टवानहम्}
{आश्चर्यं मे महाबाहो दानमाश्रित्य केवलम्} %7-76-52


॥इत्यार्षे श्रीमद्रामायणे वाल्मीकीये आदिकाव्ये उत्तरकाण्डे शम्बूकवधः नाम षट्सप्ततितमः सर्गः ॥७-७६॥
