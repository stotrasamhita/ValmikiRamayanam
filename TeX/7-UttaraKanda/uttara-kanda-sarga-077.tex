\sect{सप्तसप्ततितमः सर्गः — स्वर्गिप्रश्नः}

\twolineshloka
{पुरा त्रेतायुगे राम बभूव बहुविस्तरम्}
{समन्ताद्योजनशतं विमृगं पक्षिवर्जितम्} %7-77-1

\twolineshloka
{तस्मिन्निर्मानुषेऽरण्ये कुर्वाणस्तप उत्तमम्}
{अहमाक्रमितुं सौम्य तदरण्यमुपागमम्} %7-77-2

\twolineshloka
{तस्य रूपमरण्यस्य निर्देष्टुं न शशाक ह}
{फलमूलैः सुखास्वादैर्बहुरूपैश्च पादपैः} %7-77-3

\twolineshloka
{तस्यारण्यस्य मध्ये तु सरो योजनमायतम्}
{हंसकारण्डवाकीर्णं चक्रवाकोपशोभितम्} %7-77-4

\twolineshloka
{पद्मोत्पलसमाकीर्णं समतिक्रान्तशैवलम्}
{तदाश्चर्यमिवात्यर्थं सुखास्वादमनुत्तमम्} %7-77-5

\twolineshloka
{अरजस्कं तथाऽक्षोभ्यं श्रीमत्पक्षिगणायुतम्}
{समीपे तस्य सरसो महदद्भुतमाश्रमम्} %7-77-6

\twolineshloka
{पुराणं पुण्यमत्यर्थं तपस्विजनवर्जितम्}
{तत्राहमवसं रात्रिं नैदाघीं पुरुषर्षभ} %7-77-7

\twolineshloka
{प्रभाते काल्यमुत्थाय सरस्तदुपचक्रमे}
{अथापश्यं शवं तत्र सुपुष्टमजरं क्वचित्} %7-77-8

\twolineshloka
{पङ्किभेदेन पुष्टाङ्गं समाश्रितसरोवरम्}
{तिष्ठन्तं परया लक्ष्म्या तस्मिंस्तोयाशये नृप} %7-77-9

\twolineshloka
{तमर्थं चिन्तयानोऽहं मुहूर्तं तत्र राघव}
{उषितोऽस्मि सरस्तीरे किन्न्विदं स्यादिति प्रभो} %7-77-10

\twolineshloka
{अथापश्यं मुहूर्तेन दिव्यमद्भुतदर्शनम्}
{विमानं परमोदारं हंसयुक्तं मनोजवम्} %7-77-11

\twolineshloka
{अत्यर्थं स्वर्गिणं त विमाने रघुनन्दन}
{उपास्तेऽप्सरसां वीर सहस्रं दिव्यभूषणम्} %7-77-12

\twolineshloka
{गायन्ति दिव्यगेयानि वादयन्ति तथाऽपराः}
{क्ष्वेलयन्ति तथा चान्या नृत्यन्ति च तथाऽपराः} %7-77-13

\twolineshloka
{अपराश्चन्द्ररश्म्याभैर्हेमदण्डैश्च चामरैः}
{दोधूयुर्वदनं तस्य पुण्डरीकनिभेक्षणम्} %7-77-14

\threelineshloka
{ततः सिंहासनं त्यक्त्वा मेरुकूटमिवांशुमान्}
{पश्यतो मे तदा राम विमानादवरुह्य च}
{तं शवं भक्षयामास स स्वर्गी रघुनन्दन} %7-77-15

\twolineshloka
{तथा भुक्त्वा यथाकामं मांसं बहु सुपीवरम्}
{अवतीर्य सरः स्वर्गी संस्प्रष्टुमुपचक्रमे} %7-77-16

\twolineshloka
{उपस्पृश्य यथान्यायं स स्वर्गी रघुपुङ्गव}
{आरोढुमुपचक्राम विमानवरमुत्तमम्} %7-77-17

\twolineshloka
{तमहं देवसङ्काशमारोहन्तमुदीक्ष्य वै}
{अथाहमब्रुवं वाक्यं स्वर्गिणं पुरुषर्षभ} %7-77-18

\twolineshloka
{को भवान्देवसङ्काश आहारश्च विगर्हितः}
{त्वयेदं भुज्यते सौम्य किमर्थं वक्तुमर्हसि} %7-77-19

\twolineshloka
{कस्य स्यादीदृशो भाव आहारो देवसम्मतः}
{आश्चर्यं वर्तते सौम्य श्रोतुमिच्छामि तत्त्वतः} %7-77-20

\onelineshloka
{नाहमौपयिकं मन्ये तव भक्ष्यमिदं शवम्} %7-77-21

\twolineshloka
{इत्येवमुक्तः स नरेन्द्र नाकी कौतूहलात्सूनृतया गिरा च}
{श्रुत्वा च वाक्यं मम सर्वमेतत्सर्वं तथा चाकथयन्ममेति} %7-77-22


॥इत्यार्षे श्रीमद्रामायणे वाल्मीकीये आदिकाव्ये उत्तरकाण्डे स्वर्गिप्रश्नः नाम सप्तसप्ततितमः सर्गः ॥७-७७॥
