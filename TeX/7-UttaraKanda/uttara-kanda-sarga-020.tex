\sect{विंशः सर्गः — रावणसन्धुक्षणम्}

\twolineshloka
{ततो वित्रासयन्मर्त्यान्पृथिव्यां राक्षसाधिपः}
{आससाद घने तस्मिन्नारदं मुनिपुङ्गवम्} %7-20-1

\twolineshloka
{तस्याभिवादनं कृत्वा दशग्रीवो निशाचरः}
{अब्रवीत्कुशलं पृष्ट्वा हेतुमागमनस्य च} %7-20-2

\twolineshloka
{नारदस्तु महातेजा देवर्षिरमितप्रभः}
{अब्रवीन्मेघपृष्ठस्थो रावणं पुष्पके स्थितम्} %7-20-3

\twolineshloka
{राक्षसाधिपते सौम्य तिष्ठ विश्रवसः सुत}
{प्रीतोऽस्म्यभिजनोपेतविक्रमैरूर्जितैस्तव} %7-20-4

\twolineshloka
{विष्णुना दैत्यघातैश्च गन्धर्वोरगधर्षणैः}
{त्वया समं विमर्दैश्च भृशं हि परितोषितः} %7-20-5

\threelineshloka
{किञ्चिद्वक्ष्यामि तावत्ते श्रोतव्यं श्रोष्यसे यदि}
{श्रुत्वा चानन्तरं कार्यं त्वया राक्षससत्तम}
{तन्मे निगदतस्तात समाधिं श्रवणे कुरु} %7-20-6

\twolineshloka
{किमयं वध्यते तात त्वयाऽवध्येन दैवतैः}
{हत एव ह्ययं लोको यदा मृत्युवशं गतः} %7-20-7

\threelineshloka
{्तम}
{देवदानवदैत्यानां यक्षगन्धर्वरक्षसाम्}
{अवध्येन त्वया लोकः क्लेष्टुं युक्तो न मानुषः} %7-20-8

\twolineshloka
{नित्यं श्रेयसि सम्मूढं महद्भिर्व्यसनैर्वृतम्}
{हन्यात्कस्तादृशं लोकं जराव्याधिशतैर्युतम्} %7-20-9

\twolineshloka
{तैस्तैरनिष्टोपगमैरजस्रं यत्र कुत्र कः}
{मतिमान्मानुषे लोके युद्धेन प्रणयी भवेत्} %7-20-10

\twolineshloka
{क्षीयमाणं दैवहतं श्रुत्पिपासाजरादिभिः}
{विषादशोकसम्मूढं लोकं त्वं क्षपयस्व मा} %7-20-11

\twolineshloka
{पश्य तावन्महाबाहो राक्षसेश्वर मानुषम्}
{मूढमेवं विचित्रार्थं यस्य न ज्ञायते गतिः} %7-20-12

\twolineshloka
{क्वचिद्वादित्रनृत्यादि सेव्यते मुदितैर्जनैः}
{रुद्यते चापरैरार्तैर्धाराश्रुनयनाननैः} %7-20-13

\twolineshloka
{मातापितृसुतस्नेहैर्भार्याबन्धुमनोरमैः}
{मोहितोऽयं जनो ध्वस्तः क्लेशं स्वं नावबुध्यते} %7-20-14

\twolineshloka
{अलमेनं परिक्लिश्य लोकं मोहनिराकृतम्}
{जित एव त्वया सौम्य मर्त्यलोको न संशयः} %7-20-15

\twolineshloka
{अवश्यमेभिः सर्वैश्च गन्तव्यं यमसादनम्}
{तन्निगृह्णीष्व पौलस्त्य यमं परपुरञ्जय} %7-20-16

\threelineshloka
{तस्मिञ्जिते जितं सर्वं भवत्येव न संशयः}
{एवमुक्तस्तु लङ्केशो दीप्यमानं स्वतेजसा}
{अब्रवीन्नारदं तत्र सम्प्रहस्याभिवाद्य च} %7-20-17

\twolineshloka
{महर्षे देवगन्धर्वविहार समरप्रिय}
{अहं समुद्यतो गन्तुं विजयार्थं रसातलम्} %7-20-18

\twolineshloka
{ततो लोकत्रयं जित्वा स्थाप्य नागान्सुरान्वशे}
{समुद्रममृतार्थं च मथिष्यामि रसालयम्} %7-20-19

\twolineshloka
{अथाब्रवीदृशग्रीवं नारदो भगवानृषिः}
{क्व खल्विदानीं मार्गेण त्वया ह्यन्येन गम्यते} %7-20-20

\twolineshloka
{अयं खलु सुदुर्गम्यः प्रेतराजपुरं प्रति}
{मार्गो गच्छति दुर्धर्षो यमस्यामित्रकर्शन} %7-20-21

\twolineshloka
{स तु शारदमेघाभं हासं मुक्त्वा दशाननः}
{उवाच कृतमित्येव वचनं चेदमब्रवीत्} %7-20-22

\twolineshloka
{तस्मादेवमहं ब्रह्मन् वैवस्वतवधोद्यतः}
{गच्छामि दक्षिणामाशां यत्र सूर्यात्मजो नृपः} %7-20-23

\twolineshloka
{मया हि भगवन्क्रोधात्प्रतिज्ञातं रणार्थिना}
{अवजेष्यामि चतुरो लोकपालानिति प्रभो} %7-20-24

\twolineshloka
{तदिह प्रस्थितोऽहं वै प्रेतराजपुरं प्रति}
{प्राणिसङ्क्लेशकर्तारं योजयिष्यामि मृत्युना} %7-20-25

\twolineshloka
{एवमुक्त्वा दशग्रीवो मुनिं तमभिवाद्य च}
{प्रययौ दक्षिणामाशां प्रविष्टः सह मन्त्रिभिः} %7-20-26

\twolineshloka
{नारदस्तु महातेजा मुहूर्तं ध्यानमास्थितः}
{चिन्तयामास विप्रेन्द्रो विधूम इव पावकः} %7-20-27

\twolineshloka
{येन लोकास्त्रयः सेन्द्राः क्लिश्यन्ते सचराचराः}
{क्षीणे चायुषि धर्मेण स कालो जेष्यते कथम्} %7-20-28

\twolineshloka
{स्वदत्तकृतसाक्षी यो द्वितीय इव पावकः}
{लब्धसंज्ञा विजेष्यन्ते लोका यस्य महात्मनः} %7-20-29

\twolineshloka
{यस्य नित्यं त्रयो लोका विद्रवन्ति भयार्दिताः}
{तं कथं राक्षसेन्द्रोऽसौ स्वयमेव गमिष्यति} %7-20-30

\twolineshloka
{यो विधाता च धाता च सुकृतं दुष्कृतं तथा}
{त्रैलोक्यं विजितं येन तं कथं विजयिष्यते} %7-20-31

\twolineshloka
{अपरं किन्तु कृत्वायं विधानं संविधास्यति}
{कौतूहलसमुत्पन्नो यास्यामि यमसादनम्} %7-20-32

\onelineshloka
{विमर्दं द्रष्टुमनयोर्यमराक्षसयोः स्वयम्} %7-20-33


॥इत्यार्षे श्रीमद्रामायणे वाल्मीकीये आदिकाव्ये उत्तरकाण्डे रावणसन्धुक्षणम् नाम विंशः सर्गः ॥७-२०॥
