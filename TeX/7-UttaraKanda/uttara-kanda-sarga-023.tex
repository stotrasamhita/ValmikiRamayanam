\sect{त्रयोविंशः सर्गः — वरुणजयः
बलिदर्शनम्
सूर्यजयघोषणा
रावणमान्धातृयुद्धम्
रावणमन्त्रेश्वरदानम्
कपिलदर्शनम्}

\twolineshloka
{ततो जित्वा दशग्रीवो यमं त्रिदशपुङ्गवम्}
{रावणस्तु रणश्लाघी स्वसहायान्ददर्श ह} %7-23-1

\twolineshloka
{ततो रुधिरसिक्ताङ्गं प्रहारैर्जर्जरीकृतम्}
{रावणं राक्षसा दृष्ट्वा ह्यष्टवत् समुपागमन्} %7-23-2

\twolineshloka
{जयेन वर्धयित्वा च मारीचप्रमुखास्ततः}
{पुष्पकं भेजिरे सर्वे सान्त्विता रावणेन तु} %7-23-3

\twolineshloka
{ततो रसातलं गच्छन् प्रविष्टः पयसां निधिम्}
{दैत्योरगगणाध्युष्टं वरुणेन सुरक्षितम्} %7-23-4

\twolineshloka
{स तु भोगवतीं गत्वा पुरीं वासुकिपालिताम्}
{कृत्वा नागान्वशे हृष्टो ययौ मणिमयीं पुरीम्} %7-23-5

\twolineshloka
{निवातकवचास्तत्र दैत्या लब्धवरा वसन्}
{राक्षसान्तान्समागम्य युद्धाय समुपाह्वयत्} %7-23-6

\twolineshloka
{ते तु सर्वे सुविक्रान्ता दैतेया बलशालिनः}
{नानाप्रहरणास्तत्र प्रहृष्टा युद्धदुर्मदाः} %7-23-7

\twolineshloka
{शूलैस्त्रिशूलैः कुलिशैः पट्टिशासिपरश्वधैः}
{अन्योन्यं बिभिदुः क्रुद्धा राक्षसा दानवास्तथा} %7-23-8

\twolineshloka
{तेषां तु युध्यमानानां साग्रः संवत्सरो गतः}
{न चान्यतरयोस्तत्र विजयो वा क्षयोऽपि वा} %7-23-9

\twolineshloka
{ततः पितामहस्तत्र त्रैलोक्यगतिरव्ययः}
{आजगाम द्रुतं देवो विमानवरमास्थितः} %7-23-10

\twolineshloka
{निवातकवचानां तु निवार्य रणकर्म तत्}
{वृद्धः पितामहो वाक्यमुवाच विदितार्थवत्} %7-23-11

\twolineshloka
{नह्ययं रावणो युद्धे शक्यो जेतुं सुरासुरैः}
{न भवन्तः क्षयं नेतुमपि सामरदानवैः} %7-23-12

\twolineshloka
{राक्षसस्य सखित्वं च भवद्भिः सह रोचते}
{अविभक्ताश्च सर्वार्थाः सुहृदां नात्र संशयः} %7-23-13

\twolineshloka
{ततोऽग्निसाक्षिकं सख्यं कृतवांस्तत्र रावणः}
{निवातकवचैः सार्धं प्रीतिमानभवत्तदा} %7-23-14

\twolineshloka
{अर्थतस्तैर्यथान्यायं संवत्सरमथोषितः}
{स्वपुरान्निर्विशेषं च प्रियं प्राप्तो दशाननः} %7-23-15

\twolineshloka
{ततोपधार्य मायानां शतमेकं समाप्तवान्}
{सलिलेन्द्रपुरान्वेषी भ्रमति स्म रसातलम्} %7-23-16

\twolineshloka
{ततोऽश्मनगरं नाम कालकेयैरधिष्ठितम्}
{गत्वा तु कालकेयांश्च हत्वा तत्र बलोत्कटान्} %7-23-17

\threelineshloka
{शूर्पणख्याश्च भर्तारमसिना प्राच्छिनत्तदा}
{श्यालं च बलवन्तं च विद्युज्जिह्वं बलोत्कटम्}
{जिह्वया संलिहन्तं च राक्षसं समरे तथा} %7-23-18

\onelineshloka
{तं विजित्य मुहूर्तेन जघ्ने दैत्यांश्चतुःशतम्} %7-23-19

\twolineshloka
{ततः पाण्डुरमेघाभं कैलासमिव भास्वरम्}
{वरुणस्यालयं दिव्यमपश्यद्राक्षसाधिपः} %7-23-20

\twolineshloka
{क्षरन्तीं च पयस्तत्र सुरभिं गामवस्थिताम्}
{यस्याः पयोभिनिष्यन्दात्क्षीरोदो नाम सागरः} %7-23-21

\twolineshloka
{ददर्श रावणस्तत्र गोवृषेन्द्रवरारणिम्}
{यस्माच्चन्द्रः प्रभवति शीतरश्मिर्निशाकरः} %7-23-22

\twolineshloka
{यं समाश्रित्य जीवन्ति फेनपाः परमर्षयः}
{अमृतं यत्र चोत्पन्नं स्वधा च स्वधभोजिनाम्} %7-23-23

\twolineshloka
{यां ब्रुवन्ति नरा लोके सुरभिं नाम नामतः}
{प्रदक्षिणं तु तां कृत्वा रावणः परमाद्भुताम्} %7-23-24

\threelineshloka
{प्रविवेश महाघोरं गुप्तं बहुविधैर्बलैः}
{ततो धाराशताकीर्णं शारदाभ्रनिभं तदा}
{नित्यप्रहृष्टं ददृशे वरुणस्य गृहोत्तमम्} %7-23-25

\twolineshloka
{ततो हत्वा बलाध्यक्षान्समरे तैश्च ताडितः}
{अब्रवीच्च ततो योधान्राजा शीघ्रं निवेद्यताम्} %7-23-26

\twolineshloka
{युद्धार्थी रावणः प्राप्तस्तस्य युद्धं प्रदीयताम्}
{वद वा न भयं तेऽस्ति निर्जितोऽस्मीति साञ्जलिः} %7-23-27

\twolineshloka
{एतस्मिन्नन्तरे क्रुद्धा वरुणस्य महात्मनः}
{पुत्राः पौत्राश्च निष्क्रामन्गौश्च पुष्कर एव च} %7-23-28

\twolineshloka
{ते तु वीर्यगुणोपेता बलैः परिवृताः स्वकैः}
{युङ्क्त्वा रथान्कामगमानुद्यद्भास्वरवर्चसः} %7-23-29

\twolineshloka
{ततो युद्धं समभवद्दारुणं रोमहर्षणम्}
{सलिलेन्द्रस्य पुत्राणां रावणस्य च धीमतः} %7-23-30

\twolineshloka
{अमात्यैश्च महावीर्यैर्दशग्रीवस्य रक्षसः}
{वारुणं तद्बलं कृत्स्नं क्षणेन विनिपातितम्} %7-23-31

\twolineshloka
{समीक्ष्य स्वबलं सङ्ख्ये वरुणस्य सुतास्तदा}
{अर्दिताः शरजालेन निवृत्ता रणकर्मणः} %7-23-32

\twolineshloka
{महीतलगतास्ते तु रावणं दृश्य पुष्पके}
{आकाशमाशु विविशुः स्यन्दनैः शीघ्रगामिभिः} %7-23-33

\twolineshloka
{महादासीत्ततस्तेषां तुल्यं स्थानमवाप्य तत्}
{आकाशयुद्धं तुमुलं देवदानवयोरिव} %7-23-34

\twolineshloka
{ततस्ते रावणं युद्धे शरैः पावकसन्निभैः}
{विमुखीकृत्य सन्तुष्टा विनेदुर्विविधान्रवान्} %7-23-35

\twolineshloka
{ततो महोदरः क्रुद्धो राजानं दृश्य धर्षितम्}
{त्यक्त्वा मृत्युभयं वीरो युद्धकाङ्क्षी व्यलोकयत्} %7-23-36

\twolineshloka
{तेन ते दारुणा युद्धे कामगाः पवनोपमाः}
{महोदरेण गदया हता वै प्रययुः क्षितिम्} %7-23-37

\twolineshloka
{तेषां वरुणपुत्राणां हत्वा योधान्हयाञ्छतान्}
{मुमोचाशु महानादं विरथान्प्रेक्ष्य तान्स्थितान्} %7-23-38

\twolineshloka
{ते तु तेषां रथाः साश्वाः सह सारथिभिर्हतैः}
{महोदरेण निहताः पतिताः पृथिवीतले} %7-23-39

\twolineshloka
{ते तु त्यक्त्वा रथान्पुत्रा वरुणस्य महात्मनः}
{आकाशे विष्ठिताः शूराः स्वप्रभावान्न विव्यथुः} %7-23-40

\twolineshloka
{धनूंषि कृत्वा सज्जानि विनिर्भिद्य महोदरम्}
{रावणं समरे क्रुद्धाः सहिताः समभिद्रवन्} %7-23-41

\twolineshloka
{सायकैश्चापविभ्रष्टैर्वज्रकल्पैः सुदारुणैः}
{दारयन्ति स्म सङ्क्रुद्धा मेघा इव महागिरिम्} %7-23-42

\twolineshloka
{ततः क्रुद्धो दशग्रीवः कालाग्निरिव निर्गतः}
{शरवर्षैर्महाघोरैस्तेषां मर्मस्वताडयत्} %7-23-43

\twolineshloka
{ततस्तेनैव सहसा सीदन्ति स्म पदातयः}
{मुसलानि विचित्राणि ततो भल्लशतानि च} %7-23-44

\twolineshloka
{पट्टिशांश्चैव शक्तीश्च शतघ्नीस्तोमरांस्तथा}
{पातयामास दुर्धर्षस्तेषामुपरि विष्ठितः} %7-23-45

\twolineshloka
{अपविद्धास्तु ते वीरा विनिष्पेतुः पदातयः}
{महापङ्कमिवासाद्य कुञ्जराष्षष्टिहायनाः} %7-23-46

\twolineshloka
{सीदमानान्सुतान्दृष्ट्वा विह्वलान्सुमहौजसः}
{ननाद रावणो हर्षान्महानम्बुधरो यथा} %7-23-47

\twolineshloka
{ततो रक्षो महानादान्मुक्त्वा हन्ति स्म वारुणान्}
{नानाप्रहरणोपेतैर्धारापातैरिवाम्बुदः} %7-23-48

\twolineshloka
{ततस्ते विमुखाः सर्वे पतिता धरणीतले}
{रणात्स्वपुरुषैः शीघ्रं गृहाण्येव प्रवेशिताः} %7-23-49

\onelineshloka
{तानब्रवीत्ततो रक्षो वरुणाय निवेद्यताम्} %7-23-50

\twolineshloka
{रावणं त्वब्रवीन्मन्त्री प्रहस्तो प्रहसो नाम वारुणः}
{गतः खलु महाराजो ब्रह्मलोकं जलेश्वरः} %7-23-51

\threelineshloka
{गान्धर्वं वरुणः श्रोतुं यं त्वमाह्वयसे युधि}
{तत्किं तव वृथा वीर परिश्रम्य गते नृपे}
{ये तु सन्निहिता वीराः कुमारास्ते पराजिताः} %7-23-52

\twolineshloka
{राक्षसेन्द्रस्तु तच्छ्रुत्वा नाम विश्राव्य चात्मनः}
{हर्षान्नादं विमुञ्चन्वै निष्क्रान्तो वरुणालयात्} %7-23-53

\twolineshloka
{आगतस्तु पथा येन तेनैव विनिवृत्य सः}
{लङ्कामभिमुखो रक्षो नभस्तलगतो ययौ} %7-23-54


॥इत्यार्षे श्रीमद्रामायणे वाल्मीकीये आदिकाव्ये उत्तरकाण्डे वरुणजयः नाम त्रयोविंशः सर्गः ॥७-२३॥
