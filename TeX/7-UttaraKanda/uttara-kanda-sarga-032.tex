\sect{द्वात्रिंशः सर्गः — रावणग्रहणम्}

\twolineshloka
{नर्मदापुलिने यत्र राक्षसेन्द्रः सुदारुणः}
{पुष्पोपहारं कुरुते तस्माद्देशाददूरतः} %7-32-1

\twolineshloka
{अर्जुनो जयतां श्रेष्ठो माहिष्मत्याः पतिः प्रभुः}
{क्रीडते सह नारीभिर्नर्मदातोयमाश्रितः} %7-32-2

\twolineshloka
{तासां मध्यगतो राजा रराज च तदार्जुनः}
{करेणूनां सहस्रस्य मध्यस्थ इव कुञ्जरः} %7-32-3

\twolineshloka
{जिज्ञासुः स तु बाहूनां सहस्रस्योत्तमं बलम्}
{रुरोध नर्मदावेगं बाहुभिर्बहुभिर्वृतः} %7-32-4

\twolineshloka
{कार्तवीर्यभुजासक्तं तज्जलं प्राप्य निर्मलम्}
{कूलोपहारं कुर्वाणं प्रतिस्रोतः प्रधावति} %7-32-5

\twolineshloka
{समीननक्रमकरः सपुष्पकुशसंस्तरः}
{स नर्मदाम्भसो वेगः प्रावृट्काल इवाबभौ} %7-32-6

\twolineshloka
{स वेगः कार्तवीर्येण सम्प्रेषित इवाम्भसः}
{पुष्पोपहारं सकलं रावणस्य जहार ह} %7-32-7

\twolineshloka
{रावणोऽर्धसमाप्तं तमुत्सृज्य नियमं तदा}
{नर्मदां पश्यते कान्तां प्रतिकूलां यथा प्रियाम्} %7-32-8

\twolineshloka
{पश्चिमेन तु तं दृष्ट्वा सागरोद्गारसन्निभम्}
{वर्धन्तमम्भसो वेगं पूर्वामाशां प्रविश्य तु} %7-32-9

\twolineshloka
{ततोऽनुभ्रान्तशकुनां स्वभावोपरमे स्थिताम्}
{निर्विकाराङ्गनाभासामपश्यद्रावणो नदीम्} %7-32-10

\twolineshloka
{सव्येतरकराङ्गुल्या सशब्दं च दशाननः}
{वेगप्रभावमन्वेष्टुं सोऽदिशच्छुकसारणौ} %7-32-11

\twolineshloka
{तौ तु रावणसन्दिष्टौ भ्रातरौ शुकसारणौ}
{व्योमान्तरगतौ वीरौ प्रस्थितौ पश्चिमामुखौ} %7-32-12

\twolineshloka
{अर्धयोजनमात्रं तु गत्वा तौ रजनीचरौ}
{पश्येतां पुरुषं तोये क्रीडन्तं सहयोषितम्} %7-32-13

\twolineshloka
{बृहत्सालप्रतीकाशं तोयव्याकुलमूर्धजम्}
{मदरक्तान्तनयनं मदव्याकुलतेजसम्} %7-32-14

\twolineshloka
{नदीं बाहुसहस्रेण रुन्धन्तमरिमर्दनम्}
{गिरिं पादसहस्रेण रुन्धन्तमिव मेदिनीम्} %7-32-15

\twolineshloka
{बालानां वरनारीणां सहस्रेण समावृतम्}
{समदानां करेणूनां सहस्रेणेव कुञ्जरम्} %7-32-16

\twolineshloka
{तमद्भुततमं दृष्ट्वा राक्षसौ शुकसारणौ}
{सन्निवृत्तावुपागम्य रावणं तमथोचतुः} %7-32-17

\twolineshloka
{बृहत्सालप्रतीकाशः कोऽप्यसौ राक्षसेश्वर}
{नर्मदां रोधवद्रुद्ध्वा क्रीडापयति योषितः} %7-32-18

\twolineshloka
{तेन बाहुसहस्रेण सन्निरुद्धजला नदी}
{सागरोद्गारसङ्काशानुद्गारान्सृजते मुहुः} %7-32-19

\twolineshloka
{इत्येवं भाषमाणौ तौ निशाम्य शुकसारणौ}
{रावणोऽर्जुन इत्युक्त्वा प्रययौ युद्धलालसः} %7-32-20

\twolineshloka
{अर्जुनाभिमुखे तस्मिन्रावणे राक्षसाधिपे}
{चण्डः प्रवाति पवनः सनादः सुरजास्तथा} %7-32-21

\twolineshloka
{सकृदेव कृतो रावः सरक्तः प्रेषितो घनैः}
{महोदरमहापार्श्वधूम्राक्षशुकसारणैः} %7-32-22

\threelineshloka
{संवृतो राक्षसेन्द्रस्तु तत्रागाद्यत्र चार्जुनः}
{अदीर्घेणैव कालेन स तदा राक्षसो बली}
{तं नर्मदाह्रदं भीममाजगामाञ्जनप्रभः} %7-32-23

\twolineshloka
{स तत्र स्त्रीपरिवृतं वाशिताभिरिव द्विपम्}
{नरेन्द्रं पश्यते राजा राक्षसानां तदार्जुनम्} %7-32-24

\onelineshloka
{स रोषाद्रक्तनयनो राक्षसेन्द्रो बलोद्धतः} %7-32-25

\twolineshloka
{इत्येवमर्जुनामात्यनाह गम्भीरया गिरा}
{अमात्याः क्षिप्रमाख्यात हैहयस्य नृपस्य वै} %7-32-26

\threelineshloka
{युद्धार्थी समनुप्राप्तो रावणो नाम राक्षसः}
{रावणस्य वचः श्रुत्वा मन्त्रिणोऽथार्जुनस्य ते}
{उत्तस्थुः सायुधास्त्राश्च रावणं वाक्यमब्रुवन्} %7-32-27

\onelineshloka
{युद्धस्य कालो विज्ञेयः साधु भोः साधु रावण} %7-32-28

\fourlineindentedshloka
{चः श्रुत्वा मन्त्रिणोऽथार्जुनस्य ते}
{यः क्षीबं स्त्रीवृतं चैव योद्धुमुत्सहते नृपम्}
{स्त्रीसमक्षं कथं यत्तद्योद्धुमुत्सहसेऽर्जुनम्}
{वाशितामध्यगं मत्तं शार्दूल इव कुञ्जरम्} %7-32-29

\twolineshloka
{क्षमस्वाद्य दशग्रीव चोष्यतां रजनी त्वया}
{युद्धे श्रद्धा च यद्यस्ति श्वस्तात समरेऽर्जुनम्} %7-32-30

\twolineshloka
{यद्यद्यास्ति मतिर्योद्धुं युद्धतृष्णासमावृता}
{निहत्यास्मांस्ततो युद्धमर्जुनेनोपयास्यसि} %7-32-31

\twolineshloka
{ततस्ते रावणामात्यैरमात्याः पार्थिवस्य तु}
{सूदिताश्चापि ते युद्धे भक्षिताश्च बुभुक्षितैः} %7-32-32

\twolineshloka
{ततो हलहलाशब्दो नर्मदातीर आबभौ}
{अर्जुनस्यानुयातऽणां रावणस्य च मन्त्रिणाम्} %7-32-33

\twolineshloka
{इषुभिस्तोमरैः शूलैस्त्रिशूलैर्वज्रकर्षणैः}
{सरावणानर्दयन्तः समन्तात्समभिद्रुताः} %7-32-34

\twolineshloka
{हैहयाधिपयोधानां वेग आसीत्सुदारुणः}
{सनक्रमीनमकरसमुद्रस्येव निःस्वनः} %7-32-35

\twolineshloka
{रावणस्य तु तेऽमात्याः प्रहस्तशुकसारणाः}
{कार्तवीर्यबलं क्रुद्धा निर्दहन्ति स्म तेजसा} %7-32-36

\twolineshloka
{अर्जुनाय तु तत्कर्म रावणस्य समन्त्रिणः}
{क्रीडमानाय कथितं पुरुषैर्द्वाररक्षिभिः} %7-32-37

\twolineshloka
{श्रुत्वा न भेतव्यमिति स्त्रीजनं तं ततोऽर्जुनः}
{उत्ततार जलात्तस्माद्गङ्गातोयादिवाञ्जनः} %7-32-38

\twolineshloka
{क्रोधदूषितनेत्रस्तु स ततोऽर्जुनपावकः}
{प्रजज्वाल महाघोरो युगान्त इव पावकः} %7-32-39

\twolineshloka
{स तूर्णतरमादाय वरहेमाङ्गदो गदाम्}
{अभिदुद्राव रक्षांसि तमांसीव दिवाकरः} %7-32-40

\twolineshloka
{बाहुविक्षेपकरणां समुद्यम्य महागदाम्}
{गारुडं वेगमास्थाय चापपातैव सोऽर्जुनः} %7-32-41

\twolineshloka
{तस्य मार्गं समारुद्ध्य विन्ध्योऽर्कस्येव पर्वतः}
{स्थितो विन्ध्य इवाकम्प्यः प्रहस्तो मुसलायुधः} %7-32-42

\twolineshloka
{ततोऽस्य मुसलं घोरं लोहबद्धं महोद्धतः}
{प्रहस्तः प्रेषयन्क्रुद्धो ररास च यथाम्बुदः} %7-32-43

\twolineshloka
{तस्याग्रे मुसलस्याग्निरशोकापीडसन्निभः}
{प्रहस्तकरमुक्तस्य बभूव प्रदहन्निव} %7-32-44

\twolineshloka
{अथायान्तं तु मुसलं कार्तवीर्यस्तदार्जुनः}
{निपुणं वञ्चयामास सगदोऽगदविक्रमः} %7-32-45

\twolineshloka
{ततस्तमभिदुद्राव प्रहस्तं हैहयाधिपः}
{भ्रामयानो गदां गुर्वीं पञ्चबाहुशतोच्छ्रयाम्} %7-32-46

\twolineshloka
{तथा हतोऽतिवेगेन प्रहस्तो गदया तदा}
{निपपात स्थितः शैलो वज्रिवज्रहतो यथा} %7-32-47

\twolineshloka
{प्रहस्तं पतितं दृष्ट्वा मारीचशुकसारणाः}
{समहोदरधूम्राक्षा ह्यपसृष्टा रणाजिरात्} %7-32-48

\twolineshloka
{अपक्रान्तेष्वमात्येषु प्रहस्ते वै निपातिते}
{रावणोऽभ्यद्रवत्तूर्णमर्जुनं नृपसत्तमम्} %7-32-49

\twolineshloka
{सहस्रबाहोस्तद्युद्धं विंशद्बाहोश्च दारुणम्}
{नृपराक्षसयोस्तत्र चारब्धं रोमहर्षणम्} %7-32-50

\twolineshloka
{सागराविव संरब्धौ चलन्मूलाविवाचलौ}
{तेजोयुक्ताविवादित्यौ प्रदहन्ताविवानलौ} %7-32-51

\twolineshloka
{बलोद्धतौ यथा नागौ वाशितार्थे यथा वृषौ}
{मेघाविव विनर्दन्तौ सिंहाविव बलोत्कटौ} %7-32-52

\twolineshloka
{रुद्रकालाविव क्रुद्धौ तदा तौ राक्षसार्जुनौ}
{परस्परं गदाभ्यां तौ ताडयामासतुर्भृशम्} %7-32-53

\twolineshloka
{वज्रप्रहारानचला यथा घोरान्विषेहिरे}
{गदाप्रहारांस्तौ तत्र सेहाते नरराक्षसौ} %7-32-54

\twolineshloka
{यथाऽ शनिरवेभ्यस्तु जायतेऽथ प्रतिश्रुतिः}
{तथा तयोर्गदापोथैर्दिशः सर्वाः प्रतिश्रुताः} %7-32-55

\twolineshloka
{अर्जुनस्य गदा सा तु पात्यमानाऽहितोरसि}
{काञ्चनाभं नभश्चक्रे विद्युत्सौदामिनी यथा} %7-32-56

\twolineshloka
{तथैव रावणेनापि पात्यमाना मुहुर्मुहुः}
{अर्जुनोरसि निर्भाति गदोल्केव महागिरौ} %7-32-57

\twolineshloka
{नार्जुनः खेदमायाति न राक्षसगणेश्वरः}
{इदमासीत्तयोर्युद्धं यथा पूर्वं बलीन्द्रयोः} %7-32-58

\twolineshloka
{शृङ्गैरिव वृषायुध्यन्दन्ताग्रैरिव कुञ्जरौ}
{परस्परं तौ निघ्नन्तौ नरराक्षससत्तमौ} %7-32-59

\twolineshloka
{ततोऽर्जुनेन क्रुद्धेन सर्वप्राणेन सा गदा}
{स्तनयोरन्तरे मुक्ता रावणस्य महोरसि} %7-32-60

\twolineshloka
{वरदानकृतत्राणे सा गदा रावणोरसि}
{दुर्बलेव यथावेगं द्विधाभूत्वाऽपतत्क्षितौ} %7-32-61

\twolineshloka
{स त्वर्जुनप्रमुक्तेन गदापातेन रावणः}
{अपासर्पद्धनुर्मात्रं निषसाद च निष्टनन्} %7-32-62

\twolineshloka
{स विह्वलं तदालक्ष्य दशग्रीवं ततोऽर्जुनः}
{सहसोत्पत्य जग्राह गरुत्मानिव पन्नगम्} %7-32-63

\twolineshloka
{स तु बाहुसहस्रेण बलाद्गृह्य दशाननम्}
{बबन्ध बलवान्राजा बलिं नारायणो यथा} %7-32-64

\twolineshloka
{बध्यमाने दशग्रीवे सिद्धचारणदेवताः}
{साध्वीतिवादिनः पुष्पैः किरन्त्यर्जुनमूर्धनि} %7-32-65

\twolineshloka
{व्याघ्रो मृगमिवादाय मृगराडिव दन्तिनम्}
{ननाद हैहयो राजा हर्षादम्बुदवन्मुहुः} %7-32-66

\twolineshloka
{प्रहस्तस्तु समाश्वस्तो दृष्ट्वा बद्धं दशाननम्}
{सह तै राक्षसैः क्रुद्धश्चाभिदुद्राव पार्थिवम्} %7-32-67

\twolineshloka
{नक्तञ्चराणां वेगस्तु तेषामापततां बभौ}
{उद्भूत आतपापाये पयोदानामिवाम्बुधौ} %7-32-68

\twolineshloka
{मुञ्च मुञ्चेति भाषन्तस्तिष्ठतिष्ठेति चासकृत्}
{मुसलानि सशूलानि सोत्ससर्ज तदार्जुने} %7-32-69

\twolineshloka
{अप्राप्तान्येव तान्याशु असम्भ्रान्तस्तदार्जुनः}
{आयुधान्यमरारीणां जग्राहारिनिषूदनः} %7-32-70

\twolineshloka
{ततस्तैरेव रक्षांसि दुर्धरैः प्रवरायुधैः}
{भित्त्वा विद्रावयामास वायुरम्बुधरानिव} %7-32-71

\twolineshloka
{राक्षसांस्त्रासयित्वा तु कार्तवीर्योऽर्जुनस्तदा}
{रावणं गृह्य नगरं प्रविवेश सुहृद्वृतः} %7-32-72

\twolineshloka
{स कीर्यमाणः कुसुमाक्षतोत्करैर्द्विजैः सपौरैः पुरुहूतसन्निभः}
{तदाऽर्जुनः सम्प्रविवेश तां पुरीं बलिं निगृह्येव सहस्रलोचनः} %7-32-73


॥इत्यार्षे श्रीमद्रामायणे वाल्मीकीये आदिकाव्ये उत्तरकाण्डे रावणग्रहणम् नाम द्वात्रिंशः सर्गः ॥७-३२॥
