\sect{द्व्यधिकशततमः सर्गः — अङ्गदचन्द्रकेतुनिवेशः}

\twolineshloka
{तच्छ्रुत्वा हर्षमापेदे राघवो भ्रातृभिः सह}
{वाक्यं चाद्भुतसङ्काशं तदा प्रोवाच लक्ष्मणम्} %7-102-1

\twolineshloka
{इमौ कुमारौ सौमित्रे तव धर्मविशारदौ}
{अङ्गदश्चन्द्रकेतुश्च राज्यार्थे दृढविक्रमौ} %7-102-2

\twolineshloka
{इमौ राज्येऽभिषेक्ष्यामि देशः साधु विधीयताम्}
{रमणीयो ह्यसम्बाधो रमेतां यत्र धन्विनौ} %7-102-3

\twolineshloka
{न राज्ञो यत्र पीडा स्यान्नाश्रमाणां विनाशनम्}
{स देशो दृश्यतां सौम्य नापराध्यामहे यथा} %7-102-4

\twolineshloka
{तथोक्तवति रामे तु भरतः प्रत्युवाच ह}
{अयं कारुपथो देशो रमणीयो निरामयः} %7-102-5

\twolineshloka
{निवेश्यतां तत्र पुरमङ्गदस्य महात्मनः}
{चन्द्रकेतोः सुरुचिरं चन्द्रकान्तं निरामयम्} %7-102-6

\twolineshloka
{तद्वाक्यं भरतेनोक्तं प्रतिजग्राह राघवः}
{तं च कृत्वा वशे देशमङ्गदस्य न्यवेशयत्} %7-102-7

\twolineshloka
{अङ्गदीया पुरी रम्याप्यङ्गदस्य निवेशिता}
{रमणीया सुगुप्ता च रामेणाक्लिष्टकर्मणा} %7-102-8

\twolineshloka
{चन्द्रकेतोश्च मल्लस्य मल्लभूम्यां निवेशिता}
{चन्द्रकान्तेति विख्याता दिव्या स्वर्गपुरी यथा} %7-102-9

\twolineshloka
{ततो रामः परां प्रीतिं लक्ष्मणो भरतस्तथा}
{ययुर्युद्धे दुराधर्षा अभिषेकं च चक्रिरे} %7-102-10

\twolineshloka
{अभिषिच्य कुमारौ स प्रस्थापयति राघवः}
{अङ्गदं पश्चिमां भूमिं चन्द्रकेतुमुदङ्मुखम्} %7-102-11

\twolineshloka
{अङ्गदं चापि सौमित्रिर्लक्ष्मणोऽनुजगाम ह}
{चन्द्रकेतोस्तु भरतः पार्ष्णिग्राहो बभूव ह} %7-102-12

\twolineshloka
{लक्ष्मणस्त्वङ्गदीयायां संवत्सरमथोषितः}
{पुत्रे स्थिते दुराधर्षे अयोध्यां पुनरागमत्} %7-102-13

\twolineshloka
{भरतोऽपि तथैवोष्य संवत्सरमतोऽधिकम्}
{अयोध्यां पुनरागम्य रामपादावुपास्त सः} %7-102-14

\twolineshloka
{उभौ सौमित्रिभरतौ रामपादावनुव्रतौ}
{कालं गतमपि स्नेहान्न जज्ञातेऽतिधार्मिकौ} %7-102-15

\twolineshloka
{एवं वर्षसहस्राणि दश तेषां ययुस्तदा}
{धर्मे प्रयतमानानां पौरकार्येषु नित्यदा} %7-102-16

\twolineshloka
{विहृत्य कालं परिपूर्णमानसाः श्रिया वृता धर्मपुरे सुसंस्थिताः}
{त्रयः समिद्धा इव दीप्ततेजसो महाध्वरे साधु हुतास्त्रयोऽग्नयः} %7-102-17


॥इत्यार्षे श्रीमद्रामायणे वाल्मीकीये आदिकाव्ये उत्तरकाण्डे अङ्गदचन्द्रकेतुनिवेशः नाम द्व्यधिकशततमः सर्गः ॥७-१०२॥
