\sect{त्रिंशः सर्गः — इन्द्रपराजयकारणकथनम्}

\twolineshloka
{जिते महेन्द्रेऽतिबले रावणस्य सुतेन वै}
{प्रजापतिं पुरस्कृत्य ययुर्लङ्कां सुरास्तदा} %7-30-1

\twolineshloka
{तत्र रावणमासाद्य पुत्रभ्रातृभिरावृतम्}
{अब्रवीद्गगने तिष्ठन्सामपूर्वं प्रजापतिः} %7-30-2

\twolineshloka
{वत्स रावण तुष्टोऽस्मि पुत्रस्य तव संयुगे}
{अहोऽस्य विक्रमौदार्यं तव तुल्योऽधिकोऽपि वा} %7-30-3

\twolineshloka
{जितं हि भवता सर्वं त्रैलोक्यं स्वेन तेजसा}
{कृता प्रतिज्ञा सफला प्रीतोऽस्मि स्वसुतेन वै} %7-30-4

\twolineshloka
{अयं च पुत्रोऽतिबलस्तव रावण वीर्यवान्}
{जगतीन्द्रजिदित्येव परिख्यातो भविष्यति} %7-30-5

\twolineshloka
{बलवान्दुर्जयश्चैव भविष्यत्येव राक्षसः}
{यं समाश्रित्य ते राजन्स्थापितास्त्रिदशा वशे} %7-30-6

\twolineshloka
{तन्मुच्यतां महाबाहो महेन्द्रः पाकशासनः}
{किञ्चास्य मोक्षणार्थाय प्रयच्छन्तु दिवौकसः} %7-30-7

\twolineshloka
{अथाब्रवीन्महातेजा इन्द्रजित्समितिञ्जयः}
{अमरत्वमहं देव वृणे यद्येष मुच्यते} %7-30-8

\twolineshloka
{चतुष्पदां खेचराणामन्येषां वा महौजसाम्}
{वृक्षगुल्मक्षुपलतातृणोपलमहीभृताम्} %7-30-9

\twolineshloka
{सर्वेऽपि जन्तवोऽन्योन्यं भेतव्ये सति बिभ्यति}
{अतोऽत्र लोके सर्वेषां सर्वस्माच्च भवेद्भयम्} %7-30-10

\threelineshloka
{ततोऽब्रवीन्महातेजा मेघनादं प्रजापतिः}
{नास्ति सर्वामरत्वं हि कस्यचित्प्राणिनो भुवि}
{चतुष्पदः पक्षिणश्च भूतानां वा महौजसाम्} %7-30-11

\onelineshloka
{श्रुत्वा पितामहेनोक्तमिन्द्रजित्प्रभुणाव्ययम्} %7-30-12

\threelineshloka
{्वं हि कस्यचित्प्राणिनो भुवि}
{अथाब्रवीत्स तत्रस्थं मेघनादो महाबलः}
{श्रूयतां वा भवेत्सिद्धिः शतक्रतुविमोक्षणे} %7-30-13

\twolineshloka
{ममेष्टं नित्यशो हर्व्यैर्मन्त्रैः सम्पूज्य पावकम्}
{सङ्ग्राममवतर्त्तुं च शत्रुनिर्जयकाङ्क्षिणः} %7-30-14

\twolineshloka
{अश्वयुक्तो रथो मह्यमुत्तिष्ठेत्तु विभावसोः}
{तत्स्थस्यामरता स्यान्मे एष मे निश्चयो वरः} %7-30-15

\twolineshloka
{तस्मिन्यद्यसमाप्ते च जप्यहोमे विभावसौ}
{युध्येयं देव सङ्ग्रामे तदा मे स्याद्विनाशनम्} %7-30-16

\twolineshloka
{सर्वो हि तपसा देव वृणोत्यमरतां पुमान्}
{विक्रमेण मया त्वेतदमरत्वं प्रवर्तितम्} %7-30-17

\twolineshloka
{एवमस्त्विति तं चाह वाक्यं देवः पितामहः}
{मुक्तश्चेन्द्रजिता शक्रो गताश्च त्रिदिवं सुराः} %7-30-18

\twolineshloka
{एतस्मिन्नन्तरे राम दीनो भ्रष्टाम्बरद्युतिः}
{इन्द्रश्चिन्तापरीतात्मा ध्यानतत्परतां गतः} %7-30-19

\twolineshloka
{तं तु दृष्ट्वा तथाभूतं प्राह देवः प्रजापतिः}
{शतक्रतो किमु पुरा करोति स्म सुदुष्कृतम्} %7-30-20

\twolineshloka
{अमरेन्द्र मया बह्व्यः प्रजाः सृष्टास्तथा प्रभो}
{एकवर्णाः समाभाषा एकरूपाश्च सर्वशः} %7-30-21

\twolineshloka
{तासां नास्ति विशेषो हि दर्शने लक्षणेऽपि वा}
{ततोऽहमेकाग्रमनास्ताः प्रजाः परिचिन्तयम्} %7-30-22

\twolineshloka
{सोऽहं तासां विशेषार्थं स्त्रियमेकां विनिर्ममे}
{यद्यत्प्रजानां प्रत्यङ्गं विशिष्टं तत्तदुद्धृतम्} %7-30-23

\twolineshloka
{ततो मया रूपगुणैरहल्या स्त्री विनिर्मिता}
{हलं नामेह वैरूपं हल्यं तत्प्रभवं भवेत्} %7-30-24

\twolineshloka
{यस्मान्न विद्यते हल्यं तेनाहल्येति विश्रुता}
{अहल्येति मया शक्र तस्या नाम प्रवर्तितम्} %7-30-25

\twolineshloka
{निर्मितायां च देवेन्द्र तस्यां नार्यां सुरर्षभ}
{भविष्यतीति कस्यैषा मम चिन्ता ततोऽभवत्} %7-30-26

\twolineshloka
{त्वं तु शक्र तदा नारीं जानीषे मनसा प्रभो}
{स्थानाधिकतया पत्नी ममैषेति पुरन्दर} %7-30-27

\twolineshloka
{सा मया न्यासभूता तु गौतमस्य महात्मनः}
{न्यस्ता बहूनि वर्षाणि तेन निर्यातिता च ह} %7-30-28

\twolineshloka
{ततस्तस्य परिज्ञाय महास्थैर्यं महामुनेः}
{ज्ञात्वा तपसि सिद्धिं च पत्न्यर्थं स्पर्शिता तदा} %7-30-29

\twolineshloka
{सङ्क्रुद्धस्त्वं हि धर्मात्मन् गत्वा तस्याश्रमं मुनेः}
{दृष्टवांश्च तदा तां स्त्रीं दीप्तामग्निशिखामिव} %7-30-30

\twolineshloka
{सा त्वया धर्षिता शक्र कामार्तेन समन्युना}
{दृष्टस्त्वं च तदा तेन ह्याश्रमे परमर्षिणा} %7-30-31

\twolineshloka
{ततः क्रुद्धेन तेनाऽसि शप्तः परमतेजसा}
{गतोऽसि येन देवेन्द्र दशाभागविपर्ययम्} %7-30-32

\twolineshloka
{यस्मान्मे धर्षिता पत्नी त्वया वासव निर्भयम्}
{तस्मात्त्वं समरे राजञ्छत्रुहस्तं गमिष्यसि} %7-30-33

\twolineshloka
{अयं तु भावो दुर्बुद्धे यस्त्वयेह प्रवर्तितः}
{मानुषेष्वपि लोकेषु भविष्यति न संशयः} %7-30-34

\twolineshloka
{तत्रार्धं तस्य यः कर्ता त्वय्यर्धं निपतिष्यति}
{न च ते स्थावरं स्थानं भविष्यति न संशयः} %7-30-35

\twolineshloka
{यश्च यश्च सुरेन्द्रः स्याद्ध्रुवः स न भविष्यति}
{एष शापो मया मुक्त इत्यसौ त्वां तदाऽब्रवीत्} %7-30-36

\twolineshloka
{तां तु भार्यां स निर्भर्त्स्य सोऽब्रवीत्सुमहातपाः}
{दुर्विनीते विनिध्वंस ममाश्रमसमीपतः} %7-30-37

\twolineshloka
{रूपयौवनसम्पन्ना यस्मात्त्वमनवस्थिता}
{तस्माद्रूपवती लोके न त्वमेका भविष्यति} %7-30-38

\twolineshloka
{रूपं च ते प्रजाः सर्वा गमिष्यन्ति न संशयः}
{यत्तदेकं समाश्रित्य विभ्रमोऽयमुपस्थितः} %7-30-39

\twolineshloka
{तदाप्रभृति भूयिष्ठं प्रजा रूपसमन्विताः}
{सा तं प्रसादयामास महर्षिं गौतमं तदा} %7-30-40

\twolineshloka
{अज्ञानाद्धर्षिता विप्र त्वद्रूपेण दिवौकसा}
{न कामकाराद्विप्रर्षे प्रसादं कर्तुमर्हसि} %7-30-41

\twolineshloka
{अहल्यया त्वेवमुक्तः प्रत्युवाच स गौतमः}
{उत्पत्स्यति महातेजा इक्ष्वाकूणां महारथः} %7-30-42

\threelineshloka
{रामो नाम श्रुतो लोके वनं चाप्युपयास्यति}
{ब्राह्मणार्थे महाबाहुर्विष्णुर्मानुषविग्रहः}
{तं द्रक्ष्यसि यदा भद्रे ततः पूता भविष्यसि} %7-30-43

\twolineshloka
{स हि पावयितुं शक्तस्त्वया यद्दुष्कृतं कृतम्}
{तस्यातिथ्यं च कृत्वा वै मत्समीपं गमिष्यसि} %7-30-44

\twolineshloka
{वत्स्यसि त्वं मया सार्धं तदा हि वरवर्णिनि}
{एवमुक्त्वा स विप्रर्षिराजगाम स्वमाश्रमम्} %7-30-45

\twolineshloka
{तपश्चाचार सुमहत्सा पत्नी ब्रह्मवादिनः}
{शापोत्सर्गाद्धि तस्येदं मुनेः सर्वमुपस्थितम्} %7-30-46

\twolineshloka
{तत्स्मर त्वं महाबाहो दुष्कृतं यत्त्वया कृतम्}
{तेन त्वं ग्रहणं शत्रोर्यातो नान्येन वासव} %7-30-47

\twolineshloka
{शीघ्रं वै यज यज्ञं त्वं वैष्णवं सुसमाहितः}
{पावितस्तेन यज्ञेन यास्यसे त्रिदिवं ततः} %7-30-48

\twolineshloka
{पुत्रश्च तव देवेन्द्र न विनष्टो महारणे}
{नीतः सन्निहितश्चैव आर्यकेण महोदधौ} %7-30-49

\twolineshloka
{एतच्छ्रुत्वा महेन्द्रस्तु यज्ञमिष्ट्वा च वैष्णवम्}
{पुनस्त्रिदिवमाक्रामदन्वशासच्च देवराट्} %7-30-50

\twolineshloka
{एतदिन्द्रजितो नाम बलं यत्कीर्तितं मया}
{निर्जितस्तेन देवेन्द्रः प्राणिनोऽन्ये तु किं पुनः} %7-30-51

\twolineshloka
{आश्चर्यमिति रामश्च लक्ष्मणश्चाब्रवीत्तदा}
{अगस्त्यवचनं श्रुत्वा वानरा राक्षसास्तदा} %7-30-52

\twolineshloka
{विभीषणस्तु रामस्य पार्श्वस्थो वाक्यमब्रवीत्}
{आश्चर्यं स्मारितोऽस्म्यद्य यत्तद्दृष्टं पुरातनम्} %7-30-53

\onelineshloka
{अगस्त्यं त्वब्रवीद्रामः सत्यमेतच्छ्रुतं च मे} %7-30-54

\twolineshloka
{एवं राम समुद्भूतो रावणो लोककण्टकः}
{सपुत्रो येन सङ्ग्रामे जितः शक्रः सुरेश्वरः} %7-30-55


॥इत्यार्षे श्रीमद्रामायणे वाल्मीकीये आदिकाव्ये उत्तरकाण्डे इन्द्रपराजयकारणकथनम् नाम त्रिंशः सर्गः ॥७-३०॥
