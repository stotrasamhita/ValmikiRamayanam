\sect{पञ्चविंशः सर्गः — मधुवधवारणम्}

\twolineshloka
{स तु दत्त्वा दशग्रीवो वनं घोरं स्वरस्य तत्}
{भगिनीं च समाश्वास्य हृष्टः स्वस्थतरोऽभवत्} %7-25-1

\twolineshloka
{ततो निकुम्भिला नाम लङ्कोपवनमुत्तमम्}
{तद्राक्षसेन्द्रो बलवान्प्रविवेश सहानुगः} %7-25-2

\twolineshloka
{ततो यूपशताकीर्णं सौम्यचैत्योपशोभितम्}
{ददर्श विष्ठितं यज्ञं श्रिया सम्प्रज्वलन्निव} %7-25-3

\twolineshloka
{ततः कृष्णाजिनधरं कमण्डलुशिखाध्वजम्}
{ददर्श स्वसुतं तत्र मेघनादं भयावहम्} %7-25-4

\twolineshloka
{तं समासाद्य लङ्केशः परिष्वज्वाथ बाहुभिः}
{अब्रवीत्किमिदं वत्स वर्तसे ब्रूहि तत्त्वतः} %7-25-5

\twolineshloka
{उशना त्वब्रवीत्तत्र यज्ञसम्पत्समृद्धये}
{रावणं राक्षसश्रेष्ठं द्विजश्रेष्ठो महातपाः} %7-25-6

\twolineshloka
{अहमाख्यामि ते राजञ्छ्रूयतां सर्वमेव तत्}
{यज्ञास्ते सप्त पुत्रेण प्राप्तास्सुबहुविस्तराः} %7-25-7

\twolineshloka
{अग्निष्टोमोऽश्वमेधश्च यज्ञो बहुसुवर्णकः}
{राजसूयस्तथा यज्ञो गोमेधो वैष्णवस्तथा} %7-25-8

\twolineshloka
{माहेश्वरे प्रवृत्ते तु यज्ञे पुम्भिः सुदुर्लभे}
{वरांस्ते लब्धवान्पुत्रः साक्षात्पशुपतेरिह} %7-25-9

\twolineshloka
{कामगं स्यन्दनं दिव्यमन्तरिक्षचरं ध्रुवम्}
{मायां च तामसीं नाम यया सम्पद्यते तमः} %7-25-10

\twolineshloka
{एतया किल सङ्ग्रामे मायया राक्षसेश्वर}
{प्रयुक्तया गतिः शक्या नहि ज्ञातुं सुरासुरैः} %7-25-11

\twolineshloka
{अक्षयाविषुधी बाणैश्चापं चापि सुदुर्जयम्}
{अस्त्रं च बलवद्राजञ्छत्रुविध्वंसनं रणे} %7-25-12

\twolineshloka
{एतान्सर्वान्वरांल्लब्ध्वा पुत्रस्तेऽयं दशानन}
{अद्य यज्ञसमाप्तौ च त्वां दिदृक्षुस्स्थितो ह्यहम्} %7-25-13

\twolineshloka
{ततोऽब्रवीदृशग्रीवो न शोभनमिदं कृतम्}
{पूजिताः शत्रवो यस्माद्द्रव्यैरिन्द्रपुरोगमाः} %7-25-14

\twolineshloka
{एहीदानीं कृतं विद्धि सुकृतं तन्न संशयः}
{आगच्छ सौम्य गच्छामः स्वमेव भवनं प्रति} %7-25-15

\twolineshloka
{ततो गत्वा दशग्रीवः सपुत्रः सविभीषणः}
{स्त्रियोऽवतारयामास सर्वास्ता बाष्पगद्गदाः} %7-25-16

\twolineshloka
{लक्षिण्यो रत्नभूताश्च देवदानवरक्षसाम्}
{तस्य तासु मतिं ज्ञात्वा धर्मत्मा वाक्यमब्रवीत्} %7-25-17

\twolineshloka
{ईदृशैस्त्वं समाचारैर्यशोर्थकुलनाशनैः}
{धर्षणं ज्ञातिनां ज्ञात्वा स्वमतेन विचेष्टसे} %7-25-18

\twolineshloka
{ज्ञातींस्तान्धर्षयित्वेमास्त्वयाऽऽनीता वराङ्गनाः}
{त्वामतिक्रम्य मधुना राजन्कुम्भीनसी हृता} %7-25-19

\twolineshloka
{रावणस्त्वब्रवीद्वाक्यं नावगच्छामि किं त्विदम्}
{कोऽयं यस्तु त्वयाऽऽख्यातो मधुरित्येव नामतः} %7-25-20

\twolineshloka
{विभीषणस्तु सङ्क्रुद्धो भ्रातरं वाक्यमब्रवीत्}
{श्रूयतामस्य पापस्य कर्मणः फलमागतम्} %7-25-21

\twolineshloka
{मातामहस्य यो भ्राता ज्येष्ठो भ्राता सुमालिनः}
{माल्यवानिति विख्यातो वृद्धः प्राज्ञो निशाचरः} %7-25-22

\twolineshloka
{पिता ज्येष्ठो जनन्या नो ह्यस्माकं चार्यकोऽभवत्}
{तस्य कुम्भीनसी नाम दुहितुर्दुहिताऽभवत्} %7-25-23

\twolineshloka
{मातृष्वसुरथास्माकं सा च कन्याऽनलोद्भवा}
{भवत्यस्माकमेवैषा भ्रातृणां धर्मतः स्वसा} %7-25-24

\twolineshloka
{सा हृता मधुना राजन्राक्षसेन बलीयसा}
{यज्ञप्रवृत्ते पुत्रे तु मयि चान्तर्जलोषिते} %7-25-25

\twolineshloka
{कुम्भकर्णे महाराज निद्रामनुभवत्यथ}
{निहत्य राक्षसश्रेष्ठानमात्यानिह सम्मतान्} %7-25-26

\twolineshloka
{धर्षयित्वा हृता सा तु सुप्ताप्यन्तःपुरे तव}
{श्रुत्वाऽपि तन्महाराज क्षान्तमेव हतो न सः} %7-25-27

\twolineshloka
{यस्मादवश्यं दातव्या कन्या भर्त्रे हि भ्रातृभिः}
{तदेतत्कर्मणो ह्यस्य फलं पापस्य दुर्मते} %7-25-28

\threelineshloka
{अस्मिन्नेवाभिसम्प्राप्तं लोके विदितमस्तु ते}
{विभीषणवचः श्रुत्वा राक्षसेन्द्रः स रावणः}
{दौरात्म्येनात्मनोद्धूतस्तप्ताम्भ इव सागरः} %7-25-29

\twolineshloka
{ततोऽब्रवीदृशग्रीवः क्रुद्धः संरक्तलोचनः}
{कल्प्यतां मे रथः शीघ्रं शूराः सज्जीभवन्तु नः} %7-25-30

\twolineshloka
{भ्राता मे कुम्भकर्णश्च ये च मुख्या निशाचराः}
{वाहनान्यधिरोहन्तु नानाप्रहरणायुधाः} %7-25-31

\twolineshloka
{अद्य तं समरे हत्वा मधुं रावणनिर्भयम्}
{सुरलोकं गमिष्यामि युद्धकाङ्क्षी सुहृद्वृतः} %7-25-32

\twolineshloka
{अक्षौहिणीसहस्राणि चत्वार्यग्र्याणि रक्षसाम्}
{नानाप्रहरणान्याशु निर्ययुर्युद्धकाङ्क्षिणाम्} %7-25-33

\twolineshloka
{इन्द्रजित्त्वग्रतः अग्रतः सैन्यात्सैनिकान्परिगृह्य च}
{जगाम रावणो मध्ये कुम्भकर्णश्च पृष्ठतः} %7-25-34

\twolineshloka
{विभीषणश्च धर्मात्मा लङ्कायां धर्मामाचरत्}
{शोषाः सर्वे महाभागा ययुर्मधुपुरं प्रति} %7-25-35

\twolineshloka
{खरैरुष्ट्रैर्हयैर्दीप्तैः शिंशुमारैर्महोरगैः}
{राक्षसाः प्रययुः सर्वे कृत्वाऽऽकाशं निरन्तरम्} %7-25-36

\twolineshloka
{दैत्याश्च शतशस्तत्र कृतवैराश्च दैवतैः}
{रावणं प्रेक्ष्य गच्छन्तमन्वगच्छन्हि पृष्ठतः} %7-25-37

\twolineshloka
{स तु गत्वा मधुपुरं प्रविश्य च दशाननः}
{न ददर्श मधुं तत्र भगिनीं तत्र दृष्टवान्} %7-25-38

\onelineshloka
{सा च प्रह्वाञ्जलिर्भूत्वा शिरसा चरणौ गता} %7-25-39

\twolineshloka
{तस्य राक्षसराजस्य त्रस्ता कुम्भीनसी तदा}
{तां समुत्थापयामास न भेतव्यमिति ब्रुवन्} %7-25-40

\threelineshloka
{रावणो राक्षसश्रेष्ठः किं चापि करवाणि ते}
{साब्रवीद्यदि मे राजन्प्रसन्नस्त्वं महाभुज}
{भर्तारं न ममेहाद्य हन्तुर्महसि मानद} %7-25-41

\twolineshloka
{न हीदृशं भयं किञ्चित्कुलस्त्रीणामिहोच्यते}
{भयानामपि सर्वेषां वैधव्यं व्यसनं महत्} %7-25-42

\twolineshloka
{सत्यवाग्भव राजेन्द्र मामवेक्षस्व याचतीम्}
{त्वयाप्युक्तं महाराज न भेतव्यमिति स्वयम्} %7-25-43

\twolineshloka
{रावणस्त्वब्रवीद्धृष्टः स्वसारं तत्र संस्थिताम्}
{क्व चासौ तव भर्ता वै मम शीघ्रं निवेद्यताम्} %7-25-44

\twolineshloka
{सह तेन गमिष्यामि सुरलोकं जयावहे}
{तव कारुण्यसौहार्दान्निवृत्तोऽस्मि मधोर्वधात्} %7-25-45

\twolineshloka
{इत्युक्ता सा समुत्थाप्य प्रसुप्तं तं निशाचरम्}
{अब्रवीत्सम्प्रहृष्टेव राक्षसी सा पतिं वचः} %7-25-46

\twolineshloka
{एष प्राप्तो दशग्रीवो मम भ्राता महाबलः}
{सुरलोकजयाकाङ्क्षी साहाय्ये त्वां वृणोति च} %7-25-47

\twolineshloka
{तदस्य त्वं सहायार्थं सबन्धुर्गच्छ राक्षस}
{स्निग्धस्य भजमानस्य युक्तमर्थाय कल्पितुम्} %7-25-48

\onelineshloka
{तस्यास्तद्वचनं श्रुत्वा तथेत्याह मधुर्वचः} %7-25-49

\twolineshloka
{ददर्श राक्षसश्रेष्ठं यथान्यायमुपेत्य सः}
{पूजयमास धर्मेण रावणं राक्षसाधिपम्} %7-25-50

\twolineshloka
{प्राप्य पूजां दशग्रीवो मधुवेश्मानि वीर्यवान्}
{तत्र चैकां निशामुष्य गमनायोपचक्रमे} %7-25-51

\twolineshloka
{ततः कैलासमासाद्य शैलं वैश्रवणालयम्}
{राक्षसेन्द्रो महेन्द्राभः सेनामुपनिवेशयत्} %7-25-52


॥इत्यार्षे श्रीमद्रामायणे वाल्मीकीये आदिकाव्ये उत्तरकाण्डे मधुवधवारणम् नाम पञ्चविंशः सर्गः ॥७-२५॥
