\sect{त्रिनवतितमः सर्गः — वाल्मीकिसन्देशः}

\twolineshloka
{वर्तमाने तथाभूते यज्ञे च परमाद्भुते}
{सशिष्य आजगामाशु वाल्मीकिर्मुनिपुङ्गवः} %7-93-1

\twolineshloka
{स दृष्ट्वा दिव्यसङ्काशं यज्ञमद्भुतदर्शनम्}
{एकान्ते ऋषिवाटानां चकार उटजाञ्छुभान्} %7-93-2

\twolineshloka
{शकटांश्च बहून्पूर्णान्फलमूलैश्च शोभनान्}
{वाल्मीकिवाटे रुचिरे स्थापयन्नविदूरतः} %7-93-3

\twolineshloka
{आसीत्सुपूजितो राज्ञा मुनिभिश्च महात्मभिः}
{वाल्मीकिः सुमहातेजा न्यवसत्परमात्मवान्} %7-93-4

\twolineshloka
{स शिष्यावब्रवीद्धृष्टौ युवां गत्वा समाहितौ}
{कृत्स्नं रामायणं काव्यं गायेथां परया मुदा} %7-93-5

\twolineshloka
{ऋषिवाटेषु पुण्येषु ब्राह्मणावसथेषु च}
{रथ्यासु राजमार्गेषु पार्थिवानां गृहेषु च} %7-93-6

\twolineshloka
{रामस्य भवनद्वारि यत्र कर्म च कुर्वते}
{ऋत्विजामग्रतश्चैव तत्र गेयं विशेषतः} %7-93-7

\twolineshloka
{इमानि च फलान्यत्र स्वादूनि विविधानि च}
{जातानि पर्वताग्रेषु चास्वाद्यास्वाद्य गायताम्} %7-93-8

\twolineshloka
{न यास्यथः श्रमं वत्सौ भक्षयित्वा फलानि वै}
{मूलानि च सुमृष्टानि नगराद्बहिरास्यथः} %7-93-9

\twolineshloka
{यदि शब्दापयेद्रामः श्रवणाय महीपतिः}
{ऋषीणामुपविष्टानां ततो गेयं प्रवर्तताम्} %7-93-10

\twolineshloka
{दिवसे विंशतिः सर्गा गेया मधुरया गिरा}
{प्रमाणैर्बहुभिस्तत्र यथोद्दिष्टा मया पुरा} %7-93-11

\twolineshloka
{लोभश्चापि न कर्तव्यः स्वल्पोऽपि धनकाङ्क्षया}
{किं धनेनाश्रमस्थानां फलमूलोपजीविना} %7-93-12

\twolineshloka
{यदि पृच्छेत्स काकुत्स्थो युवां कस्येति दारकौ}
{आवां वाल्मीकिशिष्यौ स्वो ब्रूतमेवं नराधिपम्} %7-93-13

\twolineshloka
{इमास्तन्त्रीः सुमधुराः स्थानं वाऽपूर्वदर्शनम्}
{मूर्च्छयित्वा सुमधुरं गायतां विगतज्वरौ} %7-93-14

\twolineshloka
{आदिप्रभृति गेयं स्यान्न चावज्ञाय पार्थिवम्}
{पिता हि सर्वभूतानां राजा भविति धर्मतः} %7-93-15

\twolineshloka
{तद्युवां हृष्टमनसौ श्वः प्रभाते समास्थितौ}
{गायेथां मधुरं गेयं तन्त्रीलयसमन्वितम्} %7-93-16

\twolineshloka
{इति सन्दिश्य बहुशो मुनिः प्राचेतसस्तदा}
{वाल्मीकिः परमोदारस्तूष्णीमासीन्महायशाः} %7-93-17

\twolineshloka
{सन्दिष्टौ मुनिना तेन तावुभौ मैथिलीसुतौ}
{तथैव करवावेति निर्जग्मतुररिन्दमौ} %7-93-18

\twolineshloka
{तामद्भुतां तौ हृदये कुमारौ निवेश्य वाणीमृषिभाषितां तदा}
{समुत्सुकौ तौ सममूषतुर्निशां यथाऽश्विनौ भार्गवनीतिसंहिताम्} %7-93-19


॥इत्यार्षे श्रीमद्रामायणे वाल्मीकीये आदिकाव्ये उत्तरकाण्डे वाल्मीकिसन्देशः नाम त्रिनवतितमः सर्गः ॥७-९३॥
