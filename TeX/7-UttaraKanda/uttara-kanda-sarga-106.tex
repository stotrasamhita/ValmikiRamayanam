\sect{षडधिकशततमः सर्गः — लक्ष्मणपरित्यागः}

\twolineshloka
{अवाङ्मुखमथो दीनं दृष्ट्वा सोममिवाप्लुतम्}
{राघवं लक्ष्मणो वाक्यं हृष्टो मधुरमब्रवीत्} %7-106-1

\twolineshloka
{न सन्तापं महाबाहो मदर्थं कर्तुमर्हसि}
{पूर्वनिर्माणबद्धा हि कालस्य गतिरीदृशी} %7-106-2

\twolineshloka
{जहि मां सौम्य विस्रब्धं प्रतिज्ञां परिपालय}
{हीनप्रतिज्ञाः काकुत्स्थ प्रयान्ति नरकं नराः} %7-106-3

\twolineshloka
{यदि प्रीतिर्महाराज यद्यनुग्राह्यता मयि}
{जहि मां निर्विशङ्कस्त्वं धर्मं वर्धय राघव} %7-106-4

\twolineshloka
{लक्ष्मणेन तथोक्तस्तु रामः प्रचलितेन्द्रियः}
{मन्त्रिणः समुपानीय तथैव च पुरोधसम्} %7-106-5

\twolineshloka
{अब्रवीच्च तदा वृत्तं तेषां मध्ये स राघवः}
{दुर्वासोभिगमं चैव प्रतिज्ञां तापसस्य च} %7-106-6

\twolineshloka
{तच्छ्रुत्वा मन्त्रिणः सर्वे सोपाध्यायाः समासत}
{वसिष्ठस्तु महातेजा वाक्यमेतदुवाच ह} %7-106-7

\twolineshloka
{दृष्टमेतन्महाबाहो क्षयं ते रोमहर्षणम्}
{लक्ष्मणेन वियोगश्च तव राम महायशः} %7-106-8

\twolineshloka
{त्यजैनं बलवान्कालो मा प्रतिज्ञां वृथा कृताः}
{विनष्टायां प्रतिज्ञायां धर्मोऽपि च लयं व्रजेत्} %7-106-9

\twolineshloka
{ततो धर्मे विनष्टे तु त्रैलोक्यं सचराचरम्}
{सदेवर्षिगणं सर्वं विनश्येत्तु न संशयः} %7-106-10

\twolineshloka
{स त्वं पुरुषशार्दूल त्रैलोक्यस्याभिपालनात्}
{लक्ष्मणेन विना चाद्य जगत्स्वस्थं कुरुष्व ह} %7-106-11

\twolineshloka
{तेषां तत्समवेतानां वाक्यं धर्मार्थसंहितम्}
{श्रुत्वा परिषदो मध्ये रामो लक्ष्मणमब्रवीत्} %7-106-12

\twolineshloka
{विसर्जये त्वां सौमित्रे मा भूद्धर्मविपर्ययः}
{त्यागो वधो वा विहितः साधूनां तूभयं समम्} %7-106-13

\twolineshloka
{रामेण भाषिते वाक्ये बाष्पव्याकुलितेन्द्रियः}
{लक्ष्मणस्त्वरितं प्रायात्स्वगृहं न विवेश ह} %7-106-14

\twolineshloka
{स गत्वा सरयूतीरमुपस्पृश्य कृताञ्जलिः}
{निगृह्य सर्वस्रोतांसि निःश्वासं न मुमोच ह} %7-106-15

\twolineshloka
{अनिःश्वसन्तं युक्तं तं सशक्राः साप्सरोगणाः}
{देवाः सर्षिगणाः सर्वे पुष्पैरभ्यकिरंस्तदा} %7-106-16

\twolineshloka
{अदृश्यं सर्वमनुजैः सशरीरं महाबलम्}
{प्रगृह्य लक्ष्मणं शक्रस्त्रिदिवं संविवेश ह} %7-106-17

\twolineshloka
{ततो विष्णोश्चतुर्भागमागतं सुरसत्तमाः}
{दृष्ट्वा प्रमुदिताः सर्वेऽपूजयन् समहर्षयः} %7-106-18


॥इत्यार्षे श्रीमद्रामायणे वाल्मीकीये आदिकाव्ये उत्तरकाण्डे लक्ष्मणपरित्यागः नाम षडधिकशततमः सर्गः ॥७-१०६॥
