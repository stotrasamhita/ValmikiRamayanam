\sect{एकोननवतितमः सर्गः — पुरूरवोजननम्}

\twolineshloka
{श्रुत्वा किम्पुरुषोत्पत्तिं लक्ष्मणो भरतस्तदा}
{आश्चर्यमिति चाब्रूतामुभौ रामं जनेश्वरम्} %7-89-1

\twolineshloka
{अथ रामः कथामेतां भूय एव महायशाः}
{कथयामास धर्मात्मा प्रजापतिसुतस्य वै} %7-89-2

\twolineshloka
{सर्वास्ता विद्रुता दृष्ट्वा किन्नरीर्ऋषिसत्तमः}
{उवाच रूपसम्पन्नां तां स्त्रियं प्रहसन्निव} %7-89-3

\twolineshloka
{सोमस्याहं सुदयितः सुतः सुरुचिरानने}
{भजस्व मां वरारोहे भक्त्या स्निग्धेन चक्षुषा} %7-89-4

\twolineshloka
{तस्य तद्वचनं श्रुत्वा शून्ये स्वजनवर्जिते}
{इला सुरुचिरप्रख्यं प्रत्युवाच महाग्रहम्} %7-89-5

\twolineshloka
{अहं कामचरी सौम्य तवास्मि वशवर्तिनी}
{प्रशाधि मां सोमसुत यथेच्छसि तथा कुरु} %7-89-6

\twolineshloka
{तस्यास्तदद्भुतप्रख्यं श्रुत्वा हर्षमुपागतः}
{स वै कामी सह तया रेमे चन्द्रमसः सुतः} %7-89-7

\twolineshloka
{बुधस्य माधवो मासस्तामिलां रुचिराननाम्}
{गतो रमयतोऽत्यर्थं क्षणवत्तस्य कामिनः} %7-89-8

\twolineshloka
{अथ मासे तु सम्पूर्णे पूर्णेन्दुसदृशाननः}
{प्रजापतिसुतः श्रीमाञ्छयने प्रत्यबुध्यत} %7-89-9

\twolineshloka
{सोऽपश्यत्सोमजं तत्र तपन्तं सलिलाशये}
{ऊर्ध्वबाहुं निरालम्बं तं राजा प्रत्यभाषत} %7-89-10

\twolineshloka
{भगवन्पर्वतं दुर्गं प्रविष्टोऽस्मि सहानुगः}
{न च पश्यामि तत्सैन्यं क्व नु ते मामका गताः} %7-89-11

\twolineshloka
{तच्छ्रुत्वा तस्य राजर्षेर्नष्टसंज्ञस्य भाषितम्}
{प्रत्युवाच शुभं वाक्यं सान्त्वयन्परया गिरा} %7-89-12

\twolineshloka
{अश्मवर्षेण महता भृत्यास्ते विनिपातिताः}
{त्वं चाश्रमपदे सुप्तो वातवर्षभयार्दितः} %7-89-13

\twolineshloka
{समाश्वसिहि भद्रं ते निर्भयो विगतज्वरः}
{फलमूलाशनो वीर निवसेह यथासुखम्} %7-89-14

\twolineshloka
{स राजा तेन वाक्येन प्रत्याश्वस्तो महामतिः}
{प्रत्युवाच ततो वाक्यं दीनो भृत्यक्षयाद् भृशम्} %7-89-15

\twolineshloka
{त्यक्ष्याम्यहं स्वकं राज्यं नाहं भृत्यैर्विनाकृतः}
{वर्तयेयं क्षणं ब्रह्मन्समनुज्ञातुमर्हसि} %7-89-16

\twolineshloka
{सुतो धर्मपरो ब्रह्मञ्ज्येष्ठो मम महायशाः}
{शशबिन्दुरिति ख्यातः स मे राज्यं प्रपत्स्यते} %7-89-17

\twolineshloka
{न हि शक्ष्याम्यहं हित्वा भृत्यदारान्सुखान्वितान्}
{प्रतिवक्तुं महातेजः किञ्चिदप्यशुभं वचः} %7-89-18

\twolineshloka
{तथा ब्रुवति राजेन्द्रे बुधः परममद्भुतम्}
{सान्त्वपूर्वमथोवाच वासस्त इह रोचताम्} %7-89-19

\twolineshloka
{न सन्तापस्त्वया कार्यः कार्दमेय महाबल}
{संवत्सरोषितस्येह कारयिष्यामि ते हितम्} %7-89-20

\twolineshloka
{तस्य तद्वचनं श्रुत्वा बुधस्याक्लिष्टकर्मणः}
{वासाय विदधे बुद्धिं यदुक्तं ब्रह्मवादिना} %7-89-21

\twolineshloka
{मासं स स्त्री तदा भूत्वा रमयत्यनिशं शुदा}
{मासं पुरुषभावेन धर्मबुद्धिं चकार सः} %7-89-22

\twolineshloka
{ततः सा नवमे मासि इला सोमसुतात्सुतम्}
{जनयामास सुश्रोणी पुरूरवसमूर्जितम्} %7-89-23

\twolineshloka
{जातमात्रं तु सुश्रोणी पितुर्हस्ते न्यवेशयत्}
{बुधस्य समवर्णभमिला पुत्रं महाबलम्} %7-89-24

\twolineshloka
{बुधस्तु पुरुषीभूतं स वै संवत्सरान्तरम्}
{कथाभी रमयामास धर्मयुक्ताभिरात्मवान्} %7-89-25


॥इत्यार्षे श्रीमद्रामायणे वाल्मीकीये आदिकाव्ये उत्तरकाण्डे पुरूरवोजननम् नाम एकोननवतितमः सर्गः ॥७-८९॥
