\sect{एकाशीतितमः सर्गः — दण्डशापः}

\twolineshloka
{स मुहूर्तादुपश्रुत्य देवर्षिरमितप्रभः}
{स्वमाश्रमं शिष्यवृतः क्षुधार्तः सन्न्यवर्तत} %7-81-1

\twolineshloka
{सोऽपश्यदरजां दीनां रजसा समभिप्लुताम्}
{ज्योत्स्नामिव ग्रहग्रस्तां प्रत्यूषे न विराजतीम्} %7-81-2

\twolineshloka
{तस्य रोषः समभवत्क्षुधार्तस्य विशेषतः}
{निर्दहन्निव लोकांस्त्रीञ्छिष्यांश्चैतदुवाच ह} %7-81-3

\twolineshloka
{पश्यध्वं विपरीतस्य दण्डस्याविजितात्मनः}
{विपत्तिं घोरसङ्काशां क्रुद्धामग्निशिखामिव} %7-81-4

\twolineshloka
{क्षयोऽस्य दुर्मतेः प्राप्तः सानुगस्य दुरात्मनः}
{यः प्रदीप्तां हुताशस्य शिखां वै स्प्रष्टुमिच्छति} %7-81-5

\twolineshloka
{यस्मात्स कृतवान्पापमीदृशं घोरसंहितम्}
{तस्मात्प्राप्स्यति दुर्मेधाः फलं पापस्य कर्मणः} %7-81-6

\twolineshloka
{सप्तरात्रेण राजासौ सभृत्यबलवाहनः}
{पापकर्मसमाचारो वधं प्राप्स्यति दुर्मतिः} %7-81-7

\twolineshloka
{समन्ताद्योजनशतं विषयं चास्य दुर्मतेः}
{धक्ष्यते पांसुवर्षेण महता पाकशासनः} %7-81-8

\twolineshloka
{सर्वसत्वानि यानीह स्थावराणि चराणि च}
{महता पांसुवर्षेण विलयं सर्वतोऽगमन्} %7-81-9

\twolineshloka
{दण्डस्य विषयो यावत्तावत्सर्वसमुच्छ्रयम्}
{पांसुवर्षमिवालक्ष्यं सप्तरात्रं भविष्यति} %7-81-10

\twolineshloka
{इत्युक्त्वा क्रोधताम्राक्षस्तदाश्रमनिवासिनम्}
{जनं जनपदान्तेषु स्थीयतामिति चाब्रवीत्} %7-81-11

\twolineshloka
{श्रुत्वा तूशनसो वाक्यं सोऽऽश्रमावसथो जनः}
{निष्क्रान्तो विषयात्तस्मात्स्थानं चक्रेऽथ बाह्यतः} %7-81-12

\twolineshloka
{स तथोक्त्वा मुनिजनमरजामिदमब्रवीत्}
{इहैव वस दुर्मेधे आश्रमे सुसमाहिता} %7-81-13

\twolineshloka
{इदं योजनपर्यन्तं सरः सुरुचिरप्रभम्}
{अरजे विज्वरा भुङ्क्ष्व कालश्चात्र प्रतीक्ष्यताम्} %7-81-14

\twolineshloka
{त्वत्समीपे च ये सत्त्वा वासमेष्यन्ति तां निशाम्}
{अवध्याः पांसुवर्षेण ते भविष्यन्ति नित्यदा} %7-81-15

\threelineshloka
{श्रुत्वा नियोगं ब्रह्मर्षेः साऽरजा भार्गवी तदा}
{तथेति पितरं प्राह भार्गवं भृशदुःखिता}
{इत्युक्त्वा भार्गवो वासमन्यत्र समकारयत्} %7-81-16

\twolineshloka
{तच्च राज्यं नरेन्द्रस्य सभृत्यबलवाहनम्}
{सप्ताहाद्भस्मसाद्भूतं यथोक्तं ब्रह्मवादिना} %7-81-17

\twolineshloka
{तस्यासौ दण्डविषयो विन्ध्यशैवलयोर्नृप}
{शप्तो ब्रह्मर्षिणा तेन वैधर्म्ये सहिते कृते} %7-81-18

\twolineshloka
{ततः प्रभृति काकुत्स्थ दण्डकारण्यमुच्यते}
{तपस्विनः स्थिता ह्यत्र जनस्थानमतोऽभवत्} %7-81-19

\twolineshloka
{एतत्ते सर्वमाख्यातं यन्मां पृच्छसि राघव}
{सन्ध्यामुपासितुं वीर समयो ह्यतिवर्तते} %7-81-20

\twolineshloka
{एते महर्षयः सर्वे पूर्णकुम्भाः समन्ततः}
{कृतोदका नरव्याघ्र आदित्यं पर्युपासते} %7-81-21

\twolineshloka
{स तैर्ब्राह्मणमभ्यस्तं सहितैर्ब्रह्मवित्तमैः}
{रविरस्तं गतो राम गच्छोदकमुपस्पृश} %7-81-22


॥इत्यार्षे श्रीमद्रामायणे वाल्मीकीये आदिकाव्ये उत्तरकाण्डे दण्डशापः नाम एकाशीतितमः सर्गः ॥७-८१॥
