\sect{प्रथमः सर्गः — रामप्रश्नः}

\twolineshloka
{प्राप्तराज्यस्य रामस्य राक्षसानां वधे कृते}
{आजग्मुर्ऋषयः सर्वे राघवं प्रतिनन्दितुम्} %7-1-1

\twolineshloka
{कौशिकोऽथ यवक्रीतो गार्ग्यो गालव एव च}
{कण्वो मेधातिथेः पुत्रः पूर्वस्यां दिशि ये श्रिताः} %7-1-2

\twolineshloka
{स्वस्त्यात्रेयोऽथ भगवान्नमुचिः प्रमुचिस्तथा}
{आजग्मुस्ते सहागस्त्या ये श्रिता दक्षिणां दिशम्} %7-1-3

\twolineshloka
{नृषद्गुः कवषो धौम्यो रौद्रेयश्च महानृषिः}
{तेऽप्याजग्मुः सशिष्या वै ये श्रिताः पश्चिमां दिशम्} %7-1-4

\twolineshloka
{वसिष्ठः कश्यपोऽथात्रिर्विश्वामित्रः सगौतमः}
{जमदग्निर्भरद्वाजस्तेऽपि सप्तर्षयस्तथा} %7-1-5

\onelineshloka
{उदीच्यां दिशि सप्तैते नित्यमेव निवासिनः} %7-1-6

\threelineshloka
{सम्प्राप्य ते महात्मानो राघवस्य निवेशनम्}
{विष्टिताः प्रतिहारार्थं हुताशनसमप्रभाः}
{वेदवेदाङ्गविदुषो नानाशास्त्रविशारदाः} %7-1-7

\twolineshloka
{द्वाःस्थं प्रोवाच धर्मात्मा अगस्त्यो मुनिसत्तमः}
{निवेद्यतां दाशरथेर्ऋषीनस्मान्समागतान्} %7-1-8

\twolineshloka
{प्रतीहारस्ततस्तूर्णमगस्त्यवचनाद्द्रुतम्}
{समीपं राघवस्याशु प्रविवेश महात्मनः} %7-1-9

\twolineshloka
{नयेङ्गितज्ञः सद्वृत्तो दक्षो धैर्यसमन्वितः}
{स रामं दृश्य सहसा पूर्णचन्द्रसमप्रभम्} %7-1-10

\onelineshloka
{अगस्त्यं कथयामास सम्प्राप्तमृषिभिः सह} %7-1-11

\twolineshloka
{श्रुत्वा प्राप्तान्मुनींस्तांस्तु बालसूर्यसमप्रभान्}
{प्रत्युवाच ततो द्वास्स्थं प्रवेशय यथासुखम्} %7-1-12

\twolineshloka
{तान् सम्प्राप्तान् मुनीन् दृष्ट्वा प्रत्युत्थाय कृताञ्जलिः}
{पाद्यार्घ्यादिभिरानर्च गां निवेद्य च सादरम्} %7-1-13

\twolineshloka
{रामोऽभिवाद्य प्रयत आसनान्यादिदेश ह}
{तेषु काञ्चनचित्रेषु महत्सु च वरेषु च} %7-1-14

\twolineshloka
{कुशान्तर्धानदत्तेषु मृगचर्मयुतेषु च}
{यथार्हमुपविष्टास्ते आसनेष्वृषिपुङ्गवाः} %7-1-15

\twolineshloka
{रामेण कुशलं पृष्टाः सशिष्याः सपुरोगमाः}
{महर्षयो वेदविदो रामं वचनमब्रुवन्} %7-1-16

\twolineshloka
{कुशलं नो महाबाहो सर्वत्र रघुनन्दन}
{त्वां तु दिष्ट्या कुशलिनं पश्यामो हतशात्रवम्} %7-1-17

\twolineshloka
{दिष्ट्या त्वया हतो राजन्रावणो लोकरावणः}
{न हि भारः स ते राम रावणः पुत्रपौत्रवान्} %7-1-18

\twolineshloka
{सधनुस्त्वं हि लोकांस्त्रीन्विजयेथा न संशयः}
{दिष्ट्या त्वया हतो राम रावणो राक्षसेश्वरः} %7-1-19

\twolineshloka
{दिष्ट्या विजयिनं त्वाद्य पश्यामः सह सीतया}
{लक्ष्मणेन च धर्मात्मन्भ्रात्रा त्वद्धितकारिणा} %7-1-20

\threelineshloka
{मातृभिर्भ्रातृसहितं पश्यामोऽद्य वयं नृप}
{दिष्ट्या प्रहस्तो विकटो विरूपाक्षो महोदरः}
{अकम्पनश्च दुर्धर्षो निहतास्ते निशाचराः} %7-1-21

\twolineshloka
{यस्य प्रमाणाद्विपुलं प्रमाणं नेह विद्यते}
{दिष्ट्या ते समरे राम कुम्भकर्णो निपातितः} %7-1-22

\twolineshloka
{त्रिशिराश्चातिकायश्च देवान्तकनरान्तकौ}
{दिष्ट्या ते निहता राम महावीर्या निशाचराः} %7-1-23

\twolineshloka
{कुम्भश्चैव निकुम्भश्च राक्षसौ भीमदर्शनौ}
{दिष्ट्या तौ निहतौ राम कुम्भकर्णसुतौ मृधे} %7-1-24

\twolineshloka
{युद्धोन्मत्तश्च मत्तश्च कालान्तकयमोपमौ}
{यज्ञकोपश्च बलवान्धूम्राक्षो नाम राक्षसः} %7-1-25

\twolineshloka
{कुर्वन्तः कदनं घोरमेते शस्त्रास्त्रपारगाः}
{अन्तकप्रतिमैर्बाणैर्दिष्ट्या विनिहतास्त्वया} %7-1-26

\twolineshloka
{दिष्ट्या त्वं राक्षसेन्द्रेण द्वन्द्वयुद्धमुपागतः}
{देवतानामवध्येन विजयं प्राप्तवानसि} %7-1-27

\twolineshloka
{सङ्ख्ये तस्य न किञ्चित्तु रावणस्य पराभवः}
{द्वन्द्वयुद्धमनुप्राप्तो दिष्ट्या ते रावणिर्हतः} %7-1-28

\twolineshloka
{दिष्ट्या तस्य महाबाहो कालस्येवाभिधावतः}
{मुक्तः सुररिपोर्वीर प्राप्तश्च विजयस्त्वया} %7-1-29

\twolineshloka
{अभिनन्दाम ते सर्वे संश्रुत्येन्द्रजितो वधम्}
{सोऽवध्यः सर्वभूतानां महामायाधरो युधि} %7-1-30

\onelineshloka
{विस्मयस्त्वेष चास्माकं तच्छ्रुत्वेन्द्रजितं हतम्} %7-1-31

\twolineshloka
{एते चान्ये च बहवो राक्षसाः कामरूपिणः}
{दिष्ट्या त्वया हता वीरा रघूणां कुलवर्द्धन} %7-1-32

\onelineshloka
{दत्त्वा पुण्यामिमां वीर सौम्यामभयदक्षिणाम् दिष्ट्या वर्धसि काकुत्स्थ जयेनामित्रकर्शन} %7-1-33

\twolineshloka
{श्रुत्वा तु तेषां वचनमृषीणां भावितात्मनाम्}
{विस्मयं परमं गत्वा रामः प्राञ्जलिरब्रवीत्} %7-1-34

\twolineshloka
{भगवन्तः कुम्भकर्णं रावणं च निशाचरम्}
{अतिक्रम्य महावीर्यौ किं प्रशंसथ रावणिम्} %7-1-35

\twolineshloka
{महोदरं प्रहस्तं च विरूपाक्षं च राक्षसम्}
{मत्तोन्मत्तौ च दुर्धर्षौ देवान्तकनरान्तकौ अतिक्रम्य महावीर्यान् किं प्रशंसथ रावणिम्} %7-1-36

\twolineshloka
{अतिकायं त्रिशिरसं धूम्राक्षं च निशाचरम्}
{अतिक्रम्य महावीर्यान्किं प्रशंसथ रावणिम्} %7-1-37

\twolineshloka
{कीदृशो वै प्रभावोऽस्य किं बलं कः पराक्रमः}
{केन वा कारणेनैष रावणादतिरिच्यते} %7-1-38

\twolineshloka
{शक्यं यदि मया श्रोतुं न खल्वाज्ञापयामि वः}
{यदि गुह्यं न चेद्वक्तुं श्रोतुमिच्छामि कथ्यताम्} %7-1-39

\twolineshloka
{शक्रोऽपि विजितस्तेन कथं लब्धवरश्च सः}
{कथं च बलवानन्पुत्रो न पिता तस्य रावणः} %7-1-40

\twolineshloka
{कथं पितुश्चाभ्यधिको महाहवे शक्रस्य जेता हि कथं स राक्षसः}
{वराश्च लब्धाः कथयस्व मेऽद्य तत्पृच्छतश्चास्य मुनीन्द्र सर्वम्} %7-1-41


॥इत्यार्षे श्रीमद्रामायणे वाल्मीकीये आदिकाव्ये उत्तरकाण्डे रामप्रश्नः नाम प्रथमः सर्गः ॥७-१॥
