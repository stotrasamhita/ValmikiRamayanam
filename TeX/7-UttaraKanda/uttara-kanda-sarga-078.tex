\sect{अष्टसप्ततितमः सर्गः — आभरणागमः}

\twolineshloka
{श्रुत्वा तु भाषितं वाक्यं मम राम शुभाक्षरम्}
{प्राञ्जलिः प्रत्युवाचेदं स स्वर्गी रघुनन्दन} %7-78-1

\twolineshloka
{शृणु ब्रह्मन्पुरा वृत्तं ममैतत्सुखदुःखयोः}
{अनतिक्रमणीयं हि यथा पृच्छसि मां द्विज} %7-78-2

\twolineshloka
{पुरा वैदर्भको राजा पिता मम महायशाः}
{सुदेव इति विख्यातस्त्रिषु लोकेषु वीर्यवान्} %7-78-3

\twolineshloka
{तस्य पुत्रद्वयं ब्रह्मन्द्वाभ्यां स्त्रीभ्यामजायत}
{अहं श्वेत इति ख्यातो यवीयान्सुरथोऽभवत्} %7-78-4

\twolineshloka
{ततः पितरि स्वर्याते पौरा मामभ्यषेचयन्}
{तत्राहं कृतवान्राज्यं धर्म्यं च सुसमाहितः} %7-78-5

\twolineshloka
{एवं वर्षसहस्राणि समतीतानि सुव्रत}
{राज्यं कारयतो ब्रह्मन्प्रजा धर्मेण रक्षतः} %7-78-6

\twolineshloka
{सोऽहं निमित्ते कस्मिंश्चिद्विज्ञातायुर्द्विजोत्तम}
{कालधर्मं हृदि न्यस्य ततो वनमुपागमम्} %7-78-7

\twolineshloka
{सोऽहं वनमिदं दुर्गं मृगपक्षिविवर्जितम्}
{तपश्चर्तुं प्रविष्टोऽस्मि समीपे सरसः शुभे} %7-78-8

\twolineshloka
{भ्रातरं सुरथं राज्ये ह्यभिषिच्य महीपतिम्}
{इदं सरः समासाद्य तपस्तप्तं मया चिरम्} %7-78-9

\twolineshloka
{सोऽहं वर्षसहस्राणि तपस्त्रीणि महावने}
{तप्त्वा सुदुष्करं प्राप्तो ब्रह्मलोकमनुत्तमम्} %7-78-10

\threelineshloka
{तस्य मे स्वर्गभूतस्य क्षुत्पिपासे द्विजोत्तम}
{बाधेते परमोदार ततोऽहं व्यथितेन्द्रियः}
{गत्वा त्रिभुवनश्रेष्ठं पितामहमुवाच ह} %7-78-11

\twolineshloka
{भगवन्ब्रह्मलोकोऽयं क्षुत्पिपासाविवर्जितः}
{कस्यायं कर्मणः पाकः क्षुत्पिपासानुगो ह्यहम्} %7-78-12

\onelineshloka
{आहारः कश्च मे देव तन्मे ब्रूहि पितामह} %7-78-13

\twolineshloka
{पितामहस्तु मामाह तवाहारः सुदेवज}
{स्वादूनि स्वानि मांसानि तानि भक्षय नित्यशः} %7-78-14

\twolineshloka
{स्वशरीरं त्वया पुष्टं कुर्वता तप उत्तमम्}
{अनुप्तं रोहते श्वेत न कदाचिन्महामते} %7-78-15

\threelineshloka
{तृप्तिर्न तेऽस्ति सूक्ष्माऽपि वने सत्वनिषेविते}
{पुरा तु भिक्षमाणाय भिक्षा वै यतये नृप}
{न हि दत्ता त्वयेन्द्राभ यस्मादतिथयेऽपि वै} %7-78-16

\twolineshloka
{दत्तं न तेऽस्ति सूक्ष्मोऽपि तप एव निषेवसे}
{तेन स्वर्गगतो वत्स बाध्यसे क्षुत्पिपासया} %7-78-17

\twolineshloka
{स त्वं सुपुष्टमाहारैः स्वशरीरमनुत्तमम्}
{भक्षयित्वामृतरसं तेन तृप्तिर्भविष्यति} %7-78-18

\twolineshloka
{यदा तु तद्वनं श्वेत अगस्त्यः सुमहानृषिः}
{आगमिष्यति दुर्धर्षस्तदा कृच्छ्राद्विमोक्ष्यते} %7-78-19

\twolineshloka
{स हि तारयितुं सौम्य शक्तः सुरगणानपि}
{किं पुनस्त्वां महाबाहो क्षुत्पिपासावशं गतम्} %7-78-20

\twolineshloka
{सोऽहं भगवतः श्रुत्वा देवदेवस्य निश्चयम्}
{आहारं गर्हितं स्वशरीरं द्विजोत्तम} %7-78-21

\twolineshloka
{बहून्वर्षगणान्ब्रह्मन्भुज्यमानमिदं मया}
{क्षयं नाभ्येति ब्रह्मर्षे तृप्तिश्चापि ममोत्तमा} %7-78-22

\twolineshloka
{तस्य मे कृच्छ्रभूतस्य कृच्छ्रादस्माद्विमोचय}
{अन्येषां न गतिर्ह्यत्र कुम्भयोनिमृते द्विजम्} %7-78-23

\twolineshloka
{इदमाभरणं सौम्य तारणार्थं द्विजोत्तम}
{प्रतिगृह्णीष्व भद्रं ते प्रसादं कर्तुमर्हसि} %7-78-24

\twolineshloka
{इदं तावत्सुवर्णं च धनं वस्त्राणि च द्विज}
{भक्ष्यं भोज्यं च ब्रह्मर्षे ददात्याभरणानि च} %7-78-25

\twolineshloka
{सर्वान्कामान्प्रयच्छामि भोगांश्च मुनिपुङ्गव}
{तारणे भगवन्मह्यं प्रसादं कर्तुमर्हसि} %7-78-26

\twolineshloka
{तस्याहं स्वर्गिणो वाक्यं श्रुत्वा दुःखसमन्वितम्}
{तारणायोपजग्राह तदाभरणमुत्तमम्} %7-78-27

\twolineshloka
{मया प्रतिगृहीते तु तस्मिन्नाभरणे शुभे}
{मानुषः पूर्वको देहो राजर्षेर्विननाश ह} %7-78-28

\twolineshloka
{प्रनष्टे तु शरीरेऽसौ राजर्षिः परया मुदा}
{तृप्तः प्रमुदितो राजा जगाम त्रिदिवं सुखम्} %7-78-29

\twolineshloka
{तेनेदं शक्रतुल्येन दिव्यमाभरणं मम}
{तस्मिन्निमित्ते काकुत्स्थ दत्तमद्भुतदर्शनम्} %7-78-30


॥इत्यार्षे श्रीमद्रामायणे वाल्मीकीये आदिकाव्ये उत्तरकाण्डे आभरणागमः नाम अष्टसप्ततितमः सर्गः ॥७-७८॥
