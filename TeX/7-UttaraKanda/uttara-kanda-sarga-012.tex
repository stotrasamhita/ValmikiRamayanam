\sect{द्वादशः सर्गः — रावणादिविवाहः}

\twolineshloka
{राक्षसेन्द्रोऽभिषिक्तस्तु भ्रातृभ्यां सहितस्तदा}
{ततः प्रदानं राक्षस्या भगिन्याः समचिन्तयत्} %7-12-1

\twolineshloka
{स्वसारं कालकेयाय दानवेन्द्राय राक्षसीम्}
{ददौ शूर्पणखां नाम विद्युज्जिह्वाय नामतः} %7-12-2

\twolineshloka
{अथ दत्त्वा स्वयं रक्षो मृगयामटते स्म तत्}
{तत्रापश्यत्ततो राम मयं नाम दितेः सुतम्} %7-12-3

\threelineshloka
{कन्यासहायं तं दृष्ट्वा दशग्रीवो निशाचरः}
{अपृच्छत्को भवानेको निर्मनुष्यमृगे वने}
{अनया मृगशावाक्ष्या किमर्थं सह तिष्ठसि} %7-12-4

\twolineshloka
{मयस्तथाब्रवीद्राम पृच्छन्तं तं निशाचरम्}
{श्रूयतां सर्वमाख्यास्ये यथावृत्तमिदं मम} %7-12-5

\twolineshloka
{हेमा नामाप्सरास्तात श्रुतपूर्वा यदि त्वया}
{दैवतैर्मम सा दत्ता पौलोमीव शतक्रतोः} %7-12-6

\twolineshloka
{तस्यां सक्तमनास्तात पञ्चवर्षशतान्यहम्}
{सा च दैवतकार्येण त्रयोदश समा गताः} %7-12-7

\twolineshloka
{वर्षं चतुर्दशं चैव ततो हेममयं पुरम्}
{वज्रवैडूर्यचित्रं च मायया निर्मितं मया} %7-12-8

\onelineshloka
{तत्राहमवसं दीनस्तया हीनः सुदुःखितः} %7-12-9

\twolineshloka
{तस्मात्पुराद्दुहितरं गृहीत्वा वनमागतः}
{इयं ममात्मजा राजंस्तस्याः कुक्षौ विवर्धिता} %7-12-10

\twolineshloka
{भर्तारमनया सार्धमस्याः प्राप्तोऽस्मि मार्गितुम्}
{कन्यापितृत्वं दुःखं हि सर्वेषां मानकाङ्क्षिणाम्} %7-12-11

\twolineshloka
{कन्या हि द्वे कुले नित्यं संशये स्थाप्य तिष्ठति}
{पुत्रद्वयं ममाप्यस्यां भार्यायां सम्बभूव ह} %7-12-12

\twolineshloka
{मायावी प्रथमस्तात दुन्दुभिस्तदनन्तरः}
{एवं ते सर्वमाख्यातं याथातथ्येन पृच्छतः} %7-12-13

\twolineshloka
{त्वामिदानीं कथं तात जानीयां को भवानिति}
{एवमुक्तस्तु तद्रक्षो विनीतमिदमब्रवीत्} %7-12-14

\twolineshloka
{अहं पौलस्त्यतनयो दशग्रीवश्च नामतः}
{मुनेर्विश्रवसो यस्तु तृतीयो ब्रह्मणोऽभवत्} %7-12-15

\threelineshloka
{एवमुक्तस्तदा राम राक्षसेन्द्रेण दानवः}
{महर्षेस्तनयं ज्ञात्वा मयो हर्षमुपागतः}
{दातुं दुहितरं तस्मै रोचयामास यत्र वै} %7-12-16

\twolineshloka
{करेण तु करं तस्या ग्राहयित्वा मयस्तदा}
{प्रहसन्प्राह दैत्येन्द्रो राक्षसेन्द्रमिदं वचः} %7-12-17

\twolineshloka
{इयं ममात्मजा राजन्हेमयाऽप्सरसा धृता}
{कन्या मन्दोदरी नाम पत्न्यर्थं प्रतिगृह्यताम्} %7-12-18

\twolineshloka
{बाढमित्येव तं राम दशग्रीवोऽभ्यभाषत}
{प्रज्चाल्य तत्र चैवाग्निमकरोत्पाणिसङ्ग्रहम्} %7-12-19

\twolineshloka
{स हि तस्य मयो राम शापाभिज्ञस्तपोधनात्}
{विदित्वा तेन सा दत्ता तस्य पैतामहं कुलम्} %7-12-20

\twolineshloka
{अमोघां तस्य शक्तिं च प्रददौ परमाद्भुताम्}
{परेण तपसा लब्धां जघ्निवाँल्लक्ष्मणं यया} %7-12-21

\twolineshloka
{एवं सीकृतदारो वै लङ्काया ईश्वरः प्रभुः}
{गत्वा तु नगरीं भार्ये भ्रातृभ्यां समुपाहरत्} %7-12-22

\twolineshloka
{वैरोचनस्य दौहित्रीं वज्रज्वालेति नामतः}
{तां भार्यां कुम्भकर्णस्य रावणः समकल्पयत्} %7-12-23

\twolineshloka
{गन्धर्वराजस्य सुतां शैलूषस्य महात्मनः}
{सरमां नाम धर्मज्ञां लेभे भार्यां विभीषणः} %7-12-24

\twolineshloka
{तीरे तु सरसो वै तु सञ्जज्ञे मानसस्य हि}
{सरस्तदा मानसं तु ववृधे जलदागमे} %7-12-25

\twolineshloka
{मात्रा तु तस्याः कन्यायाः स्नेहेनाक्रन्दितं वचः}
{सरो मा वर्धयस्वेति ततः सा सरमाऽभवत्} %7-12-26

\twolineshloka
{एवं ते कृतदारा वै रेमिरे तत्र राक्षसाः}
{स्वां स्वां भार्यामुपागम्य गन्धर्वा इव नन्दने} %7-12-27

\twolineshloka
{ततो मन्दोदरी पुत्रं मेघनादमजीजनत्}
{स एष इन्द्रजिन्नाम युष्माभिरभिधीयते} %7-12-28

\twolineshloka
{जातमात्रेण हि पुरा तेन रावणसूनुना}
{रुदता सुमहान्मुक्तो नादो जलधरोपमः} %7-12-29

\twolineshloka
{जडीकृता च सा लङ्का तस्य नादेन राघव}
{पिता तस्याकरोन्नाम मेघनाद इति स्वयम्} %7-12-30

\twolineshloka
{सोऽवर्धत तदा राम रावणान्तःपुरे शुभे}
{रक्ष्यमाणो वरस्त्रीभिश्छन्नः काष्ठैरिवानलः} %7-12-31

\onelineshloka
{मातापित्रोर्महाहर्षं जनयन्रावणात्मजः} %7-12-32


॥इत्यार्षे श्रीमद्रामायणे वाल्मीकीये आदिकाव्ये उत्तरकाण्डे रावणादिविवाहः नाम द्वादशः सर्गः ॥७-१२॥
