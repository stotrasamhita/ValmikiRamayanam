\sect{दशमः सर्गः — रावणादिवरदानम्}

\twolineshloka
{अथाब्रवीन्मुनिं रामः कथं ते भ्रातरो वने}
{कीदृशं तु तदा ब्रह्मंस्तपस्तेपुर्महाबलाः} %7-10-1

\twolineshloka
{अगस्त्यस्त्वब्रवीत्तत्र रामं सुप्रीतमानसम्}
{तांस्तान्धर्मविधींस्तत्र भ्रातरस्ते समाविशन्} %7-10-2

\twolineshloka
{कुम्भकर्णस्ततो यत्तो नित्यं धर्मपथे स्थितः}
{तताप ग्रीष्मकाले तु पञ्चाग्नीन्परितः स्थितः} %7-10-3

\twolineshloka
{मेघाम्बुसिक्तो वर्षासु वीरासनमसेवत}
{नित्यं च शिशिरे काले जलमध्यप्रतिश्रयः} %7-10-4

\twolineshloka
{एवं वर्षसहस्राणि दश तस्यातिचक्रमुः}
{धर्मे प्रयतमानस्य सत्पथे निष्ठितस्य च} %7-10-5

\twolineshloka
{विभीषणस्तु धर्मात्मा नित्यं धर्मपरः शुचिः}
{पञ्चवर्षसहस्राणि पादेनैकेन तस्थिवान्} %7-10-6

\twolineshloka
{समाप्ते नियमे तस्य ननृतुश्चाप्सरोगणाः}
{पपात पुष्पवर्षं च क्षुभिताश्चापि देवताः} %7-10-7

\twolineshloka
{पञ्चवर्षसहस्राणि सूर्यं चैवान्ववर्तत}
{तस्थौ चोर्ध्वशिरोबाहुः स्वाध्यायधृतमानसः} %7-10-8

\twolineshloka
{एवं विभीषणस्यापि स्वर्गस्थस्येव नन्दने}
{दशवर्षसहस्राणि गतानि नियतात्मनः} %7-10-9

\twolineshloka
{दशवर्षसहस्रं तु निराहारो दशाननः}
{पूर्णे वर्षसहस्रे तु शिरश्चाग्नौ जुहाव सः} %7-10-10

\twolineshloka
{एवं वर्षसहस्राणि नव तस्यातिचक्रमुः}
{शिरांसि नव चाप्यस्य प्रविष्टानि हुताशनम्} %7-10-11

\twolineshloka
{अथ वर्षसहस्रे तु दशमे दशमं शिरः}
{छेत्तुकामे दशग्रीवे प्राप्तस्तत्र पितामहः} %7-10-12

\twolineshloka
{पितामहस्तु सुप्रीतः सार्धं देवैरुपस्थितः}
{तव तावद्दशग्रीव प्रीतोऽस्मीत्यभ्यभाषत} %7-10-13

\twolineshloka
{शीघ्रं वरय धर्मज्ञ वरो यस्तेऽभिकाङ्क्षितः}
{कं ते कामं करोम्यद्य न वृथा ते परिश्रमः} %7-10-14

\twolineshloka
{अथाब्रवीदृशग्रीवः प्रहृष्टेनान्तरात्मना}
{प्रणम्य शिरसा देवं हर्षगद्गदया गिरा} %7-10-15

\twolineshloka
{भगवन्प्राणिनां नित्यं नान्यत्र मरणाद्भयम्}
{नास्ति मृत्युसमः शत्रुरमरत्वमहं वृणे} %7-10-16

\twolineshloka
{एवमुक्तस्तदा ब्रह्मा दशग्रीवमुवाच ह}
{नास्ति सर्वामरत्वं ते वरमन्यं वृणीष्व मे} %7-10-17

\twolineshloka
{एवमुक्ते तदा राम ब्रह्मणा लोककर्तृणा}
{दशग्रीव उवाचेदं कृताञ्जलिरथाग्रतः} %7-10-18

\twolineshloka
{सुपर्णनागयक्षाणां दैत्यदानवरक्षसाम्}
{अवध्योऽहं प्रजाध्यक्ष देवतानां च शाश्वत} %7-10-19

\twolineshloka
{नहि चिन्ता ममान्येषु प्राणिष्वमरपूजित}
{तृणभूता हि ते मन्ये प्राणिनो मानुषादयः} %7-10-20

\twolineshloka
{एवमुक्तस्तु धर्मात्मा दशग्रीवेण रक्षसा}
{उवाच वचनं देवः सह देवैः पितामहः} %7-10-21

\twolineshloka
{भविष्यत्येवमेतत्ते वचो राक्षसपुङ्गव}
{एवमुक्त्वा तु तं राम दशग्रीवं पितामहः} %7-10-22

\twolineshloka
{शृणु चापि वरो भूयः प्रीतस्येह शुभो मम}
{हुतानि यानि शीर्षाणि पूर्वमग्नौ त्वयानघ} %7-10-23

\twolineshloka
{पुनस्तानि भविष्यन्ति तथैव तव राक्षस}
{वितरामीह ते सौम्य वरं चान्यं दुरासदम्} %7-10-24

\twolineshloka
{छन्दतस्तव रूपं च मनसा यद्यथेप्सितम्}
{भविष्यति न सन्देहो मद्वरात्तवराक्षस} %7-10-25

\twolineshloka
{एवं पितामहोक्तस्य दशग्रीवस्य रक्षसः}
{अग्नौ हुतानि शीर्षाणि पुनस्तान्युत्थितानि वै} %7-10-26

\twolineshloka
{एवमुक्त्वा तु तं राम दशग्रीवं पितामहः}
{विभीषणमथोवाच वाक्यं लोकपितामहः} %7-10-27

\twolineshloka
{विभीषण त्वया वत्स धर्मसंहितबुद्धिना}
{परितुष्टोऽस्मि धर्मात्मन्वरं वरय सुव्रत} %7-10-28

\twolineshloka
{विभीषणस्तु धर्मात्मा वचनं प्राह साञ्जलिः}
{वृतः सर्वगुणैर्नित्यं चन्द्रमा रश्मिभिर्यथा} %7-10-29

\twolineshloka
{भगवन्कृतकृत्योऽहं यन्मे लोकगुरुः स्वयम्}
{प्रीतेन यदि दातव्यो वरो मे शृणु सुव्रत} %7-10-30

\twolineshloka
{परमापद्गतस्यापि धर्मे मम मतिर्भवेत्}
{अशिक्षितं च ब्रह्मास्त्रं भगवन्प्रतिभातु मे} %7-10-31

\twolineshloka
{या या मे जायते बुद्धिर्येषु येष्वाश्रमेषु च}
{सा सा भवतु धर्मिष्ठा तं तु धर्मं च पालये} %7-10-32

\twolineshloka
{एष मे परमोदार वरः परमको मतः}
{नहि धर्माभिरक्तानां लोके किञ्चन दुर्लभम्} %7-10-33

\twolineshloka
{पुनः प्रजापतिः प्रीतो विभीषणमुवाच ह}
{धर्मिष्ठस्त्वं यथा वत्स तथा चैतद्भविष्यति} %7-10-34

\twolineshloka
{यस्माद्राक्षसयोनौ ते जातस्यामित्रनाशन}
{नाधर्मे जायते बुद्धिरमरत्वं ददामि ते} %7-10-35

\twolineshloka
{इत्युक्त्वा कुम्भकर्णाय वरं दातुमुपस्थितम्}
{प्रजापतिं सुराः सर्वे वाक्यं प्राञ्जलयोऽब्रुवन्} %7-10-36

\twolineshloka
{न तावत्कुम्भकर्णाय प्रदातव्यो वरस्त्वया}
{जानीषे हि यथा लोकांस्त्रासयत्येष दुर्मतिः} %7-10-37

\twolineshloka
{नन्दनेऽप्सरसः सप्त महेन्द्रानुचरा दश}
{अनेन भक्षिता ब्रह्मन्नृषयो मानुषास्तथा} %7-10-38

\twolineshloka
{अलब्धवरपूर्वेण यत्कृतं राक्षसेन तु}
{तदेष वरलब्धः स्याद्भक्षयेद्भुवनत्रयम्} %7-10-39

\twolineshloka
{वरव्याजेन मोहोऽस्मै दीयताममितप्रभ}
{लोकानां स्वस्ति चैवं स्याद्भवेदस्य च सन्नतिः} %7-10-40

\twolineshloka
{एवमुक्तः सुरैर्ब्रह्माऽचिन्तयत् पद्मसम्भवः}
{चिन्तिता चोपतस्थेऽस्य पार्श्वं देवी सरस्वती} %7-10-41

\twolineshloka
{प्राञ्जलिः सा तु पार्श्वस्था प्राह वाक्यं सरस्वती}
{इयमस्म्यागता देव किं कार्यं करवाण्यहम्} %7-10-42

\twolineshloka
{प्रजापतिस्तुं तां प्राप्तां प्राह वाक्यं सरस्वतीम्}
{वाणि त्वं राक्षसेन्द्रस्य भव या देवतेप्सिता} %7-10-43

\twolineshloka
{तथेत्युक्त्वा प्रविष्टा सा प्रजापतिरथाब्रवीत्}
{कुम्भकर्ण महाबाहो वरं वरय यो मतः} %7-10-44

\twolineshloka
{कुम्भकर्णस्तु तद्वाक्यं श्रुत्वा वचनमब्रवीत्}
{स्वप्तुं वर्षाण्यनेकानि देवदेव ममेप्सितम्} %7-10-45

\twolineshloka
{एवमस्त्विति तं चोक्त्वा प्रायाद्ब्रह्मा सुरैः समम्}
{देवी सरस्वती चैव राक्षसं तं जहौ पुनः} %7-10-46

\twolineshloka
{ब्रह्मणा सह देवेषु गतेषु च नभःस्थलम्}
{विमुक्तोऽसौ सरस्वत्या स्वां संज्ञां च ततो गतः} %7-10-47

\twolineshloka
{कुम्भकर्णस्तु दुष्टात्मा चिन्तयामास दुःखितः}
{ईदृशं किमिदं वाक्यं ममाद्य वदनाच्च्युतम्} %7-10-48

\threelineshloka
{अहं व्यामोहितो देवैरिति मन्ये तदाऽऽगतैः}
{एवं लब्धवराः सर्वे भ्रातरो दीप्ततेजसः}
{श्लेष्मातकवनं गत्वा तत्र ते न्यवसन्सुखम्} %7-10-49


॥इत्यार्षे श्रीमद्रामायणे वाल्मीकीये आदिकाव्ये उत्तरकाण्डे रावणादिवरदानम् नाम दशमः सर्गः ॥७-१०॥
