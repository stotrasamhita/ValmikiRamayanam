\sect{अष्टाधिकशततमः सर्गः — विभीषणाद्यादेशः}

\twolineshloka
{ते दूता रामवाक्येन चोदिता लघुविक्रमाः}
{प्रजग्मुर्मधुरां शीघ्रं चक्रुर्वासं न चाध्वनि} %7-108-1

\twolineshloka
{ते तु त्रिभिरहोरात्रैः सम्प्राप्य मधुरामथ}
{शत्रुघ्नाय यथातत्त्वमाचख्युः सर्वमेव तत्} %7-108-2

\twolineshloka
{लक्ष्मणस्य परित्यागं प्रतिज्ञां राघवस्य च}
{पुत्रयोरभिषेकं च पौरानुगमनं तथा} %7-108-3

\threelineshloka
{कुशस्य नगरी रम्या विन्ध्यपर्वतरोधसि}
{कुशावतीति नाम्ना सा कृता रामेण धीमता}
{श्रावस्तीति पुरी रम्या श्राविता च लवस्य ह} %7-108-4

\twolineshloka
{अयोध्यां विजनां कृत्वा राघवो भरतस्तथा}
{स्वर्गस्य गमनोद्योगं कृतवन्तौ महारथौ} %7-108-5

\twolineshloka
{एवं सर्वं निवेद्याशु शत्रुघ्नाय महात्मने}
{विरेमुस्ते ततो दूतास्त्वर राजेति चाब्रुवन्} %7-108-6

\twolineshloka
{तच्छ्रुत्वा घोरसङ्काशं कुलक्षयमुपस्थितम्}
{प्रकृतीस्तु समानीय काञ्चनं च पुरोधसम्} %7-108-7

\twolineshloka
{तेषां सर्वं यथावृत्तमब्रवीद्रघनन्दनः}
{आत्मनश्च विपर्यासं भविष्यं भ्रातृभिः सह} %7-108-8

\twolineshloka
{ततः पुत्रद्वयं वीरः सोऽभ्यषिञ्चन्नराधिपः}
{सुबाहुर्मधुरां लेभे शत्रुघाती च वैदिशम्} %7-108-9

\twolineshloka
{द्विधा कृत्वा तु तां सेनां माधुरीं पुत्रयोर्द्वयोः}
{धनं च युक्तं कृत्वा वै स्थापयामास पार्थिवः} %7-108-10

\twolineshloka
{सुबाहुं मधुरायां च वैदेशे शत्रुघातिनम्}
{ययौ स्थाप्य तदाऽयोध्यां रथेनैकेन राघवः} %7-108-11

\twolineshloka
{स ददर्श महात्मानं ज्वलन्तमिव पावकम्}
{सूक्ष्मक्षौमाम्बरधरं मुनिभिः सार्धमक्षयैः} %7-108-12

\twolineshloka
{सोऽभिवाद्य ततो रामं प्राञ्जलिः प्रयतेन्द्रियः}
{उवाच वाक्यं धर्मज्ञं धर्ममेवानुचिन्तयन्} %7-108-13

\twolineshloka
{कृत्वाभिषेकं सुतयोर्द्वयो राघवनन्दन}
{तवानुगमने राजन्विद्धि मां कृतनिश्चयम्} %7-108-14

\twolineshloka
{न चान्यदपि वक्तव्यमतो वीर न शासनम्}
{विलोक्यमानमिच्छामि मद्विधेन विशेषतः} %7-108-15

\twolineshloka
{तस्य तां बुद्धिमक्लीबां विज्ञाय रघुनन्दनः}
{बाढमित्येव शत्रुघ्नं रामो वाक्यमुवाच ह} %7-108-16

\twolineshloka
{तस्य वाक्यस्य वाक्यान्ते वानराः कामरूपिणः}
{ऋक्षराक्षससङ्घाश्च समापेतुरनेकशः} %7-108-17

\twolineshloka
{सुग्रीवं ते पुरस्कृत्य सर्व एव समागताः}
{तं रामं द्रष्टुमनसः स्वर्गायाभिमुखं स्थितम्} %7-108-18

\twolineshloka
{देवपुत्रा ऋषिसुता गन्धर्वाणां सुतास्तथा}
{रामक्षयं विदित्वा ते सर्व एव समागताः} %7-108-19

\twolineshloka
{ते राममभिवाद्योचुः सर्वे वानरराक्षसाः}
{तवानुगमने राजन्सम्प्राप्ताः कृतनिश्चयाः} %7-108-20

\twolineshloka
{यदि राम विनास्माभिर्गच्छेस्त्वं पुरुषोत्तम}
{यमदण्डमिवोद्यम्य त्वया स्म विनिपातिताः} %7-108-21

\onelineshloka
{तैरेवमुक्तः काकुत्स्थो बाढमित्यब्रवीत् स्मयन्} %7-108-22

\twolineshloka
{एतस्मिन्नन्तरे रामं सुग्रीवोऽपि महाबलः}
{प्रणम्य विधिवद्वीरं विज्ञापयितुमुद्यतः} %7-108-23

\twolineshloka
{अभिषिच्याङ्गदं वीरमागतोऽस्मि नरेश्वर}
{तवानुगमने राजन्विद्धि मां कृतनिश्चयम्} %7-108-24

\twolineshloka
{तस्य तद्वचनं श्रुत्वा रामो रमयतां वरः}
{वानरेन्द्रमथोवाचं मैत्रं तस्यानुचिन्तयन्} %7-108-25

\twolineshloka
{सखे शृणुष्व सुग्रीव न त्वयाऽहं विनाकृतः}
{गच्छेयं देवलोकं वा परमं वा पदं महत्} %7-108-26

\threelineshloka
{बिभीषणमथोवाच राक्षसेन्द्रं महायशाः}
{यावत्प्रजा धरिष्यन्ति तावत्त्वं वै बिभीषण}
{राक्षसेन्द्र महावीर्य लङ्कास्थस्त्वं धरिष्यसि} %7-108-27

\twolineshloka
{यावच्चन्द्रश्च सूर्यश्च यावत्तिष्ठति मेदिनी}
{यावच्च मत्कथा लोके तावद्राज्यं तवास्त्विह} %7-108-28

\twolineshloka
{शासितस्त्वं सखित्वेन कार्यं ते मम शासनम्}
{प्रजाः संरक्ष धर्मेण नोत्तरं वक्तुमर्हसि} %7-108-29

\onelineshloka
{किञ्चान्यद्वक्तुमिच्छामि राक्षसेन्द्र महामते} %7-108-30

\twolineshloka
{आराधय जगन्नाथमिक्ष्वाकुकुलदैवतम्}
{आराधनीयमनिशं सर्वैर्दैवैः सवासवैः} %7-108-31

\twolineshloka
{तथेति प्रतिजग्राह रामवाक्यं विभीषणः}
{राजा राक्षसमुख्यानां राघवाज्ञामनुस्मरन्} %7-108-32

\twolineshloka
{तमेवमुक्त्वा काकुत्स्थो हनूमन्तमथाब्रवीत्}
{जीविते कृतबुद्धिस्त्वं मा प्रतिज्ञां विलोपय} %7-108-33

\twolineshloka
{मत्कथाः प्रचरिष्यन्ति यावल्लोके हरीश्वर}
{तावद्रमस्व सुप्रीतो मद्वाक्यमनुपालयन्} %7-108-34

\twolineshloka
{एवमुक्तस्तु हनुमान्राघवेण महात्मना}
{वाक्यं विज्ञापयामास परं हर्षमवाप्य च} %7-108-35

\twolineshloka
{यावत्तव कथा लोके विचरिष्यति पावनी}
{तावत्स्थास्यामि मेदिन्यां तवाज्ञामनुपालयन्} %7-108-36

\threelineshloka
{जाम्बवन्तं तथोक्त्वा तु वृद्धं ब्रह्मसुतं तथा}
{मैन्दं च द्विविदं चैव पञ्च जाम्बवता सह}
{यावत्कलिश्च सम्प्राप्तस्तावज्जीवत सर्वदा} %7-108-37

\twolineshloka
{तानेवमुक्त्वा काकुत्स्थः सर्वांस्तानृक्षवानरान्}
{उवाच बाढं गच्छध्वं मया सार्धं यथेप्सितम्} %7-108-38


॥इत्यार्षे श्रीमद्रामायणे वाल्मीकीये आदिकाव्ये उत्तरकाण्डे विभीषणाद्यादेशः नाम अष्टाधिकशततमः सर्गः ॥७-१०८॥
