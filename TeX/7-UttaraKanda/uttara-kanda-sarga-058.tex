\sect{अष्टपञ्चाशः सर्गः — ययातिशापः}

\twolineshloka
{एवं ब्रुवति रामे तु लक्ष्मणः परवीरहा}
{प्रत्युवाच महात्मानं ज्वलन्तमिव तेजसा} %7-58-1

\twolineshloka
{महदद्भुतमाश्चर्यं विदेहस्य पुरातनम्}
{निर्वृत्तं राजशार्दूल वसिष्ठस्य निमेस्सह} %7-58-2

\twolineshloka
{निमिस्तु क्षत्रियः शूरो विशेषेण च दीक्षितः}
{न क्षमां कृतवान्राजा वसिष्ठस्य महात्मनः} %7-58-3

\twolineshloka
{एवमुक्तस्तु तेनायं श्रीमान् क्षत्रियपुङ्गवः}
{उवाच लक्ष्मणं वाक्यं सर्वशास्त्रविशारदम्} %7-58-4

\twolineshloka
{रामो रमयतां श्रेष्ठो भ्रातरं दीप्ततेजसम्}
{न सर्वत्र क्षमा वीर पुरुषेषु प्रदृश्यते} %7-58-5

\twolineshloka
{सौमित्रे दुःसहो रोषो यथा क्षान्तो ययातिना}
{सत्त्वानुगं पुरस्कृत्य तन्निबोध समाहितः} %7-58-6

\twolineshloka
{नहुषस्य सुतो राजा ययातिः पौरवर्धनः}
{तस्य भार्याद्वयं सौम्य रूपेणाप्रतिमं भुवि} %7-58-7

\twolineshloka
{एका तु तस्य राजर्षेर्नाहुषस्य पुरस्कृता}
{शर्मिष्ठा नाम दैतेयी दुहिता वृषपर्वणः} %7-58-8

\twolineshloka
{अन्या तूशनसः पत्नी ययातेः पुरुषर्षभ}
{न तु सा दयिता राज्ञो देवयानी सुमध्यमा} %7-58-9

\twolineshloka
{तयोः पुत्रौ तु सम्भूतौ रूपवन्तौ समाहितौ}
{शर्मिष्ठाऽजनयत्पूरुं देवयानी यदुं तदा} %7-58-10

\twolineshloka
{पूरुस्तु दयितो राज्ञो गुणैर्मातृकृतेन च}
{ततो दुःखसमाविष्टो यदुर्मातरमब्रवीत्} %7-58-11

\twolineshloka
{भार्गवस्य कुले जाता देवस्याक्लिष्टकर्मणः}
{सहसे हृद्गतं दुःखमवमानं च दुःसहम्} %7-58-12

\twolineshloka
{आवां च सहितौ देवि प्रविशाव हुताशनम्}
{राजा तु रमतां सार्धं दैत्यपुत्र्या बहुक्षपाः} %7-58-13

\twolineshloka
{यदि वा सहनीयं ते मामनुज्ञातुमर्हसि}
{क्षम त्वं न क्षमिष्येऽहं मरिष्यामि न संशयः} %7-58-14

\twolineshloka
{पुत्रस्य भाषितं श्रुत्वा परमार्तस्य रोदतः}
{देवयानी तु सङ्क्रुद्धा सस्मार पितरं तदा} %7-58-15

\twolineshloka
{इङ्गितं तदभिज्ञाय दुहितुर्भार्गवस्तदा}
{आगतस्त्वरितं तत्र देवयानी तु यत्र सा} %7-58-16

\twolineshloka
{दृष्ट्वा चाप्रकृतिस्थां तामप्रहृष्टामचेतनाम्}
{पिता दुहितरं वाक्यं किमेतदिति चाब्रवीत्} %7-58-17

\twolineshloka
{पृच्छन्तमसकृत्तं वै भार्गवं दीप्ततेजसम्}
{देवयानी तु सङ्क्रुद्धा पितरं वाक्यमब्रवीत्} %7-58-18

\twolineshloka
{अहमग्निं विषं तीक्ष्णमपो वा मुनिसत्तम}
{भक्षयिष्ये प्रवेक्ष्यामि न तु शक्ष्यामि जीवितुम्} %7-58-19

\twolineshloka
{न मां त्वमवजानीषे दुःखितामवमानिताम्}
{वृक्षस्यावज्ञया ब्रह्मंश्छिद्यन्ते वृक्षजीविनः} %7-58-20

\twolineshloka
{अवज्ञया च राजर्षिः परिभूय च भार्गव}
{मय्यवज्ञां प्रयुङ्क्ते हि न च मां बहु मन्यते} %7-58-21

\twolineshloka
{तस्यास्तद्वचनं श्रुत्वा कोपेनाभिपरिप्लुतः}
{व्याहर्तुमुपचक्राम भार्गवो नहुषात्मजम्} %7-58-22

\twolineshloka
{यस्मान्मामवजानीषे नाहुष त्वं दुरात्मवान्}
{जरया परया जीर्णः शैथिल्यमुपयास्यसि} %7-58-23

\twolineshloka
{एवमुक्त्वा दुहितरं समाश्वास्य च भार्गवः}
{पुनर्जगाम ब्रह्मर्षिर्भवनं स्वं महायशाः} %7-58-24

\twolineshloka
{स एवमुक्त्वा द्विजपुङ्गवाग्र्यः सुतां समाश्वास्य च देवयानीम्}
{पुनर्ययौ सूर्यसमानतेजा दत्त्वा च शापं नहुषात्मजाय} %7-58-25


॥इत्यार्षे श्रीमद्रामायणे वाल्मीकीये आदिकाव्ये उत्तरकाण्डे ययातिशापः नाम अष्टपञ्चाशः सर्गः ॥७-५८॥
