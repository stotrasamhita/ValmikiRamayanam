\sect{पञ्चचत्वारिंशः सर्गः — सीतासमुत्सर्गादेशः}

\twolineshloka
{तेषां समुपविष्टानां सर्वेषां दीनचेतसाम्}
{उवाच वाक्यं काकुत्स्थो मुखेन परिशुष्यता} %7-45-1

\twolineshloka
{सर्वे शृणुत भद्रं वो मा कुरुध्वं मनोऽन्यथा}
{पौराणां मम सीतायां यादृशी वर्तते कथा} %7-45-2

\twolineshloka
{पौरापवादः सुमहांस्तथा जनपदस्य च}
{वर्तते मयि बीभत्सा सा मे मर्माणि कृन्तति} %7-45-3

\twolineshloka
{अहं किल कुले जात इक्ष्वाकूणां महात्मनाम्}
{सीताऽपि सत्कुले जाता जनकानां महात्मनाम्} %7-45-4

\twolineshloka
{जानासि त्वं यथा सौम्य दण्डके विजने वने}
{रावणेन हृता सीता स च विध्वंसितो मया} %7-45-5

\twolineshloka
{तत्र मे बुद्धिरुत्पन्ना जनकस्य सुतां प्रति}
{अत्रोषितामिमां सीतामानयेयं कथं पुरीम्} %7-45-6

\threelineshloka
{प्रत्ययार्थं ततः सीता विवेश ज्वलनं तदा}
{प्रत्यक्षं तव सौमित्रे देवानां हव्यवाहनः}
{अपापां मैथिलीमाह वायुश्चाकाशगोचरः} %7-45-7

\twolineshloka
{चन्द्रादित्यौ च शंसेते सुराणां सन्निधौ पुरा}
{ऋषीणां चैव सर्वेषामपापां जनकात्मजाम्} %7-45-8

\twolineshloka
{एवं शुद्धसमाचारा देवगन्धर्वसन्निधौ}
{लङ्काद्वीपे महेन्द्रेण मम हस्ते निवेशिता} %7-45-9

\twolineshloka
{अन्तरात्मा च मे वेत्ति सीतां शुद्धां यशस्विनीम्}
{ततो गृहीत्वा वैदेहीमयोध्यामहमागतः} %7-45-10

\twolineshloka
{अयं तु मे महान्वादः शोकश्च हृदि वर्तते}
{पौरापवादः सुमहांस्तथा जनपदस्य च} %7-45-11

\twolineshloka
{अकीर्तिर्यस्य गीयेत लोके भूतस्य कस्यचित्}
{पतत्येवाधमाँल्लोकान्यावच्छब्दः प्रकीर्त्यते} %7-45-12

\twolineshloka
{अकीर्तिर्निन्द्यते देवैः कीर्तिर्लोकेषु पूज्यते}
{कीर्त्यर्थं तु समारम्भः सर्वेषां सुमहात्मनाम्} %7-45-13

\twolineshloka
{अप्यहं जीवितं जह्यां युष्मान्वा पुरुषर्षभाः}
{अपवादभयाद्भीतः किं पुनर्जनकात्मजाम्} %7-45-14

\twolineshloka
{तस्माद्भवन्तः पश्यन्तु पतितं शोकसागरे}
{नहि पश्याम्यहं भूतं किञ्चिद्दुःखमतोऽधिकम्} %7-45-15

\twolineshloka
{श्वस्त्वं प्रभाते सौमित्रे सुमन्त्राधिष्ठितं रथम्}
{आरुह्य सीतामारोप्य विषयान्ते समुत्सृज} %7-45-16

\twolineshloka
{गङ्गायास्तु परे पारे वाल्मीकेस्तु महात्मनः}
{आश्रमो दिव्यसङ्काशस्तमसातीरमाश्रितः} %7-45-17

\twolineshloka
{तत्रैनां विजने देशे विसृज्य रघुनन्दन}
{शीघ्रमागच्छ भद्रं ते कुरुष्व वचनं मम} %7-45-18

\twolineshloka
{न चास्मिन् प्रतिवक्तव्यः सीतां प्रति कथञ्चन}
{तस्मात्त्वं गच्छ सौमित्रे नात्र कार्या विचारणा} %7-45-19

\twolineshloka
{अप्रीतिर्हि परा मह्यं त्वयेतत्प्रतिवारिते}
{शापिता हि मया यूयं भुजाभ्यां जीवितेन च} %7-45-20

\twolineshloka
{ये मां वाक्यान्तरे ब्रूयुरनुनेतुं कथञ्चन}
{अहिता नाम ते नित्यं मदभीष्टविघातनात्} %7-45-21

\twolineshloka
{मानयन्तु भवन्तो मां यदि मच्छासने स्थिताः}
{इतोऽद्य नीयतां सीता कुरुष्व वचनं मम} %7-45-22

\twolineshloka
{पूर्वमुक्तोऽहमनया गङ्गातीरेऽहमाश्रमान्}
{पश्येयमिति तस्याश्च कामः संवर्त्यतामयम्} %7-45-23

\twolineshloka
{एवमुक्त्वा तु काकुत्स्थो बाष्पेण पिहिताननः}
{प्रविवेश स धर्मात्मा भ्रातृभिः परिवारितः} %7-45-24

\onelineshloka
{शोकसंलग्नहृदयो निशश्वास यथा द्विपः} %7-45-25


॥इत्यार्षे श्रीमद्रामायणे वाल्मीकीये आदिकाव्ये उत्तरकाण्डे सीतासमुत्सर्गादेशः नाम पञ्चचत्वारिंशः सर्गः ॥७-४५॥
