\sect{सप्तचत्वारिंशः सर्गः — रामशासनकथनम्}

\twolineshloka
{अथ नावं सुविस्तीर्णां नैषादीं राघवानुजः}
{आरुरोह समायुक्तां पूर्वमारोप्य मैथिलीम्} %7-47-1

\twolineshloka
{सुमन्त्रं चैव सरथं स्थीयतामिति लक्ष्मणः}
{उवाच शोकसन्तप्तः प्रयाहीति च नाविकम्} %7-47-2

\twolineshloka
{ततस्तीरमुपागम्य भागीरथ्याः स लक्ष्मणः}
{उवाच मैथिलीं वाक्यं प्राञ्जलिर्बाष्पसम्प्लुतः} %7-47-3

\twolineshloka
{हृद्गतं मे महच्छल्यं यस्मादार्येण धीमता}
{अस्मिन्निमित्ते वैदेहि लोकस्य वचनीकृतः} %7-47-4

\twolineshloka
{श्रेयो हि मरणं मेऽद्य मृत्युर्वा यत्परं भवेत्}
{न चास्मिन्नीदृशे कार्ये नियोज्यो लोकनिन्दिते} %7-47-5

\twolineshloka
{प्रसीद च न मे पापं कर्तुमर्हसि शोभने}
{इत्यञ्जलिकृतो भूमौ निपपात स लक्ष्मणः} %7-47-6

\twolineshloka
{रुदन्तं प्राञ्जलिं दृष्ट्वा काङ्क्षन्तं मृत्युमात्मनः}
{मैथिली भृशसंविग्ना लक्ष्मणं वाक्यमब्रवीत्} %7-47-7

\twolineshloka
{किमिदं नावगच्छामि ब्रूहि तत्त्वेन लक्ष्मण}
{पश्यामि त्वां न च स्वस्थमपि क्षेमं महीपतेः} %7-47-8

\twolineshloka
{शापितोऽसि नरेन्द्रेण यत्त्वं सन्तापमागतः}
{तद्ब्रूयाः सन्निधौ मह्यमहमाज्ञापयामि ते} %7-47-9

\twolineshloka
{वैदेह्या चोद्यमानस्तु लक्ष्मणो दीनचेतनः}
{अवाङ्मुखो बाष्पकलं वाक्यमेतदुवाच ह} %7-47-10

\threelineshloka
{श्रुत्वा परिषदो मध्ये ह्यपवादं सुदारुणम्}
{पुरे जनपदे चैव त्वत्कृते जनकात्मजे}
{रामः सन्तप्तहृदयो मा निवेद्य गृहं गतः} %7-47-11

\twolineshloka
{न तानि वचनीयानि मया देवि तवाग्रतः}
{यानि राज्ञा हृदि न्यस्तान्यमर्षात्पृष्ठतः कृतः} %7-47-12

\twolineshloka
{सा त्वं त्यक्ता नृपतिना निर्दोषा मम सन्निधौ}
{पौरापवादभीतेन ग्राह्यं देवि न तेऽन्यथा} %7-47-13

\twolineshloka
{आश्रमान्तेषु च मया त्यक्तव्या त्वं भविष्यसि}
{राज्ञः शासनमाज्ञाय तवेदं किल दौर्हृदम्} %7-47-14

\twolineshloka
{तदेतज्जाह्नवीतीरे ब्रह्मर्षीणां तपोवनम्}
{पुण्यं च रमणीयं च मा विषादं कृथाः शुभे} %7-47-15

\twolineshloka
{राज्ञो दशरथस्येष्टः पितुर्मे मुनिपुङ्गवः}
{सखा परमको विप्रो वाल्मीकिः सुमहायशाः} %7-47-16

\twolineshloka
{पादच्छायामुपागम्य सुखमस्य महात्मनः}
{उपवासपरैकाग्रा वस त्वं जनकात्मजे} %7-47-17

\twolineshloka
{पतिव्रतात्वमास्थाय रामं कृत्वा सदा हृदि}
{श्रेयस्ते परमं देवि तथा कृत्वा भविष्यति} %7-47-18


॥इत्यार्षे श्रीमद्रामायणे वाल्मीकीये आदिकाव्ये उत्तरकाण्डे रामशासनकथनम् नाम सप्तचत्वारिंशः सर्गः ॥७-४७॥
