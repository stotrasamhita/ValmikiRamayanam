\sect{द्विषष्ठितमः सर्गः — शत्रुघ्नप्रार्थना}

\twolineshloka
{तथोक्ते तानृषीन्रामः प्रत्युवाच कृताञ्जलिः}
{किमाहारः किमाचारो लवणः क्व च वर्तते} %7-62-1

\twolineshloka
{राघवस्य वचः श्रुत्वा ऋषयः सर्व एव ते}
{ततो निवेदयामासुर्लवणो ववृधे यथा} %7-62-2

\twolineshloka
{आहारः सर्वसत्त्वानि विशेषेण च तापसाः}
{आचारो रौद्रता नित्यं वासो मधुवने तथा} %7-62-3

\twolineshloka
{हत्वा बहुसहस्राणि सिंहव्याघ्रमृगद्विपान्}
{मानुषांश्चैव कुरुते नित्यमाहारमाह्निकम्} %7-62-4

\twolineshloka
{ततोऽन्तराणि सत्त्वानि खादते स महाबलः}
{संहारे समनुप्राप्ते व्यादितास्य इवान्तकः} %7-62-5

\twolineshloka
{तच्छ्रुत्वा राघवो वाक्यमुवाच स महामुनीन्}
{घातयिष्यामि तद्रक्षो ह्यपगच्छतु वो भयम्} %7-62-6

\twolineshloka
{प्रतिज्ञाय तथा तेषां मुनीनामुग्रतेजसाम्}
{स भ्रातऽन्सहितान्सर्वानुवाच रघुनन्दनः} %7-62-7

\twolineshloka
{को हन्ता लवणं वीरः कस्यांशः स विधीयताम्}
{भरतस्य महाबाहोः शत्रुघ्नस्य च धीमतः} %7-62-8

\twolineshloka
{राघवेणैवमुक्तस्तु भरतो वाक्यमब्रवीत्}
{अहमेनं वधिष्यामि ममांशः स विधीयताम्} %7-62-9

\twolineshloka
{भरतस्य वचः श्रुत्वा धैर्यशौर्यसमन्वितम्}
{लक्ष्मणावरजस्तस्थौ हित्वा सौवर्णमासनम्} %7-62-10

\twolineshloka
{शत्रुघ्नस्त्वब्रवीद्वाक्यं प्रणिपत्य नराधिपम्}
{कृतकर्मा महाबाहुर्मध्यमो रघुनन्दनः} %7-62-11

\twolineshloka
{आर्येण हि पुरा शून्या त्वयोध्या परिपालिता}
{सन्तापं हृदये कृत्वा आर्यस्यागमनं प्रति} %7-62-12

\twolineshloka
{दुःखानि च बहूनीह ह्यनुभूतानि पार्थिव}
{शयानो दुःखशय्यासु नन्दिग्रामेऽवसत्पुरा} %7-62-13

\threelineshloka
{फलमूलाशनो भूत्वा जटी चीरधरस्तथा}
{अनुभूयेदृशं दुःखमेष राघवनन्दनः}
{प्रेष्ये मयि स्थिते राजन्न भूयः क्लेशमाप्नुयात्} %7-62-14

\twolineshloka
{तथा ब्रुवति शत्रुघ्ने राघवः पुनरब्रवीत्}
{एवं भवतु काकुत्स्थ क्रियतां मम शासनम्} %7-62-15

\twolineshloka
{राज्ये त्वामभिषेक्ष्यामि मधोस्तु नगरे शुभे}
{निवेशय महाबाहो भरतं यद्यवेक्षसे} %7-62-16

\onelineshloka
{शूरस्त्वं कृतविद्यश्च समर्थश्च निवेशने} %7-62-17

\twolineshloka
{यो हि शत्रुं समुत्पाट्य पार्थिवस्य पुनः क्षये}
{न विधत्ते नृपं तत्र नरकं स हि गच्छति} %7-62-18

\twolineshloka
{स त्वं हत्वा मधुसुतं लवणं पापनिश्चयम्}
{राज्यं प्रशाधि धर्मेण वाक्यं मे यद्यवेक्षसे} %7-62-19

\twolineshloka
{उत्तरं च न वक्तव्यं शूर वाक्यान्तरे मम}
{बालेन पूर्वजस्याज्ञा कर्तव्या नात्र संशयः} %7-62-20

\twolineshloka
{अभिषेकं च काकुत्स्थ प्रतीच्छस्व मयोद्यतम्}
{वसिष्ठप्रमुखैर्विप्रैर्विधिमन्त्रपुरस्कृतम्} %7-62-21


॥इत्यार्षे श्रीमद्रामायणे वाल्मीकीये आदिकाव्ये उत्तरकाण्डे शत्रुघ्नप्रार्थना नाम द्विषष्ठितमः सर्गः ॥७-६२॥
