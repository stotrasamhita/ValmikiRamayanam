\sect{एकोनाशीतितमः सर्गः — दण्डराज्यनिवेशः}

\twolineshloka
{तदद्भुततमं वाक्यं श्रुत्वागस्त्यस्य राघवः}
{गौरवाद्विस्मयाच्चैव पुनः प्रष्टुं प्रचक्रमे} %7-79-1

\twolineshloka
{भगवंस्तद्वनं घोरं तपस्तप्यति यत्र सः}
{श्वेतो वैदर्भको राजा कथं स्यादमृगद्विजम्} %7-79-2

\twolineshloka
{तद्वनं स कथ राजा शून्यं मनुजवर्जितम्}
{तपश्चर्तुं प्रविष्टः स श्रोतुमिच्छामि तत्त्वतः} %7-79-3

\twolineshloka
{रामस्य वचनं श्रुत्वा कौतूहलसमन्वितम्}
{वाक्यं परमतेजस्वी वक्तुमेवोपचक्रमे} %7-79-4

\twolineshloka
{पुरा कृतयुगे राम मनुर्दण्डधरः प्रभुः}
{तस्य पुत्रो महानासीदिक्ष्वाकुः कुलनन्दनः} %7-79-5

\twolineshloka
{तं पुत्रं पूर्वंकं राज्ये निक्षिप्य भुवि दुर्जयम्}
{पृथिव्यां राजवंशानां भव कर्तेत्युवाच ह} %7-79-6

\twolineshloka
{तथैवेति प्रतिज्ञातं पितुः पुत्रेण राघव}
{ततः परमसन्तुष्टो मनुः पुत्रमुवाच ह} %7-79-7

\twolineshloka
{प्रीतोऽस्मि परमोदार त्वं कर्ताऽसि न संशयः}
{दण्डेन च प्रजा रक्ष मा च दण्डमकारणे} %7-79-8

\twolineshloka
{अपराधिषु यो दण्डः पात्यते मानवेषु वै}
{स दण्डो विधिवन्मुक्तः स्वर्गं नयति पार्थिवम्} %7-79-9

\twolineshloka
{तस्माद्दण्डे महाबाहो यत्नवान्भव पुत्रक}
{धर्मो हि परमो लोके कुर्वतस्ते भविष्यति} %7-79-10

\twolineshloka
{इति तं बहु सन्दिश्य मनुः पुत्रं समाधिना}
{जगाम त्रिदिवं हृष्टो ब्रह्मलोकं सनातनम्} %7-79-11

\twolineshloka
{प्रयाते त्रिदिवं तस्मिन्निक्ष्वाकुरमितप्रभः}
{जनयिष्ये कथं पुत्रानिति चिन्तापरोऽभवत्} %7-79-12

\twolineshloka
{कर्मभिर्बहुरूपैश्च तैस्तैर्मनुसुतस्सुतान्}
{जनयामास धर्मात्मा शतं देवसुतोपमान्} %7-79-13

\twolineshloka
{तेषामवरजस्तात सर्वेषां रघुनन्दन}
{मूढश्चाकृतविद्यश्च न शुश्रूषति पूर्वजान्} %7-79-14

\twolineshloka
{नाम तस्य च दण्डेति पिता चक्रेऽल्पमेधसः}
{अवश्यं दण्डपतनं शरीरेऽस्य भविष्यति} %7-79-15

\twolineshloka
{अपश्यमानस्तं देशं घोरं पुत्रस्य राघव}
{विन्ध्यशैवलयोर्मध्ये राज्यं प्रादादरिन्दम} %7-79-16

\twolineshloka
{स दण्डस्तत्र राजाभूद्रम्ये पर्वतरोधसि}
{पुरं चाप्रतिमं राम न्यवेशयदनुत्तमम्} %7-79-17

\twolineshloka
{पुरस्य चाकरोन्नाम मधुमन्तमिति प्रभो}
{पुरोहितं तूशनसं वरयामास सुव्रतम्} %7-79-18

\twolineshloka
{एवं स राजा तद्राज्यमकरोत्सपुरोहितः}
{प्रहृष्टमनुजाकीर्णं देवराजं यथा वृषा} %7-79-19

\twolineshloka
{ततः स राजा मनुजेन्द्रपुत्रः सार्धं च तेनोशनसा तदानीम्}
{चकार राज्यं सुमहान्महात्मा शक्रो दिवीवोशनसा समेतः} %7-79-20


॥इत्यार्षे श्रीमद्रामायणे वाल्मीकीये आदिकाव्ये उत्तरकाण्डे दण्डराज्यनिवेशः नाम एकोनाशीतितमः सर्गः ॥७-७९॥
