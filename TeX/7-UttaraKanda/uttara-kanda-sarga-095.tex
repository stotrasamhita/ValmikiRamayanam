\sect{पञ्चनवतितमः सर्गः — वाल्मीकिदूतप्रेषणम्}

\twolineshloka
{रामो बहून्यहान्येवं तद्गीतं परमं शुभम्}
{शुश्राव मुनिभिः सार्धं पार्थिवैः सह वानरैः} %7-95-1

\threelineshloka
{तस्मिन्गीते तु विज्ञाय सीतापुत्रौ कुशीलवौ}
{तस्याः परिषदो मध्ये रामो वचनमब्रवीत्}
{दूताञ्छुद्धसमाचारानाहूयात्ममनीषया} %7-95-2

\onelineshloka
{मद्वचो ब्रूत गच्छध्वमितो भगवतोऽन्तिकम्} %7-95-3

\threelineshloka
{परिषदो मध्ये रामो वचनमब्रवीत्}
{यदि शुद्धसमाचारा यदि वा वीतकल्मषा}
{करोत्विहात्मनः शुद्धिमनुमान्य महामुनिम्} %7-95-4

\twolineshloka
{छन्दं मुनेश्च विज्ञाय सीतायाश्च मनोगतम्}
{प्रत्ययं दातुकामायास्ततः शंसत मे लघु} %7-95-5

\twolineshloka
{श्वः प्रभाते तु शपथं मैथिली जनकात्मजा}
{करोतु परिषन्मध्ये शोधनार्थं ममैव च} %7-95-6

\twolineshloka
{श्रुत्वा तु ऱागवस्यैतद्वचः परममद्भुतम्}
{दूताः सम्प्रययुर्बाटं यत्रास्ते मुनिपुङ्गवः} %7-95-7

\twolineshloka
{ते प्रणम्य महात्मानं ज्वलन्तममितप्रभम्}
{ऊचुस्ते रामवाक्यानि मृदूनि मधुराणि च} %7-95-8

\twolineshloka
{तेषां तद्व्याहृतं श्रुत्वा रामस्य च मनोगतम्}
{विज्ञाय सुमहातेजा मुनिर्वाक्यमथाब्रवीत्} %7-95-9

\twolineshloka
{एवं भवतु भद्रं वो यथा वदति राघवः}
{तथा करिष्यते सीता दैवतं हि पतिः स्त्रियाः} %7-95-10

\twolineshloka
{तथोक्ता मुनिना सर्वे राजदूता महौजसम्}
{प्रत्येत्य राघवं क्षिप्रं मुनिवाक्यं बभाषिरे} %7-95-11

\twolineshloka
{ततः प्रहृष्टः काकुत्स्थः श्रुत्वा वाक्यं महात्मनः}
{ऋषींस्तत्र समेतांश्च राज्ञश्चैवाभ्यभाषत} %7-95-12

\twolineshloka
{भगवन्तः सशिष्या वै सानुगाश्च नराधिपाः}
{पश्यन्तु सीताशपथं यश्चैवान्योऽपि काङ्क्षते} %7-95-13

\twolineshloka
{तस्य तद्वचनं श्रुत्वा राघवस्य महात्मनः}
{सर्वेषामृषिमुख्यानां साधुवादो महानभूत्} %7-95-14

\twolineshloka
{राजानश्च महात्मानः प्रशंसन्ति स्म राघवम्}
{उपपन्नं नरश्रेष्ठ त्वय्येव भुवि नान्यतः} %7-95-15

\twolineshloka
{एवं विनिश्चयं कृत्वा श्वोभूत इति राघवः}
{विसर्जयामास तदा सर्वांस्ताञ्छत्रुसूदनः} %7-95-16

\twolineshloka
{इति सम्प्रविचार्य राजसिंहः श्वोभूते शपथस्य निश्चयं वै}
{विससर्ज मुनीन्नृपांश्च सर्वान्स महात्मा महतो महानुभावः} %7-95-17


॥इत्यार्षे श्रीमद्रामायणे वाल्मीकीये आदिकाव्ये उत्तरकाण्डे वाल्मीकिदूतप्रेषणम् नाम पञ्चनवतितमः सर्गः ॥७-९५॥
