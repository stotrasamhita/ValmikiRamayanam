\sect{चतुःसप्ततितमः सर्गः — नारदवचनम्}

\twolineshloka
{तथा तु करुणं तस्य द्विजस्य परिदेवनम्}
{शुश्राव राघवः सर्वं दुःखशोकसमन्वितम्} %7-74-1

\twolineshloka
{स दुःखेन च सन्तप्तो मन्त्रिणस्तानुपाह्वयत्}
{वसिष्ठं वामदेवं च भ्रातरौ सह नैगमान्} %7-74-2

\twolineshloka
{ततो द्विजा वसिष्ठेन सार्धमष्टौ प्रवेशिताः}
{राजानं देवसङ्काशं वर्धस्वेति ततोऽब्रुवन्} %7-74-3

\threelineshloka
{मार्कण्डेयोऽथ मौद्गल्यो वामदेवश्च काश्यपः}
{कात्यायनोऽथ जाबालिर्गौतमो नारदस्तथा}
{एते द्विजर्षभाः सर्वे आसनेषूपवेशिताः} %7-74-4

\twolineshloka
{महर्षीन्समनुप्राप्तानभिवाद्य कृताञ्जलिः}
{मन्त्रिणो नैगमांश्चैव यथार्हमनुकूलतः} %7-74-5

\twolineshloka
{तेषां समुपविष्टानां सर्वेषां दीप्ततेजसाम्}
{राघवः सर्वमाचष्टे द्विजोऽयमुपरोधते} %7-74-6

\twolineshloka
{तस्य तद्वचनं श्रुत्वा राज्ञो दीनस्य नारदः}
{प्रत्युवाच शुभं वाक्यमृषीणां सन्निधौ नृपम्} %7-74-7

\twolineshloka
{शृणु राजन्यथाकाले प्राप्तो बालस्य सङ्क्षयः}
{श्रुत्वा कर्तव्यतां राजन्कुरुष्वरघुनन्दन} %7-74-8

\twolineshloka
{पुरा कृतयुगे राजन्ब्राह्मणा वै तपस्विनः}
{अब्राह्मणस्तदा राजन्न तपस्वी कथञ्चन} %7-74-9

\twolineshloka
{तस्मिन्युगे प्रज्वलिते ब्रह्मभूते त्वनावृते}
{अमृत्यवस्तदा सर्वे जज्ञिरे दीर्घदर्शिनः} %7-74-10

\twolineshloka
{ततस्त्रेतायुगं नाम मानवानां वपुष्मताम्}
{क्षत्रियास्तत्र जायन्ते पूर्णेन तपसाऽन्विताः} %7-74-11

\twolineshloka
{वीर्येण तपसा चैव तेऽधिकाः पूर्वजन्मनि}
{मानवा ये महात्मानस्त्वत्र त्रेतायुगे युगे} %7-74-12

\twolineshloka
{ब्रह्मक्षत्रं च तत्सर्वं यत्पूर्वमपरं च यत्}
{युगयोरुभयोरासीत्समवीर्यसमन्वितम्} %7-74-13

\twolineshloka
{अपश्यंस्तु न ते सर्वे विशेषमधिकं ततः}
{स्थापनं चक्रिरे तत्र चातुर्वर्ण्यस्य सम्मतम्} %7-74-14

\twolineshloka
{तस्मिन्युगे प्रज्वलिते धर्मभूते ह्यनावृते}
{अधर्मः पादमेकं तु पातयत्पृथिवीतले} %7-74-15

\onelineshloka
{अधर्मेण हि संयुक्तस्तेजो मन्दं भविष्यति} %7-74-16

\twolineshloka
{आमिषं यच्च पूर्वेषां राजसं च मलं भृशम्}
{अनृतं नाम तद्भूतं पादेन पृथिवीतले} %7-74-17

\twolineshloka
{अनृतं पातयित्वा तु पादमेकमधर्मतः}
{ततः प्रादुष्कृतं पूर्वमायुषः परिनिष्ठितम्} %7-74-18

\twolineshloka
{पतिते त्वनृते तस्मिन्नधर्मे च महीतले}
{शुभान्येवाचरँल्लोकः सत्यधर्मपरायणः} %7-74-19

\twolineshloka
{त्रेतायुगे च वर्तन्ते ब्राह्मणाः क्षत्रियाश्च ये}
{तपोऽतप्यन्त ते सर्वे शुश्रूषामपरे जनाः} %7-74-20

\twolineshloka
{स धर्मः परमस्तेषां वैश्यशूद्रं समागमत्}
{पूजां च सर्ववर्णानां शूद्राश्चक्रुर्विशेषतः} %7-74-21

\twolineshloka
{एतस्मिन्नन्तरे तेषामधर्मे चानृते च ह}
{ततः पूर्वे भृशं ह्रासमगमन्नृपसत्तम} %7-74-22

\twolineshloka
{ततः पादमधर्मस्स द्वितीयमवतारयत्}
{ततो द्वापरसञ्ज्ञाऽस्य युगस्य समजायत} %7-74-23

\twolineshloka
{तस्मिन्द्वापरसञ्ज्ञे तु वर्तमाने युगक्षये}
{अधर्मश्चानृतं चैव ववृधे पुरुषर्षभ} %7-74-24

\twolineshloka
{तस्मिन्द्वापरसङ्ख्याते तपो वैश्यान्समाविशत्}
{त्रिभ्यो युगेभ्यस्त्रीन्वर्णान्क्रमाद्वै तप आविशत्} %7-74-25

\threelineshloka
{त्रिभ्यो युगेभ्यस्त्रीन्वर्णान्धर्मश्च परिनिष्ठितः}
{न शुद्धो लभते धर्मं युगतस्तु नरर्षभ}
{हीनवर्णो नृपश्रेष्ठ तप्यते सुमहत्तपः} %7-74-26

\twolineshloka
{भविष्यच्छूद्रयोन्यां वै तपश्चर्या कलौ युगे}
{अधर्मः परमो राजन्द्वापरे शूद्रजन्मनः} %7-74-27

\twolineshloka
{स वै विषयपर्यन्ते तव राजन्महातपाः}
{अद्य तप्यति दुर्बुद्धिस्तेन बालवधो ह्ययम्} %7-74-28

\threelineshloka
{यो ह्यधर्ममकार्यं वा विषये पार्थिवस्य तु}
{करोति चाश्रीमूलं तत्पुरे वा दुर्मतिर्नरः}
{क्षिप्रं च नरकं याति स च राजा न संशयः} %7-74-29

\twolineshloka
{अधीतस्य च तप्तस्य कर्मणः सुकृतस्य च}
{षष्ठं भजति भागं तु प्रजा धर्मेण पालयन्} %7-74-30

\onelineshloka
{षङ्भागस्य न भोक्ताऽसौ रक्षते न प्रजाः कथम्} %7-74-31

\twolineshloka
{स त्वं पुरुषशार्दूल मार्गस्व विषयं स्वकम्}
{दुष्कृतं यत्र पश्येथास्तत्र यत्नं समाचर} %7-74-32

\twolineshloka
{एवं चेद्धर्मवृद्धिश्च नृणां चायुर्विवर्धनम्}
{भविष्यति नरश्रेष्ठ बालस्यास्य च जीवितम्} %7-74-33


॥इत्यार्षे श्रीमद्रामायणे वाल्मीकीये आदिकाव्ये उत्तरकाण्डे नारदवचनम् नाम चतुःसप्ततितमः सर्गः ॥७-७४॥
