\sect{एकोनचत्वारिंशः सर्गः — वानरप्रीणनम्}

\twolineshloka
{ते प्रयाता महात्मानः पार्थिवास्ते प्रहृष्टवत्}
{गजवाजिसहस्रौघैः कम्पयन्तो वसुन्धराम्} %7-39-1

\twolineshloka
{अक्षौहिण्यो हि तत्रासन्राघवार्थे समुद्यताः}
{भरतस्याज्ञयाऽनेकाः प्रहृष्टा बलवाहनाः} %7-39-2

\twolineshloka
{ऊचुस्ते च महीपाला बलदर्पसमन्विताः}
{न रामरावणं युद्धे पश्यामः पुरतः स्थितम्} %7-39-3

\twolineshloka
{भरतेन वयं पश्चात्समानीता निरर्थकम्}
{हता हि राक्षसाः क्षिप्रं पार्थिवैः स्युर्न संशयः} %7-39-4

\twolineshloka
{रामस्य बाहुवीर्येण रक्षिता लक्ष्मणस्य च}
{सुखं पारेसमुद्रस्य युध्येम विगतज्वराः} %7-39-5

\twolineshloka
{एताश्चान्याश्च राजानः कथास्तत्र सहस्रशः}
{कथयन्तः स्वराज्यानि जग्मुर्हर्षसमन्विताः} %7-39-6

\twolineshloka
{स्वानि राज्यानि मुख्यानि ऋद्धानि मुदितानि च}
{समृद्धधनधान्यानि पूर्णानि वसुमन्ति च} %7-39-7

\twolineshloka
{यथापुराणि ते गत्वा रत्नानि विविधान्यथ}
{रामस्य प्रियकामार्थमुपहारान् नृपा ददुः} %7-39-8

\twolineshloka
{अश्वान्यानानि रत्नानि हस्तिनश्च मदोत्कटान्}
{चन्दनानि च मुख्यानि दिव्यान्याभरणानि च} %7-39-9

\twolineshloka
{मणिमुक्ताप्रवालांस्तु दास्यो रूपसमन्विताः}
{अजाविकांश्च विविधान् रथांस्तु विविधान्ददुः} %7-39-10

\twolineshloka
{भरतो लक्ष्मणश्चैव शत्रुघ्नश्च महाबलाः}
{आदाय तानि रत्नानि स्वां पुरीं पुनरागताः} %7-39-11

\twolineshloka
{आगम्य च पुरीं रम्यामयोध्यां पुरुषर्षभाः}
{तानि रत्नानि चित्राणि रामाय समुपाहरन्} %7-39-12

\twolineshloka
{प्रतिगृह्य च तत्सर्वं रामः प्रीतिसमन्वितः}
{सुग्रीवाय ददौ राज्ञे महात्मा कृतकर्मणे} %7-39-13

\twolineshloka
{बिभीषणाय च ददौ तथान्येभ्योऽपि राघवः}
{राक्षसेभ्यः कपिभ्यश्च यैर्वृतो जयमाप्तवान्} %7-39-14

\twolineshloka
{ते सर्वे रामदत्तानि रत्नानि कपिराक्षसाः}
{शिरोभिर्धारयामासुर्बाहुभिश्च महाबलाः} %7-39-15

\twolineshloka
{हनूमन्तं च नृपतिरिक्ष्वाकूणां महारथः}
{अङ्गदं च महाबाहुमङ्कमारोप्य वीर्यवान्} %7-39-16

\twolineshloka
{रामः कमलपत्राक्षः सुग्रीवमिदमब्रवीत्}
{अङ्गदस्ते सुपुत्रोऽयं मन्त्री चाप्यनिलात्मजः} %7-39-17

\twolineshloka
{सुग्रीवमन्त्रिते युक्तौ मम चापि हिते रतौ}
{अर्हतो विविधां पूजां त्वत्कृते वै हरीश्वर} %7-39-18

\twolineshloka
{इत्युक्त्वा व्यवमुच्याङ्गाद्भूषणानि महायशाः}
{स बबन्ध महार्हाणि तदाङ्गदहनूमतोः} %7-39-19

\twolineshloka
{आभाष्य च महावीर्यान्राघवो यूथपर्षभान्}
{नीलं नलं केसरिणं कुमुदं गन्धमादनम्} %7-39-20

\twolineshloka
{सुषेणं पनसं वीरं मैन्दं द्विविदमेव च}
{जाम्बवन्तं गवाक्षं च विनतं धूम्रमेव च} %7-39-21

\twolineshloka
{वलीमुखं प्रजङ्घं च सन्नादं च महाबलम्}
{दरीमुखं दधिमुखमिन्द्रजानुं च यूथपम्} %7-39-22

\twolineshloka
{मधुरं श्लक्ष्णया वाचा नेत्राभ्यामापिबन्निव}
{सुहृदो मे भवन्तश्च शरीरं भ्रातरस्तथा} %7-39-23

\twolineshloka
{युष्माभिरुद्धृतश्चाहं व्यसनात्काननौकसः}
{धन्यो राजा च सुग्रीवो भवद्भिः सुहृदां वरैः} %7-39-24

\twolineshloka
{एवमुक्त्वा ददौ तेभ्यो भूषणानि यथार्हतः}
{वज्राणि च महार्हाणि सस्वजे च नरर्षभः} %7-39-25

\twolineshloka
{ते पिबन्तः सुगन्धीनि मधूनि मधुपिङ्गलाः}
{मांसानि च सुमृष्टानि मूलानि च फलानि च} %7-39-26

\twolineshloka
{एवं तेषां निवसतां मासः साग्रो ययौ तदा}
{मुहूर्तमिव ते सर्वे रामभक्त्या च मेनिरे} %7-39-27

\twolineshloka
{रामोऽपि रेमे तैः सार्धं वानरैः कामरूपिभिः}
{राक्षसैश्च महावीर्यैर्ऋक्षैश्चैव महाबलैः} %7-39-28

\twolineshloka
{एवं तेषां ययौ मासो द्वितीयः शिशिरः सुखम्}
{वानराणां प्रहृष्टानां राक्षसानां च सर्वशः} %7-39-29

\twolineshloka
{इक्ष्वाकुनगरे रम्ये परां प्रीतिमुपासताम्}
{रामस्य प्रीतिकरणैः कालस्तेषां सुखं ययौ} %7-39-30


॥इत्यार्षे श्रीमद्रामायणे वाल्मीकीये आदिकाव्ये उत्तरकाण्डे वानरप्रीणनम् नाम एकोनचत्वारिंशः सर्गः ॥७-३९॥
