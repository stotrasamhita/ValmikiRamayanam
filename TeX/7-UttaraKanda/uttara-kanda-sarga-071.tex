\sect{एकसप्ततितमः सर्गः — शत्रुघ्नप्रशंसा}

\twolineshloka
{ततो द्वादशमे वर्षे शत्रघ्नो रामपालिताम्}
{अयोध्यां चकमे गन्तुमल्पभृत्यबलानुगः} %7-71-1

\twolineshloka
{ततो मन्त्रिपुरोगांश्च बलमुख्यान्निवर्त्य च}
{जगाम हयमुख्यैश्च रथानां च शतेन सः} %7-71-2

\twolineshloka
{स गत्वा गणितान्वासान्सप्ताष्टौ रघुनन्दनः}
{वाल्मीक्याश्रममागत्य वासं चक्रे महायशाः} %7-71-3

\twolineshloka
{सोऽभिवाद्य ततः पादौ वाल्मीकेः पुरुषर्षभः}
{पाद्यमर्ध्यं तथातिथ्यं जग्राह मुनिहस्ततः} %7-71-4

\twolineshloka
{बहुरूपाः सुमधुराः कथास्तत्र सहस्रशः}
{कथयामास स मुनिः शत्रुघ्नाय महात्मने} %7-71-5

\twolineshloka
{उवाच च मुनिर्वाक्यं लवणस्य वधाश्रितम्}
{सुदुष्करं कृतं कर्म लवणं निघ्नता त्वया} %7-71-6

\twolineshloka
{बहवः पार्थिवाः सौम्य हताः सबलवाहनाः}
{लवणेन महाबाहो युध्यमाना महाबलाः} %7-71-7

\twolineshloka
{स त्वया निहतः पापो लीलया पुरुषर्षभ}
{जगतश्च भयं तत्र प्रशान्तं तव तेजसा} %7-71-8

\twolineshloka
{रावणस्य वधो घोरो यत्नेन महता कृतः}
{इदं तु सुमहत्कर्म त्वया कृतमयत्नतः} %7-71-9

\twolineshloka
{प्रीतिश्चास्मिन्परा जाता देवानां लवणे हते}
{भूतानां चैव सर्वेषां जगतश्च प्रियं कृतम्} %7-71-10

\twolineshloka
{तच्च युद्धं मया दृष्टं यथावत्पुरुषर्षभ}
{सभायां वासवस्याथ उपविष्टेन राघव} %7-71-11

\twolineshloka
{ममापि परमा प्रीतिर्हृदि शत्रुघ्न वर्तते}
{उपाघ्रास्यामि ते मूर्ध्नि स्नेहस्यैषा परा गतिः} %7-71-12

\twolineshloka
{इत्युक्त्वा मूर्ध्नि शत्रुघ्नमुपाघ्राय महामुनिः}
{आतिथ्यमकरोत्तस्य ये च तस्य पदानुगाः} %7-71-13

\twolineshloka
{स भुक्तवान्नरश्रेष्ठो गीतमाधुर्यमुत्तमम्}
{शुश्राव रामचरितं तस्मिन्काले यथाकृमम्} %7-71-14

\threelineshloka
{तन्त्रीलयसमायुक्तं त्रिस्थानकरणान्वितम्}
{संस्कृतं लक्षणोपेतं समतालसमन्वितम्}
{शुश्राव रामचरितं तस्मिन्काले पुरा कृतम्} %7-71-15

\twolineshloka
{तान्यक्षराणि सत्यानि यथावृत्तानि पूर्वशः}
{श्रुत्वा पुरुषशार्दूलो विसंज्ञो बाष्पलोचनः} %7-71-16

\twolineshloka
{स मुहूर्तमिवासंज्ञो विनिःश्वस्य मुहुर्मुहुः}
{तस्मिन् गीते यथावृत्तं वर्तमानमिवाशृणोत्} %7-71-17

\twolineshloka
{पदानुगाश्च ये राज्ञस्तां श्रुत्वा गीतिसम्पदम्}
{अवाङ्मुखाश्च दीनाश्च ह्याश्चर्यमिति चाब्रुवन्} %7-71-18

\twolineshloka
{परस्परं च ये तत्र सैनिकाः सम्बभाषिरे}
{किमिदं क्व च वर्तामः किमेतत्स्वप्नदर्शनम्} %7-71-19

\twolineshloka
{अर्थो यो नः पुरा दृष्टस्तमाश्रमपदे पुनः}
{शृणुमः किमिदं स्वप्नो गीतबन्धं श्रियो भवेत्} %7-71-20

\twolineshloka
{विस्मयं ते परं गत्वा शत्रुघ्नमिदमब्रुवन्}
{साधु पृच्छ नरश्रेष्ठ वाल्मीकिं मुनिपुङ्गवम्} %7-71-21

\twolineshloka
{शत्रुघ्नस्त्वब्रवीत्सर्वान्कौतूहलसमन्वितान्}
{सैनिका न क्षमोऽस्माकं परिप्रष्टुमिहेदृशः} %7-71-22

\twolineshloka
{आश्चर्याणि बहूनीह भवन्त्यस्याश्रमे मुनेः}
{न तु कौतूहलाद्युक्तमन्वेष्टुं तं महामुनिम्} %7-71-23

\twolineshloka
{एवं तद्वाक्यमुक्त्वा च सैनिकान्रघुनन्दनः}
{अभिवाद्य महर्षिं तं स्वं निवेशं ययौ तदा} %7-71-24


॥इत्यार्षे श्रीमद्रामायणे वाल्मीकीये आदिकाव्ये उत्तरकाण्डे शत्रुघ्नप्रशंसा नाम एकसप्ततितमः सर्गः ॥७-७१॥
