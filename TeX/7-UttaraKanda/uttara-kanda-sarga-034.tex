\sect{चतुस्त्रिंशः सर्गः — वालिरावणसख्यम्}

\twolineshloka
{अर्जुनेन विमुक्तस्तु रावणो राक्षसाधिपः}
{चचार पृथिवीं सर्वामनिर्विण्णस्तथा कृतः} %7-34-1

\twolineshloka
{राक्षसं वा मनुष्यं वा शृणुतेऽयं बलाधिकम्}
{रावणस्तं समासाद्य युद्धे ह्वयति दर्पितः} %7-34-2

\twolineshloka
{ततः कदाचित्किष्किन्धां नगरीं वालिपालिताम्}
{गत्वा ह्वयति युद्धाय वालिनं हेममालिनम्} %7-34-3

\twolineshloka
{ततस्तु वानरामात्यस्तारस्तारापिता प्रभुः}
{उवाच वानरो वाक्यं युद्धप्रेप्सुमुपागतम्} %7-34-4

\twolineshloka
{राक्षसेन्द गतो वाली यस्ते प्रतिबलो भवेत्}
{कोऽन्यः प्रमुखतः स्थातुं तव शक्तः प्लवङ्गमः} %7-34-5

\twolineshloka
{चतुर्भ्योऽपि समुद्रेभ्यः सन्ध्यामन्वास्य रावण}
{इमं मुहूर्तमायाति वाली तिष्ठ मुहूर्तकम्} %7-34-6

\twolineshloka
{एतानस्थिचयान्पश्य य एते शङ्खपाण्डुराः}
{युद्धार्थिनामिमे राजन्वानराधिपतेजसा} %7-34-7

\twolineshloka
{यद्वाऽमृतरसः पीतस्त्वया रावण राक्षस}
{तदा वालिनमासाद्य तदन्तं तव जीवितम्} %7-34-8

\twolineshloka
{पश्येदानीं जगच्चित्रमिमं विश्रवसः सुत}
{इदं मुहूर्तं तिष्ठस्व दुर्लभं ते भविष्यति} %7-34-9

\twolineshloka
{अथवा त्वरसे मर्तुं गच्छ दक्षिणसागरम्}
{वालिनं द्रक्ष्यसे तत्र भूमिस्थमिव पावकम्} %7-34-10

\twolineshloka
{स तु तारं विनिर्भर्त्स्य रावणो लोकरावणः}
{पुष्पकं तत्समारुह्य प्रययौ दक्षिणार्णवम्} %7-34-11

\twolineshloka
{तत्र हेमगिरिप्रख्यं तरुणार्कनिभाननम्}
{रावणो वालिनं दृष्ट्वा सन्ध्योपासनतत्परम्} %7-34-12

\twolineshloka
{पुष्पकादवरुह्याथ रावणोऽञ्जनसन्निभः}
{ग्रहीतुं वालिनं तूर्णं निःशब्दपदमाव्रजत्} %7-34-13

\twolineshloka
{यदृच्छया तदा दृष्टो वालिनापि स रावणः}
{पापाभिप्रायवान् दृष्ट्वा चकार न तु सम्भ्रमम्} %7-34-14

\twolineshloka
{शशमालक्ष्य सिंहो वा पन्नगं गरुडो यथा}
{न चिन्तयति तं वाली रावणं पापनिश्चयम्} %7-34-15

\twolineshloka
{जिघृक्षमाणमायान्तं रावणं पापचेतसम्}
{कक्षावलम्बिनं कृत्वा गमिष्ये त्रीन्महार्णवान्} %7-34-16

\twolineshloka
{द्रक्ष्यन्त्यरिं ममाङ्कस्थं स्रंसदूरुकराम्बरम्}
{लम्बमानं दशग्रीवं गरुडस्येव पन्नगम्} %7-34-17

\twolineshloka
{इत्येवं मतिमास्थाय वाली कर्णमुपाश्रितः}
{जपन्वै नैगमान्मन्त्रांस्तस्थौ पर्वतराडिव} %7-34-18

\twolineshloka
{तावन्योन्यं जिघृक्षन्तौ हरिराक्षसपार्थिवौ}
{प्रयत्नवन्तौ तत्कर्म ईहतुर्बलदर्पितौ} %7-34-19

\twolineshloka
{हस्तग्राहं तु तं मत्वा पादशब्देन रावणम्}
{पराङ्मुखोऽपि जग्राह वाली सर्पमिवाण्डजः} %7-34-20

\twolineshloka
{ग्रहीतुकामं तं गृह्य रक्षसामीश्वरं हरिः}
{खमुत्पपात वेगेन कृत्वा कक्षावलम्बिनम्} %7-34-21

\twolineshloka
{तं चापीडयमानं तु वितुदन्तं नखैर्मुहुः}
{जहार रावणं वाली पवनस्तोयदं यथा} %7-34-22

\twolineshloka
{अथ ते राक्षसामात्या ह्रियमाणं दशाननम्}
{मुमोक्षयिषवो वालिं रवमाणा अभिद्रुताः} %7-34-23

\twolineshloka
{अन्वीयमानस्तैर्वाली भ्राजतेऽम्बरमध्यगः}
{अन्वीयमानो मेघौघैरम्बरस्थ इवांशुमान्} %7-34-24

\twolineshloka
{तेऽशक्नुवन्तः सम्प्राप्तुं वालिनं राक्षसोत्तमाः}
{तस्य बाहूरुवेगेन परिश्रान्ता व्यवस्थिताः} %7-34-25

\twolineshloka
{वालिमार्गादपाक्रामन्पर्वतेन्द्रो हि गच्छतः}
{किं पुनर्जीवितप्रेप्सुर्बिभ्रद्वै मांसशोणितम्} %7-34-26

\twolineshloka
{अपक्षिगणसम्पातान्वानरेन्द्रो महाजवः}
{क्रमशः सागरान्सर्वान्सन्ध्याकालमवन्दत} %7-34-27

\twolineshloka
{सभाज्यमानो भूतैस्तु खैचरैः खैचरोत्तमः}
{पश्चिमं सागरं वाली ह्याजगाम सरावणः} %7-34-28

\twolineshloka
{तस्मिन्सन्ध्यामुपासित्वा स्नात्वा जप्त्वा च वानरः}
{उत्तरं सागरं प्रायाद्वहमानो दशाननम्} %7-34-29

\twolineshloka
{बहुयोजनसाहस्रं वहमानो महाहरिः}
{वायुवच्च मनोवच्च जगाम सह शत्रुणा} %7-34-30

\twolineshloka
{उत्तरे सागरे सन्ध्यामुपासित्वा दशाननम्}
{वहमानोऽगमद्वाली पूर्वं वै स महोदधिम्} %7-34-31

\twolineshloka
{तत्रापि सन्ध्यामन्वास्य वासविः सहरीश्वरः}
{किष्किन्धामभितो गृह्य रावणं पुनरागमत्} %7-34-32

\twolineshloka
{चतुर्ष्वपि समुद्रेषु सन्ध्यामन्वास्य वानरः}
{रावणोद्वहनश्रान्तः किष्किन्धोपवनेऽपतत्} %7-34-33

\twolineshloka
{रावणं तु मुमोचाथ स्वकक्षात्कपिसत्तमः}
{कुतस्त्वमिति चोवाच प्रहसन्रावणं मुहुः} %7-34-34

\twolineshloka
{विस्मयं तु महद्गत्वा श्रमलोलनिरीक्षणः}
{राक्षसेन्द्रो हरीन्द्रं तमिदं वचनमब्रवीत्} %7-34-35

\twolineshloka
{वानरेन्द्र महेन्द्राभ राक्षसेन्द्रोऽस्मि रावणः}
{युद्धेप्सुरिह सम्प्राप्तः स चाद्यासादितस्त्वया} %7-34-36

\twolineshloka
{अहो बलमहो वीर्यमहो गाम्भीर्यमेव च}
{येनाहं पशुवद्गृह्य भ्रामितश्चतुरोऽर्णवान्} %7-34-37

\twolineshloka
{एवमश्रान्तवद्वीर शीघ्रमेव महार्णवान्}
{मां चैवोद्वहमानस्तु कोऽन्यो वीरः क्रमिष्यति} %7-34-38

\twolineshloka
{त्रयाणामेव भूतानां गतिरेषा प्लवङ्गम}
{मनोनिलसुपर्णानां तव चात्र न संशयः} %7-34-39

\twolineshloka
{सोऽहं दृष्टबलस्तुभ्यमिच्छामि हरिपुङ्गव}
{त्वया सह चिरं सख्यं सुस्निग्धं पावकाग्रतः} %7-34-40

\twolineshloka
{दाराः पुत्राः पुरं राष्ट्रं भोगाच्छादनभोजनम्}
{सर्वमेवाविभक्तं नौ भविष्यति हरीश्वर} %7-34-41

\twolineshloka
{ततः प्रज्वालयित्वाग्निं तावुभौ हरीराक्षसौ}
{भ्रातृत्वमुपसम्पन्नौ परिष्वज्य परस्परम्} %7-34-42

\twolineshloka
{अन्योन्यं लम्बितकरौ ततस्तौ हरिराक्षसौ}
{किष्किन्धां विशतुर्हृष्टौ सिंहौ गिरिगुहामिव} %7-34-43

\twolineshloka
{स तत्र मासमुषितः सुग्रीव इव रावणः}
{अमात्यैरागतैर्नीतस्त्रैलोक्योत्सादनार्थिभिः} %7-34-44

\twolineshloka
{एवमेतत्पुरावृत्तं वालिना रावणः प्रभो}
{धर्षितश्च कृतश्चापि भ्राता पावकसन्निधौ} %7-34-45

\twolineshloka
{बलमप्रतिमं राम वालिनोऽभवदुत्तमम्}
{सोऽपि त्वया विनिर्दग्धः शलभो वह्निना यथा} %7-34-46


॥इत्यार्षे श्रीमद्रामायणे वाल्मीकीये आदिकाव्ये उत्तरकाण्डे वालिरावणसख्यम् नाम चतुस्त्रिंशः सर्गः ॥७-३४॥
