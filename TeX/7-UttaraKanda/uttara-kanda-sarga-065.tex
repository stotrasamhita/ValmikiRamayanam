\sect{पञ्चषष्ठितमः सर्गः — सौदासकथा}

\twolineshloka
{प्रस्थाप्य च बलं सर्वं मासमात्रोषितः पथि}
{एक एवाशु शत्रुघ्नो जगाम त्वरितं तदा} %7-65-1

\twolineshloka
{द्विरात्रमन्तरे शूर उष्य राघवनन्दनः}
{वाल्मीकेराश्रमं पुण्यमगच्छद्वासमुत्तमम्} %7-65-2

\twolineshloka
{सोऽभिवाद्य महात्मानं वाल्मीकिं मुनिसत्तमम्}
{कृताञ्जलिरथो भूत्वा वाक्यमेतदुवाच ह} %7-65-3

\twolineshloka
{भगवन्वस्तुमिच्छामि गुरोः कृत्यादिहागतः}
{श्वः प्रभाते गमिष्यामि प्रतीचीं वारुणीं दिशम्} %7-65-4

\twolineshloka
{शत्रुघ्नस्य वचः श्रुत्वा प्रहस्य मुनिपुङ्गवः}
{प्रत्युवाच महात्मानं स्वागतं ते महायशः} %7-65-5

\twolineshloka
{स्वमाश्रममिदं सौम्य राघवाणां कुलस्य हि}
{आसनं पाद्यमर्ध्यं च निर्विशङ्कः प्रतीच्छ मे} %7-65-6

\twolineshloka
{प्रतिगृह्य तदा पूजां फलमूलं च भोजनम्}
{भक्षयामास काकुत्स्थस्तृप्तिं च परमां गतः} %7-65-7

\twolineshloka
{स भुक्त्वा फलमूलं च महर्षिं तमुवाच ह}
{इयं यज्ञविभूतिस्ते कस्याश्रमसमीपतः} %7-65-8

\twolineshloka
{तत्तस्य भाषितं श्रुत्वा वाल्मीकिर्वाक्यमब्रवीत्}
{शत्रुघ्न शृणु यस्येदं बभूवायतनं पुरा} %7-65-9

\twolineshloka
{युष्माकं पूर्वको राजा सौदासस्तस्य भूपतेः}
{पुत्रो वीरसहो नाम वीर्यवानतिधार्मिकः} %7-65-10

\twolineshloka
{स बाल एव सौदासो मृगयामुपचक्रमे}
{चञ्चूर्यमाणं ददृशे स शूरो राक्षसद्वयम्} %7-65-11

\twolineshloka
{शार्दूलरूपिणौ घोरौ मृगान्बहुसहस्रशः}
{भक्षमाणावसन्तुष्टौ पर्याप्तिं नैव जग्मतुः} %7-65-12

\twolineshloka
{स तु तौ राक्षसौ दृष्ट्वा निर्मृगं च वनं कृतम्}
{क्रोधेन महताऽऽविष्टो जघानैकं महेषुणा} %7-65-13

\twolineshloka
{विनिपात्य तमेकं तु सौदासः पुरुषर्षभः}
{विज्वरो विगतामार्षो हतं रक्षो ह्युदैक्षत} %7-65-14

\twolineshloka
{निरीक्षमाणं तं दृष्ट्वा सहायं तस्य रक्षसः}
{सन्तापमकरोद्घोरं सौदासं चेदमब्रवीत्} %7-65-15

\twolineshloka
{यस्मादनपराधं त्वं सहायं मम जघ्निवान्}
{तस्मात्तवापि पापिष्ठ प्रदास्यामि प्रतिक्रियाम्} %7-65-16

\twolineshloka
{एवमुक्त्वा तु तद्राक्षस्तत्रैवान्तरधीयत}
{कालपर्याययोगेन राजा मित्रसहोऽभवत्} %7-65-17

\twolineshloka
{राजापि यजते यज्ञमस्याश्रमसमीपतः}
{अश्वमेधं महायज्ञं तं वसिष्ठोऽभ्यपालयत्} %7-65-18

\twolineshloka
{तत्र यज्ञो महानासीद्बहुवर्षगणायुतः}
{समृद्धः परया लक्ष्म्या देवयज्ञसमोऽभवत्} %7-65-19

\twolineshloka
{अथावसाने यज्ञस्य पूर्ववैरमनुस्मरन्}
{वसिष्ठरूपी राजानमिति होवाच राक्षसः} %7-65-20

\twolineshloka
{अस्य यज्ञस्य जातोऽन्तः सामिषं भोजनं मम}
{दीयतामिह शीघ्रं वै नात्र कार्या विचारणा} %7-65-21

\twolineshloka
{तच्छ्रुत्वा व्याहृतं वाक्यं रक्षसा ब्रह्मरूपिणा}
{भक्ष्यसंस्कारकुशलमुवाच पृथिवीपतिः} %7-65-22

\twolineshloka
{हविष्यं सामिषं स्वादु यथा भवति भोजनम्}
{तथा कुरुष्व शीघ्रं वै परितुष्येद्यथा गुरुः} %7-65-23

\twolineshloka
{शासनात्पार्थिवेन्द्रस्य सूदः सम्भ्रान्तमानसः}
{स राक्षसः पुनस्तत्र सूदवेषमथाकरोत्} %7-65-24

\twolineshloka
{स मानुषमथो मांसं पार्थिवाय न्यवेदयत्}
{इदं स्वादु हविष्यं च सामिषं चान्नमाहृतम्} %7-65-25

\twolineshloka
{स भोजनं वसिष्ठाय पत्न्या सार्धमुपाहरत्}
{मदयन्त्या नरव्याघ्र सामिषं रक्षसा हृतम्} %7-65-26

\twolineshloka
{ज्ञात्वा तदामिषं विप्रो मानुषं भाजनं गतम्}
{क्रोधेन महताऽऽविष्टो व्याहर्तुमुपचक्रमे} %7-65-27

\twolineshloka
{यस्मात्त्वं भोजनं राजन्ममैतद्दातुमिच्छसि}
{तस्माद्भोजनमेतत्ते भविष्यति न संशयः} %7-65-28

\twolineshloka
{ततः क्रुद्धस्तु सौदासस्तोयं जग्राह पाणिना}
{वसिष्ठं शप्तुमारेभे भार्या चैनमवारयत्} %7-65-29

\twolineshloka
{राजन्प्रभुर्यतोऽस्माकं वसिष्ठो भगवानृषिः}
{प्रतिशप्तुं न शक्तस्त्वं देवतुल्यं पुरोधसम्} %7-65-30

\twolineshloka
{ततः क्रोधमयं तोयं तेजोबलसमन्वितम्}
{व्यसर्जयत धर्मात्मा ततः पादौ सिषेच च} %7-65-31

\threelineshloka
{तेनास्य राज्ञस्तौ पादौ तदा कल्माषतां गतौ}
{तदाप्रभृति राजाऽसौ सौदासः सुमहायशाः}
{कल्माषपादः संवृत्तः ख्यातश्चैव तथा नृपः} %7-65-32

\twolineshloka
{स राजा सह पत्न्या वै प्रणिपत्य मुहुर्मुहुः}
{पुनर्वसिष्ठं प्रोवाच यदुक्तं ब्रह्मरूपिणा} %7-65-33

\twolineshloka
{तच्छ्रुत्वा पार्थिवेन्द्रस्य रक्षसा विकृतं च तत्}
{पुनः प्रोवाच राजानं वसिष्ठः पुरुषर्षभम्} %7-65-34

\twolineshloka
{मया रोषपरीतेन यदिदं व्याहृतं वचः}
{नैतच्छक्यं वृथा कर्तुं प्रदास्यामि च ते वरम्} %7-65-35

\twolineshloka
{कालो द्वादश वर्षाणि शापस्यान्तो भविष्यति}
{मत्प्रासादाच्च राजेन्द्र व्यतीतं न स्मरिष्यसि} %7-65-36

\twolineshloka
{एवं स राजा तं शापमुपभुज्यारिसूदनः}
{प्रतिलेभे पुना राज्यं प्रजाश्चैवान्वपालयत्} %7-65-37

\twolineshloka
{तस्य कल्माषपादस्य यज्ञस्यायतनं शुभम्}
{आश्रमस्य समीपेऽस्य यन्मां पृच्छसि राघव} %7-65-38

\twolineshloka
{तस्य तां पार्थिवेन्द्रस्य कथां श्रुत्वा सुदारुणाम्}
{विवेश पर्णशालायां महर्षिमभिवाद्य च} %7-65-39


॥इत्यार्षे श्रीमद्रामायणे वाल्मीकीये आदिकाव्ये उत्तरकाण्डे सौदासकथा नाम पञ्चषष्ठितमः सर्गः ॥७-६५॥
