\sect{पञ्चाशः सर्गः — सुमन्त्ररहस्यकथनम्}

\twolineshloka
{दृष्ट्वा तु मैथिलीं सीतामाश्रमे सम्प्रवेशिताम्}
{सन्तापमगमद्घोरं लक्ष्मणो दीनचेतनः} %7-50-1

\twolineshloka
{अब्रवीच्च महातेजाः सुमन्त्रं मन्त्रसारथिम्}
{सीतासन्तापजं दुःखं पश्य रामस्य सारथे} %7-50-2

\twolineshloka
{ततो दुःखतरं किं नु राघवस्य भविष्यति}
{पत्नीं शुद्धसमाचारां विसृज्य जनकात्मजाम्} %7-50-3

\twolineshloka
{व्यक्तं दैवादहं मन्ये राघवस्य विना भवम्}
{वैदेह्या सारथे नित्यं दैवं हि दुरतिक्रमम्} %7-50-4

\twolineshloka
{यो हि देवान्सगन्धर्वानसुरान्सह राक्षसैः}
{निहन्याद्राघवः क्रुद्धः स दैवमनुवर्तते} %7-50-5

\twolineshloka
{पुरा रामः पितुर्वाक्याद्दण्डके विजने वने}
{उषित्वा नव वर्षाणि पञ्च चैव महावने} %7-50-6

\twolineshloka
{ततो दुःखतरं भूयः सीताया विप्रवासनम्}
{पौराणां वचनं श्रुत्वा नृशंसं प्रतिभाति मे} %7-50-7

\twolineshloka
{को नु धर्माश्रयः सूत कर्मण्यस्मिन्यशोहरे}
{मैथिलीं प्रतिसम्प्राप्तः पौरैर्हीनार्थवादिभिः} %7-50-8

\twolineshloka
{एता वाचो बहुविधाः श्रुत्वा लक्ष्मणभाषिताः}
{सुमन्त्रः श्रद्धया प्राज्ञो वाक्यमेतदुवाच ह} %7-50-9

\twolineshloka
{न सन्तापस्त्वया कार्यः सौमित्रे मैथिलीं प्रति}
{दृष्टमेतत्पुरा विप्रैः पितुस्ते लक्ष्मणाग्रतः} %7-50-10

\twolineshloka
{भविष्यति दृढं रामो दुःखप्रायोऽपि सौख्यभाक्}
{प्राप्स्यते च महाबाहुर्विप्रयोगं प्रियैर्ध्रुवम्} %7-50-11

\twolineshloka
{त्वां चैव मैथिलीं चैव शत्रुघ्नभरतावुभौ}
{सन्त्यजिष्यति धर्मात्मा कालेन महता महान्} %7-50-12

\twolineshloka
{इदं त्वयि न वक्तव्यं सौमित्रे भरतेऽपि वा}
{राज्ञा वो व्याहृतं वाक्यं दुर्वासा यदुवाच ह} %7-50-13

\twolineshloka
{महाजनसमीपे च मम चैव नरर्षभ}
{ऋषिणा व्याहृतं वाक्यं वसिष्ठस्य च सन्निधौ} %7-50-14

\twolineshloka
{ऋषेस्तु वचनं श्रुत्वा मामाह पुरुषर्षभः}
{सूत न क्वचिदेवं ते वक्तव्यं जनसन्निधौ} %7-50-15

\twolineshloka
{तस्याहं लोकपालस्य वाक्यं तत्सुसमाहितः}
{नैव जात्वनृतं कार्यमिति मे सौम्य दर्शनम्} %7-50-16

\twolineshloka
{सर्वथैव न वक्तव्यं मया सौम्य तवाग्रतः}
{यदि ते श्रवणे श्रद्धा श्रूयतां रघुनन्दन} %7-50-17

\twolineshloka
{यद्यप्यहं नरेन्द्रेण रहस्यं श्रावितः पुरा}
{तथाप्युदाहरिष्यामि दैवं हि दुरतिक्रमम्} %7-50-18

\twolineshloka
{येनेदमीदृशं प्राप्तं दुःखं शोकसमन्वितम्}
{न त्वया भरते वाच्यं शत्रुघ्नस्यापि सन्निधौ} %7-50-19

\twolineshloka
{तच्छ्रुत्वा भाषितं तस्य गम्भीरार्थपदं महत्}
{तथ्यं ब्रूहीति सौमित्रिः सूतं वाक्यमथाब्रवीत्} %7-50-20


॥इत्यार्षे श्रीमद्रामायणे वाल्मीकीये आदिकाव्ये उत्तरकाण्डे सुमन्त्ररहस्यकथनम् नाम पञ्चाशः सर्गः ॥७-५०॥
