\sect{सप्ताशीतितमः सर्गः — इलस्त्रीत्वप्राप्तिः}

\twolineshloka
{तच्छ्रुत्वा लक्ष्मणेनोक्तं वाक्यं वाक्यविशारदः}
{प्रत्युवाच महातेजाः प्रहसन्राघवो वचः} %7-87-1

\twolineshloka
{एवमेव नरश्रेष्ठ यथा वदसि लक्ष्मण}
{वृत्रघातमशेषेण वाजिमेधफलं च यत्} %7-87-2

\twolineshloka
{श्रूयते हि पुरा सौम्य कर्दमस्य प्रजापतेः}
{पुत्रो बाह्लीश्वरः श्रीमानिलो नाम महाशयाः} %7-87-3

\twolineshloka
{स राजा पृथिवीं सर्वां वशे कृत्वा सुधार्मिकः}
{राज्यं चैव नरव्याघ्र पुत्रवत्पर्यपालयत्} %7-87-4

\twolineshloka
{सुरैश्च परमोदारैर्दैतेयैश्च महाधनैः}
{नागराक्षसगन्धर्वैर्यक्षैश्च सुमहात्मभिः} %7-87-5

\twolineshloka
{पूज्यते नित्यशः सौम्य भयार्तै रघुनन्दन}
{अबिभ्यंश्च त्रयो लोकाः सरोषस्य महात्मनः} %7-87-6

\twolineshloka
{स राजा तादृशो ह्यासीद्धर्मे वीर्ये च निष्ठितः}
{बुद्ध्या च परमोदारो बाह्लीकेशो महायशाः} %7-87-7

\twolineshloka
{स प्रचक्रे महाबाहुर्मृगयां रुचिरे वने}
{चैत्रे मनोरमे मासि सभृत्यबलवाहनः} %7-87-8

\twolineshloka
{प्रजघ्ने च नृपोऽरण्ये मृगाञ्छतसहस्रशः}
{हत्वैव तृप्तिर्नाभूच्च राज्ञस्तस्य महात्मनः} %7-87-9

\twolineshloka
{नानामृगाणामयुतं वध्यमानं महात्मना}
{यत्र जातो माहासेनस्तं देशमुपचक्रमे} %7-87-10

\twolineshloka
{तस्मिन्प्रदेशे देवेश शैलराजसुतां हरः}
{रमयामास दुर्धर्षः सर्वैरनुचरैः सह} %7-87-11

\twolineshloka
{कृत्वा स्त्रीरूपमात्मानमुमेशो गोपतिध्वजः}
{देव्याः प्रियचिकीर्षुः संस्तस्मिन्पर्वतनिर्झरे} %7-87-12

\twolineshloka
{ये तु तत्र वनोद्देशे सत्त्वाः पुरुषवादिनः}
{वृक्षाः पुरुषनामानस्तेऽभवन् स्त्रीजनास्तदा} %7-87-13

\threelineshloka
{यच्च किञ्चन तत्सर्वं नारीसंज्ञं बभूव ह}
{एतस्मिन्नन्तरे राजा स इलः कर्दमात्मजः}
{निघ्नन्मृगसहस्राणि तं देशमुपचक्रमे} %7-87-14

\twolineshloka
{स दृष्ट्वा स्त्रीकृतं सर्वं सव्यालमृगपक्षकम्}
{आत्मनं स्त्रीकृतं चैव सानुगं रघुनन्दन} %7-87-15

\twolineshloka
{तस्य दुःखं महच्चासीद्दृष्ट्वाऽऽत्मानं तथागतम्}
{उमापतेश्च तत्कर्म ज्ञात्वा त्रासमुपागमत्} %7-87-16

\twolineshloka
{ततो देवं महात्मानं शितिकण्ठं कपर्दिनम्}
{जगाम शरणं राजा सभृत्यबलवाहनः} %7-87-17

\twolineshloka
{ततः प्रहस्य वरदः सह देव्या महेश्वरः}
{प्रजापतिसुतं वाक्यमुवाच वरदः स्वयम्} %7-87-18

\twolineshloka
{उत्तिष्ठोत्तिष्ठ राजर्षे कार्दमेय महाबल}
{पुरुषत्वमृते सौम्य वरं वरय सुव्रत} %7-87-19

\twolineshloka
{ततः स राजा दुःखार्तः प्रत्याख्यातो महात्मना}
{न च जग्राह स्त्रीभूतो वरमन्यं सुरोत्तमात्} %7-87-20

\twolineshloka
{ततः शोकेन महता शैलराजसुतां नृपः}
{प्रणिपत्य ह्युमां देवीं सर्वेणैवान्तरात्मना} %7-87-21

\twolineshloka
{ईशे वराणां वरदे लोकानामसि भामिनी}
{अमोघदर्शने देवी भज सौम्येन चक्षुषा} %7-87-22

\twolineshloka
{हृद्गतं तस्य राजर्षेर्विज्ञाय हरसन्निधौ}
{प्रत्युवाच शुभं वाक्यं देवी रुद्रस्य सम्मता} %7-87-23

\twolineshloka
{अर्धस्य देवो वरदो वरार्धस्य तव ह्यहम्}
{तस्मादर्धं गृहाण त्वं स्त्रीपुंसोर्यावदिच्छसि} %7-87-24

\twolineshloka
{तदद्भुततरं श्रुत्वा देव्या वरमनुत्तमम्}
{सम्प्रहृष्टमना भूत्वा राजा वाक्यमथाब्रवीत्} %7-87-25

\twolineshloka
{यदि देवि प्रसन्ना मे रूपेणाप्रतिमा भुवि}
{मासं स्त्रीत्वमुपासित्वा मासं स्यां पुरुषः पुनः} %7-87-26

\twolineshloka
{ईप्सितं तस्य विज्ञाय देवी सुरुचिरानना}
{प्रत्युवाच शुभं वाक्यमेवमेव भविष्यति} %7-87-27

\twolineshloka
{राजन्पुरुषभूतस्त्वं स्त्रीभावं न स्मरिष्यसि}
{स्त्रीभूतश्च परं मासं न स्मरिष्यसि पौरुषम्} %7-87-28

\twolineshloka
{एवं स राजा पुरुषो मासं भूत्वाथ कार्दमिः}
{त्रैलोक्यसुन्दरी नारी मासमेकमिलाऽभवत्} %7-87-29


॥इत्यार्षे श्रीमद्रामायणे वाल्मीकीये आदिकाव्ये उत्तरकाण्डे इलस्त्रीत्वप्राप्तिः नाम सप्ताशीतितमः सर्गः ॥७-८७॥
