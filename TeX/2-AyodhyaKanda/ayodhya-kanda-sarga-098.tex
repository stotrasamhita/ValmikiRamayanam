\sect{अष्टनवतितमः सर्गः — रामान्वेषणम्}

\twolineshloka
{निवेश्य सेनां तु विभुः पद्भ्यां पादवतां वरः}
{अभिगन्तुं स काकुत्स्थमियेष गुरुवर्त्तकम्} %2-98-1

\twolineshloka
{निविष्टमात्रे सैन्ये तु यथोद्देशं विनीतवत्}
{भरतो भ्रातरं वाक्यं शत्रुघ्नमिदमब्रवीत्} %2-98-2

\twolineshloka
{क्षिप्रं वनमिदं सौम्य नरसङ्घैः समन्ततः}
{लुब्धैश्च सहितैरेभिस्त्वमन्वेषितुमर्हसि} %2-98-3

\twolineshloka
{गुहो ज्ञातिसहस्रेण शरचापासिधारिणा}
{समन्वेषतु काकुत्स्थावस्मिन् परिवृतः स्वयम्} %2-98-4

\twolineshloka
{अमात्यैः सह पौरैश्च गुरुभिश्च द्विजातिभिः}
{वनं सर्वं चरिष्यामि पद्भ्यां परिवृतः स्वयम्} %2-98-5

\twolineshloka
{यावन्न रामं द्रक्ष्यामि लक्ष्मणं वा महाबलम्}
{वैदेहीं वा महाभागां न मे शान्तिर्भविष्यति} %2-98-6

\twolineshloka
{यावन्न चन्द्रसङ्काशं द्रक्ष्यामि शुभमाननम्}
{भ्रातुः पद्मपलाशाक्षं न मे शान्तिर्भविष्यति} %2-98-7

\twolineshloka
{यावन्न चरणौ भ्रातुः पार्थिवव्यञ्जनान्वितौ}
{शिरसा धारयिष्यामि न मे शान्तिर्भविष्यति} %2-98-8

\twolineshloka
{यावन्न राज्ये राज्यार्हः पितृपैतामहे स्थितः}
{अभिषेकजलक्लिन्नो न मे शान्तिर्भविष्यति} %2-98-9

\twolineshloka
{सिद्धार्थः खलु सौमित्रिर्यश्चन्द्रविमलोपमम्}
{मुखं पश्यति रामस्य राजीवाक्षं महाद्युति} %2-98-10

\twolineshloka
{कृतकत्या महाभागा वैदेही जनकात्मजा}
{भर्तारं सागरान्तायाः पृथिव्या याऽनुगच्छति} %2-98-11

\twolineshloka
{सुभगश्चित्रकूटोऽसौ गिरिराजोपमो गिरिः}
{यस्मिन् वसति काकुत्स्थः कुबेर इव नन्दने} %2-98-12

\twolineshloka
{कृतकार्यमिदं दुर्गं वनं व्यालनिषेवितम्}
{यदध्यास्ते महातेजा रामः शस्त्रभृतां वरः} %2-98-13

\twolineshloka
{एवमुक्त्वा महातेजा भरतः पुरुषर्षभः}
{पद्भ्यामेव महाबाहुः प्रविवेश महद्वनम्} %2-98-14

\twolineshloka
{स तानि द्रुमजालानि जातानि गिरिसानुषु}
{पुष्पिताग्राणि मध्येन जगाम वदतां वरः} %2-98-15

\onelineshloka
{समीपत्वात्तन्मूलदर्शनमिति न पुनरुक्तिः} %2-98-16

\twolineshloka
{तं दृष्ट्वा भरतः श्रीमान् मुमोह सहबान्धवः}
{अत्र राम इति ज्ञात्वा गतः पारमिवाम्भसः} %2-98-17

\twolineshloka
{स चित्रकूटे तु गिरौ निशम्य रामाश्रमं पुण्यजनोपपन्नम्}
{गुहेन सार्द्धं त्वरितो जगाम पुनर्निवेश्यैव चमूं महात्मा} %2-98-18


॥इत्यार्षे श्रीमद्रामायणे वाल्मीकीये आदिकाव्ये अयोध्याकाण्डे रामान्वेषणम् नाम अष्टनवतितमः सर्गः ॥२-९८॥
