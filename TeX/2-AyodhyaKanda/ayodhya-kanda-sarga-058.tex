\sect{अष्टपञ्चाशः सर्गः — रामसंदेशाख्यानम्}

\twolineshloka
{प्रत्याश्वस्तः यदा राजा मोहात् प्रत्यागतः पुनः}
{थाजुहाव तम् सूतम् राम वृत्त अन्त कारणात्} %2-58-1

\twolineshloka
{तदा सूतो महाराज कृताञ्जलिरुपस्थितः}
{राममेव अनुशोचन्तं दुःखशोकसमन्वितम्} %2-58-2

\twolineshloka
{वृद्धम् परम सम्तप्तम् नव ग्रहम् इव द्विपम्}
{विनिःश्वसन्तम् ध्यायन्तम् अस्वस्थम् इव कुन्जरम्} %2-58-3

\twolineshloka
{राजा तु रजसा सूतम् ध्वस्त अङ्गम् समुपस्थितम्}
{अश्रु पूर्ण मुखम् दीनम् उवाच परम आर्तवत्} %2-58-4

\twolineshloka
{क्व नु वत्स्यति धर्म आत्मा वृक्ष मूलम् उपाश्रितः}
{सो अत्यन्त सुखितः सूत किम् अशिष्यति राघवः} %2-58-5

\twolineshloka
{दुःखस्यानुचितो दुःखम् सुमन्त्र शयनोचितः}
{भूमि पाल आत्मजो भूमौ शेते कथम् अनाथवत्} %2-58-6

\twolineshloka
{यम् यान्तम् अनुयान्ति स्म पदाति रथ कुण्जराः}
{स वत्स्यति कथम् रामः विजनम् वनम् आश्रितः} %2-58-7

\twolineshloka
{व्याऐः मृगैः आचरितम् कृष्ण सर्प निषेवितम्}
{कथम् कुमारौ वैदेह्या सार्धम् वनम् उपस्थितौ} %2-58-8

\twolineshloka
{सुकुमार्या तपस्विन्या सुमन्त्र सह सीतया}
{राज पुत्रौ कथम् पादैः अवरुह्य रथात् गतौ} %2-58-9

\twolineshloka
{सिद्ध अर्थः खलु सूत त्वम् येन दृष्टौ मम आत्मजौ}
{वन अन्तम् प्रविशन्तौ ताव् अश्विनाव् इव मन्दरम्} %2-58-10

\twolineshloka
{किम् उवाच वचो रामः किम् उवाच च लक्ष्मणः}
{सुमन्त्र वनम् आसाद्य किम् उवाच च मैथिली} %2-58-11

\twolineshloka
{आसितम् शयितम् भुक्तम् सूत रामस्य कीर्तय}
{जीविष्याम्यहमेतेन ययातिरिव साधुषु} %2-58-12

\twolineshloka
{इति सूतः नर इन्द्रेण चोदितः सज्जमानया}
{उवाच वाचा राजानम् स बाष्प परिर्बद्धया} %2-58-13

\twolineshloka
{अब्रवीन् माम् महा राज धर्मम् एव अनुपालयन्}
{अन्जलिम् राघवः कृत्वा शिरसा अभिप्रणम्य च} %2-58-14

\twolineshloka
{सूत मद्वचनात् तस्य तातस्य विदित आत्मनः}
{शिरसा वन्दनीयस्य वन्द्यौ पादौ महात्मनः} %2-58-15

\twolineshloka
{सर्वम् अन्तः पुरम् वाच्यम् सूत मद्वचनात्त्वया}
{आरोग्यम् अविशेषेण यथा अर्हम् च अभिवादनम्} %2-58-16

\twolineshloka
{माता च मम कौसल्या कुशलम् च अभिवादनम्}
{अप्रमादम् च वक्तव्या ब्रूयाश्चैमिदम् वचः} %2-58-17

\twolineshloka
{धर्मनित्या यथाकालमग्न्यगारपरा भव}
{देवि देवस्य पादौ च देववत् परिपालय} %2-58-18

\twolineshloka
{अभिमानम् च मानम् च त्यक्त्वा वर्तस्व मातृषु}
{अनु राजान मार्याम् च कैकेयीमम्ब कारय} %2-58-19

\twolineshloka
{कुमारे भरते वृत्तिर्वर्तितव्याच राजवत्}
{अर्थज्येष्ठा हि राजानो राजधर्ममनुस्मर} %2-58-20

\twolineshloka
{भरतः कुशलम् वाच्यो वाच्यो मद् वचनेन च}
{सर्वास्व एव यथा न्यायम् वृत्तिम् वर्तस्व मातृषु} %2-58-21

\twolineshloka
{वक्तव्यः च महा बाहुर् इक्ष्वाकु कुल नन्दनः}
{पितरम् यौवराज्यस्थो राज्यस्थम् अनुपालय} %2-58-22

\twolineshloka
{अतिक्रान्तवया राजा मास्मैनम् व्यवरोरुधः}
{कुमारराज्ये जीव त्वम् तस्यैवाज्ञ्प्रवर्तनाम्} %2-58-23

\twolineshloka
{अब्रवीच्चापि माम् भूयो भृशमश्रूणि वर्तयन्}
{मातेव मम माता ते द्रष्टव्या पुत्रगर्धिनी} %2-58-24

\twolineshloka
{इति एवम् माम् महाराज बृवन्न् एव महा यशाः}
{रामः राजीव ताम्र अक्षो भृशम् अश्रूणि अवर्तयत्} %2-58-25

\twolineshloka
{लक्ष्मणः तु सुसम्क्रुद्धो निह्श्वसन् वाक्यम् अब्रवीत्}
{केन अयम् अपराधेन राज पुत्रः विवासितः} %2-58-26

\twolineshloka
{राज्ञा तु खलु कैकेय्या लघु त्वाश्रित्य शासनम्}
{कृतम् कार्यमकार्यम् वा वयम् येनाभिपीडिताः} %2-58-27

\twolineshloka
{यदि प्रव्राजितः रामः लोभ कारण कारितम्}
{वर दान निमित्तम् वा सर्वथा दुष्कृतम् कृतम्} %2-58-28

\twolineshloka
{इदम् तावद्यथाकाममीश्वरस्य कृते कृतम्}
{रामस्य तु परित्यागे न हेतुम् उपलक्षये} %2-58-29

\twolineshloka
{असमीक्ष्य समारब्धम् विरुद्धम् बुद्धि लाघवात्}
{जनयिष्यति सम्क्रोशम् राघवस्य विवासनम्} %2-58-30

\twolineshloka
{अहम् तावन् महा राजे पितृत्वम् न उपलक्षये}
{भ्राता भर्ता च बन्धुः च पिता च मम राघवः} %2-58-31

\twolineshloka
{सर्व लोक प्रियम् त्यक्त्वा सर्व लोक हिते रतम्}
{सर्व लोको अनुरज्येत कथम् त्वा अनेन कर्मणा} %2-58-32

\twolineshloka
{सर्वप्रजाभिरामम् हि रामम् प्रव्राज्य धार्मिकम्}
{सर्वलोकम् विरुध्येमम् कथम् राजा भविष्यसि} %2-58-33

\twolineshloka
{जानकी तु महा राज निःश्वसन्ती तपस्विनी}
{भूत उपहत चित्ता इव विष्ठिता वृष्मृता स्थिता} %2-58-34

\twolineshloka
{अदृष्ट पूर्व व्यसना राज पुत्री यशस्विनी}
{तेन दुह्खेन रुदती न एव माम् किम्चित् अब्रवीत्} %2-58-35

\twolineshloka
{उद्वीक्षमाणा भर्तारम् मुखेन परिशुष्यता}
{मुमोच सहसा बाष्पम् माम् प्रयान्तम् उदीक्ष्य सा} %2-58-36

\fourlineindentedshloka
{तथैव रामः अश्रु मुखः कृत अन्जलिः}
{स्थितः अभवल् लक्ष्मण बाहु पालितः स्थितः}
{तथैव सीता रुदती तपस्विनी}
{निरीक्षते राज रथम् तथैव माम्} %2-58-37


॥इत्यार्षे श्रीमद्रामायणे वाल्मीकीये आदिकाव्ये अयोध्याकाण्डे रामसंदेशाख्यानम् नाम अष्टपञ्चाशः सर्गः ॥२-५८॥
