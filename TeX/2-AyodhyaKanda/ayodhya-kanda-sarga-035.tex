\sect{पञ्चत्रिंशः सर्गः — सुमन्त्रगर्हणम्}

\twolineshloka
{ततो निधूय सहसा शिरो निःश्वस्य चासकृत्}
{पाणिं पाणौ विनिष्पिष्य दन्तान् कटकटाय्य च} %2-35-1

\twolineshloka
{लोचने कोपसंरक्ते वर्णं पूर्वोचितं जहत्}
{कोपाभिभूतः सहसा सन्तापमशुभं गतः} %2-35-2

\twolineshloka
{मनः समीक्षमाणश्च सूतो दशरथस्य च}
{कम्पयन्निव कैकेय्या हृदयं वाक्शरैः शितैः} %2-35-3

\twolineshloka
{वाक्यवज्रैरनुपमैर्निर्भिन्दन्निव चाशुभैः}
{कैकेय्याः सर्वमर्माणि सुमन्त्रः प्रत्यभाषत} %2-35-4

\twolineshloka
{यस्यास्तव पतिस्त्यक्तो राजा दशरथः स्वयम्}
{भर्ता सर्वस्य जगतः स्थावरस्य चरस्य च} %2-35-5

\twolineshloka
{नह्यकार्यतमं किञ्चित्तव देवीह विद्यते}
{पतिघ्नीं त्वामहं मन्ये कुलघ्नीमपि चान्ततः} %2-35-6

\twolineshloka
{यन्महेन्द्रमिवाजय्यं दुष्प्रकम्प्यमिवाचलम्}
{महोदधिमिवाक्षोभ्यं सन्तापयसि कर्मभिः} %2-35-7

\twolineshloka
{मावमंस्था दशरथं भर्तारं वरदं पतिम्}
{भर्तुरिच्छा हि नारीणां पुत्रकोट्या विशिष्यते} %2-35-8

\twolineshloka
{यथावयो हि राज्यानि प्राप्नुवन्ति नृपक्षये}
{इक्ष्वाकुकुलनाथेऽस्मिंस्तं लोपयितुमिच्छसि} %2-35-9

\twolineshloka
{राजा भवतु ते पुत्रो भरतः शास्तु मेदिनीम्}
{वयं तत्र गमिष्यामो यत्र रामो गमिष्यति} %2-35-10

\twolineshloka
{न च ते विषये कश्चिद् ब्राह्मणो वस्तुमर्हति}
{तादृशं त्वममर्यादमद्य कर्म करिष्यसि} %2-35-11

\twolineshloka
{नूनं सर्वे गमिष्यामो मार्गं रामनिषेवितम्}
{त्यक्ता या बान्धवैः सर्वैर्ब्राह्मणैः साधुभिः सदा} %2-35-12

\twolineshloka
{का प्रीती राज्यलाभेन तव देवि भविष्यति}
{तादृशं त्वममर्यादं कर्म कर्तुं चिकीर्षसि} %2-35-13

\twolineshloka
{आश्चर्यमिव पश्यामि यस्यास्ते वृत्तमीदृशम्}
{आचरन्त्या न विवृता सद्यो भवति मेदिनी} %2-35-14

\twolineshloka
{महाब्रह्मर्षिसृष्टा वा ज्वलन्तो भीमदर्शनाः}
{धिग्वाग्दण्डा न हिंसन्ति रामप्रव्राजने स्थिताम्} %2-35-15

\twolineshloka
{आम्रं छित्त्वा कुठारेण निम्बं परिचरेत् तु कः}
{यश्चैनं पयसा सिञ्चेन्नैवास्य मधुरो भवेत्} %2-35-16

\twolineshloka
{आभिजात्यं हि ते मन्ये यथा मातुस्तथैव च}
{न हि निम्बात् स्रवेत् क्षौद्रं लोके निगदितं वचः} %2-35-17

\twolineshloka
{तव मातुरसद्ग्राहं विद्म पूर्वं यथा श्रुतम्}
{पितुस्ते वरदः कश्चिद् ददौ वरमनुत्तमम्} %2-35-18

\twolineshloka
{सर्वभूतरुतं तस्मात् सञ्जज्ञे वसुधाधिपः}
{तेन तिर्यग्गतानां च भूतानां विदितं वचः} %2-35-19

\twolineshloka
{ततो जृम्भस्य शयने विरुताद् भूरिवर्चसः}
{पितुस्ते विदितो भावः स तत्र बहुधाहसत्} %2-35-20

\twolineshloka
{तत्र ते जननी क्रुद्धा मृत्युपाशमभीप्सती}
{हासं ते नृपते सौम्य जिज्ञासामीति चाब्रवीत्} %2-35-21

\twolineshloka
{नृपश्चोवाच तां देवीं हासं शंसामि ते यदि}
{ततो मे मरणं सद्यो भविष्यति न संशयः} %2-35-22

\twolineshloka
{माता ते पितरं देवि पुनः केकयमब्रवीत्}
{शंस मे जीव वा मा वा न मां त्वं प्रहसिष्यसि} %2-35-23

\twolineshloka
{प्रियया च तथोक्तः स केकयः पृथिवीपतिः}
{तस्मै तं वरदायार्थं कथयामास तत्त्वतः} %2-35-24

\twolineshloka
{ततः स वरदः साधू राजानं प्रत्यभाषत}
{म्रियतां ध्वंसतां वेयं मा शंसीस्त्वं महीपते} %2-35-25

\twolineshloka
{स तच्छ्रुत्वा वचस्तस्य प्रसन्नमनसो नृपः}
{मातरं ते निरस्याशु विजहार कुबेरवत्} %2-35-26

\twolineshloka
{तथा त्वमपि राजानं दुर्जनाचरिते पथि}
{असद्ग्राहमिमं मोहात् कुरुषे पापदर्शिनी} %2-35-27

\twolineshloka
{सत्यश्चात्र प्रवादोऽयं लौकिकः प्रतिभाति मा}
{पितॄन् समनुजायन्ते नरा मातरमङ्गनाः} %2-35-28

\twolineshloka
{नैवं भव गृहाणेदं यदाह वसुधाधिपः}
{भर्तुरिच्छामुपास्वेह जनस्यास्य गतिर्भव} %2-35-29

\twolineshloka
{मा त्वं प्रोत्साहिता पापैर्देवराजसमप्रभम्}
{भर्तारं लोकभर्तारमसद्धर्ममुपादध} %2-35-30

\twolineshloka
{नहि मिथ्या प्रतिज्ञातं करिष्यति तवानघः}
{श्रीमान् दशरथो राजा देवि राजीवलोचनः} %2-35-31

\twolineshloka
{ज्येष्ठो वदान्यः कर्मण्यः स्वधर्मस्यापि रक्षिता}
{रक्षिता जीवलोकस्य बली रामोऽभिषिच्यताम्} %2-35-32

\twolineshloka
{परिवादो हि ते देवि महाँल्लोके चरिष्यति}
{यदि रामो वनं याति विहाय पितरं नृपम्} %2-35-33

\twolineshloka
{स्वराज्यं राघवः पातु भव त्वं विगतज्वरा}
{नहि ते राघवादन्यः क्षमः पुरवरे वसन्} %2-35-34

\twolineshloka
{रामे हि यौवराज्यस्थे राजा दशरथो वनम्}
{प्रवेक्ष्यति महेष्वासः पूर्ववृत्तमनुस्मरन्} %2-35-35

\twolineshloka
{इति सान्त्वैश्च तीक्ष्णैश्च कैकेयीं राजसंसदि}
{भूयः सङ्क्षोभयामास सुमन्त्रस्तु कृताञ्जलिः} %2-35-36

\twolineshloka
{नैव सा क्षुभ्यते देवी न च स्म परिदूयते}
{न चास्या मुखवर्णस्य लक्ष्यते विक्रिया तदा} %2-35-37


॥इत्यार्षे श्रीमद्रामायणे वाल्मीकीये आदिकाव्ये अयोध्याकाण्डे सुमन्त्रगर्हणम् नाम पञ्चत्रिंशः सर्गः ॥२-३५॥
