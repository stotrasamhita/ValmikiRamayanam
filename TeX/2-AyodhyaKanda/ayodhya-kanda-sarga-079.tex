\sect{एकोनाशीतितमः सर्गः — सचिवप्रार्थनाप्रतिषेधः}

\twolineshloka
{ततः प्रभात समये दिवसे अथ चतुर्दशे}
{समेत्य राज कर्तारः भरतम् वाक्यम् अब्रुवन्} %2-79-1

\twolineshloka
{गतः दशरथः स्वर्गम् यो नो गुरुतरः गुरुः}
{रामम् प्रव्राज्य वै ज्येष्ठम् लक्ष्मणम् च महा बलम्} %2-79-2

\twolineshloka
{त्वम् अद्य भव नो राजा राज पुत्र महा यशः}
{सम्गत्या न अपराध्नोति राज्यम् एतत् अनायकम्} %2-79-3

\twolineshloka
{आभिषेचनिकम् सर्वम् इदम् आदाय राघव}
{प्रतीक्षते त्वाम् स्व जनः श्रेणयः च नृप आत्मज} %2-79-4

\twolineshloka
{राज्यम् गृहाण भरत पितृ पैतामहम् महत्}
{अभिषेचय च आत्मानम् पाहि च अस्मान् नर ऋषभ} %2-79-5

\twolineshloka
{आभिषेचनिकम् भाण्डम् कृत्वा सर्वम् प्रदक्षिणम्}
{भरतः तम् जनम् सर्वम् प्रत्युवाच धृत व्रतः} %2-79-6

\twolineshloka
{ज्येष्ठस्य राजता नित्यम् उचिता हि कुलस्य नः}
{न एवम् भवन्तः माम् वक्तुम् अर्हन्ति कुशला जनाः} %2-79-7

\twolineshloka
{रामः पूर्वो हि नो भ्राता भविष्यति मही पतिः}
{अहम् तु अरण्ये वत्स्यामि वर्षाणि नव पन्च च} %2-79-8

\twolineshloka
{युज्यताम् महती सेना चतुर् अन्ग महा बला}
{आनयिष्याम्य् अहम् ज्येष्ठम् भ्रातरम् राघवम् वनात्} %2-79-9

\twolineshloka
{आभिषेचनिकम् चैव सर्वम् एतत् उपस्कृतम्}
{पुरः कृत्य गमिष्यामि राम हेतोर् वनम् प्रति} %2-79-10

\twolineshloka
{तत्र एव तम् नर व्याघ्रम् अभिषिच्य पुरः कृतम्}
{आनेष्यामि तु वै रामम् हव्य वाहम् इव अध्वरात्} %2-79-11

\twolineshloka
{न सकामा करिष्यामि स्वम् इमाम् मातृ गन्धिनीम्}
{वने वत्स्याम्य् अहम् दुर्गे रामः राजा भविष्यति} %2-79-12

\twolineshloka
{क्रियताम् शिल्पिभिः पन्थाः समानि विषमाणि च}
{रक्षिणः च अनुसम्यान्तु पथि दुर्ग विचारकाः} %2-79-13

\twolineshloka
{एवम् सम्भाषमाणम् तम् राम हेतोर् नृप आत्मजम्}
{प्रत्युवाच जनः सर्वः श्रीमद् वाक्यम् अनुत्तमम्} %2-79-14

\twolineshloka
{एवम् ते भाषमाणस्य पद्मा श्रीर् उपतिष्ठताम्}
{यः त्वम् ज्येष्ठे नृप सुते पृथिवीम् दातुम् इच्चसि} %2-79-15

\fourlineindentedshloka
{अनुत्तमम् तत् वचनम् नृप आत्मज}
{प्रभाषितम् सम्श्रवणे निशम्य च}
{प्रहर्षजाः तम् प्रति बाष्प बिन्दवो}
{निपेतुर् आर्य आनन नेत्र सम्भवाः} %2-79-16

\fourlineindentedshloka
{ऊचुस् ते वचनम् इदम् निशम्य हृष्टाः}
{सामात्याः सपरिषदो वियात शोकाः}
{पन्थानम् नर वर भक्तिमान् जनः च}
{व्यादिष्टः तव वचनाच् च शिल्पि वर्गः} %2-79-17


॥इत्यार्षे श्रीमद्रामायणे वाल्मीकीये आदिकाव्ये अयोध्याकाण्डे सचिवप्रार्थनाप्रतिषेधः नाम एकोनाशीतितमः सर्गः ॥२-७९॥
