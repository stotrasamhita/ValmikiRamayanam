\sect{चतुर्नवतितमः सर्गः — चित्रकूटवर्णनम्}

\twolineshloka
{दीर्घकालोषितस्तस्मिन् गिरौ गिरिवनप्रियः}
{वैदेह्याः प्रियमाकांक्षन् स्वं च चित्तं विलोभयन्} %2-94-1

\twolineshloka
{अथ दाशरथिश्चित्रं चित्रकूटमदर्शयत्}
{भार्य्याममरसङ्काशः शचीमिव पुरन्दरः} %2-94-2

\twolineshloka
{न राज्याद्भ्रंशनं भद्रे न सुहृद्भिर्विनाभवः}
{मनो मे बाधते दृष्ट्वा रमणीयमिमं गिरिम्} %2-94-3

\twolineshloka
{पश्येममचलं भद्रे नानाद्विजगणायुतम्}
{शिखरैः खमिवोद्विद्धैर्द्धातुमद्भिर्विभूषितम्} %2-94-4

\twolineshloka
{केचिद्रजतसङ्काशाः केचित् क्षतजसन्निभाः}
{पीतमाञ्जिष्ठवर्णाश्च केचिन्मणिवरप्रभाः} %2-94-5

\twolineshloka
{पुष्पार्ककेतकाभाश्च केचिज्ज्योतीरसप्रभाः}
{विराजन्तेऽचलेन्द्रस्य देशा धातुविभूषिताः} %2-94-6

\twolineshloka
{नानामृगगणद्वीपितरक्ष्वृक्षगणैर्वृतः}
{अदुष्टैर्भात्ययं शैलो बहुपक्षिसमायुतः} %2-94-7

\twolineshloka
{आम्रजम्ब्वसनैर्लोध्रैः प्रियालैः पनसैर्धवैः}
{अङ्कोलैर्भव्यतिनिशैर्बिल्वतिन्दुकवेणुभिः} %2-94-8

\twolineshloka
{काश्मर्यरिष्टवरुणैर्मधूकैस्तिलकैस्तथा}
{बदर्य्यामलकैर्नीपैर्वेत्रधन्वनबीजकैः} %2-94-9

\twolineshloka
{पुष्पवद्भिः फलोपेतैश्छायावद्भिर्मनोरमैः}
{एवमादिभिराकीर्णः श्रियं पुष्यत्ययं गिरिः} %2-94-10

\twolineshloka
{शैलप्रस्थेषु रम्येषु पश्येमान् रोमहर्षणान्}
{किन्नरान् द्वन्द्वशो भद्रे रममाणान् मनस्विनः} %2-94-11

\twolineshloka
{शाखावसक्तान् खङ्गांश्च प्रवराण्यम्बराणि च}
{पश्च विद्याधरस्त्रीणां क्रीडोद्देशान् मनोरमान्} %2-94-12

\twolineshloka
{जलप्रपातैरुद्भेदैर्निष्यन्दैश्च क्वचित् क्वचित्}
{स्रवद्भिर्भात्ययं शैलः स्रवन्मद इव द्विपः} %2-94-13

\twolineshloka
{गुहासमीरणो गन्धान् नानापुष्पभवान् वहन्}
{घ्राणतर्प्पणमभ्येत्य कं नरं न प्रहर्षयेत्} %2-94-14

\twolineshloka
{यदीह शरदोऽनेकास्त्वया सार्द्धमनिन्दिते}
{लक्ष्मणेन च वत्स्यामि न मां शोकः प्रधक्ष्यति} %2-94-15

\twolineshloka
{बहुपुष्पफले रम्ये नानाद्विजगणायुते}
{विचित्रशिखरे ह्यस्मिन् रतवानस्मि भामिनि} %2-94-16

\twolineshloka
{अनेन वनवासेन मया प्राप्तं फलद्वयम्}
{पितुश्चानृणता धर्मे भरतस्य प्रियं तथा} %2-94-17

\twolineshloka
{वैदेहि रमसे कच्चिच्चित्रकूटे मया सह}
{पश्यन्ती विविधान् भावान् मनोवाक्कायसंयतान्} %2-94-18

\twolineshloka
{इदमेवामृतं प्राहू राज्ञि राजर्षयः परे}
{वनवासं भवार्थाय प्रेत्य मे प्रपितामहाः} %2-94-19

\twolineshloka
{शिलाः शैलस्य शोभन्ते विशालाः शतशोऽभितः}
{बहुला बहुलैर्वर्णैर्नीलपीतसितारुणैः} %2-94-20

\twolineshloka
{निशि भान्त्यचलेन्द्रस्य हुताशनशिखा इव}
{ओषध्यः स्वप्रभालक्ष्या भ्राजमानाः सहस्रशः} %2-94-21

\twolineshloka
{केचित् क्षयनिभा देशाः केचिदुद्यानसन्निभाः}
{केचिदेकशिला भान्ति पर्वतस्यास्य भामिनि} %2-94-22

\twolineshloka
{भित्त्वेव वसुधां भाति चित्रकूटः समुत्थितः}
{चित्रकूटस्य कूटोऽसौ दृश्यते सर्वतः शुभः} %2-94-23

\twolineshloka
{कुष्ठपुन्नागस्ऺथगरभूर्जपत्रोत्तरच्छदान्}
{कामिनां स्वास्तरान् पश्य कुशेशयदलायुतान्} %2-94-24

\twolineshloka
{मृदिताश्चापविद्धाश्च दृश्यन्ते कमलस्रजः}
{कामिभिर्वनिते पश्य फलानि विविधानि च} %2-94-25

\twolineshloka
{वस्वौकसारां नलिनीमत्येतीवोत्तरान् कुरून्}
{पर्वतश्चित्रकूटोऽसौ बहुमूलफलोदकः} %2-94-26

\twolineshloka
{इमं तु कालं वनिते विजह्रिवांस्त्वया च सीते सह लक्ष्मणेन च}
{रतिं प्रपत्स्ये कुलधर्मवर्द्धनीं सतां पथि स्वैर्नियमैः परैः स्थितः} %2-94-27


॥इत्यार्षे श्रीमद्रामायणे वाल्मीकीये आदिकाव्ये अयोध्याकाण्डे चित्रकूटवर्णनम् नाम चतुर्नवतितमः सर्गः ॥२-९४॥
