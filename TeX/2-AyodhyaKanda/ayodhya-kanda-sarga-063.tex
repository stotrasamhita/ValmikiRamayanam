\sect{त्रिषष्ठितमः सर्गः — ऋषिकुमारवधाख्यानम्}

\twolineshloka
{प्रतिबुद्धो मुहुर् तेन शोक उपहत चेतनः}
{अथ राजा दशरथः स चिन्ताम् अभ्यपद्यत} %2-63-1

\twolineshloka
{राम लक्ष्मणयोः चैव विवासात् वासव उपमम्}
{आविवेश उपसर्गः तम् तमः सूर्यम् इव आसुरम्} %2-63-2

\twolineshloka
{सभार्ये निर्गते रामे कौसल्याम् कोसलेश्वरः}
{विवक्षुरसितापाङ्गाम् स्मृवा दुष्कृतमात्मनः} %2-63-3

\twolineshloka
{स राजा रजनीम् षष्ठीम् रामे प्रव्रजिते वनम्}
{अर्ध रात्रे दशरथः सम्स्मरन् दुष्कृतम् कृतम्} %2-63-4

\twolineshloka
{स राजा पुत्रशोकार्तः स्मरन् दुष्कृतमात्मनः}
{कौसल्याम् पुत्र शोक आर्ताम् इदम् वचनम् अब्रवीत्} %2-63-5

\twolineshloka
{यद् आचरति कल्याणि शुभम् वा यदि वा अशुभम्}
{तत् एव लभते भद्रे कर्ता कर्मजम् आत्मनः} %2-63-6

\twolineshloka
{गुरु लाघवम् अर्थानाम् आरम्भे कर्मणाम् फलम्}
{दोषम् वा यो न जानाति स बालैति ह उच्यते} %2-63-7

\twolineshloka
{कश्चित् आम्र वणम् चित्त्वा पलाशामः च निषिन्चति}
{पुष्पम् दृष्ट्वा फले गृध्नुः स शोचति फल आगमे} %2-63-8

\twolineshloka
{अविज्ञाय फलम् यो हि कर्म त्वेवानुधावति}
{स शोचेत्फलवेआयाम् यथा किम्शुकसेचकः} %2-63-9

\twolineshloka
{सो अहम् आम्र वणम् चित्त्वा पलाशामः च न्यषेचयम्}
{रामम् फल आगमे त्यक्त्वा पश्चात् शोचामि दुर्मतिः} %2-63-10

\twolineshloka
{लब्ध शब्देन कौसल्ये कुमारेण धनुष्मता}
{कुमारः शब्द वेधी इति मया पापम् इदम् कृतम्} %2-63-11

\twolineshloka
{तत् इदम् मे अनुसम्प्राप्तम् देवि दुह्खम् स्वयम् कृतम्}
{सम्मोहात् इह बालेन यथा स्यात् भक्षितम् विषम्} %2-63-12

\twolineshloka
{यथान्यः पुरुषः कश्चित्पलाशैर्मोओहितो भवेत्}
{एवम् मम अपि अविज्ञातम् शब्द वेध्यमयम् फलम्} %2-63-13

\twolineshloka
{देव्य् अनूढा त्वम् अभवो युव राजो भवाम्य् अहम्}
{ततः प्रावृड् अनुप्राप्ता मद काम विवर्धिनी} %2-63-14

\twolineshloka
{उपास्यहि रसान् भौमाम्स् तप्त्वा च जगद् अम्शुभिः}
{परेत आचरिताम् भीमाम् रविर् आविशते दिशम्} %2-63-15

\twolineshloka
{उष्णम् अन्तर् दधे सद्यः स्निग्धा ददृशिरे घनाः}
{ततः जहृषिरे सर्वे भेक सारन्ग बर्हिणः} %2-63-16

\twolineshloka
{क्लिन्नपक्षोत्तराः स्नाताः कृच्च्रादिव वतत्रिणः}
{वृष्टिवातावधूताग्रान् पादपानभिपेदिरे} %2-63-17

\twolineshloka
{पतितेन अम्भसा चन्नः पतमानेन च असकृत्}
{आबभौ मत्त सारन्गः तोय राशिर् इव अचलः} %2-63-18

\twolineshloka
{पाण्डुरारुणवर्णानि स्रोओताम्सि विमलान्यपि}
{सुस्रुवुर्गिरिधातुभ्यः सभस्मानि भुजङ्गवत्} %2-63-19

\twolineshloka
{आकुलारुणतोयानि स्रोओताम्सि विमलान्यपि}
{उन्मार्गजलवाहीनि बभूवुर्जलदागमे} %2-63-20

\twolineshloka
{तस्मिन्न् अतिसुखे काले धनुष्मान् इषुमान् रथी}
{व्यायाम कृत सम्कल्पः सरयूम् अन्वगाम् नदीम्} %2-63-21

\twolineshloka
{निपाने महिषम् रात्रौ गजम् वा अभ्यागतम् नदीम्}
{अन्यम् वा श्वा पदम् कम्चिज् जिघाम्सुर् अजित इन्द्रियः} %2-63-22

\twolineshloka
{तस्मिम्स्तत्राहमेकान्ते रात्रौ विवृतकार्मुकः}
{तत्राहम् सम्वृतम् वन्यम् हतवाम्स्तीरमागतम्} %2-63-23

\twolineshloka
{अन्यम् चापि मृगम् हिम्स्रम् शब्दम् श्रुत्वाभु पागतम्}
{अथ अन्ध कारे तु अश्रौषम् जले कुम्भस्य पर्यतः} %2-63-24

\twolineshloka
{अचक्षुर् विषये घोषम् वारणस्य इव नर्दतः}
{ततः अहम् शरम् उद्धृत्य दीप्तम् आशी विष उपमम्} %2-63-25

\twolineshloka
{शब्दम् प्रति गजप्रेप्सुरभिलक्ष्य त्वपातयम्}
{अमुन्चम् निशितम् बाणम् अहम् आशी विष उपमम्} %2-63-26

\twolineshloka
{तत्र वाग् उषसि व्यक्ता प्रादुर् आसीद् वन ओकसः}
{हा हा इति पततः तोये बाणाभिहतमर्मणः} %2-63-27

\twolineshloka
{तस्मिन्निपतिते बाणे वागभूत्तत्र मानुषी}
{कथम् अस्मद् विधे शस्त्रम् निपतेत् तु तपस्विनि} %2-63-28

\twolineshloka
{प्रविविक्ताम् नदीम् रात्राव् उदाहारः अहम् आगतः}
{इषुणा अभिहतः केन कस्य वा किम् कृतम् मया} %2-63-29

\twolineshloka
{ऋषेर् हि न्यस्त दण्डस्य वने वन्येन जीवतः}
{कथम् नु शस्त्रेण वधो मद् विधस्य विधीयते} %2-63-30

\twolineshloka
{जटा भार धरस्य एव वल्कल अजिन वाससः}
{को वधेन मम अर्थी स्यात् किम् वा अस्य अपकृतम् मया} %2-63-31

\twolineshloka
{एवम् निष्फलम् आरब्धम् केवल अनर्थ सम्हितम्}
{न कश्चित् साधु मन्येत यथैव गुरु तल्पगम्} %2-63-32

\twolineshloka
{नहम् तथा अनुशोचामि जीवित क्षयम् आत्मनः}
{मातरम् पितरम् च उभाव् अनुशोचामि मद् विधे} %2-63-33

\twolineshloka
{तत् एतान् मिथुनम् वृद्धम् चिर कालभृतम् मया}
{मयि पन्चत्वम् आपन्ने काम् वृत्तिम् वर्तयिष्यति} %2-63-34

\twolineshloka
{वृद्धौ च माता पितराव् अहम् च एक इषुणा हतः}
{केन स्म निहताः सर्वे सुबालेन अकृत आत्मना} %2-63-35

\twolineshloka
{तम् गिरम् करुणाम् श्रुत्वा मम धर्म अनुकान्क्षिणः}
{कराभ्याम् सशरम् चापम् व्यथितस्य अपतत् भुवि} %2-63-36

\twolineshloka
{तस्याहम् करुणम् श्रुत्वा निशि लालपतो बहु}
{सम्भ्रानतः शोकवेगेन भृशमास विचेतनः} %2-63-37

\twolineshloka
{तम् देशम् अहम् आगम्य दीन सत्त्वः सुदुर्मनाः}
{अपश्यम् इषुणा तीरे सरय्वाः तापसम् हतम्} %2-63-38

\twolineshloka
{अवकीर्णजटाभारम् प्रविद्धकलशोदकम्}
{पासुशोणितदिग्धाङ्गम् शयानम् शल्यपीडितम्} %2-63-39

\twolineshloka
{स माम् उद्वीक्ष्य नेत्राभ्याम् त्रस्तम् अस्वस्थ चेतसम्}
{इति उवाच वचः क्रूरम् दिधक्षन्न् इव तेजसा} %2-63-40

\twolineshloka
{किम् तव अपकृतम् राजन् वने निवसता मया}
{जिहीर्षिउर् अम्भो गुर्व् अर्थम् यद् अहम् ताडितः त्वया} %2-63-41

\twolineshloka
{एकेन खलु बाणेन मर्मणि अभिहते मयि}
{द्वाव् अन्धौ निहतौ वृद्धौ माता जनयिता च मे} %2-63-42

\twolineshloka
{तौ नूनम् दुर्बलाव् अन्धौ मत् प्रतीक्षौ पिपासितौ}
{चिरम् आशा कृताम् तृष्णाम् कष्टाम् सम्धारयिष्यतः} %2-63-43

\twolineshloka
{न नूनम् तपसो वा अस्ति फल योगः श्रुतस्य वा}
{पिता यन् माम् न जानाति शयानम् पतितम् भुवि} %2-63-44

\twolineshloka
{जानन्न् अपि च किम् कुर्यात् अशक्तिर् अपरिक्रमः}
{चिद्यमानम् इव अशक्तः त्रातुम् अन्यो नगो नगम्} %2-63-45

\twolineshloka
{पितुस् त्वम् एव मे गत्वा शीघ्रम् आचक्ष्व राघव}
{न त्वाम् अनुदहेत् क्रुद्धो वनम् वह्निर् इव एधितः} %2-63-46

\twolineshloka
{इयम् एक पदी राजन् यतः मे पितुर् आश्रमः}
{तम् प्रसादय गत्वा त्वम् न त्वाम् स कुपितः शपेत्} %2-63-47

\twolineshloka
{विशल्यम् कुरु माम् राजन् मर्म मे निशितः शरः}
{रुणद्धि मृदु स उत्सेधम् तीरम् अम्बु रयो यथा} %2-63-48

\twolineshloka
{सशल्यः क्लिश्यते प्राणैर्विशल्यो विनशिष्यति}
{इति मामविशच्चिन्ता तस्य शल्यापकर्षणे} %2-63-49

\twolineshloka
{दुःखितस्य च दीनस्य मम शोकातुरस्य च}
{लक्ष्यामास हृदये चिन्ताम् मुनिसुत स्तदा} %2-63-50

\twolineshloka
{ताम्यमानः स माम् दुःखादुवाच परमार्तवत्}
{सीदमानो विवृत्ताङ्गो वेष्टमानो गतः क्षयम्} %2-63-51

\twolineshloka
{सम्स्तभ्य धैर्येण स्थिरचित्तो भवाम्यहम्}
{ब्रह्महत्याकृतम् पापम् हृदयादपनीयताम्} %2-63-52

\twolineshloka
{न द्विजातिर् अहम् राजन् मा भूत् ते मनसो व्यथा}
{शूद्रायाम् अस्मि वैश्येन जातः जन पद अधिप} %2-63-53

\twolineshloka
{इति इव वदतः कृच्च्रात् बाण अभिहत मर्मणः}
{विघूर्णतो विचेष्टस्य वेपमाचस्य भूतले} %2-63-54

\twolineshloka
{तस्य तु आनम्यमानस्य तम् बाणम् अहम् उद्धरम्}
{तस्य त्वानम्यमानस्य तम् बाणामहमुद्धरम्} %2-63-55

\fourlineindentedshloka
{जल आर्द्र गात्रम् तु विलप्य कृच्चान्}
{मर्म व्रणम् सम्ततम् उच्चसन्तम्}
{ततः सरय्वाम् तम् अहम् शयानम्}
{समीक्ष्य भद्रे सुभृशम् विषण्णः} %2-63-56


॥इत्यार्षे श्रीमद्रामायणे वाल्मीकीये आदिकाव्ये अयोध्याकाण्डे ऋषिकुमारवधाख्यानम् नाम त्रिषष्ठितमः सर्गः ॥२-६३॥
