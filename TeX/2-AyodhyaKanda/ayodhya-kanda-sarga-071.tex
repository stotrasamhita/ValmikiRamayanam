\sect{एकसप्ततितमः सर्गः — अयोध्यागमनम्}

\twolineshloka
{स प्रान् मुखो राज गृहात् अभिनिर्याय वीर्यवान्}
{ततः सुदामाम् द्युतिमान् सम्तीर्वावेक्ष्य ताम् नदीम्} %2-71-1

\twolineshloka
{ह्लादिनीम् दूर पाराम् च प्रत्यक् स्रोतः तरन्गिणीम्}
{शतद्रूम् अतरत् श्रीमान् नदीम् इक्ष्वाकु नन्दनः} %2-71-2

\twolineshloka
{एल धाने नदीम् तीर्त्वा प्राप्य च अपर पर्पटान्}
{शिलाम् आकुर्वतीम् तीर्त्वाआग्नेयम् शल्य कर्तनम्} %2-71-3

\twolineshloka
{सत्य सम्धः शुचिः श्रीमान् प्रेक्षमाणः शिला वहाम्}
{अत्ययात् स महा शैलान् वनम् चैत्र रथम् प्रति} %2-71-4

\twolineshloka
{सरस्वतीम् च गङ्गाम् च उग्मेन प्रतिपद्य च}
{उत्तरम् वीरमत्स्यानाम् भारुण्डम् प्राविशद्वनम्} %2-71-5

\twolineshloka
{वेगिनीम् च कुलिन्ग आख्याम् ह्रादिनीम् पर्वत आवृताम्}
{यमुनाम् प्राप्य सम्तीर्णो बलम् आश्वासयत् तदा} %2-71-6

\twolineshloka
{शीतीकृत्य तु गात्राणि क्लान्तान् आश्वास्य वाजिनः}
{तत्र स्नात्वा च पीत्वा च प्रायात् आदाय च उदकम्} %2-71-7

\twolineshloka
{राज पुत्रः महा अरण्यम् अनभीक्ष्ण उपसेवितम्}
{भद्रः भद्रेण यानेन मारुतः खम् इव अत्ययात्} %2-71-8

\twolineshloka
{भागीरथीम् दुष्प्रतरामम्शुधाने महानदीम्}
{उपायाद्राघवस्तूर्णम् प्राग्वटे विश्रुते पुरे} %2-71-9

\twolineshloka
{स गङ्गाम् प्राग्वट् तीर्त्वे समायात्कुटिकोष्ठिकाम्}
{सबलस्ताम् स तीर्त्वाथ समायाद्धर्मवर्धनम्} %2-71-10

\twolineshloka
{तोरणम् दक्षिण अर्धेन जम्बू प्रस्थम् उपागमत्}
{वरूथम् च ययौ रम्यम् ग्रामम् दशरथ आत्मजः} %2-71-11

\twolineshloka
{तत्र रम्ये वने वासम् कृत्वा असौ प्रान् मुखो ययौ}
{उद्यानम् उज्जिहानायाः प्रियका यत्र पादपाः} %2-71-12

\twolineshloka
{सालाम्स् तु प्रियकान् प्राप्य शीघ्रान् आस्थाय वाजिनः}
{अनुज्ञाप्य अथ भरतः वाहिनीम् त्वरितः ययौ} %2-71-13

\twolineshloka
{वासम् कृत्वा सर्व तीर्थे तीर्त्वा च उत्तानकाम् नदीम्}
{अन्या नदीः च विविधाः पार्वतीयैअः तुरम् गमैः} %2-71-14

\twolineshloka
{हस्ति पृष्ठकम् आसाद्य कुटिकाम् अत्यवर्तत}
{ततार च नर व्याघ्रः लौहित्ये स कपीवतीम्} %2-71-15

\twolineshloka
{एक साले स्थाणुमतीम् विनते गोमतीम् नदीम्}
{कलिन्ग नगरे च अपि प्राप्य साल वनम् तदा} %2-71-16

\twolineshloka
{भरतः क्षिप्रम् आगच्चत् सुपरिश्रान्त वाहनः}
{वनम् च समतीत्य आशु शर्वर्याम् अरुण उदये} %2-71-17

\twolineshloka
{अयोध्याम् मनुना राज्ञा निर्मिताम् स ददर्श ह}
{ताम् पुरीम् पुरुष व्याघ्रः सप्त रात्र उषिटः पथि} %2-71-18

\twolineshloka
{अयोध्याम् अग्रतः दृष्ट्वा रथे सारथिम् अब्रवीत्}
{एषा न अतिप्रतीता मे पुण्य उद्याना यशस्विनी} %2-71-19

\twolineshloka
{अयोध्या दृश्यते दूरात् सारथे पाण्डु मृत्तिका}
{यज्वभिर् गुण सम्पन्नैः ब्राह्मणैः वेद पारगैः} %2-71-20

\twolineshloka
{भूयिष्ठम् ऋषैः आकीर्णा राज ऋषि वर पालिता}
{अयोध्यायाम् पुरा शब्दः श्रूयते तुमुलो महान्} %2-71-21

\twolineshloka
{समन्तान् नर नारीणाम् तम् अद्य न शृणोम्य् अहम्}
{उद्यानानि हि साय अह्ने क्रीडित्वा उपरतैः नरैः} %2-71-22

\twolineshloka
{समन्तात् विप्रधावद्भिः प्रकाशन्ते मम अन्यदा}
{तानि अद्य अनुरुदन्ति इव परित्यक्तानि कामिभिः} %2-71-23

\twolineshloka
{अरण्य भूता इव पुरी सारथे प्रतिभाति मे}
{न हि अत्र यानैः दृश्यन्ते न गजैः न च वाजिभिः} %2-71-24

\twolineshloka
{निर्यान्तः वा अभियान्तः वा नर मुख्या यथा पुरम्}
{उद्यानानि पुरा भान्ति मत्तप्रमुदितानि च} %2-71-25

\twolineshloka
{जनानाम् रतिसम्योगेष्वत्यन्तगुणवन्ति च}
{तान्येतान्यद्य वश्यामि निरानन्दानि सर्वशः} %2-71-26

\twolineshloka
{स्रस्तपर्णैरनुपथम् विक्रोशद्भिरिव द्रुमैः}
{नाद्यापि श्रूयते शब्दो मत्तानाम् मृगपक्षिणाम्} %2-71-27

\twolineshloka
{सम्रक्ताम् मधुराम् वाणीम् कलम् व्याहरताम् बहु}
{चन्दनागुरुसम्पृक्तो धूपसम्मूर्चितोऽतुलः} %2-71-28

\twolineshloka
{प्रवाति पवनः श्रीमान् किम् नु नाद्य यथापुरम्}
{भेरीमृदङ्गवीणानाम् कोणसम्घट्टितः पुनः} %2-71-29

\twolineshloka
{किमद्य शब्दो विरतः सदाऽदीनगतिः पुरा}
{अनिष्टानि च पापानि पश्यामि विविधानि च} %2-71-30

\twolineshloka
{निमित्तानि अमनोज्ञानि तेन सीदति ते मनः}
{सर्वथा कुशलम् सूत दुर्लभम् मम बन्धुषु} %2-71-31

\twolineshloka
{तथा ह्यसति सम्मोहे हृदयम् सीदतीव मे}
{विषण्णः शान्तहृदयस्त्रस्तः सुलुलितेन्द्रियः} %2-71-32

\twolineshloka
{भरतः प्रविवेशाशु पुरीमिक्ष्वाकुपालिताम्}
{द्वारेण वैजयन्तेन प्राविशत् श्रान्त वाहनः} %2-71-33

\twolineshloka
{द्वाह्स्थैः उत्थाय विजयम् पृष्टः तैः सहितः ययौ}
{स तु अनेक अग्र हृदयो द्वाह्स्थम् प्रत्यर्च्य तम् जनम्} %2-71-34

\twolineshloka
{सूतम् अश्व पतेः क्लान्तम् अब्रवीत् तत्र राघवः}
{किमहम् त्वरयानीतः कारणेन विनानघ} %2-71-35

\twolineshloka
{अशुभाशङ्कि हृदयम् शीलम् च पततीव मे}
{श्रुता नो यादृशाः पूर्वम् नृपतीनाम् विनाशने} %2-71-36

\twolineshloka
{आकाराः तान् अहम् सर्वान् इह पश्यामि सारथे}
{सम्मार्जनविहीनानि परुषाण्युपलक्षये} %2-71-37

\twolineshloka
{असम्यतकवाटानि श्रीविहीनानि सर्वशः}
{बलिकर्मविहीनानि धूपसम्मेदनेन च} %2-71-38

\twolineshloka
{अनाशितकुटुम्बानि प्रभाहीनजनानि च}
{अलक्स्मीकानि पश्यामि कुटुम्बिभवनान्यहम्} %2-71-39

\twolineshloka
{अपेतमाल्यशोभानि असम्मृष्टाजिराणि च}
{देवागाराणि शून्यानि न चाभान्ति यथापुरम्} %2-71-40

\twolineshloka
{देवतार्चाः प्रविद्धाश्च यज्ञ्गोष्ठ्यस्तथाविधाः}
{माल्यापणेषु राजन्ते नाद्य पण्यानि वा तथा} %2-71-41

\twolineshloka
{दृश्यन्ते वणिजोऽप्यद्य न यथापूर्वमत्रवै}
{ध्यानसम्विग्नहृदयाः नष्टव्यापारयन्त्रिताः} %2-71-42

\onelineshloka
{देवायतनचैत्येषुदीनाः पक्षिगणास्तथा} %2-71-43

\twolineshloka
{मलिनम् च अश्रु पूर्ण अक्षम् दीनम् ध्यान परम् कृशम्}
{सस्त्री पुम्सम् च पश्यामि जनम् उत्कण्ठितम् पुरे} %2-71-44

\twolineshloka
{इति एवम् उक्त्वा भरतः सूतम् तम् दीन मानसः}
{तानि अनिष्टानि अयोध्यायाम् प्रेक्ष्य राज गृहम् ययौ} %2-71-45

\fourlineindentedshloka
{ताम् शून्य शृन्ग अटक वेश्म रथ्याम्}
{रजो अरुण द्वार कपाट यन्त्राम्}
{दृष्ट्वा पुरीम् इन्द्र पुरी प्रकाशाम्}
{दुह्खेन सम्पूर्णतरः बभूव} %2-71-46

\fourlineindentedshloka
{बहूनि पश्यन् मनसो अप्रियाणि}
{यानि अन्न्यदा न अस्य पुरे बभूवुः}
{अवाक् शिरा दीन मना नहृष्टः}
{पितुर् महात्मा प्रविवेश वेश्म} %2-71-47


॥इत्यार्षे श्रीमद्रामायणे वाल्मीकीये आदिकाव्ये अयोध्याकाण्डे अयोध्यागमनम् नाम एकसप्ततितमः सर्गः ॥२-७१॥
