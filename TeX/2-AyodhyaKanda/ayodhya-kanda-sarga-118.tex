\sect{अष्टादशाधिकशततमः सर्गः — दिव्यालङ्कारग्रहणम्}

\twolineshloka
{सा त्वेवमुक्ता वैदेही त्वनसूयाऽनसूयया}
{प्रतिपूज्य वचो मन्दं प्रवक्तुमुपचक्रमे} %2-118-1

\twolineshloka
{नैतदाश्चर्य्यमार्याया यन्मां त्वमनुभाषसे}
{विदितं तु ममाप्येतद्यथा नार्य्याः पतिर्गुरुः} %2-118-2

\twolineshloka
{यद्यप्येष भवेद्भर्ता ममार्ये वृत्तवर्जितः}
{अद्वैधमुपचर्तव्यस्तथाप्येष मया भवेत्} %2-118-3

\twolineshloka
{किं पुनर्यो गुणः श्लाघ्यः सानुक्रोशो जितेन्द्रियः}
{स्थिरानुरागो धर्मात्मा मातृवत्पितृवत्प्रियः} %2-118-4

\twolineshloka
{यां वृत्तिं वर्त्तते रामः कौसल्यायां महाबलः}
{तामेव नृपनारीणामन्यासामपि वर्त्तते} %2-118-5

\twolineshloka
{सकृद्दृष्टास्वपि स्त्रीषु नृपेण नृपवत्सलः}
{मातृवद्वर्त्तते वीरो मानमुत्सृज्य धर्मवित्} %2-118-6

\twolineshloka
{आगच्छन्त्याश्च विजनं वनमेवं भयावहम्}
{समाहितं मे श्वश्र्वा च हृदये तद्धृतं महत्} %2-118-7

\twolineshloka
{पाणिप्रदानकाले च यत्पुरा त्वग्निसन्निधौ}
{अनुशिष्टा जनन्याऽस्मि वाक्यं तदपि मे धृतम्} %2-118-8

\twolineshloka
{नवीकृतं च तत्सर्वं वाक्यैस्ते धर्मचारिणि}
{पतिशुश्रूषणान्नार्य्यास्तपो नान्यद्विधीयते} %2-118-9

\twolineshloka
{सावित्री पतिशुश्रूषां कृत्वा स्वर्गे महीयते}
{तथावृत्तिश्च याता त्वं पतिशुश्रूषया दिवम्} %2-118-10

\twolineshloka
{वरिष्ठा सर्वनारीणामेषा च दिवि देवता}
{रोहिणी न विना चन्द्रं मुहूर्त्तमपि दृश्यते} %2-118-11

\twolineshloka
{एवंविधाश्च प्रवराः स्त्रियो भर्तृदृढव्रताः}
{देवलोके महीयन्ते पुण्येन स्वेन कर्मणा} %2-118-12

\twolineshloka
{ततोऽनसूया संहृष्टा श्रुत्वोक्तं सीतया वचः}
{शिरस्याघ्राय चोवाच मैथिलीं हर्षयन्त्युत} %2-118-13

\twolineshloka
{नियमैर्विविधैराप्तं तपो हि महदस्ति मे}
{तत्संश्रित्य बलं सीते छन्दये त्वां शुचिस्मिते} %2-118-14

\twolineshloka
{उपपन्नं मनोज्ञं च वचनं तव मैथिलि}
{प्रीता चास्म्युचितं किं ते करवाणि ब्रवीहि मे} %2-118-15

\onelineshloka
{तस्यास्तद्वचनं श्रुत्वा विस्मिता मन्दविस्मया कृतमित्यब्रवीत्सीता तपोबलसमन्विताम्} %2-118-16

\twolineshloka
{सा त्वेवमुक्ता धर्मज्ञा तया प्रीततराऽभवत्}
{सफलं च प्रहर्षं ते हन्त सीते करोम्यहम्} %2-118-17

\twolineshloka
{इदं दिव्यं वरं माल्यं वस्त्रमाभरणानि च}
{अङ्गरागं च वैदेहि महार्हं चानुलेपनम्} %2-118-18

\twolineshloka
{मया दत्तमिदं सीते तव गात्राणि शोभयेत्}
{अनुरूपमसंक्लिष्टं नित्यमेव भविष्यति} %2-118-19

\twolineshloka
{अङ्गरागेण दिव्येन लिप्ताङ्गी जनकात्मजे}
{शोभयिष्यसि भर्त्तारं यथा श्रीविष्णुमव्ययम्} %2-118-20

\twolineshloka
{सा वस्त्रमङ्गरागं च भूषणानि स्रजस्तथा}
{मैथिली प्रतिजग्राह प्रीतिदानमनुत्तमम्} %2-118-21

\twolineshloka
{प्रतिगृह्य च तत् सीता प्रीतिदानं यशस्विनी}
{श्लिष्टाञ्जलिपुटा तत्र समुपास्त तपोधनाम्} %2-118-22

\twolineshloka
{तथा सीतासुपासीनामनसूया दृढव्रता}
{वचनं प्रष्टुमारेभे काञ्चित् प्रियकथामनु} %2-118-23

\twolineshloka
{स्वयंवरे किल प्राप्ता त्वमनेन यशस्विना}
{राघवेणेति मे सीते कथा श्रुतिमुपागता} %2-118-24

\twolineshloka
{तां कथां श्रोतुमिच्छामि विस्तरेण च मैथिलि}
{यथानुभूतं कार्त्स्न्येन तन्मे त्वं वक्तुमर्हसि} %2-118-25

\twolineshloka
{एवमुक्ता तु सा सीता तां ततो धर्मचारिणीम्}
{श्रूयतामिति चोक्त्वा वै कथयामास तां कथाम्} %2-118-26

\twolineshloka
{मिथिलाधिपतिर्वीरो जनको नाम धर्मवित्}
{क्षत्रधर्मे ह्यभिरतो न्यायतः शास्ति मेदिनीम्} %2-118-27

\twolineshloka
{तस्य लाङ्गलहस्तस्य कर्षतः क्षेत्रमण्डलम्}
{अहं किलोत्थिता भित्त्वा जगतीं नृपतेः सुता} %2-118-28

\twolineshloka
{स मां दृष्ट्वा नरपतिर्मुष्टिविक्षेपतत्परः}
{पांसुकुण्ठितसर्वाङ्गीं जनको विस्मितोऽभवत्} %2-118-29

\twolineshloka
{अनपत्येन च स्नेहादङ्कमारोप्य च स्वयम्}
{ममेयं तनयेत्युक्त्वा स्नेहो मयि निपातितः} %2-118-30

\twolineshloka
{अन्तरिक्षे च वागुक्ता प्रति माऽमानुषी किल}
{एवमेतन्नरपते धर्मेण तनया तव} %2-118-31

\twolineshloka
{ततः प्रहृष्टो धर्मात्मा पिता मे मिथिलाधिपः}
{अवाप्तो विपुलामृद्धिं मामवाप्य नराधिपः} %2-118-32

\twolineshloka
{दत्ता चास्मीष्टवद्देव्यै ज्येष्ठायै पुण्यकर्मणा}
{तया सम्भाविता चास्मि स्निग्धया मातृसौहृदात्} %2-118-33

\twolineshloka
{पतिसंयोगसुलभं वयो दृष्ट्वा तु मे पिता}
{चिन्तामभ्यगमद्दीनो वित्तनाशादिवाधनः} %2-118-34

\twolineshloka
{सदृशाच्चापकृष्टाच्च लोके कन्यापिता जनात्}
{प्रधर्षणमवाप्नोति शक्रेणापि समो भुवि} %2-118-35

\twolineshloka
{तां धर्षणामदूरस्थां दृष्ट्वा चात्मनि पार्थिवः}
{चिन्तार्णवगतः पारं नाससादाप्लवो यथा} %2-118-36

\twolineshloka
{अयोनिजां हि मां ज्ञात्वा नाध्यगच्छद्विचिन्तयन्}
{सदृशं चानुरूपं च महीपालः पतिं मम} %2-118-37

\twolineshloka
{तस्य बुद्धिरियं जाता चिन्तयानस्य सन्ततम्}
{स्वयं वरं तनूजायाः करिष्यामीति धीमतः} %2-118-38

\twolineshloka
{महायज्ञे तदा तस्य वरुणेन महात्मना}
{दत्तं धनुर्वरं प्रीत्या तूणी चाक्षयसायकौ} %2-118-39

\twolineshloka
{असञ्चाल्यं मनुष्यैश्च यत्नेनापि च गौरवात्}
{तन्न शक्ता नमयितुं स्वप्नेष्वपि नराधिपाः} %2-118-40

\twolineshloka
{तद्धनुः प्राप्य मे पित्रा व्याहृतं सत्यवादिना}
{समवाये नरेन्द्राणां पूर्वमामन्त्र्य पार्थिवान्} %2-118-41

\twolineshloka
{इदं च धनुरुद्यम्य सज्यं यः कुरुते नरः}
{तस्य मे दुहिता भार्या भविष्यति न संशयः} %2-118-42

\twolineshloka
{तच्च दृष्ट्वा धनुः श्रेष्ठं गौरवाद्गिरिसन्निभम्}
{अभिवाद्य नृपा जग्मुरशक्तास्तस्य तोलने} %2-118-43

\threelineshloka
{सुदीर्घस्य तु कालस्य राघवोऽयं महाद्युतिः}
{विश्वामित्रेण सहितो यज्ञं द्रष्टुं समागतः}
{लक्ष्मणेन सह भ्रात्रा रामः सत्यपराक्रमः} %2-118-44

\twolineshloka
{विश्वामित्रस्तु धर्मात्मा मम पित्रा सुपूजितः}
{प्रोवाच पितरं तत्र भ्रातरौ रामलक्ष्मणौ} %2-118-45

\twolineshloka
{सुतौ दशरथस्येमौ धनुर्दर्शनकांक्षिणौ}
{धनुर्दर्शय रामाय राजपुत्राय दैविकम्} %2-118-46

\twolineshloka
{इत्युक्तस्तेन विप्रेण तद्धनुः समुपानयत्}
{निमेषान्तरमात्रेण तदानम्य स वीर्य्यवान्} %2-118-47

\onelineshloka
{ज्यां समारोप्य झटिति पूरयामास वीर्यवत्} %2-118-48

\twolineshloka
{तेन पूरयता वेगान्मध्ये भग्नं द्विधा धनुः}
{तस्य शब्दोऽभवद्भीमः पतितस्याशनेरिव} %2-118-49

\twolineshloka
{ततोऽहं तत्र रामाय पित्रा सत्याभिसन्धिना}
{निश्चिता दातुमुद्यम्य जलभाजनमुत्तमम्} %2-118-50

\twolineshloka
{दीयमानां न तु तदा प्रतिजग्राह राघवः}
{अविज्ञाय पितुश्छन्दमयोध्याधिपतेः प्रभोः} %2-118-51

\twolineshloka
{ततः श्वशुरमामन्त्र्य वृद्धं दशरथं नृपम्}
{मम पित्रा त्वहं दत्ता रामाय विदितात्मने} %2-118-52

\twolineshloka
{मम चैवानुजा साध्वी ऊर्मिला प्रियदर्शना}
{भार्य्यार्थे लक्ष्मणस्यापि दत्ता पित्रा मम स्वयम्} %2-118-53

\twolineshloka
{एवं दत्तास्मि रामाय तदा तस्मिन् स्वयम्वरे}
{अनुरक्तास्मि धर्मेण पतिं वीर्यवतां वरम्} %2-118-54


॥इत्यार्षे श्रीमद्रामायणे वाल्मीकीये आदिकाव्ये अयोध्याकाण्डे दिव्यालङ्कारग्रहणम् नाम अष्टादशाधिकशततमः सर्गः ॥२-११८॥
