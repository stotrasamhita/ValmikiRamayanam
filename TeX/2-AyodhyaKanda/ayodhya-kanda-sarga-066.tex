\sect{षट्षष्ठितमः सर्गः — तैलद्रोण्यधिशयनम्}

\twolineshloka
{तम् अग्निम् इव सम्शान्तम् अम्बु हीनम् इव अर्णवम्}
{हतप्रभम् इव आदित्यम् स्वर्गथम् प्रेक्ष्य भूमिपम्} %2-66-1

\twolineshloka
{कौसल्या बाष्प पूर्ण अक्षी विविधम् शोक कर्शिता}
{उपगृह्य शिरः राज्ञः कैकेयीम् प्रत्यभाषत} %2-66-2

\twolineshloka
{सकामा भव कैकेयि भुन्क्ष्व राज्यम् अकण्टकम्}
{त्यक्त्वा राजानम् एक अग्रा नृशम्से दुष्ट चारिणि} %2-66-3

\twolineshloka
{विहाय माम् गतः रामः भर्ता च स्वर् गतः मम}
{विपथे सार्थ हीना इव न अहम् जीवितुम् उत्सहे} %2-66-4

\twolineshloka
{भर्तारम् तम् परित्यज्य का स्त्री दैवतम् आत्मनः}
{इच्चेज् जीवितुम् अन्यत्र कैकेय्याः त्यक्त धर्मणः} %2-66-5

\twolineshloka
{न लुब्धो बुध्यते दोषान् किम् पाकम् इव भक्षयन्}
{कुब्जा निमित्तम् कैकेय्या राघवाणान् कुलम् हतम्} %2-66-6

\twolineshloka
{अनियोगे नियुक्तेन राज्ञा रामम् विवासितम्}
{सभार्यम् जनकः श्रुत्वा पतितप्स्यति अहम् यथा} %2-66-7

\twolineshloka
{स मामनाथाम् विधवाम् नाद्य जानाति धार्मिकः}
{रामः कमल पत्र अक्षो जीव नाशम् इतः गतः} %2-66-8

\twolineshloka
{विदेह राजस्य सुता तहा सीता तपस्विनी}
{दुह्खस्य अनुचिता दुह्खम् वने पर्युद्विजिष्यति} %2-66-9

\twolineshloka
{नदताम् भीम घोषाणाम् निशासु मृग पक्षिणाम्}
{निशम्य नूनम् सम्स्त्रस्ता राघवम् सम्श्रयिष्यति} %2-66-10

\twolineshloka
{वृद्धः चैव अल्प पुत्रः च वैदेहीम् अनिचिन्तयन्}
{सो अपि शोक समाविष्टः ननु त्यक्ष्यति जीवितम्} %2-66-11

\twolineshloka
{साहमद्यैव दिष्टान्तम् गमिष्यामि पतिव्रता}
{इदम् शरीरमालिङ्ग्य प्रवेक्ष्यामि हुताशनम्} %2-66-12

\twolineshloka
{ताम् ततः सम्परिष्वज्य विलपन्तीम् तपस्विनीम्}
{व्यपनिन्युः सुदुह्ख आर्ताम् कौसल्याम् व्यावहारिकाः} %2-66-13

\twolineshloka
{तैल द्रोण्याम् अथ अमात्याः सम्वेश्य जगती पतिम्}
{राज्ञः सर्वाणि अथ आदिष्टाः चक्रुः कर्माणि अनन्तरम्} %2-66-14

\twolineshloka
{न तु सम्कलनम् राज्ञो विना पुत्रेण मन्त्रिणः}
{सर्वज्ञाः कर्तुम् ईषुस् ते ततः रक्षन्ति भूमिपम्} %2-66-15

\twolineshloka
{तैल द्रोण्याम् तु सचिवैः शायितम् तम् नर अधिपम्}
{हा मृतः अयम् इति ज्ञात्वा स्त्रियः ताः पर्यदेवयन्} %2-66-16

\twolineshloka
{बाहून् उद्यम्य कृपणा नेत्र प्रस्रवणैः मुखैः}
{रुदन्त्यः शोक सम्तप्ताः कृपणम् पर्यदेवयन्} %2-66-17

\twolineshloka
{हा महाराज रामेण सततम् प्रियवादिना}
{विहीनाः सत्यसन्धेन किमर्थम् विजहासि नः} %2-66-18

\twolineshloka
{कैकेय्या दुष्टभावाया राघवेण वियोजिताः}
{कथम् पतिघ्न्या वत्स्यामः समीपे विधवा वयम्} %2-66-19

\twolineshloka
{स हि नाथः सदास्माकम् तव च प्रभुरात्मवान्}
{वनम् रामो गतः श्रीमान् विहाय नृपतिश्रियम्} %2-66-20

\twolineshloka
{त्वया तेन च वीरेण विना व्यसनमोहिताः}
{कथम् वयम् निवत्स्यामः कैकेय्या च विदूषिताः} %2-66-21

\twolineshloka
{यया तु राजा रामश्च लक्ष्मणश्च महाबलः}
{सीतया सह सम्त्य्क्ताः सा कमन्यम् न हास्यति} %2-66-22

\twolineshloka
{ता बाष्पेण च सम्वीताः शोकेन विपुलेन च}
{व्यवेष्टन्त निरानन्दा राघवस्य वरस्त्रीयः} %2-66-23

\twolineshloka
{निशा नक्षत्र हीना इव स्त्री इव भर्तृ विवर्जिता}
{पुरी न अराजत अयोध्या हीना राज्ञा महात्मना} %2-66-24

\twolineshloka
{बाष्प पर्याकुल जना हाहा भूत कुल अन्गना}
{शून्य चत्वर वेश्म अन्ता न बभ्राज यथा पुरम्} %2-66-25

\fourlineindentedshloka
{गत प्रभा द्यौर् इव भास्करम् विना}
{व्यपेत नक्षत्र गणा इव शर्वरी}
{निवृत्तचारः सहसा गतो रविः}
{प्रवृत्तचारा राजनी ह्युपस्थिता} %2-66-26

\fourlineindentedshloka
{ऋते तु पुत्राद्दहनम् महीपते}
{र्नरोचयन्ते सुहृदः समागताः}
{इतीव तस्मिन् शयने न्यवेशय}
{न्विचिन्त्य राजानमचिन्त्य दर्शनम्} %2-66-27

\fourlineindentedshloka
{गतप्रभा द्यौरिव भास्करम् विना}
{व्यपेतनक्षत्रगणेव शर्वरी}
{पुरी बभासे रहिता मह आत्मना}
{न च अस्र कण्ठ आकुल मार्ग चत्वरा} %2-66-28

\fourlineindentedshloka
{नराः च नार्यः च समेत्य सम्घशो}
{विगर्हमाणा भरतस्य मातरम्}
{तदा नगर्याम् नर देव सम्क्षये}
{बभूवुर् आर्ता न च शर्म लेभिरे} %2-66-29


॥इत्यार्षे श्रीमद्रामायणे वाल्मीकीये आदिकाव्ये अयोध्याकाण्डे तैलद्रोण्यधिशयनम् नाम षट्षष्ठितमः सर्गः ॥२-६६॥
