\sect{पञ्चसप्ततितमः सर्गः — भरतशपथः}

\twolineshloka
{दीर्घकालात्समुत्थाय सम्ज्ञाम् लब्ध्वा च वीर्यवान्}
{नेत्राभ्यामश्रुपूर्णाभ्याम् दीनामुद्वीक्ष्य मातरम्} %2-75-1

\twolineshloka
{सोऽमात्यमध्येभरतो जननीमभ्यकुत्सयत्}
{राज्यम् न कामये जातु मन्त्रये नापि मातरम्} %2-75-2

\twolineshloka
{अभिषेकम् न जानामि योभूद्रज्ना समीक्षितः}
{विप्रकृष्टे ह्यहम् देशे शत्रुघ्न सहितोऽवसम्} %2-75-3

\twolineshloka
{वनवासम् न जानामि रामस्यहम् महात्मनः}
{विवासनम् वा सौमित्रेः सीतायाश्च यथाभवत्} %2-75-4

\twolineshloka
{तथैव क्रोशतः तस्य भरतस्य महात्मनः}
{कौसल्या शब्दम् आज्ञाय सुमित्राम् इदम् अब्रवीत्} %2-75-5

\twolineshloka
{आगतः क्रूर कार्यायाः कैकेय्या भरतः सुतः}
{तम् अहम् द्रष्टुम् इच्चामि भरतम् दीर्घ दर्शिनम्} %2-75-6

\twolineshloka
{एवम् उक्त्वा सुमित्राम् सा विवर्णा मलिन अम्बरा}
{प्रतस्थे भरतः यत्र वेपमाना विचेतना} %2-75-7

\twolineshloka
{स तु राम अनुजः च अपि शत्रुघ्न सहितः तदा}
{प्रतस्थे भरतः यत्र कौसल्याया निवेशनम्} %2-75-8

\twolineshloka
{ततः शत्रुघ्न भरतौ कौसल्याम् प्रेक्ष्य दुह्खितौ}
{पर्यष्वजेताम् दुह्ख आर्ताम् पतिताम् नष्ट चेतनाम्} %2-75-9

\twolineshloka
{रुदन्तौ रुदतीम् दुःखात्समेत्यार्याम् मनस्स्विनीम्}
{भरतम् प्रत्युवाच इदम् कौसल्या भृश दुह्खिता} %2-75-10

\twolineshloka
{इदम् ते राज्य कामस्य राज्यम् प्राप्तम् अकण्टकम्}
{सम्प्राप्तम् बत कैकेय्या शीघ्रम् क्रूरेण कर्मणा} %2-75-11

\twolineshloka
{प्रस्थाप्य चीर वसनम् पुत्रम् मे वन वासिनम्}
{कैकेयी कम् गुणम् तत्र पश्यति क्रूर दर्शिनी} %2-75-12

\twolineshloka
{क्षिप्रम् माम् अपि कैकेयी प्रस्थापयितुम् अर्हति}
{हिरण्य नाभो यत्र आस्ते सुतः मे सुमहा यशाः} %2-75-13

\twolineshloka
{अथवा स्वयम् एव अहम् सुमित्र अनुचरा सुखम्}
{अग्नि होत्रम् पुरः कृत्य प्रस्थास्ये यत्र राघवः} %2-75-14

\twolineshloka
{कामम् वा स्वयम् एव अद्य तत्र माम् नेतुम् अर्हसि}
{यत्र असौ पुरुष व्याघ्रः तप्यते मे तपः सुतः} %2-75-15

\twolineshloka
{इदम् हि तव विस्तीर्णम् धन धान्य समाचितम्}
{हस्ति अश्व रथ सम्पूर्णम् राज्यम् निर्यातितम् तया} %2-75-16

\twolineshloka
{इत्यादिबहुभिर्वाक्यैः क्रूरैः सम्भर्स्तितोऽनघः}
{विव्यथे भरतस्तीव्रम् व्रणे तुद्येव सूचिना} %2-75-17

\twolineshloka
{पपात चरणौ तस्यास्तदा सम्भ्रान्तचेतनः}
{विलप्य बहुधाऽसम्ज्ञो लब्धसम्ज्ञ्स्ततः स्थितः} %2-75-18

\twolineshloka
{एवम् विलपमानाम् ताम् भरतः प्रान्जलिस् तदा}
{कौसल्याम् प्रत्युवाच इदम् शोकैः बहुभिर् आवृताम्} %2-75-19

\twolineshloka
{आर्ये कस्मात् अजानन्तम् गर्हसे माम् अकिल्बिषम्}
{विपुलाम् च मम प्रीतिम् स्थिराम् जानासि राघवे} %2-75-20

\twolineshloka
{कृता शास्त्र अनुगा बुद्धिर् मा भूत् तस्य कदाचन}
{सत्य सम्धः सताम् श्रेष्ठो यस्य आर्यो अनुमते गतः} %2-75-21

\twolineshloka
{प्रैष्यम् पापीयसाम् यातु सूर्यम् च प्रति मेहतु}
{हन्तु पादेन गाम् सुप्ताम् यस्य आर्यो अनुमते गतः} %2-75-22

\twolineshloka
{कारयित्वा महत् कर्म भर्ता भृत्यम् अनर्थकम्}
{अधर्मः यो अस्य सो अस्याः तु यस्य आर्यो अनुमते गतः} %2-75-23

\twolineshloka
{परिपालयमानस्य राज्ञो भूतानि पुत्रवत्}
{ततः तु द्रुह्यताम् पापम् यस्य आर्यो अनुमते गतः} %2-75-24

\twolineshloka
{बलि षड् भागम् उद्धृत्य नृपस्य अरक्षतः प्रजाः}
{अधर्मः यो अस्य सो अस्य अस्तु यस्य आर्यो अनुमते गतः} %2-75-25

\twolineshloka
{सम्श्रुत्य च तपस्विभ्यः सत्रे वै यज्ञ दक्षिणाम्}
{ताम् विप्रलपताम् पापम् यस्य आर्यो अनुमते गतः} %2-75-26

\twolineshloka
{हस्ति अश्व रथ सम्बाधे युद्धे शस्त्र समाकुले}
{मा स्म कार्षीत् सताम् धर्मम् यस्य आर्यो अनुमते गतः} %2-75-27

\twolineshloka
{उपदिष्टम् सुसूक्ष्म अर्थम् शास्त्रम् यत्नेन धीमता}
{स नाशयतु दुष्ट आत्मा यस्य आर्यो अनुमते गतः} %2-75-28

\twolineshloka
{मा च तम् प्यूढबाह्वम्सम् चन्द्रार्कसम्तेजनम्}
{द्राक्षीद्राज्यस्थमासीनम् यस्यार्योऽनुमते गतः} %2-75-29

\twolineshloka
{पायसम् कृसरम् चागम् वृथा सो अश्नातु निर्घृणः}
{गुरूमः च अपि अवजानातु यस्य आर्यो अनुमते गतः} %2-75-30

\twolineshloka
{गाश्च स्पृशतु पादेन गुरून् परिवदेत्स्वयम्}
{मित्रे द्रुह्येत सोऽत्यन्तम् यस्यार्योऽनुमते गतः} %2-75-31

\twolineshloka
{विश्वासात्कथितम् किम्चित्परिवादम् मिथः क्वचित्}
{विवृणोतु स दुष्टात्मा यस्यार्योओऽनुमते गतः} %2-75-32

\twolineshloka
{अकर्ता ह्यकृतज्ञश्च त्यक्तात्मा निरपत्रपः}
{लोके भवतु विद्वेष्यो यस्यार्योऽनुमते गतः} %2-75-33

\twolineshloka
{पुत्रैः दारैः च भृत्यैः च स्व गृहे परिवारितः}
{स एको मृष्टम् अश्नातु यस्य आर्यो अनुमते गतः} %2-75-34

\twolineshloka
{अप्राप्य सदृशान् दाराननपत्यः प्रमीयताम्}
{अनवाप्य क्रियाम् धर्म्याम् यश्यार्योऽनुमते गतः} %2-75-35

\twolineshloka
{मात्मनः सम्ततिम् द्राक्षीत्स्वेषु दारेषु दुःखितः}
{आयुः समग्रमप्राप्य यस्यार्योऽनुमते गतः} %2-75-36

\twolineshloka
{राज स्त्री बाल वृद्धानाम् वधे यत् पापम् उच्यते}
{भृत्य त्यागे च यत् पापम् तत् पापम् प्रतिपद्यताम्} %2-75-37

\twolineshloka
{लाक्षया मधुमाम्सेन लोहेन च विषेण च}
{सदैव बिभृयाद्भृत्यान् यस्यार्योऽसुमते गतः} %2-75-38

\twolineshloka
{सम्ग्रामे समुपोढे स शत्रुपक्ष्भयम्करे}
{पलायामानो वध्येत यस्यार्योऽनुमे गतः} %2-75-39

\twolineshloka
{कपालपाणिः पृथिवीमटताम् चीरसम्वृतः}
{भिक्समाणो यथोन्मत्तो यस्यार्योऽनुमते गतह्} %2-75-40

\twolineshloka
{पाने प्रसक्तो भवतु स्त्रीष्वक्षेषु च नित्यशः}
{काम्क्रोधाभिभूतस्तु यस्यार्योऽनुमते गतः} %2-75-41

\twolineshloka
{यस्य धर्मे मनो भूयादधर्मम् स निषेवताम्}
{अपात्रवर्षी भवतु यस्यार्योऽनुमते गतः} %2-75-42

\twolineshloka
{सम्चितान्यस्य वित्तानि विविधानि सहस्रशः}
{दस्युभिर्विप्रलुप्यन्ताम् यश्यार्योऽनुमते गतः} %2-75-43

\twolineshloka
{उभे सम्ध्ये शयानस्य यत् पापम् परिकल्प्यते}
{तच् च पापम् भवेत् तस्य यस्य आर्यो अनुमते गतः} %2-75-44

\twolineshloka
{यद् अग्नि दायके पापम् यत् पापम् गुरु तल्पगे}
{मित्र द्रोहे च यत् पापम् तत् पापम् प्रतिपद्यताम्} %2-75-45

\twolineshloka
{देवतानाम् पितृऋणाम् च माता पित्रोस् तथैव च}
{मा स्म कार्षीत् स शुश्रूषाम् यस्य आर्यो अनुमते गतः} %2-75-46

\twolineshloka
{सताम् लोकात् सताम् कीर्त्याः सज् जुष्टात् कर्मणः तथा}
{भ्रश्यतु क्षिप्रम् अद्य एव यस्य आर्यो अनुमते गतः} %2-75-47

\twolineshloka
{अपास्य मातृशुश्रूषामनर्थे सोऽवतिष्ठताम्}
{दीर्घबाहुर्महावक्षा यस्यार्योऽसुमते गतः} %2-75-48

\twolineshloka
{बहुपुत्रो दरिद्रश्च ज्वररोगसमन्वितः}
{स भूयात्सततक्लेशी यस्यार्योऽनुमते गतः} %2-75-49

\twolineshloka
{आशामाशम् समानानाम् दीनानामूर्ध्वचक्षुषाम्}
{आर्थिनाम् वितथाम् कुर्याद्यस्यार्योऽनुमते गतः} %2-75-50

\twolineshloka
{मायया रमताम् नित्यम् परुषः पिशुनोऽशुचिः}
{राज्नो भीत स्त्वधर्मात्मा यस्यार्योऽनुमते गतः} %2-75-51

\twolineshloka
{ऋतुस्नाताम् सतीम् भार्यामृतुकालानुरोधिनीम्}
{अतिवर्तेत दुष्टात्मा यस्यार्योऽनुमते गतः} %2-75-52

\twolineshloka
{धर्मदारान् परित्यज्य परदारान्नि षेवताम्}
{त्यक्तधर्मरतिर्मूढो यस्यार्योऽनुमते गतः} %2-75-53

\twolineshloka
{विप्रलु प्तप्रजातस्य दुष्कृतम् ब्राह्मणस्य यत्}
{तदेव प्रतिपद्येत यस्यार्योऽनुमते गतः} %2-75-54

\twolineshloka
{पानीयदूषके पापम् तथैव विषदायके}
{यत्तदेकः स लभताम् यस्यार्योऽनुमते गतः} %2-75-55

\twolineshloka
{ब्राह्मणायोद्यताम् पूजाम् विहन्तु कलुषेन्द्रियः}
{बालवत्साम् च गाम् दोग्दु यस्यर्योऽनुमते गतः} %2-75-56

\twolineshloka
{तृष्णार्तम् सति पानीये विप्रलम्भेन योजयेत्}
{लभेत तस्य यत्पापम् यस्यार्योऽनुमते गतः} %2-75-57

\twolineshloka
{भक्त्या विवदमानेषु मार्गमाश्रित्य पश्यतः}
{तस्य पापेन युज्येत यस्यार्योऽनुमते गतः} %2-75-58

\twolineshloka
{विहीनाम् पति पुत्राभ्याम् कौसल्याम् पार्थिव आत्मजः}
{एवम् आश्वसयन्न् एव दुह्ख आर्तः निपपात ह} %2-75-59

\twolineshloka
{तथा तु शपथैः कष्टैः शपमानम् अचेतनम्}
{भरतम् शोक सम्तप्तम् कौसल्या वाक्यम् अब्रवीत्} %2-75-60

\twolineshloka
{मम दुह्खम् इदम् पुत्र भूयः समुपजायते}
{शपथैः शपमानो हि प्राणान् उपरुणत्सि मे} %2-75-61

\twolineshloka
{दिष्ट्या न चलितः धर्मात् आत्मा ते सह लक्ष्मणः}
{वत्स सत्य प्रतिज्ञो मे सताम् लोकान् अवाप्स्यसि} %2-75-62

\twolineshloka
{इत्युक्त्वा चाङ्कमानीय भरतम् भ्रातृवत्सलम्}
{परिष्वज्य महाबाहुम् रुरोद भृशदुःखिता} %2-75-63

\twolineshloka
{एवम् विलपमानस्य दुह्ख आर्तस्य महात्मनः}
{मोहाच् च शोक सम्रोधात् बभूव लुलितम् मनः} %2-75-64

\fourlineindentedshloka
{लालप्यमानस्य विचेतनस्य}
{प्रनष्ट बुद्धेः पतितस्य भूमौ}
{मुहुर् मुहुर् निह्श्वसतः च दीर्घम्}
{सा तस्य शोकेन जगाम रात्रिः} %2-75-65


॥इत्यार्षे श्रीमद्रामायणे वाल्मीकीये आदिकाव्ये अयोध्याकाण्डे भरतशपथः नाम पञ्चसप्ततितमः सर्गः ॥२-७५॥
