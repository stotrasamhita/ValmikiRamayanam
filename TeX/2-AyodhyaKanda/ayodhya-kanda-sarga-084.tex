\sect{चतुरशीतितमः सर्गः — गुहागमनम्}

\twolineshloka
{ततः निविष्टाम् ध्वजिनीम् गन्गाम् अन्वाश्रिताम् नदीम्}
{निषाद राजो दृष्ट्वा एव ज्ञातीन् सम्त्वरितः अब्रवीत्} %2-84-1

\twolineshloka
{महती इयम् अतः सेना सागर आभा प्रदृश्यते}
{न अस्य अन्तम् अवगच्चामि मनसा अपि विचिन्तयन्} %2-84-2

\twolineshloka
{यथा तु खलु दुर्भद्धिर्भरतः स्वयमागतः}
{स एष हि महा कायः कोविदार ध्वजो रथे} %2-84-3

\twolineshloka
{बन्धयिष्यति वा दाशान् अथ वा अस्मान् वधिष्यति}
{अथ दाशरथिम् रामम् पित्रा राज्यात् विवासितम्} %2-84-4

\twolineshloka
{सम्पन्नाम् श्रियमन्विच्चम्स्तस्य राज्ञः सुदुर्लभाम्}
{भरतः कैकेयी पुत्रः हन्तुम् समधिगच्चति} %2-84-5

\twolineshloka
{भर्ता चैव सखा चैव रामः दाशरथिर् मम}
{तस्य अर्थ कामाः सम्नद्धा गन्गा अनूपे अत्र तिष्ठत} %2-84-6

\twolineshloka
{तिष्ठन्तु सर्व दाशाः च गन्गाम् अन्वाश्रिता नदीम्}
{बल युक्ता नदी रक्षा माम्स मूल फल अशनाः} %2-84-7

\twolineshloka
{नावाम् शतानाम् पन्चानाम् कैवर्तानाम् शतम् शतम्}
{सम्नद्धानाम् तथा यूनाम् तिष्ठन्तु अत्यभ्यचोदयत्} %2-84-8

\twolineshloka
{यदा तुष्टः तु भरतः रामस्य इह भविष्यति}
{सा इयम् स्वस्तिमयी सेना गन्गाम् अद्य तरिष्यति} %2-84-9

\twolineshloka
{इति उक्त्वा उपायनम् गृह्य मत्स्य माम्स मधूनि च}
{अभिचक्राम भरतम् निषाद अधिपतिर् गुहः} %2-84-10

\twolineshloka
{तम् आयान्तम् तु सम्प्रेक्ष्य सूत पुत्रः प्रतापवान्}
{भरताय आचचक्षे अथ विनयज्ञो विनीतवत्} %2-84-11

\twolineshloka
{एष ज्ञाति सहस्रेण स्थपतिः परिवारितः}
{कुशलो दण्डक अरण्ये वृद्धो भ्रातुः च ते सखा} %2-84-12

\twolineshloka
{तस्मात् पश्यतु काकुत्स्थ त्वाम् निषाद अधिपो गुहः}
{असम्शयम् विजानीते यत्र तौ राम लक्ष्मणौ} %2-84-13

\twolineshloka
{एतत् तु वचनम् श्रुत्वा सुमन्त्रात् भरतः शुभम्}
{उवाच वचनम् शीघ्रम् गुहः पश्यतु माम् इति} %2-84-14

\twolineshloka
{लब्ध्वा अभ्यनुज्ञाम् सम्हृष्टः ज्ञातिभिः परिवारितः}
{आगम्य भरतम् प्रह्वो गुहो वचनम् अब्रवीइत्} %2-84-15

\twolineshloka
{निष्कुटः चैव देशो अयम् वन्चिताः च अपि ते वयम्}
{निवेदयामः ते सर्वे स्वके दाश कुले वस} %2-84-16

\twolineshloka
{अस्ति मूलम् फलम् चैव निषादैः समुपाहृतम्}
{आर्द्रम् च माम्सम् शुष्कम् च वन्यम् च उच्च अवचम् महत्} %2-84-17

\twolineshloka
{आशम्से स्वाशिता सेना वत्स्यति इमाम् विभावरीम्}
{अर्चितः विविधैः कामैः श्वः ससैन्यो गमिष्यसि} %2-84-18


॥इत्यार्षे श्रीमद्रामायणे वाल्मीकीये आदिकाव्ये अयोध्याकाण्डे गुहागमनम् नाम चतुरशीतितमः सर्गः ॥२-८४॥
