\sect{षण्णवतितमः सर्गः — लक्ष्मणक्रोधः}

\twolineshloka
{तां तथा दर्शयित्वा तु मैथिलीं गिरिनिम्नगाम्}
{निषसाद गिरिप्रस्थे सीतां मांसेन छन्दयन्} %2-96-1

\twolineshloka
{इदं मेध्यमिदं स्वादु निष्टप्तमिदमग्निना}
{एवमास्ते स धर्मात्मा सीतया सह राघवः} %2-96-2

\twolineshloka
{तथा तत्रासतस्तस्य भरतस्योपयायिनः}
{सैन्यरेणुश्च शब्दश्च प्रादुरास्तां नभस्स्पृशौ} %2-96-3

\twolineshloka
{एतस्मिन्नन्तरे त्रस्ताः शब्देन महता ततः}
{अर्दिता यूथपा मत्ताः सयूथा दुद्रुवुर्दिशः} %2-96-4

\twolineshloka
{स तं सैन्यसमुद्धूतं शब्दं शुश्राव राघवः}
{तांश्च विप्रद्रुतान् सर्वान् यूथपानन्ववैक्षत} %2-96-5

\twolineshloka
{तांश्च विद्रवतो दृष्ट्वा तं च श्रुत्वा च निस्वनम्}
{उवाच रामः सौमित्रिं लक्ष्मणं दीप्ततेजसम्} %2-96-6

\twolineshloka
{हन्त लक्ष्मण पश्येह सुमित्रा सुप्रजास्त्वया}
{भीमस्तनितगम्भीरस्तुमुलः श्रूयते स्वनः} %2-96-7

\twolineshloka
{गजयूथानि वाऽरण्ये महिषा वा महावने}
{वित्रासिता मृगाः सिंहैः सहसा प्रद्रुता दिशः} %2-96-8

\twolineshloka
{राजा वा राजमात्रो वा मृगयामटते वने}
{अन्यद्वा श्वापदं किञ्चित् सौमित्रे ज्ञातुमर्हसि} %2-96-9

\twolineshloka
{सुदुश्चरो गिरिश्चायं पक्षिणामपि लक्ष्मण}
{सर्वमेतद्यथातत्त्वमचिराज्ज्ञातुमर्हसि} %2-96-10

\twolineshloka
{स लक्ष्मणः सन्त्वरितः सालमारुह्य पुष्पितम्}
{प्रेक्षमाणो दिशः सर्वाः पूर्वां दिशमुदैक्षत} %2-96-11

\twolineshloka
{उदङ्मुखः प्रेक्षमाणो ददर्श महतीं चमूम्}
{रथाश्वगजसम्बाधां यत्तैर्युक्तां पदातिभिः} %2-96-12

\twolineshloka
{तामश्वगजसम्पूर्णां रथध्वजविभूषिताम्}
{शशंस सेनां रामाय वचनं चेदमब्रीत्} %2-96-13

\twolineshloka
{अग्निं संशमयत्वार्यः सीता च भजतां गुहाम्}
{सज्यं कुरुष्व चापं च शरांश्च कवचं तथा} %2-96-14

\twolineshloka
{तं रामः पुरुषव्याघ्रो लक्ष्मणं प्रत्युवाच ह}
{अङ्गावेक्षस्व सौमित्रे कस्येमां मन्यसे चमूम्} %2-96-15

\twolineshloka
{एवमुक्तस्तु रामेण लक्ष्मणो वाक्यमब्रवीत्}
{दिधक्षन्निव तां सेनां रुषितः पावको यथा} %2-96-16

\twolineshloka
{सम्पन्नं राज्यमिच्छंस्तु व्यक्तं प्राप्याभिषेचनम्}
{आवां हन्तुं समभ्येति कैकेय्या भरतः सुतः} %2-96-17

\twolineshloka
{एष वै सुमहान् श्रीमान् विटपी सम्प्रकाशते}
{विराजत्युद्गतस्कन्धः कोविदारध्वजो रथे} %2-96-18

\twolineshloka
{भजन्त्येते यथा काममश्वानारुह्य शीघ्रगान्}
{एते भ्राजन्ति संहृष्टा गजानारुह्य सादिनः} %2-96-19

\twolineshloka
{गृहीतधनुषौ चावां गिरिं वीरश्रयावहै}
{अथवेहैव तिष्ठावः सन्नद्धावुद्यतायुधौ} %2-96-20

\onelineshloka
{अपि नौ वशमागच्छेत् कोविदारध्वजो दणे} %2-96-21

\twolineshloka
{अपि द्रक्ष्यामि भरतं यत्कृते व्यसनं महत्}
{त्वया राघव सम्प्राप्तं सीतया च मया तथा} %2-96-22

\twolineshloka
{यन्निमित्तं भवान् राज्याच्च्युतो राघव शाश्वतात्}
{सम्प्राप्तोऽयमरिर्वीर भरतो वध्य एव मे} %2-96-23

\twolineshloka
{भरतस्य वधे दोषं नाहं पश्यामि राघव}
{पूर्वापकारिणां त्यागे न ह्यधर्मो विधीयते} %2-96-24

\twolineshloka
{पूर्वापकारी भरतस्त्यक्तधर्मश्च राघव}
{एतस्मिन्निहते कृत्स्नामनुशाधि वसुन्धराम्} %2-96-25

\twolineshloka
{अद्य पुत्रं हतं सङ्ख्ये कैकेयी राज्यकामुका}
{मया पश्येत् सुदुःखार्त्ता हस्तिभग्नमिव द्रुमम्} %2-96-26

\twolineshloka
{कैकेयीं च वधिष्यामि सानुबन्धां सबान्धवाम्}
{कलुषेणाद्य महता मेदिनी परिमुच्यताम्} %2-96-27

\twolineshloka
{अद्येमं संयतं क्रोधमसत्कारं च मानद}
{मोक्ष्यामि शत्रुसैन्येषु कक्षेष्विव हुताशनम्} %2-96-28

\twolineshloka
{अद्यैतच्चित्रकूटस्य काननं निशितैः शरैः}
{भिन्दन् शत्रुशरीराणि करिष्ये शोणितोक्षितम्} %2-96-29

\twolineshloka
{शरैर्निर्भिन्नहृदयान् कुञ्जरांस्तुरगांस्तथा}
{श्वापदाः परिकर्षन्तु नरांश्च निहतान् मया} %2-96-30

\twolineshloka
{शराणां धनुषश्चाहमनृणोऽस्मि महामृधे}
{ससैन्यं भरतं हत्वा भविष्यामि न संशयः} %2-96-31


॥इत्यार्षे श्रीमद्रामायणे वाल्मीकीये आदिकाव्ये अयोध्याकाण्डे लक्ष्मणक्रोधः नाम षण्णवतितमः सर्गः ॥२-९६॥
