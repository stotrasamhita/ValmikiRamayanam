\sect{एकत्रिंशः सर्गः — लक्ष्मणवनानुगमनाभ्यनुज्ञा}

\twolineshloka
{एवं श्रुत्वा स संवादं लक्ष्मणः पूर्वमागतः}
{बाष्पपर्याकुलमुखः शोकं सोढुमशक्नुवन्} %2-31-1

\twolineshloka
{स भ्रातुश्चरणौ गाढं निपीड्य रघुनन्दनः}
{सीतामुवाचातियशां राघवं च महाव्रतम्} %2-31-2

\twolineshloka
{यदि गन्तुं कृता बुद्धिर्वनं मृगगजायुतम्}
{अहं त्वानुगमिष्यामि वनमग्रे धनुर्धरः} %2-31-3

\twolineshloka
{मया समेतोऽरण्यानि रम्याणि विचरिष्यसि}
{पक्षिभिर्मृगयूथैश्च संघुष्टानि समन्ततः} %2-31-4

\twolineshloka
{न देवलोकाक्रमणं नामरत्वमहं वृणे}
{ऐश्वर्यं चापि लोकानां कामये न त्वया विना} %2-31-5

\twolineshloka
{एवं ब्रुवाणः सौमित्रिर्वनवासाय निश्चितः}
{रामेण बहुभिः सान्त्वैर्निषिद्धः पुनरब्रवीत्} %2-31-6

\twolineshloka
{अनुज्ञातस्तु भवता पूर्वमेव यदस्म्यहम्}
{किमिदानीं पुनरपि क्रियते मे निवारणम्} %2-31-7

\twolineshloka
{यदर्थं प्रतिषेधो मे क्रियते गन्तुमिच्छतः}
{एतदिच्छामि विज्ञातुं संशयो हि ममानघ} %2-31-8

\twolineshloka
{ततोऽब्रवीन्महातेजा रामो लक्ष्मणमग्रतः}
{स्थितं प्राग्गामिनं धीरं याचमानं कृताञ्जलिम्} %2-31-9

\twolineshloka
{स्निग्धो धर्मरतो धीरः सततं सत्पथे स्थितः}
{प्रियः प्राणसमो वश्यो विजेयश्च सखा च मे} %2-31-10

\twolineshloka
{मयाद्य सह सौमित्रे त्वयि गच्छति तद्वनम्}
{को भजिष्यति कौसल्यां सुमित्रां वा यशस्विनीम्} %2-31-11

\twolineshloka
{अभिवर्षति कामैर्यः पर्जन्यः पृथिवीमिव}
{स कामपाशपर्यस्तो महातेजा महीपतिः} %2-31-12

\twolineshloka
{सा हि राज्यमिदं प्राप्य नृपस्याश्वपतेः सुता}
{दुःखितानां सपत्नीनां न करिष्यति शोभनम्} %2-31-13

\twolineshloka
{न भरिष्यति कौसल्यां सुमित्रां च सुदुःखिताम्}
{भरतो राज्यमासाद्य कैकेय्यां पर्यवस्थितः} %2-31-14

\twolineshloka
{तामार्यां स्वयमेवेह राजानुग्रहणेन वा}
{सौमित्रे भर कौसल्यामुक्तमर्थममुं चर} %2-31-15

\twolineshloka
{एवं मयि च ते भक्तिर्भविष्यति सुदर्शिता}
{धर्मज्ञगुरुपूजायां धर्मश्चाप्यतुलो महान्} %2-31-16

\twolineshloka
{एवं कुरुष्व सौमित्रे मत्कृते रघुनन्दन}
{अस्माभिर्विप्रहीणाया मातुर्नो न भवेत् सुखम्} %2-31-17

\twolineshloka
{एवमुक्तस्तु रामेण लक्ष्मणः श्लक्ष्णया गिरा}
{प्रत्युवाच तदा रामं वाक्यज्ञो वाक्यकोविदम्} %2-31-18

\twolineshloka
{तवैव तेजसा वीर भरतः पूजयिष्यति}
{कौसल्यां च सुमित्रां च प्रयतो नास्ति संशयः} %2-31-19

\twolineshloka
{यदि दुःस्थो न रक्षेत भरतो राज्यमुत्तमम्}
{प्राप्य दुर्मनसा वीर गर्वेण च विशेषतः} %2-31-20

\twolineshloka
{तमहं दुर्मतिं क्रूरं वधिष्यामि न संशयः}
{तत्पक्षानपि तान् सर्वांस्त्रैलोक्यमपि किं तु सा} %2-31-21

\twolineshloka
{कौसल्या बिभृयादार्या सहस्रं मद्विधानपि}
{यस्याः सहस्रं ग्रामाणां सम्प्राप्तमुपजीविनाम्} %2-31-22

\twolineshloka
{तदात्मभरणे चैव मम मातुस्तथैव च}
{पर्याप्ता मद्विधानां च भरणाय मनस्विनी} %2-31-23

\twolineshloka
{कुरुष्व मामनुचरं वैधर्म्यं नेह विद्यते}
{कृतार्थोऽहं भविष्यामि तव चार्थः प्रकल्प्यते} %2-31-24

\twolineshloka
{धनुरादाय सगुणं खनित्रपिटकाधरः}
{अग्रतस्ते गमिष्यामि पन्थानं तव दर्शयन्} %2-31-25

\twolineshloka
{आहरिष्यामि ते नित्यं मूलानि च फलानि च}
{वन्यानि च तथान्यानि स्वाहार्हाणि तपस्विनाम्} %2-31-26

\twolineshloka
{भवांस्तु सह वैदेह्या गिरिसानुषु रंस्यसे}
{अहं सर्वं करिष्यामि जाग्रतः स्वपतश्च ते} %2-31-27

\twolineshloka
{रामस्त्वनेन वाक्येन सुप्रीतः प्रत्युवाच तम्}
{व्रजापृच्छस्व सौमित्रे सर्वमेव सुहृज्जनम्} %2-31-28

\twolineshloka
{ये च राज्ञो ददौ दिव्ये महात्मा वरुणः स्वयम्}
{जनकस्य महायज्ञे धनुषी रौद्रदर्शने} %2-31-29

\twolineshloka
{अभेद्ये कवचे दिव्ये तूणी चाक्षय्यसायकौ}
{आदित्यविमलाभौ द्वौ खड्गौ हेमपरिष्कृतौ} %2-31-30

\twolineshloka
{सत्कृत्य निहितं सर्वमेतदाचार्यसद्मनि}
{सर्वमायुधमादाय क्षिप्रमाव्रज लक्ष्मण} %2-31-31

\twolineshloka
{स सुहृज्जनमामन्त्र्य वनवासाय निश्चितः}
{इक्ष्वाकुगुरुमागम्य जग्राहायुधमुत्तमम्} %2-31-32

\twolineshloka
{तद् दिव्यं राजशार्दूलः सत्कृतं माल्यभूषितम्}
{रामाय दर्शयामास सौमित्रिः सर्वमायुधम्} %2-31-33

\twolineshloka
{तमुवाचात्मवान् रामः प्रीत्या लक्ष्मणमागतम्}
{काले त्वमागतः सौम्य कांक्षिते मम लक्ष्मण} %2-31-34

\twolineshloka
{अहं प्रदातुमिच्छामि यदिदं मामकं धनम्}
{ब्राह्मणेभ्यस्तपस्विभ्यस्त्वया सह परंतप} %2-31-35

\twolineshloka
{वसन्तीह दृढं भक्त्या गुरुषु द्विजसत्तमाः}
{तेषामपि च मे भूयः सर्वेषां चोपजीविनाम्} %2-31-36

\twolineshloka
{वसिष्ठपुत्रं तु सुयज्ञमार्यं त्वमानयाशु प्रवरं द्विजानाम्}
{अपि प्रयास्यामि वनं समस्तानभ्यर्च्य शिष्टानपरान् द्विजातीन्} %2-31-37


॥इत्यार्षे श्रीमद्रामायणे वाल्मीकीये आदिकाव्ये अयोध्याकाण्डे लक्ष्मणवनानुगमनाभ्यनुज्ञा नाम एकत्रिंशः सर्गः ॥२-३१॥
