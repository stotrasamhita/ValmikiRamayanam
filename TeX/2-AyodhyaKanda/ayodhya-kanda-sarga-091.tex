\sect{एकनवतितमः सर्गः — भरद्वाजातिथ्यम्}

\twolineshloka
{कृतबुद्धिं निवासाय तत्रैव स मुनिस्तदा}
{भरतं कैकयीपुत्रमातिथ्येन न्यमन्त्रयत्} %2-91-1

\twolineshloka
{अब्रवीद्भरतस्त्वेनं नन्विदं भवता कृतम्}
{पाद्यमर्घ्यं तथातिथ्यं वने यदुपपद्यते} %2-91-2

\twolineshloka
{अथोवाच भरद्वाजो भरतं प्रहसन्निव}
{जाने त्वां प्रीतिसंयुक्तं तुष्येस्त्वं येन केनचित्} %2-91-3

\twolineshloka
{सेनायास्तु तवैतस्याः कर्तुमिच्छामि भोजनम्}
{मम प्रीतिर्यथारूपा त्वमर्हो मनुजाधिप} %2-91-4

\twolineshloka
{किमर्थं चापि निक्षिप्य दूरे बलमिहागतः}
{कस्मान्नेहोपयातोसि सबलः पुरुषर्षभ} %2-91-5

\twolineshloka
{भरतः प्रत्युवाचेदं प्राञ्जलिस्तं तपोधनम्}
{ससैन्यो नोपयातोऽस्मि भगवन् भगवद्भयात्} %2-91-6

\twolineshloka
{राज्ञा च भगवन् नित्यं राजपुत्रेण वा सदा}
{यत्नतः परिहर्त्तव्या विषयेषु तपस्विनः} %2-91-7

\twolineshloka
{वाजिमुख्या मनुष्याश्च मत्ताश्च वरवारणाः}
{प्रच्छाद्य भगवन् भूमिं महतीमनुयान्ति माम्} %2-91-8

\twolineshloka
{ते वृक्षानुदकं भूमिमाश्रमेषूटजांस्तथा}
{न हिंस्युरिति तेनाहमेक एव समागतः} %2-91-9

\twolineshloka
{आनीयतामितः सेनेत्याज्ञप्तः परमर्षिणा}
{ततस्तु चक्रे भरतः सेनायाः समुपागमम्} %2-91-10

\twolineshloka
{अग्निशालां प्रविश्याथ पीत्वापः परिमृज्य च}
{आतिथ्यस्य क्रियाहेतोर्विश्वकर्माणमाह्वयत्} %2-91-11

\twolineshloka
{आह्वये विश्वकर्माणमहं त्वष्टारमेव च}
{आतिथ्यं कर्तुमिच्छामि तत्र मे संविधीयताम्} %2-91-12

\twolineshloka
{आह्वये लोकपालांस्त्रीन् देवान् शक्रमुखांस्तथा}
{आतिथ्यं कर्तुमिच्छामि तत्र मे संविधीयताम्} %2-91-13

\twolineshloka
{प्राक्स्रोतसश्च या नद्यः प्रत्यक्स्रोतस एव च}
{पृथिव्यामन्तरिक्षे च समायान्त्वद्य सर्वशः} %2-91-14

\twolineshloka
{अन्याः स्रवन्तु मैरेयं सुरामन्याः सुनिष्ठिताम्}
{अपराश्चोदकं शीतमिक्षुकाण्डरसोपमम्} %2-91-15

\twolineshloka
{आह्वये देवगन्धर्वान् विश्वावसुहहाहुहून्}
{तथैवाप्सरसो देवीर्गन्धर्व्वीश्चापि सर्वशः} %2-91-16

\twolineshloka
{घृताचीमथ विश्वाचीं मिश्रकेशीमलम्बुसाम्}
{नागदन्तां च हेमां च हिमामद्रिकृतस्थलाम्} %2-91-17

\twolineshloka
{शक्रं याश्चोपतिष्ठन्ति ब्रह्माणं याश्च योषितः}
{सर्वास्तुम्बुरुणा सार्द्धमाह्वये सपरिच्छदाः} %2-91-18

\twolineshloka
{वनं कुरुषु यद्दिव्यं वासोभूषणपत्रवत्}
{दिव्यनारीफलं शश्वत्तत्कौबेरमिहैतु च} %2-91-19

\twolineshloka
{इह मे भगवान् सोमो विधत्तामन्नमुत्तमम्}
{भक्ष्यं भोज्यं च चोष्यं च लेह्यं च विविधं बहु} %2-91-20

\twolineshloka
{विचित्राणि च माल्यानि पादपप्रच्युतानि च}
{सुरादीनि च पेयानि मांसानि विविधानि च} %2-91-21

\twolineshloka
{एवं समाधिना युक्तस्तेजसाऽप्रतिमेन च}
{शीक्षास्वरसमायुक्तं तपसा चाब्रवीन्मुनिः} %2-91-22

\twolineshloka
{मनसा ध्यायतस्तस्य प्राङ्मुखस्य कृताञ्जलेः}
{आजग्मुस्तानि सर्वाणि दैवतानि पृथक्पृथक्} %2-91-23

\twolineshloka
{मलयं दर्दुरं चैव ततः स्वेदनुदऽऽनिलः}
{उपस्पृश्य ववौ युक्त्या सुप्रियात्मा सुखः शिवः} %2-91-24

\twolineshloka
{ततोभ्यवर्तन्त घना दिव्याः कुसुमवृष्टयः}
{दिव्यदुन्दुभिघोषश्च दिक्षु सर्वासु शुश्रुवे} %2-91-25

\twolineshloka
{प्रववुश्चोत्तमा वाताननृतुश्चाप्सरोगणाः}
{प्रजगुर्देवगन्धर्वा वीणाः प्रमुमुचुः स्वरान्} %2-91-26

\twolineshloka
{स शब्दो द्यां च भूमिं च प्राणिनां श्रवणानि च}
{विवेशोच्चारितः श्लक्ष्णः समो लयगुणान्वितः} %2-91-27

\twolineshloka
{तस्मिन्नुपरते शब्दे दिव्ये श्रोतृत्रसुखे नृणाम्}
{ददर्श भारतं सैन्यं विधानं विश्वकर्मणः} %2-91-28

\twolineshloka
{बभूव हि समा भूमिः समन्तात्पञ्चयोजना}
{शाद्वलैर्बहुभिश्छन्ना नीलवैडूर्य्यसन्निभैः} %2-91-29

\twolineshloka
{तस्मिन् बिल्वाः कपित्थाश्च पनसा बीजपूरकाः}
{आमलक्यो बभूवुश्च चूताश्च फलभूषणाः} %2-91-30

\twolineshloka
{उत्तरेभ्यः कुरुभ्यश्च वनं दिव्योपभोगवत्}
{आजगाम नदी दिव्या तीजैर्बहुभिर्वृता} %2-91-31

\twolineshloka
{चतुःशालानि शुभ्राणि शालाश्च गजवाजिनाम्}
{हर्म्यप्रासादसम्बाधास्तोरणानि शुभानि च} %2-91-32

\twolineshloka
{सितमेघनिभं चापि राजवेश्मसु तोरणम्}
{दिव्यमाल्यकृताकारं दिव्यगन्धसमुक्षितम्} %2-91-33

\threelineshloka
{चतुरश्रमसम्बाधं शयनासनयानवत्}
{दिव्यैः सर्वरसैर्युक्तं दिव्यभोजनवस्त्रवत्}
{उपकल्पितसर्वान्नं धौतनिर्मलभाजनम्} %2-91-34

\threelineshloka
{क्लृप्तसर्वासनं श्रीमत् स्वास्तीर्णशयनोत्तमम्}
{प्रविवेश महाबाहुरनुज्ञातो महर्षिणा}
{वेश्म तद्रत्नसम्पूर्णं भरतः केकयीसुतः} %2-91-35

\twolineshloka
{अनुजग्मुश्च तं सर्वे मन्त्रिणः सपुरोहिताः}
{बभूवुश्च मुदा युक्ता दृष्ट्वा तं वेश्मसंविधिम्} %2-91-36

\twolineshloka
{तत्र राजासनं दिव्यं व्यजनं छत्रमेव च}
{भरतो मन्त्रिभिः सार्द्धमभ्यवर्त्तत राजवत्} %2-91-37

\twolineshloka
{आसनं पूजयामास रामायाभिप्रणम्य च}
{वालव्यजनमादाय न्यषीदत् सचिवासने} %2-91-38

\twolineshloka
{आनुपूर्व्यानिषेदुश्च सर्वे मन्त्रिपुरोहिताः}
{ततः सेनापतिः पश्चात् प्रशास्ता च निषेदतुः} %2-91-39

\twolineshloka
{ततस्तत्र मुहूर्त्तेन नद्यः पायसकर्दमाः}
{उपातिष्ठन्त भरतं भरद्वाजस्य शासनात्} %2-91-40

\twolineshloka
{तासामुभयतःकूलं पाण्डुमृत्तिकलेपनाः}
{रम्याश्चावसथा दिव्या ब्रह्मणस्तु प्रसादजाः} %2-91-41

\twolineshloka
{तेनैव च मुहूर्त्तेन दिव्याभरणभूषिताः}
{आगुर्विंशतिसाहस्रा ब्रह्मणा प्रहिताः स्त्रियः} %2-91-42

\twolineshloka
{सुवर्णमणिमुक्तेन प्रवालेन च शोभिताः}
{आगुर्विंशतिसाहस्राः कुबेरप्रहिताः स्त्रियः} %2-91-43

\twolineshloka
{याभिर्गृहीतपुरुषः सोन्माद इव लक्ष्यते}
{आगुर्विंशतिसाहस्रा नन्दनादप्सरोगणाः} %2-91-44

\twolineshloka
{नारदस्तुम्बुरुर्गोपः प्रवराः सूर्य्यवर्चसः}
{एते गन्धर्वराजानो भरतस्याग्रतो जगुः} %2-91-45

\twolineshloka
{अलम्बुसा मिश्रकेशी पुण्डरीकाथ वामना}
{उपानृत्यंस्तु भरत भरद्वाजस्य शासनात्} %2-91-46

\twolineshloka
{यानि माल्यानि देवेषु यानि चैत्ररथे वने}
{प्रयागे तान्यदृश्यन्त भरद्वाजस्य तेजसा} %2-91-47

\twolineshloka
{बिल्वा मार्दङ्गिका आसन् शम्याग्राहा विभीतकाः}
{अश्वत्थानर्त्तकाश्चासन् भरद्वाजस्य शासनात्} %2-91-48

\twolineshloka
{ततः सरलतालाश्च तिलका नक्तमालकाः}
{प्रहृष्टास्तत्र सम्पेतुः कुब्जा भूत्वाथ वामनाः} %2-91-49

\twolineshloka
{शिंशुपामलकीजम्ब्वो याश्चान्याः काननेषु ताः}
{मालती मल्लिका जातिर्याश्चान्याः कानने लताः} %2-91-50

\threelineshloka
{प्रमदाविग्रहं कृत्वा भरद्वाजाश्रमेऽवदन्}
{सुराः सुरापाः पिबत पायसं च बुभुक्षिताः}
{मांसानि च सुमेध्यानि भक्ष्यन्तां यावदिच्छथ} %2-91-51

\twolineshloka
{उच्छाद्य स्नापयन्ति स्म नदीतीरेषु वल्गुषु}
{अप्येकमेकं पुरुषं प्रमदाः सप्त चाष्ट च} %2-91-52

\twolineshloka
{संवाहन्त्यः समापेतुर्नार्यो रुचिरलोचनाः}
{परिमृज्य तथान्योन्यं पाययन्ति वराङ्गनाः} %2-91-53

\twolineshloka
{हयान् गजान् खरानुष्ट्रांस्तथैव सुरभेः सुतान्}
{अभोजयन् वाहनपास्तेषां भोज्यं यथाविधि} %2-91-54

\twolineshloka
{इक्षूंश्च मधुलाजांश्च भोजयन्ति स्म वाहनान्}
{इक्ष्वाकुवरयोधानां चोदयन्तो महाबलाः} %2-91-55

\twolineshloka
{नाश्वबन्धोऽश्वमाजानान्न गजं कुञ्जरग्रहः}
{मत्तप्रमत्तमुदिता नमूः सा तत्र सम्बभौ} %2-91-56

\twolineshloka
{तर्पिताः सर्वकामैस्ते रक्तचन्दनरूषिताः}
{अप्सरोगणसंयुक्ताः सैन्या वाचमुदैरयन्} %2-91-57

\twolineshloka
{नैवायोध्यां गमिष्यामो न गमिष्याम दण्डकान्}
{कुशलं भरतस्यास्तु रामस्यास्तु तथा सुखम्} %2-91-58

\twolineshloka
{इति पादातयोधाश्च हस्त्यश्वारोह बन्धकाः}
{अनाथास्तं विधिं लब्ध्वा वाचमेतामुदैरयन्} %2-91-59

\twolineshloka
{सम्प्रहृष्टा विनेदुस्ते नरास्तत्र सहस्रशः}
{भरतस्यानुयातारः स्वर्गोयमिति चाब्रुवन्} %2-91-60

\twolineshloka
{नृत्यन्ति स्म हसन्ति स्म गायन्ति स्म च सैनिकाः}
{समन्तात् परिधावन्ति माल्योपेताः सहस्रशः} %2-91-61

\twolineshloka
{ततो भुक्तवतां तेषां तदन्नममृतोपमम्}
{दिव्यानुद्वीक्ष्य भक्ष्यांस्तानभवद्भक्षणे मतिः} %2-91-62

\twolineshloka
{प्रेष्याश्चेष्ट्यश्च वध्वश्च बलस्थाश्च सहस्रशः}
{बभूवुस्ते भृशं दृप्ताः सर्वे चाहतवाससः} %2-91-63

\twolineshloka
{कुञ्जराश्च खरोष्ट्राश्च गोश्वाश्च मृगपक्षिणः}
{बभूवुः सुभृतास्तत्र नान्यो ह्यन्यमकल्पयत्} %2-91-64

\twolineshloka
{नाशुक्लवासास्तत्रासीत् क्षुधितो मलिनोऽपि वा}
{रजसा ध्वस्तकेशो वा नरः कश्चिददृश्यत} %2-91-65

\twolineshloka
{आजैश्चापि च वाराहैर्निष्ठानवरसञ्चयैः}
{फलनिर्यूहसंसिद्धैः सूपैर्गन्धरसान्वितैः} %2-91-66

\twolineshloka
{पुष्पध्वजवतीः पूर्णाः शुक्लस्यान्नस्य चाभितः}
{ददृशुर्विस्मितास्तत्र नरा लौहीः सहस्रशः} %2-91-67

\twolineshloka
{बभूवुर्वनपार्श्वेषु कूपाः पायसकर्दमाः}
{ताश्च कामदुघा गावो द्रुमाश्चासन् मधुस्रुतः} %2-91-68

\twolineshloka
{वाप्यो मैरेयपूर्णाश्च मृष्टमांसचयैर्वृताः}
{प्रतप्तपिठरैश्चापि मार्गमायूरकौक्कुटैः} %2-91-69

\twolineshloka
{पात्रीणां च सहस्राणि स्थालीनां नियुतानि च}
{न्यर्बुदानि च पात्राणि शातकुम्भमयानि च} %2-91-70

\twolineshloka
{स्थाल्यः कुम्भ्यः करम्भ्यश्च दधिपूर्णाः सुसंस्कृताः}
{यौवनस्थस्य गौरस्य कपित्थस्य सुगन्धिनः} %2-91-71

\twolineshloka
{ह्रदाः पूर्णा रसालस्य दघ्नः श्वेतस्य चापरे}
{बभूवुः पायसस्यान्ये शर्करायाश्च सञ्चयाः} %2-91-72

\twolineshloka
{कल्कांश्चूर्णकषायांश्च स्नानानि विविधानि च}
{ददृशुर्भाजनस्थानि तीर्थेषु सरितां नराः} %2-91-73

\twolineshloka
{शुक्लानंशुमतश्चापि दन्तधावनसञ्चयान्}
{शुक्लांश्चन्दनकल्कांश्च समुद्गेष्ववतिष्ठतः} %2-91-74

\twolineshloka
{दर्पणान् परिमृष्टांश्च वाससां चापि सञ्चयान्}
{पादुकोपानहश्चैव युग्मानि च सहस्रशः} %2-91-75

\twolineshloka
{आञ्जनीः कङ्कतान् कूर्चान् शस्त्राणि च धनूंषि च}
{मर्मत्राणानि चित्राणि शयनान्यासनानि च} %2-91-76

\twolineshloka
{प्रतिपानह्रदान् पूर्णान् खरोष्ट्रगजवाजिनाम्}
{अवगाह्य सुतीर्थांश्च ह्रदान् सोत्पलपुष्करान्} %2-91-77

\twolineshloka
{आकाशवर्णप्रतिमान् स्वच्छतोयान् सुखप्लवान्}
{नीलवैडूर्य्यवर्णांश्च मृदून् यवससञ्चयान्} %2-91-78

\onelineshloka
{निर्वापार्थान् पशूनां ते ददृशुस्तत्र सर्वशः} %2-91-79

\twolineshloka
{व्यस्मयन्त मनुष्यास्ते स्वप्नकल्पं तदद्भुतम्}
{दृष्ट्वातिथ्यं कृतं तादृक् भरतस्य महर्षिणा} %2-91-80

\twolineshloka
{इत्येवं रममाणानां देवानामिव नन्दने}
{भरद्वाजाश्रमे रम्ये सा रात्रिर्व्यत्यवर्त्तत} %2-91-81

\twolineshloka
{प्रतिजग्मुश्च ता नद्यो गन्धर्वाश्च यथागतम्}
{भरद्वाजमनुज्ञाप्य ताश्च सर्वा वराङ्गनाः} %2-91-82

\twolineshloka
{तथैव मत्ता मदिरोत्कटा नरास्तथैव दिव्यागुरुचन्दनोक्षिताः}
{तथैव दिव्या विविधाः स्रगुत्तमाः पृथक् प्रकीर्णा मनुजैः प्रमर्दिताः} %2-91-83


॥इत्यार्षे श्रीमद्रामायणे वाल्मीकीये आदिकाव्ये अयोध्याकाण्डे भरद्वाजातिथ्यम् नाम एकनवतितमः सर्गः ॥२-९१॥
