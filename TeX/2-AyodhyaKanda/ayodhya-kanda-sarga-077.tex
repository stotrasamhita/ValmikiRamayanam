\sect{सप्तसप्ततितमः सर्गः — भरतशत्रुघ्नविलापः}

\twolineshloka
{ततः दश अहे अतिगते कृत शौचो नृप आत्मजः}
{द्वादशे अहनि सम्प्राप्ते श्राद्ध कर्माणि अकारयत्} %2-77-1

\twolineshloka
{ब्राह्मणेभ्यो ददौ रत्नम् धनम् अन्नम् च पुष्कलम्}
{वासाम्सि च महार्हाणि रत्नानि विविधानि च} %2-77-2

\twolineshloka
{बास्तिकम् बहु शुक्लम् च गाः च अपि शतशः तथा}
{दासी दासम् च यानम् च वेश्मानि सुमहान्ति च} %2-77-3

\twolineshloka
{ब्राह्मणेभ्यो ददौ पुत्रः राज्ञः तस्य और्ध्वदैहिकम्}
{ततः प्रभात समये दिवसे अथ त्रयोदशे} %2-77-4

\twolineshloka
{विललाप महा बाहुर् भरतः शोक मूर्चितः}
{शब्द अपिहित कण्ठः च शोधन अर्थम् उपागतः} %2-77-5

\twolineshloka
{चिता मूले पितुर् वाक्यम् इदम् आह सुदुह्खितः}
{तात यस्मिन् निषृष्टः अहम् त्वया भ्रातरि राघवे} %2-77-6

\twolineshloka
{तस्मिन् वनम् प्रव्रजिते शून्ये त्यक्तः अस्म्य् अहम् त्वया}
{यथा गतिर् अनाथायाः पुत्रः प्रव्राजितः वनम्} %2-77-7

\twolineshloka
{ताम् अम्बाम् तात कौसल्याम् त्यक्त्वा त्वम् क्व गतः नृप}
{दृष्ट्वा भस्म अरुणम् तच् च दग्ध अस्थि स्थान मण्डलम्} %2-77-8

\twolineshloka
{पितुः शरीर निर्वाणम् निष्टनन् विषसाद ह}
{स तु दृष्ट्वा रुदन् दीनः पपात धरणी तले} %2-77-9

\twolineshloka
{उत्थाप्यमानः शक्रस्य यन्त्र ध्वजैव च्युतः}
{अभिपेतुस् ततः सर्वे तस्य अमात्याः शुचि व्रतम्} %2-77-10

\twolineshloka
{अन्त काले निपतितम् ययातिम् ऋषयो यथा}
{शत्रुघ्नः च अपि भरतम् दृष्ट्वा शोक परिप्लुतम्} %2-77-11

\twolineshloka
{विसम्ज्ञो न्यपतत् भूमौ भूमि पालम् अनुस्मरन्}
{उन्मत्तैव निश्चेता विललाप सुदुह्खितः} %2-77-12

\twolineshloka
{स्मृत्वा पितुर् गुण अन्गानि तनि तानि तदा तदा}
{मन्थरा प्रभवः तीव्रः कैकेयी ग्राह सम्कुलः} %2-77-13

\twolineshloka
{वर दानमयो अक्षोभ्यो अमज्जयत् शोक सागरः}
{सुकुमारम् च बालम् च सततम् लालितम् त्वया} %2-77-14

\twolineshloka
{क्व तात भरतम् हित्वा विलपन्तम् गतः भवान्}
{ननु भोज्येषु पानेषु वस्त्रेष्व् आभरणेषु च} %2-77-15

\twolineshloka
{प्रवारयसि नः सर्वाम्स् तन् नः को अद्य करिष्यति}
{अवदारण काले तु पृथिवी न अवदीर्यते} %2-77-16

\twolineshloka
{विहीना या त्वया राज्ञा धर्मज्ञेन महात्मना}
{पितरि स्वर्गम् आपन्ने रामे च अरण्यम् आश्रिते} %2-77-17

\twolineshloka
{किम् मे जीवित सामर्थ्यम् प्रवेक्ष्यामि हुत अशनम्}
{हीनो भ्रात्रा च पित्रा च शून्याम् इक्ष्वाकु पालिताम्} %2-77-18

\twolineshloka
{अयोध्याम् न प्रवेक्ष्यामि प्रवेक्ष्यामि तपो वनम्}
{तयोः विलपितम् श्रुत्वा व्यसनम् च अन्ववेक्ष्य तत्} %2-77-19

\twolineshloka
{भृशम् आर्ततरा भूयः सर्वएव अनुगामिनः}
{ततः विषण्णौ श्रान्तौ च शत्रुघ्न भरताव् उभौ} %2-77-20

\twolineshloka
{धरण्याम् सम्व्यचेष्टेताम् भग्न शृन्गाव् इव ऋषभौ}
{ततः प्रकृतिमान् वैद्यः पितुर् एषाम् पुरोहितः} %2-77-21

\twolineshloka
{वसिष्ठो भरतम् वाक्यम् उत्थाप्य तम् उवाच ह}
{त्रयोदशोऽयम् दिवसः पितुर्वृत्तस्य ते विभो} %2-77-22

\twolineshloka
{सावशेषास्थिनिचये किमिह त्वम् विलम्बसे}
{त्रीणि द्वन्द्वानि भूतेषु प्रवृत्तानि अविशेषतः} %2-77-23

\twolineshloka
{तेषु च अपरिहार्येषु न एवम् भवितुम् अर्हति}
{सुमन्त्रः च अपि शत्रुघ्नम् उत्थाप्य अभिप्रसाद्य च} %2-77-24

\twolineshloka
{श्रावयाम् आस तत्त्वज्ञः सर्व भूत भव अभवौ}
{उत्थितौ तौ नर व्याघ्रौ प्रकाशेते यशस्विनौ} %2-77-25

\twolineshloka
{वर्ष आतप परिक्लिन्नौ पृथग् इन्द्र ध्वजाव् इव}
{अश्रूणि परिमृद्नन्तौ रक्त अक्षौ दीन भाषिणौ} %2-77-26


॥इत्यार्षे श्रीमद्रामायणे वाल्मीकीये आदिकाव्ये अयोध्याकाण्डे भरतशत्रुघ्नविलापः नाम सप्तसप्ततितमः सर्गः ॥२-७७॥
