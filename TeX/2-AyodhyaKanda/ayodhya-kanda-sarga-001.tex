\sect{प्रथमः सर्गः — रामाभिषेकव्यवसायः}

\twolineshloka
{गच्छता मातुलकुलं भरतेन तदानघः}
{शत्रुघ्नो नित्यशत्रुघ्नो नीतः प्रीतिपुरस्कृतः} %2-1-1

\twolineshloka
{स तत्र न्यवसद् भ्रात्रा सह सत्कारसत्कृतः}
{मातुलेनाश्वपतिना पुत्रस्नेहेन लालितः} %2-1-2

\twolineshloka
{तत्रापि निवसन्तौ तौ तर्प्यमाणौ च कामतः}
{भ्रातरौ स्मरतां वीरौ वृद्धं दशरथं नृपम्} %2-1-3

\twolineshloka
{राजापि तौ महातेजाः सस्मार प्रोषितौ सुतौ}
{उभौ भरतशत्रुघ्नौ महेन्द्रवरुणोपमौ} %2-1-4

\twolineshloka
{सर्व एव तु तस्येष्टाश्चत्वारः पुरुषर्षभाः}
{स्वशरीराद् विनिर्वृत्ताश्चत्वार इव बाहवः} %2-1-5

\twolineshloka
{तेषामपि महातेजा रामो रतिकरः पितुः}
{स्वयम्भूरिव भूतानां बभूव गुणवत्तरः} %2-1-6

\twolineshloka
{स हि देवैरुदीर्णस्य रावणस्य वधार्थिभिः}
{अर्थितो मानुषे लोके जज्ञे विष्णुः सनातनः} %2-1-7

\twolineshloka
{कौसल्या शुशुभे तेन पुत्रेणामिततेजसा}
{यथा वरेण देवानामदितिर्वज्रपाणिना} %2-1-8

\twolineshloka
{स हि रूपोपपन्नश्च वीर्यवाननसूयकः}
{भूमावनुपमः सूनुर्गुणैर्दशरथोपमः} %2-1-9

\twolineshloka
{स च नित्यं प्रशान्तात्मा मृदुपूर्वं च भाषते}
{उच्यमानोऽपि परुषं नोत्तरं प्रतिपद्यते} %2-1-10

\twolineshloka
{कदाचिदुपकारेण कृतेनैकेन तुष्यति}
{न स्मरत्यपकाराणां शतमप्यात्मवत्तया} %2-1-11

\twolineshloka
{शीलवृद्धैर्ज्ञानवृद्धैर्वयोवृद्धैश्च सज्जनैः}
{कथयन्नास्त वै नित्यमस्त्रयोग्यान्तरेष्वपि} %2-1-12

\twolineshloka
{बुद्धिमान् मधुराभाषी पूर्वभाषी प्रियंवदः}
{वीर्यवान्न च वीर्येण महता स्वेन विस्मितः} %2-1-13

\twolineshloka
{न चानृतकथो विद्वान् वृद्धानां प्रतिपूजकः}
{अनुरक्तः प्रजाभिश्च प्रजाश्चाप्यनुरज्यते} %2-1-14

\twolineshloka
{सानुक्रोशो जितक्रोधो ब्राह्मणप्रतिपूजकः}
{दीनानुकम्पी धर्मज्ञो नित्यं प्रग्रहवान् शुचिः} %2-1-15

\twolineshloka
{कुलोचितमतिः क्षात्रं स्वधर्मं बहु मन्यते}
{मन्यते परया प्रीत्या महत् स्वर्गफलं ततः} %2-1-16

\twolineshloka
{नाश्रेयसि रतो यश्च न विरुद्धकथारुचिः}
{उत्तरोत्तरयुक्तीनां वक्ता वाचस्पतिर्यथा} %2-1-17

\twolineshloka
{अरोगस्तरुणो वाग्मी वपुष्मान् देशकालवित्}
{लोके पुरुषसारज्ञः साधुरेको विनिर्मितः} %2-1-18

\twolineshloka
{स तु श्रेष्ठैर्गुणैर्युक्तः प्रजानां पार्थिवात्मजः}
{बहिश्चर इव प्राणो बभूव गुणतः प्रियः} %2-1-19

\twolineshloka
{सर्वविद्याव्रतस्नातो यथावत् साङ्गवेदवित्}
{इष्वस्त्रे च पितुः श्रेष्ठो बभूव भरताग्रजः} %2-1-20

\twolineshloka
{कल्याणाभिजनः साधुरदीनः सत्यवागृजुः}
{वृद्धैरभिविनीतश्च द्विजैर्धर्मार्थदर्शिभिः} %2-1-21

\twolineshloka
{धर्मकामार्थतत्त्वज्ञः स्मृतिमान् प्रतिभानवान्}
{लौकिके समयाचारे कृतकल्पो विशारदः} %2-1-22

\twolineshloka
{निभृतः संवृताकारो गुप्तमन्त्रः सहायवान्}
{अमोघक्रोधहर्षश्च त्यागसंयमकालवित्} %2-1-23

\twolineshloka
{दृढभक्तिः स्थिरप्रज्ञो नासद्ग्राही न दुर्वचः}
{निस्तन्द्रीरप्रमत्तश्च स्वदोषपरदोषवित्} %2-1-24

\twolineshloka
{शास्त्रज्ञश्च कृतज्ञश्च पुरुषान्तरकोविदः}
{यः प्रग्रहानुग्रहयोर्यथान्यायं विचक्षणः} %2-1-25

\twolineshloka
{सत्सङ्ग्रहानुग्रहणे स्थानविन्निग्रहस्य च}
{आयकर्मण्युपायज्ञः सन्दृष्टव्ययकर्मवित्} %2-1-26

\twolineshloka
{श्रैष्ठ्यं चास्त्रसमूहेषु प्राप्तो व्यामिश्रकेषु च}
{अर्थधर्मौ च सङ्गृह्य सुखतन्त्रो न चालसः} %2-1-27

\twolineshloka
{वैहारिकाणां शिल्पानां विज्ञातार्थविभागवित्}
{आरोहे विनये चैव युक्तो वारणवाजिनाम्} %2-1-28

\twolineshloka
{धनुर्वेदविदां श्रेष्ठो लोकेऽतिरथसम्मतः}
{अभियाता प्रहर्ता च सेनानयविशारदः} %2-1-29

\twolineshloka
{अप्रधृष्यश्च सङ्ग्रामे क्रुद्धैरपि सुरासुरैः}
{अनसूयो जितक्रोधो न दृप्तो न च मत्सरी} %2-1-30

\twolineshloka
{नावज्ञेयश्च भूतानां न च कालवशानुगः}
{एवं श्रेष्ठैर्गुणैर्युक्तः प्रजानां पार्थिवात्मजः} %2-1-31

\twolineshloka
{सम्मतस्त्रिषु लोकेषु वसुधायाः क्षमागुणैः}
{बुद्ध्या बृहस्पतेस्तुल्यो वीर्ये चापि शचीपतेः} %2-1-32

\twolineshloka
{तथा सर्वप्रजाकान्तैः प्रीतिसञ्जननैः पितुः}
{गुणैर्विरुरुचे रामो दीप्तः सूर्य इवांशुभिः} %2-1-33

\twolineshloka
{तमेवंवृत्तसम्पन्नमप्रधृष्यपराक्रमम्}
{लोकनाथोपमं नाथमकामयत मेदिनी} %2-1-34

\twolineshloka
{एतैस्तु बहुभिर्युक्तं गुणैरनुपमैः सुतम्}
{दृष्ट्वा दशरथो राजा चक्रे चिन्तां परन्तपः} %2-1-35

\twolineshloka
{अथ राज्ञो बभूवैव वृद्धस्य चिरजीविनः}
{प्रीतिरेषा कथं रामो राजा स्यान्मयि जीवति} %2-1-36

\twolineshloka
{एषा ह्यस्य परा प्रीतिर्हृदि सम्परिवर्तते}
{कदा नाम सुतं द्रक्ष्याम्यभिषिक्तमहं प्रियम्} %2-1-37

\twolineshloka
{वृद्धिकामो हि लोकस्य सर्वभूतानुकम्पकः}
{मत्तः प्रियतरो लोके पर्जन्य इव वृष्टिमान्} %2-1-38

\twolineshloka
{यमशक्रसमो वीर्ये बृहस्पतिसमो मतौ}
{महीधरसमो धृत्यां मत्तश्च गुणवत्तरः} %2-1-39

\twolineshloka
{महीमहमिमां कृत्स्नामधितिष्ठन्तमात्मजम्}
{अनेन वयसा दृष्ट्वा यथा स्वर्गमवाप्नुयाम्} %2-1-40

\twolineshloka
{इत्येवं विविधैस्तैस्तैरन्यपार्थिवदुर्लभैः}
{शिष्टैरपरिमेयैश्च लोके लोकोत्तरैर्गुणैः} %2-1-41

\twolineshloka
{तं समीक्ष्य तदा राजा युक्तं समुदितैर्गुणैः}
{निश्चित्य सचिवैः सार्धं यौवराज्यममन्यत} %2-1-42

\twolineshloka
{दिव्यन्तरिक्षे भूमौ च घोरमुत्पातजं भयम्}
{सञ्चचक्षेऽथ मेधावी शरीरे चात्मनो जराम्} %2-1-43

\twolineshloka
{पूर्णचन्द्राननस्याथ शोकापनुदमात्मनः}
{लोके रामस्य बुबुधे सम्प्रियत्वं महात्मनः} %2-1-44

\twolineshloka
{आत्मनश्च प्रजानां च श्रेयसे च प्रियेण च}
{प्राप्ते काले स धर्मात्मा भक्त्या त्वरितवान् नृपः} %2-1-45

\twolineshloka
{नानानगरवास्तव्यान् पृथग्जानपदानपि}
{समानिनाय मेदिन्यां प्रधानान् पृथिवीपतिः} %2-1-46

\twolineshloka
{तान् वेश्मनानाभरणैर्यथार्हं प्रतिपूजितान्}
{ददर्शालङ्कृतो राजा प्रजापतिरिव प्रजाः} %2-1-47

\twolineshloka
{न तु केकयराजानं जनकं वा नराधिपः}
{त्वरया चानयामास पश्चात्तौ श्रोष्यतः प्रियम्} %2-1-48

\twolineshloka
{अथोपविष्टे नृपतौ तस्मिन् परपुरार्दने}
{ततः प्रविविशुः शेषा राजानो लोकसम्मताः} %2-1-49

\twolineshloka
{अथ राजवितीर्णेषु विविधेष्वासनेषु च}
{राजानमेवाभिमुखा निषेदुर्नियता नृपाः} %2-1-50

\twolineshloka
{स लब्धमानैर्विनयान्वितैर्नृपैः पुरालयैर्जानपदैश्च मानवैः}
{उपोपविष्टैर्नृपतिर्वृतो बभौ सहस्रचक्षुर्भगवानिवामरैः} %2-1-51


॥इत्यार्षे श्रीमद्रामायणे वाल्मीकीये आदिकाव्ये अयोध्याकाण्डे रामाभिषेकव्यवसायः नाम प्रथमः सर्गः ॥२-१॥
