\sect{त्र्यधिकशततमः सर्गः — मातृदर्शनम्}

\twolineshloka
{वसिष्ठः पुरतः कृत्वा दारान् दशरथस्य च}
{अभिचक्राम तं देशं रामदर्शनतर्षितः} %2-103-1

\twolineshloka
{राजपत्न्यश्च गच्छन्त्यो मन्दं मन्दाकिनीं प्रति}
{ददृशुस्तत्र तत्तीर्थं रामलक्ष्मणसेवितम्} %2-103-2

\twolineshloka
{कौसल्या बाष्पपूर्णेन मुखेन परिशुष्यता}
{सुमित्रामब्रवीद्दीना याश्ऺचान्या राजयोषितः} %2-103-3

\twolineshloka
{इदं तेषामनाथानां क्लिष्टमक्लिष्टकर्मणाम्}
{वने प्राक्केवलं तीर्थं ये ते निर्विषयीकृताः} %2-103-4

\twolineshloka
{इतः सुमित्रे पुत्रस्ते सदा जलमतन्द्रितः}
{स्वयं हरति सौमित्रिर्मम पुत्रस्य कारणात्} %2-103-5

\twolineshloka
{जघन्यमपि ते पुत्रः कृतवान्न तु गर्हितः}
{भ्रातुर्यदर्थसहितं सर्वं तद्विहितं गुणैः} %2-103-6

\twolineshloka
{अद्यायमपि ते पुत्रः क्लेशानामतथोचितः}
{नीचानर्थसमाचारं सज्जं कर्म प्रमुञ्चतु} %2-103-7

\twolineshloka
{दक्षिणाग्रेषु दर्भेषु सा ददर्श महीतले}
{पितुरिङ्गुदिपिण्याकं न्यस्तमायतलोचना} %2-103-8

\twolineshloka
{तं भूमौ पितुरार्तेन न्यस्तं रामेण वीक्ष्य सा}
{उवाच देवी कौसल्या सर्वा दशरथस्त्रियः} %2-103-9

\twolineshloka
{इदमिक्ष्वाकुनाथस्य राघवस्य महात्मनः}
{राघवेण पितुर्दत्तं पश्यतैतद्यथाविधि} %2-103-10

\twolineshloka
{तस्य देवसमानस्य पार्थिवस्य महात्मनः}
{नैतदौपयिकं मन्ये भुक्तभोगस्य भोजनम्} %2-103-11

\twolineshloka
{चतुरन्तां महीं भुक्त्वा महेन्द्रसदृशो विभुः}
{कथमिङ्गुदिपिण्याकं स भुङ्क्ते वसुधाधिपः} %2-103-12

\twolineshloka
{अतो दुःखतरं लोके न किञ्चित् प्रतिभाति मा}
{यत्र रामः पितुर्दद्यादिङ्गुदीक्षोदमृद्धिमान्} %2-103-13

\twolineshloka
{रामेणेङ्गुदिपिण्याकं पितुर्दत्तं समीक्ष्य मे}
{कथं दुःखेन हृदयं न स्फोटति सहस्रधा} %2-103-14

\twolineshloka
{श्रुतिस्तु खल्वियं सत्या लौकिकी प्रतिभाति मा}
{यदन्नः पुरुषो भवति तदन्नास्तस्य देवताः} %2-103-15

\twolineshloka
{एवमार्त्तां सपत्न्यस्ता जग्मुराश्वास्य तां तदा}
{ददृशुश्चाश्रमे रामं स्वर्गच्युतमिवामरम्} %2-103-16

\twolineshloka
{सर्वभोगैः परित्यक्तं रामं सम्प्रेक्ष्य मातरः}
{आर्त्ता मुमुचुरश्रूणि सस्वरं शोककर्शिताः} %2-103-17

\twolineshloka
{तासां रामः समुत्थाय जग्राह चरणान् शुभान्}
{मातऽणां मनुजव्याघ्रः सर्वासां सत्यसङ्गरः} %2-103-18

\twolineshloka
{ताः पाणिभिः सुखस्पर्शैर्मृद्वङ्गुलितलैः शुभैः}
{प्रममार्जू रजः पृष्ठाद्रामस्यायतलोचनाः} %2-103-19

\twolineshloka
{सौमित्रिरपि ताः सर्वाः मातऽः सम्प्रेक्ष्य दुःखितः}
{अभ्यवादयतासक्तं शनै रामादनन्तरम्} %2-103-20

\twolineshloka
{यथा रामे तथा तस्मिन् सर्वा ववृतिरे स्त्रियः}
{वृत्तिं दशरथाज्जाते लक्ष्मणे शुभलक्षणे} %2-103-21

\twolineshloka
{सीतापि चरणांस्तासामुपसंगृह्य दुःखिता}
{श्वश्रूणामश्रुपूर्णाक्षी सा बभूवाग्रतः स्थिता} %2-103-22

\twolineshloka
{तां परिष्वज्य दुःखार्त्तां माता दुहितरं यथा}
{वनवासकृशां दीनां कौसल्या वाक्यमब्रवीत्} %2-103-23

\twolineshloka
{विदेहराजस्य सुता स्नुषा दशरथस्य च}
{रामपत्नी कथं दुःखं सम्प्राप्ता निर्जने वने} %2-103-24

\twolineshloka
{पद्ममातपसन्तप्तं परिक्लिष्टमिवोत्पलम्}
{काञ्चनं रजसा ध्वस्तं क्लिष्टं चन्द्रमिवाम्बुदैः} %2-103-25

\onelineshloka
{मुखं ते प्रेक्ष्य मां शोको दहत्यग्निरिवाश्रयम् भृशं मनसि वैदेहि व्यसनारणिसम्भवः} %2-103-26

\twolineshloka
{ब्रुवन्त्यामेवमार्त्तायां जनन्यां भरताग्रजः}
{पादावासाद्य जग्राह वसिष्ठस्य च राघवः} %2-103-27

\twolineshloka
{पुरोहितस्याग्निसमस्य वै तदा बृहस्पतेरिन्द्र इवामराधिपः}
{प्रगृह्य पादौ सुसमृद्धतेजसः सहैव तेनोपविवेश राघवः} %2-103-28

\twolineshloka
{ततो जघन्यं सहितैः समन्त्रिभिः पुरप्रधानैश्च सहैव सैनिकैः}
{जनेन धर्मज्ञतमेन धर्मवानुपोपविष्टो भरतस्तदाग्रजम्} %2-103-29

\twolineshloka
{उपोपविष्टस्तु तथा स वीर्यवांस्तपस्विवेषेण समीक्ष्य राघवम्}
{श्रिया ज्वलन्तं भरतः कृताञ्जलिर्यथा महेन्द्रः प्रयतः प्रजापतिम्} %2-103-30

\twolineshloka
{किमेष वाक्यं भरतोऽद्य राघवं प्रणम्य सत्कृत्य च साधु वक्ष्यति}
{इतीव तस्यार्यजनस्य तत्त्वतो बभूव कौतूहलमुत्तमं तदा} %2-103-31

\twolineshloka
{स राघवः सत्यधृतिश्च लक्ष्मणो महानुभवो भरतश्च धार्मिकः}
{वृताः सुहृद्भिश्च विरेजुरध्वरे यथा सदस्यैः सहितास्त्रयोऽग्नयः} %2-103-32


॥इत्यार्षे श्रीमद्रामायणे वाल्मीकीये आदिकाव्ये अयोध्याकाण्डे मातृदर्शनम् नाम त्र्यधिकशततमः सर्गः ॥२-१०३॥
