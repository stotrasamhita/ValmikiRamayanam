\sect{अशीतितमः सर्गः — मार्गसंस्कारः}

\twolineshloka
{अथ भूमि प्रदेशज्ञाः सूत्र कर्म विशारदाः}
{स्व कर्म अभिरताः शूराः खनका यन्त्रकाः तथा} %2-80-1

\twolineshloka
{कर्म अन्तिकाः स्थपतयः पुरुषा यन्त्र कोविदाः}
{तथा वर्धकयः चैव मार्गिणो वृक्ष तक्षकाः} %2-80-2

\twolineshloka
{कूप काराः सुधा कारा वम्श कर्म कृतः तथा}
{समर्था ये च द्रष्टारः पुरतः ते प्रतस्थिरे} %2-80-3

\twolineshloka
{स तु हर्षात् तम् उद्देशम् जन ओघो विपुलः प्रयान्}
{अशोभत महा वेगः सागरस्य इव पर्वणि} %2-80-4

\twolineshloka
{ते स्व वारम् समास्थाय वर्त्म कर्माणि कोविदाः}
{करणैः विविध उपेतैः पुरस्तात् सम्प्रतस्थिरे} %2-80-5

\twolineshloka
{लता वल्लीः च गुल्मामः च स्थाणून् अश्मनएव च}
{जनाः ते चक्रिरे मार्गम् चिन्दन्तः विविधान् द्रुमान्} %2-80-6

\twolineshloka
{अवृक्षेषु च देशेषु केचित् वृक्षान् अरोपयन्}
{केचित् कुठारैअः टन्कैः च दात्रैः चिन्दन् क्वचित् क्वचित्} %2-80-7

\twolineshloka
{अपरे वीरण स्तम्बान् बलिनो बलवत्तराः}
{विधमन्ति स्म दुर्गाणि स्थलानि च ततः ततः} %2-80-8

\twolineshloka
{अपरे अपूरयन् कूपान् पाम्सुभिः श्वभ्रम् आयतम्}
{निम्न भागाम्स् तथा केचित् समामः चक्रुः समन्ततः} %2-80-9

\twolineshloka
{बबन्धुर् बन्धनीयामः च क्षोद्यान् सम्चुक्षुदुस् तदा}
{बिभिदुर् भेदनीयामः च ताम्स् तान् देशान् नराः तदा} %2-80-10

\twolineshloka
{अचिरेण एव कालेन परिवाहान् बहु उदकान्}
{चक्रुर् बहु विध आकारान् सागर प्रतिमान् बहून्} %2-80-11

\twolineshloka
{निर्जलेषु च देशेषु खानयामासुरुत्तमान्}
{उदपानान् बहुविधान् वेदिका परिमण्डितान्} %2-80-12

\twolineshloka
{ससुधा कुट्टिम तलः प्रपुष्पित मही रुहः}
{मत्त उद्घुष्ट द्विज गणः पताकाभिर् अलम्कृतः} %2-80-13

\twolineshloka
{चन्दन उदक सम्सिक्तः नाना कुसुम भूषितः}
{बह्व् अशोभत सेनायाः पन्थाः स्वर्ग पथ उपमः} %2-80-14

\twolineshloka
{आज्ञाप्य अथ यथा आज्ञप्ति युक्ताः ते अधिकृता नराः}
{रमणीयेषु देशेषु बहु स्वादु फलेषु च} %2-80-15

\twolineshloka
{यो निवेशः तु अभिप्रेतः भरतस्य महात्मनः}
{भूयः तम् शोभयाम् आसुर् भूषाभिर् भूषण उपमम्} %2-80-16

\twolineshloka
{नक्षत्रेषु प्रशस्तेषु मुहूर्तेषु च तद्विदः}
{निवेशम् स्थापयाम् आसुर् भरतस्य महात्मनः} %2-80-17

\twolineshloka
{बहु पाम्सु चयाः च अपि परिखा परिवारिताः}
{तन्त्र इन्द्र कील प्रतिमाः प्रतोली वर शोभिताः} %2-80-18

\twolineshloka
{प्रासाद माला सम्युक्ताः सौध प्राकार सम्वृताः}
{पताका शोभिताः सर्वे सुनिर्मित महा पथाः} %2-80-19

\twolineshloka
{विसर्पत्भिर् इव आकाशे विटन्क अग्र विमानकैः}
{समुच्च्रितैः निवेशाः ते बभुः शक्र पुर उपमाः} %2-80-20

\twolineshloka
{जाह्नवीम् तु समासाद्य विविध द्रुम काननाम्}
{शीतल अमल पानीयाम् महा मीन समाकुलाम्} %2-80-21

\fourlineindentedshloka
{सचन्द्र तारा गण मण्डितम् यथा}
{नभः क्षपायाम् अमलम् विराजते}
{नर इन्द्र मार्गः स तथा व्यराजत}
{क्रमेण रम्यः शुभ शिल्पि निर्मितः} %2-80-22


॥इत्यार्षे श्रीमद्रामायणे वाल्मीकीये आदिकाव्ये अयोध्याकाण्डे मार्गसंस्कारः नाम अशीतितमः सर्गः ॥२-८०॥
