\sect{चत्वारिंशः सर्गः — पौरद्यनुव्रज्या}

\twolineshloka
{अथ रामश्च सीता च लक्ष्मणश्च कृताञ्जलिः}
{उपसङ्गृह्य राजानं चक्रुर्दीनाः प्रदक्षिणम्} %2-40-1

\twolineshloka
{तं चापि समनुज्ञाप्य धर्मज्ञः सह सीतया}
{राघवः शोकसम्मूढो जननीमभ्यवादयत्} %2-40-2

\twolineshloka
{अन्वक्षं लक्ष्मणो भ्रातुः कौसल्यामभ्यवादयत्}
{अपि मातुः सुमित्राया जग्राह चरणौ पुनः} %2-40-3

\twolineshloka
{तं वन्दमानं रुदती माता सौमित्रिमब्रवीत्}
{हितकामा महाबाहुं मूर्ध्न्युपाघ्राय लक्ष्मणम्} %2-40-4

\twolineshloka
{सृष्टस्त्वं वनवासाय स्वनुरक्तः सुहृज्जने}
{रामे प्रमादं मा कार्षीः पुत्र भ्रातरि गच्छति} %2-40-5

\twolineshloka
{व्यसनी वा समृद्धो वा गतिरेष तवानघ}
{एष लोके सतां धर्मो यज्ज्येष्ठवशगो भवेत्} %2-40-6

\twolineshloka
{इदं हि वृत्तमुचितं कुलस्यास्य सनातनम्}
{दानं दीक्षा च यज्ञेषु तनुत्यागो मृधेषु हि} %2-40-7

\twolineshloka
{लक्ष्मणं त्वेवमुक्त्वासौ संसिद्धं प्रियराघवम्}
{सुमित्रा गच्छ गच्छेति पुनः पुनरुवाच तम्} %2-40-8

\twolineshloka
{रामं दशरथं विद्धि मां विद्धि जनकात्मजाम्}
{अयोध्यामटवीं विद्धि गच्छ तात यथासुखम्} %2-40-9

\twolineshloka
{ततः सुमन्त्रः काकुत्स्थं प्राञ्जलिर्वाक्यमब्रवीत्}
{विनीतो विनयज्ञश्च मातलिर्वासवं यथा} %2-40-10

\twolineshloka
{रथमारोह भद्रं ते राजपुत्र महायशः}
{क्षिप्रं त्वां प्रापयिष्यामि यत्र मां राम वक्ष्यसे} %2-40-11

\twolineshloka
{चतुर्दश हि वर्षाणि वस्तव्यानि वने त्वया}
{तान्युपक्रमितव्यानि यानि देव्या प्रचोदितः} %2-40-12

\twolineshloka
{तं रथं सूर्यसङ्काशं सीता हृष्टेन चेतसा}
{आरुरोह वरारोहा कृत्वालङ्कारमात्मनः} %2-40-13

\twolineshloka
{वनवासं हि सङ्ख्याय वासांस्याभरणानि च}
{भर्तारमनुगच्छन्त्यै सीतायै श्वशुरो ददौ} %2-40-14

\twolineshloka
{तथैवायुधजातानि भ्रातृभ्यां कवचानि च}
{रथोपस्थे प्रविन्यस्य सचर्म कठिनं च यत्} %2-40-15

\twolineshloka
{अथो ज्वलनसङ्काशं चामीकरविभूषितम्}
{तमारुरुहतुस्तूर्णं भ्रातरौ रामलक्ष्मणौ} %2-40-16

\twolineshloka
{सीतातृतीयानारूढान् दृष्ट्वा रथमचोदयत्}
{सुमन्त्रः सम्मतानश्वान् वायुवेगसमाञ्जवे} %2-40-17

\twolineshloka
{प्रयाते तु महारण्यं चिररात्राय राघवे}
{बभूव नगरे मूर्च्छा बलमूर्च्छा जनस्य च} %2-40-18

\twolineshloka
{तत् समाकुलसम्भ्रान्तं मत्तसङ्कुपितद्विपम्}
{हयसिञ्जितनिर्घोषं पुरमासीन्महास्वनम्} %2-40-19

\twolineshloka
{ततः सबालवृद्धा सा पुरी परमपीडिता}
{राममेवाभिदुद्राव घर्मार्तः सलिलं यथा} %2-40-20

\twolineshloka
{पार्श्वतः पृष्ठतश्चापि लम्बमानास्तदुन्मुखाः}
{बाष्पपूर्णमुखाः सर्वे तमूचुर्भृशनिःस्वनाः} %2-40-21

\twolineshloka
{संयच्छ वाजिनां रश्मीन् सूत याहि शनैः शनैः}
{मुखं द्रक्ष्याम रामस्य दुर्दर्शं नो भविष्यति} %2-40-22

\twolineshloka
{आयसं हृदयं नूनं राममातुरसंशयम्}
{यद् देवगर्भप्रतिमे वनं याति न भिद्यते} %2-40-23

\twolineshloka
{कृतकृत्या हि वैदेही छायेवानुगता पतिम्}
{न जहाति रता धर्मे मेरुमर्कप्रभा यथा} %2-40-24

\twolineshloka
{अहो लक्ष्मण सिद्धार्थः सततं प्रियवादिनम्}
{भ्रातरं देवसङ्काशं यस्त्वं परिचरिष्यसि} %2-40-25

\twolineshloka
{महत्येषा हि ते बुद्धिरेष चाभ्युदयो महान्}
{एष स्वर्गस्य मार्गश्च यदेनमनुगच्छसि} %2-40-26

\twolineshloka
{एवं वदन्तस्ते सोढुं न शेकुर्बाष्पमागतम्}
{नरास्तमनुगच्छन्ति प्रियमिक्ष्वाकुनन्दनम्} %2-40-27

\twolineshloka
{अथ राजा वृतः स्त्रीभिर्दीनाभिर्दीनचेतनः}
{निर्जगाम प्रियं पुत्रं द्रक्ष्यामीति ब्रुवन् गृहात्} %2-40-28

\twolineshloka
{शुश्रुवे चाग्रतः स्त्रीणां रुदतीनां महास्वनः}
{यथा नादः करेणूनां बद्धे महति कुञ्जरे} %2-40-29

\twolineshloka
{पिता हि राजा काकुत्स्थः श्रीमान् सन्नस्तदा बभौ}
{परिपूर्णः शशी काले ग्रहेणोपप्लुतो यथा} %2-40-30

\twolineshloka
{स च श्रीमानचिन्त्यात्मा रामो दशरथात्मजः}
{सूतं सञ्चोदयामास त्वरितं वाह्यतामिति} %2-40-31

\twolineshloka
{रामो याहीति तं सूतं तिष्ठेति च जनस्तथा}
{उभयं नाशकत् सूतः कर्तुमध्वनि चोदितः} %2-40-32

\twolineshloka
{निर्गच्छति महाबाहौ रामे पौरजनाश्रुभिः}
{पतितैरभ्यवहितं प्रणनाश महीरजः} %2-40-33

\twolineshloka
{रुदिताश्रुपरिद्यूनं हाहाकृतमचेतनम्}
{प्रयाणे राघवस्यासीत् पुरं परमपीडितम्} %2-40-34

\twolineshloka
{सुस्राव नयनैः स्त्रीणामस्रमायाससम्भवम्}
{मीनसङ्क्षोभचलितैः सलिलं पङ्कजैरिव} %2-40-35

\twolineshloka
{दृष्ट्वा तु नृपतिः श्रीमानेकचित्तगतं पुरम्}
{निपपातैव दुःखेन कृत्तमूल इव द्रुमः} %2-40-36

\twolineshloka
{ततो हलहलाशब्दो जज्ञे रामस्य पृष्ठतः}
{नराणां प्रेक्ष्य राजानं सीदन्तं भृशदुःखितम्} %2-40-37

\twolineshloka
{हा रामेति जनाः केचिद् राममातेति चापरे}
{अन्तःपुरसमृद्धं च क्रोशन्तं पर्यदेवयन्} %2-40-38

\twolineshloka
{अन्वीक्षमाणो रामस्तु विषण्णं भ्रान्तचेतसम्}
{राजानं मातरं चैव ददर्शानुगतौ पथि} %2-40-39

\twolineshloka
{स बद्ध इव पाशेन किशोरो मातरं यथा}
{धर्मपाशेन संयुक्तः प्रकाशं नाभ्युदैक्षत} %2-40-40

\twolineshloka
{पदातिनौ च यानार्हावदुःखार्हौ सुखोचितौ}
{दृष्ट्वा सञ्चोदयामास शीघ्रं याहीति सारथिम्} %2-40-41

\twolineshloka
{नहि तत् पुरुषव्याघ्रो दुःखजं दर्शनं पितुः}
{मातुश्च सहितुं शक्तस्तोत्त्रैर्नुन्न इव द्विपः} %2-40-42

\twolineshloka
{प्रत्यगारमिवायान्ती सवत्सा वत्सकारणात्}
{बद्धवत्सा यथा धेनू राममाताभ्यधावत} %2-40-43

\twolineshloka
{तथा रुदन्तीं कौसल्यां रथं तमनुधावतीम्}
{क्रोशन्तीं राम रामेति हा सीते लक्ष्मणेति च} %2-40-44

\twolineshloka
{रामलक्ष्मणसीतार्थं स्रवन्तीं वारि नेत्रजम्}
{असकृत् प्रैक्षत स तां नृत्यन्तीमिव मातरम्} %2-40-45

\twolineshloka
{तिष्ठेति राजा चुक्रोश याहि याहीति राघवः}
{सुमन्त्रस्य बभूवात्मा चक्रयोरिव चान्तरा} %2-40-46

\twolineshloka
{नाश्रौषमिति राजानमुपालब्धोऽपि वक्ष्यसि}
{चिरं दुःखस्य पापिष्ठमिति रामस्तमब्रवीत्} %2-40-47

\twolineshloka
{स रामस्य वचः कुर्वन्ननुज्ञाप्य च तं जनम्}
{व्रजतोऽपि हयान् शीघ्रं चोदयामास सारथिः} %2-40-48

\twolineshloka
{न्यवर्तत जनो राज्ञो रामं कृत्वा प्रदक्षिणम्}
{मनसाप्याशुवेगेन न न्यवर्तत मानुषम्} %2-40-49

\twolineshloka
{यमिच्छेत् पुनरायातं नैनं दूरमनुव्रजेत्}
{इत्यमात्या महाराजमूचुर्दशरथं वचः} %2-40-50

\twolineshloka
{तेषां वचः सर्वगुणोपपन्नः प्रस्विन्नगात्रः प्रविषण्णरूपः}
{निशम्य राजा कृपणः सभार्यो व्यवस्थितस्तं सुतमीक्षमाणः} %2-40-51


॥इत्यार्षे श्रीमद्रामायणे वाल्मीकीये आदिकाव्ये अयोध्याकाण्डे पौरद्यनुव्रज्या नाम चत्वारिंशः सर्गः ॥२-४०॥
