\sect{पञ्चविंशः सर्गः — मातृस्वस्त्ययनम्}

\twolineshloka
{सा विनीय तमायासमुपस्पृश्य जलं शुचि}
{चकार माता रामस्य मङ्गलानि मनस्विनी} %2-25-1

\twolineshloka
{न शक्यसे वारयितुं गच्छेदानीं रघूत्तम}
{शीघ्रं च विनिवर्तस्व वर्तस्व च सतां क्रमे} %2-25-2

\twolineshloka
{यं पालयसि धर्मं त्वं प्रीत्या च नियमेन च}
{स वै राघवशार्दूल धर्मस्त्वामभिरक्षतु} %2-25-3

\twolineshloka
{येभ्यः प्रणमसे पुत्र देवेष्वायतनेषु च}
{ते च त्वामभिरक्षन्तु वने सह महर्षिभिः} %2-25-4

\twolineshloka
{यानि दत्तानि तेऽस्त्राणि विश्वामित्रेण धीमता}
{तानि त्वामभिरक्षन्तु गुणैः समुदितं सदा} %2-25-5

\twolineshloka
{पितृशुश्रूषया पुत्र मातृशुश्रूषया तथा}
{सत्येन च महाबाहो चिरं जीवाभिरक्षितः} %2-25-6

\threelineshloka
{समित्कुशपवित्राणि वेद्यश्चायतनानि च}
{स्थण्डिलानि च विप्राणां शैला वृक्षाः क्षुपा ह्रदाः}
{पतङ्गाः पन्नगाः सिंहास्त्वां रक्षन्तु नरोत्तम} %2-25-7

\twolineshloka
{स्वस्ति साध्याश्च विश्वे च मरुतश्च महर्षिभिः}
{स्वस्ति धाता विधाता च स्वस्ति पूषा भगोऽर्यमा} %2-25-8

\twolineshloka
{लोकपालाश्च ते सर्वे वासवप्रमुखास्तथा}
{ऋतवः षट् च ते सर्वे मासाः संवत्सराः क्षपाः} %2-25-9

\twolineshloka
{दिनानि च मुहूर्ताश्च स्वस्ति कुर्वन्तु ते सदा}
{श्रुतिः स्मृतिश्च धर्मश्च पातु त्वां पुत्र सर्वतः} %2-25-10

\twolineshloka
{स्कन्दश्च भगवान् देवः सोमश्च सबृहस्पतिः}
{सप्तर्षयो नारदश्च ते त्वां रक्षन्तु सर्वतः} %2-25-11

\twolineshloka
{ते चापि सर्वतः सिद्धा दिशश्च सदिगीश्वराः}
{स्तुता मया वने तस्मिन् पान्तु त्वां पुत्र नित्यशः} %2-25-12

\twolineshloka
{शैलाः सर्वे समुद्राश्च राजा वरुण एव च}
{द्यौरन्तरिक्षं पृथिवी वायुश्च सचराचरः} %2-25-13

\twolineshloka
{नक्षत्राणि च सर्वाणि ग्रहाश्च सह दैवतैः}
{अहोरात्रे तथा संध्ये पान्तु त्वां वनमाश्रितम्} %2-25-14

\twolineshloka
{ऋतवश्चापि षट् चान्ये मासाः संवत्सरास्तथा}
{कलाश्च काष्ठाश्च तथा तव शर्म दिशन्तु ते} %2-25-15

\twolineshloka
{महावनेऽपि चरतो मुनिवेषस्य धीमतः}
{तथा देवाश्च दैत्याश्च भवन्तु सुखदाः सदा} %2-25-16

\twolineshloka
{राक्षसानां पिशाचानां रौद्राणां क्रूरकर्मणाम्}
{क्रव्यादानां च सर्वेषां मा भूत् पुत्रक ते भयम्} %2-25-17

\twolineshloka
{प्लवगा वृश्चिका दंशा मशकाश्चैव कानने}
{सरीसृपाश्च कीटाश्च मा भूवन् गहने तव} %2-25-18

\twolineshloka
{महाद्विपाश्च सिंहाश्च व्याघ्रा ऋक्षाश्च दंष्ट्रिणः}
{महिषाः शृङ्गिणो रौद्रा न ते द्रुह्यन्तु पुत्रक} %2-25-19

\twolineshloka
{नृमांसभोजना रौद्रा ये चान्ये सर्वजातयः}
{मा च त्वां हिंसिषुः पुत्र मया सम्पूजितास्त्विह} %2-25-20

\twolineshloka
{आगमास्ते शिवाः सन्तु सिध्यन्तु च पराक्रमाः}
{सर्वसम्पत्तयो राम स्वस्तिमान् गच्छ पुत्रक} %2-25-21

\twolineshloka
{स्वस्ति तेऽस्त्वान्तरिक्षेभ्यः पार्थिवेभ्यः पुनः पुनः}
{सर्वेभ्यश्चैव देवेभ्यो ये च ते परिपन्थिनः} %2-25-22

\twolineshloka
{शुक्रः सोमश्च सूर्यश्च धनदोऽथ यमस्तथा}
{पान्तु त्वामर्चिता राम दण्डकारण्यवासिनम्} %2-25-23

\twolineshloka
{अग्निर्वायुस्तथा धूमो मन्त्राश्चर्षिमुखच्युताः}
{उपस्पर्शनकाले तु पान्तु त्वां रघुनन्दन} %2-25-24

\twolineshloka
{सर्वलोकप्रभुर्ब्रह्मा भूतकर्तृ तथर्षयः}
{ये च शेषाः सुरास्ते तु रक्षन्तु वनवासिनम्} %2-25-25

\twolineshloka
{इति माल्यैः सुरगणान् गन्धैश्चापि यशस्विनी}
{स्तुतिभिश्चानुरूपाभिरानर्चायतलोचना} %2-25-26

\twolineshloka
{ज्वलनं समुपादाय ब्राह्मणेन महात्मना}
{हावयामास विधिना राममङ्गलकारणात्} %2-25-27

\twolineshloka
{घृतं श्वेतानि माल्यानि समिधश्चैव सर्षपान्}
{उपसम्पादयामास कौसल्या परमाङ्गना} %2-25-28

\twolineshloka
{उपाध्यायः स विधिना हुत्वा शान्तिमनामयम्}
{हुतहव्यावशेषेण बाह्यं बलिमकल्पयत्} %2-25-29

\twolineshloka
{मधुदध्यक्षतघृतैः स्वस्तिवाच्यं द्विजांस्ततः}
{वाचयामास रामस्य वने स्वस्त्ययनक्रियाम्} %2-25-30

\twolineshloka
{ततस्तस्मै द्विजेन्द्राय राममाता यशस्विनी}
{दक्षिणां प्रददौ काम्यां राघवं चेदमब्रवीत्} %2-25-31

\twolineshloka
{यन्मङ्गलं सहस्राक्षे सर्वदेवनमस्कृते}
{वृत्रनाशे समभवत् तत् ते भवतु मङ्गलम्} %2-25-32

\twolineshloka
{यन्मङ्गलं सुपर्णस्य विनताकल्पयत् पुरा}
{अमृतं प्रार्थयानस्य तत् ते भवतु मङ्गलम्} %2-25-33

\twolineshloka
{अमृतोत्पादने दैत्यान् घ्नतो वज्रधरस्य यत्}
{अदितिर्मङ्गलं प्रादात् तत् ते भवतु मङ्गलम्} %2-25-34

\twolineshloka
{त्रिविक्रमान् प्रक्रमतो विष्णोरतुलतेजसः}
{यदासीन्मङ्गलं राम तत् ते भवतु मङ्गलम्} %2-25-35

\twolineshloka
{ऋषयः सागरा द्वीपा वेदा लोका दिशश्च ते}
{मङ्गलानि महाबाहो दिशन्तु शुभमङ्गलम्} %2-25-36

\twolineshloka
{इति पुत्रस्य शेषाश्च कृत्वा शिरसि भामिनी}
{गन्धैश्चापि समालभ्य राममायतलोचना} %2-25-37

\twolineshloka
{औषधीं च सुसिद्धार्थां विशल्यकरणीं शुभाम्}
{चकार रक्षां कौसल्या मन्त्रैरभिजजाप च} %2-25-38

\twolineshloka
{उवाचापि प्रहृष्टेव सा दुःखवशवर्तिनी}
{वाङ्मात्रेण न भावेन वाचा संसज्जमानया} %2-25-39

\twolineshloka
{आनम्य मूर्ध्नि चाघ्राय परिष्वज्य यशस्विनी}
{अवदत् पुत्रमिष्टार्थो गच्छ राम यथासुखम्} %2-25-40

\twolineshloka
{अरोगं सर्वसिद्धार्थमयोध्यां पुनरागतम्}
{पश्यामि त्वां सुखं वत्स संधितं राजवर्त्मसु} %2-25-41

\twolineshloka
{प्रणष्टदुःखसंकल्पा हर्षविद्योतितानना}
{द्रक्ष्यामि त्वां वनात् प्राप्तं पूर्णचन्द्रमिवोदितम्} %2-25-42

\twolineshloka
{भद्रासनगतं राम वनवासादिहागतम्}
{द्रक्ष्यामि च पुनस्त्वां तु तीर्णवन्तं पितुर्वचः} %2-25-43

\twolineshloka
{मङ्गलैरुपसम्पन्नो वनवासादिहागतः}
{वध्वाश्च मम नित्यं त्वं कामान् संवर्ध याहि भोः} %2-25-44

\twolineshloka
{मयार्चिता देवगणाः शिवादयो महर्षयो भूतगणाः सुरोरगाः}
{अभिप्रयातस्य वनं चिराय ते हितानि कांक्षन्तु दिशश्च राघव} %2-25-45

\twolineshloka
{अतीव चाश्रुप्रतिपूर्णलोचना समाप्य च स्वस्त्ययनं यथाविधि}
{प्रदक्षिणं चापि चकार राघवं पुनः पुनश्चापि निरीक्ष्य सस्वजे} %2-25-46

\twolineshloka
{तया हि देव्या च कृतप्रदक्षिणो निपीड्य मातुश्चरणौ पुनः पुनः}
{जगाम सीतानिलयं महायशाः स राघवः प्रज्वलितस्तया श्रिया} %2-25-47


॥इत्यार्षे श्रीमद्रामायणे वाल्मीकीये आदिकाव्ये अयोध्याकाण्डे मातृस्वस्त्ययनम् नाम पञ्चविंशः सर्गः ॥२-२५॥
