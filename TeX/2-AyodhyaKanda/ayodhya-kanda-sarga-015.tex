\sect{पञ्चदशः सर्गः — सुमन्त्रप्रेषणम्}

\twolineshloka
{ते तु तां रजनीमुष्य ब्राह्मणा वेदपारगाः}
{उपतस्थुरुपस्थानं सह राजपुरोहिताः} %2-15-1

\twolineshloka
{अमात्या बलमुख्याश्च मुख्या ये निगमस्य च}
{राघवस्याभिषेकार्थे प्रीयमाणाः सुसङ्गताः} %2-15-2

\twolineshloka
{उदिते विमले सूर्ये पुष्ये चाभ्यागतेऽहनि}
{लग्ने कर्कटके प्राप्ते जन्म रामस्य च स्थिते} %2-15-3

\twolineshloka
{अभिषेकाय रामस्य द्विजेन्द्रैरुपकल्पितम्}
{काञ्चना जलकुम्भाश्च भद्रपीठं स्वलङ्कृतम्} %2-15-4

\twolineshloka
{रथश्च सम्यगास्तीर्णो भास्वता व्याघ्रचर्मणा}
{गङ्गायमुनयोः पुण्यात् सङ्गमादाहृतं जलम्} %2-15-5

\twolineshloka
{याश्चान्याः सरितः पुण्या ह्रदाः कूपाः सरांसि च}
{प्राग्वहाश्चोर्ध्ववाहाश्च तिर्यग्वाहाश्च क्षीरिणः} %2-15-6

\twolineshloka
{ताभ्यश्चैवाहृतं तोयं समुद्रेभ्यश्च सर्वशः}
{क्षौद्रं दधि घृतं लाजा दर्भाः सुमनसः पयः} %2-15-7

\twolineshloka
{अष्टौ च कन्या रुचिरा मत्तश्च वरवारणः}
{सजलाः क्षीरिभिश्छन्ना घटाः काञ्चनराजताः} %2-15-8

\twolineshloka
{पद्मोत्पलयुता भान्ति पूर्णाः परमवारिणा}
{चन्द्रांशुविकचप्रख्यं पाण्डुरं रत्नभूषितम्} %2-15-9

\twolineshloka
{सज्जं तिष्ठति रामस्य वालव्यजनमुत्तमम्}
{चन्द्रमण्डलसङ्काशमातपत्रं च पाण्डुरम्} %2-15-10

\twolineshloka
{सज्जं द्युतिकरं श्रीमदभिषेकपुरस्सरम्}
{पाण्डुरश्च वृषः सज्जः पाण्डुराश्वश्च संस्थितः} %2-15-11

\twolineshloka
{वादित्राणि च सर्वाणि वन्दिनश्च तथापरे}
{इक्ष्वाकूणां यथा राज्ये सम्भ्रियेताभिषेचनम्} %2-15-12

\twolineshloka
{तथाजातीयमादाय राजपुत्राभिषेचनम्}
{ते राजवचनात् तत्र समवेता महीपतिम्} %2-15-13

\twolineshloka
{अपश्यन्तोऽब्रुवन् को नु राज्ञो नः प्रतिवेदयेत्}
{न पश्यामश्च राजानमुदितश्च दिवाकरः} %2-15-14

\twolineshloka
{यौवराज्याभिषेकश्च सज्जो रामस्य धीमतः}
{इति तेषु ब्रुवाणेषु सर्वांस्तांश्च महीपतीन्} %2-15-15

\twolineshloka
{अब्रवीत् तानिदं वाक्यं सुमन्त्रो राजसत्कृतः}
{रामं राज्ञो नियोगेन त्वरया प्रस्थितो ह्यहम्} %2-15-16

\twolineshloka
{पूज्या राज्ञो भवन्तश्च रामस्य तु विशेषतः}
{अयं पृच्छामि वचनात् सुखमायुष्मतामहम्} %2-15-17

\twolineshloka
{राज्ञः सम्प्रतिबुद्धस्य चानागमनकारणम्}
{इत्युक्त्वान्तःपुरद्वारमाजगाम पुराणवित्} %2-15-18

\twolineshloka
{सदा सक्तं च तद् वेश्म सुमन्त्रः प्रविवेश ह}
{तुष्टावास्य तदा वंशं प्रविश्य स विशाम्पतेः} %2-15-19

\twolineshloka
{शयनीयं नरेन्द्रस्य तदासाद्य व्यतिष्ठत}
{सोऽत्यासाद्य तु तद् वेश्म तिरस्करणिमन्तरा} %2-15-20

\twolineshloka
{आशीर्भिर्गुणयुक्ताभिरभितुष्टाव राघवम्}
{सोमसूर्यौ च काकुत्स्थ शिववैश्रवणावपि} %2-15-21

\twolineshloka
{वरुणश्चाग्निरिन्द्रश्च विजयं प्रदिशन्तु ते}
{गता भगवती रात्रिरहः शिवमुपस्थितम्} %2-15-22

\twolineshloka
{बुद्ध्यस्व राजशार्दूल कुरु कार्यमनन्तरम्}
{ब्राह्मणा बलमुख्याश्च नैगमाश्चागतास्त्विह} %2-15-23

\twolineshloka
{दर्शनं तेऽभिकाङ्क्षन्ते प्रतिबुद्ध्यस्व राघव}
{स्तुवन्तं तं तदा सूतं सुमन्त्रं मन्त्रकोविदम्} %2-15-24

\twolineshloka
{प्रतिबुद्ध्य ततो राजा इदं वचनमब्रवीत्}
{राममानय सूतेति यदस्यभिहितो मया} %2-15-25

\twolineshloka
{किमिदं कारणं येन ममाज्ञा प्रतिवाह्यते}
{न चैव सम्प्रसुप्तोऽहमानयेहाशु राघवम्} %2-15-26

\twolineshloka
{इति राजा दशरथः सूतं तत्रान्वशात् पुनः}
{स राजवचनं श्रुत्वा शिरसा प्रतिपूज्य तम्} %2-15-27

\twolineshloka
{निर्जगाम नृपावासान्मन्यमानः प्रियं महत्}
{प्रपन्नो राजमार्गं च पताकाध्वजशोभितम्} %2-15-28

\twolineshloka
{हृष्टः प्रमुदितः सूतो जगामाशु विलोकयन्}
{स सूतस्तत्र शुश्राव रामाधिकरणाः कथाः} %2-15-29

\twolineshloka
{अभिषेचनसंयुक्ताः सर्वलोकस्य हृष्टवत्}
{ततो ददर्श रुचिरं कैलाससदृशप्रभम्} %2-15-30

\twolineshloka
{रामवेश्म सुमन्त्रस्तु शक्रवेश्मसमप्रभम्}
{महाकपाटपिहितं वितर्दिशतशोभितम्} %2-15-31

\twolineshloka
{काञ्चनप्रतिमैकाग्रं मणिविद्रुमतोरणम्}
{शारदाभ्रघनप्रख्यं दीप्तं मेरुगुहासमम्} %2-15-32

\twolineshloka
{मणिभिर्वरमाल्यानां सुमहद्भिरलङ्कृतम्}
{मुक्तामणिभिराकीर्णं चन्दनागुरुभूषितम्} %2-15-33

\twolineshloka
{गन्धान् मनोज्ञान् विसृजद् दार्दुरं शिखरं यथा}
{सारसैश्च मयूरैश्च विनदद्भिर्विराजितम्} %2-15-34

\twolineshloka
{सुकृतेहामृगाकीर्णमुत्कीर्णं भक्तिभिस्तथा}
{मनश्चक्षुश्च भूतानामाददत् तिग्मतेजसा} %2-15-35

\twolineshloka
{चन्द्रभास्करसङ्काशं कुबेरभवनोपमम्}
{महेन्द्रधामप्रतिमं नानापक्षिसमाकुलम्} %2-15-36

\twolineshloka
{मेरुशृङ्गसमं सूतो रामवेश्म ददर्श ह}
{उपस्थितैः समाकीर्णं जनैरञ्जलिकारिभिः} %2-15-37

\twolineshloka
{उपादाय समाक्रान्तैस्तदा जानपदैर्जनैः}
{रामाभिषेकसुमुखैरुन्मुखैः समलङ्कृतम्} %2-15-38

\twolineshloka
{महामेघसमप्रख्यमुदग्रं सुविराजितम्}
{नानारत्नसमाकीर्णं कुब्जकैरपि चावृतम्} %2-15-39

\twolineshloka
{स वाजियुक्तेन रथेन सारथिः समाकुलं राजकुलं विराजयन्}
{वरूथिना राजगृहाभिपातिना पुरस्य सर्वस्य मनांसि हर्षयन्} %2-15-40

\twolineshloka
{ततः समासाद्य महाधनं महत् प्रहृष्टरोमा स बभूव सारथिः}
{मृगैर्मयूरैश्च समाकुलोल्बणं गृहं वरार्हस्य शचीपतेरिव} %2-15-41

\twolineshloka
{स तत्र कैलासनिभाः स्वलङ्कृताः प्रविश्य कक्ष्यास्त्रिदशालयोपमाः}
{प्रियान् वरान् राममते स्थितान् बहून् व्यपोह्य शुद्धान्तमुपस्थितौ रथी} %2-15-42

\twolineshloka
{स तत्र शुश्राव च हर्षयुक्ता रामाभिषेकार्थकृतां जनानाम्}
{नरेन्द्रसूनोरभिमङ्गलार्थाः सर्वस्य लोकस्य गिरः प्रहृष्टाः} %2-15-43

\twolineshloka
{महेन्द्रसद्मप्रतिमं च वेश्म रामस्य रम्यं मृगपक्षिजुष्टम्}
{ददर्श मेरोरिव शृङ्गमुच्चं विभ्राजमानं प्रभया सुमन्त्रः} %2-15-44

\twolineshloka
{उपस्थितैरञ्जलिकारिभिश्च सोपायनैर्जानपदैर्जनैश्च}
{कोट्या परार्धैश्च विमुक्तयानैः समाकुलं द्वारपदं ददर्श} %2-15-45

\twolineshloka
{ततो महामेघमहीधराभं प्रभिन्नमत्यङ्कुशमत्यसह्यम्}
{रामोपवाह्यं रुचिरं ददर्श शत्रुञ्जयं नागमुदग्रकायम्} %2-15-46

\twolineshloka
{स्वलङ्कृतान् साश्वरथान् सकुञ्जरानमात्यमुख्यांश्च ददर्श वल्लभान्}
{व्यपोह्य सूतः सहितान् समन्ततः समृद्धमन्तःपुरमाविवेश ह} %2-15-47

\twolineshloka
{ततोऽद्रिकूटाचलमेघसन्निभं महाविमानोपमवेश्मसंयुतम्}
{अवार्यमाणः प्रविवेश सारथिः प्रभूतरत्नं मकरो यथार्णवम्} %2-15-48


॥इत्यार्षे श्रीमद्रामायणे वाल्मीकीये आदिकाव्ये अयोध्याकाण्डे सुमन्त्रप्रेषणम् नाम पञ्चदशः सर्गः ॥२-१५॥
