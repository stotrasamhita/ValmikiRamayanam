\sect{पञ्चचत्वारिंशः सर्गः — पौरयाचनम्}

\twolineshloka
{अनुरक्ता महात्मानं रामं सत्यपराक्रमम्}
{अनुजग्मुः प्रयान्तं तं वनवासाय मानवाः} %2-45-1

\twolineshloka
{निवर्तितेऽतीव बलात् सुहृद्धर्मेण राजनि}
{नैव ते सन्न्यवर्तन्त रामस्यानुगता रथम्} %2-45-2

\twolineshloka
{अयोध्यानिलयानां हि पुरुषाणां महायशाः}
{बभूव गुणसम्पन्नः पूर्णचन्द्र इव प्रियः} %2-45-3

\twolineshloka
{स याच्यमानः काकुत्स्थस्ताभिः प्रकृतिभिस्तदा}
{कुर्वाणः पितरं सत्यं वनमेवान्वपद्यत} %2-45-4

\twolineshloka
{अवेक्षमाणः सस्नेहं चक्षुषा प्रपिबन्निव}
{उवाच रामः सस्नेहं ताः प्रजाः स्वाः प्रजा इव} %2-45-5

\twolineshloka
{या प्रीतिर्बहुमानश्च मय्ययोध्यानिवासिनाम्}
{मत्प्रियार्थं विशेषेण भरते सा विधीयताम्} %2-45-6

\twolineshloka
{स हि कल्याणचारित्रः कैकेय्यानन्दवर्धनः}
{करिष्यति यथावद् वः प्रियाणि च हितानि च} %2-45-7

\twolineshloka
{ज्ञानवृद्धो वयोबालो मृदुर्वीर्यगुणान्वितः}
{अनुरूपः स वो भर्ता भविष्यति भयापहः} %2-45-8

\twolineshloka
{स हि राजगुणैर्युक्तो युवराजः समीक्षितः}
{अपि चापि मया शिष्टैः कार्यं वो भर्तृशासनम्} %2-45-9

\twolineshloka
{न सन्तप्येद् यथा चासौ वनवासं गते मयि}
{महाराजस्तथा कार्यो मम प्रियचिकीर्षया} %2-45-10

\twolineshloka
{यथा यथा दाशरथिर्धर्ममेवाश्रितो भवेत्}
{तथा तथा प्रकृतयो रामं पतिमकामयन्} %2-45-11

\twolineshloka
{बाष्पेण पिहितं दीनं रामः सौमित्रिणा सह}
{चकर्षेव गुणैर्बद्धं जनं पुरनिवासिनम्} %2-45-12

\twolineshloka
{ते द्विजास्त्रिविधं वृद्धा ज्ञानेन वयसौजसा}
{वयःप्रकम्पशिरसो दूरादूचुरिदं वचः} %2-45-13

\twolineshloka
{वहन्तो जवना रामं भो भो जात्यास्तुरङ्गमाः}
{निवर्तध्वं न गन्तव्यं हिता भवत भर्तरि} %2-45-14

\twolineshloka
{कर्णवन्ति हि भूतानि विशेषेण तुरङ्गमाः}
{यूयं तस्मान्निवर्तध्वं याचनां प्रतिवेदिताः} %2-45-15

\twolineshloka
{धर्मतः स विशुद्धात्मा वीरः शुभदृढव्रतः}
{उपवाह्यस्तु वो भर्ता नापवाह्यः पुराद् वनम्} %2-45-16

\twolineshloka
{एवमार्तप्रलापांस्तान् वृद्धान् प्रलपतो द्विजान्}
{अवेक्ष्य सहसा रामो रथादवततार ह} %2-45-17

\twolineshloka
{पद्भ्यामेव जगामाथ ससीतः सहलक्ष्मणः}
{सन्निकृष्टपदन्यासो रामो वनपरायणः} %2-45-18

\twolineshloka
{द्विजातीन् हि पदातींस्तान् रामश्चारित्रवत्सलः}
{न शशाक घृणाचक्षुः परिमोक्तुं रथेन सः} %2-45-19

\twolineshloka
{गच्छन्तमेव तं दृष्ट्वा रामं सम्भ्रान्तमानसाः}
{ऊचुः परमसन्तप्ता रामं वाक्यमिदं द्विजाः} %2-45-20

\twolineshloka
{ब्राह्मण्यं कृत्स्नमेतत् त्वां ब्रह्मण्यमनुगच्छति}
{द्विजस्कन्धाधिरूढास्त्वामग्नयोऽप्यनुयान्त्वमी} %2-45-21

\twolineshloka
{वाजपेयसमुत्थानि च्छत्राण्येतानि पश्य नः}
{पृष्ठतोऽनुप्रयातानि मेघानिव जलात्यये} %2-45-22

\twolineshloka
{अनवाप्तातपत्रस्य रश्मिसन्तापितस्य ते}
{एभिश्छायां करिष्यामः स्वैश्छत्रैर्वाजपेयकैः} %2-45-23

\twolineshloka
{या हि नः सततं बुद्धिर्वेदमन्त्रानुसारिणी}
{त्वत्कृते सा कृता वत्स वनवासानुसारिणी} %2-45-24

\twolineshloka
{हृदयेष्ववतिष्ठन्ते वेदा ये नः परं धनम्}
{वत्स्यन्त्यपि गृहेष्वेव दाराश्चारित्ररक्षिताः} %2-45-25

\twolineshloka
{पुनर्न निश्चयः कार्यस्त्वद्गतौ सुकृता मतिः}
{त्वयि धर्मव्यपेक्षे तु किं स्याद् धर्मपथे स्थितम्} %2-45-26

\twolineshloka
{याचितो नो निवर्तस्व हंसशुक्लशिरोरुहैः}
{शिरोभिर्निभृताचार महीपतनपांसुलैः} %2-45-27

\twolineshloka
{बहूनां वितता यज्ञा द्विजानां य इहागताः}
{तेषां समाप्तिरायत्ता तव वत्स निवर्तने} %2-45-28

\twolineshloka
{भक्तिमन्तीह भूतानि जङ्गमाजङ्गमानि च}
{याचमानेषु तेषु त्वं भक्तिं भक्तेषु दर्शय} %2-45-29

\twolineshloka
{अनुगन्तुमशक्तास्त्वां मूलैरुद्धतवेगिनः}
{उन्नता वायुवेगेन विक्रोशन्तीव पादपाः} %2-45-30

\twolineshloka
{निश्चेष्टाहारसञ्चारा वृक्षैकस्थाननिश्चिताः}
{पक्षिणोऽपि प्रयाचन्ते सर्वभूतानुकम्पिनम्} %2-45-31

\twolineshloka
{एवं विक्रोशतां तेषां द्विजातीनां निवर्तने}
{ददृशे तमसा तत्र वारयन्तीव राघवम्} %2-45-32

\twolineshloka
{ततः सुमन्त्रोऽपि रथाद् विमुच्य श्रान्तान् हयान् सम्परिवर्त्य शीघ्रम्}
{पीतोदकांस्तोयपरिप्लुताङ्गानचारयद् वै तमसाविदूरे} %2-45-33


॥इत्यार्षे श्रीमद्रामायणे वाल्मीकीये आदिकाव्ये अयोध्याकाण्डे पौरयाचनम् नाम पञ्चचत्वारिंशः सर्गः ॥२-४५॥
