\sect{त्रिंशः सर्गः — वनगमनाभ्युपपत्तिः}

\twolineshloka
{सान्त्व्यमाना तु रामेण मैथिली जनकात्मजा}
{वनवासनिमित्तार्थं भर्तारमिदमब्रवीत्} %2-30-1

\twolineshloka
{सा तमुत्तमसंविग्ना सीता विपुलवक्षसम्}
{प्रणयाच्चाभिमानाच्च परिचिक्षेप राघवम्} %2-30-2

\twolineshloka
{किं त्वामन्यत वैदेहः पिता मे मिथिलाधिपः}
{राम जामातरं प्राप्य स्त्रियं पुरुषविग्रहम्} %2-30-3

\twolineshloka
{अनृतं बत लोकोऽयमज्ञानाद् यदि वक्ष्यति}
{तेजो नास्ति परं रामे तपतीव दिवाकरे} %2-30-4

\twolineshloka
{किं हि कृत्वा विषण्णस्त्वं कुतो वा भयमस्ति ते}
{यत् परित्यक्तुकामस्त्वं मामनन्यपरायणाम्} %2-30-5

\twolineshloka
{द्युमत्सेनसुतं वीरं सत्यवन्तमनुव्रताम्}
{सावित्रीमिव मां विद्धि त्वमात्मवशवर्तिनीम्} %2-30-6

\twolineshloka
{न त्वहं मनसा त्वन्यं द्रष्टास्मि त्वदृतेऽनघ}
{त्वया राघव गच्छेयं यथान्या कुलपांसनी} %2-30-7

\twolineshloka
{स्वयं तु भार्यां कौमारीं चिरमध्युषितां सतीम्}
{शैलूष इव मां राम परेभ्यो दातुमिच्छसि} %2-30-8

\twolineshloka
{यस्य पथ्यञ्चरामात्थ यस्य चार्थेऽवरुध्यसे}
{त्वं तस्य भव वश्यश्च विधेयश्च सदानघ} %2-30-9

\twolineshloka
{स मामनादाय वनं न त्वं प्रस्थितुमर्हसि}
{तपो वा यदि वारण्यं स्वर्गो वा स्यात् त्वया सह} %2-30-10

\twolineshloka
{न च मे भविता तत्र कश्चित् पथि परिश्रमः}
{पृष्ठतस्तव गच्छन्त्या विहारशयनेष्विव} %2-30-11

\twolineshloka
{कुशकाशशरेषीका ये च कण्टकिनो द्रुमाः}
{तूलाजिनसमस्पर्शा मार्गे मम सह त्वया} %2-30-12

\twolineshloka
{महावातसमुद्भूतं यन्मामवकरिष्यति}
{रजो रमण तन्मन्ये परार्घ्यमिव चन्दनम्} %2-30-13

\twolineshloka
{शाद्वलेषु यदा शिश्ये वनान्तर्वनगोचरा}
{कुथास्तरणयुक्तेषु किं स्यात् सुखतरं ततः} %2-30-14

\twolineshloka
{पत्रं मूलं फलं यत्तु अल्पं वा यदि वा बहु}
{दास्यसे स्वयमाहृत्य तन्मेऽमृतरसोपमम्} %2-30-15

\twolineshloka
{न मातुर्न पितुस्तत्र स्मरिष्यामि न वेश्मनः}
{आर्तवान्युपभुञ्जाना पुष्पाणि च फलानि च} %2-30-16

\twolineshloka
{न च तत्र ततः किञ्चिद् द्रष्टुमर्हसि विप्रियम्}
{मत्कृते न च ते शोको न भविष्यामि दुर्भरा} %2-30-17

\twolineshloka
{यस्त्वया सह स स्वर्गो निरयो यस्त्वया विना}
{इति जानन् परां प्रीतिं गच्छ राम मया सह} %2-30-18

\twolineshloka
{अथ मामेवमव्यग्रां वनं नैव नयिष्यसे}
{विषमद्यैव पास्यामि मा वशं द्विषतां गमम्} %2-30-19

\twolineshloka
{पश्चादपि हि दुःखेन मम नैवास्ति जीवितम्}
{उज्झितायास्त्वया नाथ तदैव मरणं वरम्} %2-30-20

\twolineshloka
{इमं हि सहितुं शोकं मुहूर्तमपि नोत्सहे}
{किं पुनर्दश वर्षाणि त्रीणि चैकं च दुःखिता} %2-30-21

\twolineshloka
{इति सा शोकसन्तप्ता विलप्य करुणं बहु}
{चुक्रोश पतिमायस्ता भृशमालिङ्ग्य सस्वरम्} %2-30-22

\twolineshloka
{सा विद्धा बहुभिर्वाक्यैर्दिग्धैरिव गजाङ्गना}
{चिरसन्नियतं बाष्पं मुमोचाग्निमिवारणिः} %2-30-23

\twolineshloka
{तस्याः स्फटिकसङ्काशं वारि सन्तापसम्भवम्}
{नेत्राभ्यां परिसुस्राव पङ्कजाभ्यामिवोदकम्} %2-30-24

\twolineshloka
{तत्सितामलचन्द्राभं मुखमायतलोचनम्}
{पर्यशुष्यत बाष्पेण जलोद्धृतमिवाम्बुजम्} %2-30-25

\twolineshloka
{तां परिष्वज्य बाहुभ्यां विसंज्ञामिव दुःखिताम्}
{उवाच वचनं रामः परिविश्वासयंस्तदा} %2-30-26

\twolineshloka
{न देवि बत दुःखेन स्वर्गमप्यभिरोचये}
{नहि मेऽस्ति भयं किञ्चित् स्वयम्भोरिव सर्वतः} %2-30-27

\twolineshloka
{तव सर्वमभिप्रायमविज्ञाय शुभानने}
{वासं न रोचयेऽरण्ये शक्तिमानपि रक्षणे} %2-30-28

\twolineshloka
{यत् सृष्टासि मया सार्धं वनवासाय मैथिलि}
{न विहातुं मया शक्या प्रीतिरात्मवता यथा} %2-30-29

\twolineshloka
{धर्मस्तु गजनासोरु सद्भिराचरितः पुरा}
{तं चाहमनुवर्तिष्ये यथा सूर्यं सुवर्चला} %2-30-30

\twolineshloka
{न खल्वहं न गच्छेयं वनं जनकनन्दिनि}
{वचनं तन्नयति मां पितुः सत्योपबृंहितम्} %2-30-31

\twolineshloka
{एष धर्मश्च सुश्रोणि पितुर्मातुश्च वश्यता}
{आज्ञां चाहं व्यतिक्रम्य नाहं जीवितुमुत्सहे} %2-30-32

\twolineshloka
{अस्वाधीनं कथं दैवं प्रकारैरभिराध्यते}
{स्वाधीनं समतिक्रम्य मातरं पितरं गुरुम्} %2-30-33

\twolineshloka
{यत्र त्रयं त्रयो लोकाः पवित्रं तत्समं भुवि}
{नान्यदस्ति शुभापाङ्गे तेनेदमभिराध्यते} %2-30-34

\twolineshloka
{न सत्यं दानमानौ वा यज्ञो वाप्याप्तदक्षिणाः}
{तथा बलकराः सीते यथा सेवा पितुर्मता} %2-30-35

\twolineshloka
{स्वर्गो धनं वा धान्यं वा विद्या पुत्राः सुखानि च}
{गुरुवृत्त्यनुरोधेन न किञ्चिदपि दुर्लभम्} %2-30-36

\twolineshloka
{देवगन्धर्वगोलोकान् ब्रह्मलोकांस्तथापरान्}
{प्राप्नुवन्ति महात्मानो मातापितृपरायणाः} %2-30-37

\twolineshloka
{स मा पिता यथा शास्ति सत्यधर्मपथे स्थितः}
{तथा वर्तितुमिच्छामि स हि धर्मः सनातनः} %2-30-38

\twolineshloka
{मम सन्ना मतिः सीते नेतुं त्वां दण्डकावनम्}
{वसिष्यामीति सा त्वं मामनुयातुं सुनिश्चिता} %2-30-39

\twolineshloka
{सा हि दिष्टानवद्याङ्गि वनाय मदिरेक्षणे}
{अनुगच्छस्व मां भीरु सहधर्मचरी भव} %2-30-40

\twolineshloka
{सर्वथा सदृशं सीते मम स्वस्य कुलस्य च}
{व्यवसायमनुक्रान्ता कान्ते त्वमतिशोभनम्} %2-30-41

\twolineshloka
{आरभस्व शुभश्रोणि वनवासक्षमाः क्रियाः}
{नेदानीं त्वदृते सीते स्वर्गोऽपि मम रोचते} %2-30-42

\twolineshloka
{ब्राह्मणेभ्यश्च रत्नानि भिक्षुकेभ्यश्च भोजनम्}
{देहि चाशंसमानेभ्यः सन्त्वरस्व च मा चिरम्} %2-30-43

\twolineshloka
{भूषणानि महार्हाणि वरवस्त्राणि यानि च}
{रमणीयाश्च ये केचित् क्रीडार्थाश्चाप्युपस्कराः} %2-30-44

\twolineshloka
{शयनीयानि यानानि मम चान्यानि यानि च}
{देहि स्वभृत्यवर्गस्य ब्राह्मणानामनन्तरम्} %2-30-45

\twolineshloka
{अनुकूलं तु सा भर्तुर्ज्ञात्वा गमनमात्मनः}
{क्षिप्रं प्रमुदिता देवी दातुमेव प्रचक्रमे} %2-30-46

\twolineshloka
{ततः प्रहृष्टा प्रतिपूर्णमानसा यशस्विनी भर्तुरवेक्ष्य भाषितम्}
{धनानि रत्नानि च दातुमङ्गना प्रचक्रमे धर्मभृतां मनस्विनी} %2-30-47


॥इत्यार्षे श्रीमद्रामायणे वाल्मीकीये आदिकाव्ये अयोध्याकाण्डे वनगमनाभ्युपपत्तिः नाम त्रिंशः सर्गः ॥२-३०॥
