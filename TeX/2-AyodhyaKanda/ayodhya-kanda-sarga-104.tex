\sect{चतुरधिकशततमः सर्गः — रामभरतसंवादः}

\twolineshloka
{तं तु रामः समाज्ञाय भ्रातरं गुरुवत्सलम्}
{लक्ष्मणेन सह भ्रात्रा प्रष्टुं समुपचक्रमे} %2-104-1

\twolineshloka
{किमेतदिच्छेयमहं श्रोतुं प्रव्याहृतं त्वया}
{यस्मात्त्वमागतो देशमिमं चीरजटाजिनी} %2-104-2

\twolineshloka
{किं निमित्तमिमं देशं कृष्णाजिनजटाधरः}
{हित्वा राज्यं प्रविष्टस्त्वं तत्सर्वं वक्तुमर्हसि} %2-104-3

\twolineshloka
{इत्युक्तः कैकयीपुत्रः काकुत्स्थेन महात्मना}
{प्रगृह्य बलवद्भूयः प्राञ्जलिर्वाक्यमब्रवीत्} %2-104-4

\twolineshloka
{आर्य्यं तातः परित्यज्य कृत्वा कर्म सुदुष्करम्}
{गतः स्वर्गं महाबाहुः पुत्रशोकाभिपीडितः} %2-104-5

\twolineshloka
{स्त्रिया नियुक्तः कैकेय्या मम मात्रा परन्तप}
{चकार सुमहत्पापमिदमात्मयशोहरम्} %2-104-6

\twolineshloka
{सा राज्यफलमप्राप्य विधवा शोककर्शिता}
{पतिष्यति महाघोरे निरये जननी मम} %2-104-7

\twolineshloka
{तस्य मे दासभूतस्य प्रसादं कर्त्तुमर्हसि}
{अभिषिञ्चस्व चाद्यैव राज्येनप मघवानिव} %2-104-8

\twolineshloka
{इमाः प्रकृतयः सर्वा विधवा मातरश्च याः}
{त्वत्सकाशमनुप्राप्ताः प्रसादं कर्त्तुमर्हसि} %2-104-9

\twolineshloka
{तदानुपूर्व्या युक्तं च युक्तं चात्मनि मानद}
{राज्यं प्राप्नुहि धर्मेण सकामान् सुहृदः कुरु} %2-104-10

\twolineshloka
{भवत्वविधवा भूमिः समग्रा पतिना त्वया}
{शशिना विमलेनेव शारदी रजनी यथा} %2-104-11

\twolineshloka
{एभिश्च सचिवैः सार्द्धं शिरसा याचितो मया}
{भ्रातुः शिष्यस्य दासस्य प्रसादं कर्त्तुमर्हसि} %2-104-12

\twolineshloka
{तदिदं शाश्वतं पित्र्यं सर्वं प्रकृतिमण्डलम्}
{पूजितं पुरुषव्याघ्र नातिक्रमितुमर्हसि} %2-104-13

\twolineshloka
{एवमुक्त्वा महाबाहुः सबाष्पः कैकयीसुतः}
{रामस्य शिरसा पादौ जग्राह विधिवत्पुनः} %2-104-14

\twolineshloka
{तं मत्तमिव मातङ्गं निःश्वसन्तं पुनःपुनः}
{भ्रातरं भरतं रामः परिष्वज्येदमब्रवीत्} %2-104-15

\twolineshloka
{कुलीनः सत्त्वसम्पन्नस्तेजस्वी चरितव्रतः}
{राज्यहेतोः कथं पापमाचरेत्त्वद्विधो जनः} %2-104-16

\twolineshloka
{न दोषं त्वयि पश्यामि सूक्ष्ममप्यरिसूदन}
{न चापि जननीं बाल्यात्त्वं विगर्हितुमर्हसि} %2-104-17

\twolineshloka
{कामकारो महाप्राज्ञ गुरूणां सर्वदाऽनघ}
{उपपन्नेषु दारेषु पुत्रेषु च विधीयते} %2-104-18

\twolineshloka
{वयमस्य यथा लोके सङ्ख्याताः सौम्य साधुभिः}
{भार्य्याः पुत्राश्च शिष्याश्च त्वमनु ज्ञातुमर्हसि} %2-104-19

\twolineshloka
{वने वा चीरवसनं सौम्य कृष्णाजिनाम्बरम्}
{राज्ये वापि महाराजो मां वासयितुमीश्वरः} %2-104-20

\twolineshloka
{यावत्पितरि धर्मज्ञे गौरवं लोकसत्कृतम्}
{तावद्धर्मभृतां श्रेष्ठ जनन्यामपि गौरवम्} %2-104-21

\twolineshloka
{एताभ्यां धर्मशीलाभ्यां वनं गच्छेति राघव}
{मातापितृभ्यामुक्तोऽहं कथमन्यत् समाचरे} %2-104-22

\twolineshloka
{त्वया राज्यमयोध्यायां प्राप्तव्यं लोकसत्कृतम्}
{वस्तव्यं दण्डकारण्ये मया वल्कलवाससा} %2-104-23

\twolineshloka
{एवं कृत्वा महाराजो विभागं लोकसन्निधौ}
{व्यादिश्य च महातेजा दिवं दशरथो गतः} %2-104-24

\twolineshloka
{स च प्रमाणं धर्मात्मा राजा लोकगुरुस्तव}
{पित्रा दत्तं यथाभागमुपभोक्तुं त्वमर्हसि} %2-104-25

\twolineshloka
{चतुर्दशसमाः सौम्य दण्डकारण्यमाश्रितः}
{उपभोक्ष्ये त्वहं दत्तं भागं पित्रा महात्मना} %2-104-26

\twolineshloka
{यदब्रवीन्मां नरलोकसत्कृतः पिता महात्मा विबुधाधिपोपमः}
{तदेव मन्ये परमात्मनो हितं न सर्वलोकेश्वरभावमप्यहम्} %2-104-27


॥इत्यार्षे श्रीमद्रामायणे वाल्मीकीये आदिकाव्ये अयोध्याकाण्डे रामभरतसंवादः नाम चतुरधिकशततमः सर्गः ॥२-१०४॥
