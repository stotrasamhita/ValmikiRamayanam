\sect{विंशः सर्गः — कौसल्याक्रन्दः}

\twolineshloka
{तस्मिंस्तु पुरुषव्याघ्रे निष्क्रामति कृताञ्जलौ}
{आर्तशब्दो महान् जज्ञे स्त्रीणामन्तःपुरे तदा} %2-20-1

\twolineshloka
{कृत्येष्वचोदितः पित्रा सर्वस्यान्तःपुरस्य च}
{गतिश्च शरणं चासीत् स रामोऽद्य प्रवत्स्यति} %2-20-2

\twolineshloka
{कौसल्यायां यथा युक्तो जनन्यां वर्तते सदा}
{तथैव वर्ततेऽस्मासु जन्मप्रभृति राघवः} %2-20-3

\twolineshloka
{न क्रुध्यत्यभिशप्तोऽपि क्रोधनीयानि वर्जयन्}
{क्रुद्धान् प्रसादयन् सर्वान् स इतोऽद्य प्रवत्स्यति} %2-20-4

\twolineshloka
{अबुद्धिर्बत नो राजा जीवलोकं चरत्ययम्}
{यो गतिं सर्वभूतानां परित्यजति राघवम्} %2-20-5

\twolineshloka
{इति सर्वा महिष्यस्ता विवत्सा इव धेनवः}
{पतिमाचुक्रुशुश्चापि सस्वनं चापि चुक्रुशुः} %2-20-6

\twolineshloka
{स हि चान्तःपुरे घोरमार्तशब्दं महीपतिः}
{पुत्रशोकाभिसंतप्तः श्रुत्वा व्यालीयतासने} %2-20-7

\twolineshloka
{रामस्तु भृशमायस्तो निःश्वसन्निव कुञ्जरः}
{जगाम सहितो भ्रात्रा मातुरन्तःपुरं वशी} %2-20-8

\twolineshloka
{सोऽपश्यत् पुरुषं तत्र वृद्धं परमपूजितम्}
{उपविष्टं गृहद्वारि तिष्ठतश्चापरान् बहून्} %2-20-9

\twolineshloka
{दृष्ट्वैव तु तदा रामं ते सर्वे समुपस्थिताः}
{जयेन जयतां श्रेष्ठं वर्धयन्ति स्म राघवम्} %2-20-10

\twolineshloka
{प्रविश्य प्रथमां कक्ष्यां द्वितीयायां ददर्श सः}
{ब्राह्मणान् वेदसम्पन्नान् वृद्धान् राज्ञाभिसत्कृतान्} %2-20-11

\twolineshloka
{प्रणम्य रामस्तान् वृद्धांस्तृतीयायां ददर्श सः}
{स्त्रियो बालाश्च वृद्धाश्च द्वाररक्षणतत्पराः} %2-20-12

\twolineshloka
{वर्धयित्वा प्रहृष्टास्ताः प्रविश्य च गृहं स्त्रियः}
{न्यवेदयन्त त्वरितं राममातुः प्रियं तदा} %2-20-13

\twolineshloka
{कौसल्यापि तदा देवी रात्रिं स्थित्वा समाहिता}
{प्रभाते चाकरोत् पूजां विष्णोः पुत्रहितैषिणी} %2-20-14

\twolineshloka
{सा क्षौमवसना हृष्टा नित्यं व्रतपरायणा}
{अग्निं जुहोति स्म तदा मन्त्रवत् कृतमङ्गला} %2-20-15

\twolineshloka
{प्रविश्य तु तदा रामो मातुरन्तःपुरं शुभम्}
{ददर्श मातरं तत्र हावयन्तीं हुताशनम्} %2-20-16

\twolineshloka
{देवकार्यनिमित्तं च तत्रापश्यत् समुद्यतम्}
{दध्यक्षतघृतं चैव मोदकान् हविषस्तथा} %2-20-17

\twolineshloka
{लाजान् माल्यानि शुक्लानि पायसं कृसरं तथा}
{समिधः पूर्णकुम्भांश्च ददर्श रघुनन्दनः} %2-20-18

\twolineshloka
{तां शुक्लक्षौमसंवीतां व्रतयोगेन कर्शिताम्}
{तर्पयन्तीं ददर्शाद्भिर्देवतां वरवर्णिनीम्} %2-20-19

\twolineshloka
{सा चिरस्यात्मजं दृष्ट्वा मातृनन्दनमागतम्}
{अभिचक्राम संहृष्टा किशोरं वडवा यथा} %2-20-20

\twolineshloka
{स मातरमुपक्रान्तामुपसंगृह्य राघवः}
{परिष्वक्तश्च बाहुभ्यामवघ्रातश्च मूर्धनि} %2-20-21

\twolineshloka
{तमुवाच दुराधर्षं राघवं सुतमात्मनः}
{कौसल्या पुत्रवात्सल्यादिदं प्रियहितं वचः} %2-20-22

\twolineshloka
{वृद्धानां धर्मशीलानां राजर्षीणां महात्मनाम्}
{प्राप्नुह्यायुश्च कीर्तिं च धर्मं चाप्युचितं कुले} %2-20-23

\twolineshloka
{सत्यप्रतिज्ञं पितरं राजानं पश्य राघव}
{अद्यैव त्वां स धर्मात्मा यौवराज्येऽभिषेक्ष्यति} %2-20-24

\twolineshloka
{दत्तमासनमालभ्य भोजनेन निमन्त्रितः}
{मातरं राघवः किंचित् प्रसार्याञ्जलिमब्रवीत्} %2-20-25

\twolineshloka
{स स्वभावविनीतश्च गौरवाच्च तथानतः}
{प्रस्थितो दण्डकारण्यमाप्रष्टुमुपचक्रमे} %2-20-26

\twolineshloka
{देवि नूनं न जानीषे महद् भयमुपस्थितम्}
{इदं तव च दुःखाय वैदेह्या लक्ष्मणस्य च} %2-20-27

\twolineshloka
{गमिष्ये दण्डकारण्यं किमनेनासनेन मे}
{विष्टरासनयोग्यो हि कालोऽयं मामुपस्थितः} %2-20-28

\twolineshloka
{चतुर्दश हि वर्षाणि वत्स्याम विजने वने}
{कन्दमूलफलैर्जीवन् हित्वा मुनिवदामिषम्} %2-20-29

\twolineshloka
{भरताय महाराजो यौवराज्यं प्रयच्छति}
{मां पुनर्दण्डकारण्यं विवासयति तापसम्} %2-20-30

\twolineshloka
{स षट् चाष्टौ च वर्षाणि वत्स्यामि विजने वने}
{आसेवमानो वन्यानि फलमूलैश्च वर्तयन्} %2-20-31

\twolineshloka
{सा निकृत्तेव सालस्य यष्टिः परशुना वने}
{पपात सहसा देवी देवतेव दिवश्च्युता} %2-20-32

\twolineshloka
{तामदुःखोचितां दृष्ट्वा पतितां कदलीमिव}
{रामस्तूत्थापयामास मातरं गतचेतसम्} %2-20-33

\twolineshloka
{उपावृत्योत्थितां दीनां वडवामिव वाहिताम्}
{पांसुगुण्ठितसर्वाङ्गीं विममर्श च पाणिना} %2-20-34

\twolineshloka
{सा राघवमुपासीनमसुखार्ता सुखोचिता}
{उवाच पुरुषव्याघ्रमुपशृण्वति लक्ष्मणे} %2-20-35

\twolineshloka
{यदि पुत्र न जायेथा मम शोकाय राघव}
{न स्म दुःखमतो भूयः पश्येयमहमप्रजाः} %2-20-36

\twolineshloka
{एक एव हि वन्ध्यायाः शोको भवति मानसः}
{अप्रजास्मीति संतापो न ह्यन्यः पुत्र विद्यते} %2-20-37

\twolineshloka
{न दृष्टपूर्वं कल्याणं सुखं वा पतिपौरुषे}
{अपि पुत्रे विपश्येयमिति रामास्थितं मया} %2-20-38

\twolineshloka
{सा बहून्यमनोज्ञानि वाक्यानि हृदयच्छिदाम्}
{अहं श्रोष्ये सपत्नीनामवराणां परा सती} %2-20-39

\twolineshloka
{अतो दुःखतरं किं नु प्रमदानां भविष्यति}
{मम शोको विलापश्च यादृशोऽयमनन्तकः} %2-20-40

\twolineshloka
{त्वयि संनिहितेऽप्येवमहमासं निराकृता}
{किं पुनः प्रोषिते तात ध्रुवं मरणमेव हि} %2-20-41

\twolineshloka
{अत्यन्तं निगृहीतास्मि भर्तुर्नित्यमसम्मता}
{परिवारेण कैकेय्याः समा वाप्यथवावरा} %2-20-42

\twolineshloka
{यो हि मां सेवते कश्चिदपि वाप्यनुवर्तते}
{कैकेय्याः पुत्रमन्वीक्ष्य स जनो नाभिभाषते} %2-20-43

\twolineshloka
{नित्यक्रोधतया तस्याः कथं नु खरवादि तत्}
{कैकेय्या वदनं द्रष्टुं पुत्र शक्ष्यामि दुर्गता} %2-20-44

\twolineshloka
{दश सप्त च वर्षाणि जातस्य तव राघव}
{अतीतानि प्रकांक्षन्त्या मया दुःखपरिक्षयम्} %2-20-45

\twolineshloka
{तदक्षयं महद्दुःखं नोत्सहे सहितुं चिरात्}
{विप्रकारं सपत्नीनामेवं जीर्णापि राघव} %2-20-46

\twolineshloka
{अपश्यन्ती तव मुखं परिपूर्णशशिप्रभम्}
{कृपणा वर्तयिष्यामि कथं कृपणजीविका} %2-20-47

\twolineshloka
{उपवासैश्च योगैश्च बहुभिश्च परिश्रमैः}
{दुःखसंवर्धितो मोघं त्वं हि दुर्गतया मया} %2-20-48

\twolineshloka
{स्थिरं नु हृदयं मन्ये ममेदं यन्न दीर्यते}
{प्रावृषीव महानद्याः स्पृष्टं कूलं नवाम्भसा} %2-20-49

\twolineshloka
{ममैव नूनं मरणं न विद्यते न चावकाशोऽस्ति यमक्षये मम}
{यदन्तकोऽद्यैव न मां जिहीर्षति प्रसह्य सिंहो रुदतीं मृगीमिव} %2-20-50

\twolineshloka
{स्थिरं हि नूनं हृदयं ममायसं न भिद्यते यद् भुवि नो विदीर्यते}
{अनेन दुःखेन च देहमर्पितं ध्रुवं ह्यकाले मरणं न विद्यते} %2-20-51

\twolineshloka
{इदं तु दुःखं यदनर्थकानि मे व्रतानि दानानि च संयमाश्च हि}
{तपश्च तप्तं यदपत्यकाम्यया सुनिष्फलं बीजमिवोप्तमूषरे} %2-20-52

\twolineshloka
{यदि ह्यकाले मरणं यदृच्छया लभेत कश्चिद् गुरुदुःखकर्शितः}
{गताहमद्यैव परेतसंसदं विना त्वया धेनुरिवात्मजेन वै} %2-20-53

\twolineshloka
{अथापि किं जीवितमद्य मे वृथा त्वया विना चन्द्रनिभाननप्रभ}
{अनुव्रजिष्यामि वनं त्वयैव गौः सुदुर्बला वत्समिवाभिकांक्षया} %2-20-54

\twolineshloka
{भृशमसुखममर्षिता तदा बहु विललाप समीक्ष्य राघवम्}
{व्यसनमुपनिशाम्य सा महत् सुतमिव बद्धमवेक्ष्य किंनरी} %2-20-55


॥इत्यार्षे श्रीमद्रामायणे वाल्मीकीये आदिकाव्ये अयोध्याकाण्डे कौसल्याक्रन्दः नाम विंशः सर्गः ॥२-२०॥
