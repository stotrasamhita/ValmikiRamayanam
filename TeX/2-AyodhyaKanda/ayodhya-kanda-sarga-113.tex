\sect{त्रयोदशाधिकशततमः सर्गः — पादुकाग्रहणम्}

\twolineshloka
{ततः शिरसि कृत्वा तु पादुके भरतस्तदा}
{आरुरोह रथं हृष्टः शत्रुघ्नेन समन्वितः} %2-113-1

\twolineshloka
{वसिष्ऺठो वामदेवश्च जाबालिश्च दृढव्रतः}
{अग्रतः प्रययुः सर्वे मन्त्रिणो मन्त्रपूजिताः} %2-113-2

\twolineshloka
{मन्दाकिनीं नदीं रम्यां प्राङ्मुखास्ते ययुस्तदा}
{प्रदक्षिणं च कुर्वाणाश्चित्रकूटं महागिरिम्} %2-113-3

\twolineshloka
{पश्यन् धातुसहस्राणि रम्याणि विविधानि च}
{प्रययौ तस्य पार्श्वेन ससैन्यो भरतस्तदा} %2-113-4

\twolineshloka
{अदूराच्चित्रकूटस्य ददर्श भरतस्तदा}
{आश्रमं यत्र स मुनिर्भरद्वाजः कृतालयः} %2-113-5

\twolineshloka
{स तमाश्रममागम्य भरद्वाजस्य बुद्धिमान्}
{अवतीर्य रथात् पादौ ववन्दे भरतस्तदा} %2-113-6

\twolineshloka
{ततो हृष्टो भरद्वाजो भरतं वाक्यमब्रवीत्}
{अपि कृत्यं कृतं तात रामेण च समागतम्} %2-113-7

\twolineshloka
{एवमुक्तः स तु ततो भरद्वाजेन धीमता}
{प्रत्युवाच भरद्वाजं भरतो भ्रातृवत्सलः} %2-113-8

\twolineshloka
{स याच्यमानो गुरुणा मया च दृढविक्रमः}
{राघवः परमप्रीतो वसिष्ठं वाक्यमब्रवीत्} %2-113-9

\twolineshloka
{पितुः प्रतिज्ञां तामेव पालयिष्यामि तत्त्वतः}
{चतुर्दश हि वर्षाणि या प्रतिज्ञा पितुर्मम} %2-113-10

\twolineshloka
{एवमुक्तो महाप्राज्ञो वसिष्ठः प्रत्युवाच ह}
{वाक्यज्ञो वाक्यकुशलं राघवं वचनं महत्} %2-113-11

\twolineshloka
{एते प्रयच्छ संहृष्टः पादुके हेमभूषिते}
{अयोध्यायां महाप्राज्ञ योगक्षेमकरे तव} %2-113-12

\twolineshloka
{एवमुक्तो वसिष्ठेन राघवः प्राङ्मुखः स्थितः}
{पादुके अधिरुह्यैते मम राज्याय वै ददौ} %2-113-13

\twolineshloka
{निवृत्तोऽहमनुज्ञातो रामेण सुमहात्मना}
{अयोध्यामेव गच्छामि गृहीत्वा पादुके शुभे} %2-113-14

\twolineshloka
{एतच्छ्रुत्वा शुभं वाक्यं भरतस्य महात्मनः}
{भरद्वाजः शुभतरं मुनिर्वाक्यमुवाच तम्} %2-113-15

\twolineshloka
{नैतच्चित्रं नरव्याघ्र शीलवृत्तवतां वर}
{यदार्यं त्वयि तिष्ठेत्तु निम्ने सृष्टमिवोदकम्} %2-113-16

\twolineshloka
{अमृतः स महाबाहुः पिता दशरथस्तव}
{यस्य त्वमीदृशः पुत्रो धर्मज्ञो धर्मवत्सलः} %2-113-17

\twolineshloka
{तमृषिं तु महात्मानमुक्तवाक्यं कृताञ्जलिः}
{आमतन्त्रयितुमारेभे चरणावुपगृह्य च} %2-113-18

\twolineshloka
{ततः प्रदक्षिणं कृत्वा भरद्वाजं पुनःपुनः}
{भरतस्तु ययौ श्रीमानयोध्यां सह मन्त्रिभिः} %2-113-19

\twolineshloka
{यानैश्च शकटैश्चैव हयैर्नागैश्च सा चमूः}
{पुनर्निवृत्ता विस्तीर्णा भरतस्यानुयायिनी} %2-113-20

\twolineshloka
{ततस्ते यमुनां दिव्यां नदीं तीर्त्वोर्मिमालिनीम्}
{ददृशुस्तां पुनः सर्वे गङ्गां शुभजलां नदीम्} %2-113-21

\twolineshloka
{तां रम्यजलसम्पूर्णां सन्तीर्य सहबान्धवः}
{शृङ्गिबेरपुरं रम्यं प्रविवेश ससैनिकः} %2-113-22

\onelineshloka
{शृङ्गिबेरपुराद्भूयस्त्वयोध्यां संददर्श ह} %2-113-23

\twolineshloka
{अयोध्यां च ततो दृष्ट्वा पित्रा भ्रात्रा विवर्जिताम्}
{भरतो दुःखसन्तप्तः सारथिं चेदमब्रवीत्} %2-113-24

\twolineshloka
{सारथे पश्य विध्वस्ता सायोध्या न प्रकाशते}
{निराकारा निरानन्दा दीना प्रतिहतस्वरा} %2-113-25


॥इत्यार्षे श्रीमद्रामायणे वाल्मीकीये आदिकाव्ये अयोध्याकाण्डे पादुकाग्रहणम् नाम त्रयोदशाधिकशततमः सर्गः ॥२-११३॥
