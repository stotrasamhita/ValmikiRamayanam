\sect{द्व्यशीतितमः सर्गः — सेनाप्रस्थापनम्}

\twolineshloka
{ताम् आर्य गण सम्पूर्णाम् भरतः प्रग्रहाम् सभाम्}
{ददर्श बुद्धि सम्पन्नः पूर्ण चन्द्राम् निशाम् इव} %2-82-1

\twolineshloka
{आसनानि यथा न्यायम् आर्याणाम् विशताम् तदा}
{अदृश्यत घन अपाये पूर्ण चन्द्रा इव शर्वरी} %2-82-2

\twolineshloka
{सा विद्वज्जनसम्पूर्णा सभा सुरुचिरा तदा}
{अदृश्यत घनापाये पूर्णचन्द्रेव शर्वरी} %2-82-3

\twolineshloka
{राज्ञः तु प्रकृतीः सर्वाः समग्राः प्रेक्ष्य धर्मवित्}
{इदम् पुरोहितः वाक्यम् भरतम् मृदु च अब्रवीत्} %2-82-4

\twolineshloka
{तात राजा दशरथः स्वर् गतः धर्मम् आचरन्}
{धन धान्यवतीम् स्फीताम् प्रदाय पृथिवीम् तव} %2-82-5

\twolineshloka
{रामः तथा सत्य धृतिः सताम् धर्मम् अनुस्मरन्}
{न अजहात् पितुर् आदेशम् शशी ज्योत्स्नाम् इव उदितः} %2-82-6

\twolineshloka
{पित्रा भ्रात्रा च ते दत्तम् राज्यम् निहत कण्टकम्}
{तत् भुन्क्ष्व मुदित अमात्यः क्षिप्रम् एव अभिषेचय} %2-82-7

\twolineshloka
{उदीच्याः च प्रतीच्याः च दाक्षिणात्याः च केवलाः}
{कोट्या अपर अन्ताः सामुद्रा रत्नानि अभिहरन्तु ते} %2-82-8

\twolineshloka
{तत् श्रुत्वा भरतः वाक्यम् शोकेन अभिपरिप्लुतः}
{जगाम मनसा रामम् धर्मज्ञो धर्म कान्क्षया} %2-82-9

\twolineshloka
{स बाष्प कलया वाचा कल हम्स स्वरः युवा}
{विललाप सभा मध्ये जगर्हे च पुरोहितम्} %2-82-10

\twolineshloka
{चरित ब्रह्मचर्यस्य विद्या स्नातस्य धीमतः}
{धर्मे प्रयतमानस्य को राज्यम् मद्विधो हरेत्} %2-82-11

\twolineshloka
{कथम् दशरथाज् जातः भवेद् राज्य अपहारकः}
{राज्यम् च अहम् च रामस्य धर्मम् वक्तुम् इह अर्हसि} %2-82-12

\twolineshloka
{ज्येष्ठः श्रेष्ठः च धर्म आत्मा दिलीप नहुष उपमः}
{लब्धुम् अर्हति काकुत्स्थो राज्यम् दशरथो यथा} %2-82-13

\twolineshloka
{अनार्य जुष्टम् अस्वर्ग्यम् कुर्याम् पापम् अहम् यदि}
{इक्ष्वाकूणाम् अहम् लोके भवेयम् कुल पाम्सनः} %2-82-14

\twolineshloka
{यद्द् हि मात्रा कृतम् पापम् न अहम् तत् अभिरोचये}
{इहस्थो वन दुर्गस्थम् नमस्यामि कृत अन्जलिः} %2-82-15

\twolineshloka
{रामम् एव अनुगच्चामि स राजा द्विपदाम् वरः}
{त्रयाणाम् अपि लोकानाम् राघवो राज्यम् अर्हति} %2-82-16

\twolineshloka
{तत् वाक्यम् धर्म सम्युक्तम् श्रुत्वा सर्वे सभासदः}
{हर्षान् मुमुचुर् अश्रूणि रामे निहित चेतसः} %2-82-17

\twolineshloka
{यदि तु आर्यम् न शक्ष्यामि विनिवर्तयितुम् वनात्}
{वने तत्र एव वत्स्यामि यथा आर्यो लक्ष्मणः तथा} %2-82-18

\twolineshloka
{सर्व उपायम् तु वर्तिष्ये विनिवर्तयितुम् बलात्}
{समक्षम् आर्य मिश्राणाम् साधूनाम् गुण वर्तिनाम्} %2-82-19

\twolineshloka
{विष्टिकर्मान्तिकाः सर्वे मार्गशोधनरक्षकाः}
{प्रस्थापिता मया पूर्वम् यात्रापि मम रोचते} %2-82-20

\twolineshloka
{एवम् उक्त्वा तु धर्म आत्मा भरतः भ्रातृ वत्सलः}
{समीपस्थम् उवाच इदम् सुमन्त्रम् मन्त्र कोविदम्} %2-82-21

\twolineshloka
{तूर्णम् उत्थाय गच्च त्वम् सुमन्त्र मम शासनात्}
{यात्राम् आज्ञापय क्षिप्रम् बलम् चैव समानय} %2-82-22

\twolineshloka
{एवम् उक्तः सुमन्त्रः तु भरतेन महात्मना}
{हृष्टः सो अदिशत् सर्वम् यथा सम्दिष्टम् इष्टवत्} %2-82-23

\twolineshloka
{ताः प्रहृष्टाः प्रकृतयो बल अध्यक्षा बलस्य च}
{श्रुत्वा यात्राम् समाज्ञप्ताम् राघवस्य निवर्तने} %2-82-24

\twolineshloka
{ततः योध अन्गनाः सर्वा भर्तृऋन् सर्वान् गृहे गृहे}
{यात्रा गमनम् आज्ञाय त्वरयन्ति स्म हर्षिताः} %2-82-25

\twolineshloka
{ते हयैः गो रथैः शीघ्रैः स्यन्दनैः च मनो जवैः}
{सह योधैः बल अध्यक्षा बलम् सर्वम् अचोदयन्} %2-82-26

\twolineshloka
{सज्जम् तु तत् बलम् दृष्ट्वा भरतः गुरु सम्निधौ}
{रथम् मे त्वरयस्व इति सुमन्त्रम् पार्श्वतः अब्रवीत्} %2-82-27

\twolineshloka
{भरतस्य तु तस्य आज्ञाम् प्रतिगृह्य प्रहर्षितः}
{रथम् गृहीत्वा प्रययौ युक्तम् परम वाजिभिः} %2-82-28

\fourlineindentedshloka
{स राघवः सत्य धृतिः प्रतापवान्}
{ब्रुवन् सुयुक्तम् दृढ सत्य विक्रमः}
{गुरुम् महा अरण्य गतम् यशस्विनम्}
{प्रसादयिष्यन् भरतः अब्रवीत् तदा} %2-82-29

\fourlineindentedshloka
{तूण समुत्थाय सुमन्त्र गच्च}
{बलस्य योगाय बल प्रधानान्}
{आनेतुम् इच्चामि हि तम् वनस्थम्}
{प्रसाद्य रामम् जगतः हिताय} %2-82-30

\fourlineindentedshloka
{स सूत पुत्रः भरतेन सम्यग्}
{आज्ञापितः सम्परिपूर्ण कामः}
{शशास सर्वान् प्रकृति प्रधानान्}
{बलस्य मुख्यामः च सुहृज् जनम् च} %2-82-31

\fourlineindentedshloka
{ततः समुत्थाय कुले कुले ते}
{राजन्य वैश्या वृषलाः च विप्राः}
{अयूयुजन्न् उष्ट्र रथान् खरामः च}
{नागान् हयामः चैव कुल प्रसूतान्} %2-82-32


॥इत्यार्षे श्रीमद्रामायणे वाल्मीकीये आदिकाव्ये अयोध्याकाण्डे सेनाप्रस्थापनम् नाम द्व्यशीतितमः सर्गः ॥२-८२॥
