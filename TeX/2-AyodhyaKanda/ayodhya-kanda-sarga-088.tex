\sect{अष्टाशीतितमः सर्गः — शय्यानुवीक्षणम्}

\twolineshloka
{तत् श्रुत्वा निपुणम् सर्वम् भरतः सह मन्त्रिभिः}
{इन्गुदी मूलम् आगम्य राम शय्याम् अवेक्ष्य ताम्} %2-88-1

\twolineshloka
{अब्रवीद् जननीः सर्वा इह तेन महात्मना}
{शर्वरी शयिता भूमाउ इदम् अस्य विमर्दितम्} %2-88-2

\twolineshloka
{महा भाग कुलीनेन महा भागेन धीमता}
{जातो दशरथेन ऊर्व्याम् न रामः स्वप्तुम् अर्हति} %2-88-3

\twolineshloka
{अजिन उत्तर सम्स्तीर्णे वर आस्तरण सम्चये}
{शयित्वा पुरुष व्याघ्रः कथम् शेते मही तले} %2-88-4

\twolineshloka
{प्रासाद अग्र विमानेषु वलभीषु च सर्वदा}
{हैम राजत भौमेषु वर आस्त्ररण शालिषु} %2-88-5

\twolineshloka
{पुष्प सम्चय चित्रेषु चन्दन अगरु गन्धिषु}
{पाण्डुर अभ्र प्रकाशेषु शुक सम्घ रुतेषु च} %2-88-6

\twolineshloka
{प्रासादवरवर्येषु शीतवत्सु सुगन्धिषु}
{उषित्वा मेरुकल्पेषु कृतकाम्चनभित्तिषु} %2-88-7

\twolineshloka
{गीत वादित्र निर्घोषैर् वर आभरण निह्स्वनैः}
{मृदन्ग वर शब्दैः च सततम् प्रतिबोधितः} %2-88-8

\twolineshloka
{बन्दिभिर् वन्दितः काले बहुभिः सूत मागधैः}
{गाथाभिर् अनुरूपाभिः स्तुतिभिः च परम्तपः} %2-88-9

\twolineshloka
{अश्रद्धेयम् इदम् लोके न सत्यम् प्रतिभाति मा}
{मुह्यते खलु मे भावः स्वप्नो अयम् इति मे मतिः} %2-88-10

\twolineshloka
{न नूनम् दैवतम् किम्चित् कालेन बलवत्तरम्}
{यत्र दाशरथी रामो भूमाउ एवम् शयीत सः} %2-88-11

\twolineshloka
{विदेह राजस्य सुता सीता च प्रिय दर्शना}
{दयिता शयिता भूमौ स्नुषा दशरथस्य च} %2-88-12

\twolineshloka
{इयम् शय्या मम भ्रातुर् इदम् हि परिवर्तितम्}
{स्थण्डिले कठिने सर्वम् गात्रैर् विमृदितम् तृणम्} %2-88-13

\twolineshloka
{मन्ये साभरणा सुप्ता सीता अस्मिन् शयने तदा}
{तत्र तत्र हि दृश्यन्ते सक्ताः कनक बिन्दवः} %2-88-14

\twolineshloka
{उत्तरीयम् इह आसक्तम् सुव्यक्तम् सीतया तदा}
{तथा ह्य् एते प्रकाशन्ते सक्ताः कौशेय तन्तवः} %2-88-15

\twolineshloka
{मन्ये भर्तुः सुखा शय्या येन बाला तपस्विनी}
{सुकुमारी सती दुह्खम् न विजानाति मैथिली} %2-88-16

\twolineshloka
{हा हन्तास्मि नृशम्सोऽहम् यत्सभार्यः कृतेमम}
{ईदृशीं राघवः शय्यामधिशेते ह्यानाथवत्} %2-88-17

\twolineshloka
{सार्वभौम कुले जातः सर्व लोक सुख आवहः}
{सर्व लोक प्रियः त्यक्त्वा राज्यम् प्रियम् अनुत्तमम्} %2-88-18

\twolineshloka
{कथम् इन्दीवर श्यामो रक्त अक्षः प्रिय दर्शनः}
{सुख भागी च दुह्ख अर्हः शयितो भुवि राघवः} %2-88-19

\twolineshloka
{धन्यः खलु महाभागो लक्ष्मणः शुभलक्षमणः}
{भ्रातरम् विषमे काले यो राममनुवर्तते} %2-88-20

\twolineshloka
{सिद्ध अर्था खलु वैदेही पतिम् या अनुगता वनम्}
{वयम् सम्शयिताः सर्वे हीनाः तेन महात्मना} %2-88-21

\twolineshloka
{अकर्ण धारा पृथिवी शून्या इव प्रतिभाति मा}
{गते दशरथे स्वर्गे रामे च अरण्यम् आश्रिते} %2-88-22

\twolineshloka
{न च प्रार्थयते कश्चिन् मनसा अपि वसुम्धराम्}
{वने अपि वसतः तस्य बाहु वीर्य अभिरक्षिताम्} %2-88-23

\twolineshloka
{शून्य सम्वरणा रक्षाम् अयन्त्रित हय द्विपाम्}
{अपावृत पुर द्वाराम् राज धानीम् अरक्षिताम्} %2-88-24

\twolineshloka
{अप्रहृष्ट बलाम् न्यूनाम् विषमस्थाम् अनावृताम्}
{शत्रवो न अभिमन्यन्ते भक्ष्यान् विष कृतान् इव} %2-88-25

\twolineshloka
{अद्य प्रभृति भूमौ तु शयिष्ये अहम् तृणेषु वा}
{फल मूल अशनो नित्यम् जटा चीराणि धारयन्} %2-88-26

\twolineshloka
{तस्य अर्थम् उत्तरम् कालम् निवत्स्यामि सुखम् वने}
{तम् प्रतिश्रवम् आमुच्य न अस्य मिथ्या भविष्यति} %2-88-27

\twolineshloka
{वसन्तम् भ्रातुर् अर्थाय शत्रुघ्नो मा अनुवत्स्यति}
{लक्ष्मणेन सह तु आर्यो अयोध्याम् पालयिष्यति} %2-88-28

\fourlineindentedshloka
{अभिषेक्ष्यन्ति काकुत्स्थम् अयोध्यायाम् द्विजातयः}
{अपि मे देवताः कुर्युर् इमम् सत्यम् मनो रथम्}
{प्रसाद्यमानः शिरसा मया स्वयम्}
{बहु प्रकारम् यदि न प्रपत्स्यते} %2-88-29

\twolineshloka
{ततोन्रुवत्सयामि चिराय राघवम्}
{वनेचरम् नह्रुति माम्रुपेक्षित्रुम्} %2-88-30


॥इत्यार्षे श्रीमद्रामायणे वाल्मीकीये आदिकाव्ये अयोध्याकाण्डे शय्यानुवीक्षणम् नाम अष्टाशीतितमः सर्गः ॥२-८८॥
