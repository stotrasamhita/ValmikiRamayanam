\sect{एकोनत्रिंशः सर्गः — वनानुगमनयाच्ञानिर्बन्धः}

\twolineshloka
{एतत् तु वचनं श्रुत्वा सीता रामस्य दुःखिता}
{प्रसक्ताश्रुमुखी मन्दमिदं वचनमब्रवीत्} %2-29-1

\twolineshloka
{ये त्वया कीर्तिता दोषा वने वस्तव्यतां प्रति}
{गुणानित्येव तान् विद्धि तव स्नेहपुरस्कृता} %2-29-2

\twolineshloka
{मृगाः सिंहा गजाश्चैव शार्दूलाः शरभास्तथा}
{चमराः सृमराश्चैव ये चान्ये वनचारिणः} %2-29-3

\twolineshloka
{अदृष्टपूर्वरूपत्वात् सर्वे ते तव राघव}
{रूपं दृष्ट्वापसर्पेयुस्तव सर्वे हि बिभ्यति} %2-29-4

\twolineshloka
{त्वया च सह गन्तव्यं मया गुरुजनाज्ञया}
{त्वद्वियोगेन मे राम त्यक्तव्यमिह जीवितम्} %2-29-5

\twolineshloka
{नहि मां त्वत्समीपस्थामपि शक्रोऽपि राघव}
{सुराणामीश्वरः शक्तः प्रधर्षयितुमोजसा} %2-29-6

\twolineshloka
{पतिहीना तु या नारी न सा शक्ष्यति जीवितुम्}
{काममेवंविधं राम त्वया मम निदर्शितम्} %2-29-7

\twolineshloka
{अथापि च महाप्राज्ञ ब्राह्मणानां मया श्रुतम्}
{पुरा पितृगृहे सत्यं वस्तव्यं किल मे वने} %2-29-8

\twolineshloka
{लक्षणिभ्यो द्विजातिभ्यः श्रुत्वाहं वचनं गृहे}
{वनवासकृतोत्साहा नित्यमेव महाबल} %2-29-9

\twolineshloka
{आदेशो वनवासस्य प्राप्तव्यः स मया किल}
{सा त्वया सह भर्त्राहं यास्यामि प्रिय नान्यथा} %2-29-10

\twolineshloka
{कृतादेशा भविष्यामि गमिष्यामि त्वया सह}
{कालश्चायं समुत्पन्नः सत्यवान् भवतु द्विजः} %2-29-11

\twolineshloka
{वनवासे हि जानामि दुःखानि बहुधा किल}
{प्राप्यन्ते नियतं वीर पुरुषैरकृतात्मभिः} %2-29-12

\twolineshloka
{कन्यया च पितुर्गेहे वनवासः श्रुतो मया}
{भिक्षिण्याः शमवृत्ताया मम मातुरिहाग्रतः} %2-29-13

\twolineshloka
{प्रसादितश्च वै पूर्वं त्वं मे बहुतिथं प्रभो}
{गमनं वनवासस्य कांक्षितं हि सह त्वया} %2-29-14

\twolineshloka
{कृतक्षणाहं भद्रं ते गमनं प्रति राघव}
{वनवासस्य शूरस्य मम चर्या हि रोचते} %2-29-15

\twolineshloka
{शुद्धात्मन् प्रेमभावाद्धि भविष्यामि विकल्मषा}
{भर्तारमनुगच्छन्ती भर्ता हि परदैवतम्} %2-29-16

\twolineshloka
{प्रेत्यभावे हि कल्याणः संगमो मे सदा त्वया}
{श्रुतिर्हि श्रूयते पुण्या ब्राह्मणानां यशस्विनाम्} %2-29-17

\twolineshloka
{इहलोके च पितृभिर्या स्त्री यस्य महाबल}
{अद्भिर्दत्ता स्वधर्मेण प्रेत्यभावेऽपि तस्य सा} %2-29-18

\twolineshloka
{एवमस्मात् स्वकां नारीं सुवृत्तां हि पतिव्रताम्}
{नाभिरोचयसे नेतुं त्वं मां केनेह हेतुना} %2-29-19

\twolineshloka
{भक्तां पतिव्रतां दीनां मां समां सुखदुःखयोः}
{नेतुमर्हसि काकुत्स्थ समानसुखदुःखिनीम्} %2-29-20

\twolineshloka
{यदि मां दुःखितामेवं वनं नेतुं न चेच्छसि}
{विषमग्निं जलं वाहमास्थास्ये मृत्युकारणात्} %2-29-21

\twolineshloka
{एवं बहुविधं तं सा याचते गमनं प्रति}
{नानुमेने महाबाहुस्तां नेतुं विजनं वनम्} %2-29-22

\twolineshloka
{एवमुक्ता तु सा चिन्तां मैथिली समुपागता}
{स्नापयन्तीव गामुष्णैरश्रुभिर्नयनच्युतैः} %2-29-23

\twolineshloka
{चिन्तयन्तीं तदा तां तु निवर्तयितुमात्मवान्}
{क्रोधाविष्टां तु वैदेहीं काकुत्स्थो बह्वसान्त्वयत्} %2-29-24


॥इत्यार्षे श्रीमद्रामायणे वाल्मीकीये आदिकाव्ये अयोध्याकाण्डे वनानुगमनयाच्ञानिर्बन्धः नाम एकोनत्रिंशः सर्गः ॥२-२९॥
