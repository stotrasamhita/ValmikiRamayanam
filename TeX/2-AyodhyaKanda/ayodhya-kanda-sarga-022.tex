\sect{द्वाविंशः सर्गः — दैवप्राबल्यम्}

\twolineshloka
{अथ तं व्यथया दीनं सविशेषममर्षितम्}
{सरोषमिव नागेन्द्रं रोषविस्फारितेक्षणम्} %2-22-1

\twolineshloka
{आसाद्य रामः सौमित्रिं सुहृदं भ्रातरं प्रियम्}
{उवाचेदं स धैर्येण धारयन् सत्त्वमात्मवान्} %2-22-2

\twolineshloka
{निगृह्य रोषं शोकं च धैर्यमाश्रित्य केवलम्}
{अवमानं निरस्यैनं गृहीत्वा हर्षमुत्तमम्} %2-22-3

\twolineshloka
{उपक्लृप्तं यदैतन्मे अभिषेकार्थमुत्तमम्}
{सर्वं निवर्तय क्षिप्रं कुरु कार्यं निरव्ययम्} %2-22-4

\twolineshloka
{सौमित्रे योऽभिषेकार्थे मम सम्भारसम्भ्रमः}
{अभिषेकनिवृत्त्यर्थे सोऽस्तु सम्भारसम्भ्रमः} %2-22-5

\twolineshloka
{यस्या मदभिषेकार्थे मानसं परितप्यते}
{माता नः सा यथा न स्यात् सविशङ्का तथा कुरु} %2-22-6

\twolineshloka
{तस्याः शङ्कामयं दुःखं मुहूर्तमपि नोत्सहे}
{मनसि प्रतिसञ्जातं सौमित्रेऽहमुपेक्षितुम्} %2-22-7

\twolineshloka
{न बुद्धिपूर्वं नाबुद्धं स्मरामीह कदाचन}
{मातॄणां वा पितुर्वाहं कृतमल्पं च विप्रियम्} %2-22-8

\twolineshloka
{सत्यः सत्याभिसन्धश्च नित्यं सत्यपराक्रमः}
{परलोकभयाद् भीतो निर्भयोऽस्तु पिता मम} %2-22-9

\twolineshloka
{तस्यापि हि भवेदस्मिन् कर्मण्यप्रतिसंहृते}
{सत्यं नेति मनस्तापस्तस्य तापस्तपेच्च माम्} %2-22-10

\twolineshloka
{अभिषेकविधानं तु तस्मात् संहृत्य लक्ष्मण}
{अन्वगेवाहमिच्छामि वनं गन्तुमितः पुरः} %2-22-11

\twolineshloka
{मम प्रव्राजनादद्य कृतकृत्या नृपात्मजा}
{सुतं भरतमव्यग्रमभिषेचयतां ततः} %2-22-12

\twolineshloka
{मयि चीराजिनधरे जटामण्डलधारिणि}
{गतेऽरण्यं च कैकेय्या भविष्यति मनः सुखम्} %2-22-13

\twolineshloka
{बुद्धिः प्रणीता येनेयं मनश्च सुसमाहितम्}
{तं नु नार्हामि सङ्क्लेष्टुं प्रव्रजिष्यामि मा चिरम्} %2-22-14

\twolineshloka
{कृतान्त एव सौमित्रे द्रष्टव्यो मत्प्रवासने}
{राज्यस्य च वितीर्णस्य पुनरेव निवर्तने} %2-22-15

\twolineshloka
{कैकेय्याः प्रतिपत्तिर्हि कथं स्यान्मम वेदने}
{यदि तस्या न भावोऽयं कृतान्तविहितो भवेत्} %2-22-16

\twolineshloka
{जानासि हि यथा सौम्य न मातृषु ममान्तरम्}
{भूतपूर्वं विशेषो वा तस्या मयि सुतेऽपि वा} %2-22-17

\twolineshloka
{सोऽभिषेकनिवृत्त्यर्थैः प्रवासार्थैश्च दुर्वचैः}
{उग्रैर्वाक्यैरहं तस्या नान्यद् दैवात् समर्थये} %2-22-18

\twolineshloka
{कथं प्रकृतिसम्पन्ना राजपुत्री तथागुणा}
{ब्रूयात् सा प्राकृतेव स्त्री मत्पीड्यं भर्तृसन्निधौ} %2-22-19

\twolineshloka
{यदचिन्त्यं तु तद् दैवं भूतेष्वपि न हन्यते}
{व्यक्तं मयि च तस्यां च पतितो हि विपर्ययः} %2-22-20

\twolineshloka
{कश्च दैवेन सौमित्रे योद्धुमुत्सहते पुमान्}
{यस्य नु ग्रहणं किञ्चित् कर्मणोऽन्यन्न दृश्यते} %2-22-21

\twolineshloka
{सुखदुःखे भयक्रोधौ लाभालाभौ भवाभवौ}
{यस्य किञ्चित् तथाभूतं ननु दैवस्य कर्म तत्} %2-22-22

\twolineshloka
{ऋषयोऽप्युग्रतपसो दैवेनाभिप्रचोदिताः}
{उत्सृज्य नियमांस्तीव्रान् भ्रश्यन्ते काममन्युभिः} %2-22-23

\twolineshloka
{असङ्कल्पितमेवेह यदकस्मात् प्रवर्तते}
{निवर्त्यारब्धमारम्भैर्ननु दैवस्य कर्म तत्} %2-22-24

\twolineshloka
{एतया तत्त्वया बुद्ध्या संस्तभ्यात्मानमात्मना}
{व्याहतेऽप्यभिषेके मे परितापो न विद्यते} %2-22-25

\twolineshloka
{तस्मादपरितापः संस्त्वमप्यनुविधाय माम्}
{प्रतिसंहारय क्षिप्रमाभिषेचनिकीं क्रियाम्} %2-22-26

\twolineshloka
{एभिरेव घटैः सर्वैरभिषेचनसम्भृतैः}
{मम लक्ष्मण तापस्ये व्रतस्नानं भविष्यति} %2-22-27

\twolineshloka
{अथवा किं मयैतेन राज्यद्रव्यमयेन तु}
{उद्धृतं मे स्वयं तोयं व्रतादेशं करिष्यति} %2-22-28

\twolineshloka
{मा च लक्ष्मण सन्तापं कार्षीर्लक्ष्म्या विपर्यये}
{राज्यं वा वनवासो वा वनवासो महोदयः} %2-22-29

\twolineshloka
{न लक्ष्मणास्मिन् मम राज्यविघ्ने माता यवीयस्यभिशङ्कितव्या}
{दैवाभिपन्ना न पिता कथञ्चिज्जानासि दैवं हि तथाप्रभावम्} %2-22-30


॥इत्यार्षे श्रीमद्रामायणे वाल्मीकीये आदिकाव्ये अयोध्याकाण्डे दैवप्राबल्यम् नाम द्वाविंशः सर्गः ॥२-२२॥
