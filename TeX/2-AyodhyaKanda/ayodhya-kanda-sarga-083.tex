\sect{त्र्यशीतितमः सर्गः — भरतवनप्रस्थानम्}

\twolineshloka
{ततः समुत्थितः काल्यम् आस्थाय स्यन्दन उत्तमम्}
{प्रययौ भरतः शीघ्रम् राम दर्शन कान्क्षया} %2-83-1

\twolineshloka
{अग्रतः प्रययुस् तस्य सर्वे मन्त्रि पुरोधसः}
{अधिरुह्य हयैः युक्तान् रथान् सूर्य रथ उपमान्} %2-83-2

\twolineshloka
{नव नाग सहस्राणि कल्पितानि यथा विधि}
{अन्वयुर् भरतम् यान्तम् इक्ष्वाकु कुल नन्दनम्} %2-83-3

\twolineshloka
{षष्ठी रथ सहस्राणि धन्विनो विविध आयुधाः}
{अन्वयुर् भरतम् यान्तम् राज पुत्रम् यशस्विनम्} %2-83-4

\twolineshloka
{शतम् सहस्राणि अश्वानाम् समारूढानि राघवम्}
{अन्वयुर् भरतम् यान्तम् राज पुत्रम् यशस्विनम्} %2-83-5

\twolineshloka
{कैकेयी च सुमित्रा च कौसल्या च यशस्विनी}
{राम आनयन सम्हृष्टा ययुर् यानेन भास्वता} %2-83-6

\twolineshloka
{प्रयाताः च आर्य सम्घाता रामम् द्रष्टुम् सलक्ष्मणम्}
{तस्य एव च कथाः चित्राः कुर्वाणा हृष्ट मानसाः} %2-83-7

\twolineshloka
{मेघ श्यामम् महा बाहुम् स्थिर सत्त्वम् दृढ व्रतम्}
{कदा द्रक्ष्यामहे रामम् जगतः शोक नाशनम्} %2-83-8

\twolineshloka
{दृष्टएव हि नः शोकम् अपनेष्यति राघवः}
{तमः सर्वस्य लोकस्य समुद्यन्न् इव भास्करः} %2-83-9

\twolineshloka
{इति एवम् कथयन्तः ते सम्प्रहृष्टाः कथाः शुभाः}
{परिष्वजानाः च अन्योन्यम् ययुर् नागरिकाः तदा} %2-83-10

\twolineshloka
{ये च तत्र अपरे सर्वे सम्मता ये च नैगमाः}
{रामम् प्रति ययुर् हृष्टाः सर्वाः प्रकृतयः तदा} %2-83-11

\twolineshloka
{मणि काराः च ये केचित् कुम्भ काराः च शोभनाः}
{सूत्र कर्म कृतः चैव ये च शस्त्र उपजीविनः} %2-83-12

\twolineshloka
{मायूरकाः क्राकचिका रोचका वेधकाः तथा}
{दन्त काराः सुधा काराः तथा गन्ध उपजीविनः} %2-83-13

\twolineshloka
{सुवर्ण काराः प्रख्याताः तथा कम्बल धावकाः}
{स्नापक आच्चादका वैद्या धूपकाः शौण्डिकाः तथा} %2-83-14

\twolineshloka
{रजकाः तुन्न वायाः च ग्राम घोष महत्तराः}
{शैलूषाः च सह स्त्रीभिर् यान्ति कैवर्तकाः तथा} %2-83-15

\twolineshloka
{समाहिता वेदविदो ब्राह्मणा वृत्त सम्मताः}
{गो रथैः भरतम् यान्तम् अनुजग्मुः सहस्रशः} %2-83-16

\twolineshloka
{सुवेषाः शुद्ध वसनाः ताम्र मृष्ट अनुलेपनाः}
{सर्वे ते विविधैः यानैः शनैः भरतम् अन्वयुः} %2-83-17

\twolineshloka
{प्रहृष्ट मुदिता सेना सान्वयात् कैकयी सुतम्}
{भ्रातुरानयने यान्तम् भरतम् भ्रातृवत्सलम्} %2-83-18

\twolineshloka
{ते गत्वा दूरमध्वानम् रथम् यानाश्वकुञ्जरैः}
{समासेदुस्ततो गङ्गाम् शृङ्गिबेरपुरम् प्रति} %2-83-19

\twolineshloka
{यत्र रामसखो वीरो गुहो ज्ञातिगणैर्वृतः}
{निवसत्यप्रमादेन देशम् तम् परिपालयन्} %2-83-20

\twolineshloka
{उपेत्य तीरम् गङ्गायाश्चक्रमाकैरलङ्कतम्}
{व्यवतिष्ठत सा सेना भरतस्य अनुयायिनी} %2-83-21

\twolineshloka
{निरीक्ष्य अनुगताम् सेनाम् ताम् च गन्गाम् शिव उदकाम्}
{भरतः सचिवान् सर्वान् अब्रवीद् वाक्य कोविदः} %2-83-22

\twolineshloka
{निवेशयत मे सैन्यम् अभिप्रायेण सर्वशः}
{विश्रान्तः प्रतरिष्यामः श्वैदानीम् महा नदीम्} %2-83-23

\twolineshloka
{दातुम् च तावद् इच्चामि स्वर् गतस्य मही पतेः}
{और्ध्वदेह निमित्त अर्थम् अवतीर्य उदकम् नदीम्} %2-83-24

\twolineshloka
{तस्य एवम् ब्रुवतः अमात्याः तथा इति उक्त्वा समाहिताः}
{न्यवेशयम्स् तामः चन्देन स्वेन स्वेन पृथक् पृथक्} %2-83-25

\fourlineindentedshloka
{निवेश्य गन्गाम् अनु ताम् महा नदीम्}
{चमूम् विधानैः परिबर्ह शोभिनीम्}
{उवास रामस्य तदा महात्मनो}
{विचिन्तयानो भरतः निवर्तनम्} %2-83-26


॥इत्यार्षे श्रीमद्रामायणे वाल्मीकीये आदिकाव्ये अयोध्याकाण्डे भरतवनप्रस्थानम् नाम त्र्यशीतितमः सर्गः ॥२-८३॥
