\sect{षट्पञ्चाशः सर्गः — चित्रकूटनिवासः}

\twolineshloka
{अथ रात्र्याम् व्यतीतायाम् अवसुप्तम् अनन्तरम्}
{प्रबोधयाम् आस शनैः लक्ष्मणम् रघु नन्दनः} %2-56-1

\twolineshloka
{सौमित्रे शृणु वन्यानाम् वल्गु व्याहरताम् स्वनम्}
{सम्प्रतिष्ठामहे कालः प्रस्थानस्य परम् तप} %2-56-2

\twolineshloka
{स सुप्तः समये भ्रात्रा लक्ष्मणः प्रतिबोधितः}
{जहौ निद्राम् च तन्द्रीम् च प्रसक्तम् च पथि श्रमम्} %2-56-3

\twolineshloka
{ततौत्थाय ते सर्वे स्पृष्ट्वा नद्याः शिवम् जलम्}
{पन्थानम् ऋषिणा उद्दिष्टम् चित्र कूटस्य तम् ययुः} %2-56-4

\twolineshloka
{ततः सम्प्रस्थितः काले रामः सौमित्रिणा सह}
{सीताम् कमल पत्र अक्षीम् इदम् वचनम् अब्रवीत्} %2-56-5

\twolineshloka
{आदीप्तान् इव वैदेहि सर्वतः पुष्पितान् नगान्}
{स्वैः पुष्पैः किम्शुकान् पश्य मालिनः शिशिर अत्यये} %2-56-6

\twolineshloka
{पश्य भल्लातकान् फुल्लान् नरैः अनुपसेवितान्}
{फल पत्रैः अवनतान् नूनम् शक्ष्यामि जीवितुम्} %2-56-7

\twolineshloka
{पश्य द्रोण प्रमाणानि लम्बमानानि लक्ष्मण}
{मधूनि मधु कारीभिः सम्भृतानि नगे नगे} %2-56-8

\twolineshloka
{एष क्रोशति नत्यूहः तम् शिखी प्रतिकूजति}
{रमणीये वन उद्देशे पुष्प सम्स्तर सम्कटे} %2-56-9

\twolineshloka
{मातम्ग यूथ अनुसृतम् पक्षि सम्घ अनुनादितम्}
{चित्र कूटम् इमम् पश्य प्रवृद्ध शिखरम् गिरिम्} %2-56-10

\twolineshloka
{समभूमितले रम्ये द्रुमैर्बहुभिरावृते}
{पुण्ये रम्स्यामहे तात चित्रकूटस्य कानने} %2-56-11

\twolineshloka
{ततः तौ पाद चारेण गच्चन्तौ सह सीतया}
{रम्यम् आसेदतुः शैलम् चित्र कूटम् मनो रमम्} %2-56-12

\twolineshloka
{तम् तु पर्वतम् आसाद्य नाना पक्षि गण आयुतम्}
{बहुमूलफलम् रम्यम् सम्पन्नम् सरसोदकम्} %2-56-13

\twolineshloka
{मनोज्नोऽयम् तिरिः सौम्य नानाद्रुमलतायतह्}
{बहुमूलफलो रम्यः स्वाजीवः प्रतिभाति मे} %2-56-14

\twolineshloka
{मनयश्च महात्मानो वसन्त्य शिलोच्चये}
{अयम् वासो भवेत् तावद् अत्र सौम्य रमेमहि} %2-56-15

\twolineshloka
{इति सीता च रामश्च लक्ष्मणश्च कृताञ्जलिः}
{अभिगम्याश्रमम् सर्वे वाल्मीकि मभिवादयन्} %2-56-16

\twolineshloka
{तान्महर्षिः प्रमुदितः पूजयामास धर्मवित्}
{आस्यतामिति चोवाच स्वागतम् तु निवेद्य च} %2-56-17

\twolineshloka
{ततोऽब्रवीन्महाबाहुर्लकमणम् लक्ष्मणाग्रजः}
{सम्निवेद्य यथान्याय मात्मानमृष्ये प्रभुः} %2-56-18

\twolineshloka
{लक्ष्मण आनय दारूणि दृढानि च वराणि च}
{कुरुष्व आवसथम् सौम्य वासे मे अभिरतम् मनः} %2-56-19

\twolineshloka
{तस्य तत् वचनम् श्रुत्वा सौमित्रिर् विविधान् द्रुमान्}
{आजहार ततः चक्रे पर्ण शालाम् अरिम् दम} %2-56-20

\twolineshloka
{ताम् निष्ठताम् बद्धकटाम् दृष्ट्वा रमः सुदर्शनाम्}
{शुश्रूषमाणम् एक अग्रम् इदम् वचनम् अब्रवीत्} %2-56-21

\twolineshloka
{ऐणेयम् माम्सम् आहृत्य शालाम् यक्ष्यामहे वयम्}
{कर्त्व्यम् वास्तुशमनम् सौमित्रे चिरजीवभिः} %2-56-22

\onelineshloka
{मृगम् हत्वाऽऽनय क्षिप्रम् लक्ष्मणेह शुभेक्षणकर्तव्यः शास्त्रदृष्टो हि विधिर्दर्ममनुस्मर} %2-56-23

\twolineshloka
{भ्रातुर्वचन माज्ञाय लक्ष्मणः परवीरहा}
{चकार स यथोक्तम् च तम् रामः पुनरब्रवीत्} %2-56-24

\twolineshloka
{इणेयम् श्रपयस्वैतच्च्चालाम् यक्ष्यमहे वयम्}
{त्वरसौम्य मुहूर्तोऽयम् ध्रुवश्च दिवसोऽप्ययम्} %2-56-25

\twolineshloka
{स लक्ष्मणः कृष्ण मृगम् हत्वा मेध्यम् पतापवान्}
{अथ चिक्षेप सौमित्रिः समिद्धे जात वेदसि} %2-56-26

\twolineshloka
{तम् तु पक्वम् समाज्ञाय निष्टप्तम् चिन्न शोणितम्}
{लक्ष्मणः पुरुष व्याघ्रम् अथ राघवम् अब्रवीत्} %2-56-27

\twolineshloka
{अयम् कृष्णः समाप्त अन्गः शृतः कृष्ण मृगो यथा}
{देवता देव सम्काश यजस्व कुशलो हि असि} %2-56-28

\twolineshloka
{रामः स्नात्वा तु नियतः गुणवान् जप्य कोविदः}
{सम्ग्रहेणाकरोत्सर्वान् मन्त्रान् सत्रावसानिकान्} %2-56-29

\twolineshloka
{इष्ट्वा देवगणान् सर्वान् विवेशावसथम् शुचिः}
{बभूव च मनोह्लादो रामस्यामिततेजसः} %2-56-30

\twolineshloka
{वैश्वदेवबलिम् कृत्वा रौद्रम् वैष्णवमेव च}
{वास्तुसम्शमनीयानि मङ्गआनि प्रवर्तयन्} %2-56-31

\twolineshloka
{जपम् च न्यायतः कृत्वा स्नात्वा नद्याम् यथाविधि}
{पाप सम्शमनम् रामः चकार बलिम् उत्तमम्} %2-56-32

\twolineshloka
{वेदिस्थलविधानानि चैत्यान्यायतनानि च}
{आश्रमस्यानुरूपाणि स्थापयामास राघवः} %2-56-33

\twolineshloka
{वन्यैर्माल्यैः फलैर्मूलैः पक्वैर्माम्सैर्यथाविधि}
{अद्भर्जपैश्च वेदोक्तै र्धर्भैश्च ससमित्कुशैः} %2-56-34

\twolineshloka
{तौ तर्पयित्वा भूतानि राघवौ सह सीतया}
{तदा विविशतुः शालाम् सुशुभाम् शुभलक्षणौ} %2-56-35

\fourlineindentedshloka
{ताम् वृक्ष पर्णच् चदनाम् मनोज्ञाम्}
{यथा प्रदेशम् सुकृताम् निवाताम्}
{वासाय सर्वे विविशुः समेताः}
{सभाम् यथा देव गणाः सुधर्माम्} %2-56-36

\fourlineindentedshloka
{अनेक नाना मृग पक्षि सम्कुले}
{विचित्र पुष्प स्तबलैः द्रुमैः युते}
{वन उत्तमे व्याल मृग अनुनादिते}
{तथा विजह्रुः सुसुखम् जित इन्द्रियाः} %2-56-37

\fourlineindentedshloka
{सुरम्यम् आसाद्य तु चित्र कूटम्}
{नदीम् च ताम् माल्यवतीम् सुतीर्थाम्}
{ननन्द हृष्टः मृग पक्षि जुष्टाम्}
{जहौ च दुह्खम् पुर विप्रवासात्} %2-56-38


॥इत्यार्षे श्रीमद्रामायणे वाल्मीकीये आदिकाव्ये अयोध्याकाण्डे चित्रकूटनिवासः नाम षट्पञ्चाशः सर्गः ॥२-५६॥
