\sect{एकोनविंशः सर्गः — रामप्रतिज्ञा}

\twolineshloka
{तदप्रियममित्रघ्नो वचनं मरणोपमम्}
{श्रुत्वा न विव्यथे रामः कैकेयीं चेदमब्रवीत्} %2-19-1

\twolineshloka
{एवमस्तु गमिष्यामि वनं वस्तुमहं त्वितः}
{जटाचीरधरो राज्ञः प्रतिज्ञामनुपालयन्} %2-19-2

\twolineshloka
{इदं तु ज्ञातुमिच्छामि किमर्थं मां महीपतिः}
{नाभिनन्दति दुर्धर्षो यथापूर्वमरिन्दमः} %2-19-3

\twolineshloka
{मन्युर्न च त्वया कार्यो देवि ब्रूमि तवाग्रतः}
{यास्यामि भव सुप्रीता वनं चीरजटाधरः} %2-19-4

\twolineshloka
{हितेन गुरुणा पित्रा कृतज्ञेन नृपेण च}
{नियुज्यमानो विस्रब्धः किं न कुर्यामहं प्रियम्} %2-19-5

\twolineshloka
{अलीकं मानसं त्वेकं हृदयं दहते मम}
{स्वयं यन्नाह मां राजा भरतस्याभिषेचनम्} %2-19-6

\twolineshloka
{अहं हि सीतां राज्यं च प्राणानिष्टान् धनानि च}
{हृष्टो भ्रात्रे स्वयं दद्यां भरतायाप्रचोदितः} %2-19-7

\twolineshloka
{किं पुनर्मनुजेन्द्रेण स्वयं पित्रा प्रचोदितः}
{तव च प्रियकामार्थं प्रतिज्ञामनुपालयन्} %2-19-8

\twolineshloka
{तथाश्वासय ह्रीमन्तं किं त्विदं यन्महीपतिः}
{वसुधासक्तनयनो मन्दमश्रूणि मुञ्चति} %2-19-9

\twolineshloka
{गच्छन्तु चैवानयितुं दूताः शीघ्रजवैर्हयैः}
{भरतं मातुलकुलादद्यैव नृपशासनात्} %2-19-10

\twolineshloka
{दण्डकारण्यमेषोऽहं गच्छाम्येव हि सत्वरः}
{अविचार्य पितुर्वाक्यं समा वस्तुं चतुर्दश} %2-19-11

\twolineshloka
{सा हृष्टा तस्य तद् वाक्यं श्रुत्वा रामस्य कैकयी}
{प्रस्थानं श्रद्दधाना सा त्वरयामास राघवम्} %2-19-12

\twolineshloka
{एवं भवतु यास्यन्ति दूताः शीघ्रजवैर्हयैः}
{भरतं मातुलकुलादिहावर्तयितुं नराः} %2-19-13

\twolineshloka
{तव त्वहं क्षमं मन्ये नोत्सुकस्य विलम्बनम्}
{राम तस्मादितः शीघ्रं वनं त्वं गन्तुमर्हसि} %2-19-14

\twolineshloka
{व्रीडान्वितः स्वयं यच्च नृपस्त्वां नाभिभाषते}
{नैतत् किञ्चिन्नरश्रेष्ठ मन्युरेषोऽपनीयताम्} %2-19-15

\twolineshloka
{यावत्त्वं न वनं यातः पुरादस्मादतित्वरम्}
{पिता तावन्न ते राम स्नास्यते भोक्ष्यतेऽपि वा} %2-19-16

\twolineshloka
{धिक्कष्टमिति निःश्वस्य राजा शोकपरिप्लुतः}
{मूर्च्छितो न्यपतत् तस्मिन् पर्यङ्के हेमभूषिते} %2-19-17

\twolineshloka
{रामोऽप्युत्थाप्य राजानं कैकेय्याभिप्रचोदितः}
{कशयेव हतो वाजी वनं गन्तुं कृतत्वरः} %2-19-18

\twolineshloka
{तदप्रियमनार्याया वचनं दारुणोदयम्}
{श्रुत्वा गतव्यथो रामः कैकेयीं वाक्यमब्रवीत्} %2-19-19

\twolineshloka
{नाहमर्थपरो देवि लोकमावस्तुमुत्सहे}
{विद्धि मामृषिभिस्तुल्यं विमलं धर्ममास्थितम्} %2-19-20

\twolineshloka
{यत् तत्रभवतः किञ्चिच्छक्यं कर्तुं प्रियं मया}
{प्राणानपि परित्यज्य सर्वथा कृतमेव तत्} %2-19-21

\twolineshloka
{न ह्यतो धर्मचरणं किञ्चिदस्ति महत्तरम्}
{यथा पितरि शुश्रूषा तस्य वा वचनक्रिया} %2-19-22

\twolineshloka
{अनुक्तोऽप्यत्रभवता भवत्या वचनादहम्}
{वने वत्स्यामि विजने वर्षाणीह चतुर्दश} %2-19-23

\twolineshloka
{न नूनं मयि कैकेयि किञ्चिदाशंससे गुणान्}
{यद् राजानमवोचस्त्वं ममेश्वरतरा सती} %2-19-24

\twolineshloka
{यावन्मातरमापृच्छे सीतां चानुनयाम्यहम्}
{ततोऽद्यैव गमिष्यामि दण्डकानां महद् वनम्} %2-19-25

\twolineshloka
{भरतः पालयेद् राज्यं शुश्रूषेच्च पितुर्यथा}
{तथा भवत्या कर्तव्यं स हि धर्मः सनातनः} %2-19-26

\twolineshloka
{रामस्य तु वचः श्रुत्वा भृशं दुःखगतः पिता}
{शोकादशक्नुवन् वक्तुं प्ररुरोद महास्वनम्} %2-19-27

\twolineshloka
{वन्दित्वा चरणौ राज्ञो विसंज्ञस्य पितुस्तदा}
{कैकेय्याश्चाप्यनार्याया निष्पपात महाद्युतिः} %2-19-28

\twolineshloka
{स रामः पितरं कृत्वा कैकेयीं च प्रदक्षिणम्}
{निष्क्रम्यान्तःपुरात् तस्मात् स्वं ददर्श सुहृज्जनम्} %2-19-29

\twolineshloka
{तं बाष्पपरिपूर्णाक्षः पृष्ठतोऽनुजगाम ह}
{लक्ष्मणः परमक्रुद्धः सुमित्रानन्दवर्धनः} %2-19-30

\twolineshloka
{आभिषेचनिकं भाण्डं कृत्वा रामः प्रदक्षिणम्}
{शनैर्जगाम सापेक्षो दृष्टिं तत्राविचालयन्} %2-19-31

\twolineshloka
{न चास्य महतीं लक्ष्मीं राज्यनाशोऽपकर्षति}
{लोककान्तस्य कान्तत्वाच्छीतरश्मेरिव क्षयः} %2-19-32

\twolineshloka
{न वनं गन्तुकामस्य त्यजतश्च वसुन्धराम्}
{सर्वलोकातिगस्येव लक्ष्यते चित्तविक्रिया} %2-19-33

\twolineshloka
{प्रतिषिध्य शुभं छत्रं व्यजने च स्वलङ्कृते}
{विसर्जयित्वा स्वजनं रथं पौरांस्तथा जनान्} %2-19-34

\twolineshloka
{धारयन् मनसा दुःखमिन्द्रियाणि निगृह्य च}
{प्रविवेशात्मवान् वेश्म मातुरप्रियशंसिवान्} %2-19-35

\twolineshloka
{सर्वोऽप्यभिजनः श्रीमान् श्रीमतः सत्यवादिनः}
{नालक्षयत रामस्य कञ्चिदाकारमानने} %2-19-36

\twolineshloka
{उचितं च महाबाहुर्न जहौ हर्षमात्मवान्}
{शारदः समुदीर्णांशुश्चन्द्रस्तेज इवात्मजम्} %2-19-37

\twolineshloka
{वाचा मधुरया रामः सर्वं सम्मानयञ्जनम्}
{मातुः समीपं धर्मात्मा प्रविवेश महायशाः} %2-19-38

\twolineshloka
{तं गुणैः समतां प्राप्तो भ्राता विपुलविक्रमः}
{सौमित्रिरनुवव्राज धारयन् दुःखमात्मजम्} %2-19-39

\twolineshloka
{प्रविश्य वेश्मातिभृशं मुदा युतं समीक्ष्य तां चार्थविपत्तिमागताम्}
{न चैव रामोऽत्र जगाम विक्रियां सुहृज्जनस्यात्मविपत्तिशङ्कया} %2-19-40


॥इत्यार्षे श्रीमद्रामायणे वाल्मीकीये आदिकाव्ये अयोध्याकाण्डे रामप्रतिज्ञा नाम एकोनविंशः सर्गः ॥२-१९॥
