\sect{सप्ताशीतितमः सर्गः — रामशयनादिप्रश्नः}

\twolineshloka
{गुहस्य वचनम् श्रुत्वा भरतो भृशम् अप्रियम्}
{ध्यानम् जगाम तत्र एव यत्र तत् श्रुतम् अप्रियम्} %2-87-1

\twolineshloka
{सुकुमारो महा सत्त्वः सिम्ह स्कन्धो महा भुजः}
{पुण्डरीक विशाल अक्षः तरुणः प्रिय दर्शनः} %2-87-2

\twolineshloka
{प्रत्याश्वस्य मुहूर्तम् तु कालम् परम दुर्मनाः}
{पपात सहसा तोत्रैर् हृदि विद्ध इव द्विपः} %2-87-3

\twolineshloka
{भरतम् मुर्च्इतम् द्रुष्ट्वा विवर्णवदनो गुहः}
{बभूव व्यथितस्तत्र भूमिकम्पे यथा द्रुमः} %2-87-4

\twolineshloka
{तद् अवस्थम् तु भरतम् शत्रुघ्नो अनन्तर स्थितः}
{परिष्वज्य रुरोद उच्चैर् विसम्ज्नः शोक कर्शितः} %2-87-5

\twolineshloka
{ततः सर्वाः समापेतुर् मातरो भरतस्य ताः}
{उपवास कृशा दीना भर्तृ व्यसन कर्शिताः} %2-87-6

\twolineshloka
{ताः च तम् पतितम् भूमौ रुदन्त्यः पर्यवारयन्}
{कौसल्या तु अनुसृत्य एनम् दुर्मनाः परिषस्वजे} %2-87-7

\twolineshloka
{वत्सला स्वम् यथा वत्सम् उपगूह्य तपस्विनी}
{परिपप्रग्च्अ भरतम् रुदन्ती शोक लालसा} %2-87-8

\twolineshloka
{पुत्र व्याधिर् न ते कच्चित् शरीरम् परिबाधते}
{अद्य राज कुलस्य अस्य त्वद् अधीनम् हि जीवितम्} %2-87-9

\twolineshloka
{त्वाम् दृष्ट्वा पुत्र जीवामि रामे सभ्रातृके गते}
{वृत्ते दशरथे राज्नि नाथ एकः त्वम् अद्य नः} %2-87-10

\twolineshloka
{कच्चिन् न लक्ष्मणे पुत्र श्रुतम् ते किम्चिद् अप्रियम्}
{पुत्र वा ह्य् एकपुत्रायाः सह भार्ये वनम् गते} %2-87-11

\twolineshloka
{स मुहूर्तम् समाश्वस्य रुदन्न् एव महा यशाः}
{कौसल्याम् परिसान्त्व्य इदम् गुहम् वचनम् अब्रवीत्} %2-87-12

\twolineshloka
{भ्राता मे क्व अवसद् रात्रिम् क्व सीता क्व च लक्ष्मणः}
{अस्वपत् शयने कस्मिन् किम् भुक्त्वा गुह शम्स मे} %2-87-13

\twolineshloka
{सो अब्रवीद् भरतम् पृष्टो निषाद अधिपतिर् गुहः}
{यद् विधम् प्रतिपेदे च रामे प्रिय हिते अतिथौ} %2-87-14

\twolineshloka
{अन्नम् उच्च अवचम् भक्ष्याः फलानि विविधानि च}
{रामाय अभ्यवहार अर्थम् बहु च उपहृतम् मया} %2-87-15

\twolineshloka
{तत् सर्वम् प्रत्यनुज्नासीद् रामः सत्य पराक्रमः}
{न हि तत् प्रत्यगृह्णात् स क्षत्र धर्मम् अनुस्मरन्} %2-87-16

\twolineshloka
{न ह्य् अस्माभिः प्रतिग्राह्यम् सखे देयम् तु सर्वदा}
{इति तेन वयम् राजन्न् अनुनीता महात्मना} %2-87-17

\twolineshloka
{लक्ष्मणेन समानीतम् पीत्वा वारि महा यशाः}
{औपवास्यम् तदा अकार्षीद् राघवः सह सीतया} %2-87-18

\twolineshloka
{ततः तु जल शेषेण लक्ष्मणो अप्य् अकरोत् तदा}
{वाग् यताः ते त्रयः सम्ध्याम् उपासत समाहिताः} %2-87-19

\twolineshloka
{सौमित्रिः तु ततः पश्चाद् अकरोत् स्वास्तरम् शुभम्}
{स्वयम् आनीय बर्हीम्षि क्षिप्रम् राघव कारणात्} %2-87-20

\twolineshloka
{तस्मिन् समाविशद् रामः स्वास्तरे सह सीतया}
{प्रक्षाल्य च तयोः पादाउ अपचक्राम लक्ष्मणः} %2-87-21

\twolineshloka
{एतत् तद् इन्गुदी मूलम् इदम् एव च तत् तृणम्}
{यस्मिन् रामः च सीता च रात्रिम् ताम् शयिताउ उभौ} %2-87-22

\fourlineindentedshloka
{नियम्य पृष्ठे तु तल अन्गुलित्रवान्}
{शरैः सुपूर्णाउ इषुधी परम् तपः}
{महद् धनुः सज्यम् उपोह्य लक्ष्मणो}
{निशाम् अतिष्ठत् परितो अस्य केवलम्} %2-87-23

\fourlineindentedshloka
{ततः तु अहम् च उत्तम बाण चापधृक्}
{स्थितो अभवम् तत्र स यत्र लक्ष्मणः}
{अतन्द्रिभिर् ज्नातिभिर् आत्त कार्मुकैर्}
{महा इन्द्र कल्पम् परिपालयमः तदा} %2-87-24


॥इत्यार्षे श्रीमद्रामायणे वाल्मीकीये आदिकाव्ये अयोध्याकाण्डे रामशयनादिप्रश्नः नाम सप्ताशीतितमः सर्गः ॥२-८७॥
