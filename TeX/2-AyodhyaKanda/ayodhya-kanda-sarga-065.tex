\sect{पञ्चषष्ठितमः सर्गः — अन्तःपुराक्रन्दः}

\twolineshloka
{अथ रात्र्याम् व्यतीतायाम् प्रातर् एव अपरे अहनि}
{वन्दिनः पर्युपातिष्ठम्स् तत् पार्थिव निवेशनम्} %2-65-1

\twolineshloka
{सूताः परमसम्स्काराः मङ्गआश्चोओत्तमश्रुताः}
{गायकाः स्तुतिशीलाश्च निगदन्तः पृथक् पृथक्} %2-65-2

\twolineshloka
{राजानम् स्तुताम् तेषामुदात्ताभिहिताशिषाम्}
{प्रासादाभोगविस्तीर्णः स्तुतिशब्दो ह्यवर्तत} %2-65-3

\twolineshloka
{ततस्तु स्तुवताम् तेषाम् सूतानाम् पाणिवादकाः}
{अवदानान्युदाहृत्य पाणिवादा नवादयन्} %2-65-4

\twolineshloka
{तेन शब्देन विहगाः प्रतिबुद्धा विसस्वनुः}
{शाखास्थाः पञ्जरस्थाश्च ये राजकुलगोचराः} %2-65-5

\twolineshloka
{व्याहृताः पुण्य्शब्दाश्च वीणानाम् चापि निस्स्वनाः}
{आशीर्गेयम् च गाथानाम् पूरयामास वेश्म तत} %2-65-6

\twolineshloka
{ततः शुचि समाचाराः पर्युपस्थान कोविदः}
{स्त्री वर्ष वर भूयिष्ठाउपतस्थुर् यथा पुरम्} %2-65-7

\twolineshloka
{हरि चन्दन सम्पृक्तम् उदकम् कान्चनैः घटैः}
{आनिन्युः स्नान शिक्षा आज्ञा यथा कालम् यथा विधि} %2-65-8

\twolineshloka
{मन्गल आलम्भनीयानि प्राशनीयान् उपस्करान्}
{उपनिन्युस् तथा अपि अन्याः कुमारी बहुलाः स्त्रियः} %2-65-9

\twolineshloka
{सर्वलक्षणसम्पन्नम् सर्वम् विधिवदर्चितम्}
{सर्वम् सुगुणलक्स्मीवत्तद्भभूवाभिहारिकम्} %2-65-10

\twolineshloka
{ततः सूर्योदयम् यावत्सर्वम् परिसमुत्सुकम्}
{तस्थावनुपसम्प्राप्तम् किम् स्विदित्युपश्} %2-65-11

\twolineshloka
{अथ याः कोसल इन्द्रस्य शयनम् प्रत्यनन्तराः}
{ताः स्त्रियः तु समागम्य भर्तारम् प्रत्यबोधयन्} %2-65-12

\twolineshloka
{तथाप्युचितवृत्तास्ता विनयेन नयेन च}
{न ह्यस्य शयनम् स्पृष्ट्वा किम् चिदप्युपलेभिरे} %2-65-13

\twolineshloka
{ताः स्त्रीयः स्वप्नशीलज्ञास्चेष्टासम्चलनादिषु}
{ता वेपथु परीताः च राज्ञः प्राणेषु शन्किताः} %2-65-14

\twolineshloka
{प्रतिस्रोतः तृण अग्राणाम् सदृशम् सम्चकम्पिरे}
{अथ सम्वेपमनानाम् स्त्रीणाम् दृष्ट्वा च पार्थिवम्} %2-65-15

\twolineshloka
{यत् तत् आशन्कितम् पापम् तस्य जज्ञे विनिश्चयः}
{कौसल्या च सुमित्रा च पुत्रशोकपराजिते} %2-65-16

\twolineshloka
{प्रसुप्ते न प्रबुध्येते यथा कालसमन्विते}
{निष्प्रभा च विवर्णा च सन्ना शोकेन सन्नता} %2-65-17

\twolineshloka
{न व्यराजत कौसल्या तारेव तिमिरावृता}
{कौसल्यानन्तरम् राज्ञः सुमित्रा तदन्तनरम्} %2-65-18

\twolineshloka
{न स्म विभ्राजते देवी शोकाश्रुलुलितानना}
{ते च दृष्ट्वा तथा सुप्ते शुभे देव्यौ च तम् नृपम्} %2-65-19

\twolineshloka
{सुप्तमे वोद्गतप्राणमन्तः पुरमन्यत}
{ततः प्रचुक्रुशुर् दीनाः सस्वरम् ता वर अन्गनाः} %2-65-20

\twolineshloka
{करेणवैव अरण्ये स्थान प्रच्युत यूथपाः}
{तासाम् आक्रन्द शब्देन सहसा उद्गत चेतने} %2-65-21

\twolineshloka
{कौसल्या च सुमित्राच त्यक्त निद्रे बभूवतुः}
{कौसल्या च सुमित्रा च दृष्ट्वा स्पृष्ट्वा च पार्थिवम्} %2-65-22

\twolineshloka
{हा नाथ इति परिक्रुश्य पेततुर् धरणी तले}
{सा कोसल इन्द्र दुहिता वेष्टमाना मही तले} %2-65-23

\twolineshloka
{न बभ्राज रजो ध्वस्ता तारा इव गगन च्युता}
{नृपे शान्तगुणे जाते कौसल्याम् पतिताम् भुवि} %2-65-24

\twolineshloka
{आपश्यम्स्ताः स्त्रियः सर्वा हताम् नागवधूमिव}
{ततः सर्वा नरेन्द्रस्य कैकेयीप्रमुखाः स्त्रियः} %2-65-25

\twolineshloka
{रुदन्त्यः शोकसन्तप्ता निपेतुर्गतचेतनाः}
{ताभिः स बलवान्नादः क्रोशन्तीभिरनुद्रुतः} %2-65-26

\twolineshloka
{येन स्फीतीकृतो भूयस्तद्गृहम् समनादयत्}
{तत् समुत्त्रस्त सम्भ्रान्तम् पर्युत्सुक जन आकुलम्} %2-65-27

\twolineshloka
{सर्वतः तुमुल आक्रन्दम् परिताप आर्त बान्धवम्}
{सद्यो निपतित आनन्दम् दीन विक्लव दर्शनम्} %2-65-28

\twolineshloka
{बभूव नर देवस्य सद्म दिष्ट अन्तम् ईयुषः}
{अतीतम् आज्ञाय तु पार्थिव ऋषभम्}
{यशस्विनम् सम्परिवार्य पत्नयः}
{भृशम् रुदन्त्यः करुणम् सुदुह्खिताः}
{प्रगृह्य बाहू व्यलपन्न् अनाथवत्} %2-65-29


॥इत्यार्षे श्रीमद्रामायणे वाल्मीकीये आदिकाव्ये अयोध्याकाण्डे अन्तःपुराक्रन्दः नाम पञ्चषष्ठितमः सर्गः ॥२-६५॥
