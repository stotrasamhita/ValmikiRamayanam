\sect{एकोनसप्ततितमः सर्गः — भरतदुःस्वप्नः}

\twolineshloka
{याम् एव रात्रिम् ते दूताः प्रविशन्ति स्म ताम् पुरीम्}
{भरतेन अपि ताम् रात्रिम् स्वप्नो दृष्टः अयम् अप्रियः} %2-69-1

\twolineshloka
{व्युष्टाम् एव तु ताम् रात्रिम् दृष्ट्वा तम् स्वप्नम् अप्रियम्}
{पुत्रः राज अधिराजस्य सुभृशम् पर्यतप्यत} %2-69-2

\twolineshloka
{तप्यमानम् समाज्ञाय वयस्याः प्रिय वादिनः}
{आयासम् हि विनेष्यन्तः सभायाम् चक्रिरे कथाः} %2-69-3

\twolineshloka
{वादयन्ति तथा शान्तिम् लासयन्ति अपि च अपरे}
{नाटकानि अपरे प्राहुर् हास्यानि विविधानि च} %2-69-4

\twolineshloka
{स तैः महात्मा भरतः सखिभिः प्रिय वादिभिः}
{गोष्ठी हास्यानि कुर्वद्भिर् न प्राहृष्यत राघवः} %2-69-5

\twolineshloka
{तम् अब्रवीत् प्रिय सखो भरतम् सखिभिर् वृतम्}
{सुहृद्भिः पर्युपासीनः किम् सखे न अनुमोदसे} %2-69-6

\twolineshloka
{एवम् ब्रुवाणम् सुहृदम् भरतः प्रत्युवाच ह}
{शृणु त्वम् यन् निमित्तम्मे दैन्यम् एतत् उपागतम्} %2-69-7

\twolineshloka
{स्वप्ने पितरम् अद्राक्षम् मलिनम् मुक्त मूर्धजम्}
{पतन्तम् अद्रि शिखरात् कलुषे गोमये ह्रदे} %2-69-8

\twolineshloka
{प्लवमानः च मे दृष्टः स तस्मिन् गोमय ह्रदे}
{पिबन्न् अन्जलिना तैलम् हसन्न् इव मुहुर् मुहुः} %2-69-9

\twolineshloka
{ततः तिलोदनम् भुक्त्वा पुनः पुनर् अधः शिराः}
{तैलेन अभ्यक्त सर्व अन्गः तैलम् एव अवगाहत} %2-69-10

\twolineshloka
{स्वप्ने अपि सागरम् शुष्कम् चन्द्रम् च पतितम् भुवि}
{सहसा च अपि सम्शन्तम् ज्वलितम् जात वेदसम्} %2-69-11

\twolineshloka
{औपवाह्यस्य नागस्य विषाणम् शकलीकृतम्}
{सहसा चापि सम्शान्तम् ज्वलितम् जातवेदसम्} %2-69-12

\twolineshloka
{अवदीर्णाम् च पृथिवीम् शुष्कामः च विविधान् द्रुमान्}
{अहम् पश्यामि विध्वस्तान् सधूमामः चैव पार्वतान्} %2-69-13

\twolineshloka
{पीठे कार्ष्णायसे च एनम् निषण्णम् कृष्ण वाससम्}
{प्रहसन्ति स्म राजानम् प्रमदाः कृष्ण पिन्गलाः} %2-69-14

\twolineshloka
{त्वरमाणः च धर्म आत्मा रक्त माल्य अनुलेपनः}
{रथेन खर युक्तेन प्रयातः दक्षिणा मुखः} %2-69-15

\twolineshloka
{प्रहसन्तीव राजानम् प्रमदा रक्तवासिनी}
{प्रकर्षन्ती मया दृष्टा राक्षसी विकृतासना} %2-69-16

\twolineshloka
{एवम् एतन् मया दृष्टम् इमाम् रात्रिम् भय आवहाम्}
{अहम् रामः अथ वा राजा लक्ष्मणो वा मरिष्यति} %2-69-17

\twolineshloka
{नरः यानेन यः स्वप्ने खर युक्तेन याति हि}
{अचिरात् तस्य धूम अग्रम् चितायाम् सम्प्रदृश्यते} %2-69-18

\twolineshloka
{एतन् निमित्तम् दीनो अहम् तन् न वः प्रतिपूजये}
{शुष्यति इव च मे कण्ठो न स्वस्थम् इव मे मनः} %2-69-19

\twolineshloka
{न पश्यामि भयस्थानम् भयम् चैवोपधारये}
{भ्रष्टश्च स्वरयोगो मे चाया चोपहता मम} %2-69-20

\twolineshloka
{जुगुप्सन्न् इव च आत्मानम् न च पश्यामि कारणम्}
{इमाम् हि दुह्स्वप्न गतिम् निशाम्य ताम्}
{अनेक रूपाम् अवितर्किताम् पुरा}
{भयम् महत् तद्द् हृदयान् न याति मे}
{विचिन्त्य राजानम् अचिन्त्य दर्शनम्} %2-69-21


॥इत्यार्षे श्रीमद्रामायणे वाल्मीकीये आदिकाव्ये अयोध्याकाण्डे भरतदुःस्वप्नः नाम एकोनसप्ततितमः सर्गः ॥२-६९॥
