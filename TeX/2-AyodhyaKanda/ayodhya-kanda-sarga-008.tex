\sect{अष्टमः सर्गः — मन्थरोपजापः}

\twolineshloka
{मन्थरा त्वभ्यसूय्यैनामुत्सृज्याभरणं हि तत्}
{उवाचेदं ततो वाक्यं कोपदुःखसमन्विता} %2-8-1

\twolineshloka
{हर्षं किमर्थमस्थाने कृतवत्यसि बालिशे}
{शोकसागरमध्यस्थं नात्मानमवबुध्यसे} %2-8-2

\twolineshloka
{मनसा प्रसहामि त्वां देवि दुःखार्दिता सती}
{यच्छोचितव्ये हृष्टासि प्राप्य त्वं व्यसनं महत्} %2-8-3

\twolineshloka
{शोचामि दुर्मतित्वं ते का हि प्राज्ञा प्रहर्षयेत्}
{अरेः सपत्नीपुत्रस्य वृद्धिं मृत्योरिवागताम्} %2-8-4

\twolineshloka
{भरतादेव रामस्य राज्यसाधारणाद् भयम्}
{तद् विचिन्त्य विषण्णास्मि भयं भीताद्धि जायते} %2-8-5

\twolineshloka
{लक्ष्मणो हि महाबाहू रामं सर्वात्मना गतः}
{शत्रुघ्नश्चापि भरतं काकुत्स्थं लक्ष्मणो यथा} %2-8-6

\twolineshloka
{प्रत्यासन्नक्रमेणापि भरतस्यैव भामिनि}
{राज्यक्रमो विसृष्टस्तु तयोस्तावद्यवीयसोः} %2-8-7

\twolineshloka
{विदुषः क्षत्रचारित्रे प्राज्ञस्य प्राप्तकारिणः}
{भयात् प्रवेपे रामस्य चिन्तयन्ती तवात्मजम्} %2-8-8

\twolineshloka
{सुभगा किल कौसल्या यस्याः पुत्रोऽभिषेक्ष्यते}
{यौवराज्येन महता श्वः पुष्येण द्विजोत्तमैः} %2-8-9

\twolineshloka
{प्राप्तां वसुमतीं प्रीतिं प्रतीतां हतविद्विषम्}
{उपस्थास्यसि कौसल्यां दासीवत् त्वं कृताञ्जलिः} %2-8-10

\twolineshloka
{एवं च त्वं सहास्माभिस्तस्याः प्रेष्या भविष्यसि}
{पुत्रश्च तव रामस्य प्रेष्यत्वं हि गमिष्यति} %2-8-11

\twolineshloka
{हृष्टाः खलु भविष्यन्ति रामस्य परमाः स्त्रियः}
{अप्रहृष्टा भविष्यन्ति स्नुषास्ते भरतक्षये} %2-8-12

\twolineshloka
{तां दृष्ट्वा परमप्रीतां ब्रुवन्तीं मन्थरां ततः}
{रामस्यैव गुणान् देवी कैकेयी प्रशशंस ह} %2-8-13

\twolineshloka
{धमर्ज्ञो गुणवान् दान्तः कृतज्ञः सत्यवान् शुचिः}
{रामो राजसुतो ज्येष्ठो यौवराज्यमतोऽर्हति} %2-8-14

\twolineshloka
{भ्रातॄन् भृत्यांश्च दीर्घायुः पितृवत् पालयिष्यति}
{सन्तप्यसे कथं कुब्जे श्रुत्वा रामाभिषेचनम्} %2-8-15

\twolineshloka
{भरतश्चापि रामस्य ध्रुवं वर्षशतात् परम्}
{पितृपैतामहं राज्यमवाप्स्यति नरर्षभः} %2-8-16

\twolineshloka
{सा त्वमभ्युदये प्राप्ते दह्यमानेव मन्थरे}
{भविष्यति च कल्याणे किमिदं परितप्यसे} %2-8-17

\twolineshloka
{यथा वै भरतो मान्यस्तथा भूयोऽपि राघवः}
{कौसल्यातोऽतिरिक्तं च मम शुश्रूषते बहु} %2-8-18

\twolineshloka
{राज्यं यदि हि रामस्य भरतस्यापि तत् तदा}
{मन्यते हि यथाऽऽत्मानं यथा भ्रातॄंस्तु राघवः} %2-8-19

\twolineshloka
{कैकेय्या वचनं श्रुत्वा मन्थरा भृशदुःखिता}
{दीर्घमुष्णं विनिःश्वस्य कैकेयीमिदमब्रवीत्} %2-8-20

\twolineshloka
{अनर्थदर्शिनी मौर्ख्यान्नात्मानमवबुध्यसे}
{शोकव्यसनविस्तीर्णे मज्जन्ती दुःखसागरे} %2-8-21

\twolineshloka
{भविता राघवो राजा राघवस्य च यः सुतः}
{राजवंशात्तु भरतः कैकेयि परिहास्यते} %2-8-22

\twolineshloka
{नहि राज्ञः सुताः सर्वे राज्ये तिष्ठन्ति भामिनि}
{स्थाप्यमानेषु सर्वेषु सुमहाननयो भवेत्} %2-8-23

\twolineshloka
{तस्माज्ज्येष्ठे हि कैकेयि राज्यतन्त्राणि पार्थिवाः}
{स्थापयन्त्यनवद्याङ्गि गुणवत्स्वितरेष्वपि} %2-8-24

\twolineshloka
{असावत्यन्तनिर्भग्नस्तव पुत्रो भविष्यति}
{अनाथवत् सुखेभ्यश्च राजवंशाच्च वत्सले} %2-8-25

\twolineshloka
{साहं त्वदर्थे सम्प्राप्ता त्वं तु मां नावबुद्ध्यसे}
{सपत्निवृद्धौ या मे त्वं प्रदेयं दातुमर्हसि} %2-8-26

\twolineshloka
{ध्रुवं तु भरतं रामः प्राप्य राज्यमकण्टकम्}
{देशान्तरं नाययिता लोकान्तरमथापि वा} %2-8-27

\twolineshloka
{बाल एव तु मातुल्यं भरतो नायितस्त्वया}
{सन्निकर्षाच्च सौहार्दं जायते स्थावरेष्विव} %2-8-28

\twolineshloka
{भरतानुवशात् सोऽपि शत्रुघ्नस्तत्समं गतः}
{लक्ष्मणो हि यथा रामं तथायं भरतं गतः} %2-8-29

\twolineshloka
{श्रूयते हि द्रुमः कश्चिच्छेत्तव्यो वनजीवनैः}
{सन्निकर्षादिषीकाभिर्मोचितः परमाद् भयात्} %2-8-30

\twolineshloka
{गोप्ता हि रामं सौमित्रिर्लक्ष्मणं चापि राघवः}
{अश्विनोरिव सौभ्रात्रं तयोर्लोकेषु विश्रुतम्} %2-8-31

\twolineshloka
{तस्मान्न लक्ष्मणे रामः पापं किञ्चित् करिष्यति}
{रामस्तु भरते पापं कुर्यादेव न संशयः} %2-8-32

\twolineshloka
{तस्माद् राजगृहादेव वनं गच्छतु राघवः}
{एतद्धि रोचते मह्यं भृशं चापि हितं तव} %2-8-33

\twolineshloka
{एवं ते ज्ञातिपक्षस्य श्रेयश्चैव भविष्यति}
{यदि चेद् भरतो धर्मात् पित्र्यं राज्यमवाप्स्यति} %2-8-34

\twolineshloka
{स ते सुखोचितो बालो रामस्य सहजो रिपुः}
{समृद्धार्थस्य नष्टार्थो जीविष्यति कथं वशे} %2-8-35

\twolineshloka
{अभिद्रुतमिवारण्ये सिंहेन गजयूथपम्}
{प्रच्छाद्यमानं रामेण भरतं त्रातुमर्हसि} %2-8-36

\twolineshloka
{दर्पान्निराकृता पूर्वं त्वया सौभाग्यवत्तया}
{राममाता सपत्नी ते कथं वैरं न यापयेत्} %2-8-37

\twolineshloka
{यदा च रामः पृथिवीमवाप्स्यते प्रभूतरत्नाकरशैलसंयुताम्}
{तदा गमिष्यस्यशुभं पराभवं सहैव दीना भरतेन भामिनि} %2-8-38

\twolineshloka
{यदा हि रामः पृथिवीमवाप्स्यते ध्रुवं प्रणष्टो भरतो भविष्यति}
{अतो हि सञ्चिन्तय राज्यमात्मजे परस्य चैवास्य विवासकारणम्} %2-8-39


॥इत्यार्षे श्रीमद्रामायणे वाल्मीकीये आदिकाव्ये अयोध्याकाण्डे मन्थरोपजापः नाम अष्टमः सर्गः ॥२-८॥
