\sect{एकाशीतितमः सर्गः — सभास्थानम्}

\twolineshloka
{ततः नान्दी मुखीम् रात्रिम् भरतम् सूत मागधाः}
{तुष्टुवुर् वाग् विशेषज्ञाः स्तवैः मन्गल सम्हितैः} %2-81-1

\twolineshloka
{सुवर्ण कोण अभिहतः प्राणदद् याम दुन्दुभिः}
{दध्मुः शन्खामः च शतशो वाद्यामः च उच्च अवच स्वरान्} %2-81-2

\twolineshloka
{स तूर्य घोषः सुमहान् दिवम् आपूरयन्न् इव}
{भरतम् शोक सम्तप्तम् भूयः शोकैः अरन्ध्रयत्} %2-81-3

\twolineshloka
{ततः प्रबुद्धो भरतः तम् घोषम् सम्निवर्त्य च}
{न अहम् राजा इति च अपि उक्त्वा शत्रुघ्नम् इदम् अब्रवीत्} %2-81-4

\twolineshloka
{पश्य शत्रुघ्न कैकेय्या लोकस्य अपकृतम् महत्}
{विसृज्य मयि दुह्खानि राजा दशरथो गतः} %2-81-5

\twolineshloka
{तस्य एषा धर्म राजस्य धर्म मूला महात्मनः}
{परिभ्रमति राज श्रीर् नौर् इव अकर्णिका जले} %2-81-6

\twolineshloka
{यो हि नः सुमहान्नाथः सोऽपि प्रव्राजितो वनम्}
{अनया धर्ममुत्सृज्य मात्रा मे राघवः स्वयम्} %2-81-7

\twolineshloka
{इति एवम् भरतम् प्रेक्ष्य विलपन्तम् विचेतनम्}
{कृपणम् रुरुदुः सर्वाः सस्वरम् योषितः तदा} %2-81-8

\twolineshloka
{तथा तस्मिन् विलपति वसिष्ठो राज धर्मवित्}
{सभाम् इक्ष्वाकु नाथस्य प्रविवेश महा यशाः} %2-81-9

\twolineshloka
{शात कुम्भमयीम् रम्याम् मणि रत्न समाकुलाम्}
{सुधर्माम् इव धर्म आत्मा सगणः प्रत्यपद्यत} %2-81-10

\twolineshloka
{स कान्चनमयम् पीठम् पर अर्ध्य आस्तरण आवृतम्}
{अध्यास्त सर्व वेदज्ञो दूतान् अनुशशास च} %2-81-11

\twolineshloka
{ब्राह्मणान् क्षत्रियान् योधान् अमात्यान् गण बल्लभान्}
{क्षिप्रम् आनयत अव्यग्राः कृत्यम् आत्ययिकम् हि नः} %2-81-12

\twolineshloka
{सराजभृत्यम् शत्रुघ्नम् भरतम् च यश्स्विनम्}
{युधाजितम् सुमन्त्रम् च ये च तत्र हिता जनाः} %2-81-13

\twolineshloka
{ततः हलहला शब्दो महान् समुदपद्यत}
{रथैः अश्वैः गजैः च अपि जनानाम् उपगच्चताम्} %2-81-14

\twolineshloka
{ततः भरतम् आयान्तम् शत क्रतुम् इव अमराः}
{प्रत्यनन्दन् प्रकृतयो यथा दशरथम् तथा} %2-81-15

\fourlineindentedshloka
{ह्रदैव तिमि नाग सम्वृतः}
{स्तिमित जलो मणि शन्ख शर्करः}
{दशरथ सुत शोभिता सभा}
{सदशरथा इव बभौ यथा पुरा} %2-81-16


॥इत्यार्षे श्रीमद्रामायणे वाल्मीकीये आदिकाव्ये अयोध्याकाण्डे सभास्थानम् नाम एकाशीतितमः सर्गः ॥२-८१॥
