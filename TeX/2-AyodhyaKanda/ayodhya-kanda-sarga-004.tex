\sect{चतुर्थः सर्गः — मात्राशीः परिग्रहः}

\twolineshloka
{गतेष्वथ नृपो भूयः पौरेषु सह मन्त्रिभिः}
{मन्त्रयित्वा ततश्चक्रे निश्चयज्ञः स निश्चयम्} %2-4-1

\twolineshloka
{श्व एव पुष्यो भविता श्वोऽभिषेच्यस्तु मे सुतः}
{रामो राजीवपत्राक्षो युवराज इति प्रभुः} %2-4-2

\twolineshloka
{अथान्तर्गृहमाविश्य राजा दशरथस्तदा}
{सूतमामन्त्रयामास रामं पुनरिहानय} %2-4-3

\twolineshloka
{प्रतिगृह्य तु तद्वाक्यं सूतः पुनरुपाययौ}
{रामस्य भवनं शीघ्रं राममानयितुं पुनः} %2-4-4

\twolineshloka
{द्वाःस्थैरावेदितं तस्य रामायागमनं पुनः}
{श्रुत्वैव चापि रामस्तं प्राप्तं शङ्कान्वितोऽभवत्} %2-4-5

\twolineshloka
{प्रवेश्य चैनं त्वरितो रामो वचनमब्रवीत्}
{यदागमनकृत्यं ते भूयस्तद्ब्रूह्यशेषतः} %2-4-6

\twolineshloka
{तमुवाच ततः सूतो राजा त्वां द्रष्टुमिच्छति}
{श्रुत्वा प्रमाणं तत्र त्वं गमनायेतराय वा} %2-4-7

\twolineshloka
{इति सूतवचः श्रुत्वा रामोऽपि त्वरयान्वितः}
{प्रययौ राजभवनं पुनर्द्रष्टुं नरेश्वरम्} %2-4-8

\twolineshloka
{तं श्रुत्वा समनुप्राप्तं रामं दशरथो नृपः}
{प्रवेशयामास गृहं विवक्षुः प्रियमुत्तमम्} %2-4-9

\twolineshloka
{प्रविशन्नेव च श्रीमान् राघवो भवनं पितुः}
{ददर्श पितरं दूरात् प्रणिपत्य कृताञ्जलिः} %2-4-10

\twolineshloka
{प्रणमन्तं तमुत्थाप्य सम्परिष्वज्य भूमिपः}
{प्रदिश्य चासनं चास्मै रामं च पुनरब्रवीत्} %2-4-11

\twolineshloka
{राम वृद्धोऽस्मि दीर्घायुर्भुक्ता भोगा यथेप्सिताः}
{अन्नवद्भिः क्रतुशतैर्यथेष्टं भूरिदक्षिणैः} %2-4-12

\twolineshloka
{जातमिष्टमपत्यं मे त्वमद्यानुपमं भुवि}
{दत्तमिष्टमधीतं च मया पुरुषसत्तम} %2-4-13

\twolineshloka
{अनुभूतानि चेष्टानि मया वीर सुखान्यपि}
{देवर्षिपितृविप्राणामनृणोऽस्मि तथाऽऽत्मनः} %2-4-14

\twolineshloka
{न किञ्चिन्मम कर्तव्यं तवान्यत्राभिषेचनात्}
{अतो यत्त्वामहं ब्रूयां तन्मे त्वं कर्तुमर्हसि} %2-4-15

\twolineshloka
{अद्य प्रकृतयः सर्वास्त्वामिच्छन्ति नराधिपम्}
{अतस्त्वां युवराजानमभिषेक्ष्यामि पुत्रक} %2-4-16

\twolineshloka
{अपि चाद्याशुभान् राम स्वप्नान् पश्यामि राघव}
{सनिर्घाता दिवोल्काश्च पतन्ति हि महास्वनाः} %2-4-17

\twolineshloka
{अवष्टब्धं च मे राम नक्षत्रं दारुणग्रहैः}
{आवेदयन्ति दैवज्ञाः सूर्याङ्गारकराहुभिः} %2-4-18

\twolineshloka
{प्रायेण च निमित्तानामीदृशानां समुद्भवे}
{राजा हि मृत्युमाप्नोति घोरां चापदमृच्छति} %2-4-19

\twolineshloka
{तद् यावदेव मे चेतो न विमुह्यति राघव}
{तावदेवाभिषिञ्चस्व चला हि प्राणिनां मतिः} %2-4-20

\twolineshloka
{अद्य चन्द्रोऽभ्युपगमत् पुष्यात् पूर्वं पुनर्वसुम्}
{श्वः पुष्ययोगं नियतं वक्ष्यन्ते दैवचिन्तकाः} %2-4-21

\twolineshloka
{तत्र पुष्येऽभिषिञ्चस्व मनस्त्वरयतीव माम्}
{श्वस्त्वाहमभिषेक्ष्यामि यौवराज्ये परन्तप} %2-4-22

\twolineshloka
{तस्मात् त्वयाद्यप्रभृति निशेयं नियतात्मना}
{सह वध्वोपवस्तव्या दर्भप्रस्तरशायिना} %2-4-23

\twolineshloka
{सुहृदश्चाप्रमत्तास्त्वां रक्षन्त्वद्य समन्ततः}
{भवन्ति बहुविघ्नानि कार्याण्येवंविधानि हि} %2-4-24

\twolineshloka
{विप्रोषितश्च भरतो यावदेव पुरादितः}
{तावदेवाभिषेकस्ते प्राप्तकालो मतो मम} %2-4-25

\twolineshloka
{कामं खलु सतां वृत्ते भ्राता ते भरतः स्थितः}
{ज्येष्ठानुवर्ती धर्मात्मा सानुक्रोशो जितेन्द्रियः} %2-4-26

\twolineshloka
{किं नु चित्तं मनुष्याणामनित्यमिति मे मतम्}
{सतां च धर्मनित्यानां कृतशोभि च राघव} %2-4-27

\twolineshloka
{इत्युक्तः सोऽभ्यनुज्ञातः श्वोभाविन्यभिषेचने}
{व्रजेति रामः पितरमभिवाद्याभ्ययाद् गृहम्} %2-4-28

\twolineshloka
{प्रविश्य चात्मनो वेश्म राज्ञाऽऽदिष्टेऽभिषेचने}
{तत्क्षणादेव निष्क्रम्य मातुरन्तःपुरं ययौ} %2-4-29

\twolineshloka
{तत्र तां प्रवणामेव मातरं क्षौमवासिनीम्}
{वाग्यतां देवतागारे ददर्शायाचतीं श्रियम्} %2-4-30

\twolineshloka
{प्रागेव चागता तत्र सुमित्रा लक्ष्मणस्तथा}
{सीता चानयिता श्रुत्वा प्रियं रामाभिषेचनम्} %2-4-31

\twolineshloka
{तस्मिन् कालेऽपि कौसल्या तस्थावामीलितेक्षणा}
{सुमित्रयान्वास्यमाना सीतया लक्ष्मणेन च} %2-4-32

\twolineshloka
{श्रुत्वा पुष्ये च पुत्रस्य यौवराज्येऽभिषेचनम्}
{प्राणायामेन पुरुषं ध्यायमाना जनार्दनम्} %2-4-33

\twolineshloka
{तथा सनियमामेव सोऽभिगम्याभिवाद्य च}
{उवाच वचनं रामो हर्षयंस्तामिदं वरम्} %2-4-34

\twolineshloka
{अम्ब पित्रा नियुक्तोऽस्मि प्रजापालनकर्मणि}
{भविता श्वोऽभिषेको मे यथा मे शासनं पितुः} %2-4-35

\twolineshloka
{सीतयाप्युपवस्तव्या रजनीयं मया सह}
{एवमुक्तमुपाध्यायैः स हि मामुक्तवान् पिता} %2-4-36

\twolineshloka
{यानि यान्यत्र योग्यानि श्वोभाविन्यभिषेचने}
{तानि मे मङ्गलान्यद्य वैदेह्याश्चैव कारय} %2-4-37

\twolineshloka
{एतच्छ्रुत्वा तु कौसल्या चिरकालाभिकाङ्क्षितम्}
{हर्षबाष्पाकुलं वाक्यमिदं राममभाषत} %2-4-38

\twolineshloka
{वत्स राम चिरं जीव हतास्ते परिपन्थिनः}
{ज्ञातीन् मे त्वं श्रिया युक्तः सुमित्रायाश्च नन्दय} %2-4-39

\twolineshloka
{कल्याणे बत नक्षत्रे मया जातोऽसि पुत्रक}
{येन त्वया दशरथो गुणैराराधितः पिता} %2-4-40

\twolineshloka
{अमोघं बत मे क्षान्तं पुरुषे पुष्करेक्षणे}
{येयमिक्ष्वाकुराजश्रीः पुत्र त्वां संश्रयिष्यति} %2-4-41

\twolineshloka
{इत्येवमुक्तो मात्रा तु रामो भ्रातरमब्रवीत्}
{प्राञ्जलिं प्रह्वमासीनमभिवीक्ष्य स्मयन्निव} %2-4-42

\twolineshloka
{लक्ष्मणेमां मया सार्धं प्रशाधि त्वं वसुन्धराम्}
{द्वितीयं मेऽन्तरात्मानं त्वामियं श्रीरुपस्थिता} %2-4-43

\twolineshloka
{सौमित्रे भुङ्क्ष्व भोगांस्त्वमिष्टान् राज्यफलानिच}
{जीवितं चापि राज्यं च त्वदर्थमभिकामये} %2-4-44

\twolineshloka
{इत्युक्त्वा लक्ष्मणं रामो मातरावभिवाद्य च}
{अभ्यनुज्ञाप्य सीतां च ययौ स्वं च निवेशनम्} %2-4-45


॥इत्यार्षे श्रीमद्रामायणे वाल्मीकीये आदिकाव्ये अयोध्याकाण्डे मात्राशीः परिग्रहः नाम चतुर्थः सर्गः ॥२-४॥
