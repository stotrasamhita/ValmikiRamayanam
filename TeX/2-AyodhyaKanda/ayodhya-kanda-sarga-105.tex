\sect{पञ्चाधिकशततमः सर्गः — रामवाक्यम्}

\twolineshloka
{ततः पुरुषसिंहानां वृतानां तैः सुहृद्गणैः}
{शोचतामेव रजनी दुःखेन व्यत्यवर्त्तत} %2-105-1

\twolineshloka
{रजन्यां सुप्रभातायां भ्रातरस्ते सुहृद्वृताः}
{मन्दाकिन्यां हुतं जप्यं कृत्वा राममुपागमन्} %2-105-2

\twolineshloka
{तूष्णीं ते समुपासीना न कश्चित्किञ्चिदब्रवीत्}
{भरतस्तु सुहृन्मध्ये रामं वचनमब्रवीत्} %2-105-3

\twolineshloka
{सान्त्विता मामिका माता दत्तं राज्यमिदं मम}
{तद्ददामि तवैवाहं भुङ्क्ष्व राज्यमकण्टकम्} %2-105-4

\twolineshloka
{महतेवाम्बुवेगेन भिन्नः सेतुर्जलागमे}
{दुरावारं त्वदन्येन राज्यखण्डमिदं महत्} %2-105-5

\twolineshloka
{गतिं खर इवाश्वस्य तार्क्ष्यस्येव पतत्ऺित्रणः}
{अनुगन्तुं न शक्तिर्मे गतिं तव महीपते} %2-105-6

\twolineshloka
{सुजीवं नित्यशस्तस्य यः परैरुपजीव्यते}
{राम तेन तु दुर्जीवं यः परानुपजीवति} %2-105-7

\twolineshloka
{यथा तु रोपितो वृक्षः पुरुषेण विवर्द्धितः}
{ह्रस्वकेन दुरारोहो रूढस्कन्धो महाद्रुमः} %2-105-8

\twolineshloka
{स यथा पुष्पितो भूत्वा फलानि न विदर्शयेत्}
{स तां नानुभवेत्प्रीतिं यस्य हेतोः प्ररोपितः} %2-105-9

\twolineshloka
{एषोपमा महाबाहो तमर्थं वेत्तुमर्हसि}
{यदि त्वमस्मान् वृषभो भर्त्ता भृत्यान्न शाधि हि} %2-105-10

\twolineshloka
{श्रेणयस्त्वां महाराज पश्यन्त्वग्र्याश्च सर्वशः}
{प्रतपन्तमिवादित्यं राज्ये स्थितमरिन्दमम्} %2-105-11

\twolineshloka
{तवाऽनुयाने काकुत्स्थ मत्ता नर्दन्तु कुञ्जराः}
{अन्तःपुरगता नार्यो नन्दन्तु सुसमाहिताः} %2-105-12

\twolineshloka
{तस्य साध्वित्यमन्यन्त नागरा विविधा जनाः}
{भरतस्य वचः श्रुत्वा रामं प्रत्यनुयाचतः} %2-105-13

\twolineshloka
{तमेवं दुःखितं प्रेक्ष्य विलपन्तं यशस्विनम्}
{रामः कृतात्मा भरतं समाश्वासय दात्मवान्} %2-105-14

\twolineshloka
{नात्मनः कामकारोऽस्ति पुरुषोऽयमनीश्वरः}
{इतश्चेतरतश्चैनं कृतान्तः परिकर्षति} %2-105-15

\twolineshloka
{सर्वे क्षयान्ता निचयाः पतनान्ताः समुच्छ्रयाः}
{संयोगा विप्रयोगान्ता मरणान्तं च जीवितम्} %2-105-16

\twolineshloka
{यथा फलानां पक्वानां नान्यत्र पतनाद्भयम्}
{एवं नरस्य जातस्य नान्यत्र मरणाद्भयम्} %2-105-17

\twolineshloka
{यथागारं दृढस्थूणं जीर्णं भूत्वाऽवसीदति}
{तथैव सीदन्ति नरा जरामृत्युवशङ्गताः} %2-105-18

\twolineshloka
{अत्येति रजनी या तु सा न प्रतिनिवर्त्तते}
{यात्येव यमुना पूर्णा समुद्रमुदकाकुलम्} %2-105-19

\twolineshloka
{अहोरात्राणि गच्छन्ति सर्वेषां प्राणिनामिह}
{आयूंषि क्षपयन्त्याशु ग्रीष्मे जलमिवांशवः} %2-105-20

\twolineshloka
{आत्मानमनुशोच त्वं किमन्यमनुशोचसि}
{आयुस्ते हीयते यस्य स्थितस्य च गतस्य च} %2-105-21

\twolineshloka
{सहैव मृत्युर्व्रजति सह मृत्युर्निषीदति}
{गत्वा सुदीर्घमध्वानं सहमृत्युर्निवर्तते} %2-105-22

\twolineshloka
{गात्रेषु वलयः प्राप्ताः श्वेताश्चैव शिरोरुहाः}
{जरया पुरुषो जीर्णः किं हि कृत्वा प्रभावयेत्} %2-105-23

\twolineshloka
{नन्दन्त्युदित आदित्ये नन्दन्त्यस्तमिते रवौ}
{आत्मनो नावबुध्यन्ते मनुष्या जीवितक्षयम्} %2-105-24

\twolineshloka
{हृष्यन्त्यृतुमखं दृष्ट्वा नवं नवमिहागतम्}
{ऋतूनां परिवर्त्तेन प्राणिनां प्राणसंक्षयः} %2-105-25

\twolineshloka
{यथा काष्ठं च काष्ठं च समेयातां महार्णवे}
{समेत्य च व्यपेयातां कालमासाद्य कञ्चन} %2-105-26

\twolineshloka
{एवं भार्याश्च पुत्राश्च ज्ञातयश्च घनानि च}
{समेत्य व्यवधावन्ति ध्रुवो ह्येषां विनाभवः} %2-105-27

\twolineshloka
{नात्र कश्चिद्यथाभावं प्राणी समभिवर्त्तते}
{तेन तस्मिन्न सामर्थ्यं प्रेतस्यास्त्यनुशोचतः} %2-105-28

\twolineshloka
{यथा हि सार्थं गच्छन्तं ब्रूयात् कश्चित् पथि स्थितः}
{अहमप्यागमिष्यामि पृष्ठतो भवतामिति} %2-105-29

\twolineshloka
{एवं पूर्वैर्गतो मार्गः पितृपैतामहो ध्रुवः}
{तमापन्नः कथं शोचेद्यस्ऺय नास्ति व्यतिक्रमः} %2-105-30

\twolineshloka
{वयसः पतमानस्य स्रोतसो वाऽनिवर्तिनः}
{आत्मा सुखे नियोक्तव्यः सुखभाजः प्रजाः स्मृताः} %2-105-31

\twolineshloka
{धर्मात्मा स शुभैः कृत्स्नैः क्रतुभिश्चाप्तदक्षिणैः}
{धूतपापो गतः स्वर्गं पिता नः पृथिवीपतिः} %2-105-32

\twolineshloka
{भृत्यानां भरणात् सम्यक् प्रजानां परिपालनात्}
{अर्थादानाच्च धर्मेण पिता नस्त्रिदिवं गतः} %2-105-33

\twolineshloka
{कर्मभिस्तु शुभैरिष्टैः क्रतुभिश्चाप्तदक्षिणैः}
{स्वर्गं दशरथः प्राप्तः पिता नः पृथिवीपतिः} %2-105-34

\twolineshloka
{इष्ट्वा बहुविधैर्यज्ञैर्भोगांश्चावाप्य पुष्कलान्}
{उत्तमं चायुरासाद्य स्वर्गतः पृथिवीपतिः} %2-105-35

\twolineshloka
{आयुरुत्तममासाद्य भोगानपि च राघवः}
{स न शोच्यः पिता तातः स्वर्गतः सत्कृतः सताम्} %2-105-36

\twolineshloka
{स जीर्णं मानुषं देहं परित्यज्य पिता हि नः}
{दैवीमृद्धिमनुप्राप्तो ब्रह्मलोकविहारिणीम्} %2-105-37

\twolineshloka
{तं तु नैवंविधः कश्चित् प्राज्ञः शोचितुमर्हति}
{तद्विधो यद्विधश्चापि श्रुतवान् बुद्धिमत्तरः} %2-105-38

\twolineshloka
{एते बहुविधाः शोका विलापरुदिते तथा}
{वर्जनीया हि धीरेण सर्वावस्थासु धीमता} %2-105-39

\twolineshloka
{स स्वस्थो भव माशोचीर्यात्वा चावस तां पुरीम्}
{तथा पित्रा नियुक्तोऽसि वशिना वदतां वर} %2-105-40

\twolineshloka
{यत्राहमपि तेनैव नियुक्तः पुण्यकर्मणा}
{तत्रैवाहं करिष्यामि पितुरार्य्यस्य शासनम्} %2-105-41

\twolineshloka
{न मया शासनं तस्य त्यक्तुं न्याय्यमरिन्दम}
{तत् त्वयापि सदा मान्यं स वै बन्धुः स नः पिता} %2-105-42

\twolineshloka
{तद्वचः पितुरेवाहं सम्मतं धर्मचारिणः}
{कर्मणा पालयिष्यामि वनवासेन राघव} %2-105-43

\twolineshloka
{धीर्मिकेणानृशंसेन नरेण गुरुवर्त्तिना}
{भवितव्यं नरव्याघ्र परलोकं जिगीषता} %2-105-44

\twolineshloka
{आत्मानमनुतिष्ठ त्वं स्वभावेन नरर्षभ}
{निशाम्य तु शुभं वृत्तं पितुर्दशरथस्य नः} %2-105-45

\twolineshloka
{इत्येवमुक्त्वा वचनं महात्मा पितुर्निदेशप्रतिपालनार्थम्}
{यवीयसं भ्रातरमर्थवच्च प्रभुर्मुहूर्ताद्विरराम रामः} %2-105-46


॥इत्यार्षे श्रीमद्रामायणे वाल्मीकीये आदिकाव्ये अयोध्याकाण्डे रामवाक्यम् नाम पञ्चाधिकशततमः सर्गः ॥२-१०५॥
