\sect{द्वादशः सर्गः — कैकेयीनिवर्तनप्रयासः}

\twolineshloka
{ततः श्रुत्वा महाराजः कैकेय्या दारुणं वचः}
{चिन्तामभिसमापेदे मुहूर्तं प्रतताप च} %2-12-1

\twolineshloka
{किं नु मेऽयं दिवास्वप्नश्चित्तमोहोऽपि वा मम}
{अनुभूतोपसर्गो वा मनसो वाप्युपद्रवः} %2-12-2

\twolineshloka
{इति संचिन्त्य तद् राजा नाध्यगच्छत् तदासुखम्}
{प्रतिलभ्य ततः संज्ञां कैकेयीवाक्यतापितः} %2-12-3

\twolineshloka
{व्यथितो विक्लवश्चैव व्याघ्रीं दृष्ट्वा यथा मृगः}
{असंवृतायामासीनो जगत्यां दीर्घमुच्छ्वसन्} %2-12-4

\twolineshloka
{मण्डले पन्नगो रुद्धो मन्त्रैरिव महाविषः}
{अहो धिगिति सामर्षो वाचमुक्त्वा नराधिपः} %2-12-5

\twolineshloka
{मोहम् आपेदिवान् भूयः शोकोपहतचेतनः}
{चिरेण तु नृपः संज्ञां प्रतिलभ्य सुदुःखितः} %2-12-6

\twolineshloka
{कैकेयीमब्रवीत् क्रुद्धो निर्दहन्निव तेजसा}
{नृशंसे दुष्टचारित्रे कुलस्यास्य विनाशिनि} %2-12-7

\twolineshloka
{किं कृतं तव रामेण पापे पापं मयापि वा}
{सदा ते जननीतुल्यां वृत्तिं वहति राघवः} %2-12-8

\twolineshloka
{तस्यैवं त्वमनर्थाय किंनिमित्तमिहोद्यता}
{त्वं मयाऽऽत्मविनाशाय भवनं स्वं निवेशिता} %2-12-9

\twolineshloka
{अविज्ञानान्नृपसुता व्याला तीक्ष्णविषा यथा}
{जीवलोको यदा सर्वो रामस्याह गुणस्तवम्} %2-12-10

\twolineshloka
{अपराधं कमुद्दिश्य त्यक्ष्यामीष्टमहं सुतम्}
{कौसल्यां च सुमित्रां च त्यजेयमपि वा श्रियम्} %2-12-11

\twolineshloka
{जीवितं चात्मनो रामं न त्वेव पितृवत्सलम्}
{परा भवति मे प्रीतिर्दृष्ट्वा तनयमग्रजम्} %2-12-12

\twolineshloka
{अपश्यतस्तु मे रामं नष्टं भवति चेतनम्}
{तिष्ठेल्लोको विना सूर्यं सस्यं वा सलिलं विना} %2-12-13

\twolineshloka
{न तु रामं विना देहे तिष्ठेत्तु मम जीवितम्}
{तदलं त्यज्यतामेष निश्चयः पापनिश्चये} %2-12-14

\twolineshloka
{अपि ते चरणौ मूर्ध्ना स्पृशाम्येष प्रसीद मे}
{किमर्थं चिन्तितं पापे त्वया परमदारुणम्} %2-12-15

\twolineshloka
{अथ जिज्ञाससे मां त्वं भरतस्य प्रियाप्रिये}
{अस्तु यत्तत्त्वया पूर्वं व्याहृतं राघवं प्रति} %2-12-16

\twolineshloka
{स मे ज्येष्ठसुतः श्रीमान् धर्मज्येष्ठ इतीव मे}
{तत् त्वया प्रियवादिन्या सेवार्थं कथितं भवेत्} %2-12-17

\twolineshloka
{तच्छ्रुत्वा शोकसंतप्ता संतापयसि मां भृशम्}
{आविष्टासि गृहे शून्ये सा त्वं परवशं गता} %2-12-18

\twolineshloka
{इक्ष्वाकूणां कुले देवि सम्प्राप्तः सुमहानयम्}
{अनयो नयसम्पन्ने यत्र ते विकृता मतिः} %2-12-19

\twolineshloka
{नहि किंचिदयुक्तं वा विप्रियं वा पुरा मम}
{अकरोस्त्वं विशालाक्षि तेन न श्रद्दधामि ते} %2-12-20

\twolineshloka
{ननु ते राघवस्तुल्यो भरतेन महात्मना}
{बहुशो हि स्म बाले त्वं कथाः कथयसे मम} %2-12-21

\twolineshloka
{तस्य धर्मात्मनो देवि वने वासं यशस्विनः}
{कथं रोचयसे भीरु नव वर्षाणि पञ्च च} %2-12-22

\twolineshloka
{अत्यन्तसुकुमारस्य तस्य धर्मे कृतात्मनः}
{कथं रोचयसे वासमरण्ये भृशदारुणे} %2-12-23

\twolineshloka
{रोचयस्यभिरामस्य रामस्य शुभलोचने}
{तव शुश्रूषमाणस्य किमर्थं विप्रवासनम्} %2-12-24

\twolineshloka
{रामो हि भरताद् भूयस्तव शुश्रूषते सदा}
{विशेषं त्वयि तस्मात् तु भरतस्य न लक्षये} %2-12-25

\twolineshloka
{शुश्रूषां गौरवं चैव प्रमाणं वचनक्रियाम्}
{कस्तु भूयस्तरं कुर्यादन्यत्र पुरुषर्षभात्} %2-12-26

\twolineshloka
{बहूनां स्त्रीसहस्राणां बहूनां चोपजीविनाम्}
{परिवादोऽपवादो वा राघवे नोपपद्यते} %2-12-27

\twolineshloka
{सान्त्वयन् सर्वभूतानि रामः शुद्धेन चेतसा}
{गृह्णाति मनुजव्याघ्रः प्रियैर्विषयवासिनः} %2-12-28

\twolineshloka
{सत्येन लोकाञ् जयति द्विजान् दानेन राघवः}
{गुरूञ् छुश्रूषया वीरो धनुषा युधि शात्रवान्} %2-12-29

\twolineshloka
{सत्यं दानं तपस्त्यागो मित्रता शौचमार्जवम्}
{विद्या च गुरुशुश्रूषा ध्रुवाण्येतानि राघवे} %2-12-30

\twolineshloka
{तस्मिन्नार्जवसम्पन्ने देवि देवोपमे कथम्}
{पापमाशंससे रामे महर्षिसमतेजसि} %2-12-31

\twolineshloka
{न स्मराम्यप्रियं वाक्यं लोकस्य प्रियवादिनः}
{स कथं त्वत्कृते रामं वक्ष्यामि प्रियमप्रियम्} %2-12-32

\twolineshloka
{क्षमा यस्मिंस्तपस्त्यागः सत्यं धर्मः कृतज्ञता}
{अप्यहिंसा च भूतानां तमृते कागतिर्मम} %2-12-33

\twolineshloka
{मम वृद्धस्य कैकेयि गतान्तस्य तपस्विनः}
{दीनं लालप्यमानस्य कारुण्यं कर्तुमर्हसि} %2-12-34

\twolineshloka
{पृथिव्यां सागरान्तायां यत् किंचिदधिगम्यते}
{तत् सर्वं तव दास्यामि मा च त्वं मन्युमाविश} %2-12-35

\twolineshloka
{अञ्जलिं कुर्मि कैकेयि पादौ चापि स्पृशामि ते}
{शरणं भव रामस्य माधर्मो मामिह स्पृशेत्} %2-12-36

\twolineshloka
{इति दुःखाभिसंतप्तं विलपन्तमचेतनम्}
{घूर्णमानं महाराजं शोकेन समभिप्लुतम्} %2-12-37

\twolineshloka
{पारं शोकार्णवस्याशु प्रार्थयन्तं पुनः पुनः}
{प्रत्युवाचाथ कैकेयी रौद्रा रौद्रतरं वचः} %2-12-38

\twolineshloka
{यदि दत्त्वा वरौ राजन् पुनः प्रत्यनुतप्यसे}
{धार्मिकत्वं कथं वीर पृथिव्यां कथयिष्यसि} %2-12-39

\twolineshloka
{यदा समेता बहवस्त्वया राजर्षयः सह}
{कथयिष्यन्ति धर्मज्ञ तत्र किं प्रतिवक्ष्यसि} %2-12-40

\twolineshloka
{यस्याः प्रसादे जीवामि या च मामभ्यपालयत्}
{तस्याः कृता मया मिथ्या कैकेय्या इति वक्ष्यसि} %2-12-41

\twolineshloka
{किल्बिषं त्वं नरेन्द्राणां करिष्यसि नराधिप}
{यो दत्त्वा वरमद्यैव पुनरन्यानि भाषसे} %2-12-42

\twolineshloka
{शैब्यः श्येनकपोतीये स्वमांसं पक्षिणे ददौ}
{अलर्कश्चक्षुषी दत्त्वा जगाम गतिमुत्तमाम्} %2-12-43

\twolineshloka
{सागरः समयं कृत्वा न वेलामतिवर्तते}
{समयं मानृतं कार्षीः पूर्ववृत्तमनुस्मरन्} %2-12-44

\twolineshloka
{स त्वं धर्मं परित्यज्य रामं राज्येऽभिषिच्य च}
{सह कौसल्यया नित्यं रन्तुमिच्छसि दुर्मते} %2-12-45

\twolineshloka
{भवत्वधर्मो धर्मो वा सत्यं वा यदि वानृतम्}
{यत्त्वया संश्रुतं मह्यं तस्य नास्ति व्यतिक्रमः} %2-12-46

\twolineshloka
{अहं हि विषमद्यैव पीत्वा बहु तवाग्रतः}
{पश्यतस्ते मरिष्यामि रामो यद्यभिषिच्यते} %2-12-47

\twolineshloka
{एकाहमपि पश्येयं यद्यहं राममातरम्}
{अञ्जलिं प्रतिगृह्णन्तीं श्रेयो ननु मृतिर्मम} %2-12-48

\twolineshloka
{भरतेनात्मना चाहं शपे ते मनुजाधिप}
{यथा नान्येन तुष्येयमृते रामविवासनात्} %2-12-49

\twolineshloka
{एतावदुक्त्वा वचनं कैकेयी विरराम ह}
{विलपन्तं च राजानं न प्रतिव्याजहार सा} %2-12-50

\twolineshloka
{श्रुत्वा तु राजा कैकेय्या वाक्यं परमशोभनम्}
{रामस्य च वने वासमैश्वर्यं भरतस्य च} %2-12-51

\twolineshloka
{नाभ्यभाषत कैकेयीं मुहूर्तं व्याकुलेन्द्रियः}
{प्रैक्षतानिमिषो देवीं प्रियामप्रियवादिनीम्} %2-12-52

\twolineshloka
{तां हि वज्रसमां वाचमाकर्ण्य हृदयाप्रियाम्}
{दुःखशोकमयीं श्रुत्वा राजा न सुखितोऽभवत्} %2-12-53

\twolineshloka
{स देव्या व्यवसायं च घोरं च शपथं कृतम्}
{ध्यात्वा रामेति निःश्वस्य च्छिन्नस्तरुरिवापतत्} %2-12-54

\twolineshloka
{नष्टचित्तो यथोन्मत्तो विपरीतो यथातुरः}
{हृततेजा यथा सर्पो बभूव जगतीपतिः} %2-12-55

\twolineshloka
{दीनयाऽऽतुरया वाचा इति होवाच कैकयीम्}
{अनर्थमिममर्थाभं केन त्वमुपदेशिता} %2-12-56

\twolineshloka
{भूतोपहतचित्तेव ब्रुवन्ती मां न लज्जसे}
{शीलव्यसनमेतत् ते नाभिजानाम्यहं पुरा} %2-12-57

\twolineshloka
{बालायास्तत् त्विदानीं ते लक्षये विपरीतवत्}
{कुतो वा ते भयं जातं या त्वमेवंविधं वरम्} %2-12-58

\twolineshloka
{राष्ट्रे भरतमासीनं वृणीषे राघवं वने}
{विरमैतेन भावेन त्वमेतेनानृतेन च} %2-12-59

\twolineshloka
{यदि भर्तुः प्रियं कार्यं लोकस्य भरतस्य च}
{नृशंसे पापसंकल्पे क्षुद्रे दुष्कृतकारिणि} %2-12-60

\twolineshloka
{किं नु दुःखमलीकं वा मयि रामे च पश्यसि}
{न कथंचिदृते रामाद् भरतो राज्यमावसेत्} %2-12-61

\twolineshloka
{रामादपि हि तं मन्ये धर्मतो बलवत्तरम्}
{कथं द्रक्ष्यामि रामस्य वनं गच्छेति भाषिते} %2-12-62

\twolineshloka
{मुखवर्णं विवर्णं तु यथैवेन्दुमुपप्लुतम्}
{तां तु मे सुकृतां बुद्धिं सुहृद्भिः सह निश्चिताम्} %2-12-63

\twolineshloka
{कथं द्रक्ष्याम्यपावृत्तां परैरिव हतां चमूम्}
{किं मां वक्ष्यन्ति राजानो नानादिग्भ्यः समागताः} %2-12-64

\twolineshloka
{बालो बतायमैक्ष्वाकश्चिरं राज्यमकारयत्}
{यदा हि बहवो वृद्धा गुणवन्तो बहुश्रुताः} %2-12-65

\twolineshloka
{परिप्रक्ष्यन्ति काकुत्स्थं वक्ष्यामीह कथं तदा}
{कैकेय्या क्लिश्यमानेन पुत्रः प्रव्राजितो मया} %2-12-66

\twolineshloka
{यदि सत्यं ब्रवीम्येतत् तदसत्यं भविष्यति}
{किं मां वक्ष्यति कौसल्या राघवे वनमास्थिते} %2-12-67

\twolineshloka
{किं चैनां प्रतिवक्ष्यामि कृत्वा विप्रियमीदृशम्}
{यदा यदा च कौसल्या दासीव च सखीव च} %2-12-68

\twolineshloka
{भार्यावद् भगिनीवच्च मातृवच्चोपतिष्ठति}
{सततं प्रियकामा मे प्रियपुत्रा प्रियंवदा} %2-12-69

\twolineshloka
{न मया सत्कृता देवी सत्कारार्हा कृते तव}
{इदानीं तत्तपति मां यन्मया सुकृतं त्वयि} %2-12-70

\twolineshloka
{अपथ्यव्यञ्जनोपेतं भुक्तमन्नमिवातुरम्}
{विप्रकारं च रामस्य सम्प्रयाणं वनस्य च} %2-12-71

\twolineshloka
{सुमित्रा प्रेक्ष्य वै भीता कथं मे विश्वसिष्यति}
{कृपणं बत वैदेही श्रोष्यति द्वयमप्रियम्} %2-12-72

\twolineshloka
{मां च पञ्चत्वमापन्नं रामं च वनमाश्रितम्}
{वैदेही बत मे प्राणान् शोचन्ती क्षपयिष्यति} %2-12-73

\twolineshloka
{हीना हिमवतः पार्श्वे किंनरेणेव किंनरी}
{नहि राममहं दृष्ट्वा प्रवसन्तं महावने} %2-12-74

\twolineshloka
{चिरं जीवितुमाशंसे रुदन्तीं चापि मैथिलीम्}
{सा नूनं विधवा राज्यं सपुत्रा कारयिष्यसि} %2-12-75

\twolineshloka
{सतीं त्वामहमत्यन्तं व्यवस्याम्यसतीं सतीम्}
{रूपिणीं विषसंयुक्तां पीत्वेव मदिरां नरः} %2-12-76

\twolineshloka
{अनृतैर्बत मां सान्त्वैः सान्त्वयन्ती स्म भाषसे}
{गीतशब्देन संरुध्य लुब्धो मृगमिवावधीः} %2-12-77

\twolineshloka
{अनार्य इति मामार्याः पुत्रविक्रायकं ध्रुवम्}
{विकरिष्यन्ति रथ्यासु सुरापं ब्राह्मणं यथा} %2-12-78

\twolineshloka
{अहो दुःखमहो कृच्छ्रं यत्र वाचः क्षमे तव}
{दुःखमेवंविधं प्राप्तं पुरा कृतमिवाशुभम्} %2-12-79

\twolineshloka
{चिरं खलु मया पापे त्वं पापेनाभिरक्षिता}
{अज्ञानादुपसम्पन्ना रज्जुरुद्बन्धनी यथा} %2-12-80

\twolineshloka
{रममाणस्त्वया सार्धं मृत्युं त्वां नाभिलक्षये}
{बालो रहसि हस्तेन कृष्णसर्पमिवास्पृशम्} %2-12-81

\twolineshloka
{तं तु मां जीवलोकोऽयं नूनमाक्रोष्टुमर्हति}
{मया ह्यपितृकः पुत्रः स महात्मा दुरात्मना} %2-12-82

\twolineshloka
{बालिशो बत कामात्मा राजा दशरथो भृशम्}
{स्त्रीकृते यः प्रियं पुत्रं वनं प्रस्थापयिष्यति} %2-12-83

\twolineshloka
{वेदैश्च ब्रह्मचर्यैश्च गुरुभिश्चोपकर्शितः}
{भोगकाले महत्कृच्छ्रं पुनरेव प्रपत्स्यते} %2-12-84

\twolineshloka
{नालं द्वितीयं वचनं पुत्रो मां प्रतिभाषितुम्}
{स वनं प्रव्रजेत्युक्तो बाढमित्येव वक्ष्यति} %2-12-85

\twolineshloka
{यदि मे राघवः कुर्याद् वनं गच्छेति चोदितः}
{प्रतिकूलं प्रियं मे स्यान्न तु वत्सः करिष्यति} %2-12-86

\twolineshloka
{राघवे हि वनं प्राप्ते सर्वलोकस्य धिक्कृतम्}
{मृत्युरक्षमणीयं मां नयिष्यति यमक्षयम्} %2-12-87

\twolineshloka
{मृते मयि गते रामे वनं मनुजपुङ्गवे}
{इष्टे मम जने शेषे किं पापं प्रतिपत्स्यसे} %2-12-88

\twolineshloka
{कौसल्या मां च रामं च पुत्रौ च यदि हास्यति}
{दुःखान्यसहती देवी मामेवानुगमिष्यति} %2-12-89

\twolineshloka
{कौसल्यां च सुमित्रां च मां च पुत्रैस्त्रिभिः सह}
{प्रक्षिप्य नरके सा त्वं कैकेयि सुखिता भव} %2-12-90

\twolineshloka
{मया रामेण च त्यक्तं शाश्वतं सत्कृतं गुणैः}
{इक्ष्वाकुकुलमक्षोभ्यमाकुलं पालयिष्यसि} %2-12-91

\twolineshloka
{प्रियं चेद् भरतस्यैतद् रामप्रव्राजनं भवेत्}
{मा स्म मे भरतः कार्षीत् प्रेतकृत्यं गतायुषः} %2-12-92

\twolineshloka
{मृते मयि गते रामे वनं पुरुषपुङ्गवे}
{सेदानीं विधवा राज्यं सपुत्रा कारयिष्यसि} %2-12-93

\threelineshloka
{त्वं राजपुत्रि दैवेन न्यवसो मम वेश्मनि}
{अकीर्तिश्चातुला लोके ध्रुवः परिभवश्च मे}
{सर्वभूतेषु चावज्ञा यथा पापकृतस्तथा} %2-12-94

\twolineshloka
{कथं रथैर्विभुर्यात्वा गजाश्वैश्च मुहुर्मुहुः}
{पद्भ्यां रामो महारण्ये वत्सो मे विचरिष्यति} %2-12-95

\twolineshloka
{यस्य चाहारसमये सूदाः कुण्डलधारिणः}
{अहंपूर्वाः पचन्ति स्म प्रसन्नाः पानभोजनम्} %2-12-96

\twolineshloka
{स कथं नु कषायाणि तिक्तानि कटुकानि च}
{भक्षयन् वन्यमाहारं सुतो मे वर्तयिष्यति} %2-12-97

\twolineshloka
{महार्हवस्त्रसम्बद्धो भूत्वा चिरसुखोचितः}
{काषायपरिधानस्तु कथं रामो भविष्यति} %2-12-98

\twolineshloka
{कस्येदं दारुणं वाक्यमेवंविधमपीरितम्}
{रामस्यारण्यगमनं भरतस्याभिषेचनम्} %2-12-99

\twolineshloka
{धिगस्तु योषितो नाम शठाः स्वार्थपरायणाः}
{न ब्रवीमि स्त्रियः सर्वा भरतस्यैव मातरम्} %2-12-100

\twolineshloka
{अनर्थभावेऽर्थपरे नृशंसे ममानुतापाय निवेशितासि}
{किमप्रियं पश्यसि मन्निमित्तं हितानुकारिण्यथवापि रामे} %2-12-101

\twolineshloka
{परित्यजेयुः पितरोऽपि पुत्रान् भार्याः पतींश्चापि कृतानुरागाः}
{कृत्स्नं हि सर्वं कुपितं जगत् स्याद् दृष्ट्वैव रामं व्यसने निमग्नम्} %2-12-102

\twolineshloka
{अहं पुनर्देवकुमाररूपमलंकृतं तं सुतमाव्रजन्तम्}
{नन्दामि पश्यन्निव दर्शनेन भवामि दृष्ट्वैव पुनर्युवेव} %2-12-103

\twolineshloka
{विना हि सूर्येण भवेत् प्रवृत्तिरवर्षता वज्रधरेण वापि}
{रामं तु गच्छन्तमितः समीक्ष्य जीवेन्न कश्चित्त्विति चेतना मे} %2-12-104

\twolineshloka
{विनाशकामामहिताममित्रामावासयं मृत्युमिवात्मनस्त्वाम्}
{चिरं बताङ्केन धृतासि सर्पी महाविषा तेन हतोऽस्मि मोहात्} %2-12-105

\twolineshloka
{मया च रामेण सलक्ष्मणेन प्रशास्तु हीनो भरतस्त्वया सह}
{पुरं च राष्ट्रं च निहत्य बान्धवान् ममाहितानां च भवाभिहर्षिणी} %2-12-106

\twolineshloka
{नृशंसवृत्ते व्यसनप्रहारिणि प्रसह्य वाक्यं यदिहाद्य भाषसे}
{न नाम ते तेन मुखात् पतन्त्यधो विशीर्यमाणा दशनाः सहस्रधा} %2-12-107

\twolineshloka
{न किंचिदाहाहितमप्रियं वचो न वेत्ति रामः परुषाणि भाषितुम्}
{कथं तु रामे ह्यभिरामवादिनि ब्रवीषि दोषान् गुणनित्यसम्मते} %2-12-108

\twolineshloka
{प्रताम्य वा प्रज्वल वा प्रणश्य वा सहस्रशो वा स्फुटितां महीं व्रज}
{न ते करिष्यामि वचः सुदारुणं ममाहितं केकयराजपांसने} %2-12-109

\twolineshloka
{क्षुरोपमां नित्यमसत्प्रियंवदां प्रदुष्टभावां स्वकुलोपघातिनीम्}
{न जीवितुं त्वां विषहेऽमनोरमां दिधक्षमाणां हृदयं सबन्धनम्} %2-12-110

\twolineshloka
{न जीवितं मेऽस्ति कुतः पुनः सुखं विनात्मजेनात्मवतां कुतो रतिः}
{ममाहितं देवि न कर्तुमर्हसि स्पृशामि पादावपि ते प्रसीद मे} %2-12-111

\twolineshloka
{स भूमिपालो विलपन्ननाथवत् स्त्रिया गृहीतो हृदयेऽतिमात्रया}
{पपात देव्याश्चरणौ प्रसारितावुभावसम्प्राप्य यथाऽऽतुरस्तथा} %2-12-112


॥इत्यार्षे श्रीमद्रामायणे वाल्मीकीये आदिकाव्ये अयोध्याकाण्डे कैकेयीनिवर्तनप्रयासः नाम द्वादशः सर्गः ॥२-१२॥
