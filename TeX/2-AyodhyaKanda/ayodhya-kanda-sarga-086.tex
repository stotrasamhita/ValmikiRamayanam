\sect{षडशीतितमः सर्गः — गुहवाक्यम्}

\onelineshloka
{भरताय अप्रमेयाय गुहो गहन गोचरः} %2-86-1

\twolineshloka
{तम् जाग्रतम् गुणैर् युक्तम् वर चाप इषु धारिणम्}
{भ्रातृ गुप्त्य् अर्थम् अत्यन्तम् अहम् लक्ष्मणम् अब्रवम्} %2-86-2

\twolineshloka
{इयम् तात सुखा शय्या त्वद् अर्थम् उपकल्पिता}
{प्रत्याश्वसिहि शेष्व अस्याम् सुखम् राघव नन्दन} %2-86-3

\twolineshloka
{उचितो अयम् जनः सर्वे दुह्खानाम् त्वम् सुख उचितः}
{धर्म आत्ममः तस्य गुप्त्य् अर्थम् जागरिष्यामहे वयम्} %2-86-4

\twolineshloka
{न हि रामात् प्रियतरो मम अस्ति भुवि कश्चन}
{मा उत्सुको भूर् ब्रवीम्य् एतद् अप्य् असत्यम् तव अग्रतः} %2-86-5

\twolineshloka
{अस्य प्रसादाद् आशम्से लोके अस्मिन् सुमहद् यशः}
{धर्म अवाप्तिम् च विपुलाम् अर्थ अवाप्तिम् च केवलाम्} %2-86-6

\twolineshloka
{सो अहम् प्रिय सखम् रामम् शयानम् सह सीतया}
{रक्षिष्यामि धनुष् पाणिः सर्वैः स्वैर् ज्नातिभिः सह} %2-86-7

\twolineshloka
{न हि मे अविदितम् किम्चिद् वने अस्मिमः चरतः सदा}
{चतुर् अन्गम् ह्य् अपि बलम् प्रसहेम वयम् युधि} %2-86-8

\twolineshloka
{एवम् अस्माभिर् उक्तेन लक्ष्मणेन महात्मना}
{अनुनीता वयम् सर्वे धर्मम् एव अनुपश्यता} %2-86-9

\twolineshloka
{कथम् दाशरथौ भूमौ शयाने सह सीतया}
{शक्या निद्रा मया लब्धुम् जीवितम् वा सुखानि वा} %2-86-10

\twolineshloka
{यो न देव असुरैः सर्वैः शक्यः प्रसहितुम् युधि}
{तम् पश्य गुह सम्विष्टम् तृणेषु सह सीतया} %2-86-11

\twolineshloka
{महता तपसा लब्धो विविधैः च परिश्रमैः}
{एको दशरथस्य एष पुत्रः सदृश लक्षणः} %2-86-12

\twolineshloka
{अस्मिन् प्रव्राजिते राजा न चिरम् वर्तयिष्यति}
{विधवा मेदिनी नूनम् क्षिप्रम् एव भविष्यति} %2-86-13

\twolineshloka
{विनद्य सुमहा नादम् श्रमेण उपरताः स्त्रियः}
{निर्घोष उपरतम् नूनम् अद्य राज निवेशनम्} %2-86-14

\twolineshloka
{कौसल्या चैव राजा च तथा एव जननी मम}
{न आशम्से यदि ते सर्वे जीवेयुः शर्वरीम् इमाम्} %2-86-15

\twolineshloka
{जीवेद् अपि हि मे माता शत्रुघ्नस्य अन्ववेक्षया}
{दुह्खिता या तु कौसल्या वीरसूर् विनशिष्यति} %2-86-16

\twolineshloka
{अतिक्रान्तम् अतिक्रान्तम् अनवाप्य मनो रथम्}
{राज्ये रामम् अनिक्षिप्य पिता मे विनशिष्यति} %2-86-17

\twolineshloka
{सिद्ध अर्थाः पितरम् वृत्तम् तस्मिन् काले ह्य् उपस्थिते}
{प्रेत कार्येषु सर्वेषु सम्स्करिष्यन्ति भूमिपम्} %2-86-18

\twolineshloka
{रम्य चत्वर सम्स्थानाम् सुविभक्त महा पथाम्}
{हर्म्य प्रासाद सम्पन्नाम् सर्व रत्न विभूषिताम्} %2-86-19

\twolineshloka
{गज अश्व रथ सम्बाधाम् तूर्य नाद विनादिताम्}
{सर्व कल्याण सम्पूर्णाम् हृष्ट पुष्ट जन आकुलाम्} %2-86-20

\twolineshloka
{आराम उद्यान सम्पूर्णाम् समाज उत्सव शालिनीम्}
{सुखिता विचरिष्यन्ति राज धानीम् पितुर् मम} %2-86-21

\twolineshloka
{अपि सत्य प्रतिज्नेन सार्धम् कुशलिना वयम्}
{निवृत्ते समये ह्य् अस्मिन् सुखिताः प्रविशेमहि} %2-86-22

\twolineshloka
{परिदेवयमानस्य तस्य एवम् सुमहात्मनः}
{तिष्ठतो राज पुत्रस्य शर्वरी सा अत्यवर्तत} %2-86-23

\twolineshloka
{प्रभाते विमले सूर्ये कारयित्वा जटा उभौ}
{अस्मिन् भागीरथी तीरे सुखम् सम्तारितौ मया} %2-86-24

\fourlineindentedshloka
{जटा धरौ तौ द्रुम चीर वाससौ}
{महा बलौ कुन्जर यूथप उपमौ}
{वर इषु चाप असि धरौ परम् तपौ}
{व्यवेक्षमाणौ सह सीतया गतौ} %2-86-25

\onelineshloka
{इति वाल्मीकि रामायणे आदि काव्ये अयोध्याकाण्डे षडशीतितमः सर्गः} %2-86-26


॥इत्यार्षे श्रीमद्रामायणे वाल्मीकीये आदिकाव्ये अयोध्याकाण्डे गुहवाक्यम् नाम षडशीतितमः सर्गः ॥२-८६॥
