\sect{नवाधिकशततमः सर्गः — सत्यप्रशंसा}

\twolineshloka
{जाबालेस्तु वचः श्रुत्वा रामः सत्यात्मनां वरः}
{उवाच परया भक्त्या स्वबुद्ध्या चाविपन्नया} %2-109-1

\twolineshloka
{भवान् मे प्रियकामार्थं वचनं यदिहोक्तवान्}
{अकार्य्यं कार्य्यसङ्काशमपथ्यं पथ्यसम्मितम्} %2-109-2

\twolineshloka
{निर्मर्यादस्तु पुरुषः पापाचारसमन्वितः}
{मानं न लभते सत्सु भिन्नचारित्रदर्शनः} %2-109-3

\twolineshloka
{कुलीनमकुलीनं वा वीरं पुरुषमानिनम्}
{चारित्रमेव व्याख्याति शुचिं वा यदि वाऽशुचिम्} %2-109-4

\twolineshloka
{अनार्यस्त्वार्यसङ्काशः शौचाद्धीनस्तथा शुचिः}
{लक्षण्यवदलक्षण्यो दुःशीलः शीलवानिव} %2-109-5

\twolineshloka
{अधर्मं धर्मवेषेण यदीमं लोकसङ्करम्}
{अभिपत्स्ये शुभं हित्वा क्रियाविधिविवर्जितम्} %2-109-6

\twolineshloka
{कश्चेतयानः पुरुषः कार्याकार्यविचक्षणः}
{बहुमंस्यति मां लोके दुर्वृत्तं लोकदूषणम्} %2-109-7

\twolineshloka
{कस्य धास्याम्यहं वृत्तं केन वा स्वर्गमाप्नुयाम्}
{अनया वर्त्तमानो हि वृत्त्या हीनप्रतिज्ञया} %2-109-8

\twolineshloka
{कामवृत्तस्त्वयं लोकः कृत्स्नः समुपवर्त्तते}
{यद्वृत्ताः सन्ति राजानस्तद्वृत्ताः सन्ति हि प्रजाः} %2-109-9

\twolineshloka
{सत्यमेवानृशंसञ्च राजवृत्तं सनातनम्}
{तस्मात्सत्यात्मकं राज्यं सत्ये लोकः प्रतिष्ठितः} %2-109-10

\twolineshloka
{ऋषयश्चैव देवाश्च सत्यमेव हि मेनिरे}
{सत्यवादी हि लोकेऽस्मिन् परमं गच्छति क्षयम्} %2-109-11

\twolineshloka
{उद्विजन्ते यथा सर्प्पान्नरादनृतवादिनः}
{धर्मः सत्यं परो लोके मूलं स्वर्गस्य चोच्यते} %2-109-12

\twolineshloka
{सत्यमेवेश्वरो लोके सत्यं पद्मा श्रिता सदा}
{सत्यमूलानि सर्वाणि सत्यान्नास्ति परं पदम्} %2-109-13

\twolineshloka
{दत्तमिष्टं हुतं चैव तप्तानि च तपांसि च}
{वेदाः सत्यप्रतिष्ठानास्तस्मात् सत्यपरो भवेत्} %2-109-14

\twolineshloka
{एकः पालयते लोकमेकः पालयते कुलम्}
{मज्जत्येको हि निरय एकः स्वर्गे महीयते} %2-109-15

\twolineshloka
{सोऽहं पितुर्नियोगन्तु किमर्थं नानुपालये}
{सत्यप्रतिश्रवः सत्यं सत्येन समयीकृतः} %2-109-16

\twolineshloka
{नैव लोभान्न मोहाद्वा न ह्यज्ञानात्तमोन्वितः}
{सेतुं सत्यस्य भेत्स्यामि गुरोः सत्यप्रतिश्रवः} %2-109-17

\twolineshloka
{असत्यसन्धस्य सतश्चलस्यास्थिरचेतसः}
{नैव देवा न पितरः प्रतीच्छन्तीति नः श्रुतम्} %2-109-18

\twolineshloka
{प्रत्यगात्ममिमं धर्मं सत्यं पश्याम्यहं स्वयम्}
{भारः सत्पुरुषाचीर्णस्तदर्थमभिमन्यते} %2-109-19

\twolineshloka
{क्षात्ऺत्रं धर्ममहं त्यक्ष्ये ह्यधर्मं धर्मसंहितम्}
{क्षुद्रैर्नृशंसैर्लुब्धैश्च सेवितं पापकर्मभिः} %2-109-20

\twolineshloka
{कायेन कुरुते पापं मनसा सम्प्रधार्य च}
{अनृतं जिह्वया चाह त्रिविधं कर्म पातकम्} %2-109-21

\twolineshloka
{भूमिः कीर्त्तिर्यशो लक्ष्मीः पुरुषं प्रार्थयन्ति हि}
{स्वर्गस्थं चानुपश्यन्ति सत्यमेव भजेत तत्} %2-109-22

\twolineshloka
{श्रेष्ठं ह्यनार्यमेव स्याद्यद्भवानवधार्य्य माम्}
{आह युक्तिकरैर्वाक्यैरिदं भद्रं कुरुष्व ह} %2-109-23

\twolineshloka
{कथं ह्यहं प्रतिज्ञाय वनवासमिमं गुरौ}
{भरतस्य करिष्यामि वचो हित्वा गुरोर्वचः} %2-109-24

\twolineshloka
{स्थिरा मया प्रतिज्ञाता प्रतिज्ञा गुरुसन्निधौ}
{प्रहृष्यमाणा सा देवी कैकेयी चाभवत्तदा} %2-109-25

\twolineshloka
{वनवासं वसन्नेवं शुचिर्नियतभोजनः}
{मूलैः पुष्पैः फलैः पुण्यैः पितऽन् देवांश्च तर्पयन्} %2-109-26

\twolineshloka
{सन्तुष्टपञ्चवर्गोऽहं लोकयात्रां प्रवर्त्तये}
{अकुहः श्रद्दधानस्सन् कार्य्याकार्य्यविचक्षणः} %2-109-27

\twolineshloka
{कर्मभूमिमिमां प्राप्य कर्त्तव्यं कर्म यच्छुभम्}
{अग्निर्वायुश्च सोमश्च कर्मणां फलभागिनः} %2-109-28

\twolineshloka
{शतं क्रतूनामाहृत्य देवराट् त्रिदिवङ्गतः}
{तपांस्युग्राणि चास्थाय दिवं याता महर्षयः} %2-109-29

\twolineshloka
{अमृष्यमाणः पुनरुग्रतेजा निशम्य तन्नास्तिकवाक्यहेतुम्}
{अथाब्रवीत्तं नृपतेस्तनूजो विगर्हमाणो वचनानि तस्य} %2-109-30

\twolineshloka
{सत्यं च धर्मं च पराक्रमं च भूतानुकम्पां प्रियवादिताञ्च}
{द्विजातिदेवातिथिपूजनं च पन्थानमाहुस्त्रिदिवस्य सन्तः} %2-109-31

\twolineshloka
{तेनैवमाज्ञाय यथावदर्थमेकोदयं सम्प्रतिपद्य विप्राः}
{धर्मं चरन्तः सकलं यथावत् कांक्षन्ति लोकागममप्रमत्ताः} %2-109-32

\twolineshloka
{निन्दाम्यहं कर्म पितुः कृतं तद्यस्त्वामगृह्णाद्विषमस्थबुद्धिम्}
{बुद्ध्यानयैवंविधया चरन्तं सुनास्तिकं धर्मपथादपेतम्} %2-109-33

\twolineshloka
{यथा हि चोरः स तथा हि बुद्धस्तथागतं नास्तिकमत्र विद्धि}
{तस्माद्धि यः शङ्क्यतमः प्रजानां न नास्तिकेनाभिमुखो बुधः स्यात्} %2-109-34

\twolineshloka
{त्वत्तो जनाः पूर्वतरे वराश्च शुभानि कर्माणि बहूनि चक्रुः}
{जित्वा सदेमं च परञ्च लोकं तस्माद्द्विजाः स्वस्ति हुतं कृतं च} %2-109-35

\twolineshloka
{धर्मे रताः सत्पुरुषैः समेतास्तेजस्विनो दानगुणप्रधानाः}
{अहिंसका वीतमलाश्च लोके भवन्ति पूज्या मुनयः प्रधानाः} %2-109-36

\twolineshloka
{इति ब्रुवन्तं वचनं सरोषं रामं महात्मानमदीनसत्त्वम्}
{उवाच तथ्यं पुनरास्तिकं च सत्यं वचः सानुनयं च विप्रः} %2-109-37

\twolineshloka
{न नास्तिकानां वचनं ब्रवीम्यहं न नास्तिकोऽहं न च नास्ति किञ्चन}
{समीक्ष्य कालं पुनरास्तिकोऽभवं भवेय काले पुनरेव नास्तिकः} %2-109-38

\twolineshloka
{स चापि कालोऽयमुपागतः शनैर्यथा मया नास्तिकवागुदीरिता}
{निवर्त्तनार्थं तव राम कारणात् प्रसादनार्थं तु मयैतदीरितम्} %2-109-39


॥इत्यार्षे श्रीमद्रामायणे वाल्मीकीये आदिकाव्ये अयोध्याकाण्डे सत्यप्रशंसा नाम नवाधिकशततमः सर्गः ॥२-१०९॥
