\sect{षड्विंशः सर्गः — सीताप्रत्यवस्थापनम्}

\twolineshloka
{अभिवाद्य तु कौसल्यां रामः सम्प्रस्थितो वनम्}
{कृतस्वस्त्ययनो मात्रा धर्मिष्ठे वर्त्मनि स्थितः} %2-26-1

\twolineshloka
{विराजयन् राजसुतो राजमार्गं नरैर्वृतम्}
{हृदयान्याममन्थेव जनस्य गुणवत्तया} %2-26-2

\twolineshloka
{वैदेही चापि तत् सर्वं न शुश्राव तपस्विनी}
{तदेव हृदि तस्याश्च यौवराज्याभिषेचनम्} %2-26-3

\twolineshloka
{देवकार्यं स्म सा कृत्वा कृतज्ञा हृष्टचेतना}
{अभिज्ञा राजधर्माणां राजपुत्री प्रतीक्षति} %2-26-4

\twolineshloka
{प्रविवेशाथ रामस्तु स्ववेश्म सुविभूषितम्}
{प्रहृष्टजनसम्पूर्णं ह्रिया किञ्चिदवाङ्मुखः} %2-26-5

\twolineshloka
{अथ सीता समुत्पत्य वेपमाना च तं पतिम्}
{अपश्यच्छोकसन्तप्तं चिन्ताव्याकुलितेन्द्रियम्} %2-26-6

\twolineshloka
{तां दृष्ट्वा स हि धर्मात्मा न शशाक मनोगतम्}
{तं शोकं राघवः सोढुं ततो विवृततां गतः} %2-26-7

\twolineshloka
{विवर्णवदनं दृष्ट्वा तं प्रस्विन्नममर्षणम्}
{आह दुःखाभिसन्तप्ता किमिदानीमिदं प्रभो} %2-26-8

\twolineshloka
{अद्य बार्हस्पतः श्रीमान् युक्तः पुष्येण राघव}
{प्रोच्यते ब्राह्मणैः प्राज्ञैः केन त्वमसि दुर्मनाः} %2-26-9

\twolineshloka
{न ते शतशलाकेन जलफेननिभेन च}
{आवृतं वदनं वल्गु च्छत्रेणाभिविराजते} %2-26-10

\twolineshloka
{व्यजनाभ्यां च मुख्याभ्यां शतपत्रनिभेक्षणम्}
{चन्द्रहंसप्रकाशाभ्यां वीज्यते न तवाननम्} %2-26-11

\twolineshloka
{वाग्मिनो वन्दिनश्चापि प्रहृष्टास्त्वां नरर्षभ}
{स्तुवन्तो नाद्य दृश्यन्ते मङ्गलैः सूतमागधाः} %2-26-12

\twolineshloka
{न ते क्षौद्रं च दधि च ब्राह्मणा वेदपारगाः}
{मूर्ध्नि मूर्धाभिषिक्तस्य ददति स्म विधानतः} %2-26-13

\twolineshloka
{न त्वां प्रकृतयः सर्वाः श्रेणीमुख्याश्च भूषिताः}
{अनुव्रजितुमिच्छन्ति पौरजानपदास्तथा} %2-26-14

\twolineshloka
{चतुर्भिर्वेगसम्पन्नैर्हयैः काञ्चनभूषणैः}
{मुख्यः पुष्परथो युक्तः किं न गच्छति तेऽग्रतः} %2-26-15

\twolineshloka
{न हस्ती चाग्रतः श्रीमान् सर्वलक्षणपूजितः}
{प्रयाणे लक्ष्यते वीर कृष्णमेघगिरिप्रभः} %2-26-16

\twolineshloka
{न च काञ्चनचित्रं ते पश्यामि प्रियदर्शन}
{भद्रासनं पुरस्कृत्य यान्तं वीर पुरःसरम्} %2-26-17

\twolineshloka
{अभिषेको यदा सज्जः किमिदानीमिदं तव}
{अपूर्वो मुखवर्णश्च न प्रहर्षश्च लक्ष्यते} %2-26-18

\twolineshloka
{इतीव विलपन्तीं तां प्रोवाच रघुनन्दनः}
{सीते तत्रभवांस्तातः प्रव्राजयति मां वनम्} %2-26-19

\twolineshloka
{कुले महति सम्भूते धर्मज्ञे धर्मचारिणि}
{शृणु जानकि येनेदं क्रमेणाद्यागतं मम} %2-26-20

\twolineshloka
{राज्ञा सत्यप्रतिज्ञेन पित्रा दशरथेन वै}
{कैकेय्यै मम मात्रे तु पुरा दत्तौ महावरौ} %2-26-21

\twolineshloka
{तयाद्य मम सज्जेऽस्मिन्नभिषेके नृपोद्यते}
{प्रचोदितः स समयो धर्मेण प्रतिनिर्जितः} %2-26-22

\twolineshloka
{चतुर्दश हि वर्षाणि वस्तव्यं दण्डके मया}
{पित्रा मे भरतश्चापि यौवराज्ये नियोजितः} %2-26-23

\twolineshloka
{सोऽहं त्वामागतो द्रष्टुं प्रस्थितो विजनं वनम्}
{भरतस्य समीपे ते नाहं कथ्यः कदाचन} %2-26-24

\twolineshloka
{ऋद्धियुक्ता हि पुरुषा न सहन्ते परस्तवम्}
{तस्मान्न ते गुणाः कथ्या भरतस्याग्रतो मम} %2-26-25

\twolineshloka
{अहं ते नानुवक्तव्यो विशेषेण कदाचन}
{अनुकूलतया शक्यं समीपे तस्य वर्तितुम्} %2-26-26

\twolineshloka
{तस्मै दत्तं नृपतिना यौवराज्यं सनातनम्}
{स प्रसाद्यस्त्वया सीते नृपतिश्च विशेषतः} %2-26-27

\twolineshloka
{अहं चापि प्रतिज्ञां तां गुरोः समनुपालयन्}
{वनमद्यैव यास्यामि स्थिरीभव मनस्विनि} %2-26-28

\twolineshloka
{याते च मयि कल्याणि वनं मुनिनिषेवितम्}
{व्रतोपवासपरया भवितव्यं त्वयानघे} %2-26-29

\twolineshloka
{कल्यमुत्थाय देवानां कृत्वा पूजां यथाविधि}
{वन्दितव्यो दशरथः पिता मम जनेश्वरः} %2-26-30

\twolineshloka
{माता च मम कौसल्या वृद्धा सन्तापकर्शिता}
{धर्ममेवाग्रतः कृत्वा त्वत्तः सम्मानमर्हति} %2-26-31

\twolineshloka
{वन्दितव्याश्च ते नित्यं याः शेषा मम मातरः}
{स्नेहप्रणयसम्भोगैः समा हि मम मातरः} %2-26-32

\twolineshloka
{भ्रातृपुत्रसमौ चापि द्रष्टव्यौ च विशेषतः}
{त्वया भरतशत्रुघ्नौ प्राणैः प्रियतरौ मम} %2-26-33

\twolineshloka
{विप्रियं च न कर्तव्यं भरतस्य कदाचन}
{स हि राजा च वैदेहि देशस्य च कुलस्य च} %2-26-34

\twolineshloka
{आराधिता हि शीलेन प्रयत्नैश्चोपसेविताः}
{राजानः सम्प्रसीदन्ति प्रकुप्यन्ति विपर्यये} %2-26-35

\twolineshloka
{औरस्यानपि पुत्रान् हि त्यजन्त्यहितकारिणः}
{समर्थान् सम्प्रगृह्णन्ति जनानपि नराधिपाः} %2-26-36

\twolineshloka
{सा त्वं वसेह कल्याणि राज्ञः समनुवर्तिनी}
{भरतस्य रता धर्मे सत्यव्रतपरायणा} %2-26-37

\twolineshloka
{अहं गमिष्यामि महावनं प्रिये त्वया हि वस्तव्यमिहैव भामिनि}
{यथा व्यलीकं कुरुषे न कस्यचित् तथा त्वया कार्यमिदं वचो मम} %2-26-38


॥इत्यार्षे श्रीमद्रामायणे वाल्मीकीये आदिकाव्ये अयोध्याकाण्डे सीताप्रत्यवस्थापनम् नाम षड्विंशः सर्गः ॥२-२६॥
