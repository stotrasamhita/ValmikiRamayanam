\sect{द्वितीयः सर्गः — परिषदनुमोदनम्}

\twolineshloka
{ततः परिषदं सर्वामामन्त्र्य वसुधाधिपः}
{हितमुद्धर्षणं चैवमुवाच प्रथितं वचः} %2-2-1

\twolineshloka
{दुन्दुभिस्वरकल्पेन गम्भीरेणानुनादिना}
{स्वरेण महता राजा जीमूत इव नादयन्} %2-2-2

\twolineshloka
{राजलक्षणयुक्तेन कान्तेनानुपमेन च}
{उवाच रसयुक्तेन स्वरेण नृपतिर्नृपान्} %2-2-3

\twolineshloka
{विदितं भवतामेतद् यथा मे राज्यमुत्तमम्}
{पूर्वकैर्मम राजेन्द्रैः सुतवत् परिपालितम्} %2-2-4

\twolineshloka
{सोऽहमिक्ष्वाकुभिः सर्वैर्नरेन्द्रैः प्रतिपालितम्}
{श्रेयसा योक्तुमिच्छामि सुखार्हमखिलं जगत्} %2-2-5

\twolineshloka
{मयाप्याचरितं पूर्वैः पन्थानमनुगच्छता}
{प्रजा नित्यमनिद्रेण यथाशक्त्यभिरक्षिताः} %2-2-6

\twolineshloka
{इदं शरीरं कृत्स्नस्य लोकस्य चरता हितम्}
{पाण्डुरस्यातपत्रस्य च्छायायां जरितं मया} %2-2-7

\twolineshloka
{प्राप्य वर्षसहस्राणि बहून्यायूंषि जीवतः}
{जीर्णस्यास्य शरीरस्य विश्रान्तिमभिरोचये} %2-2-8

\twolineshloka
{राजप्रभावजुष्टां च दुर्वहामजितेन्द्रियैः}
{परिश्रान्तोऽस्मि लोकस्य गुर्वीं धर्मधुरं वहन्} %2-2-9

\twolineshloka
{सोऽहं विश्राममिच्छामि पुत्रं कृत्वा प्रजाहिते}
{संनिकृष्टानिमान् सर्वाननुमान्य द्विजर्षभान्} %2-2-10

\twolineshloka
{अनुजातो हि मां सर्वैर्गुणैः श्रेष्ठो ममात्मजः}
{पुरन्दरसमो वीर्ये रामः परपुरंजयः} %2-2-11

\twolineshloka
{तं चन्द्रमिव पुष्येण युक्तं धर्मभृतां वरम्}
{यौवराज्ये नियोक्तास्मि प्रातः पुरुषपुङ्गवम्} %2-2-12

\twolineshloka
{अनुरूपः स वो नाथो लक्ष्मीवाँल्लक्ष्मणाग्रजः}
{त्रैलोक्यमपि नाथेन येन स्यान्नाथवत्तरम्} %2-2-13

\twolineshloka
{अनेन श्रेयसा सद्यः संयोक्ष्येऽहमिमां महीम्}
{गतक्लेशो भविष्यामि सुते तस्मिन् निवेश्य वै} %2-2-14

\twolineshloka
{यदिदं मेऽनुरूपार्थं मया साधु सुमन्त्रितम्}
{भवन्तो मेऽनुमन्यन्तां कथं वा करवाण्यहम्} %2-2-15

\twolineshloka
{यद्यप्येषा मम प्रीतिर्हितमन्यद् विचिन्त्यताम्}
{अन्या मध्यस्थचिन्ता तु विमर्दाभ्यधिकोदया} %2-2-16

\twolineshloka
{इति ब्रुवन्तं मुदिताः प्रत्यनन्दन् नृपा नृपम्}
{वृष्टिमन्तं महामेघं नर्दन्त इव बर्हिणः} %2-2-17

\twolineshloka
{स्निग्धोऽनुनादः संजज्ञे ततो हर्षसमीरितः}
{जनौघोद्घुष्टसंनादो मेदिनीं कम्पयन्निव} %2-2-18

\twolineshloka
{तस्य धर्मार्थविदुषो भावमाज्ञाय सर्वशः}
{ब्राह्मणा बलमुख्याश्च पौरजानपदैः सह} %2-2-19

\twolineshloka
{समेत्य ते मन्त्रयितुं समतागतबुद्धयः}
{ऊचुश्च मनसा ज्ञात्वा वृद्धं दशरथं नृपम्} %2-2-20

\twolineshloka
{अनेकवर्षसाहस्रो वृद्धस्त्वमसि पार्थिव}
{स रामं युवराजानमभिषिञ्चस्व पार्थिवम्} %2-2-21

\twolineshloka
{इच्छामो हि महाबाहुं रघुवीरं महाबलम्}
{गजेन महता यान्तं रामं छत्रावृताननम्} %2-2-22

\twolineshloka
{इति तद्वचनं श्रुत्वा राजा तेषां मनःप्रियम्}
{अजानन्निव जिज्ञासुरिदं वचनमब्रवीत्} %2-2-23

\twolineshloka
{श्रुत्वैतद् वचनं यन्मे राघवं पतिमिच्छथ}
{राजानः संशयोऽयं मे तदिदं ब्रूत तत्त्वतः} %2-2-24

\twolineshloka
{कथं नु मयि धर्मेण पृथिवीमनुशासति}
{भवन्तो द्रष्टुमिच्छन्ति युवराजं महाबलम्} %2-2-25

\twolineshloka
{ते तमूचुर्महात्मानः पौरजानपदैः सह}
{बहवो नृप कल्याणगुणाः सन्ति सुतस्य ते} %2-2-26

\twolineshloka
{गुणान् गुणवतो देव देवकल्पस्य धीमतः}
{प्रियानानन्दनान् कृत्स्नान् प्रवक्ष्यामोऽद्य तान्शृणु} %2-2-27

\twolineshloka
{दिव्यैर्गुणैः शक्रसमो रामः सत्यपराक्रमः}
{इक्ष्वाकुभ्योऽपि सर्वेभ्यो ह्यतिरिक्तो विशाम्पते} %2-2-28

\twolineshloka
{रामः सत्पुरुषो लोके सत्यः सत्यपरायणः}
{साक्षाद् रामाद् विनिर्वृत्तो धर्मश्चापि श्रिया सह} %2-2-29

\twolineshloka
{प्रजासुखत्वे चन्द्रस्य वसुधायाः क्षमागुणैः}
{बुद्ध्या बृहस्पतेस्तुल्यो वीर्ये साक्षाच्छचीपतेः} %2-2-30

\twolineshloka
{धर्मज्ञः सत्यसंधश्च शीलवाननसूयकः}
{क्षान्तः सान्त्वयिता श्लक्ष्णः कृतज्ञो विजितेन्द्रियः} %2-2-31

\twolineshloka
{मृदुश्च स्थिरचित्तश्च सदा भव्योऽनसूयकः}
{प्रियवादी च भूतानां सत्यवादी च राघवः} %2-2-32

\twolineshloka
{बहुश्रुतानां वृद्धानां ब्राह्मणानामुपासिता}
{तेनास्येहातुला कीर्तिर्यशस्तेजश्च वर्धते} %2-2-33

\twolineshloka
{देवासुरमनुष्याणां सर्वास्त्रेषु विशारदः}
{सम्यग् विद्याव्रतस्नातो यथावत् साङ्गवेदवित्} %2-2-34

\twolineshloka
{गान्धर्वे च भुवि श्रेष्ठो बभूव भरताग्रजः}
{कल्याणाभिजनः साधुरदीनात्मा महामतिः} %2-2-35

\twolineshloka
{द्विजैरभिविनीतश्च श्रेष्ठैर्धर्मार्थनैपुणैः}
{यदा व्रजति संग्रामं ग्रामार्थे नगरस्य वा} %2-2-36

\twolineshloka
{गत्वा सौमित्रिसहितो नाविजित्य निवर्तते}
{संग्रामात् पुनरागत्य कुञ्जरेण रथेन वा} %2-2-37

\twolineshloka
{पौरान् स्वजनवन्नित्यं कुशलं परिपृच्छति}
{पुत्रेष्वग्निषु दारेषु प्रेष्यशिष्यगणेषु च} %2-2-38

\twolineshloka
{निखिलेनानुपूर्व्या च पिता पुत्रानिवौरसान्}
{शुश्रूषन्ते च वः शिष्याः कच्चिद् वर्मसु दंशिताः} %2-2-39

\twolineshloka
{इति वः पुरुषव्याघ्रः सदा रामोऽभिभाषते}
{व्यसनेषु मनुष्याणां भृशं भवति दुःखितः} %2-2-40

\twolineshloka
{उत्सवेषु च सर्वेषु पितेव परितुष्यति}
{सत्यवादी महेष्वासो वृद्धसेवी जितेन्द्रियः} %2-2-41

\twolineshloka
{स्मितपूर्वाभिभाषी च धर्मं सर्वात्मनाश्रितः}
{सम्यग्योक्ता श्रेयसां च न विगृह्यकथारुचिः} %2-2-42

\twolineshloka
{उत्तरोत्तरयुक्तौ च वक्ता वाचस्पतिर्यथा}
{सुभ्रूरायतताम्राक्षः साक्षाद् विष्णुरिव स्वयम्} %2-2-43

\twolineshloka
{रामो लोकाभिरामोऽयं शौर्यवीर्यपराक्रमैः}
{प्रजापालनसंयुक्तो न रागोपहतेन्द्रियः} %2-2-44

\twolineshloka
{शक्तस्त्रैलोक्यमप्येष भोक्तुं किं नु महीमिमाम्}
{नास्य क्रोधः प्रसादश्च निरर्थोऽस्ति कदाचन} %2-2-45

\twolineshloka
{हन्त्येष नियमाद् वध्यानवध्येषु न कुप्यति}
{युनक्त्यर्थैः प्रहृष्टश्च तमसौ यत्र तुष्यति} %2-2-46

\twolineshloka
{दान्तैः सर्वप्रजाकान्तैः प्रीतिसंजननैर्नृणाम्}
{गुणैर्विरोचते रामो दीप्तः सूर्य इवांशुभिः} %2-2-47

\twolineshloka
{तमेवंगुणसम्पन्नं रामं सत्यपराक्रमम्}
{लोकपालोपमं नाथमकामयत मेदिनी} %2-2-48

\twolineshloka
{वत्सः श्रेयसि जातस्ते दिष्ट्यासौ तव राघवः}
{दिष्ट्या पुत्रगुणैर्युक्तो मारीच इव कश्यपः} %2-2-49

\twolineshloka
{बलमारोग्यमायुश्च रामस्य विदितात्मनः}
{देवासुरमनुष्येषु सगन्धर्वोरगेषु च} %2-2-50

\twolineshloka
{आशंसते जनः सर्वो राष्ट्रे पुरवरे तथा}
{आभ्यन्तरश्च बाह्यश्च पौरजानपदो जनः} %2-2-51

\threelineshloka
{स्त्रियो वृद्धास्तरुण्यश्च सायं प्रातः समाहिताः}
{सर्वा देवान्नमस्यन्ति रामस्यार्थे मनस्विनः}
{तेषां तद् याचितं देव त्वत्प्रसादात्समृद्ध्यताम्} %2-2-52

\twolineshloka
{राममिन्दीवरश्यामं सर्वशत्रुनिबर्हणम्}
{पश्यामो यौवराज्यस्थं तव राजोत्तमात्मजम्} %2-2-53

\twolineshloka
{तं देवदेवोपममात्मजं ते सर्वस्य लोकस्य हिते निविष्टम्}
{हिताय नः क्षिप्रमुदारजुष्टं मुदाभिषेक्तुं वरद त्वमर्हसि} %2-2-54


॥इत्यार्षे श्रीमद्रामायणे वाल्मीकीये आदिकाव्ये अयोध्याकाण्डे परिषदनुमोदनम् नाम द्वितीयः सर्गः ॥२-२॥
