\sect{नवमः सर्गः — रामप्रवासनोपायचिन्ता}

\twolineshloka
{एवमुक्ता तु कैकेयी क्रोधेन ज्वलितानना}
{दीर्घमुष्णं विनिःश्वस्य मन्थरामिदमब्रवीत्} %2-9-1

\twolineshloka
{अद्य राममितः क्षिप्रं वनं प्रस्थापयाम्यहम्}
{यौवराज्येन भरतं क्षिप्रमद्याभिषेचये} %2-9-2

\twolineshloka
{इदं त्विदानीं सम्पश्य केनोपायेन साधये}
{भरतः प्राप्नुयाद् राज्यं न तु रामः कथञ्चन} %2-9-3

\twolineshloka
{एवमुक्ता तु सा देव्या मन्थरा पापदर्शिनी}
{रामार्थमुपहिंसन्ती कैकेयीमिदमब्रवीत्} %2-9-4

\twolineshloka
{हन्तेदानीं प्रपश्य त्वं कैकेयि श्रूयतां वचः}
{यथा ते भरतो राज्यं पुत्रः प्राप्स्यति केवलम्} %2-9-5

\twolineshloka
{किं न स्मरसि कैकेयि स्मरन्ती वा निगूहसे}
{यदुच्यमानमात्मार्थं मत्तस्त्वं श्रोतुमिच्छसि} %2-9-6

\twolineshloka
{मयोच्यमानं यदि ते श्रोतुं छन्दो विलासिनि}
{श्रूयतामभिधास्यामि श्रुत्वा चैतद् विधीयताम्} %2-9-7

\twolineshloka
{श्रुत्वैवं वचनं तस्या मन्थरायास्तु कैकयी}
{किञ्चिदुत्थाय शयनात् स्वास्तीर्णादिदमब्रवीत्} %2-9-8

\twolineshloka
{कथयस्व ममोपायं केनोपायेन मन्थरे}
{भरतः प्राप्नुयाद् राज्यं न तु रामः कथञ्चन} %2-9-9

\twolineshloka
{एवमुक्ता तदा देव्या मन्थरा पापदर्शिनी}
{रामार्थमुपहिंसन्ती कैकेयीमिदमब्रवीत्} %2-9-10

\twolineshloka
{पुरा देवासुरे युद्धे सह राजर्षिभिः पतिः}
{अगच्छत् त्वामुपादाय देवराजस्य साह्यकृत्} %2-9-11

\twolineshloka
{दिशमास्थाय कैकेयि दक्षिणां दण्डकान् प्रति}
{वैजयन्तमिति ख्यातं पुरं यत्र तिमिध्वजः} %2-9-12

\twolineshloka
{स शम्बर इति ख्यातः शतमायो महासुरः}
{ददौ शक्रस्य सङ्ग्रामं देवसङ्घैरनिर्जितः} %2-9-13

\twolineshloka
{तस्मिन् महति सङ्ग्रामे पुरुषान् क्षतविक्षतान्}
{रात्रौ प्रसुप्तान् घ्नन्ति स्म तरसापास्य राक्षसाः} %2-9-14

\twolineshloka
{तत्राकरोन्महायुद्धं राजा दशरथस्तदा}
{असुरैश्च महाबाहुः शस्त्रैश्च शकलीकृतः} %2-9-15

\twolineshloka
{अपवाह्य त्वया देवि सङ्ग्रामान्नष्टचेतनः}
{तत्रापि विक्षतः शस्त्रैः पतिस्ते रक्षितस्त्वया} %2-9-16

\twolineshloka
{तुष्टेन तेन दत्तौ ते द्वौ वरौ शुभदर्शने}
{स त्वयोक्तः पतिर्देवि यदेच्छेयं तदा वरम्} %2-9-17

\twolineshloka
{गृह्णीयां तु तदा भर्तस्तथेत्युक्तं महात्मना}
{अनभिज्ञा ह्यहं देवि त्वयैव कथितं पुरा} %2-9-18

\twolineshloka
{कथैषा तव तु स्नेहान्मनसा धार्यते मया}
{रामाभिषेकसम्भारान्निगृह्य विनिवर्तय} %2-9-19

\twolineshloka
{तौ च याचस्व भर्तारं भरतस्याभिषेचनम्}
{प्रव्राजनं च रामस्य वर्षाणि च चतुर्दश} %2-9-20

\twolineshloka
{चतुर्दश हि वर्षाणि रामे प्रव्राजिते वनम्}
{प्रजाभावगतस्नेहः स्थिरः पुत्रो भविष्यति} %2-9-21

\twolineshloka
{क्रोधागारं प्रविश्याद्य क्रुद्धेवाश्वपतेः सुते}
{शेष्वानन्तर्हितायां त्वं भूमौ मलिनवासिनी} %2-9-22

\twolineshloka
{मा स्मैनं प्रत्युदीक्षेथा मा चैनमभिभाषथाः}
{रुदन्ती पार्थिवं दृष्ट्वा जगत्यां शोकलालसा} %2-9-23

\twolineshloka
{दयिता त्वं सदा भर्तुरत्र मे नास्ति संशयः}
{त्वत्कृते च महाराजो विशेदपि हुताशनम्} %2-9-24

\twolineshloka
{न त्वां क्रोधयितुं शक्तो न क्रुद्धां प्रत्युदीक्षितुम्}
{तव प्रियार्थं राजा तु प्राणानपि परित्यजेत्} %2-9-25

\twolineshloka
{न ह्यतिक्रमितुं शक्तस्तव वाक्यं महीपतिः}
{मन्दस्वभावे बुध्यस्व सौभाग्यबलमात्मनः} %2-9-26

\twolineshloka
{मणिमुक्तासुवर्णानि रत्नानि विविधानि च}
{दद्याद् दशरथो राजा मा स्म तेषु मनः कृथाः} %2-9-27

\twolineshloka
{यौ तौ देवासुरे युद्धे वरौ दशरथो ददौ}
{तौ स्मारय महाभागे सोऽर्थो न त्वा क्रमेदति} %2-9-28

\twolineshloka
{यदा तु ते वरं दद्यात् स्वयमुत्थाप्य राघवः}
{व्यवस्थाप्य महाराजं त्वमिमं वृणुया वरम्} %2-9-29

\twolineshloka
{रामप्रव्रजनं दूरं नव वर्षाणि पञ्च च}
{भरतः क्रियतां राजा पृथिव्यां पार्थिवर्षभ} %2-9-30

\twolineshloka
{चतुर्दश हि वर्षाणि रामे प्रव्राजिते वनम्}
{रूढश्च कृतमूलश्च शेषं स्थास्यति ते सुतः} %2-9-31

\twolineshloka
{रामप्रव्राजनं चैव देवि याचस्व तं वरम्}
{एवं सेत्स्यन्ति पुत्रस्य सर्वार्थास्तव कामिनि} %2-9-32

\twolineshloka
{एवं प्रव्राजितश्चैव रामोऽरामो भविष्यति}
{भरतश्च गतामित्रस्तव राजा भविष्यति} %2-9-33

\twolineshloka
{येन कालेन रामश्च वनात् प्रत्यागमिष्यति}
{अन्तर्बहिश्च पुत्रस्ते कृतमूलो भविष्यति} %2-9-34

\twolineshloka
{सङ्गृहीतमनुष्यश्च सुहृद्भिः साकमात्मवान्}
{प्राप्तकालं नु मन्येऽहं राजानं वीतसाध्वसा} %2-9-35

\twolineshloka
{रामाभिषेकसङ्कल्पान्निगृह्य विनिवर्तय}
{अनर्थमर्थरूपेण ग्राहिता सा ततस्तया} %2-9-36

\twolineshloka
{हृष्टा प्रतीता कैकेयी मन्थरामिदमब्रवीत्}
{सा हि वाक्येन कुब्जायाः किशोरीवोत्पथं गता} %2-9-37

\twolineshloka
{कैकेयी विस्मयं प्राप्य परं परमदर्शना}
{प्रज्ञां ते नावजानामि श्रेष्ठे श्रेष्ठाभिधायिनि} %2-9-38

\twolineshloka
{पृथिव्यामसि कुब्जानामुत्तमा बुद्धिनिश्चये}
{त्वमेव तु ममार्थेषु नित्ययुक्ता हितैषिणी} %2-9-39

\twolineshloka
{नाहं समवबुद्ध्येयं कुब्जे राज्ञश्चिकीर्षितम्}
{सन्ति दुःसंस्थिताः कुब्जाः वक्राः परमपापिकाः} %2-9-40

\twolineshloka
{त्वं पद्ममिव वातेन सन्नता प्रियदर्शना}
{उरस्तेऽभिनिविष्टं वै यावत् स्कन्धात् समुन्नतम्} %2-9-41

\twolineshloka
{अधस्ताच्चोदरं शान्तं सुनाभमिव लज्जितम्}
{प्रतिपूर्णं च जघनं सुपीनौ च पयोधरौ} %2-9-42

\twolineshloka
{विमलेन्दुसमं वक्त्रमहो राजसि मन्थरे}
{जघनं तव निर्मृष्टं रशनादामभूषितम्} %2-9-43

\twolineshloka
{जङ्घे भृशमुपन्यस्ते पादौ च व्यायतावुभौ}
{त्वमायताभ्यां सक्थिभ्यां मन्थरे क्षौमवासिनी} %2-9-44

\twolineshloka
{अग्रतो मम गच्छन्ती राजसेऽतीव शोभने}
{आसन् याः शम्बरे मायाः सहस्रम् असुराधिपे} %2-9-45

\twolineshloka
{हृदये ते निविष्टास् ता भूयश् चान्याः सहस्रशः}
{तदेव स्थगुपृष्ठवक्रभागः यद् दीर्घं रथघोणम् इवायतम्} %2-9-46

\twolineshloka
{मतयः क्षत्रविद्याश् च मायाश्चात्र वसन्ति ते}
{अत्र तेऽहं प्रमोक्ष्यामि मालां कुब्जे हिरण्मयीम्} %2-9-47

\twolineshloka
{अभिषिक्ते च भरते राघवे च वनं गते}
{जात्येन च सुवर्णेन सुनिष्टप्तेन सुन्दरि} %2-9-48

\twolineshloka
{लब्धार्था च प्रतीता च लेपयिष्यामि ते स्थगु}
{मुखे च तिलकं चित्रं जातरूपमयं शुभम्} %2-9-49

\twolineshloka
{कारयिष्यामि ते कुब्जे शुभान्याभरणानि च}
{परिधाय शुभे वस्त्रे देवतेव चरिष्यसि} %2-9-50

\twolineshloka
{चन्द्रमाह्वयमानेन मुखेनाप्रतिमानना}
{गमिष्यसि गतिं मुख्यां गर्वयन्ती द्विषज्जने} %2-9-51

\twolineshloka
{तवापि कुब्जाः कुब्जायाः सर्वाभरणभूषिताः}
{पादौ परिचरिष्यन्ति यथैव त्वं सदा मम} %2-9-52

\twolineshloka
{इति प्रशस्यमाना सा कैकेयीमिदमब्रवीत्}
{शयानां शयने शुभ्रे वेद्यामग्निशिखामिव} %2-9-53

\twolineshloka
{गतोदके सेतुबन्धो न कल्याणि विधीयते}
{उत्तिष्ठ कुरु कल्याणं राजानमनुदर्शय} %2-9-54

\twolineshloka
{तथा प्रोत्साहिता देवी गत्वा मन्थरया सह}
{क्रोधागारं विशालाक्षी सौभाग्यमदगर्विता} %2-9-55

\twolineshloka
{अनेकशतसाहस्रं मुक्ताहारं वराङ्गना}
{अवमुच्य वरार्हाणि शुभान्याभरणानि च} %2-9-56

\twolineshloka
{तदा हेमोपमा तत्र कुब्जावाक्यवशङ्गता}
{संविश्य भूमौ कैकेयी मन्थरामिदमब्रवीत्} %2-9-57

\twolineshloka
{इह वा मां मृतां कुब्जे नृपायावेदयिष्यसि}
{वनं तु राघवे प्राप्ते भरतः प्राप्स्यते क्षितिम्} %2-9-58

\twolineshloka
{सुवर्णेन न मे ह्यर्थो न रत्नैर्न च भोजनैः}
{एष मे जीवितस्यान्तो रामो यद्यभिषिच्यते} %2-9-59

\twolineshloka
{ततः पुनस्तां महिषीं महीक्षितो वचोभिरत्यर्थमहापराक्रमैः}
{उवाच कुब्जा भरतस्य मातरं हितं वचो राममुपेत्य चाहितम्} %2-9-60

\twolineshloka
{प्रपत्स्यते राज्यमिदं हि राघवो यदि ध्रुवं त्वं ससुता च तप्स्यसे}
{ततो हि कल्याणि यतस्व तत् तथा यथा सुतस्ते भरतोऽभिषेक्ष्यते} %2-9-61

\twolineshloka
{तथातिविद्धा महिषीति कुब्जया समाहता वागिषुभिर्मुहुर्मुहुः}
{विधाय हस्तौ हृदयेऽतिविस्मिता शशंस कुब्जां कुपिता पुनः पुनः} %2-9-62

\twolineshloka
{यमस्य वा मां विषयं गतामितो निशम्य कुब्जे प्रतिवेदयिष्यसि}
{वनं गते वा सुचिराय राघवे समृद्धकामो भरतो भविष्यति} %2-9-63

\twolineshloka
{अहं हि नैवास्तरणानि न स्रजो न चन्दनं नाञ्जनपानभोजनम्}
{न किञ्चिदिच्छामि न चेह जीवनं न चेदितो गच्छति राघवो वनम्} %2-9-64

\twolineshloka
{अथैवमुक्त्वा वचनं सुदारुणं निधाय सर्वाभरणानि भामिनी}
{असंस्कृतामास्तरणेन मेदिनीं तदाधिशिश्ये पतितेव किन्नरी} %2-9-65

\twolineshloka
{उदीर्णसंरम्भतमोवृतानना तदावमुक्तोत्तममाल्यभूषणा}
{नरेन्द्रपत्नी विमना बभूव सा तमोवृता द्यौरिव मग्नतारका} %2-9-66


॥इत्यार्षे श्रीमद्रामायणे वाल्मीकीये आदिकाव्ये अयोध्याकाण्डे रामप्रवासनोपायचिन्ता नाम नवमः सर्गः ॥२-९॥
