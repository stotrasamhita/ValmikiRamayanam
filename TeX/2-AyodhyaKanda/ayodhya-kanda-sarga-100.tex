\sect{शततमः सर्गः — कच्चित्सर्गः}

\twolineshloka
{जटिलं चीरवसनं प्राञ्जलिं पतितं भुवि}
{ददर्श रामो दुर्दर्शं युगान्ते भास्करं यथा} %2-100-1

\twolineshloka
{कथञ्चिदभिविज्ञाय विवर्णवदनं कृशम्}
{भ्रातरं भरतं रामः परिजग्राह बाहुना} %2-100-2

\twolineshloka
{आघ्राय रामस्तं मूर्ध्नि परिष्वज्य च राघवः}
{अङ्के भरतमारोप्य पर्यपृच्छत् समाहितः} %2-100-3

\twolineshloka
{क्व नु तेऽभूत् पिता तात यदरण्यं त्वमागतः}
{न हि त्वं जीवतस्तस्य वनमागन्तुमर्हसि} %2-100-4

\twolineshloka
{चिरस्य बत पश्यामि दूराद्भरतमागतम्}
{दुष्प्रतीकमरण्येऽस्मिन् किं तात वनमागतः} %2-100-5

\twolineshloka
{कच्चिद्धारयते तात राजा यत्त्वमिहागतः}
{कच्चिन्न दीनः सहसा राजा लोकान्तरं गतः} %2-100-6

\onelineshloka
{कच्चित् सौम्य न ते राज्यं भ्रष्टं बालस्य शाश्वतम्} %2-100-7

\onelineshloka
{कच्चिच्छुश्रूषसे तात पितरं सत्यविक्रमम्} %2-100-8

\twolineshloka
{कच्चिद्दशरथो राजा कुशली सत्यसङ्गरः}
{राजसूयाश्वमेधानामाहर्त्ता धर्मनिश्चयः} %2-100-9

\twolineshloka
{स कच्चिद् ब्राह्मणो विद्वान् धर्मनित्यो महाद्युतिः}
{इक्ष्वाकूणामुपाध्यायो यथावत्तात पूज्यते} %2-100-10

\twolineshloka
{सा तात कच्चित्कौसल्या सुमित्रा च प्रजावती}
{सुखिनी कच्चिदार्य्या च देवी नन्दति कैकयी} %2-100-11

\twolineshloka
{कच्चिद्विनयसम्पन्नः कुलपुत्रो बहुश्रुतः}
{अनसूयुरनुद्रष्टा सत्कृतस्ते पुरोहितः} %2-100-12

\twolineshloka
{कच्चिदग्निषु ते युक्तो विधिज्ञो मतिमानृजुः}
{हुतं च होष्यमाणं च काले वेदयते सदा} %2-100-13

\twolineshloka
{कच्चिद्देवान् पितऽन् मातऽर्गुरून् पितृसमानपि}
{वृद्धांश्च तत वैद्यांश्च ब्राह्मणांश्चाभिमन्यसे} %2-100-14

\twolineshloka
{इष्वस्त्रवरसम्पन्नमर्थशास्त्रविशारदम्}
{सुधन्वानमुपाध्यायं कच्चित्त्वं तात मन्यसे} %2-100-15

\twolineshloka
{कच्चिदात्मसमाः शूराः श्रुतवन्तो जितेन्द्रियाः}
{कुलीनाश्चेङ्गितज्ञाश्च कृतास्ते तात मन्त्रिणः} %2-100-16

\twolineshloka
{मन्त्रो विजयमूलं हि राज्ञां भवति राघव}
{सुसंवृतो मन्त्रधरैरमात्यैः शास्त्रकोविदैः} %2-100-17

\twolineshloka
{कच्चिन्निद्रावशं नैषीः कच्चित्काले प्रबुध्यसे}
{कच्चिच्चापररात्रेषु चिन्तयस्यर्थनैपुणम्} %2-100-18

\twolineshloka
{कच्चिन्मन्त्रयसे नैकः कच्चिन्न बहुभिः सह}
{कच्चित्ते मन्त्रितो मन्त्रो राष्ट्रं न परिधावति} %2-100-19

\twolineshloka
{कच्चिदर्थं विनिश्चित्य लघुमूलं महोदयम्}
{क्षिप्रमारभसे कर्त्तुं न दीर्घयसि राघव} %2-100-20

\twolineshloka
{कच्चित्ते सुकृतान्येव कृतरूपाणि वा पुनः}
{विदुस्ते सर्वकार्याणि न कर्त्तव्यानि पार्थिवाः} %2-100-21

\twolineshloka
{कच्चिन्न तर्कैर्युक्त्या वा ये चाप्यपरिकीर्तिताः}
{त्वया वा वाऽमात्यैर्बुध्यते तात मन्त्रितम्} %2-100-22

\twolineshloka
{कच्चित् सहस्रान् मूर्खाणामेकमिच्छसि पण्डितम्}
{पण्डितो ह्यर्थकृच्छ्रेषु कुर्य्यान्निःश्रेयसं महत्} %2-100-23

\twolineshloka
{सहस्राण्यपि मूर्खाणां यद्युपास्ते महीपतिः}
{अथवाप्ययुतान्येव नास्ति तेषु सहायता} %2-100-24

\twolineshloka
{एकोप्यमात्यो मेधावी शूरो दक्षो विचक्षणः}
{राजानं राजमात्रं वा प्रापयेन्महतीं श्रियम्} %2-100-25

\twolineshloka
{कच्चिन्मुख्या महत्स्वेव मध्यमेषु च मध्यमाः}
{जघन्यास्तु जघन्येषु भृत्याः कर्मसु योजिताः} %2-100-26

\twolineshloka
{अमात्यानुपधातीतान् पितृपैतामहाञ्छुचीन्}
{श्रेष्ठान् श्रेष्ठेषु कच्चित्त्वं नियोजयसि कर्मसु} %2-100-27

\twolineshloka
{कच्चिन्नोग्रेण दण्डेन भृशमुद्वेजितप्रजम्}
{राष्ट्रं तवानुजानन्ति मन्त्रिणः कैकयीसुत} %2-100-28

\twolineshloka
{कच्चित्त्वां नावजानन्ति याजकाः पतितं यथा}
{उग्रप्रतिग्रहीतारं कामयानमिव स्त्रियः} %2-100-29

\twolineshloka
{उपायकुशलं वैद्यं भृत्यसन्दूषणे रतम्}
{शूरमैश्वर्यकामं च यो न हन्ति स वध्यते} %2-100-30

\twolineshloka
{कच्चिद्धृष्टश्च शूरश्च मतिमान् धृतिमाञ्छुचिः}
{कुलीनश्चानुरक्तश्च दक्षः सेनापतिः कृतः} %2-100-31

\twolineshloka
{बलवन्तश्च कच्चित्ते मुख्या युद्धविशारदाः}
{दृष्टापदाना विक्रान्तास्त्वया सत्कृत्य मानिताः} %2-100-32

\twolineshloka
{कच्चिद्बलस्य भक्तं च वेतनं च यथोचितम्}
{सम्प्राप्तकालं दातव्यं ददासि न विलम्बसे} %2-100-33

\twolineshloka
{कालातिक्रमणाच्चैव भक्तवेतनयोर्भृताः}
{भर्त्तुः कुप्यन्ति दुष्यन्ति सोऽनर्थः सुमहान् स्मृतः} %2-100-34

\twolineshloka
{कच्चित् सर्वेऽनुरक्तास्त्वां कुलपुत्राः प्रधानतः}
{कच्चित्प्राणांस्तवार्थेषु सन्त्यजन्ति समाहिताः} %2-100-35

\twolineshloka
{कच्चिज्जानपदो विद्वान् दक्षिणः प्रतिभानवान्}
{यथोक्तवादी दूतस्ते कृतो भरत पण्डितः} %2-100-36

\twolineshloka
{कच्चिदष्टादशान्येषु स्वपक्षे दश पञ्च च}
{त्रिभिस्त्रिभिरविज्ञातैर्वेत्सि तीर्थानि चारकैः} %2-100-37

\twolineshloka
{कच्चिद्व्यपास्तानहितान् प्रतियातांश्च सर्वदा}
{दुर्बलाननवज्ञाय वर्त्तसे रिपुसूदन} %2-100-38

\twolineshloka
{कच्चिन्न लोकायतिकान् ब्राह्मणांस्तात सेवसे}
{अनर्थकुशला ह्येते बालाः पण्डितमानिनः} %2-100-39

\twolineshloka
{धर्मशास्त्रेषु मुख्येषु विद्यमानेषु दुर्बुधाः}
{बुद्धिमान्वीक्षिकीं प्राप्य निरर्थं प्रवदन्ति ते} %2-100-40

\twolineshloka
{वीरैरध्युषितां पूर्वमस्माकं तात पूर्वकैः}
{सत्यनामां दृढद्वारां हस्त्यश्वरथसङ्कुलाम्} %2-100-41

\twolineshloka
{ब्राह्मणैः क्षत्ऺित्रयैर्वैश्यैः स्वकर्मनिरतैः सदा}
{जितेन्द्रियैर्महोत्साहैर्वृतामार्यैः सहस्रशः} %2-100-42

\twolineshloka
{प्रासादैर्विविधाकारैर्वृतां वैद्यजनाकुलाम्}
{कच्चित्सुमुदितां स्फीतामयोध्यां परिरक्षसि} %2-100-43

\twolineshloka
{कच्चिच्चित्यशतैर्जुष्टः सुनिविष्टजनाकुलः}
{देवस्थानैः प्रपाभिश्च तटाकैश्चोपशोभितः} %2-100-44

\twolineshloka
{प्रहृष्टनरनारीकः समाजोत्सवशोभितः}
{सुकृष्टसीमा पशुमान् हिंसाभिः परिवर्जितः} %2-100-45

\twolineshloka
{अदेवमातृको रम्यः श्वापदैः परिवर्जितः}
{परित्यक्तो भयैः सर्वैः खनिभिश्चोपशोभितः} %2-100-46

\twolineshloka
{विवर्जितो नरैः पापैर्मम पूर्वैः सुरक्षितः}
{कच्चिज्जनपदः स्फीतः सुखं वसति राघव} %2-100-47

\twolineshloka
{कच्चित्ते दयिताः सर्वे कृषिगोरक्षजीविनः}
{वार्त्तायां संश्रितस्तात लोको हि सुखमेधते} %2-100-48

\twolineshloka
{तेषां गुप्तिपरीहारैः कच्चित्ते भरणं कृतम्}
{रक्ष्या हि राज्ञा धर्मेण सर्वे विषयवासिनः} %2-100-49

\twolineshloka
{कच्चित् स्त्रियः सान्त्वयसि कच्चित्ताश्च सुरक्षिताः}
{कच्चिन्न श्रद्दधास्यासां कच्चिद् गुह्यं न भाषसे} %2-100-50

\twolineshloka
{कच्चिन्नागवनं गुप्तं कच्चित्ते सन्ति धेनुकाः}
{कच्चिन्न गणिकाश्वानां कुञ्जराणां च तृप्यसि} %2-100-51

\twolineshloka
{कच्चिद्दर्शयसे नित्यं मनुष्याणां विभूषितम्}
{उत्थायोत्थाय पूर्वाह्णे राजपुत्र महापथे} %2-100-52

\twolineshloka
{कच्चिन्न सर्वे कर्मान्ताः प्रत्यक्षास्तेऽविशङ्कया}
{सर्वे वा पुनरुत्सृष्टा मध्यमेवात्र कारणम्} %2-100-53

\twolineshloka
{कच्चित् सर्वाणि दुर्गाणि धनधान्यायुधोदकैः}
{यन्त्रैश्च परिपूर्णानि तथा शिल्पिधनुर्द्धरै} %2-100-54

\twolineshloka
{आयस्ते विपुलः कच्चित् कच्चिदल्पतरो व्ययः}
{अपात्रेषु न ते कच्चित्कोशो गच्छति राघव} %2-100-55

\twolineshloka
{देवतार्थे च पित्रर्थे ब्राह्मणाभ्यागतेषु च}
{योधेषु मित्रवर्गेषु कच्चिद्गच्छति ते व्ययः} %2-100-56

\twolineshloka
{कच्चिदार्य्यो विशुद्धात्माऽऽक्षारितश्चोरकर्मणा}
{अपृष्टः शास्त्रकुशलैर्न लोभाद्वध्यते शुचिः} %2-100-57

\twolineshloka
{गृहीतश्चैव पृष्टश्च काले दृष्टः सकारणः}
{कच्चिन्न मुच्यते चोरो धनलोभान्नरर्षभ} %2-100-58

\twolineshloka
{व्यसने कच्चिदाढ्यस्य दुर्गतस्य च राघव}
{अर्थं विरागाः पश्यन्ति तवामात्या बहुश्रुताः} %2-100-59

\twolineshloka
{यानि मिथ्याभिशस्तानां पतन्त्यस्राणि राघव}
{तानि पुत्रपशून् घ्नन्ति प्रीत्यर्थमनुशासतः} %2-100-60

\twolineshloka
{कच्चिद् वृद्धांश्च बालांश्च वैद्यमुख्यांश्च राघव}
{दानेन मनसा वाचा त्रिभिरेतैर्बुभूषसे} %2-100-61

\twolineshloka
{कच्चिद्गुरूंश्च वृद्धांश्च तापसान् देवतातिथीन्}
{चैत्यांश्च सर्वान् सिद्धार्थान् ब्राह्मणांश्च नमस्यसि} %2-100-62

\twolineshloka
{कच्चिदर्थेन वा धर्ममर्थं धर्मेण वा पुनः}
{उभौ वा प्रीतिलोभेन कामेन च न बाधसे} %2-100-63

\twolineshloka
{कच्चिदर्थं च धर्मं च कामं च जयतां वर}
{विभज्य काले कालज्ञ सर्वान् वरद सेवसे} %2-100-64

\twolineshloka
{कच्चित्ते ब्राह्मणाः शर्म सर्वशास्त्रार्थकोविदाः}
{आशंसन्ते महाप्राज्ञ पौरजानपदैः सह} %2-100-65

\twolineshloka
{नास्तिक्यमनृतं क्रोधं प्रमादं दीर्घसूत्रताम्}
{अदर्शनं ज्ञानवतामालस्यं पञ्चवृत्तिताम्} %2-100-66

\twolineshloka
{एकचिन्तनमर्थानामनर्थज्ञैश्च मन्त्रणम्}
{निश्चितानामनारम्भं मन्त्रस्यापरिरक्षणम्} %2-100-67

\twolineshloka
{मङ्गलस्याप्रयोगं च प्रत्युत्थानं च सर्वतः}
{कच्चित्त्वं वर्जयस्येतान् राजदोषांश्चतुर्दश} %2-100-68

\twolineshloka
{दश पञ्च चतुर्वर्गान् सप्तवर्गं च तत्त्वतः}
{अष्टवर्गं त्रिवर्गं च विद्यास्तिस्रश्च राघव} %2-100-69

\twolineshloka
{इन्द्रियाणां जयं बुद्ध्वा षाङ्गुण्यं दैवमानुषम्}
{कृत्यं विंशतिवर्गञ्च तथा प्रकृतिमण्डलम्} %2-100-70

\twolineshloka
{यात्रादण्डविधानञ्च द्वियोनी सन्धिविग्रहौ}
{कच्चिदेतान् महाप्राज्ञ यथावदनुमन्यसे} %2-100-71

\twolineshloka
{मन्त्रिभिस्त्वं यथोद्दिष्टैश्चतुर्भिस्त्रिभिरेव वा}
{कच्चित् समस्तैर्व्यस्तैश्च मन्त्र मन्त्रयसे मिथः} %2-100-72

\twolineshloka
{कच्चित्ते सफला वेदाः कच्चित्ते सफलाः क्रियाः}
{कच्चित्ते सफला दाराः कच्चित्ते सफलं श्रुतम्} %2-100-73

\twolineshloka
{कच्चिदेषैव ते बुद्धिर्यथोक्ता मम राघव}
{आयुष्या च यशस्या च धर्मकामार्थसंहिता} %2-100-74

\twolineshloka
{यां वृत्तिं वर्त्तते तातो यां च नः प्रपितामहाः}
{तां वृत्तिं वर्त्तसे कच्चिद्या च सत्पथगा शुभा} %2-100-75

\twolineshloka
{कच्चित् स्वादु कृतं भोज्यमेको नाश्नासि राघव}
{कच्चिदाशंसमानेभ्यो मित्रेभ्यः सम्प्रयच्छसि} %2-100-76

\twolineshloka
{राजा तु धर्मेण हि पालयित्वा महामतिर्दण्डधरः प्रजानाम्}
{अवाप्य कृत्स्नां वसुधां यथावदितश्च्युतः स्वर्गमुपैति विद्वान्} %2-100-77


॥इत्यार्षे श्रीमद्रामायणे वाल्मीकीये आदिकाव्ये अयोध्याकाण्डे कच्चित्सर्गः नाम शततमः सर्गः ॥२-१००॥
