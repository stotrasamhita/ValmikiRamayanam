\sect{त्रयोदशः सर्गः — दशरथविलापः}

\twolineshloka
{अतदर्हं महाराजं शयानमतथोचितम्}
{ययातिमिव पुण्यान्ते देवलोकात् परिच्युतम्} %2-13-1

\twolineshloka
{अनर्थरूपासिद्धार्था ह्यभीता भयदर्शिनी}
{पुनराकारयामास तमेव वरमङ्गना} %2-13-2

\twolineshloka
{त्वं कत्थसे महाराज सत्यवादी दृढव्रतः}
{मम चेदं वरं कस्माद् विधारयितुमिच्छसि} %2-13-3

\twolineshloka
{एवमुक्तस्तु कैकेय्या राजा दशरथस्तदा}
{प्रत्युवाच ततः क्रुद्धो मुहूर्तं विह्वलन्निव} %2-13-4

\twolineshloka
{मृते मयि गते रामे वनं मनुजपुङ्गवे}
{हन्तानार्ये ममामित्रे सकामा सुखिनी भव} %2-13-5

\twolineshloka
{स्वर्गेऽपि खलु रामस्य कुशलं दैवतैरहम्}
{प्रत्यादेशादभिहितं धारयिष्ये कथं बत} %2-13-6

\twolineshloka
{कैकेय्याः प्रियकामेन रामः प्रव्राजितो वनम्}
{यदि सत्यं ब्रवीम्येतत् तदसत्यं भविष्यति} %2-13-7

\twolineshloka
{अपुत्रेण मया पुत्रः श्रमेण महता महान्}
{रामो लब्धो महातेजाः स कथं त्यज्यते मया} %2-13-8

\twolineshloka
{शूरश्च कृतविद्यश्च जितक्रोधः क्षमापरः}
{कथं कमलपत्राक्षो मया रामो विवास्यते} %2-13-9

\twolineshloka
{कथमिन्दीवरश्यामं दीर्घबाहुं महाबलम्}
{अभिराममहं रामं स्थापयिष्यामि दण्डकान्} %2-13-10

\twolineshloka
{सुखानामुचितस्यैव दुःखैरनुचितस्य च}
{दुःखं नामानुपश्येयं कथं रामस्य धीमतः} %2-13-11

\twolineshloka
{यदि दुःखमकृत्वा तु मम सङ्क्रमणं भवेत्}
{अदुःखार्हस्य रामस्य ततः सुखमवाप्नुयाम्} %2-13-12

\twolineshloka
{नृशंसे पापसङ्कल्पे रामं सत्यपराक्रमम्}
{किं विप्रियेण कैकेयि प्रियं योजयसे मम} %2-13-13

\twolineshloka
{अकीर्तिरतुला लोके ध्रुवं परिभविष्यति}
{तथा विलपतस्तस्य परिभ्रमितचेतसः} %2-13-14

\twolineshloka
{अस्तमभ्यागमत् सूर्यो रजनी चाभ्यवर्तत}
{सा त्रियामा तदार्तस्य चन्द्रमण्डलमण्डिता} %2-13-15

\twolineshloka
{राज्ञो विलपमानस्य न व्यभासत शर्वरी}
{सदैवोष्णं विनिःश्वस्य वृद्धो दशरथो नृपः} %2-13-16

\twolineshloka
{विललापार्तवद् दुःखं गगनासक्तलोचनः}
{न प्रभातं त्वयेच्छामि निशे नक्षत्रभूषिते} %2-13-17

\twolineshloka
{क्रियतां मे दया भद्रे मयायं रचितोऽञ्जलिः}
{अथवा गम्यतां शीघ्रं नाहमिच्छामि निर्घृणाम्} %2-13-18

\twolineshloka
{नृशंसां केकयीं द्रष्टुं यत्कृते व्यसनं मम}
{एवमुक्त्वा ततो राजा कैकेयीं संयताञ्जलिः} %2-13-19

\twolineshloka
{प्रसादयामास पुनः कैकेयीं राजधर्मवित्}
{साधुवृत्तस्य दीनस्य त्वद्गतस्य गतायुषः} %2-13-20

\twolineshloka
{प्रसादः क्रियतां भद्रे देवि राज्ञो विशेषतः}
{शून्ये न खलु सुश्रोणि मयेदं समुदाहृतम्} %2-13-21

\twolineshloka
{कुरु साधुप्रसादं मे बाले सहृदया ह्यसि}
{प्रसीद देवि रामो मे त्वद्दत्तं राज्यमव्ययम्} %2-13-22

\threelineshloka
{लभतामसितापाङ्गे यशः परमवाप्स्यसि}
{मम रामस्य लोकस्य गुरूणां भरतस्य च}
{प्रियमेतद् गुरुश्रोणि कुरु चारुमुखेक्षणे} %2-13-23

\twolineshloka
{विशुद्धभावस्य हि दुष्टभावा दीनस्य ताम्राश्रुकलस्य राज्ञः}
{श्रुत्वा विचित्रं करुणं विलापं भर्तुर्नृशंसा न चकार वाक्यम्} %2-13-24

\twolineshloka
{ततः स राजा पुनरेव मूर्च्छितः प्रियामतुष्टां प्रतिकूलभाषिणीम्}
{समीक्ष्य पुत्रस्य विवासनं प्रति क्षितौ विसंज्ञो निपपात दुःखितः} %2-13-25

\twolineshloka
{इतीव राज्ञो व्यथितस्य सा निशा जगाम घोरं श्वसतो मनस्विनः}
{विबोध्यमानः प्रतिबोधनं तदा निवारयामास स राजसत्तमः} %2-13-26


॥इत्यार्षे श्रीमद्रामायणे वाल्मीकीये आदिकाव्ये अयोध्याकाण्डे दशरथविलापः नाम त्रयोदशः सर्गः ॥२-१३॥
