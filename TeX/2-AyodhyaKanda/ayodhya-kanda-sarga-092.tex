\sect{द्विनवतितमः सर्गः — भरद्वाजामन्त्रणम्}

\twolineshloka
{ततस्तां रजनीं व्युष्य भरतः सपरिच्छदः}
{कृतातिथ्यो भरद्वाजं कामादभिजगाम ह} %2-92-1

\twolineshloka
{तमृषिः पुरुषव्याघ्रं प्राञ्जलिं प्रेक्ष्य चागतम्}
{हुताग्निहोत्रो भरतं भरद्वाजोऽभ्यभाषत} %2-92-2

\twolineshloka
{कच्चिदत्र सुखा रात्रिस्तवास्मद्विषये गता}
{समग्रस्ते जनः कच्चिदातिथ्ये शंस मेऽनघ} %2-92-3

\twolineshloka
{तमुवाचाञ्जलिं कृत्वा भरतोऽभिप्रणम्य च}
{आश्रमादभिनिष्क्रान्तमृषिमुत्तमतेजसम्} %2-92-4

\twolineshloka
{सुखोषितोऽस्मि भगवन् समग्रबलवाहनः}
{तर्पितः सर्वकामैश्च सामात्यो बलवत्त्वया} %2-92-5

\twolineshloka
{अपेतक्लमसन्तापाः सुभिक्षाः सुप्रतिश्रयाः}
{अपि प्रेष्यानुपादाय सर्वे स्म सुसुखोषिताः} %2-92-6

\twolineshloka
{आमन्त्रयेऽहं भगवन् कामं त्वामृषिसत्तम}
{समीपं प्रस्थितं भ्रातुर्मैत्रेणेक्षस्व चक्षुषा} %2-92-7

\twolineshloka
{आश्रमं तस्य धर्मज्ञ धार्मिकस्य महात्मनः}
{आचक्ष्व कतमो मार्गः कियानिति च शंस मे} %2-92-8

\twolineshloka
{इति पृष्टस्तु भरतं भ्रातृदर्शनलालसम्}
{प्रत्युवाच महातेजा भरद्वाजो महातपाः} %2-92-9

\twolineshloka
{भरतार्द्धतृतीयेषु योजनेष्वजने वने}
{चित्रकूटो गिरिस्तत्र रम्यनिर्दरकाननः} %2-92-10

\twolineshloka
{उत्तरं पार्श्वमासाद्य तस्य मन्दाकिनी नदी}
{पुष्पितद्रुमसञ्छन्ना रम्यपुष्पितकानना} %2-92-11

\twolineshloka
{अनन्तरं तत्सरितश्चित्रकूटश्च पर्वतः}
{तयोः पर्णकुटी तात तत्र तौ वसतो ध्रुवम्} %2-92-12

\threelineshloka
{दक्षिणेनैव मार्गेण सव्यदक्षिणमेव वा}
{गजवाजिरथाकीर्णां वाहिनीं वाहिनीपते}
{वाहयस्व महाभाग ततो द्रक्ष्यसि राघवम्} %2-92-13

\twolineshloka
{प्रयाणमिति तच्छ्रुत्वा राजराजस्य योषितः}
{हित्वा यानानि यानार्हा ब्राह्मणं पर्य्यवारयन्} %2-92-14

\twolineshloka
{वेपमाना कृशा दीना सह देव्या सुमित्रया}
{कौसल्या तत्र जग्राह कराभ्यां चरणौ मुनेः} %2-92-15

\twolineshloka
{असमृद्धेन कामेन सर्वलोकस्य गर्हिता}
{कैकेयी तस्य जग्राह चरणौ सव्यपत्रपा} %2-92-16

\twolineshloka
{तं प्रदक्षिणमागम्य भगवन्तं महामुनिम्}
{अदूराद्भरतस्यैव तस्थौ दीनमनास्तदा} %2-92-17

\twolineshloka
{ततः पप्रच्छ भरतं भरद्वाजो दृढव्रतः}
{विशेषं ज्ञातुमिच्छामि मातऽणां तव राघव} %2-92-18

\twolineshloka
{एवमुक्तस्तु भरतो भरद्वाजेन धार्मिकः}
{उवाच प्राञ्जलिर्भूत्वा वाक्यं वचनकोविदः} %2-92-19

\twolineshloka
{यामिमां भगवन् दीनां शोकानशनकर्शिताम्}
{पितुर्हि महिषीं देवीं देवतामिव पश्यसि} %2-92-20

\twolineshloka
{एषा तं पुरुषव्याघ्रं सिंहविक्रान्तगामिनम्}
{कौसल्या सुषुवे रामं धातारमदितिर्यथा} %2-92-21

\twolineshloka
{अस्यावामभुजं श्लिष्टा यैषा तिष्ठति दुर्मनाः}
{कर्णिकारस्य शाखेव शीर्णपुष्पा वनान्तरे} %2-92-22

\twolineshloka
{एतस्यास्तु सुतौ देव्याः कुमारौ देववर्णिनौ}
{उभौ लक्ष्मणशत्रुघ्नौ वीरौ सत्यपराक्रमौ} %2-92-23

\twolineshloka
{यस्याः कृते नरव्याघ्रौ जीवनाशमितो गतौ}
{राजपुत्रविहीनश्च स्वर्गं दशरथो गतः} %2-92-24

\twolineshloka
{क्रोधनामकृतप्रज्ञां दृप्तां सुभगमानिनीम्}
{ऐश्वर्यकामां कैकेयीमनार्य्यामार्यरूपिणीम्} %2-92-25

\twolineshloka
{ममैतां मातरं विद्धि नृशंसां पापनिश्चयाम्}
{यतो मूलं हि पश्यामि व्यसनं महदात्मनः} %2-92-26

\twolineshloka
{इत्युक्त्वा नरशार्दूलो बाष्पगद्गदया गिरा}
{स निशश्वास ताम्राक्षो नागः क्रुद्ध इव श्वसन्} %2-92-27

\twolineshloka
{भरद्वाजो महर्षिस्तं ब्रुवन्तं भरतं तथा}
{प्रत्युवाच महाबुद्धिरिदं वचनमर्थवत्} %2-92-28

\twolineshloka
{न दोषेणावगन्तव्या कैकेयी भरत त्वया}
{रामप्रव्राजनं ह्येतत् सुखोदर्कं भविष्यति} %2-92-29

\twolineshloka
{देवानां दानवानां च ऋषीणां भावितात्मनाम्}
{हितमेव भविष्यद्धि रामप्रव्राजनादिह} %2-92-30

\twolineshloka
{अभिवाद्य तु संसिद्धः कृत्वा चैनं प्रदक्षिणम्}
{आमन्त्र्य भरतः सेन्यं युज्यतामित्यचोदयत्} %2-92-31

\twolineshloka
{ततो वाजिरथान् युक्त्वा दिव्यान् हेमपरिष्कृतान्}
{अध्यारोहत् प्रयाणार्थी बहून् बहुविधो जनः} %2-92-32

\twolineshloka
{गजकन्या गजाश्चैव हेमकक्ष्याः पताकिनः}
{जीमूता इव घर्मान्ते सघोषाः सम्प्रतस्थिरे} %2-92-33

\twolineshloka
{विविधान्यपि यानानि महान्ति च लघूनि च}
{प्रययुः सुमहार्हाणि पादैरेव पदातयः} %2-92-34

\twolineshloka
{अथ यानप्रवेकैस्तु कौसल्याप्रमुखाः स्त्रियः}
{रामदर्शनकाङ्क्षिण्यः प्रययुर्मुदितास्तदा} %2-92-35

\twolineshloka
{चन्द्रार्कतरुणाभासां नियुक्तां शिबिकां शुभाम्}
{आस्थाय प्रययौ श्रीमान् भरतः सपरिच्छदः} %2-92-36

\twolineshloka
{सा प्रयाता महासेना गजवाजिरथाकुला}
{दक्षिणां दिशमावृत्य महामेघ इवोत्थितः} %2-92-37

\twolineshloka
{वनानि तु व्यतिक्रम्य जुष्टानि मृगपक्षिभिः}
{गङ्गायाः परवेलायां गिरिष्वपि नदीषु च} %2-92-38

\twolineshloka
{सा सम्प्रहृष्टद्विजवाजियोधा वित्रासयन्ती मृगपक्षिसङ्घान्}
{महद्वनं तत्प्रतिगाहमाना रराज सेना भरतस्य तत्र} %2-92-39


॥इत्यार्षे श्रीमद्रामायणे वाल्मीकीये आदिकाव्ये अयोध्याकाण्डे भरद्वाजामन्त्रणम् नाम द्विनवतितमः सर्गः ॥२-९२॥
