\sect{चतुःषष्ठितमः सर्गः — दशरथदिष्टान्तः}

\twolineshloka
{वधमप्रतिरूपम् तु महर्षेस्तस्य राघवः}
{विलपन्ने व धर्मात्मा कौसल्याम् पुन रब्रवीत्} %2-64-1

\twolineshloka
{तत् अज्ञानान् महत् पापम् कृत्वा सम्कुलित इन्द्रियः}
{एकः तु अचिन्तयम् बुद्ध्या कथम् नु सुकृतम् भवेत्} %2-64-2

\twolineshloka
{ततः तम् घटम् आदय पूर्णम् परम वारिणा}
{आश्रमम् तम् अहम् प्राप्य यथा आख्यात पथम् गतः} %2-64-3

\twolineshloka
{तत्र अहम् दुर्बलाव् अन्धौ वृद्धाव् अपरिणायकौ}
{अपश्यम् तस्य पितरौ लून पक्षाव् इव द्विजौ} %2-64-4

\twolineshloka
{तन् निमित्ताभिर् आसीनौ कथाभिर् अपरिक्रमौ}
{ताम् आशाम् मत् कृते हीनाव् उदासीनाव् अनाथवत्} %2-64-5

\twolineshloka
{शोकोपहतचित्तश्च भयसम्त्रस्तचेतनः}
{तच्चाश्रमपदम् गत्वा भूयः शोकमहम् गतः} %2-64-6

\twolineshloka
{पद शब्दम् तु मे श्रुत्वा मुनिर् वाक्यम् अभाषत}
{किम् चिरायसि मे पुत्र पानीयम् क्षिप्रम् आनय} %2-64-7

\twolineshloka
{यन् निमित्तम् इदम् तात सलिले क्रीडितम् त्वया}
{उत्कण्ठिता ते माता इयम् प्रविश क्षिप्रम् आश्रमम्} %2-64-8

\twolineshloka
{यद् व्यलीकम् कृतम् पुत्र मात्रा ते यदि वा मया}
{न तन् मनसि कर्तव्यम् त्वया तात तपस्विना} %2-64-9

\twolineshloka
{त्वम् गतिस् तु अगतीनाम् च चक्षुस् त्वम् हीन चक्षुषाम्}
{समासक्ताः त्वयि प्राणाः किम्चिन् नौ न अभिभाषसे} %2-64-10

\twolineshloka
{मुनिम् अव्यक्तया वाचा तम् अहम् सज्जमानया}
{हीन व्यन्जनया प्रेक्ष्य भीतः भीतैव अब्रुवम्} %2-64-11

\twolineshloka
{मनसः कर्म चेष्टाभिर् अभिसम्स्तभ्य वाग् बलम्}
{आचचक्षे तु अहम् तस्मै पुत्र व्यसनजम् भयम्} %2-64-12

\twolineshloka
{क्षत्रियो अहम् दशरथो न अहम् पुत्रः महात्मनः}
{सज्जन अवमतम् दुह्खम् इदम् प्राप्तम् स्व कर्मजम्} %2-64-13

\twolineshloka
{भगवमः च अपहस्तः अहम् सरयू तीरम् आगतः}
{जिघाम्सुः श्वा पदम् किम्चिन् निपाने वा आगतम् गजम्} %2-64-14

\twolineshloka
{ततः श्रुतः मया शब्दो जले कुम्भस्य पूर्यतः}
{द्विपो अयम् इति मत्वा हि बाणेन अभिहतः मया} %2-64-15

\twolineshloka
{गत्वा नद्याः ततः तीरम् अपश्यम् इषुणा हृदि}
{विनिर्भिन्नम् गत प्राणम् शयानम् भुवि तापसम्} %2-64-16

\twolineshloka
{भगवन् शब्दम् आलक्ष्य मया गज जिघाम्सुना}
{विसृष्टः अम्भसि नाराचः तेन ते निहतः सुतः} %2-64-17

\twolineshloka
{ततस्तस्यैव वचनादुपेत्य परितप्यतः}
{स मया सहसा बण उद्धृतो मर्मतस्तदा} %2-64-18

\twolineshloka
{स च उद्धृतेन बाणेन तत्र एव स्वर्गम् आस्थितः}
{भगवन्ताव् उभौ शोचन्न् अन्धाव् इति विलप्य च} %2-64-19

\twolineshloka
{अज्ञानात् भवतः पुत्रः सहसा अभिहतः मया}
{शेषम् एवम् गते यत् स्यात् तत् प्रसीदतु मे मुनिः} %2-64-20

\twolineshloka
{स तत् श्रुत्वा वचः क्रूरम् निह्श्वसन् शोक कर्शितः}
{नाशकत्तीव्रमायासमकर्तुम् भगवानृषिः} %2-64-21

\twolineshloka
{सबाष्पपूर्णवदनो निःश्वसन् शोककर्शितः}
{माम् उवाच महा तेजाः कृत अन्जलिम् उपस्थितम्} %2-64-22

\twolineshloka
{यद्य् एतत् अशुभम् कर्म न स्म मे कथयेः स्वयम्}
{फलेन् मूर्धा स्म ते राजन् सद्यः शत सहस्रधा} %2-64-23

\twolineshloka
{क्षत्रियेण वधो राजन् वानप्रस्थे विशेषतः}
{ज्ञान पूर्वम् कृतः स्थानाच् च्यावयेद् अपि वज्रिणम्} %2-64-24

\twolineshloka
{सप्तधा तु फलेन्मूर्धा मुनौ तपसि तिष्ठति}
{ज्ञानाद्विसृजतः शस्त्रम् तादृशे ब्रह्मचारिणि} %2-64-25

\twolineshloka
{अज्ञानाद्द् हि कृतम् यस्मात् इदम् तेन एव जीवसि}
{अपि हि अद्य कुलम् नस्यात् राघवाणाम् कुतः भवान्} %2-64-26

\twolineshloka
{नय नौ नृप तम् देशम् इति माम् च अभ्यभाषत}
{अद्य तम् द्रष्टुम् इच्चावः पुत्रम् पश्चिम दर्शनम्} %2-64-27

\twolineshloka
{रुधिरेण अवसित अन्गम् प्रकीर्ण अजिन वाससम्}
{शयानम् भुवि निह्सम्ज्ञम् धर्म राज वशम् गतम्} %2-64-28

\twolineshloka
{अथ अहम् एकः तम् देशम् नीत्वा तौ भृश दुह्खितौ}
{अस्पर्शयम् अहम् पुत्रम् तम् मुनिम् सह भार्यया} %2-64-29

\twolineshloka
{तौ पुत्रम् आत्मनः स्पृष्ट्वा तम् आसाद्य तपस्विनौ}
{निपेततुः शरीरे अस्य पिता च अस्य इदम् अब्रवीत्} %2-64-30

\twolineshloka
{न न्व् अहम् ते प्रियः पुत्र मातरम् पश्य धार्मिक}
{किम् नु न आलिन्गसे पुत्र सुकुमार वचो वद} %2-64-31

\twolineshloka
{न त्वहम् ते प्रियः पुत्र मातरम् पस्य धार्मिक}
{किम् नु नालिङ्गसे पुत्र सुकुमार वचो वद} %2-64-32

\twolineshloka
{कस्य वा अपर रात्रे अहम् श्रोष्यामि हृदयम् गमम्}
{अधीयानस्य मधुरम् शास्त्रम् वा अन्यद् विशेषतः} %2-64-33

\twolineshloka
{को माम् सम्ध्याम् उपास्य एव स्नात्वा हुत हुत अशनः}
{श्लाघयिष्यति उपासीनः पुत्र शोक भय अर्दितम्} %2-64-34

\twolineshloka
{कन्द मूल फलम् हृत्वा को माम् प्रियम् इव अतिथिम्}
{भोजयिष्यति अकर्मण्यम् अप्रग्रहम् अनायकम्} %2-64-35

\twolineshloka
{इमाम् अन्धाम् च वृद्धाम् च मातरम् ते तपस्विनीम्}
{कथम् पुत्र भरिष्यामि कृपणाम् पुत्र गर्धिनीम्} %2-64-36

\twolineshloka
{तिष्ठ मा मा गमः पुत्र यमस्य सदनम् प्रति}
{श्वो मया सह गन्ता असि जनन्या च समेधितः} %2-64-37

\twolineshloka
{उभाव् अपि च शोक आर्ताव् अनाथौ कृपणौ वने}
{क्षिप्रम् एव गमिष्यावः त्वया हीनौ यम क्षयम्} %2-64-38

\twolineshloka
{ततः वैवस्वतम् दृष्ट्वा तम् प्रवक्ष्यामि भारतीम्}
{क्षमताम् धर्म राजो मे बिभृयात् पितराव् अयम्} %2-64-39

\twolineshloka
{दातुमर्हति धर्मात्मा लोकपालो महायशाः}
{ईदृषस्य ममाक्षय्या मेकामभयदक्षिणाम्} %2-64-40

\twolineshloka
{अपापो असि यथा पुत्र निहतः पाप कर्मणा}
{तेन सत्येन गच्च आशु ये लोकाः शस्त्र योधिनाम्} %2-64-41

\twolineshloka
{यान्ति शूरा गतिम् याम् च सम्ग्रामेष्व् अनिवर्तिनः}
{हताः तु अभिमुखाः पुत्र गतिम् ताम् परमाम् व्रज} %2-64-42

\twolineshloka
{याम् गतिम् सगरः शैब्यो दिलीपो जनमेजयः}
{नहुषो धुन्धुमारः च प्राप्ताः ताम् गच्च पुत्रक} %2-64-43

\twolineshloka
{या गतिः सर्व साधूनाम् स्वाध्यायात् पतसः च या}
{भूमिदस्य आहित अग्नेः चएक पत्नी व्रतस्य च} %2-64-44

\twolineshloka
{गो सहस्र प्रदातृऋणाम् या या गुरुभृताम् अपि}
{देह न्यास कृताम् या च ताम् गतिम् गच्च पुत्रक} %2-64-45

\twolineshloka
{न हि तु अस्मिन् कुले जातः गच्चति अकुशलाम् गतिम्}
{स तु यास्यति येन त्वम् निहतो मम बान्धवः} %2-64-46

\twolineshloka
{एवम् स कृपणम् तत्र पर्यदेवयत असकृत्}
{ततः अस्मै कर्तुम् उदकम् प्रवृत्तः सह भार्यया} %2-64-47

\twolineshloka
{स तु दिव्येन रूपेण मुनि पुत्रः स्व कर्मभिः}
{स्वर्गमाध्यारुहत् ख्षिप्रम् शक्रेण सह खर्मवित्} %2-64-48

\twolineshloka
{आबभाषे च वृद्धौ तौ सह शक्रेण तापसः}
{आश्वास्य च मुहूर्तम् तु पितरौ वाक्यम् अब्रवीत्} %2-64-49

\twolineshloka
{स्थानम् अस्मि महत् प्राप्तः भवतोह् परिचारणात्}
{भवन्ताव् अपि च क्षिप्रम् मम मूलम् उपैष्यतः} %2-64-50

\twolineshloka
{एवम् उक्त्वा तु दिव्येन विमानेन वपुष्मता}
{आरुरोह दिवम् क्षिप्रम् मुनि पुत्रः जित इन्द्रियः} %2-64-51

\twolineshloka
{स कृत्वा तु उदकम् तूर्णम् तापसः सह भार्यया}
{माम् उवाच महा तेजाः कृत अन्जलिम् उपस्थितम्} %2-64-52

\twolineshloka
{अद्य एव जहि माम् राजन् मरणे न अस्ति मे व्यथा}
{यत् शरेण एक पुत्रम् माम् त्वम् अकार्षीर् अपुत्रकम्} %2-64-53

\twolineshloka
{त्वया तु यद् अविज्ञानान् निहतः मे सुतः शुचिः}
{तेन त्वाम् अभिशप्स्यामि सुदुह्खम् अतिदारुणम्} %2-64-54

\twolineshloka
{पुत्र व्यसनजम् दुह्खम् यद् एतन् मम साम्प्रतम्}
{एवम् त्वम् पुत्र शोकेन राजन् कालम् करिष्यसि} %2-64-55

\twolineshloka
{अज्ञानात्तु हतो यस्मात् क्षत्रियेण त्वया मुनिः}
{तस्मात्त्वाम् नाविशत्याशु ब्रह्महत्या नराधिप} %2-64-56

\twolineshloka
{त्वामप्येतादृशो भावः क्षिप्रमेव गमिष्यति}
{जीवितान्तकरो घोरो दातारमिव दक्षिणा} %2-64-57

\twolineshloka
{एवम् शापम् मयि न्यस्य विलप्य करुणम् बहु}
{चितामारोप्य देहम् तन्मिथुनम् स्वर्गमभ्ययात्} %2-64-58

\twolineshloka
{तदेतच्चिन्तयानेन स्मऋतम् पापम् मया स्वयम्}
{तदा बाल्यात्कृतम् देवि शब्दवेध्यनुकर्षिणा} %2-64-59

\twolineshloka
{तस्यायम् कर्मणो देवि विपाकः समुपस्थितः}
{अपथ्यैः सह सम्भुक्ते व्याधिरन्नरसे यथा} %2-64-60

\twolineshloka
{तस्मान् माम् आगतम् भद्रे तस्य उदारस्य तत् वचः}
{यद् अहम् पुत्र शोकेन सम्त्यक्ष्याम्य् अद्य जीवितम्} %2-64-61

\twolineshloka
{चक्षुर्भ्याम् त्वाम् न पश्यामि कौसल्ये साधु माम्स्फृश}
{इत्युक्त्वा स रुदम्स्त्रस्तो भार्यामाह च भूमिपः} %2-64-62

\twolineshloka
{एतन्मे सदृशम् देवि यन्मया राघवे कृतम्}
{सदृशम् तत्तु तस्यैव यदनेन कृतम् मयि} %2-64-63

\twolineshloka
{दुर्वृत्तमपि कः पुत्रम् त्यजेद्भुवि विचक्षणः}
{कश्च प्रव्राज्यमानो वा नासूयेत्पितरम् सुतः} %2-64-64

\twolineshloka
{यदि माम् सम्स्पृशेद् रामः सकृदद्य लभेत वा}
{यमक्षयमनुप्राप्ता द्रक्ष्यन्ति न हि मानवाः} %2-64-65

\twolineshloka
{चक्षुषा त्वाम् न पश्यामि स्मृतिर् मम विलुप्यते}
{दूता वैवस्वतस्य एते कौसल्ये त्वरयन्ति माम्} %2-64-66

\twolineshloka
{अतः तु किम् दुह्खतरम् यद् अहम् जीवित क्षये}
{न हि पश्यामि धर्मज्ञम् रामम् सत्य पराक्यमम्} %2-64-67

\twolineshloka
{तस्यादर्शनजः शोकः सुतस्याप्रतिकर्मणः}
{उच्चोषयति मे प्राणान्वारि स्तोकमिवातवः} %2-64-68

\twolineshloka
{न ते मनुष्या देवाः ते ये चारु शुभ कुण्डलम्}
{मुखम् द्रक्ष्यन्ति रामस्य वर्षे पन्च दशे पुनः} %2-64-69

\twolineshloka
{पद्म पत्र ईक्षणम् सुभ्रु सुदम्ष्ट्रम् चारु नासिकम्}
{धन्या द्रक्ष्यन्ति रामस्य तारा अधिप निभम् मुखम्} %2-64-70

\twolineshloka
{सदृशम् शारदस्य इन्दोह् फुल्लस्य कमलस्य च}
{सुगन्धि मम नाथस्य धन्या द्रक्ष्यन्ति तन् मुखम्} %2-64-71

\twolineshloka
{निवृत्त वन वासम् तम् अयोध्याम् पुनर् आगतम्}
{द्रक्ष्यन्ति सुखिनो रामम् शुक्रम् मार्ग गतम् यथा} %2-64-72

\twolineshloka
{कौसल्ये चित्त मोहेन हृदयम् सीदतीव मे}
{वेदये न च समुक्तान् शब्दस्पर्शरसानहम्} %2-64-73

\twolineshloka
{चित्तनाशाद्विपद्यन्ते सर्वाण्येवेन्द्रियाणि मे}
{क्षिणस्नेहस्य दीपस्य सम्सक्ता रश्मयो यथा} %2-64-74

\twolineshloka
{अयम् आत्म भवः शोको माम् अनाथम् अचेतनम्}
{सम्सादयति वेगेन यथा कूलम् नदी रयः} %2-64-75

\twolineshloka
{हा राघव महा बाहो हा मम आयास नाशन}
{हा पितृप्रिय मे नाथ हाद्य क्वासि गतः सुत} %2-64-76

\twolineshloka
{हा कौसल्ये नशिष्यामि हा सुमित्रे तपस्विनि}
{हा नृशम्से ममामित्रे कैकेयि कुलपाम्सनि} %2-64-77

\twolineshloka
{इति रामस्य मातुश्च सुमित्रायाश्च सन्निधौ}
{राजा दशरथः शोचन् जीवित अन्तम् उपागमत्} %2-64-78

\fourlineindentedshloka
{यथा तु दीनम् कथयन् नर अधिपः}
{प्रियस्य पुत्रस्य विवासन आतुरः}
{गते अर्ध रात्रे भृश दुह्ख पीडितः}
{तदा जहौ प्राणम् उदार दर्शनः} %2-64-79


॥इत्यार्षे श्रीमद्रामायणे वाल्मीकीये आदिकाव्ये अयोध्याकाण्डे दशरथदिष्टान्तः नाम चतुःषष्ठितमः सर्गः ॥२-६४॥
