\sect{त्रिसप्ततितमः सर्गः — कैकेयीविगर्हणम्}

\twolineshloka
{श्रुत्वा तु पितरम् वृत्तम् भ्रातरु च विवासितौ}
{भरतः दुह्ख सम्तप्तैदम् वचनम् अब्रवीत्} %2-73-1

\twolineshloka
{किम् नुण्कार्यम् हतस्य इह मम राज्येन शोचतः}
{विहीनस्य अथ पित्रा च भ्रात्रा पितृ समेन च} %2-73-2

\twolineshloka
{दुह्खे मे दुह्खम् अकरोर् व्रणे क्षारम् इव आदधाः}
{राजानम् प्रेत भावस्थम् कृत्वा रामम् च तापसम्} %2-73-3

\twolineshloka
{कुलस्य त्वम् अभावाय काल रात्रिर् इव आगता}
{अन्गारम् उपगूह्य स्म पिता मे न अवबुद्धवान्} %2-73-4

\twolineshloka
{मृत्युमापादितो राजा त्वया मे पापदर्शिनि}
{सुखम् परिहृतम् मोहात्कुलेऽस्मिन् कुलपाम्सनि} %2-73-5

\twolineshloka
{त्वाम् प्राप्य हि पिता मेद्य सत्यसन्धो महायशाः}
{तीव्रदुःखाभिसम्तप्तो वृत्तो दशरथो नृपः} %2-73-6

\twolineshloka
{विनाशितो महाराजः पिता मे धर्मवत्सलः}
{कस्मात्प्रव्राजितो रामः कस्मादेव वनम् गतः} %2-73-7

\twolineshloka
{कौसल्या च सुमित्रा च पुत्र शोक अभिपीडिते}
{दुष्करम् यदि जीवेताम् प्राप्य त्वाम् जननीम् मम} %2-73-8

\twolineshloka
{ननु तु आर्यो अपि धर्म आत्मा त्वयि वृत्तिम् अनुत्तमाम्}
{वर्तते गुरु वृत्तिज्ञो यथा मातरि वर्तते} %2-73-9

\twolineshloka
{तथा ज्येष्ठा हि मे माता कौसल्या दीर्घ दर्शिनी}
{त्वयि धर्मम् समास्थाय भगिन्याम् इव वर्तते} %2-73-10

\twolineshloka
{तस्याः पुत्रम् कृत आत्मानम् चीर वल्कल वाससम्}
{प्रस्थाप्य वन वासाय कथम् पापे न शोचसि} %2-73-11

\twolineshloka
{अपाप दर्शिनम् शूरम् कृत आत्मानम् यशस्विनम्}
{प्रव्राज्य चीर वसनम् किम् नु पश्यसि कारणम्} %2-73-12

\twolineshloka
{लुब्धाया विदितः मन्ये न ते अहम् राघवम् प्रति}
{तथा हि अनर्थो राज्य अर्थम् त्वया नीतः महान् अयम्} %2-73-13

\twolineshloka
{अहम् हि पुरुष व्याघ्राव् अपश्यन् राम लक्ष्मणौ}
{केन शक्ति प्रभावेन राज्यम् रक्षितुम् उत्सहे} %2-73-14

\twolineshloka
{तम् हि नित्यम् महा राजो बलवन्तम् महा बलः}
{उअपाश्रितः अभूद् धर्म आत्मा मेरुर् मेरु वनम् यथा} %2-73-15

\twolineshloka
{सो अहम् कथम् इमम् भारम् महा धुर्य समुद्यतम्}
{दम्यो धुरम् इव आसाद्य सहेयम् केन च ओजसा} %2-73-16

\twolineshloka
{अथ वा मे भवेत् शक्तिर् योगैः बुद्धि बलेन वा}
{सकामाम् न करिष्यामि त्वाम् अहम् पुत्र गर्धिनीम्} %2-73-17

\twolineshloka
{न मे विकाङ्खा जायेत त्यक्तुम् त्वाम् पापनिश्चयाम्}
{यदि रामस्य नावेक्षा त्वयि स्यान्मातृवत्सदा} %2-73-18

\twolineshloka
{उत्पन्ना तु कथम् बुद्धिस्तवेयम् पापदर्शिनि}
{साधुचारित्रविभ्राष्टे पूर्वेषाम् नो विगर्हिता} %2-73-19

\twolineshloka
{अस्मिन् कुले हि सर्वेषाम् ज्येष्ठो राज्येऽभिषिच्यते}
{अपरे भ्रातरस्तस्मिन् प्रवर्तन्ते समाहिताः} %2-73-20

\twolineshloka
{न हि मन्ये नृशसे त्वम् राजधर्ममवेक्षसे}
{गतिम् वा न विजानासि राजवृत्तस्य शाश्वतीम्} %2-73-21

\twolineshloka
{सततम् राजवृत्ते हि ज्येष्ठो राज्येऽभिषिच्यते}
{राज्ञामेतत्समम् तत्स्यादिक्ष्वाकूणाम् विशेषतः} %2-73-22

\twolineshloka
{तेषाम् धर्मैकरक्षाणाम् कुलचारित्रयोगिनाम्}
{अत्र चारित्रशौण्डीर्यम् त्वाम् प्राप्य विनिवर्ततम्} %2-73-23

\twolineshloka
{तवापि सुमहाभागा जनेन्द्राः कुलपूर्वगाः}
{बुद्धेर्मोहः कथमयम् सम्भूतस्त्वयि गर्हितः} %2-73-24

\twolineshloka
{न तु कामम् करिष्यामि तवाऽह्म् पापनिश्चये}
{त्वया व्यसनमारब्धम् जीवितान्तकरम् मम} %2-73-25

\twolineshloka
{एष त्विदानीमेवाहमप्रियार्थम् तवनघम्}
{निवर्तयिष्यामि वनात् भ्रातरम् स्वजन प्रियम्} %2-73-26

\twolineshloka
{निवर्तयित्वा रामम् च तस्याहम् दीप्ततेजनः}
{दासभूतो भविष्यामि सुस्थिरेणान्तरात्मना} %2-73-27

\fourlineindentedshloka
{इति एवम् उक्त्वा भरतः महात्मा}
{प्रिय इतरैः वाक्य गणैअः तुदम्स् ताम्}
{शोक आतुरः च अपि ननाद भूयः}
{सिम्हो यथा पर्वत गह्वरस्थः} %2-73-28


॥इत्यार्षे श्रीमद्रामायणे वाल्मीकीये आदिकाव्ये अयोध्याकाण्डे कैकेयीविगर्हणम् नाम त्रिसप्ततितमः सर्गः ॥२-७३॥
