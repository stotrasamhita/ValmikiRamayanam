\sect{एकादशाधिकशततमः सर्गः — भरतानुशासनम्}

\twolineshloka
{वसिष्ठस्तु तदा राममुक्त्वा राजपुरोऺहितः}
{अब्रवीद्धर्मसंयुक्तं पुनरेवापरं वचः} %2-111-1

\twolineshloka
{पुरुषस्येह जातस्य भवन्ति गुरवस्त्रयः}
{आचार्य्यश्चैव काकुत्स्थ पिता माता च राघव} %2-111-2

\twolineshloka
{पिता ह्येनं जनयति पुरुषं पुरुषर्षभ}
{प्रज्ञां ददाति चाचार्यस्तस्मात्स गुरुरुच्यते} %2-111-3

\twolineshloka
{सोऽहं ते पितुराचार्य्यस्तव चैव परन्तप}
{मम त्वं वचनं कुर्वन् नातिवर्त्तेः सताङ्गतिम्} %2-111-4

\twolineshloka
{इमा हि ते परिषदः श्रेणयश्च द्विजास्तथा}
{एषु तात चरन् धर्मं नातिवर्त्तेः सताङ्गतिम्} %2-111-5

\twolineshloka
{वृद्धाया धर्मशीलाया मातुर्नार्हस्यवर्त्तितुम्}
{अस्यास्तु वचनं कुर्वन् नातिवर्त्तेः सताङ्गतिम्} %2-111-6

\twolineshloka
{भरतस्य वचः कुर्वन् याचमानस्य राघव}
{आत्मानं नातिवर्त्तेस्त्वं सत्यधर्मपराक्रम} %2-111-7

\twolineshloka
{एवं मधुरमुक्तस्तु गुरुणा राघवः स्वयम्}
{प्रत्युवाच समासीनं वसिष्ठं पुरुषर्षभः} %2-111-8

\twolineshloka
{यन्मातापितरौ वृत्तं तनये कुरुतः सदा}
{न सुप्रतिकरं तत्तु मात्रा पित्रा च यत्कृतम्} %2-111-9

\twolineshloka
{यथाशक्ति प्रदानेन स्नापनोच्छादनेन च}
{नित्यं च प्रियवादेन तथा संवर्द्धनेन च} %2-111-10

\twolineshloka
{स हि राजा जनयिता पिता दशरथो मम}
{आज्ञातं यन्मया तस्य न तन्मिथ्या भविष्यति} %2-111-11

\twolineshloka
{एवमुक्तस्तु रामेण भरतः प्रत्यनन्तरम्}
{उवाच परमोदारः सूतं परमदुर्मनाः} %2-111-12

\twolineshloka
{इह मे स्थण्डिले शीघ्रं कुशानास्तर सारथे}
{आर्य्यं प्रत्युपवेक्ष्यामि यावन्मे न प्रसीदति} %2-111-13

\twolineshloka
{अनाहारो निरालोको धनहीनो यथा द्विजः}
{शेष्ये पुरस्तात् शालाया यावन्न प्रतियास्यति} %2-111-14

\twolineshloka
{स तु राममवेक्षन्तं सुमन्त्रं प्रेक्ष्य दुर्मनाः}
{कुशोत्तरमुपस्थाप्य भूमावेवास्तरत् स्वयम्} %2-111-15

\twolineshloka
{तमुवाच महातेजा रामो राजर्षिसत्तमः}
{किं मां भरत कुर्वाणं तात प्रत्युपवेक्ष्यसि} %2-111-16

\twolineshloka
{ब्राह्मणो ह्येकपार्श्वेन नरान् रोद्धुमिहार्हति}
{न तु मूर्द्धाभिषिक्तानां विधिः प्रत्युपवेशने} %2-111-17

\twolineshloka
{उत्तिष्ठ नरशार्दूल हित्वैतद्दारुणं व्रतम्}
{पुरवर्य्यामितः क्षिप्रमयोध्यां याहि राघव} %2-111-18

\twolineshloka
{आसीनस्त्वेव भरतः पौरजानपदं जनम्}
{उवाच सर्वतः प्रेक्ष्य किमार्यं नानुशासथ} %2-111-19

\twolineshloka
{ते तमूचुर्महात्मानं पौरजानपदा जनाः}
{काकुत्स्थमभिजानीमः सम्यग्वदति राघवः} %2-111-20

\twolineshloka
{एषोऽपि हि महाभागः पितुर्वचसि तिष्ठति}
{अत एव न शक्ताः स्मो व्यावर्त्तयितुमञ्जसा} %2-111-21

\twolineshloka
{तेषामाज्ञाय वचनं रामो वचनमब्रवीत्}
{एवं निबोध वचनं सुहृदां धर्मचक्षुषाम्} %2-111-22

\twolineshloka
{एतच्चैवोभयं श्रुत्वा सम्यक् सम्पश्य राघव}
{उत्तिष्ठ त्वं महाबाहो मां च स्पृश तथोदकम्} %2-111-23

\twolineshloka
{अथोत्थाय जलं स्पृष्ट्वा भरतो वाक्यमब्रवीत्}
{श्रृण्वन्तु मे परिषदो मन्त्रिणः श्रेणयस्तथा} %2-111-24

\twolineshloka
{न याचे पितरं राज्यं नानुशासामि मातरम्}
{आर्यं परमधर्मज्ञं नानुजानामि राघवम्} %2-111-25

\twolineshloka
{यदि त्ववश्यं वस्तव्यं कर्त्तव्यं च पितुर्वचः}
{अहमेव निवत्स्यामि चतुर्दश समा वने} %2-111-26

\twolineshloka
{धर्मात्मा तस्य तथ्येन भ्रातुर्वाक्येन विस्मितः}
{उवाच रामः सम्प्रेक्ष्य पौरजानपदं जनम्} %2-111-27

\twolineshloka
{विक्रीतमाहितं क्रीतं यत् पित्रा जीवता मम}
{न तल्लोपयितुं शक्यं मया वा भरतेन वा} %2-111-28

\twolineshloka
{अपधिर्न मया कार्य्यो वनवासे जुगुप्सितः}
{युक्तमुक्तं च कैकेय्या पित्रा मे सुकृतं कृतम्} %2-111-29

\twolineshloka
{जानामि भरतं क्षान्तं गुरुसत्कारकारिणम्}
{सर्वमेवात्र कल्याणं सत्यसन्धे महात्मनि} %2-111-30

\twolineshloka
{अनेन धर्मशीलेन वनात् प्रत्यागतः पुनः}
{भ्रात्रा सह भविष्यामि पृथिव्याः पतिरुत्तमः} %2-111-31

\twolineshloka
{वृतो राजा हि कैकेय्या मया तद्वचनं कृतम्}
{अनृतन्मोचयानेन पितरं तं महीपतिम्} %2-111-32


॥इत्यार्षे श्रीमद्रामायणे वाल्मीकीये आदिकाव्ये अयोध्याकाण्डे भरतानुशासनम् नाम एकादशाधिकशततमः सर्गः ॥२-१११॥
