\sect{एकादशः सर्गः — वरद्वयनिर्बन्धः}

\twolineshloka
{तं मन्मथशरैर्विद्धं कामवेगवशानुगम्}
{उवाच पृथिवीपालं कैकेयी दारुणं वचः} %2-11-1

\twolineshloka
{नास्मि विप्रकृता देव केनचिन्नावमानिता}
{अभिप्रायस्तु मे कश्चित् तमिच्छामि त्वया कृतम्} %2-11-2

\twolineshloka
{प्रतिज्ञां प्रतिजानीष्व यदि त्वं कर्तुमिच्छसि}
{अथ ते व्याहरिष्यामि यथाभिप्रार्थितं मया} %2-11-3

\twolineshloka
{तामुवाच महाराजः कैकेयीमीषदुत्स्मयः}
{कामी हस्तेन संगृह्य मूर्धजेषु भुवि स्थिताम्} %2-11-4

\twolineshloka
{अवलिप्ते न जानासि त्वत्तः प्रियतरो मम}
{मनुजो मनुजव्याघ्राद् रामादन्यो न विद्यते} %2-11-5

\twolineshloka
{तेनाजय्येन मुख्येन राघवेण महात्मना}
{शपे ते जीवनार्हेण ब्रूहि यन्मनसेप्सितम्} %2-11-6

\twolineshloka
{यं मुहूर्तमपश्यंस्तु न जीवे तमहं ध्रुवम्}
{तेन रामेण कैकेयि शपे ते वचनक्रियाम्} %2-11-7

\twolineshloka
{आत्मना चात्मजैश्चान्यैर्वृणे यं मनुजर्षभम्}
{तेन रामेण कैकेयि शपे ते वचनक्रियाम्} %2-11-8

\twolineshloka
{भद्रे हृदयमप्येतदनुमृश्योद्धरस्व मे}
{एतत् समीक्ष्य कैकेयि ब्रूहि यत् साधु मन्यसे} %2-11-9

\twolineshloka
{बलमात्मनि पश्यन्ती न विशङ्कितुमर्हसि}
{करिष्यामि तव प्रीतिं सुकृतेनापि ते शपे} %2-11-10

\twolineshloka
{सा तदर्थमना देवी तमभिप्रायमागतम्}
{निर्माध्यस्थ्याच्च हर्षाच्च बभाषे दुर्वचं वचः} %2-11-11

\twolineshloka
{तेन वाक्येन संहृष्टा तमभिप्रायमात्मनः}
{व्याजहार महाघोरमभ्यागतमिवान्तकम्} %2-11-12

\twolineshloka
{यथा क्रमेण शपसे वरं मम ददासि च}
{तच्छृण्वन्तु त्रयस्त्रिंशद् देवाः सेन्द्रपुरोगमाः} %2-11-13

\twolineshloka
{चन्द्रादित्यौ नभश्चैव ग्रहा रात्र्यहनी दिशः}
{जगच्च पृथिवी चेयं सगन्धर्वाः सराक्षसाः} %2-11-14

\twolineshloka
{निशाचराणि भूतानि गृहेषु गृहदेवताः}
{यानि चान्यानि भूतानि जानीयुर्भाषितं तव} %2-11-15

\twolineshloka
{सत्यसंधो महातेजा धर्मज्ञः सत्यवाक्शुचिः}
{वरं मम ददात्येष सर्वे शृण्वन्तु दैवताः} %2-11-16

\twolineshloka
{इति देवी महेष्वासं परिगृह्याभिशस्य च}
{ततः परमुवाचेदं वरदं काममोहितम्} %2-11-17

\twolineshloka
{स्मर राजन् पुरा वृत्तं तस्मिन् देवासुरे रणे}
{तत्र त्वां च्यावयच्छत्रुस्तव जीवितमन्तरा} %2-11-18

\twolineshloka
{तत्र चापि मया देव यत् त्वं समभिरक्षितः}
{जाग्रत्या यतमानायास्ततो मे प्रददौ वरौ} %2-11-19

\twolineshloka
{तौ दत्तौ च वरौ देव निक्षेपौ मृगयाम्यहम्}
{तवैव पृथिवीपाल सकाशे रघुनन्दन} %2-11-20

\twolineshloka
{तत् प्रतिश्रुत्य धर्मेण न चेद् दास्यसि मे वरम्}
{अद्यैव हि प्रहास्यामि जीवितं त्वद्विमानिता} %2-11-21

\twolineshloka
{वाङ्मात्रेण तदा राजा कैकेय्या स्ववशे कृतः}
{प्रचस्कन्द विनाशाय पाशं मृग इवात्मनः} %2-11-22

\twolineshloka
{ततः परमुवाचेदं वरदं काममोहितम्}
{वरौ देयौ त्वया देव तदा दत्तौ महीपते} %2-11-23

\twolineshloka
{तौ तावदहमद्यैव वक्ष्यामि शृणु मे वचः}
{अभिषेकसमारम्भो राघवस्योपकल्पितः} %2-11-24

\twolineshloka
{अनेनैवाभिषेकेण भरतो मेऽभिषिच्यताम्}
{यो द्वितीयो वरो देव दत्तः प्रीतेन मे त्वया} %2-11-25

\twolineshloka
{तदा देवासुरे युद्धे तस्य कालोऽयमागतः}
{नव पञ्च च वर्षाणि दण्डकारण्यमाश्रितः} %2-11-26

\twolineshloka
{चीराजिनधरो धीरो रामो भवतु तापसः}
{भरतो भजतामद्य यौवराज्यमकण्टकम्} %2-11-27

\twolineshloka
{एष मे परमः कामो दत्तमेव वरं वृणे}
{अद्य चैव हि पश्येयं प्रयान्तं राघवं वने} %2-11-28

\twolineshloka
{स राजराजो भव सत्यसंगरः कुलं च शीलं च हि जन्म रक्ष च}
{परत्र वासे हि वदन्त्यनुत्तमं तपोधनाः सत्यवचो हितं नृणाम्} %2-11-29


॥इत्यार्षे श्रीमद्रामायणे वाल्मीकीये आदिकाव्ये अयोध्याकाण्डे वरद्वयनिर्बन्धः नाम एकादशः सर्गः ॥२-११॥
