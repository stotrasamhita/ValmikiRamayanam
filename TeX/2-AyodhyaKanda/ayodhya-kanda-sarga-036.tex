\sect{षड्त्रिंशः सर्गः — सिद्धार्थप्रतिबोधनम्}

\twolineshloka
{ततः सुमन्त्रमैक्ष्वाकः पीडितोऽत्र प्रतिज्ञया}
{सबाष्पमतिनिःश्वस्य जगादेदं पुनर्वचः} %2-36-1

\twolineshloka
{सूत रत्नसुसम्पूर्णा चतुर्विधबला चमूः}
{राघवस्यानुयात्रार्थं क्षिप्रं प्रतिविधीयताम्} %2-36-2

\twolineshloka
{रूपाजीवाश्च वादिन्यो वणिजश्च महाधनाः}
{शोभयन्तु कुमारस्य वाहिनीः सुप्रसारिताः} %2-36-3

\twolineshloka
{ये चैनमुपजीवन्ति रमते यैश्च वीर्यतः}
{तेषां बहुविधं दत्त्वा तानप्यत्र नियोजय} %2-36-4

\twolineshloka
{आयुधानि च मुख्यानि नागराः शकटानि च}
{अनुगच्छन्तु काकुत्स्थं व्याधाश्चारण्यकोविदाः} %2-36-5

\twolineshloka
{निघ्नन् मृगान् कुञ्जरांश्च पिबंश्चारण्यकं मधु}
{नदीश्च विविधाः पश्यन् न राज्यं संस्मरिष्यति} %2-36-6

\twolineshloka
{धान्यकोशश्च यः कश्चिद् धनकोशश्च मामकः}
{तौ राममनुगच्छेतां वसन्तं निर्जने वने} %2-36-7

\twolineshloka
{यजन् पुण्येषु देशेषु विसृजंश्चाप्तदक्षिणाः}
{ऋषिभिश्चापि सङ्गम्य प्रवत्स्यति सुखं वने} %2-36-8

\twolineshloka
{भरतश्च महाबाहुरयोध्यां पालयिष्यति}
{सर्वकामैः पुनः श्रीमान् रामः संसाध्यतामिति} %2-36-9

\twolineshloka
{एवं ब्रुवति काकुत्स्थे कैकेय्या भयमागतम्}
{मुखं चाप्यगमच्छोषं स्वरश्चापि व्यरुध्यत} %2-36-10

\twolineshloka
{सा विषण्णा च सन्त्रस्ता मुखेन परिशुष्यता}
{राजानमेवाभिमुखी कैकेयी वाक्यमब्रवीत्} %2-36-11

\twolineshloka
{राज्यं गतधनं साधो पीतमण्डां सुरामिव}
{निरास्वाद्यतमं शून्यं भरतो नाभिपत्स्यते} %2-36-12

\twolineshloka
{कैकेय्यां मुक्तलज्जायां वदन्त्यामतिदारुणम्}
{राजा दशरथो वाक्यमुवाचायतलोचनाम्} %2-36-13

\twolineshloka
{वहन्तं किं तुदसि मां नियुज्य धुरि माहिते}
{अनार्ये कृत्यमारब्धं किं न पूर्वमुपारुधः} %2-36-14

\twolineshloka
{तस्यैतत् क्रोधसंयुक्तमुक्तं श्रुत्वा वराङ्गना}
{कैकेयी द्विगुणं क्रुद्धा राजानमिदमब्रवीत्} %2-36-15

\twolineshloka
{तवैव वंशे सगरो ज्येष्ठपुत्रमुपारुधत्}
{असमञ्ज इति ख्यातं तथायं गन्तुमर्हति} %2-36-16

\twolineshloka
{एवमुक्तो धिगित्येव राजा दशरथोऽब्रवीत्}
{व्रीडितश्च जनः सर्वः सा च तन्नावबुध्यत} %2-36-17

\twolineshloka
{तत्र वृद्धो महामात्रः सिद्धार्थो नाम नामतः}
{शुचिर्बहुमतो राज्ञः कैकेयीमिदमब्रवीत्} %2-36-18

\twolineshloka
{असमञ्जो गृहीत्वा तु क्रीडतः पथि दारकान्}
{सरय्वां प्रक्षिपन्नप्सु रमते तेन दुर्मतिः} %2-36-19

\twolineshloka
{तं दृष्ट्वा नागराः सर्वे क्रुद्धा राजानमब्रुवन्}
{असमञ्जं वृणीष्वैकमस्मान् वा राष्ट्रवर्धन} %2-36-20

\twolineshloka
{तानुवाच ततो राजा किन्निमित्तमिदं भयम्}
{ताश्चापि राज्ञा सम्पृष्टा वाक्यं प्रकृतयोऽब्रुवन्} %2-36-21

\twolineshloka
{क्रीडतस्त्वेष नः पुत्रान् बालानुद्भ्रान्तचेतसः}
{सरय्वां प्रक्षिपन्मौर्ख्यादतुलां प्रीतिमश्नुते} %2-36-22

\twolineshloka
{स तासां वचनं श्रुत्वा प्रकृतीनां नराधिपः}
{तं तत्याजाहितं पुत्रं तासां प्रियचिकीर्षया} %2-36-23

\twolineshloka
{तं यानं शीघ्रमारोप्य सभार्यं सपरिच्छदम्}
{यावज्जीवं विवास्योऽयमिति तानन्वशात् पिता} %2-36-24

\twolineshloka
{स फालपिटकं गृह्य गिरिदुर्गाण्यलोकयत्}
{दिशः सर्वास्त्वनुचरन् स यथा पापकर्मकृत्} %2-36-25

\twolineshloka
{इत्येनमत्यजद् राजा सगरो वै सुधार्मिकः}
{रामः किमकरोत् पापं येनैवमुपरुध्यते} %2-36-26

\twolineshloka
{नहि कञ्चन पश्यामो राघवस्यागुणं वयम्}
{दुर्लभो ह्यस्य निरयः शशाङ्कस्येव कल्मषम्} %2-36-27

\twolineshloka
{अथवा देवि त्वं कञ्चिद् दोषं पश्यसि राघवे}
{तमद्य ब्रूहि तत्त्वेन तदा रामो विवास्यते} %2-36-28

\twolineshloka
{अदुष्टस्य हि सन्त्यागः सत्पथे निरतस्य च}
{निर्दहेदपि शक्रस्य द्युतिं धर्मविरोधवान्} %2-36-29

\twolineshloka
{तदलं देवि रामस्य श्रिया विहतया त्वया}
{लोकतोऽपि हि ते रक्ष्यः परिवादः शुभानने} %2-36-30

\twolineshloka
{श्रुत्वा तु सिद्धार्थवचो राजा श्रान्ततरस्वरः}
{शोकोपहतया वाचा कैकेयीमिदमब्रवीत्} %2-36-31

\twolineshloka
{एतद्वचो नेच्छसि पापरूपे हितं न जानासि ममात्मनोऽथवा}
{आस्थाय मार्गं कृपणं कुचेष्टा चेष्टा हि ते साधुपथादपेता} %2-36-32

\twolineshloka
{अनुव्रजिष्याम्यहमद्य रामं राज्यं परित्यज्य सुखं धनं च}
{सर्वे च राज्ञा भरतेन च त्वं यथासुखं भुङ्क्ष्व चिराय राज्यम्} %2-36-33


॥इत्यार्षे श्रीमद्रामायणे वाल्मीकीये आदिकाव्ये अयोध्याकाण्डे सिद्धार्थप्रतिबोधनम् नाम षड्त्रिंशः सर्गः ॥२-३६॥
