\sect{एकोननवतितमः सर्गः — गङ्गातरणम्}

\twolineshloka
{व्युष्य रात्रिं तु तत्रैव गङ्गाकूले स राघवः}
{भरतः काल्यमुत्थाय शत्रुघ्नमिदमब्रवीत्} %2-89-1

\twolineshloka
{शत्रुघ्नोत्तिष्ठ किं शेषे निषादाधिपतिं गुहम्}
{शीघ्रमानय भद्रं ते तारयिष्यति वाहिनीम्} %2-89-2

\twolineshloka
{जागर्मि नाहं स्वपिमि तमेवार्य्यं विचिन्तयन्}
{इत्येवमब्रवीद्भ्रात्रा शत्रुघ्नोपि प्रचोदितः} %2-89-3

\twolineshloka
{इति संवदतोरेवमन्योन्यं नरसिंहयोः}
{आगम्य प्राञ्जलिः काले गुहो भरतमब्रवीत्} %2-89-4

\twolineshloka
{कच्चित्सुखं नदीतीरेऽवात्सीः काकुत्स्थ शर्वरीम्}
{कच्चित्ते सहसैन्यस्य तावत्सर्वमनामयम्} %2-89-5

\twolineshloka
{गुहस्य तत्तु वचनं श्रुत्वा स्नेहादुदीरितम्}
{रामस्यानुवशो वाक्यं भरतोऽपीदमब्रवीत्} %2-89-6

\twolineshloka
{सुखा नः शर्वरी राजन् पूजिताश्चापि ते वयम्}
{गङ्गां तु नौभिर्बह्वीभिर्दाशाः सन्तारयन्तु नः} %2-89-7

\twolineshloka
{ततो गुहः सन्त्वरितं श्रुत्वा भरतशासनम्}
{प्रतिप्रविश्य नगरं तं ज्ञातिजनमब्रवीत्} %2-89-8

\twolineshloka
{उत्तिष्ठत प्रबुध्यध्वं भद्रमस्तु च वः सदा}
{नावः समनुकर्षध्वं तारयिष्याम वाहिनीम्} %2-89-9

\twolineshloka
{ते तथोक्ताः समुत्थाय त्वरिता राजशासनात्}
{पञ्च नावां शतान्याशु समानिन्युः समन्ततः} %2-89-10

\twolineshloka
{अन्याः स्वस्तिकविज्ञेया महाघण्टाधरा वराः}
{शोभमानाः पताकाभिर्युक्तवाताः सुसंहताः} %2-89-11

\threelineshloka
{ततः स्वस्तिकविज्ञेयां पाण्डुकम्बलसंवृताम्}
{सनन्दिघोषां कल्याणीं गुहो नावमुपाहरत्}
{तामारुरोह भरतः शत्रुघ्नश्च महाबलः} %2-89-12

\threelineshloka
{कौसल्या च सुमित्रा च याश्चान्या राजयोषितः}
{पुरोहितश्च तत्पूर्वं गुरवो ब्राह्मणाश्च ये}
{अनन्तरं राजदारास्तथैव शकटापणाः} %2-89-13

\twolineshloka
{आवासमादीपयतां तीर्थं चाप्यवगाहताम्}
{भाण्डानि चाददानानां घोषस्त्रिदिवमस्पृशत्} %2-89-14

\twolineshloka
{पताकिन्यस्तु ता नावः स्वयं दाशैरधिष्ठिताः}
{वहन्त्यो जनमारूढं तदा सम्पेतुराशुगाः} %2-89-15

\twolineshloka
{नारीणामभिपूर्णास्तु काश्चित् काश्चिच्च वाजिनाम्}
{काश्चिदत्र वहन्ति स्म यानयुग्यं महाधनम्} %2-89-16

\twolineshloka
{ताः स्म गत्वा परं तीरमवरोप्य च तं जनम्}
{निवृत्ताः काण्डचित्राणि क्रियन्ते दाशबन्धुभिः} %2-89-17

\twolineshloka
{सवैजयन्तास्तु गजा गजारोहप्रचोदिताः}
{तरन्तः स्म प्रकाशन्ते सध्वजा इव पर्वताः} %2-89-18

\twolineshloka
{नावश्चारुरुहुश्चान्ये प्लवैस्तेरुस्तथापरे}
{अन्ये कुम्भघटैस्तेरुरन्ये तेरुश्च बाहुभिः} %2-89-19

\twolineshloka
{सा पुण्या ध्वजिनी गङ्गा दाशैः सन्तारिता स्वयम्}
{मैत्रे मुहूर्त्ते प्रययौ प्रयागवनमुत्तमम्} %2-89-20

\twolineshloka
{आश्वासयित्वा च चमूं महात्मा निवेशयित्वा च यथोपजोषम्}
{द्रष्टुं भरद्वाजमृषिप्रवर्य्यमृत्विग्वृतः सन् भरतः प्रतस्थे} %2-89-21

\twolineshloka
{स ब्राह्मणस्याश्रममभ्युपेत्य महात्मनो देवपुरोहितस्य}
{ददर्श रम्योटजवृक्षषण्डं महद्वनं विप्रवरस्य रम्यम्} %2-89-22


॥इत्यार्षे श्रीमद्रामायणे वाल्मीकीये आदिकाव्ये अयोध्याकाण्डे गङ्गातरणम् नाम एकोननवतितमः सर्गः ॥२-८९॥
