\sect{त्रिचत्वारिशः सर्गः — कौसल्यापरिदेवितम्}

\twolineshloka
{ततः समीक्ष्य शयने सन्नं शोकेन पार्थिवम्}
{कौसल्या पुत्रशोकार्ता तमुवाच महीपतिम्} %2-43-1

\twolineshloka
{राघवे नरशार्दूले विषं मुक्त्वाहिजिह्मगा}
{विचरिष्यति कैकेयी निर्मुक्तेव हि पन्नगी} %2-43-2

\twolineshloka
{विवास्य रामं सुभगा लब्धकामा समाहिता}
{त्रासयिष्यति मां भूयो दुष्टाहिरिव वेश्मनि} %2-43-3

\twolineshloka
{अथास्मिन् नगरे रामश्चरन् भैक्षं गृहे वसेत्}
{कामकारो वरं दातुमपि दासं ममात्मजम्} %2-43-4

\twolineshloka
{पातयित्वा तु कैकेय्या रामं स्थानाद् यथेष्टतः}
{प्रविद्धो रक्षसां भागः पर्वणीवाहिताग्निना} %2-43-5

\twolineshloka
{नागराजगतिर्वीरो महाबाहुर्धनुर्धरः}
{वनमाविशते नूनं सभार्यः सहलक्ष्मणः} %2-43-6

\twolineshloka
{वने त्वदृष्टदुःखानां कैकेय्यनुमते त्वया}
{त्यक्तानां वनवासाय कान्यावस्था भविष्यति} %2-43-7

\twolineshloka
{ते रत्नहीनास्तरुणाः फलकाले विवासिताः}
{कथं वत्स्यन्ति कृपणाः फलमूलैः कृताशनाः} %2-43-8

\twolineshloka
{अपीदानीं स कालः स्यान्मम शोकक्षयः शिवः}
{सहभार्यं सह भ्रात्रा पश्येयमिह राघवम्} %2-43-9

\twolineshloka
{श्रुत्वैवोपस्थितौ वीरौ कदायोध्या भविष्यति}
{यशस्विनी हृष्टजना सूच्छ्रितध्वजमालिनी} %2-43-10

\twolineshloka
{कदा प्रेक्ष्य नरव्याघ्रावरण्यात् पुनरागतौ}
{भविष्यति पुरी हृष्टा समुद्र इव पर्वणि} %2-43-11

\twolineshloka
{कदायोध्यां महाबाहुः पुरीं वीरः प्रवेक्ष्यति}
{पुरस्कृत्य रथे सीतां वृषभो गोवधूमिव} %2-43-12

\twolineshloka
{कदा प्राणिसहस्राणि राजमार्गे ममात्मजौ}
{लाजैरवकरिष्यन्ति प्रविशन्तावरिन्दमौ} %2-43-13

\twolineshloka
{प्रविशन्तौ कदायोध्यां द्रक्ष्यामि शुभकुण्डलौ}
{उदग्रायुधनिस्त्रिंशौ सशृङ्गाविव पर्वतौ} %2-43-14

\twolineshloka
{कदा सुमनसः कन्या द्विजातीनां फलानि च}
{प्रदिशन्त्यः पुरीं हृष्टाः करिष्यन्ति प्रदक्षिणम्} %2-43-15

\twolineshloka
{कदा परिणतो बुद्ध्या वयसा चामरप्रभाः}
{अभ्युपैष्यति धर्मात्मा सुवर्ष इव लालयन्} %2-43-16

\twolineshloka
{निःसंशयं मया मन्ये पुरा वीर कदर्यया}
{पातुकामेषु वत्सेषु मातॄणां शातिताः स्तनाः} %2-43-17

\twolineshloka
{साहं गौरिव सिंहेन विवत्सा वत्सला कृता}
{कैकेय्या पुरुषव्याघ्र बालवत्सेव गौर्बलात्} %2-43-18

\twolineshloka
{नहि तावद्गुणैर्जुष्टं सर्वशास्त्रविशारदम्}
{एकपुत्रा विना पुत्रमहं जीवितुमुत्सहे} %2-43-19

\twolineshloka
{न हि मे जीविते किञ्चित् सामर्थ्यमिह कल्प्यते}
{अपश्यन्त्याः प्रियं पुत्रं लक्ष्मणं च महाबलम्} %2-43-20

\twolineshloka
{अयं हि मां दीपयतेऽद्य वह्निस्तनूजशोकप्रभवो महाहितः}
{महीमिमां रश्मिभिरुत्तमप्रभो यथा निदाघे भगवान् दिवाकरः} %2-43-21


॥इत्यार्षे श्रीमद्रामायणे वाल्मीकीये आदिकाव्ये अयोध्याकाण्डे कौसल्यापरिदेवितम् नाम त्रिचत्वारिशः सर्गः ॥२-४३॥
