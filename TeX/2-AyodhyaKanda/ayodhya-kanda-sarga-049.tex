\sect{एकोनपञ्चाशः सर्गः — जानपदाक्रोशः}

\twolineshloka
{रामः अपि रात्रि शेषेण तेन एव महद् अन्तरम्}
{जगाम पुरुष व्याघ्रः पितुर् आज्ञाम् अनुस्मरन्} %2-49-1

\twolineshloka
{तथैव गच्चतः तस्य व्यपायात् रजनी शिवा}
{उपास्य स शिवाम् सम्ध्याम् विषय अन्तम् व्यगाहत} %2-49-2

\twolineshloka
{ग्रामान् विकृष्ट सीमान् तान् पुष्पितानि वनानि च}
{पश्यन्न् अतिययौ शीघ्रम् शरैः इव हय उत्तमैः} %2-49-3

\twolineshloka
{शृण्वन् वाचो मनुष्याणाम् ग्राम सम्वास वासिनाम्}
{राजानम् धिग् दशरथम् कामस्य वशम् आगतम्} %2-49-4

\twolineshloka
{हा नृशम्स अद्य कैकेयी पापा पाप अनुबन्धिनी}
{तीक्ष्णा सम्भिन्न मर्यादा तीक्ष्णे कर्मणि वर्तते} %2-49-5

\twolineshloka
{या पुत्रम् ईदृशम् राज्ञः प्रवासयति धार्मिकम्}
{वन वासे महा प्राज्ञम् सानुक्रोशम् अतन्द्रितम्} %2-49-6

\twolineshloka
{कथम् नाम महाभागा सीता जनकनन्दिनी}
{सदा सुखेष्वभिरता दुःखान्यनुभविष्यति} %2-49-7

\twolineshloka
{अहो दशरथो राजा निस्नेहः स्वसुत प्रियम्}
{प्रजानामनघम् रामम् परित्यक्तुमिहेच्छति} %2-49-8

\twolineshloka
{एता वाचो मनुष्याणाम् ग्राम सम्वास वासिनाम्}
{शृण्वन्न् अति ययौ वीरः कोसलान् कोसल ईश्वरः} %2-49-9

\twolineshloka
{ततः वेद श्रुतिम् नाम शिव वारि वहाम् नदीम्}
{उत्तीर्य अभिमुखः प्रायात् अगस्त्य अध्युषिताम् दिशम्} %2-49-10

\twolineshloka
{गत्वा तु सुचिरम् कालम् ततः शीत जलाम् नदीम्}
{गोमतीम् गोयुत अनूपाम् अतरत् सागरम् गमाम्} %2-49-11

\twolineshloka
{गोमतीम् च अपि अतिक्रम्य राघवः शीघ्रगैः हयैः}
{मयूर हम्स अभिरुताम् ततार स्यन्दिकाम् नदीम्} %2-49-12

\twolineshloka
{स महीम् मनुना राज्ञा दत्ताम् इक्ष्वाकवे पुरा}
{स्फीताम् राष्ट्र आवृताम् रामः वैदेहीम् अन्वदर्शयत्} %2-49-13

\twolineshloka
{सूतैति एव च आभाष्य सारथिम् तम् अभीक्ष्णशः}
{हम्स मत्त स्वरः श्रीमान् उवाच पुरुष ऋषभः} %2-49-14

\twolineshloka
{कदा अहम् पुनर् आगम्य सरय्वाः पुष्पिते वने}
{मृगयाम् पर्याटष्यामि मात्रा पित्रा च सम्गतः} %2-49-15

\twolineshloka
{न अत्यर्थम् अभिकान्क्षामि मृगयाम् सरयू वने}
{रतिर् हि एषा अतुला लोके राज ऋषि गण सम्मता} %2-49-16

\twolineshloka
{राजर्षीणाम् हि लोकेऽस्मिन् रत्यर्थम् मृगया वने}
{काले कृताम् ताम् मनुजैर्धन्विनामभिकाङ्क्षिताम्} %2-49-17

\onelineshloka
{स तम् अध्वानम् ऐक्ष्वाकः सूतम् मधुरया गिरातम् तम् अर्थम् अभिप्रेत्य ययौवाक्यम् उदीरयन्} %2-49-18


॥इत्यार्षे श्रीमद्रामायणे वाल्मीकीये आदिकाव्ये अयोध्याकाण्डे जानपदाक्रोशः नाम एकोनपञ्चाशः सर्गः ॥२-४९॥
