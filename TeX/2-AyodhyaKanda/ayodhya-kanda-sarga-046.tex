\sect{षड्चत्वारिंशः सर्गः — पौरमोहनम्}

\twolineshloka
{ततस्तु तमसातीरं रम्यमाश्रित्य राघवः}
{सीतामुद्वीक्ष्य सौमित्रिमिदं वचनमब्रवीत्} %2-46-1

\twolineshloka
{इयमद्य निशा पूर्वा सौमित्रे प्रहिता वनम्}
{वनवासस्य भद्रं ते न चोत्कण्ठितुमर्हसि} %2-46-2

\twolineshloka
{पश्य शून्यान्यरण्यानि रुदन्तीव समन्ततः}
{यथानिलयमायद्भिर्निलीनानि मृगद्विजैः} %2-46-3

\twolineshloka
{अद्यायोध्या तु नगरी राजधानी पितुर्मम}
{सस्त्रीपुंसा गतानस्मान् शोचिष्यति न संशयः} %2-46-4

\twolineshloka
{अनुरक्ता हि मनुजा राजानं बहुभिर्गुणैः}
{त्वां च मां च नरव्याघ्र शत्रुघ्नभरतौ तथा} %2-46-5

\twolineshloka
{पितरं चानुशोचामि मातरं च यशस्विनीम्}
{अपि नान्धौ भवेतां नौ रुदन्तौ तावभीक्ष्णशः} %2-46-6

\twolineshloka
{भरतः खलु धर्मात्मा पितरं मातरं च मे}
{धर्मार्थकामसहितैर्वाक्यैराश्वासयिष्यति} %2-46-7

\twolineshloka
{भरतस्यानृशंसत्वं संचिन्त्याहं पुनः पुनः}
{नानुशोचामि पितरं मातरं च महाभुज} %2-46-8

\twolineshloka
{त्वया कार्यं नरव्याघ्र मामनुव्रजता कृतम्}
{अन्वेष्टव्या हि वैदेह्या रक्षणार्थं सहायता} %2-46-9

\twolineshloka
{अद्भिरेव हि सौमित्रे वत्स्याम्यद्य निशामिमाम्}
{एतद्धि रोचते मह्यं वन्येऽपि विविधे सति} %2-46-10

\twolineshloka
{एवमुक्त्वा तु सौमित्रिं सुमन्त्रमपि राघवः}
{अप्रमत्तस्त्वमश्वेषु भव सौम्येत्युवाच ह} %2-46-11

\twolineshloka
{सोऽश्वान् सुमन्त्रः संयम्य सूर्येऽस्तं समुपागते}
{प्रभूतयवसान् कृत्वा बभूव प्रत्यनन्तरः} %2-46-12

\twolineshloka
{उपास्य तु शिवां संध्यां दृष्ट्वा रात्रिमुपागताम्}
{रामस्य शयनं चक्रे सूतः सौमित्रिणा सह} %2-46-13

\twolineshloka
{तां शय्यां तमसातीरे वीक्ष्य वृक्षदलैर्वृताम्}
{रामः सौमित्रिणा सार्धं सभार्यः संविवेश ह} %2-46-14

\twolineshloka
{सभार्यं सम्प्रसुप्तं तु श्रान्तं सम्प्रेक्ष्य लक्ष्मणः}
{कथयामास सूताय रामस्य विविधान् गुणान्} %2-46-15

\twolineshloka
{जाग्रतोरेव तां रात्रिं सौमित्रेरुदितो रविः}
{सूतस्य तमसातीरे रामस्य ब्रुवतो गुणान्} %2-46-16

\twolineshloka
{गोकुलाकुलतीरायास्तमसाया विदूरतः}
{अवसत् तत्र तां रात्रिं रामः प्रकृतिभिः सह} %2-46-17

\twolineshloka
{उत्थाय च महातेजाः प्रकृतीस्ता निशाम्य च}
{अब्रवीद् भ्रातरं रामो लक्ष्मणं पुण्यलक्षणम्} %2-46-18

\twolineshloka
{अस्मद्व्यपेक्षान् सौमित्रे निर्व्यपेक्षान् गृहेष्वपि}
{वृक्षमूलेषु संसक्तान् पश्य लक्ष्मण साम्प्रतम्} %2-46-19

\twolineshloka
{यथैते नियमं पौराः कुर्वन्त्यस्मन्निवर्तने}
{अपि प्राणान् न्यसिष्यन्ति न तु त्यक्ष्यन्ति निश्चयम्} %2-46-20

\twolineshloka
{यावदेव तु संसुप्तास्तावदेव वयं लघु}
{रथमारुह्य गच्छामः पन्थानमकुतोभयम्} %2-46-21

\twolineshloka
{अतो भूयोऽपि नेदानीमिक्ष्वाकुपुरवासिनः}
{स्वपेयुरनुरक्ता मा वृक्षमूलेषु संश्रिताः} %2-46-22

\twolineshloka
{पौरा ह्यात्मकृताद् दुःखाद् विप्रमोच्या नृपात्मजैः}
{न तु खल्वात्मना योज्या दुःखेन पुरवासिनः} %2-46-23

\twolineshloka
{अब्रवील्लक्ष्मणो रामं साक्षाद् धर्ममिव स्थितम्}
{रोचते मे तथा प्राज्ञ क्षिप्रमारुह्यतामिति} %2-46-24

\twolineshloka
{अथ रामोऽब्रवीत् सूतं शीघ्रं संयुज्यतां रथः}
{गमिष्यामि ततोऽरण्यं गच्छ शीघ्रमितः प्रभो} %2-46-25

\twolineshloka
{सूतस्ततः संत्वरितः स्यन्दनं तैर्हयोत्तमैः}
{योजयित्वा तु रामस्य प्राञ्जलिः प्रत्यवेदयत्} %2-46-26

\twolineshloka
{अयं युक्तो महाबाहो रथस्ते रथिनां वर}
{त्वरयाऽऽरोह भद्रं ते ससीतः सहलक्ष्मणः} %2-46-27

\twolineshloka
{तं स्यन्दनमधिष्ठाय राघवः सपरिच्छदः}
{शीघ्रगामाकुलावर्तां तमसामतरन्नदीम्} %2-46-28

\twolineshloka
{स संतीर्य महाबाहुः श्रीमान् शिवमकण्टकम्}
{प्रापद्यत महामार्गमभयं भयदर्शिनाम्} %2-46-29

\twolineshloka
{मोहनार्थं तु पौराणां सूतं रामोऽब्रवीद् वचः}
{उदङ्मुखः प्रयाहि त्वं रथमारुह्य सारथे} %2-46-30

\twolineshloka
{मुहूर्तं त्वरितं गत्वा निवर्तय रथं पुनः}
{यथा न विद्युः पौरा मां तथा कुरु समाहितः} %2-46-31

\twolineshloka
{रामस्य तु वचः श्रुत्वा तथा चक्रे च सारथिः}
{प्रत्यागम्य च रामस्य स्यन्दनं प्रत्यवेदयत्} %2-46-32

\twolineshloka
{तौ सम्प्रयुक्तं तु रथं समास्थितौ तदा ससीतौ रघुवंशवर्धनौ}
{प्रचोदयामास ततस्तुरंगमान् स सारथिर्येन पथा तपोवनम्} %2-46-33

\twolineshloka
{ततः समास्थाय रथं महारथः ससारथिर्दाशरथिर्वनं ययौ}
{उदङ्मुखं तं तु रथं चकार प्रयाणमाङ्गल्यनिमित्तदर्शनात्} %2-46-34


॥इत्यार्षे श्रीमद्रामायणे वाल्मीकीये आदिकाव्ये अयोध्याकाण्डे पौरमोहनम् नाम षड्चत्वारिंशः सर्गः ॥२-४६॥
