\sect{एकविंशः सर्गः — कौसल्यालक्ष्मणप्रतिबोधनम्}

\twolineshloka
{तथा तु विलपन्तीं तां कौसल्यां राममातरम्}
{उवाच लक्ष्मणो दीनस्तत्कालसदृशं वचः} %2-21-1

\twolineshloka
{न रोचते ममाप्येतदार्ये यद् राघवो वनम्}
{त्यक्त्वा राज्यश्रियं गच्छेत् स्त्रिया वाक्यवशंगतः} %2-21-2

\twolineshloka
{विपरीतश्च वृद्धश्च विषयैश्च प्रधर्षितः}
{नृपः किमिव न ब्रूयाच्चोद्यमानः समन्मथः} %2-21-3

\twolineshloka
{नास्यापराधं पश्यामि नापि दोषं तथाविधम्}
{येन निर्वास्यते राष्ट्राद् वनवासाय राघवः} %2-21-4

\twolineshloka
{न तं पश्याम्यहं लोके परोक्षमपि यो नरः}
{स्वमित्रोऽपि निरस्तोऽपि योऽस्य दोषमुदाहरेत्} %2-21-5

\twolineshloka
{देवकल्पमृजुं दान्तं रिपूणामपि वत्सलम्}
{अवेक्षमाणः को धर्मं त्यजेत् पुत्रमकारणात्} %2-21-6

\twolineshloka
{तदिदं वचनं राज्ञः पुनर्बाल्यमुपेयुषः}
{पुत्रः को हृदये कुर्याद् राजवृत्तमनुस्मरन्} %2-21-7

\twolineshloka
{यावदेव न जानाति कश्चिदर्थमिमं नरः}
{तावदेव मया सार्धमात्मस्थं कुरु शासनम्} %2-21-8

\twolineshloka
{मया पार्श्वे सधनुषा तव गुप्तस्य राघव}
{कः समर्थोऽधिकं कर्तुं कृतान्तस्येव तिष्ठतः} %2-21-9

\twolineshloka
{निर्मनुष्यामिमां सर्वामयोध्यां मनुजर्षभ}
{करिष्यामि शरैस्तीक्ष्णैर्यदि स्थास्यति विप्रिये} %2-21-10

\twolineshloka
{भरतस्याथ पक्ष्यो वा यो वास्य हितमिच्छति}
{सर्वांस्तांश्च वधिष्यामि मृदुर्हि परिभूयते} %2-21-11

\twolineshloka
{प्रोत्साहितोऽयं कैकेय्या संतुष्टो यदि नः पिता}
{अमित्रभूतो निःसङ्गं वध्यतां वध्यतामपि} %2-21-12

\twolineshloka
{गुरोरप्यवलिप्तस्य कार्याकार्यमजानतः}
{उत्पथं प्रतिपन्नस्य कार्यं भवति शासनम्} %2-21-13

\twolineshloka
{बलमेष किमाश्रित्य हेतुं वा पुरुषोत्तम}
{दातुमिच्छति कैकेय्यै उपस्थितमिदं तव} %2-21-14

\twolineshloka
{त्वया चैव मया चैव कृत्वा वैरमनुत्तमम्}
{कास्य शक्तिः श्रियं दातुं भरतायारिशासन} %2-21-15

\twolineshloka
{अनुरक्तोऽस्मि भावेन भ्रातरं देवि तत्त्वतः}
{सत्येन धनुषा चैव दत्तेनेष्टेन ते शपे} %2-21-16

\twolineshloka
{दीप्तमग्निमरण्यं वा यदि रामः प्रवेक्ष्यति}
{प्रविष्टं तत्र मां देवि त्वं पूर्वमवधारय} %2-21-17

\twolineshloka
{हरामि वीर्याद् दुःखं ते तमः सूर्य इवोदितः}
{देवी पश्यतु मे वीर्यं राघवश्चैव पश्यतु} %2-21-18

\twolineshloka
{हनिष्ये पितरं वृद्धं कैकेय्यासक्तमानसम्}
{कृपणं च स्थितं बाल्ये वृद्धभावेन गर्हितम्} %2-21-19

\twolineshloka
{एतत् तु वचनं श्रुत्वा लक्ष्मणस्य महात्मनः}
{उवाच रामं कौसल्या रुदती शोकलालसा} %2-21-20

\twolineshloka
{भ्रातुस्ते वदतः पुत्र लक्ष्मणस्य श्रुतं त्वया}
{यदत्रानन्तरं तत्त्वं कुरुष्व यदि रोचते} %2-21-21

\twolineshloka
{न चाधर्म्यं वचः श्रुत्वा सपत्न्या मम भाषितम्}
{विहाय शोकसंतप्तां गन्तुमर्हसि मामितः} %2-21-22

\twolineshloka
{धर्मज्ञ इति धर्मिष्ठ धर्मं चरितुमिच्छसि}
{शुश्रूष मामिहस्थस्त्वं चर धर्ममनुत्तमम्} %2-21-23

\twolineshloka
{शुश्रूषुर्जननीं पुत्र स्वगृहे नियतो वसन्}
{परेण तपसा युक्तः काश्यपस्त्रिदिवं गतः} %2-21-24

\twolineshloka
{यथैव राजा पूज्यस्ते गौरवेण तथा ह्यहम्}
{त्वां साहं नानुजानामि न गन्तव्यमितो वनम्} %2-21-25

\twolineshloka
{त्वद्वियोगान्न मे कार्यं जीवितेन सुखेन च}
{त्वया सह मम श्रेयस्तृणानामपि भक्षणम्} %2-21-26

\twolineshloka
{यदि त्वं यास्यसि वनं त्यक्त्वा मां शोकलालसाम्}
{अहं प्रायमिहासिष्ये न च शक्ष्यामि जीवितुम्} %2-21-27

\twolineshloka
{ततस्त्वं प्राप्स्यसे पुत्र निरयं लोकविश्रुतम्}
{ब्रह्महत्यामिवाधर्मात् समुद्रः सरितां पतिः} %2-21-28

\twolineshloka
{विलपन्तीं तथा दीनां कौसल्यां जननीं ततः}
{उवाच रामो धर्मात्मा वचनं धर्मसंहितम्} %2-21-29

\twolineshloka
{नास्ति शक्तिः पितुर्वाक्यं समतिक्रमितुं मम}
{प्रसादये त्वां शिरसा गन्तुमिच्छाम्यहं वनम्} %2-21-30

\twolineshloka
{ऋषिणा च पितुर्वाक्यं कुर्वता वनचारिणा}
{गौर्हता जानताधर्मं कण्डुना च विपश्चिता} %2-21-31

\twolineshloka
{अस्माकं तु कुले पूर्वं सगरस्याज्ञया पितुः}
{खनद्भिः सागरैर्भूमिमवाप्तः सुमहान् वधः} %2-21-32

\twolineshloka
{जामदग्न्येन रामेण रेणुका जननी स्वयम्}
{कृत्ता परशुनारण्ये पितुर्वचनकारणात्} %2-21-33

\twolineshloka
{एतैरन्यैश्च बहुभिर्देवि देवसमैः कृतम्}
{पितुर्वचनमक्लीबं करिष्यामि पितुर्हितम्} %2-21-34

\twolineshloka
{न खल्वेतन्मयैकेन क्रियते पितृशासनम्}
{एतैरपि कृतं देवि ये मया परिकीर्तिताः} %2-21-35

\twolineshloka
{नाहं धर्ममपूर्वं ते प्रतिकूलं प्रवर्तये}
{पूर्वैरयमभिप्रेतो गतो मार्गोऽनुगम्यते} %2-21-36

\twolineshloka
{तदेतत् तु मया कार्यं क्रियते भुवि नान्यथा}
{पितुर्हि वचनं कुर्वन् न कश्चिन्नाम हीयते} %2-21-37

\twolineshloka
{तामेवमुक्त्वा जननीं लक्ष्मणं पुनरब्रवीत्}
{वाक्यं वाक्यविदां श्रेष्ठः श्रेष्ठः सर्वधनुष्मताम्} %2-21-38

\twolineshloka
{तव लक्ष्मण जानामि मयि स्नेहमनुत्तमम्}
{विक्रमं चैव सत्त्वं च तेजश्च सुदुरासदम्} %2-21-39

\twolineshloka
{मम मातुर्महद् दुःखमतुलं शुभलक्षण}
{अभिप्रायं न विज्ञाय सत्यस्य च शमस्य च} %2-21-40

\twolineshloka
{धर्मो हि परमो लोके धर्मे सत्यं प्रतिष्ठितम्}
{धर्मसंश्रितमप्येतत् पितुर्वचनमुत्तमम्} %2-21-41

\twolineshloka
{संश्रुत्य च पितुर्वाक्यं मातुर्वा ब्राह्मणस्य वा}
{न कर्तव्यं वृथा वीर धर्ममाश्रित्य तिष्ठता} %2-21-42

\twolineshloka
{सोऽहं न शक्ष्यामि पुनर्नियोगमतिवर्तितुम्}
{पितुर्हि वचनाद् वीर कैकेय्याहं प्रचोदितः} %2-21-43

\twolineshloka
{तदेतां विसृजानार्यां क्षत्रधर्माश्रितां मतिम्}
{धर्ममाश्रय मा तैक्ष्ण्यं मद्बुद्धिरनुगम्यताम्} %2-21-44

\twolineshloka
{तमेवमुक्त्वा सौहार्दाद् भ्रातरं लक्ष्मणाग्रजः}
{उवाच भूयः कौसल्यां प्राञ्जलिः शिरसा नतः} %2-21-45

\twolineshloka
{अनुमन्यस्व मां देवि गमिष्यन्तमितो वनम्}
{शापितासि मम प्राणैः कुरु स्वस्त्ययनानि मे} %2-21-46

\twolineshloka
{तीर्णप्रतिज्ञश्च वनात् पुनरेष्याम्यहं पुरीम्}
{ययातिरिव राजर्षिः पुरा हित्वा पुनर्दिवम्} %2-21-47

\twolineshloka
{शोकः संधार्यतां मातर्हृदये साधु मा शुचः}
{वनवासादिहैष्यामि पुनः कृत्वा पितुर्वचः} %2-21-48

\twolineshloka
{त्वया मया च वैदेह्या लक्ष्मणेन सुमित्रया}
{पितुर्नियोगे स्थातव्यमेष धर्मः सनातनः} %2-21-49

\twolineshloka
{अम्ब सम्भृत्य सम्भारान् दुःखं हृदि निगृह्य च}
{वनवासकृता बुद्धिर्मम धर्म्यानुवर्त्यताम्} %2-21-50

\twolineshloka
{एतद् वचस्तस्य निशम्य माता सुधर्म्यमव्यग्रमविक्लवं च}
{मृतेव संज्ञां प्रतिलभ्य देवी समीक्ष्य रामं पुनरित्युवाच} %2-21-51

\twolineshloka
{यथैव ते पुत्र पिता तथाहं गुरुः स्वधर्मेण सुहृत्तया च}
{न त्वानुजानामि न मां विहाय सुदुःखितामर्हसि पुत्र गन्तुम्} %2-21-52

\twolineshloka
{किं जीवितेनेह विना त्वया मे लोकेन वा किं स्वधयामृतेन}
{श्रेयो मुहूर्तं तव संनिधानं ममैव कृत्स्नादपि जीवलोकात्} %2-21-53

\twolineshloka
{नरैरिवोल्काभिरपोह्यमानो महागजो ध्वान्तमभिप्रविष्टः}
{भूयः प्रजज्वाल विलापमेवं निशम्य रामः करुणं जनन्याः} %2-21-54

\twolineshloka
{स मातरं चैव विसंज्ञकल्पामार्तं च सौमित्रिमभिप्रतप्तम्}
{धर्मे स्थितो धर्म्यमुवाच वाक्यं यथा स एवार्हति तत्र वक्तुम्} %2-21-55

\twolineshloka
{अहं हि ते लक्ष्मण नित्यमेव जानामि भक्तिं च पराक्रमं च}
{मम त्वभिप्रायमसंनिरीक्ष्य मात्रा सहाभ्यर्दसि मा सुदुःखम्} %2-21-56

\twolineshloka
{धर्मार्थकामाः खलु जीवलोके समीक्षिता धर्मफलोदयेषु}
{ये तत्र सर्वे स्युरसंशयं मे भार्येव वश्याभिमता सपुत्रा} %2-21-57

\twolineshloka
{यस्मिंस्तु सर्वे स्युरसंनिविष्टा धर्मो यतः स्यात् तदुपक्रमेत}
{द्वेष्यो भवत्यर्थपरो हि लोके कामात्मता खल्वपि न प्रशस्ता} %2-21-58

\twolineshloka
{गुरुश्च राजा च पिता च वृद्धः क्रोधात् प्रहर्षादथवापि कामात्}
{यद् व्यादिशेत् कार्यमवेक्ष्य धर्मं कस्तं न कुर्यादनृशंसवृत्तिः} %2-21-59

\twolineshloka
{न तेन शक्नोमि पितुः प्रतिज्ञामिमां न कर्तुं सकलां यथावत्}
{स ह्यावयोस्तात गुरुर्नियोगे देव्याश्च भर्ता स गतिश्च धर्मः} %2-21-60

\twolineshloka
{तस्मिन् पुनर्जीवति धर्मराजे विशेषतः स्वे पथि वर्तमाने}
{देवी मया सार्धमितोऽभिगच्छेत् कथंस्विदन्या विधवेव नारी} %2-21-61

\twolineshloka
{सा मानुमन्यस्व वनं व्रजन्तं कुरुष्व नः स्वस्त्ययनानि देवि}
{यथा समाप्ते पुनराव्रजेयं यथा हि सत्येन पुनर्ययातिः} %2-21-62

\twolineshloka
{यशो ह्यहं केवलराज्यकारणान्न पृष्ठतः कर्तुमलं महोदयम्}
{अदीर्घकालेन तु देवि जीविते वृणेऽवरामद्य महीमधर्मतः} %2-21-63

\twolineshloka
{प्रसादयन्नरवृषभः स मातरं पराक्रमाज्जिगमिषुरेव दण्डकान्}
{अथानुजं भृशमनुशास्य दर्शनं चकार तां हृदि जननीं प्रदक्षिणम्} %2-21-64


॥इत्यार्षे श्रीमद्रामायणे वाल्मीकीये आदिकाव्ये अयोध्याकाण्डे कौसल्यालक्ष्मणप्रतिबोधनम् नाम एकविंशः सर्गः ॥२-२१॥
