\sect{त्रयस्त्रिंशः सर्गः — पौरवाक्यम्}

\twolineshloka
{दत्त्वा तु सह वैदेह्या ब्राह्मणेभ्यो धनं बहु}
{जग्मतुः पितरं द्रष्टुं सीतया सह राघवौ} %2-33-1

\twolineshloka
{ततो गृहीते प्रेष्याभ्यामशोभेतां तदायुधे}
{मालादामभिरासक्ते सीतया समलंकृते} %2-33-2

\twolineshloka
{ततः प्रासादहर्म्याणि विमानशिखराणि च}
{अभिरुह्य जनः श्रीमानुदासीनो व्यलोकयत्} %2-33-3

\twolineshloka
{न हि रथ्याः सुशक्यन्ते गन्तुं बहुजनाकुलाः}
{आरुह्य तस्मात् प्रासादाद् दीनाः पश्यन्ति राघवम्} %2-33-4

\twolineshloka
{पदातिं सानुजं दृष्ट्वा ससीतं च जनास्तदा}
{ऊचुर्बहुजना वाचः शोकोपहतचेतसः} %2-33-5

\twolineshloka
{यं यान्तमनुयाति स्म चतुरङ्गबलं महत्}
{तमेकं सीतया सार्धमनुयाति स्म लक्ष्मणः} %2-33-6

\twolineshloka
{ऐश्वर्यस्य रसज्ञः सन् कामानां चाकरो महान्}
{नेच्छत्येवानृतं कर्तुं वचनं धर्मगौरवात्} %2-33-7

\twolineshloka
{या न शक्या पुरा द्रष्टुं भूतैराकाशगैरपि}
{तामद्य सीतां पश्यन्ति राजमार्गगता जनाः} %2-33-8

\twolineshloka
{अङ्गरागोचितां सीतां रक्तचन्दनसेविनीम्}
{वर्षमुष्णं च शीतं च नेष्यत्याशु विवर्णताम्} %2-33-9

\twolineshloka
{अद्य नूनं दशरथः सत्त्वमाविश्य भाषते}
{नहि राजा प्रियं पुत्रं विवासयितुमर्हति} %2-33-10

\twolineshloka
{निर्गुणस्यापि पुत्रस्य कथं स्याद् विनिवासनम्}
{किं पुनर्यस्य लोकोऽयं जितो वृत्तेन केवलम्} %2-33-11

\twolineshloka
{आनृशंस्यमनुक्रोशः श्रुतं शीलं दमः शमः}
{राघवं शोभयन्त्येते षड्गुणाः पुरुषर्षभम्} %2-33-12

\twolineshloka
{तस्मात् तस्योपघातेन प्रजाः परमपीडिताः}
{औदकानीव सत्त्वानि ग्रीष्मे सलिलसंक्षयात्} %2-33-13

\twolineshloka
{पीडया पीडितं सर्वं जगदस्य जगत्पतेः}
{मूलस्येवोपघातेन वृक्षः पुष्पफलोपगः} %2-33-14

\twolineshloka
{मूलं ह्येष मनुष्याणां धर्मसारो महाद्युतिः}
{पुष्पं फलं च पत्रं च शाखाश्चास्येतरे जनाः} %2-33-15

\twolineshloka
{ते लक्ष्मण इव क्षिप्रं सपत्न्यः सहबान्धवाः}
{गच्छन्तमनुगच्छामो येन गच्छति राघवः} %2-33-16

\twolineshloka
{उद्यानानि परित्यज्य क्षेत्राणि च गृहाणि च}
{एकदुःखसुखा राममनुगच्छाम धार्मिकम्} %2-33-17

\twolineshloka
{समुद्धृतनिधानानि परिध्वस्ताजिराणि च}
{उपात्तधनधान्यानि हृतसाराणि सर्वशः} %2-33-18

\twolineshloka
{रजसाभ्यवकीर्णानि परित्यक्तानि दैवतैः}
{मूषकैः परिधावद्भिरुद्बिलैरावृतानि च} %2-33-19

\twolineshloka
{अपेतोदकधूमानि हीनसम्मार्जनानि च}
{प्रणष्टबलिकर्मेज्यामन्त्रहोमजपानि च} %2-33-20

\twolineshloka
{दुष्कालेनेव भग्नानि भिन्नभाजनवन्ति च}
{अस्मत्त्यक्तानि कैकेयी वेश्मानि प्रतिपद्यताम्} %2-33-21

\twolineshloka
{वनं नगरमेवास्तु येन गच्छति राघवः}
{अस्माभिश्च परित्यक्तं पुरं सम्पद्यतां वनम्} %2-33-22

\twolineshloka
{बिलानि दंष्ट्रिणः सर्वे सानूनि मृगपक्षिणः}
{त्यजन्त्वस्मद्भयाद्भीता गजाः सिंहा वनान्यपि} %2-33-23

\twolineshloka
{अस्मत्त्यक्तं प्रपद्यन्तु सेव्यमानं त्यजन्तु च}
{तृणमांसफलादानां देशं व्यालमृगद्विजम्} %2-33-24

\twolineshloka
{प्रपद्यतां हि कैकेयी सपुत्रा सह बान्धवैः}
{राघवेण वयं सर्वे वने वत्स्याम निर्वृताः} %2-33-25

\twolineshloka
{इत्येवं विविधा वाचो नानाजनसमीरिताः}
{शुश्राव राघवः श्रुत्वा न विचक्रेऽस्य मानसम्} %2-33-26

\twolineshloka
{स तु वेश्म पुनर्मातुः कैलासशिखरप्रभम्}
{अभिचक्राम धर्मात्मा मत्तमातङ्गविक्रमः} %2-33-27

\twolineshloka
{विनीतवीरपुरुषं प्रविश्य तु नृपालयम्}
{ददर्शावस्थितं दीनं सुमन्त्रमविदूरतः} %2-33-28

\twolineshloka
{प्रतीक्षमाणोऽभिजनं तदार्तमनार्तरूपः प्रहसन्निवाथ}
{जगाम रामः पितरं दिदृक्षुः पितुर्निदेशं विधिवच्चिकीर्षुः} %2-33-29

\twolineshloka
{तत्पूर्वमैक्ष्वाकसुतो महात्मा रामो गमिष्यन् नृपमार्तरूपम्}
{व्यतिष्ठत प्रेक्ष्य तदा सुमन्त्रं पितुर्महात्मा प्रतिहारणार्थम्} %2-33-30

\twolineshloka
{पितुर्निदेशेन तु धर्मवत्सलो वनप्रवेशे कृतबुद्धिनिश्चयः}
{स राघवः प्रेक्ष्य सुमन्त्रमब्रवीन्निवेदयस्वागमनं नृपाय मे} %2-33-31


॥इत्यार्षे श्रीमद्रामायणे वाल्मीकीये आदिकाव्ये अयोध्याकाण्डे पौरवाक्यम् नाम त्रयस्त्रिंशः सर्गः ॥२-३३॥
