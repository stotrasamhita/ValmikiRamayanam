\sect{चतुर्विंशः सर्गः — कौसल्यार्तिसमाश्वासनम्}

\twolineshloka
{तं समीक्ष्य व्यवसितं पितुर्निर्देशपालने}
{कौसल्या बाष्पसंरुद्धा वचो धर्मिष्ठमब्रवीत्} %2-24-1

\twolineshloka
{अदृष्टदुःखो धर्मात्मा सर्वभूतप्रियंवदः}
{मयि जातो दशरथात् कथमुञ्छेन वर्तयेत्} %2-24-2

\twolineshloka
{यस्य भृत्याश्च दासाश्च मृष्टान्यन्नानि भुञ्जते}
{कथं स भोक्ष्यते रामो वने मूलफलान्ययम्} %2-24-3

\twolineshloka
{क एतच्छ्रद्दधेच्छ्रुत्वा कस्य वा न भवेद् भयम्}
{गुणवान् दयितो राज्ञः काकुत्स्थो यद् विवास्यते} %2-24-4

\twolineshloka
{नूनं तु बलवाँल्लोके कृतान्तः सर्वमादिशन्}
{लोके रामाभिरामस्त्वं वनं यत्र गमिष्यसि} %2-24-5

\twolineshloka
{अयं तु मामात्मभवस्तवादर्शनमारुतः}
{विलापदुःखसमिधो रुदिताश्रुहुताहुतिः} %2-24-6

\twolineshloka
{चिन्ताबाष्पमहाधूमस्तवागमनचिन्तजः}
{कर्शयित्वाधिकं पुत्र निःश्वासायाससम्भवः} %2-24-7

\twolineshloka
{त्वया विहीनामिह मां शोकाग्निरतुलो महान्}
{प्रधक्ष्यति यथा कक्ष्यं चित्रभानुर्हिमात्यये} %2-24-8

\twolineshloka
{कथं हि धेनुः स्वं वत्सं गच्छन्तमनुगच्छति}
{अहं त्वानुगमिष्यामि यत्र वत्स गमिष्यसि} %2-24-9

\twolineshloka
{यथा निगदितं मात्रा तद् वाक्यं पुरुषर्षभः}
{श्रुत्वा रामोऽब्रवीद् वाक्यं मातरं भृशदुःखिताम्} %2-24-10

\twolineshloka
{कैकेय्या वञ्चितो राजा मयि चारण्यमाश्रिते}
{भवत्या च परित्यक्तो न नूनं वर्तयिष्यति} %2-24-11

\twolineshloka
{भर्तुः किल परित्यागो नृशंसः केवलं स्त्रियाः}
{स भवत्या न कर्तव्यो मनसापि विगर्हितः} %2-24-12

\twolineshloka
{यावज्जीवति काकुत्स्थः पिता मे जगतीपतिः}
{शुश्रूषा क्रियतां तावत् स हि धर्मः सनातनः} %2-24-13

\twolineshloka
{एवमुक्ता तु रामेण कौसल्या शुभदर्शना}
{तथेत्युवाच सुप्रीता राममक्लिष्टकारिणम्} %2-24-14

\twolineshloka
{एवमुक्तस्तु वचनं रामो धर्मभृतां वरः}
{भूयस्तामब्रवीद् वाक्यं मातरं भृशदुःखिताम्} %2-24-15

\twolineshloka
{मया चैव भवत्या च कर्तव्यं वचनं पितुः}
{राजा भर्ता गुरुः श्रेष्ठः सर्वेषामीश्वरः प्रभुः} %2-24-16

\twolineshloka
{इमानि तु महारण्ये विहृत्य नव पञ्च च}
{वर्षाणि परमप्रीत्या स्थास्यामि वचने तव} %2-24-17

\twolineshloka
{एवमुक्ता प्रियं पुत्रं बाष्पपूर्णानना तदा}
{उवाच परमार्ता तु कौसल्या सुतवत्सला} %2-24-18

\twolineshloka
{आसां राम सपत्नीनां वस्तुं मध्ये न मे क्षमम्}
{नय मामपि काकुत्स्थ वनं वन्यां मृगीमिव} %2-24-19

\twolineshloka
{यदि ते गमने बुद्धिः कृता पितरपेक्षया}
{तां तथा रुदतीं रामो रुदन् वचनमब्रवीत्} %2-24-20

\twolineshloka
{जीवन्त्या हि स्त्रिया भर्ता दैवतं प्रभुरेव च}
{भवत्या मम चैवाद्य राजा प्रभवति प्रभुः} %2-24-21

\twolineshloka
{न ह्यनाथा वयं राज्ञा लोकनाथेन धीमता}
{भरतश्चापि धर्मात्मा सर्वभूतप्रियंवदः} %2-24-22

\twolineshloka
{भवतीमनुवर्तेत स हि धर्मरतः सदा}
{दारुणश्चाप्ययं शोको यथैनं न विनाशयेत्} %2-24-23

\twolineshloka
{राज्ञो वृद्धस्य सततं हितं चर समाहिता}
{व्रतोपवासनिरता या नारी परमोत्तमा} %2-24-24

\twolineshloka
{भर्तारं नानुवर्तेत सा च पापगतिर्भवेत्}
{भर्तुः शुश्रूषया नारी लभते स्वर्गमुत्तमम्} %2-24-25

\twolineshloka
{अपि या निर्नमस्कारा निवृत्ता देवपूजनात्}
{शुश्रूषामेव कुर्वीत भर्तुः प्रियहिते रता} %2-24-26

\twolineshloka
{एष धर्मः स्त्रिया नित्यो वेदे लोके श्रुतः स्मृतः}
{अग्निकार्येषु च सदा सुमनोभिश्च देवताः} %2-24-27

\twolineshloka
{पूज्यास्ते मत्कृते देवि ब्राह्मणाश्चैव सत्कृताः}
{एवं कालं प्रतीक्षस्व ममागमनकाङ्क्षिणी} %2-24-28

\twolineshloka
{नियता नियताहारा भर्तृशुश्रूषणे रता}
{प्राप्स्यसे परमं कामं मयि पर्यागते सति} %2-24-29

\twolineshloka
{यदि धर्मभृतां श्रेष्ठो धारयिष्यति जीवितम्}
{एवमुक्ता तु रामेण बाष्पपर्याकुलेक्षणा} %2-24-30

\twolineshloka
{कौसल्या पुत्रशोकार्ता रामं वचनमब्रवीत्}
{गमने सुकृतां बुद्धिं न ते शक्नोमि पुत्रक} %2-24-31

\twolineshloka
{विनिवर्तयितुं वीर नूनं कालो दुरत्ययः}
{गच्छ पुत्र त्वमेकाग्रो भद्रं तेऽस्तु सदा विभो} %2-24-32

\threelineshloka
{पुनस्त्वयि निवृत्ते तु भविष्यामि गतक्लमा}
{प्रत्यागते महाभागे कृतार्थे चरितव्रते}
{पितुरानृण्यतां प्राप्ते स्वपिष्ये परमं सुखम्} %2-24-33

\twolineshloka
{कृतान्तस्य गतिः पुत्र दुर्विभाव्या सदा भुवि}
{यस्त्वां सञ्चोदयति मे वच आविध्य राघव} %2-24-34

\twolineshloka
{गच्छेदानीं महाबाहो क्षेमेण पुनरागतः}
{नन्दयिष्यसि मां पुत्र साम्ना श्लक्ष्णेन चारुणा} %2-24-35

\twolineshloka
{अपीदानीं स कालः स्याद् वनात् प्रत्यागतं पुनः}
{यत् त्वां पुत्रक पश्येयं जटावल्कलधारिणम्} %2-24-36

\twolineshloka
{तथा हि रामं वनवासनिश्चितं ददर्श देवी परमेण चेतसा}
{उवाच रामं शुभलक्षणं वचो बभूव च स्वस्त्ययनाभिकाङ्क्षिणी} %2-24-37


॥इत्यार्षे श्रीमद्रामायणे वाल्मीकीये आदिकाव्ये अयोध्याकाण्डे कौसल्यार्तिसमाश्वासनम् नाम चतुर्विंशः सर्गः ॥२-२४॥
