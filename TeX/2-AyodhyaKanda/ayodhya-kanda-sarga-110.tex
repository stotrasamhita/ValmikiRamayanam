\sect{दशाधिकशततमः सर्गः — इक्ष्वाकुवंशकीर्तनम्}

\twolineshloka
{क्रुद्धमाज्ञाय रामं तं वसिष्ठः प्रत्युवाच ह}
{जाबालिरपि जानीते लोकस्यास्य गतागतिम्} %2-110-1

\twolineshloka
{निवर्त्तयितुकामस्तु त्वामेतद्वाक्यमुक्तवान्}
{इमां लोकसमुत्पत्तिं लोकनाथ निबोध मे} %2-110-2

\threelineshloka
{सर्वं सलिलमेवासीत् पृथिवी यत्र निर्मिता}
{ततः समभवद्ब्रह्मा स्वयम्भूर्दैवतैः सह}
{स वराहस्ततो भूत्वा प्रोज्जहार वसुन्धराम्} %2-110-3

\twolineshloka
{असृजच्च जगत् सर्वं सह पुत्रैः कृतात्मभिः}
{आकाशप्रभवो ब्रह्मा शाश्वतो नित्य अव्ययः} %2-110-4

\onelineshloka
{तस्मान्मरीचिः सञ्जज्ञे मरीचेः काश्यपः सुतः} %2-110-5

\twolineshloka
{विवस्वान् काश्यपाज्जज्ञे मनुर्वैवस्वतस्सुतः}
{स तु प्रजापतिः पूर्वमिक्ष्वाकुस्तु मनोः सुतः} %2-110-6

\twolineshloka
{यस्येयं प्रथमं दत्ता समृद्धा मनुना मही}
{तमिक्ष्वाकुमयोध्यायां राजानं विद्धि पूर्वकम्} %2-110-7

\twolineshloka
{इक्ष्वाकोस्तु सुतः श्रीमान् कुक्षिरेवेति विश्रुतः}
{कुक्षेरथात्मजो वीरो विकुक्षिरुदपद्यत} %2-110-8

\twolineshloka
{विकुक्षेस्तु महातेजा बाणः पुत्रः प्रतापवान्}
{बाणस्य तु महाबाहुरनरण्यो महायशाः} %2-110-9

\twolineshloka
{नानावृष्टिर्बभूवास्मिन्न दुर्भिक्षं सतां वरे}
{अनरण्ये महाराजे तस्करो नापि कश्चन} %2-110-10

\threelineshloka
{अनरण्यान्महाबाहुः पृथू राजा बभूव ह}
{तस्मात् पृथोर्महाराजस्त्रिशङ्कुरुदपद्यत}
{स सत्यवचनाद्वीरः सशरीरो दिवङ्गतः} %2-110-11

\twolineshloka
{त्रिशङ्कोरभवत्सूनुर्धुन्धुमारो महायशाः}
{धुन्धुमारो महातेजा युवनाश्वो व्यजायत} %2-110-12

\twolineshloka
{युवनाश्वसुतः श्रीमान् मान्धाता समपद्यत}
{मान्धातुस्तु महातेजाः सुसन्धिरुदपद्यत} %2-110-13

\twolineshloka
{सुसन्धेरपि पुत्रौ द्वौ ध्रुवसन्धिः प्रसेनजित्}
{यशस्वी ध्रुवसन्धेस्तु भरतो रिपुसूदनः} %2-110-14

\threelineshloka
{भरतात्तु महाबाहोरसितो नाम जायत}
{यस्यैते प्रतिराजान उदपद्यन्त शत्रवः}
{हैहयास्तालजङ्घाश्च शूराश्च शशिबिन्दवः} %2-110-15

\twolineshloka
{तांस्तु सर्वान् प्रतिव्यूह्य युद्धे राजा प्रवासितः}
{स च शैलवरे रम्ये बभूवाभिरतो मुनिः} %2-110-16

\twolineshloka
{द्वे चास्य भार्ये गर्भिण्यौ बभूवतुरिति श्रुतिः}
{एका गर्भविनाशाय सपत्न्यै सगरं ददौ} %2-110-17

\twolineshloka
{भार्गवश्च्यवनो नाम हिमवन्तमुपाश्रितः}
{तमृषिं समुपागम्य कालिन्दी त्वभ्यवादयत्} %2-110-18

\threelineshloka
{स तामभ्यवदद्विप्रो वरेप्सुं पुत्रजन्मनि}
{पुत्रस्ते भविता देवि महात्मा लोकविश्रुतः}
{धार्मिकश्च सुशीलश्च वंशकर्त्ताऽरिसूदनः} %2-110-19

\threelineshloka
{कृत्वा प्रदक्षिणं हृष्टा मुनिं तमनुमान्य च}
{पद्मपत्रसमानाक्षं पद्मगर्भसमप्रभम्}
{ततः सा गृहमागम्य देवी पुत्रं व्यजायत} %2-110-20

\twolineshloka
{सपत्न्या तु गरस्तस्यै दत्तो गर्भजिघांसया}
{गरेण सह तेनैव जातः स सगरोऽभवत्} %2-110-21

\twolineshloka
{स राजा सगरो नाम यः समुद्रमखानयत्}
{इष्ट्वा पर्वणि वेगेन त्रासयन्तमिमाः प्रजाः} %2-110-22

\twolineshloka
{असमञ्जस्तु पुत्रोभूत् सगरस्येति नः श्रुतम्}
{जीवन्नेव स पित्रा तु निरस्तः पापकर्मकृत्} %2-110-23

\twolineshloka
{अंशुमानपि पुत्रोऽभूदसमञ्जस्य वीर्य्यवान्}
{दिलीपोंऽशुमतः पुत्रो दिलीपस्य भगीरथः} %2-110-24

\twolineshloka
{भगीरथात् ककुत्स्थस्तु काकुत्स्था येन विश्रुताः}
{ककुत्स्थस्य च पुत्रोऽभूद्रघुर्येन तु राघवाः} %2-110-25

\twolineshloka
{रघोस्तु पुत्रस्तेजस्वी प्रवृद्धः पुरुषादकः}
{कल्माषपादः सौदास इत्येवं प्रथितो भुवि} %2-110-26

\twolineshloka
{कल्माषपादपुत्रोऽभूच्छङ्खणस्त्विति विश्रुतः}
{यस्तु तद्वीर्यमासाद्य सहसैन्यो व्यनीनशत्} %2-110-27

\twolineshloka
{शङ्खणस्य च पुत्रोऽभूच्छूरः श्रीमान् सुदर्शनः}
{सुदर्शनस्याग्निवर्ण अग्निवर्णस्य शीघ्रगः} %2-110-28

\twolineshloka
{शीघ्रगस्य मरुः पुत्रो मरोः पुत्रः प्रशुश्रुकः}
{प्रशुश्रुकस्य पुत्रोऽभूदम्बरीषो महाद्युतिः} %2-110-29

\twolineshloka
{अम्बरीषस्य पुत्रोऽभून्नहुषः सत्यविक्रमः}
{नहुषस्य च नाभागः पुत्रः परमधार्मिकः} %2-110-30

\twolineshloka
{अजश्च सुव्रतश्चैव नाभागस्य सुतावुभौ}
{अजस्य चैव धर्मात्मा राजा दशरथः सुतः} %2-110-31

\twolineshloka
{तस्य ज्येष्ठोऽसि दायादो राम इत्यभिविश्रुतः}
{तद्गृहाण स्वकं राज्यमवेक्षस्व जनं नृप} %2-110-32

\twolineshloka
{इक्ष्वाकूणां हि सर्वेषां राजा भवति पूर्वजः}
{पूर्वजे नावरः पुत्रो ज्येष्ठो राज्येऽभिषिच्यते} %2-110-33

\twolineshloka
{स राघवाणां कुलधर्ममात्मनः सनातनं नाद्य विहन्तुमर्हसि}
{प्रभूतरत्नामनुशाधि मेदिनीं प्रभूतराष्ट्रां पितृवन्महायशः} %2-110-34


॥इत्यार्षे श्रीमद्रामायणे वाल्मीकीये आदिकाव्ये अयोध्याकाण्डे इक्ष्वाकुवंशकीर्तनम् नाम दशाधिकशततमः सर्गः ॥२-११०॥
