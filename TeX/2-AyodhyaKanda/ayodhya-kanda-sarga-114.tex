\sect{चतुर्दशाधिकशततमः सर्गः — अयोध्याप्रवेशः}

\twolineshloka
{स्त्रिग्धगम्भीरघोषेण स्यन्दनेनोपयान् प्रभुः}
{अयोध्यां भरतः क्षिप्रं प्रविवेश महायशाः} %2-114-1

\twolineshloka
{बिडालोलूकचरितामालीननरवारणाम्}
{तिमिराभ्याहतां कालीमप्रकाशां निशामिव} %2-114-2

\twolineshloka
{राहुशत्रोः प्रियां पत्नीं श्रिया प्रज्वलितप्रभाम्}
{ग्रहेणाभ्युत्थिते नैकां रोहिणीमिव पीडिताम्} %2-114-3

\twolineshloka
{अल्पोष्णक्षुब्धसलिलां धर्मोत्तप्तविहङ्गमाम्}
{लीनमीनझषग्राहां कृशां गिरिनदीमिव} %2-114-4

\twolineshloka
{विधूमामिव हेमाभामध्वराग्नेः समुत्थिताम्}
{हविरभ्युक्षितां पश्चात् शिखां विप्रलयं गताम्} %2-114-5

\twolineshloka
{विध्वस्तकवचां रुग्णजवाजिरथध्वजाम्}
{हतप्रवीरामापन्नां चमूमिव महाहवे} %2-114-6

\twolineshloka
{सफेना सस्वना भूत्वा सागरस्य समुत्थिताम्}
{प्रशान्तमारुतोद्घातां जलोर्मिमिव निस्वनाम्} %2-114-7

\twolineshloka
{त्यक्तां यज्ञायुधैः सर्वैरभिरूपैश्च याजकैः}
{सुत्याकाले विनिर्वृत्ते वेदिं गतरवामिव} %2-114-8

\twolineshloka
{गोष्ठमध्ये स्थितामार्त्तामचरन्तीं तृणं नवम्}
{गोवृषेण परित्यक्तां गवां पत्तिमिवोत्सुकाम्} %2-114-9

\twolineshloka
{प्रभाकराद्यैः सुस्निग्धैः प्रज्वलद्भिरिवोत्तमैः}
{वियुक्तां मणिभिर्जात्यैर्नवां मुक्तावलीमिव} %2-114-10

\twolineshloka
{सहसा चलितां स्थानान्महीं पुण्यक्षयाद्गताम्}
{संहृतद्युतिविस्तारां तारामिव दिवश्च्युताम्} %2-114-11

\twolineshloka
{पुष्पनद्धां वसन्तान्ते मत्तभ्रमरनादिताम्}
{द्रुतदावाग्निविप्लुष्टां क्लान्तां वनलतामिव} %2-114-12

\twolineshloka
{सम्मूढनिगमां स्तब्धां सङ्क्षिप्तविपणापणाम्}
{प्रच्छन्नशशिनक्षत्रां द्यामिवाम्बुधरैर्वृताम्} %2-114-13

\twolineshloka
{क्षीणपानोत्तमैर्भिन्नैः शरावैरभिसंवृताम्}
{हतशौण्डामिवाकाशे पानभूमिमसंस्कृताम्} %2-114-14

\twolineshloka
{वृक्णभूमितलां निम्नां वृक्णपात्रैः समावृताम्}
{उपयुक्तोदकां भग्नां प्रपां निपतितामिव} %2-114-15

\twolineshloka
{विपुलां विततां चैव युक्तपाशां तरस्विनाम्}
{भूमौ बाणैर्विनिष्कृत्तां पतितां ज्यामिवायुधात्} %2-114-16

\twolineshloka
{सहसा युद्धशौण्डेन हयारोहेण वाहिताम्}
{निक्षिप्तभाण्डामुत्सृष्टां किशोरीमिव दुर्बलाम्} %2-114-17

\twolineshloka
{शुष्कतोयां महामत्स्यैः कूर्मैश्च बहुभिर्वृताम्}
{प्रभिन्नतटविस्तीर्णां वापीमिव हृतोत्पलाम्} %2-114-18

\twolineshloka
{पुरुषस्याप्रहृष्टस्य प्रतिषिद्धानुलेपनाम्}
{सन्तप्तामिव शोकेन गात्रयष्टिमभूषणाम्} %2-114-19

\twolineshloka
{प्रावृषि प्रविगाढायां प्रविष्टस्याभ्रमण्डलम्}
{प्रच्छन्नां नीलजीमूतैर्भास्करस्य प्रभामिव} %2-114-20

\twolineshloka
{भरतस्तु रथस्थः सन् श्रीमान् दशरथात्मजः}
{वाहयन्तं रथश्रेष्ठं सारथिं वाक्यमब्रवीत्} %2-114-21

\twolineshloka
{किं नु खल्वद्य गम्भीरो मूर्च्छितो न निशम्यते}
{यथापुरमयोध्यायां गीतवादित्रनिस्वनः} %2-114-22

\twolineshloka
{वारुणीमदगन्धश्च माल्यगन्धश्च मूर्च्छितः}
{धूपितागरुगन्धश्च न प्रवाति समन्ततः} %2-114-23

\threelineshloka
{यानप्रवरघोषश्च स्निग्धश्च हयनिस्वनः}
{प्रमत्तगजनादश्च महांश्च रथनिस्वनः}
{नेदानीं श्रूयते पुर्यामस्यां रामे विवासिते} %2-114-24

\twolineshloka
{चन्दनागरुगन्धांश्च महार्हाश्च नवस्रजः}
{गते हि रामे तरुणाः सन्तप्ता नोपभुञ्जते} %2-114-25

\twolineshloka
{बहिर्यात्रां न गच्छन्ति चित्रमाल्यधरा नराः}
{नोत्सवाः सम्प्रवर्त्तन्ते रामशोकार्दिते पुरे} %2-114-26

\twolineshloka
{सह नूनं मम भ्रात्रा पुरस्यास्यद्युतिर्गता}
{नहि राजत्ययोध्येयं सासारेवार्जुनी क्षपा} %2-114-27

\twolineshloka
{कदा नु खलु मे भ्राता महोत्सव इवागतः}
{जनयिष्यत्ययोध्यायां हर्षं ग्रीष्म इवाम्बुदः} %2-114-28

\twolineshloka
{तरुणैश्चारुवेषैश्च नरैरुन्नतगामिभिः}
{सम्पतद्भिरयोध्यायां नाभिभान्ति महापथाः} %2-114-29

\twolineshloka
{एवं बहुविधं जल्पन् विवेश वसतिं पितुः}
{तेन हीनां नरेन्द्रेण सिंहहीनां गुहामिव} %2-114-30

\twolineshloka
{तदा तदन्तःपुरमुज्झितप्रभं सुरैरिवोत्सृष्टमभास्करं दिनम्}
{निरीक्ष्य सर्वन्तु विविक्तमात्मवान् मुमोच बाष्पं भरतः सुदुःखितः} %2-114-31


॥इत्यार्षे श्रीमद्रामायणे वाल्मीकीये आदिकाव्ये अयोध्याकाण्डे अयोध्याप्रवेशः नाम चतुर्दशाधिकशततमः सर्गः ॥२-११४॥
