\sect{एकोनविंशत्यधिकशततमः सर्गः — दण्डकारण्यप्रवेशः}

\twolineshloka
{अनसूया तु धर्मज्ञा श्रुत्वा तां महतीं कथाम्}
{पर्य्यष्वजत बाहुभ्यां शिरस्याघ्राय मैथिलीम्} %2-119-1

\threelineshloka
{व्यक्ताक्षरपदं चित्रं भाषितं मधुरं त्वया}
{यथा स्वयम्वरं वृत्तं तत्सर्वं हि श्रुतं मया}
{रमेऽहं कथया ते तु दृढं मधुरभाषिणि} %2-119-2

\threelineshloka
{रविरस्तं गतः श्रीमानुपोह्य रजनीं शिवाम्}
{दिवसं प्रतिकीर्णानामाहारार्थं पतित्रणाम्}
{सन्ध्याकाले निलीनानां निद्रार्थं श्रूयते ध्वनिः} %2-119-3

\twolineshloka
{एते चाप्यभिषेकार्द्रा मुनयः कलशोद्यताः}
{सहिता उपवर्तन्ते सलिलाप्लुतवल्कलाः} %2-119-4

\twolineshloka
{ऋषीणामग्निहोत्रेषु हुतेषु विधिपूर्वकम्}
{कपोताङ्गारुणो धूमो दृश्यते पवनोद्धतः} %2-119-5

\twolineshloka
{अल्पपर्णाहि तरवो घनीभूताः समन्ततः}
{विप्रकृष्टेपि देशेऽस्मिन्न प्रकाशन्ति वै दिशः} %2-119-6

\twolineshloka
{रजनीचरसत्त्वानि प्रचरन्ति समन्ततः}
{तपोवनमृगा ह्येते वेदितीर्थेषु शेरते} %2-119-7

\twolineshloka
{सम्प्रवृद्धा निशा सीते नक्षत्रसमलङ्कृता}
{जोत्स्नाप्रावरणश्चन्द्रो दृश्यतेऽभ्युदितोऽम्बरे} %2-119-8

\twolineshloka
{गम्यतामनुजानामि रामस्यानुचरी भव}
{कथयन्त्या हि मधुरं त्वयाहं परितोषिता} %2-119-9

\twolineshloka
{अलङ्कुरु च तावत्त्वं प्रत्यक्षं मम मैथिलि}
{प्रीतिं जनय मे वत्से दिव्यालङ्कारशोभिता} %2-119-10

\twolineshloka
{सा तथा समलङ्कृत्य सीता सुरसुतोपमा}
{प्रणम्य शिरसा तस्यै रामं त्वभिमुखी ययौ} %2-119-11

\twolineshloka
{तथा तु भूषितां सीतां ददर्श वदतां वरः}
{राघवः प्रीतिदानेन तपस्विन्या जहर्ष च} %2-119-12

\twolineshloka
{न्यवेदयत्ततः सर्वं सीता रामाय मैथिली}
{प्रीतिदानं तपस्विन्या वसनाभरणस्रजम्} %2-119-13

\twolineshloka
{प्रहृष्टस्त्वभवद्रामो लक्ष्मणश्च महारथः}
{मैथिल्याः सत्क्रियां दृष्ट्वा मानुषेषु सुदुर्लभाम्} %2-119-14

\twolineshloka
{ततस्तां शर्वरीं प्रीतः पुण्यां शशिनिभाननः}
{अर्चितस्तापसैः सिद्धैरुवास रघुनन्दनः} %2-119-15

\twolineshloka
{तस्यां रात्र्यां व्यतीतायामभिषिच्य हुताग्निकान्}
{आपृच्छेतां नरव्याघ्रौ तापसान् वनगोचरान्} %2-119-16

\twolineshloka
{तावूचुस्ते वनचरास्तापसा धर्मचारिणः}
{वनस्य तस्य सञ्चारं राक्षसैः समभिप्लुतम्} %2-119-17

\twolineshloka
{रक्षांसि पुरुषादानि नानारूपाणि राघव}
{वसन्त्यस्मिन् महारण्ये व्यालाश्च रुधिराशनाः} %2-119-18

\twolineshloka
{उच्छिष्टं वा प्रमत्तं वा तापसं धर्मचारिणम्}
{अदन्त्यस्मिन् महारण्ये तान्निवारय राघव} %2-119-19

\twolineshloka
{एष पन्था महर्षीणां फलान्याहरतां वने}
{अनेन तु वनं दुर्गं गन्तुं राघव ते क्षमम्} %2-119-20

\twolineshloka
{इतीव तैः प्राञ्जलिभिस्तपस्विभिर्द्विजैः कृतः स्वस्त्ययनः परं तपः}
{वनं सभार्य्यः प्रविवेश राघवः सलक्ष्मणः सूर्य्यमिवाभ्रमण्डलम्} %2-119-21


॥इत्यार्षे श्रीमद्रामायणे वाल्मीकीये आदिकाव्ये अयोध्याकाण्डे दण्डकारण्यप्रवेशः नाम एकोनविंशत्यधिकशततमः सर्गः ॥२-११९॥
