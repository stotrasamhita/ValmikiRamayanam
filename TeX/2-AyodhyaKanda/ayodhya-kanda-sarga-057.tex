\sect{सप्तपञ्चाशः सर्गः — सुमन्त्रोपावर्तनम्}

\twolineshloka
{कथयित्वा सुदुह्ख आर्तः सुमन्त्रेण चिरम् सह}
{रामे दक्षिण कूलस्थे जगाम स्व गृहम् गुहः} %2-57-1

\twolineshloka
{भरद्वाजाभिगमनम् प्रयागे च सहासनम्}
{आगिरेर्गमनम् तेषाम् तत्रस्थैरभिलक्षितम्} %2-57-2

\twolineshloka
{अनुज्ञातः सुमन्त्रः अथ योजयित्वा हय उत्तमान्}
{अयोध्याम् एव नगरीम् प्रययौ गाढ दुर्मनाः} %2-57-3

\twolineshloka
{स वनानि सुगन्धीनि सरितः च सराम्सि च}
{पश्यन्न् अतिययौ शीघ्रम् ग्रामाणि नगराणि च} %2-57-4

\twolineshloka
{ततः साय अह्न समये तृतीये अहनि सारथिः}
{अयोध्याम् समनुप्राप्य निरानन्दाम् ददर्श ह} %2-57-5

\twolineshloka
{स शून्याम् इव निह्शब्दाम् दृष्ट्वा परम दुर्मनाः}
{सुमन्त्रः चिन्तयाम् आस शोक वेग समाहतः} %2-57-6

\twolineshloka
{कच्चिन् न सगजा साश्वा सजना सजन अधिपा}
{राम सम्ताप दुह्खेन दग्धा शोक अग्निना पुरी} %2-57-7

\twolineshloka
{इति चिन्ता परः सूतः वाजिभिः श्रीघ्रपातिभिः}
{नगरद्वारमासाद्य त्वरितः प्रविवेश ह} %2-57-8

\twolineshloka
{सुमन्त्रम् अभियान्तम् तम् शतशो अथ सहस्रशः}
{क्व रामैति पृच्चन्तः सूतम् अभ्यद्रवन् नराः} %2-57-9

\twolineshloka
{तेषाम् शशम्स गङ्गायाम् अहम् आपृच्च्य राघवम्}
{अनुज्ञातः निवृत्तः अस्मि धार्मिकेण महात्मना} %2-57-10

\twolineshloka
{ते तीर्णाइति विज्ञाय बाष्प पूर्ण मुखा जनाः}
{अहो धिग् इति निश्श्वस्य हा राम इति च चुक्रुशुः} %2-57-11

\twolineshloka
{शुश्राव च वचः तेषाम् बृन्दम् बृन्दम् च तिष्ठताम्}
{हताः स्म खलु ये न इह पश्यामैति राघवम्} %2-57-12

\twolineshloka
{दान यज्ञ विवाहेषु समाजेषु महत्सु च}
{न द्रक्ष्यामः पुनर् जातु धार्मिकम् रामम् अन्तरा} %2-57-13

\twolineshloka
{किम् समर्थम् जनस्य अस्य किम् प्रियम् किम् सुख आवहम्}
{इति रामेण नगरम् पितृवत् परिपालितम्} %2-57-14

\twolineshloka
{वात अयन गतानाम् च स्त्रीणाम् अन्वन्तर आपणम्}
{राम शोक अभितप्तानाम् शुश्राव परिदेवनम्} %2-57-15

\twolineshloka
{स राज मार्ग मध्येन सुमन्त्रः पिहित आननः}
{यत्र राजा दशरथः तत् एव उपययौ गृहम्} %2-57-16

\twolineshloka
{सो अवतीर्य रथात् शीघ्रम् राज वेश्म प्रविश्य च}
{कक्ष्याः सप्त अभिचक्राम महा जन समाकुलाः} %2-57-17

\twolineshloka
{हर्म्यैर्विमानैः प्रासादैरवेक्ष्याथ समागतम्}
{हाहाकारकृता नार्यो रामदर्शनकर्शिताः} %2-57-18

\twolineshloka
{आयतैर्विमलैर्नेत्रैरश्रुवेगपरिप्लुतैः}
{अन्योन्यमभिवीक्षन्तेऽव्यक्तमार्ततराः स्त्रीयः} %2-57-19

\twolineshloka
{ततः दशरथ स्त्रीणाम् प्रासादेभ्यः ततः ततः}
{राम शोक अभितप्तानाम् मन्दम् शुश्राव जल्पितम्} %2-57-20

\twolineshloka
{सह रामेण निर्यातः विना रामम् इह आगतः}
{सूतः किम् नाम कौसल्याम् शोचन्तीम् प्रति वक्ष्यति} %2-57-21

\twolineshloka
{यथा च मन्ये दुर्जीवम् एवम् न सुकरम् ध्रुवम्}
{आच्चिद्य पुत्रे निर्याते कौसल्या यत्र जीवति} %2-57-22

\twolineshloka
{सत्य रूपम् तु तत् वाक्यम् राज्ञः स्त्रीणाम् निशामयन्}
{प्रदीप्तम् इव शोकेन विवेश सहसा गृहम्} %2-57-23

\twolineshloka
{स प्रविश्य अष्टमीम् कक्ष्याम् राजानम् दीनम् आतुलम्}
{पुत्र शोक परिम्लानम् अपश्यत् पाण्डुरे गृहे} %2-57-24

\twolineshloka
{अभिगम्य तम् आसीनम् नर इन्द्रम् अभिवाद्य च}
{सुमन्त्रः राम वचनम् यथा उक्तम् प्रत्यवेदयत्} %2-57-25

\twolineshloka
{स तूष्णीम् एव तत् श्रुत्वा राजा विभ्रान्त चेतनः}
{मूर्चितः न्यपतत् भूमौ राम शोक अभिपीडितः} %2-57-26

\twolineshloka
{ततः अन्तः पुरम् आविद्धम् मूर्चिते पृथिवी पतौ}
{उद्धृत्य बाहू चुक्रोश नृपतौ पतिते क्षितौ} %2-57-27

\twolineshloka
{सुमित्रया तु सहिता कौसल्या पतितम् पतिम्}
{उत्थापयाम् आस तदा वचनम् च इदम् अब्रवीत्} %2-57-28

\twolineshloka
{इमम् तस्य महा भाग दूतम् दुष्कर कारिणः}
{वन वासात् अनुप्राप्तम् कस्मान् न प्रतिभाषसे} %2-57-29

\twolineshloka
{अद्य इमम् अनयम् कृत्वा व्यपत्रपसि राघव}
{उत्तिष्ठ सुकृतम् ते अस्तु शोके न स्यात् सहायता} %2-57-30

\twolineshloka
{देव यस्या भयात् रामम् न अनुपृच्चसि सारथिम्}
{न इह तिष्ठति कैकेयी विश्रब्धम् प्रतिभाष्यताम्} %2-57-31

\twolineshloka
{सा तथा उक्त्वा महा राजम् कौसल्या शोक लालसा}
{धरण्याम् निपपात आशु बाष्प विप्लुत भाषिणी} %2-57-32

\twolineshloka
{एवम् विलपतीम् दृष्ट्वा कौसल्याम् पतिताम् भुवि}
{पतिम् च अवेक्ष्य ताः सर्वाः सुस्वरम् रुरुदुः स्त्रियः} %2-57-33

\fourlineindentedshloka
{ततः तम् अन्तः पुर नादम् उत्थितम्}
{समीक्ष्य वृद्धाः तरुणाः च मानवाः}
{स्त्रियः च सर्वा रुरुदुः समन्ततः}
{पुरम् तदा आसीत् पुनर् एव सम्कुलम्} %2-57-34


॥इत्यार्षे श्रीमद्रामायणे वाल्मीकीये आदिकाव्ये अयोध्याकाण्डे सुमन्त्रोपावर्तनम् नाम सप्तपञ्चाशः सर्गः ॥२-५७॥
