\sect{सप्तमः सर्गः — मन्थरापरिदेवनम्}

\twolineshloka
{ज्ञातिदासी यतो जाता कैकेय्या तु सहोषिता}
{प्रासादं चन्द्रसंकाशमारुरोह यदृच्छया} %2-7-1

\twolineshloka
{सिक्तराजपथां कृत्स्नां प्रकीर्णकमलोत्पलाम्}
{अयोध्यां मन्थरा तस्मात् प्रासादादन्ववैक्षत} %2-7-2

\twolineshloka
{पताकाभिर्वरार्हाभिर्ध्वजैश्च समलंकृताम्}
{सिक्तां चन्दनतोयैश्च शिरःस्नातजनैर्युताम्} %2-7-3

\twolineshloka
{माल्यमोदकहस्तैश्च द्विजेन्द्रैरभिनादिताम्}
{शुक्लदेवगृहद्वारां सर्ववादित्रनादिताम्} %2-7-4

\twolineshloka
{सम्प्रहृष्टजनाकीर्णां ब्रह्मघोषनिनादिताम्}
{प्रहृष्टवरहस्त्यश्वां सम्प्रणर्दितगोवृषाम्} %2-7-5

\twolineshloka
{हृष्टप्रमुदितैः पौरैरुच्छ्रितध्वजमालिनीम्}
{अयोध्यां मन्थरा दृष्ट्वा परं विस्मयमागता} %2-7-6

\twolineshloka
{सा हर्षोत्फुल्लनयनां पाण्डुरक्षौमवासिनीम्}
{अविदूरे स्थितां दृष्ट्वा धात्रीं पप्रच्छ मन्थरा} %2-7-7

\twolineshloka
{उत्तमेनाभिसंयुक्ता हर्षेणार्थपरा सती}
{राममाता धनं किं नु जनेभ्यः सम्प्रयच्छति} %2-7-8

\twolineshloka
{अतिमात्रं प्रहर्षः किं जनस्यास्य च शंस मे}
{कारयिष्यति किं वापि सम्प्रहृष्टो महीपतिः} %2-7-9

\twolineshloka
{विदीर्यमाणा हर्षेण धात्री तु परया मुदा}
{आचचक्षेऽथ कुब्जायै भूयसीं राघवे श्रियम्} %2-7-10

\twolineshloka
{श्वः पुष्येण जितक्रोधं यौवराज्येन चानघम्}
{राजा दशरथो राममभिषेक्ता हि राघवम्} %2-7-11

\twolineshloka
{धात्र्यास्तु वचनं श्रुत्वा कुब्जा क्षिप्रममर्षितः}
{कैलासशिखराकारात् प्रासादादवरोहत} %2-7-12

\twolineshloka
{सा दह्यमाना क्रोधेन मन्थरा पापदर्शिनी}
{शयानामेव कैकेयीमिदं वचनमब्रवीत्} %2-7-13

\twolineshloka
{उत्तिष्ठ मूढे किं शेषे भयं त्वामभिवर्तते}
{उपप्लुतमघौघेन नात्मानमवबुध्यसे} %2-7-14

\twolineshloka
{अनिष्टे सुभगाकारे सौभाग्येन विकत्थसे}
{चलं हि तव सौभाग्यं नद्याः स्रोत इवोष्णगे} %2-7-15

\twolineshloka
{एवमुक्ता तु कैकेयी रुष्टया परुषं वचः}
{कुब्जया पापदर्शिन्या विषादमगमत् परम्} %2-7-16

\twolineshloka
{कैकेयी त्वब्रवीत् कुब्जां कच्चित् क्षेमं न मन्थरे}
{विषण्णवदनां हि त्वां लक्षये भृशदुःखिताम्} %2-7-17

\twolineshloka
{मन्थरा तु वचः श्रुत्वा कैकेय्या मधुराक्षरम्}
{उवाच क्रोधसंयुक्ता वाक्यं वाक्यविशारदा} %2-7-18

\twolineshloka
{सा विषण्णतरा भूत्वा कुब्जा तस्यां हितैषिणी}
{विषादयन्ती प्रोवाच भेदयन्ती च राघवम्} %2-7-19

\twolineshloka
{अक्षयं सुमहद् देवि प्रवृत्तं त्वद्विनाशनम्}
{रामं दशरथो राजा यौवराज्येऽभिषेक्ष्यति} %2-7-20

\twolineshloka
{सास्म्यगाधे भये मग्ना दुःखशोकसमन्विता}
{दह्यमानानलेनेव त्वद्धितार्थमिहागता} %2-7-21

\twolineshloka
{तव दुःखेन कैकेयि मम दुःखं महद् भवेत्}
{त्वद्वृद्धौ मम वृद्धिश्च भवेदिह न संशयः} %2-7-22

\twolineshloka
{नराधिपकुले जाता महिषी त्वं महीपतेः}
{उग्रत्वं राजधर्माणां कथं देवि न बुध्यसे} %2-7-23

\twolineshloka
{धर्मवादी शठो भर्ता श्लक्ष्णवादी च दारुणः}
{शुद्धभावेन जानीषे तेनैवमतिसंधिता} %2-7-24

\twolineshloka
{उपस्थितः प्रयुञ्जानस्त्वयि सान्त्वमनर्थकम्}
{अर्थेनैवाद्य ते भर्ता कौसल्यां योजयिष्यति} %2-7-25

\twolineshloka
{अपवाह्य तु दुष्टात्मा भरतं तव बन्धुषु}
{काल्ये स्थापयिता रामं राज्ये निहतकण्टके} %2-7-26

\twolineshloka
{शत्रुः पतिप्रवादेन मात्रेव हितकाम्यया}
{आशीविषसर्प इवाङ्गेन बाले परिधृतस्त्वया} %2-7-27

\twolineshloka
{यथा हि कुर्याच्छत्रुर्वा सर्पो वा प्रत्युपेक्षितः}
{राज्ञा दशरथेनाद्य सपुत्रा त्वं तथा कृता} %2-7-28

\twolineshloka
{पापेनानृतसान्त्वेन बाले नित्यं सुखोचिता}
{रामं स्थापयता राज्ये सानुबन्धा हता ह्यसि} %2-7-29

\twolineshloka
{सा प्राप्तकालं कैकेयि क्षिप्रं कुरु हितं तव}
{त्रायस्व पुत्रमात्मानं मां च विस्मयदर्शने} %2-7-30

\twolineshloka
{मन्थराया वचः श्रुत्वा शयनात् सा शुभानना}
{उत्तस्थौ हर्षसम्पूर्णा चन्द्रलेखेव शारदी} %2-7-31

\twolineshloka
{अतीव सा तु संतुष्टा कैकेयी विस्मयान्विता}
{दिव्यमाभरणं तस्यै कुब्जायै प्रददौ शुभम्} %2-7-32

\twolineshloka
{दत्त्वा त्वाभरणं तस्यै कुब्जायै प्रमदोत्तमा}
{कैकेयी मन्थरां हृष्टा पुनरेवाब्रवीदिदम्} %2-7-33

\twolineshloka
{इदं तु मन्थरे मह्यमाख्यातं परमं प्रियम्}
{एतन्मे प्रियमाख्यातं किं वा भूयः करोमि ते} %2-7-34

\twolineshloka
{रामे वा भरते वाहं विशेषं नोपलक्षये}
{तस्मात् तुष्टास्मि यद् राजा रामं राज्येऽभिषेक्ष्यति} %2-7-35

\twolineshloka
{न मे परं किंचिदितो वरं पुनः प्रियं प्रियार्हे सुवचं वचोऽमृतम्}
{तथा ह्यवोचस्त्वमतः प्रियोत्तरं वरं परं ते प्रददामि तं वृणु} %2-7-36


॥इत्यार्षे श्रीमद्रामायणे वाल्मीकीये आदिकाव्ये अयोध्याकाण्डे मन्थरापरिदेवनम् नाम सप्तमः सर्गः ॥२-७॥
