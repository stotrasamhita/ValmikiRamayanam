\sect{सप्ततितमः सर्गः — भरतप्रस्थानम्}

\twolineshloka
{भरते ब्रुवति स्वप्नम् दूताः ते क्लान्त वाहनाः}
{प्रविश्य असह्य परिखम् रम्यम् राज गृहम् पुरम्} %2-70-1

\twolineshloka
{समागम्य तु राज्ञा च राज पुत्रेण च अर्चिताः}
{राज्ञः पादौ गृहीत्वा तु तम् ऊचुर् भरतम् वचः} %2-70-2

\twolineshloka
{पुरोहितः त्वा कुशलम् प्राह सर्वे च मन्त्रिणः}
{त्वरमाणः च निर्याहि कृत्यम् आत्ययिकम् त्वया} %2-70-3

\twolineshloka
{इमानि च महार्हाणि वस्त्राण्आभरणानि च}
{प्रतिगृह्य विशालाक्ष मातुलस्य च दापय} %2-70-4

\twolineshloka
{अत्र विम्शति कोट्यः तु नृपतेर् मातुलस्य ते}
{दश कोट्यः तु सम्पूर्णाः तथैव च नृप आत्मज} %2-70-5

\twolineshloka
{प्रतिगृह्य च तत् सर्वम् स्वनुरक्तः सुहृज् जने}
{दूतान् उवाच भरतः कामैः सम्प्रतिपूज्य तान्} %2-70-6

\twolineshloka
{कच्चित् सुकुशली राजा पिता दशरथो मम}
{कच्चिच् च अरागता रामे लक्ष्मणे वा महात्मनि} %2-70-7

\twolineshloka
{आर्या च धर्म निरता धर्मज्ञा धर्म दर्शिनी}
{अरोगा च अपि कौसल्या माता रामस्य धीमतः} %2-70-8

\twolineshloka
{कच्चित् सुमित्रा धर्मज्ञा जननी लक्ष्मणस्य या}
{शत्रुघ्नस्य च वीरस्य सारोगा च अपि मध्यमा} %2-70-9

\twolineshloka
{आत्म कामा सदा चण्डी क्रोधना प्राज्ञ मानिनी}
{अरोगा च अपि कैकेयी माता मे किम् उवाच ह} %2-70-10

\twolineshloka
{एवम् उक्ताः तु ते दूता भरतेन महात्मना}
{ऊचुः सम्प्रश्रितम् वाक्यम् इदम् तम् भरतम् तदा} %2-70-11

\twolineshloka
{कुशलाः ते नर व्याघ्र येषाम् कुशलम् इच्चसि}
{श्रीश्च त्वाम् वृणुते पद्मा युज्यताम् चापि ते रकः} %2-70-12

\twolineshloka
{भरतः च अपि तान् दूतान् एवम् उक्तः अभ्यभाषत}
{आपृच्चे अहम् महा राजम् दूताः सम्त्वरयन्ति माम्} %2-70-13

\twolineshloka
{एवम् उक्त्वा तु तान् दूतान् भरतः पार्थिव आत्मजः}
{दूतैः सम्चोदितः वाक्यम् मातामहम् उवाच ह} %2-70-14

\twolineshloka
{राजन् पितुर् गमिष्यामि सकाशम् दूत चोदितः}
{पुनर् अपि अहम् एष्यामि यदा मे त्वम् स्मरिष्यसि} %2-70-15

\twolineshloka
{भरतेन एवम् उक्तः तु नृपो मातामहः तदा}
{तम् उवाच शुभम् वाक्यम् शिरस्य् आघ्राय राघवम्} %2-70-16

\twolineshloka
{गच्च तात अनुजाने त्वाम् कैकेयी सुप्रजाः त्वया}
{मातरम् कुशलम् ब्रूयाः पितरम् च परम् तप} %2-70-17

\twolineshloka
{पुरोहितम् च कुशलम् ये च अन्ये द्विज सत्तमाः}
{तौ च तात महा इष्वासौ भ्रातरु राम लक्ष्मणौ} %2-70-18

\twolineshloka
{तस्मै हस्ति उत्तमामः चित्रान् कम्बलान् अजिनानि च}
{अभिसत्कृत्य कैकेयो भरताय धनम् ददौ} %2-70-19

\twolineshloka
{रुक्म निष्क सहस्रे द्वे षोडश अश्व शतानि च}
{सत्कृत्य कैकेयी पुत्रम् केकयो धनम् आदिशत्} %2-70-20

\twolineshloka
{तथा अमात्यान् अभिप्रेतान् विश्वास्यामः च गुण अन्वितान्}
{ददाव् अश्व पतिः शीघ्रम् भरताय अनुयायिनः} %2-70-21

\twolineshloka
{ऐरावतान् ऐन्द्र शिरान् नागान् वै प्रिय दर्शनान्}
{खरान् शीघ्रान् सुसम्युक्तान् मातुलो अस्मै धनम् ददौ} %2-70-22

\twolineshloka
{अन्तः पुरे अतिसम्वृद्धान् व्याघ्र वीर्य बल अन्वितान्}
{दम्ष्ट्र आयुधान् महा कायान् शुनः च उपायनम् ददौ} %2-70-23

\twolineshloka
{स मातामहम् आपृच्च्य मातुलम् च युधा जितम्}
{रथम् आरुह्य भरतः शत्रुघ्न सहितः ययौ} %2-70-24

\twolineshloka
{बभूव ह्यस्य हृदते चिन्ता सुमहती तदा}
{त्वरया चापि दूतानाम् स्वप्नस्यापि च दर्शनात्} %2-70-25

\twolineshloka
{स स्ववेश्माभ्यतिक्रम्य नरनागश्वसम्वृतम्}
{प्रपेदे सुमहच्छ्रीमान् राजमार्गमनुत्तमम्} %2-70-26

\twolineshloka
{अभ्यतीत्य ततोऽपश्यदन्तः पुरमुदारधीः}
{ततस्तद्भरतः श्रीमानाविवेशानिवारितः} %2-70-27

\twolineshloka
{स माता महमापृच्च्य मातुलम् च युधाजितम्}
{रथमारुह्य भरतः शत्रुघ्नसहितो ययौ} %2-70-28

\twolineshloka
{रथान् मण्डल चक्रामः च योजयित्वा परः शतम्}
{उष्ट्र गो अश्व खरैः भृत्या भरतम् यान्तम् अन्वयुः} %2-70-29

\fourlineindentedshloka
{बलेन गुप्तः भरतः महात्मा}
{सह आर्यकस्य आत्म समैः अमात्यैः}
{आदाय शत्रुघ्नम् अपेत शत्रुर्}
{गृहात् ययौ सिद्धैव इन्द्र लोकात्} %2-70-30


॥इत्यार्षे श्रीमद्रामायणे वाल्मीकीये आदिकाव्ये अयोध्याकाण्डे भरतप्रस्थानम् नाम सप्ततितमः सर्गः ॥२-७०॥
