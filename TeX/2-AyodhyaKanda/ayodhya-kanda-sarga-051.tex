\sect{एकपञ्चाशः सर्गः — गुहलक्ष्मणजागरणम्}

\twolineshloka
{तम् जाग्रतम् अदम्भेन भ्रातुर् अर्थाय लक्ष्मणम्}
{गुहः सम्ताप सम्तप्तः राघवम् वाक्यम् अब्रवीत्} %2-51-1

\twolineshloka
{इयम् तात सुखा शय्या त्वद् अर्थम् उपकल्पिता}
{प्रत्याश्वसिहि साध्व् अस्याम् राज पुत्र यथा सुखम्} %2-51-2

\twolineshloka
{उचितः अयम् जनः सर्वः क्लेशानाम् त्वम् सुख उचितः}
{गुप्ति अर्थम् जागरिष्यामः काकुत्स्थस्य वयम् निशाम्} %2-51-3

\twolineshloka
{न हि रामात् प्रियतरः मम अस्ति भुवि कश्चन}
{ब्रवीम्य् एतत् अहम् सत्यम् सत्येन एव च ते शपे} %2-51-4

\twolineshloka
{अस्य प्रसादात् आशम्से लोके अस्मिन् सुमहद् यशः}
{धर्म अवाप्तिम् च विपुलाम् अर्थ अवाप्तिम् च केवलाम्} %2-51-5

\twolineshloka
{सो अहम् प्रिय सखम् रामम् शयानम् सह सीतया}
{रक्षिष्यामि धनुष् पाणिः सर्वतः ज्ञातिभिः सह} %2-51-6

\twolineshloka
{न हि मे अविदितम् किम्चित् वने अस्मिमः चरतः सदा}
{चतुर् अन्गम् हि अपि बलम् सुमहत् प्रसहेमहि} %2-51-7

\twolineshloka
{लक्ष्मणः तम् तदा उवाच रक्ष्यमाणाः त्वया अनघ}
{न अत्र भीता वयम् सर्वे धर्मम् एव अनुपश्यता} %2-51-8

\twolineshloka
{कथम् दाशरथौ भूमौ शयाने सह सीतया}
{शक्या निद्रा मया लब्धुम् जीवितम् वा सुखानि वा} %2-51-9

\twolineshloka
{यो न देव असुरैः सर्वैः शक्यः प्रसहितुम् युधि}
{तम् पश्य सुख सम्विष्टम् तृणेषु सह सीतया} %2-51-10

\twolineshloka
{यो मन्त्र तपसा लब्धो विविधैः च परिश्रमैः}
{एको दशरथस्य एष पुत्रः सदृश लक्षणः} %2-51-11

\twolineshloka
{अस्मिन् प्रव्रजितः राजा न चिरम् वर्तयिष्यति}
{विधवा मेदिनी नूनम् क्षिप्रम् एव भविष्यति} %2-51-12

\twolineshloka
{विनद्य सुमहा नादम् श्रमेण उपरताः स्त्रियः}
{निर्घोष उपरतम् तात मन्ये राज निवेशनम्} %2-51-13

\twolineshloka
{कौसल्या चैव राजा च तथैव जननी मम}
{न आशम्से यदि जीवन्ति सर्वे ते शर्वरीम् इमाम्} %2-51-14

\twolineshloka
{जीवेद् अपि हि मे माता शत्रुघ्नस्य अन्ववेक्षया}
{तत् दुह्खम् यत् तु कौसल्या वीरसूर् विनशिष्यति} %2-51-15

\twolineshloka
{अनुरक्त जन आकीर्णा सुख आलोक प्रिय आवहा}
{राज व्यसन सम्सृष्टा सा पुरी विनशिष्यति} %2-51-16

\twolineshloka
{कथम् पुत्रम् महात्मानम् ज्येष्ठम् प्रियमपस्यतः}
{शरीरम् धारयुष्यान्ति प्राणा राज्ञो महात्मनः} %2-51-17

\twolineshloka
{विनष्टे नृपतौ पश्चात्कौसल्या विनशिष्यति}
{अनन्तरम् च माताऽपि मम नाशमुपैष्यति} %2-51-18

\twolineshloka
{अतिक्रान्तम् अतिक्रान्तम् अनवाप्य मनोरथम्}
{राज्ये रामम् अनिक्षिप्य पिता मे विनशिष्यति} %2-51-19

\twolineshloka
{सिद्ध अर्थाः पितरम् वृत्तम् तस्मिन् काले हि उपस्थिते}
{प्रेत कार्येषु सर्वेषु सम्स्करिष्यन्ति भूमिपम्} %2-51-20

\twolineshloka
{रम्य चत्वर सम्स्थानाम् सुविभक्त महा पथाम्}
{हर्म्य प्रसाद सम्पन्नाम् गणिका वर शोभिताम्} %2-51-21

\twolineshloka
{रथ अश्व गज सम्बाधाम् तूर्य नाद विनादिताम्}
{सर्व कल्याण सम्पूर्णाम् हृष्ट पुष्ट जन आकुलाम्} %2-51-22

\twolineshloka
{आराम उद्यान सम्पन्नाम् समाज उत्सव शालिनीम्}
{सुखिता विचरिष्यन्ति राज धानीम् पितुर् मम} %2-51-23

\twolineshloka
{अपि जीवेद्धशरथो वनवासात्पुनर्वयम्}
{प्रत्यागम्य महात्मानमपि पश्येम सुव्रतम्} %2-51-24

\twolineshloka
{अपि सत्य प्रतिज्ञेन सार्धम् कुशलिना वयम्}
{निवृत्ते वन वासे अस्मिन्न् अयोध्याम् प्रविशेमहि} %2-51-25

\twolineshloka
{परिदेवयमानस्य दुह्ख आर्तस्य महात्मनः}
{तिष्ठतः राज पुत्रस्य शर्वरी सा अत्यवर्तत} %2-51-26

\fourlineindentedshloka
{तथा हि सत्यम् ब्रुवति प्रजा हिते}
{नर इन्द्र पुत्रे गुरु सौहृदात् गुहः}
{मुमोच बाष्पम् व्यसन अभिपीडितः}
{ज्वरा आतुरः नागैव व्यथा आतुरः} %2-51-27


॥इत्यार्षे श्रीमद्रामायणे वाल्मीकीये आदिकाव्ये अयोध्याकाण्डे गुहलक्ष्मणजागरणम् नाम एकपञ्चाशः सर्गः ॥२-५१॥
