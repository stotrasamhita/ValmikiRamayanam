\sect{षष्ठितमः सर्गः — कौसल्यासमाश्वसनम्}

\twolineshloka
{ततः भूत उपसृष्टा इव वेपमाना पुनः पुनः}
{धरण्याम् गत सत्त्वा इव कौसल्या सूतम् अब्रवीत्} %2-60-1

\twolineshloka
{नय माम् यत्र काकुत्स्थः सीता यत्र च लक्ष्मणः}
{तान् विना क्षणम् अपि अत्र जीवितुम् न उत्सहे हि अहम्} %2-60-2

\twolineshloka
{निवर्तय रथम् शीघ्रम् दण्डकान् नय माम् अपि}
{अथ तान् न अनुगच्चामि गमिष्यामि यम क्षयम्} %2-60-3

\twolineshloka
{बाष्प वेगौपहतया स वाचा सज्जमानया}
{इदम् आश्वासयन् देवीम् सूतः प्रान्जलिर् अब्रवीत्} %2-60-4

\twolineshloka
{त्यज शोकम् च मोहम् च सम्भ्रमम् दुह्खजम् तथा}
{व्यवधूय च सम्तापम् वने वत्स्यति राघवः} %2-60-5

\twolineshloka
{लक्ष्मणः च अपि रामस्य पादौ परिचरन् वने}
{आराधयति धर्मज्ञः पर लोकम् जित इन्द्रियः} %2-60-6

\twolineshloka
{विजने अपि वने सीता वासम् प्राप्य गृहेष्व् इव}
{विस्रम्भम् लभते अभीता रामे सम्न्यस्त मानसा} %2-60-7

\twolineshloka
{न अस्या दैन्यम् कृतम् किम्चित् सुसूक्ष्मम् अपि लक्षये}
{उचिता इव प्रवासानाम् वैदेही प्रतिभाति मा} %2-60-8

\twolineshloka
{नगर उपवनम् गत्वा यथा स्म रमते पुरा}
{तथैव रमते सीता निर्जनेषु वनेष्व् अपि} %2-60-9

\twolineshloka
{बाला इव रमते सीता बाल चन्द्र निभ आनना}
{रामा रामे हि अदीन आत्मा विजने अपि वने सती} %2-60-10

\twolineshloka
{तत् गतम् हृदयम् हि अस्याः तत् अधीनम् च जीवितम्}
{अयोध्या अपि भवेत् तस्या राम हीना तथा वनम्} %2-60-11

\twolineshloka
{परि पृच्चति वैदेही ग्रामामः च नगराणि च}
{गतिम् दृष्ट्वा नदीनाम् च पादपान् विविधान् अपि} %2-60-12

\twolineshloka
{रामम् हि लक्ष्मनम् वापि पृष्ट्वा जानाति जानती}
{अयोध्याक्रोशमात्रे तु विहारमिव सम्श्रिता} %2-60-13

\twolineshloka
{इदमेव स्मराम्यस्याः सहसैवोपजल्पितम्}
{कैकेयीसम्श्रितम् वाक्यम् नेदानीम् प्रतिभाति माम्} %2-60-14

\twolineshloka
{ध्वम्सयित्वा तु तद्वाक्यम् प्रमादात्पर्युपस्थितम्}
{ह्लदनम् वचनम् सूतो देव्या मधुरमब्रवीत्} %2-60-15

\twolineshloka
{अध्वना वात वेगेन सम्भ्रमेण आतपेन च}
{न हि गच्चति वैदेह्याः चन्द्र अम्शु सदृशी प्रभा} %2-60-16

\twolineshloka
{सदृशम् शत पत्रस्य पूर्ण चन्द्र उपम प्रभम्}
{वदनम् तत् वदान्याया वैदेह्या न विकम्पते} %2-60-17

\twolineshloka
{अलक्त रस रक्त अभाव् अलक्त रस वर्जितौ}
{अद्य अपि चरणौ तस्याः पद्म कोश सम प्रभौ} %2-60-18

\twolineshloka
{नूपुर उद्घुष्ट हेला इव खेलम् गच्चति भामिनी}
{इदानीम् अपि वैदेही तत् रागा न्यस्त भूषणा} %2-60-19

\twolineshloka
{गजम् वा वीक्ष्य सिम्हम् वा व्याघ्रम् वा वनम् आश्रिता}
{न आहारयति सम्त्रासम् बाहू रामस्य सम्श्रिता} %2-60-20

\twolineshloka
{न शोच्याः ते न च आत्मा ते शोच्यो न अपि जन अधिपः}
{इदम् हि चरितम् लोके प्रतिष्ठास्यति शाश्वतम्} %2-60-21

\fourlineindentedshloka
{विधूय शोकम् परिहृष्ट मानसा}
{महर्षि याते पथि सुव्यवस्थिताः}
{वने रता वन्य फल अशनाः पितुः}
{शुभाम् प्रतिज्ञाम् परिपालयन्ति ते} %2-60-22

\fourlineindentedshloka
{तथा अपि सूतेन सुयुक्त वादिना}
{निवार्यमाणा सुत शोक कर्शिता}
{न चैव देवी विरराम कूजितात्}
{प्रिय इति पुत्र इति च राघव इति च} %2-60-23


॥इत्यार्षे श्रीमद्रामायणे वाल्मीकीये आदिकाव्ये अयोध्याकाण्डे कौसल्यासमाश्वसनम् नाम षष्ठितमः सर्गः ॥२-६०॥
