\sect{एकोनचत्वारिंशः सर्गः — वनगमनापृच्छा}

\twolineshloka
{रामस्य तु वचः श्रुत्वा मुनिवेषधरं च तम्}
{समीक्ष्य सह भार्याभी राजा विगतचेतनः} %2-39-1

\twolineshloka
{नैनं दुःखेन सन्तप्तः प्रत्यवैक्षत राघवम्}
{न चैनमभिसम्प्रेक्ष्य प्रत्यभाषत दुर्मनाः} %2-39-2

\twolineshloka
{स मुहूर्तमिवासंज्ञो दुःखितश्च महीपतिः}
{विललाप महाबाहू राममेवानुचिन्तयन्} %2-39-3

\twolineshloka
{मन्ये खलु मया पूर्वं विवत्सा बहवः कृताः}
{प्राणिनो हिंसिता वापि तन्मामिदमुपस्थितम्} %2-39-4

\twolineshloka
{न त्वेवानागते काले देहाच्च्यवति जीवितम्}
{कैकेय्या क्लिश्यमानस्य मृत्युर्मम न विद्यते} %2-39-5

\twolineshloka
{योऽहं पावकसङ्काशं पश्यामि पुरतः स्थितम्}
{विहाय वसने सूक्ष्मे तापसाच्छादमात्मजम्} %2-39-6

\twolineshloka
{एकस्याः खलु कैकेय्याः कृतेऽयं खिद्यते जनः}
{स्वार्थे प्रयतमानायाः संश्रित्य निकृतिं त्विमाम्} %2-39-7

\twolineshloka
{एवमुक्त्वा तु वचनं बाष्पेण विहतेन्द्रियः}
{रामेति सकृदेवोक्त्वा व्याहर्तुं न शशाक सः} %2-39-8

\twolineshloka
{संज्ञां तु प्रतिलभ्यैव मुहूर्तात् स महीपतिः}
{नेत्राभ्यामश्रुपूर्णाभ्यां सुमन्त्रमिदमब्रवीत्} %2-39-9

\twolineshloka
{औपवाह्यं रथं युक्त्वा त्वमायाहि हयोत्तमैः}
{प्रापयैनं महाभागमितो जनपदात् परम्} %2-39-10

\twolineshloka
{एवं मन्ये गुणवतां गुणानां फलमुच्यते}
{पित्रा मात्रा च यत्साधुर्वीरो निर्वास्यते वनम्} %2-39-11

\twolineshloka
{राज्ञो वचनमाज्ञाय सुमन्त्रः शीघ्रविक्रमः}
{योजयित्वा ययौ तत्र रथमश्वैरलङ्कृतम्} %2-39-12

\twolineshloka
{तं रथं राजपुत्राय सूतः कनकभूषितम्}
{आचचक्षेऽञ्जलिं कृत्वा युक्तं परमवाजिभिः} %2-39-13

\twolineshloka
{राजा सत्वरमाहूय व्यापृतं वित्तसञ्चये}
{उवाच देशकालज्ञो निश्चितं सर्वतः शुचिः} %2-39-14

\twolineshloka
{वासांसि च वरार्हाणि भूषणानि महान्ति च}
{वर्षाण्येतानि सङ्ख्याय वैदेह्याः क्षिप्रमानय} %2-39-15

\twolineshloka
{नरेन्द्रेणैवमुक्तस्तु गत्वा कोशगृहं ततः}
{प्रायच्छत् सर्वमाहृत्य सीतायै क्षिप्रमेव तत्} %2-39-16

\twolineshloka
{सा सुजाता सुजातानि वैदेही प्रस्थिता वनम्}
{भूषयामास गात्राणि तैर्विचित्रैर्विभूषणैः} %2-39-17

\twolineshloka
{व्यराजयत वैदेही वेश्म तत् सुविभूषिता}
{उद्यतोंऽशुमतः काले खं प्रभेव विवस्वतः} %2-39-18

\twolineshloka
{तां भुजाभ्यां परिष्वज्य श्वश्रूर्वचनमब्रवीत्}
{अनाचरन्तीं कृपणं मूर्ध्न्युपाघ्राय मैथिलीम्} %2-39-19

\twolineshloka
{असत्यः सर्वलोकेऽस्मिन् सततं सत्कृताः प्रियैः}
{भर्तारं नानुमन्यन्ते विनिपातगतं स्त्रियः} %2-39-20

\twolineshloka
{एष स्वभावो नारीणामनुभूय पुरा सुखम्}
{अल्पामप्यापदं प्राप्य दुष्यन्ति प्रजहत्यपि} %2-39-21

\twolineshloka
{असत्यशीला विकृता दुर्गा अहृदयाः सदा}
{असत्यः पापसङ्कल्पाः क्षणमात्रविरागिणः} %2-39-22

\twolineshloka
{न कुलं न कृतं विद्या न दत्तं नापि सङ्ग्रहः}
{स्त्रीणां गृह्णाति हृदयमनित्यहृदया हि ताः} %2-39-23

\twolineshloka
{साध्वीनां तु स्थितानां तु शीले सत्ये श्रुते स्थिते}
{स्त्रीणां पवित्रं परमं पतिरेको विशिष्यते} %2-39-24

\twolineshloka
{स त्वया नावमन्तव्यः पुत्रः प्रव्राजितो वनम्}
{तव देवसमस्त्वेष निर्धनः सधनोऽपि वा} %2-39-25

\twolineshloka
{विज्ञाय वचनं सीता तस्या धर्मार्थसंहितम्}
{कृत्वाञ्जलिमुवाचेदं श्वश्रूमभिमुखे स्थिता} %2-39-26

\twolineshloka
{करिष्ये सर्वमेवाहमार्या यदनुशास्ति माम्}
{अभिज्ञास्मि यथा भर्तुर्वर्तितव्यं श्रुतं च मे} %2-39-27

\twolineshloka
{न मामसज्जनेनार्या समानयितुमर्हति}
{धर्माद् विचलितुं नाहमलं चन्द्रादिव प्रभा} %2-39-28

\twolineshloka
{नातन्त्री वाद्यते वीणा नाचक्रो विद्यते रथः}
{नापतिः सुखमेधेत या स्यादपि शतात्मजा} %2-39-29

\twolineshloka
{मितं ददाति हि पिता मितं भ्राता मितं सुतः}
{अमितस्य तु दातारं भर्तारं का न पूजयेत्} %2-39-30

\twolineshloka
{साहमेवङ्गता श्रेष्ठा श्रुतधर्मपरावरा}
{आर्ये किमवमन्येयं स्त्रिया भर्ता हि दैवतम्} %2-39-31

\twolineshloka
{सीताया वचनं श्रुत्वा कौसल्या हृदयङ्गमम्}
{शुद्धसत्त्वा मुमोचाश्रु सहसा दुःखहर्षजम्} %2-39-32

\twolineshloka
{तां प्राञ्जलिरभिप्रेक्ष्य मातृमध्येऽतिसत्कृताम्}
{रामः परमधर्मात्मा मातरं वाक्यमब्रवीत्} %2-39-33

\twolineshloka
{अम्ब मा दुःखिता भूत्वा पश्येस्त्वं पितरं मम}
{क्षयोऽपि वनवासस्य क्षिप्रमेव भविष्यति} %2-39-34

\twolineshloka
{सुप्तायास्ते गमिष्यन्ति नव वर्षाणि पञ्च च}
{समग्रमिह सम्प्राप्तं मां द्रक्ष्यसि सुहृद्वृतम्} %2-39-35

\twolineshloka
{एतावदभिनीतार्थमुक्त्वा स जननीं वचः}
{त्रयः शतशतार्धा हि ददर्शावेक्ष्य मातरः} %2-39-36

\twolineshloka
{ताश्चापि स तथैवार्ता मातॄर्दशरथात्मजः}
{धर्मयुक्तमिदं वाक्यं निजगाद कृताञ्जलिः} %2-39-37

\twolineshloka
{संवासात् परुषं किञ्चिदज्ञानादपि यत् कृतम्}
{तन्मे समुपजानीत सर्वाश्चामन्त्रयामि वः} %2-39-38

\twolineshloka
{वचनं राघवस्यैतद् धर्मयुक्तं समाहितम्}
{शुश्रुवुस्ताः स्त्रियः सर्वाः शोकोपहतचेतसः} %2-39-39

\twolineshloka
{जज्ञोऽथ तासां सन्नादः क्रौञ्चीनामिव निःस्वनः}
{मानवेन्द्रस्य भार्याणामेवं वदति राघवे} %2-39-40

\twolineshloka
{मुरजपणवमेघघोषवद् दशरथवेश्म बभूव यत् पुरा}
{विलपितपरिदेवनाकुलं व्यसनगतं तदभूत् सुदुःखितम्} %2-39-41


॥इत्यार्षे श्रीमद्रामायणे वाल्मीकीये आदिकाव्ये अयोध्याकाण्डे वनगमनापृच्छा नाम एकोनचत्वारिंशः सर्गः ॥२-३९॥
