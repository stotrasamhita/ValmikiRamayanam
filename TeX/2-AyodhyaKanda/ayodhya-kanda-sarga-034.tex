\sect{चतुस्त्रिंशः सर्गः — दशरथसमाश्वासनम्}

\twolineshloka
{ततः कमलपत्राक्षः श्यामो निरुपमो महान्}
{उवाच रामस्तं सूतं पितुराख्याहि मामिति} %2-34-1

\twolineshloka
{स रामप्रेषितः क्षिप्रं संतापकलुषेन्द्रियम्}
{प्रविश्य नृपतिं सूतो निःश्वसन्तं ददर्श ह} %2-34-2

\twolineshloka
{उपरक्तमिवादित्यं भस्मच्छन्नमिवानलम्}
{तटाकमिव निस्तोयमपश्यज्जगतीपतिम्} %2-34-3

\twolineshloka
{आबोध्य च महाप्राज्ञः परमाकुलचेतनम्}
{राममेवानुशोचन्तं सूतः प्राञ्जलिरब्रवीत्} %2-34-4

\twolineshloka
{तं वर्धयित्वा राजानं पूर्वं सूतो जयाशिषा}
{भयविक्लवया वाचा मन्दया श्लक्ष्णयाब्रवीत्} %2-34-5

\twolineshloka
{अयं स पुरुषव्याघ्रो द्वारि तिष्ठति ते सुतः}
{ब्राह्मणेभ्यो धनं दत्त्वा सर्वं चैवोपजीविनाम्} %2-34-6

\twolineshloka
{स त्वां पश्यतु भद्रं ते रामः सत्यपराक्रमः}
{सर्वान् सुहृद आपृच्छ्य त्वां हीदानीं दिदृक्षते} %2-34-7

\twolineshloka
{गमिष्यति महारण्यं तं पश्य जगतीपते}
{वृतं राजगुणैः सर्वैरादित्यमिव रश्मिभिः} %2-34-8

\twolineshloka
{स सत्यवाक्यो धर्मात्मा गाम्भीर्यात् सागरोपमः}
{आकाश इव निष्पङ्को नरेन्द्रः प्रत्युवाच तम्} %2-34-9

\twolineshloka
{सुमन्त्रानय मे दारान् ये केचिदिह मामकाः}
{दारैः परिवृतः सर्वैर्द्रष्टुमिच्छामि राघवम्} %2-34-10

\twolineshloka
{सोऽन्तःपुरमतीत्यैव स्त्रियस्ता वाक्यमब्रवीत्}
{आर्यो ह्वयति वो राजा गम्यतां तत्र मा चिरम्} %2-34-11

\twolineshloka
{एवमुक्ताः स्त्रियः सर्वाः सुमन्त्रेण नृपाज्ञया}
{प्रचक्रमुस्तद् भवनं भर्तुराज्ञाय शासनम्} %2-34-12

\twolineshloka
{अर्धसप्तशतास्तत्र प्रमदास्ताम्रलोचनाः}
{कौसल्यां परिवार्याथ शनैर्जग्मुर्धृतव्रताः} %2-34-13

\twolineshloka
{आगतेषु च दारेषु समवेक्ष्य महीपतिः}
{उवाच राजा तं सूतं सुमन्त्रानय मे सुतम्} %2-34-14

\twolineshloka
{स सूतो राममादाय लक्ष्मणं मैथिलीं तथा}
{जगामाभिमुखस्तूर्णं सकाशं जगतीपतेः} %2-34-15

\twolineshloka
{स राजा पुत्रमायान्तं दृष्ट्वा चारात् कृताञ्जलिम्}
{उत्पपातासनात् तूर्णमार्तः स्त्रीजनसंवृतः} %2-34-16

\twolineshloka
{सोऽभिदुद्राव वेगेन रामं दृष्ट्वा विशाम्पतिः}
{तमसम्प्राप्य दुःखार्तः पपात भुवि मूर्च्छितः} %2-34-17

\twolineshloka
{तं रामोऽभ्यपतत् क्षिप्रं लक्ष्मणश्च महारथः}
{विसंज्ञमिव दुःखेन सशोकं नृपतिं तथा} %2-34-18

\twolineshloka
{स्त्रीसहस्रनिनादश्च संजज्ञे राजवेश्मनि}
{हा हा रामेति सहसा भूषणध्वनिमिश्रितः} %2-34-19

\twolineshloka
{तं परिष्वज्य बाहुभ्यां तावुभौ रामलक्ष्मणौ}
{पर्यङ्के सीतया सार्धं रुदन्तः समवेशयन्} %2-34-20

\twolineshloka
{अथ रामो मुहूर्तस्य लब्धसंज्ञं महीपतिम्}
{उवाच प्राञ्जलिर्बाष्पशोकार्णवपरिप्लुतम्} %2-34-21

\twolineshloka
{आपृच्छे त्वां महाराज सर्वेषामीश्वरोऽसि नः}
{प्रस्थितं दण्डकारण्यं पश्य त्वं कुशलेन माम्} %2-34-22

\twolineshloka
{लक्ष्मणं चानुजानीहि सीता चान्वेतु मां वनम्}
{कारणैर्बहुभिस्तथ्यैर्वार्यमाणौ न चेच्छतः} %2-34-23

\twolineshloka
{अनुजानीहि सर्वान् नः शोकमुत्सृज्य मानद}
{लक्ष्मणं मां च सीतां च प्रजापतिरिवात्मजान्} %2-34-24

\twolineshloka
{प्रतीक्षमाणमव्यग्रमनुज्ञां जगतीपतेः}
{उवाच राजा सम्प्रेक्ष्य वनवासाय राघवम्} %2-34-25

\twolineshloka
{अहं राघव कैकेय्या वरदानेन मोहितः}
{अयोध्यायां त्वमेवाद्य भव राजा निगृह्य माम्} %2-34-26

\twolineshloka
{एवमुक्तो नृपतिना रामो धर्मभृतां वरः}
{प्रत्युवाचाञ्जलिं कृत्वा पितरं वाक्यकोविदः} %2-34-27

\twolineshloka
{भवान् वर्षसहस्राय पृथिव्या नृपते पतिः}
{अहं त्वरण्ये वत्स्यामि न मे राज्यस्य कांक्षिता} %2-34-28

\twolineshloka
{नव पञ्च च वर्षाणि वनवासे विहृत्य ते}
{पुनः पादौ ग्रहीष्यामि प्रतिज्ञान्ते नराधिप} %2-34-29

\twolineshloka
{रुदन्नार्तः प्रियं पुत्रं सत्यपाशेन संयुतः}
{कैकेय्या चोद्यमानस्तु मिथो राजा तमब्रवीत्} %2-34-30

\twolineshloka
{श्रेयसे वृद्धये तात पुनरागमनाय च}
{गच्छस्वारिष्टमव्यग्रः पन्थानमकुतोभयम्} %2-34-31

\twolineshloka
{न हि सत्यात्मनस्तात धर्माभिमनसस्तव}
{संनिवर्तयितुं बुद्धिः शक्यते रघुनन्दन} %2-34-32

\twolineshloka
{अद्य त्विदानीं रजनीं पुत्र मा गच्छ सर्वथा}
{एकाहं दर्शनेनापि साधु तावच्चराम्यहम्} %2-34-33

\twolineshloka
{मातरं मां च सम्पश्यन् वसेमामद्य शर्वरीम्}
{तर्पितः सर्वकामैस्त्वं श्वः काल्ये साधयिष्यसि} %2-34-34

\twolineshloka
{दुष्करं क्रियते पुत्र सर्वथा राघव प्रिय}
{त्वया हि मत्प्रियार्थं तु वनमेवमुपाश्रितम्} %2-34-35

\twolineshloka
{न चैतन्मे प्रियं पुत्र शपे सत्येन राघव}
{छन्नया चलितस्त्वस्मि स्त्रिया भस्माग्निकल्पया} %2-34-36

\twolineshloka
{वञ्चना या तु लब्धा मे तां त्वं निस्तर्तुमिच्छसि}
{अनया वृत्तसादिन्या कैकेय्याभिप्रचोदितः} %2-34-37

\twolineshloka
{न चैतदाश्चर्यतमं यत् त्वं ज्येष्ठः सुतो मम}
{अपानृतकथं पुत्र पितरं कर्तुमिच्छसि} %2-34-38

\twolineshloka
{अथ रामस्तदा श्रुत्वा पितुरार्तस्य भाषितम्}
{लक्ष्मणेन सह भ्रात्रा दीनो वचनमब्रवीत्} %2-34-39

\twolineshloka
{प्राप्स्यामि यानद्य गुणान् को मे श्वस्तान् प्रदास्यति}
{अपक्रमणमेवातः सर्वकामैरहं वृणे} %2-34-40

\twolineshloka
{इयं सराष्ट्रा सजना धनधान्यसमाकुला}
{मया विसृष्टा वसुधा भरताय प्रदीयताम्} %2-34-41

\twolineshloka
{वनवासकृता बुद्धिर्न च मेऽद्य चलिष्यति}
{यस्तु युद्धे वरो दत्तः कैकेय्यै वरद त्वया} %2-34-42

\twolineshloka
{दीयतां निखिलेनैव सत्यस्त्वं भव पार्थिव}
{अहं निदेशं भवतो यथोक्तमनुपालयन्} %2-34-43

\twolineshloka
{चतुर्दश समा वत्स्ये वने वनचरैः सह}
{मा विमर्शो वसुमती भरताय प्रदीयताम्} %2-34-44

\twolineshloka
{नहि मे कांक्षितं राज्यं सुखमात्मनि वा प्रियम्}
{यथानिदेशं कर्तुं वै तवैव रघुनन्दन} %2-34-45

\twolineshloka
{अपगच्छतु ते दुःखं मा भूर्बाष्पपरिप्लुतः}
{नहि क्षुभ्यति दुर्धर्षः समुद्रः सरितां पतिः} %2-34-46

\twolineshloka
{नैवाहं राज्यमिच्छामि न सुखं न च मेदिनीम्}
{नैव सर्वानिमान् कामान् न स्वर्गं न च जीवितुम्} %2-34-47

\twolineshloka
{त्वामहं सत्यमिच्छामि नानृतं पुरुषर्षभ}
{प्रत्यक्षं तव सत्येन सुकृतेन च ते शपे} %2-34-48

\twolineshloka
{न च शक्यं मया तात स्थातुं क्षणमपि प्रभो}
{स शोकं धारयस्वेमं नहि मेऽस्ति विपर्ययः} %2-34-49

\twolineshloka
{अर्थितो ह्यस्मि कैकेय्या वनं गच्छेति राघव}
{मया चोक्तं व्रजामीति तत्सत्यमनुपालये} %2-34-50

\twolineshloka
{मा चोत्कण्ठां कृथा देव वने रंस्यामहे वयम्}
{प्रशान्तहरिणाकीर्णे नानाशकुनिनादिते} %2-34-51

\twolineshloka
{पिता हि दैवतं तात देवतानामपि स्मृतम्}
{तस्माद् दैवतमित्येव करिष्यामि पितुर्वचः} %2-34-52

\twolineshloka
{चतुर्दशसु वर्षेषु गतेषु नृपसत्तम}
{पुनर्द्रक्ष्यसि मां प्राप्तं संतापोऽयं विमुच्यताम्} %2-34-53

\twolineshloka
{येन संस्तम्भनीयोऽयं सर्वो बाष्पकलो जनः}
{स त्वं पुरुषशार्दूल किमर्थं विक्रियां गतः} %2-34-54

\twolineshloka
{पुरं च राष्ट्रं च मही च केवला मया विसृष्टा भरताय दीयताम्}
{अहं निदेशं भवतोऽनुपालयन् वनं गमिष्यामि चिराय सेवितुम्} %2-34-55

\twolineshloka
{मया विसृष्टां भरतो महीमिमां सशैलखण्डां सपुरोपकाननाम्}
{शिवासु सीमास्वनुशास्तु केवलं त्वया यदुक्तं नृपते तथास्तु तत्} %2-34-56

\twolineshloka
{न मे तथा पार्थिव धीयते मनो महत्सु कामेषु न चात्मनः प्रिये}
{यथा निदेशे तव शिष्टसम्मते व्यपैतु दुःखं तव मत्कृतेऽनघ} %2-34-57

\twolineshloka
{तदद्य नैवानघ राज्यमव्ययं न सर्वकामान् वसुधां न मैथिलीम्}
{न चिन्तितं त्वामनृतेन योजयन् वृणीय सत्यं व्रतमस्तु ते तथा} %2-34-58

\twolineshloka
{फलानि मूलानि च भक्षयन् वने गिरींश्च पश्यन् सरितः सरांसि च}
{वनं प्रविश्यैव विचित्रपादपं सुखी भविष्यामि तवास्तु निर्वृतिः} %2-34-59

\twolineshloka
{एवं स राजा व्यसनाभिपन्नस्तापेन दुःखेन च पीड्यमानः}
{आलिङ्ग्य पुत्रं सुविनष्टसंज्ञो भूमिं गतो नैव चिचेष्ट किंचित्} %2-34-60

\twolineshloka
{देव्यः समस्ता रुरुदुः समेतास्तां वर्जयित्वा नरदेवपत्नीम्}
{रुदन् सुमन्त्रोऽपि जगाम मूर्च्छां हाहाकृतं तत्र बभूव सर्वम्} %2-34-61


॥इत्यार्षे श्रीमद्रामायणे वाल्मीकीये आदिकाव्ये अयोध्याकाण्डे दशरथसमाश्वासनम् नाम चतुस्त्रिंशः सर्गः ॥२-३४॥
