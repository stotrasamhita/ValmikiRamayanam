\sect{एकोनशततमः सर्गः — रामसमागमः}

\twolineshloka
{निविष्टायां तु सेनायामुत्सुको भरतस्तदा}
{जगाम भ्रातरं द्रष्टुं शत्रुध्नमनुदर्शयन्} %2-99-1

\twolineshloka
{ऋषिं वसिष्ठं सन्दिश्य मातऽर्मे शीघ्रमानय}
{इति त्वरितमग्रे स जगाम गुरुवत्सलः} %2-99-2

\twolineshloka
{सुमन्त्रस्त्वपि शत्रुघ्नमदूरादन्वपद्यत}
{रामदर्शनजस्तर्षो भरतस्येव तस्य च} %2-99-3

\twolineshloka
{गच्छन्नेवाथ भरतस्तापसालयसंस्थिताम्}
{भ्रातुः पर्णकुटीं श्रीमानुटजं च ददर्श ह} %2-99-4

\twolineshloka
{शालायास्त्वग्रतस्तस्या ददर्श भरतस्तदा}
{काष्ठानि चावभग्नानि पुष्पाण्युपचितानि च} %2-99-5

\twolineshloka
{सलक्ष्मणस्य रामस्य ददर्शाश्रममीयुषः}
{कृतं वृक्षेष्वभिज्ञानं कुशचीरैः क्वचित् क्वचित्} %2-99-6

\twolineshloka
{ददर्श च वने तस्मिन् महतः सञ्चयान् कृतान्}
{मृगाणां महिषाणां च करीषैः शीतकारणात्} %2-99-7

\twolineshloka
{गच्छन्नेव महाबाहुर्द्युतिमान् भरतस्तदा}
{शत्रुघ्नं चाब्रवीद्धृष्टस्तानमात्यांश्च सर्वशः} %2-99-8

\twolineshloka
{मन्ये प्राप्ताः स्म तं देशं भरद्वाजो यमब्रवीत्}
{नातिदूरे हि मन्येऽहं नदीं मन्दाकिनीमितः} %2-99-9

\twolineshloka
{उच्चैर्बद्धानि चीराणि लक्ष्मणेन भवेदयम्}
{अभिज्ञानकृतः पन्था विकाले गन्तुमिच्छता} %2-99-10

\twolineshloka
{इदं चोदात्तदन्तानां कुञ्जराणां तरस्विनाम्}
{शैलपार्श्वे परिक्रान्तमन्योन्यमभिगर्जताम्} %2-99-11

\twolineshloka
{यमेवाधातुमिच्छन्ति तापसाः सततं वने}
{तस्यासौ दृश्यते धूमः सङ्कुलः कृष्णवर्त्मनः} %2-99-12

\twolineshloka
{अत्राहं पुरुषव्याघ्रं गुरुसंस्कारकारिणम्}
{आर्यं द्रक्ष्यामि संहृष्टो महर्षिमिव राघवम्} %2-99-13

\twolineshloka
{अथ गत्वा मुहूर्तं तु चित्रकूटं स राघवः}
{मन्दाकिनीमनुप्राप्तस्तं जनं चेदमब्रवीत्} %2-99-14

\twolineshloka
{जगत्यां पुरुषव्याघ्र आस्ते वीरासने रतः}
{जनेन्द्रो निर्जनं प्राप्य धिङ्मे जन्म सजीवितम्} %2-99-15

\twolineshloka
{मत्कृते व्यसनं प्राप्तो लोकनाथो महाद्युतिः}
{सर्वान् कामान् परित्यज्य वने वसति राघवः} %2-99-16

\twolineshloka
{इति लोकसमाक्रुष्टः पादेष्वद्य प्रसादयन्}
{रामस्य निपतिष्यामि सीताया लक्ष्मणस्य च} %2-99-17

\twolineshloka
{एवं स विलपंस्तस्मिन् वने दशरथात्मजः}
{ददर्श महतीं पुण्यां पर्णशालां मनोरमाम्} %2-99-18

\twolineshloka
{सालतालाश्वकर्णानां पर्णैर्बहुभिरावृताम्}
{विशालां मृदुभिस्तीर्णां कुशैर्वेदिमिवाध्वरे} %2-99-19

\twolineshloka
{शक्रायुधनिकाशैश्च कार्मुकैर्भारसाधनैः}
{रुक्मपृष्टष्ठैर्महासारैः शोभितां शत्रुबाधकैः} %2-99-20

\twolineshloka
{अर्करश्मिप्रतीकाशैर्घोरैस्तूणीगतैः शरैः}
{शोभितां दीप्तवदनैः सर्प्पैर्भोगवतीमिव} %2-99-21

\twolineshloka
{महारजतवासोभ्यामसिभ्यां च विराजिताम्}
{रुक्मबिन्दुविचित्राभ्यां चर्मभ्यां चापि शोभिताम्} %2-99-22

\twolineshloka
{गोधाङ्गुलित्रैरासक्तैश्चित्रैः काञ्चनभूषितैः}
{अरिसङ्घैरनाधृष्यां मृगैः सिंहगुहामिव} %2-99-23

\twolineshloka
{प्रागुदक्प्रवणां वेदिं विशालां दीप्तपावकाम्}
{ददर्श भरतस्तत्र पुण्यां रामनिवेशने} %2-99-24

\twolineshloka
{निरीक्ष्य स मुहूर्त्तं तु ददर्श भरतो गुरम्}
{उटजे राममासीनं जटामण्डलधारिणम्} %2-99-25

\twolineshloka
{तं तु कृष्णाजिनधरं चीरवल्कलवाससम्}
{ददर्श राममासीनमभितः पावकोपमम्} %2-99-26

\twolineshloka
{सिंहस्कन्धं महाबाहुं पुण्डरीकनिभेक्षणम्}
{पृथिव्याः सागरान्ताया भर्त्तारं धर्मचारिणम्} %2-99-27

\twolineshloka
{उपविष्टं महाबाहुं ब्रह्माणमिव शाश्वतम्}
{स्थण्डिले दर्भसंस्तीर्णे सीतया लक्ष्मणेन च} %2-99-28

\twolineshloka
{तं दृष्ट्वा भरतः श्रीमान् दुःखशोकपरिप्लुतः}
{अभ्यधावत धर्मात्मा भरतः कैकयीसुतः} %2-99-29

\twolineshloka
{दृष्ट्वैव विललापार्त्तो बाष्पसन्दिग्धया गिरा}
{अशक्नुवन् धारयितुं धैर्याद्वचनमब्रवीत्} %2-99-30

\twolineshloka
{यः संसदि प्रकृतिभिर्भवेद्युक्त उपासितुम्}
{वन्यैर्मृगैरुपासीनः सोऽयमास्ते ममाग्रजः} %2-99-31

\twolineshloka
{वासोभिर्बहुसाहस्रैर्यो महात्मा पुरोचितः}
{मृगाजिने सोऽयमिह प्रवस्ते धर्ममाचरन्} %2-99-32

\onelineshloka
{अधारयद्यो विविधाश्चित्राः सुमनसस्तदा सोऽयं जटाभारमिमं वहते राघवः कथम्} %2-99-33

\twolineshloka
{यस्य यज्ञैर्यथोद्दिष्टैर्युक्तो धर्मस्य सञ्चयः}
{शरीरक्लेशसम्भूतं स धर्मं परिमार्गते} %2-99-34

\twolineshloka
{चन्दनेन महार्हेण यस्याङ्गमुपसेवितम्}
{मलेन तस्याङ्गमिदं कथमार्यस्य सेव्यते} %2-99-35

\twolineshloka
{मन्निमित्तमिदं दुःखं प्राप्तो रामः सुखोचितः}
{धिग्जीवितं नृशंसस्य मम लोकविगर्हितम्} %2-99-36

\twolineshloka
{इत्येवं विलपन् दीनः प्रस्विन्नमुखपङ्कजः}
{पादावप्राप्य रामस्य पपात भरतो रुदन्} %2-99-37

\twolineshloka
{दुःखाभितप्तो भरतो राजपुत्रो महाबलः}
{उक्त्वार्येति सकृद्दीनं पुनर्नोवाच किञ्चन} %2-99-38

\twolineshloka
{बाष्पापिहितकण्ठश्च प्रेक्ष्य रामं यशस्विनम्}
{आर्येत्येवाथ सङ्क्रुश्य व्याहर्त्तुं नाशकत्तदा} %2-99-39

\twolineshloka
{शत्रुघ्नश्चापि रामस्य ववन्दे चरणौ रुदन्}
{तावुभौ स समालिङ्ग्य रामश्चाश्रूण्यवर्त्तयत्} %2-99-40

\twolineshloka
{ततः सुमन्त्रेण गुहेन चैव समीयतू राजसुतावरण्ये}
{दिवाकरश्चैव निशाकरश्च यथाम्बरे शुक्रबृहस्पतिभ्याम्} %2-99-41

\twolineshloka
{तान् पार्थिवान् वारणयूथपाभान् समागतांस्तत्र महत्यरण्ये}
{वनौकसस्तेऽपि समीक्ष्य सर्वेप्यश्रूण्यमुञ्चन् प्रविहाय हर्षम्} %2-99-42


॥इत्यार्षे श्रीमद्रामायणे वाल्मीकीये आदिकाव्ये अयोध्याकाण्डे रामसमागमः नाम एकोनशततमः सर्गः ॥२-९९॥
