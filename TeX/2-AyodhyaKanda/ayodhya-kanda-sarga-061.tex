\sect{एकषष्ठितमः सर्गः — कौसल्योपालम्भः}

\twolineshloka
{वनम् गते धर्म परे रामे रमयताम् वरे}
{कौसल्या रुदती स्वार्ता भर्तारम् इदम् अब्रवीत्} %2-61-1

\twolineshloka
{यद्यपि त्रिषु लोकेषु प्रथितम् ते मयद् यशः}
{सानुक्रोशो वदान्यः च प्रिय वादी च राघवः} %2-61-2

\twolineshloka
{कथम् नर वर श्रेष्ठ पुत्रौ तौ सह सीतया}
{दुह्खितौ सुख सम्वृद्धौ वने दुह्खम् सहिष्यतः} %2-61-3

\twolineshloka
{सा नूनम् तरुणी श्यामा सुकुमारी सुख उचिता}
{कथम् उष्णम् च शीतम् च मैथिली प्रसहिष्यते} %2-61-4

\twolineshloka
{भुक्त्वा अशनम् विशाल अक्षी सूप दम्श अन्वितम् शुभम्}
{वन्यम् नैवारम् आहारम् कथम् सीता उपभोक्ष्यते} %2-61-5

\twolineshloka
{गीत वादित्र निर्घोषम् श्रुत्वा शुभम् अनिन्दिता}
{कथम् क्रव्य अद सिम्हानाम् शब्दम् श्रोष्यति अशोभनम्} %2-61-6

\twolineshloka
{महा इन्द्र ध्वज सम्काशः क्व नु शेते महा भुजः}
{भुजम् परिघ सम्काशम् उपधाय महा बलः} %2-61-7

\twolineshloka
{पद्म वर्णम् सुकेश अन्तम् पद्म निह्श्वासम् उत्तमम्}
{कदा द्रक्ष्यामि रामस्य वदनम् पुष्कर ईक्षणम्} %2-61-8

\twolineshloka
{वज्र सारमयम् नूनम् हृदयम् मे न सम्शयः}
{अपश्यन्त्या न तम् यद् वै फलति इदम् सहस्रधा} %2-61-9

\twolineshloka
{यत्त्वया करुणम् कर्म व्यपोह्य मम बान्धवाः}
{निरस्ता परिधावन्ति सुखार्हः कृपणा वने} %2-61-10

\twolineshloka
{यदि पञ्चदशे वर्षे राघवः पुनरेष्यति}
{जह्याद्राज्यम् च कोशम् च भरतो नोपल्स्ख्यते} %2-61-11

\twolineshloka
{भोजयन्ति किल श्राद्धे केचित्स्वनेव बान्धवान्}
{ततः पश्चात्समीक्षन्ते कृतकार्या द्विजर्षभान्} %2-61-12

\twolineshloka
{तत्र ये गुणवन्तश्च विद्वाम्सश्च द्विजातयः}
{न पश्चात्तेऽभिमन्यन्ते सुधामपि सुरोपमाः} %2-61-13

\twolineshloka
{ब्राह्मणेष्वपि तृप्तेषु पश्चाद्भोक्तुम् द्विजर्षभाः}
{नाभ्युपैतुमलम् प्राज्ञाः शृङ्गच्चेदमिवर्ष्भाः} %2-61-14

\twolineshloka
{एवम् कनीयसा भ्रात्रा भुक्तम् राज्यम् विशाम् पते}
{भ्राता ज्येष्ठा वरिष्ठाः च किम् अर्थम् न अवमम्स्यते} %2-61-15

\twolineshloka
{न परेण आहृतम् भक्ष्यम् व्याघ्रः खादितुम् इच्चति}
{एवम् एव नर व्याघ्रः पर लीढम् न मम्स्यते} %2-61-16

\twolineshloka
{हविर् आज्यम् पुरोडाशाः कुशा यूपाः च खादिराः}
{न एतानि यात यामानि कुर्वन्ति पुनर् अध्वरे} %2-61-17

\twolineshloka
{तथा हि आत्तम् इदम् राज्यम् हृत साराम् सुराम् इव}
{न अभिमन्तुम् अलम् रामः नष्ट सोमम् इव अध्वरम्} %2-61-18

\twolineshloka
{न एवम् विधम् असत्कारम् राघवो मर्षयिष्यति}
{बलवान् इव शार्दूलो बालधेर् अभिमर्शनम्} %2-61-19

\twolineshloka
{नैतस्य सहिता लोका भयम् कुर्युर्महामृधे}
{अधर्मम् त्विह धर्मात्मा लोकम् धर्मेण योजयेत्} %2-61-20

\twolineshloka
{नन्वसौ काञ्चनैर्बाणैर्महावीर्यो महाभुजः}
{युगान्त इव भूतानि सागरानपि निर्दहेत्} %2-61-21

\twolineshloka
{स तादृशः सिम्ह बलो वृषभ अक्षो नर ऋषभः}
{स्वयम् एव हतः पित्रा जलजेन आत्मजो यथा} %2-61-22

\twolineshloka
{द्विजाति चरितः धर्मः शास्त्र दृष्टः सनातनः}
{यदि ते धर्म निरते त्वया पुत्रे विवासिते} %2-61-23

\twolineshloka
{गतिर् एवाक् पतिर् नार्या द्वितीया गतिर् आत्मजः}
{तृतीया ज्ञातयो राजमः चतुर्थी न इह विद्यते} %2-61-24

\twolineshloka
{तत्र त्वम् चैव मे न अस्ति रामः च वनम् आश्रितः}
{न वनम् गन्तुम् इच्चामि सर्वथा हि हता त्वया} %2-61-25

\fourlineindentedshloka
{हतम् त्वया राज्यम् इदम् सराष्ट्रम्}
{हतः तथा आत्मा सह मन्त्रिभिः च}
{हता सपुत्रा अस्मि हताः च पौराः}
{सुतः च भार्या च तव प्रहृष्टौ} %2-61-26

\fourlineindentedshloka
{इमाम् गिरम् दारुण शब्द सम्श्रिताम्}
{निशम्य राजा अपि मुमोह दुह्खितः}
{ततः स शोकम् प्रविवेश पार्थिवः}
{स्वदुष्कृतम् च अपि पुनः तदा अस्मरत्} %2-61-27


॥इत्यार्षे श्रीमद्रामायणे वाल्मीकीये आदिकाव्ये अयोध्याकाण्डे कौसल्योपालम्भः नाम एकषष्ठितमः सर्गः ॥२-६१॥
