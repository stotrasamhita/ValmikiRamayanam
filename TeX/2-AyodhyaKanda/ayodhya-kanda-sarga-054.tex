\sect{चतुःपञ्चाशः सर्गः — भरद्वाजाश्रमाभिगमनम्}

\twolineshloka
{ते तु तस्मिन् महा वृक्षौषित्वा रजनीम् शिवाम्}
{विमले अभ्युदिते सूर्ये तस्मात् देशात् प्रतस्थिरे} %2-54-1

\twolineshloka
{यत्र भागीरथी गन्गा यमुनाम् अभिवर्तते}
{जग्मुस् तम् देशम् उद्दिश्य विगाह्य सुमहद् वनम्} %2-54-2

\twolineshloka
{ते भूमिम् आगान् विविधान् देशामः च अपि मनो रमान्}
{अदृष्ट पूर्वान् पश्यन्तः तत्र तत्र यशस्विनः} %2-54-3

\twolineshloka
{यथा क्षेमेण गच्चन् स पश्यमः च विविधान् द्रुमान्}
{निवृत्त मात्रे दिवसे रामः सौमित्रिम् अब्रवीत्} %2-54-4

\twolineshloka
{प्रयागम् अभितः पश्य सौमित्रे धूमम् उन्नतम्}
{अग्नेर् भगवतः केतुम् मन्ये सम्निहितः मुनिः} %2-54-5

\twolineshloka
{नूनम् प्राप्ताः स्म सम्भेदम् गन्गा यमुनयोः वयम्}
{तथा हि श्रूयते शम्ब्दो वारिणा वारि घट्टितः} %2-54-6

\twolineshloka
{दारूणि परिभिन्नानि वनजैः उपजीविभिः}
{भरद्वाज आश्रमे च एते दृश्यन्ते विविधा द्रुमाः} %2-54-7

\twolineshloka
{धन्विनौ तौ सुखम् गत्वा लम्बमाने दिवा करे}
{गन्गा यमुनयोह् सम्धौ प्रापतुर् निलयम् मुनेः} %2-54-8

\twolineshloka
{रामः तु आश्रमम् आसाद्य त्रासयन् मृग पक्षिणः}
{गत्वा मुहूर्तम् अध्वानम् भरद्वाजम् उपागमत्} %2-54-9

\twolineshloka
{ततः तु आश्रमम् आसाद्य मुनेर् दर्शन कान्क्षिणौ}
{सीतया अनुगतौ वीरौ दूरात् एव अवतस्थतुः} %2-54-10

\twolineshloka
{स प्रविश्य महात्मानमृषिम् शिष्यगणैर्वऋतम्}
{सम्शितव्रतमेकाग्रम् तपसा लब्धचक्षुषम्} %2-54-11

\twolineshloka
{हुत अग्नि होत्रम् दृष्ट्वा एव महा भागम् कृत अन्जलिः}
{रामः सौमित्रिणा सार्धम् सीतया च अभ्यवादयत्} %2-54-12

\twolineshloka
{न्यवेदयत च आत्मानम् तस्मै लक्ष्मण पूर्वजः}
{पुत्रौ दशरथस्य आवाम् भगवन् राम लक्ष्मणौ} %2-54-13

\twolineshloka
{भार्या मम इयम् वैदेही कल्याणी जनक आत्मजा}
{माम् च अनुयाता विजनम् तपो वनम् अनिन्दिता} %2-54-14

\twolineshloka
{पित्रा प्रव्राज्यमानम् माम् सौमित्रिर् अनुजः प्रियः}
{अयम् अन्वगमद् भ्राता वनम् एव दृढ व्रतः} %2-54-15

\twolineshloka
{पित्रा नियुक्ता भगवन् प्रवेष्यामः तपो वनम्}
{धर्मम् एव आचरिष्यामः तत्र मूल फल अशनाः} %2-54-16

\twolineshloka
{तस्य तत् वचनम् श्रुत्वा राज पुत्रस्य धीमतः}
{उपानयत धर्म आत्मा गाम् अर्घ्यम् उदकम् ततः} %2-54-17

\twolineshloka
{नानाविधानन्नरसान् वन्यमूलफलाश्रयान्}
{तेभ्यो ददौ तप्ततपा वासम् चैवाभ्यकल्पयत्} %2-54-18

\twolineshloka
{मृग पक्षिभिर् आसीनो मुनिभिः च समन्ततः}
{रामम् आगतम् अभ्यर्च्य स्वागतेन आह तम् मुनिः} %2-54-19

\twolineshloka
{प्रतिगृह्य च ताम् अर्चाम् उपविष्टम् स राघवम्}
{भरद्वाजो अब्रवीद् वाक्यम् धर्म युक्तम् इदम् तदा} %2-54-20

\twolineshloka
{चिरस्य खलु काकुत्स्थ पश्यामि त्वाम् इह आगतम्}
{श्रुतम् तव मया च इदम् विवासनम् अकारणम्} %2-54-21

\twolineshloka
{अवकाशो विविक्तः अयम् महा नद्योह् समागमे}
{पुण्यः च रमणीयः च वसतु इह भगान् सुखम्} %2-54-22

\twolineshloka
{एवम् उक्तः तु वचनम् भरद्वाजेन राघवः}
{प्रत्युवाच शुभम् वाक्यम् रामः सर्व हिते रतः} %2-54-23

\twolineshloka
{भगवन्न् इताअसन्नः पौर जानपदो जनः}
{सुदर्शमिह माम् प्रेक्ष्य मन्येऽह मिममाश्रमम्} %2-54-24

\twolineshloka
{आगमिष्यति वैदेहीम् माम् च अपि प्रेक्षको जनः}
{अनेन कारणेन अहम् इह वासम् न रोचये} %2-54-25

\twolineshloka
{एक अन्ते पश्य भगवन्न् आश्रम स्थानम् उत्तमम्}
{रमते यत्र वैदेही सुख अर्हा जनक आत्मजा} %2-54-26

\twolineshloka
{एतत् श्रुत्वा शुभम् वाक्यम् भरद्वाजो महा मुनिः}
{राघवस्य ततः वाक्यम् अर्थ ग्राहकम् अब्रवीत्} %2-54-27

\twolineshloka
{दश क्रोशैतः तात गिरिर् यस्मिन् निवत्स्यसि}
{महर्षि सेवितः पुण्यः सर्वतः सुख दर्शनः} %2-54-28

\twolineshloka
{गो लान्गूल अनुचरितः वानर ऋष्क निषेवितः}
{चित्र कूटैति ख्यातः गन्ध मादन सम्निभः} %2-54-29

\twolineshloka
{यावता चित्र कूटस्य नरः शृन्गाणि अवेक्षते}
{कल्याणानि समाधत्ते न पापे कुरुते मनः} %2-54-30

\twolineshloka
{ऋषयः तत्र बहवो विहृत्य शरदाम् शतम्}
{तपसा दिवम् आरूधाः कपाल शिरसा सह} %2-54-31

\twolineshloka
{प्रविविक्तम् अहम् मन्ये तम् वासम् भवतः सुखम्}
{इह वा वन वासाय वस राम मया सह} %2-54-32

\twolineshloka
{स रामम् सर्व कामैअः तम् भरद्वाजः प्रिय अतिथिम्}
{सभार्यम् सह च भ्रात्रा प्रतिजग्राह धर्मवित्} %2-54-33

\twolineshloka
{तस्य प्रयागे रामस्य तम् महर्षिम् उपेयुषः}
{प्रपन्ना रजनी पुण्या चित्राः कथयतः कथाः} %2-54-34

\twolineshloka
{सीतातृतीय काकुत्स्थह् परिश्रान्तः सुखोचितः}
{भरद्वाजाश्रमे रम्ये ताम् रात्रि मवस्त्सुखम्} %2-54-35

\twolineshloka
{प्रभातायाम् रजन्याम् तु भरद्वाजम् उपागमत्}
{उवाच नर शार्दूलो मुनिम् ज्वलित तेजसम्} %2-54-36

\twolineshloka
{शर्वरीम् भवनन्न् अद्य सत्य शील तव आश्रमे}
{उषिताः स्म इह वसतिम् अनुजानातु नो भवान्} %2-54-37

\twolineshloka
{रात्र्याम् तु तस्याम् व्युष्टायाम् भरद्वाजो अब्रवीद् इदम्}
{मधु मूल फल उपेतम् चित्र कूटम् व्रज इति ह} %2-54-38

\twolineshloka
{वासमौपयिकम् मन्ये तव राम महाबल}
{नानानगगणोपेतः किन्नरोरगसेवितह्} %2-54-39

\twolineshloka
{मयूरनादाभिरुतो गजराजनिषेवितः}
{गम्यताम् भवता शैलश्चित्रकूटः स विश्रुतः} %2-54-40

\twolineshloka
{पुण्यश्च रमणीयश्च बहुमूलफलायुतः}
{तत्र कुन्जर यूथानि मृग यूथानि च अभितः} %2-54-41

\twolineshloka
{विचरन्ति वन अन्तेषु तानि द्रक्ष्यसि राघव}
{सरित्प्रस्रवणप्रस्थान् दरीकन्धरनिर्घरान्} %2-54-42

\twolineshloka
{चरतः सीतया सार्धम् नन्दिष्यति मनस्तव}
{प्रहृष्ट कोयष्टिक कोकिल स्वनैः}
{र्विनादितम् तम् वसुधा धरम् शिवम्}
{मृगैः च मत्तैः बहुभिः च कुन्जरैः}
{सुरम्यम् आसाद्य समावस आश्रमम्} %2-54-43


॥इत्यार्षे श्रीमद्रामायणे वाल्मीकीये आदिकाव्ये अयोध्याकाण्डे भरद्वाजाश्रमाभिगमनम् नाम चतुःपञ्चाशः सर्गः ॥२-५४॥
