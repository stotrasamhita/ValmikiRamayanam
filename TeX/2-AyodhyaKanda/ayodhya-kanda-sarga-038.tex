\sect{अष्टात्रिंशः सर्गः — जनाक्रोशः}

\twolineshloka
{तस्यां चीरं वसानायां नाथवत्यामनाथवत्}
{प्रचुक्रोश जनः सर्वो धिक् त्वां दशरथं त्विति} %2-38-1

\twolineshloka
{तेन तत्र प्रणादेन दुःखितः स महीपतिः}
{चिच्छेद जीविते श्रद्धां धर्मे यशसि चात्मनः} %2-38-2

\twolineshloka
{स निःश्वस्योष्णमैक्ष्वाकस्तां भार्यामिदमब्रवीत्}
{कैकेयि कुशचीरेण न सीता गन्तुमर्हति} %2-38-3

\twolineshloka
{सुकुमारी च बाला च सततं च सुखोचिता}
{नेयं वनस्य योग्येति सत्यमाह गुरुर्मम} %2-38-4

\twolineshloka
{इयं हि कस्यापि करोति किञ्चित् तपस्विनी राजवरस्य पुत्री}
{या चीरमासाद्य जनस्य मध्ये स्थिता विसंज्ञा श्रमणीव काचित्} %2-38-5

\twolineshloka
{चीराण्यपास्याज्जनकस्य कन्या नेयं प्रतिज्ञा मम दत्तपूर्वा}
{यथासुखं गच्छतु राजपुत्री वनं समग्रा सह सर्वरत्नैः} %2-38-6

\twolineshloka
{अजीवनार्हेण मया नृशंसा कृता प्रतिज्ञा नियमेन तावत्}
{त्वया हि बाल्यात् प्रतिपन्नमेतत् तन्मा दहेद् वेणुमिवात्मपुष्पम्} %2-38-7

\twolineshloka
{रामेण यदि ते पापे किञ्चित्कृतमशोभनम्}
{अपकारः क इह ते वैदेह्या दर्शितोऽधमे} %2-38-8

\twolineshloka
{मृगीवोत्फुल्लनयना मृदुशीला मनस्विनी}
{अपकारं कमिव ते करोति जनकात्मजा} %2-38-9

\twolineshloka
{ननु पर्याप्तमेवं ते पापे रामविवासनम्}
{किमेभिः कृपणैर्भूयः पातकैरपि ते कृतैः} %2-38-10

\twolineshloka
{प्रतिज्ञातं मया तावत् त्वयोक्तं देवि शृण्वता}
{रामं यदभिषेकाय त्वमिहागतमब्रवीः} %2-38-11

\twolineshloka
{तत्त्वेतत् समतिक्रम्य निरयं गन्तुमिच्छसि}
{मैथिलीमपि या हि त्वमीक्षसे चीरवासिनीम्} %2-38-12

\twolineshloka
{एवं ब्रुवन्तं पितरं रामः सम्प्रस्थितो वनम्}
{अवाक्शिरसमासीनमिदं वचनमब्रवीत्} %2-38-13

\twolineshloka
{इयं धार्मिक कौसल्या मम माता यशस्विनी}
{वृद्धा चाक्षुद्रशीला च न च त्वां देव गर्हते} %2-38-14

\twolineshloka
{मया विहीनां वरद प्रपन्नां शोकसागरम्}
{अदृष्टपूर्वव्यसनां भूयः सम्मन्तुमर्हसि} %2-38-15

\twolineshloka
{पुत्रशोकं यथा नर्च्छेत् त्वया पूज्येन पूजिता}
{मां हि सञ्चिन्तयन्ती सा त्वयि जीवेत् तपस्विनी} %2-38-16

\twolineshloka
{इमां महेन्द्रोपम जातगर्धिनीं तथा विधातुं जननीं ममार्हसि}
{यथा वनस्थे मयि शोककर्शिता न जीवितं न्यस्य यमक्षयं व्रजेत्} %2-38-17


॥इत्यार्षे श्रीमद्रामायणे वाल्मीकीये आदिकाव्ये अयोध्याकाण्डे जनाक्रोशः नाम अष्टात्रिंशः सर्गः ॥२-३८॥
