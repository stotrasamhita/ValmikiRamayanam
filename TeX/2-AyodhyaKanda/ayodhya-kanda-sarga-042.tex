\sect{द्विचत्वारिंशः सर्गः — दशरथाक्रन्दः}

\twolineshloka
{यावत् तु निर्यतस्तस्य रजोरूपमदृश्यत}
{नैवेक्ष्वाकुवरस्तावत् संजहारात्मचक्षुषी} %2-42-1

\twolineshloka
{यावद् राजा प्रियं पुत्रं पश्यत्यत्यन्तधार्मिकम्}
{तावद् व्यवर्धतेवास्य धरण्यां पुत्रदर्शने} %2-42-2

\twolineshloka
{न पश्यति रजोऽप्यस्य यदा रामस्य भूमिपः}
{तदार्तश्च निषण्णश्च पपात धरणीतले} %2-42-3

\twolineshloka
{तस्य दक्षिणमन्वागात् कौसल्या बाहुमङ्गना}
{परं चास्यान्वगात् पार्श्वं कैकेयी सा सुमध्यमा} %2-42-4

\twolineshloka
{तां नयेन च सम्पन्नो धर्मेण विनयेन च}
{उवाच राजा कैकेयीं समीक्ष्य व्यथितेन्द्रियः} %2-42-5

\twolineshloka
{कैकेयि मामकाङ्गानि मा स्प्राक्षीः पापनिश्चये}
{नहि त्वां द्रष्टुमिच्छामि न भार्या न च बान्धवी} %2-42-6

\twolineshloka
{ये च त्वामनुजीवन्ति नाहं तेषां न ते मम}
{केवलार्थपरां हि त्वां त्यक्तधर्मां त्यजाम्यहम्} %2-42-7

\twolineshloka
{अगृह्णां यच्च ते पाणिमग्निं पर्यणयं च यत्}
{अनुजानामि तत् सर्वमस्मिंल्लोके परत्र च} %2-42-8

\twolineshloka
{भरतश्चेत् प्रतीतः स्याद् राज्यं प्राप्यैतदव्ययम्}
{यन्मे स दद्यात् पित्रर्थं मा मां तद्दत्तमागमत्} %2-42-9

\twolineshloka
{अथ रेणुसमुद्ध्वस्तं समुत्थाप्य नराधिपम्}
{न्यवर्तत तदा देवी कौसल्या शोककर्शिता} %2-42-10

\twolineshloka
{हत्वेव ब्राह्मणं कामात् स्पृष्ट्वाग्निमिव पाणिना}
{अन्वतप्यत धर्मात्मा पुत्रं संचिन्त्य राघवम्} %2-42-11

\twolineshloka
{निवृत्यैव निवृत्यैव सीदतो रथवर्त्मसु}
{राज्ञो नातिबभौ रूपं ग्रस्तस्यांशुमतो यथा} %2-42-12

\twolineshloka
{विललाप स दुःखार्तः प्रियं पुत्रमनुस्मरन्}
{नगरान्तमनुप्राप्तं बुद्ध्वा पुत्रमथाब्रवीत्} %2-42-13

\twolineshloka
{वाहनानां च मुख्यानां वहतां तं ममात्मजम्}
{पदानि पथि दृश्यन्ते स महात्मा न दृश्यते} %2-42-14

\twolineshloka
{यः सुखेनोपधानेषु शेते चन्दनरूषितः}
{वीज्यमानो महार्हाभिः स्त्रीभिर्मम सुतोत्तमः} %2-42-15

\twolineshloka
{स नूनं क्वचिदेवाद्य वृक्षमूलमुपाश्रितः}
{काष्ठं वा यदि वाश्मानमुपधाय शयिष्यते} %2-42-16

\twolineshloka
{उत्थास्यति च मेदिन्याः कृपणः पांसुगुण्ठितः}
{विनिःश्वसन् प्रस्रवणात् करेणूनामिवर्षभः} %2-42-17

\twolineshloka
{द्रक्ष्यन्ति नूनं पुरुषा दीर्घबाहुं वनेचराः}
{राममुत्थाय गच्छन्तं लोकनाथमनाथवत्} %2-42-18

\twolineshloka
{सा नूनं जनकस्येष्टा सुता सुखसदोचिता}
{कण्टकाक्रमणक्लान्ता वनमद्य गमिष्यति} %2-42-19

\twolineshloka
{अनभिज्ञा वनानां सा नूनं भयमुपैष्यति}
{श्वपदानर्दितं श्रुत्वा गम्भीरं रोमहर्षणम्} %2-42-20

\twolineshloka
{सकामा भव कैकेयि विधवा राज्यमावस}
{नहि तं पुरुषव्याघ्रं विना जीवितुमुत्सहे} %2-42-21

\twolineshloka
{इत्येवं विलपन् राजा जनौघेनाभिसंवृतः}
{अपस्नात इवारिष्टं प्रविवेश गृहोत्तमम्} %2-42-22

\twolineshloka
{शून्यचत्वरवेश्मान्तां संवृतापणवेदिकाम्}
{क्लान्तदुर्बलदुःखार्तां नात्याकीर्णमहापथाम्} %2-42-23

\twolineshloka
{तामवेक्ष्य पुरीं सर्वां राममेवानुचिन्तयन्}
{विलपन् प्राविशद् राजा गृहं सूर्य इवाम्बुदम्} %2-42-24

\twolineshloka
{महाह्रदमिवाक्षोभ्यं सुपर्णेन हृतोरगम्}
{रामेण रहितं वेश्म वैदेह्या लक्ष्मणेन च} %2-42-25

\twolineshloka
{अथ गद्गदशब्दस्तु विलपन् वसुधाधिपः}
{उवाच मृदु मन्दार्थं वचनं दीनमस्वरम्} %2-42-26

\twolineshloka
{कौसल्याया गृहं शीघ्रं राममातुर्नयन्तु माम्}
{नह्यन्यत्र ममाश्वासो हृदयस्य भविष्यति} %2-42-27

\twolineshloka
{इति ब्रुवन्तं राजानमनयन् द्वारदर्शिनः}
{कौसल्याया गृहं तत्र न्यवेस्यत विनीतवत्} %2-42-28

\twolineshloka
{ततस्तत्र प्रविष्टस्य कौसल्याया निवेशनम्}
{अधिरुह्यापि शयनं बभूव लुलितं मनः} %2-42-29

\twolineshloka
{पुत्रद्वयविहीनं च स्नुषया च विवर्जितम्}
{अपश्यद् भवनं राजा नष्टचन्द्रमिवाम्बरम्} %2-42-30

\twolineshloka
{तच्च दृष्ट्वा महाराजो भुजमुद्यम्य वीर्यवान्}
{उच्चैःस्वरेण प्राक्रोशद्धा राम विजहासि नौ} %2-42-31

\twolineshloka
{सुखिता बत तं कालं जीविष्यन्ति नरोत्तमाः}
{परिष्वजन्तो ये रामं द्रक्ष्यन्ति पुनरागतम्} %2-42-32

\twolineshloka
{अथ रात्र्यां प्रपन्नायां कालरात्र्यामिवात्मनः}
{अर्धरात्रे दशरथः कौसल्यामिदमब्रवीत्} %2-42-33

\twolineshloka
{न त्वां पश्यामि कौसल्ये साधु मां पाणिना स्पृश}
{रामं मेऽनुगता दृष्टिरद्यापि न निवर्तते} %2-42-34

\twolineshloka
{तं राममेवानुविचिन्तयन्तं समीक्ष्य देवी शयने नरेन्द्रम्}
{उपोपविश्याधिकमार्तरूपा विनिश्वसन्तं विललाप कृच्छ्रम्} %2-42-35


॥इत्यार्षे श्रीमद्रामायणे वाल्मीकीये आदिकाव्ये अयोध्याकाण्डे दशरथाक्रन्दः नाम द्विचत्वारिंशः सर्गः ॥२-४२॥
