\sect{त्रिपञ्चाशः सर्गः — रामसंक्षोभः}

\twolineshloka
{स तम् वृक्षम् समासाद्य सम्ध्याम् अन्वास्य पश्चिमाम्}
{रामः रमयताम् श्रेष्ठैति ह उवाच लक्ष्मणम्} %2-53-1

\twolineshloka
{अद्य इयम् प्रथमा रात्रिर् याता जन पदात् बहिः}
{या सुमन्त्रेण रहिता ताम् न उत्कण्ठितुम् अर्हसि} %2-53-2

\twolineshloka
{जागर्तव्यम् अतन्द्रिभ्याम् अद्य प्रभृति रात्रिषु}
{योग क्षेमः हि सीताया वर्तते लक्ष्मण आवयोह्} %2-53-3

\twolineshloka
{रात्रिम् कथम्चित् एव इमाम् सौमित्रे वर्तयामहे}
{उपावर्तामहे भूमाव् आस्तीर्य स्वयम् आर्जितैः} %2-53-4

\twolineshloka
{स तु सम्विश्य मेदिन्याम् महा अर्ह शयन उचितः}
{इमाः सौमित्रये रामः व्याजहार कथाः शुभाः} %2-53-5

\twolineshloka
{ध्रुवम् अद्य महा राजो दुह्खम् स्वपिति लक्ष्मण}
{कृत कामा तु कैकेयी तुष्टा भवितुम् अर्हति} %2-53-6

\twolineshloka
{सा हि देवी महा राजम् कैकेयी राज्य कारणात्}
{अपि न च्यावयेत् प्राणान् दृष्ट्वा भरतम् आगतम्} %2-53-7

\twolineshloka
{अनाथः चैव वृद्धः च मया चैव विनाकृतः}
{किम् करिष्यति काम आत्मा कैकेय्या वशम् आगतः} %2-53-8

\twolineshloka
{इदम् व्यसनम् आलोक्य राज्ञः च मति विभ्रमम्}
{कामएव अर्ध धर्माभ्याम् गरीयान् इति मे मतिः} %2-53-9

\twolineshloka
{को हि अविद्वान् अपि पुमान् प्रमदायाः कृते त्यजेत्}
{चन्द अनुवर्तिनम् पुत्रम् तातः माम् इव लक्ष्मण} %2-53-10

\twolineshloka
{सुखी बत सभार्यः च भरतः केकयी सुतः}
{मुदितान् कोसलान् एको यो भोक्ष्यति अधिराजवत्} %2-53-11

\twolineshloka
{स हि सर्वस्य राज्यस्य मुखम् एकम् भविष्यति}
{ताते च वयसा अतीते मयि च अरण्यम् आश्रिते} %2-53-12

\twolineshloka
{अर्थ धर्मौ परित्यज्य यः कामम् अनुवर्तते}
{एवम् आपद्यते क्षिप्रम् राजा दशरथो यथा} %2-53-13

\twolineshloka
{मन्ये दशरथ अन्ताय मम प्रव्राजनाय च}
{कैकेयी सौम्य सम्प्राप्ता राज्याय भरतस्य च} %2-53-14

\twolineshloka
{अपि इदानीम् न कैकेयी सौभाग्य मद मोहिता}
{कौसल्याम् च सुमित्राम् च सम्प्रबाधेत मत् कृते} %2-53-15

\twolineshloka
{मा स्म मत् कारणात् देवी सुमित्रा दुह्खम् आवसेत्}
{अयोध्याम् इतएव त्वम् काले प्रविश लक्ष्मण} %2-53-16

\twolineshloka
{अहम् एको गमिष्यामि सीतया सह दण्डकान्}
{अनाथाया हि नाथः त्वम् कौसल्याया भविष्यसि} %2-53-17

\twolineshloka
{क्षुद्र कर्मा हि कैकेयी द्वेषात् अन्याय्यम् आचरेत्}
{परिदद्या हि धर्मज्ञे भरते मम मातरम्} %2-53-18

\twolineshloka
{नूनम् जाति अन्तरे कस्मिम्स् स्त्रियः पुत्रैः वियोजिताः}
{जनन्या मम सौमित्रे तत् अपि एतत् उपस्थितम्} %2-53-19

\twolineshloka
{मया हि चिर पुष्टेन दुह्ख सम्वर्धितेन च}
{विप्रायुज्यत कौसल्या फल काले धिग् अस्तु माम्} %2-53-20

\twolineshloka
{मा स्म सीमन्तिनी काचिज् जनयेत् पुत्रम् ईदृशम्}
{सौमित्रे यो अहम् अम्बाया दद्मि शोकम् अनन्तकम्} %2-53-21

\twolineshloka
{मन्ये प्रीति विशिष्टा सा मत्तः लक्ष्मण सारिका}
{यस्याः तत् श्रूयते वाक्यम् शुक पादम् अरेर् दश} %2-53-22

\twolineshloka
{शोचन्त्याः च अल्प भाग्याया न किम्चित् उपकुर्वता}
{पुर्त्रेण किम् अपुत्राया मया कार्यम् अरिम् दम} %2-53-23

\twolineshloka
{अल्प भाग्या हि मे माता कौसल्या रहिता मया}
{शेते परम दुह्ख आर्ता पतिता शोक सागरे} %2-53-24

\twolineshloka
{एको हि अहम् अयोध्याम् च पृथिवीम् च अपि लक्ष्मण}
{तरेयम् इषुभिः क्रुद्धो ननु वीर्यम् अकारणम्} %2-53-25

\twolineshloka
{अधर्म भय भीतः च पर लोकस्य च अनघ}
{तेन लक्ष्मण न अद्य अहम् आत्मानम् अभिषेचये} %2-53-26

\twolineshloka
{एतत् अन्यच् च करुणम् विलप्य विजने बहु}
{अश्रु पूर्ण मुखो रामः निशि तूष्णीम् उपाविशत्} %2-53-27

\twolineshloka
{विलप्य उपरतम् रामम् गत अर्चिषम् इव अनलम्}
{समुद्रम् इव निर्वेगम् आश्वासयत लक्ष्मणः} %2-53-28

\twolineshloka
{ध्रुवम् अद्य पुरी रामायोध्या युधिनाम् वर}
{निष्प्रभा त्वयि निष्क्रान्ते गत चन्द्रा इव शर्वरी} %2-53-29

\twolineshloka
{न एतत् औपयिकम् राम यद् इदम् परितप्यसे}
{विषादयसि सीताम् च माम् चैव पुरुष ऋषभ} %2-53-30

\twolineshloka
{न च सीता त्वया हीना न च अहम् अपि राघव}
{मुहूर्तम् अपि जीवावो जलान् मत्स्याव् इव उद्धृतौ} %2-53-31

\twolineshloka
{न हि तातम् न शत्रुघ्नम् न सुमित्राम् परम् तप}
{द्रष्टुम् इच्चेयम् अद्य अहम् स्वर्गम् वा अपि त्वया विना} %2-53-32

\twolineshloka
{ततस्तत्र सुखासीने नातिदूरे निरीक्ष्य ताम्}
{न्यग्रोधे सुकृताम् शय्याम् भेजाते धर्मवत्सलौ} %2-53-33

\fourlineindentedshloka
{स लक्ष्मणस्य उत्तम पुष्कलम् वचो}
{निशम्य च एवम् वन वासम् आदरात्}
{समाः समस्ता विदधे परम् तपः}
{प्रपद्य धर्मम् सुचिराय राघवः} %2-53-34

\fourlineindentedshloka
{ततस्तु तस्मिन् विजने वने तदा}
{महाबलौ राघववम्शवर्धनौ}
{न तौ भयम् सम्भ्रममभ्युपेयतु}
{र्यथैव सिम्हौ गिरिसानुगोचरौ} %2-53-35


॥इत्यार्षे श्रीमद्रामायणे वाल्मीकीये आदिकाव्ये अयोध्याकाण्डे रामसंक्षोभः नाम त्रिपञ्चाशः सर्गः ॥२-५३॥
