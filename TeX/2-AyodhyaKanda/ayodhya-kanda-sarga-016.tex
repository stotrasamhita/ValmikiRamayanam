\sect{षोडशः सर्गः — रामप्रस्थानम्}

\twolineshloka
{स तदन्तःपुरद्वारं समतीत्य जनाकुलम्}
{प्रविविक्तां ततः कक्ष्यामाससाद पुराणवित्} %2-16-1

\twolineshloka
{प्रासकार्मुकबिभ्रद्भिर्युवभिर्मृष्टकुण्डलैः}
{अप्रमादिभिरेकाग्रैः स्वानुरक्तैरधिष्ठिताम्} %2-16-2

\twolineshloka
{तत्र काषायिणो वृद्धान् वेत्रपाणीन् स्वलङ्कृतान्}
{ददर्श विष्ठितान् द्वारि स्त्र्यध्यक्षान् समाहितान्} %2-16-3

\twolineshloka
{ते समीक्ष्य समायान्तं रामप्रियचिकीर्षवः}
{सहसोत्पतिताः सर्वे ह्यासनेभ्यः ससम्भ्रमाः} %2-16-4

\twolineshloka
{तानुवाच विनीतात्मा सूतपुत्रः प्रदक्षिणः}
{क्षिप्रमाख्यात रामाय सुमन्त्रो द्वारि तिष्ठति} %2-16-5

\twolineshloka
{ते राममुपसङ्गम्य भर्तुः प्रियचिकीर्षवः}
{सहभार्याय रामाय क्षिप्रमेवाचचक्षिरे} %2-16-6

\twolineshloka
{प्रतिवेदितमाज्ञाय सूतमभ्यन्तरं पितुः}
{तत्रैवानाययामास राघवः प्रियकाम्यया} %2-16-7

\twolineshloka
{तं वैश्रवणसङ्काशमुपविष्टं स्वलङ्कृतम्}
{ददर्श सूतः पर्यङ्के सौवर्णे सोत्तरच्छदे} %2-16-8

\twolineshloka
{वराहरुधिराभेण शुचिना च सुगन्धिना}
{अनुलिप्तं परार्घ्येन चन्दनेन परन्तपम्} %2-16-9

\twolineshloka
{स्थितया पार्श्वतश्चापि वालव्यजनहस्तया}
{उपेतं सीतया भूयश्चित्रया शशिनं यथा} %2-16-10

\twolineshloka
{तं तपन्तमिवादित्यमुपपन्नं स्वतेजसा}
{ववन्दे वरदं वन्दी विनयज्ञो विनीतवत्} %2-16-11

\twolineshloka
{प्राञ्जलिः सुमुखं दृष्ट्वा विहारशयनासने}
{राजपुत्रमुवाचेदं सुमन्त्रो राजसत्कृतः} %2-16-12

\twolineshloka
{कौसल्या सुप्रजा राम पिता त्वां द्रष्टुमिच्छति}
{महिष्यापि हि कैकेय्या गम्यतां तत्र मा चिरम्} %2-16-13

\twolineshloka
{एवमुक्तस्तु संहृष्टो नरसिंहो महाद्युतिः}
{ततः सम्मानयामास सीतामिदमुवाच ह} %2-16-14

\twolineshloka
{देवि देवश्च देवी च समागम्य मदन्तरे}
{मन्त्रयेते ध्रुवं किञ्चिदभिषेचनसंहितम्} %2-16-15

\twolineshloka
{लक्षयित्वा ह्यभिप्रायं प्रियकामा सुदक्षिणा}
{सञ्चोदयति राजानं मदर्थमसितेक्षणा} %2-16-16

\twolineshloka
{सा प्रहृष्टा महाराजं हितकामानुवर्तिनी}
{जननी चार्थकामा मे केकयाधिपतेः सुता} %2-16-17

\twolineshloka
{दिष्ट्या खलु महाराजो महिष्या प्रियया सह}
{सुमन्त्रं प्राहिणोद् दूतमर्थकामकरं मम} %2-16-18

\twolineshloka
{यादृशी परिषत् तत्र तादृशो दूत आगतः}
{ध्रुवमद्यैव मां राजा यौवराज्येऽभिषेक्ष्यति} %2-16-19

\twolineshloka
{हन्त शीघ्रमितो गत्वा द्रक्ष्यामि च महीपतिम्}
{सह त्वं परिवारेण सुखमास्स्व रमस्व च} %2-16-20

\twolineshloka
{पतिसम्मानिता सीता भर्तारमसितेक्षणा}
{आ द्वारमनुवव्राज मङ्गलान्यभिदध्युषी} %2-16-21

\twolineshloka
{राज्यं द्विजातिभिर्जुष्टं राजसूयाभिषेचनम्}
{कर्तुमर्हति ते राजा वासवस्येव लोककृत्} %2-16-22

\twolineshloka
{दीक्षितं व्रतसम्पन्नं वराजिनधरं शुचिम्}
{कुरङ्गशृङ्गपाणिं च पश्यन्ती त्वां भजाम्यहम्} %2-16-23

\twolineshloka
{पूर्वां दिशं वज्रधरो दक्षिणां पातु ते यमः}
{वरुणः पश्चिमामाशां धनेशस्तूत्तरां दिशम्} %2-16-24

\twolineshloka
{अथ सीतामनुज्ञाप्य कृतकौतुकमङ्गलः}
{निश्चक्राम सुमन्त्रेण सह रामो निवेशनात्} %2-16-25

\twolineshloka
{पर्वतादिव निष्क्रम्य सिंहो गिरिगुहाशयः}
{लक्ष्मणं द्वारि सोऽपश्यत् प्रह्वाञ्जलिपुटं स्थितम्} %2-16-26

\twolineshloka
{अथ मध्यमकक्ष्यायां समागच्छत् सुहृज्जनैः}
{स सर्वानर्थिनो दृष्ट्वा समेत्य प्रतिनन्द्य च} %2-16-27

\twolineshloka
{ततः पावकसङ्काशमारुरोह रथोत्तमम्}
{वैयाघ्रं पुरुषव्याघ्रो राजितं राजनन्दनः} %2-16-28

\twolineshloka
{मेघनादमसम्बाधं मणिहेमविभूषितम्}
{मुष्णन्तमिव चक्षूंषि प्रभया मेरुवर्चसम्} %2-16-29

\twolineshloka
{करेणुशिशुकल्पैश्च युक्तं परमवाजिभिः}
{हरियुक्तं सहस्राक्षो रथमिन्द्र इवाशुगम्} %2-16-30

\twolineshloka
{प्रययौ तूर्णमास्थाय राघवो ज्वलितः श्रिया}
{स पर्जन्य इवाकाशे स्वनवानभिनादयन्} %2-16-31

\twolineshloka
{निकेतान्निर्ययौ श्रीमान् महाभ्रादिव चन्द्रमाः}
{चित्रचामरपाणिस्तु लक्ष्मणो राघवानुजः} %2-16-32

\twolineshloka
{जुगोप भ्रातरं भ्राता रथमास्थाय पृष्ठतः}
{ततो हलहलाशब्दस्तुमुलः समजायत} %2-16-33

\twolineshloka
{तस्य निष्क्रममाणस्य जनौघस्य समन्ततः}
{ततो हयवरा मुख्या नागाश्च गिरिसन्निभाः} %2-16-34

\twolineshloka
{अनुजग्मुस्तथा रामं शतशोऽथ सहस्रशः}
{अग्रतश्चास्य सन्नद्धाश्चन्दनागुरुभूषिताः} %2-16-35

\twolineshloka
{खड्गचापधराः शूरा जग्मुराशंसवो जनाः}
{ततो वादित्रशब्दाश्च स्तुतिशब्दाश्च वन्दिनाम्} %2-16-36

\twolineshloka
{सिंहनादाश्च शूराणां ततः शुश्रुविरे पथि}
{हर्म्यवातायनस्थाभिर्भूषिताभिः समन्ततः} %2-16-37

\twolineshloka
{कीर्यमाणः सुपुष्पौघैर्ययौ स्त्रीभिररिन्दमः}
{रामं सर्वानवद्याङ्ग्यो रामपिप्रीषया ततः} %2-16-38

\twolineshloka
{वचोभिरग्र्यैर्हर्म्यस्थाः क्षितिस्थाश्च ववन्दिरे}
{नूनं नन्दति ते माता कौसल्या मातृनन्दन} %2-16-39

\twolineshloka
{पश्यन्ती सिद्धयात्रं त्वां पित्र्यं राज्यमुपस्थितम्}
{सर्वसीमन्तिनीभ्यश्च सीतां सीमन्तिनीं वराम्} %2-16-40

\twolineshloka
{अमन्यन्त हि ता नार्यो रामस्य हृदयप्रियाम्}
{तया सुचरितं देव्या पुरा नूनं महत् तपः} %2-16-41

\threelineshloka
{रोहिणीव शशाङ्केन रामसंयोगमाप या}
{इति प्रासादशृङ्गेषु प्रमदाभिर्नरोत्तमः}
{शुश्राव राजमार्गस्थः प्रिया वाच उदाहृताः} %2-16-42

\twolineshloka
{स राघवस्तत्र तदा प्रलापान् शुश्राव लोकस्य समागतस्य}
{आत्माधिकारा विविधाश्च वाचः प्रहृष्टरूपस्य पुरे जनस्य} %2-16-43

\twolineshloka
{एष श्रियं गच्छति राघवोऽद्य राजप्रसादाद् विपुलां गमिष्यन्}
{एते वयं सर्वसमृद्धकामा येषामयं नो भविता प्रशास्ता} %2-16-44

\twolineshloka
{लाभो जनस्यास्य यदेष सर्वं प्रपत्स्यते राष्ट्रमिदं चिराय}
{न ह्यप्रियं किञ्चन जातु कश्चित् पश्येन्न दुःखं मनुजाधिपेऽस्मिन्} %2-16-45

\twolineshloka
{स घोषवद्भिश्च हयैः सनागैः पुरःसरैः स्वस्तिकसूतमागधैः}
{महीयमानः प्रवरैश्च वादकैरभिष्टुतो वैश्रवणो यथा ययौ} %2-16-46

\twolineshloka
{करेणुमातङ्गरथाश्वसङ्कुलं महाजनौघैः परिपूर्णचत्वरम्}
{प्रभूतरत्नं बहुपण्यसञ्चयं ददर्श रामो विमलं महापथम्} %2-16-47


॥इत्यार्षे श्रीमद्रामायणे वाल्मीकीये आदिकाव्ये अयोध्याकाण्डे रामप्रस्थानम् नाम षोडशः सर्गः ॥२-१६॥
