\sect{पञ्चपञ्चाशः सर्गः — यमुनातरणम्}

\twolineshloka
{उषित्वा रजनीम् तत्र राजपुत्रावरिम्दमौ}
{महर्षिमभिवाद्याथ जग्मतुस्तम् गिरिम् प्रति} %2-55-1

\twolineshloka
{तेषाम् चैव स्वस्त्ययनम् महर्षिः स चकार ह}
{प्रस्थिताम्श्चैव तान् प्रेक्ष्यपिता पुत्रानिवान्वगात्} %2-55-2

\twolineshloka
{ततः प्रचक्रमे वक्तुम् वचनम् स महामुनिः}
{भर्द्वाजो महातेजा रामम् सत्यपराक्रमम्} %2-55-3

\twolineshloka
{गङ्गायमुनयोः सन्धिमासाद्य मनुजर्षभौ}
{कालिन्दीमनुगच्छेताम् नदीम् पश्चान्मुखाश्रिताम्} %2-55-4

\twolineshloka
{अथासाद्य तु कालिन्दीं शीघ्रस्रोतसमापगाम् प्रतिस्रोतःसमागताम् पाभे}
{तस्यास्तीर्थम् प्रचरितम् पुराणम् प्रेक्ष्य राघवौ} %2-55-5

\twolineshloka
{तत्र यूयम् प्लवम् कृत्वा तरतांशुमतीं नदीम्}
{ततो न्यग्रोधमासाद्य महान्तम् हरितच्छदम्} %2-55-6

\twolineshloka
{विवृद्धम् बहुभिर्वऋक्षैह् श्यामम् सिद्धोपसेवितम्}
{तस्मै सीताञ्जलिम् कृत्वा प्रयुञ्जीताशिषः शिवाः} %2-55-7

\twolineshloka
{समासाद्य तु तम् वृक्षम् वसेद्वातिक्रमेत वा}
{क्रोशमात्रम् ततो गत्वा नीलम् द्रक्ष्यथ काननम्} %2-55-8

\twolineshloka
{पलाशबदरीमिश्रम् रम्यम् वम्शैश्च यामुनैः}
{स पन्थाश्चित्रकूटस्य गतः सुबहुशो मया} %2-55-9

\twolineshloka
{रम्ये मार्दवयुक्तश्च वनदावैर्विपर्जितः}
{इति पन्थानमावेद्य महर्षः स न्यवर्तत} %2-55-10

\twolineshloka
{अभिवाद्य तथेत्युक्त्वा रामेण विनिवर्तितः}
{उपावृत्ते मुनौ तस्मिन् रामो लक्ष्मणमब्रवीत्} %2-55-11

\twolineshloka
{कृतपुण्याः स्म सौमित्रे मुनिर्यन्नोऽनुकम्पते}
{इति तौ पुरुषव्याघ्रौ मन्त्रयित्वा मनस्विनौ} %2-55-12

\twolineshloka
{सीतामेवाग्रतः कृत्वा काइन्दीम् जग्मतुर्नदीम्}
{अथा साद्य तु काइन्दीम् शीघ्रस्रोतोवहाम् नदीम्} %2-55-13

\onelineshloka
{तौ काष्ठसम्घातमथो चक्रतुस्तु महाप्लवम्} %2-55-14

\twolineshloka
{शुष्कैर्वम्शैः समास्तीर्णमुईरैश्च समावृतम्}
{ततो वेतसशाखाश्च जम्बूशाखाश्च वीर्यवान्} %2-55-15

\twolineshloka
{चकार लक्ष्मणश्छित्वा सीतायाः सुखमासनम्}
{तत्र श्रियमिवाचिन्त्याम् रामो दाशरथिः प्रियाम्} %2-55-16

\twolineshloka
{ईष्त्सन्कह्हनाबान् तानग्तारिओअतत् प्लवम्}
{पार्श्वे च तत्र वैदेह्या वसने चूष्णानि च} %2-55-17

\twolineshloka
{प्लवे कठिनकाजम् च रामश्चक्रे सहायुधैः}
{आरोप्य प्रथमम् सीताम् सम्घाटम् प्रतिगृह्य तौ} %2-55-18

\twolineshloka
{ततः प्रतेरतुर्य त्तौ वीरौ दशरथात्मजौ}
{काइन्दीमध्यमायाता सीता त्वेनामवन्दत} %2-55-19

\twolineshloka
{स्वस्ति देवि तरामि त्वाम् पार्येन्मे पतिर्वतम्}
{यक्ष्ये त्वाम् गोनहस्रेण सुराघटशतेन च} %2-55-20

\twolineshloka
{स्वस्ति प्रत्यागते रामे पुरीमिक्ष्वाकुपालिताम्}
{काइन्दीमथ सीता तु याचमाना कृताञ्जलिः} %2-55-21

\twolineshloka
{तीरमेवाभिसम्प्राप्ता दक्षिणम् वरवर्णिनी}
{ततः प्लवेनाम्शुमतीम् शीघ्रगामूर्मिमालिनीम्} %2-55-22

\twolineshloka
{तीरजैर्बहुभिर्वृक्षैः सम्तेरुर्यमुनाम् नदीम्}
{ते तीर्णाः प्लवमुत्सृज्य प्रस्थाय यमुनावनात्} %2-55-23

\twolineshloka
{श्यामम् न्यग्रोधमासेदुः शीतलम् हरितच्छदम्}
{न्य्ग्रोधम् तमुपागम्य वैदेहि वाक्यमब्रवीत्} %2-55-24

\twolineshloka
{नमस्तेऽन्तु महावृक्ष पारयेन्मे पतिर्वतम्}
{कौसल्याम् चैव पश्येयम् सुमित्राम् च यशस्विनीम्} %2-55-25

\twolineshloka
{इति सीताञ्जलिम् कृत्वा पर्यगच्छद्वनस्पतिम्}
{अवलोक्य ततः सीतामायाचन्तीमनिन्दिताम्} %2-55-26

\twolineshloka
{दयिताम् च विधेयम् च रामो लक्ष्मणमब्रवीत्}
{सीतामादाय गच्छ त्वमग्रतो भरतानुज} %2-55-27

\twolineshloka
{पृष्ठतोऽहम् गमिष्यामि सायुधो द्विपदाम् वर}
{यद्यत्फलम् प्रार्थयते पुष्पम् वा जनकात्मजा} %2-55-28

\twolineshloka
{तत्तत्प्रदद्या वैदेह्या यत्रास्य रमते मनः}
{गच्चतोस्तु तयोर्मध्ये बभूव जनकात्मजा} %2-55-29

\twolineshloka
{मातङ्गयोर्मद्यगता शुभा नागवधूरिव}
{एकैकम् पादपम् गुल्मम् लताम् वा पुष्पशालिनीम्} %2-55-30

\twolineshloka
{अदृष्टपूर्वाम् पश्यन्ती रामम् पप्रच्छ साऽबला}
{रमणीयान् बहुविधान् पादपान् कुसुमोत्कटान्} %2-55-31

\twolineshloka
{सीतावचनसम्रब्द अनयामास लक्स्मणः}
{विचित्रवालुकजलाम् हससारसनादिताम्} %2-55-32

\twolineshloka
{रेमे जनकराजस्य तदा प्रेक्ष्य सुता नदीम्}
{क्रोशमात्रम् ततो गत्वा भ्रातरौ रामलक्ष्मनौ} %2-55-33

\twolineshloka
{बहून्मेध्यान् मृगान् हत्वा चेरतुर्यमुनावने}
{विहृत्य ते बर्हिणपूगनादिते}
{शुभे वने वानरवारणायुते}
{समम् नदीवप्रमुपेत्य सम्मतम्}
{निवासमाजग्मु रदीनदर्शनाः} %2-55-34


॥इत्यार्षे श्रीमद्रामायणे वाल्मीकीये आदिकाव्ये अयोध्याकाण्डे यमुनातरणम् नाम पञ्चपञ्चाशः सर्गः ॥२-५५॥
