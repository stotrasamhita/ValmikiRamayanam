\sect{द्विसप्ततितमः सर्गः — भरतसन्तापः}

\twolineshloka
{अपश्यम्स् तु ततः तत्र पितरम् पितुर् आलये}
{जगाम भरतः द्रष्टुम् मातरम् मातुर् आलये} %2-72-1

\twolineshloka
{अनुप्राप्तम् तु तम् दृष्ट्वा कैकेयी प्रोषितम् सुतम्}
{उत्पपात तदा हृष्टा त्यक्त्वा सौवर्ण मानसम्} %2-72-2

\twolineshloka
{स प्रविश्य एव धर्म आत्मा स्व गृहम् श्री विवर्जितम्}
{भरतः प्रेक्ष्य जग्राह जनन्याः चरणौ शुभौ} %2-72-3

\twolineshloka
{सा तम् मूर्ध्नि समुपाघ्राय परिष्वज्य यशस्विनम्}
{अङ्के भरतम् आरोप्य प्रष्टुम् समुपचक्रमे} %2-72-4

\twolineshloka
{अद्य ते कतिचित् रात्र्यः च्युतस्य आर्यक वेश्मनः}
{अपि न अध्व श्रमः शीघ्रम् रथेन आपततः तव} %2-72-5

\twolineshloka
{आर्यकः ते सुकुशलो युधा जिन् मातुलः तव}
{प्रवासाच् च सुखम् पुत्र सर्वम् मे वक्तुम् अर्हसि} %2-72-6

\twolineshloka
{एवम् पृष्ठः तु कैकेय्या प्रियम् पार्थिव नन्दनः}
{आचष्ट भरतः सर्वम् मात्रे राजीव लोचनः} %2-72-7

\twolineshloka
{अद्य मे सप्तमी रात्रिः च्युतस्य आर्यक वेश्मनः}
{अम्बायाः कुशली तातः युधाजिन् मातुलः च मे} %2-72-8

\twolineshloka
{यन् मे धनम् च रत्नम् च ददौ राजा परम् तपः}
{परिश्रान्तम् पथि अभवत् ततः अहम् पूर्वम् आगतः} %2-72-9

\twolineshloka
{राज वाल्य हरैः दूतैअः त्वर्यमाणो अहम् आगतः}
{यद् अहम् प्रष्टुम् इच्चामि तत् अम्बा वक्तुम् अर्हसि} %2-72-10

\twolineshloka
{शून्यो अयम् शयनीयः ते पर्यन्को हेम भूषितः}
{न च अयम् इक्ष्वाकु जनः प्रहृष्टः प्रतिभाति मे} %2-72-11

\twolineshloka
{राजा भवति भूयिष्ठ्गम् इह अम्बाया निवेशने}
{तम् अहम् न अद्य पश्यामि द्रष्टुम् इच्चन्न् इह आगतः} %2-72-12

\twolineshloka
{पितुर् ग्रहीष्ये चरणौ तम् मम आख्याहि पृच्चतः}
{आहोस्विद् अम्ब ज्येष्ठायाः कौसल्याया निवेशने} %2-72-13

\twolineshloka
{तम् प्रत्युवाच कैकेयी प्रियवद् घोरम् अप्रियम्}
{अजानन्तम् प्रजानन्ती राज्य लोभेन मोहिता} %2-72-14

\twolineshloka
{या गतिः सर्व भूतानाम् ताम् गतिम् ते पिता गतः}
{राजा महात्मा तेजस्वी यायजूकः सताम् गतिः} %2-72-15

\twolineshloka
{तत् श्रुत्वा भरतः वाक्यम् धर्म अभिजनवान् शुचिः}
{पपात सहसा भूमौ पितृ शोक बल अर्दितः} %2-72-16

\twolineshloka
{हा हातोऽस्मीति कृपणाम् दीनाम् वाचमुदीरयन्}
{निपपात महाबाहुर्बाहु विक्षिप्य वीर्यवान्} %2-72-17

\twolineshloka
{ततः शोकेन सम्वीतः पितुर् मरण दुह्खितः}
{विललाप महा तेजा भ्रान्त आकुलित चेतनः} %2-72-18

\twolineshloka
{एतत् सुरुचिरम् भाति पितुर् मे शयनम् पुरा}
{शशिनेवामलम् रात्रौ गगनम् तोयदात्यये} %2-72-19

\twolineshloka
{तत् इदम् न विभाति अद्य विहीनम् तेन धीमता}
{व्योमेव श्शिना हीनमप्भुष्क इव सागरः} %2-72-20

\twolineshloka
{बाष्पमुत्सृज्य कण्ठे स्वात्मना परिपीडितः}
{आच्चाद्य वदनम् श्रीमद्वस्त्रेण जयताम् वरः} %2-72-21

\twolineshloka
{तम् आर्तम् देव सम्काशम् समीक्ष्य पतितम् भुवि}
{निकृत्तमिव सालस्य स्कन्धम् परशुना वने} %2-72-22

\twolineshloka
{मत्तमातङ्गसम्काशम् चन्द्रार्कसदृशम् भुवः}
{उत्थापयित्वा शोक आर्तम् वचनम् च इदम् अब्रवीत्} %2-72-23

\twolineshloka
{उत्तिष्ठ उत्तिष्ठ किम् शेषे राज पुत्र महा यशः}
{त्वद् विधा न हि शोचन्ति सन्तः सदसि सम्मताः} %2-72-24

\twolineshloka
{दानयज्ञाधिकारा हि शीलश्रुतिवचोनुगा}
{बुद्धिस्ते बुद्धिसम्पन्न प्रभेवार्कस्य मन्दिरे} %2-72-25

\twolineshloka
{स रुदत्या चिरम् कालम् भूमौ विपरिवृत्य च}
{जननीम् प्रत्युवाच इदम् शोकैः बहुभिर् आवृतः} %2-72-26

\twolineshloka
{अभिषेक्ष्यति रामम् तु राजा यज्ञम् नु यक्ष्यति}
{इति अहम् कृत सम्कल्पो हृष्टः यात्राम् अयासिषम्} %2-72-27

\twolineshloka
{तत् इदम् हि अन्यथा भूतम् व्यवदीर्णम् मनो मम}
{पितरम् यो न पश्यामि नित्यम् प्रिय हिते रतम्} %2-72-28

\twolineshloka
{अम्ब केन अत्यगात् राजा व्याधिना मय्य् अनागते}
{धन्या राम आदयः सर्वे यैः पिता सम्स्कृतः स्वयम्} %2-72-29

\twolineshloka
{न नूनम् माम् महा राजः प्राप्तम् जानाति कीर्तिमान्}
{उपजिघ्रेद्द् हि माम् मूर्ध्नि तातः सम्नम्य सत्वरम्} %2-72-30

\twolineshloka
{क्व स पाणिः सुख स्पर्शः तातस्य अक्लिष्ट कर्मणः}
{येन माम् रजसा ध्वस्तम् अभीक्ष्णम् परिमार्जति} %2-72-31

\twolineshloka
{यो मे भ्राता पिता बन्धुर् यस्य दासो अस्मि धीमतः}
{तस्य माम् शीघ्रम् आख्याहि रामस्य अक्लिष्ट कर्मणः} %2-72-32

\twolineshloka
{पिता हि भवति ज्येष्ठो धर्मम् आर्यस्य जानतः}
{तस्य पादौ ग्रहीष्यामि स हि इदानीम् गतिर् मम} %2-72-33

\twolineshloka
{धर्मविद्धर्मनित्यश्च सत्यसन्धो दृढव्रतः}
{आर्ये किम् अब्रवीद् राजा पिता मे सत्य विक्रमः} %2-72-34

\twolineshloka
{पश्चिमम् साधु सम्देशम् इच्चामि श्रोतुम् आत्मनः}
{इति पृष्टा यथा तत्त्वम् कैकेयी वाक्यम् अब्रवीत्} %2-72-35

\twolineshloka
{राम इति राजा विलपन् हा सीते लक्ष्मण इति च}
{स महात्मा परम् लोकम् गतः गतिमताम् वरः} %2-72-36

\twolineshloka
{इमाम् तु पश्चिमाम् वाचम् व्याजहार पिता तव}
{काल धर्म परिक्षिप्तः पाशैः इव महा गजः} %2-72-37

\twolineshloka
{सिद्ध अर्थाः तु नरा रामम् आगतम् सीतया सह}
{लक्ष्मणम् च महा बाहुम् द्रक्ष्यन्ति पुनर् आगतम्} %2-72-38

\twolineshloka
{तत् श्रुत्वा विषसाद एव द्वितीया प्रिय शम्सनात्}
{विषण्ण वदनो भूत्वा भूयः पप्रच्च मातरम्} %2-72-39

\twolineshloka
{क्व च इदानीम् स धर्म आत्मा कौसल्य आनन्द वर्धनः}
{लक्ष्मणेन सह भ्रात्रा सीतया च समम् गतः} %2-72-40

\twolineshloka
{तथा पृष्टा यथा तत्त्वम् आख्यातुम् उपचक्रमे}
{माता अस्य युगपद् वाक्यम् विप्रियम् प्रिय शन्कया} %2-72-41

\twolineshloka
{स हि राज सुतः पुत्र चीर वासा महा वनम्}
{दण्डकान् सह वैदेह्या लक्ष्मण अनुचरः गतः} %2-72-42

\twolineshloka
{तत् श्रुत्वा भरतः त्रस्तः भ्रातुः चारित्र शन्कया}
{स्वस्य वम्शस्य माहात्म्यात् प्रष्टुम् समुपचक्रमे} %2-72-43

\twolineshloka
{कच्चिन् न ब्राह्मण वधम् हृतम् रामेण कस्यचित्}
{कच्चिन् न आढ्यो दरिद्रः वा तेन अपापो विहिम्सितः} %2-72-44

\twolineshloka
{कच्चिन् न पर दारान् वा राज पुत्रः अभिमन्यते}
{कस्मात् स दण्डक अरण्ये भ्रूणहा इव विवासितः} %2-72-45

\twolineshloka
{अथ अस्य चपला माता तत् स्व कर्म यथा तथम्}
{तेन एव स्त्री स्वभावेन व्याहर्तुम् उपचक्रमे} %2-72-46

\twolineshloka
{एवमुक्ता तु कैकेयी भरतेन महात्मना}
{उवाच वचनम् हृष्टा मूढा पण्डितमानिनी} %2-72-47

\twolineshloka
{न ब्राह्मण धनम् किम्चिद्द् हृतम् रामेण कस्यचित्}
{कश्चिन् न आढ्यो दरिद्रः वा तेन अपापो विहिम्सितः} %2-72-48

\twolineshloka
{न रामः पर दारामः च चक्षुर्भ्याम् अपि पश्यति}
{मया तु पुत्र श्रुत्वा एव रामस्य एव अभिषेचनम्} %2-72-49

\twolineshloka
{याचितः ते पिता राज्यम् रामस्य च विवासनम्}
{स स्व वृत्तिम् समास्थाय पिता ते तत् तथा अकरोत्} %2-72-50

\twolineshloka
{रामः च सह सौमित्रिः प्रेषितः सह सीतया}
{तम् अपश्यन् प्रियम् पुत्रम् मही पालो महा यशाः} %2-72-51

\twolineshloka
{पुत्र शोक परिद्यूनः पन्चत्वम् उपपेदिवान्}
{त्वया तु इदानीम् धर्मज्ञ राजत्वम् अवलम्ब्यताम्} %2-72-52

\twolineshloka
{त्वत् कृते हि मया सर्वम् इदम् एवम् विधम् कृतम्}
{मा शोकम् मा च सम्तापम् धैर्यमाश्रय पुत्रक} %2-72-53

\twolineshloka
{त्वदधीना हि नगरी राज्यम् चैतदनामयम्}
{तत् पुत्र शीघ्रम् विधिना विधिज्ञैः}
{वसिष्ठ मुख्यैः सहितः द्विज इन्द्रैः}
{सम्काल्य राजानम् अदीन सत्त्वम्}
{आत्मानम् उर्व्याम् अभिषेचयस्व} %2-72-54


॥इत्यार्षे श्रीमद्रामायणे वाल्मीकीये आदिकाव्ये अयोध्याकाण्डे भरतसन्तापः नाम द्विसप्ततितमः सर्गः ॥२-७२॥
