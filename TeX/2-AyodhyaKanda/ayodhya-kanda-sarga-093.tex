\sect{त्रिनवतितमः सर्गः — चित्रकूटवनप्रेक्षणम्}

\twolineshloka
{तया महत्या यायिन्या ध्वजिन्या वनवासिनः}
{अर्द्दिता यूथपा मत्ताः सयूथाः सम्प्रदुद्रुवुः} %2-93-1

\twolineshloka
{ऋक्षाः पृषतसङ्घाश्च रुरवश्च समतन्तः}
{दृश्यन्ते वनराजीषु गिरिष्वपि नदीषु च} %2-93-2

\twolineshloka
{स सम्प्रतस्थे धर्मात्मा प्रीतो दशरथात्मजः}
{वृतो महत्या नादिन्या सेनया चतुरङ्गया} %2-93-3

\twolineshloka
{सागरौघनिभा सेना भरतस्य महात्मनः}
{महीं सञ्छादयामास प्रावृषि द्यामिवाम्बुदः} %2-93-4

\onelineshloka
{चिरकालमित्यनेन कदाचिल्लक्ष्यत इति गम्यते} %2-93-5

\twolineshloka
{स यात्वा दूरमध्वानं सुपरिश्रान्तवाहनः}
{उवाच भरतः श्रीमान् वसिष्ठं मन्त्रिणां वरम्} %2-93-6

\twolineshloka
{यादृशं लक्ष्यते रूपं यथा चैव श्रुतं मया}
{व्यक्तं प्राप्ताः स्म तं देशं भरद्वाजो यमब्रवीत्} %2-93-7

\twolineshloka
{अयं गिरिश्चित्रकूट इयं मन्दाकिनी नदी}
{एतत् प्रकाशते दूरान्नीलमेघनिभं वनम्} %2-93-8

\twolineshloka
{गिरेः सानूनि रम्याणि चित्रकूटस्य सम्प्रति}
{वारणैरवमृद्यन्ते मामकैः पर्वतोपमैः} %2-93-9

\twolineshloka
{मुञ्चन्ति कुसुमान्येते नगाः पर्वतसानुषु}
{नीला इवातपापाये तोयं तोयधरा घनाः} %2-93-10

\twolineshloka
{किन्नराचरितं देशं पश्य शत्रुघ्न पर्वतम्}
{मृगैः समन्तादाकीर्णं मकरैरिव सागरम्} %2-93-11

\twolineshloka
{एते मृगगणा भान्ति शीघ्रवेगाः प्रचोदिताः}
{वायुप्रविद्धा शरदि मेघराजिरिवाम्बरे} %2-93-12

\twolineshloka
{कुर्वन्ति कुसुमापीडान् शिरस्सु सुरभीनमी}
{मेघप्रकाशैः फलकैर्दाक्षिणात्या यथा नराः} %2-93-13

\twolineshloka
{निष्कूजमिव भूत्वेदं वनं घोरप्रदर्शनम्}
{अयोध्येव जनाकीर्णा सम्प्रति प्रतिभाति मा} %2-93-14

\twolineshloka
{खुरैरुदीरितो रेणुर्दिवं प्रच्छाद्य तिष्ठति}
{तं वहत्यनिलः शीघ्रं कुर्वन्निव मम प्रियम्} %2-93-15

\twolineshloka
{स्यन्दनांस्तुरगोपेतान् सूतमुख्यैरधिष्ठितान्}
{एतान् सम्पततः शीघ्रं पश्य शत्रुघ्न कानने} %2-93-16

\twolineshloka
{एतान् वित्रासितान् पश्य बर्हिणः प्रियदर्शनान्}
{एतमाविशतः शीघ्रमधिवासं पतत्ऺित्रणः} %2-93-17

\twolineshloka
{अतिमात्रमयं देशो मनोज्ञः प्रतिभाति मा}
{तापसानां निवासोऽयं व्यक्तं स्वर्गपथो यथा} %2-93-18

\twolineshloka
{मृगा मृगीभिः सहिता बहवः पृषता वने}
{मनोज्ञरूपा लक्ष्यन्ते कुसुमैरिव चित्रिताः} %2-93-19

\twolineshloka
{साधुसैन्याः प्रतिष्ठन्तां विचिन्वन्तु च कानने}
{यथा तौ पुरुषव्याघ्रौ दृश्येते रामलक्ष्मणौ} %2-93-20

\twolineshloka
{भरतस्य वचः श्रुत्वा पुरुषाः शस्त्रपाणयः}
{विविशुस्तद्वनं शूरा धूमं च ददृशुस्ततः} %2-93-21

\twolineshloka
{ते समालोक्य धूमाग्रमूचुर्भरतमागताः}
{नामनुष्ये भवत्याग्निर्व्यक्तमत्रैव राघवौ} %2-93-22

\twolineshloka
{अथ नात्र नरव्याघ्रौ राजपुत्रौ परन्तपौ}
{मन्ये रामोपमाः सन्ति व्यक्तमत्र तपस्विनः} %2-93-23

\twolineshloka
{तच्छ्रुत्वा भरतस्तेषां वचनं साधुसम्मतम्}
{सैन्यानुवाच सर्वांस्तानमित्रबलमर्दनः} %2-93-24

\twolineshloka
{यत्ता भवन्तस्तिष्ठन्तु नेतो गन्तव्यमग्रतः}
{अहमेव गमिष्यामि सुमन्त्रो गुरुरेव च} %2-93-25

\twolineshloka
{एवमुक्तास्ततः सर्वे तत्र तस्थुः समन्ततः}
{भरतो यत्र धूमाग्रं तत्र दृष्टिं समादधात्} %2-93-26

\twolineshloka
{व्यवस्थिता या भरतेन सा चमूर्निरीक्षमाणापि च धूममग्रतः}
{बभूव हृष्टा नचिरेण जानती प्रियस्य रामस्य समागमं तदा} %2-93-27


॥इत्यार्षे श्रीमद्रामायणे वाल्मीकीये आदिकाव्ये अयोध्याकाण्डे चित्रकूटवनप्रेक्षणम् नाम त्रिनवतितमः सर्गः ॥२-९३॥
