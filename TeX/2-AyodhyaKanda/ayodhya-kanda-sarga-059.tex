\sect{एकोनषष्ठितमः सर्गः — दशरथविलापः}

\twolineshloka
{मम तु अश्वा निवृत्तस्य न प्रावर्तन्त वर्त्मनि}
{उष्णम् अश्रु विमुन्चन्तः रामे सम्प्रस्थिते वनम्} %2-59-1

\twolineshloka
{उभाभ्याम् राज पुत्राभ्याम् अथ कृत्वा अहम् ज्ञलिम्}
{प्रस्थितः रथम् आस्थाय तत् दुह्खम् अपि धारयन्} %2-59-2

\twolineshloka
{गुहा इव सार्धम् तत्र एव स्थितः अस्मि दिवसान् बहून्}
{आशया यदि माम् रामः पुनः शब्दापयेद् इति} %2-59-3

\twolineshloka
{विषये ते महा राज माम व्यसन कर्शिताः}
{अपि वृक्षाः परिम्लानः सपुष्प अन्कुर कोरकाः} %2-59-4

\twolineshloka
{उपतप्तोदका नद्यः पल्वलानि सराम्सि च}
{परिष्कुपलाशानि वनान्युपवनानि च} %2-59-5

\twolineshloka
{न च सर्पन्ति सत्त्वानि व्याला न प्रसरन्ति च}
{राम शोक अभिभूतम् तन् निष्कूजम् अभवद् वनम्} %2-59-6

\twolineshloka
{लीन पुष्कर पत्राः च नर इन्द्र कलुष उदकाः}
{सम्तप्त पद्माः पद्मिन्यो लीन मीन विहम्गमाः} %2-59-7

\twolineshloka
{जलजानि च पुष्पाणि माल्यानि स्थलजानि च}
{न अद्य भान्ति अल्प गन्धीनि फलानि च यथा पुरम्} %2-59-8

\twolineshloka
{अत्रोद्यानानि शून्यानि प्रलीनविहगानि च}
{न चाभिरामानारामान् पश्यामि मनुजर्षभ} %2-59-9

\twolineshloka
{प्रविशन्तम् अयोध्याम् माम् न कश्चित् अभिनन्दति}
{नरा रामम् अपश्यन्तः निह्श्वसन्ति मुहुर् मुहुः} %2-59-10

\twolineshloka
{देव राजरथम् दृष्ट्वा विना राममिहागतम्}
{दुःखादश्रुमुखः सर्वो राजमार्गगतो जनः} %2-59-11

\twolineshloka
{हर्म्यैः विमानैः प्रासादैः अवेक्ष्य रथम् आगतम्}
{हाहा कार कृता नार्यो राम अदर्शन कर्शिताः} %2-59-12

\twolineshloka
{आयतैः विमलैः नेत्रैः अश्रु वेग परिप्लुतैः}
{अन्योन्यम् अभिवीक्षन्ते व्यक्तम् आर्ततराः स्त्रियः} %2-59-13

\twolineshloka
{न अमित्राणाम् न मित्राणाम् उदासीन जनस्य च}
{अहम् आर्ततया कम्चित् विशेषम् न उपलक्षये} %2-59-14

\twolineshloka
{अप्रहृष्ट मनुष्या च दीन नाग तुरम्गमा}
{आर्त स्वर परिम्लाना विनिह्श्वसित निह्स्वना} %2-59-15

\twolineshloka
{निरानन्दा महा राज राम प्रव्राजन आतुला}
{कौसल्या पुत्र हीना इवायोध्या प्रतिभाति मा मा} %2-59-16

\twolineshloka
{सूतस्य वचनम् श्रुत्वा वाचा परम दीनया}
{बाष्प उपहतया राजा तम् सूतम् इदम् अब्रवीत्} %2-59-17

\twolineshloka
{कैकेय्या विनियुक्तेन पाप अभिजन भावया}
{मया न मन्त्र कुशलैः वृद्धैः सह समर्थितम्} %2-59-18

\twolineshloka
{न सुहृद्भिर् न च अमात्यैः मन्त्रयित्वा न नैगमैः}
{मया अयम् अर्थः सम्मोहात् स्त्री हेतोह् सहसा कृतः} %2-59-19

\twolineshloka
{भवितव्यतया नूनम् इदम् वा व्यसनम् महत्}
{कुलस्य अस्य विनाशाय प्राप्तम् सूत यदृच्चया} %2-59-20

\twolineshloka
{सूत यद्य् अस्ति ते किम्चिन् मया अपि सुकृतम् कृतम्}
{त्वम् प्रापय आशु माम् रामम् प्राणाः सम्त्वरयन्ति माम्} %2-59-21

\twolineshloka
{यद् यद् या अपि मम एव आज्ञा निवर्तयतु राघवम्}
{न शक्ष्यामि विना राम मुहूर्तम् अपि जीवितुम्} %2-59-22

\twolineshloka
{अथवा अपि महा बाहुर् गतः दूरम् भविष्यति}
{माम् एव रथम् आरोप्य शीघ्रम् रामाय दर्शय} %2-59-23

\twolineshloka
{वृत्त दम्ष्ट्रः महा इष्वासः क्व असौ लक्ष्मण पूर्वजः}
{यदि जीवामि साध्व् एनम् पश्येयम् सह सीतया} %2-59-24

\twolineshloka
{लोहित अक्षम् महा बाहुम् आमुक्त मणि कुण्डलम्}
{रामम् यदि न पश्यामि गमिष्यामि यम क्षयम्} %2-59-25

\twolineshloka
{अतः नु किम् दुह्खतरम् यो अहम् इक्ष्वाकु नन्दनम्}
{इमाम् अवस्थाम् आपन्नो न इह पश्यामि राघवम्} %2-59-26

\twolineshloka
{हा राम राम अनुज हा हा वैदेहि तपस्विनी}
{न माम् जानीत दुह्खेन म्रियमाणम् अनाथवत्} %2-59-27

\twolineshloka
{स तेन राजा दुःखेन भृशमर्पितचेतनः}
{अवगाढः सुदुष्पारम् शोकसागमब्रवीत्} %2-59-28

\twolineshloka
{रामशोकमहाभोगः सीताविरहपारगः}
{श्वसितोर्मिमहावर्तो बाष्पफेनजलाविलः} %2-59-29

\twolineshloka
{बाहुविक्षेपमीनौघो विक्रन्दितमहास्वनः}
{प्रकीर्णकेशशैवालः कैकेयीबडबामुखः} %2-59-30

\twolineshloka
{ममाश्रुवेगप्रभवः कुब्जावाक्यमहाग्रहः}
{वरवेलो नृशम्साया रामप्रव्राजनायतः} %2-59-31

\twolineshloka
{यस्मिन् बत निमग्नोऽहम् कौसल्ये राघवम् विना}
{दुस्तरः जीवता देवि मया अयम् शोक सागरः} %2-59-32

\twolineshloka
{अशोभनम् यो अहम् इह अद्य राघवम्}
{दिदृक्षमाणो न लभे सलक्ष्मणम्इति इव राजा विलपन् महा यहाशःपपात तूर्णम् शयने स मूर्चितः} %2-59-33

\fourlineindentedshloka
{इति विलपति पार्थिवे प्रनष्टे}
{करुणतरम् द्विगुणम् च राम हेतोः}
{वचनम् अनुनिशम्य तस्य देवी}
{भयम् अगमत् पुनर् एव राम माता} %2-59-34


॥इत्यार्षे श्रीमद्रामायणे वाल्मीकीये आदिकाव्ये अयोध्याकाण्डे दशरथविलापः नाम एकोनषष्ठितमः सर्गः ॥२-५९॥
