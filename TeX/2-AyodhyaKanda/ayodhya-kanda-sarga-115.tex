\sect{पञ्चदशाधिकशततमः सर्गः — नन्दिग्रामनिवासः}

\twolineshloka
{ततो निक्षिप्य मातऽः स अयोध्यायां दृढव्रतः}
{भरतः शोकसन्तप्तो गुरूनिदमथाब्रवीत्} %2-115-1

\twolineshloka
{नन्दिग्रामं गमिष्यामि सर्वानामन्त्रयेऽद्य वः}
{तत्र दुःखमिदं सर्वं सहिष्ये राघवं विना} %2-115-2

\twolineshloka
{गतश्च हि दिवं राजा वनस्थश्च गुरुर्मम}
{रामं प्रतीक्षे राज्याय स हि राजा महायशाः} %2-115-3

\twolineshloka
{एतच्छ्रुत्वा शुभं वाक्यं भरतस्य महात्मनः}
{अब्रुवन् मन्त्रिणः सर्वे वसिष्ठश्च पुरोहितः} %2-115-4

\twolineshloka
{सुभृशं श्लाघनीयं च यदुक्तं भरत त्वया}
{वचनं भ्रातृवात्सल्यादनुरूपं तवैव तत्} %2-115-5

\twolineshloka
{नित्यं ते बन्धुलुब्धस्य तिष्ठतो भ्रातृसौहृदे}
{आर्यमार्गं प्रपन्नस्य नानुमन्येत कः पुमान्} %2-115-6

\twolineshloka
{मन्त्रिणां वचनं श्रुत्वा यथाभिलषितं प्रियम्}
{अब्रवीत्सारथिं वाक्यं रथो मे युज्यतामिति} %2-115-7

\twolineshloka
{प्रहृष्टवदनः सर्वा मातऽः समभिवाद्य सः}
{आरुरोह रथं श्रीमान् शत्रुघ्नेन समन्वितः} %2-115-8

\twolineshloka
{आरुह्य च रथं शीघ्रं शत्रुघ्नभरतावुभौ}
{ययतुः परमप्रीतौ वृतौ मन्त्रिपुरोहितैः} %2-115-9

\twolineshloka
{अग्रतो गुरवस्तत्र वसिष्ठप्रमुखा द्विजाः}
{प्रययुः प्राङ्मुखाः सर्वे नन्दिग्रामो यतोऽभवत्} %2-115-10

\twolineshloka
{बलं च तदनाहूतं गजाश्वरथसङ्कुलम्}
{प्रययौ भरते याते सर्वे च पुरवासिनः} %2-115-11

\twolineshloka
{रथस्थः स हि धर्मात्मा भरतो भ्रातृवत्सलः}
{नन्दिग्रामं ययौ तूर्णं शिरस्याधाय पादुके} %2-115-12

\twolineshloka
{ततस्तु भरतः क्षिप्रं नन्दिग्रामं प्रविश्य सः}
{अवतीर्य्य रथात्तूर्णं गुरूनिदमुवाच ह} %2-115-13

\twolineshloka
{एतद्राज्यं मम भ्रात्रा दत्तं संन्यासवत् स्वयम्}
{योगक्षेमवहे चेमे पादुके हेमभूषिते} %2-115-14

\twolineshloka
{भरतः शिरसा कृत्वा संन्यासं पादुके ततः}
{अब्रवीद्दुःखसन्तप्तः सर्वं प्रकृतिमण्डलम्} %2-115-15

\twolineshloka
{छत्रं धारयत क्षिप्रमार्यपादाविमौ मतौ}
{आभ्यां राज्ये स्थितो धर्मः पादुकाभ्यां गुरोर्मम} %2-115-16

\twolineshloka
{भ्रात्रा हि मयि संन्यासो निक्षिप्तः सौहृदादयम्}
{तमिमं पालयिष्यामि राघवागमनं प्रति} %2-115-17

\twolineshloka
{क्षिप्रं संयोजयित्वा तु राघवस्य पुनः स्वयम्}
{चरणौ तौ तु रामस्य द्रक्ष्यामि सहपादुकौ} %2-115-18

\twolineshloka
{ततो निक्षिप्तभारोऽहं राघवेण समागतः}
{निवेद्य गुरवे राज्यं भजिष्ये गुरुवृत्तिताम्} %2-115-19

\twolineshloka
{राघवाय च संन्यासं दत्त्वे मे वरपादुके}
{राज्यं चेदमयोध्यां च धूतपापो भवामि च} %2-115-20

\twolineshloka
{अभिषिक्ते तु काकुत्स्थे प्रहृष्टमुदिते जने}
{प्रीतिर्मम यशश्चैव भवेद्राज्याच्चतुर्गुणम्} %2-115-21

\twolineshloka
{एवं तु विलपन् दीनो भरतः स महायशाः}
{नन्दिग्रामेऽकरोद्राज्यं दुःखितो मन्त्रिभिः सह} %2-115-22

\twolineshloka
{स वल्कलजटाधारी मुनिवेषधरः प्रभुः}
{नन्दिग्रामेऽवसद्वीरः ससैन्यो भरतस्तदा} %2-115-23

\twolineshloka
{रामागमनमाकाङ्क्षन् भरतो भ्रातृवत्सलः}
{भ्रातुर्वचनकारी च प्रतिज्ञापारगस्तथा} %2-115-24

\twolineshloka
{पादुके त्वभिषिच्याथ नन्द्रिग्रामेऽवसत्तदा}
{भरतः शासनं सर्वं पादुकाभ्यां न्यवेदयत्} %2-115-25

\twolineshloka
{ततस्तु भरतः श्रीमानभिषिच्यार्य्यपादुके}
{तदधीनस्तदा राज्यं कारयामास सर्वदा} %2-115-26

\twolineshloka
{तदा हि यत्कार्य्यमुपैति किञ्चिदुपायनं चोपहृतं महार्हम्}
{स पादुकाभ्यां प्रथमं निवेद्य चकार पश्चाद्भरतो यथावत्} %2-115-27


॥इत्यार्षे श्रीमद्रामायणे वाल्मीकीये आदिकाव्ये अयोध्याकाण्डे नन्दिग्रामनिवासः नाम पञ्चदशाधिकशततमः सर्गः ॥२-११५॥
