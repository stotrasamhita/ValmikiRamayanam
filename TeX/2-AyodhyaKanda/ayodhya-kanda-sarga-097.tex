\sect{सप्तनवतितमः सर्गः — भरतगुणप्रशंसा}

\twolineshloka
{सुसंरब्धं तु सौमित्रिं लक्ष्मणं क्रोधमूर्च्छितम्}
{रामस्तु परिसान्त्व्याथ वचनं चेदमब्रवीत्} %2-97-1

\twolineshloka
{किमत्र धनुषा कार्यमसिना वा सचर्मणा}
{महेष्वासे महाप्राज्ञे भरते स्वयमागते} %2-97-2

\twolineshloka
{पितुः सत्यं प्रतिश्रुत्य हत्वा भरतमागतम्}
{किं करिष्यामि राज्येन सापवादेन लक्ष्मण} %2-97-3

\twolineshloka
{यद्द्रव्यं बान्धवानां वा मित्राणां वा क्षये भवेत्}
{नाहं तत् प्रतिगृह्णीयां भक्षान् विषकृतानिव} %2-97-4

\twolineshloka
{धर्ममर्थं च कामं च पृथिवीं चापि लक्ष्मण}
{इच्छामि भवतामर्थे एतत् प्रतिश्रृणोमि ते} %2-97-5

\twolineshloka
{भ्रातऽणां सङ्ग्रहार्थं च सुखार्थं चापि लक्ष्मण}
{राज्यमप्यहमिच्छामि सत्येनायुधमालभे} %2-97-6

\twolineshloka
{नेयं मम मही सौम्य दुर्ल्लभा सागराम्बरा}
{नहीच्छेयमधर्मेण शक्रत्वमपि लक्ष्मण} %2-97-7

\twolineshloka
{यद्विना भरतं त्वां च शत्रुघ्नं चापि मानद}
{भवेन्मम सुखं किञ्चिद्भस्म तत् कुरुतां शिखी} %2-97-8

\twolineshloka
{मन्येऽहमागतोऽयोध्यां भरतो भ्रातृवत्सलः}
{मम प्राणात् प्रियतरः कुलधर्ममनुस्मरन्} %2-97-9

\twolineshloka
{श्रुत्वा प्रव्राजितं मां हि जटावल्कलधारिणम्}
{जानक्या सहितं वीर त्वया च पुरुषर्षभ} %2-97-10

\twolineshloka
{स्नेहेनाक्रान्तहृदयः शोकेनाकुलितेन्द्रियः}
{द्रष्टुमभ्यागतो ह्येष भरतो नान्यथा गतः} %2-97-11

\twolineshloka
{अम्बां च कैकयीं रुष्य परुषं चाप्रियं वदन्}
{प्रसाद्य पितरं श्रीमान् राज्यं मे दातुमागतः} %2-97-12

\twolineshloka
{प्राप्तकालं यदेषोऽस्मान् भरतो द्रष्टुमिच्छति}
{अस्मासु मनसाप्येष नाप्रियं किञ्चिदाचरेत्} %2-97-13

\twolineshloka
{विप्रियं कृतपूर्वं ते भरतेन कदा नु किम्}
{ईदृशं वा भयं तेऽद्य भरतं योऽत्र शङ्कसे} %2-97-14

\twolineshloka
{नहि ते निष्ठुरं वाच्यो भरतो नाप्रियं वचः}
{अहं ह्यप्रियमुक्तः स्यां भरतस्याप्रिये कृते} %2-97-15

\twolineshloka
{कथं नु पुत्राः पितरं हन्युः कस्याञ्चिदापदि}
{भ्राता वा भ्रातरं हन्यात् सौमित्रे प्राणमात्मनः} %2-97-16

\twolineshloka
{यदि राज्यस्य हेतोस्त्वमिमां वाचं प्रभाषसे}
{वक्ष्यामि भरतं दृष्ट्वा राज्यमस्मै प्रदीयताम्} %2-97-17

\twolineshloka
{उच्यमानोऽपि भरतो मया लक्ष्मण तत्त्वतः}
{राज्यमस्मै प्रयच्छेति बाढमित्येव वक्ष्यति} %2-97-18

\twolineshloka
{तथोक्तो धर्मशीलेन भ्रात्रा तस्य हिते रतः}
{लक्ष्मणः प्रविवेशेव स्वानि गात्राणि लज्जया} %2-97-19

\twolineshloka
{तद्वाक्यं लक्ष्मणः श्रुत्वा व्रीडितः प्रत्युवाच ह}
{त्वां मन्ये द्रष्टुमायातः पिता दशरथः स्वयम्} %2-97-20

\twolineshloka
{व्रीडितं लक्ष्मणं दृष्ट्वा राघवः प्रत्युवाच ह}
{एष मन्ये महाबाहुरिहास्मान् द्रष्टुमागतः} %2-97-21

\twolineshloka
{अथवा नौ ध्रुवं मन्ये मन्यमानः सुखोचितौ}
{वनवासमनुध्याय गृहाय प्रतिनेष्यति} %2-97-22

\twolineshloka
{इमां वाप्येष वैदेहीमत्यन्तसुखसेविनीम्}
{पिता मे राघवः श्रीमान् वनादादाय यास्यति} %2-97-23

\twolineshloka
{एतौ तौ सम्प्रकाशेते गोत्रवन्तौ मनोरमौ}
{वायुवेगसमौ वीर जवनौ तुरगोत्तमौ} %2-97-24

\twolineshloka
{स एष सुमहाकायः कम्पते वाहिनीमुखे}
{नागः शत्रुञ्जयो नाम वृद्धस्तातस्य धीमतः} %2-97-25

\twolineshloka
{न तु पश्यामि तच्छत्रं पाण्डरं लोकसत्कृतम्}
{पितुर्दिव्यं महाबाहो संशयो भवतीह मे} %2-97-26

\onelineshloka
{प्रथममर्धमुत्तरार्धेन योजनीयम्} %2-97-27

\twolineshloka
{अवतीर्य्य तु सालाग्रात्तस्मात्स समितिञ्जयः}
{लक्ष्मणः प्राञ्जलिर्भूत्वा तस्थौ रामस्य पार्श्वतः} %2-97-28

\twolineshloka
{भरतेनापि सन्दिष्टा सम्मर्दो न भवेदिति}
{समन्तात्तस्य शैलस्य सेना वासमकल्पयत्} %2-97-29

\twolineshloka
{अध्यर्द्धमिक्ष्वाकुचमूर्योजनं पर्वतस्य सा}
{पार्श्वे न्यविशदावृत्य गजवाजिरथाकुला} %2-97-30

\twolineshloka
{सा चित्रकूटे भरतेन सेना धर्मं पुरस्कृत्य विधूय दर्प्पम्}
{प्रसादनार्थं रघुनन्दनस्य विराजते नीतिमता प्रणीता} %2-97-31


॥इत्यार्षे श्रीमद्रामायणे वाल्मीकीये आदिकाव्ये अयोध्याकाण्डे भरतगुणप्रशंसा नाम सप्तनवतितमः सर्गः ॥२-९७॥
