\sect{दशमः सर्गः — कैकेय्यनुनयः}

\twolineshloka
{विदर्शिता यदा देवी कुब्जया पापया भृशम्}
{तदा शेते स्म सा भूमौ दिग्धविद्धेव किन्नरी} %2-10-1

\twolineshloka
{निश्चित्य मनसा कृत्यं सा सम्यगिति भामिनी}
{मन्थरायै शनैः सर्वमाचचक्षे विचक्षणा} %2-10-2

\twolineshloka
{सा दीना निश्चयं कृत्वा मन्थरावाक्यमोहिता}
{नागकन्येव निःश्वस्य दीर्घमुष्णं च भामिनी} %2-10-3

\twolineshloka
{मुहूर्तं चिन्तयामास मार्गमात्मसुखावहम्}
{सा सुहृच्चार्थकामा च तं निशम्य विनिश्चयम्} %2-10-4

\twolineshloka
{बभूव परमप्रीता सिद्धिं प्राप्येव मन्थरा}
{अथ सा रुषिता देवी सम्यक् कृत्वा विनिश्चयम्} %2-10-5

\twolineshloka
{संविवेशाबला भूमौ निवेश्य भ्रुकुटिं मुखे}
{ततश्चित्राणि माल्यानि दिव्यान्याभरणानि च} %2-10-6

\twolineshloka
{अपविद्धानि कैकेय्या तानि भूमिं प्रपेदिरे}
{तया तान्य् अपविद्धानि माल्यान्य् आभरणानि च} %2-10-7

\twolineshloka
{अशोभयन्त वसुधां नक्षत्राणि यथा नभः}
{क्रोधागारे च पतिता सा बभौ मलिनाम्बरा} %2-10-8

\twolineshloka
{एकवेणीं दृढां बद्ध्वा गतसत्त्वेव किन्नरी}
{आज्ञाप्य तु महाराजो राघवस्याभिषेचनम्} %2-10-9

\twolineshloka
{उपस्थानमनुज्ञाप्य प्रविवेश निवेशनम्}
{अद्य रामाभिषेको वै प्रसिद्ध इति जज्ञिवान्} %2-10-10

\twolineshloka
{प्रियार्हां प्रियमाख्यातुं विवेशान्तःपुरं वशी}
{स कैकेय्या गृहं श्रेष्ठं प्रविवेश महायशाः} %2-10-11

\twolineshloka
{पाण्डुराभ्रम् इवाकाशं राहुयुक्तं निशाकरः}
{शुकबर्हिसमायुक्तं क्रौञ्चहंसरुतायुतम्} %2-10-12

\twolineshloka
{वादित्ररवसङ्घुष्टं कुब्जावामनिकायुतम्}
{लतागृहैश्चित्रगृहैश्चम्पकाशोकशोभितैः} %2-10-13

\twolineshloka
{दान्तराजतसौवर्णवेदिकाभिः समायुतम्}
{नित्यपुष्पफलैर्वृक्षैर्वापीभिरुपशोभितम्} %2-10-14

\twolineshloka
{दान्तराजतसौवर्णैः संवृतं परमासनैः}
{विविधैरन्नपानैश्च भक्ष्यैश्च विविधैरपि} %2-10-15

\twolineshloka
{उपपन्नं महार्हैश्च भूषणैस्त्रिदिवोपमम्}
{स प्रविश्य महाराजः स्वमन्तःपुरमृद्धिमत्} %2-10-16

\twolineshloka
{न ददर्श स्त्रियं राजा कैकेयीं शयनोत्तमे}
{स कामबलसंयुक्तो रत्यर्थी मनुजाधिपः} %2-10-17

\twolineshloka
{अपश्यन् दयितां भार्यां पप्रच्छ विषसाद च}
{नहि तस्य पुरा देवी तां वेलामत्यवर्तत} %2-10-18

\twolineshloka
{न च राजा गृहं शून्यं प्रविवेश कदाचन}
{ततो गृहगतो राजा कैकेयीं पर्यपृच्छत} %2-10-19

\twolineshloka
{यथापुरम् अविज्ञाय स्वार्थलिप्सुम् अपण्डिताम्}
{प्रतिहारी त्वथोवाच सन्त्रस्ता तु कृताञ्जलिः} %2-10-20

\twolineshloka
{देव देवी भृशं क्रुद्धा क्रोधागारमभिद्रुता}
{प्रतीहार्या वचः श्रुत्वा राजा परमदुर्मनाः} %2-10-21

\twolineshloka
{विषसाद पुनर्भूयो लुलितव्याकुलेन्द्रियः}
{तत्र तां पतितां भूमौ शयानामतथोचिताम्} %2-10-22

\twolineshloka
{प्रतप्त इव दुःखेन सोऽपश्यज्जगतीपतिः}
{स वृद्धस्तरुणीं भार्यां प्राणेभ्योऽपि गरीयसीम्} %2-10-23

\twolineshloka
{अपापः पापसङ्कल्पां ददर्श धरणीतले}
{लतामिव विनिष्कृत्तां पतितां देवतामिव} %2-10-24

\twolineshloka
{किन्नरीमिव निर्धूतां च्युताम् अप्सरसं यथा}
{मायाम् इव परिभ्रष्टां हरिणीम् इव संयताम्} %2-10-25

\twolineshloka
{करेणुम् इव दिग्धेन विद्धां मृगयुना वने}
{महागज इवारण्ये स्नेहात् परमदुःखिताम्} %2-10-26

\twolineshloka
{परिमृज्य च पाणिभ्यामभिसन्त्रस्तचेतनः}
{कामी कमलपत्राक्षीमुवाच वनितामिदम्} %2-10-27

\twolineshloka
{न तेऽहमभिजानामि क्रोधमात्मनि संश्रितम्}
{देवि केनाभियुक्तासि केन वासि विमानिता} %2-10-28

\twolineshloka
{यदिदं मम दुःखाय शेषे कल्याणि पांसुषु}
{भूमौ शेषे किमर्थं त्वं मयि कल्याणचेतसि} %2-10-29

\twolineshloka
{भूतोपहतचित्तेव मम चित्तप्रमाथिनि}
{सन्ति मे कुशला वैद्यास्त्वभितुष्टाश्च सर्वशः} %2-10-30

\twolineshloka
{सुखितां त्वां करिष्यन्ति व्याधिमाचक्ष्व भामिनि}
{कस्य वापि प्रियं कार्यं केन वा विप्रियं कृतम्} %2-10-31

\twolineshloka
{कः प्रियं लभतामद्य को वा सुमहदप्रियम्}
{मा रौत्सीर्मा च कार्षीस्त्वं देवि सम्परिशोषणम्} %2-10-32

\twolineshloka
{अवध्यो वध्यतां को वा वध्यः को वा विमुच्यताम्}
{दरिद्रः को भवेदाढ्यो द्रव्यवान् वाप्यकिञ्चनः} %2-10-33

\twolineshloka
{अहं च हि मदीयाश्च सर्वे तव वशानुगाः}
{न ते कञ्चिदभिप्रायं व्याहन्तुमहमुत्सहे} %2-10-34

\twolineshloka
{आत्मनो जीवितेनापि ब्रूहि यन्मनसि स्थितम्}
{बलमात्मनि जानन्ती न मां शङ्कितुमर्हसि} %2-10-35

\twolineshloka
{करिष्यामि तव प्रीतिं सुकृतेनापि ते शपे}
{यावदावर्तते चक्रं तावती मे वसुन्धरा} %2-10-36

\twolineshloka
{द्राविडाः सिन्धुसौवीराः सौराष्ट्रा दक्षिणापथाः}
{वङ्गाङ्गमगधा मत्स्याः समृद्धाः काशिकोसलाः} %2-10-37

\twolineshloka
{तत्र जातं बहु द्रव्यं धनधान्यमजाविकम्}
{ततो वृणीष्व कैकेयि यद् यत् त्वं मनसेच्छसि} %2-10-38

\threelineshloka
{किमायासेन ते भीरु उत्तिष्ठोत्तिष्ठ शोभने}
{तत्त्वं मे ब्रूहि कैकेयि यतस्ते भयमागतम्}
{तत् ते व्यपनयिष्यामि नीहारमिव रश्मिवान्} %2-10-39

\twolineshloka
{तथोक्ता सा समाश्वस्ता वक्तुकामा तदप्रियम्}
{परिपीडयितुं भूयो भर्तारमुपचक्रमे} %2-10-40


॥इत्यार्षे श्रीमद्रामायणे वाल्मीकीये आदिकाव्ये अयोध्याकाण्डे कैकेय्यनुनयः नाम दशमः सर्गः ॥२-१०॥
