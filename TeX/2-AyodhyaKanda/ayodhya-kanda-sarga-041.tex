\sect{एकचत्वारिंशः सर्गः — नगरसङ्क्षोभः}

\twolineshloka
{तस्मिंस्तु पुरुषव्याघ्रे निष्क्रामति कृताञ्जलौ}
{आर्तशब्दो हि सञ्जज्ञे स्त्रीणामन्तःपुरे महान्} %2-41-1

\twolineshloka
{अनाथस्य जनस्यास्य दुर्बलस्य तपस्विनः}
{यो गतिः शरणं चासीत् स नाथः क्व नु गच्छति} %2-41-2

\twolineshloka
{न क्रुध्यत्यभिशस्तोऽपि क्रोधनीयानि वर्जयन्}
{क्रुद्धान् प्रसादयन् सर्वान् समदुःखः क्व गच्छति} %2-41-3

\twolineshloka
{कौसल्यायां महातेजा यथा मातरि वर्तते}
{तथा यो वर्ततेऽस्मासु महात्मा क्व नु गच्छति} %2-41-4

\twolineshloka
{कैकेय्या क्लिश्यमानेन राज्ञा सञ्चोदितो वनम्}
{परित्राता जनस्यास्य जगतः क्व नु गच्छति} %2-41-5

\twolineshloka
{अहो निश्चेतनो राजा जीवलोकस्य सङ्क्षयम्}
{धर्म्यं सत्यव्रतं रामं वनवासे प्रवत्स्यति} %2-41-6

\twolineshloka
{इति सर्वा महिष्यस्ता विवत्सा इव धेनवः}
{रुरुदुश्चैव दुःखार्ताः सस्वरं च विचुक्रुशुः} %2-41-7

\twolineshloka
{स तमन्तःपुरे घोरमार्तशब्दं महीपतिः}
{पुत्रशोकाभिसन्तप्तः श्रुत्वा चासीत् सुदुःखितः} %2-41-8

\twolineshloka
{नाग्निहोत्राण्यहूयन्त नापचन् गृहमेधिनः}
{अकुर्वन् न प्रजाः कार्यं सूर्यश्चान्तरधीयत} %2-41-9

\twolineshloka
{व्यसृजन् कवलान् नागा गावो वत्सान् न पाययन्}
{पुत्रां प्रथमजं लब्ध्वा जननी नाभ्यनन्दत} %2-41-10

\twolineshloka
{त्रिशङ्कुर्लोहिताङ्गश्च बृहस्पतिबुधावपि}
{दारुणाः सोममभ्येत्य ग्रहाः सर्वे व्यवस्थिताः} %2-41-11

\twolineshloka
{नक्षत्राणि गतार्चींषि ग्रहाश्च गततेजसः}
{विशाखाश्च सधूमाश्च नभसि प्रचकाशिरे} %2-41-12

\twolineshloka
{कालिकानिलवेगेन महोदधिरिवोत्थितः}
{रामे वनं प्रव्रजिते नगरं प्रचचाल तत्} %2-41-13

\twolineshloka
{दिशः पर्याकुलाः सर्वास्तिमिरेणेव संवृताः}
{न ग्रहो नापि नक्षत्रं प्रचकाशे न किञ्चन} %2-41-14

\twolineshloka
{अकस्मान्नागरः सर्वो जनो दैन्यमुपागमत्}
{आहारे वा विहारे वा न कश्चिदकरोन्मनः} %2-41-15

\twolineshloka
{शोकपर्यायसन्तप्तः सततं दीर्घमुच्छ्वसन्}
{अयोध्यायां जनः सर्वश्चुक्रोश जगतीपतिम्} %2-41-16

\twolineshloka
{बाष्पपर्याकुलमुखो राजमार्गगतो जनः}
{न हृष्टो लभ्यते कश्चित् सर्वः शोकपरायणः} %2-41-17

\twolineshloka
{न वाति पवनः शीतो न शशी सौम्यदर्शनः}
{न सूर्यस्तपते लोकं सर्वं पर्याकुलं जगत्} %2-41-18

\twolineshloka
{अनर्थिनः सुताः स्त्रीणां भर्तारो भ्रातरस्तथा}
{सर्वे सर्वं परित्यज्य राममेवान्वचिन्तयन्} %2-41-19

\twolineshloka
{ये तु रामस्य सुहृदः सर्वे ते मूढचेतसः}
{शोकभारेण चाक्रान्ताः शयनं नैव भेजिरे} %2-41-20

\twolineshloka
{ततस्त्वयोध्या रहिता महात्मना पुरन्दरेणेव मही सपर्वता}
{चचाल घोरं भयशोकदीपिता सनागयोधाश्वगणा ननाद च} %2-41-21


॥इत्यार्षे श्रीमद्रामायणे वाल्मीकीये आदिकाव्ये अयोध्याकाण्डे नगरसङ्क्षोभः नाम एकचत्वारिंशः सर्गः ॥२-४१॥
