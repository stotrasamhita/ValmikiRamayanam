\sect{अष्टाविंशः सर्गः — वनदुःखप्रतिबोधनम्}

\twolineshloka
{स एवं ब्रुवतीं सीतां धर्मज्ञां धर्मवत्सलः}
{न नेतुं कुरुते बुद्धिं वने दुःखानि चिन्तयन्} %2-28-1

\twolineshloka
{सान्त्वयित्वा ततस्तां तु बाष्पदूषितलोचनाम्}
{निवर्तनार्थे धर्मात्मा वाक्यमेतदुवाच ह} %2-28-2

\twolineshloka
{सीते महाकुलीनासि धर्मे च निरता सदा}
{इहाचरस्व धर्मं त्वं यथा मे मनसः सुखम्} %2-28-3

\twolineshloka
{सीते यथा त्वां वक्ष्यामि तथा कार्यं त्वयाबले}
{वने दोषा हि बहवो वसतस्तान् निबोध मे} %2-28-4

\twolineshloka
{सीते विमुच्यतामेषा वनवासकृता मतिः}
{बहुदोषं हि कान्तारं वनमित्यभिधीयते} %2-28-5

\twolineshloka
{हितबुद्ध्या खलु वचो मयैतदभिधीयते}
{सदा सुखं न जानामि दुःखमेव सदा वनम्} %2-28-6

\twolineshloka
{गिरिनिर्झरसम्भूता गिरिनिर्दरिवासिनाम्}
{सिंहानां निनदा दुःखाः श्रोतुं दुःखमतो वनम्} %2-28-7

\twolineshloka
{क्रीडमानाश्च विस्रब्धा मत्ताः शून्ये तथा मृगाः}
{दृष्ट्वा समभिवर्तन्ते सीते दुःखमतो वनम्} %2-28-8

\twolineshloka
{सग्राहाः सरितश्चैव पङ्कवत्यस्तु दुस्तराः}
{मत्तैरपि गजैर्नित्यमतो दुःखतरं वनम्} %2-28-9

\twolineshloka
{लताकण्टकसङ्कीर्णाः कृकवाकूपनादिताः}
{निरपाश्च सुदुःखाश्च मार्गा दुःखमतो वनम्} %2-28-10

\twolineshloka
{सुप्यते पर्णशय्यासु स्वयम्भग्नासु भूतले}
{रात्रिषु श्रमखिन्नेन तस्माद् दुःखमतो वनम्} %2-28-11

\twolineshloka
{अहोरात्रं च सन्तोषः कर्तव्यो नियतात्मना}
{फलैर्वृक्षावपतितैः सीते दुःखमतो वनम्} %2-28-12

\twolineshloka
{उपवासश्च कर्तव्यो यथा प्राणेन मैथिलि}
{जटाभारश्च कर्तव्यो वल्कलाम्बरधारणम्} %2-28-13

\twolineshloka
{देवतानां पितॄणां च कर्तव्यं विधिपूर्वकम्}
{प्राप्तानामतिथीनां च नित्यशः प्रतिपूजनम्} %2-28-14

\twolineshloka
{कार्यस्त्रिरभिषेकश्च काले काले च नित्यशः}
{चरतां नियमेनैव तस्माद् दुःखतरं वनम्} %2-28-15

\twolineshloka
{उपहारश्च कर्तव्यः कुसुमैः स्वयमाहृतैः}
{आर्षेण विधिना वेद्यां सीते दुःखमतो वनम्} %2-28-16

\twolineshloka
{यथालब्धेन कर्तव्यः सन्तोषस्तेन मैथिलि}
{यताहारैर्वनचरैः सीते दुःखमतो वनम्} %2-28-17

\twolineshloka
{अतीव वातस्तिमिरं बुभुक्षा चाति नित्यशः}
{भयानि च महान्त्यत्र ततो दुःखतरं वनम्} %2-28-18

\twolineshloka
{सरीसृपाश्च बहवो बहुरूपाश्च भामिनि}
{चरन्ति पथि ते दर्पात् ततो दुःखतरं वनम्} %2-28-19

\twolineshloka
{नदीनिलयनाः सर्पा नदीकुटिलगामिनः}
{तिष्ठन्त्यावृत्य पन्थानमतो दुःखतरंवनम्} %2-28-20

\twolineshloka
{पतङ्गा वृश्चिकाः कीटा दंशाश्च मशकैः सह}
{बाधन्ते नित्यमबले सर्वं दुःखमतो वनम्} %2-28-21

\twolineshloka
{द्रुमाः कण्टकिनश्चैव कुशाः काशाश्च भामिनि}
{वने व्याकुलशाखाग्रास्तेन दुःखमतो वनम्} %2-28-22

\twolineshloka
{कायक्लेशाश्च बहवो भयानि विविधानि च}
{अरण्यवासे वसतो दुःखमेव सदा वनम्} %2-28-23

\twolineshloka
{क्रोधलोभौ विमोक्तव्यौ कर्तव्या तपसे मतिः}
{न भेतव्यं च भेतव्ये दुःखं नित्यमतो वनम्} %2-28-24

\twolineshloka
{तदलं ते वनं गत्वा क्षेमं नहि वनं तव}
{विमृशन्निव पश्यामि बहुदोषकरं वनम्} %2-28-25

\twolineshloka
{वनं तु नेतुं न कृता मतिर्यदा बभूव रामेण तदा महात्मना}
{न तस्य सीता वचनं चकार तं ततोऽब्रवीद् राममिदं सुदुःखिता} %2-28-26


॥इत्यार्षे श्रीमद्रामायणे वाल्मीकीये आदिकाव्ये अयोध्याकाण्डे वनदुःखप्रतिबोधनम् नाम अष्टाविंशः सर्गः ॥२-२८॥
