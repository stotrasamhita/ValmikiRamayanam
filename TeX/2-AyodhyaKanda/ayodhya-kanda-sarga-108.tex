\sect{अष्टाधिकशततमः सर्गः — जाबालिवाक्यम्}

\twolineshloka
{आश्वासयन्तं भरतं जाबालिर्ब्राह्मणोत्तमः}
{उवाच रामं धर्मज्ञं धर्मापेतमिदं वचः} %2-108-1

\twolineshloka
{साधु राघव माभूत्ते बुद्धिरेवं निरर्थिका}
{प्राकृतस्य नरस्येव ह्यार्यबुद्धेर्मनस्विनः} %2-108-2

\twolineshloka
{कः कस्य पुरुषो बन्धुः किमाप्यं कस्य केनचित्}
{यदेको जायते जन्तुरेक एव विनश्यति} %2-108-3

\twolineshloka
{तस्मान्माता पिता चेति राम सज्जेतयो नरः}
{उन्मत्त इव स ज्ञेयो नास्ति कश्चिद्धि कस्यचित्} %2-108-4

\twolineshloka
{यथा ग्रामान्तरं गच्छन् नरः कश्चित् क्वचिद्वसेत्}
{उत्सृज्य च तमावासं प्रतिष्ठेतापरेऽहनि} %2-108-5

\twolineshloka
{एवमेव मनुष्याणां पिता माता गृहं वसु}
{आवासमात्रं काकुत्स्थ सज्जन्ते नात्र सज्जनाः} %2-108-6

\twolineshloka
{पित्र्यं राज्यं परित्यज्य स नार्हसि नरोत्तम}
{आस्थातुं कापथं दुःखं विषमं बहुकण्टकम्} %2-108-7

\twolineshloka
{समृद्धायामयोध्यायामात्मानमभिषेचय}
{एकवेणी धरा हि त्वां नगरी सम्प्रतीक्षते} %2-108-8

\twolineshloka
{राजभोगाननुभवन् महार्हान् पार्थिवात्मज}
{विहर त्वमयोध्यायां यथा शक्रस्त्रिविष्टपे} %2-108-9

\twolineshloka
{न ते कश्चिद्दशरथस्त्वं च तस्य न कश्चन}
{अन्यो राजा त्वमन्यः स तस्मात् कुरु यदुच्यते} %2-108-10

\twolineshloka
{बीजमात्रं पिता जन्तोः शुक्लं रुधिरमेव च}
{संयुक्तमृतुमन्मात्रा पुरुषस्येह जन्म तत्} %2-108-11

\twolineshloka
{गतः स नृपतिस्तत्र गन्तव्यं यत्र तेन वै}
{प्रवृत्तिरेषा मर्त्त्यानां त्वं तु मिथ्या विहन्यसे} %2-108-12

\twolineshloka
{अर्थधर्मपरा ये ये तांस्तान् शोचामि नेतरान्}
{ते हि दुःखमिह प्राप्य विनाशं प्रेत्य भेजिरे} %2-108-13

\twolineshloka
{अष्टका पितृदैवत्यमित्ययं प्रसृतो जनः}
{अन्नस्योपद्रवं पश्य मृतो हि किमशिष्यति} %2-108-14

\twolineshloka
{यदि भुक्तमिहान्येन देहमन्यस्य गच्छति}
{दद्यात् प्रवसतः श्राद्धं न तत् पथ्यशनं भवेत्} %2-108-15

\twolineshloka
{दानसंवनना ह्येते ग्रन्था मेधाविभिः कृताः}
{यजस्व देहि दीक्षस्व तपस्तप्यस्व सन्त्यज} %2-108-16

\twolineshloka
{स नास्ति परमित्येव कुरु बुद्धिं महामते}
{प्रत्यक्षं यत्तदातिष्ठ परोक्षं पृष्ठतः कुरु} %2-108-17

\twolineshloka
{स तां बुद्धिं पुरस्कृत्य सर्वलोकनिदर्शिनीम्}
{राज्यं त्वं प्रतिगृह्णीष्व भरतेन प्रसादितः} %2-108-18


॥इत्यार्षे श्रीमद्रामायणे वाल्मीकीये आदिकाव्ये अयोध्याकाण्डे जाबालिवाक्यम् नाम अष्टाधिकशततमः सर्गः ॥२-१०८॥
