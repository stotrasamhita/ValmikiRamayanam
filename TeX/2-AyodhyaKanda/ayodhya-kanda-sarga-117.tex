\sect{सप्तदशाधिकशततमः सर्गः — सीतापातिव्रत्यप्रशंसा}

\twolineshloka
{राघवस्त्वथ यातेषु तपस्विषु विचिन्तयन्}
{न तत्रारोचयद्वासं कारणैर्बहुभिस्तदा} %2-117-1

\twolineshloka
{इह मे भरतो दृष्टो मातरश्च सनागराः}
{सा च मे स्मृतिरन्वेति तान्नित्यमनुशोचतः} %2-117-2

\twolineshloka
{स्कन्धावारनिवेशेन तेन तस्य महात्मनः}
{हयहस्तिकरीषैश्च उपमर्द्दः कृतो भृशम्} %2-117-3

\twolineshloka
{तस्मादन्यत्र गच्छाम इति सञ्चिन्त्य राघवः}
{प्रातिष्ठत स वैदेह्या लक्ष्मणेन च सङ्गतः} %2-117-4

\twolineshloka
{सोऽत्रेराश्रममासाद्य तं ववन्दे महायशाः}
{तं चापि भगवानत्रिः पुत्रवत् प्रत्यपद्यत} %2-117-5

\twolineshloka
{स्वयमातिथ्यमादिश्य सर्वमस्य सुसत्कृतम्}
{सौमित्रिं च महाभागां सीतां च समसान्त्वयत्} %2-117-6

\twolineshloka
{पत्नीं च समनुप्राप्तां वृद्धामामन्त्र्य सत्कृताम्}
{सान्त्वयामास धर्मज्ञः सर्वभूतहिते रतः} %2-117-7

\threelineshloka
{अनसूयां महाभागां तापसीं धर्मचारिणीम्}
{प्रतिगृह्णीष्व वैदेहीमब्रवीदृषिसत्तमः}
{रामाय चाचचक्षे तां तापसीं धर्मचारिणीम्} %2-117-8

\twolineshloka
{दशवर्षाण्यनावृष्ट्या दग्धे लोके निरन्तरम्}
{यया मूलफले सृष्टे जाह्नवी च प्रवर्त्तिता} %2-117-9

\twolineshloka
{उग्रेण तपसा युक्ता नियमैश्चाप्यलङ्कृता}
{दशवर्षसहस्राणि यया तप्तं महत्तपः} %2-117-10

\threelineshloka
{अनसूया व्रतैः स्नाता प्रत्यूहाश्च निवर्त्तिताः}
{देवकार्यनिमित्तं च यया सन्त्वरमाणया}
{दशरात्रं कृता रात्रिः सेयं मातेव तेऽनघ} %2-117-11

\threelineshloka
{तामिमां सर्वभूतानां नमस्कार्य्यां यशस्विनीम्}
{अभिगच्छतु वैदेही वृद्धामक्रोधनां सदा}
{अनसूयेति या लोके कर्मभिः ख्यातिमागता} %2-117-12

\twolineshloka
{एवं ब्रुवाणं तमृषिं तथेत्युक्त्वा स राघवः}
{सीतामुवाच धर्मज्ञामिदं वचनमुत्तमम्} %2-117-13

\twolineshloka
{राजपुत्रि श्रुतं त्वेतन्मुनेरस्य समीरितम्}
{श्रेयोर्थमात्मनः श्रीघ्रमभिगच्छ तपस्विनीम्} %2-117-14

\twolineshloka
{सीता त्वेतद्वचः श्रुत्वा राघवस्य हितैषिणः}
{तामत्रिपत्नीं धर्मज्ञामभिचक्राम मैथिली} %2-117-15

\twolineshloka
{शिथिलां वलितां वृद्धां जरापाण्डरमूर्द्धजाम्}
{सततं वेपमानाङ्गी प्रवाते कदली यथा} %2-117-16

\twolineshloka
{तां तु सीता महाभागामनसूयां पतिव्रताम्}
{अभ्यवादयदव्यग्रा स्वनाम समुदाहरत्} %2-117-17

\twolineshloka
{अभिवाद्य च वैदेही तापसीं तामनिन्दिताम्}
{बद्धाञ्जलिपुटा हृष्टा पर्यपृच्छदनामयम्} %2-117-18

\twolineshloka
{ततः सीतां महाभागां दृष्ट्वा तां धर्मचारिणीम्}
{सान्त्वयन्त्यब्रवीद्धृष्टा दिष्ट्या धर्ममवेक्षसे} %2-117-19

\twolineshloka
{त्यक्त्वा ज्ञातिजनं सीते मानमृद्धिं च भामिनि}
{अवरुद्धं वने रामं दिष्ट्या त्वमनुगच्छसि} %2-117-20

\twolineshloka
{नगरस्थो वनस्थो वा पापो वा यदि वा शुभः}
{यासां स्त्रीणां प्रियो भर्ता तासां लोका महोदयाः} %2-117-21

\twolineshloka
{दुःशीलः कामवृत्तो वा धनैवा परिवर्जितः}
{स्त्रीणामार्यस्वभावानां परमं दैवतं पतिः} %2-117-22

\twolineshloka
{नातो विशिष्टं पश्यामि बान्धवं विमृशन्त्यहम्}
{सर्वत्रयोग्यं वैदेहि तपःकृतमिवाव्ययम्} %2-117-23

\twolineshloka
{न त्वेनमवगच्छन्ति गुणदोषमसत्स्त्रियः}
{कामवक्तव्यहृदया भर्तृनाथा श्चरन्ति याः} %2-117-24

\twolineshloka
{प्राप्नुवन्त्ययशश्चैव धर्मभ्रंशं च मैथिलि}
{अकार्यवशमापन्नाः स्त्रियो याः खलु तद्विधाः} %2-117-25

\twolineshloka
{त्वद्विधास्तु गुणैर्युक्ता दृष्टलोकपरावराः}
{स्त्रियः स्वग चरिष्यन्ति यथा धर्मकृतस्तथा} %2-117-26

\twolineshloka
{तदेवमेनं त्वमनुव्रता सती पतिव्रतानां समयानुवर्तिनी}
{भवस्व भर्त्तुः सहधर्मचारिणी यशश्च धर्मं च ततः समाप्स्यसि} %2-117-27


॥इत्यार्षे श्रीमद्रामायणे वाल्मीकीये आदिकाव्ये अयोध्याकाण्डे सीतापातिव्रत्यप्रशंसा नाम सप्तदशाधिकशततमः सर्गः ॥२-११७॥
