\sect{सप्तषष्ठितमः सर्गः — अराजकदुरवस्थावर्णनम्}

\twolineshloka
{आक्रन्दितनिरानन्दा सास्रकम्ठजनाविला}
{आयोध्यायामतितता सा व्यतीयाय शर्वरी} %2-67-1

\twolineshloka
{व्यतीतायाम् तु शर्वर्याम् आदित्यस्य उदये ततः}
{समेत्य राज कर्तारः सभाम् ईयुर् द्विजातयः} %2-67-2

\twolineshloka
{मार्कण्डेयो अथ मौद्गल्यो वामदेवः च काश्यपः}
{कात्ययनो गौतमः च जाबालिः च महा यशाः} %2-67-3

\twolineshloka
{एते द्विजाः सह अमात्यैः पृथग् वाचम् उदीरयन्}
{वसिष्ठम् एव अभिमुखाः श्रेष्ठः राज पुरोहितम्} %2-67-4

\twolineshloka
{अतीता शर्वरी दुह्खम् या नो वर्ष शत उपमा}
{अस्मिन् पन्चत्वम् आपन्ने पुत्र शोकेन पार्थिवे} %2-67-5

\twolineshloka
{स्वर् गतः च महा राजो रामः च अरण्यम् आश्रितः}
{लक्ष्मणः च अपि तेजस्वी रामेण एव गतः सह} %2-67-6

\twolineshloka
{उभौ भरत शत्रुघ्नौ क्केकयेषु परम् तपौ}
{पुरे राज गृहे रम्ये मातामह निवेशने} %2-67-7

\twolineshloka
{इक्ष्वाकूणाम् इह अद्य एव कश्चित् राजा विधीयताम्}
{अराजकम् हि नो राष्ट्रम् न विनाशम् अवाप्नुयात्} %2-67-8

\twolineshloka
{न अराजले जन पदे विद्युन् माली महा स्वनः}
{अभिवर्षति पर्जन्यो महीम् दिव्येन वारिणा} %2-67-9

\twolineshloka
{न अराजके जन पदे बीज मुष्टिः प्रकीर्यते}
{न अराकके पितुः पुत्रः भार्या वा वर्तते वशे} %2-67-10

\twolineshloka
{अराजके धनम् न अस्ति न अस्ति भार्या अपि अराजके}
{इदम् अत्याहितम् च अन्यत् कुतः सत्यम् अराजके} %2-67-11

\twolineshloka
{न अराजके जन पदे कारयन्ति सभाम् नराः}
{उद्यानानि च रम्याणि हृष्टाः पुण्य गृहाणि च} %2-67-12

\twolineshloka
{न अराजके जन पदे यज्ञ शीला द्विजातयः}
{सत्राणि अन्वासते दान्ता ब्राह्मणाः सम्शित व्रताः} %2-67-13

\twolineshloka
{न अराजके जनपदे महायज्ञेषु यज्वनः}
{ब्राह्मणा वसुसम्पन्ना विसृजन्त्याप्तदक्षिणाः} %2-67-14

\twolineshloka
{न अराजके जन पदे प्रभूत नट नर्तकाः}
{उत्सवाः च समाजाः च वर्धन्ते राष्ट्र वर्धनाः} %2-67-15

\twolineshloka
{न अरजके जन पदे सिद्ध अर्था व्यवहारिणः}
{कथाभिर् अनुरज्यन्ते कथा शीलाः कथा प्रियैः} %2-67-16

\twolineshloka
{न अराजके जनपदे उद्यानानि समागताः}
{सायाह्ने क्रीडितुम् यान्ति कुमार्यो हेमभूषिताः} %2-67-17

\twolineshloka
{न अराजके जन पदे वाहनैः शीघ्र गामिभिः}
{नरा निर्यान्ति अरण्यानि नारीभिः सह कामिनः} %2-67-18

\twolineshloka
{न अराकजे जन पदे धनवन्तः सुरक्षिताः}
{शेरते विवृत द्वाराः कृषि गो रक्ष जीविनः} %2-67-19

\twolineshloka
{न अराजके जनपदे बद्दघण्टा विषाणीनः}
{आटन्ति राजमार्गेषु कुञ्जराः षष्टिहायनाः} %2-67-20

\twolineshloka
{न अराजके जनपदे शरान् सम्ततमस्यताम्}
{श्रूयते तलनिर्घोष इष्वस्त्राणामुपासने} %2-67-21

\twolineshloka
{न अराजके जन पदे वणिजो दूर गामिनः}
{गच्चन्ति क्षेमम् अध्वानम् बहु पुण्य समाचिताः} %2-67-22

\twolineshloka
{न अराजके जन पदे चरति एक चरः वशी}
{भावयन्न् आत्मना आत्मानम् यत्र सायम् गृहो मुनिः} %2-67-23

\twolineshloka
{न अराजके जन पदे योग क्षेमम् प्रवर्तते}
{न च अपि अराजके सेना शत्रून् विषहते युधि} %2-67-24

\twolineshloka
{न अराजके जनपदे हृष्टैः परमवाजिभिः}
{नराः सम्यान्ति सहसा रथैश्च परिमण्डिताः} %2-67-25

\twolineshloka
{न अराजके जनपदे नराः शास्त्रविशारदाः}
{सम्पदन्तोऽवतिष्ठन्ते वनेषूपवनेषु च} %2-67-26

\twolineshloka
{न अराजके जनपदे माल्यमोदकदक्षिणाः}
{देवताभ्यर्चनार्थय कल्प्यन्ते नियतैर्जनैः} %2-67-27

\twolineshloka
{न अराजके जनपदे चन्दनागुरुरूषिताः}
{राजपुत्रा विराजन्ते वसन्त इव शाखिनः} %2-67-28

\twolineshloka
{यथा हि अनुदका नद्यो यथा वा अपि अतृणम् वनम्}
{अगोपाला यथा गावः तथा राष्ट्रम् अराजकम्} %2-67-29

\twolineshloka
{ध्वजो रथस्य प्रज्ञानम् धूमो ज्ञानम् विभावसोः}
{तेषाम् यो नो ध्वजो राज स देवत्वमितो गतः} %2-67-30

\twolineshloka
{न अराजके जन पदे स्वकम् भवति कस्यचित्}
{मत्स्याइव नरा नित्यम् भक्षयन्ति परस्परम्} %2-67-31

\twolineshloka
{येहि सम्भिन्न मर्यादा नास्तिकाः चिन्न सम्शयाः}
{ते अपि भावाय कल्पन्ते राज दण्ड निपीडिताः} %2-67-32

\twolineshloka
{यथा दृष्टिः शरीरस्य नित्यमेवप्रवर्तते}
{तथा नरेन्द्रो राष्ट्रस्य प्रभवः सत्यधर्मयोः} %2-67-33

\twolineshloka
{राजा सत्यम् च धर्मश्च राजा कुलवताम् कुलम्}
{राजा माता पिता चैव राजा हितकरो नृणाम्} %2-67-34

\twolineshloka
{यमो वैश्रवणः शक्रो वरुणश्च महाबलः}
{विशेष्यन्ते नरेन्द्रेण वृत्तेन महाता ततः} %2-67-35

\twolineshloka
{अहो तमैव इदम् स्यान् न प्रज्ञायेत किम्चन}
{राजा चेन् न भवेन् लोके विभजन् साध्व् असाधुनी} %2-67-36

\twolineshloka
{जीवति अपि महा राजे तव एव वचनम् वयम्}
{न अतिक्रमामहे सर्वे वेलाम् प्राप्य इव सागरः} %2-67-37

\fourlineindentedshloka
{स नः समीक्ष्य द्विज वर्य वृत्तम्}
{नृपम् विना राज्यम् अरण्य भूतम्}
{कुमारम् इक्ष्वाकु सुतम् वदान्यम्}
{त्वम् एव राजानम् इह अभिषिन्चय} %2-67-38


॥इत्यार्षे श्रीमद्रामायणे वाल्मीकीये आदिकाव्ये अयोध्याकाण्डे अराजकदुरवस्थावर्णनम् नाम सप्तषष्ठितमः सर्गः ॥२-६७॥
