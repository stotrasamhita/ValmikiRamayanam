\sect{चतुर्दशः सर्गः — कैकेय्युपालम्भः}

\twolineshloka
{पुत्रशोकार्दितं पापा विसंज्ञं पतितं भुवि}
{विचेष्टमानमुत्प्रेक्ष्य ऐक्ष्वाकमिदमब्रवीत्} %2-14-1

\twolineshloka
{पापं कृत्वेव किमिदं मम संश्रुत्य संश्रवम्}
{शेषे क्षितितले सन्नः स्थित्यां स्थातुं त्वमर्हसि} %2-14-2

\twolineshloka
{आहुः सत्यं हि परमं धर्मं धर्मविदो जनाः}
{सत्यमाश्रित्य च मया त्वं धर्मं प्रतिचोदितः} %2-14-3

\twolineshloka
{संश्रुत्य शैब्यः श्येनाय स्वां तनुं जगतीपतिः}
{प्रदाय पक्षिणे राजा जगाम गतिमुत्तमाम्} %2-14-4

\twolineshloka
{तथा ह्यलर्कस्तेजस्वी ब्राह्मणे वेदपारगे}
{याचमाने स्वके नेत्रे उद्धृत्याविमना ददौ} %2-14-5

\twolineshloka
{सरितां तु पतिः स्वल्पां मर्यादां सत्यमन्वितः}
{सत्यानुरोधात् समये वेलां स्वां नातिवर्तते} %2-14-6

\twolineshloka
{सत्यमेकपदं ब्रह्म सत्ये धर्मः प्रतिष्ठितः}
{सत्यमेवाक्षया वेदाः सत्येनावाप्यते परम्} %2-14-7

\twolineshloka
{सत्यं समनुवर्तस्व यदि धर्मे धृता मतिः}
{स वरः सफलो मेऽस्तु वरदो ह्यसि सत्तम} %2-14-8

\twolineshloka
{धर्मस्यैवाभिकामार्थं मम चैवाभिचोदनात्}
{प्रव्राजय सुतं रामं त्रिः खलु त्वां ब्रवीम्यहम्} %2-14-9

\twolineshloka
{समयं च ममार्येमं यदि त्वं न करिष्यसि}
{अग्रतस्ते परित्यक्ता परित्यक्ष्यामि जीवितम्} %2-14-10

\twolineshloka
{एवं प्रचोदितो राजा कैकेय्या निर्विशङ्कया}
{नाशकत् पाशमुन्मोक्तुं बलिरिन्द्रकृतं यथा} %2-14-11

\twolineshloka
{उद्भ्रान्तहृदयश्चापि विवर्णवदनोऽभवत्}
{स धुर्यो वै परिस्पन्दन् युगचक्रान्तरं यथा} %2-14-12

\twolineshloka
{विकलाभ्यां च नेत्राभ्यामपश्यन्निव भूमिपः}
{कृच्छ्राद् धैर्येण संस्तभ्य कैकेयीमिदमब्रवीत्} %2-14-13

\twolineshloka
{यस्ते मन्त्रकृतः पाणिरग्नौ पापे मया धृतः}
{संत्यजामि स्वजं चैव तव पुत्रं सह त्वया} %2-14-14

\twolineshloka
{प्रयाता रजनी देवि सूर्यस्योदयनं प्रति}
{अभिषेकाय हि जनस्त्वरयिष्यति मां ध्रुवम्} %2-14-15

\twolineshloka
{रामाभिषेकसम्भारैस्तदर्थमुपकल्पितैः}
{रामः कारयितव्यो मे मृतस्य सलिलक्रियाम्} %2-14-16

\twolineshloka
{सपुत्रया त्वया नैव कर्तव्या सलिलक्रिया}
{व्याहन्तास्यशुभाचारे यदि रामाभिषेचनम्} %2-14-17

\twolineshloka
{न शक्तोऽद्यास्म्यहं द्रष्टुं दृष्ट्वा पूर्वं तथामुखम्}
{हतहर्षं तथानन्दं पुनर्जनमवाङ्मुखम्} %2-14-18

\twolineshloka
{तां तथा ब्रुवतस्तस्य भूमिपस्य महात्मनः}
{प्रभाता शर्वरी पुण्या चन्द्रनक्षत्रमालिनी} %2-14-19

\twolineshloka
{ततः पापसमाचारा कैकेयी पार्थिवं पुनः}
{उवाच परुषं वाक्यं वाक्यज्ञा रोषमूर्च्छिता} %2-14-20

\twolineshloka
{किमिदं भाषसे राजन् वाक्यं गररुजोपमम्}
{आनाययितुमक्लिष्टं पुत्रं राममिहार्हसि} %2-14-21

\twolineshloka
{स्थाप्य राज्ये मम सुतं कृत्वा रामं वनेचरम्}
{निःसपत्नां च मां कृत्वा कृतकृत्यो भविष्यसि} %2-14-22

\twolineshloka
{स तुन्न इव तीक्ष्णेन प्रतोदेन हयोत्तमः}
{राजा प्रचोदितोऽभीक्ष्णं कैकेय्या वाक्यमब्रवीत्} %2-14-23

\twolineshloka
{धर्मबन्धेन बद्धोऽस्मि नष्टा च मम चेतना}
{ज्येष्ठं पुत्रं प्रियं रामं द्रष्टुमिच्छामि धार्मिकम्} %2-14-24

\twolineshloka
{ततः प्रभातां रजनीमुदिते च दिवाकरे}
{पुण्ये नक्षत्रयोगे च मुहूर्ते च समागते} %2-14-25

\twolineshloka
{वसिष्ठो गुणसम्पन्नः शिष्यैः परिवृतस्तथा}
{उपगृह्याशु सम्भारान् प्रविवेश पुरोत्तमम्} %2-14-26

\twolineshloka
{सिक्तसम्मार्जितपथां पताकोत्तमभूषिताम्}
{संहृष्टमनुजोपेतां समृद्धविपणापणाम्} %2-14-27

\twolineshloka
{महोत्सवसमायुक्तां राघवार्थे समुत्सुकाम्}
{चन्दनागुरुधूपैश्च सर्वतः परिधूमिताम्} %2-14-28

\twolineshloka
{तां पुरीं समतिक्रम्य पुरंदरपुरोपमाम्}
{ददर्शान्तःपुरं श्रीमान् नानाध्वजगणायुतम्} %2-14-29

\twolineshloka
{पौरजानपदाकीर्णं ब्राह्मणैरुपशोभितम्}
{यष्टिमद्भिः सुसम्पूर्णं सदश्वैः परमार्चितैः} %2-14-30

\twolineshloka
{तदन्तःपुरमासाद्य व्यतिचक्राम तं जनम्}
{वसिष्ठः परमप्रीतः परमर्षिभिरावृतः} %2-14-31

\twolineshloka
{स त्वपश्यद् विनिष्क्रान्तं सुमन्त्रं नाम सारथिम्}
{द्वारे मनुजसिंहस्य सचिवं प्रियदर्शनम्} %2-14-32

\twolineshloka
{तमुवाच महातेजाः सूतपुत्रं विशारदम्}
{वसिष्ठः क्षिप्रमाचक्ष्व नृपतेर्मामिहागतम्} %2-14-33

\twolineshloka
{इमे गङ्गोदकघटाः सागरेभ्यश्च काञ्चनाः}
{औदुम्बरं भद्रपीठमभिषेकार्थमाहृतम्} %2-14-34

\twolineshloka
{सर्वबीजानि गन्धाश्च रत्नानि विविधानि च}
{क्षौद्रं दधि घृतं लाजा दर्भाः सुमनसः पयः} %2-14-35

\twolineshloka
{अष्टौ च कन्या रुचिरा मत्तश्च वरवारणः}
{चतुरश्वो रथः श्रीमान् निस्त्रिंशो धनुरुत्तमम्} %2-14-36

\twolineshloka
{वाहनं नरसंयुक्तं छत्रं च शशिसंनिभम्}
{श्वेते च वालव्यजने भृङ्गारं च हिरण्मयम्} %2-14-37

\twolineshloka
{हेमदामपिनद्धश्च ककुद्मान् पाण्डुरो वृषः}
{केसरी च चतुर्दंष्ट्रो हरिश्रेष्ठो महाबलः} %2-14-38

\twolineshloka
{सिंहासनं व्याघ्रतनुः समिधश्च हुताशनः}
{सर्वे वादित्रसङ्घाश्च वेश्याश्चालंकृताः स्त्रियः} %2-14-39

\twolineshloka
{आचार्या ब्राह्मणा गावः पुण्याश्च मृगपक्षिणः}
{पौरजानपदश्रेष्ठा नैगमाश्च गणैः सह} %2-14-40

\twolineshloka
{एते चान्ये च बहवः प्रीयमाणाः प्रियंवदाः}
{अभिषेकाय रामस्य सह तिष्ठन्ति पार्थिवैः} %2-14-41

\twolineshloka
{त्वरयस्व महाराजं यथा समुदितेऽहनि}
{पुष्ये नक्षत्रयोगे च रामो राज्यमवाप्नुयात्} %2-14-42

\twolineshloka
{इति तस्य वचः श्रुत्वा सूतपुत्रो महाबलः}
{स्तुवन् नृपतिशार्दूलं प्रविवेश निवेशनम्} %2-14-43

\twolineshloka
{तं तु पूर्वोदितं वृद्धं द्वारस्था राजसम्मताः}
{न शेकुरभिसंरोद्धुं राज्ञः प्रियचिकीर्षवः} %2-14-44

\twolineshloka
{स समीपस्थितो राज्ञस्तामवस्थामजज्ञिवान्}
{वाग्भिः परमतुष्टाभिरभिष्टोतुं प्रचक्रमे} %2-14-45

\twolineshloka
{ततः सूतो यथापूर्वं पार्थिवस्य निवेशने}
{सुमन्त्रः प्राञ्जलिर्भूत्वा तुष्टाव जगतीपतिम्} %2-14-46

\twolineshloka
{यथा नन्दति तेजस्वी सागरो भास्करोदये}
{प्रीतः प्रीतेन मनसा तथा नन्दय नस्ततः} %2-14-47

\twolineshloka
{इन्द्रमस्यां तु वेलायामभितुष्टाव मातलिः}
{सोऽजयद् दानवान् सर्वांस्तथा त्वां बोधयाम्यहम्} %2-14-48

\twolineshloka
{वेदाः सहाङ्गा विद्याश्च यथा ह्यात्मभुवं प्रभुम्}
{ब्रह्माणं बोधयन्त्यद्य तथा त्वां बोधयाम्यहम्} %2-14-49

\twolineshloka
{आदित्यः सह चन्द्रेण यथा भूतधरां शुभाम्}
{बोधयत्यद्य पृथिवीं तथा त्वां बोधयाम्यहम्} %2-14-50

\twolineshloka
{उत्तिष्ठ सुमहाराज कृतकौतुकमङ्गलः}
{विराजमानो वपुषा मेरोरिव दिवाकरः} %2-14-51

\twolineshloka
{सोमसूर्यौ च काकुत्स्थ शिववैश्रवणावपि}
{वरुणश्चाग्निरिन्द्रश्च विजयं प्रदिशन्तु ते} %2-14-52

\twolineshloka
{गता भगवती रात्रिः कृतं कृत्यमिदं तव}
{बुध्यस्व नृपशार्दूल कुरु कार्यमनन्तरम्} %2-14-53

\twolineshloka
{उदतिष्ठत रामस्य समग्रमभिषेचनम्}
{पौरजानपदाश्चापि नैगमश्च कृताञ्जलिः} %2-14-54

\twolineshloka
{स्वयं वसिष्ठो भगवान् ब्राह्मणैः सह तिष्ठति}
{क्षिप्रमाज्ञाप्यतां राजन् राघवस्याभिषेचनम्} %2-14-55

\twolineshloka
{यथा ह्यपालाः पशवो यथा सेना ह्यनायका}
{यथा चन्द्रं विना रात्रिर्यथा गावो विना वृषम्} %2-14-56

\twolineshloka
{एवं हि भविता राष्ट्रं यत्र राजा न दृश्यते}
{एवं तस्य वचः श्रुत्वा सान्त्वपूर्वमिवार्थवत्} %2-14-57

\twolineshloka
{अभ्यकीर्यत शोकेन भूय एव महीपतिः}
{ततस्तु राजा तं सूतं सन्नहर्षः सुतं प्रति} %2-14-58

\twolineshloka
{शोकरक्तेक्षणः श्रीमानुद्वीक्ष्योवाच धार्मिकः}
{वाक्यैस्तु खलु मर्माणि मम भूयो निकृन्तसि} %2-14-59

\twolineshloka
{सुमन्त्रः करुणं श्रुत्वा दृष्ट्वा दीनं च पार्थिवम्}
{प्रगृहीताञ्जलिः किंचित् तस्माद् देशादपाक्रमत्} %2-14-60

\twolineshloka
{यदा वक्तुं स्वयं दैन्यान्न शशाक महीपतिः}
{तदा सुमन्त्रं मन्त्रज्ञा कैकेयी प्रत्युवाच ह} %2-14-61

\twolineshloka
{सुमन्त्र राजा रजनीं रामहर्षसमुत्सुकः}
{प्रजागरपरिश्रान्तो निद्रावशमुपागतः} %2-14-62

\twolineshloka
{तद् गच्छ त्वरितं सूत राजपुत्रं यशस्विनम्}
{राममानय भद्रं ते नात्र कार्या विचारणा} %2-14-63

\twolineshloka
{अश्रुत्वा राजवचनं कथं गच्छामि भामिनि}
{तच्छ्रुत्वा मन्त्रिणो वाक्यं राजा मन्त्रिणमब्रवीत्} %2-14-64

\twolineshloka
{सुमन्त्र रामं द्रक्ष्यामि शीघ्रमानय सुन्दरम्}
{स मन्यमानः कल्याणं हृदयेन ननन्द च} %2-14-65

\twolineshloka
{निर्जगाम च स प्रीत्या त्वरितो राजशासनात्}
{सुमन्त्रश्चिन्तयामास त्वरितं चोदितस्तया} %2-14-66

\twolineshloka
{व्यक्तं रामाभिषेकार्थे इहायास्यति धर्मराट्}
{इति सूतो मतिं कृत्वा हर्षेण महता पुनः} %2-14-67

\threelineshloka
{निर्जगाम महातेजा राघवस्य दिदृक्षया}
{सागरह्रदसंकाशात्सुमन्त्रोऽन्तःपुराच्छुभात्}
{निष्क्रम्य जनसम्बाधं ददर्श द्वारमग्रतः} %2-14-68

\twolineshloka
{ततः पुरस्तात् सहसा विनिःसृतो महीपतेर्द्वारगतान् विलोकयन्}
{ददर्श पौरान् विविधान् महाधनानुपस्थितान् द्वारमुपेत्य विष्ठितान्} %2-14-69


॥इत्यार्षे श्रीमद्रामायणे वाल्मीकीये आदिकाव्ये अयोध्याकाण्डे कैकेय्युपालम्भः नाम चतुर्दशः सर्गः ॥२-१४॥
