\sect{अष्टसप्ततितमः सर्गः — कुब्जाविक्षेपः}

\twolineshloka
{अत्र यात्राम् समीहन्तम् शत्रुघ्नः लक्ष्मण अनुजः}
{भरतम् शोक सम्तप्तम् इदम् वचनम् अब्रवीत्} %2-78-1

\twolineshloka
{गतिर् यः सर्व भूतानाम् दुह्खे किम् पुनर् आत्मनः}
{स रामः सत्त्व सम्पन्नः स्त्रिया प्रव्राजितः वनम्} %2-78-2

\twolineshloka
{बलवान् वीर्य सम्पन्नो लक्ष्मणो नाम यो अपि असौ}
{किम् न मोचयते रामम् कृत्वा अपि पितृ निग्रहम्} %2-78-3

\twolineshloka
{पूर्वम् एव तु निग्राह्यः समवेक्ष्य नय अनयौ}
{उत्पथम् यः समारूढो नार्या राजा वशम् गतः} %2-78-4

\twolineshloka
{इति सम्भाषमाणे तु शत्रुघ्ने लक्ष्मण अनुजे}
{प्राग् द्वारे अभूत् तदा कुब्जा सर्व आभरण भूषिता} %2-78-5

\twolineshloka
{लिप्ता चन्दन सारेण राज वस्त्राणि बिभ्रती}
{विविधम् विविधैस्तैस्तैर्भूषणैश्च विभूषिता} %2-78-6

\twolineshloka
{मेखला दामभिः चित्रै रज्जु बद्धा इव वानरी}
{बभासे बहुभिर्बद्धा रज्जुबद्देव वानरी} %2-78-7

\twolineshloka
{ताम् समीक्ष्य तदा द्वाह्स्थो भृशम् पापस्य कारिणीम्}
{गृहीत्वा अकरुणम् कुब्जाम् शत्रुघ्नाय न्यवेदयत्} %2-78-8

\twolineshloka
{यस्याः कृते वने रामः न्यस्त देहः च वः पिता}
{सा इयम् पापा नृशम्सा च तस्याः कुरु यथा मति} %2-78-9

\twolineshloka
{शत्रुघ्नः च तत् आज्ञाय वचनम् भृश दुह्खितः}
{अन्तः पुर चरान् सर्वान् इति उवाच धृत व्रतः} %2-78-10

\twolineshloka
{तीव्रम् उत्पादितम् दुह्खम् भ्रातृऋणाम् मे तथा पितुः}
{यया सा इयम् नृशम्सस्य कर्मणः फलम् अश्नुताम्} %2-78-11

\twolineshloka
{एवम् उक्ता च तेन आशु सखी जन समावृता}
{गृहीता बलवत् कुब्जा सा तत् गृहम् अनादयत्} %2-78-12

\twolineshloka
{ततः सुभृश सम्तप्तः तस्याः सर्वः सखी जनः}
{क्रुद्धम् आज्ञाय शत्रुघ्नम् व्यपलायत सर्वशः} %2-78-13

\twolineshloka
{अमन्त्रयत कृत्स्नः च तस्याः सर्व सखी जनः}
{यथा अयम् समुपक्रान्तः निह्शेषम् नः करिष्यति} %2-78-14

\twolineshloka
{सानुक्रोशाम् वदान्याम् च धर्मज्ञाम् च यशस्विनीम्}
{कौसल्याम् शरणम् यामः सा हि नो अस्तु ध्रुवा गतिः} %2-78-15

\twolineshloka
{स च रोषेण ताम्र अक्षः शत्रुघ्नः शत्रु तापनः}
{विचकर्ष तदा कुब्जाम् क्रोशन्तीम् पृथिवी तले} %2-78-16

\twolineshloka
{तस्या हि आकृष्यमाणाया मन्थरायाः ततः ततः}
{चित्रम् बहु विधम् भाण्डम् पृथिव्याम् तत् व्यशीर्यत} %2-78-17

\twolineshloka
{तेन भाण्डेन सम्कीर्णम् श्रीमद् राज निवेशनम्}
{अशोभत तदा भूयः शारदम् गगनम् यथा} %2-78-18

\twolineshloka
{स बली बलवत् क्रोधात् गृहीत्वा पुरुष ऋषभः}
{कैकेयीम् अभिनिर्भर्त्स्य बभाषे परुषम् वचः} %2-78-19

\twolineshloka
{तैः वाक्यैः परुषैः दुह्खैः कैकेयी भृश दुह्हिता}
{शत्रुघ्न भय सम्त्रस्ता पुत्रम् शरणम् आगता} %2-78-20

\twolineshloka
{ताम् प्रेक्ष्य भरतः क्रुद्धम् शत्रुघ्नम् इदम् अब्रवीत्}
{अवध्याः सर्व भूतानाम् प्रमदाः क्षम्यताम् इति} %2-78-21

\twolineshloka
{हन्याम् अहम् इमाम् पापाम् कैकेयीम् दुष्ट चारिणीम्}
{यदि माम् धार्मिको रामः न असूयेन् मातृ घातकम्} %2-78-22

\twolineshloka
{इमाम् अपि हताम् कुब्जाम् यदि जानाति राघवः}
{त्वाम् च माम् चैव धर्म आत्मा न अभिभाषिष्यते ध्रुवम्} %2-78-23

\twolineshloka
{भरतस्य वचः श्रुत्वा शत्रुघ्नः लक्ष्मण अनुजः}
{न्यवर्तत ततः रोषात् ताम् मुमोच च मन्थराम्} %2-78-24

\twolineshloka
{सा पाद मूले कैकेय्या मन्थरा निपपात ह}
{निह्श्वसन्ती सुदुह्ख आर्ता कृपणम् विललाप च} %2-78-25

\fourlineindentedshloka
{शत्रुघ्न विक्षेप विमूढ सम्ज्ञाम्}
{समीक्ष्य कुब्जाम् भरतस्य माता}
{शनैः समाश्वासयद् आर्त रूपाम्}
{क्रौन्चीम् विलग्नाम् इव वीक्षमाणाम्} %2-78-26


॥इत्यार्षे श्रीमद्रामायणे वाल्मीकीये आदिकाव्ये अयोध्याकाण्डे कुब्जाविक्षेपः नाम अष्टसप्ततितमः सर्गः ॥२-७८॥
