\sect{त्रयोविंशः सर्गः — लक्ष्मणक्रोधः}

\twolineshloka
{इति ब्रुवति रामे तु लक्ष्मणोऽवाक् शिरा इव}
{ध्यात्वा मध्यं जगामाशु सहसा दैन्यहर्षयोः} %2-23-1

\twolineshloka
{तदा तु बद्ध्वा भ्रुकुटीं भ्रुवोर्मध्ये नरर्षभः}
{निशश्वास महासर्पो बिलस्थ इव रोषितः} %2-23-2

\twolineshloka
{तस्य दुष्प्रतिवीक्ष्यं तद् भ्रुकुटीसहितं तदा}
{बभौ क्रुद्धस्य सिंहस्य मुखस्य सदृशं मुखम्} %2-23-3

\twolineshloka
{अग्रहस्तं विधुन्वंस्तु हस्ती हस्तमिवात्मनः}
{तिर्यगूर्ध्वं शरीरे च पातयित्वा शिरोधराम्} %2-23-4

\twolineshloka
{अग्राक्ष्णा वीक्षमाणस्तु तिर्यग्भ्रातरमब्रवीत्}
{अस्थाने सम्भ्रमो यस्य जातो वै सुमहानयम्} %2-23-5

\twolineshloka
{धर्मदोषप्रसङ्गेन लोकस्यानतिशङ्कया}
{कथं ह्येतदसम्भ्रान्तस्त्वद्विधो वक्तुमर्हति} %2-23-6

\twolineshloka
{यथा ह्येवमशौण्डीरं शौण्डीरः क्षत्रियर्षभः}
{किं नाम कृपणं दैवमशक्तमभिशंससि} %2-23-7

\twolineshloka
{पापयोस्ते कथं नाम तयोः शङ्का न विद्यते}
{सन्ति धर्मोपधासक्ता धर्मात्मन् किं न बुध्यसे} %2-23-8

\threelineshloka
{तयोः सुचरितं स्वार्थं शाठ्यात् परिजिहीर्षतोः}
{यदि नैवं व्यवसितं स्याद्धि प्रागेव राघव}
{तयोः प्रागेव दत्तश्च स्याद् वरः प्रकृतश्च सः} %2-23-9

\twolineshloka
{लोकविद्विष्टमारब्धं त्वदन्यस्याभिषेचनम्}
{नोत्सहे सहितुं वीर तत्र मे क्षन्तुमर्हसि} %2-23-10

\twolineshloka
{येनैवमागता द्वैधं तव बुद्धिर्महामते}
{सोऽपि धर्मो मम द्वेष्यो यत्प्रसङ्गाद् विमुह्यसि} %2-23-11

\twolineshloka
{कथं त्वं कर्मणा शक्तः कैकेयीवशवर्तिनः}
{करिष्यसि पितुर्वाक्यमधर्मिष्ठं विगर्हितम्} %2-23-12

\twolineshloka
{यदयं किल्बिषाद् भेदः कृतोऽप्येवं न गृह्यते}
{जायते तत्र मे दुःखं धर्मसङ्गश्च गर्हितः} %2-23-13

\threelineshloka
{तवायं धर्मसंयोगो लोकस्यास्य विगर्हितः}
{मनसापि कथं कामं कुर्यात् त्वां कामवृत्तयोः}
{तयोस्त्वहितयोर्नित्यं शत्र्वोः पित्रभिधानयोः} %2-23-14

\twolineshloka
{यद्यपि प्रतिपत्तिस्ते दैवी चापि तयोर्मतम्}
{तथाप्युपेक्षणीयं ते न मे तदपि रोचते} %2-23-15

\twolineshloka
{विक्लवो वीर्यहीनो यः स दैवमनुवर्तते}
{वीराः सम्भावितात्मानो न दैवं पर्युपासते} %2-23-16

\twolineshloka
{दैवं पुरुषकारेण यः समर्थः प्रबाधितुम्}
{न दैवेन विपन्नार्थः पुरुषः सोऽवसीदति} %2-23-17

\twolineshloka
{द्रक्ष्यन्ति त्वद्य दैवस्य पौरुषं पुरुषस्य च}
{दैवमानुषयोरद्य व्यक्ता व्यक्तिर्भविष्यति} %2-23-18

\twolineshloka
{अद्य मे पौरुषहतं दैवं द्रक्ष्यन्ति वै जनाः}
{यैर्दैवादाहतं तेऽद्य दृष्टं राज्याभिषेचनम्} %2-23-19

\twolineshloka
{अत्यङ्कुशमिवोद्दामं गजं मदजलोद्धतम्}
{प्रधावितमहं दैवं पौरुषेण निवर्तये} %2-23-20

\twolineshloka
{लोकपालाः समस्तास्ते नाद्य रामाभिषेचनम्}
{न च कृत्स्नास्त्रयो लोका विहन्युः किं पुनः पिता} %2-23-21

\twolineshloka
{यैर्विवासस्तवारण्ये मिथो राजन् समर्थितः}
{अरण्ये ते विवत्स्यन्ति चतुर्दश समास्तथा} %2-23-22

\twolineshloka
{अहं तदाशां धक्ष्यामि पितुस्तस्याश्च या तव}
{अभिषेकविघातेन पुत्रराज्याय वर्तते} %2-23-23

\twolineshloka
{मद्बलेन विरुद्धाय न स्याद् दैवबलं तथा}
{प्रभविष्यति दुःखाय यथोग्रं पौरुषं मम} %2-23-24

\twolineshloka
{ऊर्ध्वं वर्षसहस्रान्ते प्रजापाल्यमनन्तरम्}
{आर्यपुत्राः करिष्यन्ति वनवासं गते त्वयि} %2-23-25

\twolineshloka
{पूर्वराजर्षिवृत्त्या हि वनवासोऽभिधीयते}
{प्रजा निक्षिप्य पुत्रेषु पुत्रवत् परिपालने} %2-23-26

\twolineshloka
{स चेद् राजन्यनेकाग्रे राज्यविभ्रमशङ्कया}
{नैवमिच्छसि धर्मात्मन् राज्यं राम त्वमात्मनि} %2-23-27

\twolineshloka
{प्रतिजाने च ते वीर मा भूवं वीरलोकभाक्}
{राज्यं च तव रक्षेयमहं वेलेव सागरम्} %2-23-28

\twolineshloka
{मङ्गलैरभिषिञ्चस्व तत्र त्वं व्यापृतो भव}
{अहमेको महीपालानलं वारयितुं बलात्} %2-23-29

\twolineshloka
{न शोभार्थाविमौ बाहू न धनुर्भूषणाय मे}
{नासिराबन्धनार्थाय न शराः स्तम्भहेतवः} %2-23-30

\twolineshloka
{अमित्रमथनार्थाय सर्वमेतच्चतुष्टयम्}
{न चाहं कामयेऽत्यर्थं यः स्याच्छत्रुर्मतो मम} %2-23-31

\twolineshloka
{असिना तीक्ष्णधारेण विद्युच्चलितवर्चसा}
{प्रगृहीतेन वै शत्रुं वज्रिणं वा न कल्पये} %2-23-32

\twolineshloka
{खड्गनिष्पेषनिष्पिष्टैर्गहना दुश्चरा च मे}
{हस्त्यश्वरथिहस्तोरुशिरोभिर्भविता मही} %2-23-33

\twolineshloka
{खड्गधाराहता मेऽद्य दीप्यमाना इवाग्नयः}
{पतिष्यन्ति द्विषो भूमौ मेघा इव सविद्युतः} %2-23-34

\twolineshloka
{बद्धगोधाङ्गुलित्राणे प्रगृहीतशरासने}
{कथं पुरुषमानी स्यात् पुरुषाणां मयि स्थिते} %2-23-35

\twolineshloka
{बहुभिश्चैकमत्यस्यन्नेकेन च बहूञ्जनान्}
{विनियोक्ष्याम्यहं बाणान्नृवाजिगजमर्मसु} %2-23-36

\twolineshloka
{अद्य मेऽस्त्रप्रभावस्य प्रभावः प्रभविष्यति}
{राज्ञश्चाप्रभुतां कर्तुं प्रभुत्वं च तव प्रभो} %2-23-37

\twolineshloka
{अद्य चन्दनसारस्य केयूरामोक्षणस्य च}
{वसूनां च विमोक्षस्य सुहृदां पालनस्य च} %2-23-38

\twolineshloka
{अनुरूपाविमौ बाहू राम कर्म करिष्यतः}
{अभिषेचनविघ्नस्य कर्तॄणां ते निवारणे} %2-23-39

\twolineshloka
{ब्रवीहि कोऽद्यैव मया वियुज्यतां तवासुहृत् प्राणयशःसुहृज्जनैः}
{यथा तवेयं वसुधा वशा भवेत् तथैव मां शाधि तवास्मि किङ्करः} %2-23-40

\twolineshloka
{विमृज्य बाष्पं परिसान्त्व्य चासकृत् स लक्ष्मणं राघववंशवर्धनः}
{उवाच पित्रोर्वचने व्यवस्थितं निबोध मामेष हि सौम्य सत्पथः} %2-23-41


॥इत्यार्षे श्रीमद्रामायणे वाल्मीकीये आदिकाव्ये अयोध्याकाण्डे लक्ष्मणक्रोधः नाम त्रयोविंशः सर्गः ॥२-२३॥
