\sect{अष्टषष्ठितमः सर्गः — दूतप्रेषणम्}

\twolineshloka
{तेषाम् तत् वचनम् श्रुत्वा वसिष्ठः प्रत्युवाच ह}
{मित्र अमात्य गणान् सर्वान् ब्राह्मणाम्स् तान् इदम् वचः} %2-68-1

\twolineshloka
{यद् असौ मातुल कुले पुरे राज गृहे सुखी}
{भरतः वसति भ्रात्रा शत्रुघ्नेन समन्वितः} %2-68-2

\twolineshloka
{तत् शीघ्रम् जवना दूता गच्चन्तु त्वरितैः हयैः}
{आनेतुम् भ्रातरौ वीरौ किम् समीक्षामहे वयम्} %2-68-3

\twolineshloka
{गच्चन्तु इति ततः सर्वे वसिष्ठम् वाक्यम् अब्रुवन्}
{तेषाम् तत् वचनम् श्रुत्वा वसिष्ठो वाक्यम् अब्रवीत्} %2-68-4

\twolineshloka
{एहि सिद्ध अर्थ विजय जयन्त अशोक नन्दन}
{श्रूयताम् इतिकर्तव्यम् सर्वान् एव ब्रवीमि वः} %2-68-5

\twolineshloka
{पुरम् राज गृहम् गत्वा शीघ्रम् शीघ्र जवैः हयैः}
{त्यक्त शोकैः इदम् वाच्यः शासनात् भरतः मम} %2-68-6

\twolineshloka
{पुरोहितः त्वाम् कुशलम् प्राह सर्वे च मन्त्रिणः}
{त्वरमाणः च निर्याहि कृत्यम् आत्ययिकम् त्वया} %2-68-7

\twolineshloka
{मा च अस्मै प्रोषितम् रामम् मा च अस्मै पितरम् मृतम्}
{भवन्तः शम्सिषुर् गत्वा राघवाणाम् इमम् क्षयम्} %2-68-8

\twolineshloka
{कौशेयानि च वस्त्राणि भूषणानि वराणि च}
{क्षिप्रम् आदाय राज्ञः च भरतस्य च गच्चत} %2-68-9

\twolineshloka
{दत्तपथ्यशना दूताजग्मुः स्वम् स्वम् निवेशनम्}
{केकयाम्स्ते गमिष्यन्तो हयानारुह्य सम्मतान्} %2-68-10

\twolineshloka
{ततः प्रास्थानिकम् कृत्वा कार्यशेषमनन्तरम्}
{वसिष्ठेनाभ्यनुज्ञाता दूताः सम्त्वरिता ययुः} %2-68-11

\twolineshloka
{न्यन्तेनापरतालस्य प्रलम्बस्योत्तरम् प्रति}
{निषेवमाणास्ते जग्मुर्नदीम् मध्येन मालिनीम्} %2-68-12

\twolineshloka
{ते हस्तिनापुरे गङ्गाम् तीर्त्वा प्रत्यङ्मुखा ययुः}
{पाञलदेशमासाद्य मध्येन कुरुजाङ्गलम्} %2-68-13

\twolineshloka
{सराम्सि च सुपूर्णानि नदीश्च विमलोदकाः}
{निरीक्षमाणास्ते जग्मुर्दूताः कार्यवशाद्द्रुतम्} %2-68-14

\twolineshloka
{ते प्रसन्नोदकाम् दिव्याम् नानाविहगसेविताम्}
{उपातिजग्मुर्वेगेन शरदण्डाम् जनाकुलाम्} %2-68-15

\twolineshloka
{निकूलवृक्षमासाद्य दिव्यम् सत्योपयाचनम्}
{अभिगम्याभिवाद्यम् तम् कुलिङ्गाम् प्राविशन् पुरीम्} %2-68-16

\twolineshloka
{अभिकालम् ततः प्राप्यते बोधिभवनाच्च्युताम्}
{पितृपैतामहीम् पुण्याम् तेरुरिक्षुमतीम् नदीम्} %2-68-17

\twolineshloka
{अवेक्स्याञ्जलिपानाम्श्च ब्राह्मणान् वेदपारगान्}
{ययुर्मध्येन बाह्लीकान् सुदामानम् च पर्वतम्} %2-68-18

\twolineshloka
{विष्णोः पदम् प्रेक्षमाणा विपाशाम् चापि शाल्मालीम्}
{नदीर्वापीस्तटाकानि पल्वलानि सराम्सि च} %2-68-19

\twolineshloka
{पस्यन्तो विविधाम्श्चापि सिमहव्याग्रमृगद्विपान्}
{ययुः पथातिमहता शासनम् भर्तुरीप्सवः} %2-68-20

\twolineshloka
{ते श्रान्त वाहना दूता विकृष्टेन सता पथा}
{गिरि व्रजम् पुर वरम् शीघ्रम् आसेदुर् अन्जसा} %2-68-21

\fourlineindentedshloka
{भर्तुः प्रिय अर्थम् कुल रक्षण अर्थम्}
{भर्तुः च वम्शस्य परिग्रह अर्थम्}
{अहेडमानाः त्वरया स्म दूता}
{रात्र्याम् तु ते तत् पुरम् एव याताः} %2-68-22


॥इत्यार्षे श्रीमद्रामायणे वाल्मीकीये आदिकाव्ये अयोध्याकाण्डे दूतप्रेषणम् नाम अष्टषष्ठितमः सर्गः ॥२-६८॥
