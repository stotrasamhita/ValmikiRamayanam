\sect{अष्टादशः सर्गः — वनवासनिर्देशः}

\twolineshloka
{स ददर्शासने रामो विषण्णं पितरं शुभे}
{कैकेय्या सहितं दीनं मुखेन परिशुष्यता} %2-18-1

\twolineshloka
{स पितुश्चरणौ पूर्वमभिवाद्य विनीतवत्}
{ततो ववन्दे चरणौ कैकेय्याः सुसमाहितः} %2-18-2

\twolineshloka
{रामेत्युक्त्वा तु वचनं बाष्पपर्याकुलेक्षणः}
{शशाक नृपतिर्दीनो नेक्षितुं नाभिभाषितुम्} %2-18-3

\twolineshloka
{तदपूर्वं नरपतेर्दृष्ट्वा रूपं भयावहम्}
{रामोऽपि भयमापन्नः पदा स्पृष्ट्वेव पन्नगम्} %2-18-4

\twolineshloka
{इन्द्रियैरप्रहृष्टैस्तं शोकसन्तापकर्शितम्}
{निःश्वसन्तं महाराजं व्यथिताकुलचेतसम्} %2-18-5

\twolineshloka
{ऊर्मिमालिनमक्षोभ्यं क्षुभ्यन्तमिव सागरम्}
{उपप्लुतमिवादित्यमुक्तानृतमृषिं यथा} %2-18-6

\twolineshloka
{अचिन्त्यकल्पं नृपतेस्तं शोकमुपधारयन्}
{बभूव संरब्धतरः समुद्र इव पर्वणि} %2-18-7

\twolineshloka
{चिन्तयामास चतुरो रामः पितृहिते रतः}
{किंस्विदद्यैव नृपतिर्न मां प्रत्यभिनन्दति} %2-18-8

\twolineshloka
{अन्यदा मां पिता दृष्ट्वा कुपितोऽपि प्रसीदति}
{तस्य मामद्य सम्प्रेक्ष्य किमायासः प्रवर्तते} %2-18-9

\twolineshloka
{स दीन इव शोकार्तो विषण्णवदनद्युतिः}
{कैकेयीमभिवाद्यैव रामो वचनमब्रवीत्} %2-18-10

\twolineshloka
{कच्चिन्मया नापराद्धमज्ञानाद् येन मे पिता}
{कुपितस्तन्ममाचक्ष्व त्वमेवैनं प्रसादय} %2-18-11

\twolineshloka
{अप्रसन्नमनाः किं नु सदा मां प्रति वत्सलः}
{विषण्णवदनो दीनः नहि मां प्रति भाषते} %2-18-12

\twolineshloka
{शारीरो मानसो वापि कच्चिदेनं न बाधते}
{सन्तापो वाभितापो वा दुर्लभं हि सदा सुखम्} %2-18-13

\twolineshloka
{कच्चिन्न किञ्चिद् भरते कुमारे प्रियदर्शने}
{शत्रुघ्ने वा महासत्त्वे मातॄणां वा ममाशुभम्} %2-18-14

\twolineshloka
{अतोषयन् महाराजमकुर्वन् वा पितुर्वचः}
{मुहूर्तमपि नेच्छेयं जीवितुं कुपिते नृपे} %2-18-15

\twolineshloka
{यतोमूलं नरः पश्येत् प्रादुर्भावमिहात्मनः}
{कथं तस्मिन् न वर्तेत प्रत्यक्षे सति दैवते} %2-18-16

\twolineshloka
{कच्चित्ते परुषं किञ्चिदभिमानात् पिता मम}
{उक्तो भवत्या रोषेण येनास्य लुलितं मनः} %2-18-17

\twolineshloka
{एतदाचक्ष्व मे देवि तत्त्वेन परिपृच्छतः}
{किन्निमित्तमपूर्वोऽयं विकारो मनुजाधिपे} %2-18-18

\twolineshloka
{एवमुक्ता तु कैकेयी राघवेण महात्मना}
{उवाचेदं सुनिर्लज्जा धृष्टमात्महितं वचः} %2-18-19

\twolineshloka
{न राजा कुपितो राम व्यसनं नास्य किञ्चन}
{किञ्चिन्मनोगतं त्वस्य त्वद्भयान्नानुभाषते} %2-18-20

\twolineshloka
{प्रियं त्वामप्रियं वक्तुं वाणी नास्य प्रवर्तते}
{तदवश्यं त्वया कार्यं यदनेनाश्रुतं मम} %2-18-21

\twolineshloka
{एष मह्यं वरं दत्त्वा पुरा मामभिपूज्य च}
{स पश्चात् तप्यते राजा यथान्यः प्राकृतस्तथा} %2-18-22

\twolineshloka
{अतिसृज्य ददानीति वरं मम विशाम्पतिः}
{स निरर्थं गतजले सेतुं बन्धितुमिच्छति} %2-18-23

\twolineshloka
{धर्ममूलमिदं राम विदितं च सतामपि}
{तत् सत्यं न त्यजेद् राजा कुपितस्त्वत्कृते यथा} %2-18-24

\twolineshloka
{यदि तद् वक्ष्यते राजा शुभं वा यदि वाशुभम्}
{करिष्यसि ततः सर्वमाख्यास्यामि पुनस्त्वहम्} %2-18-25

\twolineshloka
{यदि त्वभिहितं राज्ञा त्वयि तन्न विपत्स्यते}
{ततोऽहमभिधास्यामि न ह्येष त्वयि वक्ष्यति} %2-18-26

\twolineshloka
{एतत् तु वचनं श्रुत्वा कैकेय्या समुदाहृतम्}
{उवाच व्यथितो रामस्तां देवीं नृपसन्निधौ} %2-18-27

\twolineshloka
{अहो धिङ् नार्हसे देवि वक्तुं मामीदृशं वचः}
{अहं हि वचनाद् राज्ञः पतेयमपि पावके} %2-18-28

\twolineshloka
{भक्षयेयं विषं तीक्ष्णं पतेयमपि चार्णवे}
{नियुक्तो गुरुणा पित्रा नृपेण च हितेन च} %2-18-29

\twolineshloka
{तद् ब्रूहि वचनं देवि राज्ञो यदभिकाङ्क्षितम्}
{करिष्ये प्रतिजाने च रामो द्विर्नाभिभाषते} %2-18-30

\twolineshloka
{तमार्जवसमायुक्तमनार्या सत्यवादिनम्}
{उवाच रामं कैकेयी वचनं भृशदारुणम्} %2-18-31

\twolineshloka
{पुरा देवासुरे युद्धे पित्रा ते मम राघव}
{रक्षितेन वरौ दत्तौ सशल्येन महारणे} %2-18-32

\twolineshloka
{तत्र मे याचितो राजा भरतस्याभिषेचनम्}
{गमनं दण्डकारण्ये तव चाद्यैव राघव} %2-18-33

\twolineshloka
{यदि सत्यप्रतिज्ञं त्वं पितरं कर्तुमिच्छसि}
{आत्मानं च नरश्रेष्ठ मम वाक्यमिदं शृणु} %2-18-34

\twolineshloka
{सन्निदेशे पितुस्तिष्ठ यथानेन प्रतिश्रुतम्}
{त्वयारण्यं प्रवेष्टव्यं नव वर्षाणि पञ्च च} %2-18-35

\twolineshloka
{भरतश्चाभिषिच्येत यदेतदभिषेचनम्}
{त्वदर्थे विहितं राज्ञा तेन सर्वेण राघव} %2-18-36

\twolineshloka
{सप्त सप्त च वर्षाणि दण्डकारण्यमाश्रितः}
{अभिषेकमिदं त्यक्त्वा जटाचीरधरो भव} %2-18-37

\twolineshloka
{भरतः कोसलपतेः प्रशास्तु वसुधामिमाम्}
{नानारत्नसमाकीर्णां सवाजिरथसङ्कुलाम्} %2-18-38

\twolineshloka
{एतेन त्वां नरेन्द्रोऽयं कारुण्येन समाप्लुतः}
{शोकैः सङ्क्लिष्टवदनो न शक्नोति निरीक्षितुम्} %2-18-39

\twolineshloka
{एतत् कुरु नरेन्द्रस्य वचनं रघुनन्दन}
{सत्येन महता राम तारयस्व नरेश्वरम्} %2-18-40

\twolineshloka
{इतीव तस्यां परुषं वदन्त्यां न चैव रामः प्रविवेश शोकम्}
{प्रविव्यथे चापि महानुभावो राजा च पुत्रव्यसनाभितप्तः} %2-18-41


॥इत्यार्षे श्रीमद्रामायणे वाल्मीकीये आदिकाव्ये अयोध्याकाण्डे वनवासनिर्देशः नाम अष्टादशः सर्गः ॥२-१८॥
