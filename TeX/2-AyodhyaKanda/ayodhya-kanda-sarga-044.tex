\sect{चतुश्चत्वारिंशः सर्गः — सुमित्राश्वासनम्}

\twolineshloka
{विलपन्तीं तथा तां तु कौसल्यां प्रमदोत्तमाम्}
{इदं धर्मे स्थिता धर्म्यं सुमित्रा वाक्यमब्रवीत्} %2-44-1

\twolineshloka
{तवार्ये सद्गुणैर्युक्तः स पुत्रः पुरुषोत्तमः}
{किं ते विलपितेनैवं कृपणं रुदितेन वा} %2-44-2

\twolineshloka
{यस्तवार्ये गतः पुत्रस्त्यक्त्वा राज्यं महाबलः}
{साधु कुर्वन् महात्मानं पितरं सत्यवादिनम्} %2-44-3

\twolineshloka
{शिष्टैराचरिते सम्यक्शश्वत् प्रेत्य फलोदये}
{रामो धर्मे स्थितः श्रेष्ठो न स शोच्यः कदाचन} %2-44-4

\twolineshloka
{वर्तते चोत्तमां वृत्तिं लक्ष्मणोऽस्मिन् सदानघः}
{दयावान् सर्वभूतेषु लाभस्तस्य महात्मनः} %2-44-5

\twolineshloka
{अरण्यवासे यद् दुःखं जानन्त्येव सुखोचिता}
{अनुगच्छति वैदेही धर्मात्मानं तवात्मजम्} %2-44-6

\twolineshloka
{कीर्तिभूतां पताकां यो लोके भ्रमयति प्रभुः}
{धर्मः सत्यव्रतपरः किं न प्राप्तस्तवात्मजः} %2-44-7

\twolineshloka
{व्यक्तं रामस्य विज्ञाय शौचं माहात्म्यमुत्तमम्}
{न गात्रमंशुभिः सूर्यः सन्तापयितुमर्हति} %2-44-8

\twolineshloka
{शिवः सर्वेषु कालेषु काननेभ्यो विनिःसृतः}
{राघवं युक्तशीतोष्णः सेविष्यति सुखोऽनिलः} %2-44-9

\twolineshloka
{शयानमनघं रात्रौ पितेवाभिपरिष्वजन्}
{घर्मघ्नः संस्पृशन् शीतश्चन्द्रमा ह्लादयिष्यति} %2-44-10

\twolineshloka
{ददौ चास्त्राणि दिव्यानि यस्मै ब्रह्मा महौजसे}
{दानवेन्द्रं हतं दृष्ट्वा तिमिध्वजसुतं रणे} %2-44-11

\twolineshloka
{स शूरः पुरुषव्याघ्रः स्वबाहुबलमाश्रितः}
{असन्त्रस्तो ह्यरण्येऽसौ वेश्मनीव निवत्स्यते} %2-44-12

\twolineshloka
{यस्येषुपथमासाद्य विनाशं यान्ति शत्रवः}
{कथं न पृथिवी तस्य शासने स्थातुमर्हति} %2-44-13

\twolineshloka
{या श्रीः शौर्यं च रामस्य या च कल्याणसत्त्वता}
{निवृत्तारण्यवासः स्वं क्षिप्रं राज्यमवाप्स्यति} %2-44-14

\twolineshloka
{सूर्यस्यापि भवेत् सूर्यो ह्यग्नेरग्नः प्रभोः प्रभुः}
{श्रियाः श्रीश्च भवेदग्र्या कीर्त्याः कीर्तिः क्षमाक्षमा} %2-44-15

\twolineshloka
{दैवतं देवतानां च भूतानां भूतसत्तमः}
{तस्य के ह्यगुणा देवि वने वाप्यथवा पुरे} %2-44-16

\twolineshloka
{पृथिव्या सह वैदेह्या श्रिया च पुरुषर्षभः}
{क्षिप्रं तिसृभिरेताभिः सह रामोऽभिषेक्ष्यते} %2-44-17

\twolineshloka
{दुःखजं विसृजत्यश्रु निष्क्रामन्तमुदीक्ष्य यम्}
{अयोध्यायां जनः सर्वः शोकवेगसमाहतः} %2-44-18

\twolineshloka
{कुशचीरधरं वीरं गच्छन्तमपराजितम्}
{सीतेवानुगता लक्ष्मीस्तस्य किं नाम दुर्लभम्} %2-44-19

\twolineshloka
{धनुर्ग्रहवरो यस्य बाणखड्गास्त्रभृत् स्वयम्}
{लक्ष्मणो व्रजति ह्यग्रे तस्य किं नाम दुर्लभम्} %2-44-20

\twolineshloka
{निवृत्तवनवासं तं द्रष्टासि पुनरागतम्}
{जहि शोकं च मोहं च देवि सत्यं ब्रवीमि ते} %2-44-21

\twolineshloka
{शिरसा चरणावेतौ वन्दमानमनिन्दिते}
{पुनर्द्रक्ष्यसि कल्याणि पुत्रं चन्द्रमिवोदितम्} %2-44-22

\twolineshloka
{पुनः प्रविष्टं दृष्ट्वा तमभिषिक्तं महाश्रियम्}
{समुत्स्रक्ष्यसि नेत्राभ्यां शीघ्रमानन्दजं जलम्} %2-44-23

\twolineshloka
{मा शोको देवि दुःखं वा न रामे दृष्यतेऽशिवम्}
{क्षिप्रं द्रक्ष्यसि पुत्रं त्वं ससीतं सहलक्ष्मणम्} %2-44-24

\twolineshloka
{त्वयाऽशेषो जनश्चायं समाश्वास्यो यतोऽनघे}
{कमिदानीमिदं देवि करोषि हृदि विक्लवम्} %2-44-25

\twolineshloka
{नार्हा त्वं शोचितुं देवि यस्यास्ते राघवः सुतः}
{नहि रामात् परो लोके विद्यते सत्पथे स्थितः} %2-44-26

\twolineshloka
{अभिवादयमानं तं दृष्ट्वा ससुहृदं सुतम्}
{मुदाश्रु मोक्ष्यसे क्षिप्रं मेघरेखेव वार्षिकी} %2-44-27

\twolineshloka
{पुत्रस्ते वरदः क्षिप्रमयोध्यां पुनरागतः}
{कराभ्यां मृदुपीनाभ्यां चरणौ पीडयिष्यति} %2-44-28

\twolineshloka
{अभिवाद्य नमस्यन्तं शूरं ससुहृदं सुतम्}
{मुदास्रैः प्रोक्षसे पुत्रं मेघराजिरिवाचलम्} %2-44-29

\twolineshloka
{आश्वासयन्ती विविधैश्च वाक्यैर्वाक्योपचारे कुशलानवद्या}
{रामस्य तां मातरमेवमुक्त्वा देवी सुमित्रा विरराम रामा} %2-44-30

\twolineshloka
{निशम्य तल्लक्ष्मणमातृवाक्यं रामस्य मातुर्नरदेवपत्न्याः}
{सद्यः शरीरे विननाश शोकः शरद्गतो मेघ इवाल्पतोयः} %2-44-31


॥इत्यार्षे श्रीमद्रामायणे वाल्मीकीये आदिकाव्ये अयोध्याकाण्डे सुमित्राश्वासनम् नाम चतुश्चत्वारिंशः सर्गः ॥२-४४॥
