\sect{एकाधिकशततमः सर्गः — पितृदिष्टान्तश्रवणम्}

\twolineshloka
{रामस्य वचनं श्रुत्वा भरतः प्रत्युवाच ह}
{किं मे धर्माद्विहीनस्य राजधर्मः करिष्यति} %2-101-1

\twolineshloka
{शाश्वतोऽयं सदा धर्म्मः स्थितोऽस्मासु नरर्षभ}
{ज्येष्ठपुत्रे स्थिते राजन्न कनीयान् नृपो भवेत्} %2-101-2

\twolineshloka
{स समृद्धां मया सार्द्धमयोध्यां गच्छ राघव}
{अभिषेचय चात्मानं कुलस्यास्य भवाय नः} %2-101-3

\twolineshloka
{राजानं मानुषं प्राहुर्देवत्वे स मतो मम}
{यस्य धर्मार्थसहितं वृत्तमाहुरमानुषम्} %2-101-4

\twolineshloka
{केकयस्थे च मयि तु त्वयि चारण्यमाश्रिते}
{दिवमार्यो गतो राजा यायजूकः सतां मतः} %2-101-5

\twolineshloka
{निष्क्रान्तमात्रे भवति सहसीते सलक्ष्मणे}
{दुःखशोकाभिभूतस्तु राजा त्रिदिवमभ्यगात्} %2-101-6

\twolineshloka
{उत्तिष्ठ पुरुषव्याघ्र क्रियतामुदकं पितुः}
{ऺअहं चायं च शत्रुघ्नः पूर्वमेव कृतोदकौ} %2-101-7

\twolineshloka
{प्रियेण खलु दत्तं हि पितृलोकेषु राघव}
{अक्षय्यं भवतीत्याहुर्भवांश्चैव पितुः प्रियः} %2-101-8

\twolineshloka
{त्वामेव शोचंस्तव दर्शनेप्सुस्त्वय्येव सक्तामनिवर्त्य बुद्धिम्}
{त्वया विहीनस्तव शोकरुग्णस्त्वां संस्मरन्नस्तमितः पिता ते} %2-101-9


॥इत्यार्षे श्रीमद्रामायणे वाल्मीकीये आदिकाव्ये अयोध्याकाण्डे पितृदिष्टान्तश्रवणम् नाम एकाधिकशततमः सर्गः ॥२-१०१॥
