\sect{षष्ठः सर्गः — पौरोत्सेकः}

\twolineshloka
{गते पुरोहिते रामः स्नातो नियतमानसः}
{सह पत्न्या विशालाक्ष्या नारायणमुपागमत्} %2-6-1

\twolineshloka
{प्रगृह्य शिरसा पात्रीं हविषो विधिवत् ततः}
{महते दैवतायाज्यं जुहाव ज्वलितानले} %2-6-2

\twolineshloka
{शेषं च हविषस्तस्य प्राश्याशास्यात्मनः प्रियम्}
{ध्यायन्नारायणं देवं स्वास्तीर्णे कुशसंस्तरे} %2-6-3

\twolineshloka
{वाग्यतः सह वैदेह्या भूत्वा नियतमानसः}
{श्रीमत्यायतने विष्णोः शिश्ये नरवरात्मजः} %2-6-4

\twolineshloka
{एकयामावशिष्टायां रात्र्यां प्रतिविबुध्य सः}
{अलंकारविधिं सम्यक् कारयामास वेश्मनः} %2-6-5

\twolineshloka
{तत्र शृण्वन् सुखा वाचः सूतमागधवन्दिनाम्}
{पूर्वां संध्यामुपासीनो जजाप सुसमाहितः} %2-6-6

\twolineshloka
{तुष्टाव प्रणतश्चैव शिरसा मधुसूदनम्}
{विमलक्षौमसंवीतो वाचयामास स द्विजान्} %2-6-7

\twolineshloka
{तेषां पुण्याहघोषोऽथ गम्भीरमधुरस्तथा}
{अयोध्यां पूरयामास तूर्यघोषानुनादितः} %2-6-8

\twolineshloka
{कृतोपवासं तु तदा वैदेह्या सह राघवम्}
{अयोध्यानिलयः श्रुत्वा सर्वः प्रमुदितो जनः} %2-6-9

\twolineshloka
{ततः पौरजनः सर्वः श्रुत्वा रामाभिषेचनम्}
{प्रभातां रजनीं दृष्ट्वा चक्रे शोभयितुं पुरीम्} %2-6-10

\twolineshloka
{सिताभ्रशिखराभेषु देवतायतनेषु च}
{चतुष्पथेषु रथ्यासु चैत्येष्वट्टालकेषु च} %2-6-11

\twolineshloka
{नानापण्यसमृद्धेषु वणिजामापणेषु च}
{कुटुम्बिनां समृद्धेषु श्रीमत्सु भवनेषु च} %2-6-12

\twolineshloka
{सभासु चैव सर्वासु वृक्षेष्वालक्षितेषु च}
{ध्वजाः समुच्छ्रिताः साधु पताकाश्चाभवंस्तथा} %2-6-13

\twolineshloka
{नटनर्तकसङ्घानां गायकानां च गायताम्}
{मनःकर्णसुखा वाचः शुश्राव जनता ततः} %2-6-14

\twolineshloka
{रामाभिषेकयुक्ताश्च कथाश्चक्रुर्मिथो जनाः}
{रामाभिषेके सम्प्राप्ते चत्वरेषु गृहेषु च} %2-6-15

\twolineshloka
{बाला अपि क्रीडमाना गृहद्वारेषु सङ्घशः}
{रामाभिषवसंयुक्ताश्चक्रुरेव कथा मिथः} %2-6-16

\twolineshloka
{कृतपुष्पोपहारश्च धूपगन्धाधिवासितः}
{राजमार्गः कृतः श्रीमान् पौरै रामाभिषेचने} %2-6-17

\twolineshloka
{प्रकाशकरणार्थं च निशागमनशङ्कया}
{दीपवृक्षांस्तथा चक्रुरनुरथ्यासु सर्वशः} %2-6-18

\twolineshloka
{अलंकारं पुरस्यैवं कृत्वा तत् पुरवासिनः}
{आकांक्षमाणा रामस्य यौवराज्याभिषेचनम्} %2-6-19

\twolineshloka
{समेत्य सङ्घशः सर्वे चत्वरेषु सभासु च}
{कथयन्तो मिथस्तत्र प्रशशंसुर्जनाधिपम्} %2-6-20

\twolineshloka
{अहो महात्मा राजायमिक्ष्वाकुकुलनन्दनः}
{ज्ञात्वा वृद्धं स्वमात्मानं रामं राज्येऽभिषेक्ष्यति} %2-6-21

\twolineshloka
{सर्वे ह्यनुगृहीताः स्म यन्नो रामो महीपतिः}
{चिराय भविता गोप्ता दृष्टलोकपरावरः} %2-6-22

\twolineshloka
{अनुद्धतमना विद्वान् धर्मात्मा भ्रातृवत्सलः}
{यथा च भ्रातृषु स्निग्धस्तथास्मास्वपि राघवः} %2-6-23

\twolineshloka
{चिरं जीवतु धर्मात्मा राजा दशरथोऽनघः}
{यत्प्रसादेनाभिषिक्तं रामं द्रक्ष्यामहे वयम्} %2-6-24

\twolineshloka
{एवंविधं कथयतां पौराणां शुश्रुवुः परे}
{दिग्भ्यो विश्रुतवृत्तान्ताः प्राप्ता जानपदा जनाः} %2-6-25

\twolineshloka
{ते तु दिग्भ्यः पुरीं प्राप्ता द्रष्टुं रामाभिषेचनम्}
{रामस्य पूरयामासुः पुरीं जानपदा जनाः} %2-6-26

\twolineshloka
{जनौघैस्तैर्विसर्पद्भिः शुश्रुवे तत्र निःस्वनः}
{पर्वसूदीर्णवेगस्य सागरस्येव निःस्वनः} %2-6-27

\twolineshloka
{ततस्तदिन्द्रक्षयसंनिभं पुरं दिदृक्षुभिर्जानपदैरुपाहितैः}
{समन्ततः सस्वनमाकुलं बभौ समुद्रयादोभिरिवार्णवोदकम्} %2-6-28


॥इत्यार्षे श्रीमद्रामायणे वाल्मीकीये आदिकाव्ये अयोध्याकाण्डे पौरोत्सेकः नाम षष्ठः सर्गः ॥२-६॥
