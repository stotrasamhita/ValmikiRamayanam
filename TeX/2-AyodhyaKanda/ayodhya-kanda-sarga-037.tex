\sect{सप्तत्रिंशः सर्गः — चीरपरिग्रहनिमित्तवसिष्ठक्रोधः}

\twolineshloka
{महामात्रवचः श्रुत्वा रामो दशरथं तदा}
{अभ्यभाषत वाक्यं तु विनयज्ञो विनीतवत्} %2-37-1

\twolineshloka
{त्यक्तभोगस्य मे राजन् वने वन्येन जीवतः}
{किं कार्यमनुयात्रेण त्यक्तसङ्गस्य सर्वतः} %2-37-2

\twolineshloka
{यो हि दत्त्वा द्विपश्रेष्ठं कक्ष्यायां कुरुते मनः}
{रज्जुस्नेहेन किं तस्य त्यजतः कुञ्जरोत्तमम्} %2-37-3

\twolineshloka
{तथा मम सतां श्रेष्ठ किं ध्वजिन्या जगत्पते}
{सर्वाण्येवानुजानामि चीराण्येवानयन्तु मे} %2-37-4

\twolineshloka
{खनित्रपिटके चोभे समानयत गच्छत}
{चतुर्दश वने वासं वर्षाणि वसतो मम} %2-37-5

\twolineshloka
{अथ चीराणि कैकेयी स्वयमाहृत्य राघवम्}
{उवाच परिधत्स्वेति जनौघे निरपत्रपा} %2-37-6

\twolineshloka
{स चीरे पुरुषव्याघ्रः कैकेय्याः प्रतिगृह्य ते}
{सूक्ष्मवस्त्रमवक्षिप्य मुनिवस्त्राण्यवस्त ह} %2-37-7

\twolineshloka
{लक्ष्मणश्चापि तत्रैव विहाय वसने शुभे}
{तापसाच्छादने चैव जग्राह पितुरग्रतः} %2-37-8

\twolineshloka
{अथात्मपरिधानार्थं सीता कौशेयवासिनी}
{सम्प्रेक्ष्य चीरं संत्रस्ता पृषती वागुरामिव} %2-37-9

\twolineshloka
{सा व्यपत्रपमाणेव प्रगृह्य च सुदुर्मनाः}
{कैकेय्याः कुशचीरे ते जानकी शुभलक्षणा} %2-37-10

\twolineshloka
{अश्रुसम्पूर्णनेत्रा च धर्मज्ञा धर्मदर्शिनी}
{गन्धर्वराजप्रतिमं भर्तारमिदमब्रवीत्} %2-37-11

\twolineshloka
{कथं नु चीरं बघ्नन्ति मुनयो वनवासिनः}
{इति ह्यकुशला सीता सा मुमोह मुहुर्मुहुः} %2-37-12

\twolineshloka
{कृत्वा कण्ठे स्म सा चीरमेकमादाय पाणिना}
{तस्थौ ह्यकुशला तत्र व्रीडिता जनकात्मजा} %2-37-13

\twolineshloka
{तस्यास्तत् क्षिप्रमागत्य रामो धर्मभृतां वरः}
{चीरं बबन्ध सीतायाः कौशेयस्योपरि स्वयम्} %2-37-14

\twolineshloka
{रामं प्रेक्ष्य तु सीताया बध्नन्तं चीरमुत्तमम्}
{अन्तःपुरचरा नार्यो मुमुचुर्वारि नेत्रजम्} %2-37-15

\twolineshloka
{ऊचुश्च परमायत्ता रामं ज्वलिततेजसम्}
{वत्स नैवं नियुक्तेयं वनवासे मनस्विनी} %2-37-16

\twolineshloka
{पितुर्वाक्यानुरोधेन गतस्य विजनं वनम्}
{तावद् दर्शनमस्या नः सफलं भवतु प्रभो} %2-37-17

\twolineshloka
{लक्ष्मणेन सहायेन वनं गच्छस्व पुत्रक}
{नेयमर्हति कल्याणि वस्तुं तापसवद् वने} %2-37-18

\twolineshloka
{कुरु नो याचनां पुत्र सीता तिष्ठतु भामिनी}
{धर्मनित्यः स्वयं स्थातुं न हीदानीं त्वमिच्छसि} %2-37-19

\twolineshloka
{तासामेवंविधा वाचः शृण्वन् दशरथात्मजः}
{बबन्धैव तथा चीरं सीतया तुल्यशीलया} %2-37-20

\twolineshloka
{चीरे गृहीते तु तया सबाष्पो नृपतेर्गुरुः}
{निवार्य सीतां कैकेयीं वसिष्ठो वाक्यमब्रवीत्} %2-37-21

\twolineshloka
{अतिप्रवृत्ते दुर्मेधे कैकेयि कुलपांसनि}
{वञ्चयित्वा तु राजानं न प्रमाणेऽवतिष्ठसि} %2-37-22

\twolineshloka
{न गन्तव्यं वनं देव्या सीतया शीलवर्जिते}
{अनुष्ठास्यति रामस्य सीता प्रकृतमासनम्} %2-37-23

\twolineshloka
{आत्मा हि दाराः सर्वेषां दारसंग्रहवर्तिनाम्}
{आत्मेयमिति रामस्य पालयिष्यति मेदिनीम्} %2-37-24

\twolineshloka
{अथ यास्यति वैदेही वनं रामेण संगता}
{वयमत्रानुयास्यामः पुरं चेदं गमिष्यति} %2-37-25

\twolineshloka
{अन्तपालाश्च यास्यन्ति सदारो यत्र राघवः}
{सहोपजीव्यं राष्ट्रं च पुरं च सपरिच्छदम्} %2-37-26

\twolineshloka
{भरतश्च सशत्रुघ्नश्चीरवासा वनेचरः}
{वने वसन्तं काकुत्स्थमनुवत्स्यति पूर्वजम्} %2-37-27

\twolineshloka
{ततः शून्यां गतजनां वसुधां पादपैः सह}
{त्वमेका शाधि दुर्वृत्ता प्रजानामहिते स्थिता} %2-37-28

\twolineshloka
{न हि तद् भविता राष्ट्रं यत्र रामो न भूपतिः}
{तद् वनं भविता राष्ट्रं यत्र रामो निवत्स्यति} %2-37-29

\twolineshloka
{न ह्यदत्तां महीं पित्रा भरतः शास्तुमिच्छति}
{त्वयि वा पुत्रवद् वस्तुं यदि जातो महीपतेः} %2-37-30

\twolineshloka
{यद्यपि त्वं क्षितितलाद् गगनं चोत्पतिष्यसि}
{पितृवंशचरित्रज्ञः सोऽन्यथा न करिष्यति} %2-37-31

\twolineshloka
{तत् त्वया पुत्रगर्धिन्या पुत्रस्य कृतमप्रियम्}
{लोके नहि स विद्येत यो न राममनुव्रतः} %2-37-32

\twolineshloka
{द्रक्ष्यस्यद्यैव कैकेयि पशुव्यालमृगद्विजान्}
{गच्छतः सह रामेण पादपांश्च तदुन्मुखान्} %2-37-33

\twolineshloka
{अथोत्तमान्याभरणानि देवि देहि स्नुषायै व्यपनीय चीरम्}
{न चीरमस्याः प्रविधीयतेति न्यवारयत् तद् वसनं वसिष्ठः} %2-37-34

\twolineshloka
{एकस्य रामस्य वने निवासस्त्वया वृतः केकयराजपुत्रि}
{विभूषितेयं प्रतिकर्मनित्या वसत्वरण्ये सह राघवेण} %2-37-35

\twolineshloka
{यानैश्च मुख्यैः परिचारकैश्च सुसंवृता गच्छतु राजपुत्री}
{वस्त्रैश्च सर्वैः सहितैर्विधानैर्नेयं वृता ते वरसम्प्रदाने} %2-37-36

\twolineshloka
{तस्मिंस्तथा जल्पति विप्रमुख्ये गुरौ नृपस्याप्रतिमप्रभावे}
{नैव स्म सीता विनिवृत्तभावा प्रियस्य भर्तुः प्रतिकारकामा} %2-37-37


॥इत्यार्षे श्रीमद्रामायणे वाल्मीकीये आदिकाव्ये अयोध्याकाण्डे चीरपरिग्रहनिमित्तवसिष्ठक्रोधः नाम सप्तत्रिंशः सर्गः ॥२-३७॥
