\sect{सप्तविंशः सर्गः — पतिव्रताध्यवसायः}

\twolineshloka
{एवमुक्ता तु वैदेही प्रियार्हा प्रियवादिनी}
{प्रणयादेव सङ्क्रुद्धा भर्तारमिदमब्रवीत्} %2-27-1

\twolineshloka
{किमिदं भाषसे राम वाक्यं लघुतया ध्रुवम्}
{त्वया यदपहास्यं मे श्रुत्वा नरवरोत्तम} %2-27-2

\twolineshloka
{वीराणां राजपुत्राणां शस्त्रास्त्रविदुषां नृप}
{अनर्हमयशस्यं च न श्रोतव्यं त्वयेरितम्} %2-27-3

\twolineshloka
{आर्यपुत्र पिता माता भ्राता पुत्रस्तथा स्नुषा}
{स्वानि पुण्यानि भुञ्जानाः स्वं स्वं भाग्यमुपासते} %2-27-4

\twolineshloka
{भर्तुर्भाग्यं तु नार्येका प्राप्नोति पुरुषर्षभ}
{अतश्चैवाहमादिष्टा वने वस्तव्यमित्यपि} %2-27-5

\twolineshloka
{न पिता नात्मजो वात्मा न माता न सखीजनः}
{इह प्रेत्य च नारीणां पतिरेको गतिः सदा} %2-27-6

\twolineshloka
{यदि त्वं प्रस्थितो दुर्गं वनमद्यैव राघव}
{अग्रतस्ते गमिष्यामि मृद्नन्ती कुशकण्टकान्} %2-27-7

\twolineshloka
{ईर्ष्यां रोषं बहिष्कृत्य भुक्तशेषमिवोदकम्}
{नय मां वीर विस्रब्धः पापं मयि न विद्यते} %2-27-8

\twolineshloka
{प्रासादाग्रे विमानैर्वा वैहायसगतेन वा}
{सर्वावस्थागता भर्तुः पादच्छाया विशिष्यते} %2-27-9

\twolineshloka
{अनुशिष्टास्मि मात्रा च पित्रा च विविधाश्रयम्}
{नास्मि सम्प्रति वक्तव्या वर्तितव्यं यथा मया} %2-27-10

\twolineshloka
{अहं दुर्गं गमिष्यामि वनं पुरुषवर्जितम्}
{नानामृगगणाकीर्णं शार्दूलगणसेवितम्} %2-27-11

\twolineshloka
{सुखं वने निवत्स्यामि यथैव भवने पितुः}
{अचिन्तयन्ती त्रीँल्लोकांश्चिन्तयन्ती पतिव्रतम्} %2-27-12

\twolineshloka
{शुश्रूषमाणा ते नित्यं नियता ब्रह्मचारिणी}
{सह रंस्ये त्वया वीर वनेषु मधुगन्धिषु} %2-27-13

\twolineshloka
{त्वं हि कर्तुं वने शक्तो राम सम्परिपालनम्}
{अन्यस्यापि जनस्येह किं पुनर्मम मानद} %2-27-14

\twolineshloka
{साहं त्वया गमिष्यामि वनमद्य न संशयः}
{नाहं शक्या महाभाग निवर्तयितुमुद्यता} %2-27-15

\twolineshloka
{फलमूलाशना नित्यं भविष्यामि न संशयः}
{न ते दुःखं करिष्यामि निवसन्ती त्वया सदा} %2-27-16

\twolineshloka
{अग्रतस्ते गमिष्यामि भोक्ष्ये भुक्तवति त्वयि}
{इच्छामि परतः शैलान् पल्वलानि सरांसि च} %2-27-17

\twolineshloka
{द्रष्टुं सर्वत्र निर्भीता त्वया नाथेन धीमता}
{हंसकारण्डवाकीर्णाः पद्मिनीः साधुपुष्पिताः} %2-27-18

\twolineshloka
{इच्छेयं सुखिनी द्रष्टुं त्वया वीरेण सङ्गता}
{अभिषेकं करिष्यामि तासु नित्यमनुव्रता} %2-27-19

\twolineshloka
{सह त्वया विशालाक्ष रंस्ये परमनन्दिनी}
{एवं वर्षसहस्राणि शतं वापि त्वया सह} %2-27-20

\threelineshloka
{व्यतिक्रमं न वेत्स्यामि स्वर्गोऽपि हि न मे मतः}
{स्वर्गेऽपि च विना वासो भविता यदि राघव}
{त्वया विना नरव्याघ्र नाहं तदपि रोचये} %2-27-21

\twolineshloka
{अहं गमिष्यामि वनं सुदुर्गमं मृगायुतं वानरवारणैश्च}
{वने निवत्स्यामि यथा पितुर्गृहे तवैव पादावुपगृह्य सम्मता} %2-27-22

\twolineshloka
{अनन्यभावामनुरक्तचेतसं त्वया वियुक्तां मरणाय निश्चिताम्}
{नयस्व मां साधु कुरुष्व याचनां नातो मया ते गुरुता भविष्यति} %2-27-23

\twolineshloka
{तथा ब्रुवाणामपि धर्मवत्सलां न च स्म सीतां नृवरो निनीषति}
{उवाच चैनां बहु सन्निवर्तने वने निवासस्य च दुःखितां प्रति} %2-27-24


॥इत्यार्षे श्रीमद्रामायणे वाल्मीकीये आदिकाव्ये अयोध्याकाण्डे पतिव्रताध्यवसायः नाम सप्तविंशः सर्गः ॥२-२७॥
