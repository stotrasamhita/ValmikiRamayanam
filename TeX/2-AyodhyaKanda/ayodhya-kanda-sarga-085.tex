\sect{पञ्चाशीतितमः सर्गः — गुहसमागमः}

\twolineshloka
{एवम् उक्तः तु भरतः निषाद अधिपतिम् गुहम्}
{प्रत्युवाच महा प्राज्ञो वाक्यम् हेतु अर्थ सम्हितम्} %2-85-1

\twolineshloka
{ऊर्जितः खलु ते कामः कृतः मम गुरोह् सखे}
{यो मे त्वम् ईदृशीम् सेनाम् एको अभ्यर्चितुम् इच्चसि} %2-85-2

\twolineshloka
{इति उक्त्वा तु महा तेजा गुहम् वचनम् उत्तमम्}
{अब्रवीद् भरतः श्रीमान् निषाद अधिपतिम् पुनः} %2-85-3

\twolineshloka
{कतरेण गमिष्यामि भरद्वाज आश्रमम् गुह}
{गहनो अयम् भृशम् देशो गन्गा अनूपो दुरत्ययः} %2-85-4

\twolineshloka
{तस्य तत् वचनम् श्रुत्वा राज पुत्रस्य धीमतः}
{अब्रवीत् प्रान्जलिर् वाक्यम् गुहो गहन गोचरः} %2-85-5

\twolineshloka
{दाशाः तु अनुगमिष्यन्ति धन्विनः सुसमाहिताः}
{अहम् च अनुगमिष्यामि राज पुत्र महा यशः} %2-85-6

\twolineshloka
{कच्चिन् न दुष्टः व्रजसि रामस्य अक्लिष्ट कर्मणः}
{इयम् ते महती सेना शन्काम् जनयति इव मे} %2-85-7

\twolineshloka
{तम् एवम् अभिभाषन्तम् आकाशैव निर्मलः}
{भरतः श्लक्ष्णया वाचा गुहम् वचनम् अब्रवीत्} %2-85-8

\twolineshloka
{मा भूत् स कालो यत् कष्टम् न माम् शन्कितुम् अर्हसि}
{राघवः स हि मे भ्राता ज्येष्ठः पितृ समः मम} %2-85-9

\twolineshloka
{तम् निवर्तयितुम् यामि काकुत्स्थम् वन वासिनम्}
{बुद्धिर् अन्या न ते कार्या गुह सत्यम् ब्रवीमि ते} %2-85-10

\twolineshloka
{स तु सम्हृष्ट वदनः श्रुत्वा भरत भाषितम्}
{पुनर् एव अब्रवीद् वाक्यम् भरतम् प्रति हर्षितः} %2-85-11

\twolineshloka
{धन्यः त्वम् न त्वया तुल्यम् पश्यामि जगती तले}
{अयत्नात् आगतम् राज्यम् यः त्वम् त्यक्तुम् इह इच्चसि} %2-85-12

\twolineshloka
{शाश्वती खलु ते कीर्तिर् लोकान् अनुचरिष्यति}
{यः त्वम् कृच्च्र गतम् रामम् प्रत्यानयितुम् इच्चसि} %2-85-13

\twolineshloka
{एवम् सम्भाषमाणस्य गुहस्य भरतम् तदा}
{बभौ नष्ट प्रभः सूर्यो रजनी च अभ्यवर्तत} %2-85-14

\twolineshloka
{सम्निवेश्य स ताम् सेनाम् गुहेन परितोषितः}
{शत्रुघ्नेन सह श्रीमान् शयनम् पुनर् आगमत्} %2-85-15

\twolineshloka
{राम चिन्तामयः शोको भरतस्य महात्मनः}
{उपस्थितः हि अनर्हस्य धर्म प्रेक्षस्य तादृशः} %2-85-16

\twolineshloka
{अन्तर् दाहेन दहनः सम्तापयति राघवम्}
{वन दाह अभिसम्तप्तम् गूढो अग्निर् इव पादपम्} %2-85-17

\twolineshloka
{प्रस्रुतः सर्व गात्रेभ्यः स्वेदः शोक अग्नि सम्भवः}
{यथा सूर्य अम्शु सम्तप्तः हिमवान् प्रस्रुतः हिमम्} %2-85-18

\twolineshloka
{ध्यान निर्दर शैलेन विनिह्श्वसित धातुना}
{दैन्य पादप सम्घेन शोक आयास अधिशृन्गिणा} %2-85-19

\twolineshloka
{प्रमोह अनन्त सत्त्वेन सम्ताप ओषधि वेणुना}
{आक्रान्तः दुह्ख शैलेन महता कैकयी सुतः} %2-85-20

\fourlineindentedshloka
{विनिश्श्वसन्वै भृशदुर्मनास्ततः}
{प्रमूढसम्ज्ञः परमापदम् गतः}
{शमम् न लेभे हृदयज्वरार्दितो}
{नरर्षभो यूथहतो यथर्षभः} %2-85-21

\fourlineindentedshloka
{गुहेन सार्धम् भरतः समागतः}
{महा अनुभावः सजनः समाहितः}
{सुदुर्मनाः तम् भरतम् तदा पुनर्}
{गुहः समाश्वासयद् अग्रजम् प्रति} %2-85-22


॥इत्यार्षे श्रीमद्रामायणे वाल्मीकीये आदिकाव्ये अयोध्याकाण्डे गुहसमागमः नाम पञ्चाशीतितमः सर्गः ॥२-८५॥
