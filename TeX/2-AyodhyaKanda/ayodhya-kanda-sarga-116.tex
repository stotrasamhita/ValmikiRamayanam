\sect{षोडशाधिकशततमः सर्गः — खरविप्रकरणकथनम्}

\twolineshloka
{प्रतिप्रयाते भरते वसन् रामस्तपोवने}
{लक्षयामास सोद्वेगमथौत्सुक्यं तपस्विनाम्} %2-116-1

\twolineshloka
{ये तत्र चित्रकूटस्य पुरस्तात्तापसाश्रमे}
{राममाश्रित्य निरतास्तानलक्षयदुत्सुकान्} %2-116-2

\twolineshloka
{नयनैर्भुकुटीभिश्च रामं निर्दिश्य शङ्किताः}
{अन्योन्यमुपजल्पन्तः शनैश्चक्रुर्मिथः कथाः} %2-116-3

\twolineshloka
{तेषामौत्सुक्यमालक्ष्य रामस्त्वात्मनि शङ्कितः}
{कृताञ्जलिरुवाचेदमृषिं कुलपतिं ततः} %2-116-4

\twolineshloka
{न कच्चिद्भगवन् किञ्चित्पूर्ववृत्तमिदं मयि}
{दृश्यते विकृतं येन विक्रियन्ते तपस्विनः} %2-116-5

\twolineshloka
{प्रमादाच्चरितं कच्चित्किञ्चिन्नावरजस्य मे}
{लक्ष्मणस्यर्षिभिर्दृष्टं नानुरूपमिवात्मनः} %2-116-6

\twolineshloka
{कच्चिच्छुश्रूषमाणा वः शुश्रूषणपरा मयि}
{प्रममादाभ्युचितां वृत्तिं सीता युक्तं न वर्तते} %2-116-7

\twolineshloka
{अथर्षिर्जरया वृद्धस्तपसा च जरां गतः}
{वेपमान इवोवाच रामं भूतदयापरम्} %2-116-8

\twolineshloka
{कुतः कल्याणसत्त्वायाः कल्याणाभिरतेस्तथा}
{चलनं तात वैदेह्यास्तपस्विषु विशेषतः} %2-116-9

\twolineshloka
{त्वन्निमित्तमिदं तावत्तापसान् प्रति वर्तते}
{रक्षोभ्यस्तेन संविग्नाः कथयन्ति मिथः कथाः} %2-116-10

\twolineshloka
{रावणावरजः कश्चित् खरो नामेह राक्षसः}
{उत्पाट्य तापसान् सर्वान् जनस्थाननिकेतनान्} %2-116-11

\twolineshloka
{धृष्टश्च जितकाशीं च नृशंसः पुरुषादकः}
{अवलिप्तश्च पापश्च त्वां च तात न मृष्यते} %2-116-12

\twolineshloka
{त्वं यदाप्रभृति ह्यस्मिन्नाश्रमे तात वर्तसे}
{तदाप्रभृति रक्षांसि विप्रकुर्वन्ति तापसान्} %2-116-13

\twolineshloka
{दर्शयन्ति हि बीभत्सैः क्रूरैर्भीषणकैरपि}
{नानारूपैर्विरूपैश्च रूपैर्विकृतदर्शनैः} %2-116-14

\twolineshloka
{अप्रशस्तैरशुचिभिः सम्प्रयोज्य च तापसान्}
{प्रतिघ्नन्त्यपरान् क्षिप्रमनार्य्याः पुरतः स्थिताः} %2-116-15

\twolineshloka
{तेषु तेष्वाश्रमस्थानेष्वबुद्धमवलीय च}
{रमन्ते तापसांस्तत्र नाशयन्तोऽल्पयेतसः} %2-116-16

\twolineshloka
{अपक्षिपन्ति स्रुग्भाण्डानग्नीन् सिञ्चन्ति वारिणा}
{कलशांश्च प्रमृद्नन्ति हवने समुपस्थिते} %2-116-17

\twolineshloka
{तैर्दुरात्मभिरामृष्टानाश्रमान् प्रजिहासवः}
{गमनायान्यदेशस्य चोदयन्त्यृषयोऽद्य माम्} %2-116-18

\twolineshloka
{तत्पुरा राम शारीरीमुपहिंसां तपस्विषु}
{दर्शयन्ति हि दुष्टास्ते त्यक्ष्याम इममाश्रमम्} %2-116-19

\twolineshloka
{बहुमूलफलं चित्रमविदूरादितो वनम्}
{पुराणाश्रममेवाहं श्रयिष्ये सगणः पुनः} %2-116-20

\twolineshloka
{खरस्त्वय्यपि चायुक्तं पुरा तात प्रवर्त्तते}
{सहास्माभिरितो गच्छ यदि बुद्धिः प्रवर्त्तते} %2-116-21

\twolineshloka
{सकलत्रस्य सन्देहो नित्यं यत्तस्य राघव}
{समर्थस्यापि वसतो वासो दुःखमिहाद्य ते} %2-116-22

\twolineshloka
{इत्युक्तवन्तं रामस्तं राजपुत्रस्तपस्विनम्}
{न शशाकोत्तरैर्वाक्यैरवरोद्धुं समुत्सुकः} %2-116-23

\twolineshloka
{अभिनन्द्य समापृच्छ्य समाधाय च राघवम्}
{स जगामाश्रमं त्यक्त्वा कुलैः कुलपतिः सह} %2-116-24

\twolineshloka
{रामः संसाध्य त्वृषिगणमनुगमनाद्देशात्तस्मात् कुलपतिमभिवाद्य ऋषिम्}
{सम्यक्प्रीतैस्तैरनुमत उपदिष्टार्थः पुण्यं वासाय स्वनिलयमुपसम्पेदे} %2-116-25

\twolineshloka
{आश्रमं त्वृषिविरहितं प्रभुः क्षणमपि न विजहौ स राघवः}
{राघवं हि सततमनुगतास्तापसाश्चार्षचरितधृतगुणाः} %2-116-26


॥इत्यार्षे श्रीमद्रामायणे वाल्मीकीये आदिकाव्ये अयोध्याकाण्डे खरविप्रकरणकथनम् नाम षोडशाधिकशततमः सर्गः ॥२-११६॥
