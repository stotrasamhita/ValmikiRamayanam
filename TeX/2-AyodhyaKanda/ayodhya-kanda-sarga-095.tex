\sect{पञ्चनवतितमः सर्गः — मन्दाकिनीवर्णनम्}

\twolineshloka
{अथ शैलाद्विनिष्क्रम्य मैथिलीं कोसलेश्वरः}
{अदर्शयच्छुभजलां रम्यां मन्दाकिनीं नदीम्} %2-95-1

\twolineshloka
{अब्रवीच्च वरारोहां चारुचन्द्रनिभाननाम्}
{विदेहराजस्य सुतां रामो राजीवलोचनः} %2-95-2

\twolineshloka
{विचित्रपुलिनां रम्यां हंससारससेविताम्}
{कमलैरुपसम्पन्नां पश्य मन्दाकिनीं नदीम्} %2-95-3

\twolineshloka
{नानाविधैस्तीररुहैर्वृतां पुष्पफलद्रुमैः}
{राजन्तीं राजराजस्य नलिनीमिव सर्वतः} %2-95-4

\twolineshloka
{मृगयूथनिपीतानि कलुषाम्भांसि साम्प्रतम्}
{तीर्थानि रमणीयानि रतिं सञ्जनयन्ति मे} %2-95-5

\twolineshloka
{जटाजिनधराः काले वल्कलोत्तरवाससः}
{ऋषयस्त्ववगाहन्ते नदीं मन्दाकिनीं प्रिये} %2-95-6

\twolineshloka
{आदित्यमुपतिष्ठन्ते नियमादूर्द्ध्वबाहवः}
{एते परे विशालाक्षि मुनयः संशितव्रताः} %2-95-7

\twolineshloka
{मारुतोद्धूतशिखरैः प्रनृत्त इव पर्वतः}
{पादपैः पत्ऺत्रपुष्पाणि सृजद्भिरभितो नदीम्} %2-95-8

\twolineshloka
{क्वचिन्मणिनिकाशोदां क्वचित्पुलिनशालिनीम्}
{क्वचित्सिद्धजनाकीर्णां पश्य मन्दाकिनीं नदीम्} %2-95-9

\twolineshloka
{निर्द्धूतान् वायुना पश्य विततान् पुष्पसञ्चयान्}
{पोप्लूयमानानपरान् पश्य त्वं जलमध्यगान्} %2-95-10

\twolineshloka
{तांश्चातिवल्गुवचसो रथाङ्गाह्वयना द्विजाः}
{अधिरोहन्ति कल्याणि विकूजन्तः शुभा गिरः} %2-95-11

\twolineshloka
{दर्शनं चित्र कूटस्य मन्दाकिन्याश्च शोभने}
{अधिकं पुरवासाच्च मन्ये च तव दर्शनात्} %2-95-12

\twolineshloka
{विधूतकलुषैः सिद्धैस्तपोदमशमान्वितैः}
{नित्यविक्षोभितजलां विगाहस्व मया सह} %2-95-13

\twolineshloka
{सखीवच्च विगाहस्व सीते मन्दाकिनीं नदीम्}
{कमलान्यवमज्जन्ती पुष्कराणि च भामिनि} %2-95-14

\twolineshloka
{त्वं पौरजनवद्व्यालानयोध्यामिव पर्वतम्}
{मन्यस्व वनिते नित्यं सरयूवदिमां नदीम्} %2-95-15

\twolineshloka
{लक्ष्मणश्चापि धर्मात्मा मन्निदेशे व्यवस्थितः}
{त्वं चानुकूला वैदेहि प्रीतिं जनयथो मम} %2-95-16

\twolineshloka
{उपस्पृशंस्त्रिषवणं मधुमूलफलाशनः}
{नायोध्यायै न राज्याय स्पृहयेऽद्य त्वया सह} %2-95-17

\twolineshloka
{इमां हि रम्यां मृगयूथशालिनीं निपीततोयां गजसिंहवानरैः}
{सुपुष्पितैः पुष्पधरैरलङ्कृतां न सोऽस्ति यः स्यादगतक्लमः सुखी} %2-95-18

\twolineshloka
{इतीव रामो बहुसङ्गतं वचः प्रियासहायः सरितं प्रति ब्रुवन्}
{चचार रम्यं नयनाञ्जनप्रभं स चित्रकूटं रघुवंशवर्द्धनः} %2-95-19


॥इत्यार्षे श्रीमद्रामायणे वाल्मीकीये आदिकाव्ये अयोध्याकाण्डे मन्दाकिनीवर्णनम् नाम पञ्चनवतितमः सर्गः ॥२-९५॥
