\sect{नवतितमः सर्गः — भरद्वाजाश्रमनिवासः}

\twolineshloka
{भरद्वाजाश्रमं दृष्ट्वा क्रोशादेव नरर्षभः}
{बलं सर्वमवस्थाप्य जगाम सह मन्त्रिभिः} %2-90-1

\twolineshloka
{पद्भ्यामेव हि धर्मज्ञो न्यस्तशस्त्रपरिच्छदः}
{वसानो वाससी क्षौमे पुरोधाय पुरोधसम्} %2-90-2

\twolineshloka
{ततः सन्दर्शने तस्य भरद्वाजस्य राघवः}
{मन्त्रिणस्तानवस्थाप्य जगामानुपुरोहितम्} %2-90-3

\twolineshloka
{वसिष्ठमथ दृष्ट्वैव भरद्वाजो महातपाः}
{सञ्चचालासनात्तूर्णं शिष्यानर्घ्यमिति ब्रुवन्} %2-90-4

\onelineshloka
{वसिष्ठसाहचर्य्यादिति भावः} %2-90-5

\twolineshloka
{ताभ्यामर्घ्यं च पाद्यं च दत्त्वा पश्चात् फलानि च}
{आनुपूर्व्याच्च धर्मज्ञः पप्रच्छ कुशलं कुले} %2-90-6

\twolineshloka
{अयोध्यायां बले कोशे मित्रेष्वपि च मन्त्रिषु}
{जानन् दशरथं वृत्तं न राजानमुदाहरत्} %2-90-7

\twolineshloka
{वसिष्ठो भरतश्चैनं पप्रच्छतुरनामयम्}
{शरीरेऽग्निषु वृक्षेषु शिष्येषु मृगपक्षिषु} %2-90-8

\twolineshloka
{तथेति तत्प्रतिज्ञाय भरद्वाजो महातपाः}
{भरतं प्रत्युवाचेदं राघवस्नेहबन्धनात्} %2-90-9

\twolineshloka
{किमिहागमने कार्य्यं तव राज्यं प्रशासतः}
{एतदाचक्ष्व मे सर्वं नहि मे शुद्ध्यते मनः} %2-90-10

\twolineshloka
{सुषुवे यममित्रघ्नं कौसल्या नन्दवर्द्धनम्}
{भ्रात्रा सह सभार्यो यश्चिरं प्रव्राजितो वनम्} %2-90-11

\twolineshloka
{नियुक्तः स्त्रीनियुक्तेन पित्रा योऽसौ महायशाः}
{वनवासी भवेतीह समाः किल चतुर्दश} %2-90-12

\twolineshloka
{कच्चिन्न तस्यापापस्य पापं कर्तुमिहेच्छसि}
{अकण्टकं भोक्तुमना राज्यं तस्यानुजस्य च} %2-90-13

\twolineshloka
{एवमुक्तो भरद्वाजं भरतः प्रत्युवाच ह}
{पर्य्यश्रुनयनो दुःखाद्वाचा संसज्जमानया} %2-90-14

\twolineshloka
{हतोऽस्मि यदि मामेवं भगवानपि मन्यते}
{मत्तो न दोषमाशङ्के नैवं मामनुशाधि हि} %2-90-15

\twolineshloka
{न चैतदिष्टं माता मे यदवोचन्मदन्तरे}
{नाहमेतेन तुष्टश्च न तद्वचनमाददे} %2-90-16

\twolineshloka
{अहं तु तं नरव्याघ्रमुपयातः प्रसादकः}
{प्रतिनेतुमयोध्यां च पादौ तस्याभिवन्दितुम्} %2-90-17

\twolineshloka
{त्वं मामेवङ्गतं मत्वा प्रसादं कर्तुमर्हसि}
{शंस मे भगवन् रामः क्व सम्प्रति महीपतिः} %2-90-18

\twolineshloka
{वसिष्ठादिभिर्ऋत्विग्भिर्याचितो भगवांस्ततः}
{उवाच तं भरद्वाजः प्रसादाद्भरतं वचः} %2-90-19

\twolineshloka
{त्वय्येतत्पुरुषव्याघ्र युक्तं राघववंशजे}
{गुरुवृत्तिर्दमश्चैव साधूनां चानुयायिता} %2-90-20

\twolineshloka
{जाने चैतन्मनःस्थं ते दृढीकरणमस्त्विति}
{अपृच्छं त्वां तथात्यर्थं कीर्त्तिं समभिवर्द्धयन्} %2-90-21

\twolineshloka
{जाने च रामं धर्म्मज्ञं ससीतं सहलक्ष्मणम्}
{असौ वसति ते भ्राता चित्रकूटे महागिरौ} %2-90-22

\twolineshloka
{श्वस्तु गन्तासि तं देशं वसाद्य सह मन्त्रिभिः}
{एतं मे कुरु सुप्राज्ञ कामं कामार्थकोविद} %2-90-23

\twolineshloka
{ततस्तथेत्येवमुदारदर्शनः प्रतीतरूपो भरतोऽब्रवीद्वचः}
{चकार बुद्धिं च तदा तदाश्रमे निशानिवासाय नराधिपात्मजः} %2-90-24


॥इत्यार्षे श्रीमद्रामायणे वाल्मीकीये आदिकाव्ये अयोध्याकाण्डे भरद्वाजाश्रमनिवासः नाम नवतितमः सर्गः ॥२-९०॥
