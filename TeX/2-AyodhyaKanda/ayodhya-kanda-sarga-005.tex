\sect{पञ्चमः सर्गः — व्रतचर्याविधानम्}

\twolineshloka
{सन्दिश्य रामं नृपतिः श्वोभाविन्यभिषेचने}
{पुरोहितं समाहूय वसिष्ठमिदमब्रवीत्} %2-5-1

\twolineshloka
{गच्छोपवासं काकुत्स्थं कारयाद्य तपोधन}
{श्रेयसे राज्यलाभाय वध्वा सह यतव्रत} %2-5-2

\twolineshloka
{तथेति च स राजानमुक्त्वा वेदविदां वरः}
{स्वयं वसिष्ठो भगवान् ययौ रामनिवेशनम्} %2-5-3

\twolineshloka
{उपवासयितुं वीरं मन्त्रविन्मन्त्रकोविदम्}
{ब्राह्मं रथवरं युक्तमास्थाय सुधृतव्रतः} %2-5-4

\twolineshloka
{स रामभवनं प्राप्य पाण्डुराभ्रघनप्रभम्}
{तिस्रः कक्ष्या रथेनैव विवेश मुनिसत्तमः} %2-5-5

\twolineshloka
{तमागतमृषिं रामस्त्वरन्निव ससम्भ्रमम्}
{मानयिष्यन् स मानार्हं निश्चक्राम निवेशनात्} %2-5-6

\twolineshloka
{अभ्येत्य त्वरमाणोऽथ रथाभ्याशं मनीषिणः}
{ततोऽवतारयामास परिगृह्य रथात् स्वयम्} %2-5-7

\twolineshloka
{स चैनं प्रश्रितं दृष्ट्वा सम्भाष्याभिप्रसाद्य च}
{प्रियार्हं हर्षयन् राममित्युवाच पुरोहितः} %2-5-8

\twolineshloka
{प्रसन्नस्ते पिता राम यत्त्वं राज्यमवाप्स्यसि}
{उपवासं भवानद्य करोतु सह सीतया} %2-5-9

\twolineshloka
{प्रातस्त्वामभिषेक्ता हि यौवराज्ये नराधिपः}
{पिता दशरथः प्रीत्या ययातिं नहुषो यथा} %2-5-10

\twolineshloka
{इत्युक्त्वा स तदा राममुपवासं यतव्रतः}
{मन्त्रवत् कारयामास वैदेह्या सहितं शुचिः} %2-5-11

\twolineshloka
{ततो यथावद् रामेण स राज्ञो गुरुरर्चितः}
{अभ्यनुज्ञाप्य काकुत्स्थं ययौ रामनिवेशनात्} %2-5-12

\twolineshloka
{सुहृद्भिस्तत्र रामोऽपि सहासीनः प्रियंवदैः}
{सभाजितो विवेशाथ ताननुज्ञाप्य सर्वशः} %2-5-13

\twolineshloka
{हृष्टनारीनरयुतं रामवेश्म तदा बभौ}
{यथा मत्तद्विजगणं प्रफुल्लनलिनं सरः} %2-5-14

\twolineshloka
{स राजभवनप्रख्यात् तस्माद् रामनिवेशनात्}
{निर्गत्य ददृशे मार्गं वसिष्ठो जनसंवृतम्} %2-5-15

\twolineshloka
{वृन्दवृन्दैरयोध्यायां राजमार्गाः समन्ततः}
{बभूवुरभिसम्बाधाः कुतूहलजनैर्वृताः} %2-5-16

\twolineshloka
{जनवृन्दोर्मिसङ्घर्षहर्षस्वनवृतस्तदा}
{बभूव राजमार्गस्य सागरस्येव निःस्वनः} %2-5-17

\twolineshloka
{सिक्तसम्मृष्टरथ्या हि तथा च वनमालिनी}
{आसीदयोध्या तदहः समुच्छ्रितगृहध्वजा} %2-5-18

\twolineshloka
{तदा ह्ययोध्यानिलयः सस्त्रीबालाकुलो जनः}
{रामाभिषेकमाकाङ्क्षन्नाकाङ्क्षन्नुदयं रवेः} %2-5-19

\twolineshloka
{प्रजालङ्कारभूतं च जनस्यानन्दवर्धनम्}
{उत्सुकोऽभूज्जनो द्रष्टुं तमयोध्यामहोत्सवम्} %2-5-20

\twolineshloka
{एवं तज्जनसम्बाधं राजमार्गं पुरोहितः}
{व्यूहन्निव जनौघं तं शनै राजकुलं ययौ} %2-5-21

\twolineshloka
{सिताभ्रशिखरप्रख्यं प्रासादमधिरुह्य च}
{समीयाय नरेन्द्रेण शक्रेणेव बृहस्पतिः} %2-5-22

\twolineshloka
{तमागतमभिप्रेक्ष्य हित्वा राजासनं नृपः}
{पप्रच्छ स्वमतं तस्मै कृतमित्यभिवेदयत्} %2-5-23

\twolineshloka
{तेन चैव तदा तुल्यं सहासीनाः सभासदः}
{आसनेभ्यः समुत्तस्थुः पूजयन्तः पुरोहितम्} %2-5-24

\twolineshloka
{गुरुणा त्वभ्यनुज्ञातो मनुजौघं विसृज्य तम्}
{विवेशान्तःपुरं राजा सिंहो गिरिगुहामिव} %2-5-25

\twolineshloka
{तदग्र्यवेषप्रमदाजनाकुलं महेन्द्रवेश्मप्रतिमं निवेशनम्}
{व्यदीपयंश्चारु विवेश पार्थिवः शशीव तारागणसङ्कुलं नभः} %2-5-26


॥इत्यार्षे श्रीमद्रामायणे वाल्मीकीये आदिकाव्ये अयोध्याकाण्डे व्रतचर्याविधानम् नाम पञ्चमः सर्गः ॥२-५॥
