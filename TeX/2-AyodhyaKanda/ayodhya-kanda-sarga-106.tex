\sect{षडधिकशततमः सर्गः — भरतवचनम्}

\threelineshloka
{एवमुक्त्वा तु विरते रामे वचनमर्थवत्}
{ततो मन्दाकिनीतीरे रामं प्रकृतिवत्सलम्}
{उवाच भरतश्चित्रं धार्मिको धार्मिकं वचः} %2-106-1

\twolineshloka
{को हि स्यादीदृशो लोके यादृशस्त्वमरिन्दम}
{न त्वां प्रव्यथयेद्दुःखं प्रीतिर्वा न प्रहर्षयेत्} %2-106-2

\onelineshloka
{सम्मतश्चासि वृद्धानां तांश्च पृच्छसि संशयान्} %2-106-3

\twolineshloka
{यथा मृतस्तथा जीवन् यथाऽसति तथा सति}
{यस्यैष बुद्धिलाभः स्यात्परितप्येत केन सः} %2-106-4

\twolineshloka
{परावरज्ञो यश्च स्यात्तथा त्वं मनुजाधिप}
{स एवं व्यसनं प्राप्य न विषीदितुमर्हति} %2-106-5

\twolineshloka
{अमरोपम सत्त्वस्त्वं महात्मा सत्यसङ्गरः}
{सर्वज्ञः सर्वदर्शी च बुद्धिमांश्चासि राघव} %2-106-6

\twolineshloka
{न त्वामेवङ्गुणैर्युक्तं प्रभवाभवकोविदम्}
{अविषह्यतमं दुःखमासादयितुमर्हति} %2-106-7

\twolineshloka
{प्रोषिते मयि यत्पापं मात्रा मत्कारणात्कृतम्}
{क्षुद्रया तदनिष्टं मे प्रसीदतु भवान्मम} %2-106-8

\twolineshloka
{धर्मबन्धेन बद्धोऽस्मि तेनेमां नेह मातरम्}
{हन्मि तीव्रेण दण्डेन दण्डार्हां पापकारिणीम्} %2-106-9

\twolineshloka
{कथं दशरथाज्जातः शुद्धाभिजनकर्मणः}
{जानन् धर्ममधर्मिष्ठं कुर्य्यां कर्म जुगुप्सितम्} %2-106-10

\twolineshloka
{गुरुः क्रियावान् वृद्धश्च राजा प्रेतः पितेति च}
{तातं न परिगर्हेयं दैवतं चेति संसदि} %2-106-11

\twolineshloka
{को हि धर्मार्थयोर्हीनमीदृशं कर्म किल्बिषम्}
{स्त्रियाः प्रियं चिकीर्षुः सन् कुर्याद्धर्मज्ञ धर्मवित्} %2-106-12

\twolineshloka
{अन्तकाले हि भूतानि मुह्यन्तीति पुराश्रुतिः}
{राज्ञैवं कुर्वता लोके प्रत्यक्षं सा श्रुतिः कृता} %2-106-13

\twolineshloka
{साध्वर्थमभिसन्धाय क्रोधान्मोहाच्च साहसात्}
{तातस्य यदतिक्रान्तं प्रत्याहरतु तद्भवान्} %2-106-14

\onelineshloka
{पितुरपतनहेतुत्वात्तदेवापत्यत्वेन सम्मतम्} %2-106-15

\twolineshloka
{तदपत्यं भवानस्तु मा भवान् दुष्कृतं पितुः}
{अभिपत्ता कृतं कर्म लोके धीरविगर्हितम्} %2-106-16

\twolineshloka
{कैकेयीं मां च तातं च सुहृदो बान्धवांश्च नः}
{पौरजानपदान् सर्वांस्त्रातु सर्वमिदं भवान्} %2-106-17

\twolineshloka
{क्व चारण्यं क्व च क्षात्रं क्व जटाः क्व च पालनम्}
{ईदृशं व्याहतं कर्म न भवान् कर्तुमर्हति} %2-106-18

\twolineshloka
{एष हि प्रथमो धर्मः क्षत्रियस्याभिषेचनम्}
{येन शक्यं महाप्राज्ञ प्रजानां परिपालनम्} %2-106-19

\twolineshloka
{कश्च प्रत्यक्षमुत्सृज्य संशयस्थमलक्षणम्}
{आयतिस्थं चरेद्धर्मं क्षत्रबन्धुरनिश्चितम्} %2-106-20

\twolineshloka
{अथ क्लेशजमेव त्वं धर्मं चरितुमिच्छसि}
{धर्मेण चतुरो वर्णान् पालयन् क्लेशमाप्नुहि} %2-106-21

\twolineshloka
{चतुर्णामाश्रमाणां हि गार्हस्थ्यं श्रेष्ठमाश्रमम्}
{प्राहुर्धर्मज्ञ धर्मज्ञास्तं कथं त्यक्तुमर्हसि} %2-106-22

\twolineshloka
{श्रुतेन बालः स्थानेन जन्मना भवतो ह्यहम्}
{स कथं पालयिष्यामि भूमिं भवति तिष्ठति} %2-106-23

\twolineshloka
{हीनबुद्धिगुणो बालो हीनः स्थानेन चाप्यहम्}
{भवता च विनाभूतो न वर्त्तयितुमुत्सहे} %2-106-24

\twolineshloka
{इदं निखिलमव्यग्रं राज्यं पित्र्यमकण्टकम्}
{अनुशाधि स्वधर्मेण धर्मज्ञ सह बान्धवैः} %2-106-25

\twolineshloka
{इहैव त्वाऽभिषिञ्चन्तु सर्वाः प्रकृतयः सह}
{ऋत्विजः सवसिष्ठाश्च मन्त्रवन्मन्त्रकोविदाः} %2-106-26

\twolineshloka
{अभिषिक्तस्त्वमस्माभिरयोध्यां पालने व्रज}
{विजित्य तरसा लोकान् मरुद्भिरिव वासवः} %2-106-27

\twolineshloka
{ऋणानि त्रीण्यपाकुर्वन् दुर्हृदः साधु निर्द्दहन्}
{सुहृदस्तर्पयन् कामैस्त्वमेवात्रानुशाधि माम्} %2-106-28

\twolineshloka
{अद्यार्य मुदिताः सन्तु सुहृदस्तेऽभिषेचने}
{अद्य भीताः पलायन्तां दुर्हृदस्ते दिशो दश} %2-106-29

\twolineshloka
{आक्रोशं मम मातुश्च प्रमृज्य पुरुषर्षभ}
{अद्य तत्रभवन्तं च पितरं रक्ष किल्बिषात्} %2-106-30

\twolineshloka
{शिरसा त्वाऽभियाचेऽहं कुरुष्व करुणां मयि}
{बान्धवेषु च सर्वेषु भूतेष्विव महेश्वरः} %2-106-31

\twolineshloka
{अथैतत् पृष्ठतः कृत्वा वनमेव भवानितः}
{गमिष्यति गमिष्यामि भवता सार्द्धमप्यहम्} %2-106-32

\twolineshloka
{तथाहि रामो भरतेन ताम्यता प्रसाद्यमानः शिरसा महीपतिः}
{न चैव चक्रे गमनाय सत्त्ववान् मतिं पितुस्तद्वचने व्यवस्थितः} %2-106-33

\twolineshloka
{तदद्भुतं स्थैर्यमवेक्ष्य राघवे समं जनो हर्षमवाप दुःखितः}
{न यात्ययोध्यामिति दुःखितोऽभवत् स्थिरप्रतिज्ञत्वमवेक्ष्य हर्षितः} %2-106-34

\twolineshloka
{तमृत्विजो नैगमयूथवल्लभास्तदा विंसज्ञाश्रुकलाश्च मातरः}
{तथा ब्रुवाणं भरतं प्रतुष्टुवुः प्रणम्य रामं च ययाचिरे सह} %2-106-35


॥इत्यार्षे श्रीमद्रामायणे वाल्मीकीये आदिकाव्ये अयोध्याकाण्डे भरतवचनम् नाम षडधिकशततमः सर्गः ॥२-१०६॥
