\sect{अष्टचत्वारिंशः सर्गः — पौराङ्गनाविलापः}

\twolineshloka
{तेषामेवम् विष्ण्णानाम्पीडितानामतीव च}
{बाष्पविप्लुतनेत्राणाम् सशोकानाम् मुमूर्षया} %2-48-1

\twolineshloka
{अनुगम्य निवृत्तानाम् रामम् नगर वासिनाम्}
{उद्गतानि इव सत्त्वानि बभूवुर् अमनस्विनाम्} %2-48-2

\twolineshloka
{स्वम् स्वम् निलयम् आगम्य पुत्र दारैः समावृताः}
{अश्रूणि मुमुचुः सर्वे बाष्पेण पिहित आननाः} %2-48-3

\twolineshloka
{न च आहृष्यन् न च अमोदन् वणिजो न प्रसारयन्}
{न च अशोभन्त पण्यानि न अपचन् गृह मेधिनः} %2-48-4

\twolineshloka
{नष्टम् दृष्ट्वा न अभ्यनन्दन् विपुलम् वा धन आगमम्}
{पुत्रम् प्रथमजम् लब्ध्वा जननी न अभ्यनन्दत} %2-48-5

\twolineshloka
{गृहे गृहे रुदन्त्यः च भर्तारम् गृहम् आगतम्}
{व्यगर्हयन्तः दुह्ख आर्ता वाग्भिस् तोत्रैः इव द्विपान्} %2-48-6

\twolineshloka
{किम् नु तेषाम् गृहैः कार्यम् किम् दारैः किम् धनेन वा}
{पुत्रैः वा किम् सुखैः वा अपि ये न पश्यन्ति राघवम्} %2-48-7

\twolineshloka
{एकः सत् पुरुषो लोके लक्ष्मणः सह सीतया}
{यो अनुगच्चति काकुत्स्थम् रामम् परिचरन् वने} %2-48-8

\twolineshloka
{आपगाः कृत पुण्याः ताः पद्मिन्यः च सराम्सि च}
{येषु स्नास्यति काकुत्स्थो विगाह्य सलिलम् शुचि} %2-48-9

\twolineshloka
{शोभयिष्यन्ति काकुत्स्थम् अटव्यो रम्य काननाः}
{आपगाः च महा अनूपाः सानुमन्तः च पर्वताः} %2-48-10

\twolineshloka
{काननम् वा अपि शैलम् वा यम् रामः अभिगमिष्यति}
{प्रिय अतिथिम् इव प्राप्तम् न एनम् शक्ष्यन्ति अनर्चितुम्} %2-48-11

\twolineshloka
{विचित्र कुसुम आपीडा बहु मन्जलि धारिणः}
{अकाले च अपि मुख्यानि पुष्पाणि च फलानि च} %2-48-12

\twolineshloka
{अकाले चापि मुख्यानि पुष्पाणि च फलानि च}
{दर्शयिष्यन्ति अनुक्रोशात् गिरयो रामम् आगतम्} %2-48-13

\twolineshloka
{प्रस्रविष्यन्ति तोयानि विमलानि महीधराः}
{विदर्शयन्तः विविधान् भूयः चित्रामः च निर्झरान्} %2-48-14

\twolineshloka
{पादपाः पर्वत अग्रेषु रमयिष्यन्ति राघवम्}
{यत्र रामः भयम् न अत्र न अस्ति तत्र पराभवः} %2-48-15

\twolineshloka
{स हि शूरः महा बाहुः पुत्रः दशरथस्य च}
{पुरा भवति नो दूरात् अनुगच्चाम राघवम्} %2-48-16

\twolineshloka
{पादच् चाया सुखा भर्तुस् तादृशस्य महात्मनः}
{स हि नाथो जनस्य अस्य स गतिः स परायणम्} %2-48-17

\twolineshloka
{वयम् परिचरिष्यामः सीताम् यूयम् तु राघवम्}
{इति पौर स्त्रियो भर्तृऋन् दुह्ख आर्ताः तत् तत् अब्रुवन्} %2-48-18

\twolineshloka
{युष्माकम् राघवो अरण्ये योग क्षेमम् विधास्यति}
{सीता नारी जनस्य अस्य योग क्षेमम् करिष्यति} %2-48-19

\twolineshloka
{को न्व् अनेन अप्रतीतेन स उत्कण्ठित जनेन च}
{सम्प्रीयेत अमनोज्ञेन वासेन हृत चेतसा} %2-48-20

\twolineshloka
{कैकेय्या यदि चेद् राज्यम् स्यात् अधर्म्यम् अनाथवत्}
{न हि नो जीवितेन अर्थः कुतः पुत्रैः कुतः धनैः} %2-48-21

\twolineshloka
{यया पुत्रः च भर्ता च त्यक्ताव् ऐश्वर्य कारणात्}
{कम् सा परिहरेद् अन्यम् कैकेयी कुल पाम्सनी} %2-48-22

\twolineshloka
{कैकेय्या न वयम् राज्ये भृतका निवसेमहि}
{जीवन्त्या जातु जीवन्त्यः पुत्रैः अपि शपामहे} %2-48-23

\twolineshloka
{या पुत्रम् पार्थिव इन्द्रस्य प्रवासयति निर्घृणा}
{कः ताम् प्राप्य सुखम् जीवेद् अधर्म्याम् दुष्ट चारिणीम्} %2-48-24

\twolineshloka
{उपद्रुतमिदम् सर्वमनालम्बमनायकम्}
{कैकेय्या हि कृते सर्वम् विनाशमुपयास्यति} %2-48-25

\twolineshloka
{न हि प्रव्रजिते रामे जीविष्यति मही पतिः}
{मृते दशरथे व्यक्तम् विलोपः तत् अनन्तरम्} %2-48-26

\twolineshloka
{ते विषम् पिबत आलोड्य क्षीण पुण्याः सुदुर्गताः}
{राघवम् वा अनुगच्चध्वम् अश्रुतिम् वा अपि गच्चत} %2-48-27

\twolineshloka
{मिथ्या प्रव्राजितः रामः सभार्यः सह लक्ष्मणः}
{भरते सम्निषृष्टाः स्मः सौनिके पशवो यथा} %2-48-28

\twolineshloka
{पूर्णचन्द्राननः श्यामो गूढजत्रुररिंदमः}
{आजानुबाहुः पद्माक्षो रामो लक्ष्मनपूर्वजः} %2-48-29

\twolineshloka
{पूर्वाभिभाषी मधुरः सत्यवादी महाबलः}
{सौम्यः सर्वस्य लोकस्य चन्द्रवत्प्रियदर्शनः} %2-48-30

\twolineshloka
{नूनम् पुरुषशार्दूलो मत्तमातङ्गविक्रमः}
{शोभयुश्यत्यरण्यानि विचरन् स महारथः} %2-48-31

\twolineshloka
{तास्तथा विलपन्त्यस्तु नगरे नागरस्त्रियः}
{चुक्रुशुर्दुःखसम्तप्तामृत्योरिव भयागमे} %2-48-32

\twolineshloka
{इत्येव विलपन्तीनाम् स्त्रीणाम् वेश्मसु राघवम्}
{जगामास्तम् दिनकरो रजनी चाभ्यवर्तत} %2-48-33

\twolineshloka
{नष्टज्वलनसम्पाता प्रशान्ताध्यायसत्कथा}
{तिमिरेणाभिलिप्तेव तदा सा नगरी बभौ} %2-48-34

\twolineshloka
{उपशान्तवणिक्पण्या नष्टहर्षा निराश्रया}
{अयोध्या नगरी चासीन्नष्टतारमिवाम्बरम्} %2-48-35

\fourlineindentedshloka
{तथा स्त्रियो राम निमित्तम् आतुरा}
{यथा सुते भ्रातरि वा विवासिते}
{विलप्य दीना रुरुदुर् विचेतसः}
{सुतैः हि तासाम् अधिको हि सो अभवत्} %2-48-36

\fourlineindentedshloka
{प्रशान्तगीतोत्सव नृत्तवादना}
{व्यपास्तहर्षा पिहितापणोदया}
{तदा ह्ययोध्या नगरी बभूव सा}
{महार्णवः सम्क्षपितोदको यथा} %2-48-37


॥इत्यार्षे श्रीमद्रामायणे वाल्मीकीये आदिकाव्ये अयोध्याकाण्डे पौराङ्गनाविलापः नाम अष्टचत्वारिंशः सर्गः ॥२-४८॥
