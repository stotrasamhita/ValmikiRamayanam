\sect{द्वादशाधिकशततमः सर्गः — पादुकाप्रदानम्}

\twolineshloka
{तमप्रतिमतेजोभ्यां भ्रातृभ्यां रोमहर्षणम्}
{विस्मिताः सङ्गमं प्रेक्ष्य समवेता महर्षयः} %2-112-1

\twolineshloka
{अन्तर्हितास्त्वृषिगणाः सिद्धाश्च परमर्षयः}
{तौ भ्रातरौ महात्मानौ महात्मानौ काकुत्स्थौ प्रशशंसिरे} %2-112-2

\twolineshloka
{स धन्यो यस्य पुत्रौ द्वौ धर्मज्ञौ धर्मविक्रमौ}
{श्रुत्वा वयं हि सम्भाषामुभयोः स्पृहयामहे} %2-112-3

\twolineshloka
{ततस्त्वृषिगणाः क्षिप्रं दशग्रीववधैषिणः}
{भरतं राजशार्दूलमित्यूचुः सङ्गता वचः} %2-112-4

\twolineshloka
{कुले जात महाप्राज्ञ महावृत्त महायशः}
{ग्राह्यं रामस्य वाक्यं ते पितरं यद्यवेक्षसे} %2-112-5

\twolineshloka
{सदानृणमिमं रामं वयमिच्छामहे पितुः}
{अनृणत्वाच्च कैकेय्याः स्वर्गं दशरथो गतः} %2-112-6

\twolineshloka
{एतावदुक्त्वा वचनं गन्धर्वाः समहर्षयः}
{राजर्षयश्चैव तदा सर्वे स्वांस्वां गतिं गताः} %2-112-7

\twolineshloka
{ह्लादितस्तेन वाक्येन शुभेन शुभदर्शनः}
{रामः संहृष्टवदनस्तानृषीनभ्यपूजयत्} %2-112-8

\twolineshloka
{स्रस्तगात्रस्तु भरतः स वाचा सज्जमानया}
{कृताञ्जलिरिदं वाक्यं राघवं पुनरब्रवीत्} %2-112-9

\twolineshloka
{राजधर्ममनुप्रेक्ष्य कुलधर्मानुसन्ततिम्}
{कर्त्तुमर्हसि काकुत्स्थ मम मातुश्च याचनाम्} %2-112-10

\twolineshloka
{रक्षितुं सुमहद्राज्यमहमेकस्तु नोत्सहे}
{पौरजानपदांश्चापि रक्तान् रञ्जयितुं तथा} %2-112-11

\twolineshloka
{ज्ञातयश्च हि योधाश्च मित्राणि सुहृदश्च नः}
{त्वामेव प्रतिकांक्षन्ते पर्जन्यमिव कर्षकाः} %2-112-12

\twolineshloka
{इदं राज्यं महाप्राज्ञ स्थापय प्रतिपद्य हि}
{शक्तिमानसि काकुत्स्थ लोकस्य परिपालने} %2-112-13

\twolineshloka
{इत्युक्त्वा न्यपतद् भ्रातुः पादयोर्भरतस्तदा}
{भृशं सम्प्रार्थयामास राममेव प्रियंवदः} %2-112-14

\twolineshloka
{तमङ्के भ्रातरं कृत्वा रामो वचनमब्रवीत्}
{श्यामं नलिनपत्राक्षं मत्तहंसस्वरं स्वयम्} %2-112-15

\twolineshloka
{आगता त्वामियं बुद्धिः स्वजा वैनयिकी च या}
{भृशमुत्सहसे तात रक्षितुं पृथिवीमपि} %2-112-16

\twolineshloka
{अमात्यैश्च सुहृद्भिश्च बुद्धिमद्भिश्च मन्त्रिभिः}
{सर्वकार्याणि सम्मन्त्र्य सुमहान्त्यपि कारय} %2-112-17

\twolineshloka
{लक्ष्मीश्चन्द्रादपेयाद्वा हिमवान् वा हिमं त्यजेत्}
{अतीयात् सागरो वेलां न प्रतिज्ञामहं पितुः} %2-112-18

\twolineshloka
{कामाद्वा तात लोभाद्वा मात्रा तुभ्यमिदं कृतम्}
{न तन्मनसि कर्त्तव्यं वर्त्तितव्यं च मातृवत्} %2-112-19

\twolineshloka
{एवं ब्रुवाणं भरतः कौसल्यासुतमब्रवीत्}
{तेजसादित्यसङ्काशं प्रतिपच्चन्द्रदर्शनम्} %2-112-20

\twolineshloka
{अधिरोहार्य पादाभ्यां पादुके हेमभूषिते}
{एते हि सर्वलोकस्य योगक्षेमं विधास्यतः} %2-112-21

\twolineshloka
{सोऽधिरुह्य नरव्याघ्रः पादुके ह्यवरुह्य च}
{प्रायच्छत् सुमहातेजा भरताय महात्मने} %2-112-22

\twolineshloka
{स पादुके सम्प्रणम्य रामं वचनमब्रवीत्}
{चतुर्दश हि वर्षाणि जटाचीरधरो ह्यहम्} %2-112-23

\twolineshloka
{फलमूलाशनो वीर भवेयं रघुनन्दन}
{तवागमनमाकांक्षन् वसन् वै नगराद्बहिः} %2-112-24

\threelineshloka
{तव पादुकयोर्न्यस्तराज्यतन्त्रः परन्तप}
{चतुर्दशे हि सम्पूर्णे वर्षेऽहनि रघूत्तम}
{न द्रक्ष्यामि यदि त्वां तु प्रवेक्ष्यामि हुताशनम्} %2-112-25

\twolineshloka
{तथेति च प्रतिज्ञाय तं परिष्वज्य सादरम्}
{शत्रुघ्नं च परिष्वज्य भरतं चेदमब्रवीत्} %2-112-26

\twolineshloka
{मातरं रक्ष कैकेयीं मा रोषं कुरु तां प्रति}
{मया च सीतया चैव शप्तोऽसि रघुसत्तम} %2-112-27

\onelineshloka
{इत्युक्त्वाऽश्रुपरीताक्षो भ्रातरं विससर्ज ऺह} %2-112-28

\twolineshloka
{स पादुके ते भरतः प्रतापवान् स्वलङ्कृते सम्परिपूज्य धर्मवित्}
{प्रदक्षिणं चैव चकार राघवं चकार चैवोत्तमनागमूर्द्धनि} %2-112-29

\twolineshloka
{अथानुपूर्व्यात् प्रतिनन्द्य तं जनं गुरूंश्च मन्त्रिप्रकृतीस्तथानुजौ}
{व्यसर्जयद्राघववंशवर्द्धनः स्थिरः स्वधर्मे हिमवानिवाचलः} %2-112-30

\twolineshloka
{तं मातरो बाष्पगृहीतकण्ठ्यो दुःखेन नामन्त्रयितुं हि शेकुः}
{स त्वेव मातऽरभिवाद्य सर्वा रुदन् कुटीं स्वां प्रविवेश रामः} %2-112-31


॥इत्यार्षे श्रीमद्रामायणे वाल्मीकीये आदिकाव्ये अयोध्याकाण्डे पादुकाप्रदानम् नाम द्वादशाधिकशततमः सर्गः ॥२-११२॥
