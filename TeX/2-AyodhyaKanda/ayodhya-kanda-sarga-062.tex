\sect{द्विषष्ठितमः सर्गः — कौसल्याप्रसादनम्}

\twolineshloka
{एवम् तु क्रुद्धया राजा राम मात्रा सशोकया}
{श्रावितः परुषम् वाक्यम् चिन्तयाम् आस दुह्खितः} %2-62-1

\twolineshloka
{चिन्तयित्वा स च नृपो मुमोह व्याकुलेन्द्रियः}
{अथ दीर्घेण कालेन सम्ज्ञामाप परतपः} %2-62-2

\twolineshloka
{स सम्ज्ञाअमुपलब्यैव दीर्घमुष्णम् च निःससन्}
{कौसल्याम् पार्श्वतो दृष्ट्वा ततश्चिन्तामुपागमत्} %2-62-3

\twolineshloka
{तस्य चिन्तयमानस्य प्रत्यभात् कर्म दुष्कृतम्}
{यद् अनेन कृतम् पूर्वम् अज्ञानात् शब्द वेधिना} %2-62-4

\twolineshloka
{अमनाः तेन शोकेन राम शोकेन च प्रभुः}
{द्वाभ्यामपि महाराजः शोकाब्यामभितप्यतो} %2-62-5

\twolineshloka
{दह्यमानः तु शोकाभ्याम् कौसल्याम् आह भू पतिः}
{वेपमानोऽञ्जलिम् कृत्वा प्रसादर्तमवाङ्मुखः} %2-62-6

\twolineshloka
{प्रसादये त्वाम् कौसल्ये रचितः अयम् मया अन्जलिः}
{वत्सला च आनृशम्सा च त्वम् हि नित्यम् परेष्व् अपि} %2-62-7

\twolineshloka
{भर्ता तु खलु नारीणाम् गुणवान् निर्गुणो अपि वा}
{धर्मम् विमृशमानानाम् प्रत्यक्षम् देवि दैवतम्} %2-62-8

\twolineshloka
{सा त्वम् धर्म परा नित्यम् दृष्ट लोक पर अवर}
{न अर्हसे विप्रियम् वक्तुम् दुह्खिता अपि सुदुह्खितम्} %2-62-9

\twolineshloka
{तत् वाक्यम् करुणम् राज्ञः श्रुत्वा दीनस्य भाषितम्}
{कौसल्या व्यसृजद् बाष्पम् प्रणाली इव नव उदकम्} %2-62-10

\twolineshloka
{स मूद्र्ह्नि बद्ध्वा रुदती राज्ञः पद्मम् इव अन्जलिम्}
{सम्भ्रमात् अब्रवीत् त्रस्ता त्वरमाण अक्षरम् वचः} %2-62-11

\twolineshloka
{प्रसीद शिरसा याचे भूमौ निततिता अस्मि ते}
{याचिता अस्मि हता देव हन्तव्या अहम् न हि त्वया} %2-62-12

\twolineshloka
{न एषा हि सा स्त्री भवति श्लाघनीयेन धीमता}
{उभयोः लोकयोः वीर पत्या या सम्प्रसाद्यते} %2-62-13

\twolineshloka
{जानामि धर्मम् धर्मज्ञ त्वाम् जाने सत्यवादिनम्}
{पुत्र शोक आर्तया तत् तु मया किम् अपि भाषितम्} %2-62-14

\twolineshloka
{शोको नाशयते धैर्यम् शोको नाशयते श्रुतम्}
{शोको नाशयते सर्वम् न अस्ति शोक समः रिपुः} %2-62-15

\twolineshloka
{शक्यम् आपतितः सोढुम् प्रहरः रिपु हस्ततः}
{सोढुम् आपतितः शोकः सुसूक्ष्मः अपि न शक्यते} %2-62-16

\twolineshloka
{दर्मज्ञाः श्रुतिमन्तोऽपि चिन्नधर्मार्थसम्शयाः}
{यतयो वीर मुह्यन्ति शोकसम्मूढचेतसः} %2-62-17

\twolineshloka
{वन वासाय रामस्य पन्च रात्रः अद्य गण्यते}
{यः शोक हत हर्षायाः पन्च वर्ष उपमः मम} %2-62-18

\twolineshloka
{तम् हि चिन्तयमानायाः शोको अयम् हृदि वर्धते}
{अदीनाम् इव वेगेन समुद्र सलिलम् महत्} %2-62-19

\twolineshloka
{एवम् हि कथयन्त्याः तु कौसल्यायाः शुभम् वचः}
{मन्द रश्मिर् अभूत् सुर्यो रजनी च अभ्यवर्तत} %2-62-20

\twolineshloka
{तथ प्रह्लादितः वाक्यैः देव्या कौसल्यया नृपः}
{शोकेन च समाक्रान्तः निद्राया वशम् एयिवान्} %2-62-21


॥इत्यार्षे श्रीमद्रामायणे वाल्मीकीये आदिकाव्ये अयोध्याकाण्डे कौसल्याप्रसादनम् नाम द्विषष्ठितमः सर्गः ॥२-६२॥
