\sect{द्वात्रिंशः सर्गः — वित्तविश्राणनम्}

\twolineshloka
{ततः शासनमाज्ञाय भ्रातुः प्रियकरं हितम्}
{गत्वा स प्रविवेशाशु सुयज्ञस्य निवेशनम्} %2-32-1

\twolineshloka
{तं विप्रमग्न्यगारस्थं वन्दित्वा लक्ष्मणोऽब्रवीत्}
{सखेऽभ्यागच्छ पश्य त्वं वेश्म दुष्करकारिणः} %2-32-2

\twolineshloka
{ततः सन्ध्यामुपास्थाय गत्वा सौमित्रिणा सह}
{ऋद्धं स प्राविशल्लक्ष्म्या रम्यं रामनिवेशनम्} %2-32-3

\twolineshloka
{जातरूपमयैर्मुख्यैरङ्गदैः कुण्डलैः शुभैः}
{सहेमसूत्रैर्मणिभिः केयूरैर्वलयैरपि} %2-32-4

\twolineshloka
{अन्यैश्च रत्नैर्बहुभिः काकुत्स्थः प्रत्यपूजयत्}
{सुयज्ञं स तदोवाच रामः सीताप्रचोदितः} %2-32-5

\twolineshloka
{हारं च हेमसूत्रं च भार्यायै सौम्य हारय}
{रशनां चाथ सा सीता दातुमिच्छति ते सखी} %2-32-6

\twolineshloka
{अङ्गदानि च चित्राणि केयूराणि शुभानि च}
{प्रयच्छति सखी तुभ्यं भार्यायै गच्छती वनम्} %2-32-7

\twolineshloka
{पर्यङ्कमग्र्यास्तरणं नानारत्नविभूषितम्}
{तमपीच्छति वैदेही प्रतिष्ठापयितुं त्वयि} %2-32-8

\twolineshloka
{नागः शत्रुञ्जयो नाम मातुलोऽयं ददौ मम}
{तं ते निष्कसहस्रेण ददामि द्विजपुङ्गव} %2-32-9

\twolineshloka
{इत्युक्तः स तु रामेण सुयज्ञः प्रतिगृह्य तत्}
{रामलक्ष्मणसीतानां प्रयुयोजाशिषः शिवाः} %2-32-10

\twolineshloka
{अथ भ्रातरमव्यग्रं प्रियं रामः प्रियंवदम्}
{सौमित्रिं तमुवाचेदं ब्रह्मेव त्रिदशेश्वरम्} %2-32-11

\twolineshloka
{अगस्त्यं कौशिकं चैव तावुभौ ब्राह्मणोत्तमौ}
{अर्चयाहूय सौमित्रे रत्नैः सस्यमिवाम्बुभिः} %2-32-12

\twolineshloka
{तर्पयस्व महाबाहो गोसहस्रेण राघव}
{सुवर्णरजतैश्चैव मणिभिश्च महाधनैः} %2-32-13

\twolineshloka
{कौसल्यां च य आशीर्भिर्भक्तः पर्युपतिष्ठति}
{आचार्यस्तैत्तिरीयाणामभिरूपश्च वेदवित्} %2-32-14

\twolineshloka
{तस्य यानं च दासीश्च सौमित्रे सम्प्रदापय}
{कौशेयानि च वस्त्राणि यावत् तुष्यति स द्विजः} %2-32-15

\twolineshloka
{सूतश्चित्ररथश्चार्यः सचिवः सुचिरोषितः}
{तोषयैनं महार्हैश्च रत्नैर्वस्त्रैर्धनैस्तथा} %2-32-16

\twolineshloka
{पशुकाभिश्च सर्वाभिर्गवां दशशतेन च}
{ये चेमे कठकालापा बहवो दण्डमाणवाः} %2-32-17

\twolineshloka
{नित्यस्वाध्यायशीलत्वान्नान्यत् कुर्वन्ति किञ्चन}
{अलसाः स्वादुकामाश्च महतां चापि सम्मताः} %2-32-18

\twolineshloka
{तेषामशीतियानानि रत्नपूर्णानि दापय}
{शालिवाहसहस्रं च द्वे शते भद्रकांस्तथा} %2-32-19

\threelineshloka
{व्यञ्जनार्थं च सौमित्रे गोसहस्रमुपाकुरु}
{मेखलीनां महासङ्घः कौसल्यां समुपस्थितः}
{तेषां सहस्रं सौमित्रे प्रत्येकं सम्प्रदापय} %2-32-20

\twolineshloka
{अम्बा यथा नो नन्देच्च कौसल्या मम दक्षिणाम्}
{तथा द्विजातींस्तान् सर्वाल्ँलक्ष्मणार्चय सर्वशः} %2-32-21

\twolineshloka
{ततः पुरुषशार्दूलस्तद् धनं लक्ष्मणः स्वयम्}
{यथोक्तं ब्राह्मणेन्द्राणामददाद् धनदो यथा} %2-32-22

\twolineshloka
{अथाब्रवीद् बाष्पगलांस्तिष्ठतश्चोपजीविनः}
{स प्रदाय बहुद्रव्यमेकैकस्योपजीवनम्} %2-32-23

\twolineshloka
{लक्ष्मणस्य च यद् वेश्म गृहं च यदिदं मम}
{अशून्यं कार्यमेकैकं यावदागमनं मम} %2-32-24

\twolineshloka
{इत्युक्त्वा दुःखितं सर्वं जनं तमुपजीविनम्}
{उवाचेदं धनाध्यक्षं धनमानीयतां मम} %2-32-25

\twolineshloka
{ततोऽस्य धनमाजह्रुः सर्व एवोपजीविनः}
{स राशिः सुमहांस्तत्र दर्शनीयो ह्यदृश्यत} %2-32-26

\twolineshloka
{ततः स पुरुषव्याघ्रस्तद् धनं सहलक्ष्मणः}
{द्विजेभ्यो बालवृद्धेभ्यः कृपणेभ्यो ह्यदापयत्} %2-32-27

\twolineshloka
{तत्रासीत् पिङ्गलो गार्ग्यस्त्रिजटो नाम वै द्विजः}
{क्षतवृत्तिर्वने नित्यं फालकुद्दाललाङ्गली} %2-32-28

\twolineshloka
{तं वृद्धं तरुणी भार्या बालानादाय दारकान्}
{अब्रवीद् ब्राह्मणं वाक्यं स्त्रीणां भर्ता हि देवता} %2-32-29

\twolineshloka
{अपास्य फालं कुद्दालं कुरुष्व वचनं मम}
{रामं दर्शय धर्मज्ञं यदि किञ्चिदवाप्स्यसि} %2-32-30

\twolineshloka
{स भार्याया वचः श्रुत्वा शाटीमाच्छाद्य दुश्छदाम्}
{स प्रातिष्ठत पन्थानं यत्र रामनिवेशनम्} %2-32-31

\twolineshloka
{भृग्वङ्गिरःसमं दीप्त्या त्रिजटं जनसंसदि}
{आपञ्चमायाः कक्ष्याया नैतं कश्चिदवारयत्} %2-32-32

\twolineshloka
{स राममासाद्य तदा त्रिजटो वाक्यमब्रवीत्}
{निर्धनो बहुपुत्रोऽस्मि राजपुत्र महाबल} %2-32-33

\twolineshloka
{क्षतवृत्तिर्वने नित्यं प्रत्यवेक्षस्व मामिति}
{तमुवाच ततो रामः परिहाससमन्वितम्} %2-32-34

\twolineshloka
{गवां सहस्रमप्येकं न च विश्राणितं मया}
{परिक्षिपसि दण्डेन यावत्तावदवाप्स्यसे} %2-32-35

\twolineshloka
{स शाटीं परितः कट्यां सम्भ्रान्तः परिवेष्ट्य ताम्}
{आविध्य दण्डं चिक्षेप सर्वप्राणेन वेगतः} %2-32-36

\twolineshloka
{स तीर्त्वा सरयूपारं दण्डस्तस्य कराच्च्युतः}
{गोव्रजे बहुसाहस्रे पपातोक्षणसन्निधौ} %2-32-37

\twolineshloka
{तं परिष्वज्य धर्मात्मा आ तस्मात् सरयूतटात्}
{आनयामास ता गावस्त्रिजटस्याश्रमं प्रति} %2-32-38

\twolineshloka
{उवाच च तदा रामस्तं गार्ग्यमभिसान्त्वयन्}
{मन्युर्न खलु कर्तव्यः परिहासो ह्ययं मम} %2-32-39

\twolineshloka
{इदं हि तेजस्तव यद् दुरत्ययं तदेव जिज्ञासितुमिच्छता मया}
{इमं भवानर्थमभिप्रचोदितो वृणीष्व किञ्चेदपरं व्यवस्यसि} %2-32-40

\twolineshloka
{ब्रवीमि सत्येन न ते स्म यन्त्रणां धनं हि यद्यन्मम विप्रकारणात्}
{भवत्सु सम्यक्प्रतिपादनेन मयार्जितं चैव यशस्करं भवेत्} %2-32-41

\twolineshloka
{ततः सभार्यस्त्रिजटो महामुनिर्गवामनीकं प्रतिगृह्य मोदितः}
{यशोबलप्रीतिसुखोपबृंहिणीस्तदाशिषः प्रत्यवदन्महात्मनः} %2-32-42

\twolineshloka
{स चापि रामः प्रतिपूर्णपौरुषो महाधनं धर्मबलैरुपार्जितम्}
{नियोजयामास सुहृज्जने चिराद् यथार्हसम्मानवचः प्रचोदितः} %2-32-43

\twolineshloka
{द्विजः सुहृद् भृत्यजनोऽथवा तदा दरिद्रभिक्षाचरणश्च यो भवेत्}
{न तत्र कश्चिन्न बभूव तर्पितो यथार्हसम्माननदानसम्भ्रमैः} %2-32-44


॥इत्यार्षे श्रीमद्रामायणे वाल्मीकीये आदिकाव्ये अयोध्याकाण्डे वित्तविश्राणनम् नाम द्वात्रिंशः सर्गः ॥२-३२॥
