\sect{पञ्चाशः सर्गः — गुहसङ्गतम्}

\twolineshloka
{विशालान् कोसलान् रम्यान् यात्वा लक्ष्मण पूर्वजः}
{अयोध्याभिमुखो धीमान् प्राञ्ञ्लिर्वाक्वमब्रवीत्} %2-50-1

\twolineshloka
{आपृच्छे त्वाम् पुरीश्रेष्ठे काकुत्स्थपरिपालिते}
{दैवतानि च यानि त्वाम् पालयन्त्यावसन्ति च} %2-50-2

\twolineshloka
{निवृत्तवनवासस्त्वामनृणो जगतीपतेः}
{पुनर्ध्रक्ष्यामि मात्रा च पित्रा च सह सम्गतः} %2-50-3

\twolineshloka
{ततो रुधिरताम्राक्षो भुजमुद्यम्य दक्षिणम्}
{अश्रुपूर्णमुखो दीनोऽब्रवीज्जानपदम् जनम्} %2-50-4

\twolineshloka
{अनुक्रोशो दया चैव यथार्हम् मयि वह् कृतः}
{चिरम् दुःखस्य पापीयो गम्यतामर्थसिद्धये} %2-50-5

\twolineshloka
{तेऽभिवाद्य महात्मानम् कृत्वा चापि प्रदक्षिणम्}
{विलपन्तो नरा घोरम् व्यतिष्ठन्त क्वचित् क्वचित्} %2-50-6

\twolineshloka
{तथा विलपताम् तेषामतृप्तानाम् च राघवः}
{अचक्षुरिषयम् प्रायाद्यथार्कः क्षणदामुखे} %2-50-7

\twolineshloka
{ततो धान्यधनोपेतान् दानशीलजनान् शिवान्}
{अकुतश्चिद्भयान् रम्याम् श्चैत्ययूपसमावृतान्} %2-50-8

\twolineshloka
{उद्यानाम्रवनोपेतान् सम्पन्नसलिलाशयान्}
{तुष्टपुष्टजनाकीर्णान् गोकुलाकुलसेवितान्} %2-50-9

\twolineshloka
{लक्षणीयान्न रेम्द्राणाम् ब्रह्मघोषाभिनादितान्}
{रथेन पुरुषव्याघ्रः कोसलानत्यवर्तत} %2-50-10

\twolineshloka
{मध्येन मुदितम् स्फीतम् रम्योद्यानसमाकुलम्}
{राज्यम् भोग्यम् नरेन्द्राणाम् ययौ धृतिमताम् वरः} %2-50-11

\twolineshloka
{तत्र त्रिपथगाम् दिव्याम् शिव तोयाम् अशैवलाम्}
{ददर्श राघवो गन्गाम् पुण्याम् ऋषि निसेविताम्} %2-50-12

\twolineshloka
{आश्रमैरविदूर्स्थैः श्रीमद्भिः समलम् कृताम्}
{कालेऽप्सरोभिर्हृष्टाभिः सेविताम्भोह्रदाम् शिवाम्} %2-50-13

\twolineshloka
{देवदानवगन्धर्वैः किन्नरैरुपशोभिताम्}
{नागगन्धर्वपत्नीभिः सेविताम् सततम् शिवाम्} %2-50-14

\twolineshloka
{देवाक्रीडशताकीर्णाम् देवोद्यानशतायुताम्}
{देवार्थमाकाशगमाम् विख्याताम् देवपद्मिनीम्} %2-50-15

\twolineshloka
{जलघाताट्टहासोग्राम् फेननिर्मलहासिनीम्}
{क्वचिद्वेणीकृतजलाम् क्वचिदावर्तशोभिताम्} %2-50-16

\twolineshloka
{क्वचित्स्तिमितगम्भीराम् क्वचिद्वेगजलाकुलाम्}
{क्वचिद्गम्भीरनिर्घोषाम् क्वचिद्भैरवनिस्वनाम्} %2-50-17

\twolineshloka
{देवसम्घाप्लुतजलाम् निर्मलोत्पलशोभिताम्}
{क्वचिदाभोगपुलिनाम् क्वचिन्नर्मलवालुकाम्} %2-50-18

\twolineshloka
{हम्स सरस सम्घुष्टाम् चक्र वाक उपकूजिताम्}
{सदामदैश्च विहगैरभिसम्नादिताम् तराम्} %2-50-19

\twolineshloka
{क्वचित्तीररुहैर्वृक्षैर्मालाभिरिव शोभिताम्}
{क्वचित्फुल्लोत्पलच्छन्नाम् क्वचित्पद्मवनाकुलाम्} %2-50-20

\twolineshloka
{क्वचित्कुमुदष्ण्डैश्च कुड्मलैरुपशोभिताम्}
{नानापुष्परजोध्वस्ताम् समदामिव च क्वचित्} %2-50-21

\twolineshloka
{व्यपेतमलसम्घाताम् मणिनिर्मलदर्शनाम्}
{दिशागजैर्वनगजैर्मत्तैश्च वरवारणैः} %2-50-22

\twolineshloka
{देवोपवाह्यश्च मुहुः सम्नादितवनान्तराम्}
{प्रमदामिव यत्ने न भूषिताम् भूषणोत्तमैः} %2-50-23

\twolineshloka
{फलैः पुष्पैः किसलयैर्वऋताम् गुल्मैद्द्विजैस्तथा}
{शिम्शुमरैः च नक्रैः च भुजम्गैः च निषेविताम्} %2-50-24

\twolineshloka
{विष्णुपादच्युताम् दिव्यामपापाम् पापनाशिनीम्}
{ताम् शङ्करजटाजूटाद्भ्रष्टाम् सागरतेजसा} %2-50-25

\twolineshloka
{समुद्रमहीषीम् गङ्गाम् सारसक्रौञ्चनादिताम्}
{आससाद महाबाहुः शृङ्गिबेरपुरम् प्रति} %2-50-26

\twolineshloka
{ताम् ऊर्मि कलिल आवर्ताम् अन्ववेक्ष्य महा रथः}
{सुमन्त्रम् अब्रवीत् सूतम् इह एव अद्य वसामहे} %2-50-27

\twolineshloka
{अविदूरात् अयम् नद्या बहु पुष्प प्रवालवान्}
{सुमहान् इन्गुदी वृक्षो वसामः अत्र एव सारथे} %2-50-28

\twolineshloka
{द्रक्ष्यामः सरिताम् श्रेष्ठाम् सम्मान्यसलिलाम् शिवाम्}
{देवदानवगन्धर्वमृगमानुषपक्षिणाम्} %2-50-29

\twolineshloka
{लक्षणः च सुमन्त्रः च बाढम् इति एव राघवम्}
{उक्त्वा तम् इन्गुदी वृक्षम् तदा उपययतुर् हयैः} %2-50-30

\twolineshloka
{रामः अभियाय तम् रम्यम् वृक्षम् इक्ष्वाकु नन्दनः}
{रथात् अवातरत् तस्मात् सभार्यः सह लक्ष्मणः} %2-50-31

\twolineshloka
{सुमन्त्रः अपि अवतीर्य एव मोचयित्वा हय उत्तमान्}
{वृक्ष मूल गतम् रामम् उपतस्थे कृत अन्जलिः} %2-50-32

\twolineshloka
{तत्र राजा गुहो नाम रामस्य आत्म समः सखा}
{निषाद जात्यो बलवान् स्थपतिः च इति विश्रुतः} %2-50-33

\twolineshloka
{स श्रुत्वा पुरुष व्याघ्रम् रामम् विषयम् आगतम्}
{वृद्धैः परिवृतः अमात्यैः ज्ञातिभिः च अपि उपागतः} %2-50-34

\twolineshloka
{ततः निषाद अधिपतिम् दृष्ट्वा दूरात् अवस्थितम्}
{सह सौमित्रिणा रामः समागच्चद् गुहेन सः} %2-50-35

\twolineshloka
{तम् आर्तः सम्परिष्वज्य गुहो राघवम् अब्रवीत्}
{यथा अयोध्या तथा इदम् ते राम किम् करवाणि ते} %2-50-36

\threelineshloka
{ईदृशम् हि महाबाहो कः प्रप्स्यत्यतिथिम् प्रियम्}
{ततः गुणवद् अन्न अद्यम् उपादाय पृथग् विधम्}
{अर्घ्यम् च उपानयत् क्षिप्रम् वाक्यम् च इदम् उवाच ह} %2-50-37

\twolineshloka
{स्वागतम् ते महा बाहो तव इयम् अखिला मही}
{वयम् प्रेष्या भवान् भर्ता साधु राज्यम् प्रशाधि नः} %2-50-38

\twolineshloka
{भक्ष्यम् भोज्यम् च पेयम् च लेह्यम् च इदम् उपस्थितम्}
{शयनानि च मुख्यानि वाजिनाम् खादनम् च ते} %2-50-39

\onelineshloka
{गुहम् एव ब्रुवाणम् तम् राघवः प्रत्युवाच ह} %2-50-40

\twolineshloka
{अर्चिताः चैव हृष्टाः च भवता सर्वथा वयम्}
{पद्भ्याम् अभिगमाच् चैव स्नेह सम्दर्शनेन च} %2-50-41

\twolineshloka
{भुजाभ्याम् साधु वृत्ताभ्याम् पीडयन् वाक्यम् अब्रवीत्}
{दिष्ट्या त्वाम् गुह पश्यामिअरोगम् सह बान्धवैः} %2-50-42

\threelineshloka
{अपि ते कूशलम् राष्ट्रे मित्रेषु च धनेषु च}
{यत् तु इदम् भवता किम्चित् प्रीत्या समुपकल्पितम्}
{सर्वम् तत् अनुजानामि न हि वर्ते प्रतिग्रहे} %2-50-43

\twolineshloka
{कुश चीर अजिन धरम् फल मूल अशनम् च माम्}
{विद्धि प्रणिहितम् धर्मे तापसम् वन गोचरम्} %2-50-44

\twolineshloka
{अश्वानाम् खादनेन अहम् अर्थी न अन्येन केनचित्}
{एतावता अत्र भवता भविष्यामि सुपूजितः} %2-50-45

\twolineshloka
{एते हि दयिता राज्ञः पितुर् दशरथस्य मे}
{एतैः सुविहितैः अश्वैः भविष्याम्य् अहम् अर्चितः} %2-50-46

\twolineshloka
{अश्वानाम् प्रतिपानम् च खादनम् चैव सो अन्वशात्}
{गुहः तत्र एव पुरुषाम्स् त्वरितम् दीयताम् इति} %2-50-47

\twolineshloka
{ततः चीर उत्तर आसन्गः सम्ध्याम् अन्वास्य पश्चिमाम्}
{जलम् एव आददे भोज्यम् लक्ष्मणेन आहृतम् स्वयम्} %2-50-48

\twolineshloka
{तस्य भूमौ शयानस्य पादौ प्रक्षाल्य लक्ष्मणः}
{सभार्यस्य ततः अभ्येत्य तस्थौ वृष्कम् उपाश्रितः} %2-50-49

\twolineshloka
{गुहो अपि सह सूतेन सौमित्रिम् अनुभाषयन्}
{अन्वजाग्रत् ततः रामम् अप्रमत्तः धनुर् धरः} %2-50-50

\fourlineindentedshloka
{तथा शयानस्य ततः अस्य धीमतः}
{यशस्विनो दाशरथेर् महात्मनः}
{अदृष्ट दुह्खस्य सुख उचितस्य सा}
{तदा व्यतीयाय चिरेण शर्वरी} %2-50-51


॥इत्यार्षे श्रीमद्रामायणे वाल्मीकीये आदिकाव्ये अयोध्याकाण्डे गुहसङ्गतम् नाम पञ्चाशः सर्गः ॥२-५०॥
