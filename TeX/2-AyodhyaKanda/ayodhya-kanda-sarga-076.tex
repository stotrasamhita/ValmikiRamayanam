\sect{षट्सप्ततितमः सर्गः — दशरथौर्ध्वदैहिकम्}

\twolineshloka
{तम् एवम् शोक सम्तप्तम् भरतम् केकयी सुतम्}
{उवाच वदताम् श्रेष्ठो वसिष्ठः श्रेष्ठ वाग् ऋषिः} %2-76-1

\twolineshloka
{अलम् शोकेन भद्रम् ते राज पुत्र महा यशः}
{प्राप्त कालम् नर पतेः कुरु सम्यानम् उत्तरम्} %2-76-2

\twolineshloka
{वसिष्ठस्य वचः श्रुत्वा भरतः धारणाम् गतः}
{प्रेत कार्याणि सर्वाणि कारयाम् आस धर्मवित्} %2-76-3

\twolineshloka
{उद्धृतम् तैल सम्क्लेदात् स तु भूमौ निवेशितम्}
{आपीत वर्ण वदनम् प्रसुप्तम् इव भूमिपम्} %2-76-4

\twolineshloka
{सम्वेश्य शयने च अग्र्ये नाना रत्न परिष्कृते}
{ततः दशरथम् पुत्रः विललाप सुदुह्खितः} %2-76-5

\twolineshloka
{किम् ते व्यवसितम् राजन् प्रोषिते मय्य् अनागते}
{विवास्य रामम् धर्मज्ञम् लक्ष्मणम् च महा बलम्} %2-76-6

\twolineshloka
{क्व यास्यसि महा राज हित्वा इमम् दुह्खितम् जनम्}
{हीनम् पुरुष सिम्हेन रामेण अक्लिष्ट कर्मणा} %2-76-7

\twolineshloka
{योग क्षेमम् तु ते राजन् को अस्मिन् कल्पयिता पुरे}
{त्वयि प्रयाते स्वः तात रामे च वनम् आश्रिते} %2-76-8

\twolineshloka
{विधवा पृथिवी राजम्स् त्वया हीना न राजते}
{हीन चन्द्रा इव रजनी नगरी प्रतिभाति माम्} %2-76-9

\twolineshloka
{एवम् विलपमानम् तम् भरतम् दीन मानसम्}
{अब्रवीद् वचनम् भूयो वसिष्ठः तु महान् ऋषिः} %2-76-10

\twolineshloka
{प्रेत कार्याणि यानि अस्य कर्तव्यानि विशाम्पतेः}
{तानि अव्यग्रम् महा बाहो क्रियताम् अविचारितम्} %2-76-11

\twolineshloka
{तथा इति भरतः वाक्यम् वसिष्ठस्य अभिपूज्य तत्}
{ऋत्विक् पुरोहित आचार्याम्स् त्वरयाम् आस सर्वशः} %2-76-12

\twolineshloka
{ये तु अग्रतः नर इन्द्रस्याग्नि अगारात् बहिष् कृताः}
{ऋत्विग्भिर् याजकैः चैव ते ह्रियन्ते यथा विधि} %2-76-13

\twolineshloka
{शिबिलायाम् अथ आरोप्य राजानम् गत चेतनम्}
{बाष्प कण्ठा विमनसः तम् ऊहुः परिचारकाः} %2-76-14

\twolineshloka
{हिरण्यम् च सुवर्णम् च वासाम्सि विविधानि च}
{प्रकिरन्तः जना मार्गम् नृपतेर् अग्रतः ययुः} %2-76-15

\twolineshloka
{चन्दन अगुरु निर्यासान् सरलम् पद्मकम् तथा}
{देव दारूणि च आहृत्य चिताम् चक्रुस् तथा अपरे} %2-76-16

\twolineshloka
{गन्धान् उच्च अवचामः च अन्याम्स् तत्र दत्त्वा अथ भूमिपम्}
{ततः सम्वेशयाम् आसुः चिता मध्ये तम् ऋत्विजः} %2-76-17

\twolineshloka
{तथा हुत अशनम् हुत्वा जेपुस् तस्य तदा ऋत्विजः}
{जगुः च ते यथा शास्त्रम् तत्र सामानि सामगाः} %2-76-18

\twolineshloka
{शिबिकाभिः च यानैः च यथा अर्हम् तस्य योषितः}
{नगरान् निर्ययुस् तत्र वृद्धैः परिवृताः तदा} %2-76-19

\twolineshloka
{प्रसव्यम् च अपि तम् चक्रुर् ऋत्विजो अग्नि चितम् नृपम्}
{स्त्रियः च शोक सम्तप्ताः कौसल्या प्रमुखाः तदा} %2-76-20

\twolineshloka
{क्रौन्चीनाम् इव नारीणाम् निनादः तत्र शुश्रुवे}
{आर्तानाम् करुणम् काले क्रोशन्तीनाम् सहस्रशः} %2-76-21

\twolineshloka
{ततः रुदन्त्यो विवशा विलप्य च पुनः पुनः}
{यानेभ्यः सरयू तीरम् अवतेरुर् वर अन्गनाः} %2-76-22

\fourlineindentedshloka
{कृत उदकम् ते भरतेन सार्धम्}
{नृप अन्गना मन्त्रि पुरोहिताः च}
{पुरम् प्रविश्य अश्रु परीत नेत्रा}
{भूमौ दश अहम् व्यनयन्त दुह्खम्} %2-76-23


॥इत्यार्षे श्रीमद्रामायणे वाल्मीकीये आदिकाव्ये अयोध्याकाण्डे दशरथौर्ध्वदैहिकम् नाम षट्सप्ततितमः सर्गः ॥२-७६॥
