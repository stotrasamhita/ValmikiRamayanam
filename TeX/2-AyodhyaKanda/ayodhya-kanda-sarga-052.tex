\sect{द्विपञ्चाशः सर्गः — गङ्गातरणम्}

\twolineshloka
{प्रभातायाम् तु शर्वर्याम् पृथु वृक्षा महा यशाः}
{उवाच रामः सौमित्रिम् लक्ष्मणम् शुभ लक्षणम्} %2-52-1

\twolineshloka
{भास्कर उदय कालो अयम् गता भगवती निशा}
{असौ सुकृष्णो विहगः कोकिलः तात कूजति} %2-52-2

\twolineshloka
{बर्हिणानाम् च निर्घोषः श्रूयते नदताम् वने}
{तराम जाह्नवीम् सौम्य शीघ्रगाम् सागरम् गमाम्} %2-52-3

\twolineshloka
{विज्ञाय रामस्य वचः सौमित्रिर् मित्र नन्दनः}
{गुहम् आमन्त्र्य सूतम् च सो अतिष्ठद् भ्रातुर् अग्रतः} %2-52-4

\twolineshloka
{स तु रामस्य वचनम् निशम्य प्रतिगृह्य च}
{स्थपतिस्तूर्णमाहुय सचिवानिदमब्रवीत्} %2-52-5

\twolineshloka
{अस्य वाहनसम्युक्ताम् कर्णग्राहवतीम् शुभाम्}
{सुप्रताराम् दृढाम् तीर्खे शीग्रम् नावमुपाहर} %2-52-6

\twolineshloka
{तम् निशम्य समादेशम् गुहामात्यगणो महान्}
{उपोह्य रुचिराम् नावम् गुहाय प्रत्यवेदयत्} %2-52-7

\twolineshloka
{ततः सप्राञ्जलिर्भूत्वा गुहो राघवमब्रवीत्}
{उपस्थितेयम् नौर्देव भूयः किम् करवाणि ते} %2-52-8

\twolineshloka
{तवामरसुतप्रख्य तर्तुम् सागरगाम् नदीम्}
{नौरियम् पुरुषव्याग्र ताम् त्वमारोह सुव्रत} %2-52-9

\twolineshloka
{अथोवाच महातेजा रामो गुहमिदम् वचः}
{कृतकामोऽस्मि भवता शीघ्रमारोप्यतामिति} %2-52-10

\twolineshloka
{ततः कलापान् सम्नह्य खड्गौ बद्ध्वा च धन्विनौ}
{जग्मतुर् येन तौ गन्गाम् सीतया सह राघवौ} %2-52-11

\twolineshloka
{रामम् एव तु धर्मज्ञम् उपगम्य विनीतवत्}
{किम् अहम् करवाणि इति सूतः प्रान्जलिर् अब्रवीत्} %2-52-12

\fourlineindentedshloka
{ततोऽब्रवीद्दाशरथिः सुमन्त्रम्}
{स्पृशन् करेणोत्तमदक्षिणेन}
{सुमन्त्र शीघ्रम् पुनरेव याहि}
{राज्ञः सकाशे भवचाप्रमत्तः} %2-52-13

\twolineshloka
{निवर्तस्व इति उवाच एनम् एतावद्द् हि कृतम् मम}
{रथम् विहाय पद्भ्याम् तु गमिष्यामि महावनम्} %2-52-14

\twolineshloka
{आत्मानम् तु अभ्यनुज्ञातम् अवेक्ष्य आर्तः स सारथिः}
{सुमन्त्रः पुरुष व्याघ्रम् ऐक्ष्वाकम् इदम् अब्रवीत्} %2-52-15

\twolineshloka
{न अतिक्रान्तम् इदम् लोके पुरुषेण इह केनचित्}
{तव सभ्रातृ भार्यस्य वासः प्राकृतवद् वने} %2-52-16

\twolineshloka
{न मन्ये ब्रह्म चर्ये अस्ति स्वधीते वा फल उदयः}
{मार्दव आर्जवयोः वा अपि त्वाम् चेद् व्यसनम् आगतम्} %2-52-17

\twolineshloka
{सह राघव वैदेह्या भ्रात्रा चैव वने वसन्}
{त्वम् गतिम् प्राप्स्यसे वीर त्रीम्ल् लोकाम्स् तु जयन्न् इव} %2-52-18

\twolineshloka
{वयम् खलु हता राम ये तया अपि उपवन्चिताः}
{कैकेय्या वशम् एष्यामः पापाया दुह्ख भागिनः} %2-52-19

\twolineshloka
{इति ब्रुवन्न् आत्म समम् सुमन्त्रः सारथिस् तदा}
{दृष्ट्वा दुर गतम् रामम् दुह्ख आर्तः रुरुदे चिरम्} %2-52-20

\twolineshloka
{ततः तु विगते बाष्पे सूतम् स्पृष्ट उदकम् शुचिम्}
{रामः तु मधुरम् वाक्यम् पुनः पुनर् उवाच तम्} %2-52-21

\twolineshloka
{इक्ष्वाकूणाम् त्वया तुल्यम् सुहृदम् न उपलक्षये}
{यथा दशरथो राजा माम् न शोचेत् तथा कुरु} %2-52-22

\twolineshloka
{शोक उपहत चेताः च वृद्धः च जगती पतिः}
{काम भार अवसन्नः च तस्मात् एतत् ब्रवीमि ते} %2-52-23

\twolineshloka
{यद् यद् आज्ञापयेत् किम्चित् स महात्मा मही पतिः}
{कैकेय्याः प्रिय काम अर्थम् कार्यम् तत् अविकान्क्षया} %2-52-24

\twolineshloka
{एतत् अर्थम् हि राज्यानि प्रशासति नर ईश्वराः}
{यद् एषाम् सर्व कृत्येषु मनो न प्रतिहन्यते} %2-52-25

\twolineshloka
{यद्यथा स महा राजो न अलीकम् अधिगच्चति}
{न च ताम्यति दुह्खेन सुमन्त्र कुरु तत् तथा} %2-52-26

\twolineshloka
{अदृष्ट दुह्खम् राजानम् वृद्धम् आर्यम् जित इन्द्रियम्}
{ब्रूयाः त्वम् अभिवाद्य एव मम हेतोर् इदम् वचः} %2-52-27

\twolineshloka
{न एव अहम् अनुशोचामि लक्ष्मणो न च मैथिली}
{अयोध्यायाः च्युताः च इति वने वत्स्यामह इति वा महेति} %2-52-28

\twolineshloka
{चतुर् दशसु वर्षेषु निवृत्तेषु पुनः पुनः}
{लक्ष्मणम् माम् च सीताम् च द्रक्ष्यसि क्षिप्रम् आगतान्} %2-52-29

\twolineshloka
{एवम् उक्त्वा तु राजानम् मातरम् च सुमन्त्र मे}
{अन्याः च देवीः सहिताः कैकेयीम् च पुनः पुनः} %2-52-30

\twolineshloka
{आरोग्यम् ब्रूहि कौसल्याम् अथ पाद अभिवन्दनम्}
{सीताया मम च आर्यस्य वचनाल् लक्ष्मणस्य च} %2-52-31

\twolineshloka
{ब्रूयाः च हि महा राजम् भरतम् क्षिप्रम् आनय}
{आगतः च अपि भरतः स्थाप्यो नृप मते पदे} %2-52-32

\twolineshloka
{भरतम् च परिष्वज्य यौवराज्ये अभिषिच्य च}
{अस्मत् सम्तापजम् दुह्खम् न त्वाम् अभिभविष्यति} %2-52-33

\twolineshloka
{भरतः च अपि वक्तव्यो यथा राजनि वर्तसे}
{तथा मातृषु वर्तेथाः सर्वास्व् एव अविशेषतः} %2-52-34

\twolineshloka
{यथा च तव कैकेयी सुमित्रा च अविशेषतः}
{तथैव देवी कौसल्या मम माता विशेषतः} %2-52-35

\twolineshloka
{तातस्य प्रियकामेन यौवराज्यमपेक्षता}
{लोकयोरुभयोः शक्यम् त्वया यत्सुखमेधितुम्} %2-52-36

\twolineshloka
{निवर्त्यमानो रामेण सुमन्त्रः शोक कर्शितः}
{तत् सर्वम् वचनम् श्रुत्वा स्नेहात् काकुत्स्थम् अब्रवीत्} %2-52-37

\twolineshloka
{यद् अहम् न उपचारेण ब्रूयाम् स्नेहात् अविक्लवः}
{भक्तिमान् इति तत् तावद् वाक्यम् त्वम् क्षन्तुम् अर्हसि} %2-52-38

\twolineshloka
{कथम् हि त्वद् विहीनो अहम् प्रतियास्यामि ताम् पुरीम्}
{तव तात वियोगेन पुत्र शोक आकुलाम् इव} %2-52-39

\twolineshloka
{सरामम् अपि तावन् मे रथम् दृष्ट्वा तदा जनः}
{विना रामम् रथम् दृष्ट्वा विदीर्येत अपि सा पुरी} %2-52-40

\twolineshloka
{दैन्यम् हि नगरी गच्चेद् दृष्ट्वा शून्यम् इमम् रथम्}
{सूत अवशेषम् स्वम् सैन्यम् हत वीरम् इव आहवे} %2-52-41

\twolineshloka
{दूरे अपि निवसन्तम् त्वाम् मानसेन अग्रतः स्थितम्}
{चिन्तयन्त्यो अद्य नूनम् त्वाम् निराहाराः कृताः प्रजाः} %2-52-42

\twolineshloka
{दृष्टं तद्धि त्वया राम यादृशम् त्वत्प्रवासने}
{प्रजानाम् सम्कुलम् वृत्तम् त्वच्छोकक्लान्तचेतसाम्} %2-52-43

\twolineshloka
{आर्त नादो हि यः पौरैः मुक्तः तत् विप्रवासने}
{रथस्थम् माम् निशाम्य एव कुर्युः शत गुणम् ततः} %2-52-44

\twolineshloka
{अहम् किम् च अपि वक्ष्यामि देवीम् तव सुतः मया}
{नीतः असौ मातुल कुलम् सम्तापम् मा कृथाइति} %2-52-45

\twolineshloka
{असत्यम् अपि न एव अहम् ब्रूयाम् वचनम् ईदृशम्}
{कथम् अप्रियम् एव अहम् ब्रूयाम् सत्यम् इदम् वचः} %2-52-46

\twolineshloka
{मम तावन् नियोगस्थाः त्वद् बन्धु जन वाहिनः}
{कथम् रथम् त्वया हीनम् प्रवक्ष्यन्ति हय उत्तमाः} %2-52-47

\twolineshloka
{तन्न शक्ष्याम्यहम् गन्तुमयोध्याम् त्वदृतेऽनघ}
{वनवासानुयानाय मामनुज्ञातुमर्हसि} %2-52-48

\twolineshloka
{यदि मे याचमानस्य त्यागम् एव करिष्यसि}
{सरथो अग्निम् प्रवेक्ष्यामि त्यक्त मात्रैह त्वया} %2-52-49

\twolineshloka
{भविष्यन्ति वने यानि तपो विघ्न कराणि ते}
{रथेन प्रतिबाधिष्ये तानि सत्त्वानि राघव} %2-52-50

\twolineshloka
{तत् कृतेन मया प्राप्तम् रथ चर्या कृतम् सुखम्}
{आशम्से त्वत् कृतेन अहम् वन वास कृतम् सुखम्} %2-52-51

\twolineshloka
{प्रसीद इच्चामि ते अरण्ये भवितुम् प्रत्यनन्तरः}
{प्रीत्या अभिहितम् इच्चामि भव मे पत्यनन्तरः} %2-52-52

\twolineshloka
{इमे चापि हया वीर यदि ते वनवासिनः}
{परिचर्याम् करिष्यन्ति प्राप्स्यन्ति परमाम् गतिम्} %2-52-53

\twolineshloka
{तव शुश्रूषणम् मूर्ध्ना करिष्यामि वने वसन्}
{अयोध्याम् देव लोकम् वा सर्वथा प्रजहाम्य् अहम्} %2-52-54

\twolineshloka
{न हि शक्या प्रवेष्टुम् सा मया अयोध्या त्वया विना}
{राज धानी महा इन्द्रस्य यथा दुष्कृत कर्मणा} %2-52-55

\twolineshloka
{वन वासे क्षयम् प्राप्ते मम एष हि मनो रथः}
{यद् अनेन रथेन एव त्वाम् वहेयम् पुरीम् पुनः} %2-52-56

\twolineshloka
{चतुर् दश हि वर्षाणि सहितस्य त्वया वने}
{क्षण भूतानि यास्यन्ति शतशः तु ततः अन्यथा} %2-52-57

\twolineshloka
{भृत्य वत्सल तिष्ठन्तम् भर्तृ पुत्र गते पथि}
{भक्तम् भृत्यम् स्थितम् स्थित्याम् त्वम् न माम् हातुम् अर्हसि} %2-52-58

\twolineshloka
{एवम् बहु विधम् दीनम् याचमानम् पुनः पुनः}
{रामः भृत्य अनुकम्पी तु सुमन्त्रम् इदम् अब्रवीत्} %2-52-59

\twolineshloka
{जानामि परमाम् भक्तिम् मयि ते भर्तृ वत्सल}
{शृणु च अपि यद् अर्थम् त्वाम् प्रेषयामि पुरीम् इतः} %2-52-60

\twolineshloka
{नगरीम् त्वाम् गतम् दृष्ट्वा जननी मे यवीयसी}
{कैकेयी प्रत्ययम् गच्चेद् इति रामः वनम् गतः} %2-52-61

\twolineshloka
{परितुष्टा हि सा देवि वन वासम् गते मयि}
{राजानम् न अतिशन्केत मिथ्या वादी इति धार्मिकम्} %2-52-62

\twolineshloka
{एष मे प्रथमः कल्पो यद् अम्बा मे यवीयसी}
{भरत आरक्षितम् स्फीतम् पुत्र राज्यम् अवाप्नुयात्} %2-52-63

\twolineshloka
{मम प्रिय अर्थम् राज्ञः च सरथः त्वम् पुरीम् व्रज}
{सम्दिष्टः च असि या अनर्थाम्स् ताम्स् तान् ब्रूयाः तथा तथा} %2-52-64

\twolineshloka
{इति उक्त्वा वचनम् सूतम् सान्त्वयित्वा पुनः पुनः}
{गुहम् वचनम् अक्लीबम् रामः हेतुमद् अब्रवीत्} %2-52-65

\twolineshloka
{नेदानीम् गुह योग्योऽयम् वसो मे सजने वने}
{अवश्यम् ह्याश्रमे वासह् कर्तव्यस्तद्गतो विधिः} %2-52-66

\twolineshloka
{सोऽहम् गृहीत्वा नियमम् तपस्विजनभूषणम्}
{हितकामः पितुर्भूयः सीताया लक्ष्मणस्य च} %2-52-67

\twolineshloka
{जटाः कृत्वा गमिष्यामि न्यग्रोध क्षीरम् आनय}
{तत् क्षीरम् राज पुत्राय गुहः क्षिप्रम् उपाहरत्} %2-52-68

\twolineshloka
{लक्ष्मणस्य आत्मनः चैव रामः तेन अकरोज् जटाः}
{दीर्घबाहुर्नरव्याघ्रो जटिलत्व मधारयत्} %2-52-69

\twolineshloka
{तौ तदा चीर वसनौ जटा मण्डल धारिणौ}
{अशोभेताम् ऋषि समौ भ्रातरौ राम रक्ष्मणौ} %2-52-70

\twolineshloka
{ततः वैखानसम् मार्गम् आस्थितः सह लक्ष्मणः}
{व्रतम् आदिष्टवान् रामः सहायम् गुहम् अब्रवीत्} %2-52-71

\twolineshloka
{अप्रमत्तः बले कोशे दुर्गे जन पदे तथा}
{भवेथा गुह राज्यम् हि दुरारक्षतमम् मतम्} %2-52-72

\twolineshloka
{ततः तम् समनुज्ञाय गुहम् इक्ष्वाकु नन्दनः}
{जगाम तूर्णम् अव्यग्रः सभार्यः सह लक्ष्मणः} %2-52-73

\twolineshloka
{स तु दृष्ट्वा नदी तीरे नावम् इक्ष्वाकु नन्दनः}
{तितीर्षुः शीघ्रगाम् गन्गाम् इदम् लक्ष्मणम् अब्रवीत्} %2-52-74

\twolineshloka
{आरोह त्वम् नर व्याघ्र स्थिताम् नावम् इमाम् शनैः}
{सीताम् च आरोपय अन्वक्षम् परिगृह्य मनस्विनीम्} %2-52-75

\twolineshloka
{स भ्रातुः शासनम् श्रुत्वा सर्वम् अप्रतिकूलयन्}
{आरोप्य मैथिलीम् पूर्वम् आरुरोह आत्मवाम्स् ततः} %2-52-76

\twolineshloka
{अथ आरुरोह तेजस्वी स्वयम् लक्ष्मण पूर्वजः}
{ततः निषाद अधिपतिर् गुहो ज्ञातीन् अचोदयत्} %2-52-77

\twolineshloka
{राघवोऽपि महातेजा नावमारुह्य ताम् ततः}
{ब्रह्मवत् क्षत्रवच्चैव जजाप हितमात्मनः} %2-52-78

\twolineshloka
{आचम्य च यथाशास्त्रम् नदीम् ताम् सह सीतया}
{प्राणमत्प्रीतिसम्हृष्टो लक्ष्मणश्चामितप्रभः} %2-52-79

\twolineshloka
{अनुज्ञाय सुमन्त्रम् च सबलम् चैव तम् गुहम्}
{आस्थाय नावम् रामः तु चोदयाम् आस नाविकान्} %2-52-80

\twolineshloka
{ततः तैः चोदिता सा नौः कर्ण धार समाहिता}
{शुभ स्फ्य वेग अभिहता शीघ्रम् सलिलम् अत्यगात्} %2-52-81

\twolineshloka
{मध्यम् तु समनुप्राप्य भागीरथ्याः तु अनिन्दिता}
{वैदेही प्रान्जलिर् भूत्वा ताम् नदीम् इदम् अब्रवीत्} %2-52-82

\twolineshloka
{पुत्रः दशरथस्य अयम् महा राजस्य धीमतः}
{निदेशम् पालयतु एनम् गन्गे त्वद् अभिरक्षितः} %2-52-83

\twolineshloka
{चतुर् दश हि वर्षाणि समग्राणि उष्य कानने}
{भ्रात्रा सह मया चैव पुनः प्रत्यागमिष्यति} %2-52-84

\twolineshloka
{ततः त्वाम् देवि सुभगे क्षेमेण पुनर् आगता}
{यक्ष्ये प्रमुदिता गन्गे सर्व काम समृद्धये} %2-52-85

\twolineshloka
{त्वम् हि त्रिपथगा देवि ब्रह्म लोकम् समीक्षसे}
{भार्या च उदधि राजस्य लोके अस्मिन् सम्प्रदृश्यसे} %2-52-86

\twolineshloka
{सा त्वाम् देवि नमस्यामि प्रशम्सामि च शोभने}
{प्राप्त राज्ये नर व्याघ्र शिवेन पुनर् आगते} %2-52-87

\twolineshloka
{गवाम् शत सहस्राणि वस्त्राणि अन्नम् च पेशलम्}
{ब्राह्मणेभ्यः प्रदास्यामि तव प्रिय चिकीर्षया} %2-52-88

\twolineshloka
{सुराघटसहस्रेण माम्सभूतोदनेन च}
{यक्ष्ये त्वाम् प्रयता देवि पुरीम् पुनरुपागता} %2-52-89

\twolineshloka
{यानि त्वत्तीरवासीनि दैवतानि च सन्ति हि}
{तानि सर्वाणि यक्ष्यामि तीर्थान्यायतनानि च} %2-52-90

\twolineshloka
{पुनरेव महाबाउर्मया भ्रात्रा च सम्गतः}
{अयोध्याम् वनवासात्तु प्रविशत्वनघोऽनघे} %2-52-91

\twolineshloka
{तथा सम्भाषमाणा सा सीता गन्गाम् अनिन्दिता}
{दक्षिणा दक्षिणम् तीरम् क्षिप्रम् एव अभ्युपागमत्} %2-52-92

\twolineshloka
{तीरम् तु समनुप्राप्य नावम् हित्वा नर ऋषभः}
{प्रातिष्ठत सह भ्रात्रा वैदेह्या च परम् तपः} %2-52-93

\twolineshloka
{अथ अब्रवीन् महा बाहुः सुमित्र आनन्द वर्धनम्}
{भव सम्रक्षणार्थाय सजने विजनेऽपि वा} %2-52-94

\twolineshloka
{अवश्यम् रक्षणम् कार्यमदृष्टे विजने वने}
{अग्रतः गच्च सौमित्रे सीता त्वाम् अनुगच्चतु} %2-52-95

\twolineshloka
{पृष्ठतः अहम् गमिष्यामि त्वाम् च सीताम् च पालयन्}
{अद्य दुह्खम् तु वैदेही वन वासस्य वेत्स्यति} %2-52-96

\twolineshloka
{न हि तावदतिक्रान्ता सुकरा काचन क्रिया}
{अद्य दुःखम् तु वैदेही वनवासस्य वेत्स्यति} %2-52-97

\twolineshloka
{प्रणष्टजनसम्बाधम् क्षेत्रारामविवर्बितम्}
{विषमम् च प्रपातम् च वनमद्य प्रवेक्ष्यति} %2-52-98

\twolineshloka
{श्रुत्वा रामस्य वचनम् प्रतिस्थे लक्ष्मण्ऽग्रतः}
{अनन्तरम् च सीताया राघवो रघनन्धनः} %2-52-99

\fourlineindentedshloka
{गतम् तु गन्गा पर पारम् आशु}
{रामम् सुमन्त्रः प्रततम् निरीक्ष्य}
{अध्व प्रकर्षात् विनिवृत्त दृष्टिर्}
{र्मुमोच बाष्पम् व्यथितः तपस्वी} %2-52-100

\fourlineindentedshloka
{स लोकपालप्रतिमप्रभाववाम्}
{स्तीर्त्वा महात्मा वरदो महानदीम्}
{ततः समृद्धान् शुभसस्यमालिनः}
{क्रमेण वत्सान् मुदितानुपागमत्} %2-52-101

\fourlineindentedshloka
{तौ तत्र हत्वा चतुरः महामृगान्}
{वराहम् ऋष्यम् पृषतम् महारुरुम्}
{आदाय मेध्यम् त्वरितम् बुभुक्षितौ}
{वासाय काले ययतुर् वनः पतिम्} %2-52-102


॥इत्यार्षे श्रीमद्रामायणे वाल्मीकीये आदिकाव्ये अयोध्याकाण्डे गङ्गातरणम् नाम द्विपञ्चाशः सर्गः ॥२-५२॥
