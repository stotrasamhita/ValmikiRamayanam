\sect{सप्तदशः सर्गः — रामागमनम्}

\twolineshloka
{स रामो रथमास्थाय सम्प्रहृष्टसुहृज्जनः}
{पताकाध्वजसम्पन्नं महार्हागुरुधूपितम्} %2-17-1

\twolineshloka
{अपश्यन्नगरं श्रीमान् नानाजनसमन्वितम्}
{स गृहैरभ्रसंकाशैः पाण्डुरैरुपशोभितम्} %2-17-2

\twolineshloka
{राजमार्गं ययौ रामो मध्येनागुरुधूपितम्}
{चन्दनानां च मुख्यानामगुरूणां च संचयैः} %2-17-3

\twolineshloka
{उत्तमानां च गन्धानां क्षौमकौशाम्बरस्य च}
{अविद्धाभिश्च मुक्ताभिरुत्तमैः स्फाटिकैरपि} %2-17-4

\twolineshloka
{शोभमानमसम्बाधं तं राजपथमुत्तमम्}
{संवृतं विविधैः पुष्पैर्भक्ष्यैरुच्चावचैरपि} %2-17-5

\twolineshloka
{ददर्श तं राजपथं दिवि देवपतिर्यथा}
{दध्यक्षतहविर्लाजैर्धूपैरगुरुचन्दनैः} %2-17-6

\twolineshloka
{नानामाल्योपगन्धैश्च सदाभ्यर्चितचत्वरम्}
{आशीर्वादान् बहून् शृण्वन् सुहृद्भिः समुदीरितान्} %2-17-7

\twolineshloka
{यथार्हं चापि सम्पूज्य सर्वानेव नरान् ययौ}
{पितामहैराचरितं तथैव प्रपितामहैः} %2-17-8

\threelineshloka
{अद्योपादाय तं मार्गमभिषिक्तोऽनुपालय}
{यथा स्म पोषिताः पित्रा यथा सर्वैः पितामहैः}
{ततः सुखतरं सर्वे रामे वत्स्याम राजनि} %2-17-9

\twolineshloka
{अलमद्य हि भुक्तेन परमार्थैरलं च नः}
{यदि पश्याम निर्यान्तं रामं राज्ये प्रतिष्ठितम्} %2-17-10

\twolineshloka
{ततो हि नः प्रियतरं नान्यत् किंचिद् भविष्यति}
{यथाभिषेको रामस्य राज्येनामिततेजसः} %2-17-11

\twolineshloka
{एताश्चान्याश्च सुहृदामुदासीनः शुभाः कथाः}
{आत्मसम्पूजनीः शृण्वन् ययौ रामो महापथम्} %2-17-12

\twolineshloka
{न हि तस्मान्मनः कश्चिच्चक्षुषी वा नरोत्तमात्}
{नरः शक्नोत्यपाक्रष्टुमतिक्रान्तेऽपि राघवे} %2-17-13

\twolineshloka
{यश्च रामं न पश्येत्तु यं च रामो न पश्यति}
{निन्दितः सर्वलोकेषु स्वात्माप्येनं विगर्हते} %2-17-14

\twolineshloka
{सर्वेषु स हि धर्मात्मा वर्णानां कुरुते दयाम्}
{चतुर्णां हि वयःस्थानां तेन ते तमनुव्रताः} %2-17-15

\twolineshloka
{चतुष्पथान् देवपथांश्चैत्यांश्चायतनानि च}
{प्रदक्षिणं परिहरज्जगाम नृपतेः सुतः} %2-17-16

\twolineshloka
{स राजकुलमासाद्य मेघसङ्घोपमैः शुभैः}
{प्रासादशृङ्गैर्विविधैः कैलासशिखरोपमैः} %2-17-17

\twolineshloka
{आवारयद्भिर्गगनं विमानैरिव पाण्डुरैः}
{वर्धमानगृहैश्चापि रत्नजालपरिष्कृतैः} %2-17-18

\twolineshloka
{तत् पृथिव्यां गृहवरं महेन्द्रसदनोपमम्}
{राजपुत्रः पितुर्वेश्म प्रविवेश श्रिया ज्वलन्} %2-17-19

\twolineshloka
{स कक्ष्या धन्विभिर्गुप्तास्तिस्रोऽतिक्रम्य वाजिभिः}
{पदातिरपरे कक्ष्ये द्वे जगाम नरोत्तमः} %2-17-20

\twolineshloka
{स सर्वाः समतिक्रम्य कक्ष्या दशरथात्मजः}
{संनिवर्त्य जनं सर्वं शुद्धान्तःपुरमत्यगात्} %2-17-21

\twolineshloka
{तस्मिन् प्रविष्टे पितुरन्तिकं तदा जनः स सर्वो मुदितो नृपात्मजे}
{प्रतीक्षते तस्य पुनः स्म निर्गमं यथोदयं चन्द्रमसः सरित्पतिः} %2-17-22


॥इत्यार्षे श्रीमद्रामायणे वाल्मीकीये आदिकाव्ये अयोध्याकाण्डे रामागमनम् नाम सप्तदशः सर्गः ॥२-१७॥
