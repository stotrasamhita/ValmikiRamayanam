\sect{द्व्यधिकशततमः सर्गः — निवापदानम्}

\twolineshloka
{तां श्रुत्वा करुणां वाचं पितुर्मरणसंहिताम्}
{राघवो भरतेनोक्तां बभूव गतचेतनः} %2-102-1

\twolineshloka
{तं तु वज्रमिवोत्सृष्टमाहवे दानवारिणा}
{वाग्वज्रं भरतेनोक्तममनोज्ञं परन्तपः} %2-102-2

\twolineshloka
{प्रगृह्य बाहू रामो वै पुष्पिताग्रो यथा द्रुमः}
{वने परशुना कृत्तस्तथा भुवि पपात ह} %2-102-3

\twolineshloka
{तथा निपतितं रामं जगत्यां जगतीपतिम्}
{कूलघातपरिश्रान्तं प्रसुप्तमिव कुञ्जरम्} %2-102-4

\twolineshloka
{भ्रातरस्ते महेष्वासं सर्वतः शोककर्शितम्}
{रुदन्तः सह वैदेह्या सिषिचुः सलिलेन वै} %2-102-5

\twolineshloka
{स तु संज्ञां पुनर्लब्ध्वा नेत्राभ्यामास्रमुत्सृजन्}
{उपाक्रामत काकुत्स्थः कृपणं बहु भाषितुम्} %2-102-6

\twolineshloka
{स रामः स्वर्गतं श्रुत्वा पितरं पृथिवीपतिम्}
{उवाच भरतं वाक्यं धर्मात्मा धर्मसंहितम्} %2-102-7

\twolineshloka
{किं करिष्याम्ययोध्यायां ताते दिष्टां गतिं गते}
{कस्तां राजवराद्धीनामयोध्यां पालयिष्यति} %2-102-8

\twolineshloka
{किं नु तस्य मया कार्य्यं दुर्जातेन महात्मनः}
{यो मृतो मम शोकेन मया चापि न संस्कृतः} %2-102-9

\twolineshloka
{अहो भरत सिद्धार्थो येन राजा त्वयाऽनघ}
{शत्रुघ्नेन च सर्वेषु प्रेतकृत्येषु सत्कृतः} %2-102-10

\twolineshloka
{निष्प्रधानामनेकाग्रां नरेन्द्रेण विना कृताम्}
{निवृत्तवनवासोपि नायोध्यां गन्तुमुत्सहे} %2-102-11

\twolineshloka
{समाप्तवनवासं मामयोध्यायां परन्तप}
{को नु शासिष्यति पुनस्ताते लोकान्तरं गते} %2-102-12

\twolineshloka
{पुरा प्रेक्ष्य सुवृत्तं मां पिता यान्याह सान्त्वयन्}
{वाक्यानि तानि श्रोष्यामि कुतः कर्णसुखान्यहम्} %2-102-13

\twolineshloka
{एवमुक्त्वा स भरतं भार्यामभ्येत्य राघवः}
{उवाच शोकसन्तप्तः पूर्णचन्द्रनिभाननाम्} %2-102-14

\twolineshloka
{सीते मृतस्ते श्वशुरः पित्रा हीनोऽसि लक्ष्मण}
{भरतो दुःखमाचष्टे स्वर्गतं पृथिवीपतिम्} %2-102-15

\twolineshloka
{ततो बहुगुणं तेषां बाष्पं नेत्रेष्वजायत}
{तथा ब्रुवति काकुत्स्थे कुमाराणां यशस्विनाम्} %2-102-16

\twolineshloka
{ततस्ते भ्रातरः सर्वे भृशमाश्वास्य राघवम्}
{अब्रुवन् जगतीभर्त्तुः क्रियतामुदकं पितुः} %2-102-17

\twolineshloka
{सा सीता श्वशुरं श्रुत्वा स्वर्गलोकगतं नृपम्}
{नेत्राभ्यामश्रुपूर्णाभ्यामशकन्नेक्षितुं पतिम्} %2-102-18

\twolineshloka
{सान्त्वयित्वा तु तां रामो रुदन्तीं जनकात्मजाम्}
{उवाच लक्ष्मणं तत्र दुःखितो दुःखितं वचः} %2-102-19

\twolineshloka
{आनयेङ्गुदिपिण्याकं चीरमाहर चोत्तरम्}
{जलक्रियार्थं तातस्य गमिष्यामि महात्मनः} %2-102-20

\twolineshloka
{सीता पुरस्ताद्व्रजतुत्वमेनामभितो व्रज}
{अहं पश्चाद्गमिष्यामि गतिर्ह्येषा सुदारुणा} %2-102-21

\twolineshloka
{ततो नित्यानुगस्तेषां विदितात्मा महामतिः}
{मृदुर्दान्तश्च शान्तश्च रामे च दृढभक्तिमान्} %2-102-22

\twolineshloka
{सुमन्त्रस्तैर्नृपसुतैः सार्द्धमाश्वास्य राघवम्}
{अवातारयदालम्ब्य नदीं मन्दाकिनीं शिवाम्} %2-102-23

\twolineshloka
{ते सुतीर्थां ततः कृच्छ्रादुपागम्य यशस्विनः}
{नदीं मन्दाकिनीं रम्यां सदा पुष्पितकाननाम्} %2-102-24

\twolineshloka
{शीघ्रस्रोतसमासाद्य तीर्थं शिवमकर्द्दमम्}
{सिषिचुस्तूदकं राज्ञे तत्रैतत्ते भवत्विति} %2-102-25

\twolineshloka
{प्रगृह्य च महीपालो जलपूरितमञ्जलिम्}
{दिशं याम्यामभिमुखो रुदन् वचनमब्रवीत्} %2-102-26

\twolineshloka
{एतत्ते राजशार्दूल विमलं तोयमक्षयम्}
{पितृलोकगतस्याद्य मद्दत्तमुपतिष्ठतु} %2-102-27

\twolineshloka
{ततो मन्दाकिनीतीरात् प्रत्युत्तीर्य्य स राघवः}
{पितुश्चकार तेजस्वी निवापं भ्रातृभिः सह} %2-102-28

\twolineshloka
{ऐङ्गुदं बदरीमिश्रं पिण्याकं दर्भसंस्तरे}
{न्यस्य रामः सुदुःखार्त्तो रुदन् वचनमब्रवीत्} %2-102-29

\twolineshloka
{इदं भुङ्क्ष्व महाराज प्रीतो यदशना वयम्}
{यदन्नः पुरुषो भवति तदन्नास्तस्य देवताः} %2-102-30

\twolineshloka
{ततस्तेनैव मार्गेण प्रत्युत्तीर्य्य नदीतटात्}
{आरुरोह नरव्याघ्रो रम्यसानुं महीधरम्} %2-102-31

\twolineshloka
{ततः पर्णकुटीद्वारमासाद्य जगतीपतिः}
{परिजग्राह बाहुभ्यामुभौ भरतलक्ष्मणौ} %2-102-32

\twolineshloka
{तेषां तु रुदतां शब्दात् प्रतिश्रुत्कोऽभवद्गिरौ}
{भ्रातऽणां सह वैदेह्याः सिंहानामिव नर्दताम्} %2-102-33

\twolineshloka
{महाबलानां रुदतां कुर्वतामुदकं पितुः}
{विज्ञाय तुमुलं शब्दं त्रस्ता भरतसैनिकाः} %2-102-34

\twolineshloka
{अब्रुवंश्चापि रामेण भरतः सङ्गतो ध्रुवम्}
{तेषामेव महाञ्छब्दः शोचतां पितरं मृतम्} %2-102-35

\twolineshloka
{अथ वासान् परित्यज्य तं सर्वेऽभिमुखाः स्वनम्}
{अप्येकमनसो जग्मुर्यथास्थानं प्रधाविताः} %2-102-36

\twolineshloka
{हयैरन्ये गजैरन्ये रथैरन्ये स्वलङ्कृतैः}
{सुकुमारास्तथैवान्ये पद्भिरेव नरा ययुः} %2-102-37

\twolineshloka
{अचिरप्रोषितं रामं चिरविप्रोषितं यथा}
{द्रष्टुकामो जनः सर्वो जगाम सहसाश्रमम्} %2-102-38

\twolineshloka
{भ्रातऽणां त्वरितास्तत्र द्रष्टुकामाः समागमम्}
{ययुर्बहुविधैर्यानैः खुरनेमिस्वनाकुलैः} %2-102-39

\twolineshloka
{सा भूमिर्बहुभिर्यानैः खुरनेमिसमाहता}
{मुमोच तुमुलं शब्दं द्यौरिवाभ्रसमागमे} %2-102-40

\twolineshloka
{तेन वित्रासिता नागाः करेणुपरिवारिताः}
{आवासयन्तो गन्धेन जग्मुरन्यद्वनं ततः} %2-102-41

\twolineshloka
{वराहवृकसङ्घाश्च महिषाः सर्प्पवानराः}
{व्याघ्रगोकर्णगवयाः वित्रेसुः पृषतैः सह} %2-102-42

\twolineshloka
{रथाङ्गसाह्वा नत्यूहाः हंसाः कारण्डवाः प्लवाः}
{तथा पुंस्कोकिलाः क्रौञ्चा विसंज्ञा भेजिरे दिशः} %2-102-43

\twolineshloka
{तेन शब्देन वित्रस्तैराकाशं पक्षिभिर्वृतम्}
{मनुष्यैरावृता भूमिरुभयं प्रबभौ तदा} %2-102-44

\twolineshloka
{ततस्तं पुरुषव्याघ्रं यशस्विनमरिन्दमम्}
{आसीनं स्थण्डिले रामं ददर्श सहसा जनः} %2-102-45

\twolineshloka
{विगर्हमाणः कैकेयीं सहितो मन्थरामपि}
{अभिगम्य जनो रामं बाष्पपूर्णमुखोऽभवत्} %2-102-46

\twolineshloka
{तान्नरान् बाष्पपूर्णाक्षान् समीक्ष्याथ सुदुःखितान्}
{पर्य्यष्वजत धर्मज्ञः पितृवन्मातृवच्च सः} %2-102-47

\twolineshloka
{स तत्र काञ्च्चित् परिषस्वजे नरान्नराश्च केचित्तु तमभ्यवादयन्}
{चकार सर्वान् सवयस्य बान्धवान् यथार्हमासाद्य तदा नृपात्मजः} %2-102-48

\twolineshloka
{स तत्र तेषां रुदतां महात्मनां भुवं च खं चानुनिनादयन् स्वनः}
{गुहा गिरीणां च दिशश्च सन्ततं मृदङ्गघोषप्रतिमः प्रशुश्रुवे} %2-102-49


॥इत्यार्षे श्रीमद्रामायणे वाल्मीकीये आदिकाव्ये अयोध्याकाण्डे निवापदानम् नाम द्व्यधिकशततमः सर्गः ॥२-१०२॥
