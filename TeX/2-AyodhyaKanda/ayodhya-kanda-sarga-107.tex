\sect{सप्ताधिकशततमः सर्गः — रामप्रतिवचनम्}

\twolineshloka
{पुनरेवं ब्रुवाणं तं भरतं लक्ष्मणाग्रजः}
{प्रत्युवाच ततः श्रीमान् ज्ञातिमध्येऽभिसत्कृतः} %2-107-1

\twolineshloka
{उपपन्नमिदं वाक्यं यत्त्वमेवमभाषथाः}
{जातः पुत्रो दशरथात् कैकेय्यां राजसत्तमात्} %2-107-2

\twolineshloka
{पुरा भ्रातः पिता नः स मातरं ते समुद्वहन्}
{मातामहे समाश्रौषीद्राज्यशुल्कमनुत्तमम्} %2-107-3

\twolineshloka
{दैवासुरे च सङ्ग्रामे जनन्यै तव पार्थिवः}
{सम्प्रहृष्टो ददौ राजा वरमाराधितः प्रभुः} %2-107-4

\twolineshloka
{ततः सा सम्प्रतिश्राव्य तव माता यशस्विनी}
{अयाचत नरश्रेष्ठं द्वौ वरौ वरवर्णिनी} %2-107-5

\twolineshloka
{तव राज्यं नरव्याघ्र मम प्रव्राजनं तथा}
{तौ च राजा तदा तस्यै नियुक्तः प्रददौ वरौ} %2-107-6

\twolineshloka
{तेन पित्राऽहमप्यत्र नियुक्तः पुरुषर्षभ}
{चतुर्दश वने वासं वर्षाणि वरदानिकम्} %2-107-7

\twolineshloka
{सोऽहं वनमिदं प्राप्तो निर्जनं लक्ष्मणान्वितः}
{सीतया चाप्रतिद्वन्द्वः सत्यवादे स्थितः पितुः} %2-107-8

\twolineshloka
{भवानपि तथेत्येव पितरं सत्यवादिनम्}
{कर्त्तुमर्हति राजेन्द्र क्षिप्रमेवाभिषेचनात्} %2-107-9

\twolineshloka
{ऋणान्मोचय राजानं मत्कृते भरत प्रभुम्}
{पितरं चापि धर्मज्ञं मातरं चाभिनन्दय} %2-107-10

\twolineshloka
{श्रूयते हि पुरा तात श्रुतिर्गीता यशस्विना}
{गयेन यजमानेन गयेष्वेव पितऽन् प्रति} %2-107-11

\twolineshloka
{पुन्नाम्नो नरकाद्यस्मात् पितरं त्रायते सुतः}
{तस्मात् पुत्र इति प्रोक्तः पितऽन् यत्पाति वा सुतः} %2-107-12

\twolineshloka
{एष्टव्या बहवः पुत्रा गुणवन्तो बहुश्रुताः}
{तेषां वै समवेतानामपि कश्चिद्गयां व्रजेत्} %2-107-13

\twolineshloka
{एवं राजर्षयः सर्वे प्रतीता राजनन्दन}
{तस्मात् त्राहि नरश्रेष्ठ पितरं नरकात् प्रभो} %2-107-14

\twolineshloka
{अयोध्यां गच्छ भरत प्रकृतीरनुरञ्जय}
{शत्रुघ्नसहितो वीर सह सर्वैर्द्विजातिभिः} %2-107-15

\twolineshloka
{प्रवेक्ष्ये दण्डकारण्यमहमप्यविलम्बयन्}
{आभ्यां तु सहितो राजन् वैदेह्या लक्ष्मणेन च} %2-107-16

\twolineshloka
{त्वं राजा भरत भव स्वयं नराणां वन्यानामहमपि राजराण्मृगाणाम्}
{गच्छत्वं पुरवरमद्य सम्प्रहृष्टः संहृष्टस्त्वहमपि दण्डकान् प्रवेक्ष्ये} %2-107-17

\twolineshloka
{छायां ते दिनकरभाः प्रबाधमानं वर्षत्रं भरत करोतु मूर्ध्नि शीताम्}
{एतेषामहमपि काननद्रुमाणां छायां तामति शयिनीं सुखी श्रयिष्ये} %2-107-18

\twolineshloka
{शत्रुघ्नः कुशलमतिस्तु ते सहायः सौमित्रिर्मम विदितः प्रधानमित्रम्}
{चत्वारस्तनयवरा वयं नरेन्द्रं सत्यस्थं भरत चराम मा विषादम्} %2-107-19


॥इत्यार्षे श्रीमद्रामायणे वाल्मीकीये आदिकाव्ये अयोध्याकाण्डे रामप्रतिवचनम् नाम सप्ताधिकशततमः सर्गः ॥२-१०७॥
