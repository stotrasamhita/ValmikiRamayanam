\sect{सप्तचत्वारिंशः सर्गः — पौरनिवृत्तिः}

\twolineshloka
{प्रभातायां तु शर्वर्यां पौरास्ते राघवं विना}
{शोकोपहतनिश्चेष्टा बभूवुर्हतचेतसः} %2-47-1

\twolineshloka
{शोकजाश्रुपरिद्यूना वीक्षमाणास्ततस्ततः}
{आलोकमपि रामस्य न पश्यन्ति स्म दुःखिताः} %2-47-2

\twolineshloka
{ते विषादार्तवदना रहितास्तेन धीमता}
{कृपणाः करुणा वाचो वदन्ति स्म मनीषिणः} %2-47-3

\twolineshloka
{धिगस्तु खलु निद्रां तां ययापहतचेतसः}
{नाद्य पश्यामहे रामं पृथूरस्कं महाभुजम्} %2-47-4

\twolineshloka
{कथं रामो महाबाहुः स तथावितथक्रियः}
{भक्तं जनमभित्यज्य प्रवासं तापसो गतः} %2-47-5

\twolineshloka
{यो नः सदा पालयति पिता पुत्रानिवौरसान्}
{कथं रघूणां स श्रेष्ठस्त्यक्त्वा नो विपिनं गतः} %2-47-6

\twolineshloka
{इहैव निधनं याम महाप्रस्थानमेव वा}
{रामेण रहितानां नो किमर्थं जीवितं हितम्} %2-47-7

\twolineshloka
{सन्ति शुष्काणि काष्ठानि प्रभूतानि महान्ति च}
{तैः प्रज्वाल्य चितां सर्वे प्रविशामोऽथवा वयम्} %2-47-8

\twolineshloka
{किं वक्ष्यामो महाबाहुरनसूयः प्रियंवदः}
{नीतः स राघवोऽस्माभिरिति वक्तुं कथं क्षमम्} %2-47-9

\twolineshloka
{सा नूनं नगरी दीना दृष्ट्वास्मान् राघवं विना}
{भविष्यति निरानन्दा सस्त्रीबालवयोऽधिका} %2-47-10

\twolineshloka
{निर्यातास्तेन वीरेण सह नित्यं महात्मना}
{विहीनास्तेन च पुनः कथं द्रक्ष्याम तां पुरीम्} %2-47-11

\twolineshloka
{इतीव बहुधा वाचो बाहुमुद्यम्य ते जनाः}
{विलपन्ति स्म दुःखार्ता हृतवत्सा इवाग्र्यगाः} %2-47-12

\twolineshloka
{ततो मार्गानुसारेण गत्वा किञ्चित् ततः क्षणम्}
{मार्गनाशाद् विषादेन महता समभिप्लुताः} %2-47-13

\twolineshloka
{रथमार्गानुसारेण न्यवर्तन्त मनस्विनः}
{किमिदं किं करिष्यामो दैवेनोपहता इति} %2-47-14

\twolineshloka
{तदा यथागतेनैव मार्गेण क्लान्तचेतसः}
{अयोध्यामगमन् सर्वे पुरीं व्यथितसज्जनाम्} %2-47-15

\twolineshloka
{आलोक्य नगरीं तां च क्षयव्याकुलमानसाः}
{आवर्तयन्त तेऽश्रूणि नयनैः शोकपीडितैः} %2-47-16

\twolineshloka
{एषा रामेण नगरी रहिता नातिशोभते}
{आपगा गरुडेनेव ह्रदादुद्धृतपन्नगा} %2-47-17

\twolineshloka
{चन्द्रहीनमिवाकाशं तोयहीनमिवार्णवम्}
{अपश्यन् निहतानन्दं नगरं ते विचेतसः} %2-47-18

\twolineshloka
{ते तानि वेश्मानि महाधनानि दुःखेन दुःखोपहता विशन्तः}
{नैव प्रजग्मुः स्वजनं परं वा निरीक्ष्यमाणाः प्रविनष्टहर्षाः} %2-47-19


॥इत्यार्षे श्रीमद्रामायणे वाल्मीकीये आदिकाव्ये अयोध्याकाण्डे पौरनिवृत्तिः नाम सप्तचत्वारिंशः सर्गः ॥२-४७॥
