\sect{तृतीयः सर्गः — पुत्रानुशासनम्}

\twolineshloka
{तेषामञ्जलिपद्मानि प्रगृहीतानि सर्वशः}
{प्रतिगृह्याब्रवीद् राजा तेभ्यः प्रियहितं वचः} %2-3-1

\twolineshloka
{अहोऽस्मि परमप्रीतः प्रभावश्चातुलो मम}
{यन्मे ज्येष्ठं प्रियं पुत्रं यौवराज्यस्थमिच्छथ} %2-3-2

\twolineshloka
{इति प्रत्यर्चितान् राजा ब्राह्मणानिदमब्रवीत्}
{वसिष्ठं वामदेवं च तेषामेवोपशृण्वताम्} %2-3-3

\twolineshloka
{चैत्रः श्रीमानयं मासः पुण्यः पुष्पितकाननः}
{यौवराज्याय रामस्य सर्वमेवोपकल्प्यताम्} %2-3-4

\twolineshloka
{राज्ञस्तूपरते वाक्ये जनघोषो महानभूत्}
{शनैस्तस्मिन् प्रशान्ते च जनघोषे जनाधिपः} %2-3-5

\twolineshloka
{वसिष्ठं मुनिशार्दूलं राजा वचनमब्रवीत्}
{अभिषेकाय रामस्य यत् कर्म सपरिच्छदम्} %2-3-6

\twolineshloka
{तदद्य भगवन् सर्वमाज्ञापयितुमर्हसि}
{तच्छ्रुत्वा भूमिपालस्य वसिष्ठो मुनिसत्तमः} %2-3-7

\twolineshloka
{आदिदेशाग्रतो राज्ञः स्थितान् युक्तान् कृताञ्जलीन्}
{सुवर्णादीनि रत्नानि बलीन् सर्वौषधीरपि} %2-3-8

\twolineshloka
{शुक्लमाल्यानि लाजांश्च पृथक् च मधुसर्पिषी}
{अहतानि च वासांसि रथं सर्वायुधान्यपि} %2-3-9

\twolineshloka
{चतुरङ्गबलं चैव गजं च शुभलक्षणम्}
{चामरव्यजने चोभे ध्वजं छत्रं च पाण्डुरम्} %2-3-10

\twolineshloka
{शतं च शातकुम्भानां कुम्भानामग्निवर्चसाम्}
{हिरण्यशृङ्गमृषभं समग्रं व्याघ्रचर्म च} %2-3-11

\twolineshloka
{यच्चान्यत् किञ्चिदेष्टव्यं तत् सर्वमुपकल्प्यताम्}
{उपस्थापयत प्रातरग्न्यगारे महीपतेः} %2-3-12

\twolineshloka
{अन्तःपुरस्य द्वाराणि सर्वस्य नगरस्य च}
{चन्दनस्रग्भिरर्च्यन्तां धूपैश्च घ्राणहारिभिः} %2-3-13

\twolineshloka
{प्रशस्तमन्नं गुणवद् दधिक्षीरोपसेचनम्}
{द्विजानां शतसाहस्रं यत्प्रकाममलं भवेत्} %2-3-14

\twolineshloka
{सत्कृत्य द्विजमुख्यानां श्वः प्रभाते प्रदीयताम्}
{घृतं दधि च लाजाश्च दक्षिणाश्चापि पुष्कलाः} %2-3-15

\twolineshloka
{सूर्येऽभ्युदितमात्रे श्वो भविता स्वस्तिवाचनम्}
{ब्राह्मणाश्च निमन्त्र्यन्तां कल्प्यन्तामासनानि च} %2-3-16

\twolineshloka
{आबध्यन्तां पताकाश्च राजमार्गश्च सिच्यताम्}
{सर्वे च तालापचरा गणिकाश्च स्वलङ्कृताः} %2-3-17

\twolineshloka
{कक्ष्यां द्वितीयामासाद्य तिष्ठन्तु नृपवेश्मनः}
{देवायतनचैत्येषु सान्नभक्ष्याः सदक्षिणाः} %2-3-18

\twolineshloka
{उपस्थापयितव्याः स्युर्माल्ययोग्याः पृथक्पृथक्}
{दीर्घासिबद्धगोधाश्च सन्नद्धा मृष्टवाससः} %2-3-19

\twolineshloka
{महाराजाङ्गनं शूराः प्रविशन्तु महोदयम्}
{एवं व्यादिश्य विप्रौ तु क्रियास्तत्र विनिष्ठितौ} %2-3-20

\twolineshloka
{चक्रतुश्चैव यच्छेषं पार्थिवाय निवेद्य च}
{कृतमित्येव चाब्रूतामभिगम्य जगत्पतिम्} %2-3-21

\twolineshloka
{यथोक्तवचनं प्रीतौ हर्षयुक्तौ द्विजोत्तमौ}
{ततः सुमन्त्रं द्युतिमान् राजा वचनमब्रवीत्} %2-3-22

\twolineshloka
{रामः कृतात्मा भवता शीघ्रमानीयतामिति}
{स तथेति प्रतिज्ञाय सुमन्त्रो राजशासनात्} %2-3-23

\twolineshloka
{रामं तत्रानयाञ्चक्रे रथेन रथिनां वरम्}
{अथ तत्र सहासीनास्तदा दशरथं नृपम्} %2-3-24

\twolineshloka
{प्राच्योदीच्या प्रतीच्याश्च दाक्षिणात्याश्च भूमिपाः}
{म्लेच्छाश्चार्याश्च ये चान्ये वनशैलान्तवासिनः} %2-3-25

\twolineshloka
{उपासाञ्चक्रिरे सर्वे तं देवा वासवं यथा}
{तेषां मध्ये स राजर्षिर्मरुतामिव वासवः} %2-3-26

\twolineshloka
{प्रासादस्थो दशरथो ददर्शायान्तमात्मजम्}
{गन्धर्वराजप्रतिमं लोके विख्यातपौरुषम्} %2-3-27

\twolineshloka
{दीर्घबाहुं महासत्त्वं मत्तमातङ्गगामिनम्}
{चन्द्रकान्ताननं राममतीव प्रियदर्शनम्} %2-3-28

\twolineshloka
{रूपौदार्यगुणैः पुंसां दृष्टिचित्तापहारिणम्}
{घर्माभितप्ताः पर्जन्यं ह्लादयन्तमिव प्रजाः} %2-3-29

\twolineshloka
{न ततर्प समायान्तं पश्यमानो नराधिपः}
{अवतार्य सुमन्त्रस्तु राघवं स्यन्दनोत्तमात्} %2-3-30

\twolineshloka
{पितुः समीपं गच्छन्तं प्राञ्जलिः पृष्ठतोऽन्वगात्}
{स तं कैलासशृङ्गाभं प्रासादं रघुनन्दनः} %2-3-31

\twolineshloka
{आरुरोह नृपं द्रष्टुं सहसा तेन राघवः}
{स प्राञ्जलिरभिप्रेत्य प्रणतः पितुरन्तिके} %2-3-32

\twolineshloka
{नाम स्वं श्रावयन् रामो ववन्दे चरणौ पितुः}
{तं दृष्ट्वा प्रणतं पार्श्वे कृताञ्जलिपुटं नृपः} %2-3-33

\twolineshloka
{गृह्याञ्जलौ समाकृष्य सस्वजे प्रियमात्मजम्}
{तस्मै चाभ्युद्यतं सम्यङ्मणिकाञ्चनभूषितम्} %2-3-34

\twolineshloka
{दिदेश राजा रुचिरं रामाय परमासनम्}
{तथाऽऽसनवरं प्राप्य व्यदीपयत राघवः} %2-3-35

\twolineshloka
{स्वयैव प्रभया मेरुमुदये विमलो रविः}
{तेन विभ्राजिता तत्र सा सभापि व्यरोचत} %2-3-36

\twolineshloka
{विमलग्रहनक्षत्रा शारदी द्यौरिवेन्दुना}
{तं पश्यमानो नृपतिस्तुतोष प्रियमात्मजम्} %2-3-37

\twolineshloka
{अलङ्कृतमिवात्मानमादर्शतलसंस्थितम्}
{स तं सुस्थितमाभाष्य पुत्रं पुत्रवतां वरः} %2-3-38

\twolineshloka
{उवाचेदं वचो राजा देवेन्द्रमिव कश्यपः}
{ज्येष्ठायामसि मे पत्न्यां सदृश्यां सदृशः सुतः} %2-3-39

\twolineshloka
{उत्पन्नस्त्वं गुणज्येष्ठो मम रामात्मजः प्रियः}
{त्वया यतः प्रजाश्चेमाः स्वगुणैरनुरञ्जिताः} %2-3-40

\twolineshloka
{तस्मात् त्वं पुष्ययोगेन यौवराज्यमवाप्नुहि}
{कामतस्त्वं प्रकृत्यैव निर्णीतो गुणवानिति} %2-3-41

\twolineshloka
{गुणवत्यपि तु स्नेहात् पुत्र वक्ष्यामि ते हितम्}
{भूयो विनयमास्थाय भव नित्यं जितेन्द्रियः} %2-3-42

\twolineshloka
{कामक्रोधसमुत्थानि त्यजस्व व्यसनानि च}
{परोक्षया वर्तमानो वृत्त्या प्रत्यक्षया तथा} %2-3-43

\twolineshloka
{अमात्यप्रभृतीः सर्वाः प्रजाश्चैवानुरञ्जय}
{कोष्ठागारायुधागारैः कृत्वा सन्निचयान् बहून्} %2-3-44

\twolineshloka
{इष्टानुरक्तप्रकृतिर्यः पालयति मेदिनीम्}
{तस्य नन्दन्ति मित्राणि लब्ध्वामृतमिवामराः} %2-3-45

\twolineshloka
{तस्मात् पुत्र त्वमात्मानं नियम्यैवं समाचर}
{तच्छ्रुत्वा सुहृदस्तस्य रामस्य प्रियकारिणः} %2-3-46

\twolineshloka
{त्वरिताः शीघ्रमागत्य कौसल्यायै न्यवेदयन्}
{सा हिरण्यं च गाश्चैव रत्नानि विविधानि च} %2-3-47

\threelineshloka
{व्यादिदेश प्रियाख्येभ्यः कौसल्या प्रमदोत्तमा}
{अथाभिवाद्य राजानं रथमारुह्य राघवः}
{ययौ स्वं द्युतिमद् वेश्म जनौघैः प्रतिपूजितः} %2-3-48

\twolineshloka
{ते चापि पौरा नृपतेर्वचस्तच्छ्रुत्वा तदा लाभमिवेष्टमाशु}
{नरेन्द्रमामन्त्र्य गृहाणि गत्वा देवान् समानर्चुरभिप्रहृष्टाः} %2-3-49


॥इत्यार्षे श्रीमद्रामायणे वाल्मीकीये आदिकाव्ये अयोध्याकाण्डे पुत्रानुशासनम् नाम तृतीयः सर्गः ॥२-३॥
