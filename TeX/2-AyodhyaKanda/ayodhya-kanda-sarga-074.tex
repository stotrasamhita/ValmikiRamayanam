\sect{चतुःसप्ततितमः सर्गः — कैकेय्याक्रोशः}

\twolineshloka
{ताम् तथा गर्हयित्वा तु मातरम् भरतः तदा}
{रोषेण महता आविष्टः पुनर् एव अब्रवीद् वचः} %2-74-1

\twolineshloka
{राज्यात् भ्रम्शस्व कैकेयि नृशम्से दुष्ट चारिणि}
{परित्यक्ता च धर्मेण मा मृतम् रुदती भव} %2-74-2

\twolineshloka
{किम् नु ते अदूषयद् राजा रामः वा भृश धार्मिकः}
{ययोः मृत्युर् विवासः च त्वत् कृते तुल्यम् आगतौ} %2-74-3

\twolineshloka
{भ्रूणहत्याम् असि प्राप्ता कुलस्य अस्य विनाशनात्}
{कैकेयि नरकम् गच्च मा च भर्तुः सलोकताम्} %2-74-4

\twolineshloka
{यत्त्वया हीदृशम् पापम् कृतम् घोरेण कर्मणा}
{सर्वलोकप्रियम् हित्वा ममाप्यापादितम् भयम्} %2-74-5

\twolineshloka
{त्वत् कृते मे पिता वृत्तः रामः च अरण्यम् आश्रितः}
{अयशो जीव लोके च त्वया अहम् प्रतिपादितः} %2-74-6

\twolineshloka
{मातृ रूपे मम अमित्रे नृशम्से राज्य कामुके}
{न ते अहम् अभिभाष्यो अस्मि दुर्वृत्ते पति घातिनि} %2-74-7

\twolineshloka
{कौसल्या च सुमित्रा च याः च अन्या मम मातरः}
{दुह्खेन महता आविष्टाः त्वाम् प्राप्य कुल दूषिणीम्} %2-74-8

\twolineshloka
{न त्वम् अश्व पतेः कन्या धर्म राजस्य धीमतः}
{राक्षसी तत्र जाता असि कुल प्रध्वम्सिनी पितुः} %2-74-9

\twolineshloka
{यत् त्वया धार्मिको रामः नित्यम् सत्य परायणः}
{वनम् प्रस्थापितः दुह्खात् पिता च त्रिदिवम् गतः} %2-74-10

\twolineshloka
{यत् प्रधाना असि तत् पापम् मयि पित्रा विना कृते}
{भ्रातृभ्याम् च परित्यक्ते सर्व लोकस्य च अप्रिये} %2-74-11

\twolineshloka
{कौसल्याम् धर्म सम्युक्ताम् वियुक्ताम् पाप निश्चये}
{कृत्वा कम् प्राप्स्यसे तु अद्य लोकम् निरय गामिनी} %2-74-12

\twolineshloka
{किम् न अवबुध्यसे क्रूरे नियतम् बन्धु सम्श्रयम्}
{ज्येष्ठम् पितृ समम् रामम् कौसल्याय आत्म सम्भवम्} %2-74-13

\twolineshloka
{अन्ग प्रत्यन्गजः पुत्रः हृदयाच् च अपि जायते}
{तस्मात् प्रियतरः मातुः प्रियत्वान् न तु बान्धवः} %2-74-14

\twolineshloka
{अन्यदा किल धर्मज्ञा सुरभिः सुर सम्मता}
{वहमानौ ददर्श उर्व्याम् पुत्रौ विगत चेतसौ} %2-74-15

\twolineshloka
{ताव् अर्ध दिवसे श्रान्तौ दृष्ट्वा पुत्रौ मही तले}
{रुरोद पुत्र शोकेन बाष्प पर्याकुल ईक्षणा} %2-74-16

\twolineshloka
{अधस्तात् व्रजतः तस्याः सुर राज्ञो महात्मनः}
{बिन्दवः पतिता गात्रे सूक्ष्माः सुरभि गन्धिनः} %2-74-17

\twolineshloka
{इन्द्रोऽप्यश्रुनिपातम् तम् स्वगात्रे पुण्यगन्धिनम्}
{सुरभिम् मन्यते दृष्ट्वा भूयसीम् ताम् सुरेश्वरः} %2-74-18

\twolineshloka
{निरीक्समाणः शक्रस्ताम् ददर्श सुरभिम् स्थिताम्}
{आकाशे विष्ठिताम् दीनाम् रुदतीम् भृशदुःखिताम्} %2-74-19

\twolineshloka
{ताम् दृष्ट्वा शोक सम्तप्ताम् वज्र पाणिर् यशस्विनीम्}
{इन्द्रः प्रान्जलिर् उद्विग्नः सुर राजो अब्रवीद् वचः} %2-74-20

\twolineshloka
{भयम् कच्चिन् न च अस्मासु कुतश्चित् विद्यते महत्}
{कुतः निमित्तः शोकः ते ब्रूहि सर्व हित एषिणि} %2-74-21

\twolineshloka
{एवम् उक्ता तु सुरभिः सुर राजेन धीमता}
{पत्युवाच ततः धीरा वाक्यम् वाक्य विशारदा} %2-74-22

\twolineshloka
{शान्तम् पातम् न वः किम्चित् कुतश्चित् अमर अधिप}
{अहम् तु मग्नौ शोचामि स्व पुत्रौ विषमे स्थितौ} %2-74-23

\twolineshloka
{एतौ दृष्ट्वा कृषौ दीनौ सूर्य रश्मि प्रतापिनौ}
{अर्ध्यमानौ बली वर्दौ कर्षकेण सुर अधिप} %2-74-24

\twolineshloka
{मम कायात् प्रसूतौ हि दुह्खितौ भार पीडितौ}
{यौ दृष्ट्वा परितप्ये अहम् न अस्ति पुत्र समः प्रियः} %2-74-25

\twolineshloka
{यस्याः पुत्र सहस्त्रैस्तु कृत्स्नम् व्याप्तमिदम् जगत्}
{ताम् दृष्ट्वा रुदतीम् शक्रो न सुतान्मन्यते परम्} %2-74-26

\twolineshloka
{सदाऽप्रतिमवृत्ताया लोकधारणकाम्यया}
{श्रीमत्या गुणनित्यायाः स्वभावपरिचेष्टया} %2-74-27

\twolineshloka
{यस्याः पुत्रसहस्राणि सापि शोचै कामधुक्}
{किम् पुनर् या विना रामम् कौसल्या वर्तयिष्यति} %2-74-28

\twolineshloka
{एक पुत्रा च साध्वी च विवत्सा इयम् त्वया कृता}
{तस्मात् त्वम् सततम् दुह्खम् प्रेत्य च इह च लप्स्यसे} %2-74-29

\twolineshloka
{अहम् हि अपचितिम् भ्रातुः पितुः च सकलाम् इमाम्}
{वर्धनम् यशसः च अपि करिष्यामि न सम्शयः} %2-74-30

\twolineshloka
{आनाययित्वा तनयम् कौसल्याया महा द्युतिम्}
{स्वयम् एव प्रवेक्ष्यामि वनम् मुनि निषेवितम्} %2-74-31

\twolineshloka
{न ह्यहम् पापसम्कल्पे पापे पापम् त्वया कृतम्}
{शक्तो धारयितुम् पौरैरश्रुकण्ठै र्निरीक्षितः} %2-74-32

\twolineshloka
{सा त्वमग्निम् प्रविश वा स्वयम् वा दण्डकान्विश}
{रज्जुम् बधान वा कण्ठे न हि तेऽन्यत्परायणम्} %2-74-33

\twolineshloka
{अहमप्यवनिम् प्राप्ते रामे सत्यपराक्रमे}
{कृतकृत्यो भविष्यामि विप्रवासितकल्मषः} %2-74-34

\twolineshloka
{इति नागैव अरण्ये तोमर अन्कुश चोदितः}
{पपात भुवि सम्क्रुद्धो निह्श्वसन्न् इव पन्नगः} %2-74-35

\fourlineindentedshloka
{सम्रक्त नेत्रः शिथिल अम्बरः तदा}
{विधूत सर्व आभरणः परम्तपः}
{बभूव भूमौ पतितः नृप आत्मजः}
{शची पतेः केतुर् इव उत्सव क्षये} %2-74-36


॥इत्यार्षे श्रीमद्रामायणे वाल्मीकीये आदिकाव्ये अयोध्याकाण्डे कैकेय्याक्रोशः नाम चतुःसप्ततितमः सर्गः ॥२-७४॥
