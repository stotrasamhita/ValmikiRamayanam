\sect{सप्तषष्ठितमः सर्गः — लङ्घनावष्टम्भः}

\twolineshloka
{तं दृष्ट्वा जृम्भमाणं ते क्रमितुं शतयोजनम्}
{वेगेनापूर्यमाणं च सहसा वानरोत्तमम्} %4-67-1

\twolineshloka
{सहसा शोकमुत्सृज्य प्रहर्षेण समन्विताः}
{विनेदुस्तुष्टुवुश्चापि हनूमन्तं महाबलम्} %4-67-2

\twolineshloka
{प्रहृष्टा विस्मिताश्चापि ते वीक्षन्ते समन्ततः}
{त्रिविक्रमं कृतोत्साहं नारायणमिव प्रजाः} %4-67-3

\twolineshloka
{संस्तूयमानो हनुमान् व्यवर्धत महाबलः}
{समाविद्ध्य च लाङ्गूलं हर्षाद् बलमुपेयिवान्} %4-67-4

\twolineshloka
{तस्य संस्तूयमानस्य वृद्धैर्वानरपुङ्गवैः}
{तेजसाऽऽपूर्यमाणस्य रूपमासीदनुत्तमम्} %4-67-5

\twolineshloka
{यथा विजृम्भते सिंहो विवृते गिरिगह्वरे}
{मारुतस्यौरसः पुत्रस्तथा सम्प्रति जृम्भते} %4-67-6

\twolineshloka
{अशोभत मुखं तस्य जृम्भमाणस्य धीमतः}
{अम्बरीषोपमं दीप्तं विधूम इव पावकः} %4-67-7

\twolineshloka
{हरीणामुत्थितो मध्यात् सम्प्रहृष्टतनूरुहः}
{अभिवाद्य हरीन् वृद्धान् हनूमानिदमब्रवीत्} %4-67-8

\twolineshloka
{आरुजन् पर्वताग्राणि हुताशनसखोऽनिलः}
{बलवानप्रमेयश्च वायुराकाशगोचरः} %4-67-9

\twolineshloka
{तस्याहं शीघ्रवेगस्य शीघ्रगस्य महात्मनः}
{मारुतस्यौरसः पुत्रः प्लवनेनास्मि तत्समः} %4-67-10

\twolineshloka
{उत्सहेयं हि विस्तीर्णमालिखन्तमिवाम्बरम्}
{मेरुं गिरिमसङ्गेन परिगन्तुं सहस्रशः} %4-67-11

\twolineshloka
{बाहुवेगप्रणुन्नेन सागरेणाहमुत्सहे}
{समाप्लावयितुं लोकं सपर्वतनदीह्रदम्} %4-67-12

\twolineshloka
{ममोरुजङ्घावेगेन भविष्यति समुत्थितः}
{समुत्थितमहाग्राहः समुद्रो वरुणालयः} %4-67-13

\twolineshloka
{पन्नगाशनमाकाशे पतन्तं पक्षिसेवितम्}
{वैनतेयमहं शक्तः परिगन्तुं सहस्रशः} %4-67-14

\twolineshloka
{उदयात् प्रस्थितं वापि ज्वलन्तं रश्मिमालिनम्}
{अनस्तमितमादित्यमहं गन्तुं समुत्सहे} %4-67-15

\twolineshloka
{ततो भूमिमसंस्पृष्ट्वा पुनरागन्तुमुत्सहे}
{प्रवेगेनैव महता भीमेन प्लवगर्षभाः} %4-67-16

\twolineshloka
{उत्सहेयमतिक्रान्तुं सर्वानाकाशगोचरान्}
{सागरान् शोषयिष्यामि दारयिष्यामि मेदिनीम्} %4-67-17

\twolineshloka
{पर्वतांश्चूर्णयिष्यामि प्लवमानः प्लवङ्गमः}
{हरिष्याम्युरुवेगेन प्लवमानो महार्णवम्} %4-67-18

\twolineshloka
{लतानां विविधं पुष्पं पादपानां च सर्वशः}
{अनुयास्यति मामद्य प्लवमानं विहायसा} %4-67-19

\twolineshloka
{भविष्यति हि मे पन्थाः स्वातेः पन्था इवाम्बरे}
{चरन्तं घोरमाकाशमुत्पतिष्यन्तमेव च} %4-67-20

\twolineshloka
{द्रक्ष्यन्ति निपतन्तं च सर्वभूतानि वानराः}
{महामेरुप्रतीकाशं मां द्रक्ष्यध्वं प्लवङ्गमाः} %4-67-21

\threelineshloka
{दिवमावृत्य गच्छन्तं ग्रसमानमिवाम्बरम्}
{विधमिष्यामि जीमूतान् कम्पयिष्यामि पर्वतान्}
{सागरं शोषयिष्यामि प्लवमानः समाहितः} %4-67-22

\threelineshloka
{वैनतेयस्य वा शक्तिर्मम वा मारुतस्य वा}
{ऋते सुपर्णराजानं मारुतं वा महाबलम्}
{न तद् भूतं प्रपश्यामि यन्मां प्लुतमनुव्रजेत्} %4-67-23

\twolineshloka
{निमेषान्तरमात्रेण निरालम्बनमम्बरम्}
{सहसा निपतिष्यामि घनाद् विद्युदिवोत्थिता} %4-67-24

\twolineshloka
{भविष्यति हि मे रूपं प्लवमानस्य सागरम्}
{विष्णोः प्रक्रममाणस्य तदा त्रीन् विक्रमानिव} %4-67-25

\twolineshloka
{बुद्ध्या चाहं प्रपश्यामि मनश्चेष्टा च मे तथा}
{अहं द्रक्ष्यामि वैदेहीं प्रमोदध्वं प्लवङ्गमाः} %4-67-26

\twolineshloka
{मारुतस्य समो वेगे गरुडस्य समो जवे}
{अयुतं योजनानां तु गमिष्यामीति मे मतिः} %4-67-27

\twolineshloka
{वासवस्य सवज्रस्य ब्रह्मणो वा स्वयम्भुवः}
{विक्रम्य सहसा हस्तादमृतं तदिहानये} %4-67-28

\twolineshloka
{लङ्कां वापि समुत्क्षिप्य गच्छेयमिति मे मतिः}
{तमेवं वानरश्रेष्ठं गर्जन्तममितप्रभम्} %4-67-29

\twolineshloka
{प्रहृष्टा हरयस्तत्र समुदैक्षन्त विस्मिताः}
{तच्चास्य वचनं श्रुत्वा ज्ञातीनां शोकनाशनम्} %4-67-30

\twolineshloka
{उवाच परिसंहृष्टो जाम्बवान् प्लवगेश्वरः}
{वीर केसरिणः पुत्र वेगवन् मारुतात्मज} %4-67-31

\twolineshloka
{ज्ञातीनां विपुलः शोकस्त्वया तात प्रणाशितः}
{तव कल्याणरुचयः कपिमुख्याः समागताः} %4-67-32

\twolineshloka
{मङ्गलान्यर्थसिद्ध्यर्थं करिष्यन्ति समाहिताः}
{ऋषीणां च प्रसादेन कपिवृद्धमतेन च} %4-67-33

\twolineshloka
{गुरूणां च प्रसादेन सम्प्लव त्वं महार्णवम्}
{स्थास्यामश्चैकपादेन यावदागमनं तव} %4-67-34

\twolineshloka
{त्वद्गतानि च सर्वेषां जीवनानि वनौकसाम्}
{ततश्च हरिशार्दूलस्तानुवाच वनौकसः} %4-67-35

\twolineshloka
{कोऽपि लोके न मे वेगं प्लवने धारयिष्यति}
{एतानीह नगस्यास्य शिलासङ्कटशालिनः} %4-67-36

\twolineshloka
{शिखराणि महेन्द्रस्य स्थिराणि च महान्ति च}
{येषु वेगं गमिष्यामि महेन्द्रशिखरेष्वहम्} %4-67-37

\twolineshloka
{नानाद्रुमविकीर्णेषु धातुनिष्पन्दशोभिषु}
{एतानि मम वेगं हि शिखराणि महान्ति च} %4-67-38

\threelineshloka
{प्लवतो धारयिष्यन्ति योजनानामितः शतम्}
{ततस्तु मारुतप्रख्यः स हरिर्मारुतात्मजः}
{आरुरोह नगश्रेष्ठं महेन्द्रमरिमर्दनः} %4-67-39

\twolineshloka
{वृतं नानाविधैः पुष्पैर्मृगसेवितशाद्वलम्}
{लताकुसुमसम्बाधं नित्यपुष्पफलद्रुमम्} %4-67-40

\twolineshloka
{सिंहशार्दूलसहितं मत्तमातङ्गसेवितम्}
{मत्तद्विजगणोद्घुष्टं सलिलोत्पीडसङ्कुलम्} %4-67-41

\twolineshloka
{महद्भिरुच्छ्रितं शृङ्गैर्महेन्द्रं स महाबलः}
{विचचार हरिश्रेष्ठो महेन्द्रसमविक्रमः} %4-67-42

\twolineshloka
{पादाभ्यां पीडितस्तेन महाशैलो महात्मना}
{ररास सिंहाभिहतो महान् मत्त इव द्विपः} %4-67-43

\twolineshloka
{मुमोच सलिलोत्पीडान् विप्रकीर्णशिलोच्चयः}
{वित्रस्तमृगमातङ्गः प्रकम्पितमहाद्रुमः} %4-67-44

\twolineshloka
{नानागन्धर्वमिथुनैः पानसंसर्गकर्कशैः}
{उत्पतद्भिर्विहङ्गैश्च विद्याधरगणैरपि} %4-67-45

\twolineshloka
{त्यज्यमानमहासानुः सन्निलीनमहोरगः}
{शैलशृङ्गशिलोत्पातस्तदाभूत् स महागिरिः} %4-67-46

\twolineshloka
{निःश्वसद्भिस्तदा तैस्तु भुजगैरर्धनिःसृतैः}
{सपताक इवाभाति स तदा धरणीधरः} %4-67-47

\twolineshloka
{ऋषिभिस्त्राससम्भ्रान्तैस्त्यज्यमानः शिलोच्चयः}
{सीदन् महति कान्तारे सार्थहीन इवाध्वगः} %4-67-48

\twolineshloka
{स वेगवान् वेगसमाहितात्मा हरिप्रवीरः परवीरहन्ता}
{मनः समाधाय महानुभावो जगाम लङ्कां मनसा मनस्वी} %4-67-49


॥इत्यार्षे श्रीमद्रामायणे वाल्मीकीये आदिकाव्ये किष्किन्धाकाण्डे लङ्घनावष्टम्भः नाम सप्तषष्ठितमः सर्गः ॥४-६७॥
