\sect{चतुर्दशः सर्गः — सुग्रीवगर्जनम्}

\twolineshloka
{सर्वे ते त्वरितं गत्वा किष्किन्धां वालिनः पुरीम्}
{वृक्षैरात्मानमावृत्य व्यतिष्ठन् गहने वने} %4-14-1

\twolineshloka
{विसार्य सर्वतो दृष्टिं कानने काननप्रियः}
{सुग्रीवो विपुलग्रीवः क्रोधमाहारयद् भृशम्} %4-14-2

\twolineshloka
{ततस्तु निनदं घोरं कृत्वा युद्धाय चाह्वयत्}
{परिवारैः परिवृतो नादैर्भिन्दन्निवाम्बरम्} %4-14-3

\twolineshloka
{गर्जन्निव महामेघो वायुवेगपुरःसरः}
{अथ बालार्कसदृशो दृप्तसिंहगतिस्ततः} %4-14-4

\twolineshloka
{दृष्ट्वा रामं क्रियादक्षं सुग्रीवो वाक्यमब्रवीत्}
{हरिवागुरया व्याप्तां तप्तकाञ्चनतोरणाम्} %4-14-5

\twolineshloka
{प्राप्ताः स्म ध्वजयन्त्राढ्यां किष्किन्धां वालिनः पुरीम्}
{प्रतिज्ञा या कृता वीर त्वया वालिवधे पुरा} %4-14-6

\twolineshloka
{सफलां कुरु तां क्षिप्रं लतां काल इवागतः}
{एवमुक्तस्तु धर्मात्मा सुग्रीवेण स राघवः} %4-14-7

\twolineshloka
{तमेवोवाच वचनं सुग्रीवं शत्रुसूदनः}
{कृताभिज्ञानचिह्नस्त्वमनया गजसाह्वया} %4-14-8

\twolineshloka
{लक्ष्मणेन समुत्पाट्य एषा कण्ठे कृता तव}
{शोभसेऽप्यधिकं वीर लतया कण्ठसक्तया} %4-14-9

\twolineshloka
{विपरीत इवाकाशे सूर्यो नक्षत्रमालया}
{अद्य वालिसमुत्थं ते भयं वैरं च वानर} %4-14-10

\twolineshloka
{एकेनाहं प्रमोक्ष्यामि बाणमोक्षेण संयुगे}
{मम दर्शय सुग्रीव वैरिणं भ्रातृरूपिणम्} %4-14-11

\twolineshloka
{वाली विनिहतो यावद्वने पांसुषु चेष्टते}
{यदि दृष्टिपथं प्राप्तो जीवन् स विनिवर्तते} %4-14-12

\twolineshloka
{ततो दोषेण मागच्छेत् सद्यो गर्हेच्च मां भवान्}
{प्रत्यक्षं सप्त ते साला मया बाणेन दारिताः} %4-14-13

\twolineshloka
{तेनावेहि बलेनाद्य वालिनं निहतं रणे}
{अनृतं नोक्तपूर्वं मे चिरं कृच्छ्रेऽपि तिष्ठता} %4-14-14

\twolineshloka
{धर्मलोभपरीतेन न च वक्ष्ये कथंचन}
{सफलां च करिष्यामि प्रतिज्ञां जहि संभ्रमम्} %4-14-15

\twolineshloka
{प्रसूतं कलमक्षेत्रं वर्षेणेव शतक्रतुः}
{तदाह्वाननिमित्तं च वालिनो हेममालिनः} %4-14-16

\twolineshloka
{सुग्रीव कुरु तं शब्दं निष्पतेद् येन वानरः}
{जितकाशी जयश्लाघी त्वया चाधर्षितः पुरात्} %4-14-17

\twolineshloka
{निष्पतिष्यत्यसङ्गेन वाली स प्रियसंयुगः}
{रिपूणां धर्षितं श्रुत्वा मर्षयन्ति न संयुगे} %4-14-18

\twolineshloka
{जानन्तस्तु स्वकं वीर्यं स्त्रीसमक्षं विशेषतः}
{स तु रामवचः श्रुत्वा सुग्रीवो हेमपिङ्गलः} %4-14-19

\twolineshloka
{ननर्द क्रूरनादेन विनिर्भिन्दन्निवाम्बरम्}
{तत्र शब्देन वित्रस्ता गावो यान्ति हतप्रभाः} %4-14-20

\threelineshloka
{राजदोषपरामृष्टाः कुलस्त्रिय इवाकुलाः}
{द्रवन्ति च मृगाः शीघ्रं भग्ना इव रणे हयाः}
{पतन्ति च खगा भूमौ क्षीणपुण्या इव ग्रहाः} %4-14-21

\twolineshloka
{ततः स जीमूतकृतप्रणादो नादं ह्यमुञ्चत् त्वरया प्रतीतः}
{सूर्यात्मजः शौर्यविवृद्धतेजाः सरित्पतिर्वाऽनिलचञ्चलोर्मिः} %4-14-22


॥इत्यार्षे श्रीमद्रामायणे वाल्मीकीये आदिकाव्ये किष्किन्धाकाण्डे सुग्रीवगर्जनम् नाम चतुर्दशः सर्गः ॥४-१४॥
