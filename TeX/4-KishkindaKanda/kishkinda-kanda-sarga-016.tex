\sect{षोडशः सर्गः — वालिसंहारः}

\twolineshloka
{तामेवं ब्रुवतीं तारां ताराधिपनिभाननाम्}
{वाली निर्भर्त्सयामास वचनं चेदमब्रवीत्} %4-16-1

\twolineshloka
{गर्जतोऽस्य सुसंरब्धं भ्रातुः शत्रोर्विशेषतः}
{मर्षयिष्यामि केनापि कारणेन वरानने} %4-16-2

\twolineshloka
{अधर्षितानां शूराणां समरेष्वनिवर्तिनाम्}
{धर्षणामर्षणं भीरु मरणादतिरिच्यते} %4-16-3

\twolineshloka
{सोढुं न च समर्थोऽहं युद्धकामस्य संयुगे}
{सुग्रीवस्य च संरम्भं हीनग्रीवस्य गर्जितम्} %4-16-4

\twolineshloka
{न च कार्यो विषादस्ते राघवं प्रति मत्कृते}
{धर्मज्ञश्च कृतज्ञश्च कथं पापं करिष्यति} %4-16-5

\twolineshloka
{निवर्तस्व सह स्त्रीभिः कथं भूयोऽनुगच्छसि}
{सौहृदं दर्शितं तावन्मयि भक्तिस्त्वया कृता} %4-16-6

\twolineshloka
{प्रतियोत्स्याम्यहं गत्वा सुग्रीवं जहि सम्भ्रमम्}
{दर्पं चास्य विनेष्यामि न च प्राणैर्वियोक्ष्यते} %4-16-7

\twolineshloka
{अहं ह्याजिस्थितस्यास्य करिष्यामि यदीप्सितम्}
{वृक्षैर्मुष्टिप्रहारैश्च पीडितः प्रतियास्यति} %4-16-8

\twolineshloka
{न मे गर्वितमायस्तं सहिष्यति दुरात्मवान्}
{कृतं तारे सहायत्वं दर्शितं सौहृदं मयि} %4-16-9

\twolineshloka
{शापितासि मम प्राणैर्निवर्तस्व जनेन च}
{अलं जित्वा निवर्तिष्ये तमहं भ्रातरं रणे} %4-16-10

\twolineshloka
{तं तु तारा परिष्वज्य वालिनं प्रियवादिनी}
{चकार रुदती मन्दं दक्षिणा सा प्रदक्षिणम्} %4-16-11

\twolineshloka
{ततः स्वस्त्ययनं कृत्वा मन्त्रविद् विजयैषिणी}
{अन्तःपुरं सह स्त्रीभिः प्रविष्टा शोकमोहिता} %4-16-12

\twolineshloka
{प्रविष्टायां तु तारायां सह स्त्रीभिः स्वमालयम्}
{नगर्या निर्ययौ क्रुद्धो महासर्प इव श्वसन्} %4-16-13

\twolineshloka
{स निःश्वस्य महारोषो वाली परमवेगवान्}
{सर्वतश्चारयन् दृष्टिं शत्रुदर्शनकांक्षया} %4-16-14

\twolineshloka
{स ददर्श ततः श्रीमान् सुग्रीवं हेमपिङ्गलम्}
{सुसंवीतमवष्टब्धं दीप्यमानमिवानलम्} %4-16-15

\twolineshloka
{तं स दृष्ट्वा महाबाहुः सुग्रीवं पर्यवस्थितम्}
{गाढं परिदधे वासो वाली परमकोपनः} %4-16-16

\twolineshloka
{स वाली गाढसंवीतो मुष्टिमुद्यम्य वीर्यवान्}
{सुग्रीवमेवाभिमुखो ययौ योद्धुं कृतक्षणः} %4-16-17

\twolineshloka
{श्लिष्टं मुष्टिं समुद्यम्य संरब्धतरमागतः}
{सुग्रीवोऽपि समुद्दिश्य वालिनं हेममालिनम्} %4-16-18

\twolineshloka
{तं वाली क्रोधताम्राक्षः सुग्रीवं रणकोविदम्}
{आपतन्तं महावेगमिदं वचनमब्रवीत्} %4-16-19

\twolineshloka
{एष मुष्टिर्महान् बद्धो गाढः सुनियताङ्गुलिः}
{मया वेगविमुक्तस्ते प्राणानादाय यास्यति} %4-16-20

\twolineshloka
{एवमुक्तस्तु सुग्रीवः क्रुद्धो वालिनमब्रवीत्}
{तव चैष हरन् प्राणान् मुष्टिः पततु मूर्धनि} %4-16-21

\twolineshloka
{ताडितस्तेन तं क्रुद्धः समभिक्रम्य वेगतः}
{अभवच्छोणितोद्गारी सापीड इव पर्वतः} %4-16-22

\twolineshloka
{सुग्रीवेण तु निःशङ्कं सालमुत्पाट्य तेजसा}
{गात्रेष्वभिहतो वाली वज्रेणेव महागिरिः} %4-16-23

\twolineshloka
{स तु वृक्षेण निर्भग्नः सालताडनविह्वलः}
{गुरुभारभराक्रान्ता नौः ससार्थेव सागरे} %4-16-24

\twolineshloka
{तौ भीमबलविक्रान्तौ सुपर्णसमवेगितौ}
{प्रवृद्धौ घोरवपुषौ चन्द्रसूर्याविवाम्बरे} %4-16-25

\twolineshloka
{परस्परममित्रघ्नौ छिद्रान्वेषणतत्परौ}
{ततोऽवर्धत वाली तु बलवीर्यसमन्वितः} %4-16-26

\twolineshloka
{सूर्यपुत्रो महावीर्यः सुग्रीवः परिहीयत}
{वालिना भग्नदर्पस्तु सुग्रीवो मन्दविक्रमः} %4-16-27

\twolineshloka
{वालिनं प्रति सामर्षो दर्शयामास राघवम्}
{वृक्षैः सशाखैः शिखरैर्वज्रकोटिनिभैर्नखैः} %4-16-28

\twolineshloka
{मुष्टिभिर्जानुभिः पद्भिर्बाहुभिश्च पुनः पुनः}
{तयोर्युद्धमभूद्घोरं वृत्रवासवयोरिव} %4-16-29

\twolineshloka
{तौ शोणिताक्तौ युध्येतां वानरौ वनचारिणौ}
{मेघाविव महाशब्दैस्तर्जमानौ परस्परम्} %4-16-30

\twolineshloka
{हीयमानमथापश्यत् सुग्रीवं वानरेश्वरम्}
{प्रेक्षमाणं दिशश्चैव राघवः स मुहुर्मुहुः} %4-16-31

\twolineshloka
{ततो रामो महातेजा आर्तं दृष्ट्वा हरीश्वरम्}
{स शरं वीक्षते वीरो वालिनो वधकांक्षया} %4-16-32

\twolineshloka
{ततो धनुषि संधाय शरमाशीविषोपमम्}
{पूरयामास तच्चापं कालचक्रमिवान्तकः} %4-16-33

\twolineshloka
{तस्य ज्यातलघोषेण त्रस्ताः पत्ररथेश्वराः}
{प्रदुद्रुवुर्मृगाश्चैव युगान्त इव मोहिताः} %4-16-34

\twolineshloka
{मुक्तस्तु वज्रनिर्घोषः प्रदीप्ताशनिसंनिभः}
{राघवेण महाबाणो वालिवक्षसि पातितः} %4-16-35

\twolineshloka
{ततस्तेन महातेजा वीर्ययुक्तः कपीश्वरः}
{वेगेनाभिहतो वाली निपपात महीतले} %4-16-36

\threelineshloka
{इन्द्रध्वज इवोद्धूतः पौर्णमास्यां महीतले}
{आश्वयुक्समये मासि गतश्रीको विचेतनः}
{बाष्पसंरुद्धकण्ठस्तु वाली चार्तस्वरः शनैः} %4-16-37

\twolineshloka
{नरोत्तमः कालयुगान्तकोपमं शरोत्तमं काञ्चनरूप्यभूषितम्}
{ससर्ज दीप्तं तममित्रमर्दनं सधूममग्निं मुखतो यथा हरः} %4-16-38

\twolineshloka
{अथोक्षितः शोणिततोयविस्रवैः सुपुष्पिताशोक इवानिलोद्धतः}
{विचेतनो वासवसूनुराहवे प्रभ्रंशितेन्द्रध्वजवत् क्षितिं गतः} %4-16-39


॥इत्यार्षे श्रीमद्रामायणे वाल्मीकीये आदिकाव्ये किष्किन्धाकाण्डे वालिसंहारः नाम षोडशः सर्गः ॥४-१६॥
