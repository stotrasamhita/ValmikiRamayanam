\sect{पञ्चाशः सर्गः — ऋक्षबिलप्रवेशः}

\twolineshloka
{सह ताराङ्गदाभ्यां तु सङ्गम्य हनुमान् कपिः}
{विचिनोति च विन्ध्यस्य गुहाश्च गहनानि च} %4-50-1

\twolineshloka
{सिंहशार्दूलजुष्टाश्च गुहाश्च परितस्तदा}
{विषमेषु नगेन्द्रस्य महाप्रस्रवणेषु च} %4-50-2

\twolineshloka
{आसेदुस्तस्य शैलस्य कोटिं दक्षिणपश्चिमाम्}
{तेषां तत्रैव वसतां स कालो व्यत्यवर्तत} %4-50-3

\twolineshloka
{स हि देशो दुरन्वेष्यो गुहागहनवान् महान्}
{तत्र वायुसुतः सर्वं विचिनोति स्म पर्वतम्} %4-50-4

\twolineshloka
{परस्परेण रहिता अन्योन्यस्याविदूरतः}
{गजो गवाक्षो गवयः शरभो गन्धमादनः} %4-50-5

\twolineshloka
{मैन्दश्च द्विविदश्चैव हनूमान् जाम्बवानपि}
{अङ्गदो युवराजश्च तारश्च वनगोचरः} %4-50-6

\twolineshloka
{गिरिजालावृतान् देशान् मार्गित्वा दक्षिणां दिशम्}
{विचिन्वन्तस्ततस्तत्र ददृशुर्विवृतं बिलम्} %4-50-7

\twolineshloka
{दुर्गमृक्षबिलं नाम दानवेनाभिरक्षितम्}
{क्षुत्पिपासापरीतास्तु श्रान्तास्तु सलिलार्थिनः} %4-50-8

\twolineshloka
{अवकीर्णं लतावृक्षैर्ददृशुस्ते महाबिलम्}
{तत्र क्रौञ्चाश्च हंसाश्च सारसाश्चापि निष्क्रमन्} %4-50-9

\twolineshloka
{जलार्द्राश्चक्रवाकाश्च रक्ताङ्गाः पद्मरेणुभिः}
{ततस्तद् बिलमासाद्य सुगन्धि दुरतिक्रमम्} %4-50-10

\twolineshloka
{विस्मयव्यग्रमनसो बभूवुर्वानरर्षभाः}
{सञ्जातपरिशङ्कास्ते तद् बिलं प्लवगोत्तमाः} %4-50-11

\twolineshloka
{अभ्यपद्यन्त संहृष्टास्तेजोवन्तो महाबलाः}
{नानासत्त्वसमाकीर्णं दैत्येन्द्रनिलयोपमम्} %4-50-12

\twolineshloka
{दुर्दर्शमिव घोरं च दुर्विगाह्यं च सर्वशः}
{ततः पर्वतकूटाभो हनूमान् मारुतात्मजः} %4-50-13

\twolineshloka
{अब्रवीद् वानरान् घोरान् कान्तारवनकोविदः}
{गिरिजालावृतान् देशान् मार्गित्वा दक्षिणां दिशम्} %4-50-14

\twolineshloka
{वयं सर्वे परिश्रान्ता न च पश्याम मैथिलीम्}
{अस्माच्चापि बिलाद्धंसाः क्रौञ्चाश्च सह सारसैः} %4-50-15

\twolineshloka
{जलार्द्राश्चक्रवाकाश्च निष्पतन्ति स्म सर्वशः}
{नूनं सलिलवानत्र कूपो वा यदि वा ह्रदः} %4-50-16

\twolineshloka
{तथा चेमे बिलद्वारे स्निग्धास्तिष्ठन्ति पादपाः}
{इत्युक्तास्तद् बिलं सर्वे विविशुस्तिमिरावृतम्} %4-50-17

\twolineshloka
{अचन्द्रसूर्यं हरयो ददृशू रोमहर्षणम्}
{निशाम्य तस्मात् सिंहांश्च तांस्तांश्च मृगपक्षिणः} %4-50-18

\twolineshloka
{प्रविष्टा हरिशार्दूला बिलं तिमिरसंवृतम्}
{न तेषां सज्जते दृष्टिर्न तेजो न पराक्रमः} %4-50-19

\twolineshloka
{वायोरिव गतिस्तेषां दृष्टिस्तमसि वर्तते}
{ते प्रविष्टास्तु वेगेन तद् बिलं कपिकुञ्जराः} %4-50-20

\twolineshloka
{प्रकाशं चाभिरामं च ददृशुर्देशमुत्तमम्}
{ततस्तस्मिन् बिले भीमे नानापादपसङ्कुले} %4-50-21

\twolineshloka
{अन्योन्यं सम्परिष्वज्य जग्मुर्योजनमन्तरम्}
{ते नष्टसंज्ञास्तृषिताः सम्भ्रान्ताः सलिलार्थिनः} %4-50-22

\twolineshloka
{परिपेतुर्बिले तस्मिन् कञ्चित् कालमतन्द्रिताः}
{ते कृशा दीनवदनाः परिश्रान्ताः प्लवङ्गमाः} %4-50-23

\twolineshloka
{आलोकं ददृशुर्वीरा निराशा जीविते यदा}
{ततस्तं देशमागम्य सौम्या वितिमिरं वनम्} %4-50-24

\twolineshloka
{ददृशुः काञ्चनान् वृक्षान् दीप्तवैश्वानरप्रभान्}
{सालांस्तालांस्तमालांश्च पुन्नागान् वञ्जुलान् धवान्} %4-50-25

\twolineshloka
{चम्पकान् नागवृक्षांश्च कर्णिकारांश्च पुष्पितान्}
{स्तबकैः काञ्चनैश्चित्रै रक्तैः किसलयैस्तथा} %4-50-26

\twolineshloka
{आपीडैश्च लताभिश्च हेमाभरणभूषितान्}
{तरुणादित्यसङ्काशान् वैदूर्यमयवेदिकान्} %4-50-27

\twolineshloka
{बिभ्राजमानान् वपुषा पादपांश्च हिरण्मयान्}
{नीलवैदूर्यवर्णाश्च पद्मिनीः पतगैर्वृताः} %4-50-28

\twolineshloka
{महद्भिः काञ्चनैर्वृक्षैर्वृता बालार्कसन्निभैः}
{जातरूपमयैर्मत्स्यैर्महद्भिश्चाथ पङ्कजैः} %4-50-29

\twolineshloka
{नलिनीस्तत्र ददृशुः प्रसन्नसलिलायुताः}
{काञ्चनानि विमानानि राजतानि तथैव च} %4-50-30

\twolineshloka
{तपनीयगवाक्षाणि मुक्ताजालावृतानि च}
{हैमराजतभौमानि वैदूर्यमणिमन्ति च} %4-50-31

\twolineshloka
{ददृशुस्तत्र हरयो गृहमुख्यानि सर्वशः}
{पुष्पितान् फलिनो वृक्षान् प्रवालमणिसन्निभान्} %4-50-32

\twolineshloka
{काञ्चनभ्रमरांश्चैव मधूनि च समन्ततः}
{मणिकाञ्चनचित्राणि शयनान्यासनानि च} %4-50-33

\twolineshloka
{विविधानि विशालानि ददृशुस्ते समन्ततः}
{हैमराजतकांस्यानां भाजनानां च राशयः} %4-50-34

\twolineshloka
{अगुरूणां च दिव्यानां चन्दनानां च सञ्चयान्}
{शुचीन्यभ्यवहाराणि मूलानि च फलानि च} %4-50-35

\twolineshloka
{महार्हाणि च यानानि मधूनि रसवन्ति च}
{दिव्यानामम्बराणां च महार्हाणां च सञ्चयान्} %4-50-36

\twolineshloka
{कम्बलानां च चित्राणामजिनानां च सञ्चयान्}
{तत्र तत्र च विन्यस्तान् दीप्तान् वैश्वानरप्रभान्} %4-50-37

\twolineshloka
{ददृशुर्वानराः शुभ्राञ्जातरूपस्य सञ्चयान्}
{तत्र तत्र विचिन्वन्तो बिले तत्र महाप्रभाः} %4-50-38

\twolineshloka
{ददृशुर्वानराः शूराः स्त्रियं काञ्चिददूरतः}
{तां च ते ददृशुस्तत्र चीरकृष्णाजिनाम्बराम्} %4-50-39

\threelineshloka
{तापसीं नियताहारां ज्वलन्तीमिव तेजसा}
{विस्मिता हरयस्तत्र व्यवतिष्ठन्त सर्वशः}
{पप्रच्छ हनुमांस्तत्र कासि त्वं कस्य वा बिलम्} %4-50-40

\twolineshloka
{ततो हनूमान् गिरिसन्निकाशः कृताञ्जलिस्तामभिवाद्य वृद्धाम्}
{पप्रच्छ का त्वं भवनं बिलं च रत्नानि चेमानि वदस्व कस्य} %4-50-41


॥इत्यार्षे श्रीमद्रामायणे वाल्मीकीये आदिकाव्ये किष्किन्धाकाण्डे ऋक्षबिलप्रवेशः नाम पञ्चाशः सर्गः ॥४-५०॥
