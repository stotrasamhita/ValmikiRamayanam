\sect{चतुःपञ्चाशः सर्गः — हनूमद्भेदनम्}

\twolineshloka
{तथा ब्रुवति तारे तु ताराधिपतिवर्चसि}
{अथ मेने हृतं राज्यं हनूमानङ्गदेन तत्} %4-54-1

\twolineshloka
{बुद्ध्या ह्यष्टाङ्गया युक्तं चतुर्बलसमन्वितम्}
{चतुर्दशगुणं मेने हनूमान् वालिनः सुतम्} %4-54-2

\twolineshloka
{आपूर्यमाणं शश्वच्च तेजोबलपराक्रमैः}
{शशिनं शुक्लपक्षादौ वर्धमानमिव श्रिया} %4-54-3

\twolineshloka
{बृहस्पतिसमं बुद्ध्या विक्रमे सदृशं पितुः}
{शुश्रूषमाणं तारस्य शुक्रस्येव पुरंदरम्} %4-54-4

\twolineshloka
{भर्तुरर्थे परिश्रान्तं सर्वशास्त्रविशारदः}
{अभिसंधातुमारेभे हनूमानङ्गदं ततः} %4-54-5

\twolineshloka
{स चतुर्णामुपायानां तृतीयमुपवर्णयन्}
{भेदयामास तान् सर्वान् वानरान् वाक्यसम्पदा} %4-54-6

\twolineshloka
{तेषु सर्वेषु भिन्नेषु ततोऽभीषयदङ्गदम्}
{भीषणैर्विविधैर्वाक्यैः कोपोपायसमन्वितैः} %4-54-7

\twolineshloka
{त्वं समर्थतरः पित्रा युद्धे तारेय वै ध्रुवम्}
{दृढं धारयितुं शक्तः कपिराज्यं यथा पिता} %4-54-8

\twolineshloka
{नित्यमस्थिरचित्ता हि कपयो हरिपुंगव}
{नाज्ञाप्यं विषहिष्यन्ति पुत्रदारं विना त्वया} %4-54-9

\twolineshloka
{त्वां नैते ह्यनुरञ्जेयुः प्रत्यक्षं प्रवदामि ते}
{यथायं जाम्बवान् नीलः सुहोत्रश्च महाकपिः} %4-54-10

\twolineshloka
{नह्यहं ते इमे सर्वे सामदानादिभिर्गुणैः}
{दण्डेन न त्वया शक्याः सुग्रीवादपकर्षितुम्} %4-54-11

\twolineshloka
{विगृह्यासनमप्याहुर्दुर्बलेन बलीयसा}
{आत्मरक्षाकरस्तस्मान्न विगृह्णीत दुर्बलः} %4-54-12

\twolineshloka
{यां चेमां मन्यसे धात्रीमेतद् बिलमिति श्रुतम्}
{एतल्लक्ष्मणबाणानामीषत् कार्यं विदारणम्} %4-54-13

\twolineshloka
{स्वल्पं हि कृतमिन्द्रेण क्षिपता ह्यशनिं पुरा}
{लक्ष्मणो निशितैर्बाणैर्भिन्द्यात् पत्रपुटं यथा} %4-54-14

\twolineshloka
{लक्ष्मणस्य च नाराचा बहवः सन्ति तद्विधाः}
{वज्राशनिसमस्पर्शा गिरीणामपि दारकाः} %4-54-15

\twolineshloka
{अवस्थानं यदैव त्वमासिष्यसि परंतप}
{तदैव हरयः सर्वे त्यक्ष्यन्ति कृतनिश्चयाः} %4-54-16

\twolineshloka
{स्मरन्तः पुत्रदाराणां नित्योद्विग्ना बुभुक्षिताः}
{खेदिता दुःखशय्याभिस्त्वां करिष्यन्ति पृष्ठतः} %4-54-17

\twolineshloka
{स त्वं हीनः सुहृद्भिश्च हितकामैश्च बन्धुभिः}
{तृणादपि भृशोद्विग्नः स्पन्दमानाद् भविष्यसि} %4-54-18

\twolineshloka
{न च जातु न हिंस्युस्त्वां घोरा लक्ष्मणसायकाः}
{अपवृत्तं जिघांसन्तो महावेगा दुरासदाः} %4-54-19

\twolineshloka
{अस्माभिस्तु गतं सार्धं विनीतवदुपस्थितम्}
{आनुपूर्व्यात्तु सुग्रीवो राज्ये त्वां स्थापयिष्यति} %4-54-20

\twolineshloka
{धर्मराजः पितृव्यस्ते प्रीतिकामो दृढव्रतः}
{शुचिः सत्यप्रतिज्ञश्च स त्वां जातु न नाशयेत्} %4-54-21

\twolineshloka
{प्रियकामश्च ते मातुस्तदर्थं चास्य जीवितम्}
{तस्यापत्यं च नास्त्यन्यत् तस्मादङ्गद गम्यताम्} %4-54-22


॥इत्यार्षे श्रीमद्रामायणे वाल्मीकीये आदिकाव्ये किष्किन्धाकाण्डे हनूमद्भेदनम् नाम चतुःपञ्चाशः सर्गः ॥४-५४॥
