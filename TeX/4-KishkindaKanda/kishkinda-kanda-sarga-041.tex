\sect{एकचत्वारिंशः सर्गः — दक्षिणाप्रेषणम्}

\twolineshloka
{ततः प्रस्थाप्य सुग्रीवस्तन्महद्वानरं बलम्}
{दक्षिणां प्रेषयामास वानरानभिलक्षितान्} %4-41-1

\twolineshloka
{नीलमग्निसुतं चैव हनूमन्तं च वानरम्}
{पितामहसुतं चैव जाम्बवन्तं महौजसम्} %4-41-2

\twolineshloka
{सुहोत्रं च शरारिं च शरगुल्मं तथैव च}
{गजं गवाक्षं गवयं सुषेणं वृषभं तथा} %4-41-3

\twolineshloka
{मैन्दं च द्विविदं चैव सुषेणं गन्धमादनम्}
{उल्कामुखमनङ्गं च हुताशनसुतावुभौ} %4-41-4

\twolineshloka
{अङ्गदप्रमुखान् वीरान् वीरः कपिगणेश्वरः}
{वेगविक्रमसम्पन्नान् सन्दिदेश विशेषवित्} %4-41-5

\twolineshloka
{तेषामग्रेसरं चैव बृहद्बलमथाङ्गदम्}
{विधाय हरिवीराणामादिशद् दक्षिणां दिशम्} %4-41-6

\twolineshloka
{ये केचन समुद्देशास्तस्यां दिशि सुदुर्गमाः}
{कपीशः कपिमुख्यानां स तेषां समुदाहरत्} %4-41-7

\twolineshloka
{सहस्रशिरसं विन्ध्यं नानाद्रुमलतायुतम्}
{नर्मदां च नदीं रम्यां महोरगनिषेविताम्} %4-41-8

\threelineshloka
{ततो गोदावरीं रम्यां कृष्णवेणीं महानदीम्}
{वरदां च महाभागां महोरगनिषेविताम्}
{मेखलानुत्कलांश्चैव दशार्णनगराण्यपि} %4-41-9

\twolineshloka
{आब्रवन्तीमवन्तीं च सर्वमेवानुपश्यत}
{विदर्भानृष्टिकांश्चैव रम्यान् माहिषकानपि} %4-41-10

\twolineshloka
{तथा वङ्गान् कलिङ्गांश्च कौशिकांश्च समन्ततः}
{अन्वीक्ष्य दण्डकारण्यं सपर्वतनदीगुहम्} %4-41-11

\twolineshloka
{नदीं गोदावरीं चैव सर्वमेवानुपश्यत}
{तथैवान्ध्रांश्च पुण्ड्रांश्च चोलान् पाण्ड्यांश्च केरलान्} %4-41-12

\twolineshloka
{अयोमुखश्च गन्तव्यः पर्वतो धातुमण्डितः}
{विचित्रशिखरः श्रीमांश्चित्रपुष्पितकाननः} %4-41-13

\twolineshloka
{सुचन्दनवनोद्देशो मार्गितव्यो महागिरिः}
{ततस्तामापगां दिव्यां प्रसन्नसलिलाशयाम्} %4-41-14

\twolineshloka
{तत्र द्रक्ष्यथ कावेरीं विहृतामप्सरोगणैः}
{तस्यासीनं नगस्याग्रे मलयस्य महौजसम्} %4-41-15

\twolineshloka
{द्रक्ष्यथादित्यसङ्काशमगस्त्यमृषिसत्तमम्}
{ततस्तेनाभ्यनुज्ञाताः प्रसन्नेन महात्मना} %4-41-16

\twolineshloka
{ताम्रपर्णीं ग्राहजुष्टां तरिष्यथ महानदीम्}
{सा चन्दनवनैश्चित्रैः प्रच्छन्नद्वीपवारिणी} %4-41-17

\twolineshloka
{कान्तेव युवती कान्तं समुद्रमवगाहते}
{ततो हेममयं दिव्यं मुक्तामणिविभूषितम्} %4-41-18

\twolineshloka
{युक्तं कवाटं पाण्ड्यानां गता द्रक्ष्यथ वानराः}
{ततः समुद्रमासाद्य सम्प्रधार्यार्थनिश्चयम्} %4-41-19

\twolineshloka
{अगस्त्येनान्तरे तत्र सागरे विनिवेशितः}
{चित्रसानुनगः श्रीमान् महेन्द्रः पर्वतोत्तमः} %4-41-20

\twolineshloka
{जातरूपमयः श्रीमानवगाढो महार्णवम्}
{नानाविधैर्नगैः फुल्लैर्लताभिश्चोपशोभितम्} %4-41-21

\twolineshloka
{देवर्षियक्षप्रवरैरप्सरोभिश्च शोभितम्}
{सिद्धचारणसङ्घैश्च प्रकीर्णं सुमनोरमम्} %4-41-22

\twolineshloka
{तमुपैति सहस्राक्षः सदा पर्वसु पर्वसु}
{द्वीपस्तस्यापरे पारे शतयोजनविस्तृतः} %4-41-23

\twolineshloka
{अगम्यो मानुषैर्दीप्तस्तं मार्गध्वं समन्ततः}
{तत्र सर्वात्मना सीता मार्गितव्या विशेषतः} %4-41-24

\twolineshloka
{स हि देशस्तु वध्यस्य रावणस्य दुरात्मनः}
{राक्षसाधिपतेर्वासः सहस्राक्षसमद्युतेः} %4-41-25

\twolineshloka
{दक्षिणस्य समुद्रस्य मध्ये तस्य तु राक्षसी}
{अङ्गारकेति विख्याता छायामाक्षिप्य भोजिनी} %4-41-26

\twolineshloka
{एवं निःसंशयान् कृत्वा संशयान्नष्टसंशयाः}
{मृगयध्वं नरेन्द्रस्य पत्नीममिततेजसः} %4-41-27

\twolineshloka
{तमतिक्रम्य लक्ष्मीवान् समुद्रे शतयोजने}
{गिरिः पुष्पितको नाम सिद्धचारणसेवितः} %4-41-28

\twolineshloka
{चन्द्रसूर्यांशुसङ्काशः सागराम्बुसमाश्रयः}
{भ्राजते विपुलैः शृङ्गैरम्बरं विलिखन्निव} %4-41-29

\threelineshloka
{तस्यैकं काञ्चनं शृङ्गं सेवते यं दिवाकरः}
{श्वेतं राजतमेकं च सेवते यन्निशाकरः}
{न तं कृतघ्नाः पश्यन्ति न नृशंसा न नास्तिकाः} %4-41-30

\twolineshloka
{प्रणम्य शिरसा शैलं तं विमार्गथ वानराः}
{तमतिक्रम्य दुर्धर्षं सूर्यवान्नाम पर्वतः} %4-41-31

\twolineshloka
{अध्वना दुर्विगाहेन योजनानि चतुर्दश}
{ततस्तमप्यतिक्रम्य वैद्युतो नाम पर्वतः} %4-41-32

\twolineshloka
{सर्वकामफलैर्वृक्षैः सर्वकालमनोहरैः}
{तत्र भुक्त्वा वरार्हाणि मूलानि च फलानि च} %4-41-33

\twolineshloka
{मधूनि पीत्वा जुष्टानि परं गच्छत वानराः}
{तत्र नेत्रमनःकान्तः कुञ्जरो नाम पर्वतः} %4-41-34

\twolineshloka
{अगस्त्यभवनं यत्र निर्मितं विश्वकर्मणा}
{तत्र योजनविस्तारमुच्छ्रितं दशयोजनम्} %4-41-35

\twolineshloka
{शरणं काञ्चनं दिव्यं नानारत्नविभूषितम्}
{तत्र भोगवती नाम सर्पाणामालयः पुरी} %4-41-36

\twolineshloka
{विशालरथ्या दुर्धर्षा सर्वतः परिरक्षिता}
{रक्षिता पन्नगैर्घोरैस्तीक्ष्णदंष्ट्रैर्महाविषैः} %4-41-37

\twolineshloka
{सर्पराजो महाघोरो यस्यां वसति वासुकिः}
{निर्याय मार्गितव्या च सा च भोगवती पुरी} %4-41-38

\twolineshloka
{तत्र चानन्तरोद्देशा ये केचन समावृताः}
{तं च देशमतिक्रम्य महानृषभसंस्थितिः} %4-41-39

\twolineshloka
{सर्वरत्नमयः श्रीमानृषभो नाम पर्वतः}
{गोशीर्षकं पद्मकं च हरिश्यामं च चन्दनम्} %4-41-40

\twolineshloka
{दिव्यमुत्पद्यते यत्र तच्चैवाग्निसमप्रभम्}
{न तु तच्चन्दनं दृष्ट्वा स्प्रष्टव्यं तु कदाचन} %4-41-41

\twolineshloka
{रोहिता नाम गन्धर्वा घोरं रक्षन्ति तद्वनम्}
{तत्र गन्धर्वपतयः पञ्च सूर्यसमप्रभाः} %4-41-42

\twolineshloka
{शैलूषो ग्रामणीः शिक्षः शुको बभ्रुस्तथैव च}
{रविसोमाग्निवपुषां निवासः पुण्यकर्मणाम्} %4-41-43

\twolineshloka
{अन्ते पृथिव्या दुर्धर्षास्ततः स्वर्गजितः स्थिताः}
{ततः परं न वः सेव्यः पितृलोकः सुदारुणः} %4-41-44

\threelineshloka
{राजधानी यमस्यैषा कष्टेन तमसाऽऽवृता}
{एतावदेव युष्माभिर्वीरा वानरपुङ्गवाः}
{शक्यं विचेतुं गन्तुं वा नातो गतिमतां गतिः} %4-41-45

\twolineshloka
{सर्वमेतत् समालोक्य यच्चान्यदपि दृश्यते}
{गतिं विदित्वा वैदेह्याः सन्निवर्तितुमर्हथ} %4-41-46

\twolineshloka
{यश्च मासान्निवृत्तोऽग्रे दृष्टा सीतेति वक्ष्यति}
{मत्तुल्यविभवो भोगैः सुखं स विहरिष्यति} %4-41-47

\twolineshloka
{ततः प्रियतरो नास्ति मम प्राणाद् विशेषतः}
{कृतापराधो बहुशो मम बन्धुर्भविष्यति} %4-41-48

\twolineshloka
{अमितबलपराक्रमा भवन्तो विपुलगुणेषु कुलेषु च प्रसूताः}
{मनुजपतिसुतां यथा लभध्वं तदधिगुणं पुरुषार्थमारभध्वम्} %4-41-49


॥इत्यार्षे श्रीमद्रामायणे वाल्मीकीये आदिकाव्ये किष्किन्धाकाण्डे दक्षिणाप्रेषणम् नाम एकचत्वारिंशः सर्गः ॥४-४१॥
