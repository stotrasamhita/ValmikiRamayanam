\sect{त्रयोदशः सर्गः — सप्तजनाश्रमप्रणामः}

\twolineshloka
{ऋष्यमूकात् स धर्मात्मा किष्किन्धां लक्ष्मणाग्रजः}
{जगाम सह सुग्रीवो वालिविक्रमपालिताम्} %4-13-1

\twolineshloka
{समुद्यम्य महच्चापं रामः काञ्चनभूषितम्}
{शरांश्चादित्यसंकाशान् गृहीत्वा रणसाधकान्} %4-13-2

\twolineshloka
{अग्रतस्तु ययौ तस्य राघवस्य महात्मनः}
{सुग्रीवः संहतग्रीवो लक्ष्मणश्च महाबलः} %4-13-3

\twolineshloka
{पृष्ठतो हनुमान् वीरो नलो नीलश्च वीर्यवान्}
{तारश्चैव महातेजा हरियूथपयूथपः} %4-13-4

\twolineshloka
{ते वीक्षमाणा वृक्षांश्च पुष्पभारावलम्बिनः}
{प्रसन्नाम्बुवहाश्चैव सरितः सागरंगमाः} %4-13-5

\twolineshloka
{कन्दराणि च शैलांश्च निर्दराणि गुहास्तथा}
{शिखराणि च मुख्यानि दरीश्च प्रियदर्शनाः} %4-13-6

\twolineshloka
{वैदूर्यविमलैस्तोयैः पद्मैश्चाकोशकुड्मलैः}
{शोभितान् सजलान् मार्गे तटाकांश्चावलोकयन्} %4-13-7

\twolineshloka
{कारण्डैः सारसैर्हंसैर्वञ्जुलैर्जलकुक्कुटैः}
{चक्रवाकैस्तथा चान्यैः शकुनैः प्रतिनादितान्} %4-13-8

\twolineshloka
{मृदुशष्पाङ्कुराहारान्निर्भयान् वनगोचरान्}
{चरतः सर्वतः पश्यन् स्थलीषु हरिणान् स्थितान्} %4-13-9

\twolineshloka
{तटाकवैरिणश्चापि शुक्लदन्तविभूषितान्}
{घोरानेकचरान् वन्यान् द्विरदान् कूलघातिनः} %4-13-10

\twolineshloka
{मत्तान् गिरितटोत्कृष्टान् पर्वतानिव जङ्गमान्}
{वानरान् द्विरदप्रख्यान् महीरेणुसमुक्षितान्} %4-13-11

\twolineshloka
{वने वनचरांश्चान्यान् खेचरांश्च विहंगमान्}
{पश्यन्तस्त्वरिता जग्मुः सुग्रीववशवर्तिनः} %4-13-12

\twolineshloka
{तेषां तु गच्छतां तत्र त्वरितं रघुनन्दनः}
{द्रुमषण्डवनं दृष्ट्वा रामः सुग्रीवमब्रवीत्} %4-13-13

\twolineshloka
{एष मेघ इवाकाशे वृक्षषण्डः प्रकाशते}
{मेघसंघातविपुलः पर्यन्तकदलीवृतः} %4-13-14

\twolineshloka
{किमेतज्ज्ञातुमिच्छामि सखे कौतूहलं मम}
{कौतूहलापनयनं कर्तुमिच्छाम्यहं त्वया} %4-13-15

\twolineshloka
{तस्य तद्वचनं श्रुत्वा राघवस्य महात्मनः}
{गच्छन् नेवाचचक्षेऽथ सुग्रीवस्तन्महद् वनम्} %4-13-16

\twolineshloka
{एतद् राघव विस्तीर्णमाश्रमं श्रमनाशनम्}
{उद्यानवनसम्पन्नं स्वादुमूलफलोदकम्} %4-13-17

\twolineshloka
{अत्र सप्तजना नाम मुनयः संशितव्रताः}
{सप्तैवासन्नधःशीर्षा नियतं जलशायिनः} %4-13-18

\twolineshloka
{सप्तरात्रे कृताहारा वायुनाचलवासिनः}
{दिवं वर्षशतैर्याताः सप्तभिः सकलेवराः} %4-13-19

\twolineshloka
{तेषामेतत्प्रभावेण द्रुमप्राकारसंवृतम्}
{आश्रमं सुदुराधर्षमपि सेन्द्रैः सुरासुरैः} %4-13-20

\twolineshloka
{पक्षिणो वर्जयन्त्येतत् तथान्ये वनचारिणः}
{विशन्ति मोहाद् येऽप्यत्र न निवर्तन्ति ते पुनः} %4-13-21

\twolineshloka
{विभूषणरवाश्चात्र श्रूयन्ते सकलाक्षराः}
{तूर्यगीतस्वनश्चापि गन्धो दिव्यश्च राघव} %4-13-22

\twolineshloka
{त्रेताग्नयोऽपि दीप्यन्ते धूमो ह्येष प्रदृश्यते}
{वेष्टयन्निव वृक्षाग्रान् कपोताङ्गारुणो घनः} %4-13-23

\twolineshloka
{एते वृक्षाः प्रकाशन्ते धूमसंसक्तमस्तकाः}
{मेघजालप्रतिच्छन्ना वैडूर्यगिरयो यथा} %4-13-24

\twolineshloka
{कुरु प्रणामं धर्मात्मंस्तेषामुद्दिश्य राघव}
{लक्ष्मणेन सह भ्रात्रा प्रयतः संहताञ्जलिः} %4-13-25

\twolineshloka
{प्रणमन्ति हि ये तेषामृषीणां भावितात्मनाम्}
{न तेषामशुभं किंचिच्छरीरे राम विद्यते} %4-13-26

\twolineshloka
{ततो रामः सह भ्रात्रा लक्ष्मणेन कृताञ्जलिः}
{समुद्दिश्य महात्मानस्तानृषीनभ्यवादयत्} %4-13-27

\twolineshloka
{अभिवाद्य च धर्मात्मा रामो भ्राता च लक्ष्मणः}
{सुग्रीवो वानराश्चैव जम्मुः संहृष्टमानसाः} %4-13-28

\twolineshloka
{ते गत्वा दूरमध्वानं तस्मात् सप्तजनाश्रमात्}
{ददृशुस्तां दुराधर्षां किष्किन्धां वालिपालिताम्} %4-13-29

\twolineshloka
{ततस्तु रामानुजरामवानराः प्रगृह्य शस्त्राण्युदितोग्रतेजसः}
{पुरीं सुरेशात्मजवीर्यपालितां वधाय शत्रोः पुनरागतास्त्विह} %4-13-30


॥इत्यार्षे श्रीमद्रामायणे वाल्मीकीये आदिकाव्ये किष्किन्धाकाण्डे सप्तजनाश्रमप्रणामः नाम त्रयोदशः सर्गः ॥४-१३॥
