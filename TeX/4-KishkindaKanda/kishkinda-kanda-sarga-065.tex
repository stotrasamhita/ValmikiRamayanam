\sect{पञ्चषष्ठितमः सर्गः — बलेयत्ताविष्करणम्}

\twolineshloka
{अथाङ्गदवचः श्रुत्वा ते सर्वे वानरर्षभाः}
{स्वं स्वं गतौ समुत्साहमूचुस्तत्र यथाक्रमम्} %4-65-1

\twolineshloka
{गजो गवाक्षो गवयः शरभो गन्धमादनः}
{मैन्दश्च द्विविदश्चैव सुषेणो जाम्बवांस्तथा} %4-65-2

\twolineshloka
{आबभाषे गजस्तत्र प्लवेयं दशयोजनम्}
{गवाक्षो योजनान्याह गमिष्यामीति विंशतिम्} %4-65-3

\twolineshloka
{शरभो वानरस्तत्र वानरांस्तानुवाच ह}
{त्रिंशतं तु गमिष्यामि योजनानां प्लवङ्गमाः} %4-65-4

\twolineshloka
{ऋषभो वानरस्तत्र वानरांस्तानुवाच ह}
{चत्वारिंशद् गमिष्यामि योजनानां न संशयः} %4-65-5

\twolineshloka
{वानरांस्तु महातेजा अब्रवीद् गन्धमादनः}
{योजनानां गमिष्यामि पञ्चाशत्तु न संशयः} %4-65-6

\twolineshloka
{मैन्दस्तु वानरस्तत्र वानरांस्तानुवाच ह}
{योजनानां परं षष्टिमहं प्लवितुमुत्सहे} %4-65-7

\twolineshloka
{ततस्तत्र महातेजा द्विविदः प्रत्यभाषत}
{गमिष्यामि न सन्देहः सप्ततिं योजनान्यहम्} %4-65-8

\twolineshloka
{सुषेणस्तु महातेजाः सत्त्ववान् कपिसत्तमः}
{अशीतिं प्रतिजानेऽहं योजनानां पराक्रमे} %4-65-9

\twolineshloka
{तेषां कथयतां तत्र सर्वांस्ताननुमान्य च}
{ततो वृद्धतमस्तेषां जाम्बवान् प्रत्यभाषत} %4-65-10

\twolineshloka
{पूर्वमस्माकमप्यासीत् कश्चिद् गतिपराक्रमः}
{ते वयं वयसः पारमनुप्राप्ताः स्म साम्प्रतम्} %4-65-11

\twolineshloka
{किं तु नैवं गते शक्यमिदं कार्यमुपेक्षितुम्}
{यदर्थं कपिराजश्च रामश्च कृतनिश्चयौ} %4-65-12

\twolineshloka
{साम्प्रतं कालमस्माकं या गतिस्तां निबोधत}
{नवतिं योजनानां तु गमिष्यामि न संशयः} %4-65-13

\twolineshloka
{तांश्च सर्वान् हरिश्रेष्ठाञ्जाम्बवानिदमब्रवीत्}
{न खल्वेतावदेवासीद् गमने मे पराक्रमः} %4-65-14

\twolineshloka
{मया वैरोचने यज्ञे प्रभविष्णुः सनातनः}
{प्रदक्षिणीकृतः पूर्वं क्रममाणस्त्रिविक्रमम्} %4-65-15

\twolineshloka
{स इदानीमहं वृद्धः प्लवने मन्दविक्रमः}
{यौवने च तदासीन्मे बलमप्रतिमं परम्} %4-65-16

\twolineshloka
{सम्प्रत्येतावदेवाद्य शक्यं मे गमने स्वतः}
{नैतावता च संसिद्धिः कार्यस्यास्य भविष्यति} %4-65-17

\twolineshloka
{अथोत्तरमुदारार्थमब्रवीदङ्गदस्तदा}
{अनुमान्य तदा प्राज्ञो जाम्बवन्तं महाकपिः} %4-65-18

\twolineshloka
{अहमेतद् गमिष्यामि योजनानां शतं महत्}
{निवर्तने तु मे शक्तिः स्यान्न वेति न निश्चितम्} %4-65-19

\twolineshloka
{तमुवाच हरिश्रेष्ठं जाम्बवान् वाक्यकोविदः}
{ज्ञायते गमने शक्तिस्तव हर्यृक्षसत्तम} %4-65-20

\twolineshloka
{कामं शतसहस्रं वा नह्येष विधिरुच्यते}
{योजनानां भवान् शक्तो गन्तुं प्रतिनिवर्तितुम्} %4-65-21

\twolineshloka
{नहि प्रेषयिता तात स्वामी प्रेष्यः कथञ्चन}
{भवतायं जनः सर्वः प्रेष्यः प्लवगसत्तम} %4-65-22

\twolineshloka
{भवान् कलत्रमस्माकं स्वामिभावे व्यवस्थितः}
{स्वामी कलत्रं सैन्यस्य गतिरेषा परन्तप} %4-65-23

\twolineshloka
{अपि वै तस्य कार्यस्य भवान् मूलमरिन्दम}
{तस्मात् कलत्रवत् तात प्रतिपाल्यः सदा भवान्} %4-65-24

\twolineshloka
{मूलमर्थस्य संरक्ष्यमेष कार्यविदां नयः}
{मूले हि सति सिध्यन्ति गुणाः सर्वे फलोदयाः} %4-65-25

\twolineshloka
{तद् भवानस्य कार्यस्य साधनं सत्यविक्रम}
{वुद्धिविक्रमसम्पन्नो हेतुरत्र परन्तप} %4-65-26

\twolineshloka
{गुरुश्च गुरुपुत्रश्च त्वं हि नः कपिसत्तम}
{भवन्तमाश्रित्य वयं समर्था ह्यर्थसाधने} %4-65-27

\twolineshloka
{उक्तवाक्यं महाप्राज्ञं जाम्बवन्तं महाकपिः}
{प्रत्युवाचोत्तरं वाक्यं वालिसूनुरथाङ्गदः} %4-65-28

\twolineshloka
{यदि नाहं गमिष्यामि नान्यो वानरपुङ्गवः}
{पुनः खल्विदमस्माभिः कार्यं प्रायोपवेशनम्} %4-65-29

\twolineshloka
{नह्यकृत्वा हरिपतेः सन्देशं तस्य धीमतः}
{तत्रापि गत्वा प्राणानां न पश्ये परिरक्षणम्} %4-65-30

\twolineshloka
{स हि प्रसादे चात्यर्थकोपे च हरिरीश्वरः}
{अतीत्य तस्य सन्देशं विनाशो गमने भवेत्} %4-65-31

\twolineshloka
{तत्तथा ह्यस्य कार्यस्य न भवत्यन्यथा गतिः}
{तद् भवानेव दृष्टार्थः सञ्चिन्तयितुमर्हति} %4-65-32

\twolineshloka
{सोऽङ्गदेन तदा वीरः प्रत्युक्तः प्लवगर्षभः}
{जाम्बवानुत्तमं वाक्यं प्रोवाचेदं ततोऽङ्गदम्} %4-65-33

\twolineshloka
{तस्य ते वीर कार्यस्य न किञ्चित् परिहास्यते}
{एष सञ्चोदयाम्येनं यः कार्यं साधयिष्यति} %4-65-34

\twolineshloka
{ततः प्रतीतं प्लवतां वरिष्ठमेकान्तमाश्रित्य सुखोपविष्टम्}
{सञ्चोदयामास हरिप्रवीरो हरिप्रवीरं हनुमन्तमेव} %4-65-35


॥इत्यार्षे श्रीमद्रामायणे वाल्मीकीये आदिकाव्ये किष्किन्धाकाण्डे बलेयत्ताविष्करणम् नाम पञ्चषष्ठितमः सर्गः ॥४-६५॥
