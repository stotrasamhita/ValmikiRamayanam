\sect{तृतीयः सर्गः — हनुमत्प्रेषणम्}

\twolineshloka
{वचो विज्ञाय हनुमान् सुग्रीवस्य महात्मनः}
{पर्वतादृष्यमूकात् तु पुप्लुवे यत्र राघवौ} %4-3-1

\twolineshloka
{कपिरूपं परित्यज्य हनुमान् मारुतात्मजः}
{भिक्षुरूपं ततो भेजे शठबुद्धितया कपिः} %4-3-2

\twolineshloka
{ततश्च हनुमान् वाचा श्लक्ष्णया सुमनोज्ञया}
{विनीतवदुपागम्य राघवौ प्रणिपत्य च} %4-3-3

\twolineshloka
{आबभाषे च तौ वीरौ यथावत् प्रशशंस च}
{सम्पूज्य विधिवद् वीरौ हनुमान् वानरोत्तमः} %4-3-4

\twolineshloka
{उवाच कामतो वाक्यं मृदु सत्यपराक्रमौ}
{राजर्षिदेवप्रतिमौ तापसौ संशितव्रतौ} %4-3-5

\twolineshloka
{देशं कथमिमं प्राप्तौ भवन्तौ वरवर्णिनौ}
{त्रासयन्तौ मृगगणानन्यांश्च वनचारिणः} %4-3-6

\twolineshloka
{पम्पातीररुहान् वृक्षान् वीक्षमाणौ समन्ततः}
{इमां नदीं शुभजलां शोभयन्तौ तरस्विनौ} %4-3-7

\twolineshloka
{धैर्यवन्तौ सुवर्णाभौ कौ युवां चीरवाससौ}
{निःश्वसन्तौ वरभुजौ पीडयन्ताविमाः प्रजाः} %4-3-8

\twolineshloka
{सिंहविप्रेक्षितौ वीरौ महाबलपराक्रमौ}
{शक्रचापनिभे चापे गृहीत्वा शत्रुनाशनौ} %4-3-9

\twolineshloka
{श्रीमन्तौ रूपसम्पन्नौ वृषभश्रेष्ठविक्रमौ}
{हस्तिहस्तोपमभुजौ द्युतिमन्तौ नरर्षभौ} %4-3-10

\twolineshloka
{प्रभया पर्वतेन्द्रोऽसौ युवयोरवभासितः}
{राज्यार्हावमरप्रख्यौ कथं देशमिहागतौ} %4-3-11

\twolineshloka
{पद्मपत्रेक्षणौ वीरौ जटामण्डलधारिणौ}
{अन्योन्यसदृशौ वीरौ देवलोकादिहागतौ} %4-3-12

\twolineshloka
{यदृच्छयेव सम्प्राप्तौ चन्द्रसूर्यौ वसुंधराम्}
{विशालवक्षसौ वीरौ मानुषौ देवरूपिणौ} %4-3-13

\twolineshloka
{सिंहस्कन्धौ महोत्साहौ समदाविव गोवृषौ}
{आयताश्च सुवृत्ताश्च बाहवः परिघोपमाः} %4-3-14

\twolineshloka
{सर्वभूषणभूषार्हाः किमर्थं न विभूषिताः}
{उभौ योग्यावहं मन्ये रक्षितुं पृथिवीमिमाम्} %4-3-15

\twolineshloka
{ससागरवनां कृत्स्नां विन्ध्यमेरुविभूषिताम्}
{इमे च धनुषी चित्रे श्लक्ष्णे चित्रानुलेपने} %4-3-16

\twolineshloka
{प्रकाशेते यथेन्द्रस्य वज्रे हेमविभूषिते}
{सम्पूर्णाश्च शितैर्बाणैस्तूणाश्च शुभदर्शनाः} %4-3-17

\twolineshloka
{जीवितान्तकरैर्घोरैर्ज्वलद्भिरिव पन्नगैः}
{महाप्रमाणौ विपुलौ तप्तहाटकभूषणौ} %4-3-18

\twolineshloka
{खड्गावेतौ विराजेते निर्मुक्तभुजगाविव}
{एवं मां परिभाषन्तं कस्माद् वै नाभिभाषतः} %4-3-19

\twolineshloka
{सुग्रीवो नाम धर्मात्मा कश्चिद् वानरपुङ्गवः}
{वीरो विनिकृतो भ्रात्रा जगद्भ्रमति दुःखितः} %4-3-20

\twolineshloka
{प्राप्तोऽहं प्रेषितस्तेन सुग्रीवेण महात्मना}
{राज्ञा वानरमुख्यानां हनुमान् नाम वानरः} %4-3-21

\twolineshloka
{युवाभ्यां स हि धर्मात्मा सुग्रीवः सख्यमिच्छति}
{तस्य मां सचिवं वित्तं वानरं पवनात्मजम्} %4-3-22

\twolineshloka
{भिक्षुरूपप्रतिच्छन्नं सुग्रीवप्रियकारणात्}
{ऋष्यमूकादिह प्राप्तं कामगं कामचारिणम्} %4-3-23

\twolineshloka
{एवमुक्त्वा तु हनुमांस्तौ वीरौ रामलक्ष्मणौ}
{वाक्यज्ञो वाक्यकुशलः पुनर्नोवाच किंचन} %4-3-24

\twolineshloka
{एतच्छ्रुत्वा वचस्तस्य रामो लक्ष्मणमब्रवीत्}
{प्रहृष्टवदनः श्रीमान् भ्रातरं पार्श्वतः स्थितम्} %4-3-25

\twolineshloka
{सचिवोऽयं कपीन्द्रस्य सुग्रीवस्य महात्मनः}
{तमेव कांक्षमाणस्य ममान्तिकमिहागतः} %4-3-26

\twolineshloka
{तमभ्यभाष सौमित्रे सुग्रीवसचिवं कपिम्}
{वाक्यज्ञं मधुरैर्वाक्यैः स्नेहयुक्तमरिंदमम्} %4-3-27

\twolineshloka
{नानृग्वेदविनीतस्य नायजुर्वेदधारिणः}
{नासामवेदविदुषः शक्यमेवं विभाषितुम्} %4-3-28

\twolineshloka
{नूनं व्याकरणं कृत्स्नमनेन बहुधा श्रुतम्}
{बहु व्याहरतानेन न किंचिदपशब्दितम्} %4-3-29

\twolineshloka
{न मुखे नेत्रयोश्चापि ललाटे च भ्रुवोस्तथा}
{अन्येष्वपि च सर्वेषु दोषः संविदितः क्वचित्} %4-3-30

\twolineshloka
{अविस्तरमसंदिग्धमविलम्बितमव्यथम्}
{उरःस्थं कण्ठगं वाक्यं वर्तते मध्यमस्वरम्} %4-3-31

\twolineshloka
{संस्कारक्रमसम्पन्नामद्भुतामविलम्बिताम्}
{उच्चारयति कल्याणीं वाचं हृदयहर्षिणीम्} %4-3-32

\twolineshloka
{अनया चित्रया वाचा त्रिस्थानव्यञ्जनस्थया}
{कस्य नाराध्यते चित्तमुद्यतासेररेरपि} %4-3-33

\twolineshloka
{एवंविधो यस्य दूतो न भवेत् पार्थिवस्य तु}
{सिद्ध्यन्ति हि कथं तस्य कार्याणां गतयोऽनघ} %4-3-34

\twolineshloka
{एवंगुणगणैर्युक्ता यस्य स्युः कार्यसाधकाः}
{तस्य सिद्ध्यन्ति सर्वेऽर्था दूतवाक्यप्रचोदिताः} %4-3-35

\twolineshloka
{एवमुक्तस्तु सौमित्रिः सुग्रीवसचिवं कपिम्}
{अभ्यभाषत वाक्यज्ञो वाक्यज्ञं पवनात्मजम्} %4-3-36

\twolineshloka
{विदिता नौ गुणा विद्वन् सुग्रीवस्य महात्मनः}
{तमेव चावां मार्गावः सुग्रीवं प्लवगेश्वरम्} %4-3-37

\twolineshloka
{यथा ब्रवीषि हनुमन् सुग्रीववचनादिह}
{तत् तथा हि करिष्यावो वचनात् तव सत्तम} %4-3-38

\twolineshloka
{तत् तस्य वाक्यं निपुणं निशम्य प्रहृष्टरूपः पवनात्मजः कपिः}
{मनः समाधाय जयोपपत्तौ सख्यं तदा कर्तुमियेष ताभ्याम्} %4-3-39


॥इत्यार्षे श्रीमद्रामायणे वाल्मीकीये आदिकाव्ये किष्किन्धाकाण्डे हनुमत्प्रेषणम् नाम तृतीयः सर्गः ॥४-३॥
