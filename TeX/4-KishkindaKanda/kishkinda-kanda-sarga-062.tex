\sect{द्विषष्ठितमः सर्गः — निशाकरभविष्याख्यानम्}

\twolineshloka
{एवमुक्त्वा मुनिश्रेष्ठमरुदं भृशदुःखितः}
{अथ ध्यात्वा मुहूर्तं च भगवानिदमब्रवीत्} %4-62-1

\twolineshloka
{पक्षौ च ते प्रपक्षौ च पुनरन्यौ भविष्यतः}
{चक्षुषी चैव प्राणाश्च विक्रमश्च बलं च ते} %4-62-2

\twolineshloka
{पुराणे सुमहत्कार्यं भविष्यं हि मया श्रुतम्}
{दृष्टं मे तपसा चैव श्रुत्वा च विदितं मम} %4-62-3

\twolineshloka
{राजा दशरथो नाम कश्चिदिक्ष्वाकुवर्धनः}
{तस्य पुत्रो महातेजा रामो नाम भविष्यति} %4-62-4

\twolineshloka
{अरण्यं च सह भ्रात्रा लक्ष्मणेन गमिष्यति}
{तस्मिन्नर्थे नियुक्तः सन् पित्रा सत्यपराक्रमः} %4-62-5

\twolineshloka
{नैर्ऋतो रावणो नाम तस्य भार्यां हरिष्यति}
{राक्षसेन्द्रो जनस्थाने अवध्यः सुरदानवैः} %4-62-6

\twolineshloka
{सा च कामैः प्रलोभ्यन्ती भक्ष्यैर्भोज्यैश्च मैथिली}
{न भोक्ष्यति महाभागा दुःखमग्ना यशस्विनी} %4-62-7

\twolineshloka
{परमान्नं च वैदेह्या ज्ञात्वा दास्यति वासवः}
{यदन्नममृतप्रख्यं सुराणामपि दुर्लभम्} %4-62-8

\twolineshloka
{तदन्नं मैथिली प्राप्य विज्ञायेन्द्रादिदं त्विति}
{अग्रमुद्धृत्य रामाय भूतले निर्वपिष्यति} %4-62-9

\twolineshloka
{यदि जीवति मे भर्ता लक्ष्मणो वापि देवरः}
{देवत्वं गच्छतोर्वापि तयोरन्नमिदं त्विति} %4-62-10

\twolineshloka
{एष्यन्ति प्रेषितास्तत्र रामदूताः प्लवङ्गमाः}
{आख्येया राममहिषी त्वया तेभ्यो विहङ्गम} %4-62-11

\twolineshloka
{सर्वथा तु न गन्तव्यमीदृशः क्व गमिष्यसि}
{देशकालौ प्रतीक्षस्व पक्षौ त्वं प्रतिपत्स्यसे} %4-62-12

\twolineshloka
{उत्सहेयमहं कर्तुमद्यैव त्वां सपक्षकम्}
{इहस्थस्त्वं हि लोकानां हितं कार्यं करिष्यसि} %4-62-13

\twolineshloka
{त्वयापि खलु तत् कार्यं तयोश्च नृपपुत्रयोः}
{ब्राह्मणानां गुरूणां च मुनीनां वासवस्य च} %4-62-14

\threelineshloka
{इच्छाम्यहमपि द्रष्टुं भ्रातरौ रामलक्ष्मणौ}
{नेच्छे चिरं धारयितुं प्राणांस्त्यक्ष्ये कलेवरम्}
{महर्षिस्त्वब्रवीदेवं दृष्टतत्त्वार्थदर्शनः} %4-62-15


॥इत्यार्षे श्रीमद्रामायणे वाल्मीकीये आदिकाव्ये किष्किन्धाकाण्डे निशाकरभविष्याख्यानम् नाम द्विषष्ठितमः सर्गः ॥४-६२॥
