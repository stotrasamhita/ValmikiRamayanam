\sect{प्रथमः सर्गः — रामविप्रलम्भावेशः}

\twolineshloka
{स तां पुष्करिणीं गत्वा पद्मोत्पलझषाकुलाम्}
{रामः सौमित्रिसहितो विललापाकुलेन्द्रियः} %4-1-1

\twolineshloka
{तत्र दृष्ट्वैव तां हर्षादिन्द्रियाणि चकम्पिरे}
{स कामवशमापन्नः सौमित्रिमिदमब्रवीत्} %4-1-2

\twolineshloka
{सौमित्रे शोभते पम्पा वैदूर्यविमलोदका}
{फुल्लपद्मोत्पलवती शोभिता विविधैर्द्रुमैः} %4-1-3

\twolineshloka
{सौमित्रे पश्य पम्पायाः काननं शुभदर्शनम्}
{यत्र राजन्ति शैला वा द्रुमाः सशिखरा इव} %4-1-4

\twolineshloka
{मां तु शोकाभिसंतप्तमाधयः पीडयन्ति वै}
{भरतस्य च दुःखेन वैदेह्या हरणेन च} %4-1-5

\twolineshloka
{शोकार्तस्यापि मे पम्पा शोभते चित्रकानना}
{व्यवकीर्णा बहुविधैः पुष्पैः शीतोदका शिवा} %4-1-6

\twolineshloka
{नलिनैरपि संछन्ना ह्यत्यर्थशुभदर्शना}
{सर्पव्यालानुचरिता मृगद्विजसमाकुला} %4-1-7

\twolineshloka
{अधिकं प्रविभात्येतन्नीलपीतं तु शाद्वलम्}
{द्रुमाणां विविधैः पुष्पैः परिस्तोमैरिवार्पितम्} %4-1-8

\twolineshloka
{पुष्पभारसमृद्धानि शिखराणि समन्ततः}
{लताभिः पुष्पिताग्राभिरुपगूढानि सर्वतः} %4-1-9

\twolineshloka
{सुखानिलोऽयं सौमित्रे कालः प्रचुरमन्मथः}
{गन्धवान् सुरभिर्मासो जातपुष्पफलद्रुमः} %4-1-10

\twolineshloka
{पश्य रूपाणि सौमित्रे वनानां पुष्पशालिनाम्}
{सृजतां पुष्पवर्षाणि वर्षं तोयमुचामिव} %4-1-11

\twolineshloka
{प्रस्तरेषु च रम्येषु विविधाः काननद्रुमाः}
{वायुवेगप्रचलिताः पुष्पैरवकिरन्ति गाम्} %4-1-12

\twolineshloka
{पतितैः पतमानैश्च पादपस्थैश्च मारुतः}
{कुसुमैः पश्य सौमित्रे क्रीडतीव समन्ततः} %4-1-13

\twolineshloka
{विक्षिपन् विविधाः शाखां नगानां कुसुमोत्कटाः}
{मारुतश्चलितस्थानैः षट्पदैरनुगीयते} %4-1-14

\twolineshloka
{मत्तकोकिलसंनादैर्नर्तयन्निव पादपान्}
{शैलकंदर निष्क्रान्तः प्रगीत इव चानिलः} %4-1-15

\twolineshloka
{तेन विक्षिपतात्यर्थं पवनेन समन्ततः}
{अमी संसक्तशाखाग्रा ग्रथिता इव पादपाः} %4-1-16

\twolineshloka
{स एव सुखसंस्पर्शो वाति चन्दनशीतलः}
{गन्धमभ्यवहन् पुण्यं श्रमापनयनोऽनिलः} %4-1-17

\twolineshloka
{अमी पवनविक्षिप्ता विनदन्तीव पादपाः}
{षट्पदैरनुकूजद्भिर्वनेषु मधुगन्धिषु} %4-1-18

\twolineshloka
{गिरिप्रस्थेषु रम्येषु पुष्पवद्भिर्मनोरमैः}
{संसक्तशिखराः शैला विराजन्ति महाद्रुमैः} %4-1-19

\twolineshloka
{पुष्पसंछन्नशिखरा मारुतोत्क्षेपचञ्चलाः}
{अमी मधुकरोत्तंसाः प्रगीता इव पादपाः} %4-1-20

\twolineshloka
{सुपुष्पितांस्तु पश्यैतान् कर्णिकारान् समन्ततः}
{हाटकप्रतिसंछन्नान् नरान् पीताम्बरानिव} %4-1-21

\twolineshloka
{अयं वसन्तः सौमित्रे नानाविहगनादितः}
{सीतया विप्रहीणस्य शोकसंदीपनो मम} %4-1-22

\twolineshloka
{मां हि शोकसमाक्रान्तं संतापयति मन्मथः}
{हृष्टं प्रवदमानश्च समाह्वयति कोकिलः} %4-1-23

\twolineshloka
{एष दात्यूहको हृष्टो रम्ये मां वननिर्झरे}
{प्रणदन्मन्मथाविष्टं शोचयिष्यति लक्ष्मण} %4-1-24

\twolineshloka
{श्रुत्वैतस्य पुरा शब्दमाश्रमस्था मम प्रिया}
{मामाहूय प्रमुदिताः परमं प्रत्यनन्दत} %4-1-25

\twolineshloka
{एवं विचित्राः पतगा नानारावविराविणः}
{वृक्षगुल्मलताः पश्य सम्पतन्ति समन्ततः} %4-1-26

\twolineshloka
{विमिश्रा विहगाः पुंभिरात्मव्यूहाभिनन्दिताः}
{भृङ्गराजप्रमुदिताः सौमित्रे मधुरस्वराः} %4-1-27

\twolineshloka
{अस्याः कूले प्रमुदिताः सङ्घशः शकुनास्त्विह}
{दात्यूहरतिविक्रन्दैः पुंस्कोकिलरुतैरपि} %4-1-28

\twolineshloka
{स्वनन्ति पादपाश्चेमे ममानङ्गप्रदीपकाः}
{अशोकस्तबकाङ्गारः षट्पदस्वननिःस्वनः} %4-1-29

\twolineshloka
{मां हि पल्लवताम्रार्चिर्वसन्ताग्निः प्रधक्ष्यति}
{नहि तां सूक्ष्मपक्ष्माक्षीं सुकेशीं मृदुभाषिणीम्} %4-1-30

\twolineshloka
{अपश्यतो मे सौमित्रे जीवितेऽस्ति प्रयोजनम्}
{अयं हि रुचिरस्तस्याः कालो रुचिरकाननः} %4-1-31

\twolineshloka
{कोकिलाकुलसीमान्तो दयिताया ममानघ}
{मन्मथायाससम्भूतो वसन्तगुणवर्धितः} %4-1-32

\twolineshloka
{अयं मां धक्ष्यति क्षिप्रं शोकाग्निर्नचिरादिव}
{अपश्यतस्तां वनितां पश्यतो रुचिरान् द्रुमान्} %4-1-33

\twolineshloka
{ममायमात्मप्रभवो भूयस्त्वमुपयास्यति}
{अदृश्यमाना वैदेही शोकं वर्धयतीह मे} %4-1-34

\twolineshloka
{दृश्यमानो वसन्तश्च स्वेदसंसर्गदूषकः}
{मां हि सा मृगशावाक्षी चिन्ताशोकबलात्कृतम्} %4-1-35

\twolineshloka
{संतापयति सौमित्रे क्रूरश्चैत्रवनानिलः}
{अमी मयूराः शोभन्ते प्रनृत्यन्तस्ततस्ततः} %4-1-36

\twolineshloka
{स्वैः पक्षैः पवनोद्धूतैर्गवाक्षैः स्फाटिकैरिव}
{शिखिनीभिः परिवृतास्त एते मदमूर्च्छिताः} %4-1-37

\twolineshloka
{मन्मथाभिपरीतस्य मम मन्मथवर्धनाः}
{पश्य लक्ष्मण नृत्यन्तं मयूरमुपनृत्यति} %4-1-38

\twolineshloka
{शिखिनी मन्मथार्तैषा भर्तारं गिरिसानुनि}
{तामेव मनसा रामां मयूरोऽप्यनुधावति} %4-1-39

\twolineshloka
{वितत्य रुचिरौ पक्षौ रुतैरुपहसन्निव}
{मयूरस्य वने नूनं रक्षसा न हृता प्रिया} %4-1-40

\twolineshloka
{तस्मान्नृत्यति रम्येषु वनेषु सह कान्तया}
{मम त्वयं विना वासः पुष्पमासे सुदुःसहः} %4-1-41

\twolineshloka
{पश्य लक्ष्मण संरागस्तिर्यग्योनिगतेष्वपि}
{यदेषा शिखिनी कामाद् भर्तारमभिवर्तते} %4-1-42

\twolineshloka
{ममाप्येवं विशालाक्षी जानकी जातसम्भ्रमा}
{मदनेनाभिवर्तेत यदि नापहृता भवेत्} %4-1-43

\twolineshloka
{पश्य लक्ष्मण पुष्पाणि निष्फलानि भवन्ति मे}
{पुष्पभारसमृद्धानां वनानां शिशिरात्यये} %4-1-44

\twolineshloka
{रुचिराण्यपि पुष्पाणि पादपानामतिश्रिया}
{निष्फलानि महीं यान्ति समं मधुकरोत्करैः} %4-1-45

\twolineshloka
{नदन्ति कामं शकुना मुदिताः सङ्घशः कलम्}
{आह्वयन्त इवान्योन्यं कामोन्मादकरा मम} %4-1-46

\twolineshloka
{वसन्तो यदि तत्रापि यत्र मे वसति प्रिया}
{नूनं परवशा सीता सापि शोचत्यहं यथा} %4-1-47

\twolineshloka
{नूनं न तु वसन्तस्तं देशं स्पृशति यत्र सा}
{कथं ह्यसितपद्माक्षी वर्तयेत् सा मया विना} %4-1-48

\twolineshloka
{अथवा वर्तते तत्र वसन्तो यत्र मे प्रिया}
{किं करिष्यति सुश्रोणी सा तु निर्भर्त्सिता परैः} %4-1-49

\twolineshloka
{श्यामा पद्मपलाशाक्षी मृदुभाषा च मे प्रिया}
{नूनं वसन्तमासाद्य परित्यक्ष्यति जीवितम्} %4-1-50

\twolineshloka
{दृढं हि हृदये बुद्धिर्मम सम्परिवर्तते}
{नालं वर्तयितुं सीता साध्वी मद्विरहं गता} %4-1-51

\twolineshloka
{मयि भावो हि वैदेह्यास्तत्त्वतो विनिवेशितः}
{ममापि भावः सीतायां सर्वथा विनिवेशितः} %4-1-52

\twolineshloka
{एष पुष्पवहो वायुः सुखस्पर्शो हिमावहः}
{तां विचिन्तयतः कान्तां पावकप्रतिमो मम} %4-1-53

\twolineshloka
{सदा सुखमहं मन्ये यं पुरा सह सीतया}
{मारुतः स विना सीतां शोकसंजननो मम} %4-1-54

\twolineshloka
{तां विनाथ विहङ्गोऽसौ पक्षी प्रणदितस्तदा}
{वायसः पादपगतः प्रहृष्टमभिकूजति} %4-1-55

\twolineshloka
{एष वै तत्र वैदेह्या विहगः प्रतिहारकः}
{पक्षी मां तु विशालाक्ष्याः समीपमुपनेष्यति} %4-1-56

\twolineshloka
{पश्य लक्ष्मण संनादं वने मदविवर्धनम्}
{पुष्पिताग्रेषु वृक्षेषु द्विजानामवकूजताम्} %4-1-57

\twolineshloka
{विक्षिप्तां पवनेनैतामसौ तिलकमञ्जरीम्}
{षट्पदः सहसाभ्येति मदोद्धूतामिव प्रियाम्} %4-1-58

\twolineshloka
{कामिनामयमत्यन्तमशोकः शोकवर्धनः}
{स्तबकैः पवनोत्क्षिप्तैस्तर्जयन्निव मां स्थितः} %4-1-59

\twolineshloka
{अमी लक्ष्मण दृश्यन्ते चूताः कुसुमशालिनः}
{विभ्रमोत्सिक्तमनसः साङ्गरागा नरा इव} %4-1-60

\twolineshloka
{सौमित्रे पश्य पम्पायाश्चित्रासु वनराजिषु}
{किंनरा नरशार्दूल विचरन्ति यतस्ततः} %4-1-61

\twolineshloka
{इमानि शुभगन्धीनि पश्य लक्ष्मण सर्वशः}
{नलिनानि प्रकाशन्ते जले तरुणसूर्यवत्} %4-1-62

\twolineshloka
{एषा प्रसन्नसलिला पद्मनीलोत्पलायुता}
{हंसकारण्डवाकीर्णा पम्पा सौगन्धिकायुता} %4-1-63

\twolineshloka
{जले तरुणसूर्याभैः षट्पदाहतकेसरैः}
{पङ्कजैः शोभते पम्पा समन्तादभिसंवृता} %4-1-64

\twolineshloka
{चक्रवाकयुता नित्यं चित्रप्रस्थवनान्तरा}
{मातङ्गमृगयूथैश्च शोभते सलिलार्थिभिः} %4-1-65

\twolineshloka
{पवनाहतवेगाभिरूर्मिभिर्विमलेऽम्भसि}
{पङ्कजानि विराजन्ते ताड्यमानानि लक्ष्मण} %4-1-66

\twolineshloka
{पद्मपत्रविशालाक्षीं सततं प्रियपङ्कजाम्}
{अपश्यतो मे वैदेहीं जीवितं नाभिरोचते} %4-1-67

\twolineshloka
{अहो कामस्य वामत्वं यो गतामपि दुर्लभाम्}
{स्मारयिष्यति कल्याणीं कल्याणतरवादिनीम्} %4-1-68

\twolineshloka
{शक्यो धारयितुं कामो भवेदभ्यागतो मया}
{यदि भूयो वसन्तो मां न हन्यात् पुष्पितद्रुमः} %4-1-69

\twolineshloka
{यानि स्म रमणीयानि तया सह भवन्ति मे}
{तान्येवारमणीयानि जायन्ते मे तया विना} %4-1-70

\twolineshloka
{पद्मकोशपलाशानि द्रष्टुं दृष्टिर्हि मन्यते}
{सीताया नेत्रकोशाभ्यां सदृशानीति लक्ष्मण} %4-1-71

\twolineshloka
{पद्मकेसरसंसृष्टो वृक्षान्तरविनिःसृतः}
{निःश्वास इव सीताया वाति वायुर्मनोहरः} %4-1-72

\twolineshloka
{सौमित्रे पश्य पम्पाया दक्षिणे गिरिसानुषु}
{पुष्पितां कर्णिकारस्य यष्टिं परमशोभिताम्} %4-1-73

\twolineshloka
{अधिकं शैलराजोऽयं धातुभिस्तु विभूषितः}
{विचित्रं सृजते रेणुं वायुवेगविघट्टितम्} %4-1-74

\twolineshloka
{गिरिप्रस्थास्तु सौमित्रे सर्वतः सम्प्रपुष्पितैः}
{निष्पत्रैः सर्वतो रम्यैः प्रदीप्ता इव किंशुकैः} %4-1-75

\twolineshloka
{पम्पातीररुहाश्चेमे संसिक्ता मधुगन्धिनः}
{मालतीमल्लिकापद्मकरवीराश्च पुष्पिताः} %4-1-76

\twolineshloka
{केतक्यः सिन्दुवाराश्च वासन्त्यश्च सुपुष्पिताः}
{माधव्यो गन्धपूर्णाश्च कुन्दगुल्माश्च सर्वशः} %4-1-77

\twolineshloka
{चिरिबिल्वा मधूकाश्च वञ्जुला बकुलास्तथा}
{चम्पकास्तिलकाश्चैव नागवृक्षाश्च पुष्पिताः} %4-1-78

\twolineshloka
{पद्मकाश्चैव शोभन्ते नीलाशोकाश्च पुष्पिताः}
{लोध्राश्च गिरिपृष्ठेषु सिंहकेसरपिञ्जराः} %4-1-79

\twolineshloka
{अङ्कोलाश्च कुरण्टाश्च चूर्णकाः पारिभद्रकाः}
{चूताः पाटलयश्चापि कोविदाराश्च पुष्पिताः} %4-1-80

\twolineshloka
{मुचुकुन्दार्जुनाश्चैव दृश्यन्ते गिरिसानुषु}
{केतकोद्दालकाश्चैव शिरीषाः शिंशपा धवाः} %4-1-81

\twolineshloka
{शाल्मल्यः किंशुकाश्चैव रक्ताः कुरबकास्तथा}
{तिनिशा नक्तमालाश्च चन्दनाः स्यन्दनास्तथा} %4-1-82

\twolineshloka
{हिन्तालास्तिलकाश्चैव नागवृक्षाश्च पुष्पिताः}
{पुष्पितान् पुष्पिताग्राभिर्लताभिः परिवेष्टितान्} %4-1-83

\twolineshloka
{द्रुमान् पश्येह सौमित्रे पम्पाया रुचिरान् बहून्}
{वातविक्षिप्तविटपान् यथासन्नान् द्रुमानिमान्} %4-1-84

\twolineshloka
{लताः समनुवर्तन्ते मत्ता इव वरस्त्रियः}
{पादपात् पादपं गच्छन् शैलाच्छैलं वनाद् वनम्} %4-1-85

\twolineshloka
{वाति नैकरसास्वादसम्मोदित इवानिलः}
{केचित् पर्याप्तकुसुमाः पादपा मधुगन्धिनः} %4-1-86

\twolineshloka
{केचिन्मुकुलसंवीताः श्यामवर्णा इवाबभुः}
{इदं मृष्टमिदं स्वादु प्रफुल्लमिदमित्यपि} %4-1-87

\threelineshloka
{रागरक्तो मधुकरः कुसुमेष्वेव लीयते}
{निलीय पुनरुत्पत्य सहसान्यत्र गच्छति}
{मधुलुब्धो मधुकरः पम्पातीरद्रुमेष्वसौ} %4-1-88

\twolineshloka
{इयं कुसुमसंघातैरुपस्तीर्णा सुखाकृता}
{स्वयं निपतितैर्भूमिः शयनप्रस्तरैरिव} %4-1-89

\twolineshloka
{विविधा विविधैः पुष्पैस्तैरेव नगसानुषु}
{विस्तीर्णाः पीतरक्ताभाः सौमित्रे प्रस्तराः कृताः} %4-1-90

\twolineshloka
{हिमान्ते पश्य सौमित्रे वृक्षाणां पुष्पसम्भवम्}
{पुष्पमासे हि तरवः संघर्षादिव पुष्पिताः} %4-1-91

\twolineshloka
{आह्वयन्त इवान्योन्यं नगाः षट्पदनादिताः}
{कुसुमोत्तंसविटपाः शोभन्ते बहु लक्ष्मण} %4-1-92

\twolineshloka
{एष कारण्डवः पक्षी विगाह्य सलिलं शुभम्}
{रमते कान्तया सार्धं काममुद्दीपयन्निव} %4-1-93

\twolineshloka
{मन्दाकिन्यास्तु यदिदं रूपमेतन्मनोरमम्}
{स्थाने जगति विख्याता गुणास्तस्या मनोरमाः} %4-1-94

\twolineshloka
{यदि दृश्येत सा साध्वी यदि चेह वसेमहि}
{स्पृहयेयं न शक्राय नायोध्यायै रघूत्तम} %4-1-95

\twolineshloka
{न ह्येवं रमणीयेषु शाद्वलेषु तया सह}
{रमतो मे भवेच्चिन्ता न स्पृहान्येषु वा भवेत्} %4-1-96

\twolineshloka
{अमी हि विविधैः पुष्पैस्तरवो विविधच्छदाः}
{काननेऽस्मिन् विना कान्तां चिन्तामुत्पादयन्ति मे} %4-1-97

\twolineshloka
{पश्य शीतजलां चेमां सौमित्रे पुष्करायुताम्}
{चक्रवाकानुचरितां कारण्डवनिषेविताम्} %4-1-98

\twolineshloka
{प्लवैः क्रौञ्चैश्च सम्पूर्णां महामृगनिषेविताम्}
{अधिकं शोभते पम्पा विकूजद्भिर्विहंगमैः} %4-1-99

\twolineshloka
{दीपयन्तीव मे कामं विविधा मुदिता द्विजाः}
{श्यामां चन्द्रमुखीं स्मृत्वा प्रियां पद्मनिभेक्षणाम्} %4-1-100

\threelineshloka
{पश्य सानुषु चित्रेषु मृगीभिः सहितान् मृगान्}
{मां पुनर्मृगशावाक्ष्या वैदेह्या विरहीकृतम्}
{व्यथयन्तीव मे चित्तं संचरन्तस्ततस्ततः} %4-1-101

\twolineshloka
{अस्मिन् सानुनि रम्ये हि मत्तद्विजगणाकुले}
{पश्येयं यदि तां कान्तां ततः स्वस्ति भवेन्मम} %4-1-102

\twolineshloka
{जीवेयं खलु सौमित्रे मया सह सुमध्यमा}
{सेवेत यदि वैदेही पम्पायाः पवनं शुभम्} %4-1-103

\twolineshloka
{पद्मसौगन्धिकवहं शिवं शोकविनाशनम्}
{धन्या लक्ष्मण सेवन्ते पम्पाया वनमारुतम्} %4-1-104

\twolineshloka
{श्यामा पद्मपलाशाक्षी प्रिया विरहिता मया}
{कथं धारयति प्राणान् विवशा जनकात्मजा} %4-1-105

\twolineshloka
{किं नु वक्ष्यामि धर्मज्ञं राजानं सत्यवादिनम्}
{जनकं पृष्टसीतं तं कुशलं जनसंसदि} %4-1-106

\twolineshloka
{या मामनुगता मन्दं पित्रा प्रस्थापितं वनम्}
{सीता धर्मं समास्थाय क्व नु सा वर्तते प्रिया} %4-1-107

\twolineshloka
{तया विहीनः कृपणः कथं लक्ष्मण धारये}
{या मामनुगता राज्याद् भ्रष्टं विहतचेतसम्} %4-1-108

\twolineshloka
{तच्चार्वाञ्चितपद्माक्षं सुगन्धि शुभमव्रणम्}
{अपश्यतो मुखं तस्याः सीदतीव मतिर्मम} %4-1-109

\twolineshloka
{स्मितहास्यान्तरयुतं गुणवन्मधुरं हितम्}
{वैदेह्या वाक्यमतुलं कदा श्रोष्यामि लक्ष्मण} %4-1-110

\twolineshloka
{प्राप्य दुःखं वने श्यामा मां मन्मथविकर्शितम्}
{नष्टदुःखेव हृष्टेव साध्वी साध्वभ्यभाषत} %4-1-111

\twolineshloka
{किं नु वक्ष्याम्ययोध्यायां कौसल्यां हि नृपात्मज}
{क्व सा स्नुषेति पृच्छन्तीं कथं चापि मनस्विनीम्} %4-1-112

\twolineshloka
{गच्छ लक्ष्मण पश्य त्वं भरतं भ्रातृवत्सलम्}
{नह्यहं जीवितुं शक्तस्तामृते जनकात्मजाम्} %4-1-113

\twolineshloka
{इति रामं महात्मानं विलपन्तमनाथवत्}
{उवाच लक्ष्मणो भ्राता वचनं युक्तमव्ययम्} %4-1-114

\twolineshloka
{संस्तम्भ राम भद्रं ते मा शुचः पुरुषोत्तम}
{नेदृशानां मतिर्मन्दा भवत्यकलुषात्मनाम्} %4-1-115

\twolineshloka
{स्मृत्वा वियोगजं दुःखं त्यज स्नेहं प्रिये जने}
{अतिस्नेहपरिष्वङ्गाद् वर्तिरार्द्रापि दह्यते} %4-1-116

\twolineshloka
{यदि गच्छति पातालं ततोऽभ्यधिकमेव वा}
{सर्वथा रावणस्तात न भविष्यति राघव} %4-1-117

\twolineshloka
{प्रवृत्तिर्लभ्यतां तावत् तस्य पापस्य रक्षसः}
{ततो हास्यति वा सीतां निधनं वा गमिष्यति} %4-1-118

\twolineshloka
{यदि याति दितेर्गर्भं रावणं सह सीतया}
{तत्राप्येनं हनिष्यामि न चेद् दास्यति मैथिलीम्} %4-1-119

\twolineshloka
{स्वास्थ्यं भद्रं भजस्वार्य त्यज्यतां कृपणा मतिः}
{अर्थो हि नष्टकार्यार्थैरयत्नेनाधिगम्यते} %4-1-120

\twolineshloka
{उत्साहो बलवानार्य नास्त्युत्साहात् परं बलम्}
{सोत्साहस्य हि लोकेषु न किंचिदपि दुर्लभम्} %4-1-121

\twolineshloka
{उत्साहवन्तः पुरुषा नावसीदन्ति कर्मसु}
{उत्साहमात्रमाश्रित्य प्रतिलप्स्याम जानकीम्} %4-1-122

\twolineshloka
{त्यजतां कामवृत्तत्वं शोकं संन्यस्य पृष्ठतः}
{महात्मानं कृतात्मानमात्मानं नावबुध्यसे} %4-1-123

\twolineshloka
{एवं सम्बोधितस्तेन शोकोपहतचेतनः}
{त्यज्य शोकं च मोहं च रामो धैर्यमुपागमत्} %4-1-124

\twolineshloka
{सोऽभ्यतिक्रामदव्यग्रस्तामचिन्त्यपराक्रमः}
{रामः पम्पां सुरुचिरां रम्यां पारिप्लवद्रुमाम्} %4-1-125

\twolineshloka
{निरीक्षमाणः सहसा महात्मा सर्वं वनं निर्झरकन्दरं च}
{उद्विग्नचेताः सह लक्ष्मणेन विचार्य दुःखोपहतः प्रतस्थे} %4-1-126

\twolineshloka
{तं मत्तमातङ्गविलासगामी गच्छन्तमव्यग्रमना महात्मा}
{स लक्ष्मणो राघवमिष्टचेष्टो ररक्ष धर्मेण बलेन चैव} %4-1-127

\twolineshloka
{तावृष्यमूकस्य समीपचारी चरन् ददर्शाद्भुतदर्शनीयौ}
{शाखामृगाणामधिपस्तरस्वी वितत्रसे नैव विचेष्ट चेष्टाम्} %4-1-128

\twolineshloka
{स तौ महात्मा गजमन्दगामी शाखामृगस्तत्र चरंश्चरन्तौ}
{दृष्ट्वा विषादं परमं जगाम चिन्तापरीतो भयभारभग्नः} %4-1-129

\twolineshloka
{तमाश्रमं पुण्यसुखं शरण्यं सदैव शाखामृगसेवितान्तम्}
{त्रस्ताश्च दृष्ट्वा हरयोऽभिजग्मुर्महौजसौ राघवलक्ष्मणौ तौ} %4-1-130


॥इत्यार्षे श्रीमद्रामायणे वाल्मीकीये आदिकाव्ये किष्किन्धाकाण्डे रामविप्रलम्भावेशः नाम प्रथमः सर्गः ॥४-१॥
