\sect{अष्टमः सर्गः — वालिवधप्रतिज्ञा}

\twolineshloka
{परितुष्टस्तु सुग्रीवस्तेन वाक्येन हर्षितः}
{लक्ष्मणस्याग्रजं शूरमिदं वचनमब्रवीत्} %4-8-1

\twolineshloka
{सर्वथाहमनुग्राह्यो देवतानां न संशयः}
{उपपन्नो गुणोपेतः सखा यस्य भवान् मम} %4-8-2

\twolineshloka
{शक्यं खलु भवेद् राम सहायेन त्वयानघ}
{सुरराज्यमपि प्राप्तुं स्वराज्यं किमुत प्रभो} %4-8-3

\twolineshloka
{सोऽहं सभाज्यो बन्धूनां सुहृदां चैव राघव}
{यस्याग्निसाक्षिकं मित्रं लब्धं राघववंशजम्} %4-8-4

\twolineshloka
{अहमप्यनुरूपस्ते वयस्यो ज्ञास्यसे शनैः}
{न तु वक्तुं समर्थोऽहं त्वयि आत्मगतान् गुणान्} %4-8-5

\twolineshloka
{महात्मनां तु भूयिष्ठं त्वद्विधानां कृतात्मनाम्}
{निश्चला भवति प्रीतिर्धैर्यमात्मवतां वर} %4-8-6

\twolineshloka
{रजतं वा सुवर्णं वा शुभान्याभरणानि च}
{अविभक्तानि साधूनामवगच्छन्ति साधवः} %4-8-7

\twolineshloka
{आढ्योवापि दरिद्रो वा दुःखितः सुखितोऽपि वा}
{निर्दोषश्च सदोषश्च वयस्यः परमा गतिः} %4-8-8

\twolineshloka
{धनत्यागः सुखत्यागो देशत्यागोऽपि वानघ}
{वयस्यार्थे प्रवर्तन्ते स्नेहं दृष्ट्वा तथाविधम्} %4-8-9

\twolineshloka
{तत् तथेत्यब्रवीद् रामः सुग्रीवं प्रियवादिनम्}
{लक्ष्मणस्याग्रतो लक्ष्म्या वासवस्येव धीमतः} %4-8-10

\twolineshloka
{ततो रामं स्थितं दृष्ट्वा लक्ष्मणं च महाबलम्}
{सुग्रीवः सर्वतश्चक्षुर्वने लोलमपातयत्} %4-8-11

\twolineshloka
{स ददर्श ततः सालमविदूरे हरीश्वरः}
{सुपुष्पमीषत्पत्राढ्यं भ्रमरैरुपशोभितम्} %4-8-12

\twolineshloka
{तस्यैकां पर्णबहुलां शाखां भङ्क्त्वा सुशोभिताम्}
{रामस्यास्तीर्य सुग्रीवो निषसाद सराघवः} %4-8-13

\twolineshloka
{तावासीनौ ततो दृष्ट्वा हनूमानपि लक्ष्मणम्}
{शालशाखां समुत्पाट्य विनीतमुपवेशयत्} %4-8-14

\twolineshloka
{सुखोपविष्टं रामं तु प्रसन्नमुदधिं यथा}
{सालपुष्पावसंकीर्णे तस्मिन् गिरिवरोत्तमे} %4-8-15

\twolineshloka
{ततः प्रहृष्टः सुग्रीवः श्लक्ष्णया शुभया गिरा}
{उवाच प्रणयाद् रामं हर्षव्याकुलिताक्षरम्} %4-8-16

\twolineshloka
{अहं विनिकृतो भ्रात्रा चराम्येष भयार्दितः}
{ऋष्यमूकं गिरिवरं हृतभार्यः सुदुःखितः} %4-8-17

\twolineshloka
{सोऽहं त्रस्तो भये मग्नो वने सम्भ्रान्तचेतनः}
{वालिना निकृतो भ्रात्रा कृतवैरश्च राघव} %4-8-18

\twolineshloka
{वालिनो मे भयार्तस्य सर्वलोकाभयंकर}
{ममापि त्वमनाथस्य प्रसादं कर्तुमर्हसि} %4-8-19

\twolineshloka
{एवमुक्तस्तु तेजस्वी धर्मज्ञो धर्मवत्सलः}
{प्रत्युवाच स काकुत्स्थः सुग्रीवं प्रहसन्निव} %4-8-20

\twolineshloka
{उपकारफलं मित्रमपकारोऽरिलक्षणम्}
{अद्यैव तं वधिष्यामि तव भार्यापहारिणम्} %4-8-21

\twolineshloka
{इमे हि मे महाभाग पत्रिणस्तिग्मतेजसः}
{कार्तिकेयवनोद्भूताः शरा हेमविभूषिताः} %4-8-22

\twolineshloka
{कङ्कपत्रपरिच्छन्ना महेन्द्राशनिसंनिभाः}
{सुपर्वाणः सुतीक्ष्णाग्राः सरोषा भुजगा इव} %4-8-23

\twolineshloka
{वालिसंज्ञममित्रं ते भ्रातरं कृतकिल्बिषम्}
{शरैर्विनिहतं पश्य विकीर्णमिव पर्वतम्} %4-8-24

\twolineshloka
{राघवस्य वचः श्रुत्वा सुग्रीवो वाहिनीपतिः}
{प्रहर्षमतुलं लेभे साधु साध्विति चाब्रवीत्} %4-8-25

\twolineshloka
{राम शोकाभिभूतोऽहं शोकार्तानां भवान् गतिः}
{वयस्य इति कृत्वा हि त्वय्यहं परिदेवये} %4-8-26

\twolineshloka
{त्वं हि पाणिप्रदानेन वयस्यो मेऽग्निसाक्षिकम्}
{कृतः प्राणैर्बहुमतः सत्येन च शपाम्यहम्} %4-8-27

\twolineshloka
{वयस्य इति कृत्वा च विस्रब्धः प्रवदाम्यहम्}
{दुःखमन्तर्गतं तन्मे मनो हरति नित्यशः} %4-8-28

\twolineshloka
{एतावदुक्त्वा वचनं बाष्पदूषितलोचनः}
{बाष्पदूषितया वाचा नोच्चैः शक्नोति भाषितुम्} %4-8-29

\twolineshloka
{बाष्पवेगं तु सहसा नदीवेगमिवागतम्}
{धारयामास धैर्येण सुग्रीवो रामसंनिधौ} %4-8-30

\twolineshloka
{स निगृह्य तु तं बाष्पं प्रमृज्य नयने शुभे}
{विनिःश्वस्य च तेजस्वी राघवं पुनरूचिवान्} %4-8-31

\twolineshloka
{पुराहं वालिना राम राज्यात् स्वादवरोपितः}
{परुषाणि च संश्राव्य निर्धूतोऽस्मि बलीयसा} %4-8-32

\twolineshloka
{हृता भार्या च मे तेन प्राणेभ्योऽपि गरीयसी}
{सुहृदश्च मदीया ये संयता बन्धनेषु ते} %4-8-33

\twolineshloka
{यत्नवांश्च स दुष्टात्मा मद्विनाशाय राघव}
{बहुशस्तप्रयुक्ताश्च वानरा निहता मया} %4-8-34

\twolineshloka
{शङ्कया त्वेतयाहं च दृष्ट्वा त्वामपि राघव}
{नोपसर्पाम्यहं भीतो भये सर्वे हि बिभ्यति} %4-8-35

\twolineshloka
{केवलं हि सहाया मे हनुमत्प्रमुखास्त्विमे}
{अतोऽहं धारयाम्यद्य प्राणान् कृच्छ्रगतोऽपि सन्} %4-8-36

\twolineshloka
{एते हि कपयः स्निग्धा मां रक्षन्ति समन्ततः}
{सह गच्छन्ति गन्तव्ये नित्यं तिष्ठन्ति चास्थिते} %4-8-37

\twolineshloka
{संक्षेपस्त्वेष मे राम किमुक्त्वा विस्तरं हि ते}
{स मे ज्येष्ठो रिपुर्भ्राता वाली विश्रुतपौरुषः} %4-8-38

\twolineshloka
{तद्विनाशेऽपि मे दुःखं प्रमृष्टं स्यादनन्तरम्}
{सुखं मे जीवितं चैव तद्विनाशनिबन्धनम्} %4-8-39

\twolineshloka
{एष मे राम शोकान्तः शोकार्तेन निवेदितः}
{दुःखितः सुखितो वापि सख्युर्नित्यं सखा गतिः} %4-8-40

\twolineshloka
{श्रुत्वैतच्च वचो रामः सुग्रीवमिदमब्रवीत्}
{किं निमित्तमभूद् वैरं श्रोतुमिच्छामि तत्त्वतः} %4-8-41

\twolineshloka
{सुखं हि कारणं श्रुत्वा वैरस्य तव वानर}
{आनन्तर्याद् विधास्यामि सम्प्रधार्य बलाबलम्} %4-8-42

\twolineshloka
{बलवान् हि ममामर्षः श्रुत्वा त्वामवमानितम्}
{वर्धते हृदयोत्कम्पी प्रावृड्वेग इवाम्भसः} %4-8-43

\twolineshloka
{हृष्टः कथय विस्रब्धो यावदारोप्यते धनुः}
{सृष्टश्च हि मया बाणो निरस्तश्च रिपुस्तव} %4-8-44

\twolineshloka
{एवमुक्तस्तु सुग्रीवः काकुत्स्थेन महात्मना}
{प्रहर्षमतुलं लेभे चतुर्भिः सह वानरैः} %4-8-45

\twolineshloka
{ततः प्रहृष्टवदनः सुग्रीवो लक्ष्मणाग्रजे}
{वैरस्य कारणं तत्त्वमाख्यातुमुपचक्रमे} %4-8-46


॥इत्यार्षे श्रीमद्रामायणे वाल्मीकीये आदिकाव्ये किष्किन्धाकाण्डे वालिवधप्रतिज्ञा नाम अष्टमः सर्गः ॥४-८॥
