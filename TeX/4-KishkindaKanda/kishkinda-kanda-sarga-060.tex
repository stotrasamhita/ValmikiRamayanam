\sect{षष्ठितमः सर्गः — सम्पातिपुरावृत्तवर्णनम्}

\twolineshloka
{ततः कृतोदकं स्नातं तं गृध्रं हरियूथपाः}
{उपविष्टा गिरौ रम्ये परिवार्य समन्ततः} %4-60-1

\twolineshloka
{तमङ्गदमुपासीनं तैः सर्वैर्हरिभिर्वृतम्}
{जनितप्रत्ययो हर्षात् सम्पातिः पुनरब्रवीत्} %4-60-2

\twolineshloka
{कृत्वा निःशब्दमेकाग्राः शृण्वन्तु हरयो मम}
{तथ्यं सङ्कीर्तयिष्यामि यथा जानामि मैथिलीम्} %4-60-3

\twolineshloka
{अस्य विन्ध्यस्य शिखरे पतितोऽस्मि पुरानघ}
{सूर्यतापपरीताङ्गो निर्दग्धः सूर्यरश्मिभिः} %4-60-4

\twolineshloka
{लब्धसंज्ञस्तु षड्रात्राद् विवशो विह्वलन्निव}
{वीक्षमाणो दिशः सर्वा नाभिजानामि किञ्चन} %4-60-5

\twolineshloka
{ततस्तु सागरान् शैलान् नदीः सर्वाः सरांसि च}
{वनानि च प्रदेशांश्च निरीक्ष्य मतिरागता} %4-60-6

\twolineshloka
{हृष्टपक्षिगणाकीर्णः कन्दरोदरकूटवान्}
{दक्षिणस्योदधेस्तीरे विन्ध्योऽयमिति निश्चितः} %4-60-7

\twolineshloka
{आसीच्चात्राश्रमं पुण्यं सुरैरपि सुपूजितम्}
{ऋषिर्निशाकरो नाम यस्मिन्नुग्रतपाऽभवत्} %4-60-8

\twolineshloka
{अष्टौ वर्षसहस्राणि तेनास्मिन्नृषिणा गिरौ}
{वसतो मम धर्मज्ञे स्वर्गते तु निशाकरे} %4-60-9

\twolineshloka
{अवतीर्य च विन्ध्याग्रात् कृच्छ्रेण विषमाच्छनैः}
{तीक्ष्णदर्भां वसुमतीं दुःखेन पुनरागतः} %4-60-10

\twolineshloka
{तमृषिं द्रष्टुकामोऽस्मि दुःखेनाभ्यागतो भृशम्}
{जटायुषा मया चैव बहुशोऽधिगतो हि सः} %4-60-11

\twolineshloka
{तस्याश्रमपदाभ्याशे ववुर्वाताः सुगन्धिनः}
{वृक्षो नापुष्पितः कश्चिदफलो वा न दृश्यते} %4-60-12

\twolineshloka
{उपेत्य चाश्रमं पुण्यं वृक्षमूलमुपाश्रितः}
{द्रष्टुकामः प्रतीक्षे च भगवन्तं निशाकरम्} %4-60-13

\twolineshloka
{अथ पश्यामि दूरस्थमृषिं ज्वलिततेजसम्}
{कृताभिषेकं दुर्धर्षमुपावृत्तमुदङ्मुखम्} %4-60-14

\twolineshloka
{तमृक्षाः सृमरा व्याघ्राः सिंहा नानासरीसृपाः}
{परिवार्योपगच्छन्ति दातारं प्राणिनो यथा} %4-60-15

\twolineshloka
{ततः प्राप्तमृषिं ज्ञात्वा तानि सत्त्वानि वै ययुः}
{प्रविष्टे राजनि यथा सर्वं सामात्यकं बलम्} %4-60-16

\twolineshloka
{ऋषिस्तु दृष्ट्वा मां तुष्टः प्रविष्टश्चाश्रमं पुनः}
{मुहूर्तमात्रान्निर्गम्य ततः कार्यमपृच्छत} %4-60-17

\twolineshloka
{सौम्य वैकल्यतां दृष्ट्वा रोम्णां ते नावगम्यते}
{अग्निदग्धाविमौ पक्षौ प्राणाश्चापि शरीरके} %4-60-18

\twolineshloka
{गृध्रौ द्वौ दृष्टपूर्वौ मे मातरिश्वसमौ जवे}
{गृध्राणां चैव राजानौ भ्रातरौ कामरूपिणौ} %4-60-19

\twolineshloka
{ज्येष्ठोऽवितस्त्वं सम्पाते जटायुरनुजस्तव}
{मानुषं रूपमास्थाय गृह्णीतां चरणौ मम} %4-60-20

\twolineshloka
{किं ते व्याधिसमुत्थानं पक्षयोः पतनं कथम्}
{दण्डो वायं धृतः केन सर्वमाख्याहि पृच्छतः} %4-60-21


॥इत्यार्षे श्रीमद्रामायणे वाल्मीकीये आदिकाव्ये किष्किन्धाकाण्डे सम्पातिपुरावृत्तवर्णनम् नाम षष्ठितमः सर्गः ॥४-६०॥
