\sect{षड्विंशः सर्गः — सुग्रीवाभिषेकः}

\twolineshloka
{ततः शोकाभिसंतप्तं सुग्रीवं क्लिन्नवाससम्}
{शाखामृगमहामात्राः परिवार्योपतस्थिरे} %4-26-1

\twolineshloka
{अभिगम्य महाबाहुं राममक्लिष्टकारिणम्}
{स्थिताः प्राञ्जलयः सर्वे पितामहमिवर्षयः} %4-26-2

\twolineshloka
{ततः काञ्चनशैलाभस्तरुणार्कनिभाननः}
{अब्रवीत् प्राञ्जलिर्वाक्यं हनूमान् मारुतात्मजः} %4-26-3

\twolineshloka
{भवत्प्रसादात् काकुत्स्थ पितृपैतामहं महत्}
{वानराणां सुदंष्ट्राणां सम्पन्नबलशालिनाम्} %4-26-4

\twolineshloka
{महात्मनां सुदुष्प्रापं प्राप्तं राज्यमिदं प्रभो}
{भवता समनुज्ञातः प्रविश्य नगरं शुभम्} %4-26-5

\twolineshloka
{संविधास्यति कार्याणि सर्वाणि ससुहृद्गणः}
{स्नातोऽयं विविधैर्गन्धैरौषधैश्च यथाविधि} %4-26-6

\twolineshloka
{अर्चयिष्यति माल्यैश्च रत्नैश्च त्वां विशेषतः}
{इमां गिरिगुहां रम्यामभिगन्तुं त्वमर्हसि} %4-26-7

\twolineshloka
{कुरुष्व स्वामिसम्बन्धं वानरान् सम्प्रहर्षय}
{एवमुक्तो हनुमता राघवः परवीरहा} %4-26-8

\twolineshloka
{प्रत्युवाच हनूमन्तं बुद्धिमान् वाक्यकोविदः}
{चतुर्दश समाः सौम्य ग्रामं वा यदि वा पुरम्} %4-26-9

\twolineshloka
{न प्रवेक्ष्यामि हनुमन् पितुर्निर्देशपालकः}
{सुसमृद्धां गुहां दिव्यां सुग्रीवो वानरर्षभः} %4-26-10

\twolineshloka
{प्रविष्टो विधिवद् वीरः क्षिप्रं राज्येऽभिषिच्यताम्}
{एवमुक्त्वा हनूमन्तं रामः सुग्रीवमब्रवीत्} %4-26-11

\twolineshloka
{वृत्तज्ञो वृत्तसम्पन्नमुदारबलविक्रमम्}
{इममप्यङ्गदं वीरं यौवराज्येऽभिषेचय} %4-26-12

\twolineshloka
{ज्येष्ठस्य हि सुतो ज्येष्ठः सदृशो विक्रमेण च}
{अङ्गदोऽयमदीनात्मा यौवराज्यस्य भाजनम्} %4-26-13

\twolineshloka
{पूर्वोऽयं वार्षिको मासः श्रावणः सलिलागमः}
{प्रवृत्ताः सौम्य चत्वारो मासा वार्षिक संज्ञिताः} %4-26-14

\twolineshloka
{नायमुद्योगसमयः प्रविश त्वं पुरीं शुभाम्}
{अस्मिन् वत्स्याम्यहं सौम्य पर्वते सहलक्ष्मणः} %4-26-15

\twolineshloka
{इयं गिरिगुहा रम्या विशाला युक्तमारुता}
{प्रभूतसलिला सौम्य प्रभूतकमलोत्पला} %4-26-16

\twolineshloka
{कार्तिके समनुप्राप्ते त्वं रावणवधे यत}
{एष नः समयः सौम्य प्रविश त्वं स्वमालयम्} %4-26-17

\twolineshloka
{अभिषिञ्चस्व राज्ये च सुहृदः सम्प्रहर्षय}
{इति रामाभ्यनुज्ञातः सुग्रीवो वानरर्षभः} %4-26-18

\twolineshloka
{प्रविवेश पुरीं रम्यां किष्किन्धां वालिपालिताम्}
{तं वानरसहस्राणि प्रविष्टं वानरेश्वरम्} %4-26-19

\twolineshloka
{अभिवार्य प्रविष्टानि सर्वतः प्लवगेश्वरम्}
{ततः प्रकृतयः सर्वा दृष्ट्वा हरिगणेश्वरम्} %4-26-20

\twolineshloka
{प्रणम्य मूर्ध्ना पतिता वसुधायं समाहिताः}
{सुग्रीवः प्रकृतीः सर्वाः सम्भाष्योत्थाप्य वीर्यवान्} %4-26-21

\twolineshloka
{भ्रातुरन्तःपुरं सौम्यं प्रविवेश महाबलः}
{प्रविष्टं भीमविक्रान्तं सुग्रीवं वानरर्षभम्} %4-26-22

\twolineshloka
{अभ्यषिञ्चन्त सुहृदः सहस्राक्षमिवामराः}
{तस्य पाण्डुरमाजह्रुश्छत्रं हेमपरिष्कृतम्} %4-26-23

\twolineshloka
{शुक्ले च वालव्यजने हेमदण्डे यशस्करे}
{तथा रत्नानि सर्वाणि सर्वबीजौषधानि च} %4-26-24

\twolineshloka
{सक्षीराणां च वृक्षाणां प्ररोहान् कुसुमानि च}
{शुक्लानि चैव वस्त्राणि श्वेतं चैवानुलेपनम्} %4-26-25

\twolineshloka
{सुगन्धीनि च माल्यानि स्थलजान्यम्बुजानि च}
{चन्दनानि च दिव्यानि गन्धांश्च विविधान् बहून्} %4-26-26

\twolineshloka
{अक्षतं जातरूपं च प्रियङ्गुं मधुसर्पिषी}
{दधि चर्म च वैयाघ्रं परार्घ्यौ चाप्युपानहौ} %4-26-27

\twolineshloka
{समालम्भनमादाय गोरोचनमनःशिलाम्}
{आजग्मुस्तत्र मुदिता वराः कन्याश्च षोडश} %4-26-28

\twolineshloka
{ततस्ते वानरश्रेष्ठमभिषेक्तुं यथाविधि}
{रत्नैर्वस्त्रैश्च भक्ष्यैश्च तोषयित्वा द्विजर्षभान्} %4-26-29

\twolineshloka
{ततः कुशपरिस्तीर्णं समिद्धं जातवेदसम्}
{मन्त्रपूतेन हविषा हुत्वा मन्त्रविदो जनाः} %4-26-30

\twolineshloka
{ततो हेमप्रतिष्ठाने वरास्तरणसंवृते}
{प्रासादशिखरे रम्ये चित्रमाल्योपशोभिते} %4-26-31

\twolineshloka
{प्राङ्मुखं विधिवन्मन्त्रैः स्थापयित्वा वरासने}
{नदीनदेभ्यः संहृत्य तीर्थेभ्यश्च समन्ततः} %4-26-32

\twolineshloka
{आहृत्य च समुद्रेभ्यः सर्वेभ्यो वानरर्षभाः}
{अपः कनककुम्भेषु निधाय विमलं जलम्} %4-26-33

\twolineshloka
{शुभैर्ऋषभशृङ्गैश्च कलशैश्चैव काञ्चनैः}
{शास्त्रदृष्टेन विधिना महर्षिविहितेन च} %4-26-34

\twolineshloka
{गजो गवाक्षो गवयः शरभो गन्धमादनः}
{मैन्दश्च द्विविदश्चैव हनूमाञ्जाम्बवांस्तथा} %4-26-35

\twolineshloka
{अभ्यषिञ्चत सुग्रीवं प्रसन्नेन सुगन्धिना}
{सलिलेन सहस्राक्षं वसवो वासवं यथा} %4-26-36

\twolineshloka
{अभिषिक्ते तु सुग्रीवे सर्वे वानरपुङ्गवाः}
{प्रचुक्रुशुर्महात्मानो हृष्टाः शतसहस्रशः} %4-26-37

\twolineshloka
{रामस्य तु वचः कुर्वन् सुग्रीवो वानरेश्वरः}
{अङ्गदं सम्परिष्वज्य यौवराज्येऽभ्यषेचयत्} %4-26-38

\twolineshloka
{अङ्गदे चाभिषिक्ते तु सानुक्रोशाः प्लवंगमाः}
{साधु साध्विति सुग्रीवं महात्मानो ह्यपूजयन्} %4-26-39

\twolineshloka
{रामं चैव महात्मानं लक्ष्मणं च पुनः पुनः}
{प्रीताश्च तुष्टुवुः सर्वे तादृशे तत्र वर्तिनि} %4-26-40

\twolineshloka
{हृष्टपुष्टजनाकीर्णा पताकाध्वजशोभिता}
{बभूव नगरी रम्या किष्किन्धा गिरिगह्वरे} %4-26-41

\twolineshloka
{निवेद्य रामाय तदा महात्मने महाभिषेकं कपिवाहिनीपतिः}
{रुमां च भार्यामुपलभ्य वीर्यवानवाप राज्यं त्रिदशाधिपो यथा} %4-26-42


॥इत्यार्षे श्रीमद्रामायणे वाल्मीकीये आदिकाव्ये किष्किन्धाकाण्डे सुग्रीवाभिषेकः नाम षड्विंशः सर्गः ॥४-२६॥
