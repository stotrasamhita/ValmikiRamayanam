\sect{चतुःषष्ठितमः सर्गः — समुद्रलङ्घनमन्त्रणम्}

\twolineshloka
{आख्याता गृध्रराजेन समुत्प्लुत्य प्लवङ्गमाः}
{संगताः प्रीतिसंयुक्ता विनेदुः सिंहविक्रमाः} %4-64-1

\twolineshloka
{सम्पातेर्वचनं श्रुत्वा हरयो रावणक्षयम्}
{हृष्टाः सागरमाजग्मुः सीतादर्शनकांक्षिणः} %4-64-2

\twolineshloka
{अभिगम्य तु तं देशं ददृशुर्भीमविक्रमाः}
{कृत्स्नं लोकस्य महतः प्रतिबिम्बमवस्थितम्} %4-64-3

\twolineshloka
{दक्षिणस्य समुद्रस्य समासाद्योत्तरां दिशम्}
{संनिवेशं ततश्चक्रुर्हरिवीरा महाबलाः} %4-64-4

\twolineshloka
{प्रसुप्तमिव चान्यत्र क्रीडन्तमिव चान्यतः}
{क्वचित् पर्वतमात्रैश्च जलराशिभिरावृतम्} %4-64-5

\twolineshloka
{संकुलं दानवेन्द्रैश्च पातालतलवासिभिः}
{रोमहर्षकरं दृष्ट्वा विषेदुः कपिकुञ्जराः} %4-64-6

\twolineshloka
{आकाशमिव दुष्पारं सागरं प्रेक्ष्य वानराः}
{विषेदुः सहिताः सर्वे कथं कार्यमिति ब्रुवन्} %4-64-7

\twolineshloka
{विषण्णां वाहिनीं दृष्ट्वा सागरस्य निरीक्षणात्}
{आश्वासयामास हरीन् भयार्तान् हरिसत्तमः} %4-64-8

\twolineshloka
{न विषादे मनः कार्यं विषादो दोषवत्तरः}
{विषादो हन्ति पुरुषं बालं क्रुद्ध इवोरगः} %4-64-9

\twolineshloka
{यो विषादं प्रसहते विक्रमे समुपस्थिते}
{तेजसा तस्य हीनस्य पुरुषार्थो न सिद्ध्यति} %4-64-10

\twolineshloka
{तस्यां रात्र्यां व्यतीतायामङ्गदो वानरैः सह}
{हरिवृद्धैः समागम्य पुनर्मन्त्रममन्त्रयत्} %4-64-11

\twolineshloka
{सा वानराणां ध्वजिनी परिवार्याङ्गदं बभौ}
{वासवं परिवार्येव मरुतां वाहिनी स्थिता} %4-64-12

\twolineshloka
{कोऽन्यस्तां वानरीं सेनां शक्तः स्तम्भयितुं भवेत्}
{अन्यत्र वालितनयादन्यत्र च हनूमतः} %4-64-13

\twolineshloka
{ततस्तान् हरिवृद्धांश्च तच्च सैन्यमरिंदमः}
{अनुमान्याङ्गदः श्रीमान् वाक्यमर्थवदब्रवीत्} %4-64-14

\twolineshloka
{क इदानीं महातेजा लङ्घयिष्यति सागरम्}
{कः करिष्यति सुग्रीवं सत्यसंधमरिंदमम्} %4-64-15

\twolineshloka
{को वीरो योजनशतं लङ्घयेत प्लवङ्गमः}
{इमांश्च यूथपान् सर्वान् मोचयेत् को महाभयात्} %4-64-16

\twolineshloka
{कस्य प्रसादाद् दारांश्च पुत्रांश्चैव गृहाणि च}
{इतो निवृत्ताः पश्येम सिद्धार्थाः सुखिनो वयम्} %4-64-17

\twolineshloka
{कस्य प्रसादाद् रामं च लक्ष्मणं च महाबलम्}
{अभिगच्छेम संहृष्टाः सुग्रीवं च वनौकसम्} %4-64-18

\twolineshloka
{यदि कश्चित् समर्थो वः सागरप्लवने हरिः}
{स ददात्विह नः शीघ्रं पुण्यामभयदक्षिणाम्} %4-64-19

\twolineshloka
{अङ्गदस्य वचः श्रुत्वा न कश्चित् किंचिदब्रवीत्}
{स्तिमितेवाभवत् सर्वा सा तत्र हरिवाहिनी} %4-64-20

\threelineshloka
{पुनरेवाङ्गदः प्राह तान् हरीन् हरिसत्तमः}
{सर्वे बलवतां श्रेष्ठा भवन्तो दृढविक्रमाः}
{व्यपदेशकुले जाताः पूजिताश्चाप्यभीक्ष्णशः} %4-64-21

\twolineshloka
{नहि वो गमने भङ्गः कदाचित् कस्यचिद् भवेत्}
{ब्रुवध्वं यस्य या शक्तिः प्लवने प्लवगर्षभाः} %4-64-22


॥इत्यार्षे श्रीमद्रामायणे वाल्मीकीये आदिकाव्ये किष्किन्धाकाण्डे समुद्रलङ्घनमन्त्रणम् नाम चतुःषष्ठितमः सर्गः ॥४-६४॥
