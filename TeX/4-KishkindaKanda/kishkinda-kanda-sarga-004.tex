\sect{चतुर्थः सर्गः — सुग्रीवसमीपगमनम्}

\twolineshloka
{ततः प्रहृष्टो हनुमान् कृत्यवानिति तद्वचः}
{श्रुत्वा मधुरभावं च सुग्रीवं मनसा गतः} %4-4-1

\twolineshloka
{भाव्यो राज्यागमस्तस्य सुग्रीवस्य महात्मनः}
{यदयं कृत्यवान् प्राप्तः कृत्यं चैतदुपागतम्} %4-4-2

\twolineshloka
{ततः परमसंहृष्टो हनूमान् प्लवगोत्तमः}
{प्रत्युवाच ततो वाक्यं रामं वाक्यविशारदः} %4-4-3

\twolineshloka
{किमर्थं त्वं वनं घोरं पम्पाकाननमण्डितम्}
{आगतः सानुजो दुर्गं नानाव्यालमृगायुतम्} %4-4-4

\twolineshloka
{तस्य तद् वचनं श्रुत्वा लक्ष्मणो रामचोदितः}
{आचचक्षे महात्मानं रामं दशरथात्मजम्} %4-4-5

\twolineshloka
{राजा दशरथो नाम द्युतिमान् धर्मवत्सलः}
{चातुर्वर्ण्यं स्वधर्मेण नित्यमेवाभिपालयन्} %4-4-6

\twolineshloka
{न द्वेष्टा विद्यते तस्य स तु द्वेष्टि न कंचन}
{स तु सर्वेषु भूतेषु पितामह इवापरः} %4-4-7

\twolineshloka
{अग्निष्टोमादिभिर्यज्ञैरिष्टवानाप्तदक्षिणैः}
{तस्यायं पूर्वजः पुत्रो रामो नाम जनैः श्रुतः} %4-4-8

\twolineshloka
{शरण्यः सर्वभूतानां पितुर्निर्देशपारगः}
{ज्येष्ठो दशरथस्यायं पुत्राणां गुणवत्तरः} %4-4-9

\twolineshloka
{राजलक्षणसंयुक्तः संयुक्तो राज्यसम्पदा}
{राज्याद् भ्रष्टो मया वस्तुं वने सार्धमिहागतः} %4-4-10

\twolineshloka
{भार्यया च महाभाग सीतयानुगतो वशी}
{दिनक्षये महातेजाः प्रभयेव दिवाकरः} %4-4-11

\twolineshloka
{अहमस्यावरो भ्राता गुणैर्दास्यमुपागतः}
{कृतज्ञस्य बहुज्ञस्य लक्ष्मणो नाम नामतः} %4-4-12

\twolineshloka
{सुखार्हस्य महार्हस्य सर्वभूतहितात्मनः}
{ऐश्वर्येण विहीनस्य वनवासे रतस्य च} %4-4-13

\twolineshloka
{रक्षसापहृता भार्या रहिते कामरूपिणा}
{तच्च न ज्ञायते रक्षः पत्नी येनास्य वा हृता} %4-4-14

\twolineshloka
{दनुर्नाम दितेः पुत्रः शापाद् राक्षसतां गतः}
{आख्यातस्तेन सुग्रीवः समर्थो वानराधिपः} %4-4-15

\twolineshloka
{स ज्ञास्यति महावीर्यस्तव भार्यापहारिणम्}
{एवमुक्त्वा दनुः स्वर्गं भ्राजमानो दिवं गतः} %4-4-16

\twolineshloka
{एतत् ते सर्वमाख्यातं याथातथ्येन पृच्छतः}
{अहं चैव च रामश्च सुग्रीवं शरणं गतौ} %4-4-17

\twolineshloka
{एष दत्त्वा च वित्तानि प्राप्य चानुत्तमं यशः}
{लोकनाथः पुरा भूत्वा सुग्रीवं नाथमिच्छति} %4-4-18

\twolineshloka
{सीता यस्य स्नुषा चासीच्छरण्यो धर्मवत्सलः}
{तस्य पुत्रः शरण्यश्च सुग्रीवं शरणं गतः} %4-4-19

\twolineshloka
{सर्वलोकस्य धर्मात्मा शरण्यः शरणं पुरा}
{गुरुर्मे राघवः सोऽयं सुग्रीवं शरणं गतः} %4-4-20

\twolineshloka
{यस्य प्रसादे सततं प्रसीदेयुरिमाः प्रजाः}
{स रामो वानरेन्द्रस्य प्रसादमभिकांक्षते} %4-4-21

\twolineshloka
{येन सर्वगुणोपेताः पृथिव्यां सर्वपार्थिवाः}
{मानिताः सततं राज्ञा सदा दशरथेन वै} %4-4-22

\twolineshloka
{तस्यायं पूर्वजः पुत्रस्त्रिषु लोकेषु विश्रुतः}
{सुग्रीवं वानरेन्द्रं तु रामः शरणमागतः} %4-4-23

\twolineshloka
{शोकाभिभूते रामे तु शोकार्ते शरणं गते}
{कर्तुमर्हति सुग्रीवः प्रसादं सह यूथपैः} %4-4-24

\twolineshloka
{एवं ब्रुवाणं सौमित्रिं करुणं साश्रुपातनम्}
{हनूमान् प्रत्युवाचेदं वाक्यं वाक्यविशारदः} %4-4-25

\twolineshloka
{ईदृशा बुद्धिसम्पन्ना जितक्रोधा जितेन्द्रियाः}
{द्रष्टव्या वानरेन्द्रेण दिष्ट्या दर्शनमागताः} %4-4-26

\twolineshloka
{स हि राज्याश्च विभ्रष्टः कृतवैरश्च वालिना}
{हृतदारो वने त्रस्तो भ्रात्रा विनिकृतो भृशम्} %4-4-27

\twolineshloka
{करिष्यति स साहाय्यं युवयोर्भास्करात्मजः}
{सुग्रीवः सह चास्माभिः सीतायाः परिमार्गणे} %4-4-28

\twolineshloka
{इत्येवमुक्त्वा हनुमान् श्लक्ष्णं मधुरया गिरा}
{बभाषे साधु गच्छामः सुग्रीवमिति राघवम्} %4-4-29

\twolineshloka
{एवं ब्रुवन्तं धर्मात्मा हनूमन्तं स लक्ष्मणः}
{प्रतिपूज्य यथान्यायमिदं प्रोवाच राघवम्} %4-4-30

\twolineshloka
{कपिः कथयते हृष्टो यथायं मारुतात्मजः}
{कृत्यवान् सोऽपि सम्प्राप्तः कृतकृत्योऽसि राघव} %4-4-31

\twolineshloka
{प्रसन्नमुखवर्णश्च व्यक्तं हृष्टश्च भाषते}
{नानृतं वक्ष्यते वीरो हनूमान् मारुतात्मजः} %4-4-32

\twolineshloka
{ततः स सुमहाप्राज्ञो हनूमान् मारुतात्मजः}
{जगामादाय तौ वीरौ हरिराजाय राघवौ} %4-4-33

\twolineshloka
{भिक्षुरूपं परित्यज्य वानरं रूपमास्थितः}
{पृष्ठमारोप्य तौ वीरौ जगाम कपिकुञ्जरः} %4-4-34

\twolineshloka
{स तु विपुलयशाः कपिप्रवीरः पवनसुतः कृतकृत्यवत् प्रहृष्टः}
{गिरिवरमुरुविक्रमः प्रयातः स शुभमतिः सह रामलक्ष्मणाभ्याम्} %4-4-35


॥इत्यार्षे श्रीमद्रामायणे वाल्मीकीये आदिकाव्ये किष्किन्धाकाण्डे सुग्रीवसमीपगमनम् नाम चतुर्थः सर्गः ॥४-४॥
