\sect{द्विपञ्चाशः सर्गः — बिलप्रवेशकारणकथनम्}

\twolineshloka
{अथ तानब्रवीत् सर्वान् विश्रान्तान् हरियूथपान्}
{इदं वचनमेकाग्रा तापसी धर्मचारिणी} %4-52-1

\twolineshloka
{वानरा यदि वः खेदः प्रणष्टः फलभक्षणात्}
{यदि चैतन्मया श्राव्यं श्रोतुमिच्छामि तां कथाम्} %4-52-2

\twolineshloka
{तस्यास्तद् वचनं श्रुत्वा हनूमान् मारुतात्मजः}
{आर्जवेन यथातत्त्वमाख्यातुमुपचक्रमे} %4-52-3

\twolineshloka
{राजा सर्वस्य लोकस्य महेन्द्रवरुणोपमः}
{रामो दाशरथिः श्रीमान् प्रविष्टो दण्डकावनम्} %4-52-4

\twolineshloka
{लक्ष्मणेन सह भ्रात्रा वैदेह्या सह भार्यया}
{तस्य भार्या जनस्थानाद् रावणेन हृता बलात्} %4-52-5

\twolineshloka
{वीरस्तस्य सखा राज्ञः सुग्रीवो नाम वानरः}
{राजा वानरमुख्यानां येन प्रस्थापिता वयम्} %4-52-6

\twolineshloka
{अगस्त्यचरितामाशां दक्षिणां यमरक्षिताम्}
{सहैभिर्वानरैर्मुख्यैरङ्गदप्रमुखैर्वयम्} %4-52-7

\twolineshloka
{रावणं सहिताः सर्वे राक्षसं कामरूपिणम्}
{सीतया सह वैदेह्या मार्गध्वमिति चोदिताः} %4-52-8

\twolineshloka
{विचित्य तु वनं सर्वं समुद्रं दक्षिणां दिशम्}
{वयं बुभुक्षिताः सर्वे वृक्षमूलमुपाश्रिताः} %4-52-9

\twolineshloka
{विवर्णवदनाः सर्वे सर्वे ध्यानपरायणाः}
{नाधिगच्छामहे पारं मग्नाश्चिन्तामहार्णवे} %4-52-10

\twolineshloka
{चारयन्तस्ततश्चक्षुर्दृष्टवन्तो महद् बिलम्}
{लतापादपसञ्छन्नं तिमिरेण समावृतम्} %4-52-11

\twolineshloka
{अस्माद्धंसा जलक्लिन्नाः पक्षैः सलिलरेणुभिः}
{कुरराः सारसाश्चैव निष्पतन्ति पतत्त्रिणः} %4-52-12

\twolineshloka
{साध्वत्र प्रविशामेति मया तूक्ताः प्लवङ्गमाः}
{तेषामपि हि सर्वेषामनुमानमुपागतम्} %4-52-13

\twolineshloka
{अस्मिन् निपतिताः सर्वेऽप्यथ कार्यत्वरान्विताः}
{ततो गाढं निपतिता गृह्य हस्तैः परस्परम्} %4-52-14

\twolineshloka
{इदं प्रविष्टाः सहसा बिलं तिमिरसंवृतम्}
{एतन्नः कार्यमेतेन कृत्येन वयमागताः} %4-52-15

\twolineshloka
{त्वां चैवोपगताः सर्वे परिद्यूना बुभुक्षिताः}
{आतिथ्यधर्मदत्तानि मूलानि च फलानि च} %4-52-16

\twolineshloka
{अस्माभिरुपयुक्तानि बुभुक्षापरिपीडितैः}
{यत् त्वया रक्षिताः सर्वे म्रियमाणा बुभुक्षया} %4-52-17

\twolineshloka
{ब्रूहि प्रत्युपकारार्थं किं ते कुर्वन्तु वानराः}
{एवमुक्ता तु सर्वज्ञा वानरैस्तैः स्वयम्प्रभा} %4-52-18

\twolineshloka
{प्रत्युवाच ततः सर्वानिदं वानरयूथपान्}
{सर्वेषां परितुष्टास्मि वानराणां तरस्विनाम्} %4-52-19

\twolineshloka
{चरन्त्या मम धर्मेण न कार्यमिह केनचित्}
{एवमुक्तः शुभं वाक्यं तापस्या धर्मसंहितम्} %4-52-20

\twolineshloka
{उवाच हनुमान् वाक्यं तामनिन्दितलोचनाम्}
{शरणं त्वां प्रपन्नाः स्मः सर्वे वै धर्मचारिणीम्} %4-52-21

\twolineshloka
{यः कृतः समयोऽस्मासु सुग्रीवेण महात्मना}
{स तु कालो व्यतिक्रान्तो बिले च परिवर्तताम्} %4-52-22

\twolineshloka
{सा त्वमस्माद् बिलादस्मानुत्तारयितुमर्हसि}
{तस्मात् सुग्रीववचनादतिक्रान्तान् गतायुषः} %4-52-23

\twolineshloka
{त्रातुमर्हसि नः सर्वान् सुग्रीवभयशङ्कितान्}
{महच्च कार्यमस्माभिः कर्तव्यं धर्मचारिणि} %4-52-24

\twolineshloka
{तच्चापि न कृतं कार्यमस्माभिरिह वासिभिः}
{एवमुक्ता हनुमता तापसी वाक्यमब्रवीत्} %4-52-25

\twolineshloka
{जीवता दुष्करं मन्ये प्रविष्टेन निवर्तितुम्}
{तपसः सुप्रभावेण नियमोपार्जितेन च} %4-52-26

\twolineshloka
{सर्वानेव बिलादस्मात् तारयिष्यामि वानरान्}
{निमीलयत चक्षूंषि सर्वे वानरपुङ्गवाः} %4-52-27

\twolineshloka
{नहि निष्क्रमितुं शक्यमनिमीलितलोचनैः}
{ततो निमीलिताः सर्वे सुकुमाराङ्गुलैः करैः} %4-52-28

\twolineshloka
{सहसा पिदधुर्दृष्टिं हृष्टा गमनकाङ्क्षया}
{वानरास्तु महात्मानो हस्तरुद्धमुखास्तदा} %4-52-29

\twolineshloka
{निमेषान्तरमात्रेण बिलादुत्तारितास्तया}
{उवाच सर्वांस्तांस्तत्र तापसी धर्मचारिणी} %4-52-30

\twolineshloka
{निःसृतान् विषमात् तस्मात् समाश्वास्येदमब्रवीत्}
{एष विन्ध्यो गिरिः श्रीमान् नानाद्रुमलतायुतः} %4-52-31

\threelineshloka
{एष प्रस्रवणः शैलः सागरोऽयं महोदधिः}
{स्वस्ति वोऽस्तु गमिष्यामि भवनं वानरर्षभाः}
{इत्युक्त्वा तद् बिलं श्रीमत् प्रविवेश स्वयम्प्रभा} %4-52-32


॥इत्यार्षे श्रीमद्रामायणे वाल्मीकीये आदिकाव्ये किष्किन्धाकाण्डे बिलप्रवेशकारणकथनम् नाम द्विपञ्चाशः सर्गः ॥४-५२॥
