\sect{पञ्चमः सर्गः — सुग्रीवसख्यम्}

\twolineshloka
{ऋष्यमूकात् तु हनुमान् गत्वा तं मलयं गिरिम्}
{आचचक्षे तदा वीरौ कपिराजाय राघवौ} %4-5-1

\twolineshloka
{अयं रामो महाप्राज्ञ सम्प्राप्तो दृढविक्रमः}
{लक्ष्मणेन सह भ्रात्रा रामोऽयं सत्यविक्रमः} %4-5-2

\twolineshloka
{इक्ष्वाकूणां कुले जातो रामो दशरथात्मजः}
{धर्मे निगदितश्चैव पितुर्निर्देशकारकः} %4-5-3

\twolineshloka
{राजसूयाश्वमेधैश्च वह्निर्येनाभितर्पितः}
{दक्षिणाश्च तथोत्सृष्टा गावः शतसहस्रशः} %4-5-4

\twolineshloka
{तपसा सत्यवाक्येन वसुधा येन पालिता}
{स्त्रीहेतोस्तस्य पुत्रोऽयं रामोऽरण्यं समागतः} %4-5-5

\twolineshloka
{तस्यास्य वसतोऽरण्ये नियतस्य महात्मनः}
{रावणेन हृता भार्या स त्वां शरणमागतः} %4-5-6

\twolineshloka
{भवता सख्यकामौ तौ भ्रातरौ रामलक्ष्मणौ}
{प्रगृह्य चार्चयस्वैतौ पूजनीयतमावुभौ} %4-5-7

\twolineshloka
{श्रुत्वा हनूमतो वाक्यं सुग्रीवो वानराधिपः}
{दर्शनीयतमो भूत्वा प्रीत्योवाच च राघवम्} %4-5-8

\twolineshloka
{भवान् धर्मविनीतश्च सुतपाः सर्ववत्सलः}
{आख्याता वायुपुत्रेण तत्त्वतो मे भवद्गुणाः} %4-5-9

\twolineshloka
{तन्ममैवैष सत्कारो लाभश्चैवोत्तमः प्रभो}
{यत्त्वमिच्छसि सौहार्दं वानरेण मया सह} %4-5-10

\twolineshloka
{रोचते यदि मे सख्यं बाहुरेष प्रसारितः}
{गृह्यतां पाणिना पाणिर्मर्यादा बध्यतां ध्रुवा} %4-5-11

\twolineshloka
{एतत् तु वचनं श्रुत्वा सुग्रीवस्य सुभाषितम्}
{सम्प्रहृष्टमना हस्तं पीडयामास पाणिना} %4-5-12

\twolineshloka
{हृष्टः सौहृदमालम्ब्य पर्यष्वजत पीडितम्}
{ततो हनूमान् सन्त्यज्य भिक्षुरूपमरिन्दमः} %4-5-13

\twolineshloka
{काष्ठयोः स्वेन रूपेण जनयामास पावकम्}
{दीप्यमानं ततो वह्निं पुष्पैरभ्यर्च्य सत्कृतम्} %4-5-14

\twolineshloka
{तयोर्मध्ये तु सुप्रीतो निदधौ सुसमाहितः}
{ततोऽग्निं दीप्यमानं तौ चक्रतुश्च प्रदक्षिणम्} %4-5-15

\twolineshloka
{सुग्रीवो राघवश्चैव वयस्यत्वमुपागतौ}
{ततः सुप्रीतमनसौ तावुभौ हरिराघवौ} %4-5-16

\twolineshloka
{अन्योन्यमभिवीक्षन्तौ न तृप्तिमभिजग्मतुः}
{त्वं वयस्योऽसि हृद्यो मे ह्येकं दुःखं सुखं च नौ} %4-5-17

\twolineshloka
{सुग्रीवो राघवं वाक्यमित्युवाच प्रहृष्टवत्}
{ततः सुपर्णबहुलां भङ्क्त्वा शाखां सुपुष्पिताम्} %4-5-18

\twolineshloka
{सालस्यास्तीर्य सुग्रीवो निषसाद सराघवः}
{लक्ष्मणायाथ संहृष्टो हनुमान् मारुतात्मजः} %4-5-19

\twolineshloka
{शाखां चन्दनवृक्षस्य ददौ परमपुष्पिताम्}
{ततः प्रहृष्टः सुग्रीवः श्लक्ष्णं मधुरया गिरा} %4-5-20

\twolineshloka
{प्रत्युवाच तदा रामं हर्षव्याकुललोचनः}
{अहं विनिकृतो राम चरामीह भयार्दितः} %4-5-21

\twolineshloka
{हृतभार्यो वने त्रस्तो दुर्गमेतदुपाश्रितः}
{सोऽहं त्रस्तो वने भीतो वसाम्युद्भ्रान्तचेतनः} %4-5-22

\twolineshloka
{वालिना निकृतो भ्रात्रा कृतवैरश्च राघव}
{वालिनो मे महाभाग भयार्तस्याभयं कुरु} %4-5-23

\twolineshloka
{कर्तुमर्हसि काकुत्स्थ भयं मे न भवेद् यथा}
{एवमुक्तस्तु तेजस्वी धर्मज्ञो धर्मवत्सलः} %4-5-24

\twolineshloka
{प्रत्यभाषत काकुत्स्थः सुग्रीवं प्रहसन्निव}
{उपकारफलं मित्रं विदितं मे महाकपे} %4-5-25

\twolineshloka
{वालिनं तं वधिष्यामि तव भार्यापहारिणम्}
{अमोघाः सूर्यसङ्काशा ममेमे निशिताः शराः} %4-5-26

\twolineshloka
{तस्मिन् वालिनि दुर्वृत्ते निपतिष्यन्ति वेगिताः}
{कङ्कपत्रप्रतिच्छन्ना महेन्द्राशनिसन्निभाः} %4-5-27

\twolineshloka
{तीक्ष्णाग्रा ऋजुपर्वाणः सरोषा भुजगा इव}
{तमद्य वालिनं पश्य तीक्ष्णैराशीविषोपमैः} %4-5-28

\threelineshloka
{शरैर्विनिहतं भूमौ प्रकीर्णमिव पर्वतम्}
{स तु तद् वचनं श्रुत्वा राघवस्यात्मनो हितम्}
{सुग्रीवः परमप्रीतः परमं वाक्यमब्रवीत्} %4-5-29

\twolineshloka
{तव प्रसादेन नृसिंह वीर प्रियां च राज्यं च समाप्नुयामहम्}
{तथा कुरु त्वं नरदेव वैरिणं यथा न हिंस्यात् स पुनर्ममाग्रजम्} %4-5-30

\twolineshloka
{सीताकपीन्द्रक्षणदाचराणां राजीवहेमज्वलनोपमानि}
{सुग्रीवरामप्रणयप्रसङ्गे वामानि नेत्राणि समं स्फुरन्ति} %4-5-31


॥इत्यार्षे श्रीमद्रामायणे वाल्मीकीये आदिकाव्ये किष्किन्धाकाण्डे सुग्रीवसख्यम् नाम पञ्चमः सर्गः ॥४-५॥
