\sect{दशमः सर्गः — राज्यनिर्वासकथनम्}

\twolineshloka
{ततः क्रोधसमाविष्टं संरब्धं तमुपागतम्}
{अहं प्रसादयांचक्रे भ्रातरं हितकाम्यया} %4-10-1

\twolineshloka
{दिष्ट्यासि कुशली प्राप्तो निहतश्च त्वया रिपुः}
{अनाथस्य हि मे नाथस्त्वमेकोऽनाथनन्दन} %4-10-2

\twolineshloka
{इदं बहुशलाकं ते पूर्णचन्द्रमिवोदितम्}
{छत्रं सवालव्यजनं प्रतीच्छस्व मया धृतम्} %4-10-3

\twolineshloka
{आर्तस्तत्र बिलद्वारि स्थितः संवत्सरं नृप}
{दृष्ट्वा च शोणितं द्वारि बिलाच्चापि समुत्थितम्} %4-10-4

\twolineshloka
{शोकसंविग्नहृदयो भृशं व्याकुलितेन्द्रियः}
{अपिधाय बिलद्वारं शैलशृङ्गेण तत् तदा} %4-10-5

\twolineshloka
{तस्माद् देशादपाक्रम्य किष्किन्धां प्राविशं पुनः}
{विषादात्त्विह मां दृष्ट्वा पौरैर्मन्त्रिभिरेव च} %4-10-6

\twolineshloka
{अभिषिक्तो न कामेन तन्मे क्षन्तुं त्वमर्हसि}
{त्वमेव राजा मानार्हः सदा चाहं यथा पुरा} %4-10-7

\twolineshloka
{राजभावे नियोगोऽयं मम त्वद्विरहात् कृतः}
{सामात्यपौरनगरं स्थितं निहतकण्टकम्} %4-10-8

\twolineshloka
{न्यासभूतमिदं राज्यं तव निर्यातयाम्यहम्}
{मा च रोषं कृथाः सौम्य मम शत्रुनिषूदन} %4-10-9

\twolineshloka
{याचे त्वां शिरसा राजन् मया बद्धोऽयमञ्जलिः}
{बलादस्मिन् समागम्य मन्त्रिभिः पुरवासिभिः} %4-10-10

\twolineshloka
{राजभावे नियुक्तोऽहं शून्यदेशजिगीषया}
{स्निग्धमेवं ब्रुवाणं मां स विनिर्भर्त्स्य वानरः} %4-10-11

\twolineshloka
{धिक्त्वामिति च मामुक्त्वा बहु तत्तदुवाच ह}
{प्रकृतीश्च समानीय मन्त्रिणश्चैव सम्मतान्} %4-10-12

\twolineshloka
{मामाह सुहृदां मध्ये वाक्यं परमगर्हितम्}
{विदितं वो मया रात्रौ मायावी स महासुरः} %4-10-13

\twolineshloka
{मां समाह्वयत क्रुद्धो युद्धाकांक्षी तदा पुरा}
{तस्य तद् भाषितं श्रुत्वा निःसृतोऽहं नृपालयात्} %4-10-14

\twolineshloka
{अनुयातश्च मां तूर्णमयं भ्राता सुदारुणः}
{स तु दृष्ट्वैव मां रात्रौ सद्वितीयं महाबलः} %4-10-15

\twolineshloka
{प्राद्रवद् भयसंत्रस्तो वीक्ष्यावां समुपागतौ}
{अभिद्रुतस्तु वेगेन विवेश स महाबिलम्} %4-10-16

\twolineshloka
{तं प्रविष्टं विदित्वा तु सुघोरं सुमहद्बिलम्}
{अयमुक्तोऽथ मे भ्राता मया तु क्रूरदर्शनः} %4-10-17

\twolineshloka
{अहत्वा नास्ति मे शक्तिः प्रतिगन्तुमितः पुरीम्}
{बिलद्वारि प्रतीक्ष त्वं यावदेनं निहन्म्यहम्} %4-10-18

\twolineshloka
{स्थितोऽयमिति मत्वाहं प्रविष्टस्तु दुरासदम्}
{तं मे मार्गयतस्तत्र गतः संवत्सरस्तदा} %4-10-19

\twolineshloka
{स तु दृष्टो मया शत्रुरनिर्वेदाद् भयावहः}
{निहतश्च मया सद्यः स सर्वैः सह बन्धुभिः} %4-10-20

\twolineshloka
{तस्यास्यात्तु प्रवृत्तेन रुधिरौघेण तद्बिलम्}
{पूर्णमासीद् दुराक्रामं स्तनतस्तस्य भूतले} %4-10-21

\twolineshloka
{सूदयित्वा तु तं शत्रुं विक्रान्तं तमहं सुखम्}
{निष्क्रामं नैव पश्यामि बिलस्य पिहितं मुखम्} %4-10-22

\twolineshloka
{विक्रोशमानस्य तु मे सुग्रीवेति पुनः पुनः}
{यतः प्रतिवचो नास्ति ततोऽहं भृशदुःखितः} %4-10-23

\twolineshloka
{पादप्रहारैस्तु मया बहुभिः परिपातितम्}
{ततोऽहं तेन निष्क्रम्य पथा पुरमुपागतः} %4-10-24

\twolineshloka
{तत्रानेनास्मि संरुद्धो राज्यं मृगयताऽऽत्मनः}
{सुग्रीवेण नृशंसेन विस्मृत्य भ्रातृसौहृदम्} %4-10-25

\twolineshloka
{एवमुक्त्वा तु मां तत्र वस्त्रेणैकेन वानरः}
{तदा निर्वासयामास वाली विगतसाध्वसः} %4-10-26

\twolineshloka
{तेनाहमपविद्धश्च हृतदारश्च राघव}
{तद्भयाच्च महीं सर्वां क्रान्तवान् सवनार्णवाम्} %4-10-27

\twolineshloka
{ऋष्यमूकं गिरिवरं भार्याहरणदुःखितः}
{प्रविष्टोऽस्मि दुराधर्षं वालिनः कारणान्तरे} %4-10-28

\twolineshloka
{एतत्ते सर्वमाख्यातं वैरानुकथनं महत्}
{अनागसा मया प्राप्तं व्यसनं पश्य राघव} %4-10-29

\twolineshloka
{वालिनश्च भयात् तस्य सर्वलोकभयापह}
{कर्तुमर्हसि मे वीर प्रसादं तस्य निग्रहात्} %4-10-30

\twolineshloka
{एवमुक्तः स तेजस्वी धर्मज्ञो धर्मसंहितम्}
{वचनं वक्तुमारेभे सुग्रीवं प्रहसन्निव} %4-10-31

\twolineshloka
{अमोघाः सूर्यसंकाशा निशिता मे शरा इमे}
{तस्मिन् वालिनि दुर्वृत्ते पतिष्यन्ति रुषान्विताः} %4-10-32

\twolineshloka
{यावत् तं नहि पश्येयं तव भार्यापहारिणम्}
{तावत् स जीवेत् पापात्मा वाली चारित्रदूषकः} %4-10-33

\twolineshloka
{आत्मानुमानात् पश्यामि मग्नस्त्वं शोकसागरे}
{त्वामहं तारयिष्यामि बाढं प्राप्स्यसि पुष्कलम्} %4-10-34

\twolineshloka
{तस्य तद् वचनं श्रुत्वा हर्षपौरुषवर्धनम्}
{सुग्रीवः परमप्रीतः सुमहद्वाक्यमब्रवीत्} %4-10-35


॥इत्यार्षे श्रीमद्रामायणे वाल्मीकीये आदिकाव्ये किष्किन्धाकाण्डे राज्यनिर्वासकथनम् नाम दशमः सर्गः ॥४-१०॥
