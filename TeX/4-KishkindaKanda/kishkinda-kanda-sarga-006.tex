\sect{षष्ठः सर्गः — भूषणप्रत्यभिज्ञानम्}

\twolineshloka
{पुनरेवाब्रवीत् प्रीतो राघवं रघुनन्दनम्}
{अयमाख्याति ते राम सचिवो मन्त्रिसत्तमः} %4-6-1

\twolineshloka
{हनुमान् यन्निमित्तं त्वं निर्जनं वनमागतः}
{लक्ष्मणेन सह भ्रात्रा वसतश्च वने तव} %4-6-2

\twolineshloka
{रक्षसापहृता भार्या मैथिली जनकात्मजा}
{त्वया वियुक्ता रुदती लक्ष्मणेन च धीमता} %4-6-3

\twolineshloka
{अन्तरं प्रेप्सुना तेन हत्वा गृध्रं जटायुषम्}
{भार्यावियोगजं दुःखं प्रापितस्तेन रक्षसा} %4-6-4

\twolineshloka
{भार्यावियोगजं दुःखं नचिरात् त्वं विमोक्ष्यसे}
{अहं तामानयिष्यामि नष्टां वेदश्रुतीमिव} %4-6-5

\twolineshloka
{रसातले वा वर्तन्तीं वर्तन्तीं वा नभस्तले}
{अहमानीय दास्यामि तव भार्यामरिंदम} %4-6-6

\twolineshloka
{इदं तथ्यं मम वचस्त्वमवेहि च राघव}
{न शक्या सा जरयितुमपि सेन्द्रैः सुरासुरैः} %4-6-7

\twolineshloka
{तव भार्या महाबाहो भक्ष्यं विषकृतं यथा}
{त्यज शोकं महाबाहो तां कान्तामानयामि ते} %4-6-8

\twolineshloka
{अनुमानात् तु जानामि मैथिली सा न संशयः}
{ह्रियमाणा मया दृष्टा रक्षसा रौद्रकर्मणा} %4-6-9

\twolineshloka
{क्रोशन्ती रामरामेति लक्ष्मणेति च विस्वरम्}
{स्फुरन्ती रावणस्याङ्के पन्नगेन्द्रवधूर्यथा} %4-6-10

\twolineshloka
{आत्मना पञ्चमं मां हि दृष्ट्वा शैलतले स्थितम्}
{उत्तरीयं तया त्यक्तं शुभान्याभरणानि च} %4-6-11

\twolineshloka
{तान्यस्माभिर्गृहीतानि निहितानि च राघव}
{आनयिष्याम्यहं तानि प्रत्यभिज्ञातुमर्हसि} %4-6-12

\twolineshloka
{तमब्रवीत् ततो रामः सुग्रीवं प्रियवादिनम्}
{आनयस्व सखे शीघ्रं किमर्थं प्रविलम्बसे} %4-6-13

\twolineshloka
{एवमुक्तस्तु सुग्रीवः शैलस्य गहनां गुहाम्}
{प्रविवेश ततः शीघ्रं राघवप्रियकाम्यया} %4-6-14

\twolineshloka
{उत्तरीयं गृहीत्वा तु स तान्याभरणानि च}
{इदं पश्येति रामाय दर्शयामास वानरः} %4-6-15

\twolineshloka
{ततो गृहीत्वा वासस्तु शुभान्याभरणानि च}
{अभवद् बाष्पसंरुद्धो नीहारेणेव चन्द्रमाः} %4-6-16

\twolineshloka
{सीतास्नेहप्रवृत्तेन स तु बाष्पेण दूषितः}
{हा प्रियेति रुदन् धैर्यमुत्सृज्य न्यपतत् क्षितौ} %4-6-17

\twolineshloka
{हृदि कृत्वा स बहुशस्तमलंकारमुत्तमम्}
{निशश्वास भृशं सर्पो बिलस्थ इव रोषितः} %4-6-18

\twolineshloka
{अविच्छिन्नाश्रुवेगस्तु सौमित्रिं प्रेक्ष्य पार्श्वतः}
{परिदेवयितुं दीनं रामः समुपचक्रमे} %4-6-19

\twolineshloka
{पश्य लक्ष्मण वैदेह्या संत्यक्तं ह्रियमाणया}
{उत्तरीयमिदं भूमौ शरीराद् भूषणानि च} %4-6-20

\twolineshloka
{शाद्वलिन्यां ध्रुवं भूम्यां सीतया ह्रियमाणया}
{उत्सृष्टं भूषणमिदं तथा रूपं हि दृश्यते} %4-6-21

\twolineshloka
{एवमुक्तस्तु रामेण लक्ष्मणो वाक्यमब्रवीत्}
{नाहं जानामि केयूरे नाहं जानामि कुण्डले} %4-6-22

\twolineshloka
{नूपुरे त्वभिजानामि नित्यं पादाभिवन्दनात्}
{ततस्तु राघवो वाक्यं सुग्रीवमिदमब्रवीत्} %4-6-23

\twolineshloka
{ब्रूहि सुग्रीव कं देशं ह्रियन्ती लक्षिता त्वया}
{रक्षसा रौद्ररूपेण मम प्राणप्रिया हृता} %4-6-24

\twolineshloka
{क्व वा वसति तद् रक्षो महद् व्यसनदं मम}
{यन्निमित्तमहं सर्वान् नाशयिष्यामि राक्षसान्} %4-6-25

\twolineshloka
{हरता मैथिलीं येन मां च रोषयता ध्रुवम्}
{आत्मनो जीवितान्ताय मृत्युद्वारमपावृतम्} %4-6-26

\twolineshloka
{मम दयिततमा हृता वनाद् रजनिचरेण विमथ्य येन सा}
{कथय मम रिपुं तमद्य वै प्लवगपते यमसंनिधिं नयामि} %4-6-27


॥इत्यार्षे श्रीमद्रामायणे वाल्मीकीये आदिकाव्ये किष्किन्धाकाण्डे भूषणप्रत्यभिज्ञानम् नाम षष्ठः सर्गः ॥४-६॥
