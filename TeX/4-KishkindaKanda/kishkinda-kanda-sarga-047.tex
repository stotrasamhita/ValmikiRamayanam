\sect{सप्तचत्वारिंशः सर्गः — कपिसेनाप्रत्यागमनम्}

\twolineshloka
{दर्शनार्थं तु वैदेह्याः सर्वतः कपिकुञ्जराः}
{व्यादिष्टाः कपिराजेन यथोक्तं जग्मुरञ्जसा} %4-47-1

\twolineshloka
{ते सरांसि सरित्कक्षानाकाशं नगराणि च}
{नदीदुर्गांस्तथा देशान् विचिन्वन्ति समन्ततः} %4-47-2

\twolineshloka
{सुग्रीवेण समाख्याताः सर्वे वानरयूथपाः}
{तत्र देशान् विचिन्वन्ति सशैलवनकाननान्} %4-47-3

\twolineshloka
{विचित्य दिवसं सर्वे सीताधिगमने धृताः}
{समायान्ति स्म मेदिन्यां निशाकालेषु वानराः} %4-47-4

\twolineshloka
{सर्वर्तुकांश्च देशेषु वानराः सफलद्रुमान्}
{आसाद्य रजनीं शय्यां चक्रुः सर्वेष्वहःसु ते} %4-47-5

\twolineshloka
{तदहः प्रथमं कृत्वा मासे प्रस्रवणं गताः}
{कपिराजेन संगम्य निराशाः कपिकुञ्जराः} %4-47-6

\twolineshloka
{विचित्य तु दिशं पूर्वां यथोक्तां सचिवैः सह}
{अदृष्ट्वा विनतः सीतामाजगाम महाबलः} %4-47-7

\twolineshloka
{दिशमप्युत्तरां सर्वां विविच्य स महाकपिः}
{आगतः सह सैन्येन भीतः शतबलिस्तदा} %4-47-8

\twolineshloka
{सुषेणः पश्चिमामाशां विविच्य सह वानरैः}
{समेत्य मासे पूर्णे तु सुग्रीवमुपचक्रमे} %4-47-9

\twolineshloka
{तं प्रस्रवणपृष्ठस्थं समासाद्याभिवाद्य च}
{आसीनं सह रामेण सुग्रीवमिदमब्रुवन्} %4-47-10

\twolineshloka
{विचिताः पर्वताः सर्वे वनानि गहनानि च}
{निम्नगाः सागरान्ताश्च सर्वे जनपदाश्च ये} %4-47-11

\twolineshloka
{गुहाश्च विचिताः सर्वा याश्च ते परिकीर्तिताः}
{विचिताश्च महागुल्मा लताविततसंतताः} %4-47-12

\threelineshloka
{गहनेषु च देशेषु दुर्गेषु विषमेषु च}
{सत्त्वान्यतिप्रमाणानि विचितानि हतानि च}
{ये चैव गहना देशा विचितास्ते पुनः पुनः} %4-47-13

\twolineshloka
{उदारसत्त्वाभिजनो हनूमान् स मैथिलीं ज्ञास्यति वानरेन्द्र}
{दिशं तु यामेव गता तु सीता तामास्थितो वायुसुतो हनूमान्} %4-47-14


॥इत्यार्षे श्रीमद्रामायणे वाल्मीकीये आदिकाव्ये किष्किन्धाकाण्डे कपिसेनाप्रत्यागमनम् नाम सप्तचत्वारिंशः सर्गः ॥४-४७॥
