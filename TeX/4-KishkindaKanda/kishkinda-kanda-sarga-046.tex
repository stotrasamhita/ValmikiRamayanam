\sect{षड्चत्वारिंशः सर्गः — भूमण्डलभ्रमणकथनम्}

\twolineshloka
{गतेषु वानरेन्द्रेषु रामः सुग्रीवमब्रवीत्}
{कथं भवान् विजानीते सर्वं वै मण्डलं भुवः} %4-46-1

\twolineshloka
{सुग्रीवश्च ततो राममुवाच प्रणतात्मवान्}
{श्रूयतां सर्वमाख्यास्ये विस्तरेण वचो मम} %4-46-2

\twolineshloka
{यदा तु दुन्दुभिं नाम दानवं महिषाकृतिम्}
{प्रतिकालयते वाली मलयं प्रति पर्वतम्} %4-46-3

\twolineshloka
{तदा विवेश महिषो मलयस्य गुहां प्रति}
{विवेश वाली तत्रापि मलयं तज्जिघांसया} %4-46-4

\twolineshloka
{ततोऽहं तत्र निक्षिप्तो गुहाद्वारि विनीतवत्}
{न च निष्क्रामते वाली तदा संवत्सरे गते} %4-46-5

\twolineshloka
{ततः क्षतजवेगेन आपुपूरे तदा बिलम्}
{तदहं विस्मितो दृष्ट्वा भ्रातुः शोकविषार्दितः} %4-46-6

\twolineshloka
{अथाहं गतबुद्धिस्तु सुव्यक्तं निहतो गुरुः}
{शिला पर्वतसंकाशा बिलद्वारि मया कृता} %4-46-7

\twolineshloka
{अशक्नुवन्निष्क्रमितुं महिषो विनशिष्यति}
{ततोऽहमागां किष्किन्धां निराशस्तस्य जीविते} %4-46-8

\twolineshloka
{राज्यं च सुमहत् प्राप्य तारां च रुमया सह}
{मित्रैश्च सहितस्तत्र वसामि विगतज्वरः} %4-46-9

\twolineshloka
{आजगाम ततो वाली हत्वा तं वानरर्षभः}
{ततोऽहमददां राज्यं गौरवाद् भययन्त्रितः} %4-46-10

\twolineshloka
{स मां जिघांसुर्दुष्टात्मा वाली प्रव्यथितेन्द्रियः}
{परिकालयते वाली धावन्तं सचिवैः सह} %4-46-11

\twolineshloka
{ततोऽहं वालिना तेन सोऽनुबद्धः प्रधावितः}
{नदीश्च विविधाः पश्यन् वनानि नगराणि च} %4-46-12

\twolineshloka
{आदर्शतलसंकाशा ततो वै पृथिवी मया}
{अलातचक्रप्रतिमा दृष्टा गोष्पदवत् कृता} %4-46-13

\twolineshloka
{पूर्वां दिशं ततो गत्वा पश्यामि विविधान् द्रुमान्}
{पर्वतान् सदरीन् रम्यान् सरांसि विविधानि च} %4-46-14

\twolineshloka
{उदयं तत्र पश्यामि पर्वतं धातुमण्डितम्}
{क्षीरोदं सागरं चैव नित्यमप्सरसालयम्} %4-46-15

\twolineshloka
{परिकाल्यमानस्तु तदा वालिनाभिद्रुतो ह्यहम्}
{पुनरावृत्य सहसा प्रस्थितोऽहं तदा विभो} %4-46-16

\twolineshloka
{दिशस्तस्यास्ततो भूयः प्रस्थितो दक्षिणां दिशम्}
{विन्ध्यपादपसंकीर्णां चन्दनद्रुमशोभिताम्} %4-46-17

\twolineshloka
{द्रुमशैलान्तरे पश्यन् भूयो दक्षिणतोऽपराम्}
{अपरां च दिशं प्राप्तो वालिना समभिद्रुतः} %4-46-18

\twolineshloka
{स पश्यन् विविधान् देशानस्तं च गिरिसत्तमम्}
{प्राप्य चास्तं गिरिश्रेष्ठमुत्तरं सम्प्रधावितः} %4-46-19

\twolineshloka
{हिमवन्तं च मेरुं च समुद्रं च तथोत्तरम्}
{यदा न विन्दे शरणं वालिना समभिद्रुतः} %4-46-20

\twolineshloka
{ततो मां बुद्धिसम्पन्नो हनुमान् वाक्यमब्रवीत्}
{इदानीं मे स्मृतं राजन् यथा वाली हरीश्वरः} %4-46-21

\twolineshloka
{मतङ्गेन तदा शप्तो ह्यस्मिन्नाश्रममण्डले}
{प्रविशेद् यदि वै वाली मूर्धास्य शतधा भवेत्} %4-46-22

\twolineshloka
{तत्र वासः सुखोऽस्माकं निरुद्विग्नो भविष्यति}
{ततः पर्वतमासाद्य ऋष्यमूकं नृपात्मज} %4-46-23

\threelineshloka
{न विवेश तदा वाली मतङ्गस्य भयात् तदा}
{एवं मया तदा राजन् प्रत्यक्षमुपलक्षितम्}
{पृथिवीमण्डलं सर्वं गुहामस्म्यागतस्ततः} %4-46-24


॥इत्यार्षे श्रीमद्रामायणे वाल्मीकीये आदिकाव्ये किष्किन्धाकाण्डे भूमण्डलभ्रमणकथनम् नाम षड्चत्वारिंशः सर्गः ॥४-४६॥
