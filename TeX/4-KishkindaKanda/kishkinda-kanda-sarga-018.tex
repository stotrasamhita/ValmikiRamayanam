\sect{अष्टादशः सर्गः — वालिवधसमर्थनम्}

\twolineshloka
{इत्युक्तः प्रश्रितं वाक्यं धर्मार्थसहितं हितम्}
{परुषं वालिना रामो निहतेन विचेतसा} %4-18-1

\twolineshloka
{तं निष्प्रभमिवादित्यं मुक्ततोयमिवाम्बुदम्}
{उक्तवाक्यं हरिश्रेष्ठमुपशान्तमिवानलम्} %4-18-2

\twolineshloka
{धर्मार्थगुणसम्पन्नं हरीश्वरमनुत्तमम्}
{अधिक्षिप्तस्तदा रामः पश्चाद् वालिनमब्रवीत्} %4-18-3

\twolineshloka
{धर्ममर्थं च कामं च समयं चापि लौकिकम्}
{अविज्ञाय कथं बाल्यान्मामिहाद्य विगर्हसे} %4-18-4

\twolineshloka
{अपृष्ट्वा बुद्धिसम्पन्नान् वृद्धानाचार्यसम्मतान्}
{सौम्य वानरचापल्यात् त्वं मां वक्तुमिहेच्छसि} %4-18-5

\twolineshloka
{इक्ष्वाकूणामियं भूमिः सशैलवनकानना}
{मृगपक्षिमनुष्याणां निग्रहानुग्रहेष्वपि} %4-18-6

\twolineshloka
{तां पालयति धर्मात्मा भरतः सत्यवानृजुः}
{धर्मकामार्थतत्त्वज्ञो निग्रहानुग्रहे रतः} %4-18-7

\twolineshloka
{नयश्च विनयश्चोभौ यस्मिन् सत्यं च सुस्थितम्}
{विक्रमश्च यथा दृष्टः स राजा देशकालवित्} %4-18-8

\twolineshloka
{तस्य धर्मकृतादेशा वयमन्ये च पार्थिवाः}
{चरामो वसुधां कृत्स्नां धर्मसंतानमिच्छवः} %4-18-9

\twolineshloka
{तस्मिन् नृपतिशार्दूले भरते धर्मवत्सले}
{पालयत्यखिलां पृथ्वीं कश्चरेद् धर्मविप्रियम्} %4-18-10

\twolineshloka
{ते वयं मार्गविभ्रष्टं स्वधर्मे परमे स्थिताः}
{भरताज्ञां पुरस्कृत्य निगृह्णीमो यथाविधि} %4-18-11

\twolineshloka
{त्वं तु संक्लिष्टधर्मश्च कर्मणा च विगर्हितः}
{कामतन्त्रप्रधानश्च न स्थितो राजवर्त्मनि} %4-18-12

\twolineshloka
{ज्येष्ठो भ्राता पिता वापि यश्च विद्यां प्रयच्छति}
{त्रयस्ते पितरो ज्ञेया धर्मे च पथि वर्तिनः} %4-18-13

\twolineshloka
{यवीयानात्मनः पुत्रः शिष्यश्चापि गुणोदितः}
{पुत्रवत्ते त्रयश्चिन्त्या धर्मश्चैवात्र कारणम्} %4-18-14

\twolineshloka
{सूक्ष्मः परमदुर्ज्ञेयः सतां धर्मः प्लवङ्गम}
{हृदिस्थः सर्वभूतानामात्मा वेद शुभाशुभम्} %4-18-15

\twolineshloka
{चपलश्चपलैः सार्धं वानरैरकृतात्मभिः}
{जात्यन्ध इव जात्यन्धैर्मन्त्रयन् प्रेक्षसे नु किम्} %4-18-16

\twolineshloka
{अहं तु व्यक्ततामस्य वचनस्य ब्रवीमि ते}
{नहि मां केवलं रोषात् त्वं विगर्हितुमर्हसि} %4-18-17

\twolineshloka
{तदेतत् कारणं पश्य यदर्थं त्वं मया हतः}
{भ्रातुर्वर्तसि भार्यायां त्यक्त्वा धर्मं सनातनम्} %4-18-18

\twolineshloka
{अस्य त्वं धरमाणस्य सुग्रीवस्य महात्मनः}
{रुमायां वर्तसे कामात् स्नुषायां पापकर्मकृत्} %4-18-19

\twolineshloka
{तद् व्यतीतस्य ते धर्मात् कामवृत्तस्य वानर}
{भ्रातृभार्याभिमर्शेऽस्मिन् दण्डोऽयं प्रतिपादितः} %4-18-20

\twolineshloka
{नहि लोकविरुद्धस्य लोकवृत्तादपेयुषः}
{दण्डादन्यत्र पश्यामि निग्रहं हरियूथप} %4-18-21

\twolineshloka
{न च ते मर्षये पापं क्षत्रियोऽहं कुलोद्गतः}
{औरसीं भगिनीं वापि भार्यां वाप्यनुजस्य यः} %4-18-22

\twolineshloka
{प्रचरेत नरः कामात् तस्य दण्डो वधः स्मृतः}
{भरतस्तु महीपालो वयं त्वादेशवर्तिनः} %4-18-23

\twolineshloka
{त्वं च धर्मादतिक्रान्तः कथं शक्यमुपेक्षितुम्}
{गुरुधर्मव्यतिक्रान्तं प्राज्ञो धर्मेण पालयन्} %4-18-24

\threelineshloka
{भरतः कामयुक्तानां निग्रहे पर्यवस्थितः}
{वयं तु भरतादेशावधिं कृत्वा हरीश्वर}
{त्वद्विधान् भिन्नमर्यादान् निग्रहीतुं व्यवस्थिताः} %4-18-25

\twolineshloka
{सुग्रीवेण च मे सख्यं लक्ष्मणेन यथा तथा}
{दारराज्यनिमित्तं च निःश्रेयस्करः स मे} %4-18-26

\twolineshloka
{प्रतिज्ञा च मया दत्ता तदा वानरसंनिधौ}
{प्रतिज्ञा च कथं शक्या मद्विधेनानवेक्षितुम्} %4-18-27

\twolineshloka
{तदेभिः कारणैः सर्वैर्महद्भिर्धर्मसंश्रितैः}
{शासनं तव यद् युक्तं तद् भवाननुमन्यताम्} %4-18-28

\twolineshloka
{सर्वथा धर्म इत्येव द्रष्टव्यस्तव निग्रहः}
{वयस्यस्योपकर्तव्यं धर्ममेवानुपश्यता} %4-18-29

\threelineshloka
{शक्यं त्वयापि तत्कार्यं धर्ममेवानुवर्तता}
{श्रूयते मनुना गीतौ श्लोकौ चारित्रवत्सलौ}
{गृहीतौ धर्मकुशलैस्तथा तच्चरितं मया} %4-18-30

\twolineshloka
{राजभिर्धृतदण्डाश्च कृत्वा पापानि मानवाः}
{निर्मलाः स्वर्गमायान्ति सन्तः सुकृतिनो यथा} %4-18-31

\twolineshloka
{शासनाद् वापि मोक्षाद् वा स्तेनः पापात् प्रमुच्यते}
{राजा त्वशासन् पापस्य तदवाप्नोति किल्बिषम्} %4-18-32

\twolineshloka
{आर्येण मम मान्धात्रा व्यसनं घोरमीप्सितम्}
{श्रमणेन कृते पापे यथा पापं कृतं त्वया} %4-18-33

\twolineshloka
{अन्यैरपि कृतं पापं प्रमत्तैर्वसुधाधिपैः}
{प्रायश्चित्तं च कुर्वन्ति तेन तच्छाम्यते रजः} %4-18-34

\twolineshloka
{तदलं परितापेन धर्मतः परिकल्पितः}
{वधो वानरशार्दूल न वयं स्ववशे स्थिताः} %4-18-35

\twolineshloka
{शृणु चाप्यपरं भूयः कारणं हरिपुंगव}
{तच्छ्रुत्वा हि महद् वीर न मन्युं कर्तुमर्हसि} %4-18-36

\twolineshloka
{न मे तत्र मनस्तापो न मन्युर्हरिपुंगव}
{वागुराभिश्च पाशैश्च कूटैश्च विविधैर्नराः} %4-18-37

\twolineshloka
{प्रतिच्छन्नाश्च दृश्याश्च गृह्णन्ति सुबहून् मृगान्}
{प्रधावितान् वा वित्रस्तान् विस्रब्धानतिविष्ठितान्} %4-18-38

\twolineshloka
{प्रमत्तानप्रमत्तान् वा नरा मांसाशिनो भृशम्}
{विध्यन्ति विमुखांश्चापि न च दोषोऽत्र विद्यते} %4-18-39

\threelineshloka
{यान्ति राजर्षयश्चात्र मृगयां धर्मकोविदाः}
{तस्मात् त्वं निहतो युद्धे मया बाणेन वानर}
{अयुध्यन् प्रतियुध्यन् वा यस्माच्छाखामृगो ह्यसि} %4-18-40

\twolineshloka
{दुर्लभस्य च धर्मस्य जीवितस्य शुभस्य च}
{राजानो वानरश्रेष्ठ प्रदातारो न संशयः} %4-18-41

\twolineshloka
{तान् न हिंस्यान्न चाक्रोशेन्नाक्षिपेन्नाप्रियं वदेत्}
{देवा मानुषरूपेण चरन्त्येते महीतले} %4-18-42

\twolineshloka
{त्वं तु धर्ममविज्ञाय केवलं रोषमास्थितः}
{विदूषयसि मां धर्मे पितृपैतामहे स्थितम्} %4-18-43

\twolineshloka
{एवमुक्तस्तु रामेण वाली प्रव्यथितो भृशम्}
{न दोषं राघवे दध्यौ धर्मेऽधिगतनिश्चयः} %4-18-44

\twolineshloka
{प्रत्युवाच ततो रामं प्राञ्जलिर्वानरेश्वरः}
{यत् त्वमात्थ नरश्रेष्ठ तत् तथैव न संशयः} %4-18-45

\twolineshloka
{प्रतिवक्तुं प्रकृष्टे हि नापकृष्टस्तु शक्नुयात्}
{यदयुक्तं मया पूर्वं प्रमादाद् वाक्यमप्रियम्} %4-18-46

\threelineshloka
{तत्रापि खलु मे दोषं कर्तुं नार्हसि राघव}
{त्वं हि दृष्टार्थतत्त्वज्ञः प्रजानां च हिते रतः}
{कार्यकारणसिद्धौ च प्रसन्ना बुद्धिरव्यया} %4-18-47

\twolineshloka
{मामप्यवगतं धर्माद् व्यतिक्रान्तपुरस्कृतम्}
{धर्मसंहितया वाचा धर्मज्ञ परिपालय} %4-18-48

\twolineshloka
{बाष्पसंरुद्धकण्ठस्तु वाली सार्तरवः शनैः}
{उवाच रामं सम्प्रेक्ष्य पङ्कलग्न इव द्विपः} %4-18-49

\twolineshloka
{न चात्मानमहं शोचे न तारां नापि बान्धवान्}
{यथा पुत्रं गुणज्येष्ठमङ्गदं कनकाङ्गदम्} %4-18-50

\twolineshloka
{स ममादर्शनाद् दीनो बाल्यात् प्रभृति लालितः}
{तटाक इव पीताम्बुरुपशोषं गमिष्यति} %4-18-51

\twolineshloka
{बालश्चाकृतबुद्धिश्च एकपुत्रश्च मे प्रियः}
{तारेयो राम भवता रक्षणीयो महाबलः} %4-18-52

\twolineshloka
{सुग्रीवे चाङ्गदे चैव विधत्स्व मतिमुत्तमाम्}
{त्वं हि गोप्ता च शास्ता च कार्याकार्यविधौ स्थितः} %4-18-53

\twolineshloka
{या ते नरपते वृत्तिर्भरते लक्ष्मणे च या}
{सुग्रीवे चाङ्गदे राजंस्तां चिन्तयितुमर्हसि} %4-18-54

\twolineshloka
{मद्दोषकृतदोषां तां यथा तारां तपस्विनीम्}
{सुग्रीवो नावमन्येत तथावस्थातुमर्हसि} %4-18-55

\twolineshloka
{त्वया ह्यनुगृहीतेन शक्यं राज्यमुपासितुम्}
{त्वद्वशे वर्तमानेन तव चित्तानुवर्तिना} %4-18-56

\twolineshloka
{शक्यं दिवं चार्जयितुं वसुधां चापि शासितुम्}
{त्वत्तोऽहं वधमाकांक्षन् वार्यमाणोऽपि तारया} %4-18-57

\twolineshloka
{सुग्रीवेण सह भ्रात्रा द्वन्द्वयुद्धमुपागतः}
{इत्युक्त्वा वानरो रामं विरराम हरीश्वरः} %4-18-58

\twolineshloka
{स तमाश्वासयद् रामो वालिनं व्यक्तदर्शनम्}
{साधुसम्मतया वाचा धर्मतत्त्वार्थयुक्तया} %4-18-59

\threelineshloka
{न संतापस्त्वया कार्य एतदर्थं प्लवङ्गम}
{न वयं भवता चिन्त्या नाप्यात्मा हरिसत्तम}
{वयं भवद्विशेषेण धर्मतः कृतनिश्चयाः} %4-18-60

\twolineshloka
{दण्ड्ये यः पातयेद् दण्डं दण्ड्यो यश्चापि दण्ड्यते}
{कार्यकारणसिद्धार्थावुभौ तौ नावसीदतः} %4-18-61

\twolineshloka
{तद् भवान् दण्डसंयोगादस्माद् विगतकल्मषः}
{गतः स्वां प्रकृतिं धर्म्यां दण्डदिष्टेन वर्त्मना} %4-18-62

\twolineshloka
{त्यज शोकं च मोहं च भयं च हृदये स्थितम्}
{त्वया विधानं हर्यग्र्य न शक्यमतिवर्तितुम्} %4-18-63

\twolineshloka
{यथा त्वय्यङ्गदो नित्यं वर्तते वानरेश्वर}
{तथा वर्तेत सुग्रीवे मयि चापि न संशयः} %4-18-64

\twolineshloka
{स तस्य वाक्यं मधुरं महात्मनः समाहितं धर्मपथानुवर्तितम्}
{निशम्य रामस्य रणावमर्दिनो वचः सुयुक्तं निजगाद वानरः} %4-18-65

\twolineshloka
{शराभितप्तेन विचेतसा मया प्रभाषितस्त्वं यदजानता विभो}
{इदं महेन्द्रोपमभीमविक्रम प्रसादितस्त्वं क्षम मे नरेश्वर} %4-18-66


॥इत्यार्षे श्रीमद्रामायणे वाल्मीकीये आदिकाव्ये किष्किन्धाकाण्डे वालिवधसमर्थनम् नाम अष्टादशः सर्गः ॥४-१८॥
