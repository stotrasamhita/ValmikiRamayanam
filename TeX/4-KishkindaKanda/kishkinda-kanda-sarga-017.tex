\sect{सप्तदशः सर्गः — रामाधिक्षेपः}

\twolineshloka
{ततः शरेणाभिहतो रामेण रणकर्कशः}
{पपात सहसा वाली निकृत्त इव पादपः} %4-17-1

\twolineshloka
{स भूमौ न्यस्तसर्वाङ्गस्तप्तकाञ्चनभूषणः}
{अपतद् देवराजस्य मुक्तरश्मिरिव ध्वजः} %4-17-2

\twolineshloka
{अस्मिन् निपतिते भूमौ हर्यृक्षाणां गणेश्वरे}
{नष्टचन्द्रमिव व्योम न व्यराजत मेदिनी} %4-17-3

\twolineshloka
{भूमौ निपतितस्यापि तस्य देहं महात्मनः}
{न श्रीर्जहाति न प्राणा न तेजो न पराक्रमः} %4-17-4

\twolineshloka
{शक्रदत्ता वरा माला काञ्चनी रत्नभूषिता}
{दधार हरिमुख्यस्य प्राणांस्तेजः श्रियं च सा} %4-17-5

\twolineshloka
{स तया मालया वीरो हैमया हरियूथपः}
{संध्यानुगतपर्यन्तः पयोधर इवाभवत्} %4-17-6

\twolineshloka
{तस्य माला च देहश्च मर्मघाती च यः शरः}
{त्रिधेव रचिता लक्ष्मीः पतितस्यापि शोभते} %4-17-7

\twolineshloka
{तदस्त्रं तस्य वीरस्य स्वर्गमार्गप्रभावनम्}
{रामबाणासनक्षिप्तमावहत् परमां गतिम्} %4-17-8

\twolineshloka
{तं तथा पतितं संख्ये गतार्चिषमिवानलम्}
{ययातिमिव पुण्यान्ते देवलोकादिह च्युतम्} %4-17-9

\twolineshloka
{आदित्यमिव कालेन युगान्ते भुवि पातितम्}
{महेन्द्रमिव दुर्धर्षमुपेन्द्रमिव दुःसहम्} %4-17-10

\twolineshloka
{महेन्द्रपुत्रं पतितं वालिनं हेममालिनम्}
{व्यूढोरस्कं महाबाहुं दीप्तास्यं हरिलोचनम्} %4-17-11

\twolineshloka
{लक्ष्मणानुचरो रामो ददर्शोपससर्प च}
{तं तथा पतितं वीरं गतार्चिषमिवानलम्} %4-17-12

\twolineshloka
{बहुमान्य च तं वीरं वीक्षमाणं शनैरिव}
{उपयातौ महावीर्यौ भ्रातरौ रामलक्ष्मणौ} %4-17-13

\twolineshloka
{तं दृष्ट्वा राघवं वाली लक्ष्मणं च महाबलम्}
{अब्रवीत् परुषं वाक्यं प्रश्रितं धर्मसंहितम्} %4-17-14

\twolineshloka
{स भूमावल्पतेजोऽसुर्निहतो नष्टचेतनः}
{अर्थसंहितया वाचा गर्वितं रणगर्वितम्} %4-17-15

\threelineshloka
{त्वं नराधिपतेः पुत्रः प्रथितः प्रियदर्शनः}
{पराङ्मुखवधं कृत्वा कोऽत्र प्राप्तस्त्वया गुणः}
{यदहं युद्धसंरब्धस्त्वत्कृते निधनं गतः} %4-17-16

\twolineshloka
{कुलीनः सत्त्वसम्पन्नस्तेजस्वी चरितव्रतः}
{रामः करुणवेदी च प्रजानां च हिते रतः} %4-17-17

\twolineshloka
{सानुक्रोशो महोत्साहः समयज्ञो दृढव्रतः}
{इत्येतत् सर्वभूतानि कथयन्ति यशो भुवि} %4-17-18

\twolineshloka
{दमः शमः क्षमा धर्मो धृतिः सत्यं पराक्रमः}
{पार्थिवानां गुणा राजन् दण्डश्चाप्यपकारिषु} %4-17-19

\twolineshloka
{तान् गुणान् सम्प्रधार्याहमग्र्यं चाभिजनं तव}
{तारया प्रतिषिद्धः सन् सुग्रीवेण समागतः} %4-17-20

\twolineshloka
{न मामन्येन संरब्धं प्रमत्तं वेद्धुमर्हसि}
{इति मे बुद्धिरुत्पन्ना बभूवादर्शने तव} %4-17-21

\twolineshloka
{स त्वां विनिहतात्मानं धर्मध्वजमधार्मिकम्}
{जाने पापसमाचारं तृणैः कूपमिवावृतम्} %4-17-22

\twolineshloka
{सतां वेषधरं पापं प्रच्छन्नमिव पावकम्}
{नाहं त्वामभिजानामि धर्मच्छद्माभिसंवृतम्} %4-17-23

\twolineshloka
{विषये वा पुरे वा ते यदा पापं करोम्यहम्}
{न च त्वामवजानेऽहं कस्मात् तं हंस्यकिल्बिषम्} %4-17-24

\twolineshloka
{फलमूलाशनं नित्यं वानरं वनगोचरम्}
{मामिहाप्रतियुध्यन्तमन्येन च समागतम्} %4-17-25

\twolineshloka
{त्वं नराधिपतेः पुत्रः प्रतीतः प्रियदर्शनः}
{लिङ्गमप्यस्ति ते राजन् दृश्यते धर्मसंहितम्} %4-17-26

\twolineshloka
{कः क्षत्रियकुले जातः श्रुतवान् नष्टसंशयः}
{धर्मलिङ्गप्रतिच्छन्नः क्रूरं कर्म समाचरेत्} %4-17-27

\twolineshloka
{त्वं राघवकुले जातो धर्मवानिति विश्रुतः}
{अभव्यो भव्यरूपेण किमर्थं परिधावसे} %4-17-28

\twolineshloka
{साम दानं क्षमा धर्मः सत्यं धृतिपराक्रमौ}
{पार्थिवानां गुणा राजन् दण्डश्चाप्यपकारिषु} %4-17-29

\twolineshloka
{वयं वनचरा राम मृगा मूलफलाशिनः}
{एषा प्रकृतिरस्माकं पुरुषस्त्वं नरेश्वर} %4-17-30

\twolineshloka
{भूमिर्हिरण्यं रूपं च विग्रहे कारणानि च}
{तत्र कस्ते वने लोभो मदीयेषु फलेषु वा} %4-17-31

\twolineshloka
{नयश्च विनयश्चोभौ निग्रहानुग्रहावपि}
{राजवृत्तिरसंकीर्णा न नृपाः कामवृत्तयः} %4-17-32

\twolineshloka
{त्वं तु कामप्रधानश्च कोपनश्चानवस्थितः}
{राजवृत्तेषु संकीर्णः शरासनपरायणः} %4-17-33

\twolineshloka
{न तेऽस्त्यपचितिर्धर्मे नार्थे बुद्धिरवस्थिता}
{इन्द्रियैः कामवृत्तः सन् कृष्यसे मनुजेश्वर} %4-17-34

\twolineshloka
{हत्वा बाणेन काकुत्स्थ मामिहानपराधिनम्}
{किं वक्ष्यसि सतां मध्ये कर्म कृत्वा जुगुप्सितम्} %4-17-35

\twolineshloka
{राजहा ब्रह्महा गोघ्नश्चोरः प्राणिवधे रतः}
{नास्तिकः परिवेत्ता च सर्वे निरयगामिनः} %4-17-36

\twolineshloka
{सूचकश्च कदर्यश्च मित्रघ्नो गुरुतल्पगः}
{लोकं पापात्मनामेते गच्छन्ते नात्र संशयः} %4-17-37

\twolineshloka
{अधार्यं चर्म मे सद्भी रोमाण्यस्थि च वर्जितम्}
{अभक्ष्याणि च मांसानि त्वद्विधैर्धर्मचारिभिः} %4-17-38

\twolineshloka
{पञ्च पञ्चनखा भक्ष्या ब्रह्मक्षत्रेण राघव}
{शल्यकः श्वाविधो गोधा शशः कूर्मश्च पञ्चमः} %4-17-39

\twolineshloka
{चर्म चास्थि च मे राम न स्पृशन्ति मनीषिणः}
{अभक्ष्याणि च मांसानि सोऽहं पञ्चनखो हतः} %4-17-40

\twolineshloka
{तारया वाक्यमुक्तोऽहं सत्यं सर्वज्ञया हितम्}
{तदतिक्रम्य मोहेन कालस्य वशमागतः} %4-17-41

\twolineshloka
{त्वया नाथेन काकुत्स्थ न सनाथा वसुंधरा}
{प्रमदा शीलसम्पूर्णा पत्येव च विधर्मणा} %4-17-42

\twolineshloka
{शठो नैकृतिकः क्षुद्रो मिथ्याप्रश्रितमानसः}
{कथं दशरथेन त्वं जातः पापो महात्मना} %4-17-43

\twolineshloka
{छिन्नचारित्र्यकक्ष्येण सतां धर्मातिवर्तिना}
{त्यक्तधर्माङ्कुशेनाहं निहतो रामहस्तिना} %4-17-44

\twolineshloka
{अशुभं चाप्ययुक्तं च सतां चैव विगर्हितम्}
{वक्ष्यसे चेदृशं कृत्वा सद्भिः सह समागतः} %4-17-45

\twolineshloka
{उदासीनेषु योऽस्मासु विक्रमोऽयं प्रकाशितः}
{अपकारिषु ते राम नैवं पश्यामि विक्रमम्} %4-17-46

\twolineshloka
{दृश्यमानस्तु युध्येथा मया युधि नृपात्मज}
{अद्य वैवस्वतं देवं पश्येस्त्वं निहतो मया} %4-17-47

\twolineshloka
{त्वयादृश्येन तु रणे निहतोऽहं दुरासदः}
{प्रसुप्तः पन्नगेनैव नरः पापवशं गतः} %4-17-48

\threelineshloka
{सुग्रीवप्रियकामेन यदहं निहतस्त्वया}
{मामेव यदि पूर्वं त्वमेतदर्थमचोदयः}
{मैथिलीमहमेकाह्ना तव चानीतवान् भवेः} %4-17-49

\twolineshloka
{राक्षसं च दुरात्मानं तव भार्यापहारिणम्}
{कण्ठे बद्ध्वा प्रदद्यां तेऽनिहतं रावणं रणे} %4-17-50

\twolineshloka
{न्यस्तां सागरतोये वा पाताले वापि मैथिलीम्}
{आनयेयं तवादेशाच्छ्वेतामश्वतरीमिव} %4-17-51

\twolineshloka
{युक्तं यत्प्राप्नुयाद् राज्यं सुग्रीवः स्वर्गते मयि}
{अयुक्तं यदधर्मेण त्वयाहं निहतो रणे} %4-17-52

\twolineshloka
{काममेवंविधो लोकः कालेन विनियुज्यते}
{क्षमं चेद्भवता प्राप्तमुत्तरं साधु चिन्त्यताम्} %4-17-53

\twolineshloka
{इत्येवमुक्त्वा परिशुष्कवक्त्रः शराभिघाताद् व्यथितो महात्मा}
{समीक्ष्य रामं रविसंनिकाशं तूष्णीं बभौ वानरराजसूनुः} %4-17-54


॥इत्यार्षे श्रीमद्रामायणे वाल्मीकीये आदिकाव्ये किष्किन्धाकाण्डे रामाधिक्षेपः नाम सप्तदशः सर्गः ॥४-१७॥
