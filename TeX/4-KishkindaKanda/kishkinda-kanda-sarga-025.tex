\sect{पञ्चविंशः सर्गः — वालिसंस्कारः}

\twolineshloka
{स सुग्रीवं च तारां च साङ्गदां सहलक्ष्मणः}
{समानशोकः काकुत्स्थः सान्त्वयन्निदमब्रवीत्} %4-25-1

\twolineshloka
{न शोकपरितापेन श्रेयसा युज्यते मृतः}
{यदत्रानन्तरं कार्यं तत् समाधातुमर्हथ} %4-25-2

\twolineshloka
{लोकवृत्तमनुष्ठेयं कृतं वो बाष्पमोक्षणम्}
{न कालादुत्तरं किंचित् कर्मशक्यमुपासितुम्} %4-25-3

\twolineshloka
{नियतिः कारणं लोके नियतिः कर्मसाधनम्}
{नियतिः सर्वभूतानां नियोगेष्विह कारणम्} %4-25-4

\twolineshloka
{न कर्ता कस्यचित् कश्चिन्नियोगे नापि चेश्वरः}
{स्वभावे वर्तते लोकस्तस्य कालः परायणम्} %4-25-5

\twolineshloka
{न कालः कालमत्येति न कालः परिहीयते}
{स्वभावं च समासाद्य न कश्चिदतिवर्तते} %4-25-6

\twolineshloka
{न कालस्यास्ति बन्धुत्वं न हेतुर्न पराक्रमः}
{न मित्रज्ञातिसम्बन्धः कारणं नात्मनो वशः} %4-25-7

\twolineshloka
{किं तु कालपरीणामो द्रष्टव्यः साधु पश्यता}
{धर्मश्चार्थश्च कामश्च कालक्रमसमाहिताः} %4-25-8

\twolineshloka
{इतः स्वां प्रकृतिं वाली गतः प्राप्तः क्रियाफलम्}
{सामदानार्थसंयोगैः पवित्रं प्लवगेश्वरः} %4-25-9

\twolineshloka
{स्वधर्मस्य च संयोगाज्जितस्तेन महात्मना}
{स्वर्गः परिगृहीतश्च प्राणानपरिरक्षता} %4-25-10

\twolineshloka
{एषा वै नियतिः श्रेष्ठा यां गतो हरियूथपः}
{तदलं परितापेन प्राप्तकालमुपास्यताम्} %4-25-11

\twolineshloka
{वचनान्ते तु रामस्य लक्ष्मणः परवीरहा}
{अवदत् प्रश्रितं वाक्यं सुग्रीवं गतचेतसम्} %4-25-12

\twolineshloka
{कुरु त्वमस्य सुग्रीव प्रेतकार्यमनन्तरम्}
{ताराङ्गदाभ्यां सहितो वालिनो दहनं प्रति} %4-25-13

\twolineshloka
{समाज्ञापय काष्ठानि शुष्काणि च बहूनि च}
{चन्दनानि च दिव्यानि वालिसंस्कारकारणात्} %4-25-14

\twolineshloka
{समाश्वासय दीनं त्वमङ्गदं दीनचेतसम्}
{मा भूर्बालिशबुद्धिस्त्वं त्वदधीनमिदं पुरम्} %4-25-15

\twolineshloka
{अङ्गदस्त्वानयेन्माल्यं वस्त्राणि विविधानि च}
{घृतं तैलमथो गन्धान् यच्चात्र समनन्तरम्} %4-25-16

\twolineshloka
{त्वं तार शिबिकां शीघ्रमादायागच्छ सम्भ्रमात्}
{त्वरा गुणवती युक्ता ह्यस्मिन् काले विशेषतः} %4-25-17

\twolineshloka
{सज्जीभवन्तु प्लवगाः शिबिकावाहनोचिताः}
{समर्था बलिनश्चैव निर्हरिष्यन्ति वालिनम्} %4-25-18

\twolineshloka
{एवमुक्त्वा तु सुग्रीवं सुमित्रानन्दवर्धनः}
{तस्थौ भ्रातृसमीपस्थो लक्ष्मणः परवीरहा} %4-25-19

\twolineshloka
{लक्ष्मणस्य वचः श्रुत्वा तारः सम्भ्रान्तमानसः}
{प्रविवेश गुहां शीघ्रं शिबिकासक्तमानसः} %4-25-20

\twolineshloka
{आदाय शिबिकां तारः स तु पर्यापतत् पुनः}
{वानरैरुह्यमानां तां शूरैरुद्वहनोचितैः} %4-25-21

\twolineshloka
{दिव्यां भद्रासनयुतां शिबिकां स्यन्दनोपमाम्}
{पक्षिकर्मभिराचित्रां द्रुमकर्मविभूषिताम्} %4-25-22

\twolineshloka
{आचितां चित्रपत्तीभिः सुनिविष्टां समन्ततः}
{विमानमिव सिद्धानां जालवातायनायुताम्} %4-25-23

\twolineshloka
{सुनियुक्तां विशालां च सुकृतां शिल्पिभिः कृताम्}
{दारुपर्वतकोपेतां चारुकर्मपरिष्कृताम्} %4-25-24

\twolineshloka
{वराभरणहारैश्च चित्रमाल्योपशोभिताम्}
{गुहागहनसंछन्नां रक्तचन्दनभूषिताम्} %4-25-25

\twolineshloka
{पुष्पौघैः समभिच्छन्नां पद्ममालाभिरेव च}
{तरुणादित्यवर्णाभिर्भ्राजमानाभिरावृताम्} %4-25-26

\twolineshloka
{ईदृशीं शिबिकां दृष्ट्वा रामो लक्ष्मणमब्रवीत्}
{क्षिप्रं विनीयतां वाली प्रेतकार्यं विधीयताम्} %4-25-27

\twolineshloka
{ततो वालिनमुद्यम्य सुग्रीवः शिबिकां तदा}
{आरोपयत विक्रोशन्नङ्गदेन सहैव तु} %4-25-28

\twolineshloka
{आरोप्य शिबिकां चैव वालिनं गतजीवितम्}
{अलंकारैश्च विविधैर्माल्यैर्वस्त्रैश्च भूषितम्} %4-25-29

\twolineshloka
{आज्ञापयत् तदा राजा सुग्रीवः प्लवगेश्वरः}
{और्ध्वदेहिकमार्यस्य क्रियतामनुकूलतः} %4-25-30

\twolineshloka
{विश्राणयन्तो रत्नानि विविधानि बहूनि च}
{अग्रतः प्लवगा यान्तु शिबिका तदनन्तरम्} %4-25-31

\twolineshloka
{राज्ञामृद्धिविशेषा हि दृश्यन्ते भुवि यादृशाः}
{तादृशैरिह कुर्वन्तु वानरा भर्तृसत्क्रियाम्} %4-25-32

\twolineshloka
{तादृशं वालिनः क्षिप्रं प्राकुर्वन्नौर्ध्वदेहिकम्}
{अङ्गदं परिरभ्याशु तारप्रभृतयस्तदा} %4-25-33

\twolineshloka
{क्रोशन्तः प्रययुः सर्वे वानरा हतबान्धवाः}
{ततः प्रणिहिताः सर्वा वानर्योऽस्य वशानुगाः} %4-25-34

\twolineshloka
{चुक्रुशुर्वीरवीरेति भूयः क्रोशन्ति ताः प्रियम्}
{ताराप्रभृतयः सर्वा वानर्यो हतबान्धवाः} %4-25-35

\twolineshloka
{अनुजग्मुश्च भर्तारं क्रोशन्त्यः करुणस्वनाः}
{तासां रुदितशब्देन वानरीणां वनान्तरे} %4-25-36

\twolineshloka
{वनानि गिरयश्चैव विक्रोशन्तीव सर्वतः}
{पुलिने गिरिनद्यास्तु विविक्ते जलसंवृते} %4-25-37

\twolineshloka
{चितां चक्रुः सुबहवो वानरा वनचारिणः}
{अवरोप्य ततः स्कन्धाच्छिबिकां वानरोत्तमाः} %4-25-38

\twolineshloka
{तस्थुरेकान्तमाश्रित्य सर्वे शोकपरायणाः}
{ततस्तारा पतिं दृष्ट्वा शिबिकातलशायिनम्} %4-25-39

\twolineshloka
{आरोप्याङ्के शिरस्तस्य विललाप सुदुःखिता}
{हा वानरमहाराज हा नाथ मम वत्सल} %4-25-40

\twolineshloka
{हा महार्ह महाबाहो हा मम प्रिय पश्य माम्}
{जनं न पश्यसीमं त्वं कस्माच्छोकाभिपीडितम्} %4-25-41

\twolineshloka
{प्रहृष्टमिह ते वक्त्रं गतासोरपि मानद}
{अस्तार्कसमवर्णं च दृश्यते जीवतो यथा} %4-25-42

\twolineshloka
{एष त्वां रामरूपेण कालः कर्षति वानर}
{येन स्म विधवाः सर्वाः कृता एकेषुणा रणे} %4-25-43

\twolineshloka
{इमास्तास्तव राजेन्द्र वानर्योऽप्लवगास्तव}
{पादैर्विकृष्टमध्वानमागताः किं न बुध्यसे} %4-25-44

\twolineshloka
{तवेष्टा ननु चैवेमा भार्याश्चन्द्रनिभाननाः}
{इदानीं नेक्षसे कस्मात् सुग्रीवं प्लवगेश्वर} %4-25-45

\twolineshloka
{एते हि सचिवा राजंस्तारप्रभृतयस्तव}
{पुरवासिजनश्चायं परिवार्य विषीदति} %4-25-46

\twolineshloka
{विसर्जयैनान् सचिवान् यथापुरमरिंदम}
{ततः क्रीडामहे सर्वा वनेषु मदनोत्कटाः} %4-25-47

\twolineshloka
{एवं विलपतीं तारां पतिशोकपरीवृताम्}
{उत्थापयन्ति स्म तदा वानर्यः शोककर्शिताः} %4-25-48

\twolineshloka
{सुग्रीवेण ततः सार्धं सोऽङ्गदः पितरं रुदन्}
{चितामारोपयामास शोकेनाभिप्लुतेन्द्रियः} %4-25-49

\twolineshloka
{ततोऽग्निं विधिवद् दत्त्वा सोऽपसव्यं चकार ह}
{पितरं दीर्घमध्वानं प्रस्थितं व्याकुलेन्द्रियः} %4-25-50

\twolineshloka
{संस्कृत्य वालिनं तं तु विधिवत् प्लवगर्षभाः}
{आजग्मुरुदकं कर्तुं नदीं शुभजलां शिवाम्} %4-25-51

\twolineshloka
{ततस्ते सहितास्तत्र ह्यङ्गदं स्थाप्य चाग्रतः}
{सुग्रीवतारासहिताः सिषिचुर्वालिने जलम्} %4-25-52

\twolineshloka
{सुग्रीवेणैव दीनेन दीनो भूत्वा महाबलः}
{समानशोकः काकुत्स्थः प्रेतकार्याण्यकारयत्} %4-25-53

\twolineshloka
{ततोऽथ तं वालिनमग्र्यपौरुषं प्रकाशमिक्ष्वाकुवरेषुणा हतम्}
{प्रदीप्य दीप्ताग्निसमौजसं तदा सलक्ष्मणं राममुपेयिवान् हरिः} %4-25-54


॥इत्यार्षे श्रीमद्रामायणे वाल्मीकीये आदिकाव्ये किष्किन्धाकाण्डे वालिसंस्कारः नाम पञ्चविंशः सर्गः ॥४-२५॥
