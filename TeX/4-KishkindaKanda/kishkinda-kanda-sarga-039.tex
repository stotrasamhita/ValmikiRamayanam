\sect{एकोनचत्वारिंशः सर्गः — सेनानिवेशः}

\twolineshloka
{इति ब्रुवाणं सुग्रीवं रामो धर्मभृतां वरः}
{बाहुभ्यां सम्परिष्वज्य प्रत्युवाच कृताञ्जलिम्} %4-39-1

\twolineshloka
{यदिन्द्रो वर्षते वर्षं न तच्चित्रं भविष्यति}
{आदित्योऽसौ सहस्रांशुः कुर्याद् वितिमिरं नभः} %4-39-2

\twolineshloka
{चन्द्रमा रजनीं कुर्यात् प्रभया सौम्य निर्मलाम्}
{त्वद्विधो वापि मित्राणां प्रीतिं कुर्यात् परंतप} %4-39-3

\twolineshloka
{एवं त्वयि न तच्चित्रं भवेद् यत् सौम्य शोभनम्}
{जानाम्यहं त्वां सुग्रीव सततं प्रियवादिनम्} %4-39-4

\twolineshloka
{त्वत्सनाथः सखे संख्ये जेतास्मि सकलानरीन्}
{त्वमेव मे सुहृन्मित्रं साहाय्यं कर्तुमर्हसि} %4-39-5

\twolineshloka
{जहारात्मविनाशाय मैथिलीं राक्षसाधमः}
{वञ्चयित्वा तु पौलोमीमनुह्लादो यथा शचीम्} %4-39-6

\twolineshloka
{नचिरात् तं वधिष्यामि रावणं निशितैः शरैः}
{पौलोम्याः पितरं दृप्तं शतक्रतुरिवारिहा} %4-39-7

\twolineshloka
{एतस्मिन्नन्तरे चैव रजः समभिवर्तत}
{उष्णतीव्रां सहस्रांशोश्छादयद् गगने प्रभाम्} %4-39-8

\twolineshloka
{दिशः पर्याकुलाश्चासंस्तमसा तेन दूषिताः}
{चचाल च मही सर्वा सशैलवनकानना} %4-39-9

\twolineshloka
{ततो नगेन्द्रसंकाशैस्तीक्ष्णदंष्ट्रैर्महाबलैः}
{कृत्स्ना संछादिता भूमिरसंख्येयैः प्लवंगमैः} %4-39-10

\twolineshloka
{निमेषान्तरमात्रेण ततस्तैर्हरियूथपैः}
{कोटीशतपरीवारैर्वानरैर्हरियूथपैः} %4-39-11

\twolineshloka
{नादेयैः पार्वतेयैश्च सामुद्रैश्च महाबलैः}
{हरिभिर्मेघनिर्ह्रादैरन्यैश्च वनवासिभिः} %4-39-12

\twolineshloka
{तरुणादित्यवर्णैश्च शशिगौरैश्च वानरैः}
{पद्मकेसरवर्णैश्च श्वेतैर्हेमकृतालयैः} %4-39-13

\twolineshloka
{कोटीसहस्रैर्दशभिः श्रीमान् परिवृतस्तदा}
{वीरः शतबलिर्नाम वानरः प्रत्यदृश्यत} %4-39-14

\twolineshloka
{ततः काञ्चनशैलाभस्ताराया वीर्यवान् पिता}
{अनेकैर्बहुसाहस्रैः कोटिभिः प्रत्यदृश्यत} %4-39-15

\twolineshloka
{तथापरेण कोटीनां सहस्रेण समन्वितः}
{पिता रुमायाः सम्प्राप्तः सुग्रीवश्वशुरो विभुः} %4-39-16

\twolineshloka
{पद्मकेसरसंकाशस्तरुणार्कनिभाननः}
{बुद्धिमान् वानरश्रेष्ठः सर्ववानरसत्तमः} %4-39-17

\twolineshloka
{अनेकैर्बहुसाहस्त्रैर्वानराणां समन्वितः}
{पिता हनुमतः श्रीमान् केसरी प्रत्यदृश्यत} %4-39-18

\twolineshloka
{गोलाङ्गूलमहाराजो गवाक्षो भीमविक्रमः}
{वृतः कोटिसहस्रेण वानराणामदृश्यत} %4-39-19

\twolineshloka
{ऋक्षाणां भीमवेगानां धूम्रः शत्रुनिबर्हणः}
{वृतः कोटिसहस्राभ्यां द्वाभ्यां समभिवर्तत} %4-39-20

\twolineshloka
{महाचलनिभैर्घोरैः पनसो नाम यूथपः}
{आजगाम महावीर्यस्तिसृभिः कोटिभिर्वृतः} %4-39-21

\twolineshloka
{नीलाञ्जनचयाकारो नीलो नामैष यूथपः}
{अदृश्यत महाकायः कोटिभिर्दशभिर्वृतः} %4-39-22

\twolineshloka
{ततः काञ्चनशैलाभो गवयो नाम यूथपः}
{आजगाम महावीर्यः कोटिभिः पञ्चभिर्वृतः} %4-39-23

\twolineshloka
{दरीमुखश्च बलवान् यूथपोऽभ्याययौ तदा}
{वृतः कोटिसहस्रेण सुग्रीवं समवस्थितः} %4-39-24

\twolineshloka
{मैन्दश्च द्विविदश्चोभावश्विपुत्रौ महाबलौ}
{कोटिकोटिसहस्रेण वानराणामदृश्यताम्} %4-39-25

\twolineshloka
{गजश्च बलवान् वीरस्तिसृभिः कोटिभिर्वृतः}
{आजगाम महातेजाः सुग्रीवस्य समीपतः} %4-39-26

\twolineshloka
{ऋक्षराजो महातेजा जाम्बवान्नाम नामतः}
{कोटिभिर्दशभिर्व्याप्तः सुग्रीवस्य वशे स्थितः} %4-39-27

\twolineshloka
{रुमणो नाम तेजस्वी विक्रान्तैर्वानरैर्वृतः}
{आगतो बलवांस्तूर्णं कोटीशतसमावृतः} %4-39-28

\twolineshloka
{ततः कोटिसहस्राणां सहस्रेण शतेन च}
{पृष्ठतोऽनुगतः प्राप्तो हरिभिर्गन्धमादनः} %4-39-29

\twolineshloka
{ततः पद्मसहस्रेण वृतः शङ्कुशतेन च}
{युवराजोऽङ्गदः प्राप्तः पितुस्तुल्यपराक्रमः} %4-39-30

\twolineshloka
{ततस्ताराद्युतिस्तारो हरिभिर्भीमविक्रमैः}
{पञ्चभिर्हरिकोटीभिर्दूरतः पर्यदृश्यत} %4-39-31

\twolineshloka
{इन्द्रजानुः कविर्वीरो यूथपः प्रत्यदृश्यत}
{एकादशानां कोटीनामीश्वरस्तैश्च संवृतः} %4-39-32

\twolineshloka
{ततो रम्भस्त्वनुप्राप्तस्तरुणादित्यसंनिभः}
{अयुतेन वृतश्चैव सहस्रेण शतेन च} %4-39-33

\twolineshloka
{ततो यूथपतिर्वीरो दुर्मुखो नाम वानरः}
{प्रत्यदृश्यत कोटीभ्यां द्वाभ्यां परिवृतो बली} %4-39-34

\twolineshloka
{कैलासशिखराकारैर्वानरैर्भीमविक्रमैः}
{वृतः कोटिसहस्रेण हनुमान् प्रत्यदृश्यत} %4-39-35

\twolineshloka
{नलश्चापि महावीर्यः संवृतो द्रुमवासिभिः}
{कोटीशतेन सम्प्राप्तः सहस्रेण शतेन च} %4-39-36

\twolineshloka
{ततो दधिमुखः श्रीमान् कोटिभिर्दशभिर्वृतः}
{सम्प्राप्तोऽभिनदंस्तस्य सुग्रीवस्य महात्मनः} %4-39-37

\twolineshloka
{शरभः कुमुदो वह्निर्वानरो रंह एव च}
{एते चान्ये च बहवो वानराः कामरूपिणः} %4-39-38

\twolineshloka
{आवृत्य पृथिवीं सर्वां पर्वतांश्च वनानि च}
{यूथपाः समनुप्राप्ता येषां संख्या न विद्यते} %4-39-39

\threelineshloka
{आगताश्च निविष्टाश्च पृथिव्यां सर्ववानराः}
{आप्लवन्तः प्लवन्तश्च गर्जन्तश्च प्लवंगमाः}
{अभ्यवर्तन्त सुग्रीवं सूर्यमभ्रगणा इव} %4-39-40

\twolineshloka
{कुर्वाणा बहुशब्दांश्च प्रकृष्टा बाहुशालिनः}
{शिरोभिर्वानरेन्द्राय सुग्रीवाय न्यवेदयन्} %4-39-41

\twolineshloka
{अपरे वानरश्रेष्ठाः संगम्य च यथोचितम्}
{सुग्रीवेण समागम्य स्थिताः प्राञ्जलयस्तदा} %4-39-42

\twolineshloka
{सुग्रीवस्त्वरितो रामे सर्वांस्तान् वानरर्षभान्}
{निवेदयित्वा धर्मज्ञः स्थितः प्राञ्जलिरब्रवीत्} %4-39-43

\twolineshloka
{यथासुखं पर्वतनिर्झरेषु वनेषु सर्वेषु च वानरेन्द्राः}
{निवेशयित्वा विधिवद् बलानि बलं बलज्ञः प्रतिपत्तुमीष्टे} %4-39-44


॥इत्यार्षे श्रीमद्रामायणे वाल्मीकीये आदिकाव्ये किष्किन्धाकाण्डे सेनानिवेशः नाम एकोनचत्वारिंशः सर्गः ॥४-३९॥
