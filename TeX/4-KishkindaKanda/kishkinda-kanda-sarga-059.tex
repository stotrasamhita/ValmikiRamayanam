\sect{एकोनषष्ठितमः सर्गः — सुपार्श्ववचनानुवादः}

\twolineshloka
{ततस्तदमृतास्वादं गृध्रराजेन भाषितम्}
{निशम्य वदता हृष्टास्ते वचः प्लवगर्षभाः} %4-59-1

\twolineshloka
{जाम्बवान् वानरश्रेष्ठः सह सर्वैः प्लवङ्गमैः}
{भूतलात् सहसोत्थाय गृध्रराजानमब्रवीत्} %4-59-2

\twolineshloka
{क्व सीता केन वा दृष्टा को वा हरति मैथिलीम्}
{तदाख्यातु भवान् सर्वं गतिर्भव वनौकसाम्} %4-59-3

\twolineshloka
{को दाशरथिबाणानां वज्रवेगनिपातिनाम्}
{स्वयं लक्ष्मणमुक्तानां न चिन्तयति विक्रमम्} %4-59-4

\twolineshloka
{स हरीन् प्रतिसम्मुक्तान् सीताश्रुतिसमाहितान्}
{पुनराश्वासयन् प्रीत इदं वचनमब्रवीत्} %4-59-5

\twolineshloka
{श्रूयतामिह वैदेह्या यथा मे हरणं श्रुतम्}
{येन चापि ममाख्यातं यत्र चायतलोचना} %4-59-6

\twolineshloka
{अहमस्मिन् गिरौ दुर्गे बहुयोजनमायते}
{चिरान्निपतितो वृद्धः क्षीणप्राणपराक्रमः} %4-59-7

\twolineshloka
{तं मामेवङ्गतं पुत्रः सुपार्श्वो नाम नामतः}
{आहारेण यथाकालं बिभर्ति पततां वरः} %4-59-8

\twolineshloka
{तीक्ष्णकामास्तु गन्धर्वास्तीक्ष्णकोपा भुजङ्गमाः}
{मृगाणां तु भयं तीक्ष्णं ततस्तीक्ष्णक्षुधा वयम्} %4-59-9

\twolineshloka
{स कदाचित् क्षुधार्तस्य ममाहाराभिकाङ्क्षिणः}
{गतसूर्येऽहनि प्राप्तो मम पुत्रो ह्यनामिषः} %4-59-10

\twolineshloka
{स मयाऽऽहारसंरोधात् पीडितः प्रीतिवर्धनः}
{अनुमान्य यथातत्त्वमिदं वचनमब्रवीत्} %4-59-11

\twolineshloka
{अहं तात यथाकालमामिषार्थी खमाप्लुतः}
{महेन्द्रस्य गिरेर्द्वारमावृत्य सुसमाश्रितः} %4-59-12

\twolineshloka
{तत्र सत्त्वसहस्राणां सागरान्तरचारिणाम्}
{पन्थानमेकोऽध्यवसं सन्निरोद्धुमवाङ्मुखः} %4-59-13

\twolineshloka
{तत्र कश्चिन्मया दृष्टः सूर्योदयसमप्रभाम्}
{स्त्रियमादाय गच्छन् वै भिन्नाञ्जनचयोपमः} %4-59-14

\twolineshloka
{सोऽहमभ्यवहारार्थं तौ दृष्ट्वा कृतनिश्चयः}
{तेन साम्ना विनीतेन पन्थानमनुयाचितः} %4-59-15

\twolineshloka
{नहि सामोपपन्नानां प्रहर्ता विद्यते भुवि}
{नीचेष्वपि जनः कश्चित् किमङ्ग बत मद्विधः} %4-59-16

\twolineshloka
{स यातस्तेजसा व्योम सङ्क्षिपन्निव वेगितः}
{अथाहं खेचरैर्भूतैरभिगम्य सभाजितः} %4-59-17

\twolineshloka
{दिष्ट्या जीवति सीतेति ह्यब्रुवन् मां महर्षयः}
{कथञ्चित् सकलत्रोऽसौ गतस्ते स्वस्त्यसंशयम्} %4-59-18

\twolineshloka
{एवमुक्तस्ततोऽहं तैः सिद्धैः परमशोभनैः}
{स च मे रावणो राजा रक्षसां प्रतिवेदितः} %4-59-19

\twolineshloka
{पश्यन् दाशरथेर्भार्यां रामस्य जनकात्मजाम्}
{भ्रष्टाभरणकौशेयां शोकवेगपराजिताम्} %4-59-20

\twolineshloka
{रामलक्ष्मणयोर्नाम क्रोशन्तीं मुक्तमूर्धजाम्}
{एष कालात्ययस्तात इति वाक्यविदां वरः} %4-59-21

\twolineshloka
{एतदर्थं समग्रं मे सुपार्श्वः प्रत्यवेदयत्}
{तच्छ्रुत्वापि हि मे बुद्धिर्नासीत् काचित् पराक्रमे} %4-59-22

\twolineshloka
{अपक्षो हि कथं पक्षी कर्म किञ्चित् समारभेत्}
{यत् तु शक्यं मया कर्तुं वाग्बुद्धिगुणवर्तिना} %4-59-23

\twolineshloka
{श्रूयतां तत्र वक्ष्यामि भवतां पौरुषाश्रयम्}
{वाङ्मतिभ्यां हि सर्वेषां करिष्यामि प्रियं हि वः} %4-59-24

\twolineshloka
{यद्धि दाशरथेः कार्यं मम तन्नात्र संशयः}
{तद् भवन्तो मतिश्रेष्ठा बलवन्तो मनस्विनः} %4-59-25

\twolineshloka
{प्रहिताः कपिराजेन देवैरपि दुरासदाः}
{रामलक्ष्मणबाणाश्च विहिताः कङ्कपत्रिणः} %4-59-26

\threelineshloka
{त्रयाणामपि लोकानां पर्याप्तास्त्राणनिग्रहे}
{कामं खलु दशग्रीवस्तेजोबलसमन्वितः}
{भवतां तु समर्थानां न किञ्चिदपि दुष्करम्} %4-59-27

\twolineshloka
{तदलं कालसङ्गेन क्रियतां बुद्धिनिश्चयः}
{नहि कर्मसु सज्जन्ते बुद्धिमन्तो भवद्विधाः} %4-59-28


॥इत्यार्षे श्रीमद्रामायणे वाल्मीकीये आदिकाव्ये किष्किन्धाकाण्डे सुपार्श्ववचनानुवादः नाम एकोनषष्ठितमः सर्गः ॥४-५९॥
