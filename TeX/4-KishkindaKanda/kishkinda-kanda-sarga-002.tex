\sect{द्वितीयः सर्गः — सुग्रीवमन्त्रः}

\twolineshloka
{तौ तु दृष्ट्वा महात्मानौ भ्रातरौ रामलक्ष्मणौ}
{वरायुधधरौ वीरौ सुग्रीवः शङ्कितोऽभवत्} %4-2-1

\twolineshloka
{उद्विग्नहृदयः सर्वा दिशः समवलोकयन्}
{न व्यतिष्ठत कस्मिंश्चिद् देशे वानरपुङ्गवः} %4-2-2

\twolineshloka
{नैव चक्रे मनः स्थातुं वीक्षमाणौ महाबलौ}
{कपेः परमभीतस्य चित्तं व्यवससाद ह} %4-2-3

\twolineshloka
{चिन्तयित्वा स धर्मात्मा विमृश्य गुरुलाघवम्}
{सुग्रीवः परमोद्विग्नः सर्वैस्तैर्वानरैः सह} %4-2-4

\twolineshloka
{ततः स सचिवेभ्यस्तु सुग्रीवः प्लवगाधिपः}
{शशंस परमोद्विग्नः पश्यंस्तौ रामलक्ष्मणौ} %4-2-5

\twolineshloka
{एतौ वनमिदं दुर्गं वालिप्रणिहितौ ध्रुवम्}
{छद्मना चीरवसनौ प्रचरन्ताविहागतौ} %4-2-6

\twolineshloka
{ततः सुग्रीवसचिवा दृष्ट्वा परमधन्विनौ}
{जग्मुर्गिरितटात् तस्मादन्यच्छिखरमुत्तमम्} %4-2-7

\twolineshloka
{ते क्षिप्रमभिगम्याथ यूथपा यूथपर्षभम्}
{हरयो वानरश्रेष्ठं परिवार्योपतस्थिरे} %4-2-8

\twolineshloka
{एवमेकायनगताः प्लवमाना गिरेर्गिरिम्}
{प्रकम्पयन्तो वेगेन गिरीणां शिखराणि च} %4-2-9

\twolineshloka
{ततः शाखामृगाः सर्वे प्लवमाना महाबलाः}
{बभञ्जुश्च नगांस्तत्र पुष्पितान् दुर्गमाश्रितान्} %4-2-10

\twolineshloka
{आप्लवन्तो हरिवराः सर्वतस्तं महागिरिम्}
{मृगमार्जारशार्दूलांस्त्रासयन्तो ययुस्तदा} %4-2-11

\twolineshloka
{ततः सुग्रीवसचिवाः पर्वतेन्द्रे समाहिताः}
{सङ्गम्य कपिमुख्येन सर्वे प्राञ्जलयः स्थिताः} %4-2-12

\twolineshloka
{ततस्तु भयसन्त्रस्तं वालिकिल्बिषशङ्कितम्}
{उवाच हनुमान् वाक्यं सुग्रीवं वाक्यकोविदः} %4-2-13

\twolineshloka
{सम्भ्रमस्त्यज्यतामेष सर्वैर्वालिकृते महान्}
{मलयोऽयं गिरिवरो भयं नेहास्ति वालिनः} %4-2-14

\twolineshloka
{यस्मादुद्विग्नचेतास्त्वं विद्रुतो हरिपुङ्गव}
{तं क्रूरदर्शनं क्रूरं नेह पश्यामि वालिनम्} %4-2-15

\twolineshloka
{यस्मात् तव भयं सौम्य पूर्वजात् पापकर्मणः}
{स नेह वाली दुष्टात्मा न ते पश्याम्यहं भयम्} %4-2-16

\twolineshloka
{अहो शाखामृगत्वं ते व्यक्तमेव प्लवङ्गम}
{लघुचित्ततयाऽऽत्मानं न स्थापयसि यो मतौ} %4-2-17

\twolineshloka
{बुद्धिविज्ञानसम्पन्न इङ्गितैः सर्वमाचर}
{नह्यबुद्धिं गतो राजा सर्वभूतानि शास्ति हि} %4-2-18

\twolineshloka
{सुग्रीवस्तु शुभं वाक्यं श्रुत्वा सर्वं हनूमतः}
{ततः शुभतरं वाक्यं हनूमन्तमुवाच ह} %4-2-19

\twolineshloka
{दीर्घबाहू विशालाक्षौ शरचापासिधारिणौ}
{कस्य न स्याद् भयं दृष्ट्वा ह्येतौ सुरसुतोपमौ} %4-2-20

\twolineshloka
{वालिप्रणिहितावेव शङ्केऽहं पुरुषोत्तमौ}
{राजानो बहुमित्राश्च विश्वासो नात्र हि क्षमः} %4-2-21

\twolineshloka
{अरयश्च मनुष्येण विज्ञेयाश्छद्मचारिणः}
{विश्वस्तानामविश्वस्ताश्छिद्रेषु प्रहरन्त्यपि} %4-2-22

\twolineshloka
{कृत्येषु वाली मेधावी राजानो बहुदर्शिनः}
{भवन्त परहन्तारस्ते ज्ञेयाः प्राकृतैर्नरैः} %4-2-23

\twolineshloka
{तौ त्वया प्राकृतेनेव गत्वा ज्ञेयौ प्लवङ्गम}
{इङ्गितानां प्रकारैश्च रूपव्याभाषणेन च} %4-2-24

\twolineshloka
{लक्षयस्व तयोर्भावं प्रहृष्टमनसौ यदि}
{विश्वासयन् प्रशंसाभिरिङ्गितैश्च पुनः पुनः} %4-2-25

\twolineshloka
{ममैवाभिमुखं स्थित्वा पृच्छ त्वं हरिपुङ्गव}
{प्रयोजनं प्रवेशस्य वनस्यास्य धनुर्धरौ} %4-2-26

\twolineshloka
{शुद्धात्मानौ यदि त्वेतौ जानीहि त्वं प्लवङ्गम}
{व्याभाषितैर्वा रूपैर्वा विज्ञेया दुष्टतानयोः} %4-2-27

\twolineshloka
{इत्येवं कपिराजेन सन्दिष्टो मारुतात्मजः}
{चकार गमने बुद्धिं यत्र तौ रामलक्ष्मणौ} %4-2-28

\twolineshloka
{तथेति सम्पूज्य वचस्तु तस्य कपेः सुभीतस्य दुरासदस्य}
{महानुभावो हनुमान् ययौ तदा स यत्र रामोऽतिबली सलक्ष्मणः} %4-2-29


॥इत्यार्षे श्रीमद्रामायणे वाल्मीकीये आदिकाव्ये किष्किन्धाकाण्डे सुग्रीवमन्त्रः नाम द्वितीयः सर्गः ॥४-२॥
