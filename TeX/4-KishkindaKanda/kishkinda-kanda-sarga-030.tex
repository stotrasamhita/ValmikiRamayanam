\sect{त्रिंशः सर्गः — शरद्वर्णनम्}

\twolineshloka
{गृहं प्रविष्टे सुग्रीवे विमुक्ते गगने घनैः}
{वर्षरात्रे स्थितो रामः कामशोकाभिपीडितः} %4-30-1

\twolineshloka
{पाण्डुरं गगनं दृष्ट्वा विमलं चन्द्रमण्डलम्}
{शारदीं रजनीं चैव दृष्ट्वा ज्योत्स्नानुलेपनाम्} %4-30-2

\twolineshloka
{कामवृत्तं च सुग्रीवं नष्टां च जनकात्मजाम्}
{दृष्ट्वा कालमतीतं च मुमोह परमातुरः} %4-30-3

\twolineshloka
{स तु संज्ञामुपागम्य मुहूर्तान्मतिमान् नृपः}
{मनःस्थामपि वैदेहीं चिन्तयामास राघवः} %4-30-4

\twolineshloka
{दृष्ट्वा च विमलं व्योम गतविद्युद्बलाहकम्}
{सारसारावसंघुष्टं विललापार्तया गिरा} %4-30-5

\twolineshloka
{आसीनः पर्वतस्याग्रे हेमधातुविभूषिते}
{शारदं गगनं दृष्ट्वा जगाम मनसा प्रियाम्} %4-30-6

\twolineshloka
{सारसारावसंनादैः सारसारावनादिनी}
{याऽऽश्रमे रमते बाला साद्य मे रमते कथम्} %4-30-7

\twolineshloka
{पुष्पितांश्चासनान् दृष्ट्वा काञ्चनानिव निर्मलान्}
{कथं सा रमते बाला पश्यन्ती मामपश्यती} %4-30-8

\twolineshloka
{या पुरा कलहंसानां कलेन कलभाषिणी}
{बुध्यते चारुसर्वाङ्गी साद्य मे रमते कथम्} %4-30-9

\twolineshloka
{निःस्वनं चक्रवाकानां निशम्य सहचारिणाम्}
{पुण्डरीकविशालाक्षी कथमेषा भविष्यति} %4-30-10

\twolineshloka
{सरांसि सरितो वापीः काननानि वनानि च}
{तां विना मृगशावाक्षीं चरन्नाद्य सुखं लभे} %4-30-11

\twolineshloka
{अपि तां मद्वियोगाच्च सौकुमार्याच्च भामिनीम्}
{सुदूरं पीडयेत् कामः शरद्गुणनिरन्तरः} %4-30-12

\twolineshloka
{एवमादि नरश्रेष्ठो विललाप नृपात्मजः}
{विहंग इव सारङ्गः सलिलं त्रिदशेश्वरात्} %4-30-13

\twolineshloka
{ततश्चञ्चूर्य रम्येषु फलार्थी गिरिसानुषु}
{ददर्श पर्युपावृत्तो लक्ष्मीवाँल्लक्ष्मणोऽग्रजम्} %4-30-14

\twolineshloka
{स चिन्तया दुस्सहया परीतं विसंज्ञमेकं विजने मनस्वी}
{भ्रातुर्विषादात् त्वरितोऽतिदीनः समीक्ष्य सौमित्रिरुवाच दीनम्} %4-30-15

\twolineshloka
{किमार्य कामस्य वशंगतेन किमात्मपौरुष्यपराभवेन}
{अयं ह्रिया संह्रियते समाधिः किमत्र योगेन निवर्तते न} %4-30-16

\twolineshloka
{क्रियाभियोगं मनसः प्रसादं समाधियोगानुगतं च कालम्}
{सहायसामर्थ्यमदीनसत्त्वः स्वकर्महेतुं च कुरुष्व तात} %4-30-17

\twolineshloka
{न जानकी मानववंशनाथ त्वया सनाथा सुलभा परेण}
{न चाग्निचूडां ज्वलितामुपेत्य न दह्यते वीर वरार्ह कश्चित्} %4-30-18

\twolineshloka
{सलक्षणं लक्ष्मणमप्रधृष्यं स्वभावजं वाक्यमुवाच रामः}
{हितं च पथ्यं च नयप्रसक्तं ससामधर्मार्थसमाहितं च} %4-30-19

\twolineshloka
{निस्संशयं कार्यमवेक्षितव्यं क्रियाविशेषोऽप्यनुवर्तितव्यः}
{न तु प्रवृद्धस्य दुरासदस्य कुमार वीर्यस्य फलं च चिन्त्यम्} %4-30-20

\twolineshloka
{अथ पद्मपलाशाक्षीं मैथिलीमनुचिन्तयन्}
{उवाच लक्ष्मणं रामो मुखेन परिशुष्यता} %4-30-21

\twolineshloka
{तर्पयित्वा सहस्राक्षः सलिलेन वसुंधराम्}
{निर्वर्तयित्वा सस्यानि कृतकर्मा व्यवस्थितः} %4-30-22

\twolineshloka
{दीर्घगम्भीरनिर्घोषाः शैलद्रुमपुरोगमाः}
{विसृज्य सलिलं मेघाः परिशान्ता नृपात्मज} %4-30-23

\twolineshloka
{नीलोत्पलदलश्यामाः श्यामीकृत्वा दिशो दश}
{विमदा इव मातङ्गाः शान्तवेगाः पयोधराः} %4-30-24

\twolineshloka
{जलगर्भा महावेगाः कुटजार्जुनगन्धिनः}
{चरित्वा विरताः सौम्य वृष्टिवाताः समुद्यताः} %4-30-25

\twolineshloka
{घनानां वारणानां च मयूराणां च लक्ष्मण}
{नादः प्रस्रवणानां च प्रशान्तः सहसानघ} %4-30-26

\twolineshloka
{अभिवृष्टा महामेघैर्निर्मलाश्चित्रसानवः}
{अनुलिप्ता इवाभान्ति गिरयश्चन्द्ररश्मिभिः} %4-30-27

\twolineshloka
{शाखासु सप्तच्छदपादपानां प्रभासु तारार्कनिशाकराणाम्}
{लीलासु चैवोत्तमवारणानां श्रियं विभज्याद्य शरत्प्रवृत्ता} %4-30-28

\twolineshloka
{सम्प्रत्यनेकाश्रयचित्रशोभा लक्ष्मीः शरत्कालगुणोपपन्ना}
{सूर्याग्रहस्तप्रतिबोधितेषु पद्माकरेष्वभ्यधिकं विभाति} %4-30-29

\twolineshloka
{सप्तच्छदानां कुसुमोपगन्धी षट्पादवृन्दैरनुगीयमानः}
{मत्तद्विपानां पवनानुसारी दर्पं विनेष्यन्नधिकं विभाति} %4-30-30

\twolineshloka
{अभ्यागतैश्चारुविशालपक्षैः स्मरप्रियैः पद्मरजोऽवकीर्णैः}
{महानदीनां पुलिनोपयातैः क्रीडन्ति हंसाः सह चक्रवाकैः} %4-30-31

\twolineshloka
{मदप्रगल्भेषु च वारणेषु गवां समूहेषु च दर्पितेषु}
{प्रसन्नतोयासु च निम्नगासु विभाति लक्ष्मीर्बहुधा विभक्ता} %4-30-32

\twolineshloka
{नभः समीक्ष्याम्बुधरैर्विमुक्तं विमुक्तबर्हाभरणा वनेषु}
{प्रियास्वरक्ता विनिवृत्तशोभा गतोत्सवा ध्यानपरा मयूराः} %4-30-33

\twolineshloka
{मनोज्ञगन्धैः प्रियकैरनल्पैः पुष्पातिभारावनताग्रशाखैः}
{सुवर्णगौरैर्नयनाभिरामैरुद्योतितानीव वनान्तराणि} %4-30-34

\twolineshloka
{प्रियान्वितानां नलिनीप्रियाणां वने प्रियाणां कुसुमोद्गतानाम्}
{मदोत्कटानां मदलालसानां गजोत्तमानां गतयोऽद्य मन्दाः} %4-30-35

\twolineshloka
{व्यक्तं नभः शस्त्रविधौतवर्णं कृशप्रवाहानि नदीजलानि}
{कह्लारशीताः पवनाः प्रवान्ति तमो विमुक्ताश्च दिशः प्रकाशाः} %4-30-36

\twolineshloka
{सूर्यातपक्रामणनष्टपङ्का भूमिश्चिरोद्घाटितसान्द्ररेणुः}
{अन्योन्यवैरेण समायुतानामुद्योगकालोऽद्य नराधिपानाम्} %4-30-37

\twolineshloka
{शरद्गुणाप्यायितरूपशोभाः प्रहर्षिताः पांसुसमुत्थिताङ्गाः}
{मदोत्कटाः सम्प्रति युद्धलुब्धा वृषा गवां मध्यगता नदन्ति} %4-30-38

\twolineshloka
{समन्मथा तीव्रतरानुरागा कुलान्विता मन्दगतिः करेणुः}
{मदान्वितं सम्परिवार्य यान्तं वनेषु भर्तारमनुप्रयाति} %4-30-39

\twolineshloka
{त्यक्त्वा वराण्यात्मविभूषितानि बर्हाणि तीरोपगता नदीनाम्}
{निर्भर्त्स्यमाना इव सारसौघैः प्रयान्ति दीना विमना मयूराः} %4-30-40

\twolineshloka
{वित्रास्य कारण्डवचक्रवाकान् महारवैर्भिन्नकटा गजेन्द्राः}
{सरस्सुबद्धाम्बुजभूषणेषु विक्षोभ्य विक्षोभ्य जलं पिबन्ति} %4-30-41

\twolineshloka
{व्यपेतपङ्कासु सवालुकासु प्रसन्नतोयासु सगोकुलासु}
{ससारसारावविनादितासु नदीषु हंसा निपतन्ति हृष्टाः} %4-30-42

\twolineshloka
{नदीघनप्रस्रवणोदकानामतिप्रवृद्धानिलबर्हिणानाम्}
{प्लवंगमानां च गतोत्सवानां ध्रुवं रवाः सम्प्रति सम्प्रणष्टाः} %4-30-43

\twolineshloka
{अनेकवर्णाः सुविनष्टकाया नवोदितेष्वम्बुधरेषु नष्टाः}
{क्षुधार्दिता घोरविषा बिलेभ्यश्चिरोषिता विप्रसरन्ति सर्पाः} %4-30-44

\twolineshloka
{चञ्चच्चन्द्रकरस्पर्शहर्षोन्मीलिततारका}
{अहो रागवती संध्या जहाति स्वयमम्बरम्} %4-30-45

\twolineshloka
{रात्रिः शशाङ्कोदितसौम्यवक्त्रा तारागणोन्मीलितचारुनेत्रा}
{ज्योत्स्नांशुकप्रावरणा विभाति नारीव शुक्लांशुकसंवृताङ्गी} %4-30-46

\twolineshloka
{विपक्वशालिप्रसवानि भुक्त्वा प्रहर्षिता सारसचारुपङ्क्तिः}
{नभः समाक्रामति शीघ्रवेगा वातावधूता ग्रथितेव माला} %4-30-47

\twolineshloka
{सुप्तैकहंसं कुमुदैरुपेतं महाह्रदस्थं सलिलं विभाति}
{घनैर्विमुक्तं निशि पूर्णचन्द्रं तारागणाकीर्णमिवान्तरिक्षम्} %4-30-48

\twolineshloka
{प्रकीर्णहंसाकुलमेखलानां प्रबुद्धपद्मोत्पलमालिनीनाम्}
{वाप्युत्तमानामधिकाद्य लक्ष्मीर्वराङ्गनानामिव भूषितानाम्} %4-30-49

\twolineshloka
{वेणुस्वरव्यञ्जिततूर्यमिश्रः प्रत्यूषकालेऽनिलसम्प्रवृत्तः}
{सम्मूर्छितो गर्गरगोवृषाणामन्योन्यमापूरयतीव शब्दः} %4-30-50

\twolineshloka
{नवैर्नदीनां कुसुमप्रहासैर्व्याधूयमानैर्मृदुमारुतेन}
{धौतामलक्षौमपटप्रकाशैः कूलानि काशैरुपशोभितानि} %4-30-51

\twolineshloka
{वनप्रचण्डा मधुपानशौण्डाः प्रियान्विताः षट्चरणाः प्रहृष्टाः}
{वनेषु मत्ताः पवनानुयात्रां कुर्वन्ति पद्मासनरेणुगौराः} %4-30-52

\twolineshloka
{जलं प्रसन्नं कुसुमप्रहासं क्रौञ्चस्वनं शालिवनं विपक्वम्}
{मृदुश्च वायुर्विमलश्च चन्द्रः शंसन्ति वर्षव्यपनीतकालम्} %4-30-53

\twolineshloka
{मीनोपसंदर्शितमेखलानां नदीवधूनां गतयोऽद्य मन्दाः}
{कान्तोपभुक्तालसगामिनीनां प्रभातकालेष्विव कामिनीनाम्} %4-30-54

\twolineshloka
{सचक्रवाकानि सशैवलानि काशैर्दुकूलैरिव संवृतानि}
{सपत्ररेखाणि सरोचनानि वधूमुखानीव नदीमुखानि} %4-30-55

\twolineshloka
{प्रफुल्लबाणासनचित्रितेषु प्रहृष्टषट्पादनिकूजितेषु}
{गृहीतचापोद्यतदण्डचण्डः प्रचण्डचापोऽद्य वनेषु कामः} %4-30-56

\twolineshloka
{लोकं सुवृष्ट्या परितोषयित्वा नदीस्तटाकानि च पूरयित्वा}
{निष्पन्नसस्यां वसुधां च कृत्वा त्यक्त्वा नभस्तोयधराः प्रणष्टाः} %4-30-57

\twolineshloka
{दर्शयन्ति शरन्नद्यः पुलिनानि शनैः शनैः}
{नवसंगमसव्रीडा जघनानीव योषितः} %4-30-58

\twolineshloka
{प्रसन्नसलिलाः सौम्य कुरराभिविनादिताः}
{चक्रवाकगणाकीर्णा विभान्ति सलिलाशयाः} %4-30-59

\twolineshloka
{अन्योन्यबद्धवैराणां जिगीषूणां नृपात्मज}
{उद्योगसमयः सौम्य पार्थिवानामुपस्थितः} %4-30-60

\twolineshloka
{इयं सा प्रथमा यात्रा पार्थिवानां नृपात्मज}
{न च पश्यामि सुग्रीवमुद्योगं च तथाविधम्} %4-30-61

\twolineshloka
{असनाः सप्तपर्णाश्च कोविदाराश्च पुष्पिताः}
{दृश्यन्ते बन्धुजीवाश्च श्यामाश्च गिरिसानुषु} %4-30-62

\twolineshloka
{हंससारसचक्राह्वैः कुररैश्च समन्ततः}
{पुलिनान्यवकीर्णानि नदीनां पश्य लक्ष्मण} %4-30-63

\twolineshloka
{चत्वारो वार्षिका मासा गता वर्षशतोपमाः}
{मम शोकाभितप्तस्य तथा सीतामपश्यतः} %4-30-64

\twolineshloka
{चक्रवाकीव भर्तारं पृष्ठतोऽनुगता वनम्}
{विषमं दण्डकारण्यमुद्यानमिव चाङ्गना} %4-30-65

\twolineshloka
{प्रियाविहीने दुःखार्ते हृतराज्ये विवासिते}
{कृपां न कुरुते राजा सुग्रीवो मयि लक्ष्मण} %4-30-66

\twolineshloka
{अनाथो हृतराज्योऽहं रावणेन च धर्षितः}
{दीनो दूरगृहः कामी मां चैव शरणं गतः} %4-30-67

\twolineshloka
{इत्येतैः कारणैः सौम्य सुग्रीवस्य दुरात्मनः}
{अहं वानरराजस्य परिभूतः परंतपः} %4-30-68

\twolineshloka
{स कालं परिसंख्याय सीतायाः परिमार्गणे}
{कृतार्थः समयं कृत्वा दुर्मतिर्नाववुध्यते} %4-30-69

\twolineshloka
{स किष्किन्धां प्रविश्य त्वं ब्रूहि वानरपुङ्गवम्}
{मूर्खं ग्राम्यसुखे सक्तं सुग्रीवं वचनान्मम} %4-30-70

\twolineshloka
{अर्थिनामुपपन्नानां पूर्वं चाप्युपकारिणाम्}
{आशां संश्रुत्य यो हन्ति स लोके पुरुषाधमः} %4-30-71

\twolineshloka
{शुभं वा यदि वा पापं यो हि वाक्यमुदीरितम्}
{सत्येन परिगृह्णाति स वीरः पुरुषोत्तमः} %4-30-72

\twolineshloka
{कृतार्था ह्यकृतार्थानां मित्राणां न भवन्ति ये}
{तान् मृतानपि क्रव्यादाः कृतघ्नान् नोपभुञ्जते} %4-30-73

\twolineshloka
{नूनं काञ्चनपृष्ठस्य विकृष्टस्य मया रणे}
{द्रष्टुमिच्छसि चापस्य रूपं विद्युद्गणोपमम्} %4-30-74

\twolineshloka
{घोरं ज्यातलनिर्घोषं क्रुद्धस्य मम संयुगे}
{निर्घोषमिव वज्रस्य पुनः संश्रोतुमिच्छसि} %4-30-75

\twolineshloka
{काममेवंगतेऽप्यस्य परिज्ञाते पराक्रमे}
{त्वत्सहायस्य मे वीर न चिन्ता स्यान्नृपात्मज} %4-30-76

\twolineshloka
{यदर्थमयमारम्भः कृतः परपुरंजय}
{समयं नाभिजानाति कृतार्थः प्लवगेश्वरः} %4-30-77

\twolineshloka
{वर्षाः समयकालं तु प्रतिज्ञाय हरीश्वरः}
{व्यतीतांश्चतुरो मासान् विहरन् नावबुध्यते} %4-30-78

\twolineshloka
{सामात्यपरिषत्क्रीडन् पानमेवोपसेवते}
{शोकदीनेषु नास्मासु सुग्रीवः कुरुते दयाम्} %4-30-79

\twolineshloka
{उच्यतां गच्छ सुग्रीवस्त्वया वीर महाबल}
{मम रोषस्य यद्रूपं ब्रूयाश्चैनमिदं वचः} %4-30-80

\twolineshloka
{न स संकुचितः पन्था येन वाली हतो गतः}
{समये तिष्ठ सुग्रीव मा वालिपथमन्वगाः} %4-30-81

\twolineshloka
{एक एव रणे वाली शरेण निहतो मया}
{त्वां तु सत्यादतिक्रान्तं हनिष्यामि सबान्धवम्} %4-30-82

\twolineshloka
{यदेवं विहिते कार्ये यद्धितं पुरुषर्षभ}
{तत् तद् ब्रूहि नरश्रेष्ठ त्वर कालव्यतिक्रमः} %4-30-83

\twolineshloka
{कुरुष्व सत्यं मम वानरेश्वर प्रतिश्रुतं धर्ममवेक्ष्य शाश्वतम्}
{मा वालिनं प्रेतगतो यमक्षये त्वमद्य पश्येर्मम चोदितः शरैः} %4-30-84

\twolineshloka
{स पूर्वजं तीव्रविवृद्धकोपं लालप्यमानं प्रसमीक्ष्य दीनम्}
{चकार तीव्रां मतिमुग्रतेजा हरीश्वरे मानववंशवर्धनः} %4-30-85


॥इत्यार्षे श्रीमद्रामायणे वाल्मीकीये आदिकाव्ये किष्किन्धाकाण्डे शरद्वर्णनम् नाम त्रिंशः सर्गः ॥४-३०॥
