\sect{पञ्चदशः सर्गः — ताराहितोक्तिः}

\twolineshloka
{अथ तस्य निनादं तं सुग्रीवस्य महात्मनः}
{शुश्रावान्तःपुरगतो वाली भ्रातुरमर्षणः} %4-15-1

\twolineshloka
{श्रुत्वा तु तस्य निनदं सर्वभूतप्रकम्पनम्}
{मदश्चैकपदे नष्टः क्रोधश्चापादितो महान्} %4-15-2

\twolineshloka
{ततो रोषपरीताङ्गो वाली स कनकप्रभः}
{उपरक्त इवादित्यः सद्यो निष्प्रभतां गतः} %4-15-3

\twolineshloka
{वाली दंष्ट्राकरालस्तु क्रोधाद् दीप्ताग्निलोचनः}
{भात्युत्पतितपद्माभः समृणाल इव ह्रदः} %4-15-4

\twolineshloka
{शब्दं दुर्मर्षणं श्रुत्वा निष्पपात ततो हरिः}
{वेगेन च पदन्यासैर्दारयन्निव मेदिनीम्} %4-15-5

\twolineshloka
{तं तु तारा परिष्वज्य स्नेहाद् दर्शितसौहृदा}
{उवाच त्रस्तसम्भ्रान्ता हितोदर्कमिदं वचः} %4-15-6

\twolineshloka
{साधु क्रोधमिमं वीर नदीवेगमिवागतम्}
{शयनादुत्थितः काल्यं त्यज भुक्तामिव स्रजम्} %4-15-7

\twolineshloka
{काल्यमेतेन संग्रामं करिष्यसि च वानर}
{वीर ते शत्रुबाहुल्यं फल्गुता वा न विद्यते} %4-15-8

\twolineshloka
{सहसा तव निष्क्रामो मम तावन्न रोचते}
{श्रूयतामभिधास्यामि यन्निमित्तं निवार्यते} %4-15-9

\twolineshloka
{पूर्वमापतितः क्रोधात् स त्वामाह्वयते युधि}
{निष्पत्य च निरस्तस्ते हन्यमानो दिशो गतः} %4-15-10

\twolineshloka
{त्वया तस्य निरस्तस्य पीडितस्य विशेषतः}
{इहैत्य पुनराह्वानं शङ्कां जनयतीव मे} %4-15-11

\twolineshloka
{दर्पश्च व्यवसायश्च यादृशस्तस्य नर्दतः}
{निनादस्य च संरम्भो नैतदल्पं हि कारणम्} %4-15-12

\twolineshloka
{नासहायमहं मन्ये सुग्रीवं तमिहागतम्}
{अवष्टब्धसहायश्च यमाश्रित्यैष गर्जति} %4-15-13

\twolineshloka
{प्रकृत्या निपुणश्चैव बुद्धिमांश्चैव वानरः}
{नापरीक्षितवीर्येण सुग्रीवः सख्यमेष्यति} %4-15-14

\twolineshloka
{पूर्वमेव मया वीर श्रुतं कथयतो वचः}
{अङ्गदस्य कुमारस्य वक्ष्याम्यद्य हितं वचः} %4-15-15

\twolineshloka
{अङ्गदस्तु कुमारोऽयं वनान्तमुपनिर्गतः}
{प्रवृत्तिस्तेन कथिता चारैरासीन्निवेदिता} %4-15-16

\twolineshloka
{अयोध्याधिपतेः पुत्रौ शूरौ समरदुर्जयौ}
{इक्ष्वाकूणां कुले जातौ प्रथितौ रामलक्ष्मणौ} %4-15-17

\twolineshloka
{सुग्रीवप्रियकामार्थं प्राप्तौ तत्र दुरासदौ}
{स ते भ्रातुर्हि विख्यातः सहायो रणकर्मणि} %4-15-18

\twolineshloka
{रामः परबलामर्दी युगान्ताग्निरिवोत्थितः}
{निवासवृक्षः साधूनामापन्नानां परा गतिः} %4-15-19

\twolineshloka
{आर्तानां संश्रयश्चैव यशसश्चैकभाजनम्}
{ज्ञानविज्ञानसम्पन्नो निदेशे निरतः पितुः} %4-15-20

\twolineshloka
{धातूनामिव शैलेन्द्रो गुणानामाकरो महान्}
{तत् क्षमो न विरोधस्ते सह तेन महात्मना} %4-15-21

\twolineshloka
{दुर्जयेनाप्रमेयेण रामेण रणकर्मसु}
{शूर वक्ष्यामि ते किंचिन्न चेच्छाम्यभ्यसूयितुम्} %4-15-22

\twolineshloka
{श्रूयतां क्रियतां चैव तव वक्ष्यामि यद्धितम्}
{यौवराज्येन सुग्रीवं तूर्णं साध्वभिषेचय} %4-15-23

\twolineshloka
{विग्रहं मा कृथा वीर भ्रात्रा राजन् यवीयसा}
{अहं हि ते क्षमं मन्ये तेन रामेण सौहृदम्} %4-15-24

\twolineshloka
{सुग्रीवेण च सम्प्रीतिं वैरमुत्सृज्य दूरतः}
{लालनीयो हि ते भ्राता यवीयानेष वानरः} %4-15-25

\twolineshloka
{तत्र वा सन्निहस्थो वा सर्वथा बन्धुरेव ते}
{नहि तेन समं बन्धुं भुवि पश्यामि कंचन} %4-15-26

\twolineshloka
{दानमानादिसत्कारैः कुरुष्व प्रत्यनन्तरम्}
{वैरमेतत् समुत्सृज्य तव पार्श्वे स तिष्ठतु} %4-15-27

\twolineshloka
{सुग्रीवो विपुलग्रीवो महाबन्धुर्मतस्तव}
{भ्रातृसौहृदमालम्ब्य नान्या गतिरिहास्ति ते} %4-15-28

\twolineshloka
{यदि ते मत्प्रियं कार्यं यदि चावैषि मां हिताम्}
{याच्यमानः प्रियत्वेन साधु वाक्यं कुरुष्व मे} %4-15-29

\twolineshloka
{प्रसीद पथ्यं शृणु जल्पितं हि मे न रोषमेवानुविधातुमर्हसि}
{क्षमो हि ते कोशलराजसूनुना न विग्रहः शक्रसमानतेजसा} %4-15-30

\twolineshloka
{तदा हि तारा हितमेव वाक्यं तं वालिनं पथ्यमिदं बभाषे}
{न रोचते तद् वचनं हि तस्य कालाभिपन्नस्य विनाशकाले} %4-15-31


॥इत्यार्षे श्रीमद्रामायणे वाल्मीकीये आदिकाव्ये किष्किन्धाकाण्डे ताराहितोक्तिः नाम पञ्चदशः सर्गः ॥४-१५॥
