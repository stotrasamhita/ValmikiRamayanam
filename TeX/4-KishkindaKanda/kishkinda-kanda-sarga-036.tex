\sect{षड्त्रिंशः सर्गः — सुग्रीवलक्ष्मणानुरोधः}

\twolineshloka
{इत्युक्तस्तारया वाक्यं प्रश्रितं धर्मसंहितम्}
{मृदुस्वभावः सौमित्रिः प्रतिजग्राह तद्वचः} %4-36-1

\twolineshloka
{तस्मिन् प्रतिगृहीते तु वाक्ये हरिगणेश्वरः}
{लक्ष्मणात् सुमहत् त्रासं वस्त्रं क्लिन्नमिवात्यजत्} %4-36-2

\twolineshloka
{ततः कण्ठगतं माल्यं चित्रं बहुगुणं महत्}
{चिच्छेद विमदश्चासीत् सुग्रीवो वानरेश्वरः} %4-36-3

\twolineshloka
{स लक्ष्मणं भीमबलं सर्ववानरसत्तमः}
{अब्रवीत् प्रश्रितं वाक्यं सुग्रीवः सम्प्रहर्षयन्} %4-36-4

\twolineshloka
{प्रणष्टा श्रीश्च कीर्तिश्च कपिराज्यं च शाश्वतम्}
{रामप्रसादात् सौमित्रे पुनश्चाप्तमिदं मया} %4-36-5

\twolineshloka
{कः शक्तस्तस्य देवस्य ख्यातस्य स्वेन कर्मणा}
{तादृशं प्रतिकुर्वीत अंशेनापि नृपात्मज} %4-36-6

\twolineshloka
{सीतां प्राप्स्यति धर्मात्मा वधिष्यति च रावणम्}
{सहायमात्रेण मया राघवः स्वेन तेजसा} %4-36-7

\twolineshloka
{सहायकृत्यं किं तस्य येन सप्त महाद्रुमाः}
{गिरिश्च वसुधा चैव बाणेनैकेन दारिताः} %4-36-8

\twolineshloka
{धनुर्विस्फारमाणस्य यस्य शब्देन लक्ष्मण}
{सशैला कम्पिता भूमिः सहायैः किं नु तस्य वै} %4-36-9

\twolineshloka
{अनुयात्रां नरेन्द्रस्य करिष्येऽहं नरर्षभ}
{गच्छतो रावणं हन्तुं वैरिणं सपुरस्सरम्} %4-36-10

\twolineshloka
{यदि किंचिदतिक्रान्तं विश्वासात् प्रणयेन वा}
{प्रेष्यस्य क्षमितव्यं मे न कश्चिन्नापराध्यति} %4-36-11

\twolineshloka
{इति तस्य ब्रुवाणस्य सुग्रीवस्य महात्मनः}
{अभवल्लक्ष्मणः प्रीतः प्रेम्णा चेदमुवाच ह} %4-36-12

\twolineshloka
{सर्वथा हि मम भ्राता सनाथो वानरेश्वर}
{त्वया नाथेन सुग्रीव प्रश्रितेन विशेषतः} %4-36-13

\twolineshloka
{यस्ते प्रभावः सुग्रीव यच्च ते शौचमीदृशम्}
{अर्हस्त्वं कपिराज्यस्य श्रियं भोक्तुमनुत्तमाम्} %4-36-14

\twolineshloka
{सहायेन च सुग्रीव त्वया रामः प्रतापवान्}
{वधिष्यति रणे शत्रूनचरान्नात्र संशयः} %4-36-15

\twolineshloka
{धर्मज्ञस्य कृतज्ञस्य संग्रामेष्वनिवर्तिनः}
{उपपन्नं च युक्तं च सुग्रीव तव भाषितम्} %4-36-16

\twolineshloka
{दोषज्ञः सति सामर्थ्ये कोऽन्यो भाषितुमर्हति}
{वर्जयित्वा मम ज्येष्ठं त्वां च वानरसत्तम} %4-36-17

\twolineshloka
{सदृशश्चासि रामेण विक्रमेण बलेन च}
{सहायो दैवतैर्दत्तश्चिराय हरिपुंगव} %4-36-18

\twolineshloka
{किं तु शीघ्रमितो वीर निष्क्रम त्वं मया सह}
{सान्त्वयस्व वयस्यं च भार्याहरणदुःखितम्} %4-36-19

\twolineshloka
{यच्च शोकाभिभूतस्य श्रुत्वा रामस्य भाषितम्}
{मया त्वं परुषाण्युक्तस्तत् क्षमस्व सखे मम} %4-36-20


॥इत्यार्षे श्रीमद्रामायणे वाल्मीकीये आदिकाव्ये किष्किन्धाकाण्डे सुग्रीवलक्ष्मणानुरोधः नाम षड्त्रिंशः सर्गः ॥४-३६॥
