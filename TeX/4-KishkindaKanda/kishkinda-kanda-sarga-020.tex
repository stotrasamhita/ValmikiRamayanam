\sect{विंशः सर्गः — ताराविलापः}

\twolineshloka
{रामचापविसृष्टेन शरेणान्तकरेण तम्}
{दृष्ट्वा विनिहतं भूमौ तारा ताराधिपानना} %4-20-1

\twolineshloka
{सा समासाद्य भर्तारं पर्यष्वजत भामिनी}
{इषुणाभिहतं दृष्ट्वा वालिनं कुञ्जरोपमम्} %4-20-2

\twolineshloka
{वानरं पर्वतेन्द्राभं शोकसंतप्तमानसा}
{तारा तरुमिवोन्मूलं पर्यदेवयतातुरा} %4-20-3

\twolineshloka
{रणे दारुणविक्रान्त प्रवीर प्लवतां वर}
{किमिदानीं पुरोभागामद्य त्वं नाभिभाषसे} %4-20-4

\twolineshloka
{उत्तिष्ठ हरिशार्दूल भजस्व शयनोत्तमम्}
{नैवंविधाः शेरते हि भूमौ नृपतिसत्तमाः} %4-20-5

\twolineshloka
{अतीव खलु ते कान्ता वसुधा वसुधाधिप}
{गतासुरपि तां गात्रैर्मां विहाय निषेवसे} %4-20-6

\twolineshloka
{व्यक्तमद्य त्वया वीर धर्मतः सम्प्रवर्तता}
{किष्किन्धेव पुरी रम्या स्वर्गमार्गे विनिर्मिता} %4-20-7

\twolineshloka
{यान्यस्माभिस्त्वया सार्धं वनेषु मधुगन्धिषु}
{विहृतानि त्वया काले तेषामुपरमः कृतः} %4-20-8

\twolineshloka
{निरानन्दा निराशाहं निमग्ना शोकसागरे}
{त्वयि पञ्चत्वमापन्ने महायूथपयूथपे} %4-20-9

\twolineshloka
{हृदयं सुस्थितं मह्यं दृष्ट्वा निपतितं भुवि}
{यन्न शोकाभिसंतप्तं स्फुटतेऽद्य सहस्रधा} %4-20-10

\twolineshloka
{सुग्रीवस्य त्वया भार्या हृता स च विवासितः}
{यत् तत् तस्य त्वया व्युष्टिः प्राप्तेयं प्लवगाधिप} %4-20-11

\twolineshloka
{निःश्रेयसपरा मोहात् त्वया चाहं विगर्हिता}
{यैषाब्रुवं हितं वाक्यं वानरेन्द्र हितैषिणी} %4-20-12

\twolineshloka
{रूपयौवनदृप्तानां दक्षिणानां च मानद}
{नूनमप्सरसामार्य चित्तानि प्रमथिष्यसि} %4-20-13

\twolineshloka
{कालो निःसंशयो नूनं जीवितान्तकरस्तव}
{बलाद् येनावपन्नोऽसि सुग्रीवस्यावशो वशम्} %4-20-14

\twolineshloka
{अस्थाने वालिनं हत्वा युध्यमानं परेण च}
{न संतप्यति काकुत्स्थः कृत्वा कर्मसुगर्हितम्} %4-20-15

\twolineshloka
{वैधव्यं शोकसंतापं कृपणाकृपणा सती}
{अदुःखोपचिता पूर्वं वर्तयिष्याम्यनाथवत्} %4-20-16

\twolineshloka
{लालितश्चाङ्गदो वीरः सुकुमारः सुखोचितः}
{वत्स्यते कामवस्थां मे पितृव्ये क्रोधमूर्च्छिते} %4-20-17

\twolineshloka
{कुरुष्व पितरं पुत्र सुदृष्टं धर्मवत्सलम्}
{दुर्लभं दर्शनं तस्य तव वत्स भविष्यति} %4-20-18

\twolineshloka
{समाश्वासय पुत्रं त्वं संदेशं संदिशस्व मे}
{मूर्ध्न्नि चैनं समाघ्राय प्रवासं प्रस्थितो ह्यसि} %4-20-19

\twolineshloka
{रामेण हि महत् कर्म कृतं त्वामभिनिघ्नता}
{आनृण्यं तु गतं तस्य सुग्रीवस्य प्रतिश्रवे} %4-20-20

\twolineshloka
{सकामो भव सुग्रीव रुमां त्वं प्रतिपत्स्यसे}
{भुङ्क्ष्व राज्यमनुद्विग्नः शस्तो भ्राता रिपुस्तव} %4-20-21

\twolineshloka
{किं मामेवं प्रलपतीं प्रियां त्वं नाभिभाषसे}
{इमाः पश्य वरा बाह्व्यो भार्यास्ते वानरेश्वर} %4-20-22

\twolineshloka
{तस्या विलपितं श्रुत्वा वानर्यः सर्वतश्च ताः}
{परिगृह्याङ्गदं दीना दुःखार्ताः प्रतिचुक्रुशुः} %4-20-23

\twolineshloka
{किमङ्गदं साङ्गदवीरबाहो विहाय यातोऽसि चिरं प्रवासम्}
{न युक्तमेवं गुणसंनिकृष्टं विहाय पुत्रं प्रियचारुवेषम्} %4-20-24

\twolineshloka
{यद्यप्रियं किंचिदसम्प्रधार्य कृतं मया स्यात् तव दीर्घबाहो}
{क्षमस्व मे तद्धरिवंशनाथ व्रजामि मूर्ध्ना तव वीर पादौ} %4-20-25

\twolineshloka
{तथा तु तारा करुणं रुदन्ती भर्तुः समीपे सह वानरीभिः}
{व्यवस्यत प्रायमनिन्द्यवर्णा उपोपवेष्टुं भुवि यत्र वाली} %4-20-26


॥इत्यार्षे श्रीमद्रामायणे वाल्मीकीये आदिकाव्ये किष्किन्धाकाण्डे ताराविलापः नाम विंशः सर्गः ॥४-२०॥
