\sect{नवमः सर्गः — वैरवृत्तान्तानुक्रमः}

\twolineshloka
{वाली नाम मम भ्राता ज्येष्ठः शत्रुनिषूदनः}
{पितुर्बहुमतो नित्यं मम चापि तथा पुरा} %4-9-1

\twolineshloka
{पितर्युपरते तस्मिन् ज्येष्ठोऽयमिति मन्त्रिभिः}
{कपीनामीश्वरो राज्ये कृतः परमसम्मतः} %4-9-2

\twolineshloka
{राज्यं प्रशासतस्तस्य पितृपैतामहं महत्}
{अहं सर्वेषु कालेषु प्रणतः प्रेष्यवत् स्थितः} %4-9-3

\twolineshloka
{मायावी नाम तेजस्वी पूर्वजो दुन्दुभेः सुतः}
{तेन तस्य महद्वैरं वालिनः स्त्रीकृतं पुरा} %4-9-4

\twolineshloka
{स तु सुप्ते जने रात्रौ किष्किन्धाद्वारमागतः}
{नर्दति स्म सुसंरब्धो वालिनं चाह्वयद् रणे} %4-9-5

\twolineshloka
{प्रसुप्तस्तु मम भ्राता नर्दतो भैरवस्वनम्}
{श्रुत्वा न ममृषे वाली निष्पपात जवात् तदा} %4-9-6

\twolineshloka
{स तु वै निःसृतः क्रोधात् तं हन्तुमसुरोत्तमम्}
{वार्यमाणस्ततः स्त्रीभिर्मया च प्रणतात्मना} %4-9-7

\twolineshloka
{स तु निर्धूय सर्वान् नो निर्जगाम महाबलः}
{ततोऽहमपि सौहार्दान्निःसृतो वालिना सह} %4-9-8

\twolineshloka
{स तु मे भ्रातरं दृष्ट्वा मां च दूरादवस्थितम्}
{असुरो जातसन्त्रासः प्रदुद्राव तदा भृशम्} %4-9-9

\twolineshloka
{तस्मिन् द्रवति सन्त्रस्ते ह्यावां द्रुततरं गतौ}
{प्रकाशोऽपि कृतो मार्गश्चन्द्रेणोद्गच्छता तदा} %4-9-10

\twolineshloka
{स तृणैरावृतं दुर्गं धरण्या विवरं महत्}
{प्रविवेशासुरो वेगादावामासाद्य विष्ठितौ} %4-9-11

\twolineshloka
{तं प्रविष्टं रिपुं दृष्ट्वा बिलं रोषवशं गतः}
{मामुवाच ततो वाली वचनं क्षुभितेन्द्रियः} %4-9-12

\twolineshloka
{इह तिष्ठाद्य सुग्रीव बिलद्वारि समाहितः}
{यावदत्र प्रविश्याहं निहन्मि समरे रिपुम्} %4-9-13

\twolineshloka
{मया त्वेतद् वचः श्रुत्वा याचितः स परन्तपः}
{शापयित्वा च मां पद्भ्यां प्रविवेश बिलं ततः} %4-9-14

\twolineshloka
{तस्य प्रविष्टस्य बिलं साग्रः संवत्सरो गतः}
{स्थितस्य च बिलद्वारि स कालो व्यत्यवर्तत} %4-9-15

\twolineshloka
{अहं तु नष्टं तं ज्ञात्वा स्नेहादागतसम्भ्रमः}
{भ्रातरं न प्रपश्यामि पापशङ्कि च मे मनः} %4-9-16

\twolineshloka
{अथ दीर्घस्य कालस्य बिलात् तस्माद् विनिःसृतम्}
{सफेनं रुधिरं दृष्ट्वा ततोऽहं भृशदुःखितः} %4-9-17

\twolineshloka
{नर्दतामसुराणां च ध्वनिर्मे श्रोत्रमागतः}
{न रतस्य च सङ्ग्रामे क्रोशतोऽपि स्वनो गुरोः} %4-9-18

\twolineshloka
{अहं त्ववगतो बुद्ध्या चिह्नैस्तैर्भ्रातरं हतम्}
{पिधाय च बिलद्वारं शिलया गिरिमात्रया} %4-9-19

\twolineshloka
{शोकार्तश्चोदकं कृत्वा किष्किन्धामागतः सखे}
{गूहमानस्य मे तत् त्वं यत्नतो मन्त्रिभिः श्रुतम्} %4-9-20

\twolineshloka
{ततोऽहं तैः समागम्य समेतैरभिषेचितः}
{राज्यं प्रशासतस्तस्य न्यायतो मम राघव} %4-9-21

\twolineshloka
{आजगाम रिपुं हत्वा दानवं स तु वानरः}
{अभिषिक्तं तु मां दृष्ट्वा क्रोधात् संरक्तलोचनः} %4-9-22

\twolineshloka
{मदीयान् मन्त्रिणो बद्ध्वा परुषं वाक्यमब्रवीत्}
{निग्रहे च समर्थस्य तं पापं प्रति राघव} %4-9-23

\twolineshloka
{न प्रावर्तत मे बुद्धिर्भ्रातृगौरवयन्त्रिता}
{हत्वा शत्रुं स मे भ्राता प्रविवेश पुरं तदा} %4-9-24

\twolineshloka
{मानयंस्तं महात्मानं यथावच्चाभिवादयम्}
{उक्ताश्च नाशिषस्तेन प्रहृष्टेनान्तरात्मना} %4-9-25

\twolineshloka
{नत्वा पादावहं तस्य मुकुटेनास्पृशं प्रभो}
{अपि वाली मम क्रोधान्न प्रसादं चकार सः} %4-9-26


॥इत्यार्षे श्रीमद्रामायणे वाल्मीकीये आदिकाव्ये किष्किन्धाकाण्डे वैरवृत्तान्तानुक्रमः नाम नवमः सर्गः ॥४-९॥
