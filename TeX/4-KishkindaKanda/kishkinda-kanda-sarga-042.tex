\sect{द्विचत्वारिंशः सर्गः — प्रतीचीप्रेषणम्}

\twolineshloka
{अथ प्रस्थाप्य स हरीन् सुग्रीवो दक्षिणां दिशम्}
{अब्रवीन्मेघसंकाशं सुषेणं नाम वानरम्} %4-42-1

\twolineshloka
{तारायाः पितरं राजा श्वशुरं भीमविक्रमम्}
{अब्रवीत् प्राञ्जलिर्वाक्यमभिगम्य प्रणम्य च} %4-42-2

\twolineshloka
{महर्षिपुत्रं मारीचमर्चिष्मन्तं महाकपिम्}
{वृतं कपिवरैः शूरैर्महेन्द्रसदृशद्युतिम्} %4-42-3

\twolineshloka
{बुद्धिविक्रमसम्पन्नं वैनतेयसमद्युतिम्}
{मरीचिपुत्रान् मारीचानर्चिर्माल्यान् महाबलान्} %4-42-4

\twolineshloka
{ऋषिपुत्रांश्च तान् सर्वान् प्रतीचीमादिशद् दिशम्}
{द्वाभ्यां शतसहस्राभ्यां कपीनां कपिसत्तमाः} %4-42-5

\twolineshloka
{सुषेणप्रमुखा यूयं वैदेहीं परिमार्गथ}
{सौराष्ट्रान् सहबाह्लीकांश्चन्द्रचित्रांस्तथैव च} %4-42-6

\twolineshloka
{स्फीताञ्जनपदान् रम्यान् विपुलानि पुराणि च}
{पुंनागगहनं कुक्षिं बकुलोद्दालकाकुलम्} %4-42-7

\twolineshloka
{तथा केतकषण्डांश्च मार्गध्वं हरिपुङ्गवाः}
{प्रत्यक्स्रोतोवहाश्चैव नद्यः शीतजलाः शिवाः} %4-42-8

\twolineshloka
{तापसानामरण्यानि कान्तारगिरयश्च ये}
{तत्र स्थलीर्मरुप्राया अत्युच्चशिशिराः शिलाः} %4-42-9

\twolineshloka
{गिरिजालावृतां दुर्गां मार्गित्वा पश्चिमां दिशम्}
{ततः पश्चिममागम्य समुद्रं द्रष्टुमर्हथ} %4-42-10

\twolineshloka
{तिमिनक्राकुलजलं गत्वा द्रक्ष्यथ वानराः}
{ततः केतकषण्डेषु तमालगहनेषु च} %4-42-11

\twolineshloka
{कपयो विहरिष्यन्ति नारिकेलवनेषु च}
{तत्र सीतां च मार्गध्वं निलयं रावणस्य च} %4-42-12

\twolineshloka
{वेलातलनिविष्टेषु पर्वतेषु वनेषु च}
{मुरवीपत्तनं चैव रम्यं चैव जटापुरम्} %4-42-13

\twolineshloka
{अवन्तीमङ्गलेपां च तथा चालक्षितं वनम्}
{राष्ट्राणि च विशालानि पत्तनानि ततस्ततः} %4-42-14

\twolineshloka
{सिन्धुसागरयोश्चैव संगमे तत्र पर्वतः}
{महान् सोमगिरिर्नाम शतशृङ्गो महाद्रुमः} %4-42-15

\twolineshloka
{तत्र प्रस्थेषु रम्येषु सिंहाः पक्षगमाः स्थिताः}
{तिमिमत्स्यगजांश्चैव नीडान्यारोपयन्ति ते} %4-42-16

\twolineshloka
{तानि नीडानि सिंहानां गिरिशृङ्गगताश्च ये}
{दृप्तास्तृप्ताश्च मातङ्गास्तोयदस्वननिःस्वनाः} %4-42-17

\twolineshloka
{विचरन्ति विशालेऽस्मिंस्तोयपूर्णे समन्ततः}
{तस्य शृङ्गं दिवस्पर्शं काञ्चनं चित्रपादपम्} %4-42-18

\twolineshloka
{सर्वमाशु विचेतव्यं कपिभिः कामरूपिभिः}
{कोटिं तत्र समुद्रस्य काञ्चनीं शतयोजनाम्} %4-42-19

\twolineshloka
{दुर्दर्शां पारियात्रस्य गत्वा द्रक्ष्यथ वानराः}
{कोट्यस्तत्र चतुर्विंशद् गन्धर्वाणां तरस्विनाम्} %4-42-20

\twolineshloka
{वसन्त्यग्निनिकाशानां घोराणां कामरूपिणाम्}
{पावकार्चिःप्रतीकाशाः समवेताः समन्ततः} %4-42-21

\twolineshloka
{नात्यासादयितव्यास्ते वानरैर्भीमविक्रमैः}
{नादेयं च फलं तस्माद् देशात् किंचित् प्लवङ्गमैः} %4-42-22

\twolineshloka
{दुरासदा हि ते वीराः सत्त्ववन्तो महाबलाः}
{फलमूलानि ते तत्र रक्षन्ते भीमविक्रमाः} %4-42-23

\twolineshloka
{तत्र यत्नश्च कर्तव्यो मार्गितव्या च जानकी}
{नहि तेभ्यो भयं किंचित् कपित्वमनुवर्तताम्} %4-42-24

\twolineshloka
{तत्र वैदूर्यवर्णाभो वज्रसंस्थानसंस्थितः}
{नानाद्रुमलताकीर्णो वज्रो नाम महागिरिः} %4-42-25

\twolineshloka
{श्रीमान् समुदितस्तत्र योजनानां शतं समम्}
{गुहास्तत्र विचेतव्याः प्रयत्नेन प्लवङ्गमाः} %4-42-26

\twolineshloka
{चतुर्भागे समुद्रस्य चक्रवान् नाम पर्वतः}
{तत्र चक्रं सहस्रारं निर्मितं विश्वकर्मणा} %4-42-27

\twolineshloka
{तत्र पञ्चजनं हत्वा हयग्रीवं च दानवम्}
{आजहार ततश्चक्रं शङ्खं च पुरुषोत्तमः} %4-42-28

\twolineshloka
{तस्य सानुषु रम्येषु विशालासु गुहासु च}
{रावणः सह वैदेह्या मार्गितव्यस्ततस्ततः} %4-42-29

\twolineshloka
{योजनानि चतुःषष्टिर्वराहो नाम पर्वतः}
{सुवर्णशृङ्गः सुमहानगाधे वरुणालये} %4-42-30

\twolineshloka
{तत्र प्राग्ज्योतिषं नाम जातरूपमयं पुरम्}
{यस्मिन् वसति दुष्टात्मा नरको नाम दानवः} %4-42-31

\twolineshloka
{तत्र सानुषु रम्येषु विशालासु गुहासु च}
{रावणः सह वैदेह्या मार्गितव्यस्ततस्ततः} %4-42-32

\twolineshloka
{तमतिक्रम्य शैलेन्द्रं काञ्चनान्तरदर्शनम्}
{पर्वतः सर्वसौवर्णो धाराप्रस्रवणायुतः} %4-42-33

\twolineshloka
{तं गजाश्च वराहाश्च सिंहा व्याघ्राश्च सर्वतः}
{अभिगर्जन्ति सततं तेन शब्देन दर्पिताः} %4-42-34

\twolineshloka
{यस्मिन् हरिहयः श्रीमान् महेन्द्रः पाकशासनः}
{अभिषिक्तः सुरै राजा मेघो नाम स पर्वतः} %4-42-35

\twolineshloka
{तमतिक्रम्य शैलेन्द्रं महेन्द्रपरिपालितम्}
{षष्टिं गिरिसहस्राणि काञ्चनानि गमिष्यथ} %4-42-36

\twolineshloka
{तरुणादित्यवर्णानि भ्राजमानानि सर्वतः}
{जातरूपमयैर्वृक्षैः शोभितानि सुपुष्पितैः} %4-42-37

\twolineshloka
{तेषां मध्ये स्थितो राजा मेरुरुत्तमपर्वतः}
{आदित्येन प्रसन्नेन शैलो दत्तवरः पुरा} %4-42-38

\twolineshloka
{तेनैवमुक्तः शैलेन्द्रः सर्व एव त्वदाश्रयाः}
{मत्प्रसादाद् भविष्यन्ति दिवा रात्रौ च काञ्चनाः} %4-42-39

\twolineshloka
{त्वयि ये चापि वत्स्यन्ति देवगन्धर्वदानवाः}
{ते भविष्यन्ति भक्ताश्च प्रभया काञ्चनप्रभाः} %4-42-40

\twolineshloka
{विश्वेदेवाश्च वसवो मरुतश्च दिवौकसः}
{आगत्य पश्चिमां संध्यां मेरुमुत्तमपर्वतम्} %4-42-41

\twolineshloka
{आदित्यमुपतिष्ठन्ति तैश्च सूर्योऽभिपूजितः}
{अदृश्यः सर्वभूतानामस्तं गच्छति पर्वतम्} %4-42-42

\twolineshloka
{योजनानां सहस्राणि दश तानि दिवाकरः}
{मुहूर्तार्धेन तं शीघ्रमभियाति शिलोच्चयम्} %4-42-43

\twolineshloka
{शृङ्गे तस्य महद्दिव्यं भवनं सूर्यसंनिभम्}
{प्रासादगणसम्बाधं विहितं विश्वकर्मणा} %4-42-44

\twolineshloka
{शोभितं तरुभिश्चित्रैर्नानापक्षिसमाकुलैः}
{निकेतं पाशहस्तस्य वरुणस्य महात्मनः} %4-42-45

\twolineshloka
{अन्तरा मेरुमस्तं च तालो दशशिरा महान्}
{जातरूपमयः श्रीमान् भ्राजते चित्रवेदिकः} %4-42-46

\twolineshloka
{तेषु सर्वेषु दुर्गेषु सरस्सु च सरित्सु च}
{रावणः सह वैदेह्या मार्गितव्यस्ततस्ततः} %4-42-47

\twolineshloka
{यत्र तिष्ठति धर्मज्ञस्तपसा स्वेन भावितः}
{मेरुसावर्णिरित्येष ख्यातो वै ब्रह्मणा समः} %4-42-48

\twolineshloka
{प्रष्टव्यो मेरुसावर्णिर्महर्षिः सूर्यसंनिभः}
{प्रणम्य शिरसा भूमौ प्रवृत्तिं मैथिलीं प्रति} %4-42-49

\twolineshloka
{एतावज्जीवलोकस्य भास्करो रजनीक्षये}
{कृत्वा वितिमिरं सर्वमस्तं गच्छति पर्वतम्} %4-42-50

\twolineshloka
{एतावद् वानरैः शक्यं गन्तुं वानरपुङ्गवाः}
{अभास्करममर्यादं न जानीमस्ततः परम्} %4-42-51

\twolineshloka
{अवगम्य तु वैदेहीं निलयं रावणस्य च}
{अस्तं पर्वतमासाद्य पूर्णे मासे निवर्तत} %4-42-52

\twolineshloka
{ऊर्ध्वं मासान्न वस्तव्यं वसन् वध्यो भवेन्मम}
{सहैव शूरो युष्माभिः श्वशुरो मे गमिष्यति} %4-42-53

\twolineshloka
{श्रोतव्यं सर्वमेतस्य भवद्भिर्दिष्टकारिभिः}
{गुरुरेष महाबाहुः श्वशुरो मे महाबलः} %4-42-54

\twolineshloka
{भवन्तश्चापि विक्रान्ताः प्रमाणं सर्व एव हि}
{प्रमाणमेनं संस्थाप्य पश्यध्वं पश्चिमां दिशम्} %4-42-55

\twolineshloka
{दृष्टायां तु नरेन्द्रस्य पत्न्याममिततेजसः}
{कृतकृत्या भविष्यामः कृतस्य प्रतिकर्मणा} %4-42-56

\twolineshloka
{अतोऽन्यदपि यत्कार्यं कार्यस्यास्य प्रियं भवेत्}
{सम्प्रधार्य भवद्भिश्च देशकालार्थसंहितम्} %4-42-57

\twolineshloka
{ततः सुषेणप्रमुखाः प्लवङ्गाः सुग्रीववाक्यं निपुणं निशम्य}
{आमन्त्र्य सर्वे प्लवगाधिपं ते जग्मुर्दिशं तां वरुणाभिगुप्ताम्} %4-42-58


॥इत्यार्षे श्रीमद्रामायणे वाल्मीकीये आदिकाव्ये किष्किन्धाकाण्डे प्रतीचीप्रेषणम् नाम द्विचत्वारिंशः सर्गः ॥४-४२॥
