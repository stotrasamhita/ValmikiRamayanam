\sect{त्रिचत्वारिशः सर्गः — उदीचीप्रेषणम्}

\twolineshloka
{ततः सन्दिश्य सुग्रीवः श्वशुरं पश्चिमां दिशम्}
{वीरं शतबलिं नाम वानरं वानरेश्वरः} %4-43-1

\twolineshloka
{उवाच राजा सर्वज्ञः सर्ववानरसत्तमः}
{वाक्यमात्महितं चैव रामस्य च हितं तदा} %4-43-2

\twolineshloka
{वृतः शतसहस्रेण त्वद्विधानां वनौकसाम्}
{वैवस्वतसुतैः सार्धं प्रविष्टः सर्वमन्त्रिभिः} %4-43-3

\twolineshloka
{दिशं ह्युदीचीं विक्रान्तां हिमशैलावतंसिकाम्}
{सर्वतः परिमार्गध्वं रामपत्नीं यशस्विनीम्} %4-43-4

\twolineshloka
{अस्मिन् कार्ये विनिर्वृत्ते कृते दाशरथेः प्रिये}
{ऋणान्मुक्ता भविष्यामः कृतार्थार्थविदां वराः} %4-43-5

\twolineshloka
{कृतं हि प्रियमस्माकं राघवेण महात्मना}
{तस्य चेत्प्रतिकारोऽस्ति सफलं जीवितं भवेत्} %4-43-6

\twolineshloka
{अर्थिनः कार्यनिर्वृत्तिमकर्तुरपि यश्चरेत्}
{तस्य स्यात् सफलं जन्म किं पुनः पूर्वकारिणः} %4-43-7

\twolineshloka
{एतां बुद्धिं समास्थाय दृश्यते जानकी यथा}
{तथा भवद्भिः कर्तव्यमस्मत्प्रियहितैषिभिः} %4-43-8

\twolineshloka
{अयं हि सर्वभूतानां मान्यस्तु नरसत्तमः}
{अस्मासु च गतः प्रीतिं रामः परपुरञ्जयः} %4-43-9

\twolineshloka
{इमानि बहुदुर्गाणि नद्यः शैलान्तराणि च}
{भवन्तः परिमार्गन्तु बुद्धिविक्रमसम्पदा} %4-43-10

\twolineshloka
{तत्र म्लेच्छान् पुलिन्दांश्च शूरसेनांस्तथैव च}
{प्रस्थलान् भरतांश्चैव कुरूंश्च सह मद्रकैः} %4-43-11

\twolineshloka
{काम्बोजयवनांश्चैव शकानां पत्तनानि च}
{अन्वीक्ष्य दरदांश्चैव हिमवन्तं विचिन्वथ} %4-43-12

\twolineshloka
{लोध्रपद्मकषण्डेषु देवदारुवनेषु च}
{रावणः सह वैदेह्या मार्गितव्यस्ततस्ततः} %4-43-13

\twolineshloka
{ततः सोमाश्रमं गत्वा देवगन्धर्वसेवितम्}
{कालं नाम महासानुं पर्वतं तं गमिष्यथ} %4-43-14

\twolineshloka
{महत्सु तस्य शैलेषु पर्वतेषु गुहासु च}
{विचिन्वत महाभागां रामपत्नीमनिन्दिताम्} %4-43-15

\twolineshloka
{तमतिक्रम्य शैलेन्द्रं हेमगर्भं महागिरिम्}
{ततः सुदर्शनं नाम पर्वतं गन्तुमर्हथ} %4-43-16

\twolineshloka
{ततो देवसखो नाम पर्वतः पतगालयः}
{नानापक्षिसमाकीर्णो विविधद्रुमभूषितः} %4-43-17

\twolineshloka
{तस्य काननषण्डेषु निर्झरेषु गुहासु च}
{रावणः सह वैदेह्या मार्गितव्यस्ततस्ततः} %4-43-18

\twolineshloka
{तमतिक्रम्य चाकाशं सर्वतः शतयोजनम्}
{अपर्वतनदीवृक्षं सर्वसत्त्वविवर्जितम्} %4-43-19

\twolineshloka
{तत्तु शीघ्रमतिक्रम्य कान्तारं रोमहर्षणम्}
{कैलासं पाण्डुरं प्राप्य हृष्टा यूयं भविष्यथ} %4-43-20

\twolineshloka
{तत्र पाण्डुरमेघाभं जाम्बूनदपरिष्कृतम्}
{कुबेरभवनं रम्यं निर्मितं विश्वकर्मणा} %4-43-21

\twolineshloka
{विशाला नलिनी यत्र प्रभूतकमलोत्पला}
{हंसकारण्डवाकीर्णा अप्सरोगणसेविता} %4-43-22

\twolineshloka
{तत्र वैश्रवणो राजा सर्वलोकनमस्कृतः}
{धनदो रमते श्रीमान् गुह्यकैः सह यक्षराट्} %4-43-23

\twolineshloka
{तस्य चन्द्रनिकाशेषु पर्वतेषु गुहासु च}
{रावणः सह वैदेह्या मार्गितव्यस्ततस्ततः} %4-43-24

\twolineshloka
{क्रौञ्चं तु गिरिमासाद्य बिलं तस्य सुदुर्गमम्}
{अप्रमत्तैः प्रवेष्टव्यं दुष्प्रवेशं हि तत् स्मृतम्} %4-43-25

\twolineshloka
{वसन्ति हि महात्मानस्तत्र सूर्यसमप्रभाः}
{देवैरभ्यर्थिताः सम्यग् देवरूपा महर्षयः} %4-43-26

\twolineshloka
{क्रौञ्चस्य तु गुहाश्चान्याः सानूनि शिखराणि च}
{निर्दराश्च नितम्बाश्च विचेतव्यास्ततस्ततः} %4-43-27

\twolineshloka
{अवृक्षं कामशैलं च मानसं विहगालयम्}
{न गतिस्तत्र भूतानां देवानां न च रक्षसाम्} %4-43-28

\twolineshloka
{स च सर्वैर्विचेतव्यः ससानुप्रस्थभूधरः}
{क्रौञ्चं गिरिमतिक्रम्य मैनाको नाम पर्वतः} %4-43-29

\twolineshloka
{मयस्य भवनं तत्र दानवस्य स्वयङ्कृतम्}
{मैनाकस्तु विचेतव्यः ससानुप्रस्थकन्दरः} %4-43-30

\twolineshloka
{स्त्रीणामश्वमुखीनां तु निकेतस्तत्र तत्र तु}
{तं देशं समतिक्रम्य आश्रमं सिद्धसेवितम्} %4-43-31

\twolineshloka
{सिद्धा वैखानसा यत्र वालखिल्याश्च तापसाः}
{वन्दितव्यास्ततः सिद्धास्तपसा वीतकल्मषाः} %4-43-32

\twolineshloka
{प्रष्टव्या चापि सीतायाः प्रवृत्तिर्विनयान्वितैः}
{हेमपुष्करसञ्छन्नं तत्र वैखानसं सरः} %4-43-33

\twolineshloka
{तरुणादित्यसङ्काशैर्हंसैर्विचरितं शुभैः}
{औपवाह्यः कुबेरस्य सार्वभौम इति स्मृतः} %4-43-34

\threelineshloka
{गजः पर्येति तं देशं सदा सह करेणुभिः}
{तत् सरः समतिक्रम्य नष्टचन्द्रदिवाकरम्}
{अनक्षत्रगणं व्योम निष्पयोदमनादितम्} %4-43-35

\twolineshloka
{गभस्तिभिरिवार्कस्य स तु देशः प्रकाश्यते}
{विश्राम्यद्भिस्तपःसिद्धैर्देवकल्पैः स्वयम्प्रभैः} %4-43-36

\twolineshloka
{तं तु देशमतिक्रम्य शैलोदा नाम निम्नगा}
{उभयोस्तीरयोस्तस्याः कीचका नाम वेणवः} %4-43-37

\twolineshloka
{ते नयन्ति परं तीरं सिद्धान् प्रत्यानयन्ति च}
{उत्तराः कुरवस्तत्र कृतपुण्यप्रतिश्रयाः} %4-43-38

\twolineshloka
{ततः काञ्चनपद्माभिः पद्मिनीभिः कृतोदकाः}
{नीलवैदूर्यपत्राढ्या नद्यस्तत्र सहस्रशः} %4-43-39

\twolineshloka
{रक्तोत्पलवनैश्चात्र मण्डिताश्च हिरण्मयैः}
{तरुणादित्यसङ्काशा भान्ति तत्र जलाशयाः} %4-43-40

\twolineshloka
{महार्हमणिपत्रैश्च काञ्चनप्रभकेसरैः}
{नीलोत्पलवनैश्चित्रैः स देशः सर्वतो वृतः} %4-43-41

\twolineshloka
{निस्तुलाभिश्च मुक्ताभिर्मणिभिश्च महाधनैः}
{उद्धूतपुलिनास्तत्र जातरूपैश्च निम्नगाः} %4-43-42

\twolineshloka
{सर्वरत्नमयैश्चित्रैरवगाढा नगोत्तमैः}
{जातरूपमयैश्चापि हुताशनसमप्रभैः} %4-43-43

\twolineshloka
{नित्यपुष्पफलास्तत्र नगाः पत्ररथाकुलाः}
{दिव्यगन्धरसस्पर्शाः सर्वकामान् स्रवन्ति च} %4-43-44

\threelineshloka
{नानाकाराणि वासांसि फलन्त्यन्ये नगोत्तमाः}
{मुक्तावैदूर्यचित्राणि भूषणानि तथैव च}
{स्त्रीणां यान्यनुरूपाणि पुरुषाणां तथैव च} %4-43-45

\twolineshloka
{सर्वर्तुसुखसेव्यानि फलन्त्यन्ये नगोत्तमाः}
{महार्हमणिचित्राणि फलन्त्यन्ये नगोत्तमाः} %4-43-46

\twolineshloka
{शयनानि प्रसूयन्ते चित्रास्तरणवन्ति च}
{मनःकान्तानि माल्यानि फलन्त्यत्रापरे द्रुमाः} %4-43-47

\twolineshloka
{पानानि च महार्हाणि भक्ष्याणि विविधानि च}
{स्त्रियश्च गुणसम्पन्ना रूपयौवनलक्षिताः} %4-43-48

\twolineshloka
{गन्धर्वाः किन्नराः सिद्धा नागा विद्याधरास्तथा}
{रमन्ते सततं तत्र नारीभिर्भास्वरप्रभाः} %4-43-49

\twolineshloka
{सर्वे सुकृतकर्माणः सर्वे रतिपरायणाः}
{सर्वे कामार्थसहिता वसन्ति सह योषितः} %4-43-50

\twolineshloka
{गीतवादित्रनिर्घोषः सोत्कृष्टहसितस्वनः}
{श्रूयते सततं तत्र सर्वभूतमनोरमः} %4-43-51

\twolineshloka
{तत्र नामुदितः कश्चिन्नात्र कश्चिदसत्प्रियः}
{अहन्यहनि वर्धन्ते गुणास्तत्र मनोरमाः} %4-43-52

\twolineshloka
{समतिक्रम्य तं देशमुत्तरः पयसां निधिः}
{तत्र सोमगिरिर्नाम मध्ये हेममयो महान्} %4-43-53

\twolineshloka
{इन्द्रलोकगता ये च ब्रह्मलोकगताश्च ये}
{देवास्तं समवेक्षन्ते गिरिराजं दिवं गताः} %4-43-54

\twolineshloka
{स तु देशो विसूर्योऽपि तस्य भासा प्रकाशते}
{सूर्यलक्ष्म्याभिविज्ञेयस्तपतेव विवस्वता} %4-43-55

\twolineshloka
{भगवांस्तत्र विश्वात्मा शम्भुरेकादशात्मकः}
{ब्रह्मा वसति देवेशो ब्रह्मर्षिपरिवारितः} %4-43-56

\twolineshloka
{न कथञ्चन गन्तव्यं कुरूणामुत्तरेण वः}
{अन्येषामपि भूतानां नानुक्रामति वै गतिः} %4-43-57

\twolineshloka
{स हि सोमगिरिर्नाम देवानामपि दुर्गमः}
{तमालोक्य ततः क्षिप्रमुपावर्तितुमर्हथ} %4-43-58

\twolineshloka
{एतावद् वानरैः शक्यं गन्तुं वानरपुङ्गवाः}
{अभास्करममर्यादं न जानीमस्ततः परम्} %4-43-59

\twolineshloka
{सर्वमेतद् विचेतव्यं यन्मया परिकीर्तितम्}
{यदन्यदपि नोक्तं च तत्रापि क्रियतां मतिः} %4-43-60

\twolineshloka
{ततः कृतं दाशरथेर्महत्प्रियं महत्प्रियं चापि ततो मम प्रियम्}
{कृतं भविष्यत्यनिलानलोपमा विदेहजादर्शनजेन कर्मणा} %4-43-61

\twolineshloka
{ततः कृतार्थाः सहिताः सबान्धवा मयार्चिताः सर्वगुणैर्मनोरमैः}
{चरिष्यथोर्वीं प्रति शान्तशत्रवः सहप्रिया भूतधराः प्लवङ्गमाः} %4-43-62


॥इत्यार्षे श्रीमद्रामायणे वाल्मीकीये आदिकाव्ये किष्किन्धाकाण्डे उदीचीप्रेषणम् नाम त्रिचत्वारिशः सर्गः ॥४-४३॥
