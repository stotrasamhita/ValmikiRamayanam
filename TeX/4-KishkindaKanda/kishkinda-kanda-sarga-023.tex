\sect{त्रयोविंशः सर्गः — अङ्गदाभिवादनम्}

\twolineshloka
{ततः समुपजिघ्रन्ती कपिराजस्य तन्मुखम्}
{पतिं लोकश्रुता तारा मृतं वचनमब्रवीत्} %4-23-1

\twolineshloka
{शेषे त्वं विषमे दुःखमकृत्वा वचनं मम}
{उपलोपचिते वीर सुदुःखे वसुधातले} %4-23-2

\twolineshloka
{मत्तः प्रियतरा नूनं वानरेन्द्र मही तव}
{शेषे हि तां परिष्वज्य मां च न प्रतिभाषसे} %4-23-3

\twolineshloka
{सुग्रीवस्य वशं प्राप्तो विधिरेष भवत्यहो}
{सुग्रीव एव विक्रान्तो वीर साहसिकप्रिय} %4-23-4

\twolineshloka
{ऋक्षवानरमुख्यास्त्वां बलिनं पर्युपासते}
{तेषां विलपितं कृच्छ्रमङ्गदस्य च शोचतः} %4-23-5

\twolineshloka
{मम चेमा गिरः श्रुत्वा किं त्वं न प्रतिबुध्यसे}
{इदं तद् वीरशयनं तत्र शेषे हतो युधि} %4-23-6

\twolineshloka
{शायिता निहता यत्र त्वयैव रिपवः पुरा}
{विशुद्धसत्त्वाभिजन प्रिययुद्ध मम प्रिय} %4-23-7

\twolineshloka
{मामनाथां विहायैकां गतस्त्वमसि मानद}
{शूराय न प्रदातव्या कन्या खलु विपश्चिता} %4-23-8

\twolineshloka
{शूरभार्यां हतां पश्य सद्यो मां विधवां कृताम्}
{अवभग्नश्च मे मानो भग्ना मे शाश्वती गतिः} %4-23-9

\twolineshloka
{अगाधे च निमग्नास्मि विपुले शोकसागरे}
{अश्मसारमयं नूनमिदं मे हृदयं दृढम्} %4-23-10

\twolineshloka
{भर्तारं निहतं दृष्ट्वा यन्नाद्य शतधा कृतम्}
{सुहृच्चैव च भर्ता च प्रकृत्या च मम प्रियः} %4-23-11

\twolineshloka
{प्रहारे च पराक्रान्तः शूरः पञ्चत्वमागतः}
{पतिहीना तु या नारी कामं भवतु पुत्रिणी} %4-23-12

\twolineshloka
{धनधान्यसमृद्धापि विधवेत्युच्यते जनैः}
{स्वगात्रप्रभवे वीर शेषे रुधिरमण्डले} %4-23-13

\twolineshloka
{कृमिरागपरिस्तोमे स्वकीये शयने यथा}
{रेणुशोणितसंवीतं गात्रं तव समन्ततः} %4-23-14

\twolineshloka
{परिरब्धुं न शक्नोमि भुजाभ्यां प्लवगर्षभ}
{कृतकृत्योऽद्य सुग्रीवो वैरेऽस्मिन्नतिदारुणे} %4-23-15

\twolineshloka
{यस्य रामविमुक्तेन हृतमेकेषुणा भयम्}
{शरेण हृदि लग्नेन गात्रसंस्पर्शने तव} %4-23-16

\twolineshloka
{वार्यामि त्वां निरीक्षन्ती त्वयि पञ्चत्वमागते}
{उद्बबर्ह शरं नीलस्तस्य गात्रगतं तदा} %4-23-17

\twolineshloka
{गिरिगह्वरसंलीनं दीप्तमाशीविषं यथा}
{तस्य निष्कृष्यमाणस्य बाणस्यापि बभौ द्युतिः} %4-23-18

\twolineshloka
{अस्तमस्तकसंरुद्धरश्मेर्दिनकरादिव}
{पेतुः क्षतजधारास्तु व्रणेभ्यस्तस्य सर्वशः} %4-23-19

\twolineshloka
{ताम्रगैरिकसम्पृक्ता धारा इव धराधरात्}
{अवकीर्णं विमार्जन्ती भर्तारं रणरेणुना} %4-23-20

\twolineshloka
{अस्रैर्नयनजैः शूरं सिषेचास्त्रसमाहतम्}
{रुधिरोक्षितसर्वाङ्गं दृष्ट्वा विनिहतं पतिम्} %4-23-21

\twolineshloka
{उवाच तारा पिङ्गाक्षं पुत्रमङ्गदमङ्गना}
{अवस्थां पश्चिमां पश्य पितुः पुत्र सुदारुणाम्} %4-23-22

\twolineshloka
{सम्प्रसक्तस्य वैरस्य गतोऽन्तः पापकर्मणा}
{बालसूर्योज्ज्वलतनुं प्रयातं यमसादनम्} %4-23-23

\twolineshloka
{अभिवादय राजानं पितरं पुत्र मानदम्}
{एवमुक्तः समुत्थाय जग्राह चरणौ पितुः} %4-23-24

\twolineshloka
{भुजाभ्यां पीनवृत्ताभ्यामङ्गदोऽहमिति ब्रुवन्}
{अभिवादयमानं त्वामङ्गदं त्वं यथा पुरा} %4-23-25

\threelineshloka
{दीर्घायुर्भव पुत्रेति किमर्थं नाभिभाषसे}
{अहं पुत्रसहाया त्वामुपासे गतचेतनम्}
{सिंहेन पातितं सद्यो गौः सवत्सेव गोवृषम्} %4-23-26

\twolineshloka
{इष्ट्वा संग्रामयज्ञेन रामप्रहरणाम्भसा}
{तस्मन्नवभृथे स्नातः कथं पत्न्या मया विना} %4-23-27

\twolineshloka
{या दत्ता देवराजेन तव तुष्टेन संयुगे}
{शातकौम्भीं प्रियां मालां तां ते पश्यामि नेह किम्} %4-23-28

\twolineshloka
{राज्यश्रीर्न जहाति त्वां गतासुमपि मानद}
{सूर्यस्यावर्तमानस्य शैलराजमिव प्रभा} %4-23-29

\twolineshloka
{न मे वचः पथ्यमिदं त्वया कृतं न चास्मि शक्ता हि निवारणे तव}
{हता सपुत्रास्मि हतेन संयुगे सह त्वया श्रीर्विजहाति मामपि} %4-23-30


॥इत्यार्षे श्रीमद्रामायणे वाल्मीकीये आदिकाव्ये किष्किन्धाकाण्डे अङ्गदाभिवादनम् नाम त्रयोविंशः सर्गः ॥४-२३॥
