\sect{त्रयस्त्रिंशः सर्गः — तारासान्त्ववचनम्}

\twolineshloka
{अथ प्रतिसमादिष्टो लक्ष्मणः परवीरहा}
{प्रविवेश गुहां रम्यां किष्किन्धां रामशासनात्} %4-33-1

\twolineshloka
{द्वारस्था हरयस्तत्र महाकाया महाबलाः}
{बभूवुर्लक्ष्मणं दृष्ट्वा सर्वे प्राञ्जलयः स्थिताः} %4-33-2

\twolineshloka
{निःश्वसन्तं तु तं दृष्ट्वा क्रुद्धं दशरथात्मजम्}
{बभूवुर्हरयस्त्रस्ता न चैनं पर्यवारयन्} %4-33-3

\twolineshloka
{स तां रत्नमयीं दिव्यां श्रीमान् पुष्पितकाननाम्}
{रम्यां रत्नसमाकीर्णां ददर्श महतीं गुहाम्} %4-33-4

\twolineshloka
{हर्म्यप्रासादसम्बाधां नानारत्नोपशोभिताम्}
{सर्वकामफलैर्वृक्षैः पुष्पितैरुपशोभिताम्} %4-33-5

\twolineshloka
{देवगन्धर्वपुत्रैश्च वानरैः कामरूपिभिः}
{दिव्यमाल्याम्बरधरैः शोभितां प्रियदर्शनैः} %4-33-6

\twolineshloka
{चन्दनागुरुपद्मानां गन्धैः सुरभिगन्धिताम्}
{मैरेयाणां मधूनां च सम्मोदितमहापथाम्} %4-33-7

\twolineshloka
{विन्ध्यमेरुगिरिप्रख्यैः प्रासादैर्नैकभूमिभिः}
{ददर्श गिरिनद्यश्च विमलास्तत्र राघवः} %4-33-8

\twolineshloka
{अङ्गदस्य गृहं रम्यं मैन्दस्य द्विविदस्य च}
{गवयस्य गवाक्षस्य गजस्य शरभस्य च} %4-33-9

\twolineshloka
{विद्युन्मालेश्च सम्पातेः सूर्याक्षस्य हनूमतः}
{वीरबाहोः सुबाहोश्च नलस्य च महात्मनः} %4-33-10

\twolineshloka
{कुमुदस्य सुषेणस्य तारजाम्बवतोस्तथा}
{दधिवक्त्रस्य नीलस्य सुपाटलसुनेत्रयोः} %4-33-11

\twolineshloka
{एतेषां कपिमुख्यानां राजमार्गे महात्मनाम्}
{ददर्श गृहमुख्यानि महासाराणि लक्ष्मणः} %4-33-12

\twolineshloka
{पाण्डुराभ्रप्रकाशानि गन्धमाल्ययुतानि च}
{प्रभूतधनधान्यानि स्त्रीरत्नैः शोभितानि च} %4-33-13

\twolineshloka
{पाण्डुरेण तु शैलेन परिक्षिप्तं दुरासदम्}
{वानरेन्द्रगृहं रम्यं महेन्द्रसदनोपमम्} %4-33-14

\twolineshloka
{शुक्लैः प्रासादशिखरैः कैलासशिखरोपमैः}
{सर्वकामफलैर्वृक्षैः पुष्पितैरुपशोभितम्} %4-33-15

\twolineshloka
{महेन्द्रदत्तैः श्रीमद्भिर्नीलजीमूतसंनिभैः}
{दिव्यपुष्पफलैर्वृक्षैः शीतच्छायैर्मनोरमैः} %4-33-16

\twolineshloka
{हरिभिः संवृतद्वारं बलिभिः शस्त्रपाणिभिः}
{दिव्यमाल्यावृतं शुभ्रं तप्तकाञ्चनतोरणम्} %4-33-17

\twolineshloka
{सुग्रीवस्य गृहं रम्यं प्रविवेश महाबलः}
{अवार्यमाणः सौमित्रिर्महाभ्रमिव भास्करः} %4-33-18

\twolineshloka
{स सप्त कक्ष्या धर्मात्मा यानासनसमावृताः}
{ददर्श सुमहद्गुप्तं ददर्शान्तःपुरं महत्} %4-33-19

\twolineshloka
{हैमराजतपर्यङ्कैर्बहुभिश्च वरासनैः}
{महार्हास्तरणोपेतैस्तत्र तत्र समावृतम्} %4-33-20

\twolineshloka
{प्रविशन्नेव सततं शुश्राव मधुरस्वनम्}
{तन्त्रीगीतसमाकीर्णं समतालपदाक्षरम्} %4-33-21

\twolineshloka
{बह्वीश्च विविधाकारा रूपयौवनगर्विताः}
{स्त्रियः सुग्रीवभवने ददर्श स महाबलः} %4-33-22

\twolineshloka
{दृष्ट्वाभिजनसम्पन्नास्तत्र माल्यकृतस्रजः}
{वरमाल्यकृतव्यग्रा भूषणोत्तमभूषिताः} %4-33-23

\twolineshloka
{नातृप्तान् नाति चाव्यग्रान् नानुदात्तपरिच्छदान्}
{सुग्रीवानुचरांश्चापि लक्षयामास लक्ष्मणः} %4-33-24

\twolineshloka
{कूजितं नूपुराणां च काञ्चीनां निःस्वनं तथा}
{स निशम्य ततः श्रीमान् सौमित्रिर्लज्जितोऽभवत्} %4-33-25

\twolineshloka
{रोषवेगप्रकुपितः श्रुत्वा चाभरणस्वनम्}
{चकार ज्यास्वनं वीरो दिशः शब्देन पूरयन्} %4-33-26

\twolineshloka
{चारित्रेण महाबाहुरपकृष्टः स लक्ष्मणः}
{तस्थावेकान्तमाश्रित्य रामकोपसमन्वितः} %4-33-27

\twolineshloka
{तेन चापस्वनेनाथ सुग्रीवः प्लवगाधिपः}
{विज्ञायागमनं त्रस्तः स चचाल वरासनात्} %4-33-28

\twolineshloka
{अङ्गदेन यथा मह्यं पुरस्तात् प्रतिवेदितम्}
{सुव्यक्तमेष सम्प्राप्तः सौमित्रिर्भ्रातृवत्सलः} %4-33-29

\twolineshloka
{अङ्गदेन समाख्यातो ज्यास्वनेन च वानरः}
{बुबुधे लक्ष्मणं प्राप्तं मुखं चास्य व्यशुष्यत} %4-33-30

\twolineshloka
{ततस्तारां हरिश्रेष्ठः सुग्रीवः प्रियदर्शनाम्}
{उवाच हितमव्यग्रस्त्राससम्भ्रान्तमानसः} %4-33-31

\twolineshloka
{किं नु रुट्कारणं सुभ्रु प्रकृत्या मृदुमानसः}
{सरोष इव सम्प्राप्तो येनायं राघवानुजः} %4-33-32

\twolineshloka
{किं पश्यसि कुमारस्य रोषस्थानमनिन्दिते}
{न खल्वकारणे कोपमाहरेन्नरपुङ्गवः} %4-33-33

\twolineshloka
{यद्यस्य कृतमस्माभिर्बुध्यसे किंचिदप्रियम्}
{तद्बुद्ध्या सम्प्रधार्याशु क्षिप्रमेवाभिधीयताम्} %4-33-34

\twolineshloka
{अथवा स्वयमेवैनं द्रष्टुमर्हसि भामिनि}
{वचनैः सान्त्वयुक्तैश्च प्रसादयितुमर्हसि} %4-33-35

\twolineshloka
{त्वद्दर्शने विशुद्धात्मा न स्म कोपं करिष्यति}
{नहि स्त्रीषु महात्मानः क्वचित् कुर्वन्ति दारुणम्} %4-33-36

\twolineshloka
{त्वया सान्त्वैरुपक्रान्तं प्रसन्नेन्द्रियमानसम्}
{ततः कमलपत्राक्षं द्रक्ष्याम्यहमरिंदमम्} %4-33-37

\twolineshloka
{सा प्रस्खलन्ती मदविह्वलाक्षी प्रलम्बकाञ्चीगुणहेमसूत्रा}
{सलक्षणा लक्ष्मणसंनिधानं जगाम तारा नमिताङ्गयष्टिः} %4-33-38

\twolineshloka
{स तां समीक्ष्यैव हरीशपत्नीं तस्थावुदासीनतया महात्मा}
{अवाङ्मुखोऽभून्मनुजेन्द्रपुत्रः स्त्रीसंनिकर्षाद् विनिवृत्तकोपः} %4-33-39

\twolineshloka
{सा पानयोगाच्च निवृत्तलज्जा दृष्टिप्रसादाच्च नरेन्द्रसूनोः}
{उवाच तारा प्रणयप्रगल्भं वाक्यं महार्थं परिसान्त्वरूपम्} %4-33-40

\twolineshloka
{किं कोपमूलं मनुजेन्द्रपुत्र कस्ते न संतिष्ठति वाङ्निदेशे}
{कः शुष्कवृक्षं वनमापतन्तं दावाग्निमासीदति निर्विशङ्कः} %4-33-41

\twolineshloka
{स तस्या वचनं श्रुत्वा सान्त्वपूर्वमशङ्कितः}
{भूयः प्रणयदृष्टार्थं लक्ष्मणो वाक्यमब्रवीत्} %4-33-42

\twolineshloka
{किमयं कामवृत्तस्ते लुप्तधर्मार्थसंग्रहः}
{भर्ता भर्तृहिते युक्ते न चैनमवबुध्यसे} %4-33-43

\twolineshloka
{न चिन्तयति राज्यार्थं सोऽस्मान् शोकपरायणान्}
{सामात्यपरिषत् तारे काममेवोपसेवते} %4-33-44

\twolineshloka
{स मासांश्चतुरः कृत्वा प्रमाणं प्लवगेश्वरः}
{व्यतीतांस्तान् मदोदग्रो विहरन् नावबुध्यते} %4-33-45

\twolineshloka
{नहि धर्मार्थसिद्ध्यर्थं पानमेवं प्रशस्यते}
{पानादर्थश्च कामश्च धर्मश्च परिहीयते} %4-33-46

\twolineshloka
{धर्मलोपो महांस्तावत् कृते ह्यप्रतिकुर्वतः}
{अर्थलोपश्च मित्रस्य नाशे गुणवतो महान्} %4-33-47

\twolineshloka
{मित्रं ह्यर्थगुणश्रेष्ठं सत्यधर्मपरायणम्}
{तद्द्वयं तु परित्यक्तं न तु धर्मे व्यवस्थितम्} %4-33-48

\twolineshloka
{तदेवं प्रस्तुते कार्ये कार्यमस्माभिरुत्तरम्}
{तत् कार्यं कार्यतत्त्वज्ञे त्वमुदाहर्तुमर्हसि} %4-33-49

\twolineshloka
{सा तस्य धर्मार्थसमाधियुक्तं निशम्य वाक्यं मधुरस्वभावम्}
{तारा गतार्थे मनुजेन्द्रकार्ये विश्वासयुक्तं तमुवाच भूयः} %4-33-50

\twolineshloka
{न कोपकालः क्षितिपालपुत्र न चापि कोपः स्वजने विधेयः}
{त्वदर्थकामस्य जनस्य तस्य प्रमादमप्यर्हसि वीर सोढुम्} %4-33-51

\twolineshloka
{कोपं कथं नाम गुणप्रकृष्टः कुमार कुर्यादपकृष्टसत्त्वे}
{कस्त्वद्विधः कोपवशं हि गच्छेत् सत्त्वावरुद्धस्तपसः प्रसूतिः} %4-33-52

\twolineshloka
{जानामि कोपं हरिवीरबन्धोर्जानामि कार्यस्य च कालसङ्गम्}
{जानामि कार्यं त्वयि यत्कृतं नस्तच्चापि जानामि यदत्र कार्यम्} %4-33-53

\twolineshloka
{तच्चापि जानामि तथाविषह्यं बलं नरश्रेष्ठ शरीरजस्य}
{जानामि यस्मिंश्च जनेऽवबद्धं कामेन सुग्रीवमसक्तमद्य} %4-33-54

\twolineshloka
{न कामतन्त्रे तव बुद्धिरस्ति त्वं वै यथा मन्युवशं प्रपन्नः}
{न देशकालौ हि यथार्थधर्माववेक्षते कामरतिर्मनुष्यः} %4-33-55

\twolineshloka
{तं कामवृत्तं मम संनिकृष्टं कामाभियोगाच्च विमुक्तलज्जम्}
{क्षमस्व तावत् परवीरहन्तस्त्वद्भ्रातरं वानरवंशनाथम्} %4-33-56

\twolineshloka
{महर्षयो धर्मतपोऽभिरामाः कामानुकामाः प्रतिबद्धमोहाः}
{अयं प्रकृत्या चपलः कपिस्तु कथं न सज्जेत सुखेषु राजा} %4-33-57

\twolineshloka
{इत्येवमुक्त्वा वचनं महार्थं सा वानरी लक्ष्मणमप्रमेयम्}
{पुनः सखेदं मदविह्वलाक्षी भर्तुर्हितं वाक्यमिदं बभाषे} %4-33-58

\twolineshloka
{उद्योगस्तु चिराज्ञप्तः सुग्रीवेण नरोत्तम}
{कामस्यापि विधेयेन तवार्थप्रतिसाधने} %4-33-59

\twolineshloka
{आगता हि महावीर्या हरयः कामरूपिणः}
{कोटीः शतसहस्राणि नानानगनिवासिनः} %4-33-60

\twolineshloka
{तदागच्छ महाबाहो चारित्रं रक्षितं त्वया}
{अच्छलं मित्रभावेन सतां दारावलोकनम्} %4-33-61

\twolineshloka
{तारया चाभ्यनुज्ञातस्त्वरया वापि चोदितः}
{प्रविवेश महाबाहुरभ्यन्तरमरिंदमः} %4-33-62

\twolineshloka
{ततः सुग्रीवमासीनं काञ्चने परमासने}
{महार्हास्तरणोपेते ददर्शादित्यसंनिभम्} %4-33-63

\twolineshloka
{दिव्याभरणचित्राङ्गं दिव्यरूपं यशस्विनम्}
{दिव्यमाल्याम्बरधरं महेन्द्रमिव दुर्जयम्} %4-33-64

\twolineshloka
{दिव्याभरणमाल्याभिः प्रमदाभिः समावृतम्}
{संरब्धतररक्ताक्षो बभूवान्तकसंनिभः} %4-33-65

\twolineshloka
{रुमां तु वीरः परिरभ्य गाढं वरासनस्थो वरहेमवर्णः}
{ददर्श सौमित्रिमदीनसत्त्वं विशालनेत्रः स विशालनेत्रम्} %4-33-66


॥इत्यार्षे श्रीमद्रामायणे वाल्मीकीये आदिकाव्ये किष्किन्धाकाण्डे तारासान्त्ववचनम् नाम त्रयस्त्रिंशः सर्गः ॥४-३३॥
