\sect{एकविंशः सर्गः — हनुमदाश्वासनम्}

\twolineshloka
{ततो निपतितां तारां च्युतां तारामिवाम्बरात्}
{शनैराश्वासयामास हनुमान् हरियूथपः} %4-21-1

\twolineshloka
{गुणदोषकृतं जन्तुः स्वकर्म फलहेतुकम्}
{अव्यग्रस्तदवाप्नोति सर्वं प्रेत्य शुभाशुभम्} %4-21-2

\twolineshloka
{शोच्या शोचसि कं शोच्यं दीनं दीनानुकम्पसे}
{कश्च कस्यानुशोच्योऽस्ति देहेऽस्मिन् बुद्बुदोपमे} %4-21-3

\twolineshloka
{अङ्गदस्तु कुमारोऽयं द्रष्टव्यो जीवपुत्रया}
{आयत्यां च विधेयानि समर्थान्यस्य चिन्तय} %4-21-4

\twolineshloka
{जानास्यनियतामेवं भूतानामागतिं गतिम्}
{तस्माच्छुभं हि कर्तव्यं पण्डिते नेह लौकिकम्} %4-21-5

\twolineshloka
{यस्मिन् हरिसहस्राणि शतानि नियुतानि च}
{वर्तयन्ति कृताशानि सोऽयं दिष्टान्तमागतः} %4-21-6

\twolineshloka
{यदयं न्यायदृष्टार्थः सामदानक्षमापरः}
{गतो धर्मजितां भूमिं नैनं शोचितुमर्हसि} %4-21-7

\twolineshloka
{सर्वे च हरिशार्दूलाः पुत्रश्चायं तवाङ्गदः}
{हर्यृक्षपतिराज्यं च त्वत्सनाथमनिन्दिते} %4-21-8

\twolineshloka
{ताविमौ शोकसंतप्तौ शनैः प्रेरय भामिनि}
{त्वया परिगृहीतोऽयमङ्गदः शास्तु मेदिनीम्} %4-21-9

\twolineshloka
{संततिश्च यथा दृष्टा कृत्यं यच्चापि साम्प्रतम्}
{राज्ञस्तत् क्रियतां सर्वमेष कालस्य निश्चयः} %4-21-10

\twolineshloka
{संस्कार्यो हरिराजस्तु अङ्गदश्चाभिषिच्यताम्}
{सिंहासनगतं पुत्रं पश्यन्ती शान्तिमेष्यसि} %4-21-11

\twolineshloka
{सा तस्य वचनं श्रुत्वा भर्तृव्यसनपीडिता}
{अब्रवीदुत्तरं तारा हनूमन्तमवस्थितम्} %4-21-12

\twolineshloka
{अङ्गदप्रतिरूपाणां पुत्राणामेकतः शतम्}
{हतस्याप्यस्य वीरस्य गात्रसंश्लेषणं वरम्} %4-21-13

\twolineshloka
{न चाहं हरिराज्यस्य प्रभवाम्यङ्गदस्य वा}
{पितृव्यस्तस्य सुग्रीवः सर्वकार्येष्वनन्तरः} %4-21-14

\twolineshloka
{नह्येषा बुद्धिरास्थेया हनूमन्नङ्गदं प्रति}
{पिता हि बन्धुः पुत्रस्य न माता हरिसत्तम} %4-21-15

\twolineshloka
{नहि मम हरिराजसंश्रयात् क्षमतरमस्ति परत्र चेह वा}
{अभिमुखहतवीरसेवितं शयनमिदं मम सेवितुं क्षमम्} %4-21-16


॥इत्यार्षे श्रीमद्रामायणे वाल्मीकीये आदिकाव्ये किष्किन्धाकाण्डे हनुमदाश्वासनम् नाम एकविंशः सर्गः ॥४-२१॥
