\sect{पञ्चपञ्चाशः सर्गः — प्रायोपवेशः}

\twolineshloka
{श्रुत्वा हनुमतो वाक्यं प्रश्रितं धर्मसंहितम्}
{स्वामिसत्कारसंयुक्तमङ्गदो वाक्यमब्रवीत्} %4-55-1

\twolineshloka
{स्थैर्यमात्ममनःशौचमानृशंस्यमथार्जवम्}
{विक्रमश्चैव धैर्यं च सुग्रीवे नोपपद्यते} %4-55-2

\twolineshloka
{भ्रातुर्ज्येष्ठस्य यो भार्यां जीवतो महिषीं प्रियाम्}
{धर्मेण मातरं यस्तु स्वीकरोति जुगुप्सितः} %4-55-3

\twolineshloka
{कथं स धर्मं जानीते येन भ्रात्रा दुरात्मना}
{युद्धायाभिनियुक्तेन बिलस्य पिहितं मुखम्} %4-55-4

\twolineshloka
{सत्यात् पाणिगृहीतश्च कृतकर्मा महायशाः}
{विस्मृतो राघवो येन स कस्य सुकृतं स्मरेत्} %4-55-5

\twolineshloka
{लक्ष्मणस्य भयेनेह नाधर्मभयभीरुणा}
{आदिष्टा मार्गितुं सीता धर्मस्तस्मिन् कथं भवेत्} %4-55-6

\twolineshloka
{तस्मिन् पापे कृतघ्ने तु स्मृतिभिन्ने चलात्मनि}
{आर्यः को विश्वसेज्जातु तत्कुलीनो विशेषतः} %4-55-7

\twolineshloka
{राज्ये पुत्रः प्रतिष्ठाप्यः सगुणो निर्गुणोऽपि वा}
{कथं शत्रुकुलीनं मां सुग्रीवो जीवयिष्यति} %4-55-8

\twolineshloka
{भिन्नमन्त्रोऽपराद्धश्च भिन्नशक्तिः कथं ह्यहम्}
{किष्किन्धां प्राप्य जीवेयमनाथ इव दुर्बलः} %4-55-9

\twolineshloka
{उपांशुदण्डेन हि मां बन्धनेनोपपादयेत्}
{शठः क्रूरो नृशंसश्च सुग्रीवो राज्यकारणात्} %4-55-10

\twolineshloka
{बन्धनाच्चावसादान्मे श्रेयः प्रायोपवेशनम्}
{अनुजानन्तु मां सर्वे गृहं गच्छन्तु वानराः} %4-55-11

\twolineshloka
{अहं वः प्रतिजानामि न गमिष्याम्यहं पुरीम्}
{इहैव प्रायमासिष्ये श्रेयो मरणमेव मे} %4-55-12

\twolineshloka
{अभिवादनपूर्वं तु राजा कुशलमेव च}
{अभिवादनपूर्वं तु राघवौ बलशालिनौ} %4-55-13

\twolineshloka
{वाच्यस्तातो यवीयान् मे सुग्रीवो वानरेश्वरः}
{आरोग्यपूर्वं कुशलं वाच्या माता रुमा च मे} %4-55-14

\twolineshloka
{मातरं चैव मे तारामाश्वासयितुमर्हथ}
{प्रकृत्या प्रियपुत्रा सा सानुक्रोशा तपस्विनी} %4-55-15

\twolineshloka
{विनष्टमिह मां श्रुत्वा व्यक्तं हास्यति जीवितम्}
{एतावदुक्त्वा वचनं वृद्धांस्तानभिवाद्य च} %4-55-16

\twolineshloka
{विवेश चाङ्गदो भूमौ रुदन् दर्भेषु दुर्मनाः}
{तस्य संविशतस्तत्र रुदन्तो वानरर्षभाः} %4-55-17

\twolineshloka
{नयनेभ्यः प्रमुमुचुरुष्णं वै वारि दुःखिताः}
{सुग्रीवं चैव निन्दन्तः प्रशंसन्तश्च वालिनम्} %4-55-18

\twolineshloka
{परिवार्याङ्गदं सर्वे व्यवसन् प्रायमासितुम्}
{तद् वाक्यं वालिपुत्रस्य विज्ञाय प्लवगर्षभाः} %4-55-19

\twolineshloka
{उपस्पृश्योदकं सर्वे प्राङ्मुखाः समुपाविशन्}
{दक्षिणाग्रेषु दर्भेषु उदक्तीरं समाश्रिताः} %4-55-20

\twolineshloka
{मुमूर्षवो हरिश्रेष्ठा एतत् क्षममिति स्म ह}
{रामस्य वनवासं च क्षयं दशरथस्य च} %4-55-21

\threelineshloka
{जनस्थानवधं चैव वधं चैव जटायुषः}
{हरणं चैव वैदेह्या वालिनश्च वधं तथा}
{रामकोपं च वदतां हरीणां भयमागतम्} %4-55-22

\twolineshloka
{स संविशद्भिर्बहुभिर्महीधरो महाद्रिकूटप्रतिमैः प्लवंगमैः}
{बभूव संनादितनिर्दरान्तरो भृशं नदद्भिर्जलदैरिवाम्बरम्} %4-55-23


॥इत्यार्षे श्रीमद्रामायणे वाल्मीकीये आदिकाव्ये किष्किन्धाकाण्डे प्रायोपवेशः नाम पञ्चपञ्चाशः सर्गः ॥४-५५॥
