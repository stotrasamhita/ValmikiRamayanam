\sect{अष्टात्रिंशः सर्गः — रामसमीपगमनम्}

\twolineshloka
{प्रतिगृह्य च तत् सर्वमुपायनमुपाहृतम्}
{वानरान् सान्त्वयित्वा च सर्वानेव व्यसर्जयत्} %4-38-1

\twolineshloka
{विसर्जयित्वा स हरीन् सहस्रान् कृतकर्मणः}
{मेने कृतार्थमात्मानं राघवं च महाबलम्} %4-38-2

\twolineshloka
{स लक्ष्मणो भीमबलं सर्ववानरसत्तमम्}
{अब्रवीत् प्रश्रितं वाक्यं सुग्रीवं सम्प्रहर्षयन्} %4-38-3

\twolineshloka
{किष्किन्धाया विनिष्क्राम यदि ते सौम्य रोचते}
{तस्य तद् वचनं श्रुत्वा लक्ष्मणस्य सुभाषितम्} %4-38-4

\twolineshloka
{सुग्रीवः परमप्रीतो वाक्यमेतदुवाच ह}
{एवं भवतु गच्छाम स्थेयं त्वच्छासने मया} %4-38-5

\twolineshloka
{तमेवमुक्त्वा सुग्रीवो लक्ष्मणं शुभलक्षणम्}
{विसर्जयामास तदा ताराद्याश्चैव योषितः} %4-38-6

\twolineshloka
{एहीत्युच्चैर्हरिवरान् सुग्रीवः समुदाहरत्}
{तस्य तद् वचनं श्रुत्वा हरयः शीघ्रमाययुः} %4-38-7

\twolineshloka
{बद्धाञ्जलिपुटाः सर्वे ये स्युः स्त्रीदर्शनक्षमाः}
{तानुवाच ततः प्राप्तान् राजार्कसदृशप्रभः} %4-38-8

\twolineshloka
{उपस्थापयत क्षिप्रं शिबिकां मम वानराः}
{श्रुत्वा तु वचनं तस्य हरयः शीघ्रविक्रमाः} %4-38-9

\twolineshloka
{समुपस्थापयामासुः शिबिकां प्रियदर्शनाम्}
{तामुपस्थापितां दृष्ट्वा शिबिकां वानराधिपः} %4-38-10

\twolineshloka
{लक्ष्मणारुह्यतां शीघ्रमिति सौमित्रिमब्रवीत्}
{इत्युक्त्वा काञ्चनं यानं सुग्रीवः सूर्यसंनिभम्} %4-38-11

\twolineshloka
{बहुभिर्हरिभिर्युक्तमारुरोह सलक्ष्मणः}
{पाण्डुरेणातपत्रेण ध्रियमाणेन मूर्धनि} %4-38-12

\twolineshloka
{शुक्लैश्च वालव्यजनैर्धूयमानैः समन्ततः}
{शंखभेरीनिनादैश्च बन्दिभिश्चाभिनन्दितः} %4-38-13

\twolineshloka
{निर्ययौ प्राप्य सुग्रीवो राज्यश्रियमनुत्तमाम्}
{स वानरशतैस्तीक्ष्णैर्बहुभिः शस्त्रपाणिभिः} %4-38-14

\twolineshloka
{परिकीर्णो ययौ तत्र यत्र रामो व्यवस्थितः}
{स तं देशमनुप्राप्य श्रेष्ठं रामनिषेवितम्} %4-38-15

\twolineshloka
{अवातरन्महातेजाः शिबिकायाः सलक्ष्मणः}
{आसाद्य च ततो रामं कृताञ्जलिपुटोऽभवत्} %4-38-16

\twolineshloka
{कृताञ्जलौ स्थिते तस्मिन् वानराश्चाभवंस्तथा}
{तटाकमिव तं दृष्ट्वा रामः कुड्मलपङ्कजम्} %4-38-17

\twolineshloka
{वानराणां महत् सैन्यं सुग्रीवे प्रीतिमानभूत्}
{पादयोः पतितं मूर्ध्ना तमुत्थाप्य हरीश्वरम्} %4-38-18

\twolineshloka
{प्रेम्णा च बहुमानाच्च राघवः परिषस्वजे}
{परिष्वज्य च धर्मात्मा निषीदेति ततोऽब्रवीत्} %4-38-19

\twolineshloka
{निषण्णं तं ततो दृष्ट्वा क्षितौ रामोऽब्रवीत् ततः}
{धर्ममर्थं च कामं च काले यस्तु निषेवते} %4-38-20

\twolineshloka
{विभज्य सततं वीर स राजा हरिसत्तम}
{हित्वा धर्मं तथार्थं च कामं यस्तु निषेवते} %4-38-21

\twolineshloka
{स वृक्षाग्रे यथा सुप्तः पतितः प्रतिबुध्यते}
{अमित्राणां वधे युक्तो मित्राणां संग्रहे रतः} %4-38-22

\twolineshloka
{त्रिवर्गफलभोक्ता च राजा धर्मेण युज्यते}
{उद्योगसमयस्त्वेष प्राप्तः शत्रुनिषूदन} %4-38-23

\twolineshloka
{संचिन्त्यतां हि पिङ्गेश हरिभिः सह मन्त्रिभिः}
{एवमुक्तस्तु सुग्रीवो रामं वचनमब्रवीत्} %4-38-24

\twolineshloka
{प्रणष्टा श्रीश्च कीर्तिश्च कपिराज्यं च शाश्वतम्}
{त्वत्प्रसादान्महाबाहो पुनः प्राप्तमिदं मया} %4-38-25

\twolineshloka
{तव देव प्रसादाच्च भ्रातुश्च जयतां वर}
{कृतं न प्रतिकुर्याद् यः पुरुषाणां हि दूषकः} %4-38-26

\twolineshloka
{एते वानरमुख्याश्च शतशः शत्रुसूदन}
{प्राप्ताश्चादाय बलिनः पृथिव्यां सर्ववानरान्} %4-38-27

\twolineshloka
{ऋक्षाश्च वानराः शूरा गोलाङ्गूलाश्च राघव}
{कान्तारवनदुर्गाणामभिज्ञा घोरदर्शनाः} %4-38-28

\twolineshloka
{देवगन्धर्वपुत्राश्च वानराः कामरूपिणः}
{स्वैः स्वैः परिवृताः सैन्यैर्वर्तन्ते पथि राघव} %4-38-29

\twolineshloka
{शतैः शतसहस्रैश्च वर्तन्ते कोटिभिस्तथा}
{अयुतैश्चावृता वीर शङ्कुभिश्च परंतप} %4-38-30

\twolineshloka
{अर्बुदैरर्बुदशतैर्मध्यैश्चान्त्यैश्च वानराः}
{समुद्राश्च परार्धाश्च हरयो हरियूथपाः} %4-38-31

\twolineshloka
{आगमिष्यन्ति ते राजन् महेन्द्रसमविक्रमाः}
{मेघपर्वतसंकाशा मेरुविन्ध्यकृतालयाः} %4-38-32

\twolineshloka
{ते त्वामभिगमिष्यन्ति राक्षसं योद्धुमाहवे}
{निहत्य रावणं युद्धे ह्यानयिष्यन्ति मैथिलीम्} %4-38-33

\twolineshloka
{ततः समुद्योगमवेक्ष्य वीर्यवान् हरिप्रवीरस्य निदेशवर्तिनः}
{बभूव हर्षाद् वसुधाधिपात्मजः प्रबुद्धनीलोत्पलतुल्यदर्शनः} %4-38-34


॥इत्यार्षे श्रीमद्रामायणे वाल्मीकीये आदिकाव्ये किष्किन्धाकाण्डे रामसमीपगमनम् नाम अष्टात्रिंशः सर्गः ॥४-३८॥
