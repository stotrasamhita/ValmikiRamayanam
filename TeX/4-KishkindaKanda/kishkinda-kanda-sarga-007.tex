\sect{सप्तमः सर्गः — रामसमाश्वासनम्}

\twolineshloka
{एवमुक्तस्तु सुग्रीवो रामेणार्तेन वानरः}
{अब्रवीत् प्राञ्जलिर्वाक्यं सबाष्पं बाष्पगद्गदः} %4-7-1

\twolineshloka
{न जाने निलयं तस्य सर्वथा पापरक्षसः}
{सामर्थ्यं विक्रमं वापि दौष्कुलेयस्य वा कुलम्} %4-7-2

\twolineshloka
{सत्यं तु प्रतिजानामि त्यज शोकमरिंदम}
{करिष्यामि तथा यत्नं यथा प्राप्स्यसि मैथिलीम्} %4-7-3

\twolineshloka
{रावणं सगणं हत्वा परितोष्यात्मपौरुषम्}
{तथास्मि कर्ता नचिराद् यथा प्रीतो भविष्यसि} %4-7-4

\twolineshloka
{अलं वैक्लव्यमालम्ब्य धैर्यमात्मगतं स्मर}
{त्वद्विधानां न सदृशमीदृशं बुद्धिलाघवम्} %4-7-5

\twolineshloka
{मयापि व्यसनं प्राप्तं भार्याविरहजं महत्}
{नाहमेवं हि शोचामि धैर्यं न च परित्यजे} %4-7-6

\twolineshloka
{नाहं तामनुशोचामि प्राकृतो वानरोऽपि सन्}
{महात्मा च विनीतश्च किं पुनर्धृतिमान् महान्} %4-7-7

\twolineshloka
{बाष्पमापतितं धैर्यान्निग्रहीतुं त्वमर्हसि}
{मर्यादां सत्त्वयुक्तानां धृतिं नोत्स्रष्टुमर्हसि} %4-7-8

\twolineshloka
{व्यसने वार्थकृच्छ्रे वा भये वा जीवितान्तगे}
{विमृशंश्च स्वयाबुद्ध्या धृतिमान् नावसीदति} %4-7-9

\twolineshloka
{बालिशस्तु नरो नित्यं वैक्लव्यं योऽनुवर्तते}
{स मज्जत्यवशः शोके भाराक्रान्तेव नौर्जले} %4-7-10

\twolineshloka
{एषोऽञ्जलिर्मया बद्धः प्रणयात् त्वां प्रसादये}
{पौरुषं श्रय शोकस्य नान्तरं दातुमर्हसि} %4-7-11

\twolineshloka
{ये शोकमनुवर्तन्ते न तेषां विद्यते सुखम्}
{तेजश्च क्षीयते तेषां न त्वं शोचितुमर्हसि} %4-7-12

\twolineshloka
{शोकेनाभिप्रपन्नस्य जीविते चापि संशयः}
{स शोकं त्यज राजेन्द्र धैर्यमाश्रय केवलम्} %4-7-13

\twolineshloka
{हितं वयस्यभावेन ब्रूहि नोपदिशामि ते}
{वयस्यतां पूजयन्मे न त्वं शोचितुमर्हसि} %4-7-14

\twolineshloka
{मधुरं सान्त्वितस्तेन सुग्रीवेण स राघवः}
{मुखमश्रुपरिक्लिन्नं वस्त्रान्तेन प्रमार्जयत्} %4-7-15

\twolineshloka
{प्रकृतिस्थस्तु काकुत्स्थः सुग्रीवचनात् प्रभुः}
{सम्परिष्वज्य सुग्रीवमिदं वचनमब्रवीत्} %4-7-16

\twolineshloka
{कर्तव्यं यद् वयस्येन स्निग्धेन च हितेन च}
{अनुरूपं च युक्तं च कृतं सुग्रीव तत् त्वया} %4-7-17

\twolineshloka
{एष च प्रकृतिस्थोऽहमनुनीतस्त्वया सखे}
{दुर्लभो हीदृशो बन्धुरस्मिन् काले विशेषतः} %4-7-18

\twolineshloka
{किं तु यत्नस्त्वया कार्यो मैथिल्याः परिमार्गणे}
{राक्षसस्य च रौद्रस्य रावणस्य दुरात्मनः} %4-7-19

\twolineshloka
{मया च यदनुष्ठेयं विस्रब्धेन तदुच्यताम्}
{वर्षास्विव च सुक्षेत्रे सर्वं सम्पद्यते तव} %4-7-20

\twolineshloka
{मया च यदिदं वाक्यमभिमानात् समीरितम्}
{तत्त्वया हरिशार्दूल तत्त्वमित्युपधार्यताम्} %4-7-21

\twolineshloka
{अनृतं नोक्तपूर्वं मे न च वक्ष्ये कदाचन}
{एतत्ते प्रतिजानामि सत्येनैव शपाम्यहम्} %4-7-22

\twolineshloka
{ततः प्रहृष्टः सुग्रीवो वानरैः सचिवैः सह}
{राघवस्य वचः श्रुत्वा प्रतिज्ञातं विशेषतः} %4-7-23

\twolineshloka
{एवमेकान्तसम्पृक्तौ ततस्तौ नरवानरौ}
{उभावन्योन्यसदृशं सुखं दुःखमभाषताम्} %4-7-24

\twolineshloka
{महानुभावस्य वचो निशम्य हरिर्नृपाणामधिपस्य तस्य}
{कृतं स मेने हरिवीरमुख्यस्तदा च कार्यं हृदयेन विद्वान्} %4-7-25


॥इत्यार्षे श्रीमद्रामायणे वाल्मीकीये आदिकाव्ये किष्किन्धाकाण्डे रामसमाश्वासनम् नाम सप्तमः सर्गः ॥४-७॥
