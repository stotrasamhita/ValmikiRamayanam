\sect{पञ्चत्रिंशः सर्गः — तारासमाधानम्}

\twolineshloka
{तथा ब्रूवाणं सौमित्रिं प्रदीप्तमिव तेजसा}
{अब्रवील्लक्ष्मणं तारा ताराधिपनिभानना} %4-35-1

\twolineshloka
{नैवं लक्ष्मण वक्तव्यो नायं परुषमर्हति}
{हरीणामीश्वरः श्रोतुं तव वक्त्राद् विशेषतः} %4-35-2

\twolineshloka
{नैवाकृतज्ञः सुग्रीवो न शठो नापि दारुणः}
{नैवानृतकथो वीर न जिह्मश्च कपीश्वरः} %4-35-3

\twolineshloka
{उपकारं कृतं वीरो नाप्ययं विस्मृतः कपिः}
{रामेण वीर सुग्रीवो यदन्यैर्दुष्करं रणे} %4-35-4

\twolineshloka
{रामप्रसादात् कीर्तिं च कपिराज्यं च शाश्वतम्}
{प्राप्तवानिह सुग्रीवो रुमां मां च परन्तप} %4-35-5

\twolineshloka
{सुदुःखशयितः पूर्वं प्राप्येदं सुखमुत्तमम्}
{प्राप्तकालं न जानीते विश्वामित्रो यथा मुनिः} %4-35-6

\twolineshloka
{घृताच्यां किल संसक्तो दश वर्षाणि लक्ष्मण}
{अहोऽमन्यत धर्मात्मा विश्वामित्रो महामुनिः} %4-35-7

\twolineshloka
{स हि प्राप्तं न जानीते कालं कालविदां वरः}
{विश्वामित्रो महातेजाः किं पुनर्यः पृथग्जनः} %4-35-8

\twolineshloka
{देहधर्मगतस्यास्य परिश्रान्तस्य लक्ष्मण}
{अवितृप्तस्य कामेषु रामः क्षन्तुमिहार्हति} %4-35-9

\twolineshloka
{न च रोषवशं तात गन्तुमर्हसि लक्ष्मण}
{निश्चयार्थमविज्ञाय सहसा प्राकृतो यथा} %4-35-10

\twolineshloka
{सत्त्वयुक्ता हि पुरुषास्त्वद्विधाः पुरुषर्षभ}
{अविमृश्य न रोषस्य सहसा यान्ति वश्यताम्} %4-35-11

\twolineshloka
{प्रसादये त्वां धर्मज्ञ सुग्रीवार्थं समाहिता}
{महान् रोषसमुत्पन्नः संरम्भस्त्यज्यतामयम्} %4-35-12

\twolineshloka
{रुमां मां चाङ्गदं राज्यं धनधान्यपशूनि च}
{रामप्रियार्थं सुग्रीवस्त्यजेदिति मतिर्मम} %4-35-13

\twolineshloka
{समानेष्यति सुग्रीवः सीतया सह राघवम्}
{शशाङ्कमिव रोहिण्या हत्वा तं राक्षसाधमम्} %4-35-14

\twolineshloka
{शतकोटिसहस्राणि लङ्कायां किल रक्षसाम्}
{अयुतानि च षट्त्रिंशत्सहस्राणि शतानि च} %4-35-15

\twolineshloka
{अहत्वा तांश्च दुर्धर्षान् राक्षसान् कामरूपिणः}
{न शक्यो रावणो हन्तुं येन सा मैथिली हृता} %4-35-16

\twolineshloka
{ते न शक्या रणे हन्तुमसहायेन लक्ष्मण}
{रावणः क्रूरकर्मा च सुग्रीवेण विशेषतः} %4-35-17

\twolineshloka
{एवमाख्यातवान् वाली स ह्यभिज्ञो हरीश्वरः}
{आगमस्तु न मे व्यक्तः श्रवात् तस्य ब्रवीम्यहम्} %4-35-18

\twolineshloka
{त्वत्सहायनिमित्तं हि प्रेषिता हरिपुङ्गवाः}
{आनेतुं वानरान् युद्धे सुबहून् हरिपुङ्गवान्} %4-35-19

\twolineshloka
{तांश्च प्रतीक्षमाणोऽयं विक्रान्तान् सुमहाबलान्}
{राघवस्यार्थसिद्ध्यर्थं न निर्याति हरीश्वरः} %4-35-20

\twolineshloka
{कृता सुसंस्था सौमित्रे सुग्रीवेण पुरा यथा}
{अद्य तैर्वानरैः सर्वैरागन्तव्यं महाबलैः} %4-35-21

\threelineshloka
{ऋक्षकोटिसहस्राणि गोलाङ्गूलशतानि च}
{अद्य त्वामुपयास्यन्ति जहि कोपमरिन्दम}
{कोट्योऽनेकास्तु काकुत्स्थ कपीनां दीप्ततेजसाम्} %4-35-22

\twolineshloka
{तव हि मुखमिदं निरीक्ष्य कोपात् क्षतजसमे नयने निरीक्षमाणाः}
{हरिवरवनिता न यान्ति शान्तिं प्रथमभयस्य हि शङ्किताः स्म सर्वाः} %4-35-23


॥इत्यार्षे श्रीमद्रामायणे वाल्मीकीये आदिकाव्ये किष्किन्धाकाण्डे तारासमाधानम् नाम पञ्चत्रिंशः सर्गः ॥४-३५॥
