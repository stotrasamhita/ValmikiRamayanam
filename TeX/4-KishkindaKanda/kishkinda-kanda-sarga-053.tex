\sect{त्रिपञ्चाशः सर्गः — अङ्गदादिनिर्वेदः}

\twolineshloka
{ततस्ते ददृशुर्घोरं सागरं वरुणालयम्}
{अपारमभिगर्जन्तं घोरैरूर्मिभिराकुलम्} %4-53-1

\twolineshloka
{मयस्य मायाविहितं गिरिदुर्गं विचिन्वताम्}
{तेषां मासो व्यतिक्रान्तो यो राज्ञा समयः कृतः} %4-53-2

\twolineshloka
{विन्ध्यस्य तु गिरेः पादे सम्प्रपुष्पितपादपे}
{उपविश्य महात्मानश्चिन्तामापेदिरे तदा} %4-53-3

\twolineshloka
{ततः पुष्पातिभाराग्राँल्लताशतसमावृतान्}
{द्रुमान् वासन्तिकान् दृष्ट्वा बभूवुर्भयशङ्किताः} %4-53-4

\twolineshloka
{ते वसन्तमनुप्राप्तं प्रतिवेद्य परस्परम्}
{नष्टसन्देशकालार्था निपेतुर्धरणीतले} %4-53-5

\twolineshloka
{ततस्तान् कपिवृद्धांश्च शिष्टांश्चैव वनौकसः}
{वाचा मधुरयाऽऽभाष्य यथावदनुमान्य च} %4-53-6

\twolineshloka
{स तु सिंहवृषस्कन्धः पीनायतभुजः कपिः}
{युवराजो महाप्राज्ञ अङ्गदो वाक्यमब्रवीत्} %4-53-7

\twolineshloka
{शासनात् कपिराजस्य वयं सर्वे विनिर्गताः}
{मासः पूर्णो बिलस्थानां हरयः किं न बुध्यत} %4-53-8

\twolineshloka
{वयमाश्वयुजे मासि कालसङ्ख्याव्यवस्थिताः}
{प्रस्थिताः सोऽपि चातीतः किमतः कार्यमुत्तरम्} %4-53-9

\twolineshloka
{भवन्तः प्रत्ययं प्राप्ता नीतिमार्गविशारदाः}
{हितेष्वभिरता भर्तुर्निसृष्टाः सर्वकर्मसु} %4-53-10

\twolineshloka
{कर्मस्वप्रतिमाः सर्वे दिक्षु विश्रुतपौरुषाः}
{मां पुरस्कृत्य निर्याताः पिङ्गाक्षप्रतिचोदिताः} %4-53-11

\twolineshloka
{इदानीमकृतार्थानां मर्तव्यं नात्र संशयः}
{हरिराजस्य सन्देशमकृत्वा कः सुखी भवेत्} %4-53-12

\twolineshloka
{अस्मिन्नतीते काले तु सुग्रीवेण कृते स्वयम्}
{प्रायोपवेशनं युक्तं सर्वेषां च वनौकसाम्} %4-53-13

\twolineshloka
{तीक्ष्णः प्रकृत्या सुग्रीवः स्वामिभावे व्यवस्थितः}
{न क्षमिष्यति नः सर्वानपराधकृतो गतान्} %4-53-14

\twolineshloka
{अप्रवृत्तौ च सीतायाः पापमेव करिष्यति}
{तस्मात् क्षममिहाद्यैव गन्तुं प्रायोपवेशनम्} %4-53-15

\twolineshloka
{त्यक्त्वा पुत्रांश्च दारांश्च धनानि च गृहाणि च}
{ध्रुवं नो हिंसते राजा सर्वान् प्रतिगतानितः} %4-53-16

\twolineshloka
{वधेनाप्रतिरूपेण श्रेयान् मृत्युरिहैव नः}
{न चाहं यौवराज्येन सुग्रीवेणाभिषेचितः} %4-53-17

\twolineshloka
{नरेन्द्रेणाभिषिक्तोऽस्मि रामेणाक्लिष्टकर्मणा}
{स पूर्वं बद्धवैरो मां राजा दृष्ट्वा व्यतिक्रमम्} %4-53-18

\threelineshloka
{घातयिष्यति दण्डेन तीक्ष्णेन कृतनिश्चयः}
{किं मे सुहृद्भिर्व्यसनं पश्यद्भिर्जीवितान्तरे}
{इहैव प्रायमासिष्ये पुण्ये सागररोधसि} %4-53-19

\twolineshloka
{एतच्छ्रुत्वा कुमारेण युवराजेन भाषितम्}
{सर्वे ते वानरश्रेष्ठाः करुणं वाक्यमब्रुवन्} %4-53-20

\twolineshloka
{तीक्ष्णः प्रकृत्या सुग्रीवः प्रियारक्तश्च राघवः}
{समीक्ष्याकृतकार्यांस्तु तस्मिंश्च समये गते} %4-53-21

\twolineshloka
{अदृष्टायां च वैदेह्यां दृष्ट्वा चैव समागतान्}
{राघवप्रियकामाय घातयिष्यत्यसंशयम्} %4-53-22

\twolineshloka
{न क्षमं चापराद्धानां गमनं स्वामिपार्श्वतः}
{प्रधानभूताश्च वयं सुग्रीवस्य समागताः} %4-53-23

\twolineshloka
{इहैव सीतामन्वीक्ष्य प्रवृत्तिमुपलभ्य वा}
{नो चेद् गच्छाम तं वीरं गमिष्यामो यमक्षयम्} %4-53-24

\twolineshloka
{प्लवङ्गमानां तु भयार्दितानां श्रुत्वा वचस्तार इदं बभाषे}
{अलं विषादेन बिलं प्रविश्य वसाम सर्वे यदि रोचते वः} %4-53-25

\twolineshloka
{इदं हि मायाविहितं सुदुर्गमं प्रभूतपुष्पोदकभोज्यपेयम्}
{इहास्ति नो नैव भयं पुरन्दरान्न राघवाद् वानरराजतोऽपि वा} %4-53-26

\twolineshloka
{श्रुत्वाङ्गदस्यापि वचोऽनुकूलमूचुश्च सर्वे हरयः प्रतीताः}
{यथा न हन्येम तथा विधानमसक्तमद्यैव विधीयतां नः} %4-53-27


॥इत्यार्षे श्रीमद्रामायणे वाल्मीकीये आदिकाव्ये किष्किन्धाकाण्डे अङ्गदादिनिर्वेदः नाम त्रिपञ्चाशः सर्गः ॥४-५३॥
