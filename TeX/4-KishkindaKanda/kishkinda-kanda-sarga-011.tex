\sect{एकादशः सर्गः — वालिबलाविष्करणम्}

\twolineshloka
{रामस्य वचनं श्रुत्वा हर्षपौरुषवर्धनम्}
{सुग्रीवः पूजयाञ्चक्रे राघवं प्रशशंस च} %4-11-1

\twolineshloka
{असंशयं प्रज्वलितैस्तीक्ष्णैर्मर्मातिगैः शरैः}
{त्वं दहेः कुपितो लोकान् युगान्त इव भास्करः} %4-11-2

\twolineshloka
{वालिनः पौरुषं यत्तद् यच्च वीर्यं धृतिश्च या}
{तन्ममैकमनाः श्रुत्वा विधत्स्व यदनन्तरम्} %4-11-3

\twolineshloka
{समुद्रात् पश्चिमात् पूर्वं दक्षिणादपि चोत्तरम्}
{क्रामत्यनुदिते सूर्ये वाली व्यपगतक्लमः} %4-11-4

\twolineshloka
{अग्राण्यारुह्य शैलानां शिखराणि महान्त्यपि}
{ऊर्ध्वमुत्पात्य तरसा प्रतिगृह्णाति वीर्यवान्} %4-11-5

\twolineshloka
{बहवः सारवन्तश्च वनेषु विविधा द्रुमाः}
{वालिना तरसा भग्ना बलं प्रथयताऽऽत्मनः} %4-11-6

\twolineshloka
{महिषो दुन्दुभिर्नाम कैलासशिखरप्रभः}
{बलं नागसहस्रस्य धारयामास वीर्यवान्} %4-11-7

\twolineshloka
{स वीर्योत्सेकदुष्टात्मा वरदानेन मोहितः}
{जगाम स महाकायः समुद्रं सरितां पतिम्} %4-11-8

\twolineshloka
{ऊर्मिमन्तमतिक्रम्य सागरं रत्नसञ्चयम्}
{मम युद्धं प्रयच्छेति तमुवाच महार्णवम्} %4-11-9

\twolineshloka
{ततः समुद्रो धर्मात्मा समुत्थाय महाबलः}
{अब्रवीद् वचनं राजन्नसुरं कालचोदितम्} %4-11-10

\twolineshloka
{समर्थो नास्मि ते दातुं युद्धं युद्धविशारद}
{श्रूयतां त्वभिधास्यामि यस्ते युद्धं प्रदास्यति} %4-11-11

\twolineshloka
{शैलराजो महारण्ये तपस्विशरणं परम्}
{शङ्करश्वशुरो नाम्ना हिमवानिति विश्रुतः} %4-11-12

\twolineshloka
{महाप्रस्रवणोपेतो बहुकन्दरनिर्झरः}
{स समर्थस्तव प्रीतिमतुलां कर्तुमर्हति} %4-11-13

\twolineshloka
{तं भीतमिति विज्ञाय समुद्रमसुरोत्तमः}
{हिमवद्वनमागम्य शरश्चापादिव च्युतः} %4-11-14

\twolineshloka
{ततस्तस्य गिरेः श्वेता गजेन्द्रप्रतिमाः शिलाः}
{चिक्षेप बहुधा भूमौ दुन्दुभिर्विननाद च} %4-11-15

\twolineshloka
{ततः श्वेताम्बुदाकारः सौम्यः प्रीतिकराकृतिः}
{हिमवानब्रवीद् वाक्यं स्व एव शिखरे स्थितः} %4-11-16

\twolineshloka
{क्लेष्टुमर्हसि मां न त्वं दुन्दुभे धर्मवत्सल}
{रणकर्मस्वकुशलस्तपस्विशरणो ह्यहम्} %4-11-17

\twolineshloka
{तस्य तद् वचनं श्रुत्वा गिरिराजस्य धीमतः}
{उवाच दुन्दुभिर्वाक्यं क्रोधात् संरक्तलोचनः} %4-11-18

\twolineshloka
{यदि युद्धेऽसमर्थस्त्वं मद्भयाद् वा निरुद्यमः}
{तमाचक्ष्व प्रदद्यान्मे यो हि युद्धं युयुत्सतः} %4-11-19

\twolineshloka
{हिमवानब्रवीद् वाक्यं श्रुत्वा वाक्यविशारदः}
{अनुक्तपूर्वं धर्मात्मा क्रोधात् तमसुरोत्तमम्} %4-11-20

\twolineshloka
{वाली नाम महाप्राज्ञ शक्रपुत्रः प्रतापवान्}
{अध्यास्ते वानरः श्रीमान् किष्किन्धामतुलप्रभाम्} %4-11-21

\twolineshloka
{स समर्थो महाप्राज्ञस्तव युद्धविशारदः}
{द्वन्द्वयुद्धं स दातुं ते नमुचेरिव वासवः} %4-11-22

\twolineshloka
{तं शीघ्रमभिगच्छ त्वं यदि युद्धमिहेच्छसि}
{स हि दुर्मर्षणो नित्यं शूरः समरकर्मणि} %4-11-23

\twolineshloka
{श्रुत्वा हिमवतो वाक्यं कोपाविष्टः स दुन्दुभिः}
{जगाम तां पुरीं तस्य किष्किन्धां वालिनस्तदा} %4-11-24

\twolineshloka
{धारयन् माहिषं रूपं तीक्ष्णशृङ्गो भयावहः}
{प्रावृषीव महामेघस्तोयपूर्णो नभस्तले} %4-11-25

\twolineshloka
{ततस्तु द्वारमागम्य किष्किन्धाया महाबलः}
{ननर्द कम्पयन् भूमिं दुन्दुभिर्दुन्दुभिर्यथा} %4-11-26

\twolineshloka
{समीपजान् द्रुमान् भञ्जन् वसुधां दारयन् खुरैः}
{विषाणेनोल्लिखन् दर्पात् तद्द्वारं द्विरदो यथा} %4-11-27

\twolineshloka
{अन्तःपुरगतो वाली श्रुत्वा शब्दममर्षणः}
{निष्पपात सह स्त्रीभिस्ताराभिरिव चन्द्रमा} %4-11-28

\twolineshloka
{मितं व्यक्ताक्षरपदं तमुवाच स दुन्दुभिम्}
{हरीणामीश्वरो वाली सर्वेषां वनचारिणाम्} %4-11-29

\twolineshloka
{किमर्थं नगरद्वारमिदं रुद्ध्वा विनर्दसे}
{दुन्दुभे विदितो मेऽसि रक्ष प्राणान् महाबल} %4-11-30

\twolineshloka
{तस्य तद् वचनं श्रुत्वा वानरेन्द्रस्य धीमतः}
{उवाच दुन्दुभिर्वाक्यं क्रोधात् संरक्तलोचनः} %4-11-31

\twolineshloka
{न त्वं स्त्रीसन्निधौ वीर वचनं वक्तुमर्हसि}
{मम युद्धं प्रयच्छाद्य ततो ज्ञास्यामि ते बलम्} %4-11-32

\twolineshloka
{अथवा धारयिष्यामि क्रोधमद्य निशामिमाम्}
{गृह्यतामुदयः स्वैरं कामभोगेषु वानर} %4-11-33

\twolineshloka
{दीयतां सम्प्रदानं च परिष्वज्य च वानरान्}
{सर्वशाखामृगेन्द्रस्त्वं संसादय सुहृज्जनम्} %4-11-34

\twolineshloka
{सुदृष्टां कुरु किष्किन्धां कुरुष्वात्मसमं पुरे}
{क्रीडस्व च समं स्त्रीभिरहं ते दर्पशासनः} %4-11-35

\twolineshloka
{यो हि मत्तं प्रमत्तं वा भग्नं वा रहितं कृशम्}
{हन्यात् स भ्रूणहा लोके त्वद्विधं मदमोहितम्} %4-11-36

\twolineshloka
{स प्रहस्याब्रवीन्मन्दं क्रोधात् तमसुरेश्वरम्}
{विसृज्य ताः स्त्रियः सर्वास्ताराप्रभृतिकास्तदा} %4-11-37

\twolineshloka
{मत्तोऽयमिति मा मंस्था यद्यभीतोऽसि संयुगे}
{मदोऽयं सम्प्रहारेऽस्मिन् वीरपानं समर्थ्यताम्} %4-11-38

\twolineshloka
{तमेवमुक्त्वा सङ्क्रुद्धो मालामुत्क्षिप्य काञ्चनीम्}
{पित्रा दत्तां महेन्द्रेण युद्धाय व्यवतिष्ठत} %4-11-39

\twolineshloka
{विषाणयोर्गृहीत्वा तं दुन्दुभिं गिरिसन्निभम्}
{आविध्यत तथा वाली विनदन् कपिकुञ्जरः} %4-11-40

\twolineshloka
{बलाद् व्यापादयाञ्चक्रे ननर्द च महास्वनम्}
{श्रोत्राभ्यामथ रक्तं तु तस्य सुस्राव पात्यतः} %4-11-41

\twolineshloka
{तयोस्तु क्रोधसंरम्भात् परस्परजयैषिणोः}
{युद्धं समभवद् घोरं दुन्दुभेर्वालिनस्तथा} %4-11-42

\twolineshloka
{अयुध्यत तदा वाली शक्रतुल्यपराक्रमः}
{मुष्टिभिर्जानुभिः पद्भिः शिलाभिः पादपैस्तथा} %4-11-43

\twolineshloka
{परस्परं घ्नतोस्तत्र वानरासुरयोस्तदा}
{आसीद्धीनोऽसुरो युद्धे शक्रसूनुर्व्यवर्धत} %4-11-44

\twolineshloka
{तं तु दुन्दुभिमुद्यम्य धरण्यामभ्यपातयत्}
{युद्धे प्राणहरे तस्मिन्निष्पिष्टो दुन्दुभिस्तदा} %4-11-45

\twolineshloka
{स्रोतोभ्यो बहु रक्तं तु तस्य सुस्राव पात्यतः}
{पपात च महाबाहुः क्षितौ पञ्चत्वमागतः} %4-11-46

\twolineshloka
{तं तोलयित्वा बाहुभ्यां गतसत्त्वमचेतनम्}
{चिक्षेप वेगवान् वाली वेगेनैकेन योजनम्} %4-11-47

\twolineshloka
{तस्य वेगप्रविद्धस्य वक्त्रात् क्षतजबिन्दवः}
{प्रपेतुर्मारुतोत्क्षिप्ता मतङ्गस्याश्रमं प्रति} %4-11-48

\twolineshloka
{तान् दृष्ट्वा पतितांस्तत्र मुनिः शोणितविप्रुषः}
{क्रुद्धस्तस्य महाभाग चिन्तयामास को न्वयम्} %4-11-49

\twolineshloka
{येनाहं सहसा स्पृष्टः शोणितेन दुरात्मना}
{कोऽयं दुरात्मा दुर्बुद्धिरकृतात्मा च बालिशः} %4-11-50

\twolineshloka
{इत्युक्त्वा स विनिष्क्रम्य ददृशे मुनिसत्तमः}
{महिषं पर्वताकारं गतासुं पतितं भुवि} %4-11-51

\twolineshloka
{स तु विज्ञाय तपसा वानरेण कृतं हि तत्}
{उत्ससर्ज महाशापं क्षेप्तारं वानरं प्रति} %4-11-52

\twolineshloka
{इह तेनाप्रवेष्टव्यं प्रविष्टस्य वधो भवेत्}
{वनं मत्संश्रयं येन दूषितं रुधिरस्रवैः} %4-11-53

\twolineshloka
{क्षिपता पादपाश्चेमे सम्भग्नाश्चासुरीं तनुम्}
{समन्तादाश्रमं पूर्णं योजनं मामकं यदि} %4-11-54

\twolineshloka
{आगमिष्यति दुर्बुद्धिर्व्यक्तं स न भविष्यति}
{ये चास्य सचिवाः केचित् संश्रिता मामकं वनम्} %4-11-55

\twolineshloka
{न च तैरिह वस्तव्यं श्रुत्वा यान्तु यथासुखम्}
{तेऽपि वा यदि तिष्ठन्ति शपिष्ये तानपि ध्रुवम्} %4-11-56

\twolineshloka
{वनेऽस्मिन् मामके नित्यं पुत्रवत् परिरक्षिते}
{पत्राङ्कुरविनाशाय फलमूलाभवाय च} %4-11-57

\twolineshloka
{दिवसश्चाद्य मर्यादा यं द्रष्टा श्वोऽस्मि वानरम्}
{बहुवर्षसहस्राणि स वै शैलो भविष्यति} %4-11-58

\twolineshloka
{ततस्ते वानराः श्रुत्वा गिरं मुनिसमीरिताम्}
{निश्चक्रमुर्वनात् तस्मात् तान् दृष्ट्वा वालिरब्रवीत्} %4-11-59

\twolineshloka
{किं भवन्तः समस्ताश्च मतङ्गवनवासिनः}
{मत्समीपमनुप्राप्ता अपि स्वस्ति वनौकसाम्} %4-11-60

\twolineshloka
{ततस्ते कारणं सर्वं तथा शापं च वालिनः}
{शशंसुर्वानराः सर्वे वालिने हेममालिने} %4-11-61

\twolineshloka
{एतच्छ्रुत्वा तदा वाली वचनं वानरेरितम्}
{स महर्षिं समासाद्य याचते स्म कृताञ्जलिः} %4-11-62

\twolineshloka
{महर्षिस्तमनादृत्य प्रविवेशाश्रमं प्रति}
{शापधारणभीतस्तु वाली विह्वलतां गतः} %4-11-63

\twolineshloka
{ततः शापभयाद् भीतो ऋष्यमूकं महागिरिम्}
{प्रवेष्टुं नेच्छति हरिर्द्रष्टुं वापि नरेश्वर} %4-11-64

\twolineshloka
{तस्याप्रवेशं ज्ञात्वाहमिदं राम महावनम्}
{विचरामि सहामात्यो विषादेन विवर्जितः} %4-11-65

\twolineshloka
{एषोऽस्थिनिचयस्तस्य दुन्दुभेः सम्प्रकाशते}
{वीर्योत्सेकान्निरस्तस्य गिरिकूटनिभो महान्} %4-11-66

\twolineshloka
{इमे च विपुलाः सालाः सप्त शाखावलम्बिनः}
{यत्रैकं घटते वाली निष्पत्रयितुमोजसा} %4-11-67

\twolineshloka
{एतदस्यासमं वीर्यं मया राम प्रकाशितम्}
{कथं तं वालिनं हन्तुं समरे शक्ष्यसे नृप} %4-11-68

\twolineshloka
{तथा ब्रुवाणं सुग्रीवं प्रहसँल्लक्ष्मणोऽब्रवीत्}
{कस्मिन् कर्मणि निर्वृत्ते श्रद्दध्या वालिनो वधम्} %4-11-69

\twolineshloka
{तमुवाचाथ सुग्रीवः सप्त सालानिमान् पुरा}
{एवमेकैकशो वाली विव्याधाथ स चासकृत्} %4-11-70

\twolineshloka
{रामो निर्दारयेदेषां बाणेनैकेन च द्रुमम्}
{वालिनं निहतं मन्ये दृष्ट्वा रामस्य विक्रमम्} %4-11-71

\twolineshloka
{हतस्य महिषस्यास्थि पादेनैकेन लक्ष्मण}
{उद्यम्य प्रक्षिपेच्चापि तरसा द्वे धनुःशते} %4-11-72

\twolineshloka
{एवमुक्त्वा तु सुग्रीवो रामं रक्तान्तलोचनम्}
{ध्यात्वा मुहूर्तं काकुत्स्थं पुनरेव वचोऽब्रवीत्} %4-11-73

\twolineshloka
{शूरश्च शूरमानी च प्रख्यातबलपौरुषः}
{बलवान् वानरो वाली संयुगेष्वपराजितः} %4-11-74

\twolineshloka
{दृश्यन्ते चास्य कर्माणि दुष्कराणि सुरैरपि}
{यानि सञ्चिन्त्य भीतोऽहमृष्यमूकमुपाश्रितः} %4-11-75

\twolineshloka
{तमजय्यमधृष्यं च वानरेन्द्रममर्षणम्}
{विचिन्तयन्न मुञ्चापि ऋष्यमूकममुं त्वहम्} %4-11-76

\twolineshloka
{उद्विग्नः शङ्कितश्चाहं विचरामि महावने}
{अनुरक्तैः सहामात्यैर्हनुमत्प्रमुखैर्वरैः} %4-11-77

\twolineshloka
{उपलब्धं च मे श्लाघ्यं सन्मित्रं मित्रवत्सल}
{त्वामहं पुरुषव्याघ्र हिमवन्तमिवाश्रितः} %4-11-78

\twolineshloka
{किं तु तस्य बलज्ञोऽहं दुर्भ्रातुर्बलशालिनः}
{अप्रत्यक्षं तु मे वीर्यं समरे तव राघव} %4-11-79

\twolineshloka
{न खल्वहं त्वां तुलये नावमन्ये न भीषये}
{कर्मभिस्तस्य भीमैश्च कातर्यं जनितं मम} %4-11-80

\twolineshloka
{कामं राघव ते वाणी प्रमाणं धैर्यमाकृतिः}
{सूचयन्ति परं तेजो भस्मच्छन्नमिवानलम्} %4-11-81

\twolineshloka
{तस्य तद् वचनं श्रुत्वा सुग्रीवस्य महात्मनः}
{स्मितपूर्वमथो रामः प्रत्युवाच हरिं प्रति} %4-11-82

\twolineshloka
{यदि न प्रत्ययोऽस्मासु विक्रमे तव वानर}
{प्रत्ययं समरे श्लाघ्यमहमुत्पादयामि ते} %4-11-83

\twolineshloka
{एवमुक्त्वा तु सुग्रीवं सान्त्वयँल्लक्ष्मणाग्रजः}
{राघवो दुन्दुभेः कायं पादाङ्गुष्ठेन लीलया} %4-11-84

\twolineshloka
{तोलयित्वा महाबाहुश्चिक्षेप दशयोजनम्}
{असुरस्य तनुं शुष्कां पादाङ्गुष्ठेन वीर्यवान्} %4-11-85

\threelineshloka
{क्षिप्तं दृष्ट्वा ततः कायं सुग्रीवः पुनरब्रवीत्}
{लक्ष्मणस्याग्रतो रामं तपन्तमिव भास्करम्}
{हरीणामग्रतो वीरमिदं वचनमर्थवत्} %4-11-86

\twolineshloka
{आर्द्रः समांसः प्रत्यग्रः क्षिप्तः कायः पुरा सखे}
{परिश्रान्तेन मत्तेन भ्रात्रा मे वालिना तदा} %4-11-87

\twolineshloka
{लघुः सम्प्रति निर्मांसस्तृणभूतश्च राघव}
{क्षिप्त एवं प्रहर्षेण भवता रघुनन्दन} %4-11-88

\twolineshloka
{नात्र शक्यं बलं ज्ञातुं तव वा तस्य वाधिकम्}
{आर्द्रं शुष्कमिति ह्येतत् सुमहद् राघवान्तरम्} %4-11-89

\twolineshloka
{स एव संशयस्तात तव तस्य च यद्बलम्}
{सालमेकं विनिर्भिद्य भवेद् व्यक्तिर्बलाबले} %4-11-90

\twolineshloka
{कृत्वैतत् कार्मुकं सज्यं हस्तिहस्तमिवाततम्}
{आकर्णपूर्णमायम्य विसृजस्व महाशरम्} %4-11-91

\twolineshloka
{इमं हि सालं प्रहितस्त्वया शरो न संशयोऽत्रास्ति विदारयिष्यति}
{अलं विमर्शेन मम प्रियं ध्रुवं कुरुष्व राजन् प्रतिशापितो मया} %4-11-92

\twolineshloka
{यथा हि तेजःसु वरः सदारविर्यथा हि शैलो हिमवान् महाद्रिषु}
{यथा चतुष्पात्सु च केसरी वरस्तथा नराणामसि विक्रमे वरः} %4-11-93


॥इत्यार्षे श्रीमद्रामायणे वाल्मीकीये आदिकाव्ये किष्किन्धाकाण्डे वालिबलाविष्करणम् नाम एकादशः सर्गः ॥४-११॥
