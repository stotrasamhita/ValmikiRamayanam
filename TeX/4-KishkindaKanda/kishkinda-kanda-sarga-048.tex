\sect{अष्टचत्वारिंशः सर्गः — कण्डूवनादिविचयः}

\twolineshloka
{सह ताराङ्गदाभ्यां तु सहसा हनुमान् कपिः}
{सुग्रीवेण यथोद्दिष्टं गन्तुं देशं प्रचक्रमे} %4-48-1

\twolineshloka
{स तु दूरमुपागम्य सर्वैस्तैः कपिसत्तमैः}
{ततो विचित्य विन्ध्यस्य गुहाश्च गहनानि च} %4-48-2

\twolineshloka
{पर्वताग्रनदीदुर्गान् सरांसि विपुलद्रुमान्}
{वृक्षषण्डांश्च विविधान् पर्वतान् वनपादपान्} %4-48-3

\twolineshloka
{अन्वेषमाणास्ते सर्वे वानराः सर्वतो दिशम्}
{न सीतां ददृशुर्वीरा मैथिलीं जनकात्मजाम्} %4-48-4

\twolineshloka
{ते भक्षयन्तो मूलानि फलानि विविधान्यपि}
{अन्वेषमाणा दुर्धर्षा न्यवसंस्तत्र तत्र ह} %4-48-5

\twolineshloka
{स तु देशो दुरन्वेषो गुहागहनवान् महान्}
{निर्जलं निर्जनं शून्यं गहनं घोरदर्शनम्} %4-48-6

\twolineshloka
{तादृशान्यप्यरण्यानि विचित्य भृशपीडिताः}
{स देशश्च दुरन्वेष्यो गुहागहनवान् महान्} %4-48-7

\twolineshloka
{त्यक्त्वा तु तं ततो देशं सर्वे वै हरियूथपाः}
{देशमन्यं दुराधर्षं विविशुश्चाकुतोभयाः} %4-48-8

\twolineshloka
{यत्र वन्ध्यफला वृक्षा विपुष्पाः पर्णवर्जिताः}
{निस्तोयाः सरितो यत्र मूलं यत्र सुदुर्लभम्} %4-48-9

\twolineshloka
{न सन्ति महिषा यत्र न मृगा न च हस्तिनः}
{शार्दूलाः पक्षिणो वापि ये चान्ये वनगोचराः} %4-48-10

\twolineshloka
{न चात्र वृक्षा नौषध्यो न वल्ल्यो नापि वीरुधः}
{स्निग्धपत्राः स्थले यत्र पद्मिन्यः फुल्लपङ्कजाः} %4-48-11

\twolineshloka
{प्रेक्षणीयाः सुगन्धाश्च भ्रमरैश्च विवर्जिताः}
{कण्डुर्नाम महाभागः सत्यवादी तपोधनः} %4-48-12

\twolineshloka
{महर्षिः परमामर्षी नियमैर्दुष्प्रधर्षणः}
{तस्य तस्मिन् वने पुत्रो बालको दशवार्षिकः} %4-48-13

\twolineshloka
{प्रणष्टो जीवितान्ताय क्रुद्धस्तेन महामुनिः}
{तेन धर्मात्मना शप्तं कृत्स्नं तत्र महद्वनम्} %4-48-14

\twolineshloka
{अशरण्यं दुराधर्षं मृगपक्षिविवर्जितम्}
{तस्य ते काननान्तांस्तु गिरीणां कन्दराणि च} %4-48-15

\twolineshloka
{प्रभवाणि नदीनां च विचिन्वन्ति समाहिताः}
{तत्र चापि महात्मानो नापश्यञ्जनकात्मजाम्} %4-48-16

\twolineshloka
{हर्तारं रावणं वापि सुग्रीवप्रियकारिणः}
{ते प्रविश्य तु तं भीमं लतागुल्मसमावृतम्} %4-48-17

\twolineshloka
{ददृशुर्भीमकर्माणमसुरं सुरनिर्भयम्}
{तं दृष्ट्वा वानरा घोरं स्थितं शैलमिवासुरम्} %4-48-18

\twolineshloka
{गाढं परिहिताः सर्वे दृष्ट्वा तं पर्वतोपमम्}
{सोऽपि तान् वानरान् सर्वान् नष्टाः स्थेत्यब्रवीद् बली} %4-48-19

\twolineshloka
{अभ्यधावत सङ्क्रुद्धो मुष्टिमुद्यम्य सङ्गतम्}
{तमापतन्तं सहसा वालिपुत्रोऽङ्गदस्तदा} %4-48-20

\twolineshloka
{रावणोऽयमिति ज्ञात्वा तलेनाभिजघान ह}
{स वालिपुत्राभिहतो वक्त्राच्छोणितमुद्वमन्} %4-48-21

\twolineshloka
{असुरो न्यपतद् भूमौ पर्यस्त इव पर्वतः}
{ते तु तस्मिन् निरुच्छ्वासे वानरा जितकाशिनः} %4-48-22

\twolineshloka
{व्यचिन्वन् प्रायशस्तत्र सर्वं ते गिरिगह्वरम्}
{विचितं तु ततः सर्वं सर्वे ते काननौकसः} %4-48-23

\threelineshloka
{अन्यदेवापरं घोरं विविशुर्गिरिगह्वरम्}
{ते विचित्य पुनः खिन्ना विनिष्पत्य समागताः}
{एकान्ते वृक्षमूले तु निषेदुर्दीनमानसाः} %4-48-24


॥इत्यार्षे श्रीमद्रामायणे वाल्मीकीये आदिकाव्ये किष्किन्धाकाण्डे कण्डूवनादिविचयः नाम अष्टचत्वारिंशः सर्गः ॥४-४८॥
