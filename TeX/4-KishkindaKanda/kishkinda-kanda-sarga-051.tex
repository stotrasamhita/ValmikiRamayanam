\sect{एकपञ्चाशः सर्गः — स्वयम्प्रभातिथ्यम्}

\twolineshloka
{इत्युक्त्वा हनुमांस्तत्र चीरकृष्णाजिनाम्बराम्}
{अब्रवीत् तां महाभागां तापसीं धर्मचारिणीम्} %4-51-1

\twolineshloka
{इदं प्रविष्टाः सहसा बिलं तिमिरसंवृतम्}
{क्षुत्पिपासापरिश्रान्ताः परिखिन्नाश्च सर्वशः} %4-51-2

\twolineshloka
{महद् धरण्या विवरं प्रविष्टाः स्म पिपासिताः}
{इमांस्त्वेवंविधान् भावान् विविधानद्भुतोपमान्} %4-51-3

\twolineshloka
{दृष्ट्वा वयं प्रव्यथिताः सम्भ्रान्ता नष्टचेतसः}
{कस्यैते काञ्चना वृक्षास्तरुणादित्यसंनिभाः} %4-51-4

\twolineshloka
{शुचीन्यभ्यवहाराणि मूलानि च फलानि च}
{काञ्चनानि विमानानि राजतानि गृहाणि च} %4-51-5

\twolineshloka
{तपनीयगवाक्षाणि मणिजालावृतानि च}
{पुष्पिताः फलवन्तश्च पुण्याः सुरभिगन्धयः} %4-51-6

\twolineshloka
{इमे जाम्बूनदमयाः पादपाः कस्य तेजसा}
{काञ्चनानि च पद्मानि जातानि विमले जले} %4-51-7

\twolineshloka
{कथं मत्स्याश्च सौवर्णा दृश्यन्ते सह कच्छपैः}
{आत्मनस्त्वनुभावाद् वा कस्य चैतत्तपोबलम्} %4-51-8

\twolineshloka
{अजानतां नः सर्वेषां सर्वमाख्यातुमर्हसि}
{एवमुक्ता हनुमता तापसी धर्मचारिणी} %4-51-9

\twolineshloka
{प्रत्युवाच हनूमन्तं सर्वभूतहिते रता}
{मयो नाम महातेजा मायावी वानरर्षभ} %4-51-10

\twolineshloka
{तेनेदं निर्मितं सर्वं मायया काञ्चनं वनम्}
{पुरा दानवमुख्यानां विश्वकर्मा बभूव ह} %4-51-11

\twolineshloka
{येनेदं काञ्चनं दिव्यं निर्मितं भवनोत्तमम्}
{स तु वर्षसहस्राणि तपस्तप्त्वा महद्वने} %4-51-12

\twolineshloka
{पितामहाद् वरं लेभे सर्वमौशनसं धनम्}
{विधाय सर्वं बलवान् सर्वकामेश्वरस्तदा} %4-51-13

\twolineshloka
{उवास सुखितः कालं कंचिदस्मिन् महावने}
{तमप्सरसि हेमायां सक्तं दानवपुङ्गवम्} %4-51-14

\twolineshloka
{विक्रम्यैवाशनिं गृह्य जघानेशः पुरंदरः}
{इदं च ब्रह्मणा दत्तं हेमायै वनमुत्तमम्} %4-51-15

\twolineshloka
{शाश्वतः कामभोगश्च गृहं चेदं हिरण्मयम्}
{दुहिता मेरुसावर्णेरहं तस्याः स्वयंप्रभा} %4-51-16

\twolineshloka
{इदं रक्षामि भवनं हेमाया वानरोत्तम}
{मम प्रियसखी हेमा नृत्तगीतविशारदा} %4-51-17

\twolineshloka
{तयादत्तवरा चास्मि रक्षामि भवनं महत्}
{किं कार्यं कस्य वा हेतोः कान्ताराणि प्रपद्यथ} %4-51-18

\threelineshloka
{कथं चेदं वनं दुर्गं युष्माभिरुपलक्षितम्}
{शुचीन्यभ्यवहाराणि मूलानि च फलानि च}
{भुक्त्वा पीत्वा च पानीयं सर्वं मे वक्तुमर्हसि} %4-51-19


॥इत्यार्षे श्रीमद्रामायणे वाल्मीकीये आदिकाव्ये किष्किन्धाकाण्डे स्वयम्प्रभातिथ्यम् नाम एकपञ्चाशः सर्गः ॥४-५१॥
