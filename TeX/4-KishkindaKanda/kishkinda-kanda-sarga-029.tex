\sect{एकोनत्रिंशः सर्गः — हनुमत्प्रतिबोधनम्}

\twolineshloka
{समीक्ष्य विमलं व्योम गतविद्युद्बलाहकम्}
{सारसाकुलसङ्घुष्टं रम्यज्योत्स्नानुलेपनम्} %4-29-1

\twolineshloka
{समृद्धार्थं च सुग्रीवं मन्दधर्मार्थसङ्ग्रहम्}
{अत्यर्थं चासतां मार्गमेकान्तगतमानसम्} %4-29-2

\twolineshloka
{निवृत्तकार्यं सिद्धार्थं प्रमदाभिरतं सदा}
{प्राप्तवन्तमभिप्रेतान् सर्वानेव मनोरथान्} %4-29-3

\twolineshloka
{स्वां च पत्नीमभिप्रेतां तारां चापि समीप्सिताम्}
{विहरन्तमहोरात्रं कृतार्थं विगतज्वरम्} %4-29-4

\twolineshloka
{क्रीडन्तमिव देवेशं गन्धर्वाप्सरसां गणैः}
{मन्त्रिषु न्यस्तकार्यं च मन्त्रिणामनवेक्षकम्} %4-29-5

\twolineshloka
{उच्छिन्नराज्यसन्देहं कामवृत्तमिव स्थितम्}
{निश्चितार्थोऽर्थतत्त्वज्ञः कालधर्मविशेषवित्} %4-29-6

\twolineshloka
{प्रसाद्य वाक्यैर्विविधैर्हेतुमद्भिर्मनोरमैः}
{वाक्यविद् वाक्यतत्त्वज्ञं हरीशं मारुतात्मजः} %4-29-7

\twolineshloka
{हितं तथ्यं च पथ्यं च सामधर्मार्थनीतिमत्}
{प्रणयप्रीतिसंयुक्तं विश्वासकृतनिश्चयम्} %4-29-8

\twolineshloka
{हरीश्वरमुपागम्य हनूमान् वाक्यमब्रवीत्}
{राज्यं प्राप्तं यशश्चैव कौली श्रीरभिवर्धिता} %4-29-9

\twolineshloka
{मित्राणां सङ्ग्रहः शेषस्तद् भवान् कर्तुमर्हति}
{यो हि मित्रेषु कालज्ञः सततं साधु वर्तते} %4-29-10

\threelineshloka
{तस्य राज्यं च कीर्तिश्च प्रतापश्चापि वर्धते}
{यस्य कोशश्च दण्डश्च मित्राण्यात्मा च भूमिप}
{समान्येतानि सर्वाणि स राज्यं महदश्नुते} %4-29-11

\twolineshloka
{तद् भवान् वृत्तसम्पन्नः स्थितः पथि निरत्यये}
{मित्रार्थमभिनीतार्थं यथावत् कर्तुमर्हति} %4-29-12

\twolineshloka
{सन्त्यज्य सर्वकर्माणि मित्रार्थे यो न वर्तते}
{सम्भ्रमाद् विकृतोत्साहः सोऽनर्थैर्नावरुध्यते} %4-29-13

\twolineshloka
{यो हि कालव्यतीतेषु मित्रकार्येषु वर्तते}
{स कृत्वा महतोऽप्यर्थान्न मित्रार्थेन युज्यते} %4-29-14

\twolineshloka
{तदिदं मित्रकार्यं नः कालातीतमरिन्दम}
{क्रियतां राघवस्यैतद् वैदेह्याः परिमार्गणम्} %4-29-15

\twolineshloka
{न च कालमतीतं ते निवेदयति कालवित्}
{त्वरमाणोऽपि स प्राज्ञस्तव राजन् वशानुगः} %4-29-16

\twolineshloka
{कुलस्य हेतुः स्फीतस्य दीर्घबन्धुश्च राघवः}
{अप्रमेयप्रभावश्च स्वयं चाप्रतिमो गुणैः} %4-29-17

\twolineshloka
{तस्य त्वं कुरु वै कार्यं पूर्वं तेन कृतं तव}
{हरीश्वर कपिश्रेष्ठानाज्ञापयितुमर्हसि} %4-29-18

\twolineshloka
{नहि तावद् भवेत् कालो व्यतीतश्चोदनादृते}
{चोदितस्य हि कार्यस्य भवेत् कालव्यतिक्रमः} %4-29-19

\twolineshloka
{अकर्तुरपि कार्यस्य भवान् कर्ता हरीश्वर}
{किं पुनः प्रतिकर्तुस्ते राज्येन च वधेन च} %4-29-20

\twolineshloka
{शक्तिमानतिविक्रान्तो वानरर्क्षगणेश्वर}
{कर्तुं दाशरथेः प्रीतिमाज्ञायां किं नु सज्जसे} %4-29-21

\twolineshloka
{कामं खलु शरैः शक्तः सुरासुरमहोरगान्}
{वशे दाशरथिः कर्तुं त्वत्प्रतिज्ञामवेक्षते} %4-29-22

\twolineshloka
{प्राणत्यागाविशङ्केन कृतं तेन महत् प्रियम्}
{तस्य मार्गाम वैदेहीं पृथिव्यामपि चाम्बरे} %4-29-23

\twolineshloka
{देवदानवगन्धर्वा असुराः समरुद्गणाः}
{न च यक्षा भयं तस्य कुर्युः किमिव राक्षसाः} %4-29-24

\twolineshloka
{तदेवं शक्तियुक्तस्य पूर्वं प्रतिकृतस्तथा}
{रामस्यार्हसि पिङ्गेश कर्तुं सर्वात्मना प्रियम्} %4-29-25

\twolineshloka
{नाधस्तादवनौ नाप्सु गतिर्नोपरि चाम्बरे}
{कस्यचित् सज्जतेऽस्माकं कपीश्वर तवाज्ञया} %4-29-26

\twolineshloka
{तदाज्ञापय कः किं ते कुतो वापि व्यवस्यतु}
{हरयो ह्यप्रधृष्यास्ते सन्ति कोट्यग्रतोऽनघ} %4-29-27

\twolineshloka
{तस्य तद् वचनं श्रुत्वा काले साधु निरूपितम्}
{सुग्रीवः सत्त्वसम्पन्नश्चकार मतिमुत्तमाम्} %4-29-28

\twolineshloka
{सन्दिदेशातिमतिमान् नीलं नित्यकृतोद्यमम्}
{दिक्षु सर्वासु सर्वेषां सैन्यानामुपसङ्ग्रहे} %4-29-29

\twolineshloka
{यथा सेना समग्रा मे यूथपालाश्च सर्वशः}
{समागच्छन्त्यसङ्गेन सेनाग्र्ये ण तथा कुरु} %4-29-30

\threelineshloka
{ये त्वन्तपालाः प्लवगाः शीघ्रगा व्यवसायिनः}
{समानयन्तु ते शीघ्रं त्वरिताः शासनान्मम}
{स्वयं चानन्तरं कार्यं भवानेवानुपश्यतु} %4-29-31

\twolineshloka
{त्रिपञ्चरात्रादूर्ध्वं यः प्राप्नुयादिह वानरः}
{तस्य प्राणान्तिको दण्डो नात्र कार्या विचारणा} %4-29-32

\twolineshloka
{हरींश्च वृद्धानुपयातु साङ्गदो भवान् ममाज्ञामधिकृत्य निश्चितम्}
{इति व्यवस्थां हरिपुङ्गवेश्वरो विधाय वेश्म प्रविवेश वीर्यवान्} %4-29-33


॥इत्यार्षे श्रीमद्रामायणे वाल्मीकीये आदिकाव्ये किष्किन्धाकाण्डे हनुमत्प्रतिबोधनम् नाम एकोनत्रिंशः सर्गः ॥४-२९॥
