\sect{पञ्चचत्वारिंशः सर्गः — वानरबलप्रतिष्ठा}

\twolineshloka
{सर्वांश्चाहूय सुग्रीवः प्लवगान् प्लवगर्षभः}
{समस्तांश्चाब्रवीद् राजा रामकार्यार्थसिद्धये} %4-45-1

\twolineshloka
{एवमेतद् विचेतव्यं भवद्भिर्वानरोत्तमैः}
{तदुग्रशासनं भर्तुर्विज्ञाय हरिपुंगवाः} %4-45-2

\twolineshloka
{शलभा इव संछाद्य मेदिनीं सम्प्रतस्थिरे}
{रामः प्रस्रवणे तस्मिन् न्यवसत् सहलक्ष्मणः} %4-45-3

\twolineshloka
{प्रतीक्षमाणस्तं मासं सीताधिगमने कृतः}
{उत्तरां तु दिशं रम्यां गिरिराजसमावृताम्} %4-45-4

\twolineshloka
{प्रतस्थे सहसा वीरो हरिः शतबलिस्तदा}
{पूर्वां दिशं प्रतिययौ विनतो हरियूथपः} %4-45-5

\twolineshloka
{ताराङ्गदादिसहितः प्लवगः पवनात्मजः}
{अगस्त्याचरितामाशां दक्षिणां हरियूथपः} %4-45-6

\twolineshloka
{पश्चिमां च दिशं घोरां सुषेणः प्लवगेश्वरः}
{प्रतस्थे हरिशार्दूलो दिशं वरुणपालिताम्} %4-45-7

\twolineshloka
{ततः सर्वा दिशो राजा चोदयित्वा यथातथम्}
{कपिसेनापतिर्वीरो मुमोद सुखितः सुखम्} %4-45-8

\twolineshloka
{एवं संचोदिताः सर्वे राज्ञा वानरयूथपाः}
{स्वां स्वां दिशमभिप्रेत्य त्वरिताः सम्प्रतस्थिरे} %4-45-9

\twolineshloka
{नदन्तश्चोन्नदन्तश्च गर्जन्तश्च प्लवंगमाः}
{क्ष्वेडन्तो धावमानाश्च विनदन्तो महाबलाः} %4-45-10

\twolineshloka
{एवं संचोदिताः सर्वे राज्ञा वानरयूथपाः}
{आनयिष्यामहे सीतां हनिष्यामश्च रावणम्} %4-45-11

\twolineshloka
{अहमेको वधिष्यामि प्राप्तं रावणमाहवे}
{ततश्चोन्मथ्य सहसा हरिष्ये जनकात्मजाम्} %4-45-12

\twolineshloka
{वेपमानां श्रमेणाद्य भवद्भिः स्थीयतामिति}
{एक एवाहरिष्यामि पातालादपि जानकीम्} %4-45-13

\twolineshloka
{विधमिष्याम्यहं वृक्षान् दारयिष्याम्यहं गिरीन्}
{धरणीं दारयिष्यामि क्षोभयिष्यामि सागरान्} %4-45-14

\twolineshloka
{अहं योजनसंख्यायाः प्लवेयं नात्र संशयः}
{शतयोजनसंख्यायाः शतं समधिकं ह्यहम्} %4-45-15

\twolineshloka
{भूतले सागरे वापि शैलेषु च वनेषु च}
{पातालस्यापि वा मध्ये न ममाच्छिद्यते गतिः} %4-45-16

\twolineshloka
{इत्येकैकस्तदा तत्र वानरा बलदर्पिताः}
{ऊचुश्च वचनं तस्य हरिराजस्य संनिधौ} %4-45-17


॥इत्यार्षे श्रीमद्रामायणे वाल्मीकीये आदिकाव्ये किष्किन्धाकाण्डे वानरबलप्रतिष्ठा नाम पञ्चचत्वारिंशः सर्गः ॥४-४५॥
