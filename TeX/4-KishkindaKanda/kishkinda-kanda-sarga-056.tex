\sect{षट्पञ्चाशः सर्गः — संपातिप्रश्नः}

\twolineshloka
{उपविष्टास्तु ते सर्वे यस्मिन् प्रायं गिरिस्थले}
{हरयो गृध्रराजश्च तं देशमुपचक्रमे} %4-56-1

\twolineshloka
{सम्पातिर्नाम नाम्ना तु चिरजीवी विहंगमः}
{भ्राता जटायुषः श्रीमान् विख्यातबलपौरुषः} %4-56-2

\twolineshloka
{कन्दरादभिनिष्क्रम्य स विन्ध्यस्य महागिरेः}
{उपविष्टान् हरीन् दृष्ट्वा हृष्टात्मा गिरमब्रवीत्} %4-56-3

\twolineshloka
{विधिः किल नरं लोके विधानेनानुवर्तते}
{यथायं विहितो भक्ष्यश्चिरान्मह्यमुपागतः} %4-56-4

\twolineshloka
{परम्पराणां भक्षिष्ये वानराणां मृतं मृतम्}
{उवाचैतद् वचः पक्षी तान् निरीक्ष्य प्लवंगमान्} %4-56-5

\twolineshloka
{तस्य तद् वचनं श्रुत्वा भक्ष्यलुब्धस्य पक्षिणः}
{अङ्गदः परमायस्तो हनूमन्तमथाब्रवीत्} %4-56-6

\twolineshloka
{पश्य सीतापदेशेन साक्षाद् वैवस्वतो यमः}
{इमं देशमनुप्राप्तो वानराणां विपत्तये} %4-56-7

\twolineshloka
{रामस्य न कृतं कार्यं न कृतं राजशासनम्}
{हरीणामियमज्ञाता विपत्तिः सहसाऽऽगता} %4-56-8

\twolineshloka
{वैदेह्याः प्रियकामेन कृतं कर्म जटायुषा}
{गृध्रराजेन यत् तत्र श्रुतं वस्तदशेषतः} %4-56-9

\twolineshloka
{तथा सर्वाणि भूतानि तिर्यग्योनिगतान्यपि}
{प्रियं कुर्वन्ति रामस्य त्यक्त्वा प्राणान् यथा वयम्} %4-56-10

\twolineshloka
{अन्योन्यमुपकुर्वन्ति स्नेहकारुण्ययन्त्रिताः}
{ततस्तस्योपकारार्थं त्यजतात्मानमात्मना} %4-56-11

\twolineshloka
{प्रियं कृतं हि रामस्य धर्मज्ञेन जटायुषा}
{राघवार्थे परिश्रान्ता वयं संत्यक्तजीविताः} %4-56-12

\threelineshloka
{कान्ताराणि प्रपन्नाः स्म न च पश्याम मैथिलीम्}
{स सुखी गृध्रराजस्तु रावणेन हतो रणे}
{मुक्तश्च सुग्रीवभयाद् गतश्च परमां गतिम्} %4-56-13

\twolineshloka
{जटायुषो विनाशेन राज्ञो दशरथस्य च}
{हरणेन च वैदेह्याः संशयं हरयो गताः} %4-56-14

\twolineshloka
{रामलक्ष्मणयोर्वासमरण्ये सह सीतया}
{राघवस्य च बाणेन वालिनश्च तथा वधः} %4-56-15

\twolineshloka
{रामकोपादशेषाणां रक्षसां च तथा वधम्}
{कैकेय्या वरदानेन इदं च विकृतं कृतम्} %4-56-16

\twolineshloka
{तदसुखमनुकीर्तितं वचो भुवि पतितांश्च निरीक्ष्य वानरान्}
{भृशचकितमतिर्महामतिः कृपणमुदाहृतवान् स गृध्रराजः} %4-56-17

\twolineshloka
{तत् तु श्रुत्वा तथा वाक्यमङ्गदस्य मुखोद्गतम्}
{अब्रवीद् वचनं गृध्रस्तीक्ष्णतुण्डो महास्वनः} %4-56-18

\twolineshloka
{कोऽयं गिरा घोषयति प्राणैः प्रियतरस्य मे}
{जटायुषो वधं भ्रातुः कम्पयन्निव मे मनः} %4-56-19

\twolineshloka
{कथमासीज्जनस्थाने युद्धं राक्षसगृध्रयोः}
{नामधेयमिदं भ्रातुश्चिरस्याद्य मया श्रुतम्} %4-56-20

\twolineshloka
{इच्छेयं गिरिदुर्गाच्च भवद्भिरवतारितुम्}
{यवीयसो गुणज्ञस्य श्लाघनीयस्य विक्रमैः} %4-56-21

\twolineshloka
{अतिदीर्घस्य कालस्य परितुष्टोऽस्मि कीर्तनात्}
{तदिच्छेयमहं श्रोतुं विनाशं वानरर्षभाः} %4-56-22

\twolineshloka
{भ्रातुर्जटायुषस्तस्य जनस्थाननिवासिनः}
{तस्यैव च मम भ्रातुः सखा दशरथः कथम्} %4-56-23

\threelineshloka
{यस्य रामः प्रियः पुत्रो ज्येष्ठो गुरुजनप्रियः}
{सूर्यांशुदग्धपक्षत्वान्न शक्नोमि विसर्पितुम्}
{इच्छेयं पर्वतादस्मादवतर्तुमरिंदमाः} %4-56-24


॥इत्यार्षे श्रीमद्रामायणे वाल्मीकीये आदिकाव्ये किष्किन्धाकाण्डे संपातिप्रश्नः नाम षट्पञ्चाशः सर्गः ॥४-५६॥
