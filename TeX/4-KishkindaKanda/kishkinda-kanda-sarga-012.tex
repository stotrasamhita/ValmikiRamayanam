\sect{द्वादशः सर्गः — सुग्रीवप्रत्ययदानम्}

\twolineshloka
{एतच्च वचनं श्रुत्वा सुग्रीवस्य सुभाषितम्}
{प्रत्ययार्थं महातेजा रामो जग्राह कार्मुकम्} %4-12-1

\twolineshloka
{स गृहीत्वा धनुर्घोरं शरमेकं च मानदः}
{सालमुद्दिश्य चिक्षेप पूरयन् स रवैर्दिशः} %4-12-2

\twolineshloka
{स विसृष्टो बलवता बाणः स्वर्णपरिष्कृतः}
{भित्त्वा सालान् गिरिप्रस्थं सप्तं भूमिं विवेश ह} %4-12-3

\twolineshloka
{सायकस्तु मुहूर्तेन सालान् भित्त्वा महाजवः}
{निष्पत्य च पुनस्तूणं तमेव प्रविवेश ह} %4-12-4

\twolineshloka
{तान् दृष्ट्वा सप्त निर्भिन्नान् सालान् वानरपुङ्गवः}
{रामस्य शरवेगेन विस्मयं परमं गतः} %4-12-5

\twolineshloka
{स मूर्ध्ना न्यपतद् भूमौ प्रलम्बीकृतभूषणः}
{सुग्रीवः परमप्रीतो राघवाय कृताञ्जलिः} %4-12-6

\twolineshloka
{इदं चोवाच धर्मज्ञं कर्मणा तेन हर्षितः}
{रामं सर्वास्त्रविदुषां श्रेष्ठं शूरमवस्थितम्} %4-12-7

\twolineshloka
{सेन्द्रानपि सुरान् सर्वांस्त्वं बाणैः पुरुषर्षभ}
{समर्थः समरे हन्तुं किं पुनर्वालिनं प्रभो} %4-12-8

\twolineshloka
{येन सप्त महासाला गिरिर्भूमिश्च दारिताः}
{बाणेनैकेन काकुत्स्थ स्थाता ते को रणाग्रतः} %4-12-9

\twolineshloka
{अद्य मे विगतः शोकः प्रीतिरद्य परा मम}
{सुहृदं त्वां समासाद्य महेन्द्रवरुणोपमम्} %4-12-10

\twolineshloka
{तमद्यैव प्रियार्थं मे वैरिणं भ्रातृरूपिणम्}
{वालिनं जहि काकुत्स्थ मया बद्धोऽयमञ्जलिः} %4-12-11

\twolineshloka
{ततो रामः परिष्वज्य सुग्रीवं प्रियदर्शनम्}
{प्रत्युवाच महाप्राज्ञो लक्ष्मणानुगतं वचः} %4-12-12

\twolineshloka
{अस्माद्गच्छाम किष्किन्धां क्षिप्रं गच्छ त्वमग्रतः}
{गत्वा चाह्वय सुग्रीव वालिनं भ्रातृगन्धिनम्} %4-12-13

\twolineshloka
{सर्वे ते त्वरितं गत्वा किष्किन्धां वालिनः पुरीम्}
{वृक्षैरात्मानमावृत्य ह्यतिष्ठन् गहने वने} %4-12-14

\twolineshloka
{सुग्रीवोऽप्यनदद् घोरं वालिनो ह्वानकारणात्}
{गाढं परिहितो वेगान्नादैर्भिन्दन्निवाम्बरम्} %4-12-15

\twolineshloka
{तं श्रुत्वा निनदं भ्रातुः क्रुद्धो वाली महाबलः}
{निष्पपात सुसंरब्धो भास्करोऽस्ततटादिव} %4-12-16

\twolineshloka
{ततः सुतुमुलं युद्धं वालिसुग्रीवयोरभूत्}
{गगने ग्रहयोर्घोरं बुधाङ्गारकयोरिव} %4-12-17

\twolineshloka
{तलैरशनिकल्पैश्च वज्रकल्पैश्च मुष्टिभिः}
{जघ्नतुः समरेऽन्योन्यं भ्रातरौ क्रोधमूर्च्छितौ} %4-12-18

\twolineshloka
{ततो रामो धनुष्पाणिस्तावुभौ समुदैक्षत}
{अन्योन्यसदृशौ वीरावुभौ देवाविवाश्विनौ} %4-12-19

\twolineshloka
{यन्नावगच्छत् सुग्रीवं वालिनं वापि राघवः}
{ततो न कृतवान् बुद्धिं मोक्तुमन्तकरं शरम्} %4-12-20

\twolineshloka
{एतस्मिन्नन्तरे भग्नः सुग्रीवस्तेन वालिना}
{अपश्यन् राघवं नाथमृष्यमूकं प्रदुद्रुवे} %4-12-21

\twolineshloka
{क्लान्तो रुधिरसिक्ताङ्गः प्रहारैर्जर्जरीकृतः}
{वालिनाभिद्रुतः क्रोधात् प्रविवेश महावनम्} %4-12-22

\twolineshloka
{तं प्रविष्टं वनं दृष्ट्वा वाली शापभयात् ततः}
{मुक्तो ह्यसि त्वमित्युक्त्वा स निवृत्तो महाबलः} %4-12-23

\twolineshloka
{राघवोऽपि सह भ्रात्रा सह चैव हनूमता}
{तदेव वनमागच्छत् सुग्रीवो यत्र वानरः} %4-12-24

\twolineshloka
{तं समीक्ष्यागतं रामं सुग्रीवः सहलक्ष्मणम्}
{ह्रीमान् दीनमुवाचेदं वसुधामवलोकयन्} %4-12-25

\twolineshloka
{आह्वयस्वेति मामुक्त्वा दर्शयित्वा च विक्रमम्}
{वैरिणा घातयित्वा च किमिदानीं त्वया कृतम्} %4-12-26

\twolineshloka
{तामेव वेलां वक्तव्यं त्वया राघव तत्त्वतः}
{वालिनं न निहन्मीति ततो नाहमितो व्रजे} %4-12-27

\twolineshloka
{तस्य चैवं ब्रुवाणस्य सुग्रीवस्य महात्मनः}
{करुणं दीनया वाचा राघवः पुनरब्रवीत्} %4-12-28

\twolineshloka
{सुग्रीव श्रूयतां तात क्रोधश्च व्यपनीयताम्}
{कारणं येन बाणोऽयं स मया न विसर्जितः} %4-12-29

\twolineshloka
{अलङ्कारेण वेषेण प्रमाणेन गतेन च}
{त्वं च सुग्रीव वाली च सदृशौ स्थः परस्परम्} %4-12-30

\twolineshloka
{स्वरेण वर्चसा चैव प्रेक्षितेन च वानर}
{विक्रमेण च वाक्यैश्च व्यक्तिं वां नोपलक्षये} %4-12-31

\twolineshloka
{ततोऽहं रूपसादृश्यान्मोहितो वानरोत्तम}
{नोत्सृजामि महावेगं शरं शत्रुनिबर्हणम्} %4-12-32

\twolineshloka
{जीवितान्तकरं घोरं सादृश्यात् तु विशङ्कितः}
{मूलघातो न नौ स्याद्धि द्वयोरिति कृतो मया} %4-12-33

\twolineshloka
{त्वयि वीर विपन्ने हि अज्ञानाल्लाघवान्मया}
{मौढ्यं च मम बाल्यं च ख्यापितं स्यात् कपीश्वर} %4-12-34

\twolineshloka
{दत्ताभयवधो नाम पातकं महदद्भुतम्}
{अहं च लक्ष्मणश्चैव सीता च वरवर्णिनी} %4-12-35

\twolineshloka
{त्वदधीना वयं सर्वे वनेऽस्मिन् शरणं भवान्}
{तस्माद् युध्यस्व भूयस्त्वं मा माशङ्कीश्च वानर} %4-12-36

\twolineshloka
{एतन्मुहूर्ते तु मया पश्य वालिनमाहवे}
{निरस्तमिषुणैकेन चेष्टमानं महीतले} %4-12-37

\twolineshloka
{अभिज्ञानं कुरुष्व त्वमात्मनो वानरेश्वर}
{येन त्वामभिजानीयां द्वन्द्वयुद्धमुपागतम्} %4-12-38

\twolineshloka
{गजपुष्पीमिमां फुल्लामुत्पाट्य शुभलक्षणाम्}
{कुरु लक्ष्मण कण्ठेऽस्य सुग्रीवस्य महात्मनः} %4-12-39

\twolineshloka
{ततो गिरितटे जातामुत्पाट्य कुसुमायुताम्}
{लक्ष्मणो गजपुष्पीं तां तस्य कण्ठे व्यसर्जयत्} %4-12-40

\twolineshloka
{स तया शुशुभे श्रीमाँल्लतया कण्ठसक्तया}
{मालयेव बलाकानां ससन्ध्य इव तोयदः} %4-12-41

\twolineshloka
{विभ्राजमानो वपुषा रामवाक्यसमाहितः}
{जगाम सह रामेण किष्किन्धां पुनराप सः} %4-12-42


॥इत्यार्षे श्रीमद्रामायणे वाल्मीकीये आदिकाव्ये किष्किन्धाकाण्डे सुग्रीवप्रत्ययदानम् नाम द्वादशः सर्गः ॥४-१२॥
