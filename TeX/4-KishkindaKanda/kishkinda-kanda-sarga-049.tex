\sect{एकोनपञ्चाशः सर्गः — रजतपर्वतविचयः}

\twolineshloka
{अथाङ्गदस्तदा सर्वान् वानरानिदमब्रवीत्}
{परिश्रान्तो महाप्राज्ञः समाश्वास्य शनैर्वचः} %4-49-1

\twolineshloka
{वनानि गिरयो नद्यो दुर्गाणि गहनानि च}
{दरी गिरिगुहाश्चैव विचिताः सर्वमन्ततः} %4-49-2

\twolineshloka
{तत्र तत्र सहास्माभिर्जानकी न च दृश्यते}
{तथा रक्षोऽपहर्ता च सीतायाश्चैव दुष्कृती} %4-49-3

\twolineshloka
{कालश्च नो महान् यातः सुग्रीवश्चोग्रशासनः}
{तस्माद् भवन्तः सहिता विचिन्वन्तु समन्ततः} %4-49-4

\twolineshloka
{विहाय तन्द्रीं शोकं च निद्रां चैव समुत्थिताम्}
{विचिनुध्वं तथा सीतां पश्यामो जनकात्मजाम्} %4-49-5

\twolineshloka
{अनिर्वेदं च दाक्ष्यं च मनसश्चापराजयम्}
{कार्यसिद्धिकराण्याहुस्तस्मादेतद् ब्रवीम्यहम्} %4-49-6

\twolineshloka
{अद्यापीदं वनं दुर्गं विचिन्वन्तु वनौकसः}
{खेदं त्यक्त्वा पुनः सर्वं वनमेव विचिन्वताम्} %4-49-7

\twolineshloka
{अवश्यं कुर्वतां तस्य दृश्यते कर्मणः फलम्}
{परं निर्वेदमागम्य नहि नोन्मीलनं क्षमम्} %4-49-8

\twolineshloka
{सुग्रीवः क्रोधनो राजा तीक्ष्णदण्डश्च वानराः}
{भेतव्यं तस्य सततं रामस्य च महात्मनः} %4-49-9

\twolineshloka
{हितार्थमेतदुक्तं वः क्रियतां यदि रोचते}
{उच्यतां हि क्षमं यत् तत् सर्वेषामेव वानराः} %4-49-10

\twolineshloka
{अङ्गदस्य वचः श्रुत्वा वचनं गन्धमादनः}
{उवाच व्यक्तया वाचा पिपासाश्रमखिन्नया} %4-49-11

\twolineshloka
{सदृशं खलु वो वाक्यमङ्गदो यदुवाच ह}
{हितं चैवानुकूलं च क्रियतामस्य भाषितम्} %4-49-12

\twolineshloka
{पुनर्मार्गामहे शैलान् कन्दरांश्च शिलांस्तथा}
{काननानि च शून्यानि गिरिप्रस्रवणानि च} %4-49-13

\twolineshloka
{यथोद्दिष्टानि सर्वाणि सुग्रीवेण महात्मना}
{विचिन्वन्तु वनं सर्वे गिरिदुर्गाणि संगताः} %4-49-14

\twolineshloka
{ततः समुत्थाय पुनर्वानरास्ते महाबलाः}
{विन्ध्यकाननसंकीर्णां विचेरुर्दक्षिणां दिशम्} %4-49-15

\twolineshloka
{ते शारदाभ्रप्रतिमं श्रीमद्रजतपर्वतम्}
{शृङ्गवन्तं दरीवन्तमधिरुह्य च वानराः} %4-49-16

\twolineshloka
{तत्र लोध्रवनं रम्यं सप्तपर्णवनानि च}
{विचिन्वन्तो हरिवराः सीतादर्शनकांक्षिणः} %4-49-17

\twolineshloka
{तस्याग्रमधिरूढास्ते श्रान्ता विपुलविक्रमाः}
{न पश्यन्ति स्म वैदेहीं रामस्य महिषीं प्रियाम्} %4-49-18

\twolineshloka
{ते तु दृष्टिगतं दृष्ट्वा तं शैलं बहुकन्दरम्}
{अध्यारोहन्त हरयो वीक्षमाणाः समन्ततः} %4-49-19

\twolineshloka
{अवरुह्य ततो भूमिं श्रान्ता विगतचेतसः}
{स्थिता मुहूर्तं तत्राथ वृक्षमूलमुपाश्रिताः} %4-49-20

\twolineshloka
{ते मुहूर्तं समाश्वस्ताः किंचिद्भग्नपरिश्रमाः}
{पुनरेवोद्यताः कृत्स्नां मार्गितुं दक्षिणां दिशम्} %4-49-21

\twolineshloka
{हनुमत्प्रमुखास्तावत् प्रस्थिताः प्लवगर्षभाः}
{विन्ध्यमेवादितः कृत्वा विचेरुश्च समन्ततः} %4-49-22


॥इत्यार्षे श्रीमद्रामायणे वाल्मीकीये आदिकाव्ये किष्किन्धाकाण्डे रजतपर्वतविचयः नाम एकोनपञ्चाशः सर्गः ॥४-४९॥
