\sect{सप्तविंशः सर्गः — माल्यवन्निवासः}

\twolineshloka
{अभिषिक्ते तु सुग्रीवे प्रविष्टे वानरे गुहाम्}
{आजगाम सह भ्रात्रा रामः प्रस्रवणं गिरिम्} %4-27-1

\twolineshloka
{शार्दूलमृगसङ्घुष्टं सिंहैर्भीमरवैर्वृतम्}
{नानागुल्मलतागूढं बहुपादपसङ्कुलम्} %4-27-2

\twolineshloka
{ऋक्षवानरगोपुच्छैर्मार्जारैश्च निषेवितम्}
{मेघराशिनिभं शैलं नित्यं शुचिकरं शिवम्} %4-27-3

\twolineshloka
{तस्य शैलस्य शिखरे महतीमायतां गुहाम्}
{प्रत्यगृह्णीत वासार्थं रामः सौमित्रिणा सह} %4-27-4

\twolineshloka
{कृत्वा च समयं रामः सुग्रीवेण सहानघः}
{कालयुक्तं महद्वाक्यमुवाच रघुनन्दनः} %4-27-5

\twolineshloka
{विनीतं भ्रातरं भ्राता लक्ष्मणं लक्ष्मिवर्धनम्}
{इयं गिरिगुहा रम्या विशाला युक्तमारुता} %4-27-6

\twolineshloka
{अस्यां वत्स्याम सौमित्रे वर्षरात्रमरिन्दम}
{गिरिशृङ्गमिदं रम्यमुत्तमं पार्थिवात्मज} %4-27-7

\twolineshloka
{श्वेताभिः कृष्णताम्राभिः शिलाभिरुपशोभितम्}
{नानाधातुसमाकीर्णं नदीदर्दुरसंयुतम्} %4-27-8

\twolineshloka
{विविधैर्वृक्षषण्डैश्च चारुचित्रलतायुतम्}
{नानाविहगसङ्घुष्टं मयूरवरनादितम्} %4-27-9

\twolineshloka
{मालतीकुन्दगुल्मैश्च सिन्दुवारैः शिरीषकैः}
{कदम्बार्जुनसर्जैश्च पुष्पितैरुपशोभितम्} %4-27-10

\twolineshloka
{इयं च नलिनी रम्या फुल्लपङ्कजमण्डिता}
{नातिदूरे गुहाया नौ भविष्यति नृपात्मज} %4-27-11

\twolineshloka
{प्रागुदक्प्रवणे देशे गुहा साधु भविष्यति}
{पश्चाच्चैवोन्नता सौम्य निवातेयं भविष्यति} %4-27-12

\twolineshloka
{गुहाद्वारे च सौमित्रे शिला समतला शिवा}
{कृष्णा चैवायता चैव भिन्नाञ्जनचयोपमा} %4-27-13

\twolineshloka
{गिरिशृङ्गमिदं तात पश्य चोत्तरतः शुभम्}
{भिन्नाञ्जनचयाकारमम्भोधरमिवोदितम्} %4-27-14

\twolineshloka
{दक्षिणस्यामपि दिशि स्थितं श्वेतमिवाम्बरम्}
{कैलासशिखरप्रख्यं नानाधातुविराजितम्} %4-27-15

\twolineshloka
{प्राचीनवाहिनीं चैव नदीं भृशमकर्दमाम्}
{गुहायाः परतः पश्य त्रिकूटे जाह्नवीमिव} %4-27-16

\twolineshloka
{चन्दनैस्तिलकैः सालैस्तमालैरतिमुक्तकैः}
{पद्मकैः सरलैश्चैव अशोकैश्चैव शोभिताम्} %4-27-17

\twolineshloka
{वानीरैस्तिमिदैश्चैव बकुलैः केतकैरपि}
{हिन्तालैस्तिनिशैर्नीपैर्वेतसैः कृतमालकैः} %4-27-18

\twolineshloka
{तीरजैः शोभिता भाति नानारूपैस्ततस्ततः}
{वसनाभरणोपेता प्रमदेवाभ्यलङ्कृता} %4-27-19

\twolineshloka
{शतशः पक्षिसङ्घैश्च नानानादविनादिता}
{एकैकमनुरक्तैश्च चक्रवाकैरलङ्कृता} %4-27-20

\twolineshloka
{पुलिनैरतिरम्यैश्च हंससारससेविता}
{प्रहसन्त्येव भात्येषा नानारत्नसमन्विता} %4-27-21

\twolineshloka
{क्वचिन्नीलोत्पलैश्छन्ना भातिरक्तोत्पलैः क्वचित्}
{क्वचिदाभाति शुक्लैश्च दिव्यैः कुमुदकुड्मलैः} %4-27-22

\twolineshloka
{पारिप्लवशतैर्जुष्टा बर्हिक्रौञ्चविनादिता}
{रमणीया नदी सौम्य मुनिसङ्घनिषेविता} %4-27-23

\twolineshloka
{पश्य चन्दनवृक्षाणां पङ्क्तिः सुरुचिरा इव}
{ककुभानां च दृश्यन्ते मनसैवोदिताः समम्} %4-27-24

\twolineshloka
{अहो सुरमणीयोऽयं देशः शत्रुनिषूदन}
{दृढं रंस्याव सौमित्रे साध्वत्र निवसावहे} %4-27-25

\twolineshloka
{इतश्च नातिदूरे सा किष्किन्धा चित्रकानना}
{सुग्रीवस्य पुरी रम्या भविष्यति नृपात्मज} %4-27-26

\twolineshloka
{गीतवादित्रनिर्घोषः श्रूयते जयतां वर}
{नदतां वानराणां च मृदङ्गाडम्बरैः सह} %4-27-27

\twolineshloka
{लब्ध्वा भार्यां कपिवरः प्राप्य राज्यं सुहृद्वृतः}
{ध्रुवं नन्दति सुग्रीवः सम्प्राप्य महतीं श्रियम्} %4-27-28

\twolineshloka
{इत्युक्त्वा न्यवसत् तत्र राघवः सहलक्ष्मणः}
{बहुदृश्यदरीकुञ्जे तस्मिन् प्रस्रवणे गिरौ} %4-27-29

\twolineshloka
{सुसुखे हि बहुद्रव्ये तस्मिन् हि धरणीधरे}
{वसतस्तस्य रामस्य रतिरल्पापि नाभवत्} %4-27-30

\twolineshloka
{हृतां हि भार्यां स्मरतः प्राणेभ्योऽपि गरीयसीम्}
{उदयाभ्युदितं दृष्ट्वा शशाङ्कं च विशेषतः} %4-27-31

\twolineshloka
{आविवेश न तं निद्रा निशासु शयनं गतम्}
{तत्समुत्थेन शोकेन बाष्पोपहतचेतनम्} %4-27-32

\twolineshloka
{तं शोचमानं काकुत्स्थं नित्यं शोकपरायणम्}
{तुल्यदुःखोऽब्रवीद्भ्राता लक्ष्मणोऽनुनयं वचः} %4-27-33

\twolineshloka
{अलं वीर व्यथां गत्वा न त्वं शोचितुमर्हसि}
{शोचतो ह्यवसीदन्ति सर्वार्था विदितं हि ते} %4-27-34

\twolineshloka
{भवान् क्रियापरो लोके भवान् देवपरायणः}
{आस्तिको धर्मशीलश्च व्यवसायी च राघव} %4-27-35

\twolineshloka
{न ह्यव्यवसितः शत्रुं राक्षसं तं विशेषतः}
{समर्थस्त्वं रणे हन्तुं विक्रमे जिह्मकारिणम्} %4-27-36

\twolineshloka
{समुन्मूलय शोकं त्वं व्यवसायं स्थिरीकुरु}
{ततः सपरिवारं तं राक्षसं हन्तुमर्हसि} %4-27-37

\twolineshloka
{पृथिवीमपि काकुत्स्थ ससागरवनाचलाम्}
{परिवर्तयितुं शक्तः किं पुनस्तं हि रावणम्} %4-27-38

\twolineshloka
{शरत्कालं प्रतीक्षस्व प्रावृट्कालोऽयमागतः}
{ततः सराष्ट्रं सगणं रावणं तं वधिष्यसि} %4-27-39

\twolineshloka
{अहं तु खलु ते वीर्यं प्रसुप्तं प्रतिबोधये}
{दीप्तैराहुतिभिः काले भस्मच्छन्नमिवानलम्} %4-27-40

\twolineshloka
{लक्ष्मणस्य हि तद् वाक्यं प्रतिपूज्य हितं शुभम्}
{राघवः सुहृदं स्निग्धमिदं वचनमब्रवीत्} %4-27-41

\twolineshloka
{वाच्यं यदनुरक्तेन स्निग्धेन च हितेन च}
{सत्यविक्रमयुक्तेन तदुक्तं लक्ष्मण त्वया} %4-27-42

\twolineshloka
{एष शोकः परित्यक्तः सर्वकार्यावसादकः}
{विक्रमेष्वप्रतिहतं तेजः प्रोत्साहयाम्यहम्} %4-27-43

\twolineshloka
{शरत्कालं प्रतीक्षिष्ये स्थितोऽस्मि वचने तव}
{सुग्रीवस्य नदीनां च प्रसादमनुपालयन्} %4-27-44

\twolineshloka
{उपकारेण वीरस्तु प्रतिकारेण युज्यते}
{अकृतज्ञोऽप्रतिकृतो हन्ति सत्त्ववतां मनः} %4-27-45

\twolineshloka
{तदेव युक्तं प्रणिधाय लक्ष्मणः कृताञ्जलिस्तत् प्रतिपूज्य भाषितम्}
{उवाच रामं स्वभिरामदर्शनं प्रदर्शयन् दर्शनमात्मनः शुभम्} %4-27-46

\twolineshloka
{यथोक्तमेतत् तव सर्वमीप्सितं नरेन्द्र कर्ता नचिरात् तु वानरः}
{शरत्प्रतीक्षः क्षमतामिमं भवान् जलप्रपातं रिपुनिग्रहे धृतः} %4-27-47

\twolineshloka
{नियम्य कोपं परिपाल्यतां शरत् क्षमस्व मासांश्चतुरो मया सह}
{वसाचलेऽस्मिन् मृगराजसेविते संवर्तयन् शत्रुवधे समर्थः} %4-27-48


॥इत्यार्षे श्रीमद्रामायणे वाल्मीकीये आदिकाव्ये किष्किन्धाकाण्डे माल्यवन्निवासः नाम सप्तविंशः सर्गः ॥४-२७॥
