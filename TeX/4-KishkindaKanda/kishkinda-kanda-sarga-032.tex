\sect{द्वात्रिंशः सर्गः — हनुमन्मन्त्रः}

\twolineshloka
{अङ्गदस्य वचः श्रुत्वा सुग्रीवः सचिवैः सह}
{लक्ष्मणं कुपितं श्रुत्वा मुमोचासनमात्मवान्} %4-32-1

\twolineshloka
{स च तानब्रवीद् वाक्यं निश्चित्य गुरुलाघवम्}
{मन्त्रज्ञान् मन्त्रकुशलो मन्त्रेषु परिनिष्ठितः} %4-32-2

\twolineshloka
{न मे दुर्व्याहृतं किंचिन्नापि मे दुरनुष्ठितम्}
{लक्ष्मणो राघवभ्राता क्रुद्धः किमिति चिन्तये} %4-32-3

\twolineshloka
{असुहृद्भिर्ममामित्रैर्नित्यमन्तरदर्शिभिः}
{मम दोषानसम्भूतान् श्रावितो राघवानुजः} %4-32-4

\twolineshloka
{अत्र तावद् यथाबुद्धिः सर्वैरेव यथाविधि}
{भावस्य निश्चयस्तावद् विज्ञेयो निपुणं शनैः} %4-32-5

\twolineshloka
{न खल्वस्ति मम त्रासो लक्ष्मणान्नापि राघवात्}
{मित्रं स्वस्थानकुपितं जनयत्येव सम्भ्रमम्} %4-32-6

\twolineshloka
{सर्वथा सुकरं मित्रं दुष्करं प्रतिपालनम्}
{अनित्यत्वात् तु चित्तानां प्रीतिरल्पेऽपि भिद्यते} %4-32-7

\twolineshloka
{अतो निमित्तं त्रस्तोऽहं रामेण तु महात्मना}
{यन्ममोपकृतं शक्यं प्रतिकर्तुं न तन्मया} %4-32-8

\twolineshloka
{सुग्रीवेणैवमुक्ते तु हनूमान् हरिपुंगवः}
{उवाच स्वेन तर्केण मध्ये वानरमन्त्रिणाम्} %4-32-9

\twolineshloka
{सर्वथा नैतदाश्चर्यं यत् त्वं हरिगणेश्वर}
{न विस्मरसि सुस्निग्धमुपकारं कृतं शुभम्} %4-32-10

\twolineshloka
{राघवेण तु वीरेण भयमुत्सृज्य दूरतः}
{त्वत्प्रियार्थं हतो वाली शक्रतुल्यपराक्रमः} %4-32-11

\twolineshloka
{सर्वथा प्रणयात् क्रुद्धो राघवो नात्र संशयः}
{भ्रातरं सम्प्रहितवाँल्लक्ष्मणं लक्ष्मिवर्धनम्} %4-32-12

\twolineshloka
{त्वं प्रमत्तो न जानीषे कालं कालविदां वर}
{फुल्लसप्तच्छदश्यामा प्रवृत्ता तु शरच्छुभा} %4-32-13

\twolineshloka
{निर्मलग्रहनक्षत्रा द्यौः प्रणष्टबलाहका}
{प्रसन्नाश्च दिशः सर्वाः सरितश्च सरांसि च} %4-32-14

\twolineshloka
{प्राप्तमुद्योगकालं तु नावैषि हरिपुंगव}
{त्वं प्रमत्त इति व्यक्तं लक्ष्मणोऽयमिहागतः} %4-32-15

\twolineshloka
{आर्तस्य हृतदारस्य परुषं पुरुषान्तरात्}
{वचनं मर्षणीयं ते राघवस्य महात्मनः} %4-32-16

\twolineshloka
{कृतापराधस्य हि ते नान्यत् पश्याम्यहं क्षमम्}
{अन्तरेणाञ्जलिं बद्ध्वा लक्ष्मणस्य प्रसादनात्} %4-32-17

\twolineshloka
{नियुक्तैर्मन्त्रिभिर्वाच्यो ह्यवश्यं पार्थिवो हितम्}
{इत एव भयं त्यक्त्वा ब्रवीम्यवधृतं वचः} %4-32-18

\twolineshloka
{अभिक्रुद्धः समर्थो हि चापमुद्यम्य राघवः}
{सदेवासुरगन्धर्वं वशे स्थापयितुं जगत्} %4-32-19

\twolineshloka
{न स क्षमः कोपयितुं यः प्रसाद्यः पुनर्भवेत्}
{पूर्वोपकारं स्मरता कृतज्ञेन विशेषतः} %4-32-20

\twolineshloka
{तस्य मूर्ध्ना प्रणम्य त्वं सपुत्रः ससुहृज्जनः}
{राजंस्तिष्ठ स्वसमये भर्तुर्भार्येव तद्वशे} %4-32-21

\twolineshloka
{न रामरामानुजशासनं त्वया कपीन्द्र युक्तं मनसाप्यपोहितुम्}
{मनो हि ते ज्ञास्यति मानुषं बलं सराघवस्यास्य सुरेन्द्रवर्चसः} %4-32-22


॥इत्यार्षे श्रीमद्रामायणे वाल्मीकीये आदिकाव्ये किष्किन्धाकाण्डे हनुमन्मन्त्रः नाम द्वात्रिंशः सर्गः ॥४-३२॥
