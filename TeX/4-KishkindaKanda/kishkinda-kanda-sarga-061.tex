\sect{एकषष्ठितमः सर्गः — सूर्यानुगमनाख्यानम्}

\twolineshloka
{ततस्तद् दारुणं कर्म दुष्करं सहसा कृतम्}
{आचचक्षे मुनेः सर्वं सूर्यानुगमनं तथा} %4-61-1

\twolineshloka
{भगवन् व्रणयुक्तत्वाल्लज्जया चाकुलेन्द्रियः}
{परिश्रान्तो न शक्नोमि वचनं परिभाषितुम्} %4-61-2

\twolineshloka
{अहं चैव जटायुश्च संघर्षाद् गर्वमोहितौ}
{आकाशं पतितौ दूराज्जिज्ञासन्तौ पराक्रमम्} %4-61-3

\twolineshloka
{कैलासशिखरे बद्ध्वा मुनीनामग्रतः पणम्}
{रविः स्यादनुयातव्यो यावदस्तं महागिरिम्} %4-61-4

\twolineshloka
{अप्यावां युगपत् प्राप्तावपश्याव महीतले}
{रथचक्रप्रमाणानि नगराणि पृथक् पृथक्} %4-61-5

\twolineshloka
{क्वचिद् वादित्रघोषश्च क्वचिद् भूषणनिःस्वनः}
{गायन्तीः स्माङ्गना बह्वीः पश्यावो रक्तवाससः} %4-61-6

\twolineshloka
{तूर्णमुत्पत्य चाकाशमादित्यपदमास्थितौ}
{आवामालोकयावस्तद् वनं शाद्वलसंस्थितम्} %4-61-7

\twolineshloka
{उपलैरिव संछन्ना दृश्यते भूः शिलोच्चयैः}
{आपगाभिश्च संवीता सूत्रैरिव वसुंधरा} %4-61-8

\twolineshloka
{हिमवांश्चैव विन्ध्यश्च मेरुश्च सुमहागिरिः}
{भूतले सम्प्रकाशन्ते नागा इव जलाशये} %4-61-9

\twolineshloka
{तीव्रः स्वेदश्च खेदश्च भयं चासीत् तदावयोः}
{समाविशत मोहश्च ततो मूर्च्छा च दारुणा} %4-61-10

\twolineshloka
{न च दिग् ज्ञायते याम्या न चाग्नेयी न वारुणी}
{युगान्ते नियतो लोको हतो दग्ध इवाग्निना} %4-61-11

\twolineshloka
{मनश्च मे हतं भूयश्चक्षुः प्राप्य तु संश्रयम्}
{यत्नेन महता ह्यस्मिन् मनः संधाय चक्षुषी} %4-61-12

\twolineshloka
{यत्नेन महता भूयो भास्करः प्रतिलोकितः}
{तुल्यपृथ्वीप्रमाणेन भास्करः प्रतिभाति नौ} %4-61-13

\twolineshloka
{जटायुर्मामनापृच्छ्य निपपात महीं ततः}
{तं दृष्ट्वा तूर्णमाकाशादात्मानं मुक्तवानहम्} %4-61-14

\twolineshloka
{पक्षाभ्यां च मया गुप्तो जटायुर्न प्रदह्यत}
{प्रमादात् तत्र निर्दग्धः पतन् वायुपथादहम्} %4-61-15

\twolineshloka
{आशङ्के तं निपतितं जनस्थाने जटायुषम्}
{अहं तु पतितो विन्ध्ये दग्धपक्षो जडीकृतः} %4-61-16

\twolineshloka
{राज्याच्च हीनो भ्रात्रा च पक्षाभ्यां विक्रमेण च}
{सर्वथा मर्तुमेवेच्छन् पतिष्ये शिखराद् गिरेः} %4-61-17


॥इत्यार्षे श्रीमद्रामायणे वाल्मीकीये आदिकाव्ये किष्किन्धाकाण्डे सूर्यानुगमनाख्यानम् नाम एकषष्ठितमः सर्गः ॥४-६१॥
