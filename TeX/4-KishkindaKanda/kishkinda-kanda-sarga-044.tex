\sect{चतुश्चत्वारिंशः सर्गः — हनुमत्संदेशः}

\twolineshloka
{विशेषेण तु सुग्रीवो हनूमत्यर्थमुक्तवान्}
{स हि तस्मिन् हरिश्रेष्ठे निश्चितार्थोऽर्थसाधने} %4-44-1

\twolineshloka
{अब्रवीच्च हनूमन्तं विक्रान्तमनिलात्मजम्}
{सुग्रीवः परमप्रीतः प्रभुः सर्ववनौकसाम्} %4-44-2

\twolineshloka
{न भूमौ नान्तरिक्षे वा नाम्बरे नामरालये}
{नाप्सु वा गतिसङ्गं ते पश्यामि हरिपुंगव} %4-44-3

\twolineshloka
{सासुराः सहगन्धर्वाः सनागनरदेवताः}
{विदिताः सर्वलोकास्ते ससागरधराधराः} %4-44-4

\twolineshloka
{गतिर्वेगश्च तेजश्च लाघवं च महाकपे}
{पितुस्ते सदृशं वीर मारुतस्य महौजसः} %4-44-5

\twolineshloka
{तेजसा वापि ते भूतं न समं भुवि विद्यते}
{तद् यथा लभ्यते सीता तत्त्वमेवानुचिन्तय} %4-44-6

\twolineshloka
{त्वय्येव हनुमन्नस्ति बलं बुद्धिः पराक्रमः}
{देशकालानुवृत्तिश्च नयश्च नयपण्डित} %4-44-7

\twolineshloka
{ततः कार्यसमासङ्गमवगम्य हनूमति}
{विदित्वा हनुमन्तं च चिन्तयामास राघवः} %4-44-8

\twolineshloka
{सर्वथा निश्चितार्थोऽयं हनूमति हरीश्वरः}
{निश्चितार्थतरश्चापि हनूमान् कार्यसाधने} %4-44-9

\twolineshloka
{तदेवं प्रस्थितस्यास्य परिज्ञातस्य कर्मभिः}
{भर्त्रा परिगृहीतस्य ध्रुवः कार्यफलोदयः} %4-44-10

\twolineshloka
{तं समीक्ष्य महातेजा व्यवसायोत्तरं हरिम्}
{कृतार्थ इव संहृष्टः प्रहृष्टेन्द्रियमानसः} %4-44-11

\twolineshloka
{ददौ तस्य ततः प्रीतः स्वनामाङ्कोपशोभितम्}
{अङ्गुलीयमभिज्ञानं राजपुत्र्याः परंतपः} %4-44-12

\twolineshloka
{अनेन त्वां हरिश्रेष्ठ चिह्नेन जनकात्मजा}
{मत्सकाशादनुप्राप्तमनुद्विग्नानुपश्यति} %4-44-13

\twolineshloka
{व्यवसायश्च ते वीर सत्त्वयुक्तश्च विक्रमः}
{सुग्रीवस्य च संदेशः सिद्धिं कथयतीव मे} %4-44-14

\twolineshloka
{स तद् गृह्य हरिश्रेष्ठः कृत्वा मूर्ध्नि कृताञ्जलिः}
{वन्दित्वा चरणौ चैव प्रस्थितः प्लवगर्षभः} %4-44-15

\twolineshloka
{स तत् प्रकर्षन् हरिणां महद् बलं बभूव वीरः पवनात्मजः कपिः}
{गताम्बुदे व्योम्नि विशुद्धमण्डलः शशीव नक्षत्रगणोपशोभितः} %4-44-16

\twolineshloka
{अतिबल बलमाश्रितस्तवाहं हरिवर विक्रम विक्रमैरनल्पैः}
{पवनसुत यथाधिगम्यते सा जनकसुता हनुमंस्तथा कुरुष्व} %4-44-17


॥इत्यार्षे श्रीमद्रामायणे वाल्मीकीये आदिकाव्ये किष्किन्धाकाण्डे हनुमत्संदेशः नाम चतुश्चत्वारिंशः सर्गः ॥४-४४॥
