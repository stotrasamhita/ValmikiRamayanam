\sect{सप्तत्रिंशः सर्गः — कपिसेनासमानयनम्}

\twolineshloka
{एवमुक्तस्तु सुग्रीवो लक्ष्मणेन महात्मना}
{हनूमन्तं स्थितं पार्श्वे वचनं चेदमब्रवीत्} %4-37-1

\twolineshloka
{महेन्द्रहिमवद्विन्ध्यकैलासशिखरेषु च}
{मन्दरे पाण्डुशिखरे पञ्चशैलेषु ये स्थिताः} %4-37-2

\twolineshloka
{तरुणादित्यवर्णेषु भ्राजमानेषु नित्यशः}
{पर्वतेषु समुद्रान्ते पश्चिमस्यां तु ये दिशि} %4-37-3

\twolineshloka
{आदित्यभवने चैव गिरौ सन्ध्याभ्रसन्निभे}
{पद्माचलवनं भीमाः संश्रिता हरिपुङ्गवाः} %4-37-4

\twolineshloka
{अञ्जनाम्बुदसङ्काशाः कुञ्जरेन्द्रमहौजसः}
{अञ्जने पर्वते चैव ये वसन्ति प्लवङ्गमाः} %4-37-5

\twolineshloka
{महाशैलगुहावासा वानराः कनकप्रभाः}
{मेरुपार्श्वगताश्चैव ये च धूम्रगिरिं श्रिताः} %4-37-6

\twolineshloka
{तरुणादित्यवर्णाश्च पर्वते ये महारुणे}
{पिबन्तो मधु मैरेयं भीमवेगाः प्लवङ्गमाः} %4-37-7

\twolineshloka
{वनेषु च सुरम्येषु सुगन्धिषु महत्सु च}
{तापसाश्रमरम्येषु वनान्तेषु समन्ततः} %4-37-8

\twolineshloka
{तांस्तांस्त्वमानय क्षिप्रं पृथिव्यां सर्ववानरान्}
{सामदानादिभिः कल्पैर्वानरैर्वेगवत्तरैः} %4-37-9

\twolineshloka
{प्रेषिताः प्रथमं ये च मयाऽऽज्ञाता महाजवाः}
{त्वरणार्थं तु भूयस्त्वं सम्प्रेषय हरीश्वरान्} %4-37-10

\twolineshloka
{ये प्रसक्ताश्च कामेषु दीर्घसूत्राश्च वानराः}
{इहानयस्व तान् शीघ्रं सर्वानेव कपीश्वरान्} %4-37-11

\twolineshloka
{अहोभिर्दशभिर्ये च नागच्छन्ति ममाज्ञया}
{हन्तव्यास्ते दुरात्मानो राजशासनदूषकाः} %4-37-12

\twolineshloka
{शतान्यथ सहस्राणि कोट्यश्च मम शासनात्}
{प्रयान्तु कपिसिंहानां निदेशे मम ये स्थिताः} %4-37-13

\twolineshloka
{मेघपर्वतसङ्काशाश्छादयन्त इवाम्बरम्}
{घोररूपाः कपिश्रेष्ठा यान्तु मच्छासनादितः} %4-37-14

\twolineshloka
{ते गतिज्ञा गतिं गत्वा पृथिव्यां सर्ववानराः}
{आनयन्तु हरीन् सर्वांस्त्वरिताः शासनान्मम} %4-37-15

\twolineshloka
{तस्य वानरराजस्य श्रुत्वा वायुसुतो वचः}
{दिक्षु सर्वासु विक्रान्तान् प्रेषयामास वानरान्} %4-37-16

\twolineshloka
{ते पदं विष्णुविक्रान्तं पतत्त्रिज्योतिरध्वगाः}
{प्रयाताः प्रहिता राज्ञा हरयस्तु क्षणेन वै} %4-37-17

\twolineshloka
{ते समुद्रेषु गिरिषु वनेषु च सरस्सु च}
{वानरा वानरान् सर्वान् रामहेतोरचोदयन्} %4-37-18

\twolineshloka
{मृत्युकालोपमस्याज्ञां राजराजस्य वानराः}
{सुग्रीवस्याययुः श्रुत्वा सुग्रीवभयशङ्किताः} %4-37-19

\twolineshloka
{ततस्तेऽञ्जनसङ्काशा गिरेस्तस्मान्महाबलाः}
{तिस्रः कोट्यः प्लवङ्गानां निर्ययुर्यत्र राघवः} %4-37-20

\twolineshloka
{अस्तं गच्छति यत्रार्कस्तस्मिन् गिरिवरे रताः}
{सन्तप्तहेमवर्णाभास्तस्मात् कोट्यो दश च्युताः} %4-37-21

\twolineshloka
{कैलासशिखरेभ्यश्च सिंहकेसरवर्चसाम्}
{ततः कोटिसहस्राणि वानराणां समागमन्} %4-37-22

\twolineshloka
{फलमूलेन जीवन्तो हिमवन्तमुपाश्रिताः}
{तेषां कोटिसहस्राणां सहस्रं समवर्तत} %4-37-23

\twolineshloka
{अङ्गारकसमानानां भीमानां भीमकर्मणाम्}
{विन्ध्याद् वानरकोटीनां सहस्राण्यपतन् द्रुतम्} %4-37-24

\twolineshloka
{क्षीरोदवेलानिलयास्तमालवनवासिनः}
{नारिकेलाशनाश्चैव तेषां सङ्ख्या न विद्यते} %4-37-25

\twolineshloka
{वनेभ्यो गह्वरेभ्यश्च सरिद्भ्यश्च महाबलाः}
{आगच्छद् वानरी सेना पिबन्तीव दिवाकरम्} %4-37-26

\twolineshloka
{ये तु त्वरयितुं याता वानराः सर्ववानरान्}
{ते वीरा हिमवच्छैले ददृशुस्तं महाद्रुमम्} %4-37-27

\twolineshloka
{तस्मिन् गिरिवरे पुण्ये यज्ञो माहेश्वरः पुरा}
{सर्वदेवमनस्तोषो बभूव सुमनोरमः} %4-37-28

\twolineshloka
{अन्ननिस्यन्दजातानि मूलानि च फलानि च}
{अमृतस्वादुकल्पानि ददृशुस्तत्र वानराः} %4-37-29

\twolineshloka
{तदन्नसम्भवं दिव्यं फलमूलं मनोहरम्}
{यः कश्चित् सकृदश्नाति मासं भवति तर्पितः} %4-37-30

\twolineshloka
{तानि मूलानि दिव्यानि फलानि च फलाशनाः}
{औषधानि च दिव्यानि जगृहुर्हरिपुङ्गवाः} %4-37-31

\twolineshloka
{तस्माच्च यज्ञायतनात् पुष्पाणि सुरभीणि च}
{आनिन्युर्वानरा गत्वा सुग्रीवप्रियकारणात्} %4-37-32

\twolineshloka
{ते तु सर्वे हरिवराः पृथिव्यां सर्ववानरान्}
{सञ्चोदयित्वा त्वरितं यूथानां जग्मुरग्रतः} %4-37-33

\twolineshloka
{ते तु तेन मुहूर्तेन कपयः शीघ्रचारिणः}
{किष्किन्धां त्वरया प्राप्ताः सुग्रीवो यत्र वानरः} %4-37-34

\twolineshloka
{ते गृहीत्वौषधीः सर्वाः फलमूलं च वानराः}
{तं प्रतिग्राहयामासुर्वचनं चेदमब्रुवन्} %4-37-35

\twolineshloka
{सर्वे परिसृताः शैलाः सरितश्च वनानि च}
{पृथिव्यां वानराः सर्वे शासनादुपयान्ति ते} %4-37-36

\twolineshloka
{एवं श्रुत्वा ततो हृष्टः सुग्रीवः प्लवगाधिपः}
{प्रतिजग्राह च प्रीतस्तेषां सर्वमुपायनम्} %4-37-37


॥इत्यार्षे श्रीमद्रामायणे वाल्मीकीये आदिकाव्ये किष्किन्धाकाण्डे कपिसेनासमानयनम् नाम सप्तत्रिंशः सर्गः ॥४-३७॥
