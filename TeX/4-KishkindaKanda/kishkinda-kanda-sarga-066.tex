\sect{षट्षष्ठितमः सर्गः — हनुमद्बलसंधुक्षणम्}

\twolineshloka
{अनेकशतसाहस्रीं विषण्णां हरिवाहिनीम्}
{जाम्बवान् समुदीक्ष्यैवं हनूमन्तमथाब्रवीत्} %4-66-1

\twolineshloka
{वीर वानरलोकस्य सर्वशास्त्रविदां वर}
{तूष्णीमेकान्तमाश्रित्य हनूमन् किं न जल्पसि} %4-66-2

\twolineshloka
{हनूमन् हरिराजस्य सुग्रीवस्य समो ह्यसि}
{रामलक्ष्मणयोश्चापि तेजसा च बलेन च} %4-66-3

\twolineshloka
{अरिष्टनेमिनः पुत्रो वैनतेयो महाबलः}
{गरुत्मानिव विख्यात उत्तमः सर्वपक्षिणाम्} %4-66-4

\twolineshloka
{बहुशो हि मया दृष्टः सागरे स महाबलः}
{भुजङ्गानुद्धरन् पक्षी महाबाहुर्महाबलः} %4-66-5

\twolineshloka
{पक्षयोर्यद् बलं तस्य भुजवीर्यबलं तव}
{विक्रमश्चापि वेगश्च न ते तेनापहीयते} %4-66-6

\twolineshloka
{बलं बुद्धिश्च तेजश्च सत्त्वं च हरिपुङ्गव}
{विशिष्टं सर्वभूतेषु किमात्मानं न सज्जसे} %4-66-7

\twolineshloka
{अप्सराऽप्सरसां श्रेष्ठा विख्याता पुञ्जिकस्थला}
{अञ्जनेति परिख्याता पत्नी केसरिणो हरेः} %4-66-8

\twolineshloka
{विख्याता त्रिषु लोकेषु रूपेणाप्रतिमा भुवि}
{अभिशापादभूत् तात कपित्वे कामरूपिणी} %4-66-9

\twolineshloka
{दुहिता वानरेन्द्रस्य कुञ्जरस्य महात्मनः}
{मानुषं विग्रहं कृत्वा रूपयौवनशालिनी} %4-66-10

\twolineshloka
{विचित्रमाल्याभरणा कदाचित् क्षौमधारिणी}
{अचरत् पर्वतस्याग्रे प्रावृडम्बुदसंनिभे} %4-66-11

\twolineshloka
{तस्या वस्त्रं विशालाक्ष्याः पीतं रक्तदशं शुभम्}
{स्थितायाः पर्वतस्याग्रे मारुतोऽपाहरच्छनैः} %4-66-12

\twolineshloka
{स ददर्श ततस्तस्या वृत्तावूरू सुसंहतौ}
{स्तनौ च पीनौ सहितौ सुजातं चारु चाननम्} %4-66-13

\twolineshloka
{तां बलादायतश्रोणीं तनुमध्यां यशस्विनीम्}
{दृष्ट्वैव शुभसर्वाङ्गीं पवनः काममोहितः} %4-66-14

\twolineshloka
{स तां भुजाभ्यां दीर्घाभ्यां पर्यष्वजत मारुतः}
{मन्मथाविष्टसर्वाङ्गो गतात्मा तामनिन्दिताम्} %4-66-15

\twolineshloka
{सा तु तत्रैव सम्भ्रान्ता सुव्रता वाक्यमब्रवीत्}
{एकपत्नीव्रतमिदं को नाशयितुमिच्छति} %4-66-16

\twolineshloka
{अञ्जनाया वचः श्रुत्वा मारुतः प्रत्यभाषत}
{न त्वां हिंसामि सुश्रोणि मा भूत् ते मनसो भयम्} %4-66-17

\twolineshloka
{मनसास्मि गतो यत् त्वां परिष्वज्य यशस्विनि}
{वीर्यवान् बुद्धिसम्पन्नस्तव पुत्रो भविष्यति} %4-66-18

\twolineshloka
{महासत्त्वो महातेजा महाबलपराक्रमः}
{लङ्घने प्लवने चैव भविष्यति मया समः} %4-66-19

\twolineshloka
{एवमुक्ता ततस्तुष्टा जननी ते महाकपे}
{गुहायां त्वां महाबाहो प्रजज्ञे प्लवगर्षभ} %4-66-20

\twolineshloka
{अभ्युत्थितं ततः सूर्यं बालो दृष्ट्वा महावने}
{फलं चेति जिघृक्षुस्त्वमुत्प्लुत्याभ्युत्पतो दिवम्} %4-66-21

\twolineshloka
{शतानि त्रीणि गत्वाथ योजनानां महाकपे}
{तेजसा तस्य निर्धूतो न विषादं गतस्ततः} %4-66-22

\twolineshloka
{त्वामप्युपगतं तूर्णमन्तरिक्षं महाकपे}
{क्षिप्तमिन्द्रेण ते वज्रं कोपाविष्टेन तेजसा} %4-66-23

\twolineshloka
{तदा शैलाग्रशिखरे वामो हनुरभज्यत}
{ततो हि नामधेयं ते हनुमानिति कीर्तितम्} %4-66-24

\twolineshloka
{ततस्त्वां निहतं दृष्ट्वा वायुर्गन्धवहः स्वयम्}
{त्रैलोक्यं भृशसंक्रुद्धो न ववौ वै प्रभञ्जनः} %4-66-25

\twolineshloka
{सम्भ्रान्ताश्च सुराः सर्वे त्रैलोक्ये क्षुभिते सति}
{प्रसादयन्ति संक्रुद्धं मारुतं भुवनेश्वराः} %4-66-26

\twolineshloka
{प्रसादिते च पवने ब्रह्मा तुभ्यं वरं ददौ}
{अशस्त्रवध्यतां तात समरे सत्यविक्रम} %4-66-27

\twolineshloka
{वज्रस्य च निपातेन विरुजं त्वां समीक्ष्य च}
{सहस्रनेत्रः प्रीतात्मा ददौ ते वरमुत्तमम्} %4-66-28

\twolineshloka
{स्वच्छन्दतश्च मरणं तव स्यादिति वै प्रभो}
{स त्वं केसरिणः पुत्रः क्षेत्रजो भीमविक्रमः} %4-66-29

\twolineshloka
{मारुतस्यौरसः पुत्रस्तेजसा चापि तत्समः}
{त्वं हि वायुसुतो वत्स प्लवने चापि तत्समः} %4-66-30

\twolineshloka
{वयमद्य गतप्राणा भवानस्मासु साम्प्रतम्}
{दाक्ष्यविक्रमसम्पन्नः कपिराज इवापरः} %4-66-31

\twolineshloka
{त्रिविक्रमे मया तात सशैलवनकानना}
{त्रिःसप्तकृत्वः पृथिवी परिक्रान्ता प्रदक्षिणम्} %4-66-32

\twolineshloka
{तथा चौषधयोऽस्माभिः संचिता देवशासनात्}
{निर्मथ्यममृतं याभिस्तदानीं नो महद्बलम्} %4-66-33

\twolineshloka
{स इदानीमहं वृद्धः परिहीनपराक्रमः}
{साम्प्रतं कालमस्माकं भवान् सर्वगुणान्वितः} %4-66-34

\twolineshloka
{तद् विजृम्भस्व विक्रान्त प्लवतामुत्तमो ह्यसि}
{त्वद्वीर्यं द्रष्टुकामा हि सर्वा वानरवाहिनी} %4-66-35

\twolineshloka
{उत्तिष्ठ हरिशार्दूल लङ्घयस्व महार्णवम्}
{परा हि सर्वभूतानां हनुमन् या गतिस्तव} %4-66-36

\twolineshloka
{विषण्णा हरयः सर्वे हनुमन् किमुपेक्षसे}
{विक्रमस्व महावेग विष्णुस्त्रीन् विक्रमानिव} %4-66-37

\twolineshloka
{ततः कपीनामृषभेण चोदितः प्रतीतवेगः पवनात्मजः कपिः}
{प्रहर्षयंस्तां हरिवीरवाहिनीं चकार रूपं महदात्मनस्तदा} %4-66-38


॥इत्यार्षे श्रीमद्रामायणे वाल्मीकीये आदिकाव्ये किष्किन्धाकाण्डे हनुमद्बलसंधुक्षणम् नाम षट्षष्ठितमः सर्गः ॥४-६६॥
