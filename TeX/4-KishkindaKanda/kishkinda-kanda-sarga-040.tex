\sect{चत्वारिंशः सर्गः — प्राचीप्रेषणम्}

\twolineshloka
{अथ राजा समृद्धार्थः सुग्रीवः प्लवगेश्वरः}
{उवाच नरशार्दूलं रामं परबलार्दनम्} %4-40-1

\twolineshloka
{आगता विनिविष्टाश्च बलिनः कामरूपिणः}
{वानरेन्द्रा महेन्द्राभा ये मद्विषयवासिनः} %4-40-2

\twolineshloka
{त इमे बहुविक्रान्तैर्बलिभिर्भीमविक्रमैः}
{आगता वानरा घोरा दैत्यदानवसन्निभाः} %4-40-3

\twolineshloka
{ख्यातकर्मापदानाश्च बलवन्तो जितक्लमाः}
{पराक्रमेषु विख्याता व्यवसायेषु चोत्तमाः} %4-40-4

\twolineshloka
{पृथिव्यम्बुचरा राम नानानगनिवासिनः}
{कोट्योघाश्च इमे प्राप्ता वानरास्तव किङ्कराः} %4-40-5

\twolineshloka
{निदेशवर्तिनः सर्वे सर्वे गुरुहिते स्थिताः}
{अभिप्रेतमनुष्ठातुं तव शक्ष्यन्त्यरिन्दम} %4-40-6

\twolineshloka
{त इमे बहुसाहस्रैरनीकैर्भीमविक्रमैः}
{आगता वानरा घोरा दैत्यदानवसन्निभाः} %4-40-7

\twolineshloka
{यन्मन्यसे नरव्याघ्र प्राप्तकालं तदुच्यताम्}
{त्वत्सैन्यं त्वद्वशे युक्तमाज्ञापयितुमर्हसि} %4-40-8

\twolineshloka
{काममेषामिदं कार्यं विदितं मम तत्त्वतः}
{तथापि तु यथायुक्तमाज्ञापयितुमर्हसि} %4-40-9

\twolineshloka
{तथा ब्रुवाणं सुग्रीवं रामो दशरथात्मजः}
{बाहुभ्यां सम्परिष्वज्य इदं वचनमब्रवीत्} %4-40-10

\twolineshloka
{ज्ञायतां सौम्य वैदेही यदि जीवति वा न वा}
{स च देशो महाप्राज्ञ यस्मिन् वसति रावणः} %4-40-11

\twolineshloka
{अधिगम्य तु वैदेहीं निलयं रावणस्य च}
{प्राप्तकालं विधास्यामि तस्मिन् काले सह त्वया} %4-40-12

\twolineshloka
{नाहमस्मिन् प्रभुः कार्ये वानरेन्द्र न लक्ष्मणः}
{त्वमस्य हेतुः कार्यस्य प्रभुश्च प्लवगेश्वर} %4-40-13

\twolineshloka
{त्वमेवाज्ञापय विभो मम कार्यविनिश्चयम्}
{त्वं हि जानासि मे कार्यं मम वीर न संशयः} %4-40-14

\twolineshloka
{सुहृद्द्वितीयो विक्रान्तः प्राज्ञः कालविशेषवित्}
{भवानस्मद्धिते युक्तः सुहृदाप्तोऽर्थवित्तमः} %4-40-15

\twolineshloka
{एवमुक्तस्तु सुग्रीवो विनतं नाम यूथपम्}
{अब्रवीद् रामसान्निध्ये लक्ष्मणस्य च धीमतः} %4-40-16

\twolineshloka
{शैलाभं मेघनिर्घोषमूर्जितं प्लवगेश्वरम्}
{सोमसूर्यनिभैः सार्धं वानरैर्वानरोत्तम} %4-40-17

\twolineshloka
{देशकालनयैर्युक्तो विज्ञः कार्यविनिश्चये}
{वृतः शतसहस्रेण वानराणां तरस्विनाम्} %4-40-18

\twolineshloka
{अधिगच्छ दिशं पूर्वां सशैलवनकाननाम्}
{तत्र सीतां च वैदेहीं निलयं रावणस्य च} %4-40-19

\twolineshloka
{मार्गध्वं गिरिदुर्गेषु वनेषु च नदीषु च}
{नदीं भागीरथीं रम्यां सरयूं कौशिकीं तथा} %4-40-20

\twolineshloka
{कालिन्दीं यमुनां रम्यां यामुनं च महागिरिम्}
{सरस्वतीं च सिन्धुं च शोणं मणिनिभोदकम्} %4-40-21

\twolineshloka
{महीं कालमहीं चापि शैलकाननशोभिताम्}
{ब्रह्ममालान् विदेहांश्च मालवान् काशिकोसलान्} %4-40-22

\twolineshloka
{मागधांश्च महाग्रामान् पुण्ड्रांस्त्वङ्गांस्तथैव च}
{भूमिं च कोशकाराणां भूमिं च रजताकराम्} %4-40-23

\twolineshloka
{सर्वं च तद् विचेतव्यं मार्गयद्भिस्ततस्ततः}
{रामस्य दयितां भार्यां सीतां दशरथस्नुषाम्} %4-40-24

\twolineshloka
{समुद्रमवगाढांश्च पर्वतान् पत्तनानि च}
{मन्दरस्य च ये कोटिं संश्रिताः केचिदालयाः} %4-40-25

\twolineshloka
{कर्णप्रावरणाश्चैव तथा चाप्योष्ठकर्णकाः}
{घोरलोहमुखाश्चैव जवनाश्चैकपादकाः} %4-40-26

\twolineshloka
{अक्षया बलवन्तश्च तथैव पुरुषादकाः}
{किरातास्तीक्ष्णचूडाश्च हेमाभाः प्रियदर्शनाः} %4-40-27

\twolineshloka
{आममीनाशनाश्चापि किराता द्वीपवासिनः}
{अन्तर्जलचरा घोरा नरव्याघ्रा इति स्मृताः} %4-40-28

\twolineshloka
{एतेषामाश्रयाः सर्वे विचेयाः काननौकसः}
{गिरिभिर्ये च गम्यन्ते प्लवनेन प्लवेन च} %4-40-29

\twolineshloka
{यत्नवन्तो यवद्वीपं सप्तराजोपशोभितम्}
{सुवर्णरूप्यकद्वीपं सुवर्णाकरमण्डितम्} %4-40-30

\twolineshloka
{यवद्वीपमतिक्रम्य शिशिरो नाम पर्वतः}
{दिवं स्पृशति शृङ्गेण देवदानवसेवितः} %4-40-31

\twolineshloka
{एतेषां गिरिदुर्गेषु प्रपातेषु वनेषु च}
{मार्गध्वं सहिताः सर्वे रामपत्नीं यशस्विनीम्} %4-40-32

\twolineshloka
{ततो रक्तजलं प्राप्य शोणाख्यं शीघ्रवाहिनम्}
{गत्वा पारं समुद्रस्य सिद्धचारणसेवितम्} %4-40-33

\twolineshloka
{तस्य तीर्थेषु रम्येषु विचित्रेषु वनेषु च}
{रावणः सह वैदेह्या मार्गितव्यस्ततस्ततः} %4-40-34

\twolineshloka
{पर्वतप्रभवा नद्यः सुभीमबहुनिष्कुटाः}
{मार्गितव्या दरीमन्तः पर्वताश्च वनानि च} %4-40-35

\twolineshloka
{ततः समुद्रद्वीपांश्च सुभीमान् द्रष्टुमर्हथ}
{ऊर्मिमन्तं महारौद्रं क्रोशन्तमनिलोद्धतम्} %4-40-36

\twolineshloka
{तत्रासुरा महाकायाश्छायां गृह्णन्ति नित्यशः}
{ब्रह्मणा समनुज्ञाता दीर्घकालं बुभुक्षिताः} %4-40-37

\twolineshloka
{तं कालमेघप्रतिमं महोरगनिषेवितम्}
{अभिगम्य महानादं तीर्थेनैव महोदधिम्} %4-40-38

\twolineshloka
{ततो रक्तजलं भीमं लोहितं नाम सागरम्}
{गत्वा प्रेक्ष्यथ तां चैव बृहतीं कूटशाल्मलीम्} %4-40-39

\twolineshloka
{गृहं च वैनतेयस्य नानारत्नविभूषितम्}
{तत्र कैलाससङ्काशं विहितं विश्वकर्मणा} %4-40-40

\twolineshloka
{तत्र शैलनिभा भीमा मन्देहा नाम राक्षसाः}
{शैलशृङ्गेषु लम्बन्ते नानारूपा भयावहाः} %4-40-41

\twolineshloka
{ते पतन्ति जले नित्यं सूर्यस्योदयनं प्रति}
{अभितप्ताः स्म सूर्येण लम्बन्ते स्म पुनः पुनः} %4-40-42

\twolineshloka
{निहता ब्रह्मतेजोभिरहन्यहनि राक्षसाः}
{ततः पाण्डुरमेघाभं क्षीरोदं नाम सागरम्} %4-40-43

\twolineshloka
{गत्वा द्रक्ष्यथ दुर्धर्षा मुक्ताहारमिवोर्मिभिः}
{तस्य मध्ये महान् श्वेतो ऋषभो नाम पर्वतः} %4-40-44

\twolineshloka
{दिव्यगन्धैः कुसुमितैराचितैश्च नगैर्वृतः}
{सरश्च राजतैः पद्मैर्ज्वलितैर्हेमकेसरैः} %4-40-45

\twolineshloka
{नाम्ना सुदर्शनं नाम राजहंसैः समाकुलम्}
{विबुधाश्चारणा यक्षाः किन्नराश्चाप्सरोगणाः} %4-40-46

\twolineshloka
{हृष्टाः समधिगच्छन्ति नलिनीं तां रिरंसवः}
{क्षीरोदं समतिक्रम्य तदा द्रक्ष्यथ वानराः} %4-40-47

\twolineshloka
{जलोदं सागरं शीघ्रं सर्वभूतभयावहम्}
{तत्र तत्कोपजं तेजः कृतं हयमुखं महत्} %4-40-48

\threelineshloka
{अस्याहुस्तन्महावेगमोदनं सचराचरम्}
{तत्र विक्रोशतां नादो भूतानां सागरौकसाम्}
{श्रूयते चासमर्थानां दृष्ट्वाभूद् वडवामुखम्} %4-40-49

\twolineshloka
{स्वादूदस्योत्तरे तीरे योजनानि त्रयोदश}
{जातरूपशिलो नाम सुमहान् कनकप्रभः} %4-40-50

\twolineshloka
{तत्र चन्द्रप्रतीकाशं पन्नगं धरणीधरम्}
{पद्मपत्रविशालाक्षं ततो द्रक्ष्यथ वानराः} %4-40-51

\twolineshloka
{आसीनं पर्वतस्याग्रे सर्वदेवनमस्कृतम्}
{सहस्रशिरसं देवमनन्तं नीलवाससम्} %4-40-52

\twolineshloka
{त्रिशिराः काञ्चनः केतुस्तालस्तस्य महात्मनः}
{स्थापितः पर्वतस्याग्रे विराजति सवेदिकः} %4-40-53

\twolineshloka
{पूर्वस्यां दिशि निर्माणं कृतं तत् त्रिदशेश्वरैः}
{ततः परं हेममयः श्रीमानुदयपर्वतः} %4-40-54

\twolineshloka
{तस्य कोटिर्दिवं स्पृष्ट्वा शतयोजनमायता}
{जातरूपमयी दिव्या विराजति सवेदिका} %4-40-55

\twolineshloka
{सालैस्तालैस्तमालैश्च कर्णिकारैश्च पुष्पितैः}
{जातरूपमयैर्दिव्यैः शोभते सूर्यसन्निभैः} %4-40-56

\twolineshloka
{तत्र योजनविस्तारमुच्छ्रितं दशयोजनम्}
{शृङ्गं सौमनसं नाम जातरूपमयं ध्रुवम्} %4-40-57

\twolineshloka
{तत्र पूर्वं पदं कृत्वा पुरा विष्णुस्त्रिविक्रमे}
{द्वितीयं शिखरे मेरोश्चकार पुरुषोत्तमः} %4-40-58

\twolineshloka
{उत्तरेण परिक्रम्य जम्बूद्वीपं दिवाकरः}
{दृश्यो भवति भूयिष्ठं शिखरं तन्महोच्छ्रयम्} %4-40-59

\twolineshloka
{तत्र वैखानसा नाम वालखिल्या महर्षयः}
{प्रकाशमाना दृश्यन्ते सूर्यवर्णास्तपस्विनः} %4-40-60

\twolineshloka
{अयं सुदर्शनो द्वीपः पुरो यस्य प्रकाशते}
{तस्मिंस्तेजश्च चक्षुश्च सर्वप्राणभृतामपि} %4-40-61

\twolineshloka
{शैलस्य तस्य पृष्ठेषु कन्दरेषु वनेषु च}
{रावणः सह वैदेह्या मार्गितव्यस्ततस्ततः} %4-40-62

\twolineshloka
{काञ्चनस्य च शैलस्य सूर्यस्य च महात्मनः}
{आविष्टा तेजसा सन्ध्या पूर्वा रक्ता प्रकाशते} %4-40-63

\twolineshloka
{पूर्वमेतत् कृतं द्वारं पृथिव्या भुवनस्य च}
{सूर्यस्योदयनं चैव पूर्वा ह्येषा दिगुच्यते} %4-40-64

\twolineshloka
{तस्य शैलस्य पृष्ठेषु निर्झरेषु गुहासु च}
{रावणः सह वैदेह्या मार्गितव्यस्ततस्ततः} %4-40-65

\twolineshloka
{ततः परमगम्या स्याद् दिक्पूर्वा त्रिदशावृता}
{रहिता चन्द्रसूर्याभ्यामदृश्या तमसावृता} %4-40-66

\twolineshloka
{शैलेषु तेषु सर्वेषु कन्दरेषु नदीषु च}
{ये च नोक्ता मयोद्देशा विचेया तेषु जानकी} %4-40-67

\twolineshloka
{एतावद् वानरैः शक्यं गन्तुं वानरपुङ्गवाः}
{अभास्करममर्यादं न जानीमस्ततः परम्} %4-40-68

\twolineshloka
{अभिगम्य तु वैदेहीं निलयं रावणस्य च}
{मासे पूर्णे निवर्तध्वमुदयं प्राप्य पर्वतम्} %4-40-69

\twolineshloka
{ऊर्ध्वं मासान्न वस्तव्यं वसन् वध्यो भवेन्मम}
{सिद्धार्थाः सन्निवर्तध्वमधिगम्य च मैथिलीम्} %4-40-70

\twolineshloka
{महेन्द्रकान्तां वनषण्डमण्डितां दिशं चरित्वा निपुणेन वानराः}
{अवाप्य सीतां रघुवंशजप्रियां ततो निवृत्ताः सुखिनो भविष्यथ} %4-40-71


॥इत्यार्षे श्रीमद्रामायणे वाल्मीकीये आदिकाव्ये किष्किन्धाकाण्डे प्राचीप्रेषणम् नाम चत्वारिंशः सर्गः ॥४-४०॥
