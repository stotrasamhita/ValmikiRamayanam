\sect{सप्तपञ्चाशः सर्गः — जटायुर्दिष्टकथनम्}

\twolineshloka
{शोकाद् भ्रष्टस्वरमपि श्रुत्वा वानरयूथपाः}
{श्रद्दधुर्नैव तद्वाक्यं कर्मणा तस्य शङ्किताः} %4-57-1

\twolineshloka
{ते प्रायमुपविष्टास्तु दृष्ट्वा गृध्रं प्लवंगमाः}
{चक्रुर्बुद्धिं तदा रौद्रां सर्वान् नो भक्षयिष्यति} %4-57-2

\twolineshloka
{सर्वथा प्रायमासीनान् यदि नो भक्षयिष्यति}
{कृतकृत्या भविष्यामः क्षिप्रं सिद्धिमितो गताः} %4-57-3

\twolineshloka
{एतां बुद्धिं ततश्चक्रुः सर्वे ते हरियूथपाः}
{अवतार्य गिरेः शृङ्गाद् गृध्रमाहाङ्गदस्तदा} %4-57-4

\twolineshloka
{बभूवर्क्षरजो नाम वानरेन्द्रः प्रतापवान्}
{ममार्यः पार्थिवः पक्षिन् धार्मिकौ तस्य चात्मजौ} %4-57-5

\twolineshloka
{सुग्रीवश्चैव वाली च पुत्रौ घनबलावुभौ}
{लोके विश्रुतकर्माभूद् राजा वाली पिता मम} %4-57-6

\twolineshloka
{राजा कृत्स्नस्य जगत इक्ष्वाकूणां महारथः}
{रामो दाशरथिः श्रीमान् प्रविष्टो दण्डकावनम्} %4-57-7

\twolineshloka
{लक्ष्मणेन सह भ्रात्रा वैदेह्या सह भार्यया}
{पितुर्निदेशनिरतो धर्मं पन्थानमाश्रितः} %4-57-8

\twolineshloka
{तस्य भार्या जनस्थानाद् रावणेन हृता बलात्}
{रामस्य तु पितुर्मित्रं जटायुर्नाम गृध्रराट्} %4-57-9

\threelineshloka
{ददर्श सीतां वैदेहीं ह्रियमाणां विहायसा}
{रावणं विरथं कृत्वा स्थापयित्वा च मैथिलीम्}
{परिश्रान्तश्च वृद्धश्च रावणेन हतो रणे} %4-57-10

\twolineshloka
{एवं गृध्रो हतस्तेन रावणेन बलीयसा}
{संस्कृतश्चापि रामेण जगाम गतिमुत्तमाम्} %4-57-11

\twolineshloka
{ततो मम पितृव्येण सुग्रीवेण महात्मना}
{चकार राघवः सख्यं सोऽवधीत् पितरं मम} %4-57-12

\twolineshloka
{मम पित्रा निरुद्धो हि सुग्रीवः सचिवैः सह}
{निहत्य वालिनं रामस्ततस्तमभिषेचयत्} %4-57-13

\twolineshloka
{स राज्ये स्थापितस्तेन सुग्रीवो वानरेश्वरः}
{राजा वानरमुख्यानां तेन प्रस्थापिता वयम्} %4-57-14

\twolineshloka
{एवं रामप्रयुक्तास्तु मार्गमाणास्ततस्ततः}
{वैदेहीं नाधिगच्छामो रात्रौ सूर्यप्रभामिव} %4-57-15

\twolineshloka
{ते वयं दण्डकारण्यं विचित्य सुसमाहिताः}
{अज्ञानात् तु प्रविष्टाः स्म धरण्या विवृतं बिलम्} %4-57-16

\twolineshloka
{मयस्य मायाविहितं तद् बिलं च विचिन्वताम्}
{व्यतीतस्तत्र नो मासो यो राज्ञा समयः कृतः} %4-57-17

\twolineshloka
{ते वयं कपिराजस्य सर्वे वचनकारिणः}
{कृतां संस्थामतिक्रान्ता भयात् प्रायमुपासिताः} %4-57-18

\twolineshloka
{क्रुद्धे तस्मिंस्तु काकुत्स्थे सुग्रीवे च सलक्ष्मणे}
{गतानामपि सर्वेषां तत्र नो नास्ति जीवितम्} %4-57-19


॥इत्यार्षे श्रीमद्रामायणे वाल्मीकीये आदिकाव्ये किष्किन्धाकाण्डे जटायुर्दिष्टकथनम् नाम सप्तपञ्चाशः सर्गः ॥४-५७॥
