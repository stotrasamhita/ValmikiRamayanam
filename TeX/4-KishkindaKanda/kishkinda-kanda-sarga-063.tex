\sect{त्रिषष्ठितमः सर्गः — संपातिपक्षप्ररोहः}

\twolineshloka
{एतैरन्यैश्च बहुभिर्वाक्यैर्वाक्यविशारदः}
{मां प्रशस्याभ्यनुज्ञाप्य प्रविष्टः स स्वमालयम्} %4-63-1

\twolineshloka
{कन्दरात् तु विसर्पित्वा पर्वतस्य शनैः शनैः}
{अहं विन्ध्यं समारुह्य भवतः प्रतिपालये} %4-63-2

\twolineshloka
{अद्य त्वेतस्य कालस्य वर्षं साग्रशतं गतम्}
{देशकालप्रतीक्षोऽस्मि हृदि कृत्वा मुनेर्वचः} %4-63-3

\twolineshloka
{महाप्रस्थानमासाद्य स्वर्गते तु निशाकरे}
{मां निर्दहति संतापो वितर्कैर्बहुभिर्वृतम्} %4-63-4

\twolineshloka
{उदितां मरणे बुद्धिं मुनिवाक्यैर्निवर्तये}
{बुद्धिर्या तेन मे दत्ता प्राणानां रक्षणे मम} %4-63-5

\twolineshloka
{सा मेऽपनयते दुःखं दीप्तेवाग्निशिखा तमः}
{बुध्यता च मया वीर्यं रावणस्य दुरात्मनः} %4-63-6

\twolineshloka
{पुत्रः संतर्जितो वाग्भिर्न त्राता मैथिली कथम्}
{तस्या विलपितं श्रुत्वा तौ च सीतावियोजितौ} %4-63-7

\twolineshloka
{न मे दशरथस्नेहात् पुत्रेणोत्पादितं प्रियम्}
{तस्य त्वेवं ब्रुवाणस्य संहतैर्वानरैः सह} %4-63-8

\twolineshloka
{उत्पेततुस्तदा पक्षौ समक्षं वनचारिणाम्}
{स दृष्ट्वा स्वां तनुं पक्षैरुद्गतैररुणच्छदैः} %4-63-9

\twolineshloka
{प्रहर्षमतुलं लेभे वानरांश्चेदमब्रवीत्}
{निशाकरस्य राजर्षेः प्रसादादमितौजसः} %4-63-10

\twolineshloka
{आदित्यरश्मिनिर्दग्धौ पक्षौ पुनरुपस्थितौ}
{यौवने वर्तमानस्य ममासीद् यः पराक्रमः} %4-63-11

\twolineshloka
{तमेवाद्यावगच्छामि बलं पौरुषमेव च}
{सर्वथा क्रियतां यत्नः सीतामधिगमिष्यथ} %4-63-12

\twolineshloka
{पक्षलाभो ममायं वः सिद्धिप्रत्ययकारकः}
{इत्युक्त्वा तान् हरीन् सर्वान् सम्पातिः पतगोत्तमः} %4-63-13

\threelineshloka
{उत्पपात गिरेः शृङ्गाज्जिज्ञासुः खगमो गतिम्}
{तस्य तद् वचनं श्रुत्वा प्रतिसंहृष्टमानसाः}
{बभूवुर्हरिशार्दूला विक्रमाभ्युदयोन्मुखाः} %4-63-14

\twolineshloka
{अथ पवनसमानविक्रमाः प्लवगवराः प्रतिलब्धपौरुषाः}
{अभिजिदभिमुखां दिशं ययुर्जनकसुतापरिमार्गणोन्मुखाः} %4-63-15


॥इत्यार्षे श्रीमद्रामायणे वाल्मीकीये आदिकाव्ये किष्किन्धाकाण्डे संपातिपक्षप्ररोहः नाम त्रिषष्ठितमः सर्गः ॥४-६३॥
