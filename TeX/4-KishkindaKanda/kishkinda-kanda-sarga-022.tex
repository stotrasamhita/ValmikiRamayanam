\sect{द्वाविंशः सर्गः — वाल्यनुशासनम्}

\twolineshloka
{वीक्षमाणस्तु मन्दासुः सर्वतो मन्दमुच्छ्वसन्}
{आदावेव तु सुग्रीवं ददर्शानुजमग्रतः} %4-22-1

\twolineshloka
{तं प्राप्तविजयं वाली सुग्रीवं प्लवगेश्वरम्}
{आभाष्य व्यक्तया वाचा सस्नेहमिदमब्रवीत्} %4-22-2

\twolineshloka
{सुग्रीव दोषेण न मां गन्तुमर्हसि किल्बिषात्}
{कृष्यमाणं भविष्येण बुद्धिमोहेन मां बलात्} %4-22-3

\twolineshloka
{युगपद् विहितं तात न मन्ये सुखमावयोः}
{सौहार्दं भ्रातृयुक्तं हि तदिदं जातमन्यथा} %4-22-4

\twolineshloka
{प्रतिपद्य त्वमद्यैव राज्यमेषां वनौकसाम्}
{मामप्यद्यैव गच्छन्तं विद्धि वैवस्वतक्षयम्} %4-22-5

\twolineshloka
{जीवितं च हि राज्यं च श्रियं च विपुलां तथा}
{प्रजहाम्येष वै तूर्णमहं चागर्हितं यशः} %4-22-6

\twolineshloka
{अस्यां त्वहमवस्थायां वीर वक्ष्यामि यद् वचः}
{यद्यप्यसुकरं राजन् कर्तुमेव त्वमर्हसि} %4-22-7

\twolineshloka
{सुखार्हं सुखसंवृद्धं बालमेनमबालिशम्}
{बाष्पपूर्णमुखं पश्य भूमौ पतितमङ्गदम्} %4-22-8

\twolineshloka
{मम प्राणैः प्रियतरं पुत्रं पुत्रमिवौरसम्}
{मया हीनमहीनार्थं सर्वतः परिपालय} %4-22-9

\twolineshloka
{त्वमप्यस्य पिता दाता परित्राता च सर्वशः}
{भयेष्वभयदश्चैव यथाहं प्लवगेश्वर} %4-22-10

\twolineshloka
{एष तारात्मजः श्रीमांस्त्वया तुल्यपराक्रमः}
{रक्षसां च वधे तेषामग्रतस्ते भविष्यति} %4-22-11

\twolineshloka
{अनुरूपाणि कर्माणि विक्रम्य बलवान् रणे}
{करिष्यत्येष तारेयस्तेजस्वी तरुणोऽङ्गदः} %4-22-12

\twolineshloka
{सुषेणदुहिता चेयमर्थसूक्ष्मविनिश्चये}
{औत्पातिके च विविधे सर्वतः परिनिष्ठिता} %4-22-13

\twolineshloka
{यदेषा साध्विति ब्रूयात् कार्यं तन्मुक्तसंशयम्}
{नहि तारामतं किञ्चिदन्यथा परिवर्तते} %4-22-14

\twolineshloka
{राघवस्य च ते कार्यं कर्तव्यमविशङ्कया}
{स्यादधर्मो ह्यकरणे त्वां च हिंस्यादमानितः} %4-22-15

\twolineshloka
{इमां च मालामाधत्स्व दिव्यां सुग्रीव काञ्चनीम्}
{उदारा श्रीः स्थिता ह्यस्यां सम्प्रजह्यान्मृते मयि} %4-22-16

\twolineshloka
{इत्येवमुक्तः सुग्रीवो वालिना भ्रातृसौहृदात्}
{हर्षं त्यक्त्वा पुनर्दीनो ग्रहग्रस्त इवोडुराट्} %4-22-17

\twolineshloka
{तद्वालिवचनाच्छान्तः कुर्वन् युक्तमतन्द्रितः}
{जग्राह सोऽभ्यनुज्ञातो मालां तां चैव काञ्चनीम्} %4-22-18

\twolineshloka
{तां मालां काञ्चनीं दत्त्वा दृष्ट्वा चैवात्मजं स्थितम्}
{संसिद्धः प्रेत्यभावाय स्नेहादङ्गदमब्रवीत्} %4-22-19

\twolineshloka
{देशकालौ भजस्वाद्य क्षममाणः प्रियाप्रिये}
{सुखदुःखसहः काले सुग्रीववशगो भव} %4-22-20

\twolineshloka
{यथा हि त्वं महाबाहो लालितः सततं मया}
{न तथा वर्तमानं त्वां सुग्रीवो बहु मन्यते} %4-22-21

\twolineshloka
{नास्यामित्रैर्गतं गच्छेर्मा शत्रुभिररिन्दम}
{भर्तुरर्थपरो दान्तः सुग्रीववशगो भव} %4-22-22

\twolineshloka
{न चातिप्रणयः कार्यः कर्तव्योऽप्रणयश्च ते}
{उभयं हि महादोषं तस्मादन्तरदृग् भव} %4-22-23

\twolineshloka
{इत्युक्त्वाथ विवृत्ताक्षः शरसम्पीडितो भृशम्}
{विवृतैर्दशनैर्भीमैर्बभूवोत्क्रान्तजीवितः} %4-22-24

\twolineshloka
{ततो विचुक्रुशुस्तत्र वानरा हतयूथपाः}
{परिदेवयमानास्ते सर्वे प्लवगसत्तमाः} %4-22-25

\twolineshloka
{किष्किन्धा ह्यद्य शून्या च स्वर्गते वानरेश्वरे}
{उद्यानानि च शून्यानि पर्वताः काननानि च} %4-22-26

\twolineshloka
{हते प्लवगशार्दूले निष्प्रभा वानराः कृताः}
{यस्य वेगेन महता काननानि वनानि च} %4-22-27

\twolineshloka
{पुष्पौघेणानुबद्ध्यन्ते करिष्यति तदद्य कः}
{येन दत्तं महद् युद्धं गन्धर्वस्य महात्मनः} %4-22-28

\twolineshloka
{गोलभस्य महाबाहोर्दश वर्षाणि पञ्च च}
{नैव रात्रौ न दिवसे तद् युद्धमुपशाम्यति} %4-22-29

\threelineshloka
{ततः षोडशमे वर्षे गोलभो विनिपातितः}
{तं हत्वा दुर्विनीतं तु वाली दंष्ट्राकरालवान्}
{सर्वाभयङ्करोऽस्माकं कथमेष निपातितः} %4-22-30

\twolineshloka
{हते तु वीरे प्लवगाधिपे तदा प्लवङ्गमास्तत्र न शर्म लेभिरे}
{वनेचराः सिंहयुते महावने यथा हि गावो निहते गवां पतौ} %4-22-31

\twolineshloka
{ततस्तु तारा व्यसनार्णवप्लुता मृतस्य भर्तुर्वदनं समीक्ष्य सा}
{जगाम भूमिं परिरभ्य वालिनं महाद्रुमं छिन्नमिवाश्रिता लता} %4-22-32


॥इत्यार्षे श्रीमद्रामायणे वाल्मीकीये आदिकाव्ये किष्किन्धाकाण्डे वाल्यनुशासनम् नाम द्वाविंशः सर्गः ॥४-२२॥
