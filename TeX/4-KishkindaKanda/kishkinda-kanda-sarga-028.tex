\sect{अष्टाविंशः सर्गः — प्रावृडुज्जृम्भणम्}

\twolineshloka
{स तदा वालिनं हत्वा सुग्रीवमभिषिच्य च}
{वसन् माल्यवतः पृष्ठे रामो लक्ष्मणमब्रवीत्} %4-28-1

\twolineshloka
{अयं स कालः सम्प्राप्तः समयोऽद्य जलागमः}
{सम्पश्य त्वं नभो मेघैः संवृतं गिरिसन्निभैः} %4-28-2

\twolineshloka
{नवमासधृतं गर्भं भास्करस्य गभस्तिभिः}
{पीत्वा रसं समुद्राणां द्यौः प्रसूते रसायनम्} %4-28-3

\twolineshloka
{शक्यमम्बरमारुह्य मेघसोपानपङ्क्तिभिः}
{कुटजार्जुनमालाभिरलङ्कर्तुं दिवाकरः} %4-28-4

\twolineshloka
{सन्ध्यारागोत्थितैस्ताम्रैरन्तेष्वपि च पाण्डुभिः}
{स्निग्धैरभ्रपटच्छेदैर्बद्धव्रणमिवाम्बरम्} %4-28-5

\twolineshloka
{मन्दमारुतिनिःश्वासं सन्ध्याचन्दनरञ्जितम्}
{आपाण्डुजलदं भाति कामातुरमिवाम्बरम्} %4-28-6

\twolineshloka
{एषा घर्मपरिक्लिष्टा नववारिपरिप्लुता}
{सीतेव शोकसन्तप्ता मही बाष्पं विमुञ्चति} %4-28-7

\twolineshloka
{मेघोदरविनिर्मुक्ताः कर्पूरदलशीतलाः}
{शक्यमञ्जलिभिः पातुं वाताः केतकगन्धिनः} %4-28-8

\twolineshloka
{एष फुल्लार्जुनः शैलः केतकैरभिवासितः}
{सुग्रीव इव शान्तारिर्धाराभिरभिषिच्यते} %4-28-9

\twolineshloka
{मेघकृष्णाजिनधरा धारायज्ञोपवीतिनः}
{मारुतापूरितगुहाः प्राधीता इव पर्वताः} %4-28-10

\twolineshloka
{कशाभिरिव हैमीभिर्विद्युद्भिरभिताडितम्}
{अन्तःस्तनितनिर्घोषं सवेदनमिवाम्बरम्} %4-28-11

\twolineshloka
{नीलमेघाश्रिता विद्युत् स्फुरन्ती प्रतिभाति मे}
{स्फुरन्ती रावणस्याङ्के वैदेहीव तपस्विनी} %4-28-12

\twolineshloka
{इमास्ता मन्मथवतां हिताः प्रतिहता दिशः}
{अनुलिप्ता इव घनैर्नष्टग्रहनिशाकराः} %4-28-13

\threelineshloka
{क्वचिद् बाष्पाभिसंरुद्धान् वर्षागमसमुत्सुकान्}
{कुटजान् पश्य सौमित्रे पुष्पितान् गिरिसानुषु}
{मम शोकाभिभूतस्य कामसन्दीपनान् स्थितान्} %4-28-14

\twolineshloka
{रजः प्रशान्तं सहिमोऽद्य वायुर्निदाघदोषप्रसराः प्रशान्ताः}
{स्थिता हि यात्रा वसुधाधिपानां प्रवासिनो यान्ति नराः स्वदेशान्} %4-28-15

\twolineshloka
{सम्प्रस्थिता मानसवासलुब्धाः प्रियान्विताः सम्प्रति चक्रवाकाः}
{अभीक्ष्णवर्षोदकविक्षतेषु यानानि मार्गेषु न सम्पतन्ति} %4-28-16

\twolineshloka
{क्वचित् प्रकाशं क्वचिदप्रकाशं नभः प्रकीर्णाम्बुधरं विभाति}
{क्वचित् क्वचित् पर्वतसन्निरुद्धं रूपं यथा शान्तमहार्णवस्य} %4-28-17

\twolineshloka
{व्यामिश्रितं सर्जकदम्बपुष्पैर्नवं जलं पर्वतधातुताम्रम्}
{मयूरकेकाभिरनुप्रयातं शैलापगाः शीघ्रतरं वहन्ति} %4-28-18

\twolineshloka
{रसाकुलं षट्पदसन्निकाशं प्रभुज्यते जम्बुफलं प्रकामम्}
{अनेकवर्णं पवनावधूतं भूमौ पतत्याम्रफलं विपक्वम्} %4-28-19

\twolineshloka
{विद्युत्पताकाः सबलाकमालाः शैलेन्द्रकूटाकृतिसन्निकाशाः}
{गर्जन्ति मेघाः समुदीर्णनादा मत्ता गजेन्द्रा इव संयुगस्थाः} %4-28-20

\twolineshloka
{वर्षोदकाप्यायितशाद्वलानि प्रवृत्तनृत्तोत्सवबर्हिणानि}
{वनानि निर्वृष्टबलाहकानि पश्यापराह्णेष्वधिकं विभान्ति} %4-28-21

\twolineshloka
{समुद्वहन्तः सलिलातिभारं बलाकिनो वारिधरा नदन्तः}
{महत्सु शृङ्गेषु महीधराणां विश्रम्य विश्रम्य पुनः प्रयान्ति} %4-28-22

\twolineshloka
{मेघाभिकामा परिसम्पतन्ती सम्मोदिता भाति बलाकपङ्क्तिः}
{वातावधूता वरपौण्डरीकी लम्बेव माला रुचिराम्बरस्य} %4-28-23

\twolineshloka
{बालेन्द्रगोपान्तरचित्रितेन विभाति भूमिर्नवशाद्वलेन}
{गात्रानुपृक्तेन शुकप्रभेण नारीव लाक्षोक्षितकम्बलेन} %4-28-24

\twolineshloka
{निद्रा शनैः केशवमभ्युपैति द्रुतं नदी सागरमभ्युपैति}
{हृष्टा बलाका घनमभ्युपैति कान्ता सकामा प्रियमभ्युपैति} %4-28-25

\twolineshloka
{जाता वनान्ताः शिखिसुप्रनृत्ता जाताः कदम्बाः सकदम्बशाखाः}
{जाता वृषा गोषु समानकामा जाता मही सस्यवनाभिरामा} %4-28-26

\twolineshloka
{वहन्ति वर्षन्ति नदन्ति भान्ति ध्यायन्ति नृत्यन्ति समाश्वसन्ति}
{नद्यो घना मत्तगजा वनान्ताः प्रियाविहीनाः शिखिनः प्लवङ्गमाः} %4-28-27

\twolineshloka
{प्रहर्षिताः केतकिपुष्पगन्धमाघ्राय मत्ता वननिर्झरेषु}
{प्रपातशब्दाकुलिता गजेन्द्राः सार्धं मयूरैः समदा नदन्ति} %4-28-28

\twolineshloka
{धारानिपातैरभिहन्यमानाः कदम्बशाखासु विलम्बमानाः}
{क्षणार्जितं पुष्परसावगाढं शनैर्मदं षट्चरणास्त्यजन्ति} %4-28-29

\twolineshloka
{अङ्गारचूर्णोत्करसन्निकाशैः फलैः सुपर्याप्तरसैः समृद्धैः}
{जम्बूद्रुमाणां प्रविभान्ति शाखा निपीयमाना इव षट्पदौघैः} %4-28-30

\twolineshloka
{तडित्पताकाभिरलङ्कृतानामुदीर्णगम्भीरमहारवाणाम्}
{विभान्ति रूपाणि बलाहकानां रणोत्सुकानामिव वारणानाम्} %4-28-31

\twolineshloka
{मार्गानुगः शैलवनानुसारी सम्प्रस्थितो मेघरवं निशम्य}
{युद्धाभिकामः प्रतिनादशङ्की मत्तो गजेन्द्रः प्रतिसन्निवृत्तः} %4-28-32

\twolineshloka
{क्वचित् प्रगीता इव षट्पदौघैः क्वचित् प्रनृत्ता इव नीलकण्ठैः}
{क्वचित् प्रमत्ता इव वारणेन्द्रैर्विभान्त्यनेकाश्रयिणो वनान्ताः} %4-28-33

\twolineshloka
{कदम्बसर्जार्जुनकन्दलाढ्या वनान्तभूमिर्मधुवारिपूर्णा}
{मयूरमत्ताभिरुतप्रनृत्तैरापानभूमिप्रतिमा विभाति} %4-28-34

\twolineshloka
{मुक्तासमाभं सलिलं पतद् वै सुनिर्मलं पत्रपुटेषु लग्नम्}
{हृष्टा विवर्णच्छदना विहङ्गाः सुरेन्द्रदत्तं तृषिताः पिबन्ति} %4-28-35

\twolineshloka
{षट्पादतन्त्रीमधुराभिधानं प्लवङ्गमोदीरितकण्ठतालम्}
{आविष्कृतं मेघमृदङ्गनादैर्वनेषु सङ्गीतमिव प्रवृत्तम्} %4-28-36

\twolineshloka
{क्वचित् प्रनृत्तैः क्वचिदुन्नदद्भिः क्वचिच्च वृक्षाग्रनिषण्णकायैः}
{व्यालम्बबर्हाभरणैर्मयूरैर्वनेषु सङ्गीतमिव प्रवृत्तम्} %4-28-37

\twolineshloka
{स्वनैर्घनानां प्लवगाः प्रबुद्धा विहाय निद्रां चिरसन्निरुद्धाम्}
{अनेकरूपाकृतिवर्णनादा नवाम्बुधाराभिहता नदन्ति} %4-28-38

\twolineshloka
{नद्यः समुद्वाहितचक्रवाकास्तटानि शीर्णान्यपवाहयित्वा}
{दृप्ता नवप्रावृतपूर्णभोगादृतं स्वभर्तारमुपोपयान्ति} %4-28-39

\twolineshloka
{नीलेषु नीला नववारिपूर्णा मेघेषु मेघाः प्रतिभान्ति सक्ताः}
{दवाग्निदग्धेषु दवाग्निदग्धाः शैलेषु शैला इव बद्धमूलाः} %4-28-40

\twolineshloka
{प्रमत्तसन्नादितबर्हिणानि सशक्रगोपाकुलशाद्वलानि}
{चरन्ति नीपार्जुनवासितानि गजाः सुरम्याणि वनान्तराणि} %4-28-41

\twolineshloka
{नवाम्बुधाराहतकेसराणि द्रुतं परित्यज्य सरोरुहाणि}
{कदम्बपुष्पाणि सकेसराणि नवानि हृष्टा भ्रमराः पिबन्ति} %4-28-42

\twolineshloka
{मत्ता गजेन्द्रा मुदिता गवेन्द्रा वनेषु विक्रान्ततरा मृगेन्द्राः}
{रम्या नगेन्द्राः निभृता नरेन्द्राः प्रक्रीडितो वारिधरैः सुरेन्द्रः} %4-28-43

\twolineshloka
{मेघाः समुद्भूतसमुद्रनादा महाजलौघैर्गगनावलम्बाः}
{नदीस्तटाकानि सरांसि वापीर्महीं च कृत्स्नामपवाहयन्ति} %4-28-44

\twolineshloka
{वर्षप्रवेगा विपुलाः पतन्ति प्रवान्ति वाताः समुदीर्णवेगाः}
{प्रणष्टकूलाः प्रवहन्ति शीघ्रं नद्यो जलं विप्रतिपन्नमार्गाः} %4-28-45

\twolineshloka
{नरैर्नरेन्द्रा इव पर्वतेन्द्राः सुरेन्द्रदत्तैः पवनोपनीतैः}
{घनाम्बुकुम्भैरभिषिच्यमाना रूपं श्रियं स्वामिव दर्शयन्ति} %4-28-46

\twolineshloka
{घनोपगूढं गगनं न तारा न भास्करो दर्शनमभ्युपैति}
{नवैर्जलौघैर्धरणी वितृप्ता तमोविलिप्ता न दिशः प्रकाशाः} %4-28-47

\twolineshloka
{महान्ति कूटानि महीधराणां धाराविधौतान्यधिकं विभान्ति}
{महाप्रमाणैर्विपुलैः प्रपातैर्मुक्ताकलापैरिव लम्बमानैः} %4-28-48

\twolineshloka
{शैलोपलप्रस्खलमानवेगाः शैलोत्तमानां विपुलाः प्रपाताः}
{गुहासु सन्नादितबर्हिणासु हारा विकीर्यन्त इवावभान्ति} %4-28-49

\twolineshloka
{शीघ्रप्रवेगा विपुलाः प्रपाता निर्धौतशृङ्गोपतला गिरीणाम्}
{मुक्ताकलापप्रतिमाः पतन्तो महागुहोत्सङ्गतलैर्ध्रियन्ते} %4-28-50

\twolineshloka
{सुरतामर्दविच्छिन्नाः स्वर्गस्त्रीहारमौक्तिकाः}
{पतन्ति चातुला दिक्षु तोयधाराः समन्ततः} %4-28-51

\twolineshloka
{विलीयमानैर्विहगैर्निमीलद्भिश्च पङ्कजैः}
{विकसन्त्या च मालत्या गतोऽस्तं ज्ञायते रविः} %4-28-52

\twolineshloka
{वृत्ता यात्रा नरेन्द्राणां सेना पथ्येव वर्तते}
{वैराणि चैव मार्गाश्च सलिलेन समीकृताः} %4-28-53

\twolineshloka
{मासि प्रौष्ठपदे ब्रह्म ब्राह्मणानां विवक्षताम्}
{अयमध्यायसमयः सामगानामुपस्थितः} %4-28-54

\twolineshloka
{निवृत्तकर्मायतनो नूनं सञ्चितसञ्चयः}
{आषाढीमभ्युपगतो भरतः कोसलाधिपः} %4-28-55

\twolineshloka
{नूनमापूर्यमाणायाः सरय्वा वर्धते रयः}
{मां समीक्ष्य समायान्तमयोध्याया इव स्वनः} %4-28-56

\twolineshloka
{इमाः स्फीतगुणा वर्षाः सुग्रीवः सुखमश्नुते}
{विजितारिः सदारश्च राज्ये महति च स्थितः} %4-28-57

\twolineshloka
{अहं तु हृतदारश्च राज्याच्च महतश्च्युतः}
{नदीकूलमिव क्लिन्नमवसीदामि लक्ष्मण} %4-28-58

\twolineshloka
{शोकश्च मम विस्तीर्णो वर्षाश्च भृशदुर्गमाः}
{रावणश्च महाञ्छत्रुरपारः प्रतिभाति मे} %4-28-59

\twolineshloka
{अयात्रां चैव दृष्ट्वेमां मार्गांश्च भृशदुर्गमान्}
{प्रणते चैव सुग्रीवे न मया किञ्चिदीरितम्} %4-28-60

\twolineshloka
{अपि चापि परिक्लिष्टं चिराद् दारैः समागतम्}
{आत्मकार्यगरीयस्त्वाद् वक्तुं नेच्छामि वानरम्} %4-28-61

\twolineshloka
{स्वयमेव हि विश्रम्य ज्ञात्वा कालमुपागतम्}
{उपकारं च सुग्रीवो वेत्स्यते नात्र संशयः} %4-28-62

\twolineshloka
{तस्मात् कालप्रतीक्षोऽहं स्थितोऽस्मि शुभलक्षण}
{सुग्रीवस्य नदीनां च प्रसादमभिकाङ्क्षयन्} %4-28-63

\twolineshloka
{उपकारेण वीरो हि प्रतीकारेण युज्यते}
{अकृतज्ञोऽप्रतिकृतो हन्ति सत्त्ववतां मनः} %4-28-64

\twolineshloka
{अथैवमुक्तः प्रणिधाय लक्ष्मणः कृताञ्जलिस्तत् प्रतिपूज्य भाषितम्}
{उवाच रामं स्वभिरामदर्शनं प्रदर्शयन् दर्शनमात्मनः शुभम्} %4-28-65

\twolineshloka
{यदुक्तमेतत् तव सर्वमीप्सितं नरेन्द्र कर्ता नचिराद्धरीश्वरः}
{शरत्प्रतीक्षः क्षमतामिमं भवान् जलप्रपातं रिपुनिग्रहे धृतः} %4-28-66


॥इत्यार्षे श्रीमद्रामायणे वाल्मीकीये आदिकाव्ये किष्किन्धाकाण्डे प्रावृडुज्जृम्भणम् नाम अष्टाविंशः सर्गः ॥४-२८॥
