\sect{एकोनविंशः सर्गः — तारागमनम्}

\twolineshloka
{स वानरमहाराजः शयानः शरपीडितः}
{प्रत्युक्तो हेतुमद्वाक्यैर्नोत्तरं प्रत्यपद्यत} %4-19-1

\twolineshloka
{अश्मभिः परिभिन्नाङ्गः पादपैराहतो भृशम्}
{रामबाणेन चाक्रान्तो जीवितान्ते मुमोह सः} %4-19-2

\twolineshloka
{तं भार्या बाणमोक्षेण रामदत्तेन संयुगे}
{हतं प्लवगशार्दूलं तारा शुश्राव वालिनम्} %4-19-3

\twolineshloka
{सा सपुत्राप्रियं श्रुत्वा वधं भर्तुः सुदारुणम्}
{निष्पपात भृशं तस्मादुद्विग्ना गिरिकन्दरात्} %4-19-4

\twolineshloka
{ये त्वङ्गदपरीवारा वानरा हि महाबलाः}
{ते सकार्मुकमालोक्य रामं त्रस्ताः प्रदुद्रुवुः} %4-19-5

\twolineshloka
{सा ददर्श ततस्त्रस्तान् हरीनापततो द्रुतम्}
{यूथादेव परिभ्रष्टान् मृगान् निहतयूथपान्} %4-19-6

\twolineshloka
{तानुवाच समासाद्य दुःखितान् दुःखिता सती}
{रामवित्रासितान् सर्वाननुबद्धानिवेषुभिः} %4-19-7

\twolineshloka
{वानरा राजसिंहस्य यस्य यूयं पुरःसराः}
{तं विहाय सुवित्रस्ताः कस्माद् द्रवत दुर्गताः} %4-19-8

\twolineshloka
{राज्यहेतोः स चेद् भ्राता भ्रात्रा क्रूरेण पातितः}
{रामेण प्रहितैर्दूरान्मार्गणैर्दूरपातिभिः} %4-19-9

\twolineshloka
{कपिपत्न्या वचः श्रुत्वा कपयः कामरूपिणः}
{प्राप्तकालमविश्लिष्टमूचुर्वचनमङ्गनाम्} %4-19-10

\twolineshloka
{जीवपुत्रे निवर्तस्व पुत्रं रक्षस्व चाङ्गदम्}
{अन्तको रामरूपेण हत्वा नयति वालिनम्} %4-19-11

\twolineshloka
{क्षिप्तान् वृक्षान् समाविध्य विपुलाश्च तथा शिलाः}
{वाली वज्रसमैर्बाणैर्वज्रेणेव निपातितः} %4-19-12

\twolineshloka
{अभिभूतमिदं सर्वं विद्रुतं वानरं बलम्}
{अस्मिन् प्लवगशार्दूले हते शक्रसमप्रभे} %4-19-13

\twolineshloka
{रक्ष्यतां नगरी शूरैरङ्गदश्चाभिषिच्यताम्}
{पदस्थं वालिनः पुत्रं भजिष्यन्ति प्लवंगमाः} %4-19-14

\twolineshloka
{अथवारुचितं स्थानमिह ते रुचिरानने}
{आविशन्ति च दुर्गाणि क्षिप्रमद्यैव वानराः} %4-19-15

\twolineshloka
{अभार्याः सहभार्याश्च सन्त्यत्र वनचारिणः}
{लुब्धेभ्यो विप्रलब्धेभ्यस्तेभ्यो नः सुमहद्भयम्} %4-19-16

\twolineshloka
{अल्पान्तरगतानां तु श्रुत्वा वचनमङ्गना}
{आत्मनः प्रतिरूपं सा बभाषे चारुहासिनी} %4-19-17

\twolineshloka
{पुत्रेण मम किं कार्यं राज्येनापि किमात्मना}
{कपिसिंहे महाभागे तस्मिन् भर्तरि नश्यति} %4-19-18

\twolineshloka
{पादमूलं गमिष्यामि तस्यैवाहं महात्मनः}
{योऽसौ रामप्रयुक्तेन शरेण विनिपातितः} %4-19-19

\twolineshloka
{एवमुक्त्वा प्रदुद्राव रुदती शोकमूर्च्छिता}
{शिरश्चोरश्च बाहुभ्यां दुःखेन समभिघ्नती} %4-19-20

\twolineshloka
{सा व्रजन्ती ददर्शाथ पतिं निपतितं भुवि}
{हन्तारं दानवेन्द्राणां समरेष्वनिवर्तिनाम्} %4-19-21

\twolineshloka
{क्षेप्तारं पर्वतेन्द्राणां वज्राणामिव वासवम्}
{महावातसमाविष्टं महामेघौघनिःस्वनम्} %4-19-22

\threelineshloka
{शक्रतुल्यपराक्रान्तं वृष्ट्वेवोपरतं घनम्}
{नर्दन्तं नर्दतां भीमं शूरं शूरेण पातितम्}
{शार्दूलेनामिषस्यार्थे मृगराजमिवाहतम्} %4-19-23

\twolineshloka
{अर्चितं सर्वलोकस्य सपताकं सवेदिकम्}
{नागहेतोः सुपर्णेन चैत्यमुन्मथितं यथा} %4-19-24

\twolineshloka
{अवष्टभ्यावतिष्ठन्तं ददर्श धनुरूर्जितम्}
{रामं रामानुजं चैव भर्तुश्चैव तथानुजम्} %4-19-25

\twolineshloka
{तानतीत्य समासाद्य भर्तारं निहतं रणे}
{समीक्ष्य व्यथिता भूमौ सम्भ्रान्ता निपपात ह} %4-19-26

\twolineshloka
{सुप्तेव पुनरुत्थाय आर्यपुत्रेति वादिनी}
{रुरोद सा पतिं दृष्ट्वा संवीतं मृत्युदामभिः} %4-19-27

\twolineshloka
{तामवेक्ष्य तु सुग्रीवः क्रोशन्तीं कुररीमिव}
{विषादमगमत् कष्टं दृष्ट्वा चाङ्गदमागतम्} %4-19-28


॥इत्यार्षे श्रीमद्रामायणे वाल्मीकीये आदिकाव्ये किष्किन्धाकाण्डे तारागमनम् नाम एकोनविंशः सर्गः ॥४-१९॥
