\sect{चतुस्त्रिंशः सर्गः — सुग्रीवतर्जनम्}

\twolineshloka
{तमप्रतिहतं क्रुद्धं प्रविष्टं पुरुषर्षभम्}
{सुग्रीवो लक्ष्मणं दृष्ट्वा बभूव व्यथितेन्द्रियः} %4-34-1

\twolineshloka
{क्रुद्धं निःश्वसमानं तं प्रदीप्तमिव तेजसा}
{भ्रातुर्व्यसनसंतप्तं दृष्ट्वा दशरथात्मजम्} %4-34-2

\twolineshloka
{उत्पपात हरिश्रेष्ठो हित्वा सौवर्णमासनम्}
{महान् महेन्द्रस्य यथा स्वलंकृत इव ध्वजः} %4-34-3

\twolineshloka
{उत्पतन्तमनूत्पेतू रुमाप्रभृतयः स्त्रियः}
{सुग्रीवं गगने पूर्णं चन्द्रं तारागणा इव} %4-34-4

\twolineshloka
{संरक्तनयनः श्रीमान् संचचार कृताञ्जलिः}
{बभूवावस्थितस्तत्र कल्पवृक्षो महानिव} %4-34-5

\twolineshloka
{रुमाद्वितीयं सुग्रीवं नारीमध्यगतं स्थितम्}
{अब्रवील्लक्ष्मणः क्रुद्धः सतारं शशिनं यथा} %4-34-6

\twolineshloka
{सत्त्वाभिजनसम्पन्नः सानुक्रोशो जितेन्द्रियः}
{कृतज्ञः सत्यवादी च राजा लोके महीयते} %4-34-7

\twolineshloka
{यस्तु राजा स्थितोऽधर्मे मित्राणामुपकारिणाम्}
{मिथ्या प्रतिज्ञां कुरुते को नृशंसतरस्ततः} %4-34-8

\twolineshloka
{शतमश्वानृते हन्ति सहस्रं तु गवानृते}
{आत्मानं स्वजनं हन्ति पुरुषः पुरुषानृते} %4-34-9

\twolineshloka
{पूर्वं कृतार्थो मित्राणां न तत्प्रतिकरोति यः}
{कृतघ्नः सर्वभूतानां स वध्यः प्लवगेश्वर} %4-34-10

\twolineshloka
{गीतोऽयं ब्रह्मणा श्लोकः सर्वलोकनमस्कृतः}
{दृष्ट्वा कृतघ्नं क्रुद्धेन तन्निबोध प्लवंगम} %4-34-11

\twolineshloka
{गोघ्ने चैव सुरापे च चौरे भग्नव्रते तथा}
{निष्कृतिर्विहिता सद्भिः कृतघ्ने नास्ति निष्कृतिः} %4-34-12

\twolineshloka
{अनार्यस्त्वं कृतघ्नश्च मिथ्यावादी च वानर}
{पूर्वं कृतार्थो रामस्य न तत्प्रतिकरोषि यत्} %4-34-13

\twolineshloka
{ननु नाम कृतार्थेन त्वया रामस्य वानर}
{सीताया मार्गणे यत्नः कर्तव्यः कृतमिच्छता} %4-34-14

\twolineshloka
{स त्वं ग्राम्येषु भोगेषु सक्तो मिथ्याप्रतिश्रवः}
{न त्वां रामो विजानीते सर्पं मण्डूकराविणम्} %4-34-15

\twolineshloka
{महाभागेन रामेण पापः करुणवेदिना}
{हरीणां प्रापितो राज्यं त्वं दुरात्मा महात्मना} %4-34-16

\twolineshloka
{कृतं चेन्नातिजानीषे राघवस्य महात्मनः}
{सद्यस्त्वं निशितैर्बाणैर्हतो द्रक्ष्यसि वालिनम्} %4-34-17

\twolineshloka
{न स संकुचितः पन्था येन वाली हतो गतः}
{समये तिष्ठ सुग्रीव मा वालिपथमन्वगाः} %4-34-18

\twolineshloka
{न नूनमिक्ष्वाकुवरस्य कार्मुकाच्छरांश्च तान् पश्यसि वज्रसंनिभान्}
{ततः सुखं नाम विषेवसे सुखी न रामकार्यं मनसाप्यवेक्षसे} %4-34-19


॥इत्यार्षे श्रीमद्रामायणे वाल्मीकीये आदिकाव्ये किष्किन्धाकाण्डे सुग्रीवतर्जनम् नाम चतुस्त्रिंशः सर्गः ॥४-३४॥
