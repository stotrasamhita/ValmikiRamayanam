\sect{अष्टपञ्चाशः सर्गः — सीताप्रवृच्युपलम्भः}

\twolineshloka
{इत्युक्तः करुणं वाक्यं वानरैस्त्यक्तजीवितैः}
{सबाष्पो वानरान् गृध्रः प्रत्युवाच महास्वनः} %4-58-1

\twolineshloka
{यवीयान् स मम भ्राता जटायुर्नाम वानराः}
{यमाख्यात हतं युद्धे रावणेन बलीयसा} %4-58-2

\twolineshloka
{वृद्धभावादपक्षत्वाच्छृण्वंस्तदपि मर्षये}
{नहि मे शक्तिरस्त्यद्य भ्रातुर्वैरविमोक्षणे} %4-58-3

\twolineshloka
{पुरा वृत्रवधे वृत्ते स चाहं च जयैषिणौ}
{आदित्यमुपयातौ स्वो ज्वलन्तं रश्मिमालिनम्} %4-58-4

\twolineshloka
{आवृत्याकाशमार्गेण जवेन स्वर्गतौ भृशम्}
{मध्यं प्राप्ते तु सूर्ये तु जटायुरवसीदति} %4-58-5

\twolineshloka
{तमहं भ्रातरं दृष्ट्वा सूर्यरश्मिभिरर्दितम्}
{पक्षाभ्यां छादयामास स्नेहात् परमविह्वलम्} %4-58-6

\twolineshloka
{निर्दग्धपत्रः पतितो विन्ध्येऽहं वानरर्षभाः}
{अहमस्मिन् वसन् भ्रातुः प्रवृत्तिं नोपलक्षये} %4-58-7

\twolineshloka
{जटायुषस्त्वेवमुक्तो भ्रात्रा सम्पातिना तदा}
{युवराजो महाप्रज्ञः प्रत्युवाचाङ्गदस्तदा} %4-58-8

\twolineshloka
{जटायुषो यदि भ्राता श्रुतं ते गदितं मया}
{आख्याहि यदि जानासि निलयं तस्य रक्षसः} %4-58-9

\twolineshloka
{अदीर्घदर्शिनं तं वै रावणं राक्षसाधमम्}
{अन्तिके यदि वा दूरे यदि जानासि शंस नः} %4-58-10

\twolineshloka
{ततोऽब्रवीन्महातेजा भ्राता ज्येष्ठो जटायुषः}
{आत्मानुरूपं वचनं वानरान् सम्प्रहर्षयन्} %4-58-11

\twolineshloka
{निर्दग्धपक्षो गृध्रोऽहं गतवीर्यः प्लवङ्गमाः}
{वाङ्मात्रेण तु रामस्य करिष्ये साह्यमुत्तमम्} %4-58-12

\twolineshloka
{जानामि वारुणाँल्लोकान् विष्णोस्त्रैविक्रमानपि}
{देवासुरविमर्दांश्च ह्यमृतस्य विमन्थनम्} %4-58-13

\twolineshloka
{रामस्य यदिदं कार्यं कर्तव्यं प्रथमं मया}
{जरया च हृतं तेजः प्राणाश्च शिथिला मम} %4-58-14

\twolineshloka
{तरुणी रूपसम्पन्ना सर्वाभरणभूषिता}
{ह्रियमाणा मया दृष्टा रावणेन दुरात्मना} %4-58-15

\twolineshloka
{क्रोशन्ती रामरामेति लक्ष्मणेति च भामिनी}
{भूषणान्यपविध्यन्ती गात्राणि च विधुन्वती} %4-58-16

\twolineshloka
{सूर्यप्रभेव शैलाग्रे तस्याः कौशेयमुत्तमम्}
{असिते राक्षसे भाति यथा वा तडिदम्बुदे} %4-58-17

\twolineshloka
{तां तु सीतामहं मन्ये रामस्य परिकीर्तनात्}
{श्रूयतां मे कथयतो निलयं तस्य रक्षसः} %4-58-18

\twolineshloka
{पुत्रो विश्रवसः साक्षाद् भ्राता वैश्रवणस्य च}
{अध्यास्ते नगरीं लङ्कां रावणो नाम राक्षसः} %4-58-19

\twolineshloka
{इतो द्वीपे समुद्रस्य सम्पूर्णे शतयोजने}
{तस्मिँल्लङ्का पुरी रम्या निर्मिता विश्वकर्मणा} %4-58-20

\twolineshloka
{जाम्बूनदमयैर्द्वारैश्चित्रैः काञ्चनवेदिकैः}
{प्रासादैर्हेमवर्णैश्च महद्भिः सुसमाकृता} %4-58-21

\twolineshloka
{प्राकारेणार्कवर्णेन महता च समन्विता}
{तस्यां वसति वैदेही दीना कौशेयवासिनी} %4-58-22

\twolineshloka
{रावणान्तःपुरे रुद्धा राक्षसीभिः सुरक्षिता}
{जनकस्यात्मजां राज्ञस्तस्यां द्रक्ष्यथ मैथिलीम्} %4-58-23

\twolineshloka
{लङ्कायामथ गुप्तायां सागरेण समन्ततः}
{सम्प्राप्य सागरस्यान्तं सम्पूर्णं शतयोजनम्} %4-58-24

\twolineshloka
{आसाद्य दक्षिणं तीरं ततो द्रक्ष्यथ रावणम्}
{तत्रैव त्वरिताः क्षिप्रं विक्रमध्वं प्लवङ्गमाः} %4-58-25

\twolineshloka
{ज्ञानेन खलु पश्यामि दृष्ट्वा प्रत्यागमिष्यथ}
{आद्यः पन्थाः कुलिङ्गानां ये चान्ये धान्यजीविनः} %4-58-26

\twolineshloka
{द्वितीयो बलिभोजानां ये च वृक्षफलाशनाः}
{भासास्तृतीयं गच्छन्ति क्रौञ्चाश्च कुररैः सह} %4-58-27

\twolineshloka
{श्येनाश्चतुर्थं गच्छन्ति गृध्रा गच्छन्ति पञ्चमम्}
{बलवीर्योपपन्नानां रूपयौवनशालिनाम्} %4-58-28

\twolineshloka
{षष्ठस्तु पन्था हंसानां वैनतेयगतिः परा}
{वैनतेयाच्च नो जन्म सर्वेषां वानरर्षभाः} %4-58-29

\twolineshloka
{गर्हितं तु कृतं कर्म येन स्म पिशिताशिनः}
{प्रतिकार्यं च मे तस्य वैरं भ्रातृकृतं भवेत्} %4-58-30

\twolineshloka
{इहस्थोऽहं प्रपश्यामि रावणं जानकीं तथा}
{अस्माकमपि सौपर्णं दिव्यं चक्षुर्बलं तथा} %4-58-31

\twolineshloka
{तस्मादाहारवीर्येण निसर्गेण च वानराः}
{आयोजनशतात् साग्राद् वयं पश्याम नित्यशः} %4-58-32

\twolineshloka
{अस्माकं विहिता वृत्तिर्निसर्गेण च दूरतः}
{विहिता वृक्षमूले तु वृत्तिश्चरणयोधिनाम्} %4-58-33

\twolineshloka
{उपायो दृश्यतां कश्चिल्लङ्घने लवणाम्भसः}
{अभिगम्य तु वैदेहीं समृद्धार्था गमिष्यथ} %4-58-34

\twolineshloka
{समुद्रं नेतुमिच्छामि भवद्भिर्वरुणालयम्}
{प्रदास्याम्युदकं भ्रातुः स्वर्गतस्य महात्मनः} %4-58-35

\twolineshloka
{ततो नीत्वा तु तं देशं तीरे नदनदीपतेः}
{निर्दग्धपक्षं सम्पातिं वानराः सुमहौजसः} %4-58-36

\twolineshloka
{तं पुनः प्रापयित्वा च तं देशं पतगेश्वरम्}
{बभूवुर्वानरा हृष्टाः प्रवृत्तिमुपलभ्य ते} %4-58-37


॥इत्यार्षे श्रीमद्रामायणे वाल्मीकीये आदिकाव्ये किष्किन्धाकाण्डे सीताप्रवृच्युपलम्भः नाम अष्टपञ्चाशः सर्गः ॥४-५८॥
