\sect{एकत्रिंशः सर्गः — लक्ष्मणक्रोधः}

\twolineshloka
{स कामिनं दीनमदीनसत्त्वं शोकाभिपन्नं समुदीर्णकोपम्}
{नरेन्द्रसूनुर्नरदेवपुत्रं रामानुजः पूर्वजमित्युवाच} %4-31-1

\twolineshloka
{न वानरः स्थास्यति साधुवृत्ते न मन्यते कर्मफलानुषङ्गान्}
{न भोक्ष्यते वानरराज्यलक्ष्मीं तथा हि नातिक्रमतेऽस्य बुद्धिः} %4-31-2

\twolineshloka
{मतिक्षयाद् ग्राम्यसुखेषु सक्तस्तव प्रसादात् प्रतिकारबुद्धिः}
{हतोऽग्रजं पश्यतु वीरवालिनं न राज्यमेवं विगुणस्य देयम्} %4-31-3

\twolineshloka
{न धारये कोपमुदीर्णवेगं निहन्मि सुग्रीवमसत्यमद्य}
{हरिप्रवीरैः सह वालिपुत्रो नरेन्द्रपुत्र्या विचयं करोतु} %4-31-4

\twolineshloka
{तमात्तबाणासनमुत्पतन्तं निवेदितार्थं रणचण्डकोपम्}
{उवाच रामः परवीरहन्ता स्ववीक्षितं सानुनयं च वाक्यम्} %4-31-5

\twolineshloka
{नहि वै त्वद्विधो लोके पापमेवं समाचरेत्}
{कोपमार्येण यो हन्ति स वीरः पुरुषोत्तमः} %4-31-6

\twolineshloka
{नेदमत्र त्वया ग्राह्यं साधुवृत्तेन लक्ष्मण}
{तां प्रीतिमनुवर्तस्व पूर्ववृत्तं च संगतम्} %4-31-7

\twolineshloka
{सामोपहितया वाचा रूक्षाणि परिवर्जयन्}
{वक्तुमर्हसि सुग्रीवं व्यतीतं कालपर्यये} %4-31-8

\twolineshloka
{सोऽग्रजेनानुशिष्टार्थो यथावत् पुरुषर्षभः}
{प्रविवेश पुरीं वीरो लक्ष्मणः परवीरहा} %4-31-9

\twolineshloka
{ततः शुभमतिः प्राज्ञो भ्रातुः प्रियहिते रतः}
{लक्ष्मणः प्रतिसंरब्धो जगाम भवनं कपेः} %4-31-10

\twolineshloka
{शक्रबाणासनप्रख्यं धनुः कालान्तकोपमम्}
{प्रगृह्य गिरिशृङ्गाभं मन्दरः सानुमानिव} %4-31-11

\twolineshloka
{यथोक्तकारी वचनमुत्तरं चैव सोत्तरम्}
{बृहस्पतिसमो बुद्ध्या मत्वा रामानुजस्तदा} %4-31-12

\twolineshloka
{कामक्रोधसमुत्थेन भ्रातुः क्रोधाग्निना वृतः}
{प्रभञ्जन इवाप्रीतः प्रययौ लक्ष्मणस्ततः} %4-31-13

\twolineshloka
{सालतालाश्वकर्णांश्च तरसा पातयन् बलात्}
{पर्यस्यन् गिरिकूटानि द्रुमानन्यांश्च वेगितः} %4-31-14

\twolineshloka
{शिलाश्च शकलीकुर्वन् पद्भ्यां गज इवाशुगः}
{दूरमेकपदं त्यक्त्वा ययौ कार्यवशाद् द्रुतम्} %4-31-15

\twolineshloka
{तामपश्यद् बलाकीर्णां हरिराजमहापुरीम्}
{दुर्गामिक्ष्वाकुशार्दूलः किष्किन्धां गिरिसंकटे} %4-31-16

\twolineshloka
{रोषात् प्रस्फुरमाणोष्ठः सुग्रीवं प्रति लक्ष्मणः}
{ददर्श वानरान् भीमान् किष्किन्धायां बहिश्चरान्} %4-31-17

\threelineshloka
{तं दृष्ट्वा वानराः सर्वे लक्ष्मणं पुरुषर्षभम्}
{शैलशृङ्गाणि शतशः प्रवृद्धांश्च महीरुहान्}
{जगृहुः कुञ्जरप्रख्या वानराः पर्वतान्तरे} %4-31-18

\twolineshloka
{तान् गृहीतप्रहरणान् सर्वान् दृष्ट्वा तु लक्ष्मणः}
{बभूव द्विगुणं क्रुद्धो बह्विन्धन इवानलः} %4-31-19

\twolineshloka
{तं ते भयपरीताङ्गा क्षुब्धं दृष्ट्वा प्लवंगमाः}
{कालमृत्युयुगान्ताभं शतशो विद्रुता दिशः} %4-31-20

\twolineshloka
{ततः सुग्रीवभवनं प्रविश्य हरिपुंगवाः}
{क्रोधमागमनं चैव लक्ष्मणस्य न्यवेदयन्} %4-31-21

\twolineshloka
{तारया सहितः कामी सक्तः कपिवृषस्तदा}
{न तेषां कपिसिंहानां शुश्राव वचनं तदा} %4-31-22

\twolineshloka
{ततः सचिवसंदिष्टा हरयो रोमहर्षणाः}
{गिरिकुञ्जरमेघाभा नगरान्निर्ययुस्तदा} %4-31-23

\twolineshloka
{नखदंष्ट्रायुधाः सर्वे वीरा विकृतदर्शनाः}
{सर्वे शार्दूलदंष्ट्राश्च सर्वे विवृतदर्शनाः} %4-31-24

\twolineshloka
{दशनागबलाः केचित् केचिद् दशगुणोत्तराः}
{केचिन्नागसहस्रस्य बभूवुस्तुल्यवर्चसः} %4-31-25

\twolineshloka
{ततस्तैः कपिभिर्व्याप्तां द्रुमहस्तैर्महाबलैः}
{अपश्यल्लक्ष्मणः क्रुद्धः किष्किन्धां तां दुरासदाम्} %4-31-26

\twolineshloka
{ततस्ते हरयः सर्वे प्राकारपरिखान्तरात्}
{निष्क्रम्योदग्रसत्त्वास्तु तस्थुराविष्कृतं तदा} %4-31-27

\twolineshloka
{सुग्रीवस्य प्रमादं च पूर्वजस्यार्थमात्मवान्}
{दृष्ट्वा क्रोधवशं वीरः पुनरेव जगाम सः} %4-31-28

\twolineshloka
{स दीर्घोष्णमहोच्छ्वासः कोपसंरक्तलोचनः}
{बभूव नरशार्दूलः सधूम इव पावकः} %4-31-29

\twolineshloka
{बाणशल्यस्फुरज्जिह्वः सायकासनभोगवान्}
{स्वतेजोविषसम्भूतः पञ्चास्य इव पन्नगः} %4-31-30

\twolineshloka
{तं दीप्तमिव कालाग्निं नागेन्द्रमिव कोपितम्}
{समासाद्याङ्गदस्त्रासाद् विषादमगमत् परम्} %4-31-31

\twolineshloka
{सोऽङ्गदं रोषताम्राक्षः संदिदेश महायशाः}
{सुग्रीवः कथ्यतां वत्स ममागमनमित्युत} %4-31-32

\twolineshloka
{एष रामानुजः प्राप्तस्त्वत्सकाशमरिंदम}
{भ्रातुर्व्यसनसंतप्तो द्वारि तिष्ठति लक्ष्मणः} %4-31-33

\twolineshloka
{तस्य वाक्यं यदि रुचिः क्रियतां साधु वानर}
{इत्युक्त्वा शीघ्रमागच्छ वत्स वाक्यमरिंदम} %4-31-34

\twolineshloka
{लक्ष्मणस्य वचः श्रुत्वा शोकाविष्टोऽङ्गदोऽब्रवीत्}
{पितुः समीपमागम्य सौमित्रिरयमागतः} %4-31-35

\twolineshloka
{अथाङ्गदस्तस्य सुतीव्रवाचा सम्भ्रान्तभावः परिदीनवक्त्रः}
{निर्गत्य पूर्वं नृपतेस्तरस्वी ततो रुमायाश्चरणौ ववन्दे} %4-31-36

\twolineshloka
{संगृह्य पादौ पितुरुग्रतेजा जग्राह मातुः पुनरेव पादौ}
{पादौ रुमायाश्च निपीडयित्वा निवेदयामास ततस्तदर्थम्} %4-31-37

\twolineshloka
{स निद्राक्लान्तसंवीतो वानरो न विबुद्धवान्}
{बभूव मदमत्तश्च मदनेन च मोहितः} %4-31-38

\twolineshloka
{ततः किलकिलां चक्रुर्लक्ष्मणं प्रेक्ष्य वानराः}
{प्रसादयन्तस्तं क्रुद्धं भयमोहितचेतसः} %4-31-39

\twolineshloka
{ते महौघनिभं दृष्ट्वा वज्राशनिसमस्वनम्}
{सिंहनादं समं चक्रुर्लक्ष्मणस्य समीपतः} %4-31-40

\twolineshloka
{तेन शब्देन महता प्रत्यबुध्यत वानरः}
{मदविह्वलताम्राक्षो व्याकुलः स्रग्विभूषणः} %4-31-41

\twolineshloka
{अथाङ्गदवचः श्रुत्वा तेनैव च समागतौ}
{मन्त्रिणौ वानरेन्द्रस्य सम्मतोदारदर्शनौ} %4-31-42

\twolineshloka
{प्लक्षश्चैव प्रभावश्च मन्त्रिणावर्थधर्मयोः}
{वक्तुमुच्चावचं प्राप्तं लक्ष्मणं तौ शशंसतुः} %4-31-43

\twolineshloka
{प्रसादयित्वा सुग्रीवं वचनैः सार्थनिश्चितैः}
{आसीनं पर्युपासीनौ यथा शक्रं मरुत्पतिम्} %4-31-44

\twolineshloka
{सत्यसंधौ महाभागौ भ्रातरौ रामलक्ष्मणौ}
{मनुष्यभावं सम्प्राप्तौ राज्यार्हौ राज्यदायिनौ} %4-31-45

\twolineshloka
{तयोरेको धनुष्पाणिर्द्वारि तिष्ठति लक्ष्मणः}
{यस्य भीताः प्रवेपन्तो नादान् मुञ्चन्ति वानराः} %4-31-46

\twolineshloka
{स एष राघवभ्राता लक्ष्मणो वाक्यसारथिः}
{व्यवसायरथः प्राप्तस्तस्य रामस्य शासनात्} %4-31-47

\twolineshloka
{अयं च तनयो राजंस्ताराया दयितोऽङ्गदः}
{लक्ष्मणेन सकाशं ते प्रेषितस्त्वरयानघ} %4-31-48

\twolineshloka
{सोऽयं रोषपरीताक्षो द्वारि तिष्ठति वीर्यवान्}
{वानरान् वानरपते चक्षुषा निर्दहन्निव} %4-31-49

\twolineshloka
{तस्य मूर्ध्ना प्रणामं त्वं सपुत्रः सहबान्धवः}
{गच्छ शीघ्रं महाराज रोषो ह्यद्योपशाम्यताम्} %4-31-50

\twolineshloka
{यथा हि रामो धर्मात्मा तत्कुरुष्व समाहितः}
{राजंस्तिष्ठ स्वसमये भव सत्यप्रतिश्रवः} %4-31-51


॥इत्यार्षे श्रीमद्रामायणे वाल्मीकीये आदिकाव्ये किष्किन्धाकाण्डे लक्ष्मणक्रोधः नाम एकत्रिंशः सर्गः ॥४-३१॥
