\sect{चतुर्विंशः सर्गः — सुग्रीवताराश्वासनम्}

\twolineshloka
{तामाशु वेगेन दुरासदेन त्वभिप्लुतां शोकमहार्णवेन}
{पश्यंस्तदा वाल्यनुजस्तरस्वी भ्रातुर्वधेनाप्रतिमेन तेपे} %4-24-1

\twolineshloka
{स बाष्पपूर्णेन मुखेन पश्यन् क्षणेन निर्विण्णमना मनस्वी}
{जगाम रामस्य शनैः समीपं भृत्यैर्वृतः सम्परिदूयमानः} %4-24-2

\twolineshloka
{स तं समासाद्य गृहीतचापमुदात्तमाशीविषतुल्यबाणम्}
{यशस्विनं लक्षणलक्षिताङ्गमवस्थितं राघवमित्युवाच} %4-24-3

\twolineshloka
{यथा प्रतिज्ञातमिदं नरेन्द्र कृतं त्वया दृष्टफलं च कर्म}
{ममाद्य भोगेषु नरेन्द्रसूनो मनो निवृत्तं हतजीवितेन} %4-24-4

\twolineshloka
{अस्यां महिष्यां तु भृशं रुदत्यां पुरेऽतिविक्रोशति दुःखतप्ते}
{हते नृपे संशयितेऽङ्गदे च न राम राज्ये रमते मनो मे} %4-24-5

\twolineshloka
{क्रोधादमर्षादतिविप्रधर्षाद् भ्रातुर्वधो मेऽनुमतः पुरस्तात्}
{हते त्विदानीं हरियूथपेऽस्मिन् सुतीक्ष्णमिक्ष्वाकुवर प्रतप्स्ये} %4-24-6

\twolineshloka
{श्रेयोऽद्य मन्ये मम शैलमुख्ये तस्मिन् हि वासश्चिरमृष्यमूके}
{यथा तथा वर्तयतः स्ववृत्त्या नेमं निहत्य त्रिदिवस्य लाभः} %4-24-7

\twolineshloka
{न त्वा जिघांसामि चरेति यन्मामयं महात्मा मतिमानुवाच}
{तस्यैव तद् राम वचोऽनुरूपमिदं वचः कर्म च मेऽनुरूपम्} %4-24-8

\twolineshloka
{भ्राता कथं नाम महागुणस्य भ्रातुर्वधं राम विरोचयेत}
{राज्यस्य दुःखस्य च वीर सारं विचिन्तयन् कामपुरस्कृतोऽपि} %4-24-9

\twolineshloka
{वधो हि मे मतो नासीत् स्वमाहात्म्यव्यतिक्रमात्}
{ममासीद् बुद्धिदौरात्म्यात् प्राणहारी व्यतिक्रमः} %4-24-10

\twolineshloka
{द्रुमशाखावभग्नोऽहं मुहूर्तं परिनिष्टनन्}
{सान्त्वयित्वा त्वनेनोक्तो न पुनः कर्तुमर्हसि} %4-24-11

\twolineshloka
{भ्रातृत्वमार्यभावश्च धर्मश्चानेन रक्षितः}
{मया क्रोधश्च कामश्च कपित्वं च प्रदर्शितम्} %4-24-12

\twolineshloka
{अचिन्तनीयं परिवर्जनीयमनीप्सनीयं स्वनवेक्षणीयम्}
{प्राप्तोऽस्मि पाप्मानमिदं वयस्य भ्रातुर्वधात् त्वाष्ट्रवधादिवेन्द्रः} %4-24-13

\twolineshloka
{पाप्मानमिन्द्रस्य मही जलं च वृक्षाश्च कामं जगृहुः स्त्रियश्च}
{को नाम पाप्मानमिमं सहेत शाखामृगस्य प्रतिपत्तुमिच्छेत्} %4-24-14

\twolineshloka
{नार्हामि सम्मानमिमं प्रजानां न यौवराज्यं कुत एव राज्यम्}
{अधर्मयुक्तं कुलनाशयुक्तमेवंविधं राघव कर्म कृत्वा} %4-24-15

\twolineshloka
{पापस्य कर्तास्मि विगर्हितस्य क्षुद्रस्य लोकापकृतस्य लोके}
{शोको महान् मामभिवर्ततेऽयं वृष्टेर्यथा निम्नमिवाम्बुवेगः} %4-24-16

\twolineshloka
{सोदर्यघातापरगात्रवालः संतापहस्ताक्षिशिरोविषाणः}
{एनोमयो मामभिहन्ति हस्ती दृप्तो नदीकूलमिव प्रवृद्धः} %4-24-17

\twolineshloka
{अंहो बतेदं नृवराविषह्यं निवर्तते मे हृदि साधुवृत्तम्}
{अग्नौ विवर्णं परितप्यमानं किट्टं यथा राघव जातरूपम्} %4-24-18

\twolineshloka
{महाबलानां हरियूथपानामिदं कुलं राघव मन्निमित्तम्}
{अस्याङ्गदस्यापि च शोकतापादर्धस्थितप्राणमितीव मन्ये} %4-24-19

\twolineshloka
{सुतः सुलभ्यः सुजनः सुवश्यः कुतस्तु पुत्रः सदृशोऽङ्गदेन}
{न चापि विद्येत स वीर देशो यस्मिन् भवेत् सोदरसंनिकर्षः} %4-24-20

\twolineshloka
{अद्याङ्गदो वीरवरो न जीवेज्जीवेत माता परिपालनार्थम्}
{विना तु पुत्रं परितापदीना सा नैव जीवेदिति निश्चितं मे} %4-24-21

\twolineshloka
{सोऽहं प्रवेक्ष्याम्यतिदीप्तमग्निं भ्रात्रा च पुत्रेण च सख्यमिच्छन्}
{इमे विचेष्यन्ति हरिप्रवीराः सीतां निदेशे परिवर्तमानाः} %4-24-22

\twolineshloka
{कृत्स्नं तु ते सेत्स्यति कार्यमेतन्मय्यप्यतीते मनुजेन्द्रपुत्र}
{कुलस्य हन्तारमजीवनार्हं रामानुजानीहि कृतागसं माम्} %4-24-23

\twolineshloka
{इत्येवमार्तस्य रघुप्रवीरः श्रुत्वा वचो वालिजघन्यजस्य}
{संजातबाष्पः परवीरहन्ता रामो मुहूर्तं विमना बभूव} %4-24-24

\twolineshloka
{तस्मिन् क्षणेऽभीक्ष्णमवेक्षमाणः क्षितिक्षमावान् भुवनस्य गोप्ता}
{रामो रुदन्तीं व्यसने निमग्नां समुत्सुकः सोऽथ ददर्श ताराम्} %4-24-25

\twolineshloka
{तां चारुनेत्रां कपिसिंहनाथां पतिं समाश्लिष्य तदा शयानाम्}
{उत्थापयामासुरदीनसत्त्वां मन्त्रिप्रधानाः कपिराजपत्नीम्} %4-24-26

\twolineshloka
{सा विस्फुरन्ती परिरभ्यमाणा भर्तुः समीपादपनीयमाना}
{ददर्श रामं शरचापपाणिं स्वतेजसा सूर्यमिव ज्वलन्तम्} %4-24-27

\twolineshloka
{सुसंवृतं पार्थिवलक्षणैश्च तं चारुनेत्रं मृगशावनेत्रा}
{अदृष्टपूर्वं पुरुषप्रधानमयं स काकुत्स्थ इति प्रजज्ञे} %4-24-28

\twolineshloka
{तस्येन्द्रकल्पस्य दुरासदस्य महानुभावस्य समीपमार्या}
{आर्तातितूर्णं व्यसनं प्रपन्ना जगाम तारा परिविह्वलन्ती} %4-24-29

\twolineshloka
{तं सा समासाद्य विशुद्धसत्त्वं शोकेन सम्भ्रान्तशरीरभावा}
{मनस्विनी वाक्यमुवाच तारा रामं रणोत्कर्षणलब्धलक्ष्यम्} %4-24-30

\twolineshloka
{त्वमप्रमेयश्च दुरासदश्च जितेन्द्रियश्चोत्तमधर्मकश्च}
{अक्षीणकीर्तिश्च विचक्षणश्च क्षितिक्षमावान् क्षतजोपमाक्षः} %4-24-31

\twolineshloka
{त्वमात्तबाणासनबाणपाणिर्महाबलः संहननोपपन्नः}
{मनुष्यदेहाभ्युदयं विहाय दिव्येन देहाभ्युदयेन युक्तः} %4-24-32

\twolineshloka
{येनैव बाणेन हतः प्रियो मे तेनैव बाणेन हि मां जहीहि}
{हता गमिष्यामि समीपमस्य न मां विना वीर रमेत वाली} %4-24-33

\twolineshloka
{स्वर्गेऽपि पद्मामलपत्रनेत्र समेत्य सम्प्रेक्ष्य च मामपश्यन्}
{न ह्येष उच्चावचताम्रचूडा विचित्रवेषाप्सरसोऽभजिष्यत्} %4-24-34

\twolineshloka
{स्वर्गेऽपि शोकं च विवर्णतां च मया विना प्राप्स्यति वीर वाली}
{रम्ये नगेन्द्रस्य तटावकाशे विदेहकन्यारहितो यथा त्वम्} %4-24-35

\twolineshloka
{त्वं वेत्थ तावद् वनिताविहीनः प्राप्नोति दुःखं पुरुषः कुमारः}
{तत् त्वं प्रजानञ्जहि मां न वाली दुःखं ममादर्शनजं भजेत} %4-24-36

\twolineshloka
{यच्चापि मन्येत भवान् महात्मा स्त्रीघातदोषस्तु भवेन्न मह्यम्}
{आत्मेयमस्येति हि मां जहि त्वं न स्त्रीवधः स्यान्मनुजेन्द्रपुत्र} %4-24-37

\twolineshloka
{शास्त्रप्रयोगाद् विविधाश्च वेदादनन्यरूपाः पुरुषस्य दाराः}
{दारप्रदानाद्धि न दानमन्यत् प्रदृश्यते ज्ञानवतां हि लोके} %4-24-38

\twolineshloka
{त्वं चापि मां तस्य मम प्रियस्य प्रदास्यसे धर्ममवेक्ष्य वीर}
{अनेन दानेन न लप्स्यसे त्वमधर्मयोगं मम वीर घातात्} %4-24-39

\threelineshloka
{आर्तामनाथामपनीयमानामेवंगतां नार्हसि मामहन्तुम्}
{अहं हि मातङ्गविलासगामिना प्लवंगमानामृषभेण धीमता}
{विना वरार्होत्तमहेममालिना चिरं न शक्ष्यामि नरेन्द्र जीवितुम्} %4-24-40

\twolineshloka
{इत्येवमुक्तस्तु विभुर्महात्मा तारां समाश्वास्य हितं बभाषे}
{मा वीरभार्ये विमतिं कुरुष्व लोको हि सर्वो विहितो विधात्रा} %4-24-41

\twolineshloka
{तं चैव सर्वं सुखदुःखयोगं लोकोऽब्रवीत् तेन कृतं विधात्रा}
{त्रयोऽपि लोका विहितं विधानं नातिक्रमन्ते वशगा हि तस्य} %4-24-42

\twolineshloka
{प्रीतिं परां प्राप्स्यसि तां तथैव पुत्रश्च ते प्राप्स्यति यौवराज्यम्}
{धात्रा विधानं विहितं तथैव न शूरपत्न्यः परिदेवयन्ति} %4-24-43

\twolineshloka
{आश्वासिता तेन महात्मना तु प्रभावयुक्तेन परंतपेन}
{सा वीरपत्नी ध्वनता मुखेन सुवेषरूपा विरराम तारा} %4-24-44


॥इत्यार्षे श्रीमद्रामायणे वाल्मीकीये आदिकाव्ये किष्किन्धाकाण्डे सुग्रीवताराश्वासनम् नाम चतुर्विंशः सर्गः ॥४-२४॥
