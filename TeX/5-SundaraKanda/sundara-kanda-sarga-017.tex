\sect{सप्तदशः सर्गः — राक्षसीपरिवारः}

\twolineshloka
{ततः कुमुदखण्डाभो निर्मलं निर्मलोदयः}
{प्रजगाम नभश्चन्द्रो हंसो नीलमिवोदकम्} %5-17-1

\twolineshloka
{साचिव्यमिव कुर्वन् स प्रभया निर्मलप्रभः}
{चन्द्रमा रश्मिभिः शीतैः सिषेवे पवनात्मजम्} %5-17-2

\twolineshloka
{स ददर्श ततः सीतां पूर्णचन्द्रनिभाननाम्}
{शोकभारैरिव न्यस्तां भारैर्नावमिवाम्भसि} %5-17-3

\twolineshloka
{दिदृक्षमाणो वैदेहीं हनूमान् मारुतात्मजः}
{स ददर्शाविदूरस्था राक्षसीर्घोरदर्शनाः} %5-17-4

\twolineshloka
{एकाक्षीमेककर्णां च कर्णप्रावरणां तथा}
{अकर्णां शङ्कुकर्णां च मस्तकोच्छ्वासनासिकाम्} %5-17-5

\twolineshloka
{अतिकायोत्तमांगीं च तनुदीर्घशिरोधराम्}
{ध्वस्तकेशीं तथाकेशीं केशकम्बलधारिणीम्} %5-17-6

\twolineshloka
{लम्बकर्णललाटां च लम्बोदरपयोधराम्}
{लम्बोष्ठीं चिबुकोष्ठीं च लम्बास्यां लम्बजानुकाम्} %5-17-7

\twolineshloka
{ह्रस्वां दीर्घां च कुब्जां च विकटां वामनां तथा}
{करालां भुग्नवक्त्रां च पिंगाक्षीं विकृताननाम्} %5-17-8

\twolineshloka
{विकृताः पिंगलाः कालीः क्रोधनाः कलहप्रियाः}
{कालायसमहाशूलकूटमुद्गरधारिणीः} %5-17-9

\twolineshloka
{वराहमृगशार्दूलमहिषाजशिवामुखाः}
{गजोष्ट्रहयपादाश्च निखातशिरसोऽपराः} %5-17-10

\twolineshloka
{एकहस्तैकपादाश्च खरकर्ण्यश्वकर्णिकाः}
{गोकर्णीर्हस्तिकर्णीश्च हरिकर्णीस्तथापराः} %5-17-11

\twolineshloka
{अतिनासाश्च काश्चिच्च तिर्यङ्नासा अनासिकाः}
{गजसंनिभनासाश्च ललाटोच्छ्वासनासिकाः} %5-17-12

\twolineshloka
{हस्तिपादा महापादा गोपादाः पादचूलिकाः}
{अतिमात्रशिरोग्रीवा अतिमात्रकुचोदरीः} %5-17-13

\twolineshloka
{अतिमात्रास्यनेत्राश्च दीर्घजिह्वाननास्तथा}
{अजामुखीर्हस्तिमुखीर्गोमुखीः सूकरीमुखीः} %5-17-14

\twolineshloka
{हयोष्ट्रखरवक्त्राश्च राक्षसीर्घोरदर्शनाः}
{शूलमुद्गरहस्ताश्च क्रोधनाः कलहप्रियाः} %5-17-15

\twolineshloka
{कराला धूम्रकेशिन्यो राक्षसीर्विकृताननाः}
{पिबन्ति सततं पानं सुरामांससदाप्रियाः} %5-17-16

\twolineshloka
{मांसशोणितदिग्धांगीर्मांसशोणितभोजनाः}
{ता ददर्श कपिश्रेष्ठो रोमहर्षणदर्शनाः} %5-17-17

\twolineshloka
{स्कन्धवन्तमुपासीनाः परिवार्य वनस्पतिम्}
{तस्याधस्ताच्च तां देवीं राजपुत्रीमनिन्दिताम्} %5-17-18

\twolineshloka
{लक्षयामास लक्ष्मीवान् हनूमाञ्जनकात्मजाम्}
{निष्प्रभां शोकसंतप्तां मलसंकुलमूर्धजाम्} %5-17-19

\twolineshloka
{क्षीणपुण्यां च्युतां भूमौ तारां निपतितामिव}
{चारित्रव्यपदेशाढ्यां भर्तृदर्शनदुर्गताम्} %5-17-20

\twolineshloka
{भूषणैरुत्तमैर्हीनां भर्तृवात्सल्यभूषिताम्}
{राक्षसाधिपसंरुद्धां बन्धुभिश्च विनाकृताम्} %5-17-21

\twolineshloka
{वियूथां सिंहसंरुद्धां बद्धां गजवधूमिव}
{चन्द्ररेखां पयोदान्ते शारदाभ्रैरिवावृताम्} %5-17-22

\twolineshloka
{क्लिष्टरूपामसंस्पर्शादयुक्तामिव वल्लकीम्}
{स तां भर्तृहिते युक्तामयुक्तां रक्षसां वशे} %5-17-23

\twolineshloka
{अशोकवनिकामध्ये शोकसागरमाप्लुताम्}
{ताभिः परिवृतां तत्र सग्रहामिव रोहिणीम्} %5-17-24

\threelineshloka
{ददर्श हनुमांस्तत्र लतामकुसुमामिव}
{सा मलेन च दिग्धांगी वपुषा चाप्यलंकृता}
{मृणाली पङ्कदिग्धेव विभाति च न भाति च} %5-17-25

\twolineshloka
{मलिनेन तु वस्त्रेण परिक्लिष्टेन भामिनीम्}
{संवृतां मृगशावाक्षीं ददर्श हनुमान् कपिः} %5-17-26

\twolineshloka
{तां देवीं दीनवदनामदीनां भर्तृतेजसा}
{रक्षितां स्वेन शीलेन सीतामसितलोचनाम्} %5-17-27

\twolineshloka
{तां दृष्ट्वा हनुमान् सीतां मृगशावनिभेक्षणाम्}
{मृगकन्यामिव त्रस्तां वीक्षमाणां समन्ततः} %5-17-28

\twolineshloka
{दहन्तीमिव निःश्वासैर्वृक्षान् पल्लवधारिणः}
{संघातमिव शोकानां दुःखस्योर्मिमिवोत्थिताम्} %5-17-29

\twolineshloka
{तां क्षमां सुविभक्तांगीं विनाभरणशोभिनीम्}
{प्रहर्षमतुलं लेभे मारुतिः प्रेक्ष्य मैथिलीम्} %5-17-30

\twolineshloka
{हर्षजानि च सोऽश्रूणि तां दृष्ट्वा मदिरेक्षणाम्}
{मुमोच हनुमांस्तत्र नमश्चक्रे च राघवम्} %5-17-31

\twolineshloka
{नमस्कृत्वाथ रामाय लक्ष्मणाय च वीर्यवान्}
{सीतादर्शनसंहृष्टो हनुमान् संवृतोऽभवत्} %5-17-32


॥इत्यार्षे श्रीमद्रामायणे वाल्मीकीये आदिकाव्ये सुन्दरकाण्डे राक्षसीपरिवारः नाम सप्तदशः सर्गः ॥५-१७॥
