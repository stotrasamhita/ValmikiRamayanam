\sect{षड्विंशः सर्गः — प्राणत्यागसंप्रधारणम्}

\twolineshloka
{प्रसक्ताश्रुमुखी त्वेवं ब्रुवती जनकात्मजा}
{अधोगतमुखी बाला विलप्तुमुपचक्रमे} %5-26-1

\twolineshloka
{उन्मत्तेव प्रमत्तेव भ्रान्तचित्तेव शोचती}
{उपावृत्ता किशोरीव विचेष्टन्ती महीतले} %5-26-2

\twolineshloka
{राघवस्य प्रमत्तस्य रक्षसा कामरूपिणा}
{रावणेन प्रमथ्याहमानीता क्रोशती बलात्} %5-26-3

\twolineshloka
{राक्षसीवशमापन्ना भर्त्स्यमाना च दारुणम्}
{चिन्तयन्ती सुदुःखार्ता नाहं जीवितुमुत्सहे} %5-26-4

\twolineshloka
{नहि मे जीवितेनार्थो नैवार्थैर्न च भूषणैः}
{वसन्त्या राक्षसीमध्ये विना रामं महारथम्} %5-26-5

\twolineshloka
{अश्मसारमिदं नूनमथवाप्यजरामरम्}
{हृदयं मम येनेदं न दुःखेन विशीर्यते} %5-26-6

\twolineshloka
{धिङ्मामनार्यामसतीं याहं तेन विना कृता}
{मुहूर्तमपि जीवामि जीवितं पापजीविका} %5-26-7

\twolineshloka
{चरणेनापि सव्येन न स्पृशेयं निशाचरम्}
{रावणं किं पुनरहं कामयेयं विगर्हितम्} %5-26-8

\twolineshloka
{प्रत्याख्यानं न जानाति नात्मानं नात्मनः कुलम्}
{यो नृशंसस्वभावेन मां प्रार्थयितुमिच्छति} %5-26-9

\twolineshloka
{छिन्ना भिन्ना प्रभिन्ना वा दीप्ता वाग्नौ प्रदीपिता}
{रावणं नोपतिष्ठेयं किं प्रलापेन वश्चिरम्} %5-26-10

\twolineshloka
{ख्यातः प्राज्ञः कृतज्ञश्च सानुक्रोशश्च राघवः}
{सद्वृत्तो निरनुक्रोशः शङ्के मद्भाग्यसंक्षयात्} %5-26-11

\twolineshloka
{राक्षसानां जनस्थाने सहस्राणि चतुर्दश}
{एकेनैव निरस्तानि स मां किं नाभिपद्यते} %5-26-12

\twolineshloka
{निरुद्धा रावणेनाहमल्पवीर्येण रक्षसा}
{समर्थः खलु मे भर्ता रावणं हन्तुमाहवे} %5-26-13

\twolineshloka
{विराधो दण्डकारण्ये येन राक्षसपुंगवः}
{रणे रामेण निहतः स मां किं नाभिपद्यते} %5-26-14

\twolineshloka
{कामं मध्ये समुद्रस्य लङ्केयं दुष्प्रधर्षणा}
{न तु राघवबाणानां गतिरोधो भविष्यति} %5-26-15

\twolineshloka
{किं नु तत् कारणं येन रामो दृढपराक्रमः}
{रक्षसापहृतां भार्यामिष्टां यो नाभिपद्यते} %5-26-16

\twolineshloka
{इहस्थां मां न जानीते शङ्के लक्ष्मणपूर्वजः}
{जानन्नपि स तेजस्वी धर्षणां मर्षयिष्यति} %5-26-17

\twolineshloka
{हृतेति मां योऽधिगत्य राघवाय निवेदयेत्}
{गृध्रराजोऽपि स रणे रावणेन निपातितः} %5-26-18

\twolineshloka
{कृतं कर्म महत् तेन मां तथाभ्यवपद्यता}
{तिष्ठता रावणवधे वृद्धेनापि जटायुषा} %5-26-19

\twolineshloka
{यदि मामिह जानीयाद् वर्तमानां हि राघवः}
{अद्य बाणैरभिक्रुद्धः कुर्याल्लोकमराक्षसम्} %5-26-20

\twolineshloka
{निर्दहेच्च पुरीं लङ्कां निर्दहेच्च महोदधिम्}
{रावणस्य च नीचस्य कीर्तिं नाम च नाशयेत्} %5-26-21

\twolineshloka
{ततो निहतनाथानां राक्षसीनां गृहे गृहे}
{यथाहमेवं रुदती तथा भूयो न संशयः} %5-26-22

\twolineshloka
{अन्विष्य रक्षसां लङ्कां कुर्याद् रामः सलक्ष्मणः}
{नहि ताभ्यां रिपुर्दृष्टो मुहूर्तमपि जीवति} %5-26-23

\twolineshloka
{चिताधूमाकुलपथा गृध्रमण्डलमण्डिता}
{अचिरेणैव कालेन श्मशानसदृशी भवेत्} %5-26-24

\twolineshloka
{अचिरेणैव कालेन प्राप्स्याम्येनं मनोरथम्}
{दुष्प्रस्थानोऽयमाभाति सर्वेषां वो विपर्ययः} %5-26-25

\twolineshloka
{यादृशानि तु दृश्यन्ते लङ्कायामशुभानि तु}
{अचिरेणैव कालेन भविष्यति हतप्रभा} %5-26-26

\twolineshloka
{नूनं लङ्का हते पापे रावणे राक्षसाधिपे}
{शोषमेष्यति दुर्धर्षा प्रमदा विधवा यथा} %5-26-27

\twolineshloka
{पुण्योत्सवसमृद्धा च नष्टभर्त्री सराक्षसा}
{भविष्यति पुरी लङ्का नष्टभर्त्री यथांगना} %5-26-28

\twolineshloka
{नूनं राक्षसकन्यानां रुदतीनां गृहे गृहे}
{श्रोष्यामि नचिरादेव दुःखार्तानामिह ध्वनिम्} %5-26-29

\twolineshloka
{सान्धकारा हतद्योता हतराक्षसपुंगवा}
{भविष्यति पुरी लङ्का निर्दग्धा रामसायकैः} %5-26-30

\twolineshloka
{यदि नाम स शूरो मां रामो रक्तान्तलोचनः}
{जानीयाद् वर्तमानां यां राक्षसस्य निवेशने} %5-26-31

\twolineshloka
{अनेन तु नृशंसेन रावणेनाधमेन मे}
{समयो यस्तु निर्दिष्टस्तस्य कालोऽयमागतः} %5-26-32

\twolineshloka
{स च मे विहितो मृत्युरस्मिन् दुष्टेन वर्तते}
{अकार्यं ये न जानन्ति नैर्ऋताः पापकारिणः} %5-26-33

\twolineshloka
{अधर्मात् तु महोत्पातो भविष्यति हि साम्प्रतम्}
{नैते धर्मं विजानन्ति राक्षसाः पिशिताशनाः} %5-26-34

\twolineshloka
{ध्रुवं मां प्रातराशार्थं राक्षसः कल्पयिष्यति}
{साहं कथं करिष्यामि तं विना प्रियदर्शनम्} %5-26-35

\twolineshloka
{रामं रक्तान्तनयनमपश्यन्ती सुदुःखिता}
{क्षिप्रं वैवस्वतं देवं पश्येयं पतिना विना} %5-26-36

\twolineshloka
{नाजानाज्जीवतीं रामः स मां भरतपूर्वजः}
{जानन्तौ तु न कुर्यातां नोर्व्यां हि परिमार्गणम्} %5-26-37

\twolineshloka
{नूनं ममैव शोकेन स वीरो लक्ष्मणाग्रजः}
{देवलोकमितो यातस्त्यक्त्वा देहं महीतले} %5-26-38

\twolineshloka
{धन्या देवाः सगन्धर्वाः सिद्धाश्च परमर्षयः}
{मम पश्यन्ति ये वीरं रामं राजीवलोचनम्} %5-26-39

\twolineshloka
{अथवा नहि तस्यार्थो धर्मकामस्य धीमतः}
{मया रामस्य राजर्षेर्भार्यया परमात्मनः} %5-26-40

\twolineshloka
{दृश्यमाने भवेत् प्रीतिः सौहृदं नास्त्यदृश्यतः}
{नाशयन्ति कृतघ्नास्तु न रामो नाशयिष्यति} %5-26-41

\twolineshloka
{किं वा मय्यगुणाः केचित् किं वा भाग्यक्षयो हि मे}
{या हि सीता वरार्हेण हीना रामेण भामिनी} %5-26-42

\twolineshloka
{श्रेयो मे जीवितान्मर्तुं विहीनाया महात्मना}
{रामादक्लिष्टचारित्राच्छूराच्छत्रुनिबर्हणात्} %5-26-43

\twolineshloka
{अथवा न्यस्तशस्त्रौ तौ वने मूलफलाशनौ}
{भ्रातरौ हि नरश्रेष्ठौ चरन्तौ वनगोचरौ} %5-26-44

\twolineshloka
{अथवा राक्षसेन्द्रेण रावणेन दुरात्मना}
{छद्मना घातितौ शूरौ भ्रातरौ रामलक्ष्मणौ} %5-26-45

\twolineshloka
{साहमेवंविधे काले मर्तुमिच्छामि सर्वतः}
{न च मे विहितो मृत्युरस्मिन् दुःखेऽतिवर्तति} %5-26-46

\twolineshloka
{धन्याः खलु महात्मानो मुनयः सत्यसम्मताः}
{जितात्मानो महाभागा येषां न स्तः प्रियाप्रिये} %5-26-47

\twolineshloka
{प्रियान्न सम्भवेद् दुःखमप्रियादधिकं भवेत्}
{ताभ्यां हि ते वियुज्यन्ते नमस्तेषां महात्मनाम्} %5-26-48

\twolineshloka
{साहं त्यक्ता प्रियेणैव रामेण विदितात्मना}
{प्राणांस्त्यक्ष्यामि पापस्य रावणस्य गता वशम्} %5-26-49


॥इत्यार्षे श्रीमद्रामायणे वाल्मीकीये आदिकाव्ये सुन्दरकाण्डे प्राणत्यागसंप्रधारणम् नाम षड्विंशः सर्गः ॥५-२६॥
