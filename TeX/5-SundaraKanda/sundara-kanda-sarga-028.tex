\sect{अष्टाविंशः सर्गः — उद्बन्धनव्यवसायः}

\twolineshloka
{सा राक्षसेन्द्रस्य वचो निशम्य तद् रावणस्याप्रियमप्रियार्ता}
{सीता वितत्रास यथा वनान्ते सिंहाभिपन्ना गजराजकन्या} %5-28-1

\twolineshloka
{सा राक्षसीमध्यगता च भीरुर्वाग्भिर्भृशं रावणतर्जिता च}
{कान्तारमध्ये विजने विसृष्टा बालेव कन्या विललाप सीता} %5-28-2

\twolineshloka
{सत्यं बतेदं प्रवदन्ति लोके नाकालमृत्युर्भवतीति सन्तः}
{यत्राहमेवं परिभर्त्स्यमाना जीवामि यस्मात् क्षणमप्यपुण्या} %5-28-3

\twolineshloka
{सुखाद् विहीनं बहुदुःखपूर्णमिदं तु नूनं हृदयं स्थिरं मे}
{विदीर्यते यन्न सहस्रधाद्य वज्राहतं शृङ्गमिवाचलस्य} %5-28-4

\twolineshloka
{नैवास्ति नूनं मम दोषमत्र वध्याहमस्याप्रियदर्शनस्य}
{भावं न चास्याहमनुप्रदातुमलं द्विजो मन्त्रमिवाद्विजाय} %5-28-5

\twolineshloka
{तस्मिन्ननागच्छति लोकनाथे गर्भस्थजन्तोरिव शल्यकृन्तः}
{नूनं ममाङ्गान्यचिरादनार्यः शस्त्रैः शितैश्छेत्स्यति राक्षसेन्द्रः} %5-28-6

\twolineshloka
{दुःखं बतेदं ननु दुःखिताया मासौ चिरायाभिगमिष्यतो द्वौ}
{बद्धस्य वध्यस्य यथा निशान्ते राजोपरोधादिव तस्करस्य} %5-28-7

\twolineshloka
{हा राम हा लक्ष्मण हा सुमित्रे हा राममातः सह मे जनन्यः}
{एषा विपद्याम्यहमल्पभाग्या महार्णवे नौरिव मूढवाता} %5-28-8

\twolineshloka
{तरस्विनौ धारयता मृगस्य सत्त्वेन रूपं मनुजेन्द्रपुत्रौ}
{नूनं विशस्तौ मम कारणात् तौ सिंहर्षभौ द्वाविव वैद्युतेन} %5-28-9

\twolineshloka
{नूनं स कालो मृगरूपधारी मामल्पभाग्यां लुलुभे तदानीम्}
{यत्रार्यपुत्रौ विससर्ज मूढा रामानुजं लक्ष्मणपूर्वजं च} %5-28-10

\twolineshloka
{हा राम सत्यव्रत दीर्घबाहो हा पूर्णचन्द्रप्रतिमानवक्त्र}
{हा जीवलोकस्य हितः प्रियश्च वध्यां न मां वेत्सि हि राक्षसानाम्} %5-28-11

\twolineshloka
{अनन्यदेवत्वमियं क्षमा च भूमौ च शय्या नियमश्च धर्मे}
{पतिव्रतात्वं विफलं ममेदं कृतं कृतघ्नेष्विव मानुषाणाम्} %5-28-12

\twolineshloka
{मोघो हि धर्मश्चरितो ममायं तथैकपत्नीत्वमिदं निरर्थकम्}
{या त्वां न पश्यामि कृशा विवर्णा हीना त्वया सङ्गमने निराशा} %5-28-13

\twolineshloka
{पितुर्निदेशं नियमेन कृत्वा वनान्निवृत्तश्चरितव्रतश्च}
{स्त्रीभिस्तु मन्ये विपुलेक्षणाभिः संरंस्यसे वीतभयः कृतार्थः} %5-28-14

\twolineshloka
{अहं तु राम त्वयि जातकामा चिरं विनाशाय निबद्धभावा}
{मोघं चरित्वाथ तपो व्रतं च त्यक्ष्यामि धिग्जीवितमल्पभाग्याम्} %5-28-15

\twolineshloka
{सञ्जीवितं क्षिप्रमहं त्यजेयं विषेण शस्त्रेण शितेन वापि}
{विषस्य दाता न तु मेऽस्ति कश्चिच्छस्त्रस्य वा वेश्मनि राक्षसस्य} %5-28-16

\twolineshloka
{शोकाभितप्ता बहुधा विचिन्त्य सीताथ वेणीग्रथनं गृहीत्वा}
{उद्बद्ध्य वेण्युद्ग्रथनेन शीघ्रमहं गमिष्यामि यमस्य मूलम्} %5-28-17

\twolineshloka
{उपस्थिता सा मृदुसर्वगात्री शाखां गृहीत्वा च नगस्य तस्य}
{तस्यास्तु रामं परिचिन्तयन्त्या रामानुजं स्वं च कुलं शुभाङ्ग्याः} %5-28-18

\twolineshloka
{तस्या विशोकानि तदा बहूनि धैर्यार्जितानि प्रवराणि लोके}
{प्रादुर्निमित्तानि तदा बभूवुः पुरापि सिद्धान्युपलक्षितानि} %5-28-19


॥इत्यार्षे श्रीमद्रामायणे वाल्मीकीये आदिकाव्ये सुन्दरकाण्डे उद्बन्धनव्यवसायः नाम अष्टाविंशः सर्गः ॥५-२८॥
