\sect{पञ्चमः सर्गः — भवनविचयः}

\twolineshloka
{ततः स मध्यङ्गतमंशुमन्तं ज्योत्स्नावितानं मुहुरुद्वमन्तम्}
{ददर्श धीमान् भुवि भानुमन्तं गोष्ठे वृषं मत्तमिव भ्रमन्तम्} %5-5-1

\twolineshloka
{लोकस्य पापानि विनाशयन्तं महोदधिं चापि समेधयन्तम्}
{भूतानि सर्वाणि विराजयन्तं ददर्श शीतांशुमथाभियान्तम्} %5-5-2

\twolineshloka
{या भाति लक्ष्मीर्भुवि मन्दरस्था यथा प्रदोषेषु च सागरस्था}
{तथैव तोयेषु च पुष्करस्था रराज सा चारुनिशाकरस्था} %5-5-3

\twolineshloka
{हंसो यथा राजतपञ्जरस्थः सिंहो यथा मन्दरकन्दरस्थः}
{वीरो यथा गर्वितकुञ्जरस्थश्चन्द्रोऽपि बभ्राज तथाम्बरस्थः} %5-5-4

\twolineshloka
{स्थितः ककुद्मानिव तीक्ष्णशृङ्गो महाचलः श्वेत इवोर्ध्वशृङ्गः}
{हस्तीव जाम्बूनदबद्धशृङ्गो विभाति चन्द्रः परिपूर्णशृङ्गः} %5-5-5

\twolineshloka
{विनष्टशीताम्बुतुषारपङ्को महाग्रहग्राहविनष्टपङ्कः}
{प्रकाशलक्ष्म्याश्रयनिर्मलाङ्को रराज चन्द्रो भगवान् शशाङ्कः} %5-5-6

\twolineshloka
{शिलातलं प्राप्य यथा मृगेन्द्रो महारणं प्राप्य यथा गजेन्द्रः}
{राज्यं समासाद्य यथा नरेन्द्रस्तथा प्रकाशो विरराज चन्द्रः} %5-5-7

\twolineshloka
{प्रकाशचन्द्रोदयनष्टदोषः प्रवृद्धरक्षःपिशिताशदोषः}
{रामाभिरामेरितचित्तदोषः स्वर्गप्रकाशो भगवान् प्रदोषः} %5-5-8

\twolineshloka
{तन्त्रीस्वराः कर्णसुखाः प्रवृत्ताः स्वपन्ति नार्यः पतिभिः सुवृत्ताः}
{नक्तञ्चराश्चापि तथा प्रवृत्ता विहर्तुमत्यद्भुतरौद्रवृत्ताः} %5-5-9

\twolineshloka
{मत्तप्रमत्तानि समाकुलानि रथाश्वभद्रासनसङ्कुलानि}
{वीरश्रिया चापि समाकुलानि ददर्श धीमान् स कपिः कुलानि} %5-5-10

\twolineshloka
{परस्परं चाधिकमाक्षिपन्ति भुजांश्च पीनानधिविक्षिपन्ति}
{मत्तप्रलापानधिविक्षिपन्ति मत्तानि चान्योन्यमधिक्षिपन्ति} %5-5-11

\twolineshloka
{रक्षांसि वक्षांसि च विक्षिपन्ति गात्राणि कान्तासु च विक्षिपन्ति}
{रूपाणि चित्राणि च विक्षिपन्ति दृढानि चापानि च विक्षिपन्ति} %5-5-12

\twolineshloka
{ददर्श कान्ताश्च समालभन्त्यस्तथापरास्तत्र पुनः स्वपन्त्यः}
{सुरूपवक्त्राश्च तथा हसन्त्यः क्रुद्धाः पराश्चापि विनिःश्वसन्त्यः} %5-5-13

\twolineshloka
{महागजैश्चापि तथा नदद्भिः सुपूजितैश्चापि तथा सुसद्भिः}
{रराज वीरैश्च विनिःश्वसद्भिर्ह्रदा भुजङ्गैरिव निःश्वसद्भिः} %5-5-14

\twolineshloka
{बुद्धिप्रधानान् रुचिराभिधानान् संश्रद्दधानाञ्जगतः प्रधानान्}
{नानाविधानान् रुचिराभिधानान् ददर्श तस्यां पुरि यातुधानान्} %5-5-15

\twolineshloka
{ननन्द दृष्ट्वा स च तान् सुरूपान् नानागुणानात्मगुणानुरूपान्}
{विद्योतमानान् स च तान् सुरूपान् ददर्श कांश्चिच्च पुनर्विरूपान्} %5-5-16

\twolineshloka
{ततो वरार्हाः सुविशुद्धभावास्तेषां स्त्रियस्तत्र महानुभावाः}
{प्रियेषु पानेषु च सक्तभावा ददर्श तारा इव सुस्वभावाः} %5-5-17

\twolineshloka
{स्त्रियो ज्वलन्तीस्त्रपयोपगूढा निशीथकाले रमणोपगूढाः}
{ददर्श काश्चित् प्रमदोपगूढा यथा विहङ्गा विहगोपगूढाः} %5-5-18

\twolineshloka
{अन्याः पुनर्हर्म्यतलोपविष्टास्तत्र प्रियाङ्केषु सुखोपविष्टाः}
{भर्तुः परा धर्मपरा निविष्टा ददर्श धीमान् मदनोपविष्टाः} %5-5-19

\twolineshloka
{अप्रावृताः काञ्चनराजिवर्णाः काश्चित्परार्घ्यास्तपनीयवर्णाः}
{पुनश्च काश्चिच्छशलक्ष्मवर्णाः कान्तप्रहीणा रुचिराङ्गवर्णाः} %5-5-20

\twolineshloka
{ततः प्रियान् प्राप्य मनोऽभिरामान् सुप्रीतियुक्ताः सुमनोऽभिरामाः}
{गृहेषु हृष्टाः परमाभिरामा हरिप्रवीरः स ददर्श रामाः} %5-5-21

\twolineshloka
{चन्द्रप्रकाशाश्च हि वक्त्रमाला वक्राः सुपक्ष्माश्च सुनेत्रमालाः}
{विभूषणानां च ददर्श मालाः शतह्रदानामिव चारुमालाः} %5-5-22

\twolineshloka
{न त्वेव सीतां परमाभिजातां पथि स्थिते राजकुले प्रजाताम्}
{लतां प्रफुल्लामिव साधुजातां ददर्श तन्वीं मनसाभिजाताम्} %5-5-23

\twolineshloka
{सनातने वर्त्मनि सन्निविष्टां रामेक्षणीं तां मदनाभिविष्टाम्}
{भर्तुर्मनः श्रीमदनुप्रविष्टां स्त्रीभ्यः पराभ्यश्च सदा विशिष्टाम्} %5-5-24

\twolineshloka
{उष्णार्दितां सानुसृतास्रकण्ठीं पुरा वरार्होत्तमनिष्ककण्ठीम्}
{सुजातपक्ष्मामभिरक्तकण्ठीं वने प्रनृत्तामिव नीलकण्ठीम्} %5-5-25

\twolineshloka
{अव्यक्तरेखामिव चन्द्रलेखां पांसुप्रदिग्धामिव हेमरेखाम्}
{क्षतप्ररूढामिव वर्णरेखां वायुप्रभुग्नामिव मेघरेखाम्} %5-5-26

\twolineshloka
{सीतामपश्यन्मनुजेश्वरस्य रामस्य पत्नीं वदतां वरस्य}
{बभूव दुःखोपहतश्चिरस्य प्लवङ्गमो मन्द इवाचिरस्य} %5-5-27


॥इत्यार्षे श्रीमद्रामायणे वाल्मीकीये आदिकाव्ये सुन्दरकाण्डे भवनविचयः नाम पञ्चमः सर्गः ॥५-५॥
