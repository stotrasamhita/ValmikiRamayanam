\sect{एकत्रिंशः सर्गः — रामवृत्तसंश्रवः}

\twolineshloka
{एवं बहुविधां चिन्तां चिन्तयित्वा महामतिः}
{संश्रवे मधुरं वाक्यं वैदेह्या व्याजहार ह} %5-31-1

\twolineshloka
{राजा दशरथो नाम रथकुञ्जरवाजिमान्}
{पुण्यशीलो महाकीर्तिरिक्ष्वाकूणां महायशाः} %5-31-2

\twolineshloka
{राजर्षीणां गुणश्रेष्ठस्तपसा चर्षिभिः समः}
{चक्रवर्तिकुले जातः पुरन्दरसमो बले} %5-31-3

\twolineshloka
{अहिंसारतिरक्षुद्रो घृणी सत्यपराक्रमः}
{मुख्यस्येक्ष्वाकुवंशस्य लक्ष्मीवाँल्लक्ष्मिवर्धनः} %5-31-4

\twolineshloka
{पार्थिवव्यञ्जनैर्युक्तः पृथुश्रीः पार्थिवर्षभः}
{पृथिव्यां चतुरन्तायां विश्रुतः सुखदः सुखी} %5-31-5

\twolineshloka
{तस्य पुत्रः प्रियो ज्येष्ठस्ताराधिपनिभाननः}
{रामो नाम विशेषज्ञः श्रेष्ठः सर्वधनुष्मताम्} %5-31-6

\twolineshloka
{रक्षिता स्वस्य वृत्तस्य स्वजनस्यापि रक्षिता}
{रक्षिता जीवलोकस्य धर्मस्य च परन्तपः} %5-31-7

\twolineshloka
{तस्य सत्याभिसन्धस्य वृद्धस्य वचनात् पितुः}
{सभार्यः सह च भ्रात्रा वीरः प्रव्रजितो वनम्} %5-31-8

\twolineshloka
{तेन तत्र महारण्ये मृगयां परिधावता}
{राक्षसा निहताः शूरा बहवः कामरूपिणः} %5-31-9

\twolineshloka
{जनस्थानवधं श्रुत्वा निहतौ खरदूषणौ}
{ततस्त्वमर्षापहृता जानकी रावणेन तु} %5-31-10

\twolineshloka
{वञ्चयित्वा वने रामं मृगरूपेण मायया}
{स मार्गमाणस्तां देवीं रामः सीतामनिन्दिताम्} %5-31-11

\twolineshloka
{आससाद वने मित्रं सुग्रीवं नाम वानरम्}
{ततः स वालिनं हत्वा रामः परपुरञ्जयः} %5-31-12

\twolineshloka
{आयच्छत् कपिराज्यं तु सुग्रीवाय महात्मने}
{सुग्रीवेणाभिसन्दिष्टा हरयः कामरूपिणः} %5-31-13

\twolineshloka
{दिक्षु सर्वासु तां देवीं विचिन्वन्तः सहस्रशः}
{अहं सम्पातिवचनाच्छतयोजनमायतम्} %5-31-14

\twolineshloka
{तस्या हेतोर्विशालाक्ष्याः समुद्रं वेगवान् प्लुतः}
{यथारूपां यथावर्णां यथालक्ष्मवतीं च ताम्} %5-31-15

\twolineshloka
{अश्रौषं राघवस्याहं सेयमासादिता मया}
{विररामैवमुक्त्वा स वाचं वानरपुङ्गवः} %5-31-16

\threelineshloka
{जानकी चापि तच्छ्रुत्वा विस्मयं परमं गता}
{ततः सा वक्रकेशान्ता सुकेशी केशसंवृतम्}
{उन्नम्य वदनं भीरुः शिंशपामन्ववैक्षत} %5-31-17

\twolineshloka
{निशम्य सीता वचनं कपेश्च दिशश्च सर्वाः प्रदिशश्च वीक्ष्य}
{स्वयं प्रहर्षं परमं जगाम सर्वात्मना राममनुस्मरन्ती} %5-31-18

\twolineshloka
{सा तिर्यगूर्ध्वं च तथा ह्यधस्तान्निरीक्षमाणा तमचिन्त्यबुद्धिम्}
{ददर्श पिङ्गाधिपतेरमात्यं वातात्मजं सूर्यमिवोदयस्थम्} %5-31-19


॥इत्यार्षे श्रीमद्रामायणे वाल्मीकीये आदिकाव्ये सुन्दरकाण्डे रामवृत्तसंश्रवः नाम एकत्रिंशः सर्गः ॥५-३१॥
