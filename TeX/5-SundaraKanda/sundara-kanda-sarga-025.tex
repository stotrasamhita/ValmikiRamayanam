\sect{पञ्चविंशः सर्गः — सीतानिर्वेदः}

\twolineshloka
{अथ तासां वदन्तीनां परुषं दारुणं बहु}
{राक्षसीनामसौम्यानां रुरोद जनकात्मजा} %5-25-1

\twolineshloka
{एवमुक्ता तु वैदेही राक्षसीभिर्मनस्विनी}
{उवाच परमत्रस्ता बाष्पगद्गदया गिरा} %5-25-2

\twolineshloka
{न मानुषी राक्षसस्य भार्या भवितुमर्हति}
{कामं खादत मां सर्वा न करिष्यामि वो वचः} %5-25-3

\twolineshloka
{सा राक्षसीमध्यगता सीता सुरसुतोपमा}
{न शर्म लेभे शोकार्ता रावणेनेव भर्त्सिता} %5-25-4

\twolineshloka
{वेपते स्माधिकं सीता विशन्तीवाङ्गमात्मनः}
{वने यूथपरिभ्रष्टा मृगी कोकैरिवार्दिता} %5-25-5

\twolineshloka
{सा त्वशोकस्य विपुलां शाखामालम्ब्य पुष्पिताम्}
{चिन्तयामास शोकेन भर्तारं भग्नमानसा} %5-25-6

\twolineshloka
{सा स्नापयन्ती विपुलौ स्तनौ नेत्रजलस्रवैः}
{चिन्तयन्ती न शोकस्य तदान्तमधिगच्छति} %5-25-7

\twolineshloka
{सा वेपमाना पतिता प्रवाते कदली यथा}
{राक्षसीनां भयत्रस्ता विवर्णवदनाभवत्} %5-25-8

\twolineshloka
{तस्याः सा दीर्घबहुला वेपन्त्याः सीतया तदा}
{ददृशे कम्पिता वेणी व्यालीव परिसर्पती} %5-25-9

\twolineshloka
{सा निःश्वसन्ती शोकार्ता कोपोपहतचेतना}
{आर्ता व्यसृजदश्रूणि मैथिली विललाप च} %5-25-10

\twolineshloka
{हा रामेति च दुःखार्ता हा पुनर्लक्ष्मणेति च}
{हा श्वश्रूर्मम कौसल्ये हा सुमित्रेति भामिनी} %5-25-11

\twolineshloka
{लोकप्रवादः सत्योऽयं पण्डितैः समुदाहृतः}
{अकाले दुर्लभो मृत्युः स्त्रिया वा पुरुषस्य वा} %5-25-12

\twolineshloka
{यत्राहमाभिः क्रूराभी राक्षसीभिरिहार्दिता}
{जीवामि हीना रामेण मुहूर्तमपि दुःखिता} %5-25-13

\twolineshloka
{एषाल्पपुण्या कृपणा विनशिष्याम्यनाथवत्}
{समुद्रमध्ये नौः पूर्णा वायुवेगैरिवाहता} %5-25-14

\twolineshloka
{भर्तारं तमपश्यन्ती राक्षसीवशमागता}
{सीदामि खलु शोकेन कूलं तोयहतं यथा} %5-25-15

\twolineshloka
{तं पद्मदलपत्राक्षं सिंहविक्रान्तगामिनम्}
{धन्याः पश्यन्ति मे नाथं कृतज्ञं प्रियवादिनम्} %5-25-16

\twolineshloka
{सर्वथा तेन हीनाया रामेण विदितात्मना}
{तीक्ष्णं विषमिवास्वाद्य दुर्लभं मम जीवनम्} %5-25-17

\twolineshloka
{कीदृशं तु महापापं मया देहान्तरे कृतम्}
{तेनेदं प्राप्यते घोरं महादुःखं सुदारुणम्} %5-25-18

\twolineshloka
{जीवितं त्यक्तुमिच्छामि शोकेन महता वृता}
{राक्षसीभिश्च रक्षन्त्या रामो नासाद्यते मया} %5-25-19

\twolineshloka
{धिगस्तु खलु मानुष्यं धिगस्तु परवश्यताम्}
{न शक्यं यत् परित्यक्तुमात्मच्छन्देन जीवितम्} %5-25-20


॥इत्यार्षे श्रीमद्रामायणे वाल्मीकीये आदिकाव्ये सुन्दरकाण्डे सीतानिर्वेदः नाम पञ्चविंशः सर्गः ॥५-२५॥
