\sect{एकपञ्चाशः सर्गः — हनुमदुपदेशः}

\twolineshloka
{तं समीक्ष्य महासत्त्वं सत्त्ववान् हरिसत्तमः}
{वाक्यमर्थवदव्यग्रस्तमुवाच दशाननम्} %5-51-1

\twolineshloka
{अहं सुग्रीवसन्देशादिह प्राप्तस्तवान्तिके}
{राक्षसेश हरीशस्त्वां भ्राता कुशलमब्रवीत्} %5-51-2

\twolineshloka
{भ्रातुः श्रृणु समादेशं सुग्रीवस्य महात्मनः}
{धर्मार्थसहितं वाक्यमिह चामुत्र च क्षमम्} %5-51-3

\twolineshloka
{राजा दशरथो नाम रथकुञ्जरवाजिमान्}
{पितेव बन्धुर्लोकस्य सुरेश्वरसमद्युतिः} %5-51-4

\twolineshloka
{ज्येष्ठस्तस्य महाबाहुः पुत्रः प्रियतरः प्रभुः}
{पितुर्निदेशान्निष्क्रान्तः प्रविष्टो दण्डकावनम्} %5-51-5

\twolineshloka
{लक्ष्मणेन सह भ्राता सीतया सह भार्यया}
{रामो नाम महातेजा धर्म्यं पन्थानमाश्रितः} %5-51-6

\twolineshloka
{तस्य भार्या जनस्थाने भ्रष्टा सीतेति विश्रुता}
{वैदेहस्य सुता राज्ञो जनकस्य महात्मनः} %5-51-7

\twolineshloka
{मार्गमाणस्तु तां देवीं राजपुत्रः सहानुजः}
{ऋष्यमूकमनुप्राप्तः सुग्रीवेण च सङ्गतः} %5-51-8

\twolineshloka
{तस्य तेन प्रतिज्ञातं सीतायाः परिमार्गणम्}
{सुग्रीवस्यापि रामेण हरिराज्यं निवेदितुम्} %5-51-9

\twolineshloka
{ततस्तेन मृधे हत्वा राजपुत्रेण वालिनम्}
{सुग्रीवः स्थापितो राज्ये हर्यृक्षाणां गणेश्वरः} %5-51-10

\twolineshloka
{त्वया विज्ञातपूर्वश्च वाली वानरपुङ्गवः}
{स तेन निहतः सङ्ख्ये शरेणैकेन वानरः} %5-51-11

\twolineshloka
{स सीतामार्गणे व्यग्रः सुग्रीवः सत्यसङ्गरः}
{हरीन् सम्प्रेषयामास दिशः सर्वा हरीश्वरः} %5-51-12

\twolineshloka
{तां हरीणां सहस्राणि शतानि नियुतानि च}
{दिक्षु सर्वासु मार्गन्ते ह्यधश्चोपरि चाम्बरे} %5-51-13

\twolineshloka
{वैनतेयसमाः केचित् केचित् तत्रानिलोपमाः}
{असङ्गगतयः शीघ्रा हरिवीरा महाबलाः} %5-51-14

\twolineshloka
{अहं तु हनुमान्नाम मारुतस्यौरसः सुतः}
{सीतायास्तु कृते तूर्णं शतयोजनमायतम्} %5-51-15

\twolineshloka
{समुद्रं लङ्घयित्वैव त्वां दिदृक्षुरिहागतः}
{भ्रमता च मया दृष्टा गृहे ते जनकात्मजा} %5-51-16

\twolineshloka
{तद् भवान् दृष्टधर्मार्थस्तपःकृतपरिग्रहः}
{परदारान् महाप्राज्ञ नोपरोद्धुं त्वमर्हसि} %5-51-17

\twolineshloka
{नहि धर्मविरुद्धेषु बह्वपायेषु कर्मसु}
{मूलघातिषु सज्जन्ते बुद्धिमन्तो भवद्विधाः} %5-51-18

\twolineshloka
{कश्च लक्ष्मणमुक्तानां रामकोपानुवर्तिनाम्}
{शराणामग्रतः स्थातुं शक्तो देवासुरेष्वपि} %5-51-19

\twolineshloka
{न चापि त्रिषु लोकेषु राजन् विद्येत कश्चन}
{राघवस्य व्यलीकं यः कृत्वा सुखमवाप्नुयात्} %5-51-20

\twolineshloka
{तत् त्रिकालहितं वाक्यं धर्म्यमर्थानुयायि च}
{मन्यस्व नरदेवाय जानकी प्रतिदीयताम्} %5-51-21

\twolineshloka
{दृष्टा हीयं मया देवी लब्धं यदिह दुर्लभम्}
{उत्तरं कर्म यच्छेषं निमित्तं तत्र राघवः} %5-51-22

\twolineshloka
{लक्षितेयं मया सीता तथा शोकपरायणा}
{गृहे यां नाभिजानासि पञ्चास्यामिव पन्नगीम्} %5-51-23

\twolineshloka
{नेयं जरयितुं शक्या सासुरैरमरैरपि}
{विषसंस्पृष्टमत्यर्थं भुक्तमन्नमिवौजसा} %5-51-24

\twolineshloka
{तपःसन्तापलब्धस्ते सोऽयं धर्मपरिग्रहः}
{न स नाशयितुं न्याय्य आत्मप्राणपरिग्रहः} %5-51-25

\twolineshloka
{अवध्यतां तपोभिर्यां भवान् समनुपश्यति}
{आत्मनः सासुरैर्देवैर्हेतुस्तत्राप्ययं महान्} %5-51-26

\threelineshloka
{सुग्रीवो न च देवोऽयं न यक्षो न च राक्षसः}
{मानुषो राघवो राजन् सुग्रीवश्च हरीश्वरः}
{तस्मात् प्राणपरित्राणं कथं राजन् करिष्यसि} %5-51-27

\twolineshloka
{न तु धर्मोपसंहारमधर्मफलसंहितम्}
{तदेव फलमन्वेति धर्मश्चाधर्मनाशनः} %5-51-28

\twolineshloka
{प्राप्तं धर्मफलं तावद् भवता नात्र संशयः}
{फलमस्याप्यधर्मस्य क्षिप्रमेव प्रपत्स्यसे} %5-51-29

\twolineshloka
{जनस्थानवधं बुद्ध्वा वालिनश्च वधं तथा}
{रामसुग्रीवसख्यं च बुद्ध्यस्व हितमात्मनः} %5-51-30

\twolineshloka
{कामं खल्वहमप्येकः सवाजिरथकुञ्जराम्}
{लङ्कां नाशयितुं शक्तस्तस्यैष तु न निश्चयः} %5-51-31

\twolineshloka
{रामेण हि प्रतिज्ञातं हर्यृक्षगणसन्निधौ}
{उत्सादनममित्राणां सीता यैस्तु प्रधर्षिता} %5-51-32

\twolineshloka
{अपकुर्वन् हि रामस्य साक्षादपि पुरन्दरः}
{न सुखं प्राप्नुयादन्यः किं पुनस्त्वद्विधो जनः} %5-51-33

\twolineshloka
{यां सीतेत्यभिजानासि येयं तिष्ठति ते गृहे}
{कालरात्रीति तां विद्धि सर्वलङ्काविनाशिनीम्} %5-51-34

\twolineshloka
{तदलं कालपाशेन सीताविग्रहरूपिणा}
{स्वयं स्कन्धावसक्तेन क्षेममात्मनि चिन्त्यताम्} %5-51-35

\twolineshloka
{सीतायास्तेजसा दग्धां रामकोपप्रदीपिताम्}
{दह्यमानामिमां पश्य पुरीं साट्टप्रतोलिकाम्} %5-51-36

\twolineshloka
{स्वानि मित्राणि मन्त्रींश्च ज्ञातीन् भ्रातॄन् सुतान्हितान्}
{भोगान् दारांश्च लङ्कां च मा विनाशमुपानय} %5-51-37

\twolineshloka
{सत्यं राक्षसराजेन्द्र शृणुष्व वचनं मम}
{रामदासस्य दूतस्य वानरस्य विशेषतः} %5-51-38

\twolineshloka
{सर्वाल्ँ लोकान् सुसंहृत्य सभूतान् सचराचरान्}
{पुनरेव तथा स्रष्टुं शक्तो रामो महायशाः} %5-51-39

\twolineshloka
{देवासुरनरेन्द्रेषु यक्षरक्षोरगेषु च}
{विद्याधरेषु नागेषु गन्धर्वेषु मृगेषु च} %5-51-40

\twolineshloka
{सिद्धेषु किन्नरेन्द्रेषु पतत्त्रिषु च सर्वतः}
{सर्वत्र सर्वभूतेषु सर्वकालेषु नास्ति सः} %5-51-41

\threelineshloka
{यो रामं प्रति युध्येत विष्णुतुल्यपराक्रमम्}
{सर्वलोकेश्वरस्येह कृत्वा विप्रियमीदृशम्}
{रामस्य राजसिंहस्य दुर्लभं तव जीवितम्} %5-51-42

\twolineshloka
{देवाश्च दैत्याश्च निशाचरेन्द्र गन्धर्वविद्याधरनागयक्षाः}
{रामस्य लोकत्रयनायकस्य स्थातुं न शक्ताः समरेषु सर्वे} %5-51-43

\twolineshloka
{ब्रह्मा स्वयम्भूश्चतुराननो वा रुद्रस्त्रिनेत्रस्त्रिपुरान्तको वा}
{इन्द्रो महेन्द्रः सुरनायको वा स्थातुं न शक्ता युधि राघवस्य} %5-51-44

\twolineshloka
{स सौष्ठवोपेतमदीनवादिनः कपेर्निशम्याप्रतिमोऽप्रियं वचः}
{दशाननः कोपविवृत्तलोचनः समादिशत् तस्य वधं महाकपेः} %5-51-45


॥इत्यार्षे श्रीमद्रामायणे वाल्मीकीये आदिकाव्ये सुन्दरकाण्डे हनुमदुपदेशः नाम एकपञ्चाशः सर्गः ॥५-५१॥
