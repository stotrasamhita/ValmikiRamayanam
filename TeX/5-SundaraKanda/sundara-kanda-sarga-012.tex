\sect{द्वादशः सर्गः — हनूमद्विषादः}

\twolineshloka
{स तस्य मध्ये भवनस्य संस्थितो लतागृहांश्चित्रगृहान् निशागृहान्}
{जगाम सीतां प्रतिदर्शनोत्सुको न चैव तां पश्यति चारुदर्शनाम्} %5-12-1

\twolineshloka
{स चिन्तयामास ततो महाकपिः प्रियामपश्यन् रघुनन्दनस्य ताम्}
{ध्रुवं न सीता ध्रियते यथा न मे विचिन्वतो दर्शनमेति मैथिली} %5-12-2

\twolineshloka
{सा राक्षसानां प्रवरेण जानकी स्वशीलसंरक्षणतत्परा सती}
{अनेन नूनं प्रति दुष्टकर्मणा हता भवेदार्यपथे परे स्थिता} %5-12-3

\twolineshloka
{विरूपरूपा विकृता विवर्चसो महानना दीर्घविरूपदर्शनाः}
{समीक्ष्य ता राक्षसराजयोषितो भयाद् विनष्टा जनकेश्वरात्मजा} %5-12-4

\twolineshloka
{सीतामदृष्ट्वा ह्यनवाप्य पौरुषं विहृत्य कालं सह वानरैश्चिरम्}
{न मेऽस्ति सुग्रीवसमीपगा गतिः सुतीक्ष्णदण्डो बलवांश्च वानरः} %5-12-5

\twolineshloka
{दृष्टमन्तःपुरं सर्वं दृष्टा रावणयोषितः}
{न सीता दृश्यते साध्वी वृथा जातो मम श्रमः} %5-12-6

\twolineshloka
{किं नु मां वानराः सर्वे गतं वक्ष्यन्ति संगताः}
{गत्वा तत्र त्वया वीर किं कृतं तद् वदस्व नः} %5-12-7

\twolineshloka
{अदृष्ट्वा किं प्रवक्ष्यामि तामहं जनकात्मजाम्}
{ध्रुवं प्रायमुपासिष्ये कालस्य व्यतिवर्तने} %5-12-8

\twolineshloka
{किं वा वक्ष्यति वृद्धश्च जाम्बवानंगदश्च सः}
{गतं पारं समुद्रस्य वानराश्च समागताः} %5-12-9

\twolineshloka
{अनिर्वेदः श्रियो मूलमनिर्वेदः परं सुखम्}
{भूयस्तत्र विचेष्यामि न यत्र विचयः कृतः} %5-12-10

\twolineshloka
{अनिर्वेदो हि सततं सर्वार्थेषु प्रवर्तकः}
{करोति सफलं जन्तोः कर्म यच्च करोति सः} %5-12-11

\twolineshloka
{तस्मादनिर्वेदकरं यत्नं चेष्टेऽहमुत्तमम्}
{अदृष्टांश्च विचेष्यामि देशान् रावणपालितान्} %5-12-12

\twolineshloka
{आपानशाला विचितास्तथा पुष्पगृहाणि च}
{चित्रशालाश्च विचिता भूयः क्रीडागृहाणि च} %5-12-13

\twolineshloka
{निष्कुटान्तररथ्याश्च विमानानि च सर्वशः}
{इति संचिन्त्य भूयोऽपि विचेतुमुपचक्रमे} %5-12-14

\twolineshloka
{भूमीगृहांश्चैत्यगृहान् गृहातिगृहकानपि}
{उत्पतन् निपतंश्चापि तिष्ठन् गच्छन् पुनः क्वचित्} %5-12-15

\twolineshloka
{अपवृण्वंश्च द्वाराणि कपाटान्यवघट्टयन्}
{प्रविशन् निष्पतंश्चापि प्रपतन्नुत्पतन्निव} %5-12-16

\threelineshloka
{सर्वमप्यवकाशं स विचचार महाकपिः}
{चतुरंगुलमात्रोऽपि नावकाशः स विद्यते}
{रावणान्तःपुरे तस्मिन् यं कपिर्न जगाम सः} %5-12-17

\twolineshloka
{प्राकारान्तरवीथ्यश्च वेदिकाश्चैत्यसंश्रयाः}
{श्वभ्राश्च पुष्करिण्यश्च सर्वं तेनावलोकितम्} %5-12-18

\twolineshloka
{राक्षस्यो विविधाकारा विरूपा विकृतास्तथा}
{दृष्टा हनुमता तत्र न तु सा जनकात्मजा} %5-12-19

\twolineshloka
{रूपेणाप्रतिमा लोके परा विद्याधरस्त्रियः}
{दृष्टा हनुमता तत्र न तु राघवनन्दिनी} %5-12-20

\twolineshloka
{नागकन्या वरारोहाः पूर्णचन्द्रनिभाननाः}
{दृष्टा हनुमता तत्र न तु सा जनकात्मजा} %5-12-21

\twolineshloka
{प्रमथ्य राक्षसेन्द्रेण नागकन्या बलाद्धृताः}
{दृष्टा हनुमता तत्र न सा जनकनन्दिनी} %5-12-22

\twolineshloka
{सोऽपश्यंस्तां महाबाहुः पश्यंश्चान्या वरस्त्रियः}
{विषसाद महाबाहुर्हनूमान् मारुतात्मजः} %5-12-23

\twolineshloka
{उद्योगं वानरेन्द्राणां प्लवनं सागरस्य च}
{व्यर्थं वीक्ष्यानिलसुतश्चिन्तां पुनरुपागतः} %5-12-24

\twolineshloka
{अवतीर्य विमानाच्च हनूमान् मारुतात्मजः}
{चिन्तामुपजगामाथ शोकोपहतचेतनः} %5-12-25


॥इत्यार्षे श्रीमद्रामायणे वाल्मीकीये आदिकाव्ये सुन्दरकाण्डे हनूमद्विषादः नाम द्वादशः सर्गः ॥५-१२॥
