\sect{अष्टषष्ठितमः सर्गः — हनूमत्समाश्वासवचनानुवादः}

\twolineshloka
{अथाहमुत्तरं देव्या पुनरुक्तः ससम्भ्रमम्}
{तव स्नेहान्नरव्याघ्र सौहार्दादनुमान्य च} %5-68-1

\twolineshloka
{एवं बहुविधं वाच्यो रामो दाशरथिस्त्वया}
{यथा मां प्राप्नुयाच्छीघ्रं हत्वा रावणमाहवे} %5-68-2

\twolineshloka
{यदि वा मन्यसे वीर वसैकाहमरिन्दम}
{कस्मिंश्चित् संवृते देशे विश्रान्तः श्वो गमिष्यसि} %5-68-3

\twolineshloka
{मम चाप्यल्पभाग्यायाः सान्निध्यात् तव वानर}
{अस्य शोकविपाकस्य मुहूर्तं स्याद् विमोक्षणम्} %5-68-4

\twolineshloka
{गते हि त्वयि विक्रान्ते पुनरागमनाय वै}
{प्राणानामपि सन्देहो मम स्यान्नात्र संशयः} %5-68-5

\twolineshloka
{तवादर्शनजः शोको भूयो मां परितापयेत्}
{दुखाद् दुःखपराभूतां दुर्गतां दुःखभागिनीम्} %5-68-6

\twolineshloka
{अयं च वीर सन्देहस्तिष्ठतीव ममाग्रतः}
{सुमहांस्त्वत्सहायेषु हर्यृक्षेषु हरीश्वर} %5-68-7

\twolineshloka
{कथं नु खलु दुष्पारं तरिष्यन्ति महोदधिम्}
{तानि हर्यृक्षसैन्यानि तौ वा नरवरात्मजौ} %5-68-8

\twolineshloka
{त्रयाणामेव भूतानां सागरस्यास्य लङ्घने}
{शक्तिः स्याद् वैनतेयस्य वायोर्वा तव चानघ} %5-68-9

\twolineshloka
{तदस्मिन् कार्यनिर्योगे वीरैवं दुरतिक्रमे}
{किं पश्यसि समाधानं ब्रूहि कार्यविदां वर} %5-68-10

\twolineshloka
{काममस्य त्वमेवैकः कार्यस्य परिसाधने}
{पर्याप्तः परवीरघ्न यशस्यस्ते बलोदयः} %5-68-11

\twolineshloka
{बलैः समग्रैर्यदि मां हत्वा रावणमाहवे}
{विजयी स्वपुरीं रामो नयेत् तत् स्याद् यशस्करम्} %5-68-12

\twolineshloka
{यथाहं तस्य वीरस्य वनादुपधिना हृता}
{रक्षसा तद्भयादेव तथा नार्हति राघवः} %5-68-13

\twolineshloka
{बलैस्तु सङ्कुलां कृत्वा लङ्कां परबलार्दनः}
{मां नयेद् यदि काकुत्स्थस्तत् तस्य सदृशं भवेत्} %5-68-14

\twolineshloka
{तद् यथा तस्य विक्रान्तमनुरूपं महात्मनः}
{भवत्याहवशूरस्य तथा त्वमुपपादय} %5-68-15

\twolineshloka
{तदर्थोपहितं वाक्यं प्रश्रितं हेतुसंहितम्}
{निशम्याहं ततः शेषं वाक्यमुत्तरमब्रवम्} %5-68-16

\twolineshloka
{देवि हर्यृक्षसैन्यानामीश्वरः प्लवतां वरः}
{सुग्रीवः सत्त्वसम्पन्नस्त्वदर्थे कृतनिश्चयः} %5-68-17

\twolineshloka
{तस्य विक्रमसम्पन्नाः सत्त्ववन्तो महाबलाः}
{मनःसङ्कल्पसदृशा निदेशे हरयः स्थिताः} %5-68-18

\twolineshloka
{येषां नोपरि नाधस्तान्न तिर्यक् सज्जते गतिः}
{न च कर्मसु सीदन्ति महत्स्वमिततेजसः} %5-68-19

\twolineshloka
{असकृत् तैर्महाभागैर्वानरैर्बलसंयुतैः}
{प्रदक्षिणीकृता भूमिर्वायुमार्गानुसारिभिः} %5-68-20

\twolineshloka
{मद्विशिष्टाश्च तुल्याश्च सन्ति तत्र वनौकसः}
{मत्तः प्रत्यवरः कश्चिन्नास्ति सुग्रीवसन्निधौ} %5-68-21

\twolineshloka
{अहं तावदिह प्राप्तः किं पुनस्ते महाबलाः}
{नहि प्रकृष्टाः प्रेष्यन्ते प्रेष्यन्ते हीतरे जनाः} %5-68-22

\twolineshloka
{तदलं परितापेन देवि मन्युरपैतु ते}
{एकोत्पातेन ते लङ्कामेष्यन्ति हरियूथपाः} %5-68-23

\twolineshloka
{मम पृष्ठगतौ तौ च चन्द्रसूर्याविवोदितौ}
{त्वत्सकाशं महाभागे नृसिंहावागमिष्यतः} %5-68-24

\twolineshloka
{अरिघ्नं सिंहसङ्काशं क्षिप्रं द्रक्ष्यसि राघवम्}
{लक्ष्मणं च धनुष्मन्तं लङ्काद्वारमुपागतम्} %5-68-25

\twolineshloka
{नखदंष्ट्रायुधान् वीरान् सिंहशार्दूलविक्रमान्}
{वानरान् वारणेन्द्राभान् क्षिप्रं द्रक्ष्यसि सङ्गतान्} %5-68-26

\twolineshloka
{शैलाम्बुदनिकाशानां लङ्कामलयसानुषु}
{नर्दतां कपिमुख्यानां नचिराच्छ्रोष्यसे स्वनम्} %5-68-27

\twolineshloka
{निवृत्तवनवासं च त्वया सार्धमरिन्दमम्}
{अभिषिक्तमयोध्यायां क्षिप्रं द्रक्ष्यसि राघवम्} %5-68-28

\twolineshloka
{ततो मया वाग्भिरदीनभाषिणी शिवाभिरिष्टाभिरभिप्रसादिता}
{उवाह शान्तिं मम मैथिलात्मजा तवातिशोकेन तथातिपीडिता} %5-68-29


॥इत्यार्षे श्रीमद्रामायणे वाल्मीकीये आदिकाव्ये सुन्दरकाण्डे हनूमत्समाश्वासवचनानुवादः नाम अष्टषष्ठितमः सर्गः ॥५-६८॥
