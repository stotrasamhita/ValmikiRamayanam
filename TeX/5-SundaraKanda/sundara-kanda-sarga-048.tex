\sect{अष्टचत्वारिंशः सर्गः — इन्द्रजिदभियोगः}

\twolineshloka
{ततस्तु रक्षोऽधिपतिर्महात्मा हनूमताक्षे निहते कुमारे}
{मनः समाधाय स देवकल्पं समादिदेशेन्द्रजितं सरोषः} %5-48-1

\twolineshloka
{त्वमस्त्रविच्छस्त्रभृतां वरिष्ठः सुरासुराणामपि शोकदाता}
{सुरेषु सेन्द्रेषु च दृष्टकर्मा पितामहाराधनसंचितास्त्रः} %5-48-2

\twolineshloka
{त्वदस्त्रबलमासाद्य ससुराः समरुद्गणाः}
{न शेकुः समरे स्थातुं सुरेश्वरसमाश्रिताः} %5-48-3

\threelineshloka
{न कश्चित् त्रिषु लोकेषु संयुगे न गतश्रमः}
{भुजवीर्याभिगुप्तश्च तपसा चाभिरक्षितः}
{देशकालप्रधानश्च त्वमेव मतिसत्तमः} %5-48-4

\twolineshloka
{न तेऽस्त्यशक्यं समरेषु कर्मणां न तेऽस्त्यकार्यं मतिपूर्वमन्त्रणे}
{न सोऽस्ति कश्चित् त्रिषु संग्रहेषु न वेद यस्तेऽस्त्रबलं बलं च} %5-48-5

\twolineshloka
{ममानुरूपं तपसो बलं च ते पराक्रमश्चास्त्रबलं च संयुगे}
{न त्वां समासाद्य रणावमर्दे मनः श्रमं गच्छति निश्चितार्थम्} %5-48-6

\twolineshloka
{निहताः किंकराः सर्वे जम्बुमाली च राक्षसः}
{अमात्यपुत्रा वीराश्च पञ्च सेनाग्रगामिनः} %5-48-7

\threelineshloka
{बलानि सुसमृद्धानि साश्वनागरथानि च}
{सहोदरस्ते दयितः कुमारोऽक्षश्च सूदितः}
{न तु तेष्वेव मे सारो यस्त्वय्यरिनिषूदन} %5-48-8

\twolineshloka
{इदं च दृष्ट्वा निहतं महद् बलं कपेः प्रभावं च पराक्रमं च}
{त्वमात्मनश्चापि निरीक्ष्य सारं कुरुष्व वेगं स्वबलानुरूपम्} %5-48-9

\twolineshloka
{बलावमर्दस्त्वयि संनिकृष्टे यथा गते शाम्यति शान्तशत्रौ}
{तथा समीक्ष्यात्मबलं परं च समारभस्वास्त्रभृतां वरिष्ठ} %5-48-10

\twolineshloka
{न वीर सेना गणशो च्यवन्ति न वज्रमादाय विशालसारम्}
{न मारुतस्यास्ति गतिप्रमाणं न चाग्निकल्पः करणेन हन्तुम्} %5-48-11

\twolineshloka
{तमेवमर्थं प्रसमीक्ष्य सम्यक् स्वकर्मसाम्याद्धि समाहितात्मा}
{स्मरंश्च दिव्यं धनुषोऽस्य वीर्यं व्रजाक्षतं कर्म समारभस्व} %5-48-12

\twolineshloka
{न खल्वियं मतिश्रेष्ठ यत्त्वां सम्प्रेषयाम्यहम्}
{इयं च राजधर्माणां क्षत्रस्य च मतिर्मता} %5-48-13

\twolineshloka
{नानाशस्त्रेषु संग्रामे वैशारद्यमरिंदम}
{अवश्यमेव बोद्धव्यं काम्यश्च विजयो रणे} %5-48-14

\twolineshloka
{ततः पितुस्तद्वचनं निशम्य प्रदक्षिणं दक्षसुतप्रभावः}
{चकार भर्तारमतित्वरेण रणाय वीरः प्रतिपन्नबुद्धिः} %5-48-15

\twolineshloka
{ततस्तैः स्वगणैरिष्टैरिन्द्रजित् प्रतिपूजितः}
{युद्धोद्धतकृतोत्साहः संग्रामं सम्प्रपद्यत} %5-48-16

\twolineshloka
{श्रीमान् पद्मविशालाक्षो राक्षसाधिपतेः सुतः}
{निर्जगाम महातेजाः समुद्र इव पर्वणि} %5-48-17

\twolineshloka
{स पक्षिराजोपमतुल्यवेगैर्व्याघ्रैश्चतुर्भिः स तु तीक्ष्णदंष्ट्रैः}
{रथं समायुक्तमसह्यवेगः समारुरोहेन्द्रजिदिन्द्रकल्पः} %5-48-18

\twolineshloka
{स रथी धन्विनां श्रेष्ठः शस्त्रज्ञोऽस्त्रविदां वरः}
{रथेनाभिययौ क्षिप्रं हनूमान् यत्र सोऽभवत्} %5-48-19

\twolineshloka
{स तस्य रथनिर्घोषं ज्यास्वनं कार्मुकस्य च}
{निशम्य हरिवीरोऽसौ सम्प्रहृष्टतरोऽभवत्} %5-48-20

\twolineshloka
{इन्द्रजिच्चापमादाय शितशल्यांश्च सायकान्}
{हनूमन्तमभिप्रेत्य जगाम रणपण्डितः} %5-48-21

\twolineshloka
{तस्मिंस्ततः संयति जातहर्षे रणाय निर्गच्छति बाणपाणौ}
{दिशश्च सर्वाः कलुषा बभूवुर्मृगाश्च रौद्रा बहुधा विनेदुः} %5-48-22

\twolineshloka
{समागतास्तत्र तु नागयक्षा महर्षयश्चक्रचराश्च सिद्धाः}
{नभः समावृत्य च पक्षिसङ्घा विनेदुरुच्चैः परमप्रहृष्टाः} %5-48-23

\twolineshloka
{आयान्तं स रथं दृष्ट्वा तूर्णमिन्द्रध्वजं कपिः}
{ननाद च महानादं व्यवर्धत च वेगवान्} %5-48-24

\twolineshloka
{इन्द्रजित् स रथं दिव्यमाश्रितश्चित्रकार्मुकः}
{धनुर्विस्फारयामास तडिदूर्जितनिःस्वनम्} %5-48-25

\twolineshloka
{ततः समेतावतितीक्ष्णवेगौ महाबलौ तौ रणनिर्विशङ्कौ}
{कपिश्च रक्षोऽधिपतेस्तनूजः सुरासुरेन्द्राविव बद्धवैरौ} %5-48-26

\twolineshloka
{स तस्य वीरस्य महारथस्य धनुष्मतः संयति सम्मतस्य}
{शरप्रवेगं व्यहनत् प्रवृद्धश्चचार मार्गे पितुरप्रमेयः} %5-48-27

\twolineshloka
{ततः शरानायततीक्ष्णशल्यान् सुपत्रिणः काञ्चनचित्रपुङ्खान्}
{मुमोच वीरः परवीरहन्ता सुसंततान् वज्रसमानवेगान्} %5-48-28

\twolineshloka
{ततः स तत्स्यन्दननिःस्वनं च मृदङ्गभेरीपटहस्वनं च}
{विकृष्यमाणस्य च कार्मुकस्य निशम्य घोषं पुनरुत्पपात} %5-48-29

\twolineshloka
{शराणामन्तरेष्वाशु व्यावर्तत महाकपिः}
{हरिस्तस्याभिलक्ष्यस्य मोक्षयँल्लक्ष्यसंग्रहम्} %5-48-30

\twolineshloka
{शराणामग्रतस्तस्य पुनः समभिवर्तत}
{प्रसार्य हस्तौ हनुमानुत्पपातानिलात्मजः} %5-48-31

\twolineshloka
{तावुभौ वेगसम्पन्नौ रणकर्मविशारदौ}
{सर्वभूतमनोग्राहि चक्रतुर्युद्धमुत्तमम्} %5-48-32

\twolineshloka
{हनूमतो वेद न राक्षसोऽन्तरं न मारुतिस्तस्य महात्मनोऽन्तरम्}
{परस्परं निर्विषहौ बभूवतुः समेत्य तौ देवसमानविक्रमौ} %5-48-33

\twolineshloka
{ततस्तु लक्ष्ये स विहन्यमाने शरेष्वमोघेषु च सम्पतत्सु}
{जगाम चिन्तां महतीं महात्मा समाधिसंयोगसमाहितात्मा} %5-48-34

\twolineshloka
{ततो मतिं राक्षसराजसूनुश्चकार तस्मिन् हरिवीरमुख्ये}
{अवध्यतां तस्य कपेः समीक्ष्य कथं निगच्छेदिति निग्रहार्थम्} %5-48-35

\twolineshloka
{ततः पैतामहं वीरः सोऽस्त्रमस्त्रविदां वरः}
{संदधे सुमहातेजास्तं हरिप्रवरं प्रति} %5-48-36

\twolineshloka
{अवध्योऽयमिति ज्ञात्वा तमस्त्रेणास्त्रतत्त्ववित्}
{निजग्राह महाबाहुं मारुतात्मजमिन्द्रजित्} %5-48-37

\twolineshloka
{तेन बद्धस्ततोऽस्त्रेण राक्षसेन स वानरः}
{अभवन्निर्विचेष्टश्च पपात च महीतले} %5-48-38

\twolineshloka
{ततोऽथ बुद्ध्वा स तदस्त्रबन्धं प्रभोः प्रभावाद् विगताल्पवेगः}
{पितामहानुग्रहमात्मनश्च विचिन्तयामास हरिप्रवीरः} %5-48-39

\twolineshloka
{ततः स्वायम्भुवैर्मन्त्रैर्ब्रह्मास्त्रं चाभिमन्त्रितम्}
{हनूमांश्चिन्तयामास वरदानं पितामहात्} %5-48-40

\twolineshloka
{न मेऽस्य बन्धस्य च शक्तिरस्ति विमोक्षणे लोकगुरोः प्रभावात्}
{इत्येवमेवं विहितोऽस्त्रबन्धो मयाऽऽत्मयोनेरनुवर्तितव्यः} %5-48-41

\twolineshloka
{स वीर्यमस्त्रस्य कपिर्विचार्य पितामहानुग्रहमात्मनश्च}
{विमोक्षशक्तिं परिचिन्तयित्वा पितामहाज्ञामनुवर्तते स्म} %5-48-42

\twolineshloka
{अस्त्रेणापि हि बद्धस्य भयं मम न जायते}
{पितामहमहेन्द्राभ्यां रक्षितस्यानिलेन च} %5-48-43

\twolineshloka
{ग्रहणे चापि रक्षोभिर्महन्मे गुणदर्शनम्}
{राक्षसेन्द्रेण संवादस्तस्माद् गृह्णन्तु मां परे} %5-48-44

\twolineshloka
{स निश्चितार्थः परवीरहन्ता समीक्ष्यकारी विनिवृत्तचेष्टः}
{परैः प्रसह्याभिगतैर्निगृह्य ननाद तैस्तैः परिभर्त्स्यमानः} %5-48-45

\twolineshloka
{ततस्ते राक्षसा दृष्ट्वा विनिश्चेष्टमरिंदमम्}
{बबन्धुः शणवल्कैश्च द्रुमचीरैश्च संहतैः} %5-48-46

\twolineshloka
{स रोचयामास परैश्च बन्धं प्रसह्य वीरैरभिगर्हणं च}
{कौतूहलान्मां यदि राक्षसेन्द्रो द्रष्टुं व्यवस्येदिति निश्चितार्थः} %5-48-47

\twolineshloka
{स बद्धस्तेन वल्केन विमुक्तोऽस्त्रेण वीर्यवान्}
{अस्त्रबन्धः स चान्यं हि न बन्धमनुवर्तते} %5-48-48

\twolineshloka
{अथेन्द्रजित् तं द्रुमचीरबद्धं विचार्य वीरः कपिसत्तमं तम्}
{विमुक्तमस्त्रेण जगाम चिन्तामन्येन बद्धोऽप्यनुवर्ततेऽस्त्रम्} %5-48-49

\twolineshloka
{अहो महत् कर्म कृतं निरर्थं न राक्षसैर्मन्त्रगतिर्विमृष्टा}
{पुनश्च नास्त्रे विहतेऽस्त्रमन्यत् प्रवर्तते संशयिताः स्म सर्वे} %5-48-50

\twolineshloka
{अस्त्रेण हनुमान् मुक्तो नात्मानमवबुध्यते}
{कृष्यमाणस्तु रक्षोभिस्तैश्च बन्धैर्निपीडितः} %5-48-51

\twolineshloka
{हन्यमानस्ततः क्रूरै राक्षसैः कालमुष्टिभिः}
{समीपं राक्षसेन्द्रस्य प्राकृष्यत स वानरः} %5-48-52

\twolineshloka
{अथेन्द्रजित् तं प्रसमीक्ष्य मुक्तमस्त्रेण बद्धं द्रुमचीरसूत्रैः}
{व्यदर्शयत् तत्र महाबलं तं हरिप्रवीरं सगणाय राज्ञे} %5-48-53

\twolineshloka
{तं मत्तमिव मातङ्गं बद्धं कपिवरोत्तमम्}
{राक्षसा राक्षसेन्द्राय रावणाय न्यवेदयन्} %5-48-54

\twolineshloka
{कोऽयं कस्य कुतो वापि किं कार्यं कोऽभ्युपाश्रयः}
{इति राक्षसवीराणां दृष्ट्वा संजज्ञिरे कथाः} %5-48-55

\twolineshloka
{हन्यतां दह्यतां वापि भक्ष्यतामिति चापरे}
{राक्षसास्तत्र संक्रुद्धाः परस्परमथाब्रुवन्} %5-48-56

\twolineshloka
{अतीत्य मार्गं सहसा महात्मा स तत्र रक्षोऽधिपपादमूले}
{ददर्श राज्ञः परिचारवृद्धान् गृहं महारत्नविभूषितं च} %5-48-57

\twolineshloka
{स ददर्श महातेजा रावणः कपिसत्तमम्}
{रक्षोभिर्विकृताकारैः कृष्यमाणमितस्ततः} %5-48-58

\twolineshloka
{राक्षसाधिपतिं चापि ददर्श कपिसत्तमः}
{तेजोबलसमायुक्तं तपन्तमिव भास्करम्} %5-48-59

\twolineshloka
{स रोषसंवर्तितताम्रदृष्टिर्दशाननस्तं कपिमन्ववेक्ष्य}
{अथोपविष्टान् कुलशीलवृद्धान् समादिशत् तं प्रति मुख्यमन्त्रीन्} %5-48-60

\twolineshloka
{यथाक्रमं तैः स कपिश्च पृष्टः कार्यार्थमर्थस्य च मूलमादौ}
{निवेदयामास हरीश्वरस्य दूतः सकाशादहमागतोऽस्मि} %5-48-61


॥इत्यार्षे श्रीमद्रामायणे वाल्मीकीये आदिकाव्ये सुन्दरकाण्डे इन्द्रजिदभियोगः नाम अष्टचत्वारिंशः सर्गः ॥५-४८॥
