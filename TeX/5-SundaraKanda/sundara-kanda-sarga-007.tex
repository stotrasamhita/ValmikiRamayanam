\sect{सप्तमः सर्गः — पुष्पकदर्शनम्}

\twolineshloka
{स वेश्मजालं बलवान् ददर्श व्यासक्तवैदूर्यसुवर्णजालम्}
{यथा महत्प्रावृषि मेघजालं विद्युत्पिनद्धं सविहङ्गजालम्} %5-7-1

\twolineshloka
{निवेशनानां विविधाश्च शालाः प्रधानशङ्खायुधचापशालाः}
{मनोहराश्चापि पुनर्विशाला ददर्श वेश्माद्रिषु चन्द्रशालाः} %5-7-2

\twolineshloka
{गृहाणि नानावसुराजितानि देवासुरैश्चापि सुपूजितानि}
{सर्वैश्च दोषैः परिवर्जितानि कपिर्ददर्श स्वबलार्जितानि} %5-7-3

\twolineshloka
{तानि प्रयत्नाभिसमाहितानि मयेन साक्षादिव निर्मितानि}
{महीतले सर्वगुणोत्तराणि ददर्श लङ्काधिपतेर्गृहाणि} %5-7-4

\twolineshloka
{ततो ददर्शोच्छ्रितमेघरूपं मनोहरं काञ्चनचारुरूपम्}
{रक्षोऽधिपस्यात्मबलानुरूपं गृहोत्तमं ह्यप्रतिरूपरूपम्} %5-7-5

\twolineshloka
{महीतले स्वर्गमिव प्रकीर्णं श्रिया ज्वलन्तं बहुरत्नकीर्णम्}
{नानातरूणां कुसुमावकीर्णं गिरेरिवाग्रं रजसावकीर्णम्} %5-7-6

\twolineshloka
{नारीप्रवेकैरिव दीप्यमानं तडिद्भिरम्भोधरमर्च्यमानम्}
{हंसप्रवेकैरिव वाह्यमानं श्रिया युतं खे सुकृतं विमानम्} %5-7-7

\twolineshloka
{यथा नगाग्रं बहुधातुचित्रं यथा नभश्च ग्रहचन्द्रचित्रम्}
{ददर्श युक्तीकृतचारुमेघचित्रं विमानं बहुरत्नचित्रम्} %5-7-8

\twolineshloka
{मही कृता पर्वतराजिपूर्णा शैलाः कृता वृक्षवितानपूर्णाः}
{वृक्षाः कृताः पुष्पवितानपूर्णाः पुष्पं कृतं केसरपत्रपूर्णम्} %5-7-9

\twolineshloka
{कृतानि वेश्मानि च पाण्डुराणि तथा सुपुष्पाण्यपि पुष्कराणि}
{पुनश्च पद्मानि सकेसराणि वनानि चित्राणि सरोवराणि} %5-7-10

\twolineshloka
{पुष्पाह्वयं नाम विराजमानं रत्नप्रभाभिश्च विघूर्णमानम्}
{वेश्मोत्तमानामपि चोच्चमानं महाकपिस्तत्र महाविमानम्} %5-7-11

\twolineshloka
{कृताश्च वैदूर्यमया विहङ्गा रूप्यप्रवालैश्च तथा विहङ्गाः}
{चित्राश्च नानावसुभिर्भुजङ्गा जात्यानुरूपास्तुरगाः शुभाङ्गाः} %5-7-12

\twolineshloka
{प्रवालजाम्बूनदपुष्पपक्षाः सलीलमावर्जितजिह्मपक्षाः}
{कामस्य साक्षादिव भान्ति पक्षाः कृता विहङ्गाः सुमुखाः सुपक्षाः} %5-7-13

\twolineshloka
{नियुज्यमानाश्च गजाः सुहस्ताः सकेसराश्चोत्पलपत्रहस्ताः}
{बभूव देवी च कृतासुहस्ता लक्ष्मीस्तथा पद्मिनि पद्महस्ता} %5-7-14

\twolineshloka
{इतीव तद्गृहमभिगम्य शोभनं सविस्मयो नगमिव चारुकन्दरम्}
{पुनश्च तत्परमसुगन्धि सुन्दरं हिमात्यये नगमिव चारुकन्दरम्} %5-7-15

\twolineshloka
{ततः स तां कपिरभिपत्य पूजितां चरन् पुरीं दशमुखबाहुपालिताम्}
{अदृश्य तां जनकसुतां सुपूजितां सुदुःखितां पतिगुणवेगनिर्जिताम्} %5-7-16

\twolineshloka
{ततस्तदा बहुविधभावितात्मनः कृतात्मनो जनकसुतां सुवर्त्मनः}
{अपश्यतोऽभवदतिदुःखितं मनः सचक्षुषः प्रविचरतो महात्मनः} %5-7-17


॥इत्यार्षे श्रीमद्रामायणे वाल्मीकीये आदिकाव्ये सुन्दरकाण्डे पुष्पकदर्शनम् नाम सप्तमः सर्गः ॥५-७॥
