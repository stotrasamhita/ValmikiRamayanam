\sect{पञ्चपञ्चाशः सर्गः — हनूमद्विभ्रमः}

\twolineshloka
{संदीप्यमानां वित्रस्तां त्रस्तरक्षोगणां पुरीम्}
{अवेक्ष्य हनुमाँल्लङ्कां चिन्तयामास वानरः} %5-55-1

\twolineshloka
{तस्याभूत् सुमहांस्त्रासः कुत्सा चात्मन्यजायत}
{लङ्कां प्रदहता कर्म किंस्वित् कृतमिदं मया} %5-55-2

\twolineshloka
{धन्याः खलु महात्मानो ये बुद्ध्या कोपमुत्थितम्}
{निरुन्धन्ति महात्मानो दीप्तमग्निमिवाम्भसा} %5-55-3

\twolineshloka
{क्रुद्धः पापं न कुर्यात् कः क्रुद्धो हन्याद् गुरूनपि}
{क्रुद्धः परुषया वाचा नरः साधूनधिक्षिपेत्} %5-55-4

\twolineshloka
{वाच्यावाच्यं प्रकुपितो न विजानाति कर्हिचित्}
{नाकार्यमस्ति क्रुद्धस्य नावाच्यं विद्यते क्वचित्} %5-55-5

\twolineshloka
{यः समुत्पतितं क्रोधं क्षमयैव निरस्यति}
{यथोरगस्त्वचं जीर्णां स वै पुरुष उच्यते} %5-55-6

\twolineshloka
{धिगस्तु मां सुदुर्बुद्धिं निर्लज्जं पापकृत्तमम्}
{अचिन्तयित्वा तां सीतामग्निदं स्वामिघातकम्} %5-55-7

\twolineshloka
{यदि दग्धा त्वियं सर्वा नूनमार्यापि जानकी}
{दग्धा तेन मया भर्तुर्हतं कार्यमजानता} %5-55-8

\twolineshloka
{यदर्थमयमारम्भस्तत्कार्यमवसादितम्}
{मया हि दहता लङ्कां न सीता परिरक्षिता} %5-55-9

\twolineshloka
{ईषत्कार्यमिदं कार्यं कृतमासीन्न संशयः}
{तस्य क्रोधाभिभूतेन मया मूलक्षयः कृतः} %5-55-10

\twolineshloka
{विनष्टा जानकी व्यक्तं न ह्यदग्धः प्रदृश्यते}
{लङ्कायाः कश्चिदुद्देशः सर्वा भस्मीकृता पुरी} %5-55-11

\twolineshloka
{यदि तद्विहतं कार्यं मया प्रज्ञाविपर्ययात्}
{इहैव प्राणसंन्यासो ममापि ह्यद्य रोचते} %5-55-12

\twolineshloka
{किमग्नौ निपताम्यद्य आहोस्विद् वडवामुखे}
{शरीरमिह सत्त्वानां दद्मि सागरवासिनाम्} %5-55-13

\twolineshloka
{कथं नु जीवता शक्यो मया द्रष्टुं हरीश्वरः}
{तौ वा पुरुषशार्दूलौ कार्यसर्वस्वघातिना} %5-55-14

\twolineshloka
{मया खलु तदेवेदं रोषदोषात् प्रदर्शितम्}
{प्रथितं त्रिषु लोकेषु कपित्वमनवस्थितम्} %5-55-15

\twolineshloka
{धिगस्तु राजसं भावमनीशमनवस्थितम्}
{ईश्वरेणापि यद् रागान्मया सीता न रक्षिता} %5-55-16

\twolineshloka
{विनष्टायां तु सीतायां तावुभौ विनशिष्यतः}
{तयोर्विनाशे सुग्रीवः सबन्धुर्विनशिष्यति} %5-55-17

\twolineshloka
{एतदेव वचः श्रुत्वा भरतो भ्रातृवत्सलः}
{धर्मात्मा सहशत्रुघ्नः कथं शक्ष्यति जीवितुम्} %5-55-18

\twolineshloka
{इक्ष्वाकुवंशे धर्मिष्ठे गते नाशमसंशयम्}
{भविष्यन्ति प्रजाः सर्वाः शोकसंतापपीडिताः} %5-55-19

\twolineshloka
{तदहं भाग्यरहितो लुप्तधर्मार्थसंग्रहः}
{रोषदोषपरीतात्मा व्यक्तं लोकविनाशनः} %5-55-20

\twolineshloka
{इति चिन्तयतस्तस्य निमित्तान्युपपेदिरे}
{पूर्वमप्युपलब्धानि साक्षात् पुनरचिन्तयत्} %5-55-21

\twolineshloka
{अथ वा चारुसर्वाङ्गी रक्षिता स्वेन तेजसा}
{न नशिष्यति कल्याणी नाग्निरग्नौ प्रवर्तते} %5-55-22

\twolineshloka
{नहि धर्मात्मनस्तस्य भार्याममिततेजसः}
{स्वचरित्राभिगुप्तां तां स्प्रष्टुमर्हति पावकः} %5-55-23

\twolineshloka
{नूनं रामप्रभावेण वैदेह्याः सुकृतेन च}
{यन्मां दहनकर्मायं नादहद्धव्यवाहनः} %5-55-24

\twolineshloka
{त्रयाणां भरतादीनां भ्रातॄणां देवता च या}
{रामस्य च मनःकान्ता सा कथं विनशिष्यति} %5-55-25

\twolineshloka
{यद् वा दहनकर्मायं सर्वत्र प्रभुरव्ययः}
{न मे दहति लाङ्गूलं कथमार्यां प्रधक्ष्यति} %5-55-26

\twolineshloka
{पुनश्चाचिन्तयत् तत्र हनूमान् विस्मितस्तदा}
{हिरण्यनाभस्य गिरेर्जलमध्ये प्रदर्शनम्} %5-55-27

\twolineshloka
{तपसा सत्यवाक्येन अनन्यत्वाच्च भर्तरि}
{असौ विनिर्दहेदग्निं न तामग्निः प्रधक्ष्यति} %5-55-28

\twolineshloka
{स तथा चिन्तयंस्तत्र देव्या धर्मपरिग्रहम्}
{शुश्राव हनुमांस्तत्र चारणानां महात्मनाम्} %5-55-29

\twolineshloka
{अहो खलु कृतं कर्म दुर्विगाहं हनूमता}
{अग्निं विसृजता तीक्ष्णं भीमं राक्षससद्मनि} %5-55-30

\twolineshloka
{प्रपलायितरक्षःस्त्रीबालवृद्धसमाकुला}
{जनकोलाहलाध्माता क्रन्दन्तीवाद्रिकन्दरैः} %5-55-31

\twolineshloka
{दग्धेयं नगरी लङ्का साट्टप्राकारतोरणा}
{जानकी न च दग्धेति विस्मयोऽद्भुत एव नः} %5-55-32

\twolineshloka
{इति शुश्राव हनुमान् वाचं ताममृतोपमाम्}
{बभूव चास्य मनसो हर्षस्तत्कालसम्भवः} %5-55-33

\twolineshloka
{स निमित्तैश्च दृष्टार्थैः कारणैश्च महागुणैः}
{ऋषिवाक्यैश्च हनुमानभवत् प्रीतमानसः} %5-55-34

\twolineshloka
{ततः कपिः प्राप्तमनोरथार्थस्तामक्षतां राजसुतां विदित्वा}
{प्रत्यक्षतस्तां पुनरेव दृष्ट्वा प्रतिप्रयाणाय मतिं चकार} %5-55-35


॥इत्यार्षे श्रीमद्रामायणे वाल्मीकीये आदिकाव्ये सुन्दरकाण्डे हनूमद्विभ्रमः नाम पञ्चपञ्चाशः सर्गः ॥५-५५॥
