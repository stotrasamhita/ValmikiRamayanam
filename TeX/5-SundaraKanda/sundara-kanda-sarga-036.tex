\sect{षड्त्रिंशः सर्गः — अङ्गुलीयकप्रदानम्}

\twolineshloka
{भूय एव महातेजा हनूमान् पवनात्मजः}
{अब्रवीत् प्रश्रितं वाक्यं सीताप्रत्ययकारणात्} %5-36-1

\twolineshloka
{वानरोऽहं महाभागे दूतो रामस्य धीमतः}
{रामनामाङ्कितं चेदं पश्य देव्यङ्गुलीयकम्} %5-36-2

\twolineshloka
{प्रत्ययार्थं तवानीतं तेन दत्तं महात्मना}
{समाश्वसिहि भद्रं ते क्षीणदुःखफला ह्यसि} %5-36-3

\twolineshloka
{गृहीत्वा प्रेक्षमाणा सा भर्तुः करविभूषितम्}
{भर्तारमिव सम्प्राप्तं जानकी मुदिताभवत्} %5-36-4

\twolineshloka
{चारु तद् वदनं तस्यास्ताम्रशुक्लायतेक्षणम्}
{बभूव हर्षोदग्रं च राहुमुक्त इवोडुराट्} %5-36-5

\twolineshloka
{ततः सा ह्रीमती बाला भर्तुः सन्देशहर्षिता}
{परितुष्टा प्रियं कृत्वा प्रशशंस महाकपिम्} %5-36-6

\twolineshloka
{विक्रान्तस्त्वं समर्थस्त्वं प्राज्ञस्त्वं वानरोत्तम}
{येनेदं राक्षसपदं त्वयैकेन प्रधर्षितम्} %5-36-7

\twolineshloka
{शतयोजनविस्तीर्णः सागरो मकरालयः}
{विक्रमश्लाघनीयेन क्रमता गोष्पदीकृतः} %5-36-8

\twolineshloka
{नहि त्वां प्राकृतं मन्ये वानरं वानरर्षभ}
{यस्य ते नास्ति सन्त्रासो रावणादपि सम्भ्रमः} %5-36-9

\twolineshloka
{अर्हसे च कपिश्रेष्ठ मया समभिभाषितुम्}
{यद्यसि प्रेषितस्तेन रामेण विदितात्मना} %5-36-10

\twolineshloka
{प्रेषयिष्यति दुर्धर्षो रामो नह्यपरीक्षितम्}
{पराक्रममविज्ञाय मत्सकाशं विशेषतः} %5-36-11

\twolineshloka
{दिष्ट्या च कुशली रामो धर्मात्मा सत्यसङ्गरः}
{लक्ष्मणश्च महातेजाः सुमित्रानन्दवर्धनः} %5-36-12

\twolineshloka
{कुशली यदि काकुत्स्थः किं न सागरमेखलाम्}
{महीं दहति कोपेन युगान्ताग्निरिवोत्थितः} %5-36-13

\twolineshloka
{अथवा शक्तिमन्तौ तौ सुराणामपि निग्रहे}
{ममैव तु न दुःखानामस्ति मन्ये विपर्ययः} %5-36-14

\twolineshloka
{कच्चिन्न व्यथते रामः कच्चिन्न परितप्यते}
{उत्तराणि च कार्याणि कुरुते पुरुषोत्तमः} %5-36-15

\twolineshloka
{कच्चिन्न दीनः सम्भ्रान्तः कार्येषु च न मुह्यति}
{कच्चित् पुरुषकार्याणि कुरुते नृपतेः सुतः} %5-36-16

\twolineshloka
{द्विविधं त्रिविधोपायमुपायमपि सेवते}
{विजिगीषुः सुहृत् कच्चिन्मित्रेषु च परन्तपः} %5-36-17

\twolineshloka
{कच्चिन्मित्राणि लभतेऽमित्रैश्चाप्यभिगम्यते}
{कच्चित् कल्याणमित्रश्च मित्रैश्चापि पुरस्कृतः} %5-36-18

\twolineshloka
{कच्चिदाशास्ति देवानां प्रसादं पार्थिवात्मजः}
{कच्चित् पुरुषकारं च दैवं च प्रतिपद्यते} %5-36-19

\twolineshloka
{कच्चिन्न विगतस्नेहो विवासान्मयि राघवः}
{कच्चिन्मां व्यसनादस्मान्मोक्षयिष्यति राघवः} %5-36-20

\twolineshloka
{सुखानामुचितो नित्यमसुखानामनूचितः}
{दुःखमुत्तरमासाद्य कच्चिद् रामो न सीदति} %5-36-21

\twolineshloka
{कौसल्यायास्तथा कच्चित् सुमित्रायास्तथैव च}
{अभीक्ष्णं श्रूयते कच्चित् कुशलं भरतस्य च} %5-36-22

\twolineshloka
{मन्निमित्तेन मानार्हः कच्चिच्छोकेन राघवः}
{कच्चिन्नान्यमना रामः कच्चिन्मां तारयिष्यति} %5-36-23

\twolineshloka
{कच्चिदक्षौहिणीं भीमां भरतो भ्रातृवत्सलः}
{ध्वजिनीं मन्त्रिभिर्गुप्तां प्रेषयिष्यति मत्कृते} %5-36-24

\twolineshloka
{वानराधिपतिः श्रीमान् सुग्रीवः कच्चिदेष्यति}
{मत्कृते हरिभिर्वीरैर्वृतो दन्तनखायुधैः} %5-36-25

\twolineshloka
{कच्चिच्च लक्ष्मणः शूरः सुमित्रानन्दवर्धनः}
{अस्त्रविच्छरजालेन राक्षसान् विधमिष्यति} %5-36-26

\twolineshloka
{रौद्रेण कच्चिदस्त्रेण रामेण निहतं रणे}
{द्रक्ष्याम्यल्पेन कालेन रावणं ससुहृज्जनम्} %5-36-27

\twolineshloka
{कच्चिन्न तद्धेमसमानवर्णं तस्याननं पद्मसमानगन्धि}
{मया विना शुष्यति शोकदीनं जलक्षये पद्ममिवातपेन} %5-36-28

\twolineshloka
{धर्मापदेशात् त्यजतः स्वराज्यं मां चाप्यरण्यं नयतः पदातेः}
{नासीद् यथा यस्य न भीर्न शोकः कच्चित् स धैर्यं हृदये करोति} %5-36-29

\twolineshloka
{न चास्य माता न पिता न चान्यः स्नेहाद् विशिष्टोऽस्ति मया समो वा}
{तावद्ध्यहं दूत जिजीविषेयं यावत् प्रवृत्तिं शृणुयां प्रियस्य} %5-36-30

\twolineshloka
{इतीव देवी वचनं महार्थं तं वानरेन्द्रं मधुरार्थमुक्त्वा}
{श्रोतुं पुनस्तस्य वचोऽभिरामं रामार्थयुक्तं विरराम रामा} %5-36-31

\twolineshloka
{सीताया वचनं श्रुत्वा मारुतिर्भीमविक्रमः}
{शिरस्यञ्जलिमाधाय वाक्यमुत्तरमब्रवीत्} %5-36-32

\twolineshloka
{न त्वामिहस्थां जानीते रामः कमललोचनः}
{तेन त्वां नानयत्याशु शचीमिव पुरन्दरः} %5-36-33

\twolineshloka
{श्रुत्वैव च वचो मह्यं क्षिप्रमेष्यति राघवः}
{चमूं प्रकर्षन् महतीं हर्यृक्षगणसंयुताम्} %5-36-34

\twolineshloka
{विष्टम्भयित्वा बाणौघैरक्षोभ्यं वरुणालयम्}
{करिष्यति पुरीं लङ्कां काकुत्स्थः शान्तराक्षसाम्} %5-36-35

\twolineshloka
{तत्र यद्यन्तरा मृत्युर्यदि देवा महासुराः}
{स्थास्यन्ति पथि रामस्य स तानपि वधिष्यति} %5-36-36

\twolineshloka
{तवादर्शनजेनार्ये शोकेन परिपूरितः}
{न शर्म लभते रामः सिंहार्दित इव द्विपः} %5-36-37

\twolineshloka
{मन्दरेण च ते देवि शपे मूलफलेन च}
{मलयेन च विन्ध्येन मेरुणा दर्दुरेण च} %5-36-38

\twolineshloka
{यथा सुनयनं वल्गु बिम्बोष्ठं चारुकुण्डलम्}
{मुखं द्रक्ष्यसि रामस्य पूर्णचन्द्रमिवोदितम्} %5-36-39

\twolineshloka
{क्षिप्रं द्रक्ष्यसि वैदेहि रामं प्रस्रवणे गिरौ}
{शतक्रतुमिवासीनं नागपृष्ठस्य मूर्धनि} %5-36-40

\twolineshloka
{न मांसं राघवो भुङ्क्ते न चैव मधु सेवते}
{वन्यं सुविहितं नित्यं भक्तमश्नाति पञ्चमम्} %5-36-41

\twolineshloka
{नैव दंशान् न मशकान् न कीटान् न सरीसृपान्}
{राघवोऽपनयेद् गात्रात् त्वद्गतेनान्तरात्मना} %5-36-42

\twolineshloka
{नित्यं ध्यानपरो रामो नित्यं शोकपरायणः}
{नान्यच्चिन्तयते किञ्चित् स तु कामवशं गतः} %5-36-43

\twolineshloka
{अनिद्रः सततं रामः सुप्तोऽपि च नरोत्तमः}
{सीतेति मधुरां वाणीं व्याहरन् प्रतिबुध्यते} %5-36-44

\twolineshloka
{दृष्ट्वा फलं वा पुष्पं वा यच्चान्यत् स्त्रीमनोहरम्}
{बहुशो हा प्रियेत्येवं श्वसंस्त्वामभिभाषते} %5-36-45

\twolineshloka
{स देवि नित्यं परितप्यमानस्त्वामेव सीतेत्यभिभाषमाणः}
{धृतव्रतो राजसुतो महात्मा तवैव लाभाय कृतप्रयत्नः} %5-36-46

\twolineshloka
{सा रामसङ्कीर्तनवीतशोका रामस्य शोकेन समानशोका}
{शरन्मुखेनाम्बुदशेषचन्द्रा निशेव वैदेहसुता बभूव} %5-36-47


॥इत्यार्षे श्रीमद्रामायणे वाल्मीकीये आदिकाव्ये सुन्दरकाण्डे अङ्गुलीयकप्रदानम् नाम षड्त्रिंशः सर्गः ॥५-३६॥
