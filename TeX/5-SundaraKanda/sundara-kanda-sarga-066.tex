\sect{षट्षष्ठितमः सर्गः — सीताभाषितप्रश्नः}

\twolineshloka
{एवमुक्तो हनुमता रामो दशरथात्मजः}
{तं मणिं हृदये कृत्वा रुरोद सहलक्ष्मणः} %5-66-1

\twolineshloka
{तं तु दृष्ट्वा मणिश्रेष्ठं राघवः शोककर्शितः}
{नेत्राभ्यामश्रुपूर्णाभ्यां सुग्रीवमिदमब्रवीत्} %5-66-2

\twolineshloka
{यथैव धेनुः स्रवति स्नेहाद् वत्सस्य वत्सला}
{तथा ममापि हृदयं मणिश्रेष्ठस्य दर्शनात्} %5-66-3

\twolineshloka
{मणिरत्नमिदं दत्तं वैदेह्याः श्वशुरेण मे}
{वधूकाले यथा बद्धमधिकं मूर्ध्नि शोभते} %5-66-4

\twolineshloka
{अयं हि जलसम्भूतो मणिः प्रवरपूजितः}
{यज्ञे परमतुष्टेन दत्तः शक्रेण धीमता} %5-66-5

\twolineshloka
{इमं दृष्ट्वा मणिश्रेष्ठं तथा तातस्य दर्शनम्}
{अद्यास्म्यवगतः सौम्य वैदेहस्य तथा विभोः} %5-66-6

\twolineshloka
{अयं हि शोभते तस्याः प्रियाया मूर्ध्नि मे मणिः}
{अद्यास्य दर्शनेनाहं प्राप्तां तामिव चिन्तये} %5-66-7

\twolineshloka
{किमाह सीता वैदेही ब्रूहि सौम्य पुनः पुनः}
{परासुमिव तोयेन सिञ्चन्ती वाक्यवारिणा} %5-66-8

\twolineshloka
{इतस्तु किं दुःखतरं यदिमं वारिसम्भवम्}
{मणिं पश्यामि सौमित्रे वैदेहीमागतां विना} %5-66-9

\twolineshloka
{चिरं जीवति वैदेही यदि मासं धरिष्यति}
{क्षणं वीर न जीवेयं विना तामसितेक्षणाम्} %5-66-10

\twolineshloka
{नय मामपि तं देशं यत्र दृष्टा मम प्रिया}
{न तिष्ठेयं क्षणमपि प्रवृत्तिमुपलभ्य च} %5-66-11

\twolineshloka
{कथं सा मम सुश्रोणी भीरुभीरुः सती तदा}
{भयावहानां घोराणां मध्ये तिष्ठति रक्षसाम्} %5-66-12

\twolineshloka
{शारदस्तिमिरोन्मुक्तो नूनं चन्द्र इवाम्बुदैः}
{आवृतो वदनं तस्या न विराजति साम्प्रतम्} %5-66-13

\twolineshloka
{किमाह सीता हनुमंस्तत्त्वतः कथयस्व मे}
{एतेन खलु जीविष्ये भेषजेनातुरो यथा} %5-66-14

\threelineshloka
{मधुरा मधुरालापा किमाह मम भामिनी}
{मद्विहीना वरारोहा हनुमन् कथयस्व मे}
{दुःखाद् दुःखतरं प्राप्य कथं जीवति जानकी} %5-66-15


॥इत्यार्षे श्रीमद्रामायणे वाल्मीकीये आदिकाव्ये सुन्दरकाण्डे सीताभाषितप्रश्नः नाम षट्षष्ठितमः सर्गः ॥५-६६॥
