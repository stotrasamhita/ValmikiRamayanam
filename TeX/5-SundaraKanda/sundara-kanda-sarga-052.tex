\sect{द्विपञ्चाशः सर्गः — दूतवधनिवारणम्}

\twolineshloka
{स तस्य वचनं श्रुत्वा वानरस्य महात्मनः}
{आज्ञापयद् वधं तस्य रावणः क्रोधमूर्च्छितः} %5-52-1

\twolineshloka
{वधे तस्य समाज्ञप्ते रावणेन दुरात्मना}
{निवेदितवतो दौत्यं नानुमेने विभीषणः} %5-52-2

\twolineshloka
{तं रक्षोऽधिपतिं क्रुद्धं तच्च कार्यमुपस्थितम्}
{विदित्वा चिन्तयामास कार्यं कार्यविधौ स्थितः} %5-52-3

\twolineshloka
{निश्चितार्थस्ततः साम्ना पूज्यं शत्रुजिदग्रजम्}
{उवाच हितमत्यर्थं वाक्यं वाक्यविशारदः} %5-52-4

\twolineshloka
{क्षमस्व रोषं त्यज राक्षसेन्द्र प्रसीद मे वाक्यमिदं शृणुष्व}
{वधं न कुर्वन्ति परावरज्ञा दूतस्य सन्तो वसुधाधिपेन्द्राः} %5-52-5

\twolineshloka
{राजन् धर्मविरुद्धं च लोकवृत्तेश्च गर्हितम्}
{तव चासदृशं वीर कपेरस्य प्रमापणम्} %5-52-6

\twolineshloka
{धर्मज्ञश्च कृतज्ञश्च राजधर्मविशारदः}
{परावरज्ञो भूतानां त्वमेव परमार्थवित्} %5-52-7

\twolineshloka
{गृह्यन्ते यदि रोषेण त्वादृशोऽपि विचक्षणाः}
{ततः शास्त्रविपश्चित्त्वं श्रम एव हि केवलम्} %5-52-8

\twolineshloka
{तस्मात् प्रसीद शत्रुघ्न राक्षसेन्द्र दुरासद}
{युक्तायुक्तं विनिश्चित्य दूतदण्डो विधीयताम्} %5-52-9

\twolineshloka
{विभीषणवचः श्रुत्वा रावणो राक्षसेश्वरः}
{कोपेन महताऽऽविष्टो वाक्यमुत्तरमब्रवीत्} %5-52-10

\twolineshloka
{न पापानां वधे पापं विद्यते शत्रुसूदन}
{तस्मादिमं वधिष्यामि वानरं पापकारिणम्} %5-52-11

\twolineshloka
{अधर्ममूलं बहुदोषयुक्तमनार्यजुष्टं वचनं निशम्य}
{उवाच वाक्यं परमार्थतत्त्वं विभीषणो बुद्धिमतां वरिष्ठः} %5-52-12

\twolineshloka
{प्रसीद लङ्केश्वर राक्षसेन्द्र धर्मार्थतत्त्वं वचनं शृणुष्व}
{दूता न वध्याः समयेषु राजन् सर्वेषु सर्वत्र वदन्ति सन्तः} %5-52-13

\twolineshloka
{असंशयं शत्रुरयं प्रवृद्धः कृतं ह्यनेनाप्रियमप्रमेयम्}
{न दूतवध्यां प्रवदन्ति सन्तो दूतस्य दृष्टा बहवो हि दण्डाः} %5-52-14

\twolineshloka
{वैरूप्यमङ्गेषु कशाभिघातो मौण्ड्यं तथा लक्षणसन्निपातः}
{एतान् हि दूते प्रवदन्ति दण्डान् वधस्तु दूतस्य न नः श्रुतोऽस्ति} %5-52-15

\twolineshloka
{कथं च धर्मार्थविनीतबुद्धिः परावरप्रत्ययनिश्चितार्थः}
{भवद्विधः कोपवशे हि तिष्ठेत् कोपं न गच्छन्ति हि सत्त्ववन्तः} %5-52-16

\twolineshloka
{न धर्मवादे न च लोकवृत्ते न शास्त्रबुद्धिग्रहणेषु वापि}
{विद्येत कश्चित्तव वीर तुल्यस्त्वं ह्युत्तमः सर्वसुरासुराणाम्} %5-52-17

\twolineshloka
{पराक्रमोत्साहमनस्विनां च सुरासुराणामपि दुर्जयेन}
{त्वयाप्रमेयेण सुरेन्द्रसङ्घा जिताश्च युद्धेष्वसकृन्नरेन्द्राः} %5-52-18

\twolineshloka
{इत्थंविधस्यामरदैत्यशत्रोः शूरस्य वीरस्य तवाजितस्य}
{कुर्वन्ति वीरा मनसाप्यलीकं प्राणैर्विमुक्ता न तु भोः पुरा ते} %5-52-19

\twolineshloka
{न चाप्यस्य कपेर्घाते कञ्चित् पश्याम्यहं गुणम्}
{तेष्वयं पात्यतां दण्डो यैरयं प्रेषितः कपिः} %5-52-20

\twolineshloka
{साधुर्वा यदि वासाधुः परैरेष समर्पितः}
{ब्रुवन् परार्थं परवान् न दूतो वधमर्हति} %5-52-21

\twolineshloka
{अपि चास्मिन् हते नान्यं राजन् पश्यामि खेचरम्}
{इह यः पुनरागच्छेत् परं पारं महोदधेः} %5-52-22

\twolineshloka
{तस्मान्नास्य वधे यत्नः कार्यः परपुरञ्जय}
{भवान् सेन्द्रेषु देवेषु यत्नमास्थातुमर्हति} %5-52-23

\twolineshloka
{अस्मिन् विनष्टे नहि भूतमन्यं पश्यामि यस्तौ नरराजपुत्रौ}
{युद्धाय युद्धप्रिय दुर्विनीतावुद्योजयेद् वै भवता विरुद्धौ} %5-52-24

\twolineshloka
{पराक्रमोत्साहमनस्विनां च सुरासुराणामपि दुर्जयेन}
{त्वया मनोनन्दन नैर्ऋतानां युद्धाय निर्नाशयितुं न युक्तम्} %5-52-25

\twolineshloka
{हिताश्च शूराश्च समाहिताश्च कुलेषु जाताश्च महागुणेषु}
{मनस्विनः शस्त्रभृतां वरिष्ठाः कोपप्रशस्ताः सुभृताश्च योधाः} %5-52-26

\twolineshloka
{तदेकदेशेन बलस्य तावत् केचित् तवादेशकृतोऽद्य यान्तु}
{तौ राजपुत्रावुपगृह्य मूढौ परेषु ते भावयितुं प्रभावम्} %5-52-27

\twolineshloka
{निशाचराणामधिपोऽनुजस्य विभीषणस्योत्तमवाक्यमिष्टम्}
{जग्राह बुद्ध्या सुरलोकशत्रुर्महाबलो राक्षसराजमुख्यः} %5-52-28


॥इत्यार्षे श्रीमद्रामायणे वाल्मीकीये आदिकाव्ये सुन्दरकाण्डे दूतवधनिवारणम् नाम द्विपञ्चाशः सर्गः ॥५-५२॥
