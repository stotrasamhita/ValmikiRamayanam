\sect{त्रिंशः सर्गः — हनुमत्कृत्याकृत्यविचिन्तनम्}

\twolineshloka
{हनुमानपि विक्रान्तः सर्वं शुश्राव तत्त्वतः}
{सीतायास्त्रिजटायाश्च राक्षसीनां च तर्जितम्} %5-30-1

\twolineshloka
{अवेक्षमाणस्तां देवीं देवतामिव नन्दने}
{ततो बहुविधां चिन्तां चिन्तयामास वानरः} %5-30-2

\twolineshloka
{यां कपीनां सहस्राणि सुबहून्ययुतानि च}
{दिक्षु सर्वासु मार्गन्ते सेयमासादिता मया} %5-30-3

\twolineshloka
{चारेण तु सुयुक्तेन शत्रोः शक्तिमवेक्षता}
{गूढेन चरता तावदवेक्षितमिदं मया} %5-30-4

\twolineshloka
{राक्षसानां विशेषश्च पुरी चेयं निरीक्षिता}
{राक्षसाधिपतेरस्य प्रभावो रावणस्य च} %5-30-5

\twolineshloka
{यथा तस्याप्रमेयस्य सर्वसत्त्वदयावतः}
{समाश्वासयितुं भार्यां पतिदर्शनकाङ्क्षिणीम्} %5-30-6

\twolineshloka
{अहमाश्वासयाम्येनां पूर्णचन्द्रनिभाननाम्}
{अदृष्टदुःखां दुःखस्य न ह्यन्तमधिगच्छतीम्} %5-30-7

\twolineshloka
{यदि ह्यहं सतीमेनां शोकोपहतचेतनाम्}
{अनाश्वास्य गमिष्यामि दोषवद् गमनं भवेत्} %5-30-8

\twolineshloka
{गते हि मयि तत्रेयं राजपुत्री यशस्विनी}
{परित्राणमपश्यन्ती जानकी जीवितं त्यजेत्} %5-30-9

\twolineshloka
{यथा च स महाबाहुः पूर्णचन्द्रनिभाननः}
{समाश्वासयितुं न्याय्यः सीतादर्शनलालसः} %5-30-10

\twolineshloka
{निशाचरीणां प्रत्यक्षमक्षमं चाभिभाषितम्}
{कथं नु खलु कर्तव्यमिदं कृच्छ्रगतो ह्यहम्} %5-30-11

\twolineshloka
{अनेन रात्रिशेषेण यदि नाश्वास्यते मया}
{सर्वथा नास्ति सन्देहः परित्यक्ष्यति जीवितम्} %5-30-12

\twolineshloka
{रामस्तु यदि पृच्छेन्मां किं मां सीताब्रवीद्वचः}
{किमहं तं प्रतिब्रूयामसम्भाष्य सुमध्यमाम्} %5-30-13

\twolineshloka
{सीतासन्देशरहितं मामितस्त्वरया गतम्}
{निर्दहेदपि काकुत्स्थः क्रोधतीव्रेण चक्षुषा} %5-30-14

\twolineshloka
{यदि वोद्योजयिष्यामि भर्तारं रामकारणात्}
{व्यर्थमागमनं तस्य ससैन्यस्य भविष्यति} %5-30-15

\twolineshloka
{अन्तरं त्वहमासाद्य राक्षसीनामवस्थितः}
{शनैराश्वासयाम्यद्य सन्तापबहुलामिमाम्} %5-30-16

\twolineshloka
{अहं ह्यतितनुश्चैव वानरश्च विशेषतः}
{वाचं चोदाहरिष्यामि मानुषीमिह संस्कृताम्} %5-30-17

\twolineshloka
{यदि वाचं प्रदास्यामि द्विजातिरिव संस्कृताम्}
{रावणं मन्यमाना मां सीता भीता भविष्यति} %5-30-18

\twolineshloka
{अवश्यमेव वक्तव्यं मानुषं वाक्यमर्थवत्}
{मया सान्त्वयितुं शक्या नान्यथेयमनिन्दिता} %5-30-19

\twolineshloka
{सेयमालोक्य मे रूपं जानकी भाषितं तथा}
{रक्षोभिस्त्रासिता पूर्वं भूयस्त्रासमुपैष्यति} %5-30-20

\twolineshloka
{ततो जातपरित्रासा शब्दं कुर्यान्मनस्विनी}
{जानाना मां विशालाक्षी रावणं कामरूपिणम्} %5-30-21

\twolineshloka
{सीतया च कृते शब्दे सहसा राक्षसीगणः}
{नानाप्रहरणो घोरः समेयादन्तकोपमः} %5-30-22

\twolineshloka
{ततो मां सम्परिक्षिप्य सर्वतो विकृताननाः}
{वधे च ग्रहणे चैव कुर्युर्यत्नं महाबलाः} %5-30-23

\twolineshloka
{तं मां शाखाः प्रशाखाश्च स्कन्धांश्चोत्तमशाखिनाम्}
{दृष्ट्वा च परिधावन्तं भवेयुः परिशङ्किताः} %5-30-24

\twolineshloka
{मम रूपं च सम्प्रेक्ष्य वने विचरतो महत्}
{राक्षस्यो भयवित्रस्ता भवेयुर्विकृतस्वराः} %5-30-25

\twolineshloka
{ततः कुर्युः समाह्वानं राक्षस्यो रक्षसामपि}
{राक्षसेन्द्रनियुक्तानां राक्षसेन्द्रनिवेशने} %5-30-26

\twolineshloka
{ते शूलशरनिस्त्रिंशविविधायुधपाणयः}
{आपतेयुर्विमर्देऽस्मिन् वेगेनोद्वेगकारणात्} %5-30-27

\twolineshloka
{संरुद्धस्तैस्तु परितो विधमे राक्षसं बलम्}
{शक्नुयां न तु सम्प्राप्तुं परं पारं महोदधेः} %5-30-28

\twolineshloka
{मां वा गृह्णीयुरावृत्य बहवः शीघ्रकारिणः}
{स्यादियं चागृहीतार्था मम च ग्रहणं भवेत्} %5-30-29

\twolineshloka
{हिंसाभिरुचयो हिंस्युरिमां वा जनकात्मजाम्}
{विपन्नं स्यात् ततः कार्यं रामसुग्रीवयोरिदम्} %5-30-30

\twolineshloka
{उद्देशे नष्टमार्गेऽस्मिन् राक्षसैः परिवारिते}
{सागरेण परिक्षिप्ते गुप्ते वसति जानकी} %5-30-31

\twolineshloka
{विशस्ते वा गृहीते वा रक्षोभिर्मयि संयुगे}
{नान्यं पश्यामि रामस्य सहायं कार्यसाधने} %5-30-32

\twolineshloka
{विमृशंश्च न पश्यामि यो हते मयि वानरः}
{शतयोजनविस्तीर्णं लङ्घयेत महोदधिम्} %5-30-33

\twolineshloka
{कामं हन्तुं समर्थोऽस्मि सहस्राण्यपि रक्षसाम्}
{न तु शक्ष्याम्यहं प्राप्तुं परं पारं महोदधेः} %5-30-34

\twolineshloka
{कामं हन्तुं समर्थोऽस्मि सहस्राण्यपि रक्षसाम्}
{न तु शक्ष्याम्यहं प्राप्तुं परं पारं महोदधेः} %5-30-35

\twolineshloka
{एष दोषो महान् हि स्यान्मम सीताभिभाषणे}
{प्राणत्यागश्च वैदेह्या भवेदनभिभाषणे} %5-30-36

\twolineshloka
{भूताश्चार्था विरुध्यन्त देशकालविरोधिताः}
{विक्लवं दूतमासाद्य तमः सूर्योदये यथा} %5-30-37

\twolineshloka
{अर्थानर्थान्तरे बुद्धिर्निश्चितापि न शोभते}
{घातयन्ति हि कार्याणि दूताः पण्डितमानिनः} %5-30-38

\twolineshloka
{न विनश्येत् कथं कार्यं वैक्लव्यं न कथं मम}
{लङ्घनं च समुद्रस्य कथं नु न वृथा भवेत्} %5-30-39

\twolineshloka
{कथं नु खलु वाक्यं मे शृणुयान्नोद्विजेत च}
{इति सञ्चिन्त्य हनुमांश्चकार मतिमान् मतिम्} %5-30-40

\twolineshloka
{राममक्लिष्टकर्माणं सुबन्धुमनुकीर्तयन्}
{नैनामुद्वेजयिष्यामि तद्बन्धुगतचेतनाम्} %5-30-41

\twolineshloka
{इक्ष्वाकूणां वरिष्ठस्य रामस्य विदितात्मनः}
{शुभानि धर्मयुक्तानि वचनानि समर्पयन्} %5-30-42

\twolineshloka
{श्रावयिष्यामि सर्वाणि मधुरां प्रब्रुवन् गिरम्}
{श्रद्धास्यति यथा सीता तथा सर्वं समादधे} %5-30-43

\twolineshloka
{इति स बहुविधं महाप्रभावो जगतिपतेः प्रमदामवेक्षमाणः}
{मधुरमवितथं जगाद वाक्यं द्रुमविटपान्तरमास्थितो हनूमान्} %5-30-44


॥इत्यार्षे श्रीमद्रामायणे वाल्मीकीये आदिकाव्ये सुन्दरकाण्डे हनुमत्कृत्याकृत्यविचिन्तनम् नाम त्रिंशः सर्गः ॥५-३०॥
