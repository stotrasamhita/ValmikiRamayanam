\sect{एकोनषष्ठितमः सर्गः — अनन्तरकार्यप्ररोचनम्}

\twolineshloka
{एतदाख्याय तत् सर्वं हनूमान् मारुतात्मजः}
{भूयः समुपचक्राम वचनं वक्तुमुत्तरम्} %5-59-1

\twolineshloka
{सफलो राघवोद्योगः सुग्रीवस्य च सम्भ्रमः}
{शीलमासाद्य सीताया मम च प्रीणितं मनः} %5-59-2

\twolineshloka
{आर्यायाः सदृशं शीलं सीतायाः प्लवगर्षभाः}
{तपसा धारयेल्लोकान् क्रुद्धा वा निर्दहेदपि} %5-59-3

\twolineshloka
{सर्वथातिप्रकृष्टोऽसौ रावणो राक्षसेश्वरः}
{यस्य तां स्पृशतो गात्रं तपसा न विनाशितम्} %5-59-4

\twolineshloka
{न तदग्निशिखा कुर्यात् संस्पृष्टा पाणिना सती}
{जनकस्य सुता कुर्याद् यत् क्रोधकलुषीकृता} %5-59-5

\threelineshloka
{जाम्बवत्प्रमुखान् सर्वाननुज्ञाप्य महाकपीन्}
{अस्मिन् नेवंगते कार्ये भवतां च निवेदिते}
{न्याय्यं स्म सह वैदेह्या द्रष्टुं तौ पार्थिवात्मजौ} %5-59-6

\twolineshloka
{अहमेकोऽपि पर्याप्तः सराक्षसगणां पुरीम्}
{तां लङ्कां तरसा हन्तुं रावणं च महाबलम्} %5-59-7

\twolineshloka
{किं पुनः सहितो वीरैर्बलवद्भिः कृतात्मभिः}
{कृतास्त्रैः प्लवगैः शक्तैर्भवद्भिर्विजयैषिभिः} %5-59-8

\twolineshloka
{अहं तु रावणं युद्धे ससैन्यं सपुरःसरम्}
{सहपुत्रं वधिष्यामि सहोदरयुतं युधि} %5-59-9

\threelineshloka
{ब्राह्ममस्त्रं च रौद्रं च वायव्यं वारुणं तथा}
{यदि शक्रजितोऽस्त्राणि दुर्निरीक्ष्याणि संयुगे}
{तान्यहं निहनिष्यामि विधमिष्यामि राक्षसान्} %5-59-10

\twolineshloka
{भवतामभ्यनुज्ञातो विक्रमो मे रुणद्धि तम्}
{मयातुला विसृष्टा हि शैलवृष्टिर्निरन्तरा} %5-59-11

\twolineshloka
{देवानपि रणे हन्यात् किं पुनस्तान् निशाचरान्}
{भवतामननुज्ञातो विक्रमो मे रुणद्धि माम्} %5-59-12

\twolineshloka
{सागरोऽप्यतियाद् वेलां मन्दरः प्रचलेदपि}
{न जाम्बवन्तं समरे कम्पयेदरिवाहिनी} %5-59-13

\twolineshloka
{सर्वराक्षससङ्घानां राक्षसा ये च पूर्वजाः}
{अलमेकोऽपि नाशाय वीरो वालिसुतः कपिः} %5-59-14

\twolineshloka
{प्लवगस्योरुवेगेन नीलस्य च महात्मनः}
{मन्दरोऽप्यवशीर्येत किं पुनर्युधि राक्षसाः} %5-59-15

\twolineshloka
{सदेवासुरयक्षेषु गन्धर्वोरगपक्षिषु}
{मैन्दस्य प्रतियोद्धारं शंसत द्विविदस्य वा} %5-59-16

\twolineshloka
{अश्विपुत्रौ महावेगावेतौ प्लवगसत्तमौ}
{एतयोः प्रतियोद्धारं न पश्यामि रणाजिरे} %5-59-17

\twolineshloka
{मयैव निहता लङ्का दग्धा भस्मीकृता पुरी}
{राजमार्गेषु सर्वेषु नाम विश्रावितं मया} %5-59-18

\twolineshloka
{जयत्यतिबलो रामो लक्ष्मणश्च महाबलः}
{राजा जयति सुग्रीवो राघवेणाभिपालितः} %5-59-19

\twolineshloka
{अहं कोसलराजस्य दासः पवनसम्भवः}
{हनूमानिति सर्वत्र नाम विश्रावितं मया} %5-59-20

\twolineshloka
{अशोकवनिकामध्ये रावणस्य दुरात्मनः}
{अधस्ताच्छिंशपामूले साध्वी करुणमास्थिता} %5-59-21

\twolineshloka
{राक्षसीभिः परिवृता शोकसंतापकर्शिता}
{मेघरेखापरिवृता चन्द्ररेखेव निष्प्रभा} %5-59-22

\twolineshloka
{अचिन्तयन्ती वैदेही रावणं बलदर्पितम्}
{पतिव्रता च सुश्रोणी अवष्टब्धा च जानकी} %5-59-23

\twolineshloka
{अनुरक्ता हि वैदेही रामे सर्वात्मना शुभा}
{अनन्यचित्ता रामेण पौलोमीव पुरन्दरे} %5-59-24

\twolineshloka
{तदेकवासःसंवीता रजोध्वस्ता तथैव च}
{सा मया राक्षसीमध्ये तर्ज्यमाना मुहुर्मुहुः} %5-59-25

\twolineshloka
{राक्षसीभिर्विरूपाभिर्दृष्टा हि प्रमदावने}
{एकवेणीधरा दीना भर्तृचिन्तापरायणा} %5-59-26

\twolineshloka
{अधःशय्या विवर्णाङ्गी पद्मिनीव हिमोदये}
{रावणाद् विनिवृत्तार्था मर्तव्यकृतनिश्चया} %5-59-27

\twolineshloka
{कथंचिन्मृगशावाक्षी विश्वासमुपपादिता}
{ततः सम्भाषिता चैव सर्वमर्थं प्रकाशिता} %5-59-28

\twolineshloka
{रामसुग्रीवसख्यं च श्रुत्वा प्रीतिमुपागता}
{नियतः समुदाचारो भक्तिर्भर्तरि चोत्तमा} %5-59-29

\twolineshloka
{यन्न हन्ति दशग्रीवं स महात्मा दशाननः}
{निमित्तमात्रं रामस्तु वधे तस्य भविष्यति} %5-59-30

\twolineshloka
{सा प्रकृत्यैव तन्वङ्गी तद्वियोगाच्च कर्शिता}
{प्रतिपत्पाठशीलस्य विद्येव तनुतां गता} %5-59-31

\twolineshloka
{एवमास्ते महाभागा सीता शोकपरायणा}
{यदत्र प्रतिकर्तव्यं तत् सर्वमुपकल्प्यताम्} %5-59-32


॥इत्यार्षे श्रीमद्रामायणे वाल्मीकीये आदिकाव्ये सुन्दरकाण्डे अनन्तरकार्यप्ररोचनम् नाम एकोनषष्ठितमः सर्गः ॥५-५९॥
