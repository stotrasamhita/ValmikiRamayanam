\sect{एकषष्ठितमः सर्गः — मधुवनप्रवेशः}

\twolineshloka
{ततो जाम्बवतो वाक्यमगृह्णन्त वनौकसः}
{अङ्गदप्रमुखा वीरा हनूमांश्च महाकपिः} %5-61-1

\twolineshloka
{प्रीतिमन्तस्ततः सर्वे वायुपुत्रपुरःसराः}
{महेन्द्राग्रात् समुत्पत्य पुप्लुवुः प्लवगर्षभाः} %5-61-2

\twolineshloka
{मेरुमन्दरसंकाशा मत्ता इव महागजाः}
{छादयन्त इवाकाशं महाकाया महाबलाः} %5-61-3

\twolineshloka
{सभाज्यमानं भूतैस्तमात्मवन्तं महाबलम्}
{हनूमन्तं महावेगं वहन्त इव दृष्टिभिः} %5-61-4

\twolineshloka
{राघवे चार्थनिर्वृत्तिं कर्तुं च परमं यशः}
{समाधाय समृद्धार्थाः कर्मसिद्धिभिरुन्नताः} %5-61-5

\twolineshloka
{प्रियाख्यानोन्मुखाः सर्वे सर्वे युद्धाभिनन्दिनः}
{सर्वे रामप्रतीकारे निश्चितार्था मनस्विनः} %5-61-6

\twolineshloka
{प्लवमानाः खमाप्लुत्य ततस्ते काननौकसः}
{नन्दनोपममासेदुर्वनं द्रुमशतायुतम्} %5-61-7

\twolineshloka
{यत् तन्मधुवनं नाम सुग्रीवस्याभिरक्षितम्}
{अधृष्यं सर्वभूतानां सर्वभूतमनोहरम्} %5-61-8

\twolineshloka
{यद् रक्षति महावीरः सदा दधिमुखः कपिः}
{मातुलः कपिमुख्यस्य सुग्रीवस्य महात्मनः} %5-61-9

\twolineshloka
{ते तद् वनमुपागम्य बभूवुः परमोत्कटाः}
{वानरा वानरेन्द्रस्य मनःकान्तं महावनम्} %5-61-10

\twolineshloka
{ततस्ते वानरा हृष्टा दृष्ट्वा मधुवनं महत्}
{कुमारमभ्ययाचन्त मधूनि मधुपिङ्गलाः} %5-61-11

\twolineshloka
{ततः कुमारस्तान् वृद्धाञ्जाम्बवत्प्रमुखान् कपीन्}
{अनुमान्य ददौ तेषां निसर्गं मधुभक्षणे} %5-61-12

\twolineshloka
{ते निसृष्टाः कुमारेण धीमता वालिसूनुना}
{हरयः समपद्यन्त द्रुमान् मधुकराकुलान्} %5-61-13

\twolineshloka
{भक्षयन्तः सुगन्धीनि मूलानि च फलानि च}
{जग्मुः प्रहर्षं ते सर्वे बभूवुश्च मदोत्कटाः} %5-61-14

\twolineshloka
{ततश्चानुमताः सर्वे सुसंहृष्टा वनौकसः}
{मुदिताश्च ततस्ते च प्रनृत्यन्ति ततस्ततः} %5-61-15

\twolineshloka
{गायन्ति केचित् प्रहसन्ति केचिन्नृत्यन्ति केचित् प्रणमन्ति केचित्}
{पतन्ति केचित् प्रचरन्ति केचित् प्लवन्ति केचित् प्रलपन्ति केचित्} %5-61-16

\twolineshloka
{परस्परं केचिदुपाश्रयन्ति परस्परं केचिदतिब्रुवन्ति}
{द्रुमाद् द्रुमं केचिदभिद्रवन्ति क्षितौ नगाग्रान्निपतन्ति केचित्} %5-61-17

\twolineshloka
{महीतलात् केचिदुदीर्णवेगा महाद्रुमाग्राण्यभिसम्पतन्ति}
{गायन्तमन्यः प्रहसन्नुपैति हसन्तमन्यः प्ररुदन्नुपैति} %5-61-18

\twolineshloka
{तुदन्तमन्यः प्रणदन्नुपैति समाकुलं तत् कपिसैन्यमासीत्}
{न चात्र कश्चिन्न बभूव मत्तो न चात्र कश्चिन्न बभूव दृप्तः} %5-61-19

\twolineshloka
{ततो वनं तत् परिभक्ष्यमाणं द्रुमांश्च विध्वंसितपत्रपुष्पान्}
{समीक्ष्य कोपाद् दधिवक्त्रनामा निवारयामास कपिः कपींस्तान्} %5-61-20

\twolineshloka
{स तैः प्रवृद्धैः परिभर्त्स्यमानो वनस्य गोप्ता हरिवृद्धवीरः}
{चकार भूयो मतिमुग्रतेजा वनस्य रक्षां प्रति वानरेभ्यः} %5-61-21

\twolineshloka
{उवाच कांश्चित् परुषाण्यभीतमसक्तमन्यांश्च तलैर्जघान}
{समेत्य कैश्चित् कलहं चकार तथैव साम्नोपजगाम कांश्चित्} %5-61-22

\twolineshloka
{स तैर्मदादप्रतिवार्यवेगैर्बलाच्च तेन प्रतिवार्यमाणैः}
{प्रधर्षणे त्यक्तभयैः समेत्य प्रकृष्यते चाप्यनवेक्ष्य दोषम्} %5-61-23

\twolineshloka
{नखैस्तुदन्तो दशनैर्दशन्तस्तलैश्च पादैश्च समापयन्तः}
{मदात् कपिं ते कपयः समन्तान्महावनं निर्विषयं च चक्रुः} %5-61-24


॥इत्यार्षे श्रीमद्रामायणे वाल्मीकीये आदिकाव्ये सुन्दरकाण्डे मधुवनप्रवेशः नाम एकषष्ठितमः सर्गः ॥५-६१॥
