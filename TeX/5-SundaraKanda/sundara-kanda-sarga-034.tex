\sect{चतुस्त्रिंशः सर्गः — रावणशङ्कानिवारणम्}

\twolineshloka
{तस्यास्तद् वचनं श्रुत्वा हनूमान् हरिपुङ्गवः}
{दुःखाद् दुःखाभिभूतायाः सान्त्वमुत्तरमब्रवीत्} %5-34-1

\twolineshloka
{अहं रामस्य सन्देशाद् देवि दूतस्तवागतः}
{वैदेहि कुशली रामः स त्वां कौशलमब्रवीत्} %5-34-2

\twolineshloka
{यो ब्राह्ममस्त्रं वेदांश्च वेद वेदविदां वरः}
{स त्वां दाशरथी रामो देवि कौशलमब्रवीत्} %5-34-3

\twolineshloka
{लक्ष्मणश्च महातेजा भर्तुस्तेऽनुचरः प्रियः}
{कृतवाञ्छोकसन्तप्तः शिरसा तेऽभिवादनम्} %5-34-4

\twolineshloka
{सा तयोः कुशलं देवी निशम्य नरसिंहयोः}
{प्रतिसंहृष्टसर्वाङ्गी हनूमन्तमथाब्रवीत्} %5-34-5

\twolineshloka
{कल्याणी बत गाथेयं लौकिकी प्रतिभाति मा}
{एति जीवन्तमानन्दो नरं वर्षशतादपि} %5-34-6

\twolineshloka
{तयोः समागमे तस्मिन् प्रीतिरुत्पादिताद्भुता}
{परस्परेण चालापं विश्वस्तौ तौ प्रचक्रतुः} %5-34-7

\twolineshloka
{तस्यास्तद् वचनं श्रुत्वा हनूमान् मारुतात्मजः}
{सीतायाः शोकतप्तायाः समीपमुपचक्रमे} %5-34-8

\twolineshloka
{यथा यथा समीपं स हनूमानुपसर्पति}
{तथा तथा रावणं सा तं सीता परिशङ्कते} %5-34-9

\twolineshloka
{अहो धिग् धिक्कृतमिदं कथितं हि यदस्य मे}
{रूपान्तरमुपागम्य स एवायं हि रावणः} %5-34-10

\twolineshloka
{तामशोकस्य शाखां तु विमुक्त्वा शोककर्शिता}
{तस्यामेवानवद्याङ्गी धरण्यां समुपाविशत्} %5-34-11

\twolineshloka
{अवन्दत महाबाहुस्ततस्तां जनकात्मजाम्}
{सा चैनं भयसन्त्रस्ता भूयो नैनमुदैक्षत} %5-34-12

\twolineshloka
{तं दृष्ट्वा वन्दमानं च सीता शशिनिभानना}
{अब्रवीद् दीर्घमुच्छ्वस्य वानरं मधुरस्वरा} %5-34-13

\twolineshloka
{मायां प्रविष्टो मायावी यदि त्वं रावणः स्वयम्}
{उत्पादयसि मे भूयः सन्तापं तन्न शोभनम्} %5-34-14

\twolineshloka
{स्वं परित्यज्य रूपं यः परिव्राजकरूपवान्}
{जनस्थाने मया दृष्टस्त्वं स एव हि रावणः} %5-34-15

\twolineshloka
{उपवासकृशां दीनां कामरूप निशाचर}
{सन्तापयसि मां भूयः सन्तापं तन्न शोभनम्} %5-34-16

\twolineshloka
{अथवा नैतदेवं हि यन्मया परिशङ्कितम्}
{मनसो हि मम प्रीतिरुत्पन्ना तव दर्शनात्} %5-34-17

\twolineshloka
{यदि रामस्य दूतस्त्वमागतो भद्रमस्तु ते}
{पृच्छामि त्वां हरिश्रेष्ठ प्रिया रामकथा हि मे} %5-34-18

\twolineshloka
{गुणान् रामस्य कथय प्रियस्य मम वानर}
{चित्तं हरसि मे सौम्य नदीकूलं यथा रयः} %5-34-19

\twolineshloka
{अहो स्वप्नस्य सुखता याहमेव चिराहृता}
{प्रेषितं नाम पश्यामि राघवेण वनौकसम्} %5-34-20

\twolineshloka
{स्वप्नेऽपि यद्यहं वीरं राघवं सहलक्ष्मणम्}
{पश्येयं नावसीदेयं स्वप्नोऽपि मम मत्सरी} %5-34-21

\twolineshloka
{नाहं स्वप्नमिमं मन्ये स्वप्ने दृष्ट्वा हि वानरम्}
{न शक्योऽभ्युदयः प्राप्तुं प्राप्तश्चाभ्युदयो मम} %5-34-22

\twolineshloka
{किं नु स्याच्चित्तमोहोऽयं भवेद् वातगतिस्त्वियम्}
{उन्मादजो विकारो वा स्यादयं मृगतृष्णिका} %5-34-23

\twolineshloka
{अथवा नायमुन्मादो मोहोऽप्युन्मादलक्षणः}
{सम्बुध्ये चाहमात्मानमिमं चापि वनौकसम्} %5-34-24

\twolineshloka
{इत्येवं बहुधा सीता सम्प्रधार्य बलाबलम्}
{रक्षसां कामरूपत्वान्मेने तं राक्षसाधिपम्} %5-34-25

\twolineshloka
{एतां बुद्धिं तदा कृत्वा सीता सा तनुमध्यमा}
{न प्रतिव्याजहाराथ वानरं जनकात्मजा} %5-34-26

\twolineshloka
{सीताया निश्चितं बुद्ध्वा हनूमान् मारुतात्मजः}
{श्रोत्रानुकूलैर्वचनैस्तदा तां सम्प्रहर्षयन्} %5-34-27

\twolineshloka
{आदित्य इव तेजस्वी लोककान्तः शशी यथा}
{राजा सर्वस्य लोकस्य देवो वैश्रवणो यथा} %5-34-28

\twolineshloka
{विक्रमेणोपपन्नश्च यथा विष्णुर्महायशाः}
{सत्यवादी मधुरवाग् देवो वाचस्पतिर्यथा} %5-34-29

\twolineshloka
{रूपवान् सुभगः श्रीमान् कन्दर्प इव मूर्तिमान्}
{स्थानक्रोधे प्रहर्ता च श्रेष्ठो लोके महारथः} %5-34-30

\twolineshloka
{बाहुच्छायामवष्टब्धो यस्य लोको महात्मनः}
{अपक्रम्याश्रमपदान्मृगरूपेण राघवम्} %5-34-31

\twolineshloka
{शून्ये येनापनीतासि तस्य द्रक्ष्यसि तत्फलम्}
{अचिराद् रावणं सङ्ख्ये यो वधिष्यति वीर्यवान्} %5-34-32

\twolineshloka
{क्रोधप्रमुक्तैरिषुभिर्ज्वलद्भिरिव पावकैः}
{तेनाहं प्रेषितो दूतस्त्वत्सकाशमिहागतः} %5-34-33

\twolineshloka
{त्वद्वियोगेन दुःखार्तः स त्वां कौशलमब्रवीत्}
{लक्ष्मणश्च महातेजाः सुमित्रानन्दवर्धनः} %5-34-34

\twolineshloka
{अभिवाद्य महाबाहुः स त्वां कौशलमब्रवीत्}
{रामस्य च सखा देवि सुग्रीवो नाम वानरः} %5-34-35

\twolineshloka
{राजा वानरमुख्यानां स त्वां कौशलमब्रवीत्}
{नित्यं स्मरति ते रामः ससुग्रीवः सलक्ष्मणः} %5-34-36

\twolineshloka
{दिष्ट्या जीवसि वैदेहि राक्षसीवशमागता}
{नचिराद् द्रक्ष्यसे रामं लक्ष्मणं च महारथम्} %5-34-37

\twolineshloka
{मध्ये वानरकोटीनां सुग्रीवं चामितौजसम्}
{अहं सुग्रीवसचिवो हनूमान् नाम वानरः} %5-34-38

\twolineshloka
{प्रविष्टो नगरीं लङ्कां लङ्घयित्वा महोदधिम्}
{कृत्वा मूर्ध्नि पदन्यासं रावणस्य दुरात्मनः} %5-34-39

\threelineshloka
{त्वां द्रष्टुमुपयातोऽहं समाश्रित्य पराक्रमम्}
{नाहमस्मि तथा देवि यथा मामवगच्छसि}
{विशङ्का त्यज्यतामेषा श्रद्धत्स्व वदतो मम} %5-34-40


॥इत्यार्षे श्रीमद्रामायणे वाल्मीकीये आदिकाव्ये सुन्दरकाण्डे रावणशङ्कानिवारणम् नाम चतुस्त्रिंशः सर्गः ॥५-३४॥
