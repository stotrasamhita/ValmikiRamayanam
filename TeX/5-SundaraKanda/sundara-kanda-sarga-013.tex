\sect{त्रयोदशः सर्गः — हनूमन्निर्वेदः}

\twolineshloka
{विमानात् तु स संक्रम्य प्राकारं हरियूथपः}
{हनूमान् वेगवानासीद् यथा विद्युद् घनान्तरे} %5-13-1

\twolineshloka
{सम्परिक्रम्य हनुमान् रावणस्य निवेशनान्}
{अदृष्ट्वा जानकीं सीतामब्रवीद् वचनं कपिः} %5-13-2

\twolineshloka
{भूयिष्ठं लोलिता लंका रामस्य चरता प्रियम्}
{न हि पश्यामि वैदेहीं सीतां सर्वांगशोभनाम्} %5-13-3

\twolineshloka
{पल्वलानि तटाकानि सरांसि सरितस्तथा}
{नद्योऽनूपवनान्ताश्च दुर्गाश्च धरणीधराः} %5-13-4

\threelineshloka
{लोलिता वसुधा सर्वा न च पश्यामि जानकीम्}
{इह सम्पातिना सीता रावणस्य निवेशने}
{आख्याता गृध्रराजेन न च सा दृश्यते न किम्} %5-13-5

\twolineshloka
{किं नु सीताथ वैदेही मैथिली जनकात्मजा}
{उपतिष्ठेत विवशा रावणेन हृता बलात्} %5-13-6

\twolineshloka
{क्षिप्रमुत्पततो मन्ये सीतामादाय रक्षसः}
{बिभ्यतो रामबाणानामन्तरा पतिता भवेत्} %5-13-7

\twolineshloka
{अथवा ह्रियमाणायाः पथि सिद्धनिषेविते}
{मन्ये पतितमार्याया हृदयं प्रेक्ष्य सागरम्} %5-13-8

\twolineshloka
{रावणस्योरुवेगेन भुजाभ्यां पीडितेन च}
{तया मन्ये विशालाक्ष्या त्यक्तं जीवितमार्यया} %5-13-9

\twolineshloka
{उपर्युपरि सा नूनं सागरं क्रमतस्तदा}
{विचेष्टमाना पतिता समुद्रे जनकात्मजा} %5-13-10

\twolineshloka
{आहो क्षुद्रेण चानेन रक्षन्ती शीलमात्मनः}
{अबन्धुर्भक्षिता सीता रावणेन तपस्विनी} %5-13-11

\twolineshloka
{अथवा राक्षसेन्द्रस्य पत्नीभिरसितेक्षणा}
{अदुष्टा दुष्टभावाभिर्भक्षिता सा भविष्यति} %5-13-12

\twolineshloka
{सम्पूर्णचन्द्रप्रतिमं पद्मपत्रनिभेक्षणम्}
{रामस्य ध्यायती वक्त्रं पञ्चत्वं कृपणा गता} %5-13-13

\twolineshloka
{हा राम लक्ष्मणेत्येवं हायोध्ये चेति मैथिली}
{विलप्य बहु वैदेही न्यस्तदेहा भविष्यति} %5-13-14

\twolineshloka
{अथवा निहिता मन्ये रावणस्य निवेशने}
{भृशं लालप्यते बाला पञ्जरस्थेव सारिका} %5-13-15

\twolineshloka
{जनकस्य कुले जाता रामपत्नी सुमध्यमा}
{कथमुत्पलपत्राक्षी रावणस्य वशं व्रजेत्} %5-13-16

\twolineshloka
{विनष्टा वा प्रणष्टा वा मृता वा जनकात्मजा}
{रामस्य प्रियभार्यस्य न निवेदयितुं क्षमम्} %5-13-17

\twolineshloka
{निवेद्यमाने दोषः स्याद् दोषः स्यादनिवेदने}
{कथं नु खलु कर्तव्यं विषमं प्रतिभाति मे} %5-13-18

\twolineshloka
{अस्मिन् नेवंगते कार्ये प्राप्तकालं क्षमं च किम्}
{भवेदिति मतिं भूयो हनुमान् प्रविचारयन्} %5-13-19

\twolineshloka
{यदि सीतामदृष्ट्वाहं वानरेन्द्रपुरीमितः}
{गमिष्यामि ततः को मे पुरुषार्थो भविष्यति} %5-13-20

\twolineshloka
{ममेदं लङ्घनं व्यर्थं सागरस्य भविष्यति}
{प्रवेशश्चैव लंकायां राक्षसानां च दर्शनम्} %5-13-21

\twolineshloka
{किं वा वक्ष्यति सुग्रीवो हरयो वापि संगताः}
{किष्किन्धामनुसम्प्राप्तं तौ वा दशरथात्मजौ} %5-13-22

\twolineshloka
{गत्वा तु यदि काकुत्स्थं वक्ष्यामि परुषं वचः}
{न दृष्टेति मया सीता ततस्त्यक्ष्यति जीवितम्} %5-13-23

\twolineshloka
{परुषं दारुणं तीक्ष्णं क्रूरमिन्द्रियतापनम्}
{सीतानिमित्तं दुर्वाक्यं श्रुत्वा स न भविष्यति} %5-13-24

\twolineshloka
{तं तु कृच्छ्रगतं दृष्ट्वा पञ्चत्वगतमानसम्}
{भृशानुरक्तमेधावी न भविष्यति लक्ष्मणः} %5-13-25

\twolineshloka
{विनष्टौ भ्रातरौ श्रुत्वा भरतोऽपि मरिष्यति}
{भरतं च मृतं दृष्ट्वा शत्रुघ्नो न भविष्यति} %5-13-26

\twolineshloka
{पुत्रान् मृतान् समीक्ष्याथ न भविष्यन्ति मातरः}
{कौसल्या च सुमित्रा च कैकेयी च न संशयः} %5-13-27

\twolineshloka
{कृतज्ञः सत्यसंधश्च सुग्रीवः प्लवगाधिपः}
{रामं तथागतं दृष्ट्वा ततस्त्यक्ष्यति जीवितम्} %5-13-28

\twolineshloka
{दुर्मना व्यथिता दीना निरानन्दा तपस्विनी}
{पीडिता भर्तृशोकेन रुमा त्यक्ष्यति जीवितम्} %5-13-29

\twolineshloka
{वालिजेन तु दुःखेन पीडिता शोककर्शिता}
{पञ्चत्वमागता राज्ञी तारापि न भविष्यति} %5-13-30

\twolineshloka
{मातापित्रोर्विनाशेन सुग्रीवव्यसनेन च}
{कुमारोऽप्यंगदस्तस्माद् विजहिष्यति जीवितम्} %5-13-31

\twolineshloka
{भर्तृजेन तु दुःखेन अभिभूता वनौकसः}
{शिरांस्यभिहनिष्यन्ति तलैर्मुष्टिभिरेव च} %5-13-32

\twolineshloka
{सान्त्वेनानुप्रदानेन मानेन च यशस्विना}
{लालिताः कपिनाथेन प्राणांस्त्यक्ष्यन्ति वानराः} %5-13-33

\twolineshloka
{न वनेषु न शैलेषु न निरोधेषु वा पुनः}
{क्रीडामनुभविष्यन्ति समेत्य कपिकुञ्जराः} %5-13-34

\twolineshloka
{सपुत्रदाराः सामात्या भर्तृव्यसनपीडिताः}
{शैलाग्रेभ्यः पतिष्यन्ति समेषु विषमेषु च} %5-13-35

\twolineshloka
{विषमुद्बन्धनं वापि प्रवेशं ज्वलनस्य वा}
{उपवासमथो शस्त्रं प्रचरिष्यन्ति वानराः} %5-13-36

\twolineshloka
{घोरमारोदनं मन्ये गते मयि भविष्यति}
{इक्ष्वाकुकुलनाशश्च नाशश्चैव वनौकसाम्} %5-13-37

\twolineshloka
{सोऽहं नैव गमिष्यामि किष्किन्धां नगरीमितः}
{नहि शक्ष्याम्यहं द्रष्टुं सुग्रीवं मैथिलीं विना} %5-13-38

\twolineshloka
{मय्यगच्छति चेहस्थे धर्मात्मानौ महारथौ}
{आशया तौ धरिष्येते वानराश्च तरस्विनः} %5-13-39

\twolineshloka
{हस्तादानो मुखादानो नियतो वृक्षमूलिकः}
{वानप्रस्थो भविष्यामि ह्यदृष्ट्वा जनकात्मजाम्} %5-13-40

\twolineshloka
{सागरानूपजे देशे बहुमूलफलोदके}
{चितिं कृत्वा प्रवेक्ष्यामि समिद्धमरणीसुतम्} %5-13-41

\twolineshloka
{उपविष्टस्य वा सम्यग् लिंगिनं साधयिष्यतः}
{शरीरं भक्षयिष्यन्ति वायसाः श्वापदानि च} %5-13-42

\twolineshloka
{इदमप्यृषिभिर्दृष्टं निर्याणमिति मे मतिः}
{सम्यगापः प्रवेक्ष्यामि न चेत् पश्यामि जानकीम्} %5-13-43

\twolineshloka
{सुजातमूला सुभगा कीर्तिमाला यशस्विनी}
{प्रभग्ना चिररात्राय मम सीतामपश्यतः} %5-13-44

\twolineshloka
{तापसो वा भविष्यामि नियतो वृक्षमूलिकः}
{नेतः प्रतिगमिष्यामि तामदृष्ट्वासितेक्षणाम्} %5-13-45

\twolineshloka
{यदि तु प्रतिगच्छामि सीतामनधिगम्य ताम्}
{अंगदः सहितः सर्वैर्वानरैर्न भविष्यति} %5-13-46

\twolineshloka
{विनाशे बहवो दोषा जीवन् प्राप्नोति भद्रकम्}
{तस्मात् प्राणान् धरिष्यामि ध्रुवो जीवति संगमः} %5-13-47

\twolineshloka
{एवं बहुविधं दुःखं मनसा धारयन् बहु}
{नाध्यगच्छत् तदा पारं शोकस्य कपिकुञ्जरः} %5-13-48

\threelineshloka
{ततो विक्रममासाद्य धैर्यवान् कपिकुञ्जरः}
{रावणं वा वधिष्यामि दशग्रीवं महाबलम्}
{काममस्तु हृता सीता प्रत्याचीर्णं भविष्यति} %5-13-49

\twolineshloka
{अथवैनं समुत्क्षिप्य उपर्युपरि सागरम्}
{रामायोपहरिष्यामि पशुं पशुपतेरिव} %5-13-50

\twolineshloka
{इति चिन्तासमापन्नः सीतामनधिगम्य ताम्}
{ध्यानशोकपरीतात्मा चिन्तयामास वानरः} %5-13-51

\twolineshloka
{यावत् सीतां न पश्यामि रामपत्नीं यशस्विनीम्}
{तावदेतां पुरीं लंकां विचिनोमि पुनः पुनः} %5-13-52

\twolineshloka
{सम्पातिवचनाच्चापि रामं यद्यानयाम्यहम्}
{अपश्यन् राघवो भार्यां निर्दहेत् सर्ववानरान्} %5-13-53

\twolineshloka
{इहैव नियताहारो वत्स्यामि नियतेन्द्रियः}
{न मत्कृते विनश्येयुः सर्वे ते नरवानराः} %5-13-54

\twolineshloka
{अशोकवनिका चापि महतीयं महाद्रुमा}
{इमामधिगमिष्यामि नहीयं विचिता मया} %5-13-55

\twolineshloka
{वसून् रुद्रांस्तथाऽऽदित्यानश्विनौ मरुतोऽपि च}
{नमस्कृत्वा गमिष्यामि रक्षसां शोकवर्धनः} %5-13-56

\twolineshloka
{जित्वा तु राक्षसान् देवीमिक्ष्वाकुकुलनन्दिनीम्}
{सम्प्रदास्यामि रामाय सिद्धीमिव तपस्विने} %5-13-57

\twolineshloka
{स मुहूर्तमिव ध्यात्वा चिन्ताविग्रथितेन्द्रियः}
{उदतिष्ठन् महाबाहुर्हनूमान् मारुतात्मजः} %5-13-58

\twolineshloka
{नमोऽस्तु रामाय सलक्ष्मणाय देव्यै च तस्यै जनकात्मजायै}
{नमोऽस्तु रुद्रेन्द्रयमानिलेभ्यो नमोऽस्तु चन्द्राग्निमरुद्गणेभ्यः} %5-13-59

\twolineshloka
{स तेभ्यस्तु नमस्कृत्वा सुग्रीवाय च मारुतिः}
{दिशः सर्वाः समालोक्य सोऽशोकवनिकां प्रति} %5-13-60

\twolineshloka
{स गत्वा मनसा पूर्वमशोकवनिकां शुभाम्}
{उत्तरं चिन्तयामास वानरो मारुतात्मजः} %5-13-61

\twolineshloka
{ध्रुवं तु रक्षोबहुला भविष्यति वनाकुला}
{अशोकवनिका पुण्या सर्वसंस्कारसंस्कृता} %5-13-62

\twolineshloka
{रक्षिणश्चात्र विहिता नूनं रक्षन्ति पादपान्}
{भगवानपि विश्वात्मा नातिक्षोभं प्रवायति} %5-13-63

\twolineshloka
{संक्षिप्तोऽयं मयाऽऽत्मा च रामार्थे रावणस्य च}
{सिद्धिं दिशन्तु मे सर्वे देवाः सर्षिगणास्त्विह} %5-13-64

\twolineshloka
{ब्रह्मा स्वयम्भूर्भगवान् देवाश्चैव तपस्विनः}
{सिद्धिमग्निश्च वायुश्च पुरुहूतश्च वज्रभृत्} %5-13-65

\twolineshloka
{वरुणः पाशहस्तश्च सोमादित्यौ तथैव च}
{अश्विनौ च महात्मानौ मरुतः सर्व एव च} %5-13-66

\twolineshloka
{सिद्धिं सर्वाणि भूतानि भूतानां चैव यः प्रभुः}
{दास्यन्ति मम ये चान्येऽप्यदृष्टाः पथि गोचराः} %5-13-67

\twolineshloka
{तदुन्नसं पाण्डुरदन्तमव्रणं शुचिस्मितं पद्मपलाशलोचनम्}
{द्रक्ष्ये तदार्यावदनं कदा न्वहं प्रसन्नताराधिपतुल्यवर्चसम्} %5-13-68

\twolineshloka
{क्षुद्रेण हीनेन नृशंसमूर्तिना सुदारुणालंकृतवेषधारिणा}
{बलाभिभूता ह्यबला तपस्विनी कथं नु मे दृष्टिपथेऽद्य सा भवेत्} %5-13-69


॥इत्यार्षे श्रीमद्रामायणे वाल्मीकीये आदिकाव्ये सुन्दरकाण्डे हनूमन्निर्वेदः नाम त्रयोदशः सर्गः ॥५-१३॥
