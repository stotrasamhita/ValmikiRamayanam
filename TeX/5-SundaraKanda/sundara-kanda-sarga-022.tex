\sect{द्वाविंशः सर्गः — मासद्वयावधिकरणम्}

\twolineshloka
{सीताया वचनं श्रुत्वा परुषं राक्षसेश्वरः}
{प्रत्युवाच ततः सीतां विप्रियं प्रियदर्शनाम्} %5-22-1

\twolineshloka
{यथा यथा सान्त्वयिता वश्यः स्त्रीणां तथा तथा}
{यथा यथा प्रियं वक्ता परिभूतस्तथा तथा} %5-22-2

\twolineshloka
{संनियच्छति मे क्रोधं त्वयि कामः समुत्थितः}
{द्रवतो मार्गमासाद्य हयानिव सुसारथिः} %5-22-3

\twolineshloka
{वामः कामो मनुष्याणां यस्मिन् किल निबध्यते}
{जने तस्मिंस्त्वनुक्रोशः स्नेहश्च किल जायते} %5-22-4

\twolineshloka
{एतस्मात् कारणान्न त्वां घातयामि वरानने}
{वधार्हामवमानार्हां मिथ्या प्रव्रजने रताम्} %5-22-5

\twolineshloka
{परुषाणि हि वाक्यानि यानि यानि ब्रवीषि माम्}
{तेषु तेषु वधो युक्तस्तव मैथिलि दारुणः} %5-22-6

\twolineshloka
{एवमुक्त्वा तु वैदेहीं रावणो राक्षसाधिपः}
{क्रोधसंरम्भसंयुक्तः सीतामुत्तरमब्रवीत्} %5-22-7

\twolineshloka
{द्वौ मासौ रक्षितव्यौ मे योऽवधिस्ते मया कृतः}
{ततः शयनमारोह मम त्वं वरवर्णिनि} %5-22-8

\twolineshloka
{द्वाभ्यामूर्ध्वं तु मासाभ्यां भर्तारं मामनिच्छतीम्}
{मम त्वां प्रातराशार्थे सूदाश्छेत्स्यन्ति खण्डशः} %5-22-9

\twolineshloka
{तां भर्त्स्यमानां सम्प्रेक्ष्य राक्षसेन्द्रेण जानकीम्}
{देवगन्धर्वकन्यास्ता विषेदुर्विकृतेक्षणाः} %5-22-10

\twolineshloka
{ओष्ठप्रकारैरपरा नेत्रैर्वक्त्रैस्तथापराः}
{सीतामाश्वासयामासुस्तर्जितां तेन रक्षसा} %5-22-11

\twolineshloka
{ताभिराश्वासिता सीता रावणं राक्षसाधिपम्}
{उवाचात्महितं वाक्यं वृत्तशौटीर्यगर्वितम्} %5-22-12

\twolineshloka
{नूनं न ते जनः कश्चिदस्मिन्निःश्रेयसि स्थितः}
{निवारयति यो न त्वां कर्मणोऽस्माद् विगर्हितात्} %5-22-13

\twolineshloka
{मां हि धर्मात्मनः पत्नीं शचीमिव शचीपतेः}
{त्वदन्यस्त्रिषु लोकेषु प्रार्थयेन्मनसापि कः} %5-22-14

\twolineshloka
{राक्षसाधम रामस्य भार्याममिततेजसः}
{उक्तवानसि यत् पापं क्व गतस्तस्य मोक्ष्यसे} %5-22-15

\twolineshloka
{यथा दृप्तश्च मातंगः शशश्च सहितौ वने}
{तथा द्विरदवद् रामस्त्वं नीच शशवत् स्मृतः} %5-22-16

\twolineshloka
{स त्वमिक्ष्वाकुनाथं वै क्षिपन्निह न लज्जसे}
{चक्षुषो विषये तस्य न यावदुपगच्छसि} %5-22-17

\twolineshloka
{इमे ते नयने क्रूरे विकृते कृष्णपिंगले}
{क्षितौ न पतिते कस्मान्मामनार्य निरीक्षतः} %5-22-18

\twolineshloka
{तस्य धर्मात्मनः पत्नी स्नुषा दशरथस्य च}
{कथं व्याहरतो मां ते न जिह्वा पाप शीर्यति} %5-22-19

\twolineshloka
{असंदेशात्तु रामस्य तपसश्चानुपालनात्}
{न त्वां कुर्मि दशग्रीव भस्म भस्मार्हतेजसा} %5-22-20

\twolineshloka
{नापहर्तुमहं शक्या तस्य रामस्य धीमतः}
{विधिस्तव वधार्थाय विहितो नात्र संशयः} %5-22-21

\twolineshloka
{शूरेण धनदभ्रात्रा बलैः समुदितेन च}
{अपोह्य रामं कस्माच्चिद् दारचौर्यं त्वया कृतम्} %5-22-22

\twolineshloka
{सीताया वचनं श्रुत्वा रावणो राक्षसाधिपः}
{विवृत्य नयने क्रूरे जानकीमन्ववैक्षत} %5-22-23

\twolineshloka
{नीलजीमूतसंकाशो महाभुजशिरोधरः}
{सिंहसत्त्वगतिः श्रीमान् दीप्तजिह्वोग्रलोचनः} %5-22-24

\twolineshloka
{चलाग्रमुकुटप्रांशुश्चित्रमाल्यानुलेपनः}
{रक्तमाल्याम्बरधरस्तप्तांगदविभूषणः} %5-22-25

\twolineshloka
{श्रोणीसूत्रेण महता मेचकेन सुसंवृतः}
{अमृतोत्पादने नद्धो भुजंगेनेव मन्दरः} %5-22-26

\twolineshloka
{ताभ्यां स परिपूर्णाभ्यां भुजाभ्यां राक्षसेश्वरः}
{शुशुभेऽचलसंकाशः शृंगाभ्यामिव मन्दरः} %5-22-27

\twolineshloka
{तरुणादित्यवर्णाभ्यां कुण्डलाभ्यां विभूषितः}
{रक्तपल्लवपुष्पाभ्यामशोकाभ्यामिवाचलः} %5-22-28

\twolineshloka
{स कल्पवृक्षप्रतिमो वसन्त इव मूर्तिमान्}
{श्मशानचैत्यप्रतिमो भूषितोऽपि भयंकरः} %5-22-29

\twolineshloka
{अवेक्षमाणो वैदेहीं कोपसंरक्तलोचनः}
{उवाच रावणः सीतां भुजंग इव निःश्वसन्} %5-22-30

\twolineshloka
{अनयेनाभिसम्पन्नमर्थहीनमनुव्रते}
{नाशयाम्यहमद्य त्वां सूर्यः संध्यामिवौजसा} %5-22-31

\twolineshloka
{इत्युक्त्वा मैथिलीं राजा रावणः शत्रुरावणः}
{संददर्श ततः सर्वा राक्षसीर्घोरदर्शनाः} %5-22-32

\twolineshloka
{एकाक्षीमेककर्णां च कर्णप्रावरणां तथा}
{गोकर्णीं हस्तिकर्णीं च लम्बकर्णीमकर्णिकाम्} %5-22-33

\twolineshloka
{हस्तिपद्यश्वपद्यौ च गोपदीं पादचूलिकाम्}
{एकाक्षीमेकपादीं च पृथुपादीमपादिकाम्} %5-22-34

\twolineshloka
{अतिमात्रशिरोग्रीवामतिमात्रकुचोदरीम्}
{अतिमात्रास्यनेत्रां च दीर्घजिह्वानखामपि} %5-22-35

\twolineshloka
{अनासिकां सिंहमुखीं गोमुखीं सूकरीमुखीम्}
{यथा मद्वशगा सीता क्षिप्रं भवति जानकी} %5-22-36

\twolineshloka
{तथा कुरुत राक्षस्यः सर्वाः क्षिप्रं समेत्य वा}
{प्रतिलोमानुलोमैश्च सामदानादिभेदनैः} %5-22-37

\twolineshloka
{आवर्जयत वैदेहीं दण्डस्योद्यमनेन च}
{इति प्रतिसमादिश्य राक्षसेन्द्रः पुनः पुनः} %5-22-38

\twolineshloka
{काममन्युपरीतात्मा जानकीं प्रति गर्जत}
{उपगम्य ततः क्षिप्रं राक्षसी धान्यमालिनी} %5-22-39

\twolineshloka
{परिष्वज्य दशग्रीवमिदं वचनमब्रवीत्}
{मया क्रीड महाराज सीतया किं तवानया} %5-22-40

\twolineshloka
{विवर्णया कृपणया मानुष्या राक्षसेश्वर}
{नूनमस्यां महाराज न देवा भोगसत्तमान्} %5-22-41

\twolineshloka
{विदधत्यमरश्रेष्ठास्तव बाहुबलार्जितान्}
{अकामां कामयानस्य शरीरमुपतप्यते} %5-22-42

\threelineshloka
{इच्छतीं कामयानस्य प्रीतिर्भवति शोभना}
{एवमुक्तस्तु राक्षस्या समुत्क्षिप्तस्ततो बली}
{प्रहसन् मेघसंकाशो राक्षसः स न्यवर्तत} %5-22-43

\twolineshloka
{प्रस्थितः स दशग्रीवः कम्पयन्निव मेदिनीम्}
{ज्वलद्भास्करसंकाशं प्रविवेश निवेशनम्} %5-22-44

\twolineshloka
{देवगन्धर्वकन्याश्च नागकन्याश्च तास्ततः}
{परिवार्य दशग्रीवं प्रविशुस्ता गृहोत्तमम्} %5-22-45

\twolineshloka
{स मैथिलीं धर्मपरामवस्थितां प्रवेपमानां परिभर्त्स्य रावणः}
{विहाय सीतां मदनेन मोहितः स्वमेव वेश्म प्रविवेश रावणः} %5-22-46


॥इत्यार्षे श्रीमद्रामायणे वाल्मीकीये आदिकाव्ये सुन्दरकाण्डे मासद्वयावधिकरणम् नाम द्वाविंशः सर्गः ॥५-२२॥
