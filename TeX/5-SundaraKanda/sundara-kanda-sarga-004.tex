\sect{चतुर्थः सर्गः — लङ्कापुरीप्रवेशः}

\twolineshloka
{स निर्जित्य पुरीं लङ्कां श्रेष्ठां तां कामरूपिणीम्}
{विक्रमेण महातेजा हनूमान् कपिसत्तमः} %5-4-1

\twolineshloka
{अद्वारेण महावीर्यः प्राकारमवपुप्लुवे}
{निशि लङ्कां महासत्त्वो विवेश कपिकुञ्जरः} %5-4-2

\twolineshloka
{प्रविश्य नगरीं लङ्कां कपिराजहितङ्करः}
{चक्रेऽथ पादं सव्यं च शत्रूणां स तु मूर्धनि} %5-4-3

\twolineshloka
{प्रविष्टः सत्त्वसम्पन्नो निशायां मारुतात्मजः}
{स महापथमास्थाय मुक्तपुष्पविराजितम्} %5-4-4

\twolineshloka
{ततस्तु तां पुरीं लङ्कां रम्यामभिययौ कपिः}
{हसितोत्कृष्टनिनदैस्तूर्यघोषपुरस्कृतैः} %5-4-5

\twolineshloka
{वज्राङ्कुशनिकाशैश्च वज्रजालविभूषितैः}
{गृहमेघैः पुरी रम्या बभासे द्यौरिवाम्बुदैः} %5-4-6

\twolineshloka
{प्रजज्वाल तदा लङ्का रक्षोगणगृहैः शुभैः}
{सिताभ्रसदृशैश्चित्रैः पद्मस्वस्तिकसंस्थितैः} %5-4-7

\twolineshloka
{वर्धमानगृहैश्चापि सर्वतः सुविभूषितैः}
{तां चित्रमाल्याभरणां कपिराजहितङ्करः} %5-4-8

\twolineshloka
{राघवार्थे चरन् श्रीमान् ददर्श च ननन्द च}
{भवनाद् भवनं गच्छन् ददर्श कपिकुञ्जरः} %5-4-9

\twolineshloka
{विविधाकृतिरूपाणि भवनानि ततस्ततः}
{शुश्राव रुचिरं गीतं त्रिस्थानस्वरभूषितम्} %5-4-10

\twolineshloka
{स्त्रीणां मदनविद्धानां दिवि चाप्सरसामिव}
{शुश्राव काञ्चीनिनदं नूपुराणां च निःस्वनम्} %5-4-11

\twolineshloka
{सोपाननिनदांश्चापि भवनेषु महात्मनाम्}
{आस्फोटितनिनादांश्च क्ष्वेडितांश्च ततस्ततः} %5-4-12

\twolineshloka
{शुश्राव जपतां तत्र मन्त्रान् रक्षोगृहेषु वै}
{स्वाध्यायनिरतांश्चैव यातुधानान् ददर्श सः} %5-4-13

\twolineshloka
{रावणस्तवसंयुक्तान् गर्जतो राक्षसानपि}
{राजमार्गं समावृत्य स्थितं रक्षोगणं महत्} %5-4-14

\twolineshloka
{ददर्श मध्यमे गुल्मे राक्षसस्य चरान् बहून्}
{दीक्षिताञ्जटिलान् मुण्डान् गोजिनाम्बरवाससः} %5-4-15

\twolineshloka
{दर्भमुष्टिप्रहरणानग्निकुण्डायुधांस्तथा}
{कूटमुद्गरपाणींश्च दण्डायुधधरानपि} %5-4-16

\twolineshloka
{एकाक्षानेकवर्णांश्च लम्बोदरपयोधरान्}
{करालान् भुग्नवक्त्रांश्च विकटान् वामनांस्तथा} %5-4-17

\twolineshloka
{धन्विनः खड्गिनश्चैव शतघ्नीमुसलायुधान्}
{परिघोत्तमहस्तांश्च विचित्रकवचोज्ज्वलान्} %5-4-18

\twolineshloka
{नातिस्थूलान् नातिकृशान् नातिदीर्घातिह्रस्वकान्}
{नातिगौरान् नातिकृष्णान्नातिकुब्जान्न वामनान्} %5-4-19

\twolineshloka
{विरूपान् बहुरूपांश्च सुरूपांश्च सुवर्चसः}
{ध्वजिनः पताकिनश्चैव ददर्श विविधायुधान्} %5-4-20

\twolineshloka
{शक्तिवृक्षायुधांश्चैव पट्टिशाशनिधारिणः}
{क्षेपणीपाशहस्तांश्च ददर्श स महाकपिः} %5-4-21

\twolineshloka
{स्रग्विणस्त्वनुलिप्तांश्च वराभरणभूषितान्}
{नानावेषसमायुक्तान् यथास्वैरचरान् बहून्} %5-4-22

\twolineshloka
{तीक्ष्णशूलधरांश्चैव वज्रिणश्च महाबलान्}
{शतसाहस्रमव्यग्रमारक्षं मध्यमं कपिः} %5-4-23

\twolineshloka
{रक्षोऽधिपतिनिर्दिष्टं ददर्शान्तःपुराग्रतः}
{स तदा तद् गृहं दृष्ट्वा महाहाटकतोरणम्} %5-4-24

\twolineshloka
{राक्षसेन्द्रस्य विख्यातमद्रिमूर्ध्नि प्रतिष्ठितम्}
{पुण्डरीकावतंसाभिः परिखाभिः समावृतम्} %5-4-25

\twolineshloka
{प्राकारावृतमत्यन्तं ददर्श स महाकपिः}
{त्रिविष्टपनिभं दिव्यं दिव्यनादविनादितम्} %5-4-26

\twolineshloka
{वाजिह्रेषितसङ्घुष्टं नादितं भूषणैस्तथा}
{रथैर्यानैर्विमानैश्च तथा हयगजैः शुभैः} %5-4-27

\twolineshloka
{वारणैश्च चतुर्दन्तैः श्वेताभ्रनिचयोपमैः}
{भूषितै रुचिरद्वारं मत्तैश्च मृगपक्षिभिः} %5-4-28

\twolineshloka
{रक्षितं सुमहावीर्यैर्यातुधानैः सहस्रशः}
{राक्षसाधिपतेर्गुप्तमाविवेश गृहं कपिः} %5-4-29

\twolineshloka
{स हेमजाम्बूनदचक्रवालं महार्हमुक्तामणि भूषितान्तम्}
{परार्घ्यकालागुरुचन्दनार्हं स रावणान्तःपुरमाविवेश} %5-4-30


॥इत्यार्षे श्रीमद्रामायणे वाल्मीकीये आदिकाव्ये सुन्दरकाण्डे लङ्कापुरीप्रवेशः नाम चतुर्थः सर्गः ॥५-४॥
