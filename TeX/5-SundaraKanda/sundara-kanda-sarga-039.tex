\sect{एकोनचत्वारिंशः सर्गः — हनूमत्सन्देशः}

\twolineshloka
{मणिं दत्त्वा ततः सीता हनूमन्तमथाब्रवीत्}
{अभिज्ञानमभिज्ञातमेतद् रामस्य तत्त्वतः} %5-39-1

\twolineshloka
{मणिं दृष्ट्वा तु रामो वै त्रयाणां संस्मरिष्यति}
{वीरो जनन्या मम च राज्ञो दशरथस्य च} %5-39-2

\twolineshloka
{स भूयस्त्वं समुत्साहचोदितो हरिसत्तम}
{अस्मिन् कार्यसमुत्साहे प्रचिन्तय यदुत्तरम्} %5-39-3

\twolineshloka
{त्वमस्मिन् कार्यनिर्योगे प्रमाणं हरिसत्तम}
{तस्य चिन्तय यो यत्नो दुःखक्षयकरो भवेत्} %5-39-4

\twolineshloka
{हनूमन् यत्नमास्थाय दुःखक्षयकरो भव}
{स तथेति प्रतिज्ञाय मारुतिर्भीमविक्रमः} %5-39-5

\twolineshloka
{शिरसाऽऽवन्द्य वैदेहीं गमनायोपचक्रमे}
{ज्ञात्वा सम्प्रस्थितं देवी वानरं पवनात्मजम्} %5-39-6

\twolineshloka
{बाष्पगद्गदया वाचा मैथिली वाक्यमब्रवीत्}
{हनूमन् कुशलं ब्रूयाः सहितौ रामलक्ष्मणौ} %5-39-7

\twolineshloka
{सुग्रीवं च सहामात्यं सर्वान् वृद्धांश्च वानरान्}
{ब्रूयास्त्वं वानरश्रेष्ठ कुशलं धर्मसंहितम्} %5-39-8

\twolineshloka
{यथा च स महाबाहुर्मां तारयति राघवः}
{अस्माद् दुःखाम्बुसंरोधात् त्वं समाधातुमर्हसि} %5-39-9

\twolineshloka
{जीवन्तीं मां यथा रामः सम्भावयति कीर्तिमान्}
{तत् त्वया हनुमन् वाच्यं वाचा धर्ममवाप्नुहि} %5-39-10

\twolineshloka
{नित्यमुत्साहयुक्तस्य वाचः श्रुत्वा मयेरिताः}
{वर्धिष्यते दाशरथेः पौरुषं मदवाप्तये} %5-39-11

\twolineshloka
{मत्संदेशयुता वाचस्त्वत्तः श्रुत्वैव राघवः}
{पराक्रमे मतिं वीरो विधिवत् संविधास्यति} %5-39-12

\twolineshloka
{सीतायास्तद् वचः श्रुत्वा हनूमान् मारुतात्मजः}
{शिरस्यञ्जलिमाधाय वाक्यमुत्तरमब्रवीत्} %5-39-13

\twolineshloka
{क्षिप्रमेष्यति काकुत्स्थो हर्यृक्षप्रवरैर्वृतः}
{यस्ते युधि विजित्यारीन् शोकं व्यपनयिष्यति} %5-39-14

\twolineshloka
{नहि पश्यामि मर्त्येषु नासुरेषु सुरेषु वा}
{यस्तस्य वमतो बाणान् स्थातुमुत्सहतेऽग्रतः} %5-39-15

\twolineshloka
{अप्यर्कमपि पर्जन्यमपि वैवस्वतं यमम्}
{स हि सोढुं रणे शक्तस्तव हेतोर्विशेषतः} %5-39-16

\twolineshloka
{स हि सागरपर्यन्तां महीं साधितुमर्हति}
{त्वन्निमित्तो हि रामस्य जयो जनकनन्दिनि} %5-39-17

\twolineshloka
{तस्य तद् वचनं श्रुत्वा सम्यक् सत्यं सुभाषितम्}
{जानकी बहु मेने तं वचनं चेदमब्रवीत्} %5-39-18

\twolineshloka
{ततस्तं प्रस्थितं सीता वीक्षमाणा पुनः पुनः}
{भर्तृस्नेहान्वितं वाक्यं सौहार्दादनुमानयत्} %5-39-19

\twolineshloka
{यदि वा मन्यसे वीर वसैकाहमरिंदम}
{कस्मिंश्चित् संवृते देशे विश्रान्तः श्वो गमिष्यसि} %5-39-20

\twolineshloka
{मम चैवाल्पभाग्यायाः सांनिध्यात् तव वानर}
{अस्य शोकस्य महतो मुहूर्तं मोक्षणं भवेत्} %5-39-21

\twolineshloka
{ततो हि हरिशार्दूल पुनरागमनाय तु}
{प्राणानामपि संदेहो मम स्यान्नात्र संशयः} %5-39-22

\twolineshloka
{तवादर्शनजः शोको भूयो मां परितापयेत्}
{दुःखादुःखपरामृष्टां दीपयन्निव वानर} %5-39-23

\twolineshloka
{अयं च वीर संदेहस्तिष्ठतीव ममाग्रतः}
{सुमहांस्त्वत्सहायेषु हर्यृक्षेषु हरीश्वर} %5-39-24

\twolineshloka
{कथं नु खलु दुष्पारं तरिष्यन्ति महोदधिम्}
{तानि हर्यृक्षसैन्यानि तौ वा नरवरात्मजौ} %5-39-25

\twolineshloka
{त्रयाणामेव भूतानां सागरस्येह लङ्घने}
{शक्तिः स्याद् वैनतेयस्य तव वा मारुतस्य वा} %5-39-26

\twolineshloka
{तदस्मिन् कार्यनिर्योगे वीरैवं दुरतिक्रमे}
{किं पश्यसे समाधानं त्वं हि कार्यविदां वरः} %5-39-27

\twolineshloka
{काममस्य त्वमेवैकः कार्यस्य परिसाधने}
{पर्याप्तः परवीरघ्न यशस्यस्ते फलोदयः} %5-39-28

\twolineshloka
{बलैः समग्रैर्युधि मां रावणं जित्य संयुगे}
{विजयी स्वपुरं यायात् तत्तस्य सदृशं भवेत्} %5-39-29

\twolineshloka
{बलैस्तु संकुलां कृत्वा लङ्कां परबलार्दनः}
{मां नयेद् यदि काकुत्स्थस्तत् तस्य सदृशं भवेत्} %5-39-30

\twolineshloka
{तद्यथा तस्य विक्रान्तमनुरूपं महात्मनः}
{भवेदाहवशूरस्य तथा त्वमुपपादय} %5-39-31

\twolineshloka
{तदर्थोपहितं वाक्यं प्रश्रितं हेतुसंहितम्}
{निशम्य हनुमान् शेषं वाक्यमुत्तरमब्रवीत्} %5-39-32

\twolineshloka
{देवि हर्यृक्षसैन्यानामीश्वरः प्लवतां वरः}
{सुग्रीवः सत्यसम्पन्नस्तवार्थे कृतनिश्चयः} %5-39-33

\twolineshloka
{स वानरसहस्राणां कोटीभिरभिसंवृतः}
{क्षिप्रमेष्यति वैदेहि राक्षसानां निबर्हणः} %5-39-34

\twolineshloka
{तस्य विक्रमसम्पन्नाः सत्त्ववन्तो महाबलाः}
{मनःसंकल्पसम्पाता निदेशे हरयः स्थिताः} %5-39-35

\twolineshloka
{येषां नोपरि नाधस्तान्न तिर्यक् सज्जते गतिः}
{न च कर्मसु सीदन्ति महत्स्वमिततेजसः} %5-39-36

\twolineshloka
{असकृत् तैर्महोत्साहैः ससागरधराधरा}
{प्रदक्षिणीकृता भूमिर्वायुमार्गानुसारिभिः} %5-39-37

\twolineshloka
{मद्विशिष्टाश्च तुल्याश्च सन्ति तत्र वनौकसः}
{मत्तः प्रत्यवरः कश्चिन्नास्ति सुग्रीवसंनिधौ} %5-39-38

\twolineshloka
{अहं तावदिह प्राप्तः किं पुनस्ते महाबलाः}
{नहि प्रकृष्टाः प्रेष्यन्ते प्रेष्यन्ते हीतरे जनाः} %5-39-39

\twolineshloka
{तदलं परितापेन देवि शोको व्यपैतु ते}
{एकोत्पातेन ते लङ्कामेष्यन्ति हरियूथपाः} %5-39-40

\twolineshloka
{मम पृष्ठगतौ तौ च चन्द्रसूर्याविवोदितौ}
{त्वत्सकाशं महासङ्घौ नृसिंहावागमिष्यतः} %5-39-41

\twolineshloka
{तौ हि वीरौ नरवरौ सहितौ रामलक्ष्मणौ}
{आगम्य नगरीं लङ्कां सायकैर्विधमिष्यतः} %5-39-42

\twolineshloka
{सगणं रावणं हत्वा राघवो रघुनन्दनः}
{त्वामादाय वरारोहे स्वपुरीं प्रति यास्यति} %5-39-43

\twolineshloka
{तदाश्वसिहि भद्रं ते भव त्वं कालकाङ्क्षिणी}
{नचिराद् द्रक्ष्यसे रामं प्रज्वलन्तमिवानलम्} %5-39-44

\twolineshloka
{निहते राक्षसेन्द्रे च सपुत्रामात्यबान्धवे}
{त्वं समेष्यसि रामेण शशाङ्केनेव रोहिणी} %5-39-45

\twolineshloka
{क्षिप्रं त्वं देवि शोकस्य पारं द्रक्ष्यसि मैथिलि}
{रावणं चैव रामेण द्रक्ष्यसे निहतं बलात्} %5-39-46

\twolineshloka
{एवमाश्वास्य वैदेहीं हनूमान् मारुतात्मजः}
{गमनाय मतिं कृत्वा वैदेहीं पुनरब्रवीत्} %5-39-47

\twolineshloka
{तमरिघ्नं कृतात्मानं क्षिप्रं द्रक्ष्यसि राघवम्}
{लक्ष्मणं च धनुष्पाणिं लङ्काद्वारमुपागतम्} %5-39-48

\twolineshloka
{नखदंष्ट्रायुधान् वीरान् सिंहशार्दूलविक्रमान्}
{वानरान् वारणेन्द्राभान् क्षिप्रं द्रक्ष्यसि संगतान्} %5-39-49

\twolineshloka
{शैलाम्बुदनिकाशानां लङ्कामलयसानुषु}
{नर्दतां कपिमुख्यानामार्ये यूथान्यनेकशः} %5-39-50

\twolineshloka
{स तु मर्मणि घोरेण ताडितो मन्मथेषुणा}
{न शर्म लभते रामः सिंहार्दित इव द्विपः} %5-39-51

\twolineshloka
{रुद मा देवि शोकेन मा भूत् ते मनसो भयम्}
{शचीव भर्त्रा शक्रेण सङ्गमेष्यसि शोभने} %5-39-52

\twolineshloka
{रामाद् विशिष्टः कोऽन्योऽस्ति कश्चित् सौमित्रिणा समः}
{अग्निमारुतकल्पौ तौ भ्रातरौ तव संश्रयौ} %5-39-53

\twolineshloka
{नास्मिंश्चिरं वत्स्यसि देवि देशे रक्षोगणैरध्युषितेऽतिरौद्रे}
{न ते चिरादागमनं प्रियस्य क्षमस्व मत्संगमकालमात्रम्} %5-39-54


॥इत्यार्षे श्रीमद्रामायणे वाल्मीकीये आदिकाव्ये सुन्दरकाण्डे हनूमत्सन्देशः नाम एकोनचत्वारिंशः सर्गः ॥५-३९॥
