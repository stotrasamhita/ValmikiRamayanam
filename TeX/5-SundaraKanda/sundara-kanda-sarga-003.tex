\sect{तृतीयः सर्गः — लङ्काधिदेवताविजयः}

\twolineshloka
{स लम्बशिखरे लम्बे लम्बतोयदसन्निभे}
{सत्त्वमास्थाय मेधावी हनुमान् मारुतात्मजः} %5-3-1

\twolineshloka
{निशि लङ्कां महासत्त्वो विवेश कपिकुञ्जरः}
{रम्यकाननतोयाढ्यां पुरीं रावणपालिताम्} %5-3-2

\twolineshloka
{शारदाम्बुधरप्रख्यैर्भवनैरुपशोभिताम्}
{सागरोपमनिर्घोषां सागरानिलसेविताम्} %5-3-3

\twolineshloka
{सुपुष्टबलसम्पुष्टां यथैव विटपावतीम्}
{चारुतोरणनिर्यूहां पाण्डुरद्वारतोरणाम्} %5-3-4

\twolineshloka
{भुजगाचरितां गुप्तां शुभां भोगवतीमिव}
{तां सविद्युद्घनाकीर्णां ज्योतिर्गणनिषेविताम्} %5-3-5

\twolineshloka
{चण्डमारुतनिर्ह्रादां यथा चाप्यमरावतीम्}
{शातकुम्भेन महता प्राकारेणाभिसंवृताम्} %5-3-6

\twolineshloka
{किङ्किणीजालघोषाभिः पताकाभिरलङ्कृताम्}
{आसाद्य सहसा हृष्टः प्राकारमभिपेदिवान्} %5-3-7

\twolineshloka
{विस्मयाविष्टहृदयः पुरीमालोक्य सर्वतः}
{जाम्बूनदमयैर्द्वारैर्वैदूर्यकृतवेदिकैः} %5-3-8

\twolineshloka
{वज्रस्फटिकमुक्ताभिर्मणिकुट्टिमभूषितैः}
{तप्तहाटकनिर्यूहै राजतामलपाण्डुरैः} %5-3-9

\twolineshloka
{वैदूर्यकृतसोपानैः स्फाटिकान्तरपांसुभिः}
{चारुसञ्जवनोपेतैः खमिवोत्पतितैः शुभैः} %5-3-10

\twolineshloka
{क्रौञ्चबर्हिणसङ्घुष्टै राजहंसनिषेवितैः}
{तूर्याभरणनिर्घोषैः सर्वतः परिनादिताम्} %5-3-11

\twolineshloka
{वस्वोकसारप्रतिमां समीक्ष्य नगरीं ततः}
{खमिवोत्पतितां लङ्कां जहर्ष हनुमान् कपिः} %5-3-12

\twolineshloka
{तां समीक्ष्य पुरीं लङ्कां राक्षसाधिपतेः शुभाम्}
{अनुत्तमामृद्धिमतीं चिन्तयामास वीर्यवान्} %5-3-13

\twolineshloka
{नेयमन्येन नगरी शक्या धर्षयितुं बलात्}
{रक्षिता रावणबलैरुद्यतायुधपाणिभिः} %5-3-14

\twolineshloka
{कुमुदाङ्गदयोर्वापि सुषेणस्य महाकपेः}
{प्रसिद्धेयं भवेद् भूमिर्मैन्दद्विविदयोरपि} %5-3-15

\twolineshloka
{विवस्वतस्तनूजस्य हरेश्च कुशपर्वणः}
{ऋक्षस्य कपिमुख्यस्य मम चैव गतिर्भवेत्} %5-3-16

\twolineshloka
{समीक्ष्य च महाबाहो राघवस्य पराक्रमम्}
{लक्ष्मणस्य च विक्रान्तमभवत् प्रीतिमान् कपिः} %5-3-17

\twolineshloka
{तां रत्नवसनोपेतां गोष्ठागारावतंसिकाम्}
{यन्त्रागारस्तनीमृद्धां प्रमदामिव भूषिताम्} %5-3-18

\twolineshloka
{तां नष्टतिमिरां दीपैर्भास्वरैश्च महाग्रहैः}
{नगरीं राक्षसेन्द्रस्य स ददर्श महाकपिः} %5-3-19

\twolineshloka
{अथ सा हरिशार्दूलं प्रविशन्तं महाकपिम्}
{नगरी स्वेन रूपेण ददर्श पवनात्मजम्} %5-3-20

\twolineshloka
{सा तं हरिवरं दृष्ट्वा लङ्का रावणपालिता}
{स्वयमेवोत्थिता तत्र विकृताननदर्शना} %5-3-21

\twolineshloka
{पुरस्तात् तस्य वीरस्य वायुसूनोरतिष्ठत}
{मुञ्चमाना महानादमब्रवीत् पवनात्मजम्} %5-3-22

\twolineshloka
{कस्त्वं केन च कार्येण इह प्राप्तो वनालय}
{कथयस्वेह यत् तत्त्वं यावत् प्राणा धरन्ति ते} %5-3-23

\twolineshloka
{न शक्यं खल्वियं लङ्का प्रवेष्टुं वानर त्वया}
{रक्षिता रावणबलैरभिगुप्ता समन्ततः} %5-3-24

\twolineshloka
{अथ तामब्रवीद् वीरो हनुमानग्रतः स्थिताम्}
{कथयिष्यामि तत् तत्त्वं यन्मां त्वं परिपृच्छसे} %5-3-25

\twolineshloka
{का त्वं विरूपनयना पुरद्वारेऽवतिष्ठसे}
{किमर्थं चापि मां क्रोधान्निर्भर्त्सयसि दारुणे} %5-3-26

\twolineshloka
{हनुमद्वचनं श्रुत्वा लङ्का सा कामरूपिणी}
{उवाच वचनं क्रुद्धा परुषं पवनात्मजम्} %5-3-27

\twolineshloka
{अहं राक्षसराजस्य रावणस्य महात्मनः}
{आज्ञाप्रतीक्षा दुर्धर्षा रक्षामि नगरीमिमाम्} %5-3-28

\twolineshloka
{न शक्यं मामवज्ञाय प्रवेष्टुं नगरीमिमाम्}
{अद्य प्राणैः परित्यक्तः स्वप्स्यसे निहतो मया} %5-3-29

\twolineshloka
{अहं हि नगरी लङ्का स्वयमेव प्लवङ्गम}
{सर्वतः परिरक्षामि अतस्ते कथितं मया} %5-3-30

\twolineshloka
{लङ्काया वचनं श्रुत्वा हनुमान् मारुतात्मजः}
{यत्नवान् स हरिश्रेष्ठः स्थितः शैल इवापरः} %5-3-31

\twolineshloka
{स तां स्त्रीरूपविकृतां दृष्ट्वा वानरपुङ्गवः}
{आबभाषेऽथ मेधावी सत्त्ववान् प्लवगर्षभः} %5-3-32

\twolineshloka
{द्रक्ष्यामि नगरीं लङ्कां साट्टप्राकारतोरणाम्}
{इत्यर्थमिह सम्प्राप्तः परं कौतूहलं हि मे} %5-3-33

\twolineshloka
{वनान्युपवनानीह लङ्कायाः काननानि च}
{सर्वतो गृहमुख्यानि द्रष्टुमागमनं हि मे} %5-3-34

\twolineshloka
{तस्य तद् वचनं श्रुत्वा लङ्का सा कामरूपिणी}
{भूय एव पुनर्वाक्यं बभाषे परुषाक्षरम्} %5-3-35

\twolineshloka
{मामनिर्जित्य दुर्बुद्धे राक्षसेश्वरपालिताम्}
{न शक्यं ह्यद्य ते द्रष्टुं पुरीयं वानराधम} %5-3-36

\twolineshloka
{ततः स हरिशार्दूलस्तामुवाच निशाचरीम्}
{दृष्ट्वा पुरीमिमां भद्रे पुनर्यास्ये यथागतम्} %5-3-37

\twolineshloka
{ततः कृत्वा महानादं सा वै लङ्का भयङ्करम्}
{तलेन वानरश्रेष्ठं ताडयामास वेगिता} %5-3-38

\twolineshloka
{ततः स हरिशार्दूलो लङ्कया ताडितो भृशम्}
{ननाद सुमहानादं वीर्यवान् मारुतात्मजः} %5-3-39

\twolineshloka
{ततः संवर्तयामास वामहस्तस्य सोऽङ्गुलीः}
{मुष्टिनाभिजघानैनां हनुमान् क्रोधर्मूच्छितः} %5-3-40

\threelineshloka
{स्त्री चेति मन्यमानेन नातिक्रोधः स्वयं कृतः}
{सा तु तेन प्रहारेण विह्वलाङ्गी निशाचरी}
{पपात सहसा भूमौ विकृताननदर्शना} %5-3-41

\twolineshloka
{ततस्तु हनुमान् वीरस्तां दृष्ट्वा विनिपातिताम्}
{कृपां चकार तेजस्वी मन्यमानः स्त्रियं च ताम्} %5-3-42

\twolineshloka
{ततो वै भृशमुद्विग्ना लङ्का सा गद्गदाक्षरम्}
{उवाचागर्वितं वाक्यं हनुमन्तं प्लवङ्गमम्} %5-3-43

\twolineshloka
{प्रसीद सुमहाबाहो त्रायस्व हरिसत्तम}
{समये सौम्य तिष्ठन्ति सत्त्ववन्तो महाबलाः} %5-3-44

\twolineshloka
{अहं तु नगरी लङ्का स्वयमेव प्लवङ्गम}
{निर्जिताहं त्वया वीर विक्रमेण महाबल} %5-3-45

\twolineshloka
{इदं च तथ्यं शृणु मे ब्रुवन्त्या वै हरीश्वर}
{स्वयं स्वयम्भुवा दत्तं वरदानं यथा मम} %5-3-46

\twolineshloka
{यदा त्वां वानरः कश्चिद् विक्रमाद् वशमानयेत्}
{तदा त्वया हि विज्ञेयं रक्षसां भयमागतम्} %5-3-47

\twolineshloka
{स हि मे समयः सौम्य प्राप्तोऽद्य तव दर्शनात्}
{स्वयम्भूविहितः सत्यो न तस्यास्ति व्यतिक्रमः} %5-3-48

\twolineshloka
{सीतानिमित्तं राज्ञस्तु रावणस्य दुरात्मनः}
{रक्षसां चैव सर्वेषां विनाशः समुपागतः} %5-3-49

\twolineshloka
{तत् प्रविश्य हरिश्रेष्ठ पुरीं रावणपालिताम्}
{विधत्स्व सर्वकार्याणि यानि यानीह वाञ्छसि} %5-3-50

\twolineshloka
{प्रविश्य शापोपहतां हरीश्वर पुरीं शुभां राक्षसमुख्यपालिताम्}
{यदृच्छया त्वं जनकात्मजां सतीं विमार्ग सर्वत्र गतो यथासुखम्} %5-3-51


॥इत्यार्षे श्रीमद्रामायणे वाल्मीकीये आदिकाव्ये सुन्दरकाण्डे लङ्काधिदेवताविजयः नाम तृतीयः सर्गः ॥५-३॥
