\sect{त्रयोविंशः सर्गः — राक्षसीप्ररोचनम्}

\twolineshloka
{इत्युक्त्वा मैथिलीं राजा रावणः शत्रुरावणः}
{संदिश्य च ततः सर्वा राक्षसीर्निर्जगाम ह} %5-23-1

\twolineshloka
{निष्क्रान्ते राक्षसेन्द्रे तु पुनरन्तःपुरं गते}
{राक्षस्यो भीमरूपास्ताः सीतां समभिदुद्रुवुः} %5-23-2

\twolineshloka
{ततः सीतामुपागम्य राक्षस्यः क्रोधमूर्च्छिताः}
{परं परुषया वाचा वैदेहीमिदमब्रुवन्} %5-23-3

\twolineshloka
{पौलस्त्यस्य वरिष्ठस्य रावणस्य महात्मनः}
{दशग्रीवस्य भार्यात्वं सीते न बहु मन्यसे} %5-23-4

\twolineshloka
{ततस्त्वेकजटा नाम राक्षसी वाक्यमब्रवीत्}
{आमन्त्र्य क्रोधताम्राक्षी सीतां करतलोदरीम्} %5-23-5

\twolineshloka
{प्रजापतीनां षण्णां तु चतुर्थोऽयं प्रजापतिः}
{मानसो ब्रह्मणः पुत्रः पुलस्त्य इति विश्रुतः} %5-23-6

\twolineshloka
{पुलस्त्यस्य तु तेजस्वी महर्षिर्मानसः सुतः}
{नाम्ना स विश्रवा नाम प्रजापतिसमप्रभः} %5-23-7

\twolineshloka
{तस्य पुत्रो विशालाक्षि रावणः शत्रुरावणः}
{तस्य त्वं राक्षसेन्द्रस्य भार्या भवितुमर्हसि} %5-23-8

\twolineshloka
{मयोक्तं चारुसर्वाङ्गि वाक्यं किं नानुमन्यसे}
{ततो हरिजटा नाम राक्षसी वाक्यमब्रवीत्} %5-23-9

\twolineshloka
{विवृत्य नयने कोपान्मार्जारसदृशेक्षणा}
{येन देवास्त्रयस्त्रिंशद् देवराजश्च निर्जितः} %5-23-10

\threelineshloka
{तस्य त्वं राक्षसेन्द्रस्य भार्या भवितुमर्हसि}
{वीर्योत्सिक्तस्य शूरस्य संग्रामेष्वनिवर्तिनः}
{बलिनो वीर्ययुक्तस्य भार्यात्वं किं न लिप्ससे} %5-23-11

\twolineshloka
{प्रियां बहुमतां भार्यां त्यक्त्वा राजा महाबलः}
{सर्वासां च महाभागां त्वामुपैष्यति रावणः} %5-23-12

\twolineshloka
{समृद्धं स्त्रीसहस्रेण नानारत्नोपशोभितम्}
{अन्तःपुरं तदुत्सृज्य त्वामुपैष्यति रावणः} %5-23-13

\threelineshloka
{अन्या तु विकटा नाम राक्षसी वाक्यमब्रवीत्}
{असकृद् भीमवीर्येण नागा गन्धर्वदानवाः}
{निर्जिताः समरे येन स ते पार्श्वमुपागतः} %5-23-14

\twolineshloka
{तस्य सर्वसमृद्धस्य रावणस्य महात्मनः}
{किमर्थं राक्षसेन्द्रस्य भार्यात्वं नेच्छसेऽधमे} %5-23-15

\threelineshloka
{ततस्तां दुर्मुखी नाम राक्षसी वाक्यमब्रवीत्}
{यस्य सूर्यो न तपति भीतो यस्य स मारुतः}
{न वाति स्मायतापाङ्गि किं त्वं तस्य न तिष्ठसे} %5-23-16

\twolineshloka
{पुष्पवृष्टिं च तरवो मुमुचुर्यस्य वै भयात्}
{शैलाः सुस्रुवुः पानीयं जलदाश्च यदेच्छति} %5-23-17

\twolineshloka
{तस्य नैर्ऋतराजस्य राजराजस्य भामिनि}
{किं त्वं न कुरुषे बुद्धिं भार्यार्थे रावणस्य हि} %5-23-18

\twolineshloka
{साधु ते तत्त्वतो देवि कथितं साधु भामिनि}
{गृहाण सुस्मिते वाक्यमन्यथा न भविष्यसि} %5-23-19


॥इत्यार्षे श्रीमद्रामायणे वाल्मीकीये आदिकाव्ये सुन्दरकाण्डे राक्षसीप्ररोचनम् नाम त्रयोविंशः सर्गः ॥५-२३॥
