\sect{त्रिचत्वारिशः सर्गः — चैत्यप्रासाददाहः}

\twolineshloka
{ततः स किङ्करान् हत्वा हनूमान् ध्यानमास्थितः}
{वनं भग्नं मया चैत्यप्रासादो न विनाशितः} %5-43-1

\twolineshloka
{तस्मात् प्रासादमद्यैवमिमं विध्वंसयाम्यहम्}
{इति सञ्चिन्त्य हनुमान् मनसादर्शयन् बलम्} %5-43-2

\twolineshloka
{चैत्यप्रासादमुत्प्लुत्य मेरुशृङ्गमिवोन्नतम्}
{आरुरोह हरिश्रेष्ठो हनूमान् मारुतात्मजः} %5-43-3

\twolineshloka
{आरुह्य गिरिसङ्काशं प्रासादं हरियूथपः}
{बभौ स सुमहातेजाः प्रतिसूर्य इवोदितः} %5-43-4

\twolineshloka
{सम्प्रधृष्य तु दुर्धर्षश्चैत्यप्रासादमुन्नतम्}
{हनूमान् प्रज्वलँल्लक्ष्म्या पारियात्रोपमोऽभवत्} %5-43-5

\twolineshloka
{स भूत्वा सुमहाकायः प्रभावान् मारुतात्मजः}
{धृष्टमास्फोटयामास लङ्कां शब्देन पूरयन्} %5-43-6

\twolineshloka
{तस्यास्फोटितशब्देन महता श्रोत्रघातिना}
{पेतुर्विहङ्गमास्तत्र चैत्यपालाश्च मोहिताः} %5-43-7

\twolineshloka
{अस्त्रविज्जयतां रामो लक्ष्मणश्च महाबलः}
{राजा जयति सुग्रीवो राघवेणाभिपालितः} %5-43-8

\twolineshloka
{दासोऽहं कोसलेन्द्रस्य रामस्याक्लिष्टकर्मणः}
{हनूमान् शत्रुसैन्यानां निहन्ता मारुतात्मजः} %5-43-9

\twolineshloka
{न रावणसहस्रं मे युद्धे प्रतिबलं भवेत्}
{शिलाभिश्च प्रहरतः पादपैश्च सहस्रशः} %5-43-10

\twolineshloka
{धर्षयित्वा पुरीं लङ्कामभिवाद्य च मैथिलीम्}
{समृद्धार्थो गमिष्यामि मिषतां सर्वरक्षसाम्} %5-43-11

\twolineshloka
{एवमुक्त्वा महाकायश्चैत्यस्थो हरियूथपः}
{ननाद भीमनिर्ह्रादो रक्षसां जनयन् भयम्} %5-43-12

\twolineshloka
{तेन नादेन महता चैत्यपालाः शतं ययुः}
{गृहीत्वा विविधानस्त्रान् प्रासान् खड्गान् परश्वधान्} %5-43-13

\twolineshloka
{विसृजन्तो महाकाया मारुतिं पर्यवारयन्}
{ते गदाभिर्विचित्राभिः परिघैः काञ्चनाङ्गदैः} %5-43-14

\twolineshloka
{आजग्मुर्वानरश्रेष्ठं बाणैश्चादित्यसन्निभैः}
{आवर्त इव गङ्गायास्तोयस्य विपुलो महान्} %5-43-15

\twolineshloka
{परिक्षिप्य हरिश्रेष्ठं स बभौ रक्षसां गणः}
{ततो वातात्मजः क्रुद्धो भीमरूपं समास्थितः} %5-43-16

\twolineshloka
{प्रासादस्य महांस्तस्य स्तम्भं हेमपरिष्कृतम्}
{उत्पाटयित्वा वेगेन हनूमान् मारुतात्मजः} %5-43-17

\twolineshloka
{ततस्तं भ्रामयामास शतधारं महाबलः}
{तत्र चाग्निः समभवत् प्रासादश्चाप्यदह्यत} %5-43-18

\twolineshloka
{दह्यमानं ततो दृष्ट्वा प्रासादं हरियूथपः}
{स राक्षसशतं हत्वा वज्रेणेन्द्र इवासुरान्} %5-43-19

\twolineshloka
{अन्तरिक्षस्थितः श्रीमानिदं वचनमब्रवीत्}
{मादृशानां सहस्राणि विसृष्टानि महात्मनाम्} %5-43-20

\twolineshloka
{बलिनां वानरेन्द्राणां सुग्रीववशवर्तिनाम्}
{अटन्ति वसुधां कृत्स्नां वयमन्ये च वानराः} %5-43-21

\twolineshloka
{दशनागबलाः केचित् केचिद् दशगुणोत्तराः}
{केचिन्नागसहस्रस्य बभूवुस्तुल्यविक्रमाः} %5-43-22

\twolineshloka
{सन्ति चौघबलाः केचित् सन्ति वायुबलोपमाः}
{अप्रमेयबलाः केचित् तत्रासन् हरियूथपाः} %5-43-23

\twolineshloka
{ईदृग्विधैस्तु हरिभिर्वृतो दन्तनखायुधैः}
{शतैः शतसहस्रैश्च कोटिभिश्चायुतैरपि} %5-43-24

\threelineshloka
{आगमिष्यति सुग्रीवः सर्वेषां वो निषूदनः}
{नेयमस्ति पुरी लङ्का न यूयं न च रावणः}
{यस्य त्विक्ष्वाकुवीरेण बद्धं वैरं महात्मना} %5-43-25


॥इत्यार्षे श्रीमद्रामायणे वाल्मीकीये आदिकाव्ये सुन्दरकाण्डे चैत्यप्रासाददाहः नाम त्रिचत्वारिशः सर्गः ॥५-४३॥
