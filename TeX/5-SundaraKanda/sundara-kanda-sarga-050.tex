\sect{पञ्चाशः सर्गः — प्रहस्तप्रश्नः}

\twolineshloka
{तमुद्वीक्ष्य महाबाहुः पिङ्गाक्षं पुरतः स्थितम्}
{रोषेण महताऽऽविष्टो रावणो लोकरावणः} %5-50-1

\twolineshloka
{शङ्काहतात्मा दध्यौ स कपीन्द्रं तेजसा वृतम्}
{किमेष भगवान् नन्दी भवेत् साक्षादिहागतः} %5-50-2

\twolineshloka
{येन शप्तोऽस्मि कैलासे मया प्रहसिते पुरा}
{सोऽयं वानरमूर्तिः स्यात्किंस्विद् बाणोऽपि वासुरः} %5-50-3

\twolineshloka
{स राजा रोषताम्राक्षः प्रहस्तं मन्त्रिसत्तमम्}
{कालयुक्तमुवाचेदं वचो विपुलमर्थवत्} %5-50-4

\twolineshloka
{दुरात्मा पृच्छ्यतामेष कुतः किं वास्य कारणम्}
{वनभङ्गे च कोऽस्यार्थो राक्षसानां च तर्जने} %5-50-5

\twolineshloka
{मत्पुरीमप्रधृष्यां वै गमने किं प्रयोजनम्}
{आयोधने वा कं कार्यं पृच्छ्यतामेष दुर्मतिः} %5-50-6

\twolineshloka
{रावणस्य वचः श्रुत्वा प्रहस्तो वाक्यमब्रवीत्}
{समाश्वसिहि भद्रं ते न भीः कार्या त्वया कपे} %5-50-7

\twolineshloka
{यदि तावत् त्वमिन्द्रेण प्रेषितो रावणालयम्}
{तत्त्वमाख्याहि मा ते भूद् भयं वानर मोक्ष्यसे} %5-50-8

\twolineshloka
{यदि वैश्रवणस्य त्वं यमस्य वरुणस्य च}
{चारुरूपमिदं कृत्वा प्रविष्टो नः पुरीमिमाम्} %5-50-9

\twolineshloka
{विष्णुना प्रेषितो वापि दूतो विजयकाङ्क्षिणा}
{नहि ते वानरं तेजो रूपमात्रं तु वानरम्} %5-50-10

\twolineshloka
{तत्त्वतः कथयस्वाद्य ततो वानर मोक्ष्यसे}
{अनृतं वदतश्चापि दुर्लभं तव जीवितम्} %5-50-11

\twolineshloka
{अथवा यन्निमित्तस्ते प्रवेशो रावणालये}
{एवमुक्तो हरिवरस्तदा रक्षोगणेश्वरम्} %5-50-12

\twolineshloka
{अब्रवीन्नास्मि शक्रस्य यमस्य वरुणस्य च}
{धनदेन न मे सख्यं विष्णुना नास्मि चोदितः} %5-50-13

\twolineshloka
{जातिरेव मम त्वेषा वानरोऽहमिहागतः}
{दर्शने राक्षसेन्द्रस्य तदिदं दुर्लभं मया} %5-50-14

\twolineshloka
{वनं राक्षसराजस्य दर्शनार्थं विनाशितम्}
{ततस्ते राक्षसाः प्राप्ता बलिनो युद्धकाङ्क्षिणः} %5-50-15

\twolineshloka
{रक्षणार्थं च देहस्य प्रतियुद्धा मया रणे}
{अस्त्रपाशैर्न शक्योऽहं बद्धुं देवासुरैरपि} %5-50-16

\twolineshloka
{पितामहादेष वरो ममापि हि समागतः}
{राजानं द्रष्टुकामेन मयास्त्रमनुवर्तितम्} %5-50-17

\twolineshloka
{विमुक्तोऽप्यहमस्त्रेण राक्षसैस्त्वभिवेदितः}
{केनचिद् रामकार्येण आगतोऽस्मि तवान्तिकम्} %5-50-18

\twolineshloka
{दूतोऽहमिति विज्ञाय राघवस्यामितौजसः}
{श्रूयतामेव वचनं मम पथ्यमिदं प्रभो} %5-50-19


॥इत्यार्षे श्रीमद्रामायणे वाल्मीकीये आदिकाव्ये सुन्दरकाण्डे प्रहस्तप्रश्नः नाम पञ्चाशः सर्गः ॥५-५०॥
