\sect{द्वात्रिंशः सर्गः — सीतवितर्कः}

\twolineshloka
{ततः शाखान्तरे लीनं दृष्ट्वा चलितमानसा}
{वेष्टितार्जुनवस्त्रं तं विद्युत्संघातपिंगलम्} %5-32-1

\twolineshloka
{सा ददर्श कपिं तत्र प्रश्रितं प्रियवादिनम्}
{फुल्लाशोकोत्कराभासं तप्तचामीकरेक्षणम्} %5-32-2

\twolineshloka
{साथ दृष्ट्वा हरिश्रेष्ठं विनीतवदवस्थितम्}
{मैथिली चिन्तयामास विस्मयं परमं गता} %5-32-3

\twolineshloka
{अहो भीममिदं सत्त्वं वानरस्य दुरासदम्}
{दुर्निरीक्ष्यमिदं मत्वा पुनरेव मुमोह सा} %5-32-4

\twolineshloka
{विललाप भृशं सीता करुणं भयमोहिता}
{राम रामेति दुःखार्ता लक्ष्मणेति च भामिनी} %5-32-5

\threelineshloka
{रुरोद सहसा सीता मन्दमन्दस्वरा सती}
{साथ दृष्ट्वा हरिवरं विनीतवदुपागतम्}
{मैथिली चिन्तयामास स्वप्नोऽयमिति भामिनी} %5-32-6

\twolineshloka
{सा वीक्षमाणा पृथुभुग्नवक्त्रं शाखामृगेन्द्रस्य यथोक्तकारम्}
{ददर्श पिंगप्रवरं महार्हं वातात्मजं बुद्धिमतां वरिष्ठम्} %5-32-7

\twolineshloka
{सा तं समीक्ष्यैव भृशं विपन्ना गतासुकल्पेव बभूव सीता}
{चिरेण संज्ञां प्रतिलभ्य चैवं विचिन्तयामास विशालनेत्रा} %5-32-8

\twolineshloka
{स्वप्नो मयायं विकृतोऽद्य दृष्टः शाखामृगः शास्त्रगणैर्निषिद्धः}
{स्वस्त्यस्तु रामाय सलक्ष्मणाय तथा पितुर्मे जनकस्य राज्ञः} %5-32-9

\twolineshloka
{स्वप्नो हि नायं नहि मेऽस्ति निद्रा शोकेन दुःखेन च पीडितायाः}
{सुखं हि मे नास्ति यतो विहीना तेनेन्दुपूर्णप्रतिमाननेन} %5-32-10

\twolineshloka
{रामेति रामेति सदैव बुद्ध्या विचिन्त्य वाचा ब्रुवती तमेव}
{तस्यानुरूपं च कथां तदर्थामेवं प्रपश्यामि तथा शृणोमि} %5-32-11

\twolineshloka
{अहं हि तस्याद्य मनोभवेन सम्पीडिता तद्गतसर्वभावा}
{विचिन्तयन्ती सततं तमेव तथैव पश्यामि तथा शृणोमि} %5-32-12

\twolineshloka
{मनोरथः स्यादिति चिन्तयामि तथापि बुद्ध्यापि वितर्कयामि}
{किं कारणं तस्य हि नास्ति रूपं सुव्यक्तरूपश्च वदत्ययं माम्} %5-32-13

\twolineshloka
{नमोऽस्तु वाचस्पतये सवज्रिणे स्वयम्भुवे चैव हुताशनाय}
{अनेन चोक्तं यदिदं ममाग्रतो वनौकसा तच्च तथास्तु नान्यथा} %5-32-14


॥इत्यार्षे श्रीमद्रामायणे वाल्मीकीये आदिकाव्ये सुन्दरकाण्डे सीतवितर्कः नाम द्वात्रिंशः सर्गः ॥५-३२॥
