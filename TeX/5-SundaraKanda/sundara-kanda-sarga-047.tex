\sect{सप्तचत्वारिंशः सर्गः — अक्षकुमारवधः}

\twolineshloka
{सेनापतीन् पञ्च स तु प्रमापितान् हनूमता सानुचरान् सवाहनान्}
{निशम्य राजा समरोद्धतोन्मुखं कुमारमक्षं प्रसमैक्षताक्षम्} %5-47-1

\twolineshloka
{स तस्य दृष्ट्यर्पणसम्प्रचोदितः प्रतापवान् काञ्चनचित्रकार्मुकः}
{समुत्पपाताथ सदस्युदीरितो द्विजातिमुख्यैर्हविषेव पावकः} %5-47-2

\twolineshloka
{ततो महान् बालदिवाकरप्रभं प्रतप्तजाम्बूनदजालसंततम्}
{रथं समास्थाय ययौ स वीर्यवान् महाहरिं तं प्रति नैर्ऋतर्षभः} %5-47-3

\twolineshloka
{ततस्तपःसंग्रहसंचयार्जितं प्रतप्तजाम्बूनदजालचित्रितम्}
{पताकिनं रत्नविभूषितध्वजं मनोजवाष्टाश्ववरैः सुयोजितम्} %5-47-4

\twolineshloka
{सुरासुराधृष्यमसङ्गचारिणं तडित्प्रभं व्योमचरं समाहितम्}
{सतूणमष्टासिनिबद्धबन्धुरं यथाक्रमावेशितशक्तितोमरम्} %5-47-5

\twolineshloka
{विराजमानं प्रतिपूर्णवस्तुना सहेमदाम्ना शशिसूर्यवर्चसा}
{दिवाकराभं रथमास्थितस्ततः स निर्जगामामरतुल्यविक्रमः} %5-47-6

\twolineshloka
{स पूरयन् खं च महीं च साचलां तुरङ्गमातङ्गमहारथस्वनैः}
{बलैः समेतैः सहतोरणस्थितं समर्थमासीनमुपागमत् कपिम्} %5-47-7

\twolineshloka
{स तं समासाद्य हरिं हरीक्षणो युगान्तकालाग्निमिव प्रजाक्षये}
{अवस्थितं विस्मितजातसम्भ्रमं समैक्षताक्षो बहुमानचक्षुषा} %5-47-8

\twolineshloka
{स तस्य वेगं च कपेर्महात्मनः पराक्रमं चारिषु रावणात्मजः}
{विचारयन् स्वं च बलं महाबलो युगक्षये सूर्य इवाभिवर्धत} %5-47-9

\twolineshloka
{स जातमन्युः प्रसमीक्ष्य विक्रमं स्थितः स्थिरः संयति दुर्निवारणम्}
{समाहितात्मा हनुमन्तमाहवे प्रचोदयामास शितैः शरैस्त्रिभिः} %5-47-10

\twolineshloka
{ततः कपिं तं प्रसमीक्ष्य गर्वितं जितश्रमं शत्रुपराजयोचितम्}
{अवैक्षताक्षः समुदीर्णमानसं सबाणपाणिः प्रगृहीतकार्मुकः} %5-47-11

\twolineshloka
{स हेमनिष्काङ्गदचारुकुण्डलः समाससादाशुपराक्रमः कपिम्}
{तयोर्बभूवाप्रतिमः समागमः सुरासुराणामपि सम्भ्रमप्रदः} %5-47-12

\twolineshloka
{ररास भूमिर्न तताप भानुमान् ववौ न वायुः प्रचचाल चाचलः}
{कपेः कुमारस्य च वीर्यसंयुगं ननाद च द्यौरुदधिश्च चुक्षुभे} %5-47-13

\twolineshloka
{स तस्य वीरः सुमुखान् पतत्रिणः सुवर्णपुङ्खान् सविषानिवोरगान्}
{समाधिसंयोगविमोक्षतत्त्वविच्छरानथ त्रीन् कपिमूर्ध्न्यताडयत्} %5-47-14

\twolineshloka
{स तैः शरैर्मूर्ध्नि समं निपातितैः क्षरन्नसृग्दिग्धविवृत्तनेत्रः}
{नवोदितादित्यनिभः शरांशुमान् व्यराजतादित्य इवांशुमालिकः} %5-47-15

\twolineshloka
{ततः प्लवङ्गाधिपमन्त्रिसत्तमः समीक्ष्य तं राजवरात्मजं रणे}
{उदग्रचित्रायुधचित्रकार्मुकं जहर्ष चापूर्यत चाहवोन्मुखः} %5-47-16

\twolineshloka
{स मन्दराग्रस्थ इवांशुमाली विवृद्धकोपो बलवीर्यसंवृतः}
{कुमारमक्षं सबलं सवाहनं ददाह नेत्राग्निमरीचिभिस्तदा} %5-47-17

\twolineshloka
{ततः स बाणासनशक्रकार्मुकः शरप्रवर्षो युधि राक्षसाम्बुदः}
{शरान् मुमोचाशु हरीश्वराचले बलाहको वृष्टिमिवाचलोत्तमे} %5-47-18

\twolineshloka
{कपिस्ततस्तं रणचण्डविक्रमं प्रवृद्धतेजोबलवीर्यसायकम्}
{कुमारमक्षं प्रसमीक्ष्य संयुगे ननाद हर्षाद् घनतुल्यनिःस्वनः} %5-47-19

\twolineshloka
{स बालभावाद् युधि वीर्यदर्पितः प्रवृद्धमन्युः क्षतजोपमेक्षणः}
{समाससादाप्रतिमं रणे कपिं गजो महाकूपमिवावृतं तृणैः} %5-47-20

\twolineshloka
{स तेन बाणैः प्रसभं निपातितैश्चकार नादं घननादनिःस्वनः}
{समुत्सहेनाशु नभः समारुजन् भुजोरुविक्षेपणघोरदर्शनः} %5-47-21

\twolineshloka
{तमुत्पतन्तं समभिद्रवद् बली स राक्षसानां प्रवरः प्रतापवान्}
{रथी रथश्रेष्ठतरः किरन् शरैः पयोधरः शैलमिवाश्मवृष्टिभिः} %5-47-22

\twolineshloka
{स ताञ्छरांस्तस्य हरिर्विमोक्षयंश्चचार वीरः पथि वायुसेविते}
{शरान्तरे मारुतवद् विनिष्पतन् मनोजवः संयति भीमविक्रमः} %5-47-23

\twolineshloka
{तमात्तबाणासनमाहवोन्मुखं खमास्तृणन्तं विविधैः शरोत्तमैः}
{अवैक्षताक्षं बहुमानचक्षुषा जगाम चिन्तां स च मारुतात्मजः} %5-47-24

\twolineshloka
{ततः शरैर्भिन्नभुजान्तरः कपिः कुमारवर्येण महात्मना नदन्}
{महाभुजः कर्मविशेषतत्त्वविद् विचिन्तयामास रणे पराक्रमम्} %5-47-25

\twolineshloka
{अबालवद् बालदिवाकरप्रभः करोत्ययं कर्म महन्महाबलः}
{न चास्य सर्वाहवकर्मशालिनः प्रमापणे मे मतिरत्र जायते} %5-47-26

\twolineshloka
{अयं महात्मा च महांश्च वीर्यतः समाहितश्चातिसहश्च संयुगे}
{असंशयं कर्मगुणोदयादयं सनागयक्षैर्मुनिभिश्च पूजितः} %5-47-27

\twolineshloka
{पराक्रमोत्साहविवृद्धमानसःसमीक्षते मां प्रमुखोऽग्रतः स्थितः}
{पराक्रमो ह्यस्य मनांसि कम्पयेत् सुरासुराणामपि शीघ्रकारिणः} %5-47-28

\twolineshloka
{न खल्वयं नाभिभवेदुपेक्षितः पराक्रमो ह्यस्य रणे विवर्धते}
{प्रमापणं ह्यस्य ममाद्य रोचते न वर्धमानोऽग्निरुपेक्षितुं क्षमः} %5-47-29

\twolineshloka
{इति प्रवेगं तु परस्य तर्कयन् स्वकर्मयोगं च विधाय वीर्यवान्}
{चकार वेगं तु महाबलस्तदा मतिं च चक्रेऽस्य वधे तदानीम्} %5-47-30

\twolineshloka
{स तस्य तानष्ट वरान् महाहयान् समाहितान् भारसहान् विवर्तने}
{जघान वीरः पथि वायुसेविते तलप्रहारैः पवनात्मजः कपिः} %5-47-31

\twolineshloka
{ततस्तलेनाभिहतो महारथः स तस्य पिङ्गाधिपमन्त्रिनिर्जितः}
{स भग्ननीडः परिवृत्तकूबरः पपात भूमौ हतवाजिरम्बरात्} %5-47-32

\twolineshloka
{स तं परित्यज्य महारथो रथं सकार्मुकः खड्गधरः खमुत्पतन्}
{ततोऽभियोगादृषिरुग्रवीर्यवान् विहाय देहं मरुतामिवालयम्} %5-47-33

\twolineshloka
{कपिस्ततस्तं विचरन्तमम्बरे पतत्त्रिराजानिलसिद्धसेविते}
{समेत्य तं मारुतवेगविक्रमः क्रमेण जग्राह च पादयोर्दृढम्} %5-47-34

\twolineshloka
{स तं समाविध्य सहस्रशः कपिर्महोरगं गृह्य इवाण्डजेश्वरः}
{मुमोच वेगात् पितृतुल्यविक्रमो महीतले संयति वानरोत्तमः} %5-47-35

\twolineshloka
{स भग्नबाहूरुकटीपयोधरः क्षरन्नसृङ्निर्मथितास्थिलोचनः}
{सम्भिन्नसंधिः प्रविकीर्णबन्धनो हतः क्षितौ वायुसुतेन राक्षसः} %5-47-36

\threelineshloka
{महाकपिर्भूमितले निपीड्य तं चकार रक्षोऽधिपतेर्महद्भयम्}
{महर्षिभिश्चक्रचरैः समागतैः समेत्य भूतैश्च सयक्षपन्नगैः}
{सुरैश्च सेन्द्रैर्भृशजातविस्मयैर्हते कुमारे स कपिर्निरीक्षितः} %5-47-37

\twolineshloka
{निहत्य तं वज्रिसुतोपमं रणे कुमारमक्षं क्षतजोपमेक्षणम्}
{तदेव वीरोऽभिजगाम तोरणं कृतक्षणः काल इव प्रजाक्षये} %5-47-38


॥इत्यार्षे श्रीमद्रामायणे वाल्मीकीये आदिकाव्ये सुन्दरकाण्डे अक्षकुमारवधः नाम सप्तचत्वारिंशः सर्गः ॥५-४७॥
