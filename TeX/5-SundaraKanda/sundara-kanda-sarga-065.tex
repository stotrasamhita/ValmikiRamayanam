\sect{पञ्चषष्ठितमः सर्गः — चूडामणिप्रदानम्}

\twolineshloka
{ततः प्रस्रवणं शैलं ते गत्वा चित्रकाननम्}
{प्रणम्य शिरसा रामं लक्ष्मणं च महाबलम्} %5-65-1

\twolineshloka
{युवराजं पुरस्कृत्य सुग्रीवमभिवाद्य च}
{प्रवृत्तिमथ सीतायाः प्रवक्तुमुपचक्रमुः} %5-65-2

\twolineshloka
{रावणान्तःपुरे रोधं राक्षसीभिश्च तर्जनम्}
{रामे समनुरागं च यथा च नियमः कृतः} %5-65-3

\twolineshloka
{एतदाख्याय ते सर्वं हरयो रामसन्निधौ}
{वैदेहीमक्षतां श्रुत्वा रामस्तूत्तरमब्रवीत्} %5-65-4

\twolineshloka
{क्व सीता वर्तते देवी कथं च मयि वर्तते}
{एतन्मे सर्वमाख्यात वैदेहीं प्रति वानराः} %5-65-5

\twolineshloka
{रामस्य गदितं श्रुत्वा हरयो रामसन्निधौ}
{चोदयन्ति हनूमन्तं सीतावृत्तान्तकोविदम्} %5-65-6

\twolineshloka
{श्रुत्वा तु वचनं तेषां हनूमान् मारुतात्मजः}
{प्रणम्य शिरसा देव्यै सीतायै तां दिशं प्रति} %5-65-7

\twolineshloka
{उवाच वाक्यं वाक्यज्ञः सीताया दर्शनं यथा}
{तं मणिं काञ्चनं दिव्यं दीप्यमानं स्वतेजसा} %5-65-8

\twolineshloka
{दत्त्वा रामाय हनुमांस्ततः प्राञ्जलिरब्रवीत्}
{समुद्रं लङ्घयित्वाहं शतयोजनमायतम्} %5-65-9

\twolineshloka
{अगच्छं जानकीं सीतां मार्गमाणो दिदृक्षया}
{तत्र लङ्केति नगरी रावणस्य दुरात्मनः} %5-65-10

\twolineshloka
{दक्षिणस्य समुद्रस्य तीरे वसति दक्षिणे}
{तत्र सीता मया दृष्टा रावणान्तःपुरे सती} %5-65-11

\twolineshloka
{त्वयि सन्न्यस्य जीवन्ती रामा राम मनोरथम्}
{दृष्टा मे राक्षसीमध्ये तर्ज्यमाना मुहुर्मुहुः} %5-65-12

\twolineshloka
{राक्षसीभिर्विरूपाभी रक्षिता प्रमदावने}
{दुःखमापद्यते देवी त्वया वीर सुखोचिता} %5-65-13

\twolineshloka
{रावणान्तःपुरे रुद्धा राक्षसीभिः सुरक्षिता}
{एकवेणीधरा दीना त्वयि चिन्तापरायणा} %5-65-14

\twolineshloka
{अधःशय्या विवर्णाङ्गी पद्मिनीव हिमागमे}
{रावणाद् विनिवृत्तार्था मर्तव्यकृतनिश्चया} %5-65-15

\twolineshloka
{देवी कथञ्चित् काकुत्स्थ त्वन्मना मार्गिता मया}
{इक्ष्वाकुवंशविख्यातिं शनैः कीर्तयतानघ} %5-65-16

\twolineshloka
{सा मया नरशार्दूल शनैर्विश्वासिता तदा}
{ततः सम्भाषिता देवी सर्वमर्थं च दर्शिता} %5-65-17

\twolineshloka
{रामसुग्रीवसख्यं च श्रुत्वा हर्षमुपागता}
{नियतः समुदाचारो भक्तिश्चास्याः सदा त्वयि} %5-65-18

\twolineshloka
{एवं मया महाभाग दृष्टा जनकनन्दिनी}
{उग्रेण तपसा युक्ता त्वद्भक्त्या पुरुषर्षभ} %5-65-19

\twolineshloka
{अभिज्ञानं च मे दत्तं यथावृत्तं तवान्तिके}
{चित्रकूटे महाप्राज्ञ वायसं प्रति राघव} %5-65-20

\twolineshloka
{विज्ञाप्यः पुनरप्येष रामो वायुसुत त्वया}
{अखिलेन यथा दृष्टमिति मामाह जानकी} %5-65-21

\twolineshloka
{अयं चास्मै प्रदातव्यो यत्नात् सुपरिरक्षितः}
{ब्रुवता वचनान्येवं सुग्रीवस्योपशृण्वतः} %5-65-22

\twolineshloka
{एष चूडामणिः श्रीमान् मया ते यत्नरक्षितः}
{मनःशिलायास्तिलकं तत् स्मरस्वेति चाब्रवीत्} %5-65-23

\twolineshloka
{एष निर्यातितः श्रीमान् मया ते वारिसम्भवः}
{एनं दृष्ट्वा प्रमोदिष्ये व्यसने त्वामिवानघ} %5-65-24

\twolineshloka
{जीवितं धारयिष्यामि मासं दशरथात्मज}
{ऊर्ध्वं मासान्न जीवेयं रक्षसां वशमागता} %5-65-25

\twolineshloka
{इति मामब्रवीत् सीता कृशाङ्गी धर्मचारिणी}
{रावणान्तःपुरे रुद्धा मृगीवोत्फुल्ललोचना} %5-65-26

\twolineshloka
{एतदेव मयाऽऽख्यातं सर्वं राघव यद् यथा}
{सर्वथा सागरजले सन्तारः प्रविधीयताम्} %5-65-27

\twolineshloka
{तौ जाताश्वासौ राजपुत्रौ विदित्वा तच्चाभिज्ञानं राघवाय प्रदाय}
{देव्या चाख्यातं सर्वमेवानुपूर्व्याद् वाचा सम्पूर्णं वायुपुत्रः शशंस} %5-65-28


॥इत्यार्षे श्रीमद्रामायणे वाल्मीकीये आदिकाव्ये सुन्दरकाण्डे चूडामणिप्रदानम् नाम पञ्चषष्ठितमः सर्गः ॥५-६५॥
