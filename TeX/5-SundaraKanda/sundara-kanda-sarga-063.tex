\sect{त्रिषष्ठितमः सर्गः — सुग्रीवहर्षः}

\twolineshloka
{ततो मूर्ध्ना निपतितं वानरं वानरर्षभः}
{दृष्ट्वैवोद्विग्नहृदयो वाक्यमेतदुवाच ह} %5-63-1

\twolineshloka
{उत्तिष्ठोत्तिष्ठ कस्मात् त्वं पादयोः पतितो मम}
{अभयं ते प्रदास्यामि सत्यमेवाभिधीयताम्} %5-63-2

\twolineshloka
{किं सम्भ्रमाद्धितं कृत्स्नं ब्रूहि यद् वक्तुमर्हसि}
{कच्चिन्मधुवने स्वस्ति श्रोतुमिच्छामि वानर} %5-63-3

\twolineshloka
{स समाश्वासितस्तेन सुग्रीवेण महात्मना}
{उत्थाय स महाप्राज्ञो वाक्यं दधिमुखोऽब्रवीत्} %5-63-4

\twolineshloka
{नैवर्क्षरजसा राजन् न त्वया न च वालिना}
{वनं निसृष्टपूर्वं ते नाशितं तत्तु वानरैः} %5-63-5

\twolineshloka
{न्यवारयमहं सर्वान् सहैभिर्वनचारिभिः}
{अचिन्तयित्वा मां हृष्टा भक्षयन्ति पिबन्ति च} %5-63-6

\twolineshloka
{एभिः प्रधर्षणायां च वारितं वनपालकैः}
{मामप्यचिन्तयन् देव भक्षयन्ति वनौकसः} %5-63-7

\twolineshloka
{शिष्टमत्रापविध्यन्ति भक्षयन्ति तथापरे}
{निवार्यमाणास्ते सर्वे भ्रुकुटिं दर्शयन्ति हि} %5-63-8

\twolineshloka
{इमे हि संरब्धतरास्तदा तैः सम्प्रधर्षिताः}
{निवार्यन्ते वनात् तस्मात् क्रुद्धैर्वानरपुङ्गवैः} %5-63-9

\twolineshloka
{ततस्तैर्बहुभिर्वीरैर्वानरैर्वानरर्षभाः}
{संरक्तनयनैः क्रोधाद्धरयः सम्प्रधर्षिताः} %5-63-10

\twolineshloka
{पाणिभिर्निहताः केचित् केचिज्जानुभिराहताः}
{प्रकृष्टाश्च तदा कामं देवमार्गं च दर्शिताः} %5-63-11

\twolineshloka
{एवमेते हताः शूरास्त्वयि तिष्ठति भर्तरि}
{कृत्स्नं मधुवनं चैव प्रकामं तैश्च भक्ष्यते} %5-63-12

\twolineshloka
{एवं विज्ञाप्यमानं तं सुग्रीवं वानरर्षभम्}
{अपृच्छत् तं महाप्राज्ञो लक्ष्मणः परवीरहा} %5-63-13

\twolineshloka
{किमयं वानरो राजन् वनपः प्रत्युपस्थितः}
{किं चार्थमभिनिर्दिश्य दुःखितो वाक्यमब्रवीत्} %5-63-14

\twolineshloka
{एवमुक्तस्तु सुग्रीवो लक्ष्मणेन महात्मना}
{लक्ष्मणं प्रत्युवाचेदं वाक्यं वाक्यविशारदः} %5-63-15

\twolineshloka
{आर्य लक्ष्मण सम्प्राह वीरो दधिमुखः कपिः}
{अङ्गदप्रमुखैर्वीरैर्भक्षितं मधु वानरैः} %5-63-16

\twolineshloka
{नैषामकृतकार्याणामीदृशः स्याद् व्यतिक्रमः}
{वनं यदभिपन्नास्ते साधितं कर्म तद् ध्रुवम्} %5-63-17

\twolineshloka
{वारयन्तो भृशं प्राप्ताः पाला जानुभिराहताः}
{तथा न गणितश्चायं कपिर्दधिमुखो बली} %5-63-18

\twolineshloka
{पतिर्मम वनस्यायमस्माभिः स्थापितः स्वयम्}
{दृष्टा देवी न सन्देहो न चान्येन हनूमता} %5-63-19

\twolineshloka
{न ह्यन्यः साधने हेतुः कर्मणोऽस्य हनूमतः}
{कार्यसिद्धिर्हनुमति मतिश्च हरिपुङ्गवे} %5-63-20

\twolineshloka
{व्यवसायश्च वीर्यं च श्रुतं चापि प्रतिष्ठितम्}
{जाम्बवान् यत्र नेता स्यादङ्गदश्च महाबलः} %5-63-21

\twolineshloka
{हनूमांश्चाप्यधिष्ठाता न तत्र गतिरन्यथा}
{अङ्गदप्रमुखैर्वीरैर्हतं मधुवनं किल} %5-63-22

\twolineshloka
{विचित्य दक्षिणामाशामागतैर्हरिपुङ्गवैः}
{आगतैश्चाप्रधृष्यं तद्धतं मधुवनं हि तैः} %5-63-23

\twolineshloka
{धर्षितं च वनं कृत्स्नमुपयुक्तं तु वानरैः}
{पातिता वनपालास्ते तदा जानुभिराहताः} %5-63-24

\twolineshloka
{एतदर्थमयं प्राप्तो वक्तुं मधुरवागिह}
{नाम्ना दधिमुखो नाम हरिः प्रख्यातविक्रमः} %5-63-25

\twolineshloka
{दृष्टा सीता महाबाहो सौमित्रे पश्य तत्त्वतः}
{अभिगम्य यथा सर्वे पिबन्ति मधु वानराः} %5-63-26

\twolineshloka
{न चाप्यदृष्ट्वा वैदेहीं विश्रुताः पुरुषर्षभ}
{वनं दत्तवरं दिव्यं धर्षयेयुर्वनौकसः} %5-63-27

\twolineshloka
{ततः प्रहृष्टो धर्मात्मा लक्ष्मणः सहराघवः}
{श्रुत्वा कर्णसुखां वाणीं सुग्रीववदनाच्च्युताम्} %5-63-28

\twolineshloka
{प्राहृष्यत भृशं रामो लक्ष्मणश्च महायशाः}
{श्रुत्वा दधिमुखस्यैवं सुग्रीवस्तु प्रहृष्य च} %5-63-29

\twolineshloka
{वनपालं पुनर्वाक्यं सुग्रीवः प्रत्यभाषत}
{प्रीतोऽस्मि सोऽहं यद्भुक्तं वनं तैः कृतकर्मभिः} %5-63-30

\threelineshloka
{धर्षितं मर्षणीयं च चेष्टितं कृतकर्मणाम्}
{गच्छ शीघ्रं मधुवनं संरक्षस्व त्वमेव हि}
{शीघ्रं प्रेषय सर्वांस्तान् हनूमत्प्रमुखान् कपीन्} %5-63-31

\twolineshloka
{इच्छामि शीघ्रं हनुमत्प्रधानान्शाखामृगांस्तान् मृगराजदर्पान्}
{प्रष्टुं कृतार्थान् सह राघवाभ्यां श्रोतुं च सीताधिगमे प्रयत्नम्} %5-63-32

\twolineshloka
{प्रीतिस्फीताक्षौ सम्प्रहृष्टौ कुमारौ दृष्ट्वा सिद्धार्थौ वानराणां च राजा}
{अङ्गैः प्रहृष्टैः कार्यसिद्धिं विदित्वा बाह्वोरासन्नामतिमात्रं ननन्द} %5-63-33


॥इत्यार्षे श्रीमद्रामायणे वाल्मीकीये आदिकाव्ये सुन्दरकाण्डे सुग्रीवहर्षः नाम त्रिषष्ठितमः सर्गः ॥५-६३॥
