\sect{अष्टात्रिंशः सर्गः — वायसवृत्तान्तकथनम्}

\twolineshloka
{ततः स कपिशार्दूलस्तेन वाक्येन तोषितः}
{सीतामुवाच तच्छ्रुत्वा वाक्यं वाक्यविशारदः} %5-38-1

\twolineshloka
{युक्तरूपं त्वया देवि भाषितं शुभदर्शने}
{सदृशं स्त्रीस्वभावस्य साध्वीनां विनयस्य च} %5-38-2

\twolineshloka
{स्त्रीत्वान्न त्वं समर्थासि सागरं व्यतिवर्तितुम्}
{मामधिष्ठाय विस्तीर्णं शतयोजनमायतम्} %5-38-3

\twolineshloka
{द्वितीयं कारणं यच्च ब्रवीषि विनयान्विते}
{रामादन्यस्य नार्हामि संसर्गमिति जानकि} %5-38-4

\twolineshloka
{एतत् ते देवि सदृशं पत्न्यास्तस्य महात्मनः}
{का ह्यन्या त्वामृते देवि ब्रूयाद् वचनमीदृशम्} %5-38-5

\twolineshloka
{श्रोष्यते चैव काकुत्स्थः सर्वं निरवशेषतः}
{चेष्टितं यत् त्वया देवि भाषितं च ममाग्रतः} %5-38-6

\twolineshloka
{कारणैर्बहुभिर्देवि रामप्रियचिकीर्षया}
{स्नेहप्रस्कन्नमनसा मयैतत् समुदीरितम्} %5-38-7

\twolineshloka
{लङ्काया दुष्प्रवेशत्वाद् दुस्तरत्वान्महोदधेः}
{सामर्थ्यादात्मनश्चैव मयैतत् समुदीरितम्} %5-38-8

\twolineshloka
{इच्छामि त्वां समानेतुमद्यैव रघुनन्दिना}
{गुरुस्नेहेन भक्त्या च नान्यथा तदुदाहृतम्} %5-38-9

\twolineshloka
{यदि नोत्सहसे यातुं मया सार्धमनिन्दिते}
{अभिज्ञानं प्रयच्छ त्वं जानीयाद् राघवो हि यत्} %5-38-10

\twolineshloka
{एवमुक्ता हनुमता सीता सुरसुतोपमा}
{उवाच वचनं मन्दं बाष्पप्रग्रथिताक्षरम्} %5-38-11

\twolineshloka
{इदं श्रेष्ठमभिज्ञानं ब्रूयास्त्वं तु मम प्रियम्}
{शैलस्य चित्रकूटस्य पादे पूर्वोत्तरे पदे} %5-38-12

\twolineshloka
{तापसाश्रमवासिन्याः प्राज्यमूलफलोदके}
{तस्मिन् सिद्धाश्रिते देशे मन्दाकिन्यविदूरतः} %5-38-13

\twolineshloka
{तस्योपवनखण्डेषु नानापुष्पसुगन्धिषु}
{विहृत्य सलिले क्लिन्नो ममाङ्के समुपाविशः} %5-38-14

\twolineshloka
{ततो मांससमायुक्तो वायसः पर्यतुण्डयत्}
{तमहं लोष्टमुद्यम्य वारयामि स्म वायसम्} %5-38-15

\twolineshloka
{दारयन् स च मां काकस्तत्रैव परिलीयते}
{न चाप्युपारमन्मांसाद् भक्षार्थी बलिभोजनः} %5-38-16

\twolineshloka
{उत्कर्षन्त्यां च रशनां क्रुद्धायां मयि पक्षिणे}
{स्रंसमाने च वसने ततो दृष्टा त्वया ह्यहम्} %5-38-17

\twolineshloka
{त्वया विहसिता चाहं क्रुद्धा संलज्जिता तदा}
{भक्ष्यगृद्धेन काकेन दारिता त्वामुपागता} %5-38-18

\twolineshloka
{ततः श्रान्ताहमुत्सङ्गमासीनस्य तवाविशम्}
{क्रुध्यन्तीव प्रहृष्टेन त्वयाहं परिसान्त्विता} %5-38-19

\twolineshloka
{बाष्पपूर्णमुखी मन्दं चक्षुषी परिमार्जती}
{लक्षिताहं त्वया नाथ वायसेन प्रकोपिता} %5-38-20

\twolineshloka
{परिश्रमाच्च सुप्ता हे राघवाङ्केऽस्म्यहं चिरम्}
{पर्यायेण प्रसुप्तश्च ममाङ्के भरताग्रजः} %5-38-21

\threelineshloka
{स तत्र पुनरेवाथ वायसः समुपागमत्}
{ततः सुप्तप्रबुद्धां मां राघवाङ्कात् समुत्थिताम्}
{वायसः सहसागम्य विददार स्तनान्तरे} %5-38-22

\twolineshloka
{पुनः पुनरथोत्पत्य विददार स मां भृशम्}
{ततः समुत्थितो रामो मुक्तैः शोणितबिन्दुभिः} %5-38-23

\twolineshloka
{स मां दृष्ट्वा महाबाहुर्वितुन्नां स्तनयोस्तदा}
{आशीविष इव क्रुद्धः श्वसन् वाक्यमभाषत} %5-38-24

\twolineshloka
{केन ते नागनासोरु विक्षतं वै स्तनान्तरम्}
{कः क्रीडति सरोषेण पञ्चवक्त्रेण भोगिना} %5-38-25

\twolineshloka
{वीक्षमाणस्ततस्तं वै वायसं समवैक्षत}
{नखैः सरुधिरैस्तीक्ष्णैर्मामेवाभिमुखं स्थितम्} %5-38-26

\twolineshloka
{पुत्रः किल स शक्रस्य वायसः पततां वरः}
{धरान्तरं गतः शीघ्रं पवनस्य गतौ समः} %5-38-27

\twolineshloka
{ततस्तस्मिन् महाबाहुः कोपसंवर्तितेक्षणः}
{वायसे कृतवान् क्रूरां मतिं मतिमतां वरः} %5-38-28

\twolineshloka
{स दर्भसंस्तराद् गृह्य ब्रह्मणोऽस्त्रेण योजयत्}
{स दीप्त इव कालाग्निर्जज्वालाभिमुखो द्विजम्} %5-38-29

\twolineshloka
{स तं प्रदीप्तं चिक्षेप दर्भं तं वायसं प्रति}
{ततस्तु वायसं दर्भः सोऽम्बरेऽनुजगाम ह} %5-38-30

\twolineshloka
{अनुसृष्टस्तदा काको जगाम विविधां गतिम्}
{त्राणकाम इमं लोकं सर्वं वै विचचार ह} %5-38-31

\twolineshloka
{स पित्रा च परित्यक्तः सर्वैश्च परमर्षिभिः}
{त्रीँल्लोकान् सम्परिक्रम्य तमेव शरणं गतः} %5-38-32

\twolineshloka
{स तं निपतितं भूमौ शरण्यः शरणागतम्}
{वधार्हमपि काकुत्स्थः कृपया पर्यपालयत्} %5-38-33

\twolineshloka
{परिद्यूनं विवर्णं च पतमानं तमब्रवीत्}
{मोघमस्त्रं न शक्यं तु ब्राह्मं कर्तुं तदुच्यताम्} %5-38-34

\twolineshloka
{ततस्तस्याक्षि काकस्य हिनस्ति स्म स दक्षिणम्}
{दत्त्वा तु दक्षिणं नेत्रं प्राणेभ्यः परिरक्षितः} %5-38-35

\twolineshloka
{स रामाय नमस्कृत्वा राज्ञे दशरथाय च}
{विसृष्टस्तेन वीरेण प्रतिपेदे स्वमालयम्} %5-38-36

\twolineshloka
{मत्कृते काकमात्रेऽपि ब्रह्मास्त्रं समुदीरितम्}
{कस्माद् यो माहरत् त्वत्तः क्षमसे तं महीपते} %5-38-37

\twolineshloka
{स कुरुष्व महोत्साहां कृपां मयि नरर्षभ}
{त्वया नाथवती नाथ ह्यनाथा इव दृश्यते} %5-38-38

\twolineshloka
{आनृशंस्यं परो धर्मस्त्वत्त एव मया श्रुतम्}
{जानामि त्वां महावीर्यं महोत्साहं महाबलम्} %5-38-39

\twolineshloka
{अपारवारमक्षोभ्यं गाम्भीर्यात् सागरोपमम्}
{भर्तारं ससमुद्राया धरण्या वासवोपमम्} %5-38-40

\twolineshloka
{एवमस्त्रविदां श्रेष्ठो बलवान् सत्त्ववानपि}
{किमर्थमस्त्रं रक्षःसु न योजयसि राघव} %5-38-41

\twolineshloka
{न नागा नापि गन्धर्वा न सुरा न मरुद्गणाः}
{रामस्य समरे वेगं शक्ताः प्रतिसमीहितुम्} %5-38-42

\twolineshloka
{तस्य वीर्यवतः कच्चिद् यद्यस्ति मयि सम्भ्रमः}
{किमर्थं न शरैस्तीक्ष्णैः क्षयं नयति राक्षसान्} %5-38-43

\twolineshloka
{भ्रातुरादेशमादाय लक्ष्मणो वा परंतपः}
{कस्य हेतोर्न मां वीरः परित्राति महाबलः} %5-38-44

\twolineshloka
{यदि तौ पुरुषव्याघ्रौ वाय्विन्द्रसमतेजसौ}
{सुराणामपि दुर्धर्षौ किमर्थं मामुपेक्षतः} %5-38-45

\twolineshloka
{ममैव दुष्कृतं किंचिन्महदस्ति न संशयः}
{समर्थावपि तौ यन्मां नावेक्षेते परंतपौ} %5-38-46

\twolineshloka
{वैदेह्या वचनं श्रुत्वा करुणं साश्रु भाषितम्}
{अथाब्रवीन्महातेजा हनूमान् हरियूथपः} %5-38-47

\twolineshloka
{त्वच्छोकविमुखो रामो देवि सत्येन ते शपे}
{रामे दुःखाभिपन्ने तु लक्ष्मणः परितप्यते} %5-38-48

\twolineshloka
{कथंचिद् भवती दृष्टा न कालः परिशोचितुम्}
{इमं मुहूर्तं दुःखानामन्तं द्रक्ष्यसि शोभने} %5-38-49

\twolineshloka
{तावुभौ पुरुषव्याघ्रौ राजपुत्रौ महाबलौ}
{त्वद्दर्शनकृतोत्साहौ लोकान् भस्मीकरिष्यतः} %5-38-50

\twolineshloka
{हत्वा च समरक्रूरं रावणं सहबान्धवम्}
{राघवस्त्वां विशालाक्षि स्वां पुरीं प्रति नेष्यति} %5-38-51

\twolineshloka
{ब्रूहि यद् राघवो वाच्यो लक्ष्मणश्च महाबलः}
{सुग्रीवो वापि तेजस्वी हरयो वा समागताः} %5-38-52

\twolineshloka
{इत्युक्तवति तस्मिंश्च सीता पुनरथाब्रवीत्}
{कौसल्या लोकभर्तारं सुषुवे यं मनस्विनी} %5-38-53

\twolineshloka
{तं ममार्थे सुखं पृच्छ शिरसा चाभिवादय}
{स्रजश्च सर्वरत्नानि प्रियायाश्च वराङ्गनाः} %5-38-54

\twolineshloka
{ऐश्वर्यं च विशालायां पृथिव्यामपि दुर्लभम्}
{पितरं मातरं चैव सम्मान्याभिप्रसाद्य च} %5-38-55

\twolineshloka
{अनुप्रव्रजितो रामं सुमित्रा येन सुप्रजाः}
{आनुकूल्येन धर्मात्मा त्यक्त्वा सुखमनुत्तमम्} %5-38-56

\twolineshloka
{अनुगच्छति काकुत्स्थं भ्रातरं पालयन् वने}
{सिंहस्कन्धो महाबाहुर्मनस्वी प्रियदर्शनः} %5-38-57

\twolineshloka
{पितृवद् वर्तते रामे मातृवन्मां समाचरत्}
{ह्रियमाणां तदा वीरो न तु मां वेद लक्ष्मणः} %5-38-58

\twolineshloka
{वृद्धोपसेवी लक्ष्मीवान् शक्तो न बहुभाषिता}
{राजपुत्रप्रियश्रेष्ठः सदृशः श्वशुरस्य मे} %5-38-59

\twolineshloka
{मत्तः प्रियतरो नित्यं भ्राता रामस्य लक्ष्मणः}
{नियुक्तो धुरि यस्यां तु तामुद्वहति वीर्यवान्} %5-38-60

\twolineshloka
{यं दृष्ट्वा राघवो नैव वृत्तमार्यमनुस्मरत्}
{स ममार्थाय कुशलं वक्तव्यो वचनान्मम} %5-38-61

\twolineshloka
{मृदुर्नित्यं शुचिर्दक्षः प्रियो रामस्य लक्ष्मणः}
{यथा हि वानरश्रेष्ठ दुःखक्षयकरो भवेत्} %5-38-62

\twolineshloka
{त्वमस्मिन् कार्यनिर्वाहे प्रमाणं हरियूथप}
{राघवस्त्वत्समारम्भान्मयि यत्नपरो भवेत्} %5-38-63

\twolineshloka
{इदं ब्रूयाश्च मे नाथं शूरं रामं पुनः पुनः}
{जीवितं धारयिष्यामि मासं दशरथात्मज} %5-38-64

\threelineshloka
{ऊर्ध्वं मासान्न जीवेयं सत्येनाहं ब्रवीमि ते}
{रावणेनोपरुद्धां मां निकृत्या पापकर्मणा}
{त्रातुमर्हसि वीर त्वं पातालादिव कौशिकीम्} %5-38-65

\twolineshloka
{ततो वस्त्रगतं मुक्त्वा दिव्यं चूडामणिं शुभम्}
{प्रदेयो राघवायेति सीता हनुमते ददौ} %5-38-66

\twolineshloka
{प्रतिगृह्य ततो वीरो मणिरत्नमनुत्तमम्}
{अङ्गुल्या योजयामास नह्यस्य प्राभवद् भुजः} %5-38-67

\twolineshloka
{मणिरत्नं कपिवरः प्रतिगृह्याभिवाद्य च}
{सीतां प्रदक्षिणं कृत्वा प्रणतः पार्श्वतः स्थितः} %5-38-68

\twolineshloka
{हर्षेण महता युक्तः सीतादर्शनजेन सः}
{हृदयेन गतो रामं लक्ष्मणं च सलक्षणम्} %5-38-69

\twolineshloka
{मणिवरमुपगृह्य तं महार्हं जनकनृपात्मजया धृतं प्रभावात्}
{गिरिवरपवनावधूतमुक्तः सुखितमनाः प्रतिसंक्रमं प्रपेदे} %5-38-70


॥इत्यार्षे श्रीमद्रामायणे वाल्मीकीये आदिकाव्ये सुन्दरकाण्डे वायसवृत्तान्तकथनम् नाम अष्टात्रिंशः सर्गः ॥५-३८॥
