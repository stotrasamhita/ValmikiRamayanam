\sect{सप्तविंशः सर्गः — त्रिजटास्वप्नः}

\twolineshloka
{इत्युक्ताः सीतया घोरं राक्षस्यः क्रोधमूर्च्छिताः}
{काश्चिज्जग्मुस्तदाख्यातुं रावणस्य दुरात्मनः} %5-27-1

\twolineshloka
{ततः सीतामुपागम्य राक्षस्यो भीमदर्शनाः}
{पुनः परुषमेकार्थमनर्थार्थमथाब्रुवन्} %5-27-2

\twolineshloka
{अद्येदानीं तवानार्ये सीते पापविनिश्चये}
{राक्षस्यो भक्षयिष्यन्ति मांसमेतद् यथासुखम्} %5-27-3

\twolineshloka
{सीतां ताभिरनार्याभिर्दृष्ट्वा संतर्जितां तदा}
{राक्षसी त्रिजटा वृद्धा प्रबुद्धा वाक्यमब्रवीत्} %5-27-4

\twolineshloka
{आत्मानं खादतानार्या न सीतां भक्षयिष्यथ}
{जनकस्य सुतामिष्टां स्नुषां दशरथस्य च} %5-27-5

\twolineshloka
{स्वप्नो ह्यद्य मया दृष्टो दारुणो रोमहर्षणः}
{राक्षसानामभावाय भर्तुरस्या भवाय च} %5-27-6

\twolineshloka
{एवमुक्तास्त्रिजटया राक्षस्यः क्रोधमूर्च्छिताः}
{सर्वा एवाब्रुवन् भीतास्त्रिजटां तामिदं वचः} %5-27-7

\twolineshloka
{कथयस्व त्वया दृष्टः स्वप्नोऽयं कीदृशो निशि}
{तासां श्रुत्वा तु वचनं राक्षसीनां मुखोद्गतम्} %5-27-8

\twolineshloka
{उवाच वचनं काले त्रिजटा स्वप्नसंश्रितम्}
{गजदन्तमयीं दिव्यां शिबिकामन्तरिक्षगाम्} %5-27-9

\twolineshloka
{युक्तां वाजिसहस्रेण स्वयमास्थाय राघवः}
{शुक्लमाल्याम्बरधरो लक्ष्मणेन समागतः} %5-27-10

\twolineshloka
{स्वप्ने चाद्य मया दृष्टा सीता शुक्लाम्बरावृता}
{सागरेण परिक्षिप्तं श्वेतपर्वतमास्थिता} %5-27-11

\twolineshloka
{रामेण संगता सीता भास्करेण प्रभा यथा}
{राघवश्च पुनर्दृष्टश्चतुर्दन्तं महागजम्} %5-27-12

\twolineshloka
{आरूढः शैलसंकाशं चकास सहलक्ष्मणः}
{ततस्तु सूर्यसंकाशौ दीप्यमानौ स्वतेजसा} %5-27-13

\twolineshloka
{शुक्लमाल्याम्बरधरौ जानकीं पर्युपस्थितौ}
{ततस्तस्य नगस्याग्रे ह्याकाशस्थस्य दन्तिनः} %5-27-14

\twolineshloka
{भर्त्रा परिगृहीतस्य जानकी स्कन्धमाश्रिता}
{भर्तुरङ्कात् समुत्पत्य ततः कमललोचना} %5-27-15

\threelineshloka
{चन्द्रसूर्यौ मया दृष्टा पाणिभ्यां परिमार्जती}
{ततस्ताभ्यां कुमाराभ्यामास्थितः स गजोत्तमः}
{सीतया च विशालाक्ष्या लङ्काया उपरि स्थितः} %5-27-16

\twolineshloka
{पाण्डुरर्षभयुक्तेन रथेनाष्टयुजा स्वयम्}
{इहोपयातः काकुत्स्थः सीतया सह भार्यया} %5-27-17

\twolineshloka
{शुक्लमाल्याम्बरधरो लक्ष्मणेन सहागतः}
{ततोऽन्यत्र मया दृष्टो रामः सत्यपराक्रमः} %5-27-18

\twolineshloka
{लक्ष्मणेन सह भ्रात्रा सीतया सह वीर्यवान्}
{आरुह्य पुष्पकं दिव्यं विमानं सूर्यसंनिभम्} %5-27-19

\twolineshloka
{उत्तरां दिशमालोच्य प्रस्थितः पुरुषोत्तमः}
{एवं स्वप्ने मया दृष्टो रामो विष्णुपराक्रमः} %5-27-20

\twolineshloka
{लक्ष्मणेन सह भ्रात्रा सीतया सह भार्यया}
{न हि रामो महातेजाः शक्यो जेतुं सुरासुरैः} %5-27-21

\twolineshloka
{राक्षसैर्वापि चान्यैर्वा स्वर्गः पापजनैरिव}
{रावणश्च मया दृष्टो मुण्डस्तैलसमुक्षितः} %5-27-22

\twolineshloka
{रक्तवासाः पिबन्मत्तः करवीरकृतस्रजः}
{विमानात् पुष्पकादद्य रावणः पतितः क्षितौ} %5-27-23

\twolineshloka
{कृष्यमाणः स्त्रिया मुण्डो दृष्टः कृष्णाम्बरः पुनः}
{रथेन खरयुक्तेन रक्तमाल्यानुलेपनः} %5-27-24

\twolineshloka
{पिबंस्तैलं हसन्नृत्यन् भ्रान्तचित्ताकुलेन्द्रियः}
{गर्दभेन ययौ शीघ्रं दक्षिणां दिशमास्थितः} %5-27-25

\twolineshloka
{पुनरेव मया दृष्टो रावणो राक्षसेश्वरः}
{पतितोऽवाक्शिरा भूमौ गर्दभाद् भयमोहितः} %5-27-26

\twolineshloka
{सहसोत्थाय सम्भ्रान्तो भयार्तो मदविह्वलः}
{उन्मत्तरूपो दिग्वासा दुर्वाक्यं प्रलपन् बहु} %5-27-27

\twolineshloka
{दुर्गन्धं दुःसहं घोरं तिमिरं नरकोपमम्}
{मलपङ्कं प्रविश्याशु मग्नस्तत्र स रावणः} %5-27-28

\twolineshloka
{प्रस्थितो दक्षिणामाशां प्रविष्टोऽकर्दमं ह्रदम्}
{कण्ठे बद्ध्वा दशग्रीवं प्रमदा रक्तवासिनी} %5-27-29

\twolineshloka
{काली कर्दमलिप्तांगी दिशं याम्यां प्रकर्षति}
{एवं तत्र मया दृष्टः कुम्भकर्णो महाबलः} %5-27-30

\twolineshloka
{रावणस्य सुताः सर्वे मुण्डास्तैलसमुक्षिताः}
{वराहेण दशग्रीवः शिशुमारेण चेन्द्रजित्} %5-27-31

\twolineshloka
{उष्ट्रेण कुम्भकर्णश्च प्रयातो दक्षिणां दिशम्}
{एकस्तत्र मया दृष्टः श्वेतच्छत्रो विभीषणः} %5-27-32

\twolineshloka
{शुक्लमाल्याम्बरधरः शुक्लगन्धानुलेपनः}
{शङ्खदुन्दुभिनिर्घोषैर्नृत्तगीतैरलंकृतः} %5-27-33

\twolineshloka
{आरुह्य शैलसंकाशं मेघस्तनितनिःस्वनम्}
{चतुर्दन्तं गजं दिव्यमास्ते तत्र विभीषणः} %5-27-34

\onelineshloka
{चतुर्भिः सचिवैः सार्धं वैहायसमुपस्थितः} %5-27-35

\twolineshloka
{समाजश्च महान् वृत्तो गीतवादित्रनिःस्वनः}
{पिबतां रक्तमाल्यानां रक्षसां रक्तवाससाम्} %5-27-36

\twolineshloka
{लङ्का चेयं पुरी रम्या सवाजिरथकुञ्जरा}
{सागरे पतिता दृष्टा भग्नगोपुरतोरणा} %5-27-37

\twolineshloka
{लङ्का दृष्टा मया स्वप्ने रावणेनाभिरक्षिता}
{दग्धा रामस्य दूतेन वानरेण तरस्विना} %5-27-38

\twolineshloka
{पीत्वा तैलं प्रमत्ताश्च प्रहसन्त्यो महास्वनाः}
{लङ्कायां भस्मरूक्षायां सर्वा राक्षसयोषितः} %5-27-39

\twolineshloka
{कुम्भकर्णादयश्चेमे सर्वे राक्षसपुंगवाः}
{रक्तं निवसनं गृह्य प्रविष्टा गोमयह्रदम्} %5-27-40

\twolineshloka
{अपगच्छत पश्यध्वं सीतामाप्नोति राघवः}
{घातयेत् परमामर्षी युष्मान् सार्धं हि राक्षसैः} %5-27-41

\twolineshloka
{प्रियां बहुमतां भार्यां वनवासमनुव्रताम्}
{भर्त्सितां तर्जितां वापि नानुमंस्यति राघवः} %5-27-42

\twolineshloka
{तदलं क्रूरवाक्यैश्च सान्त्वमेवाभिधीयताम्}
{अभियाचाम वैदेहीमेतद्धि मम रोचते} %5-27-43

\twolineshloka
{यस्या ह्येवंविधः स्वप्नो दुःखितायाः प्रदृश्यते}
{सा दुःखैर्बहुभिर्मुक्ता प्रियं प्राप्नोत्यनुत्तमम्} %5-27-44

\twolineshloka
{भर्त्सितामपि याचध्वं राक्षस्यः किं विवक्षया}
{राघवाद्धि भयं घोरं राक्षसानामुपस्थितम्} %5-27-45

\twolineshloka
{प्रणिपातप्रसन्ना हि मैथिली जनकात्मजा}
{अलमेषा परित्रातुं राक्षस्यो महतो भयात्} %5-27-46

\twolineshloka
{अपि चास्या विशालाक्ष्या न किंचिदुपलक्षये}
{विरूपमपि चांगेषु सुसूक्ष्ममपि लक्षणम्} %5-27-47

\twolineshloka
{छायावैगुण्यमात्रं तु शङ्के दुःखमुपस्थितम्}
{अदुःखार्हामिमां देवीं वैहायसमुपस्थिताम्} %5-27-48

\twolineshloka
{अर्थसिद्धिं तु वैदेह्याः पश्याम्यहमुपस्थिताम्}
{राक्षसेन्द्रविनाशं च विजयं राघवस्य च} %5-27-49

\twolineshloka
{निमित्तभूतमेतत् तु श्रोतुमस्या महत् प्रियम्}
{दृश्यते च स्फुरच्चक्षुः पद्मपत्रमिवायतम्} %5-27-50

\twolineshloka
{ईषद्धि हृषितो वास्या दक्षिणाया ह्यदक्षिणः}
{अकस्मादेव वैदेह्या बाहुरेकः प्रकम्पते} %5-27-51

\twolineshloka
{करेणुहस्तप्रतिमः सव्यश्चोरुरनुत्तमः}
{वेपन् कथयतीवास्या राघवं पुरतः स्थितम्} %5-27-52

\twolineshloka
{पक्षी च शाखानिलयं प्रविष्टः पुनः पुनश्चोत्तमसान्त्ववादी}
{सुस्वागतां वाचमुदीरयाणः पुनः पुनश्चोदयतीव हृष्टः} %5-27-53

\twolineshloka
{ततः सा ह्रीमती बाला भर्तुर्विजयहर्षिता}
{अवोचद् यदि तत् तथ्यं भवेयं शरणं हि वः} %5-27-54


॥इत्यार्षे श्रीमद्रामायणे वाल्मीकीये आदिकाव्ये सुन्दरकाण्डे त्रिजटास्वप्नः नाम सप्तविंशः सर्गः ॥५-२७॥
