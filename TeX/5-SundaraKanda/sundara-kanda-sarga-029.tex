\sect{एकोनत्रिंशः सर्गः — शुभनिमित्तानि}

\twolineshloka
{तथागतां तां व्यथितामनिन्दितां व्यतीतहर्षां परिदीनमानसाम्}
{शुभां निमित्तानि शुभानि भेजिरे नरं श्रिया जुष्टमिवोपसेविनः} %5-29-1

\twolineshloka
{तस्याः शुभं वाममरालपक्ष्मराज्यावृतं कृष्णविशालशुक्लम्}
{प्रास्पन्दतैकं नयनं सुकेश्या मीनाहतं पद्ममिवाभिताम्रम्} %5-29-2

\twolineshloka
{भुजश्च चार्वञ्चितवृत्तपीनः परार्घ्यकालागुरुचन्दनार्हः}
{अनुत्तमेनाघ्युषितः प्रियेण चिरेण वामः समवेपताशु} %5-29-3

\twolineshloka
{गजेन्द्रहस्तप्रतिमश्च पीनस्तयोर्द्वयोः संहतयोस्तु जातः}
{प्रस्पन्दमानः पुनरूरुरस्या रामं पुरस्तात् स्थितमाचचक्षे} %5-29-4

\twolineshloka
{शुभं पुनर्हेमसमानवर्णमीषद्रजोध्वस्तमिवातुलाक्ष्याः}
{वासः स्थितायाः शिखराग्रदन्त्याः किंचित् परिस्रंसत चारुगात्र्याः} %5-29-5

\twolineshloka
{एतैर्निमित्तैरपरैश्च सुभ्रूः संचोदिता प्रागपि साधुसिद्धैः}
{वातातपक्लान्तमिव प्रणष्टं वर्षेण बीजं प्रतिसंजहर्ष} %5-29-6

\twolineshloka
{तस्याः पुनर्बिम्बफलोपमोष्ठं स्वक्षिभ्रुकेशान्तमरालपक्ष्म}
{वक्त्रं बभासे सितशुक्लदंष्ट्रं राहोर्मुखाच्चन्द्र इव प्रमुक्तः} %5-29-7

\twolineshloka
{सा वीतशोका व्यपनीततन्द्रा शान्तज्वरा हर्षविबुद्धसत्त्वा}
{अशोभतार्या वदनेन शुक्ले शीतांशुना रात्रिरिवोदितेन} %5-29-8


॥इत्यार्षे श्रीमद्रामायणे वाल्मीकीये आदिकाव्ये सुन्दरकाण्डे शुभनिमित्तानि नाम एकोनत्रिंशः सर्गः ॥५-२९॥
