\sect{चतुःपञ्चाशः सर्गः — लङ्कादाहः}

\twolineshloka
{वीक्षमाणस्ततो लङ्कां कपिः कृतमनोरथः}
{वर्धमानसमुत्साहः कार्यशेषमचिन्तयत्} %5-54-1

\twolineshloka
{किं नु खल्ववशिष्टं मे कर्तव्यमिह साम्प्रतम्}
{यदेषां रक्षसां भूयः सन्तापजननं भवेत्} %5-54-2

\twolineshloka
{वनं तावत्प्रमथितं प्रकृष्टा राक्षसा हताः}
{बलैकदेशः क्षपितः शेषं दुर्गविनाशनम्} %5-54-3

\twolineshloka
{दुर्गे विनाशिते कर्म भवेत् सुखपरिश्रमम्}
{अल्पयत्नेन कार्येऽस्मिन् मम स्यात् सफलः श्रमः} %5-54-4

\twolineshloka
{यो ह्ययं मम लाङ्गूले दीप्यते हव्यवाहनः}
{अस्य सन्तर्पणं न्याय्यं कर्तुमेभिर्गृहोत्तमैः} %5-54-5

\twolineshloka
{ततः प्रदीप्तलाङ्गूलः सविद्युदिव तोयदः}
{भवनाग्रेषु लङ्काया विचचार महाकपिः} %5-54-6

\twolineshloka
{गृहाद् गृहं राक्षसानामुद्यानानि च वानरः}
{वीक्षमाणो ह्यसन्त्रस्तः प्रासादांश्च चचार सः} %5-54-7

\twolineshloka
{अवप्लुत्य महावेगः प्रहस्तस्य निवेशनम्}
{अग्निं तत्र विनिक्षिप्य श्वसनेन समो बली} %5-54-8

\twolineshloka
{ततोऽन्यत् पुप्लुवे वेश्म महापार्श्वस्य वीर्यवान्}
{मुमोच हनुमानग्निं कालानलशिखोपमम्} %5-54-9

\twolineshloka
{वज्रदंष्ट्रस्य च तथा पुप्लुवे स महाकपिः}
{शुकस्य च महातेजाः सारणस्य च धीमतः} %5-54-10

\twolineshloka
{तथा चेन्द्रजितो वेश्म ददाह हरियूथपः}
{जम्बुमालेः सुमालेश्च ददाह भवनं ततः} %5-54-11

\twolineshloka
{रश्मिकेतोश्च भवनं सूर्यशत्रोस्तथैव च}
{ह्रस्वकर्णस्य दंष्ट्रस्य रोमशस्य च रक्षसः} %5-54-12

\twolineshloka
{युद्धोन्मत्तस्य मत्तस्य ध्वजग्रीवस्य रक्षसः}
{विद्युज्जिह्वस्य घोरस्य तथा हस्तिमुखस्य च} %5-54-13

\twolineshloka
{करालस्य विशालस्य शोणिताक्षस्य चैव हि}
{कुम्भकर्णस्य भवनं मकराक्षस्य चैव हि} %5-54-14

\twolineshloka
{नरान्तकस्य कुम्भस्य निकुम्भस्य दुरात्मनः}
{यज्ञशत्रोश्च भवनं ब्रह्मशत्रोस्तथैव च} %5-54-15

\twolineshloka
{वर्जयित्वा महातेजा विभीषणगृहं प्रति}
{क्रममाणः क्रमेणैव ददाह हरिपुङ्गवः} %5-54-16

\twolineshloka
{तेषु तेषु महार्हेषु भवनेषु महायशाः}
{गृहेष्वृद्धिमतामृद्धिं ददाह कपिकुञ्जरः} %5-54-17

\twolineshloka
{सर्वेषां समतिक्रम्य राक्षसेन्द्रस्य वीर्यवान्}
{आससादाथ लक्ष्मीवान् रावणस्य निवेशनम्} %5-54-18

\twolineshloka
{ततस्तस्मिन् गृहे मुख्ये नानारत्नविभूषिते}
{मेरुमन्दरसङ्काशे नानामङ्गलशोभिते} %5-54-19

\twolineshloka
{प्रदीप्तमग्निमुत्सृज्य लाङ्गूलाग्रे प्रतिष्ठितम्}
{ननाद हनुमान् वीरो युगान्तजलदो यथा} %5-54-20

\twolineshloka
{श्वसनेन च संयोगादतिवेगो महाबलः}
{कालाग्निरिव जज्वाल प्रावर्धत हुताशनः} %5-54-21

\twolineshloka
{प्रदीप्तमग्निं पवनस्तेषु वेश्मसु चारयन्}
{तानि काञ्चनजालानि मुक्तामणिमयानि च} %5-54-22

\twolineshloka
{भवनानि व्यशीर्यन्त रत्नवन्ति महान्ति च}
{तानि भग्नविमानानि निपेतुर्वसुधातले} %5-54-23

\twolineshloka
{भवनानीव सिद्धानामम्बरात् पुण्यसङ्क्षये}
{सञ्जज्ञे तुमुलः शब्दो राक्षसानां प्रधावताम्} %5-54-24

\twolineshloka
{स्वे स्वे गृहपरित्राणे भग्नोत्साहोज्झितश्रियाम्}
{नूनमेषोऽग्निरायातः कपिरूपेण हा इति} %5-54-25

\twolineshloka
{क्रन्दन्त्यः सहसा पेतुः स्तनन्धयधराः स्त्रियः}
{काश्चिदग्निपरीताङ्ग्यो हर्म्येभ्यो मुक्तमूर्धजाः} %5-54-26

\twolineshloka
{पतन्त्योरेजिरेऽभ्रेभ्यः सौदामन्य इवाम्बरात्}
{वज्रविद्रुमवैदूर्यमुक्तारजतसंहतान्} %5-54-27

\twolineshloka
{विचित्रान् भवनाद्धातून् स्यन्दमानान् ददर्श सः}
{नाग्निस्तृप्यति काष्ठानां तृणानां च यथा तथा} %5-54-28

\twolineshloka
{हनूमान् राक्षसेन्द्राणां वधे किञ्चिन्न तृप्यति}
{न हनूमद्विशस्तानां राक्षसानां वसुन्धरा} %5-54-29

\twolineshloka
{हनूमता वेगवता वानरेण महात्मना}
{लङ्कापुरं प्रदग्धं तद् रुद्रेण त्रिपुरं यथा} %5-54-30

\twolineshloka
{ततः स लङ्कापुरपर्वताग्रे समुत्थितो भीमपराक्रमोऽग्निः}
{प्रसार्य चूडावलयं प्रदीप्तो हनूमता वेगवतोपसृष्टः} %5-54-31

\twolineshloka
{युगान्तकालानलतुल्यरूपः समारुतोऽग्निर्ववृधे दिवस्पृक्}
{विधूमरश्मिर्भवनेषु सक्तो रक्षःशरीराज्यसमर्पितार्चिः} %5-54-32

\twolineshloka
{आदित्यकोटीसदृशः सुतेजा लङ्कां समस्तां परिवार्य तिष्ठन्}
{शब्दैरनेकैरशनिप्ररूढैर्भिन्दन्निवाण्डं प्रबभौ महाग्निः} %5-54-33

\twolineshloka
{तत्राम्बरादग्निरतिप्रवृद्धो रूक्षप्रभः किंशुकपुष्पचूडः}
{निर्वाणधूमाकुलराजयश्च नीलोत्पलाभाः प्रचकाशिरेऽभ्राः} %5-54-34

\twolineshloka
{वज्री महेन्द्रस्त्रिदशेश्वरो वा साक्षाद् यमो वा वरुणोऽनिलो वा}
{रौद्रोऽग्निरर्को धनदश्च सोमो न वानरोऽयं स्वयमेव कालः} %5-54-35

\twolineshloka
{किं ब्रह्मणः सर्वपितामहस्य लोकस्य धातुश्चतुराननस्य}
{इहागतो वानररूपधारी रक्षोपसंहारकरः प्रकोपः} %5-54-36

\twolineshloka
{किं वैष्णवं वा कपिरूपमेत्य रक्षोविनाशाय परं सुतेजः}
{अचिन्त्यमव्यक्तमनन्तमेकं स्वमायया साम्प्रतमागतं वा} %5-54-37

\twolineshloka
{इत्येवमूचुर्बहवो विशिष्टा रक्षोगणास्तत्र समेत्य सर्वे}
{सप्राणिसङ्घां सगृहां सवृक्षां दग्धां पुरीं तां सहसा समीक्ष्य} %5-54-38

\twolineshloka
{ततस्तु लङ्का सहसा प्रदग्धा सराक्षसा साश्वरथा सनागा}
{सपक्षिसङ्घा समृगा सवृक्षा रुरोद दीना तुमुलं सशब्दम्} %5-54-39

\twolineshloka
{हा तात हा पुत्रक कान्त मित्र हा जीवितेशाङ्ग हतं सुपुण्यम्}
{रक्षोभिरेवं बहुधा ब्रुवद्भिः शब्दः कृतो घोरतरः सुभीमः} %5-54-40

\twolineshloka
{हुताशनज्वालसमावृता सा हतप्रवीरा परिवृत्तयोधा}
{हनूमतः क्रोधबलाभिभूता बभूव शापोपहतेव लङ्का} %5-54-41

\twolineshloka
{ससम्भ्रमं त्रस्तविषण्णराक्षसां समुज्ज्वलज्ज्वालहुताशनाङ्किताम्}
{ददर्श लङ्कां हनुमान् महामनाः स्वयम्भुरोषोपहतामिवावनिम्} %5-54-42

\twolineshloka
{भङ्क्त्वा वनं पादपरत्नसङ्कुलं हत्वा तु रक्षांसि महान्ति संयुगे}
{दग्ध्वा पुरीं तां गृहरत्नमालिनीं तस्थौ हनूमान् पवनात्मजः कपिः} %5-54-43

\twolineshloka
{स राक्षसांस्तान् सुबहूंश्च हत्वा वनं च भङ्क्त्वा बहुपादपं तत्}
{विसृज्य रक्षोभवनेषु चाग्निं जगाम रामं मनसा महात्मा} %5-54-44

\twolineshloka
{ततस्तु तं वानरवीरमुख्यं महाबलं मारुततुल्यवेगम्}
{महामतिं वायुसुतं वरिष्ठं प्रतुष्टुवुर्देवगणाश्च सर्वे} %5-54-45

\twolineshloka
{देवाश्च सर्वे मुनिपुङ्गवाश्च गन्धर्वविद्याधरपन्नगाश्च}
{भूतानि सर्वाणि महान्ति तत्र जग्मुः परां प्रीतिमतुल्यरूपाम्} %5-54-46

\twolineshloka
{भङ्क्त्वा वनं महातेजा हत्वा रक्षांसि संयुगे}
{दग्ध्वा लङ्कापुरीं भीमां रराज स महाकपिः} %5-54-47

\twolineshloka
{गृहाग्र्यशृङ्गाग्रतले विचित्रे प्रतिष्ठितो वानरराजसिंहः}
{प्रदीप्तलाङ्गूलकृतार्चिमाली व्यराजतादित्य इवार्चिमाली} %5-54-48

\twolineshloka
{लङ्कां समस्तां सम्पीड्य लाङ्गूलाग्निं महाकपिः}
{निर्वापयामास तदा समुद्रे हरिपुङ्गवः} %5-54-49

\twolineshloka
{ततो देवाः सगन्धर्वाः सिद्धाश्च परमर्षयः}
{दृष्ट्वा लङ्कां प्रदग्धां तां विस्मयं परमं गताः} %5-54-50

\twolineshloka
{तं दृष्ट्वा वानरश्रेष्ठं हनूमन्तं महाकपिम्}
{कालाग्निरिति सञ्चिन्त्य सर्वभूतानि तत्रसुः} %5-54-51


॥इत्यार्षे श्रीमद्रामायणे वाल्मीकीये आदिकाव्ये सुन्दरकाण्डे लङ्कादाहः नाम चतुःपञ्चाशः सर्गः ॥५-५४॥
