\sect{षड्चत्वारिंशः सर्गः — सेनापतिपञ्चकवधः}

\twolineshloka
{हतान् मन्त्रिसुतान् बुद्ध्वा वानरेण महात्मना}
{रावणः संवृताकारश्चकार मतिमुत्तमाम्} %5-46-1

\twolineshloka
{स विरूपाक्षयूपाक्षौ दुर्धरं चैव राक्षसम्}
{प्रघसं भासकर्णं च पञ्च सेनाग्रनायकान्} %5-46-2

\twolineshloka
{संदिदेश दशग्रीवो वीरान् नयविशारदान्}
{हनूमद्ग्रहणेऽव्यग्रान् वायुवेगसमान् युधि} %5-46-3

\twolineshloka
{यात सेनाग्रगाः सर्वे महाबलपरिग्रहाः}
{सवाजिरथमातङ्गाः स कपिः शास्यतामिति} %5-46-4

\twolineshloka
{यत्तैश्च खलु भाव्यं स्यात् तमासाद्य वनालयम्}
{कर्म चापि समाधेयं देशकालाविरोधितम्} %5-46-5

\twolineshloka
{न ह्यहं तं कपिं मन्ये कर्मणा प्रति तर्कयन्}
{सर्वथा तन्महद् भूतं महाबलपरिग्रहम्} %5-46-6

\twolineshloka
{वानरोऽयमिति ज्ञात्वा नहि शुद्ध्यति मे मनः}
{नैवाहं तं कपिं मन्ये यथेयं प्रस्तुता कथा} %5-46-7

\twolineshloka
{भवेदिन्द्रेण वा सृष्टमस्मदर्थं तपोबलात्}
{सनागयक्षगन्धर्वदेवासुरमहर्षयः} %5-46-8

\twolineshloka
{युष्माभिः प्रहितैः सर्वैर्मया सह विनिर्जिताः}
{तैरवश्यं विधातव्यं व्यलीकं किंचिदेव नः} %5-46-9

\twolineshloka
{तदेव नात्र संदेहः प्रसह्य परिगृह्यताम्}
{यात सेनाग्रगाः सर्वे महाबलपरिग्रहाः} %5-46-10

\twolineshloka
{सवाजिरथमातङ्गाः स कपिः शास्यतामिति}
{नावमन्यो भवद्भिश्च कपिर्धीरपराक्रमः} %5-46-11

\twolineshloka
{दृष्टा हि हरयः पूर्वे मया विपुलविक्रमाः}
{वाली च सह सुग्रीवो जाम्बवांश्च महाबलः} %5-46-12

\twolineshloka
{नीलः सेनापतिश्चैव ये चान्ये द्विविदादयः}
{नैव तेषां गतिर्भीमा न तेजो न पराक्रमः} %5-46-13

\twolineshloka
{न मतिर्न बलोत्साहो न रूपपरिकल्पनम्}
{महत्सत्त्वमिदं ज्ञेयं कपिरूपं व्यवस्थितम्} %5-46-14

\twolineshloka
{प्रयत्नं महदास्थाय क्रियतामस्य निग्रहः}
{कामं लोकास्त्रयः सेन्द्राः ससुरासुरमानवाः} %5-46-15

\twolineshloka
{भवतामग्रतः स्थातुं न पर्याप्ता रणाजिरे}
{तथापि तु नयज्ञेन जयमाकाङ्क्षता रणे} %5-46-16

\twolineshloka
{आत्मा रक्ष्यः प्रयत्नेन युद्धसिद्धिर्हि चञ्चला}
{ते स्वामिवचनं सर्वे प्रतिगृह्य महौजसः} %5-46-17

\twolineshloka
{समुत्पेतुर्महावेगा हुताशसमतेजसः}
{रथैश्च मत्तैर्नागैश्च वाजिभिश्च महाजवैः} %5-46-18

\twolineshloka
{शस्त्रैश्च विविधैस्तीक्ष्णैः सर्वैश्चोपहिता बलैः}
{ततस्तु ददृशुर्वीरा दीप्यमानं महाकपिम्} %5-46-19

\twolineshloka
{रश्मिमन्तमिवोद्यन्तं स्वतेजोरश्मिमालिनम्}
{तोरणस्थं महावेगं महासत्त्वं महाबलम्} %5-46-20

\twolineshloka
{महामतिं महोत्साहं महाकायं महाभुजम्}
{तं समीक्ष्यैव ते सर्वे दिक्षु सर्वास्ववस्थिताः} %5-46-21

\threelineshloka
{तैस्तैः प्रहरणैर्भीमैरभिपेतुस्ततस्ततः}
{तस्य पञ्चायसास्तीक्ष्णाः सिताः पीतमुखाः शराः}
{शिरस्युत्पलपत्राभा दुर्धरेण निपातिताः} %5-46-22

\twolineshloka
{स तैः पञ्चभिराविद्धः शरैः शिरसि वानरः}
{उत्पपात नदन् व्योम्नि दिशो दश विनादयन्} %5-46-23

\twolineshloka
{ततस्तु दुर्धरो वीरः सरथः सज्जकार्मुकः}
{किरन् शरशतैर्नैकैरभिपेदे महाबलः} %5-46-24

\twolineshloka
{स कपिर्वारयामास तं व्योम्नि शरवर्षिणम्}
{वृष्टिमन्तं पयोदान्ते पयोदमिव मारुतः} %5-46-25

\twolineshloka
{अर्द्यमानस्ततस्तेन दुर्धरेणानिलात्मजः}
{चकार निनदं भूयो व्यवर्धत च वीर्यवान्} %5-46-26

\twolineshloka
{स दूरं सहसोत्पत्य दुर्धरस्य रथे हरिः}
{निपपात महावेगो विद्युद्राशिर्गिराविव} %5-46-27

\twolineshloka
{ततः स मथिताष्टाश्वं रथं भग्नाक्षकूबरम्}
{विहाय न्यपतद् भूमौ दुर्धरस्त्यक्तजीवितः} %5-46-28

\twolineshloka
{तं विरूपाक्षयूपाक्षौ दृष्ट्वा निपतितं भुवि}
{तौ जातरोषौ दुर्धर्षावुत्पेततुररिंदमौ} %5-46-29

\twolineshloka
{स ताभ्यां सहसोत्प्लुत्य विष्ठितो विमलेऽम्बरे}
{मुद्गराभ्यां महाबाहुर्वक्षस्यभिहतः कपिः} %5-46-30

\twolineshloka
{तयोर्वेगवतोर्वेगं निहत्य स महाबलः}
{निपपात पुनर्भूमौ सुपर्ण इव वेगितः} %5-46-31

\twolineshloka
{स सालवृक्षमासाद्य समुत्पाट्य च वानरः}
{तावुभौ राक्षसौ वीरौ जघान पवनात्मजः} %5-46-32

\twolineshloka
{ततस्तांस्त्रीन् हतान् ज्ञात्वा वानरेण तरस्विना}
{अभिपेदे महावेगः प्रहस्य प्रघसो बली} %5-46-33

\twolineshloka
{भासकर्णश्च संक्रुद्धः शूलमादाय वीर्यवान्}
{एकतः कपिशार्दूलं यशस्विनमवस्थितौ} %5-46-34

\twolineshloka
{पट्टिशेन शिताग्रेण प्रघसः प्रत्यपोथयत्}
{भासकर्णश्च शूलेन राक्षसः कपिकुञ्जरम्} %5-46-35

\twolineshloka
{स ताभ्यां विक्षतैर्गात्रैरसृग्दिग्धतनूरुहः}
{अभवद् वानरः क्रुद्धो बालसूर्यसमप्रभः} %5-46-36

\threelineshloka
{समुत्पाट्य गिरेः शृङ्गं समृगव्यालपादपम्}
{जघान हनुमान् वीरो राक्षसौ कपिकुञ्जरः}
{गिरिशृङ्गसुनिष्पिष्टौ तिलशस्तौ बभूवतुः} %5-46-37

\twolineshloka
{ततस्तेष्ववसन्नेषु सेनापतिषु पञ्चसु}
{बलं तदवशेषं तु नाशयामास वानरः} %5-46-38

\twolineshloka
{अश्वैरश्वान् गजैर्नागान् योधैर्योधान् रथै रथान्}
{स कपिर्नाशयामास सहस्राक्ष इवासुरान्} %5-46-39

\twolineshloka
{हयैर्नागैस्तुरंगैश्च भग्नाक्षैश्च महारथैः}
{हतैश्च राक्षसैर्भूमी रुद्धमार्गा समन्ततः} %5-46-40

\twolineshloka
{ततः कपिस्तान् ध्वजिनीपतीन् रणे निहत्य वीरान् सबलान् सवाहनान्}
{तथैव वीरः परिगृह्य तोरणं कृतक्षणः काल इव प्रजाक्षये} %5-46-41


॥इत्यार्षे श्रीमद्रामायणे वाल्मीकीये आदिकाव्ये सुन्दरकाण्डे सेनापतिपञ्चकवधः नाम षड्चत्वारिंशः सर्गः ॥५-४६॥
