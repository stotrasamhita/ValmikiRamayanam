\sect{त्रिपञ्चाशः सर्गः — पावकशैत्यम्}

\twolineshloka
{तस्य तद् वचनं श्रुत्वा दशग्रीवो महात्मनः}
{देशकालहितं वाक्यं भ्रातुरुत्तरमब्रवीत्} %5-53-1

\twolineshloka
{सम्यगुक्तं हि भवता दूतवध्या विगर्हिता}
{अवश्यं तु वधायान्यः क्रियतामस्य निग्रहः} %5-53-2

\twolineshloka
{कपीनां किल लाङ्गूलमिष्टं भवति भूषणम्}
{तदस्य दीप्यतां शीघ्रं तेन दग्धेन गच्छतु} %5-53-3

\twolineshloka
{ततः पश्यन्त्वमुं दीनमङ्गवैरूप्यकर्शितम्}
{सुमित्रज्ञातयः सर्वे बान्धवाः ससुहृज्जनाः} %5-53-4

\twolineshloka
{आज्ञापयद् राक्षसेन्द्रः पुरं सर्वं सचत्वरम्}
{लाङ्गूलेन प्रदीप्तेन रक्षोभिः परिणीयताम्} %5-53-5

\twolineshloka
{तस्य तद् वचनं श्रुत्वा राक्षसाः कोपकर्कशाः}
{वेष्टन्ते तस्य लाङ्गूलं जीर्णैः कार्पासिकैः पटैः} %5-53-6

\twolineshloka
{संवेष्ट्यमाने लाङ्गूले व्यवर्धत महाकपिः}
{शुष्कमिन्धनमासाद्य वनेष्विव हुताशनम्} %5-53-7

\twolineshloka
{तैलेन परिषिच्याथ तेऽग्निं तत्रोपपादयन्}
{लाङ्गूलेन प्रदीप्तेन राक्षसांस्तानताडयत्} %5-53-8

\twolineshloka
{रोषामर्षपरीतात्मा बालसूर्यसमाननः}
{स भूयः संगतैः क्रूरै राक्षसैर्हरिपुङ्गवः} %5-53-9

\twolineshloka
{सहस्त्रीबालवृद्धाश्च जग्मुः प्रीतिं निशाचराः}
{निबद्धः कृतवान् वीरस्तत्कालसदृशीं मतिम्} %5-53-10

\twolineshloka
{कामं खलु न मे शक्ता निबद्धस्यापि राक्षसाः}
{छित्त्वा पाशान् समुत्पत्य हन्यामहमिमान् पुनः} %5-53-11

\twolineshloka
{यदि भर्तृहितार्थाय चरन्तं भर्तृशासनात्}
{निबध्नन्ते दुरात्मानो न तु मे निष्कृतिः कृता} %5-53-12

\twolineshloka
{सर्वेषामेव पर्याप्तो राक्षसानामहं युधि}
{किं तु रामस्य प्रीत्यर्थं विषहिष्येऽहमीदृशम्} %5-53-13

\twolineshloka
{लङ्का चारयितव्या मे पुनरेव भवेदिति}
{रात्रौ नहि सुदृष्टा मे दुर्गकर्मविधानतः} %5-53-14

\twolineshloka
{अवश्यमेव द्रष्टव्या मया लङ्का निशाक्षये}
{कामं बध्नन्तु मे भूयः पुच्छस्योद्दीपनेन च} %5-53-15

\twolineshloka
{पीडां कुर्वन्ति रक्षांसि न मेऽस्ति मनसः श्रमः}
{ततस्ते संवृताकारं सत्त्ववन्तं महाकपिम्} %5-53-16

\twolineshloka
{परिगृह्य ययुर्हृष्टा राक्षसाः कपिकुञ्जरम्}
{शङ्खभेरीनिनादैश्च घोषयन्तः स्वकर्मभिः} %5-53-17

\twolineshloka
{राक्षसाः क्रूरकर्माणश्चारयन्ति स्म तां पुरीम्}
{अन्वीयमानो रक्षोभिर्ययौ सुखमिरंदमः} %5-53-18

\twolineshloka
{हनूमांश्चारयामास राक्षसानां महापुरीम्}
{अथापश्यद् विमानानि विचित्राणि महाकपिः} %5-53-19

\twolineshloka
{संवृतान् भूमिभागांश्च सुविभक्तांश्च चत्वरान्}
{रथ्याश्च गृहसम्बाधाः कपिः शृङ्गाटकानि च} %5-53-20

\twolineshloka
{तथा रथ्योपरथ्याश्च तथैव च गृहान्तरान्}
{चत्वरेषु चतुष्केषु राजमार्गे तथैव च} %5-53-21

\twolineshloka
{घोषयन्ति कपिं सर्वे चार इत्येव राक्षसाः}
{स्त्रीबालवृद्धा निर्जग्मुस्तत्र तत्र कुतूहलात्} %5-53-22

\twolineshloka
{तं प्रदीपितलाङ्गूलं हनूमन्तं दिदृक्षवः}
{दीप्यमाने ततस्तस्य लाङ्गूलाग्रे हनूमतः} %5-53-23

\twolineshloka
{राक्षस्यस्ता विरूपाक्ष्यः शंसुर्देव्यास्तदप्रियम्}
{यस्त्वया कृतसंवादः सीते ताम्रमुखः कपिः} %5-53-24

\twolineshloka
{लाङ्गूलेन प्रदीप्तेन स एष परिणीयते}
{श्रुत्वा तद् वचनं क्रूरमात्मापहरणोपमम्} %5-53-25

\twolineshloka
{वैदेही शोकसंतप्ता हुताशनमुपागमत्}
{मङ्गलाभिमुखी तस्य सा तदासीन्महाकपेः} %5-53-26

\threelineshloka
{उपतस्थे विशालाक्षी प्रयता हव्यवाहनम्}
{यद्यस्ति पतिशुश्रूषा यद्यस्ति चरितं तपः}
{यदि वा त्वेकपत्नीत्वं शीतो भव हनूमतः} %5-53-27

\twolineshloka
{यदि किंचिदनुक्रोशस्तस्य मय्यस्ति धीमतः}
{यदि वा भाग्यशेषो मे शीतो भव हनूमतः} %5-53-28

\twolineshloka
{यदि मां वृत्तसम्पन्नां तत्समागमलालसाम्}
{स विजानाति धर्मात्मा शीतो भव हनूमतः} %5-53-29

\twolineshloka
{यदि मां तारयेदार्यः सुग्रीवः सत्यसंगरः}
{अस्माद् दुःखाम्बुसंरोधाच्छीतो भव हनूमतः} %5-53-30

\twolineshloka
{ततस्तीक्ष्णार्चिरव्यग्रः प्रदक्षिणशिखोऽनलः}
{जज्वाल मृगशावाक्ष्याः शंसन्निव शुभं कपेः} %5-53-31

\twolineshloka
{हनूमज्जनकश्चैव पुच्छानलयुतोऽनिलः}
{ववौ स्वास्थ्यकरो देव्याः प्रालेयानिलशीतलः} %5-53-32

\twolineshloka
{दह्यमाने च लाङ्गूले चिन्तयामास वानरः}
{प्रदीप्तोऽग्निरयं कस्मान्न मां दहति सर्वतः} %5-53-33

\twolineshloka
{दृश्यते च महाज्वालः करोति च न मे रुजम्}
{शिशिरस्येव सम्पातो लाङ्गूलाग्रे प्रतिष्ठितः} %5-53-34

\twolineshloka
{अथ वा तदिदं व्यक्तं यद् दृष्टं प्लवता मया}
{रामप्रभावादाश्चर्यं पर्वतः सरितां पतौ} %5-53-35

\twolineshloka
{यदि तावत् समुद्रस्य मैनाकस्य च धीमतः}
{रामार्थं सम्भ्रमस्तादृक् किमग्निर्न करिष्यति} %5-53-36

\twolineshloka
{सीतायाश्चानृशंस्येन तेजसा राघवस्य च}
{पितुश्च मम सख्येन न मां दहति पावकः} %5-53-37

\twolineshloka
{भूयः स चिन्तयामास मुहूर्तं कपिकुञ्जरः}
{कथमस्मद्विधस्येह बन्धनं राक्षसाधमैः} %5-53-38

\twolineshloka
{प्रतिक्रियास्य युक्ता स्यात् सति मह्यं पराक्रमे}
{ततश्छित्त्वा च तान् पाशान् वेगवान् वै महाकपिः} %5-53-39

\twolineshloka
{उत्पपाताथ वेगेन ननाद च महाकपिः}
{पुरद्वारं ततः श्रीमान् शैलशृङ्गमिवोन्नतम्} %5-53-40

\twolineshloka
{विभक्तरक्षःसम्बाधमाससादानिलात्मजः}
{स भूत्वा शैलसंकाशः क्षणेन पुनरात्मवान्} %5-53-41

\twolineshloka
{ह्रस्वतां परमां प्राप्तो बन्धनान्यवशातयत्}
{विमुक्तश्चाभवच्छ्रीमान् पुनः पर्वतसंनिभः} %5-53-42

\threelineshloka
{वीक्षमाणश्च ददृशे परिघं तोरणाश्रितम्}
{स तं गृह्य महाबाहुः कालायसपरिष्कृतम्}
{रक्षिणस्तान् पुनः सर्वान् सूदयामास मारुतिः} %5-53-43

\twolineshloka
{स तान् निहत्वा रणचण्डविक्रमः समीक्षमाणः पुनरेव लङ्काम्}
{प्रदीप्तलाङ्गूलकृतार्चिमाली प्रकाशितादित्य इवार्चिमाली} %5-53-44


॥इत्यार्षे श्रीमद्रामायणे वाल्मीकीये आदिकाव्ये सुन्दरकाण्डे पावकशैत्यम् नाम त्रिपञ्चाशः सर्गः ॥५-५३॥
