\sect{षष्ठः सर्गः — रावणगृहावेक्षणम्}

\twolineshloka
{स निकामं विमानेषु विचरन् कामरूपधृक्}
{विचचार कपिर्लङ्कां लाघवेन समन्वितः} %5-6-1

\twolineshloka
{आससाद च लक्ष्मीवान् राक्षसेन्द्रनिवेशनम्}
{प्राकारेणार्कवर्णेन भास्वरेणाभिसंवृतम्} %5-6-2

\twolineshloka
{रक्षितं राक्षसैर्भीमैः सिंहैरिव महद् वनम्}
{समीक्षमाणो भवनं चकाशे कपिकुञ्जरः} %5-6-3

\twolineshloka
{रूप्यकोपहितैश्चित्रैस्तोरणैर्हेमभूषणैः}
{विचित्राभिश्च कक्ष्याभिर्द्वारैश्च रुचिरैर्वृतम्} %5-6-4

\twolineshloka
{गजास्थितैर्महामात्रैः शूरैश्च विगतश्रमैः}
{उपस्थितमसंहार्यैर्हयैः स्यन्दनयायिभिः} %5-6-5

\twolineshloka
{सिंहव्याघ्रतनुत्राणैर्दान्तकाञ्चनराजतीः}
{घोषवद्भिर्विचित्रैश्च सदा विचरितं रथैः} %5-6-6

\twolineshloka
{बहुरत्नसमाकीर्णं परार्घ्यासनभूषितम्}
{महारथसमावापं महारथमहासनम्} %5-6-7

\twolineshloka
{दृश्यैश्च परमोदारैस्तैस्तैश्च मृगपक्षिभिः}
{विविधैर्बहुसाहस्रैः परिपूर्णं समन्ततः} %5-6-8

\twolineshloka
{विनीतैरन्तपालैश्च रक्षोभिश्च सुरक्षितम्}
{मुख्याभिश्च वरस्त्रीभिः परिपूर्णं समन्ततः} %5-6-9

\twolineshloka
{मुदितप्रमदारत्नं राक्षसेन्द्रनिवेशनम्}
{वराभरणसंह्रादैः समुद्रस्वननिःस्वनम्} %5-6-10

\twolineshloka
{तद् राजगुणसम्पन्नं मुख्यैश्च वरचन्दनैः}
{महाजनसमाकीर्णं सिंहैरिव महद् वनम्} %5-6-11

\twolineshloka
{भेरीमृदङ्गाभिरुतं शङ्खघोषविनादितम्}
{नित्यार्चितं पर्वसुतं पूजितं राक्षसैः सदा} %5-6-12

\twolineshloka
{समुद्रमिव गम्भीरं समुद्रसमनिःस्वनम्}
{महात्मनो महद् वेश्म महारत्नपरिच्छदम्} %5-6-13

\twolineshloka
{महारत्नसमाकीर्णं ददर्श स महाकपिः}
{विराजमानं वपुषा गजाश्वरथसङ्कुलम्} %5-6-14

\twolineshloka
{लङ्काभरणमित्येव सोऽमन्यत महाकपिः}
{चचार हनुमांस्तत्र रावणस्य समीपतः} %5-6-15

\twolineshloka
{गृहाद् गृहं राक्षसानामुद्यानानि च सर्वशः}
{वीक्षमाणोऽप्यसन्त्रस्तः प्रासादांश्च चचार सः} %5-6-16

\twolineshloka
{अवप्लुत्य महावेगः प्रहस्तस्य निवेशनम्}
{ततोऽन्यत् पुप्लुवे वेश्म महापार्श्वस्य वीर्यवान्} %5-6-17

\twolineshloka
{अथ मेघप्रतीकाशं कुम्भकर्णनिवेशनम्}
{विभीषणस्य च तथा पुप्लुवे स महाकपिः} %5-6-18

\twolineshloka
{महोदरस्य च तथा विरूपाक्षस्य चैव हि}
{विद्युज्जिह्वस्य भवनं विद्युन्मालेस्तथैव च} %5-6-19

\twolineshloka
{वज्रदंष्ट्रस्य च तथा पुप्लुवे स महाकपिः}
{शुकस्य च महावेगः सारणस्य च धीमतः} %5-6-20

\twolineshloka
{तथा चेन्द्रजितो वेश्म जगाम हरियूथपः}
{जम्बुमालेः सुमालेश्च जगाम हरिसत्तमः} %5-6-21

\twolineshloka
{रश्मिकेतोश्च भवनं सूर्यशत्रोस्तथैव च}
{वज्रकायस्य च तथा पुप्लुवे स महाकपिः} %5-6-22

\twolineshloka
{धूम्राक्षस्याथ सम्पातेर्भवनं मारुतात्मजः}
{विद्युद्रूपस्य भीमस्य घनस्य विघनस्य च} %5-6-23

\twolineshloka
{शुकनाभस्य चक्रस्य शठस्य कपटस्य च}
{ह्रस्वकर्णस्य दंष्ट्रस्य लोमशस्य च रक्षसः} %5-6-24

\twolineshloka
{युद्धोन्मत्तस्य मत्तस्य ध्वजग्रीवस्य सादिनः}
{विद्युज्जिह्वद्विजिह्वानां तथा हस्तिमुखस्य च} %5-6-25

\twolineshloka
{करालस्य पिशाचस्य शोणिताक्षस्य चैव हि}
{प्लवमानः क्रमेणैव हनुमान् मारुतात्मजः} %5-6-26

\twolineshloka
{तेषु तेषु महार्हेषु भवनेषु महायशाः}
{तेषामृद्धिमतामृद्धिं ददर्श स महाकपिः} %5-6-27

\twolineshloka
{सर्वेषां समतिक्रम्य भवनानि समन्ततः}
{आससादाथ लक्ष्मीवान् राक्षसेन्द्रनिवेशनम्} %5-6-28

\twolineshloka
{रावणस्योपशायिन्यो ददर्श हरिसत्तमः}
{विचरन् हरिशार्दूलो राक्षसीर्विकृतेक्षणाः} %5-6-29

\twolineshloka
{शूलमुद्गरहस्तांश्च शक्तितोमरधारिणः}
{ददर्श विविधान्गुल्मांस्तस्य रक्षःपतेर्गृहे} %5-6-30

\twolineshloka
{राक्षसांश्च महाकायान् नानाप्रहरणोद्यतान्}
{रक्तान् श्वेतान् सितांश्चापि हरींश्चापि महाजवान्} %5-6-31

\twolineshloka
{कुलीनान् रूपसम्पन्नान् गजान् परगजारुजान्}
{शिक्षितान् गजशिक्षायामैरावतसमान् युधि} %5-6-32

\twolineshloka
{निहन्तॄन् परसैन्यानां गृहे तस्मिन् ददर्श सः}
{क्षरतश्च यथा मेघान् स्रवतश्च यथा गिरीन्} %5-6-33

\twolineshloka
{मेघस्तनितनिर्घोषान् दुर्धर्षान् समरे परैः}
{सहस्रं वाहिनीस्तत्र जाम्बूनदपरिष्कृताः} %5-6-34

\twolineshloka
{हेमजालैरविच्छिन्नास्तरुणादित्यसन्निभाः}
{ददर्श राक्षसेन्द्रस्य रावणस्य निवेशने} %5-6-35

\twolineshloka
{शिबिका विविधाकाराः स कपिर्मारुतात्मजः}
{लतागृहाणि चित्राणि चित्रशालागृहाणि च} %5-6-36

\twolineshloka
{क्रीडागृहाणि चान्यानि दारुपर्वतकानि च}
{कामस्य गृहकं रम्यं दिवागृहकमेव च} %5-6-37

\twolineshloka
{ददर्श राक्षसेन्द्रस्य रावणस्य निवेशने}
{स मन्दरसमप्रख्यं मयूरस्थानसङ्कुलम्} %5-6-38

\threelineshloka
{ध्वजयष्टिभिराकीर्णं ददर्श भवनोत्तमम्}
{अनन्तरत्ननिचयं निधिजालं समन्ततः}
{धीरनिष्ठितकर्माङ्गं गृहं भूतपतेरिव} %5-6-39

\twolineshloka
{अर्चिर्भिश्चापि रत्नानां तेजसा रावणस्य च}
{विरराज च तद् वेश्म रश्मिवानिव रश्मिभिः} %5-6-40

\twolineshloka
{जाम्बूनदमयान्येव शयनान्यासनानि च}
{भाजनानि च शुभ्राणि ददर्श हरियूथपः} %5-6-41

\twolineshloka
{मध्वासवकृतक्लेदं मणिभाजनसङ्कुलम्}
{मनोरममसम्बाधं कुबेरभवनं यथा} %5-6-42

\twolineshloka
{नूपुराणां च घोषेण काञ्चीनां निःस्वनेन च}
{मृदङ्गतलनिर्घोषैर्घोषवद्भिर्विनादितम्} %5-6-43

\twolineshloka
{प्रासादसङ्घातयुतं स्त्रीरत्नशतसङ्कुलम्}
{सुव्यूढकक्ष्यं हनुमान् प्रविवेश महागृहम्} %5-6-44


॥इत्यार्षे श्रीमद्रामायणे वाल्मीकीये आदिकाव्ये सुन्दरकाण्डे रावणगृहावेक्षणम् नाम षष्ठः सर्गः ॥५-६॥
