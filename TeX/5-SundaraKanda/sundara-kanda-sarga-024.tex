\sect{चतुर्विंशः सर्गः — राक्षसीनिर्भर्त्सनम्}

\twolineshloka
{ततः सीतां समस्तास्ता राक्षस्यो विकृताननाः}
{परुषं परुषानर्हामूचुस्तद्वाक्यमप्रियम्} %5-24-1

\twolineshloka
{किं त्वमन्तःपुरे सीते सर्वभूतमनोरमे}
{महार्हशयनोपेते न वासमनुमन्यसे} %5-24-2

\twolineshloka
{मानुषी मानुषस्यैव भार्यात्वं बहु मन्यसे}
{प्रत्याहर मनो रामान्नैवं जातु भविष्यति} %5-24-3

\twolineshloka
{त्रैलोक्यवसुभोक्तारं रावणं राक्षसेश्वरम्}
{भर्तारमुपसङ्गम्य विहरस्व यथासुखम्} %5-24-4

\twolineshloka
{मानुषी मानुषं तं तु राममिच्छसि शोभने}
{राज्याद् भ्रष्टमसिद्धार्थं विक्लवन्तमनिन्दिते} %5-24-5

\twolineshloka
{राक्षसीनां वचः श्रुत्वा सीता पद्मनिभेक्षणा}
{नेत्राभ्यामश्रुपूर्णाभ्यामिदं वचनमब्रवीत्} %5-24-6

\twolineshloka
{यदिदं लोकविद्विष्टमुदाहरत सङ्गताः}
{नैतन्मनसि वाक्यं मे किल्बिषं प्रतितिष्ठति} %5-24-7

\twolineshloka
{न मानुषी राक्षसस्य भार्या भवितुमर्हति}
{कामं खादत मां सर्वा न करिष्यामि वो वचः} %5-24-8

\twolineshloka
{दीनो वा राज्यहीनो वा यो मे भर्ता स मे गुरुः}
{तं नित्यमनुरक्तास्मि यथा सूर्यं सुवर्चला} %5-24-9

\twolineshloka
{यथा शची महाभागा शक्रं समुपतिष्ठति}
{अरुन्धती वसिष्ठं च रोहिणी शशिनं यथा} %5-24-10

\twolineshloka
{लोपामुद्रा यथागस्त्यं सुकन्या च्यवनं यथा}
{सावित्री सत्यवन्तं च कपिलं श्रीमती यथा} %5-24-11

\twolineshloka
{सौदासं मदयन्तीव केशिनी सगरं यथा}
{नैषधं दमयन्तीव भैमी पतिमनुव्रता} %5-24-12

\threelineshloka
{तथाहमिक्ष्वाकुवरं रामं पतिमनुव्रता}
{सीताया वचनं श्रुत्वा राक्षस्यः क्रोधमूर्च्छिताः}
{भर्त्सयन्ति स्म परुषैर्वाक्यै रावणचोदिताः} %5-24-13

\twolineshloka
{अवलीनः स निर्वाक्यो हनुमान् शिंशपाद्रुमे}
{सीतां सन्तर्जयन्तीस्ता राक्षसीरशृणोत् कपिः} %5-24-14

\twolineshloka
{तामभिक्रम्य संरब्धा वेपमानां समन्ततः}
{भृशं संलिलिहुर्दीप्तान् प्रलम्बान् दशनच्छदान्} %5-24-15

\twolineshloka
{ऊचुश्च परमक्रुद्धाः प्रगृह्याशु परश्वधान्}
{नेयमर्हति भर्तारं रावणं राक्षसाधिपम्} %5-24-16

\twolineshloka
{सा भर्त्स्यमाना भीमाभी राक्षसीभिर्वराङ्गना}
{सा बाष्पमपमार्जन्ती शिंशपां तामुपागमत्} %5-24-17

\twolineshloka
{ततस्तां शिंशपां सीता राक्षसीभिः समावृता}
{अभिगम्य विशालाक्षी तस्थौ शोकपरिप्लुता} %5-24-18

\twolineshloka
{तां कृशां दीनवदनां मलिनाम्बरवासिनीम्}
{भर्त्सयाञ्चक्रिरे भीमा राक्षस्यस्ताः समन्ततः} %5-24-19

\twolineshloka
{ततस्तु विनता नाम राक्षसी भीमदर्शना}
{अब्रवीत् कुपिताकारा कराला निर्णतोदरी} %5-24-20

\twolineshloka
{सीते पर्याप्तमेतावद् भर्तुः स्नेहः प्रदर्शितः}
{सर्वत्रातिकृतं भद्रे व्यसनायोपकल्पते} %5-24-21

\twolineshloka
{परितुष्टास्मि भद्रं ते मानुषस्ते कृतो विधिः}
{ममापि तु वचः पथ्यं ब्रुवन्त्याः कुरु मैथिलि} %5-24-22

\twolineshloka
{रावणं भज भर्तारं भर्तारं सर्वरक्षसाम्}
{विक्रान्तमापतन्तं च सुरेशमिव वासवम्} %5-24-23

\twolineshloka
{दक्षिणं त्यागशीलं च सर्वस्य प्रियवादिनम्}
{मानुषं कृपणं रामं त्यक्त्वा रावणमाश्रय} %5-24-24

\twolineshloka
{दिव्याङ्गरागा वैदेहि दिव्याभरणभूषिता}
{अद्यप्रभृति लोकानां सर्वेषामीश्वरी भव} %5-24-25

\twolineshloka
{अग्नेः स्वाहा यथा देवी शची वेन्द्रस्य शोभने}
{किं ते रामेण वैदेहि कृपणेन गतायुषा} %5-24-26

\twolineshloka
{एतदुक्तं च मे वाक्यं यदि त्वं न करिष्यसि}
{अस्मिन् मुहूर्ते सर्वास्त्वां भक्षयिष्यामहे वयम्} %5-24-27

\twolineshloka
{अन्या तु विकटा नाम लम्बमानपयोधरा}
{अब्रवीत् कुपिता सीतां मुष्टिमुद्यम्य तर्जती} %5-24-28

\twolineshloka
{बहून्यप्रतिरूपाणि वचनानि सुदुर्मते}
{अनुक्रोशान्मृदुत्वाच्च सोढानि तव मैथिलि} %5-24-29

\twolineshloka
{न च नः कुरुषे वाक्यं हितं कालपुरस्कृतम्}
{आनीतासि समुद्रस्य पारमन्यैर्दुरासदम्} %5-24-30

\twolineshloka
{रावणान्तःपुरे घोरे प्रविष्टा चासि मैथिलि}
{रावणस्य गृहे रुद्धा अस्माभिस्त्वभिरक्षिता} %5-24-31

\twolineshloka
{न त्वां शक्तः परित्रातुमपि साक्षात् पुरन्दरः}
{कुरुष्व हितवादिन्या वचनं मम मैथिलि} %5-24-32

\twolineshloka
{अलमश्रुनिपातेन त्यज शोकमनर्थकम्}
{भज प्रीतिं प्रहर्षं च त्यजन्ती नित्यदैन्यताम्} %5-24-33

\twolineshloka
{सीते राक्षसराजेन परिक्रीड यथासुखम्}
{जानीमहे यथा भीरु स्त्रीणां यौवनमध्रुवम्} %5-24-34

\twolineshloka
{यावन्न ते व्यतिक्रामेत् तावत् सुखमवाप्नुहि}
{उद्यानानि च रम्याणि पर्वतोपवनानि च} %5-24-35

\twolineshloka
{सह राक्षसराजेन चर त्वं मदिरेक्षणे}
{स्त्रीसहस्राणि ते देवि वशे स्थास्यन्ति सुन्दरि} %5-24-36

\twolineshloka
{रावणं भज भर्तारं भर्तारं सर्वरक्षसाम्}
{उत्पाट्य वा ते हृदयं भक्षयिष्यामि मैथिलि} %5-24-37

\twolineshloka
{यदि मे व्याहृतं वाक्यं न यथावत् करिष्यसि}
{ततश्चण्डोदरी नाम राक्षसी क्रूरदर्शना} %5-24-38

\twolineshloka
{भ्रामयन्ती महच्छूलमिदं वचनमब्रवीत्}
{इमां हरिणशावाक्षीं त्रासोत्कम्पपयोधराम्} %5-24-39

\twolineshloka
{रावणेन हृतां दृष्ट्वा दौर्हृदो मे महानयम्}
{यकृत्प्लीहं महत् क्रोडं हृदयं च सबन्धनम्} %5-24-40

\twolineshloka
{गात्राण्यपि तथा शीर्षं खादेयमिति मे मतिः}
{ततस्तु प्रघसा नाम राक्षसी वाक्यमब्रवीत्} %5-24-41

\twolineshloka
{कण्ठमस्या नृशंसायाः पीडयामः किमास्यते}
{निवेद्यतां ततो राज्ञे मानुषी सा मृतेति ह} %5-24-42

\twolineshloka
{नात्र कश्चन सन्देहः खादतेति स वक्ष्यति}
{ततस्त्वजामुखी नाम राक्षसी वाक्यमब्रवीत्} %5-24-43

\twolineshloka
{विशस्येमां ततः सर्वान् समान् कुरुत पिण्डकान्}
{विभजाम ततः सर्वा विवादो मे न रोचते} %5-24-44

\twolineshloka
{पेयमानीयतां क्षिप्रं माल्यं च विविधं बहु}
{ततः शूर्पणखा नाम राक्षसी वाक्यमब्रवीत्} %5-24-45

\twolineshloka
{अजामुख्या यदुक्तं वै तदेव मम रोचते}
{सुरा चानीयतां क्षिप्रं सर्वशोकविनाशिनी} %5-24-46

\threelineshloka
{मानुषं मांसमास्वाद्य नृत्यामोऽथ निकुम्भिलाम्}
{एवं निर्भर्त्स्यमाना सा सीता सुरसुतोपमा}
{राक्षसीभिर्विरूपाभिर्धैर्यमुत्सृज्य रोदिति} %5-24-47


॥इत्यार्षे श्रीमद्रामायणे वाल्मीकीये आदिकाव्ये सुन्दरकाण्डे राक्षसीनिर्भर्त्सनम् नाम चतुर्विंशः सर्गः ॥५-२४॥
