\sect{अष्टपञ्चाशः सर्गः — हनूमद्वृत्तानुकथनम्}

\twolineshloka
{ततस्तस्य गिरेः शृङ्गे महेन्द्रस्य महाबलाः}
{हनुमत्प्रमुखाः प्रीतिं हरयो जग्मुरुत्तमाम्} %5-58-1

\twolineshloka
{प्रीतिमत्सूपविष्टेषु वानरेषु महात्मसु}
{तं ततः प्रतिसंहृष्टः प्रीतियुक्तं महाकपिम्} %5-58-2

\twolineshloka
{जाम्बवान् कार्यवृत्तान्तमपृच्छदनिलात्मजम्}
{कथं दृष्टा त्वया देवी कथं वा तत्र वर्तते} %5-58-3

\twolineshloka
{तस्यां चापि कथं वृत्तः क्रूरकर्मा दशाननः}
{तत्त्वतः सर्वमेतन्नः प्रब्रूहि त्वं महाकपे} %5-58-4

\twolineshloka
{सम्मार्गिता कथं देवी किं च सा प्रत्यभाषत}
{श्रुतार्थाश्चिन्तयिष्यामो भूयः कार्यविनिश्चयम्} %5-58-5

\twolineshloka
{यश्चार्थस्तत्र वक्तव्यो गतैरस्माभिरात्मवान्}
{रक्षितव्यं च यत्तत्र तद् भवान् व्याकरोतु नः} %5-58-6

\twolineshloka
{स नियुक्तस्ततस्तेन सम्प्रहृष्टतनूरुहः}
{नमस्यन् शिरसा देव्यै सीतायै प्रत्यभाषत} %5-58-7

\twolineshloka
{प्रत्यक्षमेव भवतां महेन्द्राग्रात् खमाप्लुतः}
{उदधेर्दक्षिणं पारं काङ्क्षमाणः समाहितः} %5-58-8

\twolineshloka
{गच्छतश्च हि मे घोरं विघ्नरूपमिवाभवत्}
{काञ्चनं शिखरं दिव्यं पश्यामि सुमनोहरम्} %5-58-9

\twolineshloka
{स्थितं पन्थानमावृत्य मेने विघ्नं च तं नगम्}
{उपसंगम्य तं दिव्यं काञ्चनं नगमुत्तमम्} %5-58-10

\twolineshloka
{कृता मे मनसा बुद्धिर्भेत्तव्योऽयं मयेति च}
{प्रहतस्य मया तस्य लाङ्गूलेन महागिरेः} %5-58-11

\twolineshloka
{शिखरं सूर्यसंकाशं व्यशीर्यत सहस्रधा}
{व्यवसायं च तं बुद्ध्वा स होवाच महागिरिः} %5-58-12

\twolineshloka
{पुत्रेति मधुरां वाणीं मनः प्रह्लादयन्निव}
{पितृव्यं चापि मां विद्धि सखायं मातरिश्वनः} %5-58-13

\twolineshloka
{मैनाकमिति विख्यातं निवसन्तं महोदधौ}
{पक्षवन्तः पुरा पुत्र बभूवुः पर्वतोत्तमाः} %5-58-14

\twolineshloka
{छन्दतः पृथिवीं चेरुर्बाधमानाः समन्ततः}
{श्रुत्वा नगानां चरितं महेन्द्रः पाकशासनः} %5-58-15

\twolineshloka
{वज्रेण भगवान् पक्षौ चिच्छेदैषां सहस्रशः}
{अहं तु मोचितस्तस्मात् तव पित्रा महात्मना} %5-58-16

\twolineshloka
{मारुतेन तदा वत्स प्रक्षिप्तो वरुणालये}
{राघवस्य मया साह्ये वर्तितव्यमरिंदम} %5-58-17

\twolineshloka
{रामो धर्मभृतां श्रेष्ठो महेन्द्रसमविक्रमः}
{एतच्छ्रुत्वा मया तस्य मैनाकस्य महात्मनः} %5-58-18

\twolineshloka
{कार्यमावेद्य च गिरेरुद्धतं वै मनो मम}
{तेन चाहमनुज्ञातो मैनाकेन महात्मना} %5-58-19

\twolineshloka
{स चाप्यन्तर्हितः शैलो मानुषेण वपुष्मता}
{शरीरेण महाशैलः शैलेन च महोदधौ} %5-58-20

\twolineshloka
{उत्तमं जवमास्थाय शेषमध्वानमास्थितः}
{ततोऽहं सुचिरं कालं जवेनाभ्यगमं पथि} %5-58-21

\twolineshloka
{ततः पश्याम्यहं देवीं सुरसां नागमातरम्}
{समुद्रमध्ये सा देवी वचनं चेदमब्रवीत्} %5-58-22

\twolineshloka
{मम भक्ष्यः प्रदिष्टस्त्वममरैर्हरिसत्तम}
{ततस्त्वां भक्षयिष्यामि विहितस्त्वं हि मे सुरैः} %5-58-23

\twolineshloka
{एवमुक्तः सुरसया प्राञ्जलिः प्रणतः स्थितः}
{विवर्णवदनो भूत्वा वाक्यं चेदमुदीरयम्} %5-58-24

\twolineshloka
{रामो दाशरथिः श्रीमान् प्रविष्टो दण्डकावनम्}
{लक्ष्मणेन सह भ्रात्रा सीतया च परंतपः} %5-58-25

\twolineshloka
{तस्य सीता हृता भार्या रावणेन दुरात्मना}
{तस्याः सकाशं दूतोऽहं गमिष्ये रामशासनात्} %5-58-26

\twolineshloka
{कर्तुमर्हसि रामस्य साहाय्यं विषये सती}
{अथवा मैथिलीं दृष्ट्वा रामं चाक्लिष्टकारिणम्} %5-58-27

\twolineshloka
{आगमिष्यामि ते वक्त्रं सत्यं प्रतिशृणोमि ते}
{एवमुक्ता मया सा तु सुरसा कामरूपिणी} %5-58-28

\twolineshloka
{अब्रवीन्नातिवर्तेत कश्चिदेष वरो मम}
{एवमुक्तः सुरसया दशयोजनमायतः} %5-58-29

\twolineshloka
{ततोऽर्धगुणविस्तारो बभूवाहं क्षणेन तु}
{मत्प्रमाणाधिकं चैव व्यादितं तु मुखं तया} %5-58-30

\twolineshloka
{तद् दृष्ट्वा व्यादितं त्वास्यं ह्रस्वं ह्यकरवं पुनः}
{तस्मिन् मुहूर्ते च पुनर्बभूवाङ्गुष्ठसम्मितः} %5-58-31

\twolineshloka
{अभिपत्याशु तद्वक्त्रं निर्गतोऽहं ततः क्षणात्}
{अब्रवीत् सुरसा देवी स्वेन रूपेण मां पुनः} %5-58-32

\twolineshloka
{अर्थसिद्धौ हरिश्रेष्ठ गच्छ सौम्य यथासुखम्}
{समानय च वैदेहीं राघवेण महात्मना} %5-58-33

\twolineshloka
{सुखी भव महाबाहो प्रीतास्मि तव वानर}
{ततोऽहं साधुसाध्वीति सर्वभूतैः प्रशंसितः} %5-58-34

\twolineshloka
{ततोऽन्तरिक्षं विपुलं प्लुतोऽहं गरुडो यथा}
{छाया मे निगृहीता च न च पश्यामि किंचन} %5-58-35

\twolineshloka
{सोऽहं विगतवेगस्तु दिशो दश विलोकयन्}
{न किंचित् तत्र पश्यामि येन मे विहता गतिः} %5-58-36

\twolineshloka
{अथ मे बुद्धिरुत्पन्ना किंनाम गमने मम}
{ईदृशो विघ्न उत्पन्नो रूपमत्र न दृश्यते} %5-58-37

\twolineshloka
{अधोभागे तु मे दृष्टिः शोचतः पतिता तदा}
{तत्राद्राक्षमहं भीमां राक्षसीं सलिलेशयाम्} %5-58-38

\twolineshloka
{प्रहस्य च महानादमुक्तोऽहं भीमया तया}
{अवस्थितमसम्भ्रान्तमिदं वाक्यमशोभनम्} %5-58-39

\twolineshloka
{क्वासि गन्ता महाकाय क्षुधिताया ममेप्सितः}
{भक्षः प्रीणय मे देहं चिरमाहारवर्जितम्} %5-58-40

\twolineshloka
{बाढमित्येव तां वाणीं प्रत्यगृह्णामहं ततः}
{आस्यप्रमाणादधिकं तस्याः कायमपूरयम्} %5-58-41

\twolineshloka
{तस्याश्चास्यं महद् भीमं वर्धते मम भक्षणे}
{न तु मां सा नु बुबुधे मम वा विकृतं कृतम्} %5-58-42

\twolineshloka
{ततोऽहं विपुलं रूपं संक्षिप्य निमिषान्तरात्}
{तस्या हृदयमादाय प्रपतामि नभःस्थलम्} %5-58-43

\twolineshloka
{सा विसृष्टभुजा भीमा पपात लवणाम्भसि}
{मया पर्वतसंकाशा निकृत्तहृदया सती} %5-58-44

\twolineshloka
{शृणोमि खगतानां च वाचः सौम्या महात्मनाम्}
{राक्षसी सिंहिका भीमा क्षिप्रं हनुमता हता} %5-58-45

\twolineshloka
{तां हत्वा पुनरेवाहं कृत्यमात्ययिकं स्मरन्}
{गत्वा च महदध्वानं पश्यामि नगमण्डितम्} %5-58-46

\twolineshloka
{दक्षिणं तीरमुदधेर्लङ्का यत्र गता पुरी}
{अस्तं दिनकरे याते रक्षसां निलयं पुरीम्} %5-58-47

\twolineshloka
{प्रविष्टोऽहमविज्ञातो रक्षोभिर्भीमविक्रमैः}
{तत्र प्रविशतश्चापि कल्पान्तघनसप्रभा} %5-58-48

\twolineshloka
{अट्टहासं विमुञ्चन्ती नारी काप्युत्थिता पुरः}
{जिघांसन्तीं ततस्तां तु ज्वलदग्निशिरोरुहाम्} %5-58-49

\twolineshloka
{सव्यमुष्टिप्रहारेण पराजित्य सुभैरवाम्}
{प्रदोषकाले प्रविशं भीतयाहं तयोदितः} %5-58-50

\twolineshloka
{अहं लङ्कापुरी वीर निर्जिता विक्रमेण ते}
{यस्मात् तस्माद् विजेतासि सर्वरक्षांस्यशेषतः} %5-58-51

\twolineshloka
{तत्राहं सर्वरात्रं तु विचरञ्जनकात्मजाम्}
{रावणान्तःपुरगतो न चापश्यं सुमध्यमाम्} %5-58-52

\twolineshloka
{ततः सीतामपश्यंस्तु रावणस्य निवेशने}
{शोकसागरमासाद्य न पारमुपलक्षये} %5-58-53

\twolineshloka
{शोचता च मया दृष्टं प्राकारेणाभिसंवृतम्}
{काञ्चनेन विकृष्टेन गृहोपवनमुत्तमम्} %5-58-54

\twolineshloka
{सप्राकारमवप्लुत्य पश्यामि बहुपादपम्}
{अशोकवनिकामध्ये शिंशपापादपो महान्} %5-58-55

\twolineshloka
{तमारुह्य च पश्यामि काञ्चनं कदलीवनम्}
{अदूराच्छिंशपावृक्षात् पश्यामि वरवर्णिनीम्} %5-58-56

\twolineshloka
{श्यामां कमलपत्राक्षीमुपवासकृशाननाम्}
{तदेकवासःसंवीतां रजोध्वस्तशिरोरुहाम्} %5-58-57

\twolineshloka
{शोकसंतापदीनाङ्गीं सीतां भर्तृहिते स्थिताम्}
{राक्षसीभिर्विरूपाभिः क्रूराभिरभिसंवृताम्} %5-58-58

\twolineshloka
{मांसशोणितभक्ष्याभिर्व्याघ्रीभिर्हरिणीं यथा}
{सा मया राक्षसीमध्ये तर्ज्यमाना मुहुर्मुहुः} %5-58-59

\twolineshloka
{एकवेणीधरा दीना भर्तृचिन्तापरायणा}
{भूमिशय्या विवर्णाङ्गी पद्मिनीव हिमागमे} %5-58-60

\twolineshloka
{रावणाद् विनिवृत्तार्था मर्तव्ये कृतनिश्चया}
{कथंचिन्मृगशावाक्षी तूर्णमासादिता मया} %5-58-61

\twolineshloka
{तां दृष्ट्वा तादृशीं नारीं रामपत्नीं यशस्विनीम्}
{तत्रैव शिंशपावृक्षे पश्यन्नहमवस्थितः} %5-58-62

\twolineshloka
{ततो हलहलाशब्दं काञ्चीनूपुरमिश्रितम्}
{शृणोम्यधिकगम्भीरं रावणस्य निवेशने} %5-58-63

\twolineshloka
{ततोऽहं परमोद्विग्नः स्वरूपं प्रत्यसंहरम्}
{अहं च शिंशपावृक्षे पक्षीव गहने स्थितः} %5-58-64

\twolineshloka
{ततो रावणदाराश्च रावणश्च महाबलः}
{तं देशमनुसम्प्राप्तो यत्र सीताभवत् स्थिता} %5-58-65

\twolineshloka
{तं दृष्ट्वाथ वरारोहा सीता रक्षोगणेश्वरम्}
{संकुच्योरू स्तनौ पीनौ बाहुभ्यां परिरभ्य च} %5-58-66

\twolineshloka
{वित्रस्तां परमोद्विग्नां वीक्ष्यमाणामितस्ततः}
{त्राणं कंचिदपश्यन्तीं वेपमानां तपस्विनीम्} %5-58-67

\twolineshloka
{तामुवाच दशग्रीवः सीतां परमदुःखिताम्}
{अवाक्शिराः प्रपतितो बहुमन्यस्व मामिति} %5-58-68

\twolineshloka
{यदि चेत्त्वं तु मां दर्पान्नाभिनन्दसि गर्विते}
{द्विमासानन्तरं सीते पास्यामि रुधिरं तव} %5-58-69

\twolineshloka
{एतच्छ्रुत्वा वचस्तस्य रावणस्य दुरात्मनः}
{उवाच परमक्रुद्धा सीता वचनमुत्तमम्} %5-58-70

\twolineshloka
{राक्षसाधम रामस्य भार्याममिततेजसः}
{इक्ष्वाकुवंशनाथस्य स्नुषां दशरथस्य च} %5-58-71

\twolineshloka
{अवाच्यं वदतो जिह्वा कथं न पतिता तव}
{किंस्विद्वीर्य तवानार्य यो मां भर्तुरसंनिधौ} %5-58-72

\twolineshloka
{अपहृत्यागतः पाप तेनादृष्टो महात्मना}
{न त्वं रामस्य सदृशो दास्येऽप्यस्य न युज्यसे} %5-58-73

\twolineshloka
{अजेयः सत्यवाक् शूरो रणश्लाघी च राघवः}
{जानक्या परुषं वाक्यमेवमुक्तो दशाननः} %5-58-74

\twolineshloka
{जज्वाल सहसा कोपाच्चितास्थ इव पावकः}
{विवृत्य नयने क्रूरे मुष्टिमुद्यम्य दक्षिणम्} %5-58-75

\twolineshloka
{मैथिलीं हन्तुमारब्धः स्त्रीभिर्हाहाकृतं तदा}
{स्त्रीणां मध्यात् समुत्पत्य तस्य भार्या दुरात्मनः} %5-58-76

\twolineshloka
{वरा मन्दोदरी नाम तया स प्रतिषेधितः}
{उक्तश्च मधुरां वाणीं तया स मदनार्दितः} %5-58-77

\twolineshloka
{सीतया तव किं कार्यं महेन्द्रसमविक्रम}
{मया सह रमस्वाद्य मद्विशिष्टा न जानकी} %5-58-78

\twolineshloka
{देवगन्धर्वकन्याभिर्यक्षकन्याभिरेव च}
{सार्धं प्रभो रमस्वेति सीतया किं करिष्यसि} %5-58-79

\twolineshloka
{ततस्ताभिः समेताभिर्नारीभिः स महाबलः}
{उत्थाप्य सहसा नीतो भवनं स्वं निशाचरः} %5-58-80

\twolineshloka
{याते तस्मिन् दशग्रीवे राक्षस्यो विकृताननाः}
{सीतां निर्भर्त्सयामासुर्वाक्यैः क्रूरैः सुदारुणैः} %5-58-81

\twolineshloka
{तृणवद् भाषितं तासां गणयामास जानकी}
{गर्जितं च तथा तासां सीतां प्राप्य निरर्थकम्} %5-58-82

\twolineshloka
{वृथा गर्जितनिश्चेष्टा राक्षस्यः पिशिताशनाः}
{रावणाय शशंसुस्ताः सीताव्यवसितं महत्} %5-58-83

\twolineshloka
{ततस्ताः सहिताः सर्वा विहताशा निरुद्यमाः}
{परिक्लिश्य समस्तास्ता निद्रावशमुपागताः} %5-58-84

\twolineshloka
{तासु चैव प्रसुप्तासु सीता भर्तृहिते रता}
{विलप्य करुणं दीना प्रशुशोच सुदुःखिता} %5-58-85

\twolineshloka
{तासां मध्यात् समुत्थाय त्रिजटा वाक्यमब्रवीत्}
{आत्मानं खादत क्षिप्रं न सीतामसितेक्षणाम्} %5-58-86

\twolineshloka
{जनकस्यात्मजां साध्वीं स्नुषां दशरथस्य च}
{स्वप्नो ह्यद्य मया दृष्टो दारुणो रोमहर्षणः} %5-58-87

\twolineshloka
{रक्षसां च विनाशाय भर्तुरस्या जयाय च}
{अलमस्मान् परित्रातुं राघवाद् राक्षसीगणम्} %5-58-88

\twolineshloka
{अभियाचाम वैदेहीमेतद्धि मम रोचते}
{यदि ह्येवंविधः स्वप्नो दुःखितायाः प्रदृश्यते} %5-58-89

\twolineshloka
{सा दुःखैर्विविधैर्मुक्ता सुखमाप्नोत्यनुत्तमम्}
{प्रणिपातप्रसन्ना हि मैथिली जनकात्मजा} %5-58-90

\twolineshloka
{अलमेषा परित्रातुं राक्षस्यो महतो भयात्}
{ततः सा ह्रीमती बाला भर्तुर्विजयहर्षिता} %5-58-91

\twolineshloka
{अवोचद् यदि तत् तथ्यं भवेयं शरणं हि वः}
{तां चाहं तादृशीं दृष्ट्वा सीताया दारुणां दशाम्} %5-58-92

\twolineshloka
{चिन्तयामास विश्रान्तो न च मे निर्वृतं मनः}
{सम्भाषणार्थे च मया जानक्याश्चिन्तितो विधिः} %5-58-93

\twolineshloka
{इक्ष्वाकुकुलवंशस्तु स्तुतो मम पुरस्कृतः}
{श्रुत्वा तु गदितां वाचं राजर्षिगणभूषिताम्} %5-58-94

\twolineshloka
{प्रत्यभाषत मां देवी बाष्पैः पिहितलोचना}
{कस्त्वं केन कथं चेह प्राप्तो वानरपुङ्गव} %5-58-95

\twolineshloka
{का च रामेण ते प्रीतिस्तन्मे शंसितुमर्हसि}
{तस्यास्तद् वचनं श्रुत्वा अहमप्यब्रुवं वचः} %5-58-96

\twolineshloka
{देवि रामस्य भर्तुस्ते सहायो भीमविक्रमः}
{सुग्रीवो नाम विक्रान्तो वानरेन्द्रो महाबलः} %5-58-97

\twolineshloka
{तस्य मां विद्धि भृत्यं त्वं हनूमन्तमिहागतम्}
{भर्त्रा सम्प्रहितस्तुभ्यं रामेणाक्लिष्टकर्मणा} %5-58-98

\twolineshloka
{इदं तु पुरुषव्याघ्रः श्रीमान् दाशरथिः स्वयम्}
{अङ्गुलीयमभिज्ञानमदात् तुभ्यं यशस्विनि} %5-58-99

\twolineshloka
{तदिच्छामि त्वयाज्ञप्तं देवि किं करवाण्यहम्}
{रामलक्ष्मणयोः पार्श्वं नयामि त्वां किमुत्तरम्} %5-58-100

\twolineshloka
{एतच्छ्रुत्वा विदित्वा च सीता जनकनन्दिनी}
{आह रावणमुत्पाट्य राघवो मां नयत्विति} %5-58-101

\twolineshloka
{प्रणम्य शिरसा देवीमहमार्यामनिन्दिताम्}
{राघवस्य मनोह्लादमभिज्ञानमयाचिषम्} %5-58-102

\twolineshloka
{अथ मामब्रवीत् सीता गृह्यतामयमुत्तमः}
{मणिर्येन महाबाहू रामस्त्वां बहु मन्यते} %5-58-103

\twolineshloka
{इत्युक्त्वा तु वरारोहा मणिप्रवरमुत्तमम्}
{प्रायच्छत् परमोद्विग्ना वाचा मां संदिदेश ह} %5-58-104

\twolineshloka
{ततस्तस्यै प्रणम्याहं राजपुत्र्यै समाहितः}
{प्रदक्षिणं परिक्राममिहाभ्युद्गतमानसः} %5-58-105

\twolineshloka
{उत्तरं पुनरेवाह निश्चित्य मनसा तदा}
{हनूमन् मम वृत्तान्तं वक्तुमर्हसि राघवे} %5-58-106

\twolineshloka
{यथा श्रुत्वैव नचिरात् तावुभौ रामलक्ष्मणौ}
{सुग्रीवसहितौ वीरावुपेयातां तथा कुरु} %5-58-107

\twolineshloka
{यदन्यथा भवेदेतद् द्वौ मासौ जीवितं मम}
{न मां द्रक्ष्यति काकुत्स्थो म्रिये साहमनाथवत्} %5-58-108

\twolineshloka
{तच्छ्रुत्वा करुणं वाक्यं क्रोधो मामभ्यवर्तत}
{उत्तरं च मया दृष्टं कार्यशेषमनन्तरम्} %5-58-109

\twolineshloka
{ततोऽवर्धत मे कायस्तदा पर्वतसंनिभः}
{युद्धाकाङ्क्षी वनं तस्य विनाशयितुमारभे} %5-58-110

\twolineshloka
{तद् भग्नं वनखण्डं तु भ्रान्तत्रस्तमृगद्विजम्}
{प्रतिबुद्ध्य निरीक्षन्ते राक्षस्यो विकृताननाः} %5-58-111

\twolineshloka
{मां च दृष्ट्वा वने तस्मिन् समागम्य ततस्ततः}
{ताः समभ्यागताः क्षिप्रं रावणायाचचक्षिरे} %5-58-112

\twolineshloka
{राजन् वनमिदं दुर्गं तव भग्नं दुरात्मना}
{वानरेण ह्यविज्ञाय तव वीर्यं महाबल} %5-58-113

\twolineshloka
{तस्य दुर्बुद्धिता राजंस्तव विप्रियकारिणः}
{वधमाज्ञापय क्षिप्रं यथासौ न पुनर्व्रजेत्} %5-58-114

\twolineshloka
{तच्छ्रुत्वा राक्षसेन्द्रेण विसृष्टा बहुदुर्जयाः}
{राक्षसाः किंकरा नाम रावणस्य मनोऽनुगाः} %5-58-115

\twolineshloka
{तेषामशीतिसाहस्रं शूलमुद्गरपाणिनाम्}
{मया तस्मिन् वनोद्देशे परिघेण निषूदितम्} %5-58-116

\twolineshloka
{तेषां तु हतशिष्टा ये ते गता लघुविक्रमाः}
{निहतं च मया सैन्यं रावणायाचचक्षिरे} %5-58-117

\twolineshloka
{ततो मे बुद्धिरुत्पन्ना चैत्यप्रासादमुत्तमम्}
{तत्रस्थान् राक्षसान् हत्वा शतं स्तम्भेन वै पुनः} %5-58-118

\twolineshloka
{ललामभूतो लङ्काया मया विध्वंसितो रुषा}
{ततः प्रहस्तस्य सुतं जम्बुमालिनमादिशत्} %5-58-119

\twolineshloka
{राक्षसैर्बहुभिः सार्धं घोररूपैर्भयानकैः}
{तमहं बलसम्पन्नं राक्षसं रणकोविदम्} %5-58-120

\twolineshloka
{परिघेणातिघोरेण सूदयामि सहानुगम्}
{तच्छ्रुत्वा राक्षसेन्द्रस्तु मन्त्रिपुत्रान् महाबलान्} %5-58-121

\twolineshloka
{पदातिबलसम्पन्नान् प्रेषयामास रावणः}
{परिघेणैव तान् सर्वान् नयामि यमसादनम्} %5-58-122

\twolineshloka
{मन्त्रिपुत्रान् हतान् श्रुत्वा समरे लघुविक्रमान्}
{पञ्च सेनाग्रगान् शूरान् प्रेषयामास रावणः} %5-58-123

\twolineshloka
{तानहं सहसैन्यान् वै सर्वानेवाभ्यसूदयम्}
{ततः पुनर्दशग्रीवः पुत्रमक्षं महाबलम्} %5-58-124

\twolineshloka
{बहुभी राक्षसैः सार्धं प्रेषयामास संयुगे}
{तं तु मन्दोदरीपुत्रं कुमारं रणपण्डितम्} %5-58-125

\twolineshloka
{सहसा खं समुद्यन्तं पादयोश्च गृहीतवान्}
{तमासीनं शतगुणं भ्रामयित्वा व्यपेषयम्} %5-58-126

\twolineshloka
{तमक्षमागतं भग्नं निशम्य स दशाननः}
{ततश्चेन्द्रजितं नाम द्वितीयं रावणः सुतम्} %5-58-127

\twolineshloka
{व्यादिदेश सुसंक्रुद्धो बलिनं युद्धदुर्मदम्}
{तच्चाप्यहं बलं सर्वं तं च राक्षसपुङ्गवम्} %5-58-128

\twolineshloka
{नष्टौजसं रणे कृत्वा परं हर्षमुपागतः}
{महतापि महाबाहुः प्रत्ययेन महाबलः} %5-58-129

\twolineshloka
{प्रहितो रावणेनैष सह वीरैर्मदोद्धतैः}
{सोऽविषह्यं हि मां बुद्ध्वा स्वसैन्यं चावमर्दितम्} %5-58-130

\twolineshloka
{ब्रह्मणोऽस्त्रेण स तु मां प्रबद्ध्वा चातिवेगिनः}
{रज्जुभिश्चापि बध्नन्ति ततो मां तत्र राक्षसाः} %5-58-131

\twolineshloka
{रावणस्य समीपं च गृहीत्वा मामुपागमन्}
{दृष्ट्वा सम्भाषितश्चाहं रावणेन दुरात्मना} %5-58-132

\twolineshloka
{पृष्टश्च लङ्कागमनं राक्षसानां च तं वधम्}
{तत्सर्वं च रणे तत्र सीतार्थमुपजल्पितम्} %5-58-133

\twolineshloka
{तस्यास्तु दर्शनाकाङ्क्षी प्राप्तस्त्वद्भवनं विभो}
{मारुतस्यौरसः पुत्रो वानरो हनुमानहम्} %5-58-134

\twolineshloka
{रामदूतं च मां विद्धि सुग्रीवसचिवं कपिम्}
{सोऽहं दौत्येन रामस्य त्वत्सकाशमिहागतः} %5-58-135

\twolineshloka
{शृणु चापि समादेशं यदहं प्रब्रवीमि ते}
{राक्षसेश हरीशस्त्वां वाक्यमाह समाहितम्} %5-58-136

\twolineshloka
{सुग्रीवश्च महाभागः स त्वां कौशलमब्रवीत्}
{धर्मार्थकामसहितं हितं पथ्यमुवाच ह} %5-58-137

\twolineshloka
{वसतो ऋष्यमूके मे पर्वते विपुलद्रुमे}
{राघवो रणविक्रान्तो मित्रत्वं समुपागतः} %5-58-138

\twolineshloka
{तेन मे कथितं राजन् भार्या मे रक्षसा हृता}
{तत्र साहाय्यहेतोर्मे समयं कर्तुमर्हसि} %5-58-139

\twolineshloka
{वालिना हृतराज्येन सुग्रीवेण सह प्रभुः}
{चक्रेऽग्निसाक्षिकं सख्यं राघवः सहलक्ष्मणः} %5-58-140

\twolineshloka
{तेन वालिनमाहत्य शरेणैकेन संयुगे}
{वानराणां महाराजः कृतः सम्प्लवतां प्रभुः} %5-58-141

\twolineshloka
{तस्य साहाय्यमस्माभिः कार्यं सर्वात्मना त्विह}
{तेन प्रस्थापितस्तुभ्यं समीपमिह धर्मतः} %5-58-142

\twolineshloka
{क्षिप्रमानीयतां सीता दीयतां राघवस्य च}
{यावन्न हरयो वीरा विधमन्ति बलं तव} %5-58-143

\twolineshloka
{वानराणां प्रभावोऽयं न केन विदितः पुरा}
{देवतानां सकाशं च ये गच्छन्ति निमन्त्रिताः} %5-58-144

\twolineshloka
{इति वानरराजस्त्वामाहेत्यभिहितो मया}
{मामैक्षत ततो रुष्टश्चक्षुषा प्रदहन्निव} %5-58-145

\twolineshloka
{तेन वध्योऽहमाज्ञप्तो रक्षसा रौद्रकर्मणा}
{मत्प्रभावमविज्ञाय रावणेन दुरात्मना} %5-58-146

\twolineshloka
{ततो विभीषणो नाम तस्य भ्राता महामतिः}
{तेन राक्षसराजश्च याचितो मम कारणात्} %5-58-147

\twolineshloka
{नैवं राक्षसशार्दूल त्यज्यतामेष निश्चयः}
{राजशास्त्रव्यपेतो हि मार्गः संलक्ष्यते त्वया} %5-58-148

\twolineshloka
{दूतवध्या न दृष्टा हि राजशास्त्रेषु राक्षस}
{दूतेन वेदितव्यं च यथाभिहितवादिना} %5-58-149

\twolineshloka
{सुमहत्यपराधेऽपि दूतस्यातुलविक्रम}
{विरूपकरणं दृष्टं न वधोऽस्ति हि शास्त्रतः} %5-58-150

\twolineshloka
{विभीषणेनैवमुक्तो रावणः संदिदेश तान्}
{राक्षसानेतदेवाद्य लाङ्गूलं दह्यतामिति} %5-58-151

\twolineshloka
{ततस्तस्य वचः श्रुत्वा मम पुच्छं समन्ततः}
{वेष्टितं शणवल्कैश्च पट्टैः कार्पासकैस्तथा} %5-58-152

\twolineshloka
{राक्षसाः सिद्धसंनाहास्ततस्ते चण्डविक्रमाः}
{तदादीप्यन्त मे पुच्छं हनन्तः काष्ठमुष्टिभिः} %5-58-153

\twolineshloka
{बद्धस्य बहुभिः पाशैर्यन्त्रितस्य च राक्षसैः}
{न मे पीडाभवत् काचिद् दिदृक्षोर्नगरीं दिवा} %5-58-154

\twolineshloka
{ततस्ते राक्षसाः शूरा बद्धं मामग्निसंवृतम्}
{अघोषयन् राजमार्गे नगरद्वारमागताः} %5-58-155

\twolineshloka
{ततोऽहं सुमहद्रूपं संक्षिप्य पुनरात्मनः}
{विमोचयित्वा तं बन्धं प्रकृतिस्थः स्थितः पुनः} %5-58-156

\twolineshloka
{आयसं परिघं गृह्य तानि रक्षांस्यसूदयम्}
{ततस्तन्नगरद्वारं वेगेन प्लुतवानहम्} %5-58-157

\twolineshloka
{पुच्छेन च प्रदीप्तेन तां पुरीं साट्टगोपुराम्}
{दहाम्यहमसम्भ्रान्तो युगान्ताग्निरिव प्रजाः} %5-58-158

\twolineshloka
{विनष्टा जानकी व्यक्तं न ह्यदग्धः प्रदृश्यते}
{लङ्कायाः कश्चिदुद्देशः सर्वा भस्मीकृता पुरी} %5-58-159

\twolineshloka
{दहता च मया लङ्कां दग्धा सीता न संशयः}
{रामस्य च महत्कार्यं मयेदं विफलीकृतम्} %5-58-160

\twolineshloka
{इति शोकसमाविष्टश्चिन्तामहमुपागतः}
{ततोऽहं वाचमश्रौषं चारणानां शुभाक्षराम्} %5-58-161

\twolineshloka
{जानकी न च दग्धेति विस्मयोदन्तभाषिणाम्}
{ततो मे बुद्धिरुत्पन्ना श्रुत्वा तामद्भुतां गिरम्} %5-58-162

\twolineshloka
{अदग्धा जानकीत्येव निमित्तैश्चोपलक्षितम्}
{दीप्यमाने तु लाङ्गूले न मां दहति पावकः} %5-58-163

\twolineshloka
{हृदयं च प्रहृष्टं मे वाताः सुरभिगन्धिनः}
{तैर्निमित्तैश्च दृष्टार्थैः कारणैश्च महागुणैः} %5-58-164

\twolineshloka
{ऋषिवाक्यैश्च दृष्टार्थैरभवं हृष्टमानसः}
{पुनर्दृष्टा च वैदेही विसृष्टश्च तया पुनः} %5-58-165

\twolineshloka
{ततः पर्वतमासाद्य तत्रारिष्टमहं पुनः}
{प्रतिप्लवनमारेभे युष्मद्दर्शनकाङ्क्षया} %5-58-166

\twolineshloka
{ततः श्वसनचन्द्रार्कसिद्धगन्धर्वसेवितम्}
{पन्थानमहमाक्रम्य भवतो दृष्टवानिह} %5-58-167

\twolineshloka
{राघवस्य प्रसादेन भवतां चैव तेजसा}
{सुग्रीवस्य च कार्यार्थं मया सर्वमनुष्ठितम्} %5-58-168

\twolineshloka
{एतत् सर्वं मया तत्र यथावदुपपादितम्}
{तत्र यन्न कृतं शेषं तत् सर्वं क्रियतामिति} %5-58-169


॥इत्यार्षे श्रीमद्रामायणे वाल्मीकीये आदिकाव्ये सुन्दरकाण्डे हनूमद्वृत्तानुकथनम् नाम अष्टपञ्चाशः सर्गः ॥५-५८॥
