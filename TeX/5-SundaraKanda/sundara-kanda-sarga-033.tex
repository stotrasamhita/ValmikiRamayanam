\sect{त्रयस्त्रिंशः सर्गः — हनूमञ्जानकीसंवादोपक्रमः}

\twolineshloka
{सोऽवतीर्य द्रुमात् तस्माद् विद्रुमप्रतिमाननः}
{विनीतवेषः कृपणः प्रणिपत्योपसृत्य च} %5-33-1

\twolineshloka
{तामब्रवीन्महातेजा हनूमान् मारुतात्मजः}
{शिरस्यञ्जलिमाधाय सीतां मधुरया गिरा} %5-33-2

\twolineshloka
{का नु पद्मपलाशाक्षि क्लिष्टकौशेयवासिनि}
{द्रुमस्य शाखामालम्ब्य तिष्ठसि त्वमनिन्दिते} %5-33-3

\twolineshloka
{किमर्थं तव नेत्राभ्यां वारि स्रवति शोकजम्}
{पुण्डरीकपलाशाभ्यां विप्रकीर्णमिवोदकम्} %5-33-4

\twolineshloka
{सुराणामसुराणां च नागगन्धर्वरक्षसाम्}
{यक्षाणां किन्नराणां च का त्वं भवसि शोभने} %5-33-5

\twolineshloka
{का त्वं भवसि रुद्राणां मरुतां वा वरानने}
{वसूनां वा वरारोहे देवता प्रतिभासि मे} %5-33-6

\twolineshloka
{किं नु चन्द्रमसा हीना पतिता विबुधालयात्}
{रोहिणी ज्योतिषां श्रेष्ठा श्रेष्ठा सर्वगुणाधिका} %5-33-7

\twolineshloka
{कोपाद् वा यदि वा मोहाद् भर्तारमसितेक्षणे}
{वसिष्ठं कोपयित्वा त्वं वासि कल्याण्यरुन्धती} %5-33-8

\twolineshloka
{को नु पुत्रः पिता भ्राता भर्ता वा ते सुमध्यमे}
{अस्माल्लोकादमुं लोकं गतं त्वमनुशोचसि} %5-33-9

\twolineshloka
{रोदनादतिनिःश्वासाद् भूमिसंस्पर्शनादपि}
{न त्वां देवीमहं मन्ये राज्ञः संज्ञावधारणात्} %5-33-10

\twolineshloka
{व्यञ्जनानि हि ते यानि लक्षणानि च लक्षये}
{महिषी भूमिपालस्य राजकन्या च मे मता} %5-33-11

\twolineshloka
{रावणेन जनस्थानाद् बलात् प्रमथिता यदि}
{सीता त्वमसि भद्रं ते तन्ममाचक्ष्व पृच्छतः} %5-33-12

\twolineshloka
{यथा हि तव वै दैन्यं रूपं चाप्यतिमानुषम्}
{तपसा चान्वितो वेषस्त्वं राममहिषी ध्रुवम्} %5-33-13

\twolineshloka
{सा तस्य वचनं श्रुत्वा रामकीर्तनहर्षिता}
{उवाच वाक्यं वैदेही हनूमन्तं द्रुमाश्रितम्} %5-33-14

\twolineshloka
{पृथिव्यां राजसिंहानां मुख्यस्य विदितात्मनः}
{स्नुषा दशरथस्याहं शत्रुसैन्यप्रणाशिनः} %5-33-15

\twolineshloka
{दुहिता जनकस्याहं वैदेहस्य महात्मनः}
{सीतेति नाम्ना चोक्ताहं भार्या रामस्य धीमतः} %5-33-16

\twolineshloka
{समा द्वादश तत्राहं राघवस्य निवेशने}
{भुञ्जाना मानुषान् भोगान् सर्वकामसमृद्धिनी} %5-33-17

\twolineshloka
{ततस्त्रयोदशे वर्षे राज्ये चेक्ष्वाकुनन्दनम्}
{अभिषेचयितुं राजा सोपाध्यायः प्रचक्रमे} %5-33-18

\twolineshloka
{तस्मिन् सम्भ्रियमाणे तु राघवस्याभिषेचने}
{कैकेयी नाम भर्तारमिदं वचनमब्रवीत्} %5-33-19

\twolineshloka
{न पिबेयं न खादेयं प्रत्यहं मम भोजनम्}
{एष मे जीवितस्यान्तो रामो यद्यभिषिच्यते} %5-33-20

\twolineshloka
{यत् तदुक्तं त्वया वाक्यं प्रीत्या नृपतिसत्तम}
{तच्चेन्न वितथं कार्यं वनं गच्छतु राघवः} %5-33-21

\twolineshloka
{स राजा सत्यवाग् देव्या वरदानमनुस्मरन्}
{मुमोह वचनं श्रुत्वा कैकेय्याः क्रूरमप्रियम्} %5-33-22

\twolineshloka
{ततस्तं स्थविरो राजा सत्यधर्मे व्यवस्थितः}
{ज्येष्ठं यशस्विनं पुत्रं रुदन् राज्यमयाचत} %5-33-23

\twolineshloka
{स पितुर्वचनं श्रीमानभिषेकात् परं प्रियम्}
{मनसा पूर्वमासाद्य वाचा प्रतिगृहीतवान्} %5-33-24

\twolineshloka
{दद्यान्न प्रतिगृह्णीयात् सत्यं ब्रूयान्न चानृतम्}
{अपि जीवितहेतोर्हि रामः सत्यपराक्रमः} %5-33-25

\twolineshloka
{स विहायोत्तरीयाणि महार्हाणि महायशाः}
{विसृज्य मनसा राज्यं जनन्यै मां समादिशत्} %5-33-26

\twolineshloka
{साहं तस्याग्रतस्तूर्णं प्रस्थिता वनचारिणी}
{नहि मे तेन हीनाया वासः स्वर्गेऽपि रोचते} %5-33-27

\twolineshloka
{प्रागेव तु महाभागः सौमित्रिर्मित्रनन्दनः}
{पूर्वजस्यानुयात्रार्थे कुशचीरैरलङ्कृतः} %5-33-28

\twolineshloka
{ते वयं भर्तुरादेशं बहुमान्य दृढव्रताः}
{प्रविष्टाः स्म पुरादृष्टं वनं गम्भीरदर्शनम्} %5-33-29

\twolineshloka
{वसतो दण्डकारण्ये तस्याहममितौजसः}
{रक्षसापहृता भार्या रावणेन दुरात्मना} %5-33-30

\twolineshloka
{द्वौ मासौ तेन मे कालो जीवितानुग्रहः कृतः}
{ऊर्ध्वं द्वाभ्यां तु मासाभ्यां ततस्त्यक्ष्यामि जीवितम्} %5-33-31


॥इत्यार्षे श्रीमद्रामायणे वाल्मीकीये आदिकाव्ये सुन्दरकाण्डे हनूमञ्जानकीसंवादोपक्रमः नाम त्रयस्त्रिंशः सर्गः ॥५-३३॥
