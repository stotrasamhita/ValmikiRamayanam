\sect{पञ्चदशः सर्गः — सीतोपलम्भः}

\twolineshloka
{स वीक्षमाणस्तत्रस्थो मार्गमाणश्च मैथिलीम्}
{अवेक्षमाणश्च महीं सर्वां तामन्ववैक्षत} %5-15-1

\twolineshloka
{सन्तानकलताभिश्च पादपैरुपशोभिताम्}
{दिव्यगन्धरसोपेतां सर्वतः समलङ्कृताम्} %5-15-2

\twolineshloka
{तां स नन्दनसङ्काशां मृगपक्षिभिरावृताम्}
{हर्म्यप्रासादसम्बाधां कोकिलाकुलनिःस्वनाम्} %5-15-3

\twolineshloka
{काञ्चनोत्पलपद्माभिर्वापीभिरुपशोभिताम्}
{बह्वासनकुथोपेतां बहुभूमिगृहायुताम्} %5-15-4

\twolineshloka
{सर्वर्तुकुसुमै रम्यैः फलवद्भिश्च पादपैः}
{पुष्पितानामशोकानां श्रिया सूर्योदयप्रभाम्} %5-15-5

\twolineshloka
{प्रदीप्तामिव तत्रस्थो मारुतिः समुदैक्षत}
{निष्पत्रशाखां विहगैः क्रियमाणामिवासकृत्} %5-15-6

\twolineshloka
{विनिष्पतद्भिः शतशश्चित्रैः पुष्पावतंसकैः}
{समूलपुष्परचितैरशोकैः शोकनाशनैः} %5-15-7

\twolineshloka
{पुष्पभारातिभारैश्च स्पृशद्भिरिव मेदिनीम्}
{कर्णिकारैः कुसुमितैः किंशुकैश्च सुपुष्पितैः} %5-15-8

\twolineshloka
{स देशः प्रभया तेषां प्रदीप्त इव सर्वतः}
{पुन्नागाः सप्तपर्णाश्च चम्पकोद्दालकास्तथा} %5-15-9

\twolineshloka
{विवृद्धमूला बहवः शोभन्ते स्म सुपुष्पिताः}
{शातकुम्भनिभाः केचित् केचिदग्निशिखप्रभाः} %5-15-10

\twolineshloka
{नीलाञ्जननिभाः केचित् तत्राशोकाः सहस्रशः}
{नन्दनं विबुधोद्यानं चित्रं चैत्ररथं यथा} %5-15-11

\twolineshloka
{अतिवृत्तमिवाचिन्त्यं दिव्यं रम्यश्रियायुतम्}
{द्वितीयमिव चाकाशं पुष्पज्योतिर्गणायुतम्} %5-15-12

\twolineshloka
{पुष्परत्नशतैश्चित्रं पञ्चमं सागरं यथा}
{सर्वर्तुपुष्पैर्निचितं पादपैर्मधुगन्धिभिः} %5-15-13

\twolineshloka
{नानानिनादैरुद्यानं रम्यं मृगगणद्विजैः}
{अनेकगन्धप्रवहं पुण्यगन्धं मनोहरम्} %5-15-14

\twolineshloka
{शैलेन्द्रमिव गन्धाढ्यं द्वितीयं गन्धमादनम्}
{अशोकवनिकायां तु तस्यां वानरपुङ्गवः} %5-15-15

\twolineshloka
{स ददर्शाविदूरस्थं चैत्यप्रासादमूर्जितम्}
{मध्ये स्तम्भसहस्रेण स्थितं कैलासपाण्डुरम्} %5-15-16

\twolineshloka
{प्रवालकृतसोपानं तप्तकाञ्चनवेदिकम्}
{मुष्णन्तमिव चक्षूंषि द्योतमानमिव श्रिया} %5-15-17

\twolineshloka
{निर्मलं प्रांशुभावत्वादुल्लिखन्तमिवाम्बरम्}
{ततो मलिनसंवीतां राक्षसीभिः समावृताम्} %5-15-18

\twolineshloka
{उपवासकृशां दीनां निःश्वसन्तीं पुनः पुनः}
{ददर्श शुक्लपक्षादौ चन्द्ररेखामिवामलाम्} %5-15-19

\twolineshloka
{मन्दप्रख्यायमानेन रूपेण रुचिरप्रभाम्}
{पिनद्धां धूमजालेन शिखामिव विभावसोः} %5-15-20

\twolineshloka
{पीतेनैकेन संवीतां क्लिष्टेनोत्तमवाससा}
{सपङ्कामनलङ्कारां विपद्मामिव पद्मिनीम्} %5-15-21

\twolineshloka
{पीडितां दुःखसन्तप्तां परिक्षीणां तपस्विनीम्}
{ग्रहेणाङ्गारकेणेव पीडितामिव रोहिणीम्} %5-15-22

\twolineshloka
{अश्रुपूर्णमुखीं दीनां कृशामनशनेन च}
{शोकध्यानपरां दीनां नित्यं दुःखपरायणाम्} %5-15-23

\twolineshloka
{प्रियं जनमपश्यन्तीं पश्यन्तीं राक्षसीगणम्}
{स्वगणेन मृगीं हीनां श्वगणेनावृतामिव} %5-15-24

\twolineshloka
{नीलनागाभया वेण्या जघनं गतयैकया}
{नीलया नीरदापाये वनराज्या महीमिव} %5-15-25

\twolineshloka
{सुखार्हां दुःखसन्तप्तां व्यसनानामकोविदाम्}
{तां विलोक्य विशालाक्षीमधिकं मलिनां कृशाम्} %5-15-26

\twolineshloka
{तर्कयामास सीतेति कारणैरुपपादिभिः}
{ह्रियमाणा तदा तेन रक्षसा कामरूपिणा} %5-15-27

\twolineshloka
{यथारूपा हि दृष्टा सा तथारूपेयमङ्गना}
{पूर्णचन्द्राननां सुभ्रूं चारुवृत्तपयोधराम्} %5-15-28

\twolineshloka
{कुर्वतीं प्रभया देवीं सर्वा वितिमिरा दिशः}
{तां नीलकण्ठीं बिम्बोष्ठीं सुमध्यां सुप्रतिष्ठिताम्} %5-15-29

\twolineshloka
{सीतां पद्मपलाशाक्षीं मन्मथस्य रतिं यथा}
{इष्टां सर्वस्य जगतः पूर्णचन्द्रप्रभामिव} %5-15-30

\twolineshloka
{भूमौ सुतनुमासीनां नियतामिव तापसीम्}
{निःश्वासबहुलां भीरुं भुजगेन्द्रवधूमिव} %5-15-31

\twolineshloka
{शोकजालेन महता विततेन न राजतीम्}
{संसक्तां धूमजालेन शिखामिव विभावसोः} %5-15-32

\twolineshloka
{तां स्मृतीमिव सन्दिग्धामृद्धिं निपतितामिव}
{विहतामिव च श्रद्धामाशां प्रतिहतामिव} %5-15-33

\twolineshloka
{सोपसर्गां यथा सिद्धिं बुद्धिं सकलुषामिव}
{अभूतेनापवादेन कीर्तिं निपतितामिव} %5-15-34

\twolineshloka
{रामोपरोधव्यथितां रक्षोगणनिपीडिताम्}
{अबलां मृगशावाक्षीं वीक्षमाणां ततस्ततः} %5-15-35

\twolineshloka
{बाष्पाम्बुपरिपूर्णेन कृष्णवक्राक्षिपक्ष्मणा}
{वदनेनाप्रसन्नेन निःश्वसन्तीं पुनः पुनः} %5-15-36

\twolineshloka
{मलपङ्कधरां दीनां मण्डनार्हाममण्डिताम्}
{प्रभां नक्षत्रराजस्य कालमेघैरिवावृताम्} %5-15-37

\twolineshloka
{तस्य सन्दिदिहे बुद्धिस्तथा सीतां निरीक्ष्य च}
{आम्नायानामयोगेन विद्यां प्रशिथिलामिव} %5-15-38

\twolineshloka
{दुःखेन बुबुधे सीतां हनुमाननलङ्कृताम्}
{संस्कारेण यथा हीनां वाचमर्थान्तरं गताम्} %5-15-39

\twolineshloka
{तां समीक्ष्य विशालाक्षीं राजपुत्रीमनिन्दिताम्}
{तर्कयामास सीतेति कारणैरुपपादयन्} %5-15-40

\twolineshloka
{वैदेह्या यानि चाङ्गेषु तदा रामोऽन्वकीर्तयत्}
{तान्याभरणजालानि गात्रशोभीन्यलक्षयत्} %5-15-41

\twolineshloka
{सुकृतौ कर्णवेष्टौ च श्वदंष्ट्रौ च सुसंस्थितौ}
{मणिविद्रुमचित्राणि हस्तेष्वाभरणानि च} %5-15-42

\twolineshloka
{श्यामानि चिरयुक्तत्वात् तथा संस्थानवन्ति च}
{तान्येवैतानि मन्येऽहं यानि रामोऽन्वकीर्तयत्} %5-15-43

\twolineshloka
{तत्र यान्यवहीनानि तान्यहं नोपलक्षये}
{यान्यस्या नावहीनानि तानीमानि न संशयः} %5-15-44

\twolineshloka
{पीतं कनकपट्टाभं स्रस्तं तद्वसनं शुभम्}
{उत्तरीयं नगासक्तं तदा दृष्टं प्लवङ्गमैः} %5-15-45

\twolineshloka
{भूषणानि च मुख्यानि दृष्टानि धरणीतले}
{अनयैवापविद्धानि स्वनवन्ति महान्ति च} %5-15-46

\twolineshloka
{इदं चिरगृहीतत्वाद् वसनं क्लिष्टवत्तरम्}
{तथाप्यनूनं तद्वर्णं तथा श्रीमद्यथेतरत्} %5-15-47

\twolineshloka
{इयं कनकवर्णाङ्गी रामस्य महिषी प्रिया}
{प्रणष्टापि सती यस्य मनसो न प्रणश्यति} %5-15-48

\twolineshloka
{इयं सा यत्कृते रामश्चतुर्भिरिह तप्यते}
{कारुण्येनानृशंस्येन शोकेन मदनेन च} %5-15-49

\twolineshloka
{स्त्री प्रणष्टेति कारुण्यादाश्रितेत्यानृशंस्यतः}
{पत्नी नष्टेति शोकेन प्रियेति मदनेन च} %5-15-50

\twolineshloka
{अस्या देव्या यथारूपमङ्गप्रत्यङ्गसौष्ठवम्}
{रामस्य च यथारूपं तस्येयमसितेक्षणा} %5-15-51

\twolineshloka
{अस्या देव्या मनस्तस्मिंस्तस्य चास्यां प्रतिष्ठितम्}
{तेनेयं स च धर्मात्मा मुहूर्तमपि जीवति} %5-15-52

\twolineshloka
{दुष्करं कृतवान् रामो हीनो यदनया प्रभुः}
{धारयत्यात्मनो देहं न शोकेनावसीदति} %5-15-53

\twolineshloka
{एवं सीतां तथा दृष्ट्वा हृष्टः पवनसम्भवः}
{जगाम मनसा रामं प्रशशंस च तं प्रभुम्} %5-15-54


॥इत्यार्षे श्रीमद्रामायणे वाल्मीकीये आदिकाव्ये सुन्दरकाण्डे सीतोपलम्भः नाम पञ्चदशः सर्गः ॥५-१५॥
