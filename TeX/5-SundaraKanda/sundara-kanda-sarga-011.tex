\sect{एकादशः सर्गः — पानभूमिविचयः}

\twolineshloka
{अवधूय च तां बुद्धिं बभूवावस्थितस्तदा}
{जगाम चापरां चिन्तां सीतां प्रति महाकपिः} %5-11-1

\twolineshloka
{न रामेण वियुक्ता सा स्वप्तुमर्हति भामिनी}
{न भोक्तुं नाप्यलङ्कर्तुं न पानमुपसेवितुम्} %5-11-2

\twolineshloka
{नान्यं नरमुपस्थातुं सुराणामपि चेश्वरम्}
{न हि रामसमः कश्चिद् विद्यते त्रिदशेष्वपि} %5-11-3

\twolineshloka
{अन्येयमिति निश्चित्य भूयस्तत्र चचार सः}
{पानभूमौ हरिश्रेष्ठः सीतासन्दर्शनोत्सुकः} %5-11-4

\twolineshloka
{क्रीडितेनापराः क्लान्ता गीतेन च तथापराः}
{नृत्येन चापराः क्लान्ताः पानविप्रहतास्तथा} %5-11-5

\twolineshloka
{मुरजेषु मृदङ्गेषु चेलिकासु च संस्थिताः}
{तथाऽऽस्तरणमुख्येषु संविष्टाश्चापराः स्त्रियः} %5-11-6

\twolineshloka
{अङ्गनानां सहस्रेण भूषितेन विभूषणैः}
{रूपसंलापशीलेन युक्तगीतार्थभाषिणा} %5-11-7

\twolineshloka
{देशकालाभियुक्तेन युक्तवाक्याभिधायिना}
{रताधिकेन संयुक्तां ददर्श हरियूथपः} %5-11-8

\twolineshloka
{अन्यत्रापि वरस्त्रीणां रूपसंलापशायिनाम्}
{सहस्रं युवतीनां तु प्रसुप्तं स ददर्श ह} %5-11-9

\twolineshloka
{देशकालाभियुक्तं तु युक्तवाक्याभिधायि तत्}
{रताविरतसंसुप्तं ददर्श हरियूथपः} %5-11-10

\twolineshloka
{तासां मध्ये महाबाहुः शुशुभे राक्षसेश्वरः}
{गोष्ठे महति मुख्यानां गवां मध्ये यथा वृषः} %5-11-11

\twolineshloka
{स राक्षसेन्द्रः शुशुभे ताभिः परिवृतः स्वयम्}
{करेणुभिर्यथारण्ये परिकीर्णो महाद्विपः} %5-11-12

\twolineshloka
{सर्वकामैरुपेतां च पानभूमिं महात्मनः}
{ददर्श कपिशार्दूलस्तस्य रक्षःपतेर्गृहे} %5-11-13

\twolineshloka
{मृगाणां महिषाणां च वराहाणां च भागशः}
{तत्र न्यस्तानि मांसानि पानभूमौ ददर्श सः} %5-11-14

\twolineshloka
{रौक्मेषु च विशालेषु भाजनेष्वप्यभक्षितान्}
{ददर्श कपिशार्दूलो मयूरान् कुक्कुटांस्तथा} %5-11-15

\twolineshloka
{वराहवाध्रीणसकान् दधिसौवर्चलायुतान्}
{शल्यान् मृगमयूरांश्च हनुमानन्ववैक्षत} %5-11-16

\twolineshloka
{कृकलान् विविधांश्छागान् शशकानर्धभक्षितान्}
{महिषानेकशल्यांश्च मेषांश्च कृतनिष्ठितान्} %5-11-17

\twolineshloka
{लेह्यानुच्चावचान् पेयान् भोज्यान्युच्चावचानि च}
{तथाम्ललवणोत्तंसैर्विविधै रागखाण्डवैः} %5-11-18

\twolineshloka
{महानूपुरकेयूरैरपविद्धैर्महाधनैः}
{पानभाजनविक्षिप्तैः फलैश्च विविधैरपि} %5-11-19

\twolineshloka
{कृतपुष्पोपहारा भूरधिकां पुष्यति श्रियम्}
{तत्र तत्र च विन्यस्तैः सुश्लिष्टशयनासनैः} %5-11-20

\twolineshloka
{पानभूमिर्विना वह्निं प्रदीप्तेवोपलक्ष्यते}
{बहुप्रकारैर्विविधैर्वरसंस्कारसंस्कृतैः} %5-11-21

\twolineshloka
{मांसैः कुशलसंयुक्तैः पानभूमिगतैः पृथक्}
{दिव्याः प्रसन्ना विविधाः सुराः कृतसुरा अपि} %5-11-22

\twolineshloka
{शर्करासवमाध्वीकाः पुष्पासवफलासवाः}
{वासचूर्णैश्च विविधैर्मृष्टास्तैस्तैः पृथक् पृथक्} %5-11-23

\twolineshloka
{सन्तता शुशुभे भूमिर्माल्यैश्च बहुसंस्थितैः}
{हिरण्मयैश्च कलशैर्भाजनैः स्फाटिकैरपि} %5-11-24

\twolineshloka
{जाम्बूनदमयैश्चान्यैः करकैरभिसंवृता}
{राजतेषु च कुम्भेषु जाम्बूनदमयेषु च} %5-11-25

\twolineshloka
{पानश्रेष्ठां तथा भूमिं कपिस्तत्र ददर्श सः}
{सोऽपश्यच्छातकुम्भानि सीधोर्मणिमयानि च} %5-11-26

\twolineshloka
{तानि तानि च पूर्णानि भाजनानि महाकपिः}
{क्वचिदर्धावशेषाणि क्वचित् पीतान्यशेषतः} %5-11-27

\twolineshloka
{क्वचिन्नैव प्रपीतानि पानानि स ददर्श ह}
{क्वचिद् भक्ष्यांश्च विविधान् क्वचित् पानानि भागशः} %5-11-28

\threelineshloka
{क्वचिदर्धावशेषाणि पश्यन् वै विचचार ह}
{शयनान्यत्र नारीणां शून्यानि बहुधा पुनः}
{परस्परं समाश्लिष्य काश्चित् सुप्ता वराङ्गनाः} %5-11-29

\twolineshloka
{काचिच्च वस्त्रमन्यस्या अपहृत्योपगुह्य च}
{उपगम्याबला सुप्ता निद्राबलपराजिता} %5-11-30

\twolineshloka
{तासामुच्छ्वासवातेन वस्त्रं माल्यं च गात्रजम्}
{नात्यर्थं स्पन्दते चित्रं प्राप्य मन्दमिवानिलम्} %5-11-31

\twolineshloka
{चन्दनस्य च शीतस्य सीधोर्मधुरसस्य च}
{विविधस्य च माल्यस्य पुष्पस्य विविधस्य च} %5-11-32

\twolineshloka
{बहुधा मारुतस्तस्य गन्धं विविधमुद्वहन्}
{स्नानानां चन्दनानां च धूपानां चैव मूर्च्छितः} %5-11-33

\twolineshloka
{प्रववौ सुरभिर्गन्धो विमाने पुष्पके तदा}
{श्यामावदातास्तत्रान्याः काश्चित् कृष्णा वराङ्गनाः} %5-11-34

\twolineshloka
{काश्चित् काञ्चनवर्णाङ्ग््यः प्रमदा राक्षसालये}
{तासां निद्रावशत्वाच्च मदनेन विमूर्च्छितम्} %5-11-35

\threelineshloka
{पद्मिनीनां प्रसुप्तानां रूपमासीद् यथैव हि}
{एवं सर्वमशेषेण रावणान्तःपुरं कपिः}
{ददर्श स महातेजा न ददर्श च जानकीम्} %5-11-36

\twolineshloka
{निरीक्षमाणश्च ततस्ताः स्त्रियः स महाकपिः}
{जगाम महतीं शङ्कां धर्मसाध्वसशङ्कितः} %5-11-37

\twolineshloka
{परदारावरोधस्य प्रसुप्तस्य निरीक्षणम्}
{इदं खलु ममात्यर्थं धर्मलोपं करिष्यति} %5-11-38

\twolineshloka
{न हि मे परदाराणां दृष्टिर्विषयवर्तिनी}
{अयं चात्र मया दृष्टः परदारपरिग्रहः} %5-11-39

\twolineshloka
{तस्य प्रादुरभूच्चिन्ता पुनरन्या मनस्विनः}
{निश्चितैकान्तचित्तस्य कार्यनिश्चयदर्शिनी} %5-11-40

\twolineshloka
{कामं दृष्टा मया सर्वा विश्वस्ता रावणस्त्रियः}
{न तु मे मनसा किञ्चिद् वैकृत्यमुपपद्यते} %5-11-41

\twolineshloka
{मनो हि हेतुः सर्वेषामिन्द्रियाणां प्रवर्तने}
{शुभाशुभास्ववस्थासु तच्च मे सुव्यवस्थितम्} %5-11-42

\twolineshloka
{नान्यत्र हि मया शक्या वैदेही परिमार्गितुम्}
{स्त्रियो हि स्त्रीषु दृश्यन्ते सदा सम्परिमार्गणे} %5-11-43

\twolineshloka
{यस्य सत्त्वस्य या योनिस्तस्यां तत् परिमार्गते}
{न शक्यं प्रमदा नष्टा मृगीषु परिमार्गितुम्} %5-11-44

\twolineshloka
{तदिदं मार्गितं तावच्छुद्धेन मनसा मया}
{रावणान्तःपुरं सर्वं दृश्यते न च जानकी} %5-11-45

\twolineshloka
{देवगन्धर्वकन्याश्च नागकन्याश्च वीर्यवान्}
{अवेक्षमाणो हनुमान् नैवापश्यत जानकीम्} %5-11-46

\twolineshloka
{तामपश्यन् कपिस्तत्र पश्यंश्चान्या वरस्त्रियः}
{अपक्रम्य तदा वीरः प्रस्थातुमुपचक्रमे} %5-11-47

\twolineshloka
{स भूयः सर्वतः श्रीमान् मारुतिर्यत्नमाश्रितः}
{आपानभूमिमुत्सृज्य तां विचेतुं प्रचक्रमे} %5-11-48


॥इत्यार्षे श्रीमद्रामायणे वाल्मीकीये आदिकाव्ये सुन्दरकाण्डे पानभूमिविचयः नाम एकादशः सर्गः ॥५-११॥
