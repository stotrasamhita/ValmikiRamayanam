\sect{पञ्चचत्वारिंशः सर्गः — अमात्यपुत्रवधः}

\twolineshloka
{ततस्ते राक्षसेन्द्रेण चोदिता मन्त्रिणः सुताः}
{निर्ययुर्भवनात् तस्मात् सप्त सप्तार्चिवर्चसः} %5-45-1

\twolineshloka
{महद्बलपरीवारा धनुष्मन्तो महाबलाः}
{कृतास्त्रास्त्रविदां श्रेष्ठाः परस्परजयैषिणः} %5-45-2

\twolineshloka
{हेमजालपरिक्षिप्तैर्ध्वजवद्भिः पताकिभिः}
{तोयदस्वननिर्घोषैर्वाजियुक्तैर्महारथैः} %5-45-3

\twolineshloka
{तप्तकाञ्चनचित्राणि चापान्यमितविक्रमाः}
{विस्फारयन्तः संहृष्टास्तडिद्वन्त इवाम्बुदाः} %5-45-4

\twolineshloka
{जनन्यस्तास्ततस्तेषां विदित्वा किंकरान् हतान्}
{बभूवुः शोकसम्भ्रान्ताः सबान्धवसुहृज्जनाः} %5-45-5

\twolineshloka
{ते परस्परसंघर्षात् तप्तकाञ्चनभूषणाः}
{अभिपेतुर्हनूमन्तं तोरणस्थमवस्थितम्} %5-45-6

\twolineshloka
{सृजन्तो बाणवृष्टिं ते रथगर्जितनिःस्वनाः}
{प्रावृट्काल इवाम्भोदा विचेरुर्नैर्ऋताम्बुदाः} %5-45-7

\twolineshloka
{अवकीर्णस्ततस्ताभिर्हनूमान् शरवृष्टिभिः}
{अभवत् संवृताकारः शैलराडिव वृष्टिभिः} %5-45-8

\twolineshloka
{स शरान् वञ्चयामास तेषामाशुचरः कपिः}
{रथवेगांश्च वीराणां विचरन् विमलेऽम्बरे} %5-45-9

\twolineshloka
{स तैः क्रीडन् धनुष्मद्भिर्व्योम्नि वीरः प्रकाशते}
{धनुष्मद्भिर्यथा मेघैर्मारुतः प्रभुरम्बरे} %5-45-10

\twolineshloka
{स कृत्वा निनदं घोरं त्रासयंस्तां महाचमूम्}
{चकार हनुमान् वेगं तेषु रक्षःसु वीर्यवान्} %5-45-11

\twolineshloka
{तलेनाभिहनत् कांश्चित् पादैः कांश्चित् परंतपः}
{मुष्टिभिश्चाहनत् कांश्चिन्नखैः कांश्चिद् व्यदारयत्} %5-45-12

\twolineshloka
{प्रममाथोरसा कांश्चिदूरुभ्यामपरानपि}
{केचित् तस्यैव नादेन तत्रैव पतिता भुवि} %5-45-13

\twolineshloka
{ततस्तेष्ववसन् नेषु भूमौ निपतितेषु च}
{तत्सैन्यमगमत् सर्वं दिशो दश भयार्दितम्} %5-45-14

\twolineshloka
{विनेदुर्विस्वरं नागा निपेतुर्भुवि वाजिनः}
{भग्ननीडध्वजच्छत्रैर्भूश्च कीर्णाभवद् रथैः} %5-45-15

\twolineshloka
{स्रवता रुधिरेणाथ स्रवन्त्यो दर्शिताः पथि}
{विविधैश्च स्वनैर्लङ्का ननाद विकृतं तदा} %5-45-16

\twolineshloka
{स तान् प्रवृद्धान् विनिहत्य राक्षसान् महाबलश्चण्डपराक्रमः कपिः}
{युयुत्सुरन्यैः पुनरेव राक्षसैस्तदेव वीरोऽभिजगाम तोरणम्} %5-45-17


॥इत्यार्षे श्रीमद्रामायणे वाल्मीकीये आदिकाव्ये सुन्दरकाण्डे अमात्यपुत्रवधः नाम पञ्चचत्वारिंशः सर्गः ॥५-४५॥
