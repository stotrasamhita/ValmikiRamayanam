\sect{नवमः सर्गः — सङ्कुलान्तःपुरम्}

\twolineshloka
{तस्यालयवरिष्ठस्य मध्ये विमलमायतम्}
{ददर्श भवनश्रेष्ठं हनुमान् मारुतात्मजः} %5-9-1

\twolineshloka
{अर्धयोजनविस्तीर्णमायतं योजनं महत्}
{भवनं राक्षसेन्द्रस्य बहुप्रासादसङ्कुलम्} %5-9-2

\twolineshloka
{मार्गमाणस्तु वैदेहीं सीतामायतलोचनाम्}
{सर्वतः परिचक्राम हनूमानरिसूदनः} %5-9-3

\twolineshloka
{उत्तमं राक्षसावासं हनुमानवलोकयन्}
{आससादाथ लक्ष्मीवान् राक्षसेन्द्रनिवेशनम्} %5-9-4

\twolineshloka
{चतुर्विषाणैर्द्विरदैस्त्रिविषाणैस्तथैव च}
{परिक्षिप्तमसम्बाधं रक्ष्यमाणमुदायुधैः} %5-9-5

\twolineshloka
{राक्षसीभिश्च पत्नीभी रावणस्य निवेशनम्}
{आहृताभिश्च विक्रम्य राजकन्याभिरावृतम्} %5-9-6

\twolineshloka
{तन्नक्रमकराकीर्णं तिमिङ्गिलझषाकुलम्}
{वायुवेगसमाधूतं पन्नगैरिव सागरम्} %5-9-7

\twolineshloka
{या हि वैश्रवणे लक्ष्मीर्या चन्द्रे हरिवाहने}
{सा रावणगृहे रम्या नित्यमेवानपायिनी} %5-9-8

\twolineshloka
{या च राज्ञः कुबेरस्य यमस्य वरुणस्य च}
{तादृशी तद्विशिष्टा वा ऋद्धी रक्षोगृहेष्विह} %5-9-9

\twolineshloka
{तस्य हर्म्यस्य मध्यस्थवेश्म चान्यत् सुनिर्मितम्}
{बहुनिर्यूहसंयुक्तं ददर्श पवनात्मजः} %5-9-10

\twolineshloka
{ब्रह्मणोऽर्थे कृतं दिव्यं दिवि यद् विश्वकर्मणा}
{विमानं पुष्पकं नाम सर्वरत्नविभूषितम्} %5-9-11

\twolineshloka
{परेण तपसा लेभे यत् कुबेरः पितामहात्}
{कुबेरमोजसा जित्वा लेभे तद् राक्षसेश्वरः} %5-9-12

\twolineshloka
{ईहामृगसमायुक्तैः कार्तस्वरहिरण्मयैः}
{सुकृतैराचितं स्तम्भैः प्रदीप्तमिव च श्रिया} %5-9-13

\twolineshloka
{मेरुमन्दरसङ्काशैरुल्लिखद्भिरिवाम्बरम्}
{कूटागारैः शुभागारैः सर्वतः समलङ्कृतम्} %5-9-14

\twolineshloka
{ज्वलनार्कप्रतीकाशैः सुकृतं विश्वकर्मणा}
{हेमसोपानयुक्तं च चारुप्रवरवेदिकम्} %5-9-15

\twolineshloka
{जालवातायनैर्युक्तं काञ्चनैः स्फाटिकैरपि}
{इन्द्रनीलमहानीलमणिप्रवरवेदिकम्} %5-9-16

\twolineshloka
{विद्रुमेण विचित्रेण मणिभिश्च महाधनैः}
{निस्तुलाभिश्च मुक्ताभिस्तलेनाभिविराजितम्} %5-9-17

\twolineshloka
{चन्दनेन च रक्तेन तपनीयनिभेन च}
{सुपुण्यगन्धिना युक्तमादित्यतरुणोपमम्} %5-9-18

\threelineshloka
{कूटागारैर्वराकारैर्विविधैः समलङ्कृतम्}
{विमानं पुष्पकं दिव्यमारुरोह महाकपिः}
{तत्रस्थः सर्वतो गन्धं पानभक्ष्यान्नसम्भवम्} %5-9-19

\twolineshloka
{दिव्यं सम्मूर्च्छितं जिघ्रन् रूपवन्तमिवानिलम्}
{स गन्धस्तं महासत्त्वं बन्धुर्बन्धुमिवोत्तमम्} %5-9-20

\twolineshloka
{इत एहीत्युवाचेव तत्र यत्र स रावणः}
{ततस्तां प्रस्थितः शालां ददर्श महतीं शिवाम्} %5-9-21

\twolineshloka
{रावणस्य महाकान्तां कान्तामिव वरस्त्रियम्}
{मणिसोपानविकृतां हेमजालविराजिताम्} %5-9-22

\twolineshloka
{स्फाटिकैरावृततलां दन्तान्तरितरूपिकाम्}
{मुक्तावज्रप्रवालैश्च रूप्यचामीकरैरपि} %5-9-23

\twolineshloka
{विभूषितां मणिस्तम्भैः सुबहुस्तम्भभूषिताम्}
{समैर्ऋजुभिरत्युच्चैः समन्तात् सुविभूषितैः} %5-9-24

\twolineshloka
{स्तम्भैः पक्षैरिवात्युच्चैर्दिवं सम्प्रस्थितामिव}
{महत्या कुथयाऽऽस्तीर्णां पृथिवीलक्षणाङ्कया} %5-9-25

\twolineshloka
{पृथिवीमिव विस्तीर्णां सराष्ट्रगृहशालिनीम्}
{नादितां मत्तविहगैर्दिव्यगन्धाधिवासिताम्} %5-9-26

\twolineshloka
{परर्घ्यास्तरणोपेतां रक्षोऽधिपनिषेविताम्}
{धूम्रामगुरुधूपेन विमलां हंसपाण्डुराम्} %5-9-27

\twolineshloka
{पत्रपुष्पोपहारेण कल्माषीमिव सुप्रभाम्}
{मनसो मोदजननीं वर्णस्यापि प्रसाधिनीम्} %5-9-28

\twolineshloka
{तां शोकनाशिनीं दिव्यां श्रियः सञ्जननीमिव}
{इन्द्रियाणीन्द्रियार्थैस्तु पञ्च पञ्चभिरुत्तमैः} %5-9-29

\threelineshloka
{तर्पयामास मातेव तदा रावणपालिता}
{स्वर्गोऽयं देवलोकोऽयमिन्द्रस्यापि पुरी भवेत्}
{सिद्धिर्वेयं परा हि स्यादित्यमन्यत मारुतिः} %5-9-30

\twolineshloka
{प्रध्यायत इवापश्यत् प्रदीपांस्तत्र काञ्चनान्}
{धूर्तानिव महाधूर्तैर्देवनेन पराजितान्} %5-9-31

\twolineshloka
{दीपानां च प्रकाशेन तेजसा रावणस्य च}
{अर्चिर्भिर्भूषणानां च प्रदीप्तेत्यभ्यमन्यत} %5-9-32

\twolineshloka
{ततोऽपश्यत् कुथासीनं नानावर्णाम्बरस्रजम्}
{सहस्रं वरनारीणां नानावेषबिभूषितम्} %5-9-33

\twolineshloka
{परिवृत्तेऽर्धरात्रे तु पाननिद्रावशङ्गतम्}
{क्रीडित्वोपरतं रात्रौ प्रसुप्तं बलवत् तदा} %5-9-34

\twolineshloka
{तत् प्रसुप्तं विरुरुचे निःशब्दान्तरभूषितम्}
{निःशब्दहंसभ्रमरं यथा पद्मवनं महत्} %5-9-35

\twolineshloka
{तासां संवृतदान्तानि मीलिताक्षीणि मारुतिः}
{अपश्यत् पद्मगन्धीनि वदनानि सुयोषिताम्} %5-9-36

\twolineshloka
{प्रबुद्धानीव पद्मानि तासां भूत्वा क्षपाक्षये}
{पुनः संवृतपत्राणि रात्राविव बभुस्तदा} %5-9-37

\twolineshloka
{इमानि मुखपद्मानि नियतं मत्तषट्पदाः}
{अम्बुजानीव फुल्लानि प्रार्थयन्ति पुनः पुनः} %5-9-38

\twolineshloka
{इति वामन्यत श्रीमानुपपत्त्या महाकपिः}
{मेने हि गुणतस्तानि समानि सलिलोद्भवैः} %5-9-39

\twolineshloka
{सा तस्य शुशुभे शाला ताभिः स्त्रीभिर्विराजिता}
{शरदीव प्रसन्ना द्यौस्ताराभिरभिशोभिता} %5-9-40

\twolineshloka
{स च ताभिः परिवृतः शुशुभे राक्षसाधिपः}
{यथा ह्युडुपतिः श्रीमांस्ताराभिरिव संवृतः} %5-9-41

\twolineshloka
{याश्च्यवन्तेऽम्बरात् ताराः पुण्यशेषसमावृताः}
{इमास्ताः सङ्गताः कृत्स्ना इति मेने हरिस्तदा} %5-9-42

\twolineshloka
{ताराणामिव सुव्यक्तं महतीनां शुभार्चिषाम्}
{प्रभावर्णप्रसादाश्च विरेजुस्तत्र योषिताम्} %5-9-43

\twolineshloka
{व्यावृत्तकचपीनस्रक्प्रकीर्णवरभूषणाः}
{पानव्यायामकालेषु निद्रोपहतचेतसः} %5-9-44

\twolineshloka
{व्यावृत्ततिलकाः काश्चित् काश्चिदुदभ्रान्तनूपुराः}
{पार्श्वे गलितहाराश्च काश्चित् परमयोषितः} %5-9-45

\twolineshloka
{मुक्ताहारवृताश्चान्याः काश्चित् प्रस्रस्तवाससः}
{व्याविद्धरशनादामाः किशोर्य इव वाहिताः} %5-9-46

\twolineshloka
{अकुण्डलधराश्चान्या विच्छिन्नमृदितस्रजः}
{गजेन्द्रमृदिताः फुल्ला लता इव महावने} %5-9-47

\twolineshloka
{चन्द्रांशुकिरणाभाश्च हाराः कासाञ्चिदुद्गताः}
{हंसा इव बभुः सुप्ताः स्तनमध्येषु योषिताम्} %5-9-48

\twolineshloka
{अपरासां च वैदूर्याः कादम्बा इव पक्षिणः}
{हेमसूत्राणि चान्यासां चक्रवाका इवाभवन्} %5-9-49

\twolineshloka
{हंसकारण्डवोपेताश्चक्रवाकोपशोभिताः}
{आपगा इव ता रेजुर्जघनैः पुलिनैरिव} %5-9-50

\twolineshloka
{किङ्किणीजालसङ्काशास्ता हेमविपुलाम्बुजाः}
{भावग्राहा यशस्तीराः सुप्ता नद्य इवाबभुः} %5-9-51

\twolineshloka
{मृदुष्वङ्गेषु कासाञ्चित् कुचाग्रेषु च संस्थिताः}
{बभूवुर्भूषणानीव शुभा भूषणराजयः} %5-9-52

\twolineshloka
{अंशुकान्ताश्च कासाञ्चिन्मुखमारुतकम्पिताः}
{उपर्युपरि वक्त्राणां व्याधूयन्ते पुनः पुनः} %5-9-53

\twolineshloka
{ताः पताका इवोद्धूताः पत्नीनां रुचिरप्रभाः}
{नानावर्णसुवर्णानां वक्त्रमूलेषु रेजिरे} %5-9-54

\twolineshloka
{ववल्गुश्चात्र कासाञ्चित् कुण्डलानि शुभार्चिषाम्}
{मुखमारुतसङ्कम्पैर्मन्दं मन्दं च योषिताम्} %5-9-55

\twolineshloka
{शर्करासवगन्धः स प्रकृत्या सुरभिः सुखः}
{तासां वदननिःश्वासः सिषेवे रावणं तदा} %5-9-56

\twolineshloka
{रावणाननशङ्काश्च काश्चिद् रावणयोषितः}
{मुखानि च सपत्नीनामुपाजिघ्रन् पुनः पुनः} %5-9-57

\twolineshloka
{अत्यर्थं सक्तमनसो रावणे ता वरस्त्रियः}
{अस्वतन्त्राः सपत्नीनां प्रियमेवाचरंस्तदा} %5-9-58

\twolineshloka
{बाहूनुपनिधायान्याः पारिहार्यविभूषितान्}
{अंशुकानि च रम्याणि प्रमदास्तत्र शिश्यिरे} %5-9-59

\twolineshloka
{अन्या वक्षसि चान्यस्यास्तस्याः काचित् पुनर्भुजम्}
{अपरा त्वङ्कमन्यस्यास्तस्याश्चाप्यपरा कुचौ} %5-9-60

\twolineshloka
{ऊरुपार्श्वकटीपृष्ठमन्योन्यस्य समाश्रिताः}
{परस्परनिविष्टाङ्ग्यो मदस्नेहवशानुगाः} %5-9-61

\twolineshloka
{अन्योन्यस्याङ्गसंस्पर्शात् प्रीयमाणाः सुमध्यमाः}
{एकीकृतभुजाः सर्वाः सुषुपुस्तत्र योषितः} %5-9-62

\twolineshloka
{अन्योन्यभुजसूत्रेण स्त्रीमाला ग्रथिता हि सा}
{मालेव ग्रथिता सूत्रे शुशुभे मत्तषट्पदा} %5-9-63

\twolineshloka
{लतानां माधवे मासि फुल्लानां वायुसेवनात्}
{अन्योन्यमालाग्रथितं संसक्तकुसुमोच्चयम्} %5-9-64

\twolineshloka
{प्रतिवेष्टितसुस्कन्धमन्योन्यभ्रमराकुलम्}
{आसीद् वनमिवोद्धूतं स्त्रीवनं रावणस्य तत्} %5-9-65

\twolineshloka
{उचितेष्वपि सुव्यक्तं न तासां योषितां तदा}
{विवेकः शक्य आधातुं भूषणाङ्गाम्बरस्रजाम्} %5-9-66

\twolineshloka
{रावणे सुखसंविष्टे ताः स्त्रियो विविधप्रभाः}
{ज्वलन्तः काञ्चना दीपाः प्रेक्षन्तो निमिषा इव} %5-9-67

\twolineshloka
{राजर्षिविप्रदैत्यानां गन्धर्वाणां च योषितः}
{रक्षसां चाभवन् कन्यास्तस्य कामवशङ्गताः} %5-9-68

\twolineshloka
{युद्धकामेन ताः सर्वा रावणेन हृताः स्त्रियः}
{समदा मदनेनैव मोहिताः काश्चिदागताः} %5-9-69

\twolineshloka
{न तत्र काश्चित् प्रमदाः प्रसह्य वीर्योपपन्नेन गुणेन लब्धाः}
{न चान्यकामापि न चान्यपूर्वा विना वरार्हां जनकात्मजां तु} %5-9-70

\twolineshloka
{न चाकुलीना न च हीनरूपा नादक्षिणा नानुपचारयुक्ता}
{भार्याभवत् तस्य न हीनसत्त्वा न चापि कान्तस्य न कामनीया} %5-9-71

\twolineshloka
{बभूव बुद्धिस्तु हरीश्वरस्य यदीदृशी राघवधर्मपत्नी}
{इमा महाराक्षसराजभार्याः सुजातमस्येति हि साधुबुद्धेः} %5-9-72

\twolineshloka
{पुनश्च सोऽचिन्तयदात्तरूपो ध्रुवं विशिष्टा गुणतो हि सीता}
{अथायमस्यां कृतवान् महात्मा लङ्केश्वरः कष्टमनार्यकर्म} %5-9-73


॥इत्यार्षे श्रीमद्रामायणे वाल्मीकीये आदिकाव्ये सुन्दरकाण्डे सङ्कुलान्तःपुरम् नाम नवमः सर्गः ॥५-९॥
