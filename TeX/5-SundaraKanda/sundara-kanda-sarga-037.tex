\sect{सप्तत्रिंशः सर्गः — सीताप्रत्यानयनानौचित्यम्}

\twolineshloka
{सा सीता वचनं श्रुत्वा पूर्णचन्द्रनिभानना}
{हनूमन्तमुवाचेदं धर्मार्थसहितं वचः} %5-37-1

\twolineshloka
{अमृतं विषसम्पृक्तं त्वया वानर भाषितम्}
{यच्च नान्यमना रामो यच्च शोकपरायणः} %5-37-2

\twolineshloka
{ऐश्वर्ये वा सुविस्तीर्णे व्यसने वा सुदारुणे}
{रज्ज्वेव पुरुषं बद्ध्वा कृतान्तः परिकर्षति} %5-37-3

\twolineshloka
{विधिर्नूनमसंहार्यः प्राणिनां प्लवगोत्तम}
{सौमित्रं मां च रामं च व्यसनैः पश्य मोहितान्} %5-37-4

\twolineshloka
{शोकस्यास्य कथं पारं राघवोऽधिगमिष्यति}
{प्लवमानः परिक्रान्तो हतनौः सागरे यथा} %5-37-5

\twolineshloka
{राक्षसानां वधं कृत्वा सूदयित्वा च रावणम्}
{लङ्कामुन्मथितां कृत्वा कदा द्रक्ष्यति मां पतिः} %5-37-6

\twolineshloka
{स वाच्यः सन्त्वरस्वेति यावदेव न पूर्यते}
{अयं संवत्सरः कालस्तावद्धि मम जीवितम्} %5-37-7

\twolineshloka
{वर्तते दशमो मासो द्वौ तु शेषौ प्लवङ्गम}
{रावणेन नृशंसेन समयो यः कृतो मम} %5-37-8

\twolineshloka
{विभीषणेन च भ्रात्रा मम निर्यातनं प्रति}
{अनुनीतः प्रयत्नेन न च तत् कुरुते मतिम्} %5-37-9

\twolineshloka
{मम प्रतिप्रदानं हि रावणस्य न रोचते}
{रावणं मार्गते सङ्ख्ये मृत्युः कालवशङ्गतम्} %5-37-10

\twolineshloka
{ज्येष्ठा कन्या कला नाम विभीषणसुता कपे}
{तया ममैतदाख्यातं मात्रा प्रहितया स्वयम्} %5-37-11

\twolineshloka
{अविन्ध्यो नाम मेधावी विद्वान् राक्षसपुङ्गवः}
{धृतिमाञ्छीलवान् वृद्धो रावणस्य सुसम्मतः} %5-37-12

\twolineshloka
{रामात् क्षयमनुप्राप्तं रक्षसां प्रत्यचोदयत्}
{न च तस्य स दुष्टात्मा शृणोति वचनं हितम्} %5-37-13

\twolineshloka
{आशंसेयं हरिश्रेष्ठ क्षिप्रं मां प्राप्स्यते पतिः}
{अन्तरात्मा हि मे शुद्धस्तस्मिंश्च बहवो गुणाः} %5-37-14

\twolineshloka
{उत्साहः पौरुषं सत्त्वमानृशंस्यं कृतज्ञता}
{विक्रमश्च प्रभावश्च सन्ति वानर राघवे} %5-37-15

\twolineshloka
{चतुर्दश सहस्राणि राक्षसानां जघान यः}
{जनस्थाने विना भ्रात्रा शत्रुः कस्तस्य नोद्विजेत्} %5-37-16

\twolineshloka
{न स शक्यस्तुलयितुं व्यसनैः पुरुषर्षभः}
{अहं तस्यानुभावज्ञा शक्रस्येव पुलोमजा} %5-37-17

\twolineshloka
{शरजालांशुमान् शूरः कपे रामदिवाकरः}
{शत्रुरक्षोमयं तोयमुपशोषं नयिष्यति} %5-37-18

\twolineshloka
{इति सञ्जल्पमानां तां रामार्थे शोककर्शिताम्}
{अश्रुसम्पूर्णवदनामुवाच हनुमान् कपिः} %5-37-19

\twolineshloka
{श्रुत्वैव च वचो मह्यं क्षिप्रमेष्यति राघवः}
{चमूं प्रकर्षन् महतीं हर्यृक्षगणसङ्कुलाम्} %5-37-20

\twolineshloka
{अथवा मोचयिष्यामि त्वामद्यैव सराक्षसात्}
{अस्माद् दुःखादुपारोह मम पृष्ठमनिन्दिते} %5-37-21

\twolineshloka
{त्वां तु पृष्ठगतां कृत्वा सन्तरिष्यामि सागरम्}
{शक्तिरस्ति हि मे वोढुं लङ्कामपि सरावणाम्} %5-37-22

\twolineshloka
{अहं प्रस्रवणस्थाय राघवायाद्य मैथिलि}
{प्रापयिष्यामि शक्राय हव्यं हुतमिवानलः} %5-37-23

\twolineshloka
{द्रक्ष्यस्यद्यैव वैदेहि राघवं सहलक्ष्मणम्}
{व्यवसायसमायुक्तं विष्णुं दैत्यवधे यथा} %5-37-24

\twolineshloka
{त्वद्दर्शनकृतोत्साहमाश्रमस्थं महाबलम्}
{पुरन्दरमिवासीनं नगराजस्य मूर्धनि} %5-37-25

\twolineshloka
{पृष्ठमारोह मे देवि मा विकाङ्क्षस्व शोभने}
{योगमन्विच्छ रामेण शशाङ्केनेव रोहिणी} %5-37-26

\twolineshloka
{कथयन्तीव शशिना सङ्गमिष्यसि रोहिणी}
{मत्पृष्ठमधिरोह त्वं तराकाशं महार्णवम्} %5-37-27

\twolineshloka
{नहि मे सम्प्रयातस्य त्वामितो नयतोऽङ्गने}
{अनुगन्तुं गतिं शक्ताः सर्वे लङ्कानिवासिनः} %5-37-28

\twolineshloka
{यथैवाहमिह प्राप्तस्तथैवाहमसंशयम्}
{यास्यामि पश्य वैदेहि त्वामुद्यम्य विहायसम्} %5-37-29

\twolineshloka
{मैथिली तु हरिश्रेष्ठाच्छ्रुत्वा वचनमद्भुतम्}
{हर्षविस्मितसर्वाङ्गी हनूमन्तमथाब्रवीत्} %5-37-30

\twolineshloka
{हनूमन् दूरमध्वानं कथं मां नेतुमिच्छसि}
{तदेव खलु ते मन्ये कपित्वं हरियूथप} %5-37-31

\twolineshloka
{कथं चाल्पशरीरस्त्वं मामितो नेतुमिच्छसि}
{सकाशं मानवेन्द्रस्य भर्तुर्मे प्लवगर्षभ} %5-37-32

\twolineshloka
{सीतायास्तु वचः श्रुत्वा हनूमान् मारुतात्मजः}
{चिन्तयामास लक्ष्मीवान् नवं परिभवं कृतम्} %5-37-33

\twolineshloka
{न मे जानाति सत्त्वं वा प्रभावं वासितेक्षणा}
{तस्मात् पश्यतु वैदेही यद् रूपं मम कामतः} %5-37-34

\twolineshloka
{इति सञ्चिन्त्य हनुमांस्तदा प्लवगसत्तमः}
{दर्शयामास सीतायाः स्वरूपमरिमर्दनः} %5-37-35

\twolineshloka
{स तस्मात् पादपाद् धीमानाप्लुत्य प्लवगर्षभः}
{ततो वर्धितुमारेभे सीताप्रत्ययकारणात्} %5-37-36

\twolineshloka
{मेरुमन्दरसङ्काशो बभौ दीप्तानलप्रभः}
{अग्रतो व्यवतस्थे च सीताया वानरर्षभः} %5-37-37

\twolineshloka
{हरिः पर्वतसङ्काशस्ताम्रवक्त्रो महाबलः}
{वज्रदंष्ट्रनखो भीमो वैदेहीमिदमब्रवीत्} %5-37-38

\twolineshloka
{सपर्वतवनोद्देशां साट्टप्राकारतोरणाम्}
{लङ्कामिमां सनाथां वा नयितुं शक्तिरस्ति मे} %5-37-39

\twolineshloka
{तदवस्थाप्यतां बुद्धिरलं देवि विकाङ्क्षया}
{विशोकं कुरु वैदेहि राघवं सहलक्ष्मणम्} %5-37-40

\twolineshloka
{तं दृष्ट्वाचलसङ्काशमुवाच जनकात्मजा}
{पद्मपत्रविशालाक्षी मारुतस्यौरसं सुतम्} %5-37-41

\twolineshloka
{तव सत्त्वं बलं चैव विजानामि महाकपे}
{वायोरिव गतिश्चापि तेजश्चाग्नेरिवाद्भुतम्} %5-37-42

\twolineshloka
{प्राकृतोऽन्यः कथं चेमां भूमिमागन्तुमर्हति}
{उदधेरप्रमेयस्य पारं वानरयूथप} %5-37-43

\twolineshloka
{जानामि गमने शक्तिं नयने चापि ते मम}
{अवश्यं सम्प्रधार्याशु कार्यसिद्धिरिवात्मनः} %5-37-44

\twolineshloka
{अयुक्तं तु कपिश्रेष्ठ मया गन्तुं त्वया सह}
{वायुवेगसवेगस्य वेगो मां मोहयेत् तव} %5-37-45

\twolineshloka
{अहमाकाशमासक्ता उपर्युपरि सागरम्}
{प्रपतेयं हि ते पृष्ठाद् भूयो वेगेन गच्छतः} %5-37-46

\twolineshloka
{पतिता सागरे चाहं तिमिनक्रझषाकुले}
{भवेयमाशु विवशा यादसामन्नमुत्तमम्} %5-37-47

\twolineshloka
{न च शक्ष्ये त्वया सार्धं गन्तुं शत्रुविनाशन}
{कलत्रवति सन्देहस्त्वयि स्यादप्यसंशयम्} %5-37-48

\twolineshloka
{ह्रियमाणां तु मां दृष्ट्वा राक्षसा भीमविक्रमाः}
{अनुगच्छेयुरादिष्टा रावणेन दुरात्मना} %5-37-49

\twolineshloka
{तैस्त्वं परिवृतः शूरैः शूलमुद्गरपाणिभिः}
{भवेस्त्वं संशयं प्राप्तो मया वीर कलत्रवान्} %5-37-50

\twolineshloka
{सायुधा बहवो व्योम्नि राक्षसास्त्वं निरायुधः}
{कथं शक्ष्यसि संयातुं मां चैव परिरक्षितुम्} %5-37-51

\twolineshloka
{युध्यमानस्य रक्षोभिस्ततस्तैः क्रूरकर्मभिः}
{प्रपतेयं हि ते पृष्ठाद् भयार्ता कपिसत्तम} %5-37-52

\twolineshloka
{अथ रक्षांसि भीमानि महान्ति बलवन्ति च}
{कथञ्चित् साम्पराये त्वां जयेयुः कपिसत्तम} %5-37-53

\twolineshloka
{अथवा युध्यमानस्य पतेयं विमुखस्य ते}
{पतितां च गृहीत्वा मां नयेयुः पापराक्षसाः} %5-37-54

\twolineshloka
{मां वा हरेयुस्त्वद्धस्ताद् विशसेयुरथापि वा}
{अनवस्थौ हि दृश्येते युद्धे जयपराजयौ} %5-37-55

\twolineshloka
{अहं वापि विपद्येयं रक्षोभिरभितर्जिता}
{त्वत्प्रयत्नो हरिश्रेष्ठ भवेन्निष्फल एव तु} %5-37-56

\twolineshloka
{कामं त्वमपि पर्याप्तो निहन्तुं सर्वराक्षसान्}
{राघवस्य यशो हीयेत् त्वया शस्तैस्तु राक्षसैः} %5-37-57

\twolineshloka
{अथवाऽऽदाय रक्षांसि न्यसेयुः संवृते हि माम्}
{यत्र ते नाभिजानीयुर्हरयो नापि राघवः} %5-37-58

\twolineshloka
{आरम्भस्तु मदर्थोऽयं ततस्तव निरर्थकः}
{त्वया हि सह रामस्य महानागमने गुणः} %5-37-59

\twolineshloka
{मयि जीवितमायत्तं राघवस्यामितौजसः}
{भ्रातॄणां च महाबाहो तव राजकुलस्य च} %5-37-60

\twolineshloka
{तौ निराशौ मदर्थं च शोकसन्तापकर्शितौ}
{सह सर्वर्क्षहरिभिस्त्यक्ष्यतः प्राणसङ्ग्रहम्} %5-37-61

\twolineshloka
{भर्तुर्भक्तिं पुरस्कृत्य रामादन्यस्य वानर}
{नाहं स्प्रष्टुं स्वतो गात्रमिच्छेयं वानरोत्तम} %5-37-62

\twolineshloka
{यदहं गात्रसंस्पर्शं रावणस्य गता बलात्}
{अनीशा किं करिष्यामि विनाथा विवशा सती} %5-37-63

\twolineshloka
{यदि रामो दशग्रीवमिह हत्वा सराक्षसम्}
{मामितो गृह्य गच्छेत तत् तस्य सदृशं भवेत्} %5-37-64

\twolineshloka
{श्रुताश्च दृष्टा हि मया पराक्रमा महात्मनस्तस्य रणावमर्दिनः}
{न देवगन्धर्वभुजङ्गराक्षसा भवन्ति रामेण समा हि संयुगे} %5-37-65

\twolineshloka
{समीक्ष्य तं संयति चित्रकार्मुकं महाबलं वासवतुल्यविक्रमम्}
{सलक्ष्मणं को विषहेत राघवं हुताशनं दीप्तमिवानिलेरितम्} %5-37-66

\twolineshloka
{सलक्ष्मणं राघवमाजिमर्दनं दिशागजं मत्तमिव व्यवस्थितम्}
{सहेत को वानरमुख्य संयुगे युगान्तसूर्यप्रतिमं शरार्चिषम्} %5-37-67

\twolineshloka
{स मे कपिश्रेष्ठ सलक्ष्मणं प्रियं सयूथपं क्षिप्रमिहोपपादय}
{चिराय रामं प्रति शोककर्शितां कुरुष्व मां वानरवीर हर्षिताम्} %5-37-68


॥इत्यार्षे श्रीमद्रामायणे वाल्मीकीये आदिकाव्ये सुन्दरकाण्डे सीताप्रत्यानयनानौचित्यम् नाम सप्तत्रिंशः सर्गः ॥५-३७॥
