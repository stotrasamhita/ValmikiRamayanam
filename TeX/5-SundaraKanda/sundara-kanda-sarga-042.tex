\sect{द्विचत्वारिंशः सर्गः — किङ्करनिषूदनम्}

\twolineshloka
{ततः पक्षिनिनादेन वृक्षभङ्गस्वनेन च}
{बभूवुस्त्राससम्भ्रान्ताः सर्वे लङ्कानिवासिनः} %5-42-1

\twolineshloka
{विद्रुताश्च भयत्रस्ता विनेदुर्मृगपक्षिणः}
{रक्षसां च निमित्तानि क्रूराणि प्रतिपेदिरे} %5-42-2

\twolineshloka
{ततो गतायां निद्रायां राक्षस्यो विकृताननाः}
{तद् वनं ददृशुर्भग्नं तं च वीरं महाकपिम्} %5-42-3

\twolineshloka
{स ता दृष्ट्वा महाबाहुर्महासत्त्वो महाबलः}
{चकार सुमहद्रूपं राक्षसीनां भयावहम्} %5-42-4

\twolineshloka
{ततस्तु गिरिसङ्काशमतिकायं महाबलम्}
{राक्षस्यो वानरं दृष्ट्वा पप्रच्छुर्जनकात्मजाम्} %5-42-5

\twolineshloka
{कोऽयं कस्य कुतो वायं किन्निमित्तमिहागतः}
{कथं त्वया सहानेन संवादः कृत इत्युत} %5-42-6

\twolineshloka
{आचक्ष्व नो विशालाक्षि मा भूत्ते सुभगे भयम्}
{संवादमसितापाङ्गि त्वया किं कृतवानयम्} %5-42-7

\twolineshloka
{अथाब्रवीत् तदा साध्वी सीता सर्वाङ्गशोभना}
{रक्षसां कामरूपाणां विज्ञाने का गतिर्मम} %5-42-8

\twolineshloka
{यूयमेवास्य जानीत योऽयं यद् वा करिष्यति}
{अहिरेव ह्यहेः पादान् विजानाति न संशयः} %5-42-9

\twolineshloka
{अहमप्यतिभीतास्मि नैव जानामि को ह्ययम्}
{वेद्मि राक्षसमेवैनं कामरूपिणमागतम्} %5-42-10

\twolineshloka
{वैदेह्या वचनं श्रुत्वा राक्षस्यो विद्रुता द्रुतम्}
{स्थिताः काश्चिद्गताः काश्चिद् रावणाय निवेदितुम्} %5-42-11

\twolineshloka
{रावणस्य समीपे तु राक्षस्यो विकृताननाः}
{विरूपं वानरं भीमं रावणाय न्यवेदिषुः} %5-42-12

\twolineshloka
{अशोकवनिकामध्ये राजन् भीमवपुः कपिः}
{सीतया कृतसंवादस्तिष्ठत्यमितविक्रमः} %5-42-13

\twolineshloka
{न च तं जानकी सीता हरिं हरिणलोचना}
{अस्माभिर्बहुधा पृष्टा निवेदयितुमिच्छति} %5-42-14

\twolineshloka
{वासवस्य भवेद् दूतो दूतो वैश्रवणस्य वा}
{प्रेषितो वापि रामेण सीतान्वेषणकाङ्क्षया} %5-42-15

\twolineshloka
{तेनैवाद्भुतरूपेण यत्तत्तव मनोहरम्}
{नानामृगगणाकीर्णं प्रमृष्टं प्रमदावनम्} %5-42-16

\twolineshloka
{न तत्र कश्चिदुद्देशो यस्तेन न विनाशितः}
{यत्र सा जानकी देवी स तेन न विनाशितः} %5-42-17

\twolineshloka
{जानकीरक्षणार्थं वा श्रमाद् वा नोपलक्ष्यते}
{अथवा कः श्रमस्तस्य सैव तेनाभिरक्षिता} %5-42-18

\twolineshloka
{चारुपल्लवपत्राढ्यं यं सीता स्वयमास्थिता}
{प्रवृद्धः शिंशपावृक्षः स च तेनाभिरक्षितः} %5-42-19

\twolineshloka
{तस्योग्ररूपस्योग्रं त्वं दण्डमाज्ञातुमर्हसि}
{सीता सम्भाषिता येन वनं तेन विनाशितम्} %5-42-20

\twolineshloka
{मनःपरिगृहीतां तां तव रक्षोगणेश्वर}
{कः सीतामभिभाषेत यो न स्यात् त्यक्तजीवितः} %5-42-21

\twolineshloka
{राक्षसीनां वचः श्रुत्वा रावणो राक्षसेश्वरः}
{चिताग्निरिव जज्वाल कोपसंवर्तितेक्षणः} %5-42-22

\twolineshloka
{तस्य क्रुद्धस्य नेत्राभ्यां प्रापतन्नश्रुबिन्दवः}
{दीप्ताभ्यामिव दीपाभ्यां सार्चिषः स्नेहबिन्दवः} %5-42-23

\twolineshloka
{आत्मनः सदृशान् वीरान् किङ्करान्नाम राक्षसान्}
{व्यादिदेश महातेजा निग्रहार्थं हनूमतः} %5-42-24

\twolineshloka
{तेषामशीतिसाहस्रं किङ्कराणां तरस्विनाम्}
{निर्ययुर्भवनात् तस्मात् कूटमुद्गरपाणयः} %5-42-25

\twolineshloka
{महोदरा महादंष्ट्रा घोररूपा महाबलाः}
{युद्धाभिमनसः सर्वे हनूमद्ग्रहणोन्मुखाः} %5-42-26

\twolineshloka
{ते कपिं तं समासाद्य तोरणस्थमवस्थितम्}
{अभिपेतुर्महावेगाः पतङ्गा इव पावकम्} %5-42-27

\twolineshloka
{ते गदाभिर्विचित्राभिः परिघैः काञ्चनाङ्गदैः}
{आजग्मुर्वानरश्रेष्ठं शरैरादित्यसन्निभैः} %5-42-28

\twolineshloka
{मुद्गरैः पट्टिशैः शूलैः प्रासतोमरपाणयः}
{परिवार्य हनूमन्तं सहसा तस्थुरग्रतः} %5-42-29

\twolineshloka
{हनूमानपि तेजस्वी श्रीमान् पर्वतसन्निभः}
{क्षितावाविद्ध्य लाङ्गूलं ननाद च महाध्वनिम्} %5-42-30

\twolineshloka
{स भूत्वा तु महाकायो हनूमान् मारुतात्मजः}
{पुच्छमास्फोटयामास लङ्कां शब्देन पूरयन्} %5-42-31

\twolineshloka
{तस्यास्फोटितशब्देन महता चानुनादिना}
{पेतुर्विहङ्गा गगनादुच्चैश्चेदमघोषयत्} %5-42-32

\twolineshloka
{जयत्यतिबलो रामो लक्ष्मणश्च महाबलः}
{राजा जयति सुग्रीवो राघवेणाभिपालितः} %5-42-33

\twolineshloka
{दासोऽहं कोसलेन्द्रस्य रामस्याक्लिष्टकर्मणः}
{हनूमान् शत्रुसैन्यानां निहन्ता मारुतात्मजः} %5-42-34

\twolineshloka
{न रावणसहस्रं मे युद्धे प्रतिबलं भवेत्}
{शिलाभिश्च प्रहरतः पादपैश्च सहस्रशः} %5-42-35

\twolineshloka
{अर्दयित्वा पुरीं लङ्कामभिवाद्य च मैथिलीम्}
{समृद्धार्थो गमिष्यामि मिषतां सर्वरक्षसाम्} %5-42-36

\twolineshloka
{तस्य सन्नादशब्देन तेऽभवन् भयशङ्किताः}
{ददृशुश्च हनूमन्तं सन्ध्यामेघमिवोन्नतम्} %5-42-37

\twolineshloka
{स्वामिसन्देशनिःशङ्कास्ततस्ते राक्षसाः कपिम्}
{चित्रैः प्रहरणैर्भीमैरभिपेतुस्ततस्ततः} %5-42-38

\twolineshloka
{स तैः परिवृतः शूरैः सर्वतः स महाबलः}
{आससादायसं भीमं परिघं तोरणाश्रितम्} %5-42-39

\twolineshloka
{स तं परिघमादाय जघान रजनीचरान्}
{सपन्नगमिवादाय स्फुरन्तं विनतासुतः} %5-42-40

\twolineshloka
{विचचाराम्बरे वीरः परिगृह्य च मारुतिः}
{सूदयामास वज्रेण दैत्यानिव सहस्रदृक्} %5-42-41

\twolineshloka
{स हत्वा राक्षसान् वीरः किङ्करान् मारुतात्मजः}
{युद्धाकाङ्क्षी महावीरस्तोरणं समवस्थितः} %5-42-42

\twolineshloka
{ततस्तस्माद् भयान्मुक्ताः कतिचित्तत्र राक्षसाः}
{निहतान् किङ्करान् सर्वान् रावणाय न्यवेदयन्} %5-42-43

\twolineshloka
{स राक्षसानां निहतं महाबलं निशम्य राजा परिवृत्तलोचनः}
{समादिदेशाप्रतिमं पराक्रमे प्रहस्तपुत्रं समरे सुदुर्जयम्} %5-42-44


॥इत्यार्षे श्रीमद्रामायणे वाल्मीकीये आदिकाव्ये सुन्दरकाण्डे किङ्करनिषूदनम् नाम द्विचत्वारिंशः सर्गः ॥५-४२॥
