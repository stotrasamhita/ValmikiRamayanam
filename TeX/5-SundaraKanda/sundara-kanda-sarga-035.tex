\sect{पञ्चत्रिंशः सर्गः — विश्वासोत्पादनम्}

\twolineshloka
{तां तु रामकथां श्रुत्वा वैदेही वानरर्षभात्}
{उवाच वचनं सान्त्वमिदं मधुरया गिरा} %5-35-1

\twolineshloka
{क्व ते रामेण संसर्गः कथं जानासि लक्ष्मणम्}
{वानराणां नराणां च कथमासीत् समागमः} %5-35-2

\twolineshloka
{यानि रामस्य चिह्नानि लक्ष्मणस्य च वानर}
{तानि भूयः समाचक्ष्व न मां शोकः समाविशेत्} %5-35-3

\twolineshloka
{कीदृशं तस्य संस्थानं रूपं तस्य च कीदृशम्}
{कथमूरू कथं बाहू लक्ष्मणस्य च शंस मे} %5-35-4

\twolineshloka
{एवमुक्तस्तु वैदेह्या हनूमान् मारुतात्मजः}
{ततो रामं यथातत्त्वमाख्यातुमुपचक्रमे} %5-35-5

\twolineshloka
{जानन्ती बत दिष्ट्या मां वैदेहि परिपृच्छसि}
{भर्तुः कमलपत्राक्षि संस्थानं लक्ष्मणस्य च} %5-35-6

\twolineshloka
{यानि रामस्य चिह्नानि लक्ष्मणस्य च यानि वै}
{लक्षितानि विशालाक्षि वदतः शृणु तानि मे} %5-35-7

\twolineshloka
{रामः कमलपत्राक्षः पूर्णचन्द्रनिभाननः}
{रूपदाक्षिण्यसम्पन्नः प्रसूतो जनकात्मजे} %5-35-8

\twolineshloka
{तेजसाऽऽदित्यसंकाशः क्षमया पृथिवीसमः}
{बृहस्पतिसमो बुद्ध्या यशसा वासवोपमः} %5-35-9

\twolineshloka
{रक्षिता जीवलोकस्य स्वजनस्य च रक्षिता}
{रक्षिता स्वस्य वृत्तस्य धर्मस्य च परंतपः} %5-35-10

\twolineshloka
{रामो भामिनि लोकस्य चातुर्वर्ण्यस्य रक्षिता}
{मर्यादानां च लोकस्य कर्ता कारयिता च सः} %5-35-11

\twolineshloka
{अर्चिष्मानर्चितोऽत्यर्थं ब्रह्मचर्यव्रते स्थितः}
{साधूनामुपकारज्ञः प्रचारज्ञश्च कर्मणाम्} %5-35-12

\twolineshloka
{राजनीत्यां विनीतश्च ब्राह्मणानामुपासकः}
{ज्ञानवान् शीलसम्पन्नो विनीतश्च परंतपः} %5-35-13

\twolineshloka
{यजुर्वेदविनीतश्च वेदविद्भिः सुपूजितः}
{धनुर्वेदे च वेदे च वेदाङ्गेषु च निष्ठितः} %5-35-14

\twolineshloka
{विपुलांसो महाबाहुः कम्बुग्रीवः शुभाननः}
{गूढजत्रुः सुताम्राक्षो रामो नाम जनैः श्रुतः} %5-35-15

\twolineshloka
{दुन्दुभिस्वननिर्घोषः स्निग्धवर्णः प्रतापवान्}
{समश्च सुविभक्ताङ्गो वर्णं श्यामं समाश्रितः} %5-35-16

\twolineshloka
{त्रिस्थिरस्त्रिप्रलम्बश्च त्रिसमस्त्रिषु चोन्नतः}
{त्रिताम्रस्त्रिषु च स्निग्धो गम्भीरस्त्रिषु नित्यशः} %5-35-17

\twolineshloka
{त्रिवलीमांस्त्र्यवनतश्चतुर्व्यङ्गस्त्रिशीर्षवान्}
{चतुष्कलश्चतुर्लेखश्चतुष्किष्कुश्चतुःसमः} %5-35-18

\twolineshloka
{चतुर्दशसमद्वन्द्वश्चतुर्दंष्ट्रश्चतुर्गतिः}
{महोष्ठहनुनासश्च पञ्चस्निग्धोऽष्टवंशवान्} %5-35-19

\twolineshloka
{दशपद्मो दशबृहत् त्रिभिर्व्याप्तो द्विशुक्लवान्}
{षडुन्नतो नवतनुस्त्रिभिर्व्याप्नोति राघवः} %5-35-20

\twolineshloka
{सत्यधर्मरतः श्रीमान् संग्रहानुग्रहे रतः}
{देशकालविभागज्ञः सर्वलोकप्रियंवदः} %5-35-21

\twolineshloka
{भ्राता चास्य च वैमात्रः सौमित्रिरमितप्रभः}
{अनुरागेण रूपेण गुणैश्चापि तथाविधः} %5-35-22

\twolineshloka
{स सुवर्णच्छविः श्रीमान् रामः श्यामो महायशाः}
{तावुभौ नरशार्दूलौ त्वद्दर्शनकृतोत्सवौ} %5-35-23

\twolineshloka
{विचिन्वन्तौ महीं कृत्स्नामस्माभिः सह संगतौ}
{त्वामेव मार्गमाणौ तौ विचरन्तौ वसुन्धराम्} %5-35-24

\twolineshloka
{ददर्शतुर्मृगपतिं पूर्वजेनावरोपितम्}
{ऋष्यमूकस्य मूले तु बहुपादपसंकुले} %5-35-25

\twolineshloka
{भ्रातुर्भयार्तमासीनं सुग्रीवं प्रियदर्शनम्}
{वयं च हरिराजं तं सुग्रीवं सत्यसङ्गरम्} %5-35-26

\twolineshloka
{परिचर्यामहे राज्यात् पूर्वजेनावरोपितम्}
{ततस्तौ चीरवसनौ धनुःप्रवरपाणिनौ} %5-35-27

\twolineshloka
{ऋष्यमूकस्य शैलस्य रम्यं देशमुपागतौ}
{स तौ दृष्ट्वा नरव्याघ्रौ धन्विनौ वानरर्षभः} %5-35-28

\twolineshloka
{अभिप्लुतो गिरेस्तस्य शिखरं भयमोहितः}
{ततः स शिखरे तस्मिन् वानरेन्द्रो व्यवस्थितः} %5-35-29

\twolineshloka
{तयोः समीपं मामेव प्रेषयामास सत्वरम्}
{तावहं पुरुषव्याघ्रौ सुग्रीववचनात् प्रभू} %5-35-30

\twolineshloka
{रूपलक्षणसम्पन्नौ कृताञ्जलिरुपस्थितः}
{तौ परिज्ञाततत्त्वार्थौ मया प्रीतिसमन्वितौ} %5-35-31

\twolineshloka
{पृष्ठमारोप्य तं देशं प्रापितौ पुरुषर्षभौ}
{निवेदितौ च तत्त्वेन सुग्रीवाय महात्मने} %5-35-32

\twolineshloka
{तयोरन्योन्यसम्भाषाद् भृशं प्रीतिरजायत}
{तत्र तौ कीर्तिसम्पन्नौ हरीश्वरनरेश्वरौ} %5-35-33

\twolineshloka
{परस्परकृताश्वासौ कथया पूर्ववृत्तया}
{तं ततः सान्त्वयामास सुग्रीवं लक्ष्मणाग्रजः} %5-35-34

\twolineshloka
{स्त्रीहेतोर्वालिना भ्रात्रा निरस्तं पुरुतेजसा}
{ततस्त्वन्नाशजं शोकं रामस्याक्लिष्टकर्मणः} %5-35-35

\twolineshloka
{लक्ष्मणो वानरेन्द्राय सुग्रीवाय न्यवेदयत्}
{स श्रुत्वा वानरेन्द्रस्तु लक्ष्मणेनेरितं वचः} %5-35-36

\twolineshloka
{तदासीन्निष्प्रभोऽत्यर्थं ग्रहग्रस्त इवांशुमान्}
{ततस्त्वद्गात्रशोभीनि रक्षसा ह्रियमाणया} %5-35-37

\twolineshloka
{यान्याभरणजालानि पातितानि महीतले}
{तानि सर्वाणि रामाय आनीय हरियूथपाः} %5-35-38

\twolineshloka
{संहृष्टा दर्शयामासुर्गतिं तु न विदुस्तव}
{तानि रामाय दत्तानि मयैवोपहृतानि च} %5-35-39

\twolineshloka
{स्वनवन्त्यवकीर्णानि तस्मिन् विहतचेतसि}
{तान्यङ्के दर्शनीयानि कृत्वा बहुविधं तदा} %5-35-40

\twolineshloka
{तेन देवप्रकाशेन देवेन परिदेवितम्}
{पश्यतस्तानि रुदतस्ताम्यतश्च पुनः पुनः} %5-35-41

\onelineshloka
{प्रादीपयद् दाशरथेस्तदा शोकहुताशनम्} %5-35-42

\twolineshloka
{शायितं च चिरं तेन दुःखार्तेन महात्मना}
{मयापि विविधैर्वाक्यैः कृच्छ्रादुत्थापितः पुनः} %5-35-43

\twolineshloka
{तानि दृष्ट्वा महार्हाणि दर्शयित्वा मुहुर्मुहुः}
{राघवः सहसौमित्रिः सुग्रीवे संन्यवेशयत्} %5-35-44

\twolineshloka
{स तवादर्शनादार्ये राघवः परितप्यते}
{महता ज्वलता नित्यमग्निनेवाग्निपर्वतः} %5-35-45

\twolineshloka
{त्वत्कृते तमनिद्रा च शोकश्चिन्ता च राघवम्}
{तापयन्ति महात्मानमग्न्यगारमिवाग्नयः} %5-35-46

\twolineshloka
{तवादर्शनशोकेन राघवः परिचाल्यते}
{महता भूमिकम्पेन महानिव शिलोच्चयः} %5-35-47

\twolineshloka
{काननानि सुरम्याणि नदीप्रस्रवणानि च}
{चरन् न रतिमाप्नोति त्वामपश्यन् नृपात्मजे} %5-35-48

\twolineshloka
{स त्वां मनुजशार्दूलः क्षिप्रं प्राप्स्यति राघवः}
{समित्रबान्धवं हत्वा रावणं जनकात्मजे} %5-35-49

\twolineshloka
{सहितौ रामसुग्रीवावुभावकुरुतां तदा}
{समयं वालिनं हन्तुं तव चान्वेषणं प्रति} %5-35-50

\twolineshloka
{ततस्ताभ्यां कुमाराभ्यां वीराभ्यां स हरीश्वरः}
{किष्किन्धां समुपागम्य वाली युद्धे निपातितः} %5-35-51

\twolineshloka
{ततो निहत्य तरसा रामो वालिनमाहवे}
{सर्वर्क्षहरिसङ्घानां सुग्रीवमकरोत् पतिम्} %5-35-52

\twolineshloka
{रामसुग्रीवयोरैक्यं देव्येवं समजायत}
{हनूमन्तं च मां विद्धि तयोर्दूतमुपागतम्} %5-35-53

\twolineshloka
{स्वं राज्यं प्राप्य सुग्रीवः स्वानानीय महाकपीन्}
{त्वदर्थं प्रेषयामास दिशो दश महाबलान्} %5-35-54

\twolineshloka
{आदिष्टा वानरेन्द्रेण सुग्रीवेण महौजसः}
{अद्रिराजप्रतीकाशाः सर्वतः प्रस्थिता महीम्} %5-35-55

\twolineshloka
{ततस्ते मार्गमाणा वै सुग्रीववचनातुराः}
{चरन्ति वसुधां कृत्स्नां वयमन्ये च वानराः} %5-35-56

\twolineshloka
{अङ्गदो नाम लक्ष्मीवान् वालिसूनुर्महाबलः}
{प्रस्थितः कपिशार्दूलस्त्रिभागबलसंवृतः} %5-35-57

\twolineshloka
{तेषां नो विप्रणष्टानां विन्ध्ये पर्वतसत्तमे}
{भृशं शोकपरीतानामहोरात्रगणा गताः} %5-35-58

\twolineshloka
{ते वयं कार्यनैराश्यात् कालस्यातिक्रमेण च}
{भयाच्च कपिराजस्य प्राणांस्त्यक्तुमुपस्थिताः} %5-35-59

\twolineshloka
{विचित्य गिरिदुर्गाणि नदीप्रस्रवणानि च}
{अनासाद्य पदं देव्याः प्राणांस्त्यक्तुं व्यवस्थिताः} %5-35-60

\twolineshloka
{ततस्तस्य गिरेर्मूर्ध्नि वयं प्रायमुपास्महे}
{दृष्ट्वा प्रायोपविष्टांश्च सर्वान् वानरपुङ्गवान्} %5-35-61

\twolineshloka
{भृशं शोकार्णवे मग्नः पर्यदेवयदङ्गदः}
{तव नाशं च वैदेहि वालिनश्च तथा वधम्} %5-35-62

\twolineshloka
{प्रायोपवेशमस्माकं मरणं च जटायुषः}
{तेषां नः स्वामिसंदेशान्निराशानां मुमूर्षताम्} %5-35-63

\twolineshloka
{कार्यहेतोरिहायातः शकुनिर्वीर्यवान् महान्}
{गृध्रराजस्य सोदर्यः सम्पातिर्नाम गृध्रराट्} %5-35-64

\twolineshloka
{श्रुत्वा भ्रातृवधं कोपादिदं वचनमब्रवीत्}
{यवीयान् केन मे भ्राता हतः क्व च निपातितः} %5-35-65

\twolineshloka
{एतदाख्यातुमिच्छामि भवद्भिर्वानरोत्तमाः}
{अङ्गदोऽकथयत् तस्य जनस्थाने महद्वधम्} %5-35-66

\twolineshloka
{रक्षसा भीमरूपेण त्वामुद्दिश्य यथार्थतः}
{जटायोस्तु वधं श्रुत्वा दुःखितः सोऽरुणात्मजः} %5-35-67

\twolineshloka
{त्वामाह स वरारोहे वसन्तीं रावणालये}
{तस्य तद् वचनं श्रुत्वा सम्पातेः प्रीतिवर्धनम्} %5-35-68

\twolineshloka
{अङ्गदप्रमुखाः सर्वे ततः प्रस्थापिता वयम्}
{विन्ध्यादुत्थाय सम्प्राप्ताः सागरस्यान्तमुत्तमम्} %5-35-69

\twolineshloka
{त्वद्दर्शने कृतोत्साहा हृष्टाः पुष्टाः प्लवङ्गमाः}
{अङ्गदप्रमुखाः सर्वे वेलोपान्तमुपागताः} %5-35-70

\twolineshloka
{चिन्तां जग्मुः पुनर्भीमां त्वद्दर्शनसमुत्सुकाः}
{अथाहं हरिसैन्यस्य सागरं दृश्य सीदतः} %5-35-71

\twolineshloka
{व्यवधूय भयं तीव्रं योजनानां शतं प्लुतः}
{लङ्का चापि मया रात्रौ प्रविष्टा राक्षसाकुला} %5-35-72

\twolineshloka
{रावणश्च मया दृष्टस्त्वं च शोकनिपीडिता}
{एतत् ते सर्वमाख्यातं यथावृत्तमनिन्दिते} %5-35-73

\twolineshloka
{अभिभाषस्व मां देवि दूतो दाशरथेरहम्}
{तन्मां रामकृतोद्योगं त्वन्निमित्तमिहागतम्} %5-35-74

\twolineshloka
{सुग्रीवसचिवं देवि बुद्ध्यस्व पवनात्मजम्}
{कुशली तव काकुत्स्थः सर्वशस्त्रभृतां वरः} %5-35-75

\twolineshloka
{गुरोराराधने युक्तो लक्ष्मणः शुभलक्षणः}
{तस्य वीर्यवतो देवि भर्तुस्तव हिते रतः} %5-35-76

\twolineshloka
{अहमेकस्तु सम्प्राप्तः सुग्रीववचनादिह}
{मयेयमसहायेन चरता कामरूपिणा} %5-35-77

\twolineshloka
{दक्षिणा दिगनुक्रान्ता त्वन्मार्गविचयैषिणा}
{दिष्ट्याहं हरिसैन्यानां त्वन्नाशमनुशोचताम्} %5-35-78

\twolineshloka
{अपनेष्यामि संतापं तवाधिगमशासनात्}
{दिष्ट्या हि न मम व्यर्थं सागरस्येह लङ्घनम्} %5-35-79

\twolineshloka
{प्राप्स्याम्यहमिदं देवि त्वद्दर्शनकृतं यशः}
{राघवश्च महावीर्यः क्षिप्रं त्वामभिपत्स्यते} %5-35-80

\twolineshloka
{सपुत्रबान्धवं हत्वा रावणं राक्षसाधिपम्}
{माल्यवान् नाम वैदेहि गिरीणामुत्तमो गिरिः} %5-35-81

\threelineshloka
{ततो गच्छति गोकर्णं पर्वतं केसरी हरिः}
{स च देवर्षिभिर्दिष्टः पिता मम महाकपिः}
{तीर्थे नदीपतेः पुण्ये शम्बसादनमुद्धरन्} %5-35-82

\twolineshloka
{यस्याहं हरिणः क्षेत्रे जातो वातेन मैथिलि}
{हनूमानिति विख्यातो लोके स्वेनैव कर्मणा} %5-35-83

\twolineshloka
{विश्वासार्थं तु वैदेहि भर्तुरुक्ता मया गुणाः}
{अचिरात् त्वामितो देवि राघवो नयिता ध्रुवम्} %5-35-84

\twolineshloka
{एवं विश्वासिता सीता हेतुभिः शोककर्शिता}
{उपपन्नैरभिज्ञानैर्दूतं तमधिगच्छति} %5-35-85

\twolineshloka
{अतुलं च गता हर्षं प्रहर्षेण तु जानकी}
{नेत्राभ्यां वक्रपक्ष्माभ्यां मुमोचानन्दजं जलम्} %5-35-86

\twolineshloka
{चारु तद् वदनं तस्यास्ताम्रशुक्लायतेक्षणम्}
{अशोभत विशालाक्ष्या राहुमुक्त इवोडुराट्} %5-35-87

\twolineshloka
{हनूमन्तं कपिं व्यक्तं मन्यते नान्यथेति सा}
{अथोवाच हनूमांस्तामुत्तरं प्रियदर्शनाम्} %5-35-88

\twolineshloka
{एतत् ते सर्वमाख्यातं समाश्वसिहि मैथिलि}
{किं करोमि कथं वा ते रोचते प्रतियाम्यहम्} %5-35-89

\twolineshloka
{हतेऽसुरे संयति शम्बसादने कपिप्रवीरेण महर्षिचोदनात्}
{ततोऽस्मि वायुप्रभवो हि मैथिलि प्रभावतस्तत्प्रतिमश्च वानरः} %5-35-90


॥इत्यार्षे श्रीमद्रामायणे वाल्मीकीये आदिकाव्ये सुन्दरकाण्डे विश्वासोत्पादनम् नाम पञ्चत्रिंशः सर्गः ॥५-३५॥
