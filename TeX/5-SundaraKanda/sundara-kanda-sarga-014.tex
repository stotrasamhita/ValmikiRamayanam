\sect{चतुर्दशः सर्गः — अशोकवनिकाविचयः}

\twolineshloka
{स मुहूर्तमिव ध्यात्वा मनसा चाधिगम्य ताम्}
{अवप्लुतो महातेजाः प्राकारं तस्य वेश्मनः} %5-14-1

\twolineshloka
{स तु संहृष्टसर्वाङ्गः प्राकारस्थो महाकपिः}
{पुष्पिताग्रान् वसन्तादौ ददर्श विविधान् द्रुमान्} %5-14-2

\twolineshloka
{सालानशोकान् भव्यांश्च चम्पकांश्च सुपुष्पितान्}
{उद्दालकान् नागवृक्षांश्चूतान् कपिमुखानपि} %5-14-3

\twolineshloka
{तथाऽऽम्रवणसम्पन्नाँल्लताशतसमन्वितान्}
{ज्यामुक्त इव नाराचः पुप्लुवे वृक्षवाटिकाम्} %5-14-4

\twolineshloka
{स प्रविश्य विचित्रां तां विहगैरभिनादिताम्}
{राजतैः काञ्चनैश्चैव पादपैः सर्वतो वृताम्} %5-14-5

\twolineshloka
{विहगैर्मृगसङ्घैश्च विचित्रां चित्रकाननाम्}
{उदितादित्यसङ्काशां ददर्श हनुमान् बली} %5-14-6

\twolineshloka
{वृतां नानाविधैर्वृक्षैः पुष्पोपगफलोपगैः}
{कोकिलैर्भृङ्गराजैश्च मत्तैर्नित्यनिषेविताम्} %5-14-7

\twolineshloka
{प्रहृष्टमनुजां काले मृगपक्षिमदाकुलाम्}
{मत्तबर्हिणसङ्घुष्टां नानाद्विजगणायुताम्} %5-14-8

\twolineshloka
{मार्गमाणो वरारोहां राजपुत्रीमनिन्दिताम्}
{सुखप्रसुप्तान् विहगान् बोधयामास वानरः} %5-14-9

\twolineshloka
{उत्पतद्भिर्द्विजगणैः पक्षैर्वातैः समाहताः}
{अनेकवर्णा विविधा मुमुचुः पुष्पवृष्टयः} %5-14-10

\twolineshloka
{पुष्पावकीर्णः शुशुभे हनूमान् मारुतात्मजः}
{अशोकवनिकामध्ये यथा पुष्पमयो गिरिः} %5-14-11

\twolineshloka
{दिशः सर्वाभिधावन्तं वृक्षखण्डगतं कपिम्}
{दृष्ट्वा सर्वाणि भूतानि वसन्त इति मेनिरे} %5-14-12

\twolineshloka
{वृक्षेभ्यः पतितैः पुष्पैरवकीर्णाः पृथग्विधैः}
{रराज वसुधा तत्र प्रमदेव विभूषिता} %5-14-13

\twolineshloka
{तरस्विना ते तरवस्तरसा बहु कम्पिताः}
{कुसुमानि विचित्राणि ससृजुः कपिना तदा} %5-14-14

\twolineshloka
{निर्धूतपत्रशिखराः शीर्णपुष्पफलद्रुमाः}
{निक्षिप्तवस्त्राभरणा धूर्ता इव पराजिताः} %5-14-15

\twolineshloka
{हनूमता वेगवता कम्पितास्ते नगोत्तमाः}
{पुष्पपत्रफलान्याशु मुमुचुः फलशालिनः} %5-14-16

\twolineshloka
{विहङ्गसङ्घैर्हीनास्ते स्कन्धमात्राश्रया द्रुमाः}
{बभूवुरगमाः सर्वे मारुतेन विनिर्धुताः} %5-14-17

\twolineshloka
{विधूतकेशी युवतिर्यथा मृदितवर्णका}
{निपीतशुभदन्तोष्ठी नखैर्दन्तैश्च विक्षता} %5-14-18

\twolineshloka
{तथा लाङ्गूलहस्तैस्तु चरणाभ्यां च मर्दिता}
{तथैवाशोकवनिका प्रभग्नवनपादपा} %5-14-19

\twolineshloka
{महालतानां दामानि व्यधमत् तरसा कपिः}
{यथा प्रावृषि वेगेन मेघजालानि मारुतः} %5-14-20

\twolineshloka
{स तत्र मणिभूमीश्च राजतीश्च मनोरमाः}
{तथा काञ्चनभूमीश्च विचरन् ददृशे कपिः} %5-14-21

\twolineshloka
{वापीश्च विविधाकाराः पूर्णाः परमवारिणा}
{महार्हैर्मणिसोपानैरुपपन्नास्ततस्ततः} %5-14-22

\twolineshloka
{मुक्ताप्रवालसिकताः स्फाटिकान्तरकुट्टिमाः}
{काञ्चनैस्तरुभिश्चित्रैस्तीरजैरुपशोभिताः} %5-14-23

\twolineshloka
{बुद्धपद्मोत्पलवनाश्चक्रवाकोपशोभिताः}
{नत्यूहरुतसङ्घुष्टा हंससारसनादिताः} %5-14-24

\twolineshloka
{दीर्घाभिर्द्रुमयुक्ताभिः सरिद्भिश्च समन्ततः}
{अमृतोपमतोयाभिः शिवाभिरुपसंस्कृताः} %5-14-25

\twolineshloka
{लताशतैरवतताः सन्तानकुसुमावृताः}
{नानागुल्मावृतवनाः करवीरकृतान्तराः} %5-14-26

\twolineshloka
{ततोऽम्बुधरसङ्काशं प्रवृद्धशिखरं गिरिम्}
{विचित्रकूटं कूटैश्च सर्वतः परिवारितम्} %5-14-27

\twolineshloka
{शिलागृहैरवततं नानावृक्षसमावृतम्}
{ददर्श कपिशार्दूलो रम्यं जगति पर्वतम्} %5-14-28

\twolineshloka
{ददर्श च नगात् तस्मान्नदीं निपतितां कपिः}
{अङ्कादिव समुत्पत्य प्रियस्य पतितां प्रियाम्} %5-14-29

\twolineshloka
{जले निपतिताग्रैश्च पादपैरुपशोभिताम्}
{वार्यमाणामिव क्रुद्धां प्रमदां प्रियबन्धुभिः} %5-14-30

\twolineshloka
{पुनरावृत्ततोयां च ददर्श स महाकपिः}
{प्रसन्नामिव कान्तस्य कान्तां पुनरुपस्थिताम्} %5-14-31

\twolineshloka
{तस्यादूरात् स पद्मिन्यो नानाद्विजगणायुताः}
{ददर्श कपिशार्दूलो हनूमान् मारुतात्मजः} %5-14-32

\twolineshloka
{कृत्रिमां दीर्घिकां चापि पूर्णां शीतेन वारिणा}
{मणिप्रवरसोपानां मुक्तासिकतशोभिताम्} %5-14-33

\twolineshloka
{विविधैर्मृगसङ्घैश्च विचित्रां चित्रकाननाम्}
{प्रासादैः सुमहद्भिश्च निर्मितैर्विश्वकर्मणा} %5-14-34

\twolineshloka
{काननैः कृत्रिमैश्चापि सर्वतः समलङ्कृताम्}
{ये केचित् पादपास्तत्र पुष्पोपगफलोपगाः} %5-14-35

\twolineshloka
{सच्छत्राः सवितर्दीकाः सर्वे सौवर्णवेदिकाः}
{लताप्रतानैर्बहुभिः पर्णैश्च बहुभिर्वृताम्} %5-14-36

\twolineshloka
{काञ्चनीं शिंशपामेकां ददर्श स महाकपिः}
{वृतां हेममयीभिस्तु वेदिकाभिः समन्ततः} %5-14-37

\twolineshloka
{सोऽपश्यद् भूमिभागांश्च नगप्रस्रवणानि च}
{सुवर्णवृक्षानपरान् ददर्श शिखिसन्निभान्} %5-14-38

\twolineshloka
{तेषां द्रुमाणां प्रभया मेरोरिव महाकपिः}
{अमन्यत तदा वीरः काञ्चनोऽस्मीति सर्वतः} %5-14-39

\twolineshloka
{तान् काञ्चनान् वृक्षगणान् मारुतेन प्रकम्पितान्}
{किङ्किणीशतनिर्घोषान् दृष्ट्वा विस्मयमागमत्} %5-14-40

\twolineshloka
{सुपुष्पिताग्रान् रुचिरांस्तरुणाङ्कुरपल्लवान्}
{तामारुह्य महावेगः शिंशपां पर्णसंवृताम्} %5-14-41

\twolineshloka
{इतो द्रक्ष्यामि वैदेहीं रामदर्शनलालसाम्}
{इतश्चेतश्च दुःखार्तां सम्पतन्तीं यदृच्छया} %5-14-42

\twolineshloka
{अशोकवनिका चेयं दृढं रम्या दुरात्मनः}
{चन्दनैश्चम्पकैश्चापि बकुलैश्च विभूषिता} %5-14-43

\twolineshloka
{इयं च नलिनी रम्या द्विजसङ्घनिषेविता}
{इमां सा राजमहिषी नूनमेष्यति जानकी} %5-14-44

\twolineshloka
{सा रामा राजमहिषी राघवस्य प्रिया सती}
{वनसञ्चारकुशला ध्रुवमेष्यति जानकी} %5-14-45

\twolineshloka
{अथवा मृगशावाक्षी वनस्यास्य विचक्षणा}
{वनमेष्यति साद्येह रामचिन्तासुकर्शिता} %5-14-46

\twolineshloka
{रामशोकाभिसन्तप्ता सा देवी वामलोचना}
{वनवासरता नित्यमेष्यते वनचारिणी} %5-14-47

\twolineshloka
{वनेचराणां सततं नूनं स्पृहयते पुरा}
{रामस्य दयिता चार्या जनकस्य सुता सती} %5-14-48

\twolineshloka
{सन्ध्याकालमनाः श्यामा ध्रुवमेष्यति जानकी}
{नदीं चेमां शुभजलां सन्ध्यार्थे वरवर्णिनी} %5-14-49

\twolineshloka
{तस्याश्चाप्यनुरूपेयमशोकवनिका शुभा}
{शुभायाः पार्थिवेन्द्रस्य पत्नी रामस्य सम्मता} %5-14-50

\twolineshloka
{यदि जीवति सा देवी ताराधिपनिभानना}
{आगमिष्यति सावश्यमिमां शीतजलां नदीम्} %5-14-51

\twolineshloka
{एवं तु मत्वा हनुमान् महात्मा प्रतीक्षमाणो मनुजेन्द्रपत्नीम्}
{अवेक्षमाणश्च ददर्श सर्वं सुपुष्पिते पर्णघने निलीनः} %5-14-52


॥इत्यार्षे श्रीमद्रामायणे वाल्मीकीये आदिकाव्ये सुन्दरकाण्डे अशोकवनिकाविचयः नाम चतुर्दशः सर्गः ॥५-१४॥
