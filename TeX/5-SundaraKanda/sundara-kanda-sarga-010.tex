\sect{दशमः सर्गः — मन्दोदरीदर्शनम्}

\twolineshloka
{तत्र दिव्योपमं मुख्यं स्फाटिकं रत्नभूषितम्}
{अवेक्षमाणो हनुमान् ददर्श शयनासनम्} %5-10-1

\twolineshloka
{दान्तकाञ्चनचित्राङ्गैर्वैदूर्यैश्च वरासनैः}
{महार्हास्तरणोपेतैरुपपन्नं महाधनैः} %5-10-2

\twolineshloka
{तस्य चैकतमे देशे दिव्यमालोपशोभितम्}
{ददर्श पाण्डुरं छत्रं ताराधिपतिसन्निभम्} %5-10-3

\twolineshloka
{जातरूपपरिक्षिप्तं चित्रभानोः समप्रभम्}
{अशोकमालाविततं ददर्श परमासनम्} %5-10-4

\twolineshloka
{वालव्यजनहस्ताभिर्वीज्यमानं समन्ततः}
{गन्धैश्च विविधैर्जुष्टं वरधूपेन धूपितम्} %5-10-5

\twolineshloka
{परमास्तरणास्तीर्णमाविकाजिनसंवृतम्}
{दामभिर्वरमाल्यानां समन्तादुपशोभितम्} %5-10-6

\twolineshloka
{तस्मिञ्जीमूतसङ्काशं प्रदीप्तोज्ज्वलकुण्डलम्}
{लोहिताक्षं महाबाहुं महारजतवाससम्} %5-10-7

\twolineshloka
{लोहितेनानुलिप्ताङ्गं चन्दनेन सुगन्धिना}
{सन्ध्यारक्तमिवाकाशे तोयदं सतडिद्गुणम्} %5-10-8

\twolineshloka
{वृतमाभरणैर्दिव्यैः सुरूपं कामरूपिणम्}
{सवृक्षवनगुल्माढ्यं प्रसुप्तमिव मन्दरम्} %5-10-9

\twolineshloka
{क्रीडित्वोपरतं रात्रौ वराभरणभूषितम्}
{प्रियं राक्षसकन्यानां राक्षसानां सुखावहम्} %5-10-10

\twolineshloka
{पीत्वाप्युपरतं चापि ददर्श स महाकपिः}
{भास्वरे शयने वीरं प्रसुप्तं राक्षसाधिपम्} %5-10-11

\twolineshloka
{निःश्वसन्तं यथा नागं रावणं वानरोत्तमः}
{आसाद्य परमोद्विग्नः सोपासर्पत् सुभीतवत्} %5-10-12

\twolineshloka
{अथारोहणमासाद्य वेदिकान्तरमाश्रितः}
{क्षीबं राक्षसशार्दूलं प्रेक्षते स्म महाकपिः} %5-10-13

\twolineshloka
{शुशुभे राक्षसेन्द्रस्य स्वपतः शयनं शुभम्}
{गन्धहस्तिनि संविष्टे यथा प्रस्रवणं महत्} %5-10-14

\twolineshloka
{काञ्चनाङ्गदसन्नद्धौ ददर्श स महात्मनः}
{विक्षिप्तौ राक्षसेन्द्रस्य भुजाविन्द्रध्वजोपमौ} %5-10-15

\twolineshloka
{ऐरावतविषाणाग्रैरापीडनकृतव्रणौ}
{वज्रोल्लिखितपीनांसौ विष्णुचक्रपरिक्षतौ} %5-10-16

\twolineshloka
{पीनौ समसुजातांसौ सङ्गतौ बलसंयुतौ}
{सुलक्षणनखाङ्गुष्ठौ स्वङ्गुलीयकलक्षितौ} %5-10-17

\twolineshloka
{संहतौ परिघाकारौ वृत्तौ करिकरोपमौ}
{विक्षिप्तौ शयने शुभ्रे पञ्चशीर्षाविवोरगौ} %5-10-18

\twolineshloka
{शशक्षतजकल्पेन सुशीतेन सुगन्धिना}
{चन्दनेन परार्घ्येन स्वनुलिप्तौ स्वलङ्कृतौ} %5-10-19

\twolineshloka
{उत्तमस्त्रीविमृदितौ गन्धोत्तमनिषेवितौ}
{यक्षपन्नगगन्धर्वदेवदानवराविणौ} %5-10-20

\twolineshloka
{ददर्श स कपिस्तस्य बाहू शयनसंस्थितौ}
{मन्दरस्यान्तरे सुप्तौ महाही रुषिताविव} %5-10-21

\twolineshloka
{ताभ्यां स परिपूर्णाभ्यामुभाभ्यां राक्षसेश्वरः}
{शुशुभेऽचलसङ्काशः शृङ्गाभ्यामिव मन्दरः} %5-10-22

\twolineshloka
{चूतपुन्नागसुरभिर्बकुलोत्तमसंयुतः}
{मृष्टान्नरससंयुक्तः पानगन्धपुरःसरः} %5-10-23

\twolineshloka
{तस्य राक्षसराजस्य निश्चक्राम महामुखात्}
{शयानस्य विनिःश्वासः पूरयन्निव तद् गृहम्} %5-10-24

\twolineshloka
{मुक्तामणिविचित्रेण काञ्चनेन विराजिता}
{मुकुटेनापवृत्तेन कुण्डलोज्ज्वलिताननम्} %5-10-25

\twolineshloka
{रक्तचन्दनदिग्धेन तथा हारेण शोभिना}
{पीनायतविशालेन वक्षसाभिविराजिता} %5-10-26

\twolineshloka
{पाण्डुरेणापविद्धेन क्षौमेण क्षतजेक्षणम्}
{महार्हेण सुसंवीतं पीतेनोत्तरवाससा} %5-10-27

\twolineshloka
{माषराशिप्रतीकाशं निःश्वसन्तं भुजङ्गवत्}
{गाङ्गे महति तोयान्ते प्रसुप्तमिव कुञ्जरम्} %5-10-28

\twolineshloka
{चतुर्भिः काञ्चनैर्दीपैर्दीप्यमानं चतुर्दिशम्}
{प्रकाशीकृतसर्वाङ्गं मेघं विद्युद्गणैरिव} %5-10-29

\twolineshloka
{पादमूलगताश्चापि ददर्श सुमहात्मनः}
{पत्नीः स प्रियभार्यस्य तस्य रक्षःपतेर्गृहे} %5-10-30

\twolineshloka
{शशिप्रकाशवदना वरकुण्डलभूषणाः}
{अम्लानमाल्याभरणा ददर्श हरियूथपः} %5-10-31

\twolineshloka
{नृत्यवादित्रकुशला राक्षसेन्द्रभुजाङ्कगाः}
{वराभरणधारिण्यो निषण्णा ददृशे कपिः} %5-10-32

\twolineshloka
{वज्रवैदूर्यगर्भाणि श्रवणान्तेषु योषिताम्}
{ददर्श तापनीयानि कुण्डलान्यङ्गदानि च} %5-10-33

\twolineshloka
{तासां चन्द्रोपमैर्वक्त्रैः शुभैर्ललितकुण्डलैः}
{विरराज विमानं तन्नभस्तारागणैरिव} %5-10-34

\twolineshloka
{मदव्यायामखिन्नास्ता राक्षसेन्द्रस्य योषितः}
{तेषु तेष्ववकाशेषु प्रसुप्तास्तनुमध्यमाः} %5-10-35

\twolineshloka
{अङ्गहारैस्तथैवान्या कोमलैर्नृत्यशालिनी}
{विन्यस्तशुभसर्वाङ्गी प्रसुप्ता वरवर्णिनी} %5-10-36

\twolineshloka
{काचिद् वीणां परिष्वज्य प्रसुप्ता सम्प्रकाशते}
{महानदीप्रकीर्णेव नलिनी पोतमाश्रिता} %5-10-37

\twolineshloka
{अन्या कक्षगतेनैव मड्डुकेनासितेक्षणा}
{प्रसुप्ता भामिनी भाति बालपुत्रेव वत्सला} %5-10-38

\twolineshloka
{पटहं चारुसर्वाङ्गी न्यस्य शेते शुभस्तनी}
{चिरस्य रमणं लब्ध्वा परिष्वज्येव कामिनी} %5-10-39

\twolineshloka
{काचिद् वीणां परिष्वज्य सुप्ता कमललोचना}
{वरं प्रियतमं गृह्य सकामेव हि कामिनी} %5-10-40

\twolineshloka
{विपञ्चीं परिगृह्यान्या नियता नृत्यशालिनी}
{निद्रावशमनुप्राप्ता सहकान्तेव भामिनी} %5-10-41

\twolineshloka
{अन्या कनकसङ्काशैर्मृदुपीनैर्मनोरमैः}
{मृदङ्गं परिविद्ध्याङ्गैः प्रसुप्ता मत्तलोचना} %5-10-42

\twolineshloka
{भुजपाशान्तरस्थेन कक्षगेन कृशोदरी}
{पणवेन सहानिन्द्या सुप्ता मदकृतश्रमा} %5-10-43

\twolineshloka
{डिण्डिमं परिगृह्यान्या तथैवासक्तडिण्डिमा}
{प्रसुप्ता तरुणं वत्समुपगुह्येव भामिनी} %5-10-44

\twolineshloka
{काचिदाडम्बरं नारी भुजसम्भोगपीडितम्}
{कृत्वा कमलपत्राक्षी प्रसुप्ता मदमोहिता} %5-10-45

\twolineshloka
{कलशीमपविद्ध्यान्या प्रसुप्ता भाति भामिनी}
{वसन्ते पुष्पशबला मालेव परिमार्जिता} %5-10-46

\twolineshloka
{पाणिभ्यां च कुचौ काचित् सुवर्णकलशोपमौ}
{उपगुह्याबला सुप्ता निद्राबलपराजिता} %5-10-47

\twolineshloka
{अन्या कमलपत्राक्षी पूर्णेन्दुसदृशानना}
{अन्यामालिङ्ग्य सुश्रोणीं प्रसुप्ता मदविह्वला} %5-10-48

\twolineshloka
{आतोद्यानि विचित्राणि परिष्वज्य वरस्त्रियः}
{निपीड्य च कुचैः सुप्ताः कामिन्यः कामुकानिव} %5-10-49

\twolineshloka
{तासामेकान्तविन्यस्ते शयानां शयने शुभे}
{ददर्श रूपसम्पन्नामथ तां स कपिः स्त्रियम्} %5-10-50

\twolineshloka
{मुक्तामणिसमायुक्तैर्भूषणैः सुविभूषिताम्}
{विभूषयन्तीमिव च स्वश्रिया भवनोत्तमम्} %5-10-51

\twolineshloka
{गौरीं कनकवर्णाभामिष्टामन्तःपुरेश्वरीम्}
{कपिर्मन्दोदरीं तत्र शयानां चारुरूपिणीम्} %5-10-52

\threelineshloka
{स तां दृष्ट्वा महाबाहुर्भूषितां मारुतात्मजः}
{तर्कयामास सीतेति रूपयौवनसम्पदा}
{हर्षेण महता युक्तो ननन्द हरियूथपः} %5-10-53

\twolineshloka
{आस्फोटयामास चुचुम्ब पुच्छं ननन्द चिक्रीड जगौ जगाम}
{स्तम्भानरोहन्निपपात भूमौ निदर्शयन् स्वां प्रकृतिं कपीनाम्} %5-10-54


॥इत्यार्षे श्रीमद्रामायणे वाल्मीकीये आदिकाव्ये सुन्दरकाण्डे मन्दोदरीदर्शनम् नाम दशमः सर्गः ॥५-१०॥
