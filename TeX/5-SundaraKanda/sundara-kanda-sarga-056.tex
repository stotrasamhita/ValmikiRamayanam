\sect{षट्पञ्चाशः सर्गः — प्रतिप्रयाणोत्पतनम्}

\twolineshloka
{ततस्तु शिंशपामूले जानकीं पर्यवस्थिताम्}
{अभिवाद्याब्रवीद् दिष्ट्या पश्यामि त्वामिहाक्षताम्} %5-56-1

\twolineshloka
{ततस्तं प्रस्थितं सीता वीक्षमाणा पुनः पुनः}
{भर्तुः स्नेहान्विता वाक्यं हनूमन्तमभाषत} %5-56-2

\twolineshloka
{यदि त्वं मन्यसे तात वसैकाहमिहानघ}
{क्वचित् सुसंवृते देशे विश्रान्तः श्वो गमिष्यसि} %5-56-3

\twolineshloka
{मम चैवाल्पभाग्यायाः सांनिध्यात् तव वानर}
{शोकस्यास्याप्रमेयस्य मुहूर्तं स्यादपि क्षयः} %5-56-4

\twolineshloka
{गते हि हरिशार्दूल पुनः सम्प्राप्तये त्वयि}
{प्राणेष्वपि न विश्वासो मम वानरपुङ्गव} %5-56-5

\twolineshloka
{अदर्शनं च ते वीर भूयो मां दारयिष्यति}
{दुःखाद् दुःखतरं प्राप्तां दुर्मनःशोककर्शिताम्} %5-56-6

\twolineshloka
{अयं च वीर संदेहस्तिष्ठतीव ममाग्रतः}
{सुमहत्सु सहायेषु हर्यृक्षेषु महाबलः} %5-56-7

\twolineshloka
{कथं नु खलु दुष्पारं संतरिष्यति सागरम्}
{तानि हर्यृक्षसैन्यानि तौ वा नरवरात्मजौ} %5-56-8

\twolineshloka
{त्रयाणामेव भूतानां सागरस्यापि लङ्घने}
{शक्तिः स्याद् वैनतेयस्य तव वा मारुतस्य वा} %5-56-9

\twolineshloka
{तदत्र कार्यनिर्बन्धे समुत्पन्ने दुरासदे}
{किं पश्यसि समाधानं त्वं हि कार्यविशारदः} %5-56-10

\twolineshloka
{काममस्य त्वमेवैकः कार्यस्य परिसाधने}
{पर्याप्तः परवीरघ्न यशस्यस्ते फलोदयः} %5-56-11

\twolineshloka
{बलैस्तु संकुलां कृत्वा लङ्कां परबलार्दनः}
{मां नयेद् यदि काकुत्स्थस्तत् तस्य सदृशं भवेत्} %5-56-12

\twolineshloka
{तद् यथा तस्य विक्रान्तमनुरूपं महात्मनः}
{भवत्याहवशूरस्य तथा त्वमुपपादय} %5-56-13

\twolineshloka
{तदर्थोपहितं वाक्यं प्रश्रितं हेतुसंहितम्}
{निशम्य हनुमान् वीरो वाक्यमुत्तरमब्रवीत्} %5-56-14

\twolineshloka
{देवि हर्यृक्षसैन्यानामीश्वरः प्लवतां वरः}
{सुग्रीवः सत्त्वसम्पन्नस्तवार्थे कृतनिश्चयः} %5-56-15

\twolineshloka
{स वानरसहस्राणां कोटीभिरभिसंवृतः}
{क्षिप्रमेष्यति वैदेहि सुग्रीवः प्लवगाधिपः} %5-56-16

\twolineshloka
{तौ च वीरौ नरवरौ सहितौ रामलक्ष्मणौ}
{आगम्य नगरीं लङ्कां सायकैर्विधमिष्यतः} %5-56-17

\twolineshloka
{सगणं राक्षसं हत्वा नचिराद् रघुनन्दनः}
{त्वामादाय वरारोहे स्वां पुरीं प्रति यास्यति} %5-56-18

\twolineshloka
{समाश्वसिहि भद्रं ते भव त्वं कालकाङ्क्षिणी}
{क्षिप्रं द्रक्ष्यसि रामेण निहतं रावणं रणे} %5-56-19

\twolineshloka
{निहते राक्षसेन्द्रे च सपुत्रामात्यबान्धवे}
{त्वं समेष्यसि रामेण शशाङ्केनेव रोहिणी} %5-56-20

\twolineshloka
{क्षिप्रमेष्यति काकुत्स्थो हर्यृक्षप्रवरैर्युतः}
{यस्ते युधि विजित्यारीञ्छोकं व्यपनयिष्यति} %5-56-21

\twolineshloka
{एवमाश्वास्य वैदेहीं हनूमान् मारुतात्मजः}
{गमनाय मतिं कृत्वा वैदेहीमभ्यवादयत्} %5-56-22

\twolineshloka
{राक्षसान् प्रवरान् हत्वा नाम विश्राव्य चात्मनः}
{समाश्वास्य च वैदेहीं दर्शयित्वा परं बलम्} %5-56-23

\twolineshloka
{नगरीमाकुलां कृत्वा वञ्चयित्वा च रावणम्}
{दर्शयित्वा बलं घोरं वैदेहीमभिवाद्य च} %5-56-24

\twolineshloka
{प्रतिगन्तुं मनश्चक्रे पुनर्मध्येन सागरम्}
{ततः स कपिशार्दूलः स्वामिसंदर्शनोत्सुकः} %5-56-25

\twolineshloka
{आरुरोह गिरिश्रेष्ठमरिष्टमरिमर्दनः}
{तुङ्गपद्मकजुष्टाभिर्नीलाभिर्वनराजिभिः} %5-56-26

\twolineshloka
{सोत्तरीयमिवाम्भोदैः शृङ्गान्तरविलम्बिभिः}
{बोध्यमानमिव प्रीत्या दिवाकरकरैः शुभैः} %5-56-27

\twolineshloka
{उन्मिषन्तमिवोद्धूतैर्लोचनैरिव धातुभिः}
{तोयौघनिःस्वनैर्मन्द्रैः प्राधीतमिव पर्वतम्} %5-56-28

\twolineshloka
{प्रगीतमिव विस्पष्टं नानाप्रस्रवणस्वनैः}
{देवदारुभिरुद्धूतैरूर्ध्वबाहुमिव स्थितम्} %5-56-29

\twolineshloka
{प्रपातजलनिर्घोषैः प्राक्रुष्टमिव सर्वतः}
{वेपमानमिव श्यामैः कम्पमानैः शरद्वनैः} %5-56-30

\twolineshloka
{वेणुभिर्मारुतोद्धूतैः कूजन्तमिव कीचकैः}
{निःश्वसन्तमिवामर्षाद् घोरैराशीविषोत्तमैः} %5-56-31

\twolineshloka
{नीहारकृतगम्भीरैर्ध्यायन्तमिव गह्वरैः}
{मेघपादनिभैः पादैः प्रक्रान्तमिव सर्वतः} %5-56-32

\twolineshloka
{जृम्भमाणमिवाकाशे शिखरैरभ्रमालिभिः}
{कूटैश्च बहुधा कीर्णं शोभितं बहुकन्दरैः} %5-56-33

\twolineshloka
{सालतालैश्च कर्णैश्च वंशैश्च बहुभिर्वृतम्}
{लतावितानैर्विततैः पुष्पवद्भिरलंकृतम्} %5-56-34

\twolineshloka
{नानामृगगणैः कीर्णं धातुनिष्यन्दभूषितम्}
{बहुप्रस्रवणोपेतं शिलासंचयसंकटम्} %5-56-35

\twolineshloka
{महर्षियक्षगन्धर्वकिंनरोरगसेवितम्}
{लतापादपसम्बाधं सिंहाधिष्ठितकन्दरम्} %5-56-36

\twolineshloka
{व्याघ्रादिभिः समाकीर्णं स्वादुमूलफलद्रुमम्}
{आरुरोहानिलसुतः पर्वतं प्लवगोत्तमः} %5-56-37

\twolineshloka
{रामदर्शनशीघ्रेण प्रहर्षेणाभिचोदितः}
{तेन पादतलक्रान्ता रम्येषु गिरिसानुषु} %5-56-38

\twolineshloka
{सघोषाः समशीर्यन्त शिलाश्चूर्णीकृतास्ततः}
{स तमारुह्य शैलेन्द्रं व्यवर्धत महाकपिः} %5-56-39

\twolineshloka
{दक्षिणादुत्तरं पारं प्रार्थयँल्लवणाम्भसः}
{अधिरुह्य ततो वीरः पर्वतं पवनात्मजः} %5-56-40

\twolineshloka
{ददर्श सागरं भीमं भीमोरगनिषेवितम्}
{स मारुत इवाकाशं मारुतस्यात्मसम्भवः} %5-56-41

\twolineshloka
{प्रपेदे हरिशार्दूलो दक्षिणादुत्तरां दिशम्}
{स तदा पीडितस्तेन कपिना पर्वतोत्तमः} %5-56-42

\twolineshloka
{ररास विविधैर्भूतैः प्राविशद् वसुधातलम्}
{कम्पमानैश्च शिखरैः पतद्भिरपि च द्रुमैः} %5-56-43

\twolineshloka
{तस्योरुवेगोन्मथिताः पादपाः पुष्पशालिनः}
{निपेतुर्भूतले भग्नाः शक्रायुधहता इव} %5-56-44

\twolineshloka
{कन्दरोदरसंस्थानां पीडितानां महौजसाम्}
{सिंहानां निनदो भीमो नभो भिन्दन् हि शुश्रुवे} %5-56-45

\twolineshloka
{त्रस्तव्याविद्धवसना व्याकुलीकृतभूषणाः}
{विद्याधर्यः समुत्पेतुः सहसा धरणीधरात्} %5-56-46

\twolineshloka
{अतिप्रमाणा बलिनो दीप्तजिह्वा महाविषाः}
{निपीडितशिरोग्रीवा व्यवेष्टन्त महाहयः} %5-56-47

\twolineshloka
{किंनरोरगगन्धर्वयक्षविद्याधरास्तथा}
{पीडितं तं नगवरं त्यक्त्वा गगनमास्थिताः} %5-56-48

\twolineshloka
{स च भूमिधरः श्रीमान् बलिना तेन पीडितः}
{सवृक्षशिखरोदग्रः प्रविवेश रसातलम्} %5-56-49

\twolineshloka
{दशयोजनविस्तारस्त्रिंशद्योजनमुच्छ्रितः}
{धरण्यां समतां यातः स बभूव धराधरः} %5-56-50

\twolineshloka
{स लिलङ्घयिषुर्भीमं सलीलं लवणार्णवम्}
{कल्लोलास्फालवेलान्तमुत्पपात नभो हरिः} %5-56-51


॥इत्यार्षे श्रीमद्रामायणे वाल्मीकीये आदिकाव्ये सुन्दरकाण्डे प्रतिप्रयाणोत्पतनम् नाम षट्पञ्चाशः सर्गः ॥५-५६॥
