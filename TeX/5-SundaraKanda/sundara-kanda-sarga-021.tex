\sect{एकविंशः सर्गः — रावणतृणीकरणम्}

\twolineshloka
{तस्य तद् वचनं श्रुत्वा सीता रौद्रस्य रक्षसः}
{आर्ता दीनस्वरा दीनं प्रत्युवाच ततः शनैः} %5-21-1

\twolineshloka
{दुःखार्ता रुदती सीता वेपमाना तपस्विनी}
{चिन्तयन्ती वरारोहा पतिमेव पतिव्रता} %5-21-2

\twolineshloka
{तृणमन्तरतः कृत्वा प्रत्युवाच शुचिस्मिता}
{निवर्तय मनो मत्तः स्वजने प्रीयतां मनः} %5-21-3

\twolineshloka
{न मां प्रार्थयितुं युक्तस्त्वं सिद्धिमिव पापकृत्}
{अकार्यं न मया कार्यमेकपत्न्या विगर्हितम्} %5-21-4

\twolineshloka
{कुलं सम्प्राप्तया पुण्यं कुले महति जातया}
{एवमुक्त्वा तु वैदेही रावणं तं यशस्विनी} %5-21-5

\twolineshloka
{रावणं पृष्ठतः कृत्वा भूयो वचनमब्रवीत्}
{नाहमौपयिकी भार्या परभार्या सती तव} %5-21-6

\twolineshloka
{साधु धर्ममवेक्षस्व साधु साधुव्रतं चर}
{यथा तव तथान्येषां रक्ष्या दारा निशाचर} %5-21-7

\threelineshloka
{आत्मानमुपमां कृत्वा स्वेषु दारेषु रम्यताम्}
{अतुष्टं स्वेषु दारेषु चपलं चपलेन्द्रियम्}
{नयन्ति निकृतिप्रज्ञं परदाराः पराभवम्} %5-21-8

\twolineshloka
{इह सन्तो न वा सन्ति सतो वा नानुवर्तसे}
{यथा हि विपरीता ते बुद्धिराचारवर्जिता} %5-21-9

\twolineshloka
{वचो मिथ्याप्रणीतात्मा पथ्यमुक्तं विचक्षणैः}
{राक्षसानामभावाय त्वं वा न प्रतिपद्यसे} %5-21-10

\twolineshloka
{अकृतात्मानमासाद्य राजानमनये रतम्}
{समृद्धानि विनश्यन्ति राष्ट्राणि नगराणि च} %5-21-11

\twolineshloka
{तथैव त्वां समासाद्य लंका रत्नौघसंकुला}
{अपराधात् तवैकस्य नचिराद् विनशिष्यति} %5-21-12

\twolineshloka
{स्वकृतैर्हन्यमानस्य रावणादीर्घदर्शिनः}
{अभिनन्दन्ति भूतानि विनाशे पापकर्मणः} %5-21-13

\twolineshloka
{एवं त्वां पापकर्माणं वक्ष्यन्ति निकृता जनाः}
{दिष्ट्यैतद् व्यसनं प्राप्तो रौद्र इत्येव हर्षिताः} %5-21-14

\twolineshloka
{शक्या लोभयितुं नाहमैश्वर्येण धनेन वा}
{अनन्या राघवेणाहं भास्करेण यथा प्रभा} %5-21-15

\twolineshloka
{उपधाय भुजं तस्य लोकनाथस्य सत्कृतम्}
{कथं नामोपधास्यामि भुजमन्यस्य कस्यचित्} %5-21-16

\twolineshloka
{अहमौपयिकी भार्या तस्यैव च धरापतेः}
{व्रतस्नातस्य विद्येव विप्रस्य विदितात्मनः} %5-21-17

\twolineshloka
{साधु रावण रामेण मां समानय दुःखिताम्}
{वने वासितया सार्धं करेण्वेव गजाधिपम्} %5-21-18

\twolineshloka
{मित्रमौपयिकं कर्तुं रामः स्थानं परीप्सता}
{बन्धं चानिच्छता घोरं त्वयासौ पुरुषर्षभः} %5-21-19

\twolineshloka
{विदितः सर्वधर्मज्ञः शरणागतवत्सलः}
{तेन मैत्री भवतु ते यदि जीवितुमिच्छसि} %5-21-20

\twolineshloka
{प्रसादयस्व त्वं चैनं शरणागतवत्सलम्}
{मां चास्मै प्रयतो भूत्वा निर्यातयितुमर्हसि} %5-21-21

\twolineshloka
{एवं हि ते भवेत् स्वस्ति सम्प्रदाय रघूत्तमे}
{अन्यथा त्वं हि कुर्वाणः परां प्राप्स्यसि चापदम्} %5-21-22

\twolineshloka
{वर्जयेद् वज्रमुत्सृष्टं वर्जयेदन्तकश्चिरम्}
{त्वद्विधं न तु संक्रुद्धो लोकनाथः स राघवः} %5-21-23

\twolineshloka
{रामस्य धनुषः शब्दं श्रोष्यसि त्वं महास्वनम्}
{शतक्रतुविसृष्टस्य निर्घोषमशनेरिव} %5-21-24

\twolineshloka
{इह शीघ्रं सुपर्वाणो ज्वलितास्या इवोरगाः}
{इषवो निपतिष्यन्ति रामलक्ष्मणलक्षिताः} %5-21-25

\twolineshloka
{रक्षांसि निहनिष्यन्तः पुर्यामस्यां न संशयः}
{असम्पातं करिष्यन्ति पतन्तः कङ्कवाससः} %5-21-26

\twolineshloka
{राक्षसेन्द्रमहासर्पान् स रामगरुडो महान्}
{उद्धरिष्यति वेगेन वैनतेय इवोरगान्} %5-21-27

\twolineshloka
{अपनेष्यति मां भर्ता त्वत्तः शीघ्रमरिंदमः}
{असुरेभ्यः श्रियं दीप्तां विष्णुस्त्रिभिरिव क्रमैः} %5-21-28

\twolineshloka
{जनस्थाने हतस्थाने निहते रक्षसां बले}
{अशक्तेन त्वया रक्षः कृतमेतदसाधु वै} %5-21-29

\twolineshloka
{आश्रमं तत्तयोः शून्यं प्रविश्य नरसिंहयोः}
{गोचरं गतयोर्भ्रात्रोरपनीता त्वयाधम} %5-21-30

\twolineshloka
{नहि गन्धमुपाघ्राय रामलक्ष्मणयोस्त्वया}
{शक्यं संदर्शने स्थातुं शुना शार्दूलयोरिव} %5-21-31

\twolineshloka
{तस्य ते विग्रहे ताभ्यां युगग्रहणमस्थिरम्}
{वृत्रस्येवेन्द्रबाहुभ्यां बाहोरेकस्य विग्रहे} %5-21-32

\twolineshloka
{क्षिप्रं तव स नाथो मे रामः सौमित्रिणा सह}
{तोयमल्पमिवादित्यः प्राणानादास्यते शरैः} %5-21-33

\twolineshloka
{गिरिं कुबेरस्य गतोऽथवाऽऽलयं सभां गतो वा वरुणस्य राज्ञः}
{असंशयं दाशरथेर्विमोक्ष्यसे महाद्रुमः कालहतोऽशनेरिव} %5-21-34


॥इत्यार्षे श्रीमद्रामायणे वाल्मीकीये आदिकाव्ये सुन्दरकाण्डे रावणतृणीकरणम् नाम एकविंशः सर्गः ॥५-२१॥
