\sect{द्विषष्ठितमः सर्गः — दधिमुखखिलीकारः}

\twolineshloka
{तानुवाच हरिश्रेष्ठो हनूमान् वानरर्षभः}
{अव्यग्रमनसो यूयं मधु सेवत वानराः} %5-62-1

\twolineshloka
{अहमावर्जयिष्यामि युष्माकं परिपन्थिनः}
{श्रुत्वा हनूमतो वाक्यं हरीणां प्रवरोऽङ्गदः} %5-62-2

\twolineshloka
{प्रत्युवाच प्रसन्नात्मा पिबन्तु हरयो मधु}
{अवश्यं कृतकार्यस्य वाक्यं हनुमतो मया} %5-62-3

\twolineshloka
{अकार्यमपि कर्तव्यं किमङ्गं पुनरीदृशम्}
{अङ्गदस्य मुखाच्छ्रुत्वा वचनं वानरर्षभाः} %5-62-4

\twolineshloka
{साधु साध्विति संहृष्टा वानराः प्रत्यपूजयन्}
{पूजयित्वाङ्गदं सर्वे वानरा वानरर्षभम्} %5-62-5

\twolineshloka
{जग्मुर्मधुवनं यत्र नदीवेग इव द्रुमम्}
{ते प्रविष्टा मधुवनं पालानाक्रम्य शक्तितः} %5-62-6

\twolineshloka
{अतिसर्गाच्च पटवो दृष्ट्वा श्रुत्वा च मैथिलीम्}
{पपुः सर्वे मधु तदा रसवत् फलमाददुः} %5-62-7

\twolineshloka
{उत्पत्य च ततः सर्वे वनपालान् समागतान्}
{ते ताडयन्तः शतशः सक्ता मधुवने तदा} %5-62-8

\twolineshloka
{मधूनि द्रोणमात्राणि बाहुभिः परिगृह्य ते}
{पिबन्ति कपयः केचित् सङ्घशस्तत्र हृष्टवत्} %5-62-9

\twolineshloka
{घ्नन्ति स्म सहिताः सर्वे भक्षयन्ति तथापरे}
{केचित् पीत्वापविध्यन्ति मधूनि मधुपिङ्गलाः} %5-62-10

\twolineshloka
{मधूच्छिष्टेन केचिच्च जघ्नुरन्योन्यमुत्कटाः}
{अपरे वृक्षमूलेषु शाखा गृह्य व्यवस्थिताः} %5-62-11

\twolineshloka
{अत्यर्थं च मदग्लानाः पर्णान्यास्तीर्य शेरते}
{उन्मत्तवेगाः प्लवगा मधुमत्ताश्च हृष्टवत्} %5-62-12

\twolineshloka
{क्षिपन्त्यपि तथान्योन्यं स्खलन्ति च तथापरे}
{केचित् क्ष्वेडान् प्रकुर्वन्ति केचित् कूजन्ति हृष्टवत्} %5-62-13

\twolineshloka
{हरयो मधुना मत्ताः केचित् सुप्ता महीतले}
{धृष्टाः केचिद्धसन्त्यन्ये केचित् कुर्वन्ति चेतरत्} %5-62-14

\twolineshloka
{कृत्वा केचिद् वदन्त्यन्ये केचिद् बुध्यन्ति चेतरत्}
{येऽप्यत्र मधुपालाः स्युः प्रेष्या दधिमुखस्य तु} %5-62-15

\twolineshloka
{तेऽपि तैर्वानरैर्भीमैः प्रतिषिद्धा दिशो गताः}
{जानुभिश्च प्रघृष्टाश्च देवमार्गं च दर्शिताः} %5-62-16

\threelineshloka
{अब्रुवन् परमोद्विग्ना गत्वा दधिमुखं वचः}
{हनूमता दत्तवरैर्हतं मधुवनं बलात्}
{वयं च जानुभिर्घृष्टा देवमार्गं च दर्शिताः} %5-62-17

\twolineshloka
{तदा दधिमुखः क्रुद्धो वनपस्तत्र वानरः}
{हतं मधुवनं श्रुत्वा सान्त्वयामास तान् हरीन्} %5-62-18

\twolineshloka
{एतागच्छत गच्छामो वानरानतिदर्पितान्}
{बलेनावारयिष्यामि प्रभुञ्जानान् मधूत्तमम्} %5-62-19

\twolineshloka
{श्रुत्वा दधिमुखस्येदं वचनं वानरर्षभाः}
{पुनर्वीरा मधुवनं तेनैव सहिता ययुः} %5-62-20

\twolineshloka
{मध्ये चैषां दधिमुखः सुप्रगृह्य महातरुम्}
{समभ्यधावन् वेगेन सर्वे ते च प्लवंगमाः} %5-62-21

\twolineshloka
{ते शिलाः पादपांश्चैव पाषाणानपि वानराः}
{गृहीत्वाभ्यागमन् क्रुद्धा यत्र ते कपिकुञ्जराः} %5-62-22

\twolineshloka
{बलान्निवारयन्तश्च आसेदुर्हरयो हरीन्}
{संदष्टौष्ठपुटाः क्रुद्धा भर्त्सयन्तो मुहुर्मुहुः} %5-62-23

\twolineshloka
{अथ दृष्ट्वा दधिमुखं क्रुद्धं वानरपुङ्गवाः}
{अभ्यधावन्त वेगेन हनुमत्प्रमुखास्तदा} %5-62-24

\twolineshloka
{सवृक्षं तं महाबाहुमापतन्तं महाबलम्}
{वेगवन्तं विजग्राह बाहुभ्यां कुपितोऽङ्गदः} %5-62-25

\twolineshloka
{मदान्धो न कृपां चक्रे आर्यकोऽयं ममेति सः}
{अथैनं निष्पिपेषाशु वेगेन वसुधातले} %5-62-26

\twolineshloka
{स भग्नबाहूरुमुखो विह्वलः शोणितोक्षितः}
{प्रमुमोह महावीरो मुहूर्तं कपिकुञ्जरः} %5-62-27

\twolineshloka
{स कथंचिद् विमुक्तस्तैर्वानरैर्वानरर्षभः}
{उवाचैकान्तमागत्य स्वान् भृत्यान् समुपागतान्} %5-62-28

\twolineshloka
{एतागच्छत गच्छामो भर्ता नो यत्र वानरः}
{सुग्रीवो विपुलग्रीवः सह रामेण तिष्ठति} %5-62-29

\twolineshloka
{सर्वं चैवाङ्गदे दोषं श्रावयिष्याम पार्थिवे}
{अमर्षी वचनं श्रुत्वा घातयिष्यति वानरान्} %5-62-30

\twolineshloka
{इष्टं मधुवनं ह्येतत् सुग्रीवस्य महात्मनः}
{पितृपैतामहं दिव्यं देवैरपि दुरासदम्} %5-62-31

\twolineshloka
{स वानरानिमान् सर्वान् मधुलुब्धान् गतायुषः}
{घातयिष्यति दण्डेन सुग्रीवः ससुहृज्जनान्} %5-62-32

\twolineshloka
{वध्या ह्येते दुरात्मानो नृपाज्ञापरिपन्थिनः}
{अमर्षप्रभवो रोषः सफलो मे भविष्यति} %5-62-33

\twolineshloka
{एवमुक्त्वा दधिमुखो वनपालान् महाबलः}
{जगाम सहसोत्पत्य वनपालैः समन्वितः} %5-62-34

\twolineshloka
{निमेषान्तरमात्रेण स हि प्राप्तो वनालयः}
{सहस्रांशुसुतो धीमान् सुग्रीवो यत्र वानरः} %5-62-35

\twolineshloka
{रामं च लक्ष्मणं चैव दृष्ट्वा सुग्रीवमेव च}
{समप्रतिष्ठां जगतीमाकाशान्निपपात ह} %5-62-36

\twolineshloka
{स निपत्य महावीरः सर्वैस्तैः परिवारितः}
{हरिर्दधिमुखः पालैः पालानां परमेश्वरः} %5-62-37

\twolineshloka
{स दीनवदनो भूत्वा कृत्वा शिरसि चाञ्जलिम्}
{सुग्रीवस्याशु तौ मूर्ध्ना चरणौ प्रत्यपीडयत्} %5-62-38


॥इत्यार्षे श्रीमद्रामायणे वाल्मीकीये आदिकाव्ये सुन्दरकाण्डे दधिमुखखिलीकारः नाम द्विषष्ठितमः सर्गः ॥५-६२॥
