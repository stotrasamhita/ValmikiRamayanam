\sect{अष्टादशः सर्गः — रावणागमनम्}

\twolineshloka
{तथा विप्रेक्षमाणस्य वनं पुष्पितपादपम्}
{विचिन्वतश्च वैदेहीं किञ्चिच्छेषा निशाभवत्} %5-18-1

\twolineshloka
{षडङ्गवेदविदुषां क्रतुप्रवरयाजिनाम्}
{शुश्राव ब्रह्मघोषान् स विरात्रे ब्रह्मरक्षसाम्} %5-18-2

\twolineshloka
{अथ मङ्गलवादित्रैः शब्दैः श्रोत्रमनोहरैः}
{प्राबोध्यत महाबाहुर्दशग्रीवो महाबलः} %5-18-3

\twolineshloka
{विबुध्य तु महाभागो राक्षसेन्द्रः प्रतापवान्}
{स्रस्तमाल्याम्बरधरो वैदेहीमन्वचिन्तयत्} %5-18-4

\twolineshloka
{भृशं नियुक्तस्तस्यां च मदनेन मदोत्कटः}
{न तु तं राक्षसः कामं शशाकात्मनि गूहितुम्} %5-18-5

\twolineshloka
{स सर्वाभरणैर्युक्तो बिभ्रच्छ्रियमनुत्तमाम्}
{तां नगैर्विविधैर्जुष्टां सर्वपुष्पफलोपगैः} %5-18-6

\twolineshloka
{वृतां पुष्करिणीभिश्च नानापुष्पोपशोभिताम्}
{सदा मत्तैश्च विहगैर्विचित्रां परमाद्भुतैः} %5-18-7

\twolineshloka
{ईहामृगैश्च विविधैर्वृतां दृष्टिमनोहरैः}
{वीथीः सम्प्रेक्षमाणश्च मणिकाञ्चनतोरणाम्} %5-18-8

\twolineshloka
{नानामृगगणाकीर्णां फलैः प्रपतितैर्वृताम्}
{अशोकवनिकामेव प्राविशत् सन्ततद्रुमाम्} %5-18-9

\twolineshloka
{अङ्गनाः शतमात्रं तु तं व्रजन्तमनुव्रजन्}
{महेन्द्रमिव पौलस्त्यं देवगन्धर्वयोषितः} %5-18-10

\twolineshloka
{दीपिकाः काञ्चनीः काश्चिज्जगृहुस्तत्र योषितः}
{वालव्यजनहस्ताश्च तालवृन्तानि चापराः} %5-18-11

\twolineshloka
{काञ्चनैश्चैव भृङ्गारैर्जह्रुः सलिलमग्रतः}
{मण्डलाग्रा बृसीश्चैव गृह्यान्याः पृष्ठतो ययुः} %5-18-12

\twolineshloka
{काचिद् रत्नमयीं पात्रीं पूर्णां पानस्य भ्राजतीम्}
{दक्षिणा दक्षिणेनैव तदा जग्राह पाणिना} %5-18-13

\twolineshloka
{राजहंसप्रतीकाशं छत्रं पूर्णशशिप्रभम्}
{सौवर्णदण्डमपरा गृहीत्वा पृष्ठतो ययौ} %5-18-14

\twolineshloka
{निद्रामदपरीताक्ष्यो रावणस्योत्तमस्त्रियः}
{अनुजग्मुः पतिं वीरं घनं विद्युल्लता इव} %5-18-15

\twolineshloka
{व्याविद्धहारकेयूराः समामृदितवर्णकाः}
{समागलितकेशान्ताः सस्वेदवदनास्तथा} %5-18-16

\twolineshloka
{घूर्णन्त्यो मदशेषेण निद्रया च शुभाननाः}
{स्वेदक्लिष्टाङ्गकुसुमाः समाल्याकुलमूर्धजाः} %5-18-17

\twolineshloka
{प्रयान्तं नैर्ऋतपतिं नार्यो मदिरलोचनाः}
{बहुमानाच्च कामाच्च प्रियभार्यास्तमन्वयुः} %5-18-18

\twolineshloka
{स च कामपराधीनः पतिस्तासां महाबलः}
{सीतासक्तमना मन्दो मन्दाञ्चितगतिर्बभौ} %5-18-19

\twolineshloka
{ततः काञ्चीनिनादं च नूपुराणां च निःस्वनम्}
{शुश्राव परमस्त्रीणां कपिर्मारुतनन्दनः} %5-18-20

\twolineshloka
{तं चाप्रतिमकर्माणमचिन्त्यबलपौरुषम्}
{द्वारदेशमनुप्राप्तं ददर्श हनुमान् कपिः} %5-18-21

\twolineshloka
{दीपिकाभिरनेकाभिः समन्तादवभासितम्}
{गन्धतैलावसिक्ताभिर्ध्रियमाणाभिरग्रतः} %5-18-22

\twolineshloka
{कामदर्पमदैर्युक्तं जिह्मताम्रायतेक्षणम्}
{समक्षमिव कन्दर्पमपविद्धशरासनम्} %5-18-23

\twolineshloka
{मथितामृतफेनाभमरजोवस्त्रमुत्तमम्}
{सपुष्पमवकर्षन्तं विमुक्तं सक्तमङ्गदे} %5-18-24

\twolineshloka
{तं पत्रविटपे लीनः पत्रपुष्पशतावृतः}
{समीपमुपसङ्क्रान्तं विज्ञातुमुपचक्रमे} %5-18-25

\twolineshloka
{अवेक्षमाणस्तु तदा ददर्श कपिकुञ्जरः}
{रूपयौवनसम्पन्ना रावणस्य वरस्त्रियः} %5-18-26

\twolineshloka
{ताभिः परिवृतो राजा सुरूपाभिर्महायशाः}
{तन्मृगद्विजसङ्घुष्टं प्रविष्टः प्रमदावनम्} %5-18-27

\twolineshloka
{क्षीबो विचित्राभरणः शङ्कुकर्णो महाबलः}
{तेन विश्रवसः पुत्रः स दृष्टो राक्षसाधिपः} %5-18-28

\twolineshloka
{वृतः परमनारीभिस्ताराभिरिव चन्द्रमाः}
{तं ददर्श महातेजास्तेजोवन्तं महाकपिः} %5-18-29

\threelineshloka
{रावणोऽयं महाबाहुरिति सञ्चिन्त्य वानरः}
{सोऽयमेव पुरा शेते पुरमध्ये गृहोत्तमे}
{अवप्लुतो महातेजा हनूमान् मारुतात्मजः} %5-18-30

\twolineshloka
{स तथाप्युग्रतेजाः स निर्धूतस्तस्य तेजसा}
{पत्रे गुह्यान्तरे सक्तो मतिमान् संवृतोऽभवत्} %5-18-31

\twolineshloka
{स तामसितकेशान्तां सुश्रोणीं संहतस्तनीम्}
{दिदृक्षुरसितापाङ्गीमुपावर्तत रावणः} %5-18-32


॥इत्यार्षे श्रीमद्रामायणे वाल्मीकीये आदिकाव्ये सुन्दरकाण्डे रावणागमनम् नाम अष्टादशः सर्गः ॥५-१८॥
