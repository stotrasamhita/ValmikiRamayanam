\sect{द्वितीयः सर्गः — निशागमप्रतीक्षा}

\twolineshloka
{स सागरमनाधृष्यमतिक्रम्य महाबलः}
{त्रिकूटस्य तटे लङ्कां स्थितः स्वस्थो ददर्श ह} %5-2-1

\twolineshloka
{ततः पादपमुक्तेन पुष्पवर्षेण वीर्यवान्}
{अभिवृष्टस्ततस्तत्र बभौ पुष्पमयो हरिः} %5-2-2

\twolineshloka
{योजनानां शतं श्रीमांस्तीर्त्वाप्युत्तमविक्रमः}
{अनिःश्वसन् कपिस्तत्र न ग्लानिमधिगच्छति} %5-2-3

\twolineshloka
{शतान्यहं योजनानां क्रमेयं सुबहून्यपि}
{किं पुनः सागरस्यान्तं सङ्ख्यातं शतयोजनम्} %5-2-4

\twolineshloka
{स तु वीर्यवतां श्रेष्ठः प्लवतामपि चोत्तमः}
{जगाम वेगवाँल्लङ्कां लङ्घयित्वा महोदधिम्} %5-2-5

\twolineshloka
{शाद्वलानि च नीलानि गन्धवन्ति वनानि च}
{मधुमन्ति च मध्येन जगाम नगवन्ति च} %5-2-6

\twolineshloka
{शैलांश्च तरुसञ्छन्नान् वनराजीश्च पुष्पिताः}
{अभिचक्राम तेजस्वी हनूमान् प्लवगर्षभः} %5-2-7

\twolineshloka
{स तस्मिन्नचले तिष्ठन् वनान्युपवनानि च}
{स नगाग्रे स्थितां लङ्कां ददर्श पवनात्मजः} %5-2-8

\twolineshloka
{सरलान् कर्णिकारांश्च खर्जूरांश्च सुपुष्पितान्}
{प्रियालान् मुचुलिन्दांश्च कुटजान् केतकानपि} %5-2-9

\twolineshloka
{प्रियङ्गून् गन्धपूर्णांश्च नीपान् सप्तच्छदांस्तथा}
{असनान् कोविदारांश्च करवीरांश्च पुष्पितान्} %5-2-10

\twolineshloka
{पुष्पभारनिबद्धांश्च तथा मुकुलितानपि}
{पादपान् विहगाकीर्णान् पवनाधूतमस्तकान्} %5-2-11

\twolineshloka
{हंसकारण्डवाकीर्णा वापीः पद्मोत्पलावृताः}
{आक्रीडान् विविधान् रम्यान् विविधांश्च जलाशयान्} %5-2-12

\twolineshloka
{सन्ततान् विविधैर्वृक्षैः सर्वर्तुफलपुष्पितैः}
{उद्यानानि च रम्याणि ददर्श कपिकुञ्जरः} %5-2-13

\twolineshloka
{समासाद्य च लक्ष्मीवाँल्लङ्कां रावणपालिताम्}
{परिखाभिः सपद्माभिः सोत्पलाभिरलङ्कृताम्} %5-2-14

\twolineshloka
{सीतापहरणात् तेन रावणेन सुरक्षिताम्}
{समन्ताद् विचरद्भिश्च राक्षसैरुग्रधन्वभिः} %5-2-15

\twolineshloka
{काञ्चनेनावृतां रम्यां प्राकारेण महापुरीम्}
{गृहैश्च गिरिसङ्काशैः शारदाम्बुदसन्निभैः} %5-2-16

\twolineshloka
{पाण्डुराभिः प्रतोलीभिरुच्चाभिरभिसंवृताम्}
{अट्टालकशताकीर्णां पताकाध्वजशोभिताम्} %5-2-17

\twolineshloka
{तोरणैः काञ्चनैर्दिव्यैर्लतापङ्क्तिविराजितैः}
{ददर्श हनुमाँल्लङ्कां देवो देवपुरीमिव} %5-2-18

\twolineshloka
{गिरिमूर्ध्नि स्थितां लङ्कां पाण्डुरैर्भवनैः शुभैः}
{ददर्श स कपिः श्रीमान् पुरीमाकाशगामिव} %5-2-19

\twolineshloka
{पालितां राक्षसेन्द्रेण निर्मितां विश्वकर्मणा}
{प्लवमानामिवाकाशे ददर्श हनुमान् कपिः} %5-2-20

\twolineshloka
{वप्रप्राकारजघनां विपुलाम्बुवनाम्बराम्}
{शतघ्नीशूलकेशान्तामट्टालकावतंसकाम्} %5-2-21

\twolineshloka
{मनसेव कृतां लङ्कां निर्मितां विश्वकर्मणा}
{द्वारमुत्तरमासाद्य चिन्तयामास वानरः} %5-2-22

\twolineshloka
{कैलासनिलयप्रख्यमालिखन्तमिवाम्बरम्}
{ध्रियमाणमिवाकाशमुच्छ्रितैर्भवनोत्तमैः} %5-2-23

\twolineshloka
{सम्पूर्णा राक्षसैर्घोरैर्नागैर्भोगवतीमिव}
{अचिन्त्यां सुकृतां स्पष्टां कुबेराध्युषितां पुरा} %5-2-24

\twolineshloka
{दंष्ट्राभिर्बहुभिः शूरैः शूलपट्टिशपाणिभिः}
{रक्षितां राक्षसैर्घोरैर्गुहामाशीविषैरिव} %5-2-25

\twolineshloka
{तस्याश्च महतीं गुप्तिं सागरं च निरीक्ष्य सः}
{रावणं च रिपुं घोरं चिन्तयामास वानरः} %5-2-26

\twolineshloka
{आगत्यापीह हरयो भविष्यन्ति निरर्थकाः}
{नहि युद्धेन वै लङ्का शक्या जेतुं सुरैरपि} %5-2-27

\twolineshloka
{इमां त्वविषमां लङ्कां दुर्गां रावणपालिताम्}
{प्राप्यापि सुमहाबाहुः किं करिष्यति राघवः} %5-2-28

\twolineshloka
{अवकाशो न साम्नस्तु राक्षसेष्वभिगम्यते}
{न दानस्य न भेदस्य नैव युद्धस्य दृश्यते} %5-2-29

\twolineshloka
{चतुर्णामेव हि गतिर्वानराणां तरस्विनाम्}
{वालिपुत्रस्य नीलस्य मम राज्ञश्च धीमतः} %5-2-30

\twolineshloka
{यावज्जानामि वैदेहीं यदि जीवति वा न वा}
{तत्रैव चिन्तयिष्यामि दृष्ट्वा तां जनकात्मजाम्} %5-2-31

\twolineshloka
{ततः स चिन्तयामास मुहूर्तं कपिकुञ्जरः}
{गिरेः शृङ्गे स्थितस्तस्मिन् रामस्याभ्युदयं ततः} %5-2-32

\twolineshloka
{अनेन रूपेण मया न शक्या रक्षसां पुरी}
{प्रवेष्टुं राक्षसैर्गुप्ता क्रूरैर्बलसमन्वितैः} %5-2-33

\twolineshloka
{महौजसो महावीर्या बलवन्तश्च राक्षसाः}
{वञ्चनीया मया सर्वे जानकीं परिमार्गता} %5-2-34

\twolineshloka
{लक्ष्यालक्ष्येण रूपेण रात्रौ लङ्कापुरी मया}
{प्राप्तकालं प्रवेष्टुं मे कृत्यं साधयितुं महत्} %5-2-35

\twolineshloka
{तां पुरीं तादृशीं दृष्ट्वा दुराधर्षां सुरासुरैः}
{हनूमांश्चिन्तयामास विनिःश्वस्य मुहुर्मुहुः} %5-2-36

\twolineshloka
{केनोपायेन पश्येयं मैथिलीं जनकात्मजाम्}
{अदृष्टो राक्षसेन्द्रेण रावणेन दुरात्मना} %5-2-37

\twolineshloka
{न विनश्येत् कथं कार्यं रामस्य विदितात्मनः}
{एकामेकस्तु पश्येयं रहिते जनकात्मजाम्} %5-2-38

\twolineshloka
{भूताश्चार्था विनश्यन्ति देशकालविरोधिताः}
{विक्लवं दूतमासाद्य तमः सूर्योदये यथा} %5-2-39

\twolineshloka
{अर्थानर्थान्तरे बुद्धिर्निश्चितापि न शोभते}
{घातयन्तीह कार्याणि दूताः पण्डितमानिनः} %5-2-40

\twolineshloka
{न विनश्येत् कथं कार्यं वैक्लव्यं न कथं भवेत्}
{लङ्घनं च समुद्रस्य कथं नु न भवेद् वृथा} %5-2-41

\twolineshloka
{मयि दृष्टे तु रक्षोभी रामस्य विदितात्मनः}
{भवेद् व्यर्थमिदं कार्यं रावणानर्थमिच्छतः} %5-2-42

\twolineshloka
{नहि शक्यं क्वचित् स्थातुमविज्ञातेन राक्षसैः}
{अपि राक्षसरूपेण किमुतान्येन केनचित्} %5-2-43

\twolineshloka
{वायुरप्यत्र नाज्ञातश्चरेदिति मतिर्मम}
{नह्यत्राविदितं किञ्चिद् रक्षसां भीमकर्मणाम्} %5-2-44

\twolineshloka
{इहाहं यदि तिष्ठामि स्वेन रूपेण संवृतः}
{विनाशमुपयास्यामि भर्तुरर्थश्च हास्यति} %5-2-45

\twolineshloka
{तदहं स्वेन रूपेण रजन्यां ह्रस्वतां गतः}
{लङ्कामभिपतिष्यामि राघवस्यार्थसिद्धये} %5-2-46

\twolineshloka
{रावणस्य पुरीं रात्रौ प्रविश्य सुदुरासदाम्}
{प्रविश्य भवनं सर्वं द्रक्ष्यामि जनकात्मजाम्} %5-2-47

\twolineshloka
{इति निश्चित्य हनुमान् सूर्यस्यास्तमयं कपिः}
{आचकाङ्क्षे तदा वीरो वैदेह्या दर्शनोत्सुकः} %5-2-48

\twolineshloka
{सूर्ये चास्तं गते रात्रौ देहं सङ्क्षिप्य मारुतिः}
{वृषदंशकमात्रोऽथ बभूवाद्भुतदर्शनः} %5-2-49

\twolineshloka
{प्रदोषकाले हनुमांस्तूर्णमुत्पत्य वीर्यवान्}
{प्रविवेश पुरीं रम्यां प्रविभक्तमहापथाम्} %5-2-50

\twolineshloka
{प्रासादमालाविततां स्तम्भैः काञ्चनसन्निभैः}
{शातकुम्भनिभैर्जालैर्गन्धर्वनगरोपमाम्} %5-2-51

\twolineshloka
{सप्तभौमाष्टभौमैश्च स ददर्श महापुरीम्}
{तलैः स्फटिकसङ्कीर्णैः कार्तस्वरविभूषितैः} %5-2-52

\twolineshloka
{वैदूर्यमणिचित्रैश्च मुक्ताजालविभूषितैः}
{तैस्तैः शुशुभिरे तानि भवनान्यत्र रक्षसाम्} %5-2-53

\twolineshloka
{काञ्चनानि विचित्राणि तोरणानि च रक्षसाम्}
{लङ्कामुद्योतयामासुः सर्वतः समलङ्कृताम्} %5-2-54

\twolineshloka
{अचिन्त्यामद्भुताकारां दृष्ट्वा लङ्कां महाकपिः}
{आसीद् विषण्णो हृष्टश्च वैदेह्या दर्शनोत्सुकः} %5-2-55

\twolineshloka
{स पाण्डुराविद्धविमानमालिनीं महार्हजाम्बूनदजालतोरणाम्}
{यशस्विनीं रावणबाहुपालितां क्षपाचरैर्भीमबलैः सुपालिताम्} %5-2-56

\twolineshloka
{चन्द्रोऽपि साचिव्यमिवास्य कुर्वंस्तारागणैर्मध्यगतो विराजन्}
{ज्योत्स्नावितानेन वितत्य लोकानुत्तिष्ठतेऽनेकसहस्ररश्मिः} %5-2-57

\twolineshloka
{शङ्खप्रभं क्षीरमृणालवर्णमुद्गच्छमानं व्यवभासमानम्}
{ददर्श चन्द्रं स कपिप्रवीरः पोप्लूयमानं सरसीव हंसम्} %5-2-58


॥इत्यार्षे श्रीमद्रामायणे वाल्मीकीये आदिकाव्ये सुन्दरकाण्डे निशागमप्रतीक्षा नाम द्वितीयः सर्गः ॥५-२॥
