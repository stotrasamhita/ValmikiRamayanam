\sect{चतुःषष्ठितमः सर्गः — हनूमदाद्यागमनम्}

\twolineshloka
{सुग्रीवेणैवमुक्तस्तु हृष्टो दधिमुखः कपिः}
{राघवं लक्ष्मणं चैव सुग्रीवं चाभ्यवादयत्} %5-64-1

\twolineshloka
{स प्रणम्य च सुग्रीवं राघवौ च महाबलौ}
{वानरैः सहितः शूरैर्दिवमेवोत्पपात ह} %5-64-2

\twolineshloka
{स यथैवागतः पूर्वं तथैव त्वरितं गतः}
{निपत्य गगनाद् भूमौ तद् वनं प्रविवेश ह} %5-64-3

\twolineshloka
{स प्रविष्टो मधुवनं ददर्श हरियूथपान्}
{विमदानुद्धतान् सर्वान् मेहमानान् मधूदकम्} %5-64-4

\twolineshloka
{स तानुपागमद् वीरो बद्ध्वा करपुटाञ्जलिम्}
{उवाच वचनं श्लक्ष्णमिदं हृष्टवदङ्गदम्} %5-64-5

\twolineshloka
{सौम्य रोषो न कर्तव्यो यदेभिः परिवारणम्}
{अज्ञानाद् रक्षिभिः क्रोधाद् भवन्तः प्रतिषेधिताः} %5-64-6

\twolineshloka
{श्रान्तो दूरादनुप्राप्तो भक्षयस्व स्वकं मधु}
{युवराजस्त्वमीशश्च वनस्यास्य महाबल} %5-64-7

\twolineshloka
{मौर्ख्यात् पूर्वं कृतो रोषस्तद् भवान् क्षन्तुमर्हति}
{यथैव हि पिता तेऽभूत् पूर्वं हरिगणेश्वरः} %5-64-8

\twolineshloka
{तथा त्वमपि सुग्रीवो नान्यस्तु हरिसत्तम}
{आख्यातं हि मया गत्वा पितृव्यस्य तवानघ} %5-64-9

\twolineshloka
{इहोपयानं सर्वेषामेतेषां वनचारिणाम्}
{भवदागमनं श्रुत्वा सहैभिर्वनचारिभिः} %5-64-10

\twolineshloka
{प्रहृष्टो न तु रुष्टोऽसौ वनं श्रुत्वा प्रधर्षितम्}
{प्रहृष्टो मां पितृव्यस्ते सुग्रीवो वानरेश्वरः} %5-64-11

\twolineshloka
{शीघ्रं प्रेषय सर्वांस्तानिति होवाच पार्थिवः}
{श्रुत्वा दधिमुखस्यैतद् वचनं श्लक्ष्णमङ्गदः} %5-64-12

\twolineshloka
{अब्रवीत् तान् हरिश्रेष्ठो वाक्यं वाक्यविशारदः}
{शङ्के श्रुतोऽयं वृत्तान्तो रामेण हरियूथपाः} %5-64-13

\twolineshloka
{अयं च हर्षादाख्याति तेन जानामि हेतुना}
{तत् क्षमं नेह नः स्थातुं कृते कार्ये परन्तपाः} %5-64-14

\twolineshloka
{पीत्वा मधु यथाकामं विक्रान्ता वनचारिणः}
{किं शेषं गमनं तत्र सुग्रीवो यत्र वानरः} %5-64-15

\twolineshloka
{सर्वे यथा मां वक्ष्यन्ति समेत्य हरिपुङ्गवाः}
{तथास्मि कर्ता कर्तव्ये भवद्भिः परवानहम्} %5-64-16

\twolineshloka
{नाज्ञापयितुमीशोऽहं युवराजोऽस्मि यद्यपि}
{अयुक्तं कृतकर्माणो यूयं धर्षयितुं बलात्} %5-64-17

\twolineshloka
{ब्रुवतश्चाङ्गदस्यैवं श्रुत्वा वचनमुत्तमम्}
{प्रहृष्टमनसो वाक्यमिदमूचुर्वनौकसः} %5-64-18

\twolineshloka
{एवं वक्ष्यति को राजन् प्रभुः सन् वानरर्षभ}
{ऐश्वर्यमदमत्तो हि सर्वोऽहमिति मन्यते} %5-64-19

\twolineshloka
{तव चेदं सुसदृशं वाक्यं नान्यस्य कस्यचित्}
{सन्नतिर्हि तवाख्याति भविष्यच्छुभयोग्यताम्} %5-64-20

\twolineshloka
{सर्वे वयमपि प्राप्तास्तत्र गन्तुं कृतक्षणाः}
{स यत्र हरिवीराणां सुग्रीवः पतिरव्ययः} %5-64-21

\twolineshloka
{त्वया ह्यनुक्तैर्हरिभिर्नैव शक्यं पदात् पदम्}
{क्वचिद् गन्तुं हरिश्रेष्ठ ब्रूमः सत्यमिदं तु ते} %5-64-22

\twolineshloka
{एवं तु वदतां तेषामङ्गदः प्रत्यभाषत}
{साधु गच्छाम इत्युक्त्वा खमुत्पेतुर्महाबलाः} %5-64-23

\twolineshloka
{उत्पतन्तमनूत्पेतुः सर्वे ते हरियूथपाः}
{कृत्वाऽऽकाशं निराकाशं यन्त्रोत्क्षिप्ता इवोपलाः} %5-64-24

\twolineshloka
{अङ्गदं पुरतः कृत्वा हनूमन्तं च वानरम्}
{तेऽम्बरं सहसोत्पत्य वेगवन्तः प्लवङ्गमाः} %5-64-25

\twolineshloka
{विनदन्तो महानादं घना वातेरिता यथा}
{अङ्गदे समनुप्राप्ते सुग्रीवो वानरेश्वरः} %5-64-26

\twolineshloka
{उवाच शोकसन्तप्तं रामं कमललोचनम्}
{समाश्वसिहि भद्रं ते दृष्टा देवी न संशयः} %5-64-27

\twolineshloka
{नागन्तुमिह शक्यं तैरतीतसमयैरिह}
{अङ्गदस्य प्रहर्षाच्च जानामि शुभदर्शन} %5-64-28

\twolineshloka
{न मत्सकाशमागच्छेत् कृत्ये हि विनिपातिते}
{युवराजो महाबाहुः प्लवतामङ्गदो वरः} %5-64-29

\twolineshloka
{यद्यप्यकृतकृत्यानामीदृशः स्यादुपक्रमः}
{भवेत् तु दीनवदनो भ्रान्तविप्लुतमानसः} %5-64-30

\twolineshloka
{पितृपैतामहं चैतत् पूर्वकैरभिरक्षितम्}
{न मे मधुवनं हन्याददृष्ट्वा जनकात्मजाम्} %5-64-31

\twolineshloka
{कौसल्या सुप्रजा राम समाश्वसिहि सुव्रत}
{दृष्टा देवी न सन्देहो न चान्येन हनूमता} %5-64-32

\twolineshloka
{नह्यन्यः कर्मणो हेतुः साधनेऽस्य हनूमतः}
{हनूमतीह सिद्धिश्च मतिश्च मतिसत्तम} %5-64-33

\twolineshloka
{व्यवसायश्च शौर्यं च श्रुतं चापि प्रतिष्ठितम्}
{जाम्बवान् यत्र नेता स्यादङ्गदश्च हरीश्वरः} %5-64-34

\twolineshloka
{हनूमांश्चाप्यधिष्ठाता न तत्र गतिरन्यथा}
{मा भूश्चिन्तासमायुक्तः सम्प्रत्यमितविक्रम} %5-64-35

\twolineshloka
{यदा हि दर्पितोदग्राः सङ्गताः काननौकसः}
{नैषामकृतकार्याणामीदृशः स्यादुपक्रमः} %5-64-36

\twolineshloka
{वनभङ्गेन जानामि मधूनां भक्षणेन च}
{ततः किलकिलाशब्दं शुश्रावासन्नमम्बरे} %5-64-37

\twolineshloka
{हनूमत्कर्मदृप्तानां नदतां काननौकसाम्}
{किष्किन्धामुपयातानां सिद्धिं कथयतामिव} %5-64-38

\twolineshloka
{ततः श्रुत्वा निनादं तं कपीनां कपिसत्तमः}
{आयताञ्चितलाङ्गूलः सोऽभवद्हृष्टमानसः} %5-64-39

\twolineshloka
{आजग्मुस्तेऽपि हरयो रामदर्शनकाङ्क्षिणः}
{अङ्गदं पुरतः कृत्वा हनूमन्तं च वानरम्} %5-64-40

\twolineshloka
{तेऽङ्गदप्रमुखा वीराः प्रहृष्टाश्च मुदान्विताः}
{निपेतुर्हरिराजस्य समीपे राघवस्य च} %5-64-41

\twolineshloka
{हनूमांश्च महाबाहुः प्रणम्य शिरसा ततः}
{नियतामक्षतां देवीं राघवाय न्यवेदयत्} %5-64-42

\twolineshloka
{दृष्टा देवीति हनुमद्वदनादमृतोपमम्}
{आकर्ण्य वचनं रामो हर्षमाप सलक्ष्मणः} %5-64-43

\twolineshloka
{निश्चितार्थं ततस्तस्मिन् सुग्रीवं पवनात्मजे}
{लक्ष्मणः प्रीतिमान् प्रीतं बहुमानादवैक्षत} %5-64-44

\twolineshloka
{प्रीत्या च परयोपेतो राघवः परवीरहा}
{बहुमानेन महता हनूमन्तमवैक्षत} %5-64-45


॥इत्यार्षे श्रीमद्रामायणे वाल्मीकीये आदिकाव्ये सुन्दरकाण्डे हनूमदाद्यागमनम् नाम चतुःषष्ठितमः सर्गः ॥५-६४॥
