\sect{अष्टमः सर्गः — पुष्पकानुवर्णनम्}

\twolineshloka
{स तस्य मध्ये भवनस्य संस्थितो महद्विमानं मणिरत्नचित्रितम्}
{प्रतप्तजाम्बूनदजालकृत्रिमं ददर्श धीमान् पवनात्मजः कपिः} %5-8-1

\twolineshloka
{तदप्रमेयप्रतिकारकृत्रिमं कृतं स्वयं साध्विति विश्वकर्मणा}
{दिवं गते वायुपथे प्रतिष्ठितं व्यराजतादित्यपथस्य लक्ष्म तत्} %5-8-2

\twolineshloka
{न तत्र किञ्चिन्न कृतं प्रयत्नतो न तत्र किञ्चिन्न महार्घरत्नवत्}
{न ते विशेषा नियताः सुरेष्वपि न तत्र किञ्चिन्न महाविशेषवत्} %5-8-3

\twolineshloka
{तपः समाधानपराक्रमार्जितं मनःसमाधानविचारचारिणम्}
{अनेकसंस्थानविशेषनिर्मितं ततस्ततस्तुल्यविशेषनिर्मितम्} %5-8-4

\twolineshloka
{मनः समाधाय तु शीघ्रगामिनं दुरासदं मारुततुल्यगामिनम्}
{महात्मनां पुण्यकृतां महर्द्धिनां यशस्विनामग्ऱ्यमुदामिवालयम्} %5-8-5

\twolineshloka
{विशेषमालम्ब्य विशेषसंस्थितं विचित्रकूटं बहुकूटमण्डितम्}
{मनोऽभिरामं शरदिन्दुनिर्मलं विचित्रकूटं शिखरं गिरेर्यथा} %5-8-6

\twolineshloka
{वहन्ति यत्कुण्डलशोभितानना महाशना व्योमचरानिशाचराः}
{विवृत्तविध्वस्तविशाललोचना महाजवा भूतगणाः सहस्रशः} %5-8-7

\twolineshloka
{वसन्तपुष्पोत्करचारुदर्शनं वसन्तमासादपि चारुदर्शनम्}
{स पुष्पकं तत्र विमानमुत्तमं ददर्श तद् वानरवीरसत्तमः} %5-8-8


॥इत्यार्षे श्रीमद्रामायणे वाल्मीकीये आदिकाव्ये सुन्दरकाण्डे पुष्पकानुवर्णनम् नाम अष्टमः सर्गः ॥५-८॥
