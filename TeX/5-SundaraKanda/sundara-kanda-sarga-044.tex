\sect{चतुश्चत्वारिंशः सर्गः — जम्बुमालिवधः}

\twolineshloka
{संदिष्टो राक्षसेन्द्रेण प्रहस्तस्य सुतो बली}
{जम्बुमाली महादंष्ट्रो निर्जगाम धनुर्धरः} %5-44-1

\twolineshloka
{रक्तमाल्याम्बरधरः स्रग्वी रुचिरकुण्डलः}
{महान् विवृत्तनयनश्चण्डः समरदुर्जयः} %5-44-2

\twolineshloka
{धनुः शक्रधनुःप्रख्यं महद्रुचिरसायकम्}
{विस्फारयानो वेगेन वज्राशनिसमस्वनम्} %5-44-3

\twolineshloka
{तस्य विस्फारघोषेण धनुषो महता दिशः}
{प्रदिशश्च नभश्चॆव सहसा समपूर्यत} %5-44-4

\twolineshloka
{रथेन खरयुक्तेन तमागतमुदीक्ष्य सः}
{हनुमान् वेगसंपन्नो जहर्ष च ननाद च} %5-44-5

\twolineshloka
{तं तोरणविटङ्कस्थं हनुमन्तं महाकपिम्}
{जम्बुमाली महाबाहुर्विव्याध निशितैः शरैः} %5-44-6

\twolineshloka
{अर्धचन्द्रेण वदने शिरस्येकेन कर्णिना}
{बाह्वोर्विव्याध नाराचैर्दशभिस्तं कपीश्वरम्} %5-44-7

\twolineshloka
{तस्य तच्छुशुभे ताम्रं शरेणाभिहतं मुखम्}
{शरदीवाम्बुजम्फुल्लं विद्धं भास्कररश्मिना} %5-44-8

\twolineshloka
{तत्तस्य रक्तं रक्तेन रञ्जितं शुशुभे मुखम्}
{यथाकाशे महापद्मं सिक्तं चन्दनबिन्दुभिः} %5-44-9

\twolineshloka
{चुकोप बाणाभिहतो राक्षसस्य महाकपिः}
{ततः पार्श्वेतिविपुलां ददर्श महतीं शिलाम्} %5-44-10

\twolineshloka
{तरसा तां समुत्पाट्य चिक्षेप बलवद्बली}
{तां शरैर्दशभिः क्रुद्धस्ताडयामास राक्षसः} %5-44-11

\twolineshloka
{विपन्नं कर्म तद्दृष्ट्वा हनुमांश्चण्डविक्रमः}
{सालं विपुलमुत्पाट्य भ्रामयामास वीर्यवान्} %5-44-12

\twolineshloka
{भ्रामयन्तं कपिं दृष्ट्वा सालवृक्षं महाबलम्}
{चिक्षेप सुबहून् बाणान् जम्बुमाली महाबलः} %5-44-13

\twolineshloka
{सालं चतुर्भिश्चिच्छेद वानरं पञ्चभिर्भुजे}
{शिरस्येकेन बाणेन दशभिस्तु स्तनान्तरे} %5-44-14

\twolineshloka
{स शरैः पूरिततनुः क्रोधेन महता वृतः}
{तमेव परिघं गृह्य भ्रामयामास वेगतः} %5-44-15

\twolineshloka
{अतिवेगोतिवेगेन भ्रामयित्वा बलोत्कटः}
{परिघं पातयामास जम्बुमालेर्महोरसि} %5-44-16

\twolineshloka
{तस्य चैव शिरो नास्ति न बाहू न च जानुनी}
{न धनुर्नरथो नाश्वास्तत्रादृश्यन्त नेषवः} %5-44-17

\twolineshloka
{स हतस्तरसा तेन जम्बुमाली महाबलः}
{पपात निहतो भूवौ चूर्णिताङ्गविभूषणः} %5-44-18

\twolineshloka
{जम्बुमालिं च निहतं किङ्करांश्च महाबलान्}
{चुक्रोध रावणः श्रुत्वा कोपसंरक्त लोचनः} %5-44-19

\twolineshloka
{स रोषसंवर्तितताम्रलोचनः प्रहस्त पुत्रे निहते महाबले}
{अमात्यपुत्रानतिवीर्यविक्रमान् समादिदेशाशु निशाचरेश्वरः} %5-44-20


॥इत्यार्षे श्रीमद्रामायणे वाल्मीकीये आदिकाव्ये सुन्दरकाण्डे जम्बुमालिवधः नाम चतुश्चत्वारिंशः सर्गः ॥५-४४॥
