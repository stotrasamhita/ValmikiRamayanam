\sect{चत्वारिंशः सर्गः — हनूमत्प्रेषणम्}

\twolineshloka
{श्रुत्वा तु वचनं तस्य वायुसूनोर्महात्मनः}
{उवाचात्महितं वाक्यं सीता सुरसुतोपमा} %5-40-1

\twolineshloka
{त्वां दृष्ट्वा प्रियवक्तारं सम्प्रहृष्यामि वानर}
{अर्धसंजातसस्येव वृष्टिं प्राप्य वसुंधरा} %5-40-2

\twolineshloka
{यथा तं पुरुषव्याघ्रं गात्रैः शोकाभिकर्शितैः}
{संस्पृशेयं सकामाहं तथा कुरु दयां मयि} %5-40-3

\twolineshloka
{अभिज्ञानं च रामस्य दद्या हरिगणोत्तम}
{क्षिप्तामिषीकां काकस्य कोपादेकाक्षिशातनीम्} %5-40-4

\twolineshloka
{मनःशिलायास्तिलको गण्डपार्श्वे निवेशितः}
{त्वया प्रणष्टे तिलके तं किल स्मर्तुमर्हसि} %5-40-5

\twolineshloka
{स वीर्यवान् कथं सीतां हृतां समनुमन्यसे}
{वसन्तीं रक्षसां मध्ये महेन्द्रवरुणोपम} %5-40-6

\twolineshloka
{एष चूडामणिर्दिव्यो मया सुपरिरक्षितः}
{एतं दृष्ट्वा प्रहृष्यामि व्यसने त्वामिवानघ} %5-40-7

\twolineshloka
{एष निर्यातितः श्रीमान् मया ते वारिसम्भवः}
{अतः परं न शक्ष्यामि जीवितुं शोकलालसा} %5-40-8

\twolineshloka
{असह्यानि च दुःखानि वाचश्च हृदयच्छिदः}
{राक्षसैः सह संवासं त्वत्कृते मर्षयाम्यहम्} %5-40-9

\twolineshloka
{धारयिष्यामि मासं तु जीवितं शत्रुसूदन}
{मासादूर्ध्वं न जीविष्ये त्वया हीना नृपात्मज} %5-40-10

\twolineshloka
{घोरो राक्षसराजोऽयं दृष्टिश्च न सुखा मयि}
{त्वां च श्रुत्वा विषज्जन्तं न जीवेयमपि क्षणम्} %5-40-11

\twolineshloka
{वैदेह्या वचनं श्रुत्वा करुणं साश्रुभाषितम्}
{अथाब्रवीन्महातेजा हनूमान् मारुतात्मजः} %5-40-12

\twolineshloka
{त्वच्छोकविमुखो रामो देवि सत्येन ते शपे}
{रामे शोकाभिभूते तु लक्ष्मणः परितप्यते} %5-40-13

\twolineshloka
{दृष्टा कथंचिद् भवती न कालः परिदेवितुम्}
{इमं मुहूर्तं दुःखानामन्तं द्रक्ष्यसि भामिनि} %5-40-14

\twolineshloka
{तावुभौ पुरुषव्याघ्रौ राजपुत्रावनिन्दितौ}
{त्वद्दर्शनकृतोत्साहौ लङ्कां भस्मीकरिष्यतः} %5-40-15

\twolineshloka
{हत्वा तु समरे रक्षो रावणं सहबान्धवैः}
{राघवौ त्वां विशालाक्षि स्वां पुरीं प्रति नेष्यतः} %5-40-16

\twolineshloka
{यत्तु रामो विजानीयादभिज्ञानमनिन्दिते}
{प्रीतिसंजननं भूयस्तस्य त्वं दातुमर्हसि} %5-40-17

\twolineshloka
{साब्रवीद् दत्तमेवाहो मयाभिज्ञानमुत्तमम्}
{एतदेव हि रामस्य दृष्ट्वा यत्नेन भूषणम्} %5-40-18

\twolineshloka
{श्रद्धेयं हनुमन् वाक्यं तव वीर भविष्यति}
{स तं मणिवरं गृह्य श्रीमान् प्लवगसत्तमः} %5-40-19

\twolineshloka
{प्रणम्य शिरसा देवीं गमनायोपचक्रमे}
{तमुत्पातकृतोत्साहमवेक्ष्य हरियूथपम्} %5-40-20

\twolineshloka
{वर्धमानं महावेगमुवाच जनकात्मजा}
{अश्रुपूर्णमुखी दीना बाष्पगद्गदया गिरा} %5-40-21

\twolineshloka
{हनूमन् सिंहसंकाशौ भ्रातरौ रामलक्ष्मणौ}
{सुग्रीवं च सहामात्यं सर्वान् ब्रूया अनामयम्} %5-40-22

\twolineshloka
{यथा च स महाबाहुर्मां तारयति राघवः}
{अस्माद् दुःखाम्बुसंरोधात् त्वं समाधातुमर्हसि} %5-40-23

\twolineshloka
{इदं च तीव्रं मम शोकवेगं रक्षोभिरेभिः परिभर्त्सनं च}
{ब्रूयास्तु रामस्य गतः समीपं शिवश्च तेऽध्वास्तु हरिप्रवीर} %5-40-24

\twolineshloka
{स राजपुत्र्या प्रतिवेदितार्थः कपिः कृतार्थः परिहृष्टचेताः}
{तदल्पशेषं प्रसमीक्ष्य कार्यं दिशं ह्युदीचीं मनसा जगाम} %5-40-25


॥इत्यार्षे श्रीमद्रामायणे वाल्मीकीये आदिकाव्ये सुन्दरकाण्डे हनूमत्प्रेषणम् नाम चत्वारिंशः सर्गः ॥५-४०॥
