\sect{सप्तपञ्चाशः सर्गः — हनूमत्प्रत्यागमनम्}

\twolineshloka
{आप्लुत्य च महावेगः पक्षवानिव पर्वतः}
{भुजङ्गयक्षगन्धर्वप्रबुद्धकमलोत्पलम्} %5-57-1

\twolineshloka
{स चन्द्रकुमुदं रम्यं सार्ककारण्डवं शुभम्}
{तिष्यश्रवणकादम्बमभ्रशैवलशाद्वलम्} %5-57-2

\twolineshloka
{पुनर्वसुमहामीनं लोहिताङ्गमहाग्रहम्}
{ऐरावतमहाद्वीपं स्वातीहंसविलासितम्} %5-57-3

\twolineshloka
{वातसंघातजालोर्मिचन्द्रांशुशिशिराम्बुमत्}
{हनूमानपरिश्रान्तः पुप्लुवे गगनार्णवम्} %5-57-4

\twolineshloka
{ग्रसमान इवाकाशं ताराधिपमिवोल्लिखन्}
{हरन्निव सनक्षत्रं गगनं सार्कमण्डलम्} %5-57-5

\twolineshloka
{अपारमपरिश्रान्तश्चाम्बुधिं समगाहत}
{हनूमान् मेघजालानि विकर्षन्निव गच्छति} %5-57-6

\twolineshloka
{पाण्डुरारुणवर्णानि नीलमाञ्जिष्ठकानि च}
{हरितारुणवर्णानि महाभ्राणि चकाशिरे} %5-57-7

\twolineshloka
{प्रविशन्नभ्रजालानि निष्क्रमंश्च पुनः पुनः}
{प्रकाशश्चाप्रकाशश्च चन्द्रमा इव दृश्यते} %5-57-8

\twolineshloka
{विविधाभ्रघनापन्नगोचरो धवलाम्बरः}
{दृश्यादृश्यतनुर्वीरस्तथा चन्द्रायतेऽम्बरे} %5-57-9

\twolineshloka
{तार्क्ष््यायमाणो गगने स बभौ वायुनन्दनः}
{दारयन् मेघवृन्दानि निष्पतंश्च पुनः पुनः} %5-57-10

\twolineshloka
{नदन् नादेन महता मेघस्वनमहास्वनः}
{प्रवरान् राक्षसान् हत्वा नाम विश्राव्य चात्मनः} %5-57-11

\twolineshloka
{आकुलां नगरीं कृत्वा व्यथयित्वा च रावणम्}
{अर्दयित्वा महावीरान् वैदेहीमभिवाद्य च} %5-57-12

\twolineshloka
{आजगाम महातेजाः पुनर्मध्येन सागरम्}
{पर्वतेन्द्रं सुनाभं च समुपस्पृश्य वीर्यवान्} %5-57-13

\twolineshloka
{ज्यामुक्त इव नाराचो महावेगोऽभ्युपागमत्}
{स किंचिदारात् सम्प्राप्तः समालोक्य महागिरिम्} %5-57-14

\twolineshloka
{महेन्द्रं मेघसंकाशं ननाद स महाकपिः}
{स पूरयामास कपिर्दिशो दश समन्ततः} %5-57-15

\twolineshloka
{नदन् नादेन महता मेघस्वनमहास्वनः}
{स तं देशमनुप्राप्तः सुहृद्दर्शनलालसः} %5-57-16

\twolineshloka
{ननाद सुमहानादं लाङ्गूलं चाप्यकम्पयत्}
{तस्य नानद्यमानस्य सुपर्णाचरिते पथि} %5-57-17

\twolineshloka
{फलतीवास्य घोषेण गगनं सार्कमण्डलम्}
{ये तु तत्रोत्तरे कूले समुद्रस्य महाबलाः} %5-57-18

\threelineshloka
{पूर्वं संविष्ठिताः शूरा वायुपुत्रदिदृक्षवः}
{महतो वायुनुन्नस्य तोयदस्येव निःस्वनम्}
{शुश्रुवुस्ते तदा घोषमूरुवेगं हनूमतः} %5-57-19

\twolineshloka
{ते दीनमनसः सर्वे शुश्रुवुः काननौकसः}
{वानरेन्द्रस्य निर्घोषं पर्जन्यनिनदोपमम्} %5-57-20

\twolineshloka
{निशम्य नदतो नादं वानरास्ते समन्ततः}
{बभूवुरुत्सुकाः सर्वे सुहृद्दर्शनकाङ्क्षिणः} %5-57-21

\twolineshloka
{जाम्बवान् स हरिश्रेष्ठः प्रीतिसंहृष्टमानसः}
{उपामन्त्र्य हरीन् सर्वानिदं वचनमब्रवीत्} %5-57-22

\twolineshloka
{सर्वथा कृतकार्योऽसौ हनूमान् नात्र संशयः}
{न ह्यस्याकृतकार्यस्य नाद एवंविधो भवेत्} %5-57-23

\twolineshloka
{तस्य बाहूरुवेगं च निनादं च महात्मनः}
{निशम्य हरयो हृष्टाः समुत्पेतुर्यतस्ततः} %5-57-24

\twolineshloka
{ते नगाग्रान्नगाग्राणि शिखराच्छिखराणि च}
{प्रहृष्टाः समपद्यन्त हनूमन्तं दिदृक्षवः} %5-57-25

\twolineshloka
{ते प्रीताः पादपाग्रेषु गृह्य शाखामवस्थिताः}
{वासांसि च प्रकाशानि समाविध्यन्त वानराः} %5-57-26

\twolineshloka
{गिरिगह्वरसंलीनो यथा गर्जति मारुतः}
{एवं जगर्ज बलवान् हनूमान् मारुतात्मजः} %5-57-27

\twolineshloka
{तमभ्रघनसंकाशमापतन्तं महाकपिम्}
{दृष्ट्वा ते वानराः सर्वे तस्थुः प्राञ्जलयस्तदा} %5-57-28

\twolineshloka
{ततस्तु वेगवान् वीरो गिरेर्गिरिनिभः कपिः}
{निपपात गिरेस्तस्य शिखरे पादपाकुले} %5-57-29

\twolineshloka
{हर्षेणापूर्यमाणोऽसौ रम्ये पर्वतनिर्झरे}
{छिन्नपक्ष इवाकाशात् पपात धरणीधरः} %5-57-30

\twolineshloka
{ततस्ते प्रीतमनसः सर्वे वानरपुङ्गवाः}
{हनूमन्तं महात्मानं परिवार्योपतस्थिरे} %5-57-31

\twolineshloka
{परिवार्य च ते सर्वे परां प्रीतिमुपागताः}
{प्रहृष्टवदनाः सर्वे तमागतमुपागमन्} %5-57-32

\twolineshloka
{उपायनानि चादाय मूलानि च फलानि च}
{प्रत्यर्चयन् हरिश्रेष्ठं हरयो मारुतात्मजम्} %5-57-33

\twolineshloka
{विनेदुर्मुदिताः केचित् केचित् किलकिलां तथा}
{हृष्टाः पादपशाखाश्च आनिन्युर्वानरर्षभाः} %5-57-34

\twolineshloka
{हनूमांस्तु गुरून् वृद्धाञ्जाम्बवत्प्रमुखांस्तदा}
{कुमारमङ्गदं चैव सोऽवन्दत महाकपिः} %5-57-35

\twolineshloka
{स ताभ्यां पूजितः पूज्यः कपिभिश्च प्रसादितः}
{दृष्टा देवीति विक्रान्तः संक्षेपेण न्यवेदयत्} %5-57-36

\twolineshloka
{निषसाद च हस्तेन गृहीत्वा वालिनः सुतम्}
{रमणीये वनोद्देशे महेन्द्रस्य गिरेस्तदा} %5-57-37

\twolineshloka
{हनूमानब्रवीत् पृष्टस्तदा तान् वानरर्षभान्}
{अशोकवनिकासंस्था दृष्टा सा जनकात्मजा} %5-57-38

\twolineshloka
{रक्ष्यमाणा सुघोराभी राक्षसीभिरनिन्दिता}
{एकवेणीधरा बाला रामदर्शनलालसा} %5-57-39

\twolineshloka
{उपवासपरिश्रान्ता मलिना जटिला कृशा}
{ततो दृष्टेति वचनं महार्थममृतोपमम्} %5-57-40

\twolineshloka
{निशम्य मारुतेः सर्वे मुदिता वानराभवन्}
{क्ष्वेडन्त्यन्ये नदन्त्यन्ये गर्जन्त्यन्ये महाबलाः} %5-57-41

\twolineshloka
{चक्रुः किलकिलामन्ये प्रतिगर्जन्ति चापरे}
{केचिदुच्छ्रितलाङ्गूलाः प्रहृष्टाः कपिकुञ्जराः} %5-57-42

\twolineshloka
{आयताञ्चितदीर्घाणि लाङ्गूलानि प्रविव्यधुः}
{अपरे तु हनूमन्तं श्रीमन्तं वानरोत्तमम्} %5-57-43

\twolineshloka
{आप्लुत्य गिरिशृङ्गेषु संस्पृशन्ति स्म हर्षिताः}
{उक्तवाक्यं हनूमन्तमङ्गदस्तु तदाब्रवीत्} %5-57-44

\twolineshloka
{सर्वेषां हरिवीराणां मध्ये वाचमनुत्तमाम्}
{सत्त्वे वीर्ये न ते कश्चित् समो वानर विद्यते} %5-57-45

\twolineshloka
{यदवप्लुत्य विस्तीर्णं सागरं पुनरागतः}
{जीवितस्य प्रदाता नस्त्वमेको वानरोत्तम} %5-57-46

\twolineshloka
{त्वत्प्रसादात् समेष्यामः सिद्धार्था राघवेण ह}
{अहो स्वामिनि ते भक्तिरहो वीर्यमहो धृतिः} %5-57-47

\twolineshloka
{दिष्ट्या दृष्टा त्वया देवी रामपत्नी यशस्विनी}
{दिष्ट्या त्यक्ष्यति काकुत्स्थः शोकं सीतावियोगजम्} %5-57-48

\twolineshloka
{ततोऽङ्गदं हनूमन्तं जाम्बवन्तं च वानराः}
{परिवार्य प्रमुदिता भेजिरे विपुलाः शिलाः} %5-57-49

\twolineshloka
{उपविष्टा गिरेस्तस्य शिलासु विपुलासु ते}
{श्रोतुकामाः समुद्रस्य लङ्घनं वानरोत्तमाः} %5-57-50

\twolineshloka
{दर्शनं चापि लङ्कायाः सीताया रावणस्य च}
{तस्थुः प्राञ्जलयः सर्वे हनूमद्वदनोन्मुखाः} %5-57-51

\twolineshloka
{तस्थौ तत्राङ्गदः श्रीमान् वानरैर्बहुभिर्वृतः}
{उपास्यमानो विबुधैर्दिवि देवपतिर्यथा} %5-57-52

\twolineshloka
{हनूमता कीर्तिमता यशस्विना तथाङ्गदेनाङ्गदनद्धबाहुना}
{मुदा तदाध्यासितमुन्नतं महन्महीधराग्रं ज्वलितं श्रियाभवत्} %5-57-53


॥इत्यार्षे श्रीमद्रामायणे वाल्मीकीये आदिकाव्ये सुन्दरकाण्डे हनूमत्प्रत्यागमनम् नाम सप्तपञ्चाशः सर्गः ॥५-५७॥
