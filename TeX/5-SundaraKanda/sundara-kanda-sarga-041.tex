\sect{एकचत्वारिंशः सर्गः — प्रमदावनभञ्जनम्}

\twolineshloka
{स च वाग्भिः प्रशस्ताभिर्गमिष्यन् पूजितस्तया}
{तस्माद् देशादपाक्रम्य चिन्तयामास वानरः} %5-41-1

\twolineshloka
{अल्पशेषमिदं कार्यं दृष्टेयमसितेक्षणा}
{त्रीनुपायानतिक्रम्य चतुर्थ इह दृश्यते} %5-41-2

\twolineshloka
{न साम रक्षःसु गुणाय कल्पते न दानमर्थोपचितेषु युज्यते}
{न भेदसाध्या बलदर्पिता जनाः पराक्रमस्त्वेष ममेह रोचते} %5-41-3

\twolineshloka
{न चास्य कार्यस्य पराक्रमादृते विनिश्चयः कश्चिदिहोपपद्यते}
{हतप्रवीराश्च रणे तु राक्षसाः कथंचिदीयुर्यदिहाद्य मार्दवम्} %5-41-4

\twolineshloka
{कार्ये कर्मणि निर्वृत्ते यो बहून्यपि साधयेत्}
{पूर्वकार्याविरोधेन स कार्यं कर्तुमर्हति} %5-41-5

\twolineshloka
{न ह्येकः साधको हेतुः स्वल्पस्यापीह कर्मणः}
{यो ह्यर्थं बहुधा वेद स समर्थोऽर्थसाधने} %5-41-6

\twolineshloka
{इहैव तावत्कृतनिश्चयो ह्यहं व्रजेयमद्य प्लवगेश्वरालयम्}
{परात्मसम्मर्दविशेषतत्त्ववित् ततः कृतं स्यान्मम भर्तृशासनम्} %5-41-7

\twolineshloka
{कथं नु खल्वद्य भवेत् सुखागतं प्रसह्य युद्धं मम राक्षसैः सह}
{तथैव खल्वात्मबलं च सारवत् समानयेन्मां च रणे दशाननः} %5-41-8

\twolineshloka
{ततः समासाद्य रणे दशाननं समन्त्रिवर्गं सबलं सयायिनम्}
{हृदि स्थितं तस्य मतं बलं च सुखेन मत्वाहमितः पुनर्व्रजे} %5-41-9

\twolineshloka
{इदमस्य नृशंसस्य नन्दनोपममुत्तमम्}
{वनं नेत्रमनःकान्तं नानाद्रुमलतायुतम्} %5-41-10

\twolineshloka
{इदं विध्वंसयिष्यामि शुष्कं वनमिवानलः}
{अस्मिन् भग्ने ततः कोपं करिष्यति स रावणः} %5-41-11

\twolineshloka
{ततो महत्साश्वमहारथद्विपं बलं समानेष्यति राक्षसाधिपः}
{त्रिशूलकालायसपट्टिशायुधं ततो महद्युद्धमिदं भविष्यति} %5-41-12

\twolineshloka
{अहं च तैः संयति चण्डविक्रमैः समेत्य रक्षोभिरभङ्गविक्रमः}
{निहत्य तद् रावणचोदितं बलं सुखं गमिष्यामि हरीश्वरालयम्} %5-41-13

\twolineshloka
{ततो मारुतवत् क्रुद्धो मारुतिर्भीमविक्रमः}
{ऊरुवेगेन महता द्रुमान् क्षेप्तुमथारभत्} %5-41-14

\twolineshloka
{ततस्तद्धनुमान् वीरो बभञ्ज प्रमदावनम्}
{मत्तद्विजसमाघुष्टं नानाद्रुमलतायुतम्} %5-41-15

\twolineshloka
{तद्वनं मथितैर्वृक्षैर्भिन्नैश्च सलिलाशयैः}
{चूर्णितैः पर्वताग्रैश्च बभूवाप्रियदर्शनम्} %5-41-16

\twolineshloka
{नानाशकुन्तविरुतैः प्रभिन्नसलिलाशयैः}
{ताम्रैः किसलयैः क्लान्तैः क्लान्तद्रुमलतायुतैः} %5-41-17

\twolineshloka
{न बभौ तद् वनं तत्र दावानलहतं यथा}
{व्याकुलावरणा रेजुर्विह्वला इव ता लताः} %5-41-18

\twolineshloka
{लतागृहैश्चित्रगृहैश्च सादितैर्व्यालैर्मृगैरार्तरवैश्च पक्षिभिः}
{शिलागृहैरुन्मथितैस्तथा गृहैः प्रणष्टरूपं तदभून्महद् वनम्} %5-41-19

\twolineshloka
{सा विह्वलाशोकलताप्रताना वनस्थली शोकलताप्रताना}
{जाता दशास्यप्रमदावनस्य कपेर्बलाद्धि प्रमदावनस्य} %5-41-20

\twolineshloka
{ततः स कृत्वा जगतीपतेर्महान् महद् व्यलीकं मनसो महात्मनः}
{युयुत्सुरेको बहुभिर्महाबलैः श्रिया ज्वलंस्तोरणमाश्रितः कपिः} %5-41-21


॥इत्यार्षे श्रीमद्रामायणे वाल्मीकीये आदिकाव्ये सुन्दरकाण्डे प्रमदावनभञ्जनम् नाम एकचत्वारिंशः सर्गः ॥५-४१॥
