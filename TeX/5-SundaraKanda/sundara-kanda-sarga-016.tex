\sect{षोडशः सर्गः — हनूमत्परीतापः}

\twolineshloka
{प्रशस्य तु प्रशस्तव्यां सीतां तां हरिपुंगवः}
{गुणाभिरामं रामं च पुनश्चिन्तापरोऽभवत्} %5-16-1

\twolineshloka
{स मुहूर्तमिव ध्यात्वा बाष्पपर्याकुलेक्षणः}
{सीतामाश्रित्य तेजस्वी हनूमान् विललाप ह} %5-16-2

\twolineshloka
{मान्या गुरुविनीतस्य लक्ष्मणस्य गुरुप्रिया}
{यदि सीता हि दुःखार्ता कालो हि दुरतिक्रमः} %5-16-3

\twolineshloka
{रामस्य व्यवसायज्ञा लक्ष्मणस्य च धीमतः}
{नात्यर्थं क्षुभ्यते देवी गंगेव जलदागमे} %5-16-4

\twolineshloka
{तुल्यशीलवयोवृत्तां तुल्याभिजनलक्षणाम्}
{राघवोऽर्हति वैदेहीं तं चेयमसितेक्षणा} %5-16-5

\twolineshloka
{तां दृष्ट्वा नवहेमाभां लोककान्तामिव श्रियम्}
{जगाम मनसा रामं वचनं चेदमब्रवीत्} %5-16-6

\twolineshloka
{अस्या हेतोर्विशालाक्ष्या हतो वाली महाबलः}
{रावणप्रतिमो वीर्ये कबन्धश्च निपातितः} %5-16-7

\twolineshloka
{विराधश्च हतः संख्ये राक्षसो भीमविक्रमः}
{वने रामेण विक्रम्य महेन्द्रेणेव शम्बरः} %5-16-8

\twolineshloka
{चतुर्दश सहस्राणि रक्षसां भीमकर्मणाम्}
{निहतानि जनस्थाने शरैरग्निशिखोपमैः} %5-16-9

\twolineshloka
{खरश्च निहतः संख्ये त्रिशिराश्च निपातितः}
{दूषणश्च महातेजा रामेण विदितात्मना} %5-16-10

\twolineshloka
{ऐश्वर्यं वानराणां च दुर्लभं वालिपालितम्}
{अस्या निमित्ते सुग्रीवः प्राप्तवाँल्लोकविश्रुतः} %5-16-11

\twolineshloka
{सागरश्च मयाऽऽक्रान्तः श्रीमान् नदनदीपतिः}
{अस्या हेतोर्विशालाक्ष्याः पुरी चेयं निरीक्षिता} %5-16-12

\twolineshloka
{यदि रामः समुद्रान्तां मेदिनीं परिवर्तयेत्}
{अस्याः कृते जगच्चापि युक्तमित्येव मे मतिः} %5-16-13

\twolineshloka
{राज्यं वा त्रिषु लोकेषु सीता वा जनकात्मजा}
{त्रैलोक्यराज्यं सकलं सीताया नाप्नुयात् कलाम्} %5-16-14

\twolineshloka
{इयं सा धर्मशीलस्य जनकस्य महात्मनः}
{सुता मैथिलराजस्य सीता भर्तृदृढव्रता} %5-16-15

\twolineshloka
{उत्थिता मेदिनीं भित्त्वा क्षेत्रे हलमुखक्षते}
{पद्मरेणुनिभैः कीर्णा शुभैः केदारपांसुभिः} %5-16-16

\twolineshloka
{विक्रान्तस्यार्यशीलस्य संयुगेष्वनिवर्तिनः}
{स्नुषा दशरथस्यैषा ज्येष्ठा राज्ञो यशस्विनी} %5-16-17

\twolineshloka
{धर्मज्ञस्य कृतज्ञस्य रामस्य विदितात्मनः}
{इयं सा दयिता भार्या राक्षसीवशमागता} %5-16-18

\twolineshloka
{सर्वान् भोगान् परित्यज्य भर्तृस्नेहबलात् कृता}
{अचिन्तयित्वा कष्टानि प्रविष्टा निर्जनं वनम्} %5-16-19

\twolineshloka
{संतुष्टा फलमूलेन भर्तृशुश्रूषणापरा}
{या परां भजते प्रीतिं वनेऽपि भवने यथा} %5-16-20

\twolineshloka
{सेयं कनकवर्णांगी नित्यं सुस्मितभाषिणी}
{सहते यातनामेतामनर्थानामभागिनी} %5-16-21

\twolineshloka
{इमां तु शीलसम्पन्नां द्रष्टुमिच्छति राघवः}
{रावणेन प्रमथितां प्रपामिव पिपासितः} %5-16-22

\twolineshloka
{अस्या नूनं पुनर्लाभाद् राघवः प्रीतिमेष्यति}
{राजा राज्यपरिभ्रष्टः पुनः प्राप्येव मेदिनीम्} %5-16-23

\twolineshloka
{कामभोगैः परित्यक्ता हीना बन्धुजनेन च}
{धारयत्यात्मनो देहं तत्समागमकाङ्क्षिणी} %5-16-24

\twolineshloka
{नैषा पश्यति राक्षस्यो नेमान् पुष्पफलद्रुमान्}
{एकस्थहृदया नूनं राममेवानुपश्यति} %5-16-25

\twolineshloka
{भर्ता नाम परं नार्याः शोभनं भूषणादपि}
{एषा हि रहिता तेन शोभनार्हा न शोभते} %5-16-26

\twolineshloka
{दुष्करं कुरुते रामो हीनो यदनया प्रभुः}
{धारयत्यात्मनो देहं न दुःखेनावसीदति} %5-16-27

\twolineshloka
{इमामसितकेशान्तां शतपत्रनिभेक्षणाम्}
{सुखार्हां दुःखितां ज्ञात्वा ममापि व्यथितं मनः} %5-16-28

\twolineshloka
{क्षितिक्षमा पुष्करसंनिभेक्षणा या रक्षिता राघवलक्ष्मणाभ्याम्}
{सा राक्षसीभिर्विकृतेक्षणाभिः संरक्ष्यते सम्प्रति वृक्षमूले} %5-16-29

\twolineshloka
{हिमहतनलिनीव नष्टशोभा व्यसनपरम्परया निपीड्यमाना}
{सहचररहितेव चक्रवाकी जनकसुता कृपणां दशां प्रपन्ना} %5-16-30

\twolineshloka
{अस्या हि पुष्पावनताग्रशाखाः शोकं दृढं वै जनयन्त्यशोकाः}
{हिमव्यपायेन च शीतरश्मिरभ्युत्थितो नैकसहस्ररश्मिः} %5-16-31

\twolineshloka
{इत्येवमर्थं कपिरन्ववेक्ष्य सीतेयमित्येव तु जातबुद्धिः}
{संश्रित्य तस्मिन् निषसाद वृक्षे बली हरीणामृषभस्तरस्वी} %5-16-32


॥इत्यार्षे श्रीमद्रामायणे वाल्मीकीये आदिकाव्ये सुन्दरकाण्डे हनूमत्परीतापः नाम षोडशः सर्गः ॥५-१६॥
