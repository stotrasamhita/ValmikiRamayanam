\sect{षष्ठितमः सर्गः — अङ्गदजाम्बवत्संवादः}

\twolineshloka
{तस्य तद् वचनं श्रुत्वा वालिसूनुरभाषत}
{अश्विपुत्रौ महावेगौ बलवन्तौ प्लवङ्गमौ} %5-60-1

\twolineshloka
{पितामहवरोत्सेकात् परमं दर्पमास्थितौ}
{अश्विनोर्माननार्थं हि सर्वलोकपितामहः} %5-60-2

\twolineshloka
{सर्वावध्यत्वमतुलमनयोर्दत्तवान् पुरा}
{वरोत्सेकेन मत्तौ च प्रमथ्य महतीं चमूम्} %5-60-3

\twolineshloka
{सुराणाममृतं वीरौ पीतवन्तौ महाबलौ}
{एतावेव हि सङ्क्रुद्धौ सवाजिरथकुञ्जराम्} %5-60-4

\twolineshloka
{लङ्कां नाशयितुं शक्तौ सर्वे तिष्ठन्तु वानराः}
{अहमेकोऽपि पर्याप्तः सराक्षसगणां पुरीम्} %5-60-5

\twolineshloka
{तां लङ्कां तरसा हन्तुं रावणं च महाबलम्}
{किं पुनः सहितो वीरैर्बलवद्भिः कृतात्मभिः} %5-60-6

\twolineshloka
{कृतास्त्रैः प्लवगैः शक्तैर्भवद्भिर्विजयैषिभिः}
{वायुसूनोर्बलेनैव दग्धा लङ्केति नः श्रुतम्} %5-60-7

\twolineshloka
{दृष्टा देवी न चानीता इति तत्र निवेदितुम्}
{न युक्तमिव पश्यामि भवद्भिः ख्यातपौरुषैः} %5-60-8

\twolineshloka
{नहि वः प्लवने कश्चिन्नापि कश्चित् पराक्रमे}
{तुल्यः सामरदैत्येषु लोकेषु हरिसत्तमाः} %5-60-9

\twolineshloka
{जित्वा लङ्कां सरक्षौघां हत्वा तं रावणं रणे}
{सीतामादाय गच्छामः सिद्धार्था हृष्टमानसाः} %5-60-10

\twolineshloka
{तेष्वेवं हतवीरेषु राक्षसेषु हनूमता}
{किमन्यदत्र कर्तव्यं गृहीत्वा याम जानकीम्} %5-60-11

\twolineshloka
{रामलक्ष्मणयोर्मध्ये न्यस्याम जनकात्मजाम्}
{किं व्यलीकैस्तु तान् सर्वान् वानरान् वानरर्षभान्} %5-60-12

\twolineshloka
{वयमेव हि गत्वा तान् हत्वा राक्षसपुङ्गवान्}
{राघवं द्रष्टुमर्हामः सुग्रीवं सहलक्ष्मणम्} %5-60-13

\twolineshloka
{तमेवं कृतसङ्कल्पं जाम्बवान् हरिसत्तमः}
{उवाच परमप्रीतो वाक्यमर्थवदर्थवित्} %5-60-14

\twolineshloka
{नैषा बुद्धिर्महाबुद्धे यद् ब्रवीषि महाकपे}
{विचेतुं वयमाज्ञप्ता दक्षिणां दिशमुत्तमाम्} %5-60-15

\twolineshloka
{नानेतुं कपिराजेन नैव रामेण धीमता}
{कथञ्चिन्निर्जितां सीतामस्माभिर्नाभिरोचयेत्} %5-60-16

\twolineshloka
{राघवो नृपशार्दूलः कुलं व्यपदिशन् स्वकम्}
{प्रतिज्ञाय स्वयं राजा सीताविजयमग्रतः} %5-60-17

\twolineshloka
{सर्वेषां कपिमुख्यानां कथं मिथ्या करिष्यति}
{विफलं कर्म च कृतं भवेत् तुष्टिर्न तस्य च} %5-60-18

\threelineshloka
{वृथा च दर्शितं वीर्यं भवेद् वानरपुङ्गवाः}
{तस्माद् गच्छाम वै सर्वे यत्र रामः सलक्ष्मणः}
{सुग्रीवश्च महातेजाः कार्यस्यास्य निवेदने} %5-60-19

\twolineshloka
{न तावदेषा मतिरक्षमा नो यथा भवान् पश्यति राजपुत्र}
{यथा तु रामस्य मतिर्निविष्टा तथा भवान् पश्यतु कार्यसिद्धिम्} %5-60-20


॥इत्यार्षे श्रीमद्रामायणे वाल्मीकीये आदिकाव्ये सुन्दरकाण्डे अङ्गदजाम्बवत्संवादः नाम षष्ठितमः सर्गः ॥५-६०॥
