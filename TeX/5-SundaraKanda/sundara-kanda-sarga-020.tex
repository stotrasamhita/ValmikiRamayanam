\sect{विंशः सर्गः — प्रणयप्रार्थना}

\twolineshloka
{स तां परिवृतां दीनां निरानन्दां तपस्विनीम्}
{साकारैर्मधुरैर्वाक्यैर्न्यदर्शयत रावणः} %5-20-1

\twolineshloka
{मां दृष्ट्वा नागनासोरु गूहमाना स्तनोदरम्}
{अदर्शनमिवात्मानं भयान्नेतुं त्वमिच्छसि} %5-20-2

\twolineshloka
{कामये त्वां विशालाक्षि बहु मन्यस्व मां प्रिये}
{सर्वाङ्गगुणसम्पन्ने सर्वलोकमनोहरे} %5-20-3

\twolineshloka
{नेह किञ्चिन्मनुष्या वा राक्षसाः कामरूपिणः}
{व्यपसर्पतु ते सीते भयं मत्तः समुत्थितम्} %5-20-4

\twolineshloka
{स्वधर्मो रक्षसां भीरु सर्वदैव न संशयः}
{गमनं वा परस्त्रीणां हरणं सम्प्रमथ्य वा} %5-20-5

\twolineshloka
{एवं चैवमकामां त्वां न च स्प्रक्ष्यामि मैथिलि}
{कामं कामः शरीरे मे यथाकामं प्रवर्तताम्} %5-20-6

\twolineshloka
{देवि नेह भयं कार्यं मयि विश्वसिहि प्रिये}
{प्रणयस्व च तत्त्वेन मैवं भूः शोकलालसा} %5-20-7

\twolineshloka
{एकवेणी अधःशय्या ध्यानं मलिनमम्बरम्}
{अस्थानेऽप्युपवासश्च नैतान्यौपयिकानि ते} %5-20-8

\twolineshloka
{विचित्राणि च माल्यानि चन्दनान्यगुरूणि च}
{विविधानि च वासांसि दिव्यान्याभरणानि च} %5-20-9

\twolineshloka
{महार्हाणि च पानानि शयनान्यासनानि च}
{गीतं नृत्यं च वाद्यं च लभ मां प्राप्य मैथिलि} %5-20-10

\twolineshloka
{स्त्रीरत्नमसि मैवं भूः कुरु गात्रेषु भूषणम्}
{मां प्राप्य हि कथं वा स्यास्त्वमनर्हा सुविग्रहे} %5-20-11

\twolineshloka
{इदं ते चारु सञ्जातं यौवनं ह्यतिवर्तते}
{यदतीतं पुनर्नैति स्रोतः स्रोतस्विनामिव} %5-20-12

\twolineshloka
{त्वां कृत्वोपरतो मन्ये रूपकर्ता स विश्वकृत्}
{नहि रूपोपमा ह्यन्या तवास्ति शुभदर्शने} %5-20-13

\twolineshloka
{त्वां समासाद्य वैदेहि रूपयौवनशालिनीम्}
{कः पुनर्नातिवर्तेत साक्षादपि पितामहः} %5-20-14

\twolineshloka
{यद् यत् पश्यामि ते गात्रं शीतांशुसदृशानने}
{तस्मिंस्तस्मिन् पृथुश्रोणि चक्षुर्मम निबध्यते} %5-20-15

\twolineshloka
{भव मैथिलि भार्या मे मोहमेतं विसर्जय}
{बह्वीनामुत्तमस्त्रीणां ममाग्रमहिषी भव} %5-20-16

\twolineshloka
{लोकेभ्यो यानि रत्नानि सम्प्रमथ्याहृतानि मे}
{तानि ते भीरु सर्वाणि राज्यं चैव ददामि ते} %5-20-17

\twolineshloka
{विजित्य पृथिवीं सर्वां नानानगरमालिनीम्}
{जनकाय प्रदास्यामि तव हेतोर्विलासिनि} %5-20-18

\twolineshloka
{नेह पश्यामि लोकेऽन्यं यो मे प्रतिबलो भवेत्}
{पश्य मे सुमहद्वीर्यमप्रतिद्वन्द्वमाहवे} %5-20-19

\twolineshloka
{असकृत् संयुगे भग्ना मया विमृदितध्वजाः}
{अशक्ताः प्रत्यनीकेषु स्थातुं मम सुरासुराः} %5-20-20

\twolineshloka
{इच्छ मां क्रियतामद्य प्रतिकर्म तवोत्तमम्}
{सुप्रभाण्यवसज्जन्तां तवाङ्गे भूषणानि हि} %5-20-21

\twolineshloka
{साधु पश्यामि ते रूपं सुयुक्तं प्रतिकर्मणा}
{प्रतिकर्माभिसंयुक्ता दाक्षिण्येन वरानने} %5-20-22

\twolineshloka
{भुङ्क्ष्व भोगान् यथाकामं पिब भीरु रमस्व च}
{यथेष्टं च प्रयच्छ त्वं पृथिवीं वा धनानि च} %5-20-23

\twolineshloka
{ललस्व मयि विस्रब्धा धृष्टमाज्ञापयस्व च}
{मत्प्रासादाल्ललन्त्याश्च ललतां बान्धवस्तव} %5-20-24

\twolineshloka
{ऋद्धिं ममानुपश्य त्वं श्रियं भद्रे यशस्विनि}
{किं करिष्यसि रामेण सुभगे चीरवासिना} %5-20-25

\twolineshloka
{निक्षिप्तविजयो रामो गतश्रीर्वनगोचरः}
{व्रती स्थण्डिलशायी च शङ्के जीवति वा न वा} %5-20-26

\twolineshloka
{नहि वैदेहि रामस्त्वां द्रष्टुं वाप्युपलभ्यते}
{पुरोबलाकैरसितैर्मेघैर्ज्योत्स्नामिवावृताम्} %5-20-27

\twolineshloka
{न चापि मम हस्तात् त्वां प्राप्तुमर्हति राघवः}
{हिरण्यकशिपुः कीर्तिमिन्द्रहस्तगतामिव} %5-20-28

\twolineshloka
{चारुस्मिते चारुदति चारुनेत्रे विलासिनि}
{मनो हरसि मे भीरु सुपर्णः पन्नगं यथा} %5-20-29

\twolineshloka
{क्लिष्टकौशेयवसनां तन्वीमप्यनलङ्कृताम्}
{त्वां दृष्ट्वा स्वेषु दारेषु रतिं नोपलभाम्यहम्} %5-20-30

\twolineshloka
{अन्तःपुरनिवासिन्यः स्त्रियः सर्वगुणान्विताः}
{यावत्यो मम सर्वासामैश्वर्यं कुरु जानकि} %5-20-31

\twolineshloka
{मम ह्यसितकेशान्ते त्रैलोक्यप्रवरस्त्रियः}
{तास्त्वां परिचरिष्यन्ति श्रियमप्सरसो यथा} %5-20-32

\twolineshloka
{यानि वैश्रवणे सुभ्रु रत्नानि च धनानि च}
{तानि लोकांश्च सुश्रोणि मया भुङ्क्ष्व यथासुखम्} %5-20-33

\twolineshloka
{न रामस्तपसा देवि न बलेन च विक्रमैः}
{न धनेन मया तुल्यस्तेजसा यशसापि वा} %5-20-34

\twolineshloka
{पिब विहर रमस्व भुङ्क्ष्व भोगान् धननिचयं प्रदिशामि मेदिनीं च}
{मयि लल ललने यथासुखं त्वं त्वयि च समेत्य ललन्तु बान्धवास्ते} %5-20-35

\twolineshloka
{कुसुमिततरुजालसन्ततानि भ्रमरयुतानि समुद्रतीरजानि}
{कनकविमलहारभूषिताङ्गी विहर मया सह भीरु काननानि} %5-20-36


॥इत्यार्षे श्रीमद्रामायणे वाल्मीकीये आदिकाव्ये सुन्दरकाण्डे प्रणयप्रार्थना नाम विंशः सर्गः ॥५-२०॥
