\sect{एकोनपञ्चाशः सर्गः — रावणप्रभावदर्शनम्}

\twolineshloka
{ततः स कर्मणा तस्य विस्मितो भीमविक्रमः}
{हनूमान् क्रोधताम्राक्षो रक्षोऽधिपमवैक्षत} %5-49-1

\twolineshloka
{भ्राजमानं महार्हेण काञ्चनेन विराजता}
{मुक्ताजालवृतेनाथ मुकुटेन महाद्युतिम्} %5-49-2

\twolineshloka
{वज्रसंयोगसंयुक्तैर्महार्हमणिविग्रहैः}
{हैमैराभरणैश्चित्रैर्मनसेव प्रकल्पितैः} %5-49-3

\twolineshloka
{महार्हक्षौमसंवीतं रक्तचन्दनरूषितम्}
{स्वनुलिप्तं विचित्राभिर्विविधाभिश्च भक्तिभिः} %5-49-4

\twolineshloka
{विचित्रं दर्शनीयैश्च रक्ताक्षैर्भीमदर्शनैः}
{दीप्ततीक्ष्णमहादंष्ट्रं प्रलम्बं दशनच्छदैः} %5-49-5

\twolineshloka
{शिरोभिर्दशभिर्वीरो भ्राजमानं महौजसम्}
{नानाव्यालसमाकीर्णैः शिखरैरिव मन्दरम्} %5-49-6

\twolineshloka
{नीलाञ्जनचयप्रख्यं हारेणोरसि राजता}
{पूर्णचन्द्राभवक्त्रेण सबालार्कमिवाम्बुदम्} %5-49-7

\twolineshloka
{बाहुभिर्बद्धकेयूरैश्चन्दनोत्तमरूषितैः}
{भ्राजमानाङ्गदैर्भीमैः पञ्चशीर्षैरिवोरगैः} %5-49-8

\twolineshloka
{महति स्फाटिके चित्रे रत्नसंयोगचित्रिते}
{उत्तमास्तरणास्तीर्णे सूपविष्टं वरासने} %5-49-9

\twolineshloka
{अलंकृताभिरत्यर्थं प्रमदाभिः समन्ततः}
{वालव्यजनहस्ताभिरारात्समुपसेवितम्} %5-49-10

\twolineshloka
{दुर्धरेण प्रहस्तेन महापार्श्वेन रक्षसा}
{मन्त्रिभिर्मन्त्रतत्त्वज्ञैर्निकुम्भेन च मन्त्रिणा} %5-49-11

\twolineshloka
{उपोपविष्टं रक्षोभिश्चतुर्भिर्बलदर्पितम्}
{कृत्स्नं परिवृतं लोकं चतुर्भिरिव सागरैः} %5-49-12

\twolineshloka
{मन्त्रिभिर्मन्त्रतत्त्वज्ञैरन्यैश्च शुभदर्शिभिः}
{आश्वास्यमानं सचिवैः सुरैरिव सुरेश्वरम्} %5-49-13

\twolineshloka
{अपश्यद् राक्षसपतिं हनूमानतितेजसम्}
{वेष्टितं मेरुशिखरे सतोयमिव तोयदम्} %5-49-14

\twolineshloka
{स तैः सम्पीड्यमानोऽपि रक्षोभिर्भीमविक्रमैः}
{विस्मयं परमं गत्वा रक्षोऽधिपमवैक्षत} %5-49-15

\twolineshloka
{भ्राजमानं ततो दृष्ट्वा हनुमान् राक्षसेश्वरम्}
{मनसा चिन्तयामास तेजसा तस्य मोहितः} %5-49-16

\twolineshloka
{अहो रूपमहो धैर्यमहो सत्त्वमहो द्युतिः}
{अहो राक्षसराजस्य सर्वलक्षणयुक्तता} %5-49-17

\twolineshloka
{यद्यधर्मो न बलवान् स्यादयं राक्षसेश्वरः}
{स्यादयं सुरलोकस्य सशक्रस्यापि रक्षिता} %5-49-18

\twolineshloka
{अस्य क्रूरैर्नृशंसैश्च कर्मभिर्लोककुत्सितैः}
{सर्वे बिभ्यति खल्वस्माल्लोकाः सामरदानवाः} %5-49-19

\threelineshloka
{अयं ह्युत्सहते क्रुद्धः कर्तुमेकार्णवं जगत्}
{इति चिन्तां बहुविधामकरोन्मतिमान् कपिः}
{दृष्ट्वा राक्षसराजस्य प्रभावममितौजसः} %5-49-20


॥इत्यार्षे श्रीमद्रामायणे वाल्मीकीये आदिकाव्ये सुन्दरकाण्डे रावणप्रभावदर्शनम् नाम एकोनपञ्चाशः सर्गः ॥५-४९॥
