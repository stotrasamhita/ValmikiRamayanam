\sect{प्रथमः सर्गः — सागरलङ्घनम्}

\twolineshloka
{ततो रावणनीतायाः सीतायाः शत्रुकर्षणः}
{इयेष पदमन्वेष्टुं चारणाचरिते पथि} %5-1-1

\twolineshloka
{दुष्करं निष्प्रतिद्वन्द्वं चिकीर्षन् कर्म वानरः}
{समुदग्रशिरोग्रीवो गवां पतिरिवाबभौ} %5-1-2

\twolineshloka
{अथ वैदूर्यवर्णेषु शाद्वलेषु महाबलः}
{धीरः सलिलकल्पेषु विचचार यथासुखम्} %5-1-3

\twolineshloka
{द्विजान् वित्रासयन् धीमानुरसा पादपान् हरन्}
{मृगांश्च सुबहून् निघ्नन् प्रवृद्ध इव केसरी} %5-1-4

\twolineshloka
{नीललोहितमाञ्जिष्ठपद्मवर्णैः सितासितैः}
{स्वभावसिद्धैर्विमलैर्धातुभिः समलंकृतम्} %5-1-5

\twolineshloka
{कामरूपिभिराविष्टमभीक्ष्णं सपरिच्छदैः}
{यक्षकिंनरगन्धर्वैर्देवकल्पैः सपन्नगैः} %5-1-6

\twolineshloka
{स तस्य गिरिवर्यस्य तले नागवरायुते}
{तिष्ठन् कपिवरस्तत्र ह्रदे नाग इवाबभौ} %5-1-7

\twolineshloka
{स सूर्याय महेन्द्राय पवनाय स्वयम्भुवे}
{भूतेभ्यश्चाञ्जलिं कृत्वा चकार गमने मतिम्} %5-1-8

\twolineshloka
{अञ्जलिं प्राङ्मुखं कुर्वन् पवनायात्मयोनये}
{ततो हि ववृधे गन्तुं दक्षिणो दक्षिणां दिशम्} %5-1-9

\twolineshloka
{प्लवगप्रवरैर्दृष्टः प्लवने कृतनिश्चयः}
{ववृधे रामवृद्ध्यर्थं समुद्र इव पर्वसु} %5-1-10

\twolineshloka
{निष्प्रमाणशरीरः सँल्लिलङ्घयिषुरर्णवम्}
{बाहुभ्यां पीडयामास चरणाभ्यां च पर्वतम्} %5-1-11

\twolineshloka
{स चचालाचलश्चाशु मुहूर्तं कपिपीडितः}
{तरूणां पुष्पिताग्राणां सर्वं पुष्पमशातयत्} %5-1-12

\twolineshloka
{तेन पादपमुक्तेन पुष्पौघेण सुगन्धिना}
{सर्वतः संवृतः शैलो बभौ पुष्पमयो यथा} %5-1-13

\twolineshloka
{तेन चोत्तमवीर्येण पीड्यमानः स पर्वतः}
{सलिलं सम्प्रसुस्राव मदमत्त इव द्विपः} %5-1-14

\twolineshloka
{पीड्यमानस्तु बलिना महेन्द्रस्तेन पर्वतः}
{रीतीर्निर्वर्तयामास काञ्चनाञ्जनराजतीः} %5-1-15

\twolineshloka
{मुमोच च शिलाः शैलो विशालाः समनःशिलाः}
{मध्यमेनार्चिषा जुष्टो धूमराजीरिवानलः} %5-1-16

\twolineshloka
{हरिणा पीड्यमानेन पीड्यमानानि सर्वतः}
{गुहाविष्टानि सत्त्वानि विनेदुर्विकृतैः स्वरैः} %5-1-17

\twolineshloka
{स महान् सत्त्वसन्नादः शैलपीडानिमित्तजः}
{पृथिवीं पूरयामास दिशश्चोपवनानि च} %5-1-18

\twolineshloka
{शिरोभिः पृथुभिर्नागा व्यक्तस्वस्तिकलक्षणैः}
{वमन्तः पावकं घोरं ददंशुर्दशनैः शिलाः} %5-1-19

\twolineshloka
{तास्तदा सविषैर्दष्टाः कुपितैस्तैर्महाशिलाः}
{जज्वलुः पावकोद्दीप्ता बिभिदुश्च सहस्रधा} %5-1-20

\twolineshloka
{यानि त्वौषधजालानि तस्मिञ्जातानि पर्वते}
{विषघ्नान्यपि नागानां न शेकुः शमितुं विषम्} %5-1-21

\twolineshloka
{भिद्यतेऽयं गिरिर्भूतैरिति मत्वा तपस्विनः}
{त्रस्ता विद्याधरास्तस्मादुत्पेतुः स्त्रीगणैः सह} %5-1-22

\twolineshloka
{पानभूमिगतं हित्वा हैममासवभाजनम्}
{पात्राणि च महार्हाणि करकांश्च हिरण्मयान्} %5-1-23

\twolineshloka
{लेह्यानुच्चावचान् भक्ष्यान् मांसानि विविधानि च}
{आर्षभाणि च चर्माणि खड्गांश्च कनकत्सरून्} %5-1-24

\twolineshloka
{कृतकण्ठगुणाः क्षीबा रक्तमाल्यानुलेपनाः}
{रक्ताक्षाः पुष्कराक्षाश्च गगनं प्रतिपेदिरे} %5-1-25

\twolineshloka
{हारनूपुरकेयूरपारिहार्यधराः स्त्रियः}
{विस्मिताः सस्मितास्तस्थुराकाशे रमणौः सह} %5-1-26

\twolineshloka
{दर्शयन्तो महाविद्यां विद्याधरमहर्षयः}
{सहितास्तस्थुराकाशे वीक्षांचक्रुश्च पर्वतम्} %5-1-27

\twolineshloka
{शुश्रुवुश्च तदा शब्दमृषीणां भावितात्मनाम्}
{चारणानां च सिद्धानां स्थितानां विमलेऽम्बरे} %5-1-28

\twolineshloka
{एष पर्वतसंकाशो हनुमान् मारुतात्मजः}
{तितीर्षति महावेगः समुद्रं वरुणालयम्} %5-1-29

\twolineshloka
{रामार्थं वानरार्थं च चिकीर्षन् कर्म दुष्करम्}
{समुद्रस्य परं पारं दुष्प्रापं प्राप्तुमिच्छति} %5-1-30

\twolineshloka
{इति विद्याधरा वाचः श्रुत्वा तेषां तपस्विनाम्}
{तमप्रमेयं ददृशुः पर्वते वानरर्षभम्} %5-1-31

\twolineshloka
{दुधुवे च स रोमाणि चकम्पे चानलोपमः}
{ननाद च महानादं सुमहानिव तोयदः} %5-1-32

\twolineshloka
{आनुपूर्व्या च वृत्तं तल्लाङ्गूलं रोमभिश्चितम्}
{उत्पतिष्यन् विचिक्षेप पक्षिराज इवोरगम्} %5-1-33

\twolineshloka
{तस्य लाङ्गूलमाविद्धमतिवेगस्य पृष्ठतः}
{ददृशे गरुडेनेव ह्रियमाणो महोरगः} %5-1-34

\twolineshloka
{बाहू संस्तम्भयामास महापरिघसंनिभौ}
{आससाद कपिः कट्यां चरणौ संचुकोच च} %5-1-35

\twolineshloka
{संहृत्य च भुजौ श्रीमांस्तथैव च शिरोधराम्}
{तेजः सत्त्वं तथा वीर्यमाविवेश स वीर्यवान्} %5-1-36

\twolineshloka
{मार्गमालोकयन् दूरादूर्ध्वप्रणिहितेक्षणः}
{रुरोध हृदये प्राणानाकाशमवलोकयन्} %5-1-37

\twolineshloka
{पद्भ्यां दृढमवस्थानं कृत्वा स कपिकुञ्जरः}
{निकुच्य कर्णौ हनुमानुत्पतिष्यन् महाबलः} %5-1-38

\twolineshloka
{वानरान् वानरश्रेष्ठ इदं वचनमब्रवीत्}
{यथा राघवनिर्मुक्तः शरः श्वसनविक्रमः} %5-1-39

\twolineshloka
{गच्छेत् तद्वद् गमिष्यामि लंकां रावणपालिताम्}
{नहि द्रक्ष्यामि यदि तां लंकायां जनकात्मजाम्} %5-1-40

\twolineshloka
{अनेनैव हि वेगेन गमिष्यामि सुरालयम्}
{यदि वा त्रिदिवे सीतां न द्रक्ष्यामि कृतश्रमः} %5-1-41

\twolineshloka
{बद्ध्वा राक्षसराजानमानयिष्यामि रावणम्}
{सर्वथा कृतकार्योऽहमेष्यामि सह सीतया} %5-1-42

\twolineshloka
{आनयिष्यामि वा लंकां समुत्पाट्य सरावणाम्}
{एवमुक्त्वा तु हनुमान् वानरो वानरोत्तमः} %5-1-43

\twolineshloka
{उत्पपाताथ वेगेन वेगवानविचारयन्}
{सुपर्णमिव चात्मानं मेने स कपिकुञ्जरः} %5-1-44

\twolineshloka
{समुत्पतति वेगात् तु वेगात् ते नगरोहिणः}
{संहृत्य विटपान् सर्वान् समुत्पेतुः समन्ततः} %5-1-45

\twolineshloka
{स मत्तकोयष्टिभकान् पादपान् पुष्पशालिनः}
{उद्वहन्नुरुवेगेन जगाम विमलेऽम्बरे} %5-1-46

\twolineshloka
{ऊरुवेगोत्थिता वृक्षा मुहूर्तं कपिमन्वयुः}
{प्रस्थितं दीर्घमध्वानं स्वबन्धुमिव बान्धवाः} %5-1-47

\twolineshloka
{तमूरुवेगोन्मथिताः सालाश्चान्ये नगोत्तमाः}
{अनुजग्मुर्हनूमन्तं सैन्या इव महीपतिम्} %5-1-48

\twolineshloka
{सुपुष्पिताग्रैर्बहुभिः पादपैरन्वितः कपिः}
{हनूमान् पर्वताकारो बभूवाद्भुतदर्शनः} %5-1-49

\twolineshloka
{सारवन्तोऽथ ये वृक्षा न्यमज्जँल्लवणाम्भसि}
{भयादिव महेन्द्रस्य पर्वता वरुणालये} %5-1-50

\twolineshloka
{स नानाकुसुमैः कीर्णः कपिः साङ्कुरकोरकैः}
{शुशुभे मेघसंकाशः खद्योतैरिव पर्वतः} %5-1-51

\twolineshloka
{विमुक्तास्तस्य वेगेन मुक्त्वा पुष्पाणि ते द्रुमाः}
{व्यवशीर्यन्त सलिले निवृत्ताः सुहृदो यथा} %5-1-52

\threelineshloka
{लघुत्वेनोपपन्नं तद् विचित्रं सागरेऽपतत्}
{द्रुमाणां विविधं पुष्पं कपिवायुसमीरितम्}
{ताराचितमिवाकाशं प्रबभौ स महार्णवः} %5-1-53

\twolineshloka
{पुष्पौघेण सुगन्धेन नानावर्णेन वानरः}
{बभौ मेघ इवोद्यन् वै विद्युद्गणविभूषितः} %5-1-54

\twolineshloka
{तस्य वेगसमुद्भूतैः पुष्पैस्तोयमदृश्यत}
{ताराभिरिव रामाभिरुदिताभिरिवाम्बरम्} %5-1-55

\twolineshloka
{तस्याम्बरगतौ बाहू ददृशाते प्रसारितौ}
{पर्वताग्राद् विनिष्क्रान्तौ पञ्चास्याविव पन्नगौ} %5-1-56

\twolineshloka
{पिबन्निव बभौ चापि सोर्मिजालं महार्णवम्}
{पिपासुरिव चाकाशं ददृशे स महाकपिः} %5-1-57

\twolineshloka
{तस्य विद्युत्प्रभाकारे वायुमार्गानुसारिणः}
{नयने विप्रकाशेते पर्वतस्थाविवानलौ} %5-1-58

\twolineshloka
{पिङ्गे पिङ्गाक्षमुख्यस्य बृहती परिमण्डले}
{चक्षुषी सम्प्रकाशेते चन्द्रसूर्याविव स्थितौ} %5-1-59

\twolineshloka
{मुखं नासिकया तस्य ताम्रया ताम्रमाबभौ}
{संध्यया समभिस्पृष्टं यथा स्यात् सूर्यमण्डलम्} %5-1-60

\twolineshloka
{लाङ्गूलं च समाविद्धं प्लवमानस्य शोभते}
{अम्बरे वायुपुत्रस्य शक्रध्वज इवोच्छ्रितम्} %5-1-61

\twolineshloka
{लाङ्गूलचक्रो हनुमान् शुक्लदंष्ट्रोऽनिलात्मजः}
{व्यरोचत महाप्राज्ञः परिवेषीव भास्करः} %5-1-62

\twolineshloka
{स्फिग्देशेनातिताम्रेण रराज स महाकपिः}
{महता दारितेनेव गिरिर्गैरिकधातुना} %5-1-63

\twolineshloka
{तस्य वानरसिंहस्य प्लवमानस्य सागरम्}
{कक्षान्तरगतो वायुर्जीमूत इव गर्जति} %5-1-64

\twolineshloka
{खे यथा निपतत्युल्का उत्तरान्ताद् विनिःसृता}
{दृश्यते सानुबन्धा च तथा स कपिकुञ्जरः} %5-1-65

\twolineshloka
{पतत्पतङ्गसंकाशो व्यायतः शुशुभे कपिः}
{प्रवृद्ध इव मातङ्गः कक्ष्यया बध्यमानया} %5-1-66

\twolineshloka
{उपरिष्टाच्छरीरेण च्छायया चावगाढया}
{सागरे मारुताविष्टा नौरिवासीत् तदा कपिः} %5-1-67

\twolineshloka
{यं यं देशं समुद्रस्य जगाम स महाकपिः}
{स तु तस्याङ्गवेगेन सोन्माद इव लक्ष्यते} %5-1-68

\twolineshloka
{सागरस्योर्मिजालानामुरसा शैलवर्ष्मणाम्}
{अभिध्नंस्तु महावेगः पुप्लुवे स महाकपिः} %5-1-69

\twolineshloka
{कपिवातश्च बलवान् मेघवातश्च निर्गतः}
{सागरं भीमनिर्ह्रादं कम्पयामासतुर्भृशम्} %5-1-70

\twolineshloka
{विकर्षन्नूर्मिजालानि बृहन्ति लवणाम्भसि}
{पुप्लुवे कपिशार्दूलो विकिरन्निव रोदसी} %5-1-71

\twolineshloka
{मेरुमन्दरसंकाशानुद्गतान् सुमहार्णवे}
{अत्यक्रामन्महावेगस्तरङ्गान् गणयन्निव} %5-1-72

\twolineshloka
{तस्य वेगसमुद्घुष्टं जलं सजलदं तदा}
{अम्बरस्थं विबभ्राजे शरदभ्रमिवाततम्} %5-1-73

\twolineshloka
{तिमिनक्रझषाः कूर्मा दृश्यन्ते विवृतास्तदा}
{वस्त्रापकर्षणेनेव शरीराणि शरीरिणाम्} %5-1-74

\twolineshloka
{क्रममाणं समीक्ष्याथ भुजगाः सागरंगमाः}
{व्योम्नि तं कपिशार्दूलं सुपर्णमिव मेनिरे} %5-1-75

\twolineshloka
{दशयोजनविस्तीर्णा त्रिंशद्योजनमायता}
{छाया वानरसिंहस्य जवे चारुतराभवत्} %5-1-76

\twolineshloka
{श्वेताभ्रघनराजीव वायुपुत्रानुगामिनी}
{तस्य सा शुशुभे छाया पतिता लवणाम्भसि} %5-1-77

\twolineshloka
{शुशुभे स महातेजा महाकायो महाकपिः}
{वायुमार्गे निरालम्बे पक्षवानिव पर्वतः} %5-1-78

\twolineshloka
{येनासौ याति बलवान् वेगेन कपिकुञ्जरः}
{तेन मार्गेण सहसा द्रोणीकृत इवार्णवः} %5-1-79

\twolineshloka
{आपाते पक्षिसङ्घानां पक्षिराज इव व्रजन्}
{हनुमान् मेघजालानि प्रकर्षन् मारुतो यथा} %5-1-80

\twolineshloka
{पाण्डुरारुणवर्णानि नीलमञ्जिष्ठकानि च}
{कपिनाऽऽकृष्यमाणानि महाभ्राणि चकाशिरे} %5-1-81

\twolineshloka
{प्रविशन्नभ्रजालानि निष्पतंश्च पुनः पुनः}
{प्रच्छन्नश्च प्रकाशश्च चन्द्रमा इव दृश्यते} %5-1-82

\twolineshloka
{प्लवमानं तु तं दृष्ट्वा प्लवगं त्वरितं तदा}
{ववृषुस्तत्र पुष्पाणि देवगन्धर्वचारणाः} %5-1-83

\twolineshloka
{तताप नहि तं सूर्यः प्लवन्तं वानरेश्वरम्}
{सिषेवे च तदा वायू रामकार्यार्थसिद्धये} %5-1-84

\twolineshloka
{ऋषयस्तुष्टुवुश्चैनं प्लवमानं विहायसा}
{जगुश्च देवगन्धर्वाः प्रशंसन्तो वनौकसम्} %5-1-85

\twolineshloka
{नागाश्च तुष्टुवुर्यक्षा रक्षांसि विविधानि च}
{प्रेक्ष्य सर्वे कपिवरं सहसा विगतक्लमम्} %5-1-86

\twolineshloka
{तस्मिन् प्लवगशार्दूले प्लवमाने हनूमति}
{इक्ष्वाकुकुलमानार्थी चिन्तयामास सागरः} %5-1-87

\twolineshloka
{साहाय्यं वानरेन्द्रस्य यदि नाहं हनूमतः}
{करिष्यामि भविष्यामि सर्ववाच्यो विवक्षताम्} %5-1-88

\twolineshloka
{अहमिक्ष्वाकुनाथेन सगरेण विवर्धितः}
{इक्ष्वाकुसचिवश्चायं तन्नार्हत्यवसादितुम्} %5-1-89

\twolineshloka
{तथा मया विधातव्यं विश्रमेत यथा कपिः}
{शेषं च मयि विश्रान्तः सुखी सोऽतितरिष्यति} %5-1-90

\twolineshloka
{इति कृत्वा मतिं साध्वीं समुद्रश्छन्नमम्भसि}
{हिरण्यनाभं मैनाकमुवाच गिरिसत्तमम्} %5-1-91

\twolineshloka
{त्वमिहासुरसङ्घानां देवराज्ञा महात्मना}
{पातालनिलयानां हि परिघः संनिवेशितः} %5-1-92

\twolineshloka
{त्वमेषां ज्ञातवीर्याणां पुनरेवोत्पतिष्यताम्}
{पातालस्याप्रमेयस्य द्वारमावृत्य तिष्ठसि} %5-1-93

\twolineshloka
{तिर्यगूर्ध्वमधश्चैव शक्तिस्ते शैल वर्धितुम्}
{तस्मात् संचोदयामि त्वामुत्तिष्ठ गिरिसत्तम} %5-1-94

\twolineshloka
{स एष कपिशार्दूलस्त्वामुपर्येति वीर्यवान्}
{हनूमान् रामकार्यार्थी भीमकर्मा खमाप्लुतः} %5-1-95

\twolineshloka
{अस्य साह्यं मया कार्यमिक्ष्वाकुकुलवर्तिनः}
{मम इक्ष्वाकवः पूज्याः परं पूज्यतमास्तव} %5-1-96

\twolineshloka
{कुरु साचिव्यमस्माकं न नः कार्यमतिक्रमेत्}
{कर्तव्यमकृतं कार्यं सतां मन्युमुदीरयेत्} %5-1-97

\twolineshloka
{सलिलादूर्ध्वमुत्तिष्ठ तिष्ठत्वेष कपिस्त्वयि}
{अस्माकमतिथिश्चैव पूज्यश्च प्लवतां वरः} %5-1-98

\twolineshloka
{चामीकरमहानाभ देवगन्धर्वसेवित}
{हनूमाँस्त्वयि विश्रान्तस्ततः शेषं गमिष्यति} %5-1-99

\twolineshloka
{काकुत्स्थस्यानृशंस्यं च मैथिल्याश्च विवासनम्}
{श्रमं च प्लवगेन्द्रस्य समीक्ष्योत्थातुमर्हसि} %5-1-100

\twolineshloka
{हिरण्यगर्भो मैनाको निशम्य लवणाम्भसः}
{उत्पपात जलात् तूर्णं महाद्रुमलतावृतः} %5-1-101

\twolineshloka
{स सागरजलं भित्त्वा बभूवात्युच्छ्रितस्तदा}
{यथा जलधरं भित्त्वा दीप्तरश्मिर्दिवाकरः} %5-1-102

\twolineshloka
{स महात्मा मुहूर्तेन पर्वतः सलिलावृतः}
{दर्शयामास शृङ्गाणि सागरेण नियोजितः} %5-1-103

\twolineshloka
{शातकुम्भमयैः शृङ्गैः सकिंनरमहोरगैः}
{आदित्योदयसंकाशैरुल्लिखद्भिरिवाम्बरम्} %5-1-104

\twolineshloka
{तस्य जाम्बूनदैः शृङ्गैः पर्वतस्य समुत्थितैः}
{आकाशं शस्त्रसंकाशमभवत् काञ्चनप्रभम्} %5-1-105

\twolineshloka
{जातरूपमयैः शृङ्गैर्भ्राजमानैर्महाप्रभैः}
{आदित्यशतसंकाशः सोऽभवद् गिरिसत्तमः} %5-1-106

\twolineshloka
{समुत्थितमसङ्गेन हनूमानग्रतः स्थितम्}
{मध्ये लवणतोयस्य विघ्नोऽयमिति निश्चितः} %5-1-107

\twolineshloka
{स तमुच्छ्रितमत्यर्थं महावेगो महाकपिः}
{उरसा पातयामास जीमूतमिव मारुतः} %5-1-108

\twolineshloka
{स तदासादितस्तेन कपिना पर्वतोत्तमः}
{बुद्ध्वा तस्य हरेर्वेगं जहर्ष च ननाद च} %5-1-109

\twolineshloka
{तमाकाशगतं वीरमाकाशे समुपस्थितः}
{प्रीतो हृष्टमना वाक्यमब्रवीत् पर्वतः कपिम्} %5-1-110

\twolineshloka
{मानुषं धारयन् रूपमात्मनः शिखरे स्थितः}
{दुष्करं कृतवान् कर्म त्वमिदं वानरोत्तम} %5-1-111

\twolineshloka
{निपत्य मम शृङ्गेषु सुखं विश्रम्य गम्यताम्}
{राघवस्य कुले जातैरुदधिः परिवर्धितः} %5-1-112

\twolineshloka
{स त्वां रामहिते युक्तं प्रत्यर्चयति सागरः}
{कृते च प्रतिकर्तव्यमेष धर्मः सनातनः} %5-1-113

\twolineshloka
{सोऽयं तत्प्रतिकारार्थी त्वत्तः सम्मानमर्हति}
{त्वन्निमित्तमनेनाहं बहुमानात् प्रचोदितः} %5-1-114

\twolineshloka
{योजनानां शतं चापि कपिरेष खमाप्लुतः}
{तव सानुषु विश्रान्तः शेषं प्रक्रमतामिति} %5-1-115

\twolineshloka
{तिष्ठ त्वं हरिशार्दूल मयि विश्रम्य गम्यताम्}
{तदिदं गन्धवत् स्वादु कन्दमूलफलं बहु} %5-1-116

\threelineshloka
{तदास्वाद्य हरिश्रेष्ठ विश्रान्तोऽथ गमिष्यसि}
{अस्माकमपि सम्बन्धः कपिमुख्य त्वयास्ति वै}
{प्रख्यातस्त्रिषु लोकेषु महागुणपरिग्रहः} %5-1-117

\twolineshloka
{वेगवन्तः प्लवन्तो ये प्लवगा मारुतात्मज}
{तेषां मुख्यतमं मन्ये त्वामहं कपिकुञ्जर} %5-1-118

\twolineshloka
{अतिथिः किल पूजार्हः प्राकृतोऽपि विजानता}
{धर्मं जिज्ञासमानेन किं पुनर्यादृशो भवान्} %5-1-119

\twolineshloka
{त्वं हि देववरिष्ठस्य मारुतस्य महात्मनः}
{पुत्रस्तस्यैव वेगेन सदृशः कपिकुञ्जर} %5-1-120

\twolineshloka
{पूजिते त्वयि धर्मज्ञे पूजां प्राप्नोति मारुतः}
{तस्मात् त्वं पूजनीयो मे शृणु चाप्यत्र कारणम्} %5-1-121

\twolineshloka
{पूर्वं कृतयुगे तात पर्वताः पक्षिणोऽभवन्}
{तेऽपि जग्मुर्दिशः सर्वा गरुडा इव वेगिनः} %5-1-122

\twolineshloka
{ततस्तेषु प्रयातेषु देवसङ्घाः सहर्षिभिः}
{भूतानि च भयं जग्मुस्तेषां पतनशङ्कया} %5-1-123

\twolineshloka
{ततः क्रुद्धः सहस्राक्षः पर्वतानां शतक्रतुः}
{पक्षांश्चिच्छेद वज्रेण ततः शतसहस्रशः} %5-1-124

\twolineshloka
{स मामुपगतः क्रुद्धो वज्रमुद्यम्य देवराट्}
{ततोऽहं सहसा क्षिप्तः श्वसनेन महात्मना} %5-1-125

\twolineshloka
{अस्मिँल्लवणतोये च प्रक्षिप्तः प्लवगोत्तम}
{गुप्तपक्षः समग्रश्च तव पित्राभिरक्षितः} %5-1-126

\twolineshloka
{ततोऽहं मानयामि त्वां मान्योऽसि मम मारुते}
{त्वया ममैष सम्बन्धः कपिमुख्य महागुणः} %5-1-127

\twolineshloka
{अस्मिन् नेवंगते कार्ये सागरस्य ममैव च}
{प्रीतिं प्रीतमनाः कर्तुं त्वमर्हसि महामते} %5-1-128

\twolineshloka
{श्रमं मोक्षय पूजां च गृहाण हरिसत्तम}
{प्रीतिं च मम मान्यस्य प्रीतोऽस्मि तव दर्शनात्} %5-1-129

\twolineshloka
{एवमुक्तः कपिश्रेष्ठस्तं नगोत्तममब्रवीत्}
{प्रीतोऽस्मि कृतमातिथ्यं मन्युरेषोऽपनीयताम्} %5-1-130

\twolineshloka
{त्वरते कार्यकालो मे अहश्चाप्यतिवर्तते}
{प्रतिज्ञा च मया दत्ता न स्थातव्यमिहान्तरा} %5-1-131

\twolineshloka
{इत्युक्त्वा पाणिना शैलमालभ्य हरिपुङ्गवः}
{जगामाकाशमाविश्य वीर्यवान् प्रहसन्निव} %5-1-132

\twolineshloka
{स पर्वतसमुद्राभ्यां बहुमानादवेक्षितः}
{पूजितश्चोपपन्नाभिराशीर्भिरभिनन्दितः} %5-1-133

\twolineshloka
{अथोर्ध्वं दूरमागत्य हित्वा शैलमहार्णवौ}
{पितुः पन्थानमासाद्य जगाम विमलेऽम्बरे} %5-1-134

\twolineshloka
{भूयश्चोर्ध्वं गतिं प्राप्य गिरिं तमवलोकयन्}
{वायुसूनुर्निरालम्बो जगाम कपिकुञ्जरः} %5-1-135

\twolineshloka
{तद् द्वितीयं हनुमतो दृष्ट्वा कर्म सुदुष्करम्}
{प्रशशंसुः सुराः सर्वे सिद्धाश्च परमर्षयः} %5-1-136

\twolineshloka
{देवताश्चाभवन् हृष्टास्तत्रस्थास्तस्य कर्मणा}
{काञ्चनस्य सुनाभस्य सहस्राक्षश्च वासवः} %5-1-137

\twolineshloka
{उवाच वचनं धीमान् परितोषात् सगद्गदम्}
{सुनाभं पर्वतश्रेष्ठं स्वयमेव शचीपतिः} %5-1-138

\twolineshloka
{हिरण्यनाभ शैलेन्द्र परितुष्टोऽस्मि ते भृशम्}
{अभयं ते प्रयच्छामि गच्छ सौम्य यथासुखम्} %5-1-139

\twolineshloka
{साह्यं कृतं ते सुमहद् विश्रान्तस्य हनूमतः}
{क्रमतो योजनशतं निर्भयस्य भये सति} %5-1-140

\twolineshloka
{रामस्यैष हितायैव याति दाशरथेः कपिः}
{सत्क्रियां कुर्वता शक्त्या तोषितोऽस्मि दृढं त्वया} %5-1-141

\twolineshloka
{स तत् प्रहर्षमलभद् विपुलं पर्वतोत्तमः}
{देवतानां पतिं दृष्ट्वा परितुष्टं शतक्रतुम्} %5-1-142

\twolineshloka
{स वै दत्तवरः शैलो बभूवावस्थितस्तदा}
{हनूमांश्च मुहूर्तेन व्यतिचक्राम सागरम्} %5-1-143

\twolineshloka
{ततो देवाः सगन्धर्वाः सिद्धाश्च परमर्षयः}
{अब्रुवन् सूर्यसंकाशां सुरसां नागमातरम्} %5-1-144

\twolineshloka
{अयं वातात्मजः श्रीमान् प्लवते सागरोपरि}
{हनूमान् नाम तस्य त्वं मुहूर्तं विघ्नमाचर} %5-1-145

\twolineshloka
{राक्षसं रूपमास्थाय सुघोरं पर्वतोपमम्}
{दंष्ट्राकरालं पिङ्गाक्षं वक्त्रं कृत्वा नभःस्पृशम्} %5-1-146

\twolineshloka
{बलमिच्छामहे ज्ञातुं भूयश्चास्य पराक्रमम्}
{त्वां विजेष्यत्युपायेन विषादं वा गमिष्यति} %5-1-147

\twolineshloka
{एवमुक्ता तु सा देवी दैवतैरभिसत्कृता}
{समुद्रमध्ये सुरसा बिभ्रती राक्षसं वपुः} %5-1-148

\twolineshloka
{विकृतं च विरूपं च सर्वस्य च भयावहम्}
{प्लवमानं हनूमन्तमावृत्येदमुवाच ह} %5-1-149

\twolineshloka
{मम भक्ष्यः प्रदिष्टस्त्वमीश्वरैर्वानरर्षभ}
{अहं त्वां भक्षयिष्यामि प्रविशेदं ममाननम्} %5-1-150

\twolineshloka
{वर एष पुरा दत्तो मम धात्रेति सत्वरा}
{व्यादाय वक्त्रं विपुलं स्थिता सा मारुतेः पुरः} %5-1-151

\threelineshloka
{एवमुक्तः सुरसया प्रहृष्टवदनोऽब्रवीत्}
{रामो दाशरथिर्नाम प्रविष्टो दण्डकावनम्}
{लक्ष्मणेन सह भ्रात्रा वैदेह्या चापि भार्यया} %5-1-152

\twolineshloka
{अन्यकार्यविषक्तस्य बद्धवैरस्य राक्षसैः}
{तस्य सीता हृता भार्या रावणेन यशस्विनी} %5-1-153

\twolineshloka
{तस्याः सकाशं दूतोऽहं गमिष्ये रामशासनात्}
{कर्तुमर्हसि रामस्य साह्यं विषयवासिनि} %5-1-154

\twolineshloka
{अथवा मैथिलीं दृष्ट्वा रामं चाक्लिष्टकारिणम्}
{आगमिष्यामि ते वक्त्रं सत्यं प्रतिशृणोमि ते} %5-1-155

\twolineshloka
{एवमुक्ता हनुमता सुरसा कामरूपिणी}
{अब्रवीन्नातिवर्तेन्मां कश्चिदेष वरो मम} %5-1-156

\twolineshloka
{तं प्रयान्तं समुद्वीक्ष्य सुरसा वाक्यमब्रवीत्}
{बलं जिज्ञासमाना सा नागमाता हनूमतः} %5-1-157

\twolineshloka
{निविश्य वदनं मेऽद्य गन्तव्यं वानरोत्तम}
{वर एष पुरा दत्तो मम धात्रेति सत्वरा} %5-1-158

\twolineshloka
{व्यादाय विपुलं वक्त्रं स्थिता सा मारुतेःपुरः}
{एवमुक्तः सुरसया क्रुद्धो वानरपुंगवः} %5-1-159

\twolineshloka
{अब्रवीत् कुरु वै वक्त्रं येन मां विषहिष्यसि}
{इत्युक्त्वा सुरसां क्रुद्धो दशयोजनमायताम्} %5-1-160

\threelineshloka
{दशयोजनविस्तारो हनूमानभवत् तदा}
{तं दृष्ट्वा मेघसंकाशं दशयोजनमायतम्}
{चकार सुरसाप्यास्यं विंशद् योजनमायतम्} %5-1-161

\twolineshloka
{हनूमांस्तु ततः क्रुद्धस्त्रिंशद् योजनमायतः}
{चकार सुरसा वक्त्रं चत्वारिंशत् तथोच्छ्रितम्} %5-1-162

\twolineshloka
{बभूव हनुमान् वीरः पञ्चाशद् योजनोच्छ्रितः}
{चकार सुरसा वक्त्रं षष्टिं योजनमुच्छ्रितम्} %5-1-163

\twolineshloka
{तदैव हनुमान् वीरः सप्ततिं योजनोच्छ्रितः}
{चकार सुरसा वक्त्रमशीतिं योजनोच्छ्रितम्} %5-1-164

\twolineshloka
{हनूमाननलप्रख्यो नवतिं योजनोच्छ्रितः}
{चकार सुरसा वक्त्रं शतयोजनमायतम्} %5-1-165

\twolineshloka
{तद् दृष्ट्वा व्यादितं त्वास्यं वायुपुत्रः स बुद्धिमान्}
{दीर्घजिह्वं सुरसया सुभीमं नरकोपमम्} %5-1-166

\twolineshloka
{स संक्षिप्यात्मनः कायं जीमूत इव मारुतिः}
{तस्मिन् मुहूर्ते हनुमान् बभूवाङ्गुष्ठमात्रकः} %5-1-167

\twolineshloka
{सोऽभिपद्याथ तद्वक्त्रं निष्पत्य च महाबलः}
{अन्तरिक्षे स्थितः श्रीमानिदं वचनमब्रवीत्} %5-1-168

\twolineshloka
{प्रविष्टोऽस्मि हि ते वक्त्रं दाक्षायणि नमोऽस्तुते}
{गमिष्ये यत्र वैदेही सत्यश्चासीद् वरस्तव} %5-1-169

\twolineshloka
{तं दृष्ट्वा वदनान्मुक्तं चन्द्रं राहुमुखादिव}
{अब्रवीत् सुरसा देवी स्वेन रूपेण वानरम्} %5-1-170

\twolineshloka
{अर्थसिद्ध्यै हरिश्रेष्ठ गच्छ सौम्य यथासुखम्}
{समानय च वैदेहीं राघवेण महात्मना} %5-1-171

\twolineshloka
{तत् तृतीयं हनुमतो दृष्ट्वा कर्म सुदुष्करम्}
{साधुसाध्विति भूतानि प्रशशंसुस्तदा हरिम्} %5-1-172

\twolineshloka
{स सागरमनाधृष्यमभ्येत्य वरुणालयम्}
{जगामाकाशमाविश्य वेगेन गरुडोपमः} %5-1-173

\twolineshloka
{सेविते वारिधाराभिः पतगैश्च निषेविते}
{चरिते कैशिकाचार्यैरैरावतनिषेविते} %5-1-174

\twolineshloka
{सिंहकुञ्जरशार्दूलपतगोरगवाहनैः}
{विमानैः सम्पतद्भिश्च विमलैः समलंकृते} %5-1-175

\twolineshloka
{वज्राशनिसमस्पर्शैः पावकैरिव शोभिते}
{कृतपुण्यैर्महाभागैः स्वर्गजिद्भिरधिष्ठिते} %5-1-176

\twolineshloka
{वहता हव्यमत्यन्तं सेविते चित्रभानुना}
{ग्रहनक्षत्रचन्द्रार्कतारागणविभूषिते} %5-1-177

\twolineshloka
{महर्षिगणगन्धर्वनागयक्षसमाकुले}
{विविक्ते विमले विश्वे विश्वावसुनिषेविते} %5-1-178

\twolineshloka
{देवराजगजाक्रान्ते चन्द्रसूर्यपथे शिवे}
{विताने जीवलोकस्य वितते ब्रह्मनिर्मिते} %5-1-179

\twolineshloka
{बहुशः सेविते वीरैर्विद्याधरगणैर्वृते}
{जगाम वायुमार्गे च गरुत्मानिव मारुतिः} %5-1-180

\twolineshloka
{हनुमान् मेघजालानि प्राकर्षन् मारुतो यथा}
{कालागुरुसवर्णानि रक्तपीतसितानि च} %5-1-181

\twolineshloka
{कपिना कृष्यमाणानि महाभ्राणि चकाशिरे}
{प्रविशन्नभ्रजालानि निष्पतंश्च पुनः पुनः} %5-1-182

\twolineshloka
{प्रावृषीन्दुरिवाभाति निष्पतन् प्रविशंस्तदा}
{प्रदृश्यमानः सर्वत्र हनूमान् मारुतात्मजः} %5-1-183

\twolineshloka
{भेजेऽम्बरं निरालम्बं पक्षयुक्त इवाद्रिराट्}
{प्लवमानं तु तं दृष्ट्वा सिंहिका नाम राक्षसी} %5-1-184

\twolineshloka
{मनसा चिन्तयामास प्रवृद्धा कामरूपिणी}
{अद्य दीर्घस्य कालस्य भविष्याम्यहमाशिता} %5-1-185

\twolineshloka
{इदं मम महासत्त्वं चिरस्य वशमागतम्}
{इति संचिन्त्य मनसा च्छायामस्य समाक्षिपत्} %5-1-186

\twolineshloka
{छायायां गृह्यमाणायां चिन्तयामास वानरः}
{समाक्षिप्तोऽस्मि सहसा पङ्गूकृतपराक्रमः} %5-1-187

\twolineshloka
{प्रतिलोमेन वातेन महानौरिव सागरे}
{तिर्यगूर्ध्वमधश्चैव वीक्षमाणस्तदा कपिः} %5-1-188

\twolineshloka
{ददर्श स महासत्त्वमुत्थितं लवणाम्भसि}
{तद् दृष्ट्वा चिन्तयामास मारुतिर्विकृताननाम्} %5-1-189

\twolineshloka
{कपिराज्ञा यथाख्यातं सत्त्वमद्भुतदर्शनम्}
{छायाग्राहि महावीर्यं तदिदं नात्र संशयः} %5-1-190

\twolineshloka
{स तां बुद्ध्वार्थतत्त्वेन सिंहिकां मतिमान् कपिः}
{व्यवर्धत महाकायः प्रावृषीव बलाहकः} %5-1-191

\twolineshloka
{तस्य सा कायमुद्वीक्ष्य वर्धमानं महाकपेः}
{वक्त्रं प्रसारयामास पातालाम्बरसंनिभम्} %5-1-192

\twolineshloka
{घनराजीव गर्जन्ती वानरं समभिद्रवत्}
{स ददर्श ततस्तस्या विकृतं सुमहन्मुखम्} %5-1-193

\twolineshloka
{कायमात्रं च मेधावी मर्माणि च महाकपिः}
{स तस्या विकृते वक्त्रे वज्रसंहननः कपिः} %5-1-194

\twolineshloka
{संक्षिप्य मुहुरात्मानं निपपात महाकपिः}
{आस्ये तस्या निमज्जन्तं ददृशुः सिद्धचारणाः} %5-1-195

\twolineshloka
{ग्रस्यमानं यथा चन्द्रं पूर्णं पर्वणि राहुणा}
{ततस्तस्या नखैस्तीक्ष्णैर्मर्माण्युत्कृत्य वानरः} %5-1-196

\twolineshloka
{उत्पपाताथ वेगेन मनःसम्पातविक्रमः}
{तां तु दिष्ट्या च धृत्या च दाक्षिण्येन निपात्य सः} %5-1-197

\threelineshloka
{कपिप्रवीरो वेगेन ववृधे पुनरात्मवान्}
{हृतहृत्सा हनुमता पपात विधुराम्भसि}
{स्वयंभुवैव हनुमान् सृष्टस्तस्या निपातने} %5-1-198

\twolineshloka
{तां हतां वानरेणाशु पतितां वीक्ष्य सिंहिकाम्}
{भूतान्याकाशचारीणि तमूचुः प्लवगोत्तमम्} %5-1-199

\twolineshloka
{भीममद्य कृतं कर्म महत्सत्त्वं त्वया हतम्}
{साधयार्थमभिप्रेतमरिष्टं प्लवतां वर} %5-1-200

\twolineshloka
{यस्य त्वेतानि चत्वारि वानरेन्द्र यथा तव}
{धृतिर्दृष्टिर्मतिर्दाक्ष्यं स कर्मसु न सीदति} %5-1-201

\twolineshloka
{स तैः सम्पूजितः पूज्यः प्रतिपन्नप्रयोजनैः}
{जगामाकाशमाविश्य पन्नगाशनवत् कपिः} %5-1-202

\twolineshloka
{प्राप्तभूयिष्ठपारस्तु सर्वतः परिलोकयन्}
{योजनानां शतस्यान्ते वनराजीं ददर्श सः} %5-1-203

\twolineshloka
{ददर्श च पतन्नेव विविधद्रुमभूषितम्}
{द्वीपं शाखामृगश्रेष्ठो मलयोपवनानि च} %5-1-204

\twolineshloka
{सागरं सागरानूपान् सागरानूपजान् द्रुमान्}
{सागरस्य च पत्नीनां मुखान्यपि विलोकयत्} %5-1-205

\twolineshloka
{स महामेघसंकाशं समीक्ष्यात्मानमात्मवान्}
{निरुन्धन्तमिवाकाशं चकार मतिमान् मतिम्} %5-1-206

\twolineshloka
{कायवृद्धिं प्रवेगं च मम दृष्ट्वैव राक्षसाः}
{मयि कौतूहलं कुर्युरिति मेने महामतिः} %5-1-207

\twolineshloka
{ततः शरीरं संक्षिप्य तन्महीधरसंनिभम्}
{पुनः प्रकृतिमापेदे वीतमोह इवात्मवान्} %5-1-208

\twolineshloka
{तद्रूपमतिसंक्षिप्य हनूमान् प्रकृतौ स्थितः}
{त्रीन् क्रमानिव विक्रम्य बलिवीर्यहरो हरिः} %5-1-209

\twolineshloka
{स चारुनानाविधरूपधारी परं समासाद्य समुद्रतीरम्}
{परैरशक्यं प्रतिपन्नरूपः समीक्षितात्मा समवेक्षितार्थः} %5-1-210

\twolineshloka
{ततः स लम्बस्य गिरेः समृद्धे विचित्रकूटे निपपात कूटे}
{सकेतकोद्दालकनारिकेले महाभ्रकूटप्रतिमो महात्मा} %5-1-211

\twolineshloka
{ततस्तु सम्प्राप्य समुद्रतीरं समीक्ष्य लंकां गिरिवर्यमूर्ध्नि}
{कपिस्तु तस्मिन् निपपात पर्वते विधूय रूपं व्यथयन्मृगद्विजान्} %5-1-212

\twolineshloka
{स सागरं दानवपन्नगायुतं बलेन विक्रम्य महोर्मिमालिनम्}
{निपत्य तीरे च महोदधेस्तदा ददर्श लंकाममरावतीमिव} %5-1-213


॥इत्यार्षे श्रीमद्रामायणे वाल्मीकीये आदिकाव्ये सुन्दरकाण्डे सागरलङ्घनम् नाम प्रथमः सर्गः ॥५-१॥
