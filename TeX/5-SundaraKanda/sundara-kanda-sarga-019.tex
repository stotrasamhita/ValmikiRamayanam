\sect{एकोनविंशः सर्गः — कृच्छ्रगतसीतोपमाः}

\twolineshloka
{तस्मिन्नेव ततः काले राजपुत्री त्वनिन्दिता}
{रूपयौवनसम्पन्नं भूषणोत्तमभूषितम्} %5-19-1

\twolineshloka
{ततो दृष्ट्वैव वैदेही रावणं राक्षसाधिपम्}
{प्रावेपत वरारोहा प्रवाते कदली यथा} %5-19-2

\twolineshloka
{ऊरुभ्यामुदरं छाद्य बाहुभ्यां च पयोधरौ}
{उपविष्टा विशालाक्षी रुदती वरवर्णिनी} %5-19-3

\twolineshloka
{दशग्रीवस्तु वैदेहीं रक्षितां राक्षसीगणैः}
{ददर्श दीनां दुःखार्तां नावं सन्नामिवार्णवे} %5-19-4

\twolineshloka
{असंवृतायामासीनां धरण्यां संशितव्रताम्}
{छिन्नां प्रपतितां भूमौ शाखामिव वनस्पतेः} %5-19-5

\twolineshloka
{मलमण्डनदिग्धांगीं मण्डनार्हाममण्डनाम्}
{मृणाली पङ्कदिग्धेव विभाति न विभाति च} %5-19-6

\twolineshloka
{समीपं राजसिंहस्य रामस्य विदितात्मनः}
{संकल्पहयसंयुक्तैर्यान्तीमिव मनोरथैः} %5-19-7

\twolineshloka
{शुष्यन्तीं रुदतीमेकां ध्यानशोकपरायणाम्}
{दुःखस्यान्तमपश्यन्तीं रामां राममनुव्रताम्} %5-19-8

\twolineshloka
{चेष्टमानामथाविष्टां पन्नगेन्द्रवधूमिव}
{धूप्यमानां ग्रहेणेव रोहिणीं धूमकेतुना} %5-19-9

\twolineshloka
{वृत्तशीले कुले जातामाचारवति धार्मिके}
{पुनः संस्कारमापन्नां जातामिव च दुष्कुले} %5-19-10

\twolineshloka
{सन्नामिव महाकीर्तिं श्रद्धामिव विमानिताम्}
{प्रज्ञामिव परिक्षीणामाशां प्रतिहतामिव} %5-19-11

\twolineshloka
{आयतीमिव विध्वस्तामाज्ञां प्रतिहतामिव}
{दीप्तामिव दिशं काले पूजामपहतामिव} %5-19-12

\twolineshloka
{पौर्णमासीमिव निशां तमोग्रस्तेन्दुमण्डलाम्}
{पद्मिनीमिव विध्वस्तां हतशूरां चमूमिव} %5-19-13

\twolineshloka
{प्रभामिव तमोध्वस्तामुपक्षीणामिवापगाम्}
{वेदीमिव परामृष्टां शान्तामग्निशिखामिव} %5-19-14

\twolineshloka
{उत्कृष्टपर्णकमलां वित्रासितविहंगमाम्}
{हस्तिहस्तपरामृष्टामाकुलामिव पद्मिनीम्} %5-19-15

\twolineshloka
{पतिशोकातुरां शुष्कां नदीं विस्रावितामिव}
{परया मृजया हीनां कृष्णपक्षे निशामिव} %5-19-16

\twolineshloka
{सुकुमारीं सुजातांगीं रत्नगर्भगृहोचिताम्}
{तप्यमानामिवोष्णेन मृणालीमचिरोद्धृताम्} %5-19-17

\twolineshloka
{गृहीतामालितां स्तम्भे यूथपेन विनाकृताम्}
{निःश्वसन्तीं सुदुःखार्तां गजराजवधूमिव} %5-19-18

\twolineshloka
{एकया दीर्घया वेण्या शोभमानामयत्नतः}
{नीलया नीरदापाये वनराज्या महीमिव} %5-19-19

\twolineshloka
{उपवासेन शोकेन ध्यानेन च भयेन च}
{परिक्षीणां कृशां दीनामल्पाहारां तपोधनाम्} %5-19-20

\twolineshloka
{आयाचमानां दुःखार्तां प्राञ्जलिं देवतामिव}
{भावेन रघुमुख्यस्य दशग्रीवपराभवम्} %5-19-21

\twolineshloka
{समीक्षमाणां रुदतीमनिन्दितां सुपक्ष्मताम्रायतशुक्ललोचनाम्}
{अनुव्रतां राममतीव मैथिलीं प्रलोभयामास वधाय रावणः} %5-19-22


॥इत्यार्षे श्रीमद्रामायणे वाल्मीकीये आदिकाव्ये सुन्दरकाण्डे कृच्छ्रगतसीतोपमाः नाम एकोनविंशः सर्गः ॥५-१९॥
