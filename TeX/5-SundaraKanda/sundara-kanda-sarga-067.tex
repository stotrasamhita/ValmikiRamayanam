\sect{सप्तषष्ठितमः सर्गः — सीताभाषितानुवचनम्}

\twolineshloka
{एवमुक्तस्तु हनुमान् राघवेण महात्मना}
{सीताया भाषितं सर्वं न्यवेदयत राघवे} %5-67-1

\twolineshloka
{इदमुक्तवती देवी जानकी पुरुषर्षभ}
{पूर्ववृत्तमभिज्ञानं चित्रकूटे यथातथम्} %5-67-2

\twolineshloka
{सुखसुप्ता त्वया सार्धं जानकी पूर्वमुत्थिता}
{वायसः सहसोत्पत्य विददार स्तनान्तरम्} %5-67-3

\twolineshloka
{पर्यायेण च सुप्तस्त्वं देव्यङ्के भरताग्रज}
{पुनश्च किल पक्षी स देव्या जनयति व्यथा} %5-67-4

\twolineshloka
{ततः पुनरुपागम्य विददार भृशं किल}
{ततस्त्वं बोधितस्तस्याः शोणितेन समुक्षितः} %5-67-5

\twolineshloka
{वायसेन च तेनैवं सततं बाध्यमानया}
{बोधितः किल देव्या त्वं सुखसुप्तः परन्तप} %5-67-6

\twolineshloka
{तां च दृष्ट्वा महाबाहो दारितां च स्तनान्तरे}
{आशीविष इव क्रुद्धस्ततो वाक्यं त्वमूचिवान्} %5-67-7

\twolineshloka
{नखाग्रैः केन ते भीरु दारितं वै स्तनान्तरम्}
{कः क्रीडति सरोषेण पञ्चवक्त्रेण भोगिना} %5-67-8

\twolineshloka
{निरीक्षमाणः सहसा वायसं समुदैक्षथाः}
{नखैः सरुधिरैस्तीक्ष्णैस्तामेवाभिमुखं स्थितम्} %5-67-9

\twolineshloka
{सुतः किल स शक्रस्य वायसः पततां वरः}
{धरान्तरगतः शीघ्रं पवनस्य गतौ समः} %5-67-10

\twolineshloka
{ततस्तस्मिन् महाबाहो कोपसंवर्तितेक्षणः}
{वायसे त्वं व्यधाः क्रूरां मतिं मतिमतां वर} %5-67-11

\twolineshloka
{स दर्भसंस्तराद् गृह्य ब्रह्मास्त्रेण न्ययोजयः}
{स दीप्त इव कालाग्निर्जज्वालाभिमुखं खगम्} %5-67-12

\twolineshloka
{स त्वं प्रदीप्तं चिक्षेप दर्भं तं वायसं प्रति}
{ततस्तु वायसं दीप्तः स दर्भोऽनुजगाम ह} %5-67-13

\twolineshloka
{भीतैश्च सम्परित्यक्तः सुरैः सर्वैश्च वायसः}
{त्रीँल्लोकान् सम्परिक्रम्य त्रातारं नाधिगच्छति} %5-67-14

\twolineshloka
{पुनरप्यागतस्तत्र त्वत्सकाशमरिन्दम}
{त्वं तं निपतितं भूमौ शरण्यः शरणागतम्} %5-67-15

\twolineshloka
{वधार्हमपि काकुत्स्थ कृपया परिपालयः}
{मोघमस्त्रं न शक्यं तु कर्तुमित्येव राघव} %5-67-16

\twolineshloka
{भवांस्तस्याक्षि काकस्य हिनस्ति स्म स दक्षिणम्}
{राम त्वां स नमस्कृत्य राज्ञो दशरथस्य च} %5-67-17

\twolineshloka
{विसृष्टस्तु तदा काकः प्रतिपेदे स्वमालयम्}
{एवमस्त्रविदां श्रेष्ठः सत्त्ववाञ्छीलवानपि} %5-67-18

\twolineshloka
{किमर्थमस्त्रं रक्षःसु न योजयसि राघव}
{न दानवा न गन्धर्वा नासुरा न मरुद्गणाः} %5-67-19

\twolineshloka
{तव राम रणे शक्तास्तथा प्रतिसमासितुम्}
{तव वीर्यवतः कश्चिन्मयि यद्यस्ति सम्भ्रमः} %5-67-20

\twolineshloka
{क्षिप्रं सुनिशितैर्बाणैर्हन्यतां युधि रावणः}
{भ्रातुरादेशमाज्ञाय लक्ष्मणो वा परन्तपः} %5-67-21

\twolineshloka
{स किमर्थं नरवरो न मां रक्षति राघवः}
{शक्तौ तौ पुरुषव्याघ्रौ वाय्वग्निसमतेजसौ} %5-67-22

\twolineshloka
{सुराणामपि दुर्धर्षौ किमर्थं मामुपेक्षतः}
{ममैव दुष्कृतं किञ्चिन्महदस्ति न संशयः} %5-67-23

\twolineshloka
{समर्थौ सहितौ यन्मां न रक्षेते परन्तपौ}
{वैदेह्या वचनं श्रुत्वा करुणं साधुभाषितम्} %5-67-24

\twolineshloka
{पुनरप्यहमार्यां तामिदं वचनमब्रुवम्}
{त्वच्छोकविमुखो रामो देवि सत्येन ते शपे} %5-67-25

\twolineshloka
{रामे दुःखाभिभूते च लक्ष्मणः परितप्यते}
{कथञ्चिद् भवती दृष्टा न कालः परिशोचितुम्} %5-67-26

\twolineshloka
{अस्मिन् मुहूर्ते दुःखानामन्तं द्रक्ष्यसि भामिनि}
{तावुभौ नरशार्दूलौ राजपुत्रौ परन्तपौ} %5-67-27

\twolineshloka
{त्वद्दर्शनकृतोत्साहौ लङ्कां भस्मीकरिष्यतः}
{हत्वा च समरे रौद्रं रावणं सहबान्धवम्} %5-67-28

\twolineshloka
{राघवस्त्वां वरारोहे स्वपुरीं नयिता ध्रुवम्}
{यत् तु रामो विजानीयादभिज्ञानमनिन्दिते} %5-67-29

\twolineshloka
{प्रीतिसञ्जननं तस्य प्रदातुं तत् त्वमर्हसि}
{साभिवीक्ष्य दिशः सर्वा वेण्युद्ग्रथनमुत्तमम्} %5-67-30

\twolineshloka
{मुक्त्वा वस्त्राद् ददौ मह्यं मणिमेतं महाबल}
{प्रतिगृह्य मणिं दोर्भ्यां तव हेतो रघुप्रिय} %5-67-31

\twolineshloka
{शिरसा सम्प्रणम्यैनामहमागमने त्वरे}
{गमने च कृतोत्साहमवेक्ष्य वरवर्णिनी} %5-67-32

\twolineshloka
{विवर्धमानं च हि मामुवाच जनकात्मजा}
{अश्रुपूर्णमुखी दीना बाष्पगद्गदभाषिणी} %5-67-33

\twolineshloka
{ममोत्पतनसम्भ्रान्ता शोकवेगसमाहता}
{मामुवाच ततः सीता सभाग्योऽसि महाकपे} %5-67-34

\twolineshloka
{यद् द्रक्ष्यसि महाबाहुं रामं कमललोचनम्}
{लक्ष्मणं च महाबाहुं देवरं मे यशस्विनम्} %5-67-35

\twolineshloka
{सीतयाप्येवमुक्तोऽहमब्रुवं मैथिलीं तथा}
{पृष्ठमारोह मे देवि क्षिप्रं जनकनन्दनि} %5-67-36

\twolineshloka
{यावत्ते दर्शयाम्यद्य ससुग्रीवं सलक्ष्मणम्}
{राघवं च महाभागे भर्तारमसितेक्षणे} %5-67-37

\twolineshloka
{साब्रवीन्मां ततो देवी नैष धर्मो महाकपे}
{यत्ते पृष्ठं सिषेवेऽहं स्ववशा हरिपुङ्गव} %5-67-38

\twolineshloka
{पुरा च यदहं वीर स्पृष्टा गात्रेषु रक्षसा}
{तत्राहं किं करिष्यामि कालेनोपनिपीडिता} %5-67-39

\twolineshloka
{गच्छ त्वं कपिशार्दूल यत्र तौ नृपतेः सुतौ}
{इत्येवं सा समाभाष्य भूयः सन्देष्टुमास्थिता} %5-67-40

\twolineshloka
{हनूमन् सिंहसङ्काशौ तावुभौ रामलक्ष्मणौ}
{सुग्रीवं च सहामात्यं सर्वान् ब्रूया अनामयम्} %5-67-41

\twolineshloka
{यथा च स महाबाहुर्मां तारयति राघवः}
{अस्माद्दुःखाम्बुसंरोधात् तत् त्वमाख्यातुमर्हसि} %5-67-42

\twolineshloka
{इदं च तीव्रं मम शोकवेगं रक्षोभिरेभिः परिभर्त्सनं च}
{ब्रूयास्तु रामस्य गतः समीपं शिवश्च तेऽध्वास्तु हरिप्रवीर} %5-67-43

\twolineshloka
{एतत् तवार्या नृप संयता सा सीता वचः प्राह विषादपूर्वम्}
{एतच्च बुद्ध्वा गदितं यथा त्वं श्रद्धत्स्व सीतां कुशलां समग्राम्} %5-67-44


॥इत्यार्षे श्रीमद्रामायणे वाल्मीकीये आदिकाव्ये सुन्दरकाण्डे सीताभाषितानुवचनम् नाम सप्तषष्ठितमः सर्गः ॥५-६७॥
