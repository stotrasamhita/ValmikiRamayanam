\sect{अष्टमः सर्गः — सुमन्त्रवाक्यम्}

\twolineshloka
{तस्य चैवं प्रभावस्य धर्मज्ञस्य महात्मनः}
{सुतार्थं तप्यमानस्य नासीद् वंशकरः सुतः} %1-8-1

\twolineshloka
{चिन्तयानस्य तस्यैवं बुद्धिरासीन्महात्मनः}
{सुतार्थं वाजिमेधेन किमर्थं न यजाम्यहम्} %1-8-2

\twolineshloka
{स निश्चितां मतिं कृत्वा यष्टव्यमिति बुद्धिमान्}
{मन्त्रिभिः सह धर्मात्मा सर्वैरेव कृतात्मभिः} %1-8-3

\twolineshloka
{ततोऽब्रवीन्महातेजाः सुमन्त्रं मन्त्रिसत्तम}
{शीघ्रमानय मे सर्वान् गुरूंस्तान् सपुरोहितान्} %1-8-4

\twolineshloka
{ततः सुमन्त्रस्त्वरितं गत्वा त्वरितविक्रमः}
{समानयत् स तान् सर्वान् समस्तान् वेदपारगान्} %1-8-5

\twolineshloka
{सुयज्ञं वामदेवं च जाबालिमथ काश्यपम्}
{पुरोहितं वशिष्ठं च ये चाप्यन्ये द्विजोत्तमाः} %1-8-6

\twolineshloka
{तान् पूजयित्वा धर्मात्मा राजा दशरथस्तदा}
{इदं धर्मार्थसहितं श्लक्ष्णं वचनमब्रवीत्} %1-8-7

\twolineshloka
{मम लालप्यमानस्य सुतार्थं नास्ति वै सुखम्}
{तदर्थं हयमेधेन यक्ष्यामीति मतिर्मम} %1-8-8

\twolineshloka
{तदहं यष्टुमिच्छामि शास्त्रदृष्टेन कर्मणा}
{कथं प्राप्स्याम्यहं कामं बुद्धिरत्रविचिन्त्यताम्} %1-8-9

\twolineshloka
{ततः साध्विति तद्वाक्यं ब्राह्मणाः प्रत्यपूजयन्}
{वसिष्ठप्रमुखाः सर्वे पार्थिवस्य मुखेरितम्} %1-8-10

\twolineshloka
{ऊचुश्च परमप्रीताः सर्वे दशरथं वचः}
{सम्भाराः सम्भ्रियन्तां ते तुरगश्च विमुच्यताम्} %1-8-11

\twolineshloka
{सरय्वाश्चोत्तरे तीरे यज्ञभूमिर्विधीयताम्}
{सर्वथा प्राप्स्यसे पुत्रानभिप्रेतांश्च पार्थिव} %1-8-12

\twolineshloka
{यस्य ते धार्मिकी बुद्धिरियं पुत्रार्थमागता}
{ततस्तुष्टोऽभवद् राजा श्रुत्वैतद् द्विजभाषितम्} %1-8-13

\twolineshloka
{अमात्यानब्रवीद् राजा हर्षव्याकुललोचनः}
{सम्भाराः सम्भ्रियन्तां मे गुरूणां वचनादिह} %1-8-14

\twolineshloka
{समर्थाधिष्ठितश्चाश्वः सोपाध्यायो विमुच्यताम्}
{सरय्वाश्चोत्तरे तीरे यज्ञभूमिर्विधीयताम्} %1-8-15

\twolineshloka
{शान्तयश्चापि वर्धन्तां यथाकल्पं यथाविधि}
{शक्यः प्राप्तुमयं यज्ञः सर्वेणापि महीक्षिता} %1-8-16

\twolineshloka
{नापराधो भवेत् कष्टो यद्यस्मिन् क्रतुसत्तमे}
{च्छिद्रं हि मृगयन्ते स्म विद्वांसो ब्रह्मराक्षसाः} %1-8-17

\twolineshloka
{विधिहीनस्य यज्ञस्य सद्यः कर्ता विनश्यति}
{तद्यथा विधिपूर्वं मे क्रतुरेष समाप्यते} %1-8-18

\twolineshloka
{तथा विधानं क्रियतां समर्थाः साधनेष्विति}
{तथेति चाब्रुवन् सर्वे मन्त्रिणः प्रतिपूजिताः} %1-8-19

\twolineshloka
{पार्थिवेन्द्रस्य तद् वाक्यं यथापूर्वं निशम्य ते}
{तथा द्विजास्ते धर्मज्ञा वर्धयन्तो नृपोत्तमम्} %1-8-20

\twolineshloka
{अनुज्ञातास्ततः सर्वे पुनर्जग्मुर्यथागतम्}
{विसर्जयित्वा तान् विप्रान् सचिवानिदमब्रवीत्} %1-8-21

\twolineshloka
{ऋत्विग्भिरुपसन्दिष्टो यथावत् क्रतुराप्यताम्}
{इत्युक्त्वा नृपशार्दूलः सचिवान् समुपस्थितान्} %1-8-22

\twolineshloka
{विसर्जयित्वा स्वं वेश्म प्रविवेश महामतिः}
{ततः स गत्वा ताः पत्नीर्नरेन्द्रो हृदयङ्गमाः} %1-8-23

\threelineshloka
{उवाच दीक्षां विशत यक्ष्येऽहं सुतकारणात्}
{तासां तेनातिकान्तेन वचनेन सुवर्चसाम्}
{मुखपद्मान्यशोभन्त पद्मानीव हिमात्यये} %1-8-24


॥इत्यार्षे श्रीमद्रामायणे वाल्मीकीये आदिकाव्ये बालकाण्डे सुमन्त्रवाक्यम् नाम अष्टमः सर्गः ॥१-८॥
