\sect{तृतीयः सर्गः — काव्यसङ्क्षेपः}

\twolineshloka
{श्रुत्वा वस्तु समग्रं तद्धर्मार्थसहितं हितम्}
{व्यक्तमन्वेषते भूयो यद् वृत्तं तस्य धीमतः} %1-3-1

\twolineshloka
{उपस्पृश्योदकं सम्यङ्मुनिः स्थित्वा कृताञ्जलिः}
{प्राचीनाग्रेषु दर्भेषु धर्मेणान्वेषते गतिम्} %1-3-2

\twolineshloka
{रामलक्ष्मणसीताभी राज्ञा दशरथेन च}
{सभार्येण सराष्ट्रेण यत् प्राप्तं तत्र तत्त्वतः} %1-3-3

\twolineshloka
{हसितं भाषितं चैव गतिर्यावच्च चेष्टितम्}
{तत् सर्वं धर्मवीर्येण यथावत् सम्प्रपश्यति} %1-3-4

\twolineshloka
{स्त्रीतृतीयेन च तथा यत् प्राप्तं चरता वने}
{सत्यसन्धेन रामेण तत् सर्वं चान्ववैक्षत} %1-3-5

\twolineshloka
{ततः पश्यति धर्मात्मा तत् सर्वं योगमास्थितः}
{पुरा यत् तत्र निर्वृत्तं पाणावामलकं यथा} %1-3-6

\twolineshloka
{तत् सर्वं तत्त्वतो दृष्ट्वा धर्मेण स महामतिः}
{अभिरामस्य रामस्य तत् सर्वं कर्तुमुद्यतः} %1-3-7

\twolineshloka
{कामार्थगुणसंयुक्तं धर्मार्थगुणविस्तरम्}
{समुद्रमिव रत्नाढ्यं सर्वश्रुतिमनोहरम्} %1-3-8

\twolineshloka
{स यथा कथितं पूर्वं नारदेन महात्मना}
{रघुवंशस्य चरितं चकार भगवान् मुनिः} %1-3-9

\twolineshloka
{जन्म रामस्य सुमहद्वीर्यं सर्वानुकूलताम्}
{लोकस्य प्रियतां क्षान्तिं सौम्यतां सत्यशीलताम्} %1-3-10

\twolineshloka
{नाना चित्राः कथाश्चान्या विश्वामित्रसहायने}
{जानक्याश्च विवाहं च धनुषश्च विभेदनम्} %1-3-11

\twolineshloka
{रामरामविवादं च गुणान् दाशरथेस्तथा}
{तथाभिषेकं रामस्य कैकेय्या दुष्टभावताम्} %1-3-12

\twolineshloka
{विघातं चाभिषेकस्य रामस्य च विवासनम्}
{राज्ञः शोकं विलापं च परलोकस्य चाश्रयम्} %1-3-13

\twolineshloka
{प्रकृतीनां विषादं च प्रकृतीनां विसर्जनम्}
{निषादाधिपसंवादं सूतोपावर्तनं तथा} %1-3-14

\twolineshloka
{गङ्गायाश्चापि सन्तारं भरद्वाजस्य दर्शनम्}
{भरद्वाजाभ्यनुज्ञानाच्चित्रकूटस्य दर्शनम्} %1-3-15

\twolineshloka
{वास्तुकर्म निवेशं च भरतागमनं तथा}
{प्रसादनं च रामस्य पितुश्च सलिलक्रियाम्} %1-3-16

\twolineshloka
{पादुकाग्र्याभिषेकं च नन्दिग्रामनिवासनम्}
{दण्डकारण्यगमनं विराधस्य वधं तथा} %1-3-17

\twolineshloka
{दर्शनं शरभङ्गस्य सुतीक्ष्णेन समागमम्}
{अनसूयासमाख्यां च अङ्गरागस्य चार्पणम्} %1-3-18

\twolineshloka
{दर्शनं चाप्यगस्त्यस्य धनुषो ग्रहणं तथा}
{शूर्पणख्याश्च संवादं विरूपकरणं तथा} %1-3-19

\twolineshloka
{वधं खरत्रिशिरसोरुत्थानं रावणस्य च}
{मारीचस्य वधं चैव वैदेह्या हरणं तथा} %1-3-20

\twolineshloka
{राघवस्य विलापं च गृध्रराजनिबर्हणम्}
{कबन्धदर्शनं चैव पम्पायाश्चापि दर्शनम्} %1-3-21

\twolineshloka
{शबरीदर्शनं चैव फलमूलाशनं तथा}
{प्रलापं चैव पम्पायां हनूमद्दर्शनं तथा} %1-3-22

\twolineshloka
{ऋष्यमूकस्य गमनं सुग्रीवेण समागमम्}
{प्रत्ययोत्पादनं सख्यं वालिसुग्रीवविग्रहम्} %1-3-23

\twolineshloka
{वालिप्रमथनं चैव सुग्रीवप्रतिपादनम्}
{ताराविलापं समयं वर्षरात्रनिवासनम्} %1-3-24

\twolineshloka
{कोपं राघवसिंहस्य बलानामुपसङ्ग्रहम्}
{दिशः प्रस्थापनं चैव पृथिव्याश्च निवेदनम्} %1-3-25

\twolineshloka
{अङ्गुलीयकदानं च ऋक्षस्य बिलदर्शनम्}
{प्रायोपवेशनं चैव सम्पातेश्चापि दर्शनम्} %1-3-26

\twolineshloka
{पर्वतारोहणं चैव सागरस्यापि लङ्घनम्}
{समुद्रवचनाच्चैव मैनाकस्य च दर्शनम्} %1-3-27

\twolineshloka
{राक्षसीतर्जनं चैव छायाग्राहस्य दर्शनम्}
{सिंहिकायाश्च निधनं लङ्कामलयदर्शनम्} %1-3-28

\twolineshloka
{रात्रौ लङ्काप्रवेशं च एकस्यापि विचिन्तनम्}
{आपानभूमिगमनमवरोधस्य दर्शनम्} %1-3-29

\twolineshloka
{दर्शनं रावणस्यापि पुष्पकस्य च दर्शनम्}
{अशोकवनिकायानं सीतायाश्चापि दर्शनम्} %1-3-30

\twolineshloka
{अभिज्ञानप्रदानं च सीतायाश्चापि भाषणम्}
{राक्षसीतर्जनं चैव त्रिजटास्वप्नदर्शनम्} %1-3-31

\twolineshloka
{मणिप्रदानं सीताया वृक्षभङ्गं तथैव च}
{राक्षसीविद्रवं चैव किङ्कराणां निबर्हणम्} %1-3-32

\twolineshloka
{ग्रहणं वायुसूनोश्च लङ्कादाहाभिगर्जनम्}
{प्रतिप्लवनमेवाथ मधूनां हरणं तथा} %1-3-33

\twolineshloka
{राघवाश्वासनं चैव मणिनिर्यातनं तथा}
{सङ्गमं च समुद्रेण नलसेतोश्च बन्धनम्} %1-3-34

\twolineshloka
{प्रतारं च समुद्रस्य रात्रौ लङ्कावरोधनम्}
{विभीषणेन संसर्गं वधोपायनिवेदनम्} %1-3-35

\twolineshloka
{कुम्भकर्णस्य निधनं मेघनादनिबर्हणम्}
{रावणस्य विनाशं च सीतावाप्तिमरेः पुरे} %1-3-36

\twolineshloka
{बिभीषणाभिषेकं च पुष्पकस्य च दर्शनम्}
{अयोध्यायाश्च गमनं भरद्वाजसमागमम्} %1-3-37

\threelineshloka
{प्रेषणं वायुपुत्रस्य भरतेन समागमम्}
{रामाभिषेकाभ्युदयं सर्वसैन्यविसर्जनम्}
{स्वराष्ट्ररञ्जनं चैव वैदेह्याश्च विसर्जनम्} %1-3-38

\twolineshloka
{अनागतं च यत् किञ्चिद् रामस्य वसुधातले}
{तच्चकारोत्तरे काव्ये वाल्मीकिर्भगवानृषिः} %1-3-39


॥इत्यार्षे श्रीमद्रामायणे वाल्मीकीये आदिकाव्ये बालकाण्डे काव्यसङ्क्षेपः नाम तृतीयः सर्गः ॥१-३॥
