\sect{एकोनपञ्चाशः सर्गः — अहल्याशापमोक्षः}

\twolineshloka
{अफलस्तु ततः शक्रो देवानग्निपुरोगमान्}
{अब्रवीत् त्रस्तनयनः सिद्धगन्धर्वचारणान्} %1-49-1

\twolineshloka
{कुर्वता तपसो विघ्नं गौतमस्य महात्मनः}
{क्रोधमुत्पाद्य हि मया सुरकार्यमिदं कृतम्} %1-49-2

\twolineshloka
{अफलोऽस्मि कृतस्तेन क्रोधात् सा च निराकृता}
{शापमोक्षेण महता तपोऽस्यापहृतं मया} %1-49-3

\twolineshloka
{तन्मां सुरवराः सर्वे सर्षिसङ्घाः सचारणाः}
{सुरकार्यकरं यूयं सफलं कर्तुमर्हथ} %1-49-4

\twolineshloka
{शतक्रतोर्वचः श्रुत्वा देवाः साग्निपुरोगमाः}
{पितृदेवानुपेत्याहुः सर्वे सह मरुद्गणैः} %1-49-5

\twolineshloka
{अयं मेषः सवृषणः शक्रो ह्यवृषणः कृतः}
{मेषस्य वृषणौ गृह्य शक्रायाशु प्रयच्छत} %1-49-6

\threelineshloka
{अफलस्तु कृतो मेषः परां तुष्टिं प्रदास्यति}
{भवतां हर्षणार्थं च ये च दास्यन्ति मानवाः}
{अक्षयं हि फलं तेषां यूयं दास्यथ पुष्कलम्} %1-49-7

\twolineshloka
{अग्नेस्तु वचनं श्रुत्वा पितृदेवाः समागताः}
{उत्पाट्य मेषवृषणौ सहस्राक्षे न्यवेशयन्} %1-49-8

\twolineshloka
{तदाप्रभृति काकुत्स्थ पितृदेवाः समागताः}
{अफलान् भुञ्जते मेषान् फलैस्तेषामयोजयन्} %1-49-9

\twolineshloka
{इन्द्रस्तु मेषवृषणस्तदाप्रभृति राघव}
{गौतमस्य प्रभावेण तपसा च महात्मनः} %1-49-10

\twolineshloka
{तदागच्छ महातेज आश्रमं पुण्यकर्मणः}
{तारयैनां महाभागामहल्यां देवरूपिणीम्} %1-49-11

\twolineshloka
{विश्वामित्रवचः श्रुत्वा राघवः सहलक्ष्मणः}
{विश्वामित्रं पुरस्कृत्य आश्रमं प्रविवेश ह} %1-49-12

\twolineshloka
{ददर्श च महाभागां तपसा द्योतितप्रभाम्}
{लोकैरपि समागम्य दुर्निरीक्ष्यां सुरासुरैः} %1-49-13

\twolineshloka
{प्रयत्नान्निर्मितां धात्रा दिव्यां मायामयीमिव}
{धूमेनाभिपरीतांगीं दीप्तामग्निशिखामिव} %1-49-14

\twolineshloka
{सतुषारावृतां साभ्रां पूर्णचन्द्रप्रभामिव}
{मध्येऽम्भसो दुराधर्षां दीप्तां सूर्यप्रभामिव} %1-49-15

\threelineshloka
{सा हि गौतमवाक्येन दुर्निरीक्ष्या बभूव ह}
{त्रयाणामपि लोकानां यावद् रामस्य दर्शनम्}
{शापस्यान्तमुपागम्य तेषां दर्शनमागता} %1-49-16

\twolineshloka
{राघवौ तु तदा तस्याः पादौ जगृहतुर्मुदा}
{स्मरन्ती गौतमवचः प्रतिजग्राह सा हि तौ} %1-49-17

\twolineshloka
{पाद्यमर्घ्यं तथाऽऽतिथ्यं चकार सुसमाहिता}
{प्रतिजग्राह काकुत्स्थो विधिदृष्टेन कर्मणा} %1-49-18

\twolineshloka
{पुष्पवृष्टिर्महत्यासीद् देवदुन्दुभिनिःस्वनैः}
{गन्धर्वाप्सरसां चैव महानासीत् समुत्सवः} %1-49-19

\twolineshloka
{साधु साध्विति देवास्तामहल्यां समपूजयन्}
{तपोबलविशुद्धाङ्गीं गौतमस्य वशानुगाम्} %1-49-20

\twolineshloka
{गौतमोऽपि महातेजा अहल्यासहितः सुखी}
{रामं सम्पूज्य विधिवत् तपस्तेपे महातपाः} %1-49-21

\twolineshloka
{रामोऽपि परमां पूजां गौतमस्य महामुनेः}
{सकाशाद् विधिवत् प्राप्य जगाम मिथिलां ततः} %1-49-22


॥इत्यार्षे श्रीमद्रामायणे वाल्मीकीये आदिकाव्ये बालकाण्डे अहल्याशापमोक्षः नाम एकोनपञ्चाशः सर्गः ॥१-४९॥
