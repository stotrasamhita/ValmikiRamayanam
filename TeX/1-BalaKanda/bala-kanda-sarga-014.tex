\sect{चतुर्दशः सर्गः — अश्वमेधः}

\twolineshloka
{अथ संवत्सरे पूर्णे तस्मिन् प्राप्ते तुरङ्गमे}
{सरय्वाश्चोत्तरे तीरे राज्ञो यज्ञोऽभ्यवर्तत} %1-14-1

\twolineshloka
{ऋष्यशृङ्गं पुरस्कृत्य कर्म चक्रुर्द्विजर्षभाः}
{अश्वमेधे महायज्ञे राज्ञोऽस्य सुमहात्मनः} %1-14-2

\twolineshloka
{कर्म कुर्वन्ति विधिवद् याजका वेदपारगाः}
{यथाविधि यथान्यायं परिक्रामन्ति शास्त्रतः} %1-14-3

\twolineshloka
{प्रवर्ग्यं शास्त्रतः कृत्वा तथैवोपसदं द्विजाः}
{चक्रुश्च विधिवत् सर्वमधिकं कर्म शास्त्रतः} %1-14-4

\twolineshloka
{अभिपूज्य ततो हृष्टाः सर्वे चक्रुर्यथाविधि}
{प्रातःसवनपूर्वाणि कर्माणि मुनिपुङ्गवाः} %1-14-5

\twolineshloka
{ऐन्द्रश्च विधिवत् दत्तो राजा चाभिषुतोऽनघः}
{मध्यन्दिनं च सवनं प्रावर्तत यथाक्रमम्} %1-14-6

\twolineshloka
{तृतीयसवनं चैव राज्ञोऽस्य सुमहात्मनः}
{चक्रुस्ते शास्त्रतो दृष्ट्वा यथा ब्राह्मणपुङ्गवाः} %1-14-7

\twolineshloka
{आह्वायाञ्चक्रिरे तत्र शक्रादीन् विबुधोत्तमान्}
{ऋष्यशृङ्गादयो मन्त्रैः शिक्षाक्षरसमन्वितैः} %1-14-8

\twolineshloka
{गीतिभिर्मधुरैः स्निग्धैर्मन्त्राह्वानैर्यथार्हतः}
{होतारो ददुरावाह्य हविर्भागान् दिवौकसाम्} %1-14-9

\twolineshloka
{न चाहुतमभूत् तत्र स्खलितं वा न किञ्चन}
{दृश्यते ब्रह्मवत् सर्वं क्षेमयुक्तं हि चक्रिरे} %1-14-10

\twolineshloka
{न तेष्वहःसु श्रान्तो वा क्षुधितो वा न दृश्यते}
{नाविद्वान् ब्राह्मणः कश्चिन्नाशतानुचरस्तथा} %1-14-11

\twolineshloka
{ब्राह्मणा भुञ्जते नित्यं नाथवन्तश्च भुञ्जते}
{तापसा भुञ्जते चापि श्रमणाश्चैव भुञ्जते} %1-14-12

\twolineshloka
{वृद्धाश्च व्याधिताश्चैव स्त्रीबालाश्च तथैव च}
{अनिशं भुञ्जमानानां न तृप्तिरुपलभ्यते} %1-14-13

\twolineshloka
{दीयतां दीयतामन्नं वासांसि विविधानि च}
{इति सञ्चोदितास्तत्र तथा चक्रुरनेकशः} %1-14-14

\twolineshloka
{अन्नकूटाश्च दृश्यन्ते बहवः पर्वतोपमाः}
{दिवसे दिवसे तत्र सिद्धस्य विधिवत् तदा} %1-14-15

\twolineshloka
{नानादेशादनुप्राप्ताः पुरुषाः स्त्रीगणास्तथा}
{अन्नपानैः सुविहितास्तस्मिन् यज्ञे महात्मनः} %1-14-16

\twolineshloka
{अन्नं हि विधिवत्स्वादु प्रशंसन्ति द्विजर्षभाः}
{अहो तृप्ताः स्म भद्रं ते इति शुश्राव राघवः} %1-14-17

\twolineshloka
{स्वलङ्कृताश्च पुरुषा ब्राह्मणान् पर्यवेषयन्}
{उपासन्ते च तानन्ये सुमृष्टमणिकुण्डलाः} %1-14-18

\twolineshloka
{कर्मान्तरे तदा विप्रा हेतुवादान् बहूनपि}
{प्राहुः सुवाग्मिनो धीराः परस्परजिगीषया} %1-14-19

\twolineshloka
{दिवसे दिवसे तत्र संस्तरे कुशला द्विजाः}
{सर्वकर्माणि चक्रुस्ते यथाशास्त्रं प्रचोदिताः} %1-14-20

\twolineshloka
{नाषडङ्गविदत्रासीन्नाव्रतो नाबहुश्रुतः}
{सदस्यास्तस्य वै राज्ञो नावादकुशलो द्विजः} %1-14-21

\twolineshloka
{प्राप्ते यूपोच्छ्रये तस्मिन् षड् बैल्वाः खादिरास्तथा}
{तावन्तो बिल्वसहिताः पर्णिनश्च तथा परे} %1-14-22

\twolineshloka
{श्लेष्मातकमयो दिष्टो देवदारुमयस्तथा}
{द्वावेव तत्र विहितौ बाहुव्यस्तपरिग्रहौ} %1-14-23

\twolineshloka
{कारिताः सर्व एवैते शास्त्रज्ञैर्यज्ञकोविदैः}
{शोभार्थं तस्य यज्ञस्य काञ्चनालङ्कृता भवन्} %1-14-24

\twolineshloka
{एकविंशतियूपास्ते एकविंशत्यरत्नयः}
{वासोभिरेकविंशद्भिरेकैकं समलङ्कृताः} %1-14-25

\twolineshloka
{विन्यस्ता विधिवत् सर्वे शिल्पिभिः सुकृता दृढाः}
{अष्टाश्रयः सर्व एव श्लक्ष्णरूपसमन्विताः} %1-14-26

\twolineshloka
{आच्छादितास्ते वासोभिः पुष्पैर्गन्धैश्च पूजिताः}
{सप्तर्षयो दीप्तिमन्तो विराजन्ते यथा दिवि} %1-14-27

\twolineshloka
{इष्टकाश्च यथान्यायं कारिताश्च प्रमाणतः}
{चितोऽग्निर्ब्राह्मणैस्तत्र कुशलैः शुल्बकर्मणि} %1-14-28

\twolineshloka
{स चित्यो राजसिंहस्य सञ्चितः कुशलैर्द्विजैः}
{गरुडो रुक्मपक्षो वै त्रिगुणोऽष्टादशात्मकः} %1-14-29

\twolineshloka
{नियुक्तास्तत्र पशवस्तत्तदुद्दिश्य दैवतम्}
{उरगाः पक्षिणश्चैव यथाशास्त्रं प्रचोदिताः} %1-14-30

\twolineshloka
{शामित्रे तु हयस्तत्र तथा जलचराश्च ये}
{ऋषिभिः सर्वमेवैतन्नियुक्तं शास्त्रतस्तदा} %1-14-31

\twolineshloka
{पशूनां त्रिशतं तत्र यूपेषु नियतं तदा}
{अश्वरत्नोत्तमं तत्र राज्ञो दशरथस्य ह} %1-14-32

\twolineshloka
{कौसल्या तं हयं तत्र परिचर्य समन्ततः}
{कृपाणैर्विससारैनं त्रिभिः परमया मुदा} %1-14-33

\twolineshloka
{पतत्रिणा तदा सार्धं सुस्थितेन च चेतसा}
{अवसद् रजनीमेकां कौसल्या धर्मकाम्यया} %1-14-34

\twolineshloka
{होताध्वर्युस्तथोद्गाता हस्तेन समयोजयन्}
{महिष्या परिवृत्त्याथ वावातामपरां तथा} %1-14-35

\twolineshloka
{पतत्रिणस्तस्य वपामुद्धृत्य नियतेन्द्रियः}
{ऋत्विक्परमसम्पन्नः श्रपयामास शास्त्रतः} %1-14-36

\twolineshloka
{धूमगन्धं वपायास्तु जिघ्रति स्म नराधिपः}
{यथाकालं यथान्यायं निर्णुदन् पापमात्मनः} %1-14-37

\twolineshloka
{हयस्य यानि चाङ्गानि तानि सर्वाणि ब्राह्मणाः}
{अग्नौ प्रास्यन्ति विधिवत् समस्ताः षोडशर्त्विजः} %1-14-38

\twolineshloka
{प्लक्षशाखासु यज्ञानामन्येषां क्रियते हविः}
{अश्वमेधस्य यज्ञस्य वैतसो भाग इष्यते} %1-14-39

\twolineshloka
{त्र्यहोऽश्वमेधः सङ्ख्यातः कल्पसूत्रेण ब्राह्मणैः}
{चतुष्टोममहस्तस्य प्रथमं परिकल्पितम्} %1-14-40

\twolineshloka
{उक्थ्यं द्वितीयं सङ्ख्यातमतिरात्रं तथोत्तरम्}
{कारितास्तत्र बहवो विहिताः शास्त्रदर्शनात्} %1-14-41

\twolineshloka
{ज्योतिष्टोमायुषी चैवमतिरात्रौ च निर्मितौ}
{अभिजिद्विश्वजिच्चैवमाप्तोर्यामौ महाक्रतुः} %1-14-42

\twolineshloka
{प्राचीं होत्रे ददौ राजा दिशं स्वकुलवर्धनः}
{अध्वर्यवे प्रतीचीं तु ब्रह्मणे दक्षिणां दिशम्} %1-14-43

\twolineshloka
{उद्गात्रे तु तथोदीचीं दक्षिणैषा विनिर्मिता}
{अश्वमेधे महायज्ञे स्वयम्भूविहिते पुरा} %1-14-44

\twolineshloka
{क्रतुं समाप्य तु तदा न्यायतः पुरुषर्षभः}
{ऋत्विग्भ्यो हि ददौ राजा धरां तां कुलवर्धनः} %1-14-45

\twolineshloka
{एवम् दत्त्वा प्रहृष्टोऽभूच्छ्रीमानिक्ष्वाकुनन्दनः}
{ऋत्विजस्त्वब्रुवन् सर्वे राजानं गतकिल्बिषम्} %1-14-46

\twolineshloka
{भवानेव महीं कृत्स्नामेको रक्षितुमर्हति}
{न भूम्या कार्यमस्माकं नहि शक्ताः स्म पालने} %1-14-47

\twolineshloka
{रताः स्वाध्यायकरणे वयं नित्यं हि भूमिप}
{निष्क्रयं किञ्चिदेवेह प्रयच्छतु भवानिति} %1-14-48

\twolineshloka
{मणिरत्नं सुवर्णं वा गावो यद्वा समुद्यतम्}
{तत् प्रयच्छ नृपश्रेष्ठ धरण्या न प्रयोजनम्} %1-14-49

\twolineshloka
{एवमुक्तो नरपतिर्ब्राह्मणैर्वेदपारगैः}
{गवां शतसहस्राणि दश तेभ्यो ददौ नृपः} %1-14-50

\twolineshloka
{दशकोटिं सुवर्णस्य रजतस्य चतुर्गुणम्}
{ऋत्विजस्तु ततः सर्वे प्रददुः सहिता वसु} %1-14-51

\twolineshloka
{ऋष्यशृङ्गाय मुनये वसिष्ठाय च धीमते}
{ततस्ते न्यायतः कृत्वा प्रविभागं द्विजोत्तमाः} %1-14-52

\twolineshloka
{सुप्रीतमनसः सर्वे प्रत्यूचुर्मुदिता भृशम्}
{ततः प्रसर्पकेभ्यस्तु हिरण्यं सुसमाहितः} %1-14-53

\twolineshloka
{जाम्बूनदं कोटिसङ्ख्यं ब्राह्मणेभ्यो ददौ तदा}
{दरिद्राय द्विजायाथ हस्ताभरणमुत्तमम्} %1-14-54

\twolineshloka
{कस्मैचित् याचमानाय ददौ राघवनन्दनः}
{ततः प्रीतेषु विधिवत् द्विजेषु द्विजवत्सलः} %1-14-55

\twolineshloka
{प्रणाममकरोत् तेषां हर्षव्याकुलितेन्द्रियः}
{तस्याशिषोऽथ विविधा ब्राह्मणैः समुदाहृताः} %1-14-56

\twolineshloka
{उदारस्य नृवीरस्य धरण्यां पतितस्य च}
{ततः प्रीतमना राजा प्राप्य यज्ञमनुत्तमम्} %1-14-57

\twolineshloka
{पापापहं स्वर्नयनं दुस्तरं पार्थिवर्षभैः}
{ततोऽब्रवीदृष्यशृङ्गं राजा दशरथस्तदा} %1-14-58

\threelineshloka
{कुलस्य वर्धनं तत् तु कर्तुमर्हसि सुव्रत}
{तथेति च स राजानमुवाच द्विजसत्तमः}
{भविष्यन्ति सुता राजंश्चत्वारस्ते कुलोद्वहाः} %1-14-59

\twolineshloka
{स तस्य वाक्यं मधुरं निशम्य प्रणम्य तस्मै प्रयतो नृपेन्द्रः}
{जगाम हर्षं परमं महात्मा तमृष्यशृङ्गं पुनरप्युवाच} %1-14-60


॥इत्यार्षे श्रीमद्रामायणे वाल्मीकीये आदिकाव्ये बालकाण्डे अश्वमेधः नाम चतुर्दशः सर्गः ॥१-१४॥
