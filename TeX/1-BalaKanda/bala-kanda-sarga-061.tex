\sect{एकषष्ठितमः सर्गः — शुनःशेपविक्रयः}

\twolineshloka
{विश्वामित्रो महातेजाः प्रस्थितान् वीक्ष्य तानृषीन्}
{अब्रवीन्नरशार्दूल सर्वांस्तान् वनवासिनः} %1-61-1

\twolineshloka
{महाविघ्नः प्रवृत्तोऽयं दक्षिणामास्थितो दिशम्}
{दिशमन्यां प्रपत्स्यामस्तत्र तप्स्यामहे तपः} %1-61-2

\twolineshloka
{पश्चिमायां विशालायां पुष्करेषु महात्मनः}
{सुखं तपश्चरिष्यामः सुखं तद्धि तपोवनम्} %1-61-3

\twolineshloka
{एवमुक्त्वा महातेजाः पुष्करेषु महामुनिः}
{तप उग्रं दुराधर्षं तेपे मूलफलाशनः} %1-61-4

\twolineshloka
{एतस्मिन्नेव काले तु अयोध्याधिपतिर्महान्}
{अम्बरीष इति ख्यातो यष्टुं समुपचक्रमे} %1-61-5

\twolineshloka
{तस्य वै यजमानस्य पशुमिन्द्रो जहार ह}
{प्रणष्टे तु पशौ विप्रो राजानमिदमब्रवीत्} %1-61-6

\twolineshloka
{पशुरभ्याहृतो राजन् प्रणष्टस्तव दुर्नयात्}
{अरक्षितारं राजानं घ्नन्ति दोषा नरेश्वर} %1-61-7

\twolineshloka
{प्रायश्चित्तं महद्ध्येतन्नरं वा पुरुषर्षभ}
{आनयस्व पशुं शीघ्रं यावत् कर्म प्रवर्तते} %1-61-8

\twolineshloka
{उपाध्यायवचः श्रुत्वा स राजा पुरुषर्षभः}
{अन्वियेष महाबुद्धिः पशुं गोभिः सहस्रशः} %1-61-9

\twolineshloka
{देशाञ्जनपदांस्तांस्तान् नगराणि वनानि च}
{आश्रमाणि च पुण्यानि मार्गमाणो महीपतिः} %1-61-10

\twolineshloka
{स पुत्रसहितं तात सभार्यं रघुनन्दन}
{भृगुतुंगे समासीनमृचीकं संददर्श ह} %1-61-11

\twolineshloka
{तमुवाच महातेजाः प्रणम्याभिप्रसाद्य च}
{महर्षिं तपसा दीप्तं राजर्षिरमितप्रभः} %1-61-12

\twolineshloka
{पृष्ट्वा सर्वत्र कुशलमृचीकं तमिदं वचः}
{गवां शतसहस्रेण विक्रीणीषे सुतं यदि} %1-61-13

\twolineshloka
{पशोरर्थे महाभाग कृतकृत्योऽस्मि भार्गव}
{सर्वे परिगता देशा यज्ञियं न लभे पशुम्} %1-61-14

\twolineshloka
{दातुमर्हसि मूल्येन सुतमेकमितो मम}
{एवमुक्तो महातेजा ऋचीकस्त्वब्रवीद् वचः} %1-61-15

\twolineshloka
{नाहं ज्येष्ठं नरश्रेष्ठ विक्रीणीयां कथंचन}
{ऋचीकस्य वचः श्रुत्वा तेषां माता महात्मनाम्} %1-61-16

\twolineshloka
{उवाच नरशार्दूलमम्बरीषमिदं वचः}
{अविक्रेयं सुतं ज्येष्ठं भगवानाह भार्गवः} %1-61-17

\twolineshloka
{ममापि दयितं विद्धि कनिष्ठं शुनकं प्रभो}
{तस्मात् कनीयसं पुत्रं न दास्ये तव पार्थिव} %1-61-18

\twolineshloka
{प्रायेण हि नरश्रेष्ठ ज्येष्ठाः पितृषु वल्लभाः}
{मातॄणां च कनीयांसस्तस्माद् रक्ष्ये कनीयसम्} %1-61-19

\twolineshloka
{उक्तवाक्ये मुनौ तस्मिन् मुनिपत्न्यां तथैव च}
{शुनःशेपः स्वयं राम मध्यमो वाक्यमब्रवीत्} %1-61-20

\twolineshloka
{पिता ज्येष्ठमविक्रेयं माता चाह कनीयसम्}
{विक्रेयं मध्यमं मन्ये राजपुत्र नयस्व माम्} %1-61-21

\twolineshloka
{अथ राजा महाबाहो वाक्यान्ते ब्रह्मवादिनः}
{हिरण्यस्य सुवर्णस्य कोटिभी रत्नराशिभिः} %1-61-22

\twolineshloka
{गवां शतसहस्रेण शुनःशेपं नरेश्वरः}
{गृहीत्वा परमप्रीतो जगाम रघुनन्दन} %1-61-23

\twolineshloka
{अम्बरीषस्तु राजर्षी रथमारोप्य सत्वरः}
{शुनःशेपं महातेजा जगामाशु महायशाः} %1-61-24


॥इत्यार्षे श्रीमद्रामायणे वाल्मीकीये आदिकाव्ये बालकाण्डे शुनःशेपविक्रयः नाम एकषष्ठितमः सर्गः ॥१-६१॥
