\sect{पञ्चदशः सर्गः — रावणवधोपायः}

\twolineshloka
{मेधावी तु ततो ध्यात्वा स किञ्चिदिदमुत्तरम्}
{लब्धसंज्ञस्ततस्तं तु वेदज्ञो नृपमब्रवीत्} %1-15-1

\twolineshloka
{इष्टिं तेऽहं करिष्यामि पुत्रीयां पुत्रकारणात्}
{अथर्वशिरसि प्रोक्तैर्मन्त्रैः सिद्धां विधानतः} %1-15-2

\twolineshloka
{ततः प्राक्रमदिष्टिं तां पुत्रीयां पुत्रकारणात्}
{जुहावाग्नौ च तेजस्वी मन्त्रदृष्टेन कर्मणा} %1-15-3

\twolineshloka
{ततो देवाः सगन्धर्वाः सिद्धाश्च परमर्षयः}
{भागप्रतिग्रहार्थं वै समवेता यथाविधि} %1-15-4

\twolineshloka
{ताः समेत्य यथान्यायं तस्मिन् सदसि देवताः}
{अब्रुवन् लोककर्तारं ब्रह्माणं वचनं ततः} %1-15-5

\twolineshloka
{भगवंस्त्वत्प्रसादेन रावणो नाम राक्षसः}
{सर्वान् नो बाधते वीर्याच्छासितुं तं न शक्नुमः} %1-15-6

\twolineshloka
{त्वया तस्मै वरो दत्तः प्रीतेन भगवंस्तदा}
{मानयन्तश्च तं नित्यं सर्वं तस्य क्षमामहे} %1-15-7

\twolineshloka
{उद्वेजयति लोकांस्त्रीनुच्छ्रितान् द्वेष्टि दुर्मतिः}
{शक्रं त्रिदशराजानं प्रधर्षयितुमिच्छति} %1-15-8

\twolineshloka
{ऋषीन्यक्षान्सगन्धर्वानसुरान्ब्राह्मणांस्तथा}
{अतिक्रामति दुर्धर्षो वरदानेन मोहितः} %1-15-9

\twolineshloka
{नैनं सूर्यः प्रतपति पार्श्वे वाति न मारुतः}
{चलोर्मिमाली तं दृष्ट्वा समुद्रोऽपि न कम्पते} %1-15-10

\twolineshloka
{तन्महन्नो भयं तस्माद् राक्षसाद् घोरदर्शनात्}
{वधार्थं तस्य भगवन्नुपायं कर्तुमर्हसि} %1-15-11

\twolineshloka
{एवमुक्तः सुरैः सर्वैश्चिन्तयित्वा ततोऽब्रवीत्}
{हन्तायं विदितस्तस्य वधोपायो दुरात्मनः} %1-15-12

\twolineshloka
{तेन गन्धर्वयक्षाणां देवदानवरक्षसाम्}
{अवध्योऽस्मीति वागुक्ता तथेत्युक्तं च तन्मया} %1-15-13

\twolineshloka
{नाकीर्तयदवज्ञानात् तद् रक्षो मानुषांस्तदा}
{तस्मात् स मानुषाद् वध्यो मृत्युर्नान्योऽस्य विद्यते} %1-15-14

\twolineshloka
{एतच्छ्रुत्वा प्रियं वाक्यं ब्रह्मणा समुदाहृतम्}
{देवा महर्षयः सर्वे प्रहृष्टास्तेऽभवंस्तदा} %1-15-15

\twolineshloka
{एतस्मिन्नन्तरे विष्णुरुपयातो महाद्युतिः}
{शङ्खचक्रगदापाणिः पीतवासा जगत्पतिः} %1-15-16

\twolineshloka
{वैनतेयं समारुह्य भास्करस्तोयदम् यथा}
{तप्तहाटककेयूरो वन्द्यमानः सुरोत्तमैः} %1-15-17

\twolineshloka
{ब्रह्मणा च समागत्य तत्र तस्थौ समाहितः}
{तमब्रुवन् सुराः सर्वे समभिष्टूय सन्नताः} %1-15-18

\twolineshloka
{त्वां नियोक्ष्यामहे विष्णो लोकानां हितकाम्यया}
{राज्ञो दशरथस्य त्वमयोध्याधिपतेर्विभो} %1-15-19

\twolineshloka
{धर्मज्ञस्य वदान्यस्य महर्षिसमतेजसः}
{तस्य भार्यासु तिसृषु ह्रीश्रीकीर्त्युपमासु च} %1-15-20

\twolineshloka
{विष्णो पुत्रत्वमागच्छ कृत्वाऽऽत्मानं चतुर्विधम्}
{तत्र त्वं मानुषो भूत्वा प्रवृद्धं लोककण्टकम्} %1-15-21

\twolineshloka
{अवध्यं दैवतैर्विष्णो समरे जहि रावणम्}
{स हि देवान् सगन्धर्वान् सिद्धांश्च ऋषिसत्तमान्} %1-15-22

\twolineshloka
{राक्षसो रावणो मूर्खो वीर्योद्रेकेण बाधते}
{ऋषयश्च ततस्तेन गन्धर्वाप्सरसस्तथा} %1-15-23

\twolineshloka
{क्रीडन्तो नन्दनवने रौद्रेण विनिपातिताः}
{वधार्थं वयमायातास्तस्य वै मुनिभिः सह} %1-15-24

\twolineshloka
{सिद्धगन्धर्वयक्षाश्च ततस्त्वां शरणं गताः}
{त्वं गतिः परमा देव सर्वेषाम् नः परन्तपः} %1-15-25

\twolineshloka
{वधाय देवशतॄणां नृणां लोके मनः कुरु}
{एवं स्तुतस्तु देवेशो विष्णुस्त्रिदशपुङ्गवः} %1-15-26

\twolineshloka
{पितामहपुरोगांस्तान् सर्वलोकनमस्कृतः}
{अब्रवीत् त्रिदशान् सर्वान् समेतान् धर्मसंहितान्} %1-15-27

\twolineshloka
{भयं त्यजत भद्रं वो हितार्थं युधि रावणम्}
{सपुत्रपौत्रं सामात्यं समन्त्रिज्ञातिबान्धवम्} %1-15-28

\twolineshloka
{हत्वा क्रूरं दुराधर्षं देवर्षीणां भयावहम्}
{दशवर्षसहस्राणि दशवर्षशतानि च} %1-15-29

\twolineshloka
{वत्स्यामि मानुषे लोके पालयन् पृथिवीमिमाम्}
{एवं दत्त्वा वरं देवो देवानां विष्णुरात्मवान्} %1-15-30

\twolineshloka
{मानुष्ये चिन्तयामास जन्मभूमिमथात्मनः}
{ततः पद्मपलाशाक्षः कृत्वात्मानं चतुर्विधम्} %1-15-31

\threelineshloka
{पितरं रोचयामास तदा दशरथं नृपम्}
{ततो देवर्षिगन्धर्वाः सरुद्राः साप्सरोगणाः}
{स्तुतिभिर्दिव्यरूपाभिस्तुष्टुवुर्मधुसूदनम्} %1-15-32

\twolineshloka
{तमुद्धतं रावणमुग्रतेजसं प्रवृद्धदर्पं त्रिदशेश्वरद्विषम्}
{विरावणं साधु तपस्विकण्टकं तपस्विनामुद्धर तं भयावहम्} %1-15-33

\twolineshloka
{तमेव हत्वा सबलं सबान्धवम् विरावणं रावणमुग्रपौरुषम्}
{स्वर्लोकमागच्छ गतज्वरश्चिरं सुरेन्द्रगुप्तं गतदोषकल्मषम्} %1-15-34


॥इत्यार्षे श्रीमद्रामायणे वाल्मीकीये आदिकाव्ये बालकाण्डे रावणवधोपायः नाम पञ्चदशः सर्गः ॥१-१५॥
