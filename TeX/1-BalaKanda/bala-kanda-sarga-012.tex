\sect{द्वादशः सर्गः — अश्वमेधसंभारः}

\twolineshloka
{ततः काले बहुतिथे कस्मिंश्चित् सुमनोहरे}
{वसन्ते समनुप्राप्ते राज्ञो यष्टुं मनोऽभवत्} %1-12-1

\twolineshloka
{ततः प्रणम्य शिरसा तं विप्रं देववर्णिनम्}
{यज्ञाय वरयामास संतानार्थं कुलस्य च} %1-12-2

\twolineshloka
{तथेति च स राजानमुवाच वसुधाधिपम्}
{सम्भाराः सम्भ्रियन्तां ते तुरगश्च विमुच्यताम्} %1-12-3

\twolineshloka
{सरय्वाश्चोत्तरे तीरे यज्ञभूमिर्विधीयताम्}
{ततोऽब्रवीन्नृपो वाक्यं ब्राह्मणान् वेदपारगान्} %1-12-4

\twolineshloka
{सुमन्त्रावाहय क्षिप्रमृत्विजो ब्रह्मवादिनः}
{सुयज्ञं वामदेवं च जाबालिमथ काश्यपम्} %1-12-5

\twolineshloka
{पुरोहितं वसिष्ठं च ये चान्ये द्विजसत्तमाः}
{ततः सुमन्त्रस्त्वरितं गत्वा त्वरितविक्रमः} %1-12-6

\twolineshloka
{समानयत् स तान् सर्वान् समस्तान् वेदपारगान्}
{तान् पूजयित्वा धर्मात्मा राजा दशरथस्तदा} %1-12-7

\twolineshloka
{धर्मार्थसहितं युक्तं श्लक्ष्णं वचनमब्रवीत्}
{मम तातप्यमानस्य पुत्रार्थं नास्ति वै सुखम्} %1-12-8

\twolineshloka
{पुत्रार्थं हयमेधेन यक्ष्यामीति मतिर्मम}
{तदहं यष्टुमिच्छामि हयमेधेन कर्मणा} %1-12-9

\twolineshloka
{ऋषिपुत्रप्रभावेण कामान् प्राप्स्यामि चाप्यहम्}
{ततः साध्विति तद्वाक्यं ब्राह्मणाः प्रत्यपूजयन्} %1-12-10

\twolineshloka
{वसिष्ठप्रमुखाः सर्वे पार्थिवस्य मुखाच्च्युतम्}
{ऋष्यशृङ्गपुरोगाश्च प्रत्यूचुर्नृपतिं तदा} %1-12-11

\twolineshloka
{सम्भाराः सम्भ्रियन्तां ते तुरगश्च विमुच्यताम्}
{सरय्वाश्चोत्तरे तीरे यज्ञभूमिर्विधीयताम्} %1-12-12

\twolineshloka
{सर्वथा प्राप्यसे पुत्रांश्चतुरोऽमितविक्रमान्}
{यस्य ते धार्मिकी बुद्धिरियं पुत्रार्थमागता} %1-12-13

\twolineshloka
{ततः प्रीतोऽभवद् राजा श्रुत्वा तु द्विजभाषितम्}
{अमात्यानब्रवीद् राजा हर्षेणेदं शुभाक्षरम्} %1-12-14

\twolineshloka
{गुरूणां वचनाच्छीघ्रं सम्भाराः सम्भ्रियन्तु मे}
{समर्थाधिष्ठितश्चाश्वः सोपाध्यायो विमुच्यताम्} %1-12-15

\twolineshloka
{सरय्वाश्चोत्तरे तीरे यज्ञभूमिर्विधीयताम्}
{शान्तयश्चाभिवर्धन्तां यथाकल्पं यथाविधि} %1-12-16

\twolineshloka
{शक्यः कर्तुमयं यज्ञः सर्वेणापि महीक्षिता}
{नापराधो भवेत् कष्टो यद्यस्मिन् क्रतुसत्तमे} %1-12-17

\twolineshloka
{छिद्रं हि मृगयन्त्येते विद्वांसो ब्रह्मराक्षसाः}
{विधिहीनस्य यज्ञस्य सद्यः कर्ता विनश्यति} %1-12-18

\twolineshloka
{तद् यथा विधिपूर्वं मे क्रतुरेष समाप्यते}
{तथा विधानं क्रियतां समर्थाः करणेष्विह} %1-12-19

\twolineshloka
{तथेति च ततः सर्वे मन्त्रिणः प्रत्यपूजयन्}
{पार्थिवेन्द्रस्य तद् वाक्यं यथाज्ञप्तमकुर्वत} %1-12-20

\twolineshloka
{ततो द्विजास्ते धर्मज्ञमस्तुवन् पार्थिवर्षभम्}
{अनुज्ञातास्ततः सर्वे पुनर्जग्मुर्यथागतम्} %1-12-21

\twolineshloka
{गतेषु तेषु विप्रेषु मन्त्रिणस्तान् नराधिपः}
{विसर्जयित्वा स्वं वेश्म प्रविवेश महामतिः} %1-12-22


॥इत्यार्षे श्रीमद्रामायणे वाल्मीकीये आदिकाव्ये बालकाण्डे अश्वमेधसंभारः नाम द्वादशः सर्गः ॥१-१२॥
