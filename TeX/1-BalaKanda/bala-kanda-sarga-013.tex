\sect{त्रयोदशः सर्गः — यज्ञशालाप्रवेशः}

\twolineshloka
{पुनः प्राप्ते वसन्ते तु पूर्णः संवत्सरोऽभवत्}
{प्रसवार्थं गतो यष्टुं हयमेधेन वीर्यवान्} %1-13-1

\twolineshloka
{अभिवाद्य वसिष्ठं च न्यायतः प्रतिपूज्य च}
{अब्रवीत् प्रश्रितं वाक्यं प्रसवार्थं द्विजोत्तमम्} %1-13-2

\twolineshloka
{यज्ञो मे क्रियतां ब्रह्मन् यथोक्तं मुनिपुङ्गव}
{यथा न विघ्नाः क्रियन्ते यज्ञाङ्गेषु विधीयताम्} %1-13-3

\twolineshloka
{भवान् स्निग्धः सुहृन्मह्यं गुरुश्च परमो महान्}
{वोढव्यो भवता चैव भारो यज्ञस्य चोद्यतः} %1-13-4

\twolineshloka
{तथेति च स राजानमब्रवीद् द्विजसत्तमः}
{करिष्ये सर्वमेवैतद् भवता यत् समर्थितम्} %1-13-5

\twolineshloka
{ततोऽब्रवीद् द्विजान् वृद्धान् यज्ञकर्मसुनिष्ठितान्}
{स्थापत्ये निष्ठितांश्चैव वृद्धान् परमधार्मिकान्} %1-13-6

\twolineshloka
{कर्मान्तिकान् शिल्पकरान्वर्धकीन् खनकानपि}
{गणकान् शिल्पिनश्चैव तथैव नटनर्तकान्} %1-13-7

\twolineshloka
{तथा शुचीन् शास्त्रविदः पुरुषान् सुबहुश्रुतान्}
{यज्ञकर्म समीहन्तां भवन्तो राजशासनात्} %1-13-8

\twolineshloka
{इष्टका बहुसाहस्री शीघ्रमानीयतामिति}
{उपकार्याः क्रियन्तां च राज्ञो बहुगुणान्विताः} %1-13-9

\twolineshloka
{ब्राह्मणावसथाश्चैव कर्तव्याः शतशः शुभाः}
{भक्ष्यान्नपानैर्बहुभिः समुपेताः सुनिष्ठिताः} %1-13-10

\twolineshloka
{तथा पौरजनस्यापि कर्तव्याश्च सुविस्तराः}
{आगतानां सुदूराच्च पार्थिवानां पृथक् पृथक्} %1-13-11

\twolineshloka
{वाजिवारणशालाश्च तथा शय्यागृहाणि च}
{भटानां महदावासा वैदेशिकनिवासिनाम्} %1-13-12

\twolineshloka
{आवासा बहुभक्ष्या वै सर्वकामैरुपस्थिताः}
{तथा पौरजनस्यापि जनस्य बहुशोभनम्} %1-13-13

\twolineshloka
{दातव्यमन्नं विधिवत् सत्कृत्य न तु लीलया}
{सर्वे वर्णा यथा पूजां प्राप्नुवन्ति सुसत्कृताः} %1-13-14

\twolineshloka
{न चावज्ञा प्रयोक्तव्या कामक्रोधवशादपि}
{यज्ञकर्मसु ये व्यग्राः पुरुषाः शिल्पिनस्तथा} %1-13-15

\twolineshloka
{तेषामपि विशेषेण पूजा कार्या यथाक्रमम्}
{ये स्युः सम्पूजिताः सर्वे वसुभिर्भोजनेन च} %1-13-16

\twolineshloka
{यथा सर्वं सुविहितं न किंचित् परिहीयते}
{तथा भवन्तः कुर्वन्तु प्रीतियुक्तेन चेतसा} %1-13-17

\twolineshloka
{ततः सर्वे समागम्य वसिष्ठमिदमब्रुवन्}
{यथेष्टं तत् सुविहितं न किंचित् परिहीयते} %1-13-18

\twolineshloka
{यथोक्तं तत् करिष्यामो न किंचित् परिहास्यते}
{ततः सुमन्त्रमाहूय वसिष्ठो वाक्यमब्रवीत्} %1-13-19

\twolineshloka
{निमन्त्रयस्व नृपतीन् पृथिव्यां ये च धार्मिकाः}
{ब्राह्मणान्क्षत्रियान्वैश्याञ्छूद्रांश्चैव सहस्रशः} %1-13-20

\twolineshloka
{समानयस्व सत्कृत्य सर्वदेशेषु मानवान्}
{मिथिलाधिपतिं शूरं जनकं सत्यवादिनम्} %1-13-21

\twolineshloka
{तमानय महाभागं स्वयमेव सुसत्कृतम्}
{पूर्वं सम्बन्धिनं ज्ञात्वा ततः पूर्वं ब्रवीमि ते} %1-13-22

\twolineshloka
{तथा काशिपतिं स्निग्धं सततं प्रियवादिनम्}
{सद्वृत्तं देवसंकाशं स्वयमेवानयस्व ह} %1-13-23

\twolineshloka
{तथा केकयराजानं वृद्धं परमधार्मिकम्}
{श्वशुरं राजसिंहस्य सपुत्रं तमिहानय} %1-13-24

\twolineshloka
{अङ्गेश्वरं महेष्वासं रोमपादं सुसत्कृतम्}
{वयस्यं राजसिंहस्य सपुत्रं तमिहानय} %1-13-25

\twolineshloka
{तथा कोसलराजानं भानुमन्तं सुसंस्कृतम्}
{मगधाधिपतिं शूरं सर्वशास्त्रविशारदम्} %1-13-26

\threelineshloka
{प्राप्तिज्ञं परमोदारं सत्कृतं पुरुषर्षभम्}
{राज्ञः शासनमादाय चोदयस्व नृपर्षभान्}
{प्राचीनान् सिन्धुसौवीरान् सौराष्ठ्रेयांश्च पार्थिवान्} %1-13-27

\twolineshloka
{दाक्षिणात्यान् नरेन्द्रांश्च समस्तानानयस्व ह}
{सन्ति स्निग्धाश्च ये चान्ये राजानः पृथिवीतले} %1-13-28

\twolineshloka
{तानानय यथा क्षिप्रं सानुगान् सहबान्धवान्}
{एतान् दूतैर्महाभागैरानयस्व नृपाज्ञया} %1-13-29

\twolineshloka
{वसिष्ठवाक्यं तच्छ्रुत्वा सुमन्त्रस्त्वरितं तदा}
{व्यादिशत् पुरुषांस्तत्र राज्ञामानयने शुभान्} %1-13-30

\twolineshloka
{स्वयमेव हि धर्मात्मा प्रयातो मुनिशासनात्}
{सुमन्त्रस्त्वरितो भूत्वा समानेतुं महामतिः} %1-13-31

\twolineshloka
{ते च कर्मान्तिकाः सर्वे वसिष्ठाय महर्षये}
{सर्वं निवेदयन्ति स्म यज्ञे यदुपकल्पितम्} %1-13-32

\twolineshloka
{ततः प्रीतो द्विजश्रेष्ठस्तान् सर्वान् मुनिरब्रवीत्}
{अवज्ञया न दातव्यं कस्यचिल्लीलयापि वा} %1-13-33

\twolineshloka
{अवज्ञया कृतं हन्याद् दातारं नात्र संशयः}
{ततः कैश्चिदहोरात्रैरुपयाता महीक्षितः} %1-13-34

\twolineshloka
{बहूनि रत्नान्यादाय राज्ञो दशरथस्य ह}
{ततो वसिष्ठः सुप्रीतो राजानमिदमब्रवीत्} %1-13-35

\twolineshloka
{उपयाता नरव्याघ्र राजानस्तव शासनात्}
{मयापि सत्कृताः सर्वे यथार्हं राजसत्तमाः} %1-13-36

\twolineshloka
{यज्ञियं च कृतं सर्वं पुरुषैः सुसमाहितैः}
{निर्यातु च भवान् यष्टुं यज्ञायतनमन्तिकात्} %1-13-37

\twolineshloka
{सर्वकामैरुपहृतैरुपेतं वै समन्ततः}
{द्रष्टुमर्हसि राजेन्द्र मनसेव विनिर्मितम्} %1-13-38

\twolineshloka
{तथा वसिष्ठवचनादृष्यशृङ्गस्य चोभयोः}
{दिवसे शुभनक्षत्रे निर्यातो जगतीपतिः} %1-13-39

\twolineshloka
{ततो वसिष्ठप्रमुखाः सर्व एव द्विजोत्तमाः}
{ऋष्यशृङ्गं पुरस्कृत्य यज्ञकर्मारभंस्तदा} %1-13-40

\twolineshloka
{यज्ञवाटं गताः सर्वे यथाशास्त्रं यथाविधि}
{श्रीमांश्च सह पत्नीभी राजा दीक्षामुपाविशत्} %1-13-41


॥इत्यार्षे श्रीमद्रामायणे वाल्मीकीये आदिकाव्ये बालकाण्डे यज्ञशालाप्रवेशः नाम त्रयोदशः सर्गः ॥१-१३॥
