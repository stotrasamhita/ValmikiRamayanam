\sect{चत्वारिंशः सर्गः — कपिलदर्शनम्}

\twolineshloka
{देवतानां वचः श्रुत्वा भगवान् वै पितामहः}
{प्रत्युवाच सुसंत्रस्तान् कृतान्तबलमोहितान्} %1-40-1

\twolineshloka
{यस्येयं वसुधा कृत्स्ना वासुदेवस्य धीमतः}
{महिषी माधवस्यैषा स एव भगवन् प्रभुः} %1-40-2

\twolineshloka
{कापिलं रूपमास्थाय धारयत्यनिशं धराम्}
{तस्य कोपाग्निना दग्धा भविष्यन्ति नृपात्मजाः} %1-40-3

\twolineshloka
{पृथिव्याश्चापि निर्भेदो दृष्ट एव सनातनः}
{सगरस्य च पुत्राणां विनाशो दीर्घदर्शिनाम्} %1-40-4

\twolineshloka
{पितामहवचः श्रुत्वा त्रयस्त्रिंशदरिंदमाः}
{देवाः परमसंहृष्टाः पुनर्जग्मुर्यथागतम्} %1-40-5

\twolineshloka
{सगरस्य च पुत्राणां प्रादुरासीन्महास्वनः}
{पृथिव्यां भिद्यमानायां निर्घातसमनिस्वनः} %1-40-6

\twolineshloka
{ततो भित्त्वा महीं सर्वां कृत्वा चापि प्रदक्षिणम्}
{सहिताः सागराः सर्वे पितरं वाक्यमब्रुवन्} %1-40-7

\twolineshloka
{परिक्रान्ता मही सर्वा सत्त्ववन्तश्च सूदिताः}
{देवदानवरक्षांसि पिशाचोरगपन्नगाः} %1-40-8

\twolineshloka
{न च पश्यामहेऽश्वं ते अश्वहर्तारमेव च}
{किं करिष्याम भद्रं ते बुद्धिरत्र विचार्यताम्} %1-40-9

\twolineshloka
{तेषां तद्वचनं श्रुत्वा पुत्राणां राजसत्तमः}
{समन्युरब्रवीद् वाक्यं सगरो रघुनन्दन} %1-40-10

\twolineshloka
{भूयः खनत भद्रं वो विभेद्य वसुधातलम्}
{अश्वहर्तारमासाद्य कृतार्थाश्च निवर्तत} %1-40-11

\twolineshloka
{पितुर्वचनमासाद्य सगरस्य महात्मनः}
{षष्टिः पुत्रसहस्राणि रसातलमभिद्रवन्} %1-40-12

\twolineshloka
{खन्यमाने ततस्तस्मिन् ददृशुः पर्वतोपमम्}
{दिशागजं विरूपाक्षं धारयन्तं महीतलम्} %1-40-13

\twolineshloka
{सपर्वतवनां कृत्स्नां पृथिवीं रघुनन्दन}
{धारयामास शिरसा विरूपाक्षो महागजः} %1-40-14

\twolineshloka
{यदा पर्वणि काकुत्स्थ विश्रमार्थं महागजः}
{खेदाच्चालयते शीर्षं भूमिकम्पस्तदा भवेत्} %1-40-15

\twolineshloka
{ते तं प्रदक्षिणं कृत्वा दिशापालं महागजम्}
{मानयन्तो हि ते राम जग्मुर्भित्त्वा रसातलम्} %1-40-16

\twolineshloka
{ततः पूर्वां दिशं भित्त्वा दक्षिणां बिभिदुः पुनः}
{दक्षिणस्यामपि दिशि ददृशुस्ते महागजम्} %1-40-17

\twolineshloka
{महापद्मं महात्मानं सुमहत्पर्वतोपमम्}
{शिरसा धारयन्तं गां विस्मयं जग्मुरुत्तमम्} %1-40-18

\twolineshloka
{ते तं प्रदक्षिणं कृत्वा सगरस्य महात्मनः}
{षष्टिः पुत्रसहस्राणि पश्चिमां बिभिदुर्दिशम्} %1-40-19

\twolineshloka
{पश्चिमायामपि दिशि महान्तमचलोपमम्}
{दिशागजं सौमनसं ददृशुस्ते महाबलाः} %1-40-20

\twolineshloka
{तं ते प्रदक्षिणं कृत्वा पृष्ट्वा चापि निरामयम्}
{खनन्तः समुपक्रान्ता दिशं सोमवतीं तदा} %1-40-21

\twolineshloka
{उत्तरस्यां रघुश्रेष्ठ ददृशुर्हिमपाण्डुरम्}
{भद्रं भद्रेण वपुषा धारयन्तं महीमिमाम्} %1-40-22

\twolineshloka
{समालभ्य ततः सर्वे कृत्वा चैनं प्रदक्षिणम्}
{षष्टिः पुत्रसहस्राणि बिभिदुर्वसुधातलम्} %1-40-23

\twolineshloka
{ततः प्रागुत्तरां गत्वा सागराः प्रथितां दिशम्}
{रोषादभ्यखनन् सर्वे पृथिवीं सगरात्मजाः} %1-40-24

\twolineshloka
{ते तु सर्वे महत्मानो भिमवेगा महबलाः}
{ददृशुः कपिलं तत्र वासुदेवं सनातनम्} %1-40-25

\twolineshloka
{हयं च तस्य देवस्य चरन्तमविदूरतः}
{प्रहर्षमतुलं प्राप्ताः सर्वे ते रघुनंदन} %1-40-26

\twolineshloka
{ते तं यज्ञहनं ज्ञात्वा क्रोधपर्याकुलेक्षणाः}
{खनित्रलाङ्गलधरा नानावृक्षशिलाधराः} %1-40-27

\twolineshloka
{अभ्यधावन्त संक्रुद्धास्तिष्ठ तिष्ठेति चाब्रुवन्}
{अस्माकं त्वं हि तुरगं यज्ञियं हृतवानसि} %1-40-28

\twolineshloka
{दुर्मेधस्त्वं हि संप्राप्तान् विद्धि नः सगरात्मजान्}
{श्रुत्वा तद्वचनं तेषां कपिलो रघुनन्दन} %1-40-29

\threelineshloka
{रोषेण महताविष्टो हुंकारमकरोत् तदा}
{ततस्तेनाप्रमेयेण कपिलेन महात्मना}
{भस्मराशीकृताः सर्वे काकुत्स्थ सगरात्मजाः} %1-40-30


॥इत्यार्षे श्रीमद्रामायणे वाल्मीकीये आदिकाव्ये बालकाण्डे कपिलदर्शनम् नाम चत्वारिंशः सर्गः ॥१-४०॥
