\sect{अष्टात्रिंशः सर्गः — सगरपुत्रजन्म}

\twolineshloka
{तां कथां कौशिको रामे निवेद्य मधुराक्षराम्}
{पुनरेवापरं वाक्यं काकुत्स्थमिदमब्रवीत्} %1-38-1

\twolineshloka
{अयोध्याधिपतिर्वीर पूर्वमासीन्नराधिपः}
{सगरो नाम धर्मात्मा प्रजाकामः स चाप्रजः} %1-38-2

\twolineshloka
{वैदर्भदुहिता राम केशिनी नाम नामतः}
{ज्येष्ठा सगरपत्नी सा धर्मिष्ठा सत्यवादिनी} %1-38-3

\twolineshloka
{अरिष्टनेमेर्दुहिता सुपर्णभगिनी तु सा}
{द्वितीया सगरस्यासीत् पत्नी सुमतिसंज्ञिता} %1-38-4

\twolineshloka
{ताभ्यां सह महाराजः पत्नीभ्यां तप्तवांस्तपः}
{हिमवन्तं समासाद्य भृगुप्रस्रवणे गिरौ} %1-38-5

\twolineshloka
{अथ वर्षशते पूर्णे तपसाऽऽराधितो मुनिः}
{सगराय वरं प्रादाद् भृगुः सत्यवतां वरः} %1-38-6

\twolineshloka
{अपत्यलाभः सुमहान् भविष्यति तवानघ}
{कीर्तिं चाप्रतिमां लोके प्राप्स्यसे पुरुषर्षभ} %1-38-7

\twolineshloka
{एका जनयिता तात पुत्रं वंशकरं तव}
{षष्टिं पुत्रसहस्राणि अपरा जनयिष्यति} %1-38-8

\twolineshloka
{भाषमाणं महात्मानं राजपुत्र्यौ प्रसाद्य तम्}
{ऊचतुः परमप्रीते कृताञ्जलिपुटे तदा} %1-38-9

\twolineshloka
{एकः कस्याः सुतो ब्रह्मन् का बहूञ्जनयिष्यति}
{श्रोतुमिच्छावहे ब्रह्मन् सत्यमस्तु वचस्तव} %1-38-10

\twolineshloka
{तयोस्तद् वचनं श्रुत्वा भृगुः परमधार्मिकः}
{उवाच परमां वाणीं स्वच्छन्दोऽत्र विधीयताम्} %1-38-11

\twolineshloka
{एको वंशकरो वास्तु बहवो वा महाबलाः}
{कीर्तिमन्तो महोत्साहाः का वा कं वरमिच्छति} %1-38-12

\twolineshloka
{मुनेस्तु वचनं श्रुत्वा केशिनी रघुनन्दन}
{पुत्रं वंशकरं राम जग्राह नृपसंनिधौ} %1-38-13

\twolineshloka
{षष्टिं पुत्रसहस्राणि सुपर्णभगिनी तदा}
{महोत्साहान् कीर्तिमतो जग्राह सुमतिः सुतान्} %1-38-14

\twolineshloka
{प्रदक्षिणमृषिं कृत्वा शिरसाभिप्रणम्य तम्}
{जगाम स्वपुरं राजा सभार्यो रघुनन्दन} %1-38-15

\twolineshloka
{अथ काले गते तस्य ज्येष्ठा पुत्रं व्यजायत}
{असमञ्ज इति ख्यातं केशिनी सगरात्मजम्} %1-38-16

\twolineshloka
{सुमतिस्तु नरव्याघ्र गर्भतुम्बं व्यजायत}
{षष्टिः पुत्रसहस्राणि तुम्बभेदाद् विनिःसृताः} %1-38-17

\twolineshloka
{घृतपूर्णेषु कुम्भेषु धात्र्यस्तान् समवर्धयन्}
{कालेन महता सर्वे यौवनं प्रतिपेदिरे} %1-38-18

\twolineshloka
{अथ दीर्घेण कालेन रूपयौवनशालिनः}
{षष्टिः पुत्रसहस्राणि सगरस्याभवंस्तदा} %1-38-19

\twolineshloka
{स च ज्येष्ठो नरश्रेष्ठ सगरस्यात्मसम्भवः}
{बालान् गृहीत्वा तु जले सरय्वा रघुनन्दन} %1-38-20

\twolineshloka
{प्रक्षिप्य प्राहसन्नित्यं मज्जतस्तान् निरीक्ष्य वै}
{एवं पापसमाचारः सज्जनप्रतिबाधकः} %1-38-21

\twolineshloka
{पौराणामहिते युक्तः पित्रा निर्वासितः पुरात्}
{तस्य पुत्रोंऽशुमान् नाम असमञ्जस्य वीर्यवान्} %1-38-22

\twolineshloka
{सम्मतः सर्वलोकस्य सर्वस्यापि प्रियंवदः}
{ततः कालेन महता मतिः समभिजायत} %1-38-23

\threelineshloka
{सगरस्य नरश्रेष्ठ यजेयमिति निश्चिता}
{स कृत्वा निश्चयं राजा सोपाध्यायगणस्तदा}
{यज्ञकर्मणि वेदज्ञो यष्टुं समुपचक्रमे} %1-38-24


॥इत्यार्षे श्रीमद्रामायणे वाल्मीकीये आदिकाव्ये बालकाण्डे सगरपुत्रजन्म नाम अष्टात्रिंशः सर्गः ॥१-३८॥
