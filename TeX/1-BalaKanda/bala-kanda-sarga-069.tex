\sect{एकोनसप्ततितमः सर्गः — दशरथजनकसमागमः}

\twolineshloka
{ततो रात्र्यां व्यतीतायां सोपाध्यायः सबान्धवः}
{राजा दशरथो हृष्टः सुमन्त्रमिदमब्रवीत्} %1-69-1

\twolineshloka
{अद्य सर्वे धनाध्यक्षा धनमादाय पुष्कलम्}
{व्रजन्त्वग्रे सुविहिता नानारत्नसमन्विताः} %1-69-2

\twolineshloka
{चतुरङ्गबलं चापि शीघ्रं निर्यातु सर्वशः}
{ममाज्ञासमकालं च यानं युग्यमनुत्तमम्} %1-69-3

\twolineshloka
{वसिष्ठो वामदेवश्च जाबालिरथ कश्यपः}
{मार्कण्डेयस्तु दीर्घायुर्ऋषिः कात्यायनस्तथा} %1-69-4

\twolineshloka
{एते द्विजाः प्रयान्त्वग्रे स्यन्दनं योजयस्व मे}
{यथा कालात्ययो न स्याद् दूता हि त्वरयन्ति माम्} %1-69-5

\twolineshloka
{वचनाच्च नरेन्द्रस्य सेना च चतुरङ्गिणी}
{राजानमृषिभिः सार्धं व्रजन्तं पृष्ठतोऽन्वयात्} %1-69-6

\twolineshloka
{गत्वा चतुरहं मार्गं विदेहानभ्युपेयिवान्}
{राजा च जनकः श्रीमान् श्रुत्वा पूजामकल्पयत्} %1-69-7

\twolineshloka
{ततो राजानमासाद्य वृद्धं दशरथं नृपम्}
{मुदितो जनको राजा प्रहर्षं परमं ययौ} %1-69-8

\twolineshloka
{उवाच वचनं श्रेष्ठो नरश्रेष्ठं मुदान्वितम्}
{स्वागतं ते नरश्रेष्ठ दिष्ट्या प्राप्तोऽसि राघव} %1-69-9

\twolineshloka
{पुत्रयोरुभयोः प्रीतिं लप्स्यसे वीर्यनिर्जिताम्}
{दिष्ट्या प्राप्तो महातेजा वसिष्ठो भगवानृषिः} %1-69-10

\twolineshloka
{सह सर्वैर्द्विजश्रेष्ठैर्देवैरिव शतक्रतुः}
{दिष्ट्या मे निर्जिता विघ्ना दिष्ट्या मे पूजितं कुलम्} %1-69-11

\twolineshloka
{राघवैः सह सम्बन्धाद् वीर्यश्रेष्ठैर्महाबलैः}
{श्वः प्रभाते नरेन्द्र त्वं संवर्तयितुमर्हसि} %1-69-12

\twolineshloka
{यज्ञस्यान्ते नरश्रेष्ठ विवाहमृषिसत्तमैः}
{तस्य तद् वचनं श्रुत्वा ऋषिमध्ये नराधिपः} %1-69-13

\twolineshloka
{वाक्यं वाक्यविदां श्रेष्ठः प्रत्युवाच महीपतिम्}
{प्रतिग्रहो दातृवशः श्रुतमेतन्मया पुरा} %1-69-14

\twolineshloka
{यथा वक्ष्यसि धर्मज्ञ तत् करिष्यामहे वयम्}
{तद् धर्मिष्ठं यशस्यं च वचनं सत्यवादिनः} %1-69-15

\twolineshloka
{श्रुत्वा विदेहाधिपतिः परं विस्मयमागतः}
{ततः सर्वे मुनिगणाः परस्परसमागमे} %1-69-16

\twolineshloka
{हर्षेण महता युक्तास्तां रात्रिमवसन् सुखम्}
{अथ रामो महातेजा लक्ष्मणेन समं ययौ} %1-69-17

\twolineshloka
{विश्वामित्रं पुरस्कृत्य पितुः पादावुपस्पृशन्}
{राजा च राघवौ पुत्रौ निशाम्य परिहर्षितः} %1-69-18

\threelineshloka
{उवास परमप्रीतो जनकेनाभिपूजितः}
{जनकोऽपि महातेजाः क्रिया धर्मेण तत्त्ववित्}
{यज्ञस्य च सुताभ्यां च कृत्वा रात्रिमुवास ह} %1-69-19


॥इत्यार्षे श्रीमद्रामायणे वाल्मीकीये आदिकाव्ये बालकाण्डे दशरथजनकसमागमः नाम एकोनसप्ततितमः सर्गः ॥१-६९॥
