\sect{एकसप्ततितमः सर्गः — कन्यादानप्रतिश्रवः}

\twolineshloka
{एवं ब्रुवाणं जनकः प्रत्युवाच कृताञ्जलिः}
{श्रोतुमर्हसि भद्रं ते कुलं नः परिकीर्तितम्} %1-71-1

\twolineshloka
{प्रदाने हि मुनिश्रेष्ठ कुलं निरवशेषतः}
{वक्तव्यं कुलजातेन तन्निबोध महामते} %1-71-2

\twolineshloka
{राजाभूत् त्रिषु लोकेषु विश्रुतः स्वेन कर्मणा}
{निमिः परमधर्मात्मा सर्वसत्त्ववतां वरः} %1-71-3

\twolineshloka
{तस्य पुत्रो मिथिर्नाम जनको मिथिपुत्रकः}
{प्रथमो जनको राजा जनकादप्युदावसुः} %1-71-4

\twolineshloka
{उदावसोस्तु धर्मात्मा जातो वै नन्दिवर्धनः}
{नन्दिवर्धसुतः शूरः सुकेतुर्नाम नामतः} %1-71-5

\twolineshloka
{सुकेतोरपि धर्मात्मा देवरातो महाबलः}
{देवरातस्य राजर्षेर्बृहद्रथ इति स्मृतः} %1-71-6

\twolineshloka
{बृहद्रथस्य शूरोऽभून्महावीरः प्रतापवान्}
{महावीरस्य धृतिमान् सुधृतिः सत्यविक्रमः} %1-71-7

\twolineshloka
{सुधृतेरपि धर्मात्मा धृष्टकेतुः सुधार्मिकः}
{धृष्टकेतोश्च राजर्षेर्हर्यश्व इति विश्रुतः} %1-71-8

\twolineshloka
{हर्यश्वस्य मरुः पुत्रो मरोः पुत्रः प्रतीन्धकः}
{प्रतीन्धकस्य धर्मात्मा राजा कीर्तिरथः सुतः} %1-71-9

\twolineshloka
{पुत्रः कीर्तिरथस्यापि देवमीढ इति स्मृतः}
{देवमीढस्य विबुधो विबुधस्य महीध्रकः} %1-71-10

\twolineshloka
{महीध्रकसुतो राजा कीर्तिरातो महाबलः}
{कीर्तिरातस्य राजर्षेर्महारोमा व्यजायत} %1-71-11

\twolineshloka
{महारोम्णस्तु धर्मात्मा स्वर्णरोमा व्यजायत}
{स्वर्णरोम्णस्तु राजर्षेर्ह्रस्वरोमा व्यजायत} %1-71-12

\twolineshloka
{तस्य पुत्रद्वयं राज्ञो धर्मज्ञस्य महात्मनः}
{ज्येष्ठोऽहमनुजो भ्राता मम वीरः कुशध्वजः} %1-71-13

\twolineshloka
{मां तु ज्येष्ठं पिता राज्ये सोऽभिषिच्य पिता मम}
{कुशध्वजं समावेश्य भारं मयि वनं गतः} %1-71-14

\twolineshloka
{वृद्धे पितरि स्वर्याते धर्मेण धुरमावहम्}
{भ्रातरं देवसङ्काशं स्नेहात् पश्यन् कुशध्वजम्} %1-71-15

\twolineshloka
{कस्यचित्त्वथ कालस्य साङ्काश्यादागतः पुरात्}
{सुधन्वा वीर्यवान् राजा मिथिलामवरोधकः} %1-71-16

\twolineshloka
{स च मे प्रेषयामास शैवं धनुरनुत्तमम्}
{सीता च कन्या पद्माक्षी मह्यं वै दीयतामिति} %1-71-17

\twolineshloka
{तस्याप्रदानान्महर्षे युद्धमासीन्मया सह}
{स हतोऽभिमुखो राजा सुधन्वा तु मया रणे} %1-71-18

\twolineshloka
{निहत्य तं मुनिश्रेष्ठ सुधन्वानं नराधिपम्}
{साङ्काश्ये भ्रातरं शूरमभ्यषिञ्चं कुशध्वजम्} %1-71-19

\twolineshloka
{कनीयानेष मे भ्राता अहं ज्येष्ठो महामुने}
{ददामि परमप्रीतो वध्वौ ते मुनिपुङ्गव} %1-71-20

\twolineshloka
{सीतां रामाय भद्रं ते ऊर्मिलां लक्ष्मणाय वै}
{वीर्यशुल्कां मम सुतां सीतां सुरसुतोपमाम्} %1-71-21

\twolineshloka
{द्वितीयामूर्मिलां चैव त्रिर्वदामि न संशयः}
{ददामि परमप्रीतो वध्वौ ते मुनिपुङ्गव} %1-71-22

\twolineshloka
{रामलक्ष्मणयो राजन् गोदानं कारयस्व ह}
{पितृकार्यं च भद्रं ते ततो वैवाहिकं कुरु} %1-71-23

\threelineshloka
{मघा ह्यद्य महाबाहो तृतीयदिवसे प्रभो}
{फल्गुन्यामुत्तरे राजंस्तस्मिन् वैवाहिकं कुरु}
{रामलक्ष्मणयोरर्थे दानं कार्यं सुखोदयम्} %1-71-24


॥इत्यार्षे श्रीमद्रामायणे वाल्मीकीये आदिकाव्ये बालकाण्डे कन्यादानप्रतिश्रवः नाम एकसप्ततितमः सर्गः ॥१-७१॥
