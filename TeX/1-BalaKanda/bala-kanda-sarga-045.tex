\sect{पञ्चचत्वारिंशः सर्गः — अमृतोत्पत्तिः}

\twolineshloka
{विश्वामित्रवचः श्रुत्वा राघवः सहलक्ष्मणः}
{विस्मयं परमं गत्वा विश्वामित्रमथाब्रवीत्} %1-45-1

\twolineshloka
{अत्यद्भुतमिदं ब्रह्मन् कथितं परमं त्वया}
{गङ्गावतरणं पुण्यं सागरस्यापि पूरणम्} %1-45-2

\twolineshloka
{क्षणभूतेव नौ रात्रिः संवृत्तेयं परन्तप}
{इमां चिन्तयतोः सर्वां निखिलेन कथां तव} %1-45-3

\twolineshloka
{तस्य सा शर्वरी सर्वा मम सौमित्रिणा सह}
{जगाम चिन्तयानस्य विश्वामित्र कथां शुभाम्} %1-45-4

\twolineshloka
{ततः प्रभाते विमले विश्वामित्रं तपोधनम्}
{उवाच राघवो वाक्यं कृताह्निकमरिन्दमः} %1-45-5

\twolineshloka
{गता भगवती रात्रिः श्रोतव्यं परमं श्रुतम्}
{तराम सरितां श्रेष्ठां पुण्यां त्रिपथगां नदीम्} %1-45-6

\twolineshloka
{नौरेषा हि सुखास्तीर्णा ऋषीणां पुण्यकर्मणाम्}
{भगवन्तमिह प्राप्तं ज्ञात्वा त्वरितमागता} %1-45-7

\twolineshloka
{तस्य तद् वचनं श्रुत्वा राघवस्य महात्मनः}
{सन्तारं कारयामास सर्षिसङ्घस्य कौशिकः} %1-45-8

\twolineshloka
{उत्तरं तीरमासाद्य सम्पूज्यर्षिगणं ततः}
{गङ्गाकूले निविष्टास्ते विशालां ददृशुः पुरीम्} %1-45-9

\twolineshloka
{ततो मुनिवरस्तूर्णं जगाम सहराघवः}
{विशालां नगरीं रम्यां दिव्यां स्वर्गोपमां तदा} %1-45-10

\twolineshloka
{अथ रामो महाप्राज्ञो विश्वामित्रं महामुनिम्}
{पप्रच्छ प्राञ्जलिर्भूत्वा विशालामुत्तमां पुरीम्} %1-45-11

\twolineshloka
{कतमो राजवंशोऽयं विशालायां महामुने}
{श्रोतुमिच्छामि भद्रं ते परं कौतूहलं हि मे} %1-45-12

\twolineshloka
{तस्य तद् वचनं श्रुत्वा रामस्य मुनिपुङ्गवः}
{आख्यातुं तत्समारेभे विशालायाः पुरातनम्} %1-45-13

\twolineshloka
{श्रूयतां राम शक्रस्य कथां कथयतः श्रुताम्}
{अस्मिन् देशे हि यद् वृत्तं शृणु तत्त्वेन राघव} %1-45-14

\twolineshloka
{पूर्वं कृतयुगे राम दितेः पुत्रा महाबलाः}
{अदितेश्च महाभागा वीर्यवन्तः सुधार्मिकाः} %1-45-15

\twolineshloka
{ततस्तेषां नरव्याघ्र बुद्धिरासीन्महात्मनाम्}
{अमरा विजराश्चैव कथं स्यामो निरामयाः} %1-45-16

\twolineshloka
{तेषां चिन्तयतां तत्र बुद्धिरासीद् विपश्चिताम्}
{क्षीरोदमथनं कृत्वा रसं प्राप्स्याम तत्र वै} %1-45-17

\twolineshloka
{ततो निश्चित्य मथनं योक्त्रं कृत्वा च वासुकिम्}
{मन्थानं मन्दरं कृत्वा ममन्थुरमितौजसः} %1-45-18

\twolineshloka
{अथ वर्षसहस्रेण योक्त्रसर्पशिरांसि च}
{वमन्तोऽतिविषं तत्र ददंशुर्दशनैः शिलाः} %1-45-19

\twolineshloka
{उत्पपाताग्निसङ्काशं हालाहलमहाविषम्}
{तेन दग्धं जगत् सर्वं सदेवासुरमानुषम्} %1-45-20

\twolineshloka
{अथ देवा महादेवं शङ्करं शरणार्थिनः}
{जग्मुः पशुपतिं रुद्रं त्राहि त्राहीति तुष्टुवुः} %1-45-21

\twolineshloka
{एवमुक्तस्ततो देवैर्देवदेवेश्वरः प्रभुः}
{प्रादुरासीत् ततोऽत्रैव शङ्खचक्रधरो हरिः} %1-45-22

\twolineshloka
{उवाचैनं स्मितं कृत्वा रुद्रं शूलधरं हरिः}
{दैवतैर्मथ्यमाने तु यत्पूर्वं समुपस्थितम्} %1-45-23

\twolineshloka
{तत् त्वदीयं सुरश्रेष्ठ सुराणामग्रतो हि यत्}
{अग्रपूजामिह स्थित्वा गृहाणेदं विषं प्रभो} %1-45-24

\twolineshloka
{इत्युक्त्वा च सुरश्रेष्ठस्तत्रैवान्तरधीयत}
{देवतानां भयं दृष्ट्वा श्रुत्वा वाक्यं तु शार्ङ्गिणः} %1-45-25

\twolineshloka
{हालाहलं विषं घोरं सञ्जग्राहामृतोपमम्}
{देवान् विसृज्य देवेशो जगाम भगवान् हरः} %1-45-26

\twolineshloka
{ततो देवासुराः सर्वे ममन्थू रघुनन्दन}
{प्रविवेशाथ पातालं मन्थानः पर्वतोत्तमः} %1-45-27

\twolineshloka
{ततो देवाः सगन्धर्वास्तुष्टुवुर्मधुसूदनम्}
{त्वं गतिः सर्वभूतानां विशेषेण दिवौकसाम्} %1-45-28

\twolineshloka
{पालयास्मान् महाबाहो गिरिमुद्धर्तुमर्हसि}
{इति श्रुत्वा हृषीकेशः कामठं रूपमास्थितः} %1-45-29

\twolineshloka
{पर्वतं पृष्ठतः कृत्वा शिष्ये तत्रोदधौ हरिः}
{पर्वताग्रं तु लोकात्मा हस्तेनाक्रम्य केशवः} %1-45-30

\twolineshloka
{देवानां मध्यतः स्थित्वा ममन्थ पुरुषोत्तमः}
{अथ वर्षसहस्रेण आयुर्वेदमयः पुमान्} %1-45-31

\twolineshloka
{उदतिष्ठत् सुधर्मात्मा सदण्डः सकमण्डलुः}
{पूर्वं धन्वन्तरिर्नाम अप्सराश्च सुवर्चसः} %1-45-32

\twolineshloka
{अप्सु निर्मथनादेव रसात् तस्माद् वरस्त्रियः}
{उत्पेतुर्मनुजश्रेष्ठ तस्मादप्सरसोऽभवन्} %1-45-33

\twolineshloka
{षष्टिः कोट्योऽभवंस्तासामप्सराणां सुवर्चसाम्}
{असङ्ख्येयास्तु काकुत्स्थ यास्तासां परिचारिकाः} %1-45-34

\twolineshloka
{न ताः स्म प्रतिगृह्णन्ति सर्वे ते देवदानवाः}
{अप्रतिग्रहणादेव ता वै साधारणाः स्मृताः} %1-45-35

\twolineshloka
{वरुणस्य ततः कन्या वारुणी रघुनन्दन}
{उत्पपात महाभागा मार्गमाणा परिग्रहम्} %1-45-36

\twolineshloka
{दितेः पुत्रा न तां राम जगृहुर्वरुणात्मजाम्}
{अदितेस्तु सुता वीर जगृहुस्तामनिन्दिताम्} %1-45-37

\twolineshloka
{असुरास्तेन दैतेयाः सुरास्तेनादितेः सुताः}
{हृष्टाः प्रमुदिताश्चासन् वारुणीग्रहणात् सुराः} %1-45-38

\twolineshloka
{उच्चैःश्रवा हयश्रेष्ठो मणिरत्नं च कौस्तुभम्}
{उदतिष्ठन्नरश्रेष्ठ तथैवामृतमुत्तमम्} %1-45-39

\twolineshloka
{अथ तस्य कृते राम महानासीत् कुलक्षयः}
{अदितेस्तु ततः पुत्रा दितिपुत्रानयोधयन्} %1-45-40

\twolineshloka
{एकतामगमन् सर्वे असुरा राक्षसैः सह}
{युद्धमासीन्महाघोरं वीर त्रैलोक्यमोहनम्} %1-45-41

\twolineshloka
{यदा क्षयं गतं सर्वं तदा विष्णुर्महाबलः}
{अमृतं सोऽहरत् तूर्णं मायामास्थाय मोहिनीम्} %1-45-42

\twolineshloka
{ये गताभिमुखं विष्णुमक्षरं पुरुषोत्तमम्}
{सम्पिष्टास्ते तदा युद्धे विष्णुना प्रभविष्णुना} %1-45-43

\twolineshloka
{अदितेरात्मजा वीरा दितेः पुत्रान् निजघ्निरे}
{अस्मिन् घोरे महायुद्धे दैतेयादित्ययोर्भृशम्} %1-45-44

\twolineshloka
{निहत्य दितिपुत्रांस्तु राज्यं प्राप्य पुरन्दरः}
{शशास मुदितो लोकान् सर्षिसङ्घान् सचारणान्} %1-45-45


॥इत्यार्षे श्रीमद्रामायणे वाल्मीकीये आदिकाव्ये बालकाण्डे अमृतोत्पत्तिः नाम पञ्चचत्वारिंशः सर्गः ॥१-४५॥
