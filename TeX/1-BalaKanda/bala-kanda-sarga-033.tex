\sect{त्रयस्त्रिंशः सर्गः — ब्रह्मदत्तविवाहः}

\twolineshloka
{तस्य तद्वचनं श्रुत्वा कुशनाभस्य धीमतः}
{शिरोभिश्चरणौ स्पृष्ट्वा कन्याशतमभाषत} %1-33-1

\twolineshloka
{वायुः सर्वात्मको राजन् प्रधर्षयितुमिच्छति}
{अशुभं मार्गमास्थाय न धर्मं प्रत्यवेक्षते} %1-33-2

\twolineshloka
{पितृमत्यः स्म भद्रं ते स्वच्छन्दे न वयं स्थिताः}
{पितरं नो वृणीष्व त्वं यदि नो दास्यते तव} %1-33-3

\twolineshloka
{तेन पापानुबन्धेन वचनं न प्रतीच्छता}
{एवं ब्रुवन्त्यः सर्वाः स्म वायुनाभिहता भृशम्} %1-33-4

\twolineshloka
{तासां तु वचनं श्रुत्वा राजा परमधार्मिकः}
{प्रत्युवाच महातेजाः कन्याशतमनुत्तमम्} %1-33-5

\twolineshloka
{क्षान्तं क्षमावतां पुत्र्यः कर्तव्यं सुमहत् कृतम्}
{ऐकमत्यमुपागम्य कुलं चावेक्षितं मम} %1-33-6

\twolineshloka
{अलङ्कारो हि नारीणां क्षमा तु पुरुषस्य वा}
{दुष्करं तच्च वै क्षान्तं त्रिदशेषु विशेषतः} %1-33-7

\twolineshloka
{यादृशी वः क्षमा पुत्र्यः सर्वासामविशेषतः}
{क्षमा दानं क्षमा सत्यं क्षमा यज्ञाश्च पुत्रिकाः} %1-33-8

\twolineshloka
{क्षमा यशः क्षमा धर्मः क्षमायां विष्ठितं जगत्}
{विसृज्य कन्याः काकुत्स्थ राजा त्रिदशविक्रमः} %1-33-9

\twolineshloka
{मन्त्रज्ञो मन्त्रयामास प्रदानं सह मन्त्रिभिः}
{देशे काले च कर्तव्यं सदृशे प्रतिपादनम्} %1-33-10

\twolineshloka
{एतस्मिन्नेव काले तु चूली नाम महाद्युतिः}
{ऊर्ध्वरेताः शुभाचारो ब्राह्मं तप उपागमत्} %1-33-11

\twolineshloka
{तपस्यन्तमृषिं तत्र गन्धर्वी पर्युपासते}
{सोमदा नाम भद्रं ते ऊर्मिलातनया तदा} %1-33-12

\twolineshloka
{सा च तं प्रणता भूत्वा शुश्रूषणपरायणा}
{उवास काले धर्मिष्ठा तस्यास्तुष्टोऽभवद् गुरुः} %1-33-13

\twolineshloka
{स च तां कालयोगेन प्रोवाच रघुनन्दन}
{परितुष्टोऽस्मि भद्रं ते किं करोमि तव प्रियम्} %1-33-14

\twolineshloka
{परितुष्टं मुनिं ज्ञात्वा गन्धर्वी मधुरस्वरम्}
{उवाच परमप्रीता वाक्यज्ञा वाक्यकोविदम्} %1-33-15

\twolineshloka
{लक्ष्म्या समुदितो ब्राह्म्या ब्रह्मभूतो महातपाः}
{ब्राह्मेण तपसा युक्तं पुत्रमिच्छामि धार्मिकम्} %1-33-16

\twolineshloka
{अपतिश्चास्मि भद्रं ते भार्या चास्मि न कस्यचित्}
{ब्राह्मेणोपगतायाश्च दातुमर्हसि मे सुतम्} %1-33-17

\twolineshloka
{तस्याः प्रसन्नो ब्रह्मर्षिर्ददौ ब्राह्ममनुत्तमम्}
{ब्रह्मदत्त इति ख्यातं मानसं चूलिनः सुतम्} %1-33-18

\twolineshloka
{स राजा ब्रह्मदत्तस्तु पुरीमध्यवसत् तदा}
{काम्पिल्यां परया लक्ष्म्या देवराजो यथा दिवम्} %1-33-19

\twolineshloka
{स बुद्धिं कृतवान् राजा कुशनाभः सुधार्मिकः}
{ब्रह्मदत्ताय काकुत्स्थ दातुं कन्याशतं तदा} %1-33-20

\twolineshloka
{तमाहूय महातेजा ब्रह्मदत्तं महीपतिः}
{ददौ कन्याशतं राजा सुप्रीतेनान्तरात्मना} %1-33-21

\twolineshloka
{यथाक्रमं तदा पाणिं जग्राह रघुनन्दन}
{ब्रह्मदत्तो महीपालस्तासां देवपतिर्यथा} %1-33-22

\twolineshloka
{स्पृष्टमात्रे तदा पाणौ विकुब्जा विगतज्वराः}
{युक्तं परमया लक्ष्म्या बभौ कन्याशतं तदा} %1-33-23

\twolineshloka
{स दृष्ट्वा वायुना मुक्ताः कुशनाभो महीपतिः}
{बभूव परमप्रीतो हर्षं लेभे पुनः पुनः} %1-33-24

\twolineshloka
{कृतोद्वाहं तु राजानं ब्रह्मदत्तं महीपतिम्}
{सदारं प्रेषयामास सोपाध्यायगणं तदा} %1-33-25

\threelineshloka
{सोमदापि सुतं दृष्ट्वा पुत्रस्य सदृशीं क्रियाम्}
{यथान्यायं च गन्धर्वी स्नुषास्ताः प्रत्यनन्दत}
{स्पृष्ट्वा स्पृष्ट्वा च ताः कन्याः कुशनाभं प्रशस्य च} %1-33-26


॥इत्यार्षे श्रीमद्रामायणे वाल्मीकीये आदिकाव्ये बालकाण्डे ब्रह्मदत्तविवाहः नाम त्रयस्त्रिंशः सर्गः ॥१-३३॥
