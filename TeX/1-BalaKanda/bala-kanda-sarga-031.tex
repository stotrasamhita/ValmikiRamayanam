\sect{एकत्रिंशः सर्गः — मिथिलाप्रस्थानम्}

\twolineshloka
{अथ तां रजनीं तत्र कृतार्थौ रामलक्ष्मणौ}
{ऊषतुर्मुदितौ वीरौ प्रहृष्टेनान्तरात्मना} %1-31-1

\twolineshloka
{प्रभातायां तु शर्वर्यां कृतपौर्वाह्णिकक्रियौ}
{विश्वामित्रमृषींश्चान्यान् सहितावभिजग्मतुः} %1-31-2

\twolineshloka
{अभिवाद्य मुनिश्रेष्ठं ज्वलन्तमिव पावकम्}
{ऊचतुर्परमोदारं वाक्यं मधुरभाषिणौ} %1-31-3

\twolineshloka
{इमौ स्म मुनिशार्दूल किङ्करौ समुपागतौ}
{आज्ञापय मुनिश्रेष्ठ शासनं करवाव किम्} %1-31-4

\twolineshloka
{एवमुक्ते तयोर्वाक्ये सर्व एव महर्षयः}
{विश्वामित्रं पुरस्कृत्य रामं वचनमब्रुवन्} %1-31-5

\twolineshloka
{मैथिलस्य नरश्रेष्ठ जनकस्य भविष्यति}
{यज्ञः परमधर्मिष्ठस्तत्र यास्यामहे वयम्} %1-31-6

\twolineshloka
{त्वं चैव नरशार्दूल सहास्माभिर्गमिष्यसि}
{अद्भुतं च धनूरत्नं तत्र त्वं द्रष्टुमर्हसि} %1-31-7

\twolineshloka
{तद्धि पूर्वं नरश्रेष्ठ दत्तं सदसि दैवतैः}
{अप्रमेयबलं घोरं मखे परमभास्वरम्} %1-31-8

\twolineshloka
{नास्य देवा न गन्धर्वा नासुरा न च राक्षसाः}
{कर्तुमारोपणं शक्ता न कथञ्चन मानुषाः} %1-31-9

\twolineshloka
{धनुषस्तस्य वीर्यं हि जिज्ञासन्तो महीक्षितः}
{न शेकुरारोपयितुं राजपुत्रा महाबलाः} %1-31-10

\twolineshloka
{तद्धनुर्नरशार्दूल मैथिलस्य महात्मनः}
{तत्र द्रक्ष्यसि काकुत्स्थ यज्ञं च परमाद्भुतम्} %1-31-11

\twolineshloka
{तद्धि यज्ञफलं तेन मैथिलेनोत्तमं धनुः}
{याचितं नरशार्दूल सुनाभं सर्वदैवतैः} %1-31-12

\threelineshloka
{आयागभूतं नृपतेस्तस्य वेश्मनि राघव}
{अर्चितं विविधैर्गन्धैर्धूपैश्चागुरुगन्धिभिः एवमुक्त्वा मुनिवरः प्रस्थानमकरोत् तदा}
{सर्षिसङ्घः सकाकुत्स्थ आमन्त्र्य वनदेवताः} %1-31-13

\twolineshloka
{स्वस्ति वोऽस्तु गमिष्यामि सिद्धः सिद्धाश्रमादहम्}
{उत्तरे जाह्नवीतीरे हिमवन्तं शिलोच्चयम्} %1-31-14

\twolineshloka
{इत्युक्त्वा मुनिशार्दूलः कौशिकः स तपोधनः}
{उत्तरां दिशमुद्दिश्य प्रस्थातुमुपचक्रमे} %1-31-15

\twolineshloka
{तं प्रयान्तं मुनिवरमन्वगादनुसारिणाम्}
{शकटीशतमात्रं तु प्रयाणे ब्रह्मवादिनाम्} %1-31-16

\twolineshloka
{मृगपक्षिगणाश्चैव सिद्धाश्रमनिवासिनः}
{अनुजग्मुर्महात्मानं विश्वामित्रं तपोधनम्} %1-31-17

\twolineshloka
{निवर्तयामास ततः सर्षि सन्घः स पक्षिणः}
{ते गत्वा दूरम् अध्वानम् लम्बमाने दिवाकरे} %1-31-18

\twolineshloka
{ते गत्वा दूरमध्वानं लम्बमाने दिवाकरे}
{वासं चक्रुर्मुनिगणाः शोणाकूले समाहिताः} %1-31-19

\twolineshloka
{तेऽस्तं गते दिनकरे स्नात्वा हुतहुताशनाः}
{विश्वामित्रं पुरस्कृत्य निषेदुरमितौजसः} %1-31-20

\twolineshloka
{रामोऽपि सहसौमित्रिर्मुनींस्तानभिपूज्य च}
{अग्रतो निषसादाथ विश्वामित्रस्य धीमतः} %1-31-21

\twolineshloka
{अथ रामो महातेजा विश्वामित्रं तपोधनम्}
{पप्रच्छ नरशार्दूलः कौतूहलसमन्वितः} %1-31-22

\twolineshloka
{भगवन् को न्वयं देशः समृद्धवनशोभितः}
{श्रोतुमिच्छामि भद्रं ते वक्तुमर्हसि तत्त्वतः} %1-31-23

\twolineshloka
{चोदितो रामवाक्येन कथयामास सुव्रतः}
{तस्य देशस्य निखिलमृषिमध्ये महातपाः} %1-31-24


॥इत्यार्षे श्रीमद्रामायणे वाल्मीकीये आदिकाव्ये बालकाण्डे मिथिलाप्रस्थानम् नाम एकत्रिंशः सर्गः ॥१-३१॥
