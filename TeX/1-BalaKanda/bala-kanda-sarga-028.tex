\sect{अष्टाविंशः सर्गः — अस्त्रसंहारग्रहणम्}

\twolineshloka
{प्रतिगृह्य ततोऽस्त्राणि प्रहृष्टवदनः शुचिः}
{गच्छन्नेव च काकुत्स्थो विश्वामित्रमथाब्रवीत्} %1-28-1

\twolineshloka
{गृहीतास्त्रोऽस्मि भगवन् दुराधर्षः सुरैरपि}
{अस्त्राणां त्वहमिच्छामि संहारान् मुनिपुङ्गव} %1-28-2

\twolineshloka
{एवं ब्रुवति काकुत्स्थे विश्वामित्रो महातपाः}
{संहारान् व्याजहाराथ धृतिमान् सुव्रतः शुचिः} %1-28-3

\twolineshloka
{सत्यवन्तं सत्यकीर्तिं धृष्टं रभसमेव च}
{प्रतिहारतरं नाम पराङ्मुखमवाङ्मुखम्} %1-28-4

\twolineshloka
{लक्ष्यालक्ष्याविमौ चैव दृढनाभसुनाभकौ}
{दशाक्षशतवक्त्रौ च दशशीर्षशतोदरौ} %1-28-5

\twolineshloka
{पद्मनाभमहानाभौ दुन्दुनाभस्वनाभकौ}
{ज्योतिषं शकुनं चैव नैराश्यविमलावुभौ} %1-28-6

\threelineshloka
{यौगन्धरविनिद्रौ च दैत्यप्रमथनौ तथा}
{शुचिबाहुर्महाबाहुर्निष्कलिर्विरुचस्तथा}
{सार्चिर्माली धृतिमाली वृत्तिमान् रुचिरस्तथा} %1-28-7

\twolineshloka
{पित्र्यः सौमनसश्चैव विधूतमकरावुभौ}
{परवीरं रतिं चैव धनधान्यौ च राघव} %1-28-8

\twolineshloka
{कामरूपं कामरुचिं मोहमावरणं तथा}
{जृम्भकं सर्पनाभं च पन्थानवरणौ तथा} %1-28-9

\twolineshloka
{कृशाश्वतनयान् राम भास्वरान् कामरूपिणः}
{प्रतीच्छ मम भद्रं ते पात्रभूतोऽसि राघव} %1-28-10

\twolineshloka
{बाढमित्येव काकुत्स्थः प्रहृष्टेनान्तरात्मना}
{दिव्यभास्वरदेहाश्च मूर्तिमन्तः सुखप्रदाः} %1-28-11

\twolineshloka
{केचिदङ्गारसदृशाः केचिद् धूमोपमास्तथा}
{चन्द्रार्कसदृशाः केचित् प्रह्वाञ्जलिपुटास्तथा} %1-28-12

\twolineshloka
{रामं प्राञ्जलयो भूत्वाऽब्रुवन् मधुरभाषिणः}
{इमे स्म नरशार्दूल शाधि किं करवाम ते} %1-28-13

\twolineshloka
{गम्यतामिति तानाह यथेष्टं रघुनन्दनः}
{मानसाः कार्यकालेषु साहाय्यं मे करिष्यथ} %1-28-14

\twolineshloka
{अथ ते राममामन्त्र्य कृत्वा चापि प्रदक्षिणम्}
{एवमस्त्विति काकुत्स्थमुक्त्वा जग्मुर्यथागतम्} %1-28-15

\twolineshloka
{स च तान् राघवो ज्ञात्वा विश्वामित्रं महामुनिम्}
{गच्छन्नेवाथ मधुरं श्लक्ष्णं वचनमब्रवीत्} %1-28-16

\twolineshloka
{किमेतन्मेघसङ्काशं पर्वतस्याविदूरतः}
{वृक्षखण्डमितो भाति परं कौतूहलं हि मे} %1-28-17

\twolineshloka
{दर्शनीयं मृगाकीर्णं मनोहरमतीव च}
{नानाप्रकारैः शकुनैर्वल्गुभाषैरलङ्कृतम्} %1-28-18

\twolineshloka
{निःसृताःस्मो मुनिश्रेष्ठ कान्ताराद् रोमहर्षणात्}
{अनया त्ववगच्छामि देशस्य सुखवत्तया} %1-28-19

\twolineshloka
{सर्वं मे शंस भगवन् कस्याश्रमपदं त्विदम्}
{सम्प्राप्ता यत्र ते पापा ब्रह्मघ्ना दुष्टचारिणः} %1-28-20

\twolineshloka
{तव यज्ञस्य विघ्नाय दुरात्मानो महामुने}
{भगवंस्तस्य को देशः सा यत्र तव याज्ञिकी} %1-28-21

\twolineshloka
{रक्षितव्या क्रिया ब्रह्मन् मया वध्याश्च राक्षसाः}
{एतत् सर्वं मुनिश्रेष्ठ श्रोतुमिच्छाम्यहं प्रभो} %1-28-22


॥इत्यार्षे श्रीमद्रामायणे वाल्मीकीये आदिकाव्ये बालकाण्डे अस्त्रसंहारग्रहणम् नाम अष्टाविंशः सर्गः ॥१-२८॥
