\sect{षड्चत्वारिंशः सर्गः — दितिगर्भभेदः}

\twolineshloka
{हतेषु तेषु पुत्रेषु दितिः परमदुःखिता}
{मारीचं कश्यपं नाम भर्तारमिदमब्रवीत्} %1-46-1

\twolineshloka
{हतपुत्रास्मि भगवंस्तव पुत्रौर्महाबलैः}
{शक्रहन्तारमिच्छामि पुत्रं दीर्घतपोर्जितम्} %1-46-2

\twolineshloka
{साहं तपश्चरिष्यामि गर्भं मे दातुमर्हसि}
{ईश्वरं शक्रहन्तारं त्वमनुज्ञातुमर्हसि} %1-46-3

\twolineshloka
{तस्यास्तद्वचनं श्रुत्वा मारीचः कश्यपस्तदा}
{प्रत्युवाच महातेजा दितिं परमदुःखिताम्} %1-46-4

\twolineshloka
{एवं भवतु भद्रं ते शुचिर्भव तपोधने}
{जनयिष्यसि पुत्रं त्वं शक्रहन्तारमाहवे} %1-46-5

\twolineshloka
{पूर्णे वर्षसहस्रे तु शुचिर्यदि भविष्यसि}
{पुत्रं त्रैलोक्यहन्तारं मत्तस्त्वं जनयिष्यसि} %1-46-6

\twolineshloka
{एवमुक्त्वा महातेजाः पाणिना सम्ममार्ज ताम्}
{तामालभ्य ततः स्वस्ति इत्युक्त्वा तपसे ययौ} %1-46-7

\twolineshloka
{गते तस्मिन् नरश्रेष्ठ दितिः परमहर्षिता}
{कुशप्लवं समासाद्य तपस्तेपे सुदारुणम्} %1-46-8

\twolineshloka
{तपस्तस्यां हि कुर्वत्यां परिचर्यां चकार ह}
{सहस्राक्षो नरश्रेष्ठ परया गुणसम्पदा} %1-46-9

\twolineshloka
{अग्निं कुशान् काष्ठमपः फलं मूलं तथैव च}
{न्यवेदयत् सहस्राक्षो यच्चान्यदपि काङ्क्षितम्} %1-46-10

\twolineshloka
{गात्रसंवाहनैश्चैव श्रमापनयनैस्तथा}
{शक्रः सर्वेषु कालेषु दितिं परिचचार ह} %1-46-11

\twolineshloka
{पूर्णे वर्षसहस्रे सा दशोने रघुनन्दन}
{दितिः परमसंहृष्टा सहस्राक्षमथाब्रवीत्} %1-46-12

\twolineshloka
{तपश्चरन्त्या वर्षाणि दश वीर्यवतां वर}
{अवशिष्टानि भद्रं ते भ्रातरं द्रक्ष्यसे ततः} %1-46-13

\twolineshloka
{यमहं त्वत्कृते पुत्र तमाधास्ये जयोत्सुकम्}
{त्रैलोक्यविजयं पुत्र सह भोक्ष्यसि विज्वर} %1-46-14

\twolineshloka
{याचितेन सुरश्रेष्ठ पित्रा तव महात्मना}
{वरो वर्षसहस्रान्ते मम दत्तः सुतं प्रति} %1-46-15

\twolineshloka
{इत्युक्त्वा च दितिस्तत्र प्राप्ते मध्यं दिनेश्वरे}
{निद्रयापहृता देवी पादौ कृत्वाथ शीर्षतः} %1-46-16

\twolineshloka
{दृष्ट्वा तामशुचिं शक्रः पादयोः कृतमूर्धजाम्}
{शिरःस्थाने कृतौ पादौ जहास च मुमोद च} %1-46-17

\twolineshloka
{तस्याः शरीरविवरं प्रविवेश पुरन्दरः}
{गर्भं च सप्तधा राम चिच्छेद परमात्मवान्} %1-46-18

\twolineshloka
{भिद्यमानस्ततो गर्भो वज्रेण शतपर्वणा}
{रुरोद सुस्वरं राम ततो दितिरबुध्यत} %1-46-19

\twolineshloka
{मा रुदो मा रुदश्चेति गर्भं शक्रोऽभ्यभाषत}
{बिभेद च महातेजा रुदन्तमपि वासवः} %1-46-20

\twolineshloka
{न हन्तव्यं न हन्तव्यमित्येव दितिरब्रवीत्}
{निष्पपात ततः शक्रो मातुर्वचनगौरवात्} %1-46-21

\twolineshloka
{प्राञ्जलिर्वज्रसहितो दितिं शक्रोऽभ्यभाषत}
{अशुचिर्देवि सुप्तासि पादयोः कृतमूर्धजा} %1-46-22

\twolineshloka
{तदन्तरमहं लब्ध्वा शक्रहन्तारमाहवे}
{अभिन्दं सप्तधा देवि तन्मे त्वं क्षन्तुमर्हसि} %1-46-23


॥इत्यार्षे श्रीमद्रामायणे वाल्मीकीये आदिकाव्ये बालकाण्डे दितिगर्भभेदः नाम षड्चत्वारिंशः सर्गः ॥१-४६॥
