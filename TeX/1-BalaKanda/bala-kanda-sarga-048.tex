\sect{अष्टचत्वारिंशः सर्गः — शक्राहल्याशापः}

\twolineshloka
{पृष्ट्वा तु कुशलं तत्र परस्परसमागमे}
{कथान्ते सुमतिर्वाक्यं व्याजहार महामुनिम्} %1-48-1

\twolineshloka
{इमौ कुमारौ भद्रं ते देवतुल्यपराक्रमौ}
{गजसिंहगती वीरौ शार्दूलवृषभोपमौ} %1-48-2

\twolineshloka
{पद्मपत्रविशालाक्षौ खड्गतूणिधनुर्धरौ}
{अश्विनाविव रूपेण समुपस्थितयौवनौ} %1-48-3

\twolineshloka
{यदृच्छयैव गां प्राप्तौ देवलोकादिवामरौ}
{कथं पद्भ्यामिह प्राप्तौ किमर्थं कस्य वा मुने} %1-48-4

\twolineshloka
{भूषयन्ताविमं देशं चन्द्रसूर्याविवाम्बरम्}
{परस्परेण सदृशौ प्रमाणेङ्गितचेष्टितैः} %1-48-5

\twolineshloka
{किमर्थं च नरश्रेष्ठौ सम्प्राप्तौ दुर्गमे पथि}
{वरायुधधरौ वीरौ श्रोतुमिच्छामि तत्त्वतः} %1-48-6

\threelineshloka
{तस्य तद् वचनं श्रुत्वा यथावृत्तं न्यवेदयत्}
{सिद्धाश्रमनिवासं च राक्षसानां वधं यथा}
{विश्वामित्रवचः श्रुत्वा राजा परमविस्मितः} %1-48-7

\twolineshloka
{अतिथी परमं प्राप्तौ पुत्रौ दशरथस्य तौ}
{पूजयामास विधिवत् सत्कारार्हौ महाबलौ} %1-48-8

\twolineshloka
{ततः परमसत्कारं सुमतेः प्राप्य राघवौ}
{उष्य तत्र निशामेकां जग्मतुर्मिथिलां ततः} %1-48-9

\twolineshloka
{तां दृष्ट्वा मुनयः सर्वे जनकस्य पुरीं शुभाम्}
{साधु साध्विति शंसन्तो मिथिलां समपूजयन्} %1-48-10

\twolineshloka
{मिथिलोपवने तत्र आश्रमं दृश्य राघवः}
{पुराणं निर्जनं रम्यं पप्रच्छ मुनिपुङ्गवम्} %1-48-11

\twolineshloka
{इदमाश्रमसंकाशं किं न्विदं मुनिवर्जितम्}
{श्रोतुमिच्छामि भगवन् कस्यायं पूर्व आश्रमः} %1-48-12

\twolineshloka
{तच्छ्रुत्वा राघवेणोक्तं वाक्यं वाक्यविशारदः}
{प्रत्युवाच महातेजा विश्वामित्रो महामुनिः} %1-48-13

\twolineshloka
{हन्त ते कथयिष्यामि शृणु तत्त्वेन राघव}
{यस्यैतदाश्रमपदं शप्तं कोपान्महात्मनः} %1-48-14

\twolineshloka
{गौतमस्य नरश्रेष्ठ पूर्वमासीन्महात्मनः}
{आश्रमो दिव्यसंकाशः सुरैरपि सुपूजितः} %1-48-15

\twolineshloka
{स चात्र तप आतिष्ठदहल्यासहितः पुरा}
{वर्षपूगान्यनेकानि राजपुत्र महायशः} %1-48-16

\twolineshloka
{तस्यान्तरं विदित्वा च सहस्राक्षः शचीपतिः}
{मुनिवेषधरो भूत्वा अहल्यामिदमब्रवीत्} %1-48-17

\twolineshloka
{ऋतुकालं प्रतीक्षन्ते नार्थिनः सुसमाहिते}
{संगमं त्वहमिच्छामि त्वया सह सुमध्यमे} %1-48-18

\twolineshloka
{मुनिवेषं सहस्राक्षं विज्ञाय रघुनन्दन}
{मतिं चकार दुर्मेधा देवराजकुतूहलात्} %1-48-19

\twolineshloka
{अथाब्रवीत् सुरश्रेष्ठं कृतार्थेनान्तरात्मना}
{कृतार्थास्मि सुरश्रेष्ठ गच्छ शीघ्रमितः प्रभो} %1-48-20

\twolineshloka
{आत्मानं मां च देवेश सर्वथा रक्ष गौतमात्}
{इन्द्रस्तु प्रहसन् वाक्यमहल्यामिदमब्रवीत्} %1-48-21

\twolineshloka
{सुश्रोणि परितुष्टोऽस्मि गमिष्यामि यथागतम्}
{एवं संगम्य तु तदा निश्चक्रामोटजात् ततः} %1-48-22

\twolineshloka
{स सम्भ्रमात् त्वरन् राम शङ्कितो गौतमं प्रति}
{गौतमं स ददर्शाथ प्रविशन्तं महामुनिम्} %1-48-23

\twolineshloka
{देवदानवदुर्धर्षं तपोबलसमन्वितम्}
{तीर्थोदकपरिक्लिन्नं दीप्यमानमिवानलम्} %1-48-24

\twolineshloka
{गृहीतसमिधं तत्र सकुशं मुनिपुंगवम्}
{दृष्ट्वा सुरपतिस्त्रस्तो विषण्णवदनोऽभवत्} %1-48-25

\twolineshloka
{अथ दृष्ट्वा सहस्राक्षं मुनिवेषधरं मुनिः}
{दुर्वृत्तं वृत्तसम्पन्नो रोषाद् वचनमब्रवीत्} %1-48-26

\twolineshloka
{मम रूपं समास्थाय कृतवानसि दुर्मते}
{अकर्तव्यमिदं यस्माद् विफलस्त्वं भविष्यसि} %1-48-27

\twolineshloka
{गौतमेनैवमुक्तस्य सुरोषेण महात्मना}
{पेततुर्वृषणौ भूमौ सहस्राक्षस्य तत्क्षणात्} %1-48-28

\twolineshloka
{तथा शप्त्वा च वै शक्रं भार्यामपि च शप्तवान्}
{इह वर्षसहस्राणि बहूनि निवसिष्यसि} %1-48-29

\twolineshloka
{वातभक्षा निराहारा तप्यन्ती भस्मशायिनी}
{अदृश्या सर्वभूतानामाश्रमेऽस्मिन् वसिष्यसि} %1-48-30

\twolineshloka
{यदा त्वेतद् वनं घोरं रामो दशरथात्मजः}
{आगमिष्यति दुर्धर्षस्तदा पूता भविष्यसि} %1-48-31

\twolineshloka
{तस्यातिथ्येन दुर्वृत्ते लोभमोहविवर्जिता}
{मत्सकाशं मुदा युक्ता स्वं वपुर्धारयिष्यसि} %1-48-32

\threelineshloka
{एवमुक्त्वा महातेजा गौतमो दुष्टचारिणीम्}
{इममाश्रममुत्सृज्य सिद्धचारणसेविते}
{हिमवच्छिखरे रम्ये तपस्तेपे महातपाः} %1-48-33


॥इत्यार्षे श्रीमद्रामायणे वाल्मीकीये आदिकाव्ये बालकाण्डे शक्राहल्याशापः नाम अष्टचत्वारिंशः सर्गः ॥१-४८॥
