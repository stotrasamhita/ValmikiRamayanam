\sect{चतुःसप्ततितमः सर्गः — जामदग्न्याभियोगः}

\twolineshloka
{अथ रात्र्यां व्यतीतायां विश्वामित्रो महामुनिः}
{आपृष्ट्वा तौ च राजानौ जगामोत्तरपर्वतम्} %1-74-1

\twolineshloka
{विश्वामित्रे गते राजा वैदेहं मिथिलाधिपम्}
{आपृष्ट्वैव जगामाशु राजा दशरथः पुरीम्} %1-74-2

\twolineshloka
{अथ राजा विदेहानां ददौ कन्याधनं बहु}
{गवां शतसहस्राणि बहूनि मिथिलेश्वरः} %1-74-3

\twolineshloka
{कम्बलानां च मुख्यानां क्षौमान् कोट्यम्बराणि च}
{हस्त्यश्वरथपादातं दिव्यरूपं स्वलंकृतम्} %1-74-4

\twolineshloka
{ददौ कन्याशतं तासां दासीदासमनुत्तमम्}
{हिरण्यस्य सुवर्णस्य मुक्तानां विद्रुमस्य च} %1-74-5

\twolineshloka
{ददौ राजा सुसंहृष्टः कन्याधनमनुत्तमम्}
{दत्त्वा बहुविधं राजा समनुज्ञाप्य पार्थिवम्} %1-74-6

\twolineshloka
{प्रविवेश स्वनिलयं मिथिलां मिथिलेश्वरः}
{राजाप्ययोध्याधिपतिः सह पुत्रैर्महात्मभिः} %1-74-7

\twolineshloka
{ऋषीन् सर्वान् पुरस्कृत्य जगाम सबलानुगः}
{गच्छन्तं तु नरव्याघ्रं सर्षिसङ्घं सराघवम्} %1-74-8

\twolineshloka
{घोरास्तु पक्षिणो वाचो व्याहरन्ति समन्ततः}
{भौमाश्चैव मृगाः सर्वे गच्छन्ति स्म प्रदक्षिणम्} %1-74-9

\twolineshloka
{तान् दृष्ट्वा राजशार्दूलो वसिष्ठं पर्यपृच्छत}
{असौम्याः पक्षिणो घोरा मृगाश्चापि प्रदक्षिणाः} %1-74-10

\twolineshloka
{किमिदं हृदयोत्कम्पि मनो मम विषीदति}
{राज्ञो दशरथस्यैतच्छ्रुत्वा वाक्यं महानृषिः} %1-74-11

\twolineshloka
{उवाच मधुरां वाणीं श्रूयतामस्य यत् फलम्}
{उपस्थितं भयं घोरं दिव्यं पक्षिमुखाच्च्युतम्} %1-74-12

\twolineshloka
{मृगाः प्रशमयन्त्येते संतापस्त्यज्यतामयम्}
{तेषां संवदतां तत्र वायुः प्रादुर्बभूव ह} %1-74-13

\twolineshloka
{कम्पयन् मेदिनीं सर्वां पातयंश्च महाद्रुमान्}
{तमसा संवृतः सूर्यः सर्वे नावेदिषुर्दिशः} %1-74-14

\twolineshloka
{भस्मना चावृतं सर्वं सम्मूढमिव तद्बलम्}
{वसिष्ठ ऋषयश्चान्ये राजा च ससुतस्तदा} %1-74-15

\twolineshloka
{ससंज्ञा इव तत्रासन् सर्वमन्यद्विचेतनम्}
{तस्मिंस्तमसि घोरे तु भस्मच्छन्नेव सा चमूः} %1-74-16

\twolineshloka
{ददर्श भीमसंकाशं जटामण्डलधारिणम्}
{भार्गवं जामदग्न्येयं राजा राजविमर्दनम्} %1-74-17

\twolineshloka
{कैलासमिव दुर्धर्षं कालाग्निमिव दुःसहम्}
{ज्वलन्तमिव तेजोभिर्दुर्निरीक्ष्यं पृथग्जनैः} %1-74-18

\twolineshloka
{स्कन्धे चासज्ज्य परशुं धनुर्विद्युद्गणोपमम्}
{प्रगृह्य शरमुग्रं च त्रिपुरघ्नं यथा शिवम्} %1-74-19

\twolineshloka
{तं दृष्ट्वा भीमसंकाशं ज्वलन्तमिव पावकम्}
{वसिष्ठप्रमुखा विप्रा जपहोमपरायणाः} %1-74-20

\twolineshloka
{संगता मुनयः सर्वे संजजल्पुरथो मिथः}
{कच्चित् पितृवधामर्षी क्षत्रं नोत्सादयिष्यति} %1-74-21

\twolineshloka
{पूर्वं क्षत्रवधं कृत्वा गतमन्युर्गतज्वरः}
{क्षत्रस्योत्सादनं भूयो न खल्वस्य चिकीर्षितम्} %1-74-22

\twolineshloka
{एवमुक्त्वार्घ्यमादाय भार्गवं भीमदर्शनम्}
{ऋषयो राम रामेति मधुरं वाक्यमब्रुवन्} %1-74-23

\twolineshloka
{प्रतिगृह्य तु तां पूजामृषिदत्तां प्रतापवान्}
{रामं दाशरथिं रामो जामदग्न्योऽभ्यभाषत} %1-74-24


॥इत्यार्षे श्रीमद्रामायणे वाल्मीकीये आदिकाव्ये बालकाण्डे जामदग्न्याभियोगः नाम चतुःसप्ततितमः सर्गः ॥१-७४॥
