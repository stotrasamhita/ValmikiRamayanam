\sect{सप्तत्रिंशः सर्गः — कुमारोत्पत्तिः}

\twolineshloka
{तप्यमाने तदा देवे सेन्द्राः साग्निपुरोगमाः}
{सेनापतिमभीप्सन्तः पितामहमुपागमन्} %1-37-1

\twolineshloka
{ततोऽब्रुवन् सुराः सर्वे भगवन्तं पितामहम्}
{प्रणिपत्य सुराराम सेन्द्राः साग्निपुरोगमाः} %1-37-2

\twolineshloka
{येन सेनापतिर्देव दत्तो भगवता पुरा}
{स तपः परमास्थाय तप्यते स्म सहोमया} %1-37-3

\twolineshloka
{यदत्रानन्तरं कार्यं लोकानां हितकाम्यया}
{संविधत्स्व विधानज्ञ त्वं हि नः परमा गतिः} %1-37-4

\twolineshloka
{देवतानां वचः श्रुत्वा सर्वलोकपितामहः}
{सान्त्वयन् मधुरैर्वाक्यैस्त्रिदशानिदमब्रवीत्} %1-37-5

\twolineshloka
{शैलपुत्र्या यदुक्तं तन्न प्रजाः स्वासु पत्निषु}
{तस्या वचनमक्लिष्टं सत्यमेव न संशयः} %1-37-6

\twolineshloka
{इयमाकाशगङ्गा च यस्यां पुत्रं हुताशनः}
{जनयिष्यति देवानां सेनापतिमरिंदमम्} %1-37-7

\twolineshloka
{ज्येष्ठा शैलेन्द्रदुहिता मानयिष्यति तं सुतम्}
{उमायास्तद्बहुमतं भविष्यति न संशयः} %1-37-8

\twolineshloka
{तच्छ्रुत्वा वचनं तस्य कृतार्था रघुनन्दन}
{प्रणिपत्य सुराः सर्वे पितामहमपूजयन्} %1-37-9

\twolineshloka
{ते गत्वा परमं राम कैलासं धातुमण्डितम्}
{अग्निं नियोजयामासुः पुत्रार्थं सर्वदेवताः} %1-37-10

\twolineshloka
{देवकार्यमिदं देव समाधत्स्व हुताशन}
{शैलपुत्र्यां महातेजो गङ्गायां तेज उत्सृज} %1-37-11

\twolineshloka
{देवतानां प्रतिज्ञाय गङ्गामभ्येत्य पावकः}
{गर्भं धारय वै देवि देवतानामिदं प्रियम्} %1-37-12

\twolineshloka
{इत्येतद् वचनं श्रुत्वा दिव्यं रूपमधारयत्}
{स तस्या महिमां दृष्ट्वा समन्तादवशीर्यत} %1-37-13

\twolineshloka
{समन्ततस्तदा देवीमभ्यषिञ्चत पावकः}
{सर्वस्रोतांसि पूर्णानि गङ्गाया रघुनन्दन} %1-37-14

\twolineshloka
{तमुवाच ततो गङ्गा सर्वदेवपुरोगमम्}
{अशक्ता धारणे देव तेजस्तव समुद्धतम्} %1-37-15

\twolineshloka
{दह्यमानाग्निना तेन सम्प्रव्यथितचेतना}
{अथाब्रवीदिदं गङ्गां सर्वदेवहुताशनः} %1-37-16

\twolineshloka
{इह हैमवते पार्श्वे गर्भोऽयं संनिवेश्यताम्}
{श्रुत्वा त्वग्निवचो गङ्गा तं गर्भमतिभास्वरम्} %1-37-17

\twolineshloka
{उत्ससर्ज महातेजाः स्रोतोभ्यो हि तदानघ}
{यदस्या निर्गतं तस्मात् तप्तजाम्बूनदप्रभम्} %1-37-18

\twolineshloka
{काञ्चनं धरणीं प्राप्तं हिरण्यमतुलप्रभम्}
{ताम्रं कार्ष्णायसं चैव तैक्ष्ण्यादेवाभिजायत} %1-37-19

\twolineshloka
{मलं तस्याभवत् तत्र त्रपु सीसकमेव च}
{तदेतद्धरणीं प्राप्य नानाधातुरवर्धत} %1-37-20

\twolineshloka
{निक्षिप्तमात्रे गर्भे तु तेजोभिरभिरञ्जितम्}
{सर्वं पर्वतसंनद्धं सौवर्णमभवद् वनम्} %1-37-21

\threelineshloka
{जातरूपमिति ख्यातं तदाप्रभृति राघव}
{सुवर्णं पुरुषव्याघ्र हुताशनसमप्रभम्}
{तृणवृक्षलतागुल्मं सर्वं भवति काञ्चनम्} %1-37-22

\twolineshloka
{तं कुमारं ततो जातं सेन्द्राः सह मरुद्गणाः}
{क्षीरसम्भावनार्थाय कृत्तिकाः समयोजयन्} %1-37-23

\twolineshloka
{ताः क्षीरं जातमात्रस्य कृत्वा समयमुत्तमम्}
{ददुः पुत्रोऽयमस्माकं सर्वासामिति निश्चिताः} %1-37-24

\twolineshloka
{ततस्तु देवताः सर्वाः कार्तिकेय इति ब्रुवन्}
{पुत्रस्त्रैलोक्यविख्यातो भविष्यति न संशयः} %1-37-25

\twolineshloka
{तेषां तद् वचनं श्रुत्वा स्कन्नं गर्भपरिस्रवे}
{स्नापयन् परया लक्ष्म्या दीप्यमानं यथानलम्} %1-37-26

\twolineshloka
{स्कन्द इत्यब्रुवन् देवाः स्कन्नं गर्भपरिस्रवे}
{कार्तिकेयं महाबाहुं काकुत्स्थ ज्वलनोपमम्} %1-37-27

\twolineshloka
{प्रादुर्भूतं ततः क्षीरं कृत्तिकानामनुत्तमम्}
{षण्णां षडाननो भूत्वा जग्राह स्तनजं पयः} %1-37-28

\twolineshloka
{गृहीत्वा क्षीरमेकाह्ना सुकुमारवपुस्तदा}
{अजयत् स्वेन वीर्येण दैत्यसैन्यगणान् विभुः} %1-37-29

\twolineshloka
{सुरसेनागणपतिमभ्यषिञ्चन्महाद्युतिम्}
{ततस्तममराः सर्वे समेत्याग्निपुरोगमाः} %1-37-30

\twolineshloka
{एष ते राम गङ्गाया विस्तरोऽभिहितो मया}
{कुमारसम्भवश्चैव धन्यः पुण्यस्तथैव च} %1-37-31

\twolineshloka
{भक्तश्च यः कार्तिकेये काकुत्स्थ भुवि मानवः}
{आयुष्मान् पुत्रपौत्रैश्च स्कन्दसालोक्यतां व्रजेत्} %1-37-32


॥इत्यार्षे श्रीमद्रामायणे वाल्मीकीये आदिकाव्ये बालकाण्डे कुमारोत्पत्तिः नाम सप्तत्रिंशः सर्गः ॥१-३७॥
