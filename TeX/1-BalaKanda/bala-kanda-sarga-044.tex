\sect{चतुश्चत्वारिंशः सर्गः — सागरोद्धारः}

\twolineshloka
{स गत्वा सागरं राजा गङ्गायानुगतस्तदा}
{प्रविवेश तलं भूमेर्यत्र ते भस्मसात्कृताः} %1-44-1

\twolineshloka
{भस्मन्यथाप्लुते राम गङ्गायाः सलिलेन वै}
{सर्वलोकप्रभुर्ब्रह्मा राजानमिदमब्रवीत्} %1-44-2

\twolineshloka
{तारिता नरशार्दूल दिवं याताश्च च देववत्}
{षष्टिः पुत्रसहस्राणि सगरस्य महात्मनः} %1-44-3

\twolineshloka
{सागरस्य जलं लोके यावत्स्थास्यति पार्थिव}
{सगरस्यात्मजाः सर्वे दिवि स्थास्यन्ति देववत्} %1-44-4

\twolineshloka
{इयं च दुहिता ज्येष्ठा तव गङ्गा भविष्यति}
{त्वत्कृतेन च नाम्नाथ लोके स्थास्यति विश्रुता} %1-44-5

\twolineshloka
{गङ्गा त्रिपथगा नाम दिव्या भागीरथीति च}
{त्रीन् पथो भावयन्तीति तस्मात् त्रिपथगा स्मृता} %1-44-6

\twolineshloka
{पितामहानां सर्वेषां त्वमत्र मनुजाधिप}
{कुरुष्व सलिलं राजन् प्रतिज्ञामपवर्जय} %1-44-7

\twolineshloka
{पूर्वकेण हि ते राजंस्तेनातियशसा तदा}
{धर्मिणां प्रवरेणाथ नैष प्राप्तो मनोरथः} %1-44-8

\twolineshloka
{तथैवांशुमता तात लोकेऽप्रतिमतेजसा}
{गङ्गां प्रार्थयता नेतुं प्रतिज्ञा नापवर्जिता} %1-44-9

\twolineshloka
{राजर्षिणा गुणवता महर्षिसमतेजसा}
{मत्तुल्यतपसा चैव क्षत्रधर्मस्थितेन च} %1-44-10

\twolineshloka
{दिलीपेन महाभाग तव पित्रातितेजसा}
{पुनर्न शकिता नेतुं गङ्गां प्रार्थयतानघ} %1-44-11

\twolineshloka
{सा त्वया समतिक्रान्ता प्रतिज्ञा पुरुषर्षभ}
{प्राप्तोऽसि परमं लोके यशः परमसम्मतम्} %1-44-12

\twolineshloka
{तच्च गङ्गावतरणं त्वया कृतमरिंदम}
{अनेन च भवान् प्राप्तो धर्मस्यायतनं महत्} %1-44-13

\twolineshloka
{प्लावयस्व त्वमात्मानं नरोत्तम सदोचिते}
{सलिले पुरुषश्रेष्ठ शुचिः पुण्यफलो भव} %1-44-14

\twolineshloka
{पितामहानां सर्वेषां कुरुष्व सलिलक्रियाम्}
{स्वस्ति तेऽस्तु गमिष्यामि स्वं लोकं गम्यतां नृप} %1-44-15

\twolineshloka
{इत्येवमुक्त्वा देवेशः सर्वलोकपितामहः}
{यथागतं तथागच्चद् देवलोकं महायशाः} %1-44-16

\twolineshloka
{भगीरथस्तु राजर्षिः कृत्वा सलिलमुत्तमम्}
{यथाक्रमं यथान्यायं सागराणां महायशाः} %1-44-17

\twolineshloka
{कृतोदकः शुची राजा स्वपुरं प्रविवेश ह}
{समृद्धार्थो नरश्रेष्ठ स्वराज्यं प्रशशास ह} %1-44-18

\twolineshloka
{प्रमुमोद च लोकस्तं नृपमासाद्य राघव}
{नष्टशोकः समृद्धार्थो बभूव विगतज्वरः} %1-44-19

\twolineshloka
{एष ते राम गंगाया विस्तरोऽभिहितो मया}
{स्वस्ति प्राप्नुहि भद्रं ते संध्याकालोऽतिवर्तते} %1-44-20

\twolineshloka
{धन्यं यशस्यमायुष्यं पुत्र्यं स्वर्ग्यमथापि च}
{यः श्रावयति विप्रेषु क्षत्रियेष्वितरेषु च} %1-44-21

\twolineshloka
{प्रीयन्ते पितरस्तस्य प्रीयन्ते दैवतानि च}
{इदमाख्यानमायुष्यं गंगावतरणं शुभम्} %1-44-22

\twolineshloka
{यः शृणोति च काकुत्स्थ सर्वान् कामानवाप्नुयात्}
{सर्वे पापाः प्रणश्यन्ति आयुः कीर्तिश्च वर्धते} %1-44-23


॥इत्यार्षे श्रीमद्रामायणे वाल्मीकीये आदिकाव्ये बालकाण्डे सागरोद्धारः नाम चतुश्चत्वारिंशः सर्गः ॥१-४४॥
