\sect{सप्तपञ्चाशः सर्गः — त्रिशङ्कुयाजनप्रार्थना}

\twolineshloka
{ततः सन्तप्तहृदयः स्मरन्निग्रहमात्मनः}
{विनिःश्वस्य विनिःश्वस्य कृतवैरो महात्मना} %1-57-1

\twolineshloka
{स दक्षिणां दिशं गत्वा महिष्या सह राघव}
{तताप परमं घोरं विश्वामित्रो महातपाः} %1-57-2

\twolineshloka
{फलमूलाशनो दान्तश्चचार परमं तपः}
{अथास्य जज्ञिरे पुत्राः सत्यधर्मपरायणाः} %1-57-3

\twolineshloka
{हविष्पन्दो मधुष्पन्दो दृढनेत्रो महारथः}
{पूर्णे वर्षसहस्रे तु ब्रह्मा लोकपितामहः} %1-57-4

\twolineshloka
{अब्रवीन्मधुरं वाक्यं विश्वामित्रं तपोधनम्}
{जिता राजर्षिलोकास्ते तपसा कुशिकात्मज} %1-57-5

\twolineshloka
{अनेन तपसा त्वां हि राजर्षिरिति विद्महे}
{एवमुक्त्वा महातेजा जगाम सह दैवतैः} %1-57-6

\twolineshloka
{त्रिविष्टपं ब्रह्मलोकं लोकानां परमेश्वरः}
{विश्वामित्रोऽपि तच्छ्रुत्वा ह्रिया किञ्चिदवाङ्मुखः} %1-57-7

\twolineshloka
{दुःखेन महताविष्टः समन्युरिदमब्रवीत्}
{तपश्च सुमहत् तप्तं राजर्षिरिति मां विदुः} %1-57-8

\twolineshloka
{देवाः सर्षिगणाः सर्वे नास्ति मन्ये तपः फलम्}
{एवं निश्चित्य मनसा भूय एव महातपाः} %1-57-9

\twolineshloka
{तपश्चचार धर्मात्मा काकुत्स्थ परमात्मवान्}
{एतस्मिन्नेव काले तु सत्यवादी जितेन्द्रियः} %1-57-10

\twolineshloka
{त्रिशङ्कुरिति विख्यात इक्ष्वाकुकुलवर्धनः}
{तस्य बुद्धिः समुत्पन्ना यजेयमिति राघव} %1-57-11

\twolineshloka
{गच्छेयं स्वशरीरेण देवतानां परां गतिम्}
{वसिष्ठं स समाहूय कथयामास चिन्तितम्} %1-57-12

\twolineshloka
{अशक्यमिति चाप्युक्तो वसिष्ठेन महात्मना}
{प्रत्याख्यातो वसिष्ठेन स ययौ दक्षिणां दिशम्} %1-57-13

\twolineshloka
{ततस्तत्कर्मसिद्ध्यर्थं पुत्रांस्तस्य गतो नृपः}
{वासिष्ठा दीर्घतपसस्तपो यत्र हि तेपिरे} %1-57-14

\twolineshloka
{त्रिशङ्कुस्तु महातेजाः शतं परमभास्वरम्}
{वसिष्ठपुत्रान् ददृशे तप्यमानान् मनस्विनः} %1-57-15

\twolineshloka
{सोऽभिगम्य महात्मानः सर्वानेव गुरोः सुतान्}
{अभिवाद्यानुपूर्वेण ह्रिया किञ्चिदवाङ्मुखः} %1-57-16

\twolineshloka
{अब्रवीत् स महात्मानः सर्वानेव कृताञ्जलिः}
{शरणं वः प्रपन्नोऽहं शरण्यान् शरणं गतः} %1-57-17

\twolineshloka
{प्रत्याख्यातो हि भद्रं वो वसिष्ठेन महात्मना}
{यष्टुकामो महायज्ञं तदनुज्ञातुमर्हथ} %1-57-18

\twolineshloka
{गुरुपुत्रानहं सर्वान् नमस्कृत्य प्रसादये}
{शिरसा प्रणतो याचे ब्राह्मणांस्तपसि स्थितान्} %1-57-19

\twolineshloka
{ते मां भवन्तः सिद्ध्यर्थं याजयन्तु समाहिताः}
{सशरीरो यथाहं वै देवलोकमवाप्नुयाम्} %1-57-20

\twolineshloka
{प्रत्याख्यातो वसिष्ठेन गतिमन्यां तपोधनाः}
{गुरुपुत्रानृते सर्वान् नाहं पश्यामि काञ्चन} %1-57-21

\twolineshloka
{इक्ष्वाकूणां हि सर्वेषां पुरोधाः परमा गतिः}
{तस्मादनन्तरं सर्वे भवन्तो दैवतं मम} %1-57-22


॥इत्यार्षे श्रीमद्रामायणे वाल्मीकीये आदिकाव्ये बालकाण्डे त्रिशङ्कुयाजनप्रार्थना नाम सप्तपञ्चाशः सर्गः ॥१-५७॥
