\sect{पञ्चाशः सर्गः — जनकसमागमः}

\twolineshloka
{ततः प्रागुत्तरां गत्वा रामः सौमित्रिणा सह}
{विश्वामित्रं पुरस्कृत्य यज्ञवाटमुपागमत्} %1-50-1

\twolineshloka
{रामस्तु मुनिशार्दूलमुवाच सहलक्ष्मणः}
{साध्वी यज्ञसमृद्धिर्हि जनकस्य महात्मनः} %1-50-2

\twolineshloka
{बहूनीह सहस्राणि नानादेशनिवासिनाम्}
{ब्राह्मणानां महाभाग वेदाध्ययनशालिनाम्} %1-50-3

\twolineshloka
{ऋषिवाटाश्च दृश्यन्ते शकटीशतसंकुलाः}
{देशो विधीयतां ब्रह्मन् यत्र वत्स्यामहे वयम्} %1-50-4

\twolineshloka
{रामस्य वचनं श्रुत्वा विश्वामित्रो महामुनिः}
{निवासमकरोद् देशे विविक्ते सलिलान्विते} %1-50-5

\twolineshloka
{विश्वामित्रमनुप्राप्तं श्रुत्वा नृपवरस्तदा}
{शतानन्दं पुरस्कृत्य पुरोहितमनिन्दितः} %1-50-6

\twolineshloka
{ऋत्विजोऽपि महात्मानस्त्वर्घ्यमादाय सत्वरम्}
{प्रत्युज्जगाम सहसा विनयेन समन्वितः} %1-50-7

\twolineshloka
{विश्वामित्राय धर्मेण ददौ धर्मपुरस्कृतम्}
{प्रतिगृह्य तु तां पूजां जनकस्य महात्मनः} %1-50-8

\twolineshloka
{पप्रच्छ कुशलं राज्ञो यज्ञस्य च निरामयम्}
{स तांश्चाथ मुनीन् पृष्ट्वा सोपाध्यायपुरोधसः} %1-50-9

\twolineshloka
{यथार्हमृषिभिः सर्वैः समागच्छत् प्रहृष्टवत्}
{अथ राजा मुनिश्रेष्ठं कृताञ्जलिरभाषत} %1-50-10

\twolineshloka
{आसने भगवानास्तां सहैभिर्मुनिपुंगवैः}
{जनकस्य वचः श्रुत्वा निषसाद महामुनिः} %1-50-11

\twolineshloka
{पुरोधा ऋत्विजश्चैव राजा च सहमन्त्रिभिः}
{आसनेषु यथान्यायमुपविष्टाः समन्ततः} %1-50-12

\twolineshloka
{दृष्ट्वा स नृपतिस्तत्र विश्वामित्रमथाब्रवीत्}
{अद्य यज्ञसमृद्धिर्मे सफला दैवतैः कृता} %1-50-13

\twolineshloka
{अद्य यज्ञफलं प्राप्तं भगवद्दर्शनान्मया}
{धन्योऽस्म्यनुगृहीतोऽस्मि यस्य मे मुनिपुंगवः} %1-50-14

\twolineshloka
{यज्ञोपसदनं ब्रह्मन् प्राप्तोऽसि मुनिभिः सह}
{द्वादशाहं तु ब्रह्मर्षे दीक्षामाहुर्मनीषिणः} %1-50-15

\twolineshloka
{ततो भागार्थिनो देवान् द्रष्टुमर्हसि कौशिक}
{इत्युक्त्वा मुनिशार्दूलं प्रहृष्टवदनस्तदा} %1-50-16

\twolineshloka
{पुनस्तं परिपप्रच्छ प्राञ्जलिः प्रयतो नृपः}
{इमौ कुमारौ भद्रं ते देवतुल्यपराक्रमौ} %1-50-17

\threelineshloka
{गजतुल्यगती वीरौ शार्दूलवृषभोपमौ}
{पद्मपत्रविशालाक्षौ खड्गतूणीधनुर्धरौ}
{अश्विनाविव रूपेण समुपस्थितयौवनौ} %1-50-18

\twolineshloka
{यदृच्छयेव गां प्राप्तौ देवलोकादिवामरौ}
{कथं पद्भ्यामिह प्राप्तौ किमर्थं कस्य वा मुने} %1-50-19

\twolineshloka
{वरायुधधरौ वीरौ कस्य पुत्रौ महामुने}
{भूषयन्ताविमं देशं चन्द्रसूर्याविवाम्बरम्} %1-50-20

\twolineshloka
{परस्परस्य सदृशौ प्रमाणेङ्गितचेष्टितैः}
{काकपक्षधरौ वीरौ श्रोतुमिच्छामि तत्त्वतः} %1-50-21

\twolineshloka
{तस्य तद् वचनं श्रुत्वा जनकस्य महात्मनः}
{न्यवेदयदमेयात्मा पुत्रौ दशरथस्य तौ} %1-50-22

\twolineshloka
{सिद्धाश्रमनिवासं च राक्षसानां वधं तथा}
{तत्रागमनमव्यग्रं विशालायाश्च दर्शनम्} %1-50-23

\twolineshloka
{अहल्यादर्शनं चैव गौतमेन समागमम्}
{महाधनुषि जिज्ञासां कर्तुमागमनं तथा} %1-50-24

\twolineshloka
{एतत् सर्वं महातेजा जनकाय महात्मने}
{निवेद्य विररामाथ विश्वामित्रो महामुनिः} %1-50-25


॥इत्यार्षे श्रीमद्रामायणे वाल्मीकीये आदिकाव्ये बालकाण्डे जनकसमागमः नाम पञ्चाशः सर्गः ॥१-५०॥
