\sect{सप्ततितमः सर्गः — कन्यावरणम्}

\twolineshloka
{ततः प्रभाते जनकः कृतकर्मा महर्षिभिः}
{उवाच वाक्यं वाक्यज्ञः शतानन्दं पुरोहितम्} %1-70-1

\twolineshloka
{भ्राता मम महातेजा वीर्यवानतिधार्मिकः}
{कुशध्वज इति ख्यातः पुरीमध्यवसच्छुभाम्} %1-70-2

\twolineshloka
{वार्याफलकपर्यन्तां पिबन्निक्षुमतीं नदीम्}
{सांकाश्यां पुण्यसंकाशां विमानमिव पुष्पकम्} %1-70-3

\twolineshloka
{तमहं द्रष्टुमिच्छामि यज्ञगोप्ता स मे मतः}
{प्रीतिं सोऽपि महातेजा इमां भोक्ता मया सह} %1-70-4

\twolineshloka
{एवमुक्ते तु वचने शतानन्दस्य संनिधौ}
{आगताः केचिदव्यग्रा जनकस्तान् समादिशत्} %1-70-5

\twolineshloka
{शासनात् तु नरेन्द्रस्य प्रययुः शीघ्रवाजिभिः}
{समानेतुं नरव्याघ्रं विष्णुमिन्द्राज्ञया यथा} %1-70-6

\twolineshloka
{सांकाश्यां ते समागम्य ददृशुश्च कुशध्वजम्}
{न्यवेदयन् यथावृत्तं जनकस्य च चिन्तितम्} %1-70-7

\twolineshloka
{तद्वृत्तं नृपतिः श्रुत्वा दूतश्रेष्ठैर्महाजवैः}
{आज्ञया तु नरेन्द्रस्य आजगाम कुशध्वजः} %1-70-8

\twolineshloka
{स ददर्श महात्मानं जनकं धर्मवत्सलम्}
{सोऽभिवाद्य शतानन्दं जनकं चातिधार्मिकम्} %1-70-9

\twolineshloka
{राजार्हं परमं दिव्यमासनं सोऽध्यरोहत}
{उपविष्टावुभौ तौ तु भ्रातरावमितद्युती} %1-70-10

\twolineshloka
{प्रेषयामासतुर्वीरौ मन्त्रिश्रेष्ठं सुदामनम्}
{गच्छ मन्त्रिपते शीघ्रमिक्ष्वाकुममितप्रभम्} %1-70-11

\twolineshloka
{आत्मजैः सह दुर्धर्षमानयस्व समन्त्रिणम्}
{औपकार्यां स गत्वा तु रघूणां कुलवर्धनम्} %1-70-12

\twolineshloka
{ददर्श शिरसा चैनमभिवाद्येदमब्रवीत्}
{अयोध्याधिपते वीर वैदेहो मिथिलाधिपः} %1-70-13

\twolineshloka
{स त्वां द्रष्टुं व्यवसितः सोपाध्यायपुरोहितम्}
{मन्त्रिश्रेष्ठवचः श्रुत्वा राजा सर्षिगणस्तथा} %1-70-14

\twolineshloka
{सबन्धुरगमत् तत्र जनको यत्र वर्तते}
{राजा च मन्त्रिसहितः सोपाध्यायः सबान्धवः} %1-70-15

\twolineshloka
{वाक्यं वाक्यविदां श्रेष्ठो वैदेहमिदमब्रवीत्}
{विदितं ते महाराज इक्ष्वाकुकुलदैवतम्} %1-70-16

\twolineshloka
{वक्ता सर्वेषु कृत्येषु वसिष्ठो भगवानृषिः}
{विश्वामित्राभ्यनुज्ञातः सह सर्वैर्महर्षिभिः} %1-70-17

\twolineshloka
{एष वक्ष्यति धर्मात्मा वसिष्ठो मे यथाक्रमम्}
{तूष्णींभूते दशरथे वसिष्ठो भगवानृषिः} %1-70-18

\twolineshloka
{उवाच वाक्यं वाक्यज्ञो वैदेहं सपुरोधसम्}
{अव्यक्तप्रभवो ब्रह्मा शाश्वतो नित्य अव्ययः} %1-70-19

\twolineshloka
{तस्मान्मरीचिः संजज्ञे मरीचेः कश्यपः सुतः}
{विवस्वान् कश्यपाज्जज्ञे मनुर्वैवस्वतः स्मृतः} %1-70-20

\twolineshloka
{मनुः प्रजापतिः पूर्वमिक्ष्वाकुश्च मनोः सुतः}
{तमिक्ष्वाकुमयोध्यायां राजानं विद्धि पूर्वकम्} %1-70-21

\twolineshloka
{इक्ष्वाकोस्तु सुतः श्रीमान् कुक्षिरित्येव विश्रुतः}
{कुक्षेरथात्मजः श्रीमान् विकुक्षिरुदपद्यत} %1-70-22

\twolineshloka
{विकुक्षेस्तु महातेजा बाणः पुत्रः प्रतापवान्}
{बाणस्य तु महातेजा अनरण्यः प्रतापवान्} %1-70-23

\twolineshloka
{अनरण्यात् पृथुर्जज्ञे त्रिशङ्कुस्तु पृथोरपि}
{त्रिशङ्कोरभवत् पुत्रो धुन्धुमारो महायशाः} %1-70-24

\twolineshloka
{धुन्धुमारान्महातेजा युवनाश्वो महारथः}
{युवनाश्वसुतश्चासीन्मान्धाता पृथिवीपतिः} %1-70-25

\twolineshloka
{मान्धातुस्तु सुतः श्रीमान् सुसन्धिरुदपद्यत}
{सुसन्धेरपि पुत्रौ द्वौ ध्रुवसन्धिः प्रसेनजित्} %1-70-26

\twolineshloka
{यशस्वी ध्रुवसन्धेस्तु भरतो नाम नामतः}
{भरतात् तु महातेजा असितो नाम जायत} %1-70-27

\twolineshloka
{यस्यैते प्रतिराजान उदपद्यन्त शत्रवः}
{हैहयास्तालजङ्घाश्च शूराश्च शशबिन्दवः} %1-70-28

\twolineshloka
{तांश्च स प्रतियुध्यन् वै युद्धे राजा प्रवासितः}
{हिमवन्तमुपागम्य भार्याभ्यां सहितस्तदा} %1-70-29

\twolineshloka
{असितोऽल्पबलो राजा कालधर्ममुपेयिवान्}
{द्वे चास्य भार्ये गर्भिण्यौ बभूवतुरिति श्रुतिः} %1-70-30

\twolineshloka
{एका गर्भविनाशार्थं सपत्न्यै सगरं ददौ}
{ततः शैलवरे रम्ये बभूवाभिरतो मुनिः} %1-70-31

\twolineshloka
{भार्गवश्च्यवनो नाम हिमवन्तमुपाश्रितः}
{तत्र चैका महाभागा भार्गवं देववर्चसम्} %1-70-32

\twolineshloka
{ववन्दे पद्मपत्राक्षी कांक्षन्ती सुतमुत्तमम्}
{तमृषिं साभ्युपागम्य कालिन्दी चाभ्यवादयत्} %1-70-33

\twolineshloka
{स तामभ्यवदद् विप्रः पुत्रेप्सुं पुत्रजन्मनि}
{तव कुक्षौ महाभागे सुपुत्रः सुमहाबलः} %1-70-34

\twolineshloka
{महावीर्यो महातेजा अचिरात् संजनिष्यति}
{गरेण सहितः श्रीमान् मा शुचः कमलेक्षणे} %1-70-35

\twolineshloka
{च्यवनं च नमस्कृत्य राजपुत्री पतिव्रता}
{पत्या विरहिता तस्मात् पुत्रं देवी व्यजायत} %1-70-36

\twolineshloka
{सपत्न्या तु गरस्तस्यै दत्तो गर्भजिघांसया}
{सह तेन गरेणैव संजातः सगरोऽभवत्} %1-70-37

\twolineshloka
{सगरस्यासमञ्जस्तु असमञ्जादथांशुमान्}
{दिलीपोंऽशुमतः पुत्रो दिलीपस्य भगीरथः} %1-70-38

\twolineshloka
{भगीरथात् ककुत्स्थश्च ककुत्स्थाच्च रघुस्तथा}
{रघोस्तु पुत्रस्तेजस्वी प्रवृद्धः पुरुषादकः} %1-70-39

\twolineshloka
{कल्माषपादोऽप्यभवत् तस्माज्जातस्तु शङ्खणः}
{सुदर्शनः शङ्खणस्य अग्निवर्णः सुदर्शनात्} %1-70-40

\twolineshloka
{शीघ्रगस्त्वग्निवर्णस्य शीघ्रगस्य मरुः सुतः}
{मरोः प्रशुश्रुकस्त्वासीदम्बरीषः प्रशुश्रुकात्} %1-70-41

\twolineshloka
{अम्बरीषस्य पुत्रोऽभून्नहुषश्च महीपतिः}
{नहुषस्य ययातिस्तु नाभागस्तु ययातिजः} %1-70-42

\twolineshloka
{नाभागस्य बभूवाज अजाद् दशरथोऽभवत्}
{अस्माद् दशरथाज्जातौ भ्रातरौ रामलक्ष्मणौ} %1-70-43

\twolineshloka
{आदिवंशविशुद्धानां राज्ञां परमधर्मिणाम्}
{इक्ष्वाकुकुलजातानां वीराणां सत्यवादिनाम्} %1-70-44

\twolineshloka
{रामलक्ष्मणयोरर्थे त्वत्सुते वरये नृप}
{सदृशाभ्यां नरश्रेष्ठ सदृशे दातुमर्हसि} %1-70-45


॥इत्यार्षे श्रीमद्रामायणे वाल्मीकीये आदिकाव्ये बालकाण्डे कन्यावरणम् नाम सप्ततितमः सर्गः ॥१-७०॥
