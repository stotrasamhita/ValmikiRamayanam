\sect{पञ्चषष्ठितमः सर्गः — ब्रह्मर्षित्वप्राप्तिः}

\twolineshloka
{अथ हैमवतीं राम दिशं त्यक्त्वा महामुनिः}
{पूर्वां दिशमनुप्राप्य तपस्तेपे सुदारुणम्} %1-65-1

\twolineshloka
{मौनं वर्षसहस्रस्य कृत्वा व्रतमनुत्तमम्}
{चकाराप्रतिमं राम तपः परमदुष्करम्} %1-65-2

\twolineshloka
{पूर्णे वर्षसहस्रे तु काष्ठभूतं महामुनिम्}
{विघ्नैर्बहुभिराधूतं क्रोधो नान्तरमाविशत्} %1-65-3

\twolineshloka
{स कृत्वा निश्चयं राम तप आतिष्ठताव्ययम्}
{तस्य वर्षसहस्रस्य व्रते पूर्णे महाव्रतः} %1-65-4

\twolineshloka
{भोक्तुमारब्धवानन्नं तस्मिन् काले रघूत्तम}
{इन्द्रो द्विजातिर्भूत्वा तं सिद्धमन्नमयाचत} %1-65-5

\twolineshloka
{तस्मै दत्त्वा तदा सिद्धं सर्वं विप्राय निश्चितः}
{निःशेषितेऽन्ने भगवानभुक्त्वैव महातपाः} %1-65-6

\twolineshloka
{न किंचिदवदद् विप्रं मौनव्रतमुपास्थितः}
{तथैवासीत् पुनर्मौनमनुच्छ्वासं चकार ह} %1-65-7

\twolineshloka
{अथ वर्षसहस्रं च नोच्छ्वसन् मुनिपुंगवः}
{तस्यानुच्छ्वसमानस्य मूर्ध्नि धूमो व्यजायत} %1-65-8

\twolineshloka
{त्रैलोक्यं येन सम्भ्रान्तमातापितमिवाभवत्}
{ततो देवर्षिगन्धर्वाः पन्नगोरगराक्षसाः} %1-65-9

\twolineshloka
{मोहितास्तपसा तस्य तेजसा मन्दरश्मयः}
{कश्मलोपहताः सर्वे पितामहमथाब्रुवन्} %1-65-10

\twolineshloka
{बहुभिः कारणैर्देव विश्वामित्रो महामुनिः}
{लोभितः क्रोधितश्चैव तपसा चाभिवर्धते} %1-65-11

\twolineshloka
{नह्यस्य वृजिनं किंचिद् दृश्यते सूक्ष्ममप्युत}
{न दीयते यदि त्वस्य मनसा यदभीप्सितम्} %1-65-12

\twolineshloka
{विनाशयति त्रैलोक्यं तपसा सचराचरम्}
{व्याकुलाश्च दिशः सर्वा न च किंचित् प्रकाशते} %1-65-13

\twolineshloka
{सागराः क्षुभिताः सर्वे विशीर्यन्ते च पर्वताः}
{प्रकम्पते च वसुधा वायुर्वातीह संकुलः} %1-65-14

\twolineshloka
{ब्रह्मन् न प्रतिजानीमो नास्तिको जायते जनः}
{सम्मूढमिव त्रैलोक्यं सम्प्रक्षुभितमानसम्} %1-65-15

\twolineshloka
{भास्करो निष्प्रभश्चैव महर्षेस्तस्य तेजसा}
{बुद्धिं न कुरुते यावन्नाशे देव महामुनिः} %1-65-16

\twolineshloka
{तावत् प्रसादो भगवन्नग्निरूपो महाद्युतिः}
{कालाग्निना यथा पूर्वं त्रैलोक्यं दह्यतेऽखिलम्} %1-65-17

\twolineshloka
{देवराज्यं चिकीर्षेत दीयतामस्य यन्मनः}
{ततः सुरगणाः सर्वे पितामहपुरोगमाः} %1-65-18

\twolineshloka
{विश्वामित्रं महात्मानं वाक्यं मधुरमब्रुवन्}
{ब्रह्मर्षे स्वागतं तेऽस्तु तपसा स्म सुतोषिताः} %1-65-19

\twolineshloka
{ब्राह्मण्यं तपसोग्रेण प्राप्तवानसि कौशिक}
{दीर्घमायुश्च ते ब्रह्मन् ददामि समरुद्गणः} %1-65-20

\twolineshloka
{स्वस्ति प्राप्नुहि भद्रं ते गच्छ सौम्य यथासुखम्}
{पितामहवचः श्रुत्वा सर्वेषां त्रिदिवौकसाम्} %1-65-21

\twolineshloka
{कृत्वा प्रणामं मुदितो व्याजहार महामुनिः}
{ब्राह्मण्यं यदि मे प्राप्तं दीर्घमायुस्तथैव च} %1-65-22

\twolineshloka
{ॐकारोऽथ वषट्कारो वेदाश्च वरयन्तु माम्}
{क्षत्रवेदविदां श्रेष्ठो ब्रह्मवेदविदामपि} %1-65-23

\twolineshloka
{ब्रह्मपुत्रो वसिष्ठो मामेवं वदतु देवताः}
{यद्येवं परमः कामः कृतो यान्तु सुरर्षभाः} %1-65-24

\twolineshloka
{ततः प्रसादितो देवैर्वसिष्ठो जपतां वरः}
{सख्यं चकार ब्रह्मर्षिरेवमस्त्विति चाब्रवीत्} %1-65-25

\twolineshloka
{ब्रह्मर्षिस्त्वं न संदेहः सर्वं सम्पद्यते तव}
{इत्युक्त्वा देवताश्चापि सर्वा जग्मुर्यथागतम्} %1-65-26

\twolineshloka
{विश्वामित्रोऽपि धर्मात्मा लब्ध्वा ब्राह्मण्यमुत्तमम्}
{पूजयामास ब्रह्मर्षिं वसिष्ठं जपतां वरम्} %1-65-27

\twolineshloka
{कृतकामो महीं सर्वां चचार तपसि स्थितः}
{एवं त्वनेन ब्राह्मण्यं प्राप्तं राम महात्मना} %1-65-28

\twolineshloka
{एष राम मुनिश्रेष्ठ एष विग्रहवांस्तपः}
{एष धर्मः परो नित्यं वीर्यस्यैष परायणम्} %1-65-29

\twolineshloka
{एवमुक्त्वा महातेजा विरराम द्विजोत्तमः}
{शतानन्दवचः श्रुत्वा रामलक्ष्मणसंनिधौ} %1-65-30

\twolineshloka
{जनकः प्राञ्जलिर्वाक्यमुवाच कुशिकात्मजम्}
{धन्योऽस्म्यनुगृहीतोऽस्मि यस्य मे मुनिपुंगव} %1-65-31

\twolineshloka
{यज्ञं काकुत्स्थसहितः प्राप्तवानसि कौशिक}
{पावितोऽहं त्वया ब्रह्मन् दर्शनेन महामुने} %1-65-32

\twolineshloka
{गुणा बहुविधाः प्राप्तास्तव संदर्शनान्मया}
{विस्तरेण च वै ब्रह्मन् कीर्त्यमानं महत्तपः} %1-65-33

\twolineshloka
{श्रुतं मया महातेजो रामेण च महात्मना}
{सदस्यैः प्राप्य च सदः श्रुतास्ते बहवो गुणाः} %1-65-34

\twolineshloka
{अप्रमेयं तपस्तुभ्यमप्रमेयं च ते बलम्}
{अप्रमेया गुणाश्चैव नित्यं ते कुशिकात्मज} %1-65-35

\twolineshloka
{तृप्तिराश्चर्यभूतानां कथानां नास्ति मे विभो}
{कर्मकालो मुनिश्रेष्ठ लम्बते रविमण्डलम्} %1-65-36

\twolineshloka
{श्वः प्रभाते महातेजो द्रष्टुमर्हसि मां पुनः}
{स्वागतं जपतां श्रेष्ठ मामनुज्ञातुमर्हसि} %1-65-37

\twolineshloka
{एवमुक्तो मुनिवरः प्रशस्य पुरुषर्षभम्}
{विससर्जाशु जनकं प्रीतं प्रीतमनास्तदा} %1-65-38

\twolineshloka
{एवमुक्त्वा मुनिश्रेष्ठं वैदेहो मिथिलाधिपः}
{प्रदक्षिणं चकाराशु सोपाध्यायः सबान्धवः} %1-65-39

\twolineshloka
{विश्वामित्रोऽपि धर्मात्मा सहरामः सलक्ष्मणः}
{स्ववासमभिचक्राम पूज्यमानो महात्मभिः} %1-65-40


॥इत्यार्षे श्रीमद्रामायणे वाल्मीकीये आदिकाव्ये बालकाण्डे ब्रह्मर्षित्वप्राप्तिः नाम पञ्चषष्ठितमः सर्गः ॥१-६५॥
