\sect{अष्टषष्ठितमः सर्गः — दशरथाह्वानम्}

\twolineshloka
{जनकेन समादिष्टा दूतास्ते क्लान्तवाहनाः}
{त्रिरात्रमुषिता मार्गे तेऽयोध्यां प्राविशन् पुरीम्} %1-68-1

\twolineshloka
{ते राजवचनाद् गत्वा राजवेश्म प्रवेशिताः}
{ददृशुर्देवसङ्काशं वृद्धं दशरथं नृपम्} %1-68-2

\twolineshloka
{बद्धाञ्जलिपुटाः सर्वे दूता विगतसाध्वसाः}
{राजानं प्रश्रितं वाक्यमब्रुवन् मधुराक्षरम्} %1-68-3

\twolineshloka
{मैथिलो जनको राजा साग्निहोत्रपुरस्कृतः}
{मुहुर्मुहुर्मधुरया स्नेहसंरक्तया गिरा} %1-68-4

\twolineshloka
{कुशलं चाव्ययं चैव सोपाध्यायपुरोहितम्}
{जनकस्त्वां महाराज पृच्छते सपुरःसरम्} %1-68-5

\twolineshloka
{पृष्ट्वा कुशलमव्यग्रं वैदेहो मिथिलाधिपः}
{कौशिकानुमते वाक्यं भवन्तमिदमब्रवीत्} %1-68-6

\twolineshloka
{पूर्वं प्रतिज्ञा विदिता वीर्यशुल्का ममात्मजा}
{राजानश्च कृतामर्षा निर्वीर्या विमुखीकृताः} %1-68-7

\twolineshloka
{सेयं मम सुता राजन् विश्वामित्रपुरस्कृतैः}
{यदृच्छयागतै राजन् निर्जिता तव पुत्रकैः} %1-68-8

\twolineshloka
{तच्च रत्नं धनुर्दिव्यं मध्ये भग्नं महात्मना}
{रामेण हि महाबाहो महत्यां जनसंसदि} %1-68-9

\twolineshloka
{अस्मै देया मया सीता वीर्यशुल्का महात्मने}
{प्रतिज्ञां तर्तुमिच्छामि तदनुज्ञातुमर्हसि} %1-68-10

\twolineshloka
{सोपाध्यायो महाराज पुरोहितपुरस्कृतः}
{शीघ्रमागच्छ भद्रं ते द्रष्टुमर्हसि राघवौ} %1-68-11

\twolineshloka
{प्रतिज्ञां मम राजेन्द्र निर्वर्तयितुमर्हसि}
{पुत्रयोरुभयोरेव प्रीतिं त्वमुपलप्स्यसे} %1-68-12

\twolineshloka
{एवं विदेहाधिपतिर्मधुरं वाक्यमब्रवीत्}
{विश्वामित्राभ्यनुज्ञातः शतानन्दमते स्थितः} %1-68-13

\twolineshloka
{दूतवाक्यं तु तच्छ्रुत्वा राजा परमहर्षितः}
{वसिष्ठं वामदेवं च मन्त्रिणश्चैवमब्रवीत्} %1-68-14

\twolineshloka
{गुप्तः कुशिकपुत्रेण कौसल्यानन्दवर्धनः}
{लक्ष्मणेन सह भ्रात्रा विदेहेषु वसत्यसौ} %1-68-15

\twolineshloka
{दृष्टवीर्यस्तु काकुत्स्थो जनकेन महात्मना}
{सम्प्रदानं सुतायास्तु राघवे कर्तुमिच्छति} %1-68-16

\twolineshloka
{यदि वो रोचते वृत्तं जनकस्य महात्मनः}
{पुरीं गच्छामहे शीघ्रं मा भूत् कालस्य पर्ययः} %1-68-17

\twolineshloka
{मन्त्रिणो बाढमित्याहुः सह सर्वैर्महर्षिभिः}
{सुप्रीतश्चाब्रवीद् राजा श्वो यात्रेति च मन्त्रिणः} %1-68-18

\twolineshloka
{मन्त्रिणस्तु नरेन्द्रस्य रात्रिं परमसत्कृताः}
{ऊषुः प्रमुदिताः सर्वे गुणैः सर्वैः समन्विताः} %1-68-19


॥इत्यार्षे श्रीमद्रामायणे वाल्मीकीये आदिकाव्ये बालकाण्डे दशरथाह्वानम् नाम अष्टषष्ठितमः सर्गः ॥१-६८॥
