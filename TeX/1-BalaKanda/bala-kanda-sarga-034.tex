\sect{चतुस्त्रिंशः सर्गः — विश्वामित्रवंशवर्णनम्}

\twolineshloka
{कृतोद्वाहे गते तस्मिन् ब्रह्मदत्ते च राघव}
{अपुत्रः पुत्रलाभाय पौत्रीमिष्टिमकल्पयत्} %1-34-1

\twolineshloka
{इष्ट्यां तु वर्तमानायां कुशनाभं महीपतिम्}
{उवाच परमोदारः कुशो ब्रह्मसुतस्तदा} %1-34-2

\twolineshloka
{पुत्रस्ते सदृशः पुत्र भविष्यति सुधार्मिकः}
{गाधिं प्राप्स्यसि तेन त्वं कीर्तिं लोके च शाश्वतीम्} %1-34-3

\twolineshloka
{एवमुक्त्वा कुशो राम कुशनाभं महीपतिम्}
{जगामाकाशमाविश्य ब्रह्मलोकं सनातनम्} %1-34-4

\twolineshloka
{कस्यचित् त्वथ कालस्य कुशनाभस्य धीमतः}
{जज्ञे परमधर्मिष्ठो गाधिरित्येव नामतः} %1-34-5

\twolineshloka
{स पिता मम काकुत्स्थ गाधिः परमधार्मिकः}
{कुशवंशप्रसूतोऽस्मि कौशिको रघुनन्दन} %1-34-6

\twolineshloka
{पूर्वजा भगिनी चापि मम राघव सुव्रता}
{नाम्ना सत्यवती नाम ऋचीके प्रतिपादिता} %1-34-7

\twolineshloka
{सशरीरा गता स्वर्गं भर्तारमनुवर्तिनी}
{कौशिकी परमोदारा सा प्रवृत्ता महानदी} %1-34-8

\twolineshloka
{दिव्या पुण्योदका रम्या हिमवन्तमुपाश्रिता}
{लोकस्य हितकार्यार्थं प्रवृत्ता भगिनी मम} %1-34-9

\twolineshloka
{ततोऽहं हिमवत्पार्श्वे वसामि नियतः सुखम्}
{भगिन्यां स्नेहसंयुक्तः कौशिक्यां रघुनन्दन} %1-34-10

\twolineshloka
{सा तु सत्यवती पुण्या सत्ये धर्मे प्रतिष्ठिता}
{पतिव्रता महाभागा कौशिकी सरितां वरा} %1-34-11

\twolineshloka
{अहं हि नियमाद्राम हित्वा तां समुपागतः}
{सिद्धाश्रममनुप्राप्तः सिद्धोऽस्मि तव तेजसा} %1-34-12

\twolineshloka
{एषा राम ममोत्पत्तिः स्वस्य वंशस्य कीर्तिता}
{देशस्य च महाबाहो यन्मां त्वं परिपृच्छसि} %1-34-13

\twolineshloka
{गतोऽर्धरात्रः काकुत्स्थ कथाः कथयतो मम}
{निद्रामभ्येहि भद्रं ते मा भूद् विघ्नोऽध्वनीह नः} %1-34-14

\twolineshloka
{निष्पन्दास्तरवः सर्वे निलीना मृगपक्षिणः}
{नैशेन तमसा व्याप्ता दिशश्च रघुनन्दन} %1-34-15

\twolineshloka
{शनैर्विसृज्यते सन्ध्या नभो नेत्रैरिवावृतम्}
{नक्षत्रतारागहनं ज्योतिर्भिरवभासते} %1-34-16

\twolineshloka
{उत्तिष्ठते च शीतांशुः शशी लोकतमोनुदः}
{ह्लादयन् प्राणिनां लोके मनांसि प्रभया स्वया} %1-34-17

\twolineshloka
{नैशानि सर्वभूतानि प्रचरन्ति ततस्ततः}
{यक्षराक्षससङ्घाश्च रौद्राश्च पिशिताशनाः} %1-34-18

\twolineshloka
{एवमुक्त्वा महातेजा विरराम महामुनिः}
{साधुसाध्विति तं सर्वे मुनयो ह्यभ्यपूजयन्} %1-34-19

\twolineshloka
{कुशिकानामयं वंशो महान् धर्मपरः सदा}
{ब्रह्मोपमा महात्मानः कुशवंश्या नरोत्तमाः} %1-34-20

\twolineshloka
{विशेषेण भवानेव विश्वामित्र महायशः}
{कौशिकी सरितां श्रेष्ठा कुलोद्योतकरी तव} %1-34-21

\twolineshloka
{मुदितैर्मुनिशार्दूलैः प्रशस्तः कुशिकात्मजः}
{निद्रामुपागमच्छ्रीमानस्तङ्गत इवांशुमान्} %1-34-22

\twolineshloka
{रामोऽपि सहसौमित्रिः किञ्चिदागतविस्मयः}
{प्रशस्य मुनिशार्दूलं निद्रां समुपसेवते} %1-34-23


॥इत्यार्षे श्रीमद्रामायणे वाल्मीकीये आदिकाव्ये बालकाण्डे विश्वामित्रवंशवर्णनम् नाम चतुस्त्रिंशः सर्गः ॥१-३४॥
