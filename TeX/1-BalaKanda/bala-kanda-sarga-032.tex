\sect{द्वात्रिंशः सर्गः — कुशनाभकन्योपाख्यानम्}

\twolineshloka
{ब्रह्मयोनिर्महानासीत् कुशो नाम महातपाः}
{अक्लिष्टव्रतधर्मज्ञः सज्जनप्रतिपूजकः} %1-32-1

\twolineshloka
{स महात्मा कुलीनायां युक्तायां सुमहाबलान्}
{वैदर्भ्यां जनयामास चतुरः सदृशान् सुतान्} %1-32-2

\twolineshloka
{कुशाम्बं कुशनाभं च असूर्तरजसं वसुम्}
{दीप्तियुक्तान् महोत्साहान् क्षत्रधर्मचिकीर्षया} %1-32-3

\twolineshloka
{तानुवाच कुशः पुत्रान् धर्मिष्ठान् सत्यवादिनः}
{क्रियतां पालनं पुत्रा धर्मं प्राप्स्यथ पुष्कलम्} %1-32-4

\twolineshloka
{कुशस्य वचनं श्रुत्वा चत्वारो लोकसत्तमाः}
{निवेशं चक्रिरे सर्वे पुराणां नृवरास्तदा} %1-32-5

\twolineshloka
{कुशाम्बस्तु महातेजाः कौशाम्बीमकरोत् पुरीम्}
{कुशनाभस्तु धर्मात्मा पुरं चक्रे महोदयम्} %1-32-6

\twolineshloka
{असूर्तरजसो नाम धर्मारण्यं महीपतिः}
{चक्रे पुरवरं राजा वसुनाम गिरिव्रजम्} %1-32-7

\twolineshloka
{एषा वसुमती राम वसोस्तस्य महात्मनः}
{एते शैलवराः पञ्च प्रकाशन्ते समन्ततः} %1-32-8

\twolineshloka
{सुमागधी नदी रम्या मागधान् विश्रुताऽऽययौ}
{पञ्चानां शैलमुख्यानां मध्ये मालेव शोभते} %1-32-9

\twolineshloka
{सैषा हि मागधी राम वसोस्तस्य महात्मनः}
{पूर्वाभिचरिता राम सुक्षेत्रा सस्यमालिनी} %1-32-10

\twolineshloka
{कुशनाभस्तु राजर्षिः कन्याशतमनुत्तमम्}
{जनयामास धर्मात्मा घृताच्यां रघुनन्दन} %1-32-11

\twolineshloka
{तास्तु यौवनशालिन्यो रूपवत्यः स्वलङ्कृताः}
{उद्यानभूमिमागम्य प्रावृषीव शतह्रदाः} %1-32-12

\twolineshloka
{गायन्त्यो नृत्यमानाश्च वादयन्त्यस्तु राघव}
{आमोदं परमं जग्मुर्वराभरणभूषिताः} %1-32-13

\twolineshloka
{अथ ताश्चारुसर्वाङ्ग्यो रूपेणाप्रतिमा भुवि}
{उद्यानभूमिमागम्य तारा इव घनान्तरे} %1-32-14

\twolineshloka
{ताः सर्वा गुणसम्पन्ना रूपयौवनसंयुताः}
{दृष्ट्वा सर्वात्मको वायुरिदं वचनमब्रवीत्} %1-32-15

\twolineshloka
{अहं वः कामये सर्वा भार्या मम भविष्यथ}
{मानुषस्त्यज्यतां भावो दीर्घमायुरवाप्स्यथ} %1-32-16

\twolineshloka
{चलं हि यौवनं नित्यं मानुषेषु विशेषतः}
{अक्षयं यौवनं प्राप्ता अमर्यश्च भविष्यथ} %1-32-17

\twolineshloka
{तस्य तद्वचनं श्रुत्वा वायोरक्लिष्टकर्मणः}
{अपहास्य ततो वाक्यं कन्याशतमथाब्रवीत्} %1-32-18

\twolineshloka
{अन्तश्चरसि भूतानां सर्वेषां सुरसत्तम}
{प्रभावज्ञाश्च ते सर्वाः किमर्थमवमन्यसे} %1-32-19

\twolineshloka
{कुशनाभसुता देव समस्ताः सुरसत्तम}
{स्थानाच्च्यावयितुं देवं रक्षामस्तु तपो वयम्} %1-32-20

\twolineshloka
{मा भूत् स कालो दुर्मेधः पितरं सत्यवादिनम्}
{अवमन्य स्वधर्मेण स्वयं वरमुपास्महे} %1-32-21

\twolineshloka
{पिता हि प्रभुरस्माकं दैवतं परमं च सः}
{यस्य नो दास्यति पिता स नो भर्ता भविष्यति} %1-32-22

\twolineshloka
{तासां तु वचनं श्रुत्वा हरिः परमकोपनः}
{प्रविश्य सर्वगात्राणि बभञ्ज भगवान् प्रभुः} %1-32-23

\threelineshloka
{अरत्निमात्राकृतयो भग्नगात्रा भयार्दिताः}
{ताः कन्या वायुना भग्ना विविशुर्नृपतेर्गृहम्}
{प्रविश्य च सुसम्भ्रान्ताः सलज्जाः सास्रलोचनाः} %1-32-24

\twolineshloka
{स च ता दयिता भग्नाः कन्याः परमशोभनाः}
{दृष्ट्वा दीनास्तदा राजा सम्भ्रान्त इदमब्रवीत्} %1-32-25

\threelineshloka
{किमिदं कथ्यतां पुत्र्यः को धर्ममवमन्यते}
{कुब्जाः केन कृताः सर्वाश्चेष्टन्त्यो नाभिभाषथ}
{एवं राजा विनिःश्वस्य समाधिं संदधे ततः} %1-32-26


॥इत्यार्षे श्रीमद्रामायणे वाल्मीकीये आदिकाव्ये बालकाण्डे कुशनाभकन्योपाख्यानम् नाम द्वात्रिंशः सर्गः ॥१-३२॥
