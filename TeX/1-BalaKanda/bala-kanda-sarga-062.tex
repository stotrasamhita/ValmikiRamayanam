\sect{द्विषष्ठितमः सर्गः — अम्बरीशयज्ञः}

\twolineshloka
{शुनःशेपं नरश्रेष्ठ गृहीत्वा तु महायशाः}
{व्यश्रमत् पुष्करे राजा मध्याह्ने रघुनन्दन} %1-62-1

\twolineshloka
{तस्य विश्रममाणस्य शुनःशेपो महायशाः}
{पुष्करं ज्येष्ठमागम्य विश्वामित्रं ददर्श ह} %1-62-2

\twolineshloka
{तप्यन्तमृषिभिः सार्धं मातुलं परमातुरः}
{विषण्णवदनो दीनस्तृष्णया च श्रमेण च} %1-62-3

\twolineshloka
{पपाताङ्के मुने राम वाक्यं चेदमुवाच ह}
{न मेऽस्ति माता न पिता ज्ञातयो बान्धवाः कुतः} %1-62-4

\twolineshloka
{त्रातुमर्हसि मां सौम्य धर्मेण मुनिपुङ्गव}
{त्राता त्वं हि नरश्रेष्ठ सर्वेषां त्वं हि भावनः} %1-62-5

\twolineshloka
{राजा च कृतकार्यः स्यादहं दीर्घायुरव्ययः}
{स्वर्गलोकमुपाश्नीयां तपस्तप्त्वा ह्यनुत्तमम्} %1-62-6

\twolineshloka
{स मे नाथो ह्यनाथस्य भव भव्येन चेतसा}
{पितेव पुत्रं धर्मात्मंस्त्रातुमर्हसि किल्बिषात्} %1-62-7

\twolineshloka
{तस्य तद् वचनं श्रुत्वा विश्वामित्रो महातपाः}
{सान्त्वयित्वा बहुविधं पुत्रानिदमुवाच ह} %1-62-8

\twolineshloka
{यत्कृते पितरः पुत्राञ्जनयन्ति शुभार्थिनः}
{परलोकहितार्थाय तस्य कालोऽयमागतः} %1-62-9

\twolineshloka
{अयं मुनिसुतो बालो मत्तः शरणमिच्छति}
{अस्य जीवितमात्रेण प्रियं कुरुत पुत्रकाः} %1-62-10

\twolineshloka
{सर्वे सुकृतकर्माणः सर्वे धर्मपरायणाः}
{पशुभूता नरेन्द्रस्य तृप्तिमग्नेः प्रयच्छत} %1-62-11

\twolineshloka
{नाथवांश्च शुनःशेपो यज्ञश्चाविघ्नतो भवेत्}
{देवतास्तर्पिताश्च स्युर्मम चापि कृतं वचः} %1-62-12

\twolineshloka
{मुनेस्तद् वचनं श्रुत्वा मधुच्छन्दादयः सुताः}
{साभिमानं नरश्रेष्ठ सलीलमिदमब्रुवन्} %1-62-13

\twolineshloka
{कथमात्मसुतान् हित्वा त्रायसेऽन्यसुतं विभो}
{अकार्यमिव पश्यामः श्वमांसमिव भोजने} %1-62-14

\twolineshloka
{तेषां तद् वचनं श्रुत्वा पुत्राणां मुनिपुङ्गवः}
{क्रोधसंरक्तनयनो व्याहर्तुमुपचक्रमे} %1-62-15

\twolineshloka
{निःसाध्वसमिदं प्रोक्तं धर्मादपि विगर्हितम्}
{अतिक्रम्य तु मद्वाक्यं दारुणं रोमहर्षणम्} %1-62-16

\twolineshloka
{श्वमांसभोजिनः सर्वे वासिष्ठा इव जातिषु}
{पूर्णं वर्षसहस्रं तु पृथिव्यामनुवत्स्यथ} %1-62-17

\twolineshloka
{कृत्वा शापसमायुक्तान् पुत्रान् मुनिवरस्तदा}
{शुनःशेपमुवाचार्तं कृत्वा रक्षां निरामयाम्} %1-62-18

\twolineshloka
{पवित्रपाशैराबद्धो रक्तमाल्यानुलेपनः}
{वैष्णवं यूपमासाद्य वाग्भिरग्निमुदाहर} %1-62-19

\twolineshloka
{इमे च गाथे द्वे दिव्ये गायेथा मुनिपुत्रक}
{अम्बरीषस्य यज्ञेऽस्मिंस्ततः सिद्धिमवाप्स्यसि} %1-62-20

\twolineshloka
{शुनःशेपो गृहीत्वा ते द्वे गाथे सुसमाहितः}
{त्वरया राजसिंहं तमम्बरीषमुवाच ह} %1-62-21

\twolineshloka
{राजसिंह महाबुद्धे शीघ्रं गच्छावहे वयम्}
{निवर्तयस्व राजेन्द्र दीक्षां च समुदाहर} %1-62-22

\twolineshloka
{तद् वाक्यमृषिपुत्रस्य श्रुत्वा हर्षसमन्वितः}
{जगाम नृपतिः शीघ्रं यज्ञवाटमतन्द्रितः} %1-62-23

\twolineshloka
{सदस्यानुमते राजा पवित्रकृतलक्षणम्}
{पशुं रक्ताम्बरं कृत्वा यूपे तं समबन्धयत्} %1-62-24

\twolineshloka
{स बद्धो वाग्भिरग्र्याभिरभितुष्टाव वै सुरौ}
{इन्द्रमिन्द्रानुजं चैव यथावन्मुनिपुत्रकः} %1-62-25

\twolineshloka
{ततः प्रीतः सहस्राक्षो रहस्यस्तुतितोषितः}
{दीर्घमायुस्तदा प्रादाच्छुनःशेपाय वासवः} %1-62-26

\twolineshloka
{स च राजा नरश्रेष्ठ यज्ञस्य च समाप्तवान्}
{फलं बहुगुणं राम सहस्राक्षप्रसादजम्} %1-62-27

\twolineshloka
{विश्वामित्रोऽपि धर्मात्मा भूयस्तेपे महातपाः}
{पुष्करेषु नरश्रेष्ठ दशवर्षशतानि च} %1-62-28


॥इत्यार्षे श्रीमद्रामायणे वाल्मीकीये आदिकाव्ये बालकाण्डे अम्बरीशयज्ञः नाम द्विषष्ठितमः सर्गः ॥१-६२॥
