\sect{त्रिचत्वारिशः सर्गः — गङ्गावतरणम्}

\twolineshloka
{देवदेवे गते तस्मिन् सोऽङ्गुष्ठाग्रनिपीडिताम्}
{कृत्वा वसुमतीं राम वत्सरं समुपासत} %1-43-1

\twolineshloka
{अथ संवत्सरे पूर्णे सर्वलोकनमस्कृतः}
{उमापतिः पशुपती राजानमिदमब्रवीत्} %1-43-2

\twolineshloka
{प्रीतस्तेऽहं नरश्रेष्ठ करिष्यामि तव प्रियम्}
{शिरसा धारयिष्यामि शैलराजसुतामहम्} %1-43-3

\twolineshloka
{ततो हैमवती ज्येष्ठा सर्वलोकनमस्कृता}
{तदा सातिमहद्रूपं कृत्वा वेगं च दुःसहम्} %1-43-4

\twolineshloka
{आकाशादपतद् राम शिवे शिवशिरस्युत}
{अचिन्तयच्च सा देवी गङ्गा परमदुर्धरा} %1-43-5

\twolineshloka
{विशाम्यहं हि पातालं स्त्रोतसा गृह्य शङ्करम्}
{तस्यावलेपनं ज्ञात्वा क्रुद्धस्तु भगवान् हरः} %1-43-6

\twolineshloka
{तिरोभावयितुं बुद्धिं चक्रे त्रिनयनस्तदा}
{सा तस्मिन् पतिता पुण्या पुण्ये रुद्रस्य मूर्धनि} %1-43-7

\twolineshloka
{हिमवत्प्रतिमे राम जटामण्डलगह्वरे}
{सा कथञ्चिन्महीं गन्तुं नाशक्नोद् यत्नमास्थिता} %1-43-8

\twolineshloka
{नैव सा निर्गमं लेभे जटामण्डलमन्ततः}
{तत्रैवाबभ्रमद् देवी संवत्सरगणान् बहून्} %1-43-9

\twolineshloka
{तामपश्यत् पुनस्तत्र तपः परममास्थितः}
{स तेन तोषितश्चासीदत्यन्तं रघुनन्दन} %1-43-10

\twolineshloka
{विससर्ज ततो गङ्गां हरो बिन्दुसरः प्रति}
{तस्यां विसृज्यमानायां सप्त स्रोतांसि जज्ञिरे} %1-43-11

\twolineshloka
{ह्लादिनी पावनी चैव नलिनी च तथैव च}
{तिस्रः प्राचीं दिशं जग्मुर्गङ्गाः शिवजलाः शुभाः} %1-43-12

\twolineshloka
{सुचक्षुश्चैव सीता च सिन्धुश्चैव महानदी}
{तिस्रश्चैता दिशं जग्मुः प्रतीचीं तु दिशं शुभाः} %1-43-13

\twolineshloka
{सप्तमी चान्वगात् तासां भगीरथरथं तदा}
{भगीरथोऽपि राजर्षिर्दिव्यं स्यन्दनमास्थितः} %1-43-14

\twolineshloka
{प्रायादग्रे महातेजा गङ्गा तं चाप्यनुव्रजत्}
{गगनाच्छङ्करशिरस्ततो धरणिमागता} %1-43-15

\twolineshloka
{असर्पत जलं तत्र तीव्रशब्दपुरस्कृतम्}
{मत्स्यकच्छपसङ्घैश्च शिंशुमारगणैस्तथा} %1-43-16

\twolineshloka
{पतद्भिः पतितैश्चैव व्यरोचत वसुन्धरा}
{ततो देवर्षिगन्धर्वा यक्षसिद्धगणास्तथा} %1-43-17

\twolineshloka
{व्यलोकयन्त ते तत्र गगनाद् गां गतां तदा}
{विमानैर्नगराकारैर्हयैर्गजवरैस्तदा} %1-43-18

\twolineshloka
{पारिप्लवगताश्चापि देवतास्तत्र विष्ठिताः}
{तदद्भुतमिमं लोके गङ्गावतरमुत्तमम्} %1-43-19

\twolineshloka
{दिदृक्षवो देवगणाः समीयुरमितौजसः}
{सम्पतद्भिः सुरगणैस्तेषां चाभरणौजसा} %1-43-20

\twolineshloka
{शतादित्यमिवाभाति गगनं गततोयदम्}
{शिंशुमारोरगगणैर्मीनैरपि च चञ्चलैः} %1-43-21

\twolineshloka
{विद्युद्भिरिव विक्षिप्तैराकाशमभवत् तदा}
{पाण्डुरैः सलिलोत्पीडैः कीर्यमाणैः सहस्रधा} %1-43-22

\twolineshloka
{शारदाभ्रैरिवाकीर्णं गगनं हंससम्प्लवैः}
{क्वचिद् द्रुततरं याति कुटिलं क्वचिदायतम्} %1-43-23

\twolineshloka
{विनतं क्वचिदुद्भूतं क्वचिद् याति शनैः शनैः}
{सलिलेनैव सलिलं क्वचिदभ्याहतं पुनः} %1-43-24

\twolineshloka
{मुहुरूर्ध्वपथं गत्वा पपात वसुधां पुनः}
{तच्छङ्करशिरोभ्रष्टं भ्रष्टं भूमितले पुनः} %1-43-25

\twolineshloka
{व्यरोचत तदा तोयं निर्मलं गतकल्मषम्}
{तत्रर्षिगणगन्धर्वा वसुधातलवासिनः} %1-43-26

\twolineshloka
{भवाङ्गपतितं तोयं पवित्रमिति पस्पृशुः}
{शापात् प्रपतिता ये च गगनाद् वसुधातलम्} %1-43-27

\twolineshloka
{कृत्वा तत्राभिषेकं ते बभूवुर्गतकल्मषाः}
{धूपपापाः पुनस्तेन तोयेनाथ शुभान्विताः} %1-43-28

\twolineshloka
{पुनराकाशमाविश्य स्वाँल्लोकान् प्रतिपेदिरे}
{मुमुदे मुदितो लोकस्तेन तोयेन भास्वता} %1-43-29

\twolineshloka
{कृताभिषेको गङ्गायां बभूव गतकल्मषः}
{भगीरथो हि राजर्षिर्दिव्यं स्यन्दनमास्थितः} %1-43-30

\twolineshloka
{प्रायादग्रे महाराजास्तं गङ्गा पृष्ठतोऽन्वगात्}
{देवाः सर्षिगणाः सर्वे दैत्यदानवराक्षसाः} %1-43-31

\twolineshloka
{गन्धर्वयक्षप्रवराः सकिन्नरमहोरगाः}
{सर्वाश्चाप्सरसो राम भगीरथरथानुगाः} %1-43-32

\twolineshloka
{गङ्गामन्वगमन् प्रीताः सर्वे जलचराश्च ये}
{यतो भगीरथो राजा ततो गङ्गा यशस्विनी} %1-43-33

\twolineshloka
{जगाम सरितां श्रेष्ठा सर्वपापप्रणाशिनी}
{ततो हि यजमानस्य जह्नोरद्भुतकर्मणः} %1-43-34

\twolineshloka
{गङ्गा सम्प्लावयामास यज्ञवाटं महत्मनः}
{तस्यावलेपनं ज्ञात्वा क्रुद्धो जह्नुश्च राघव} %1-43-35

\twolineshloka
{अपिबत् तु जलं सर्वं गङ्गयाः परमाद्भुतम्}
{ततो देवाः सगन्घर्वा ऋषयश्च सुविस्मिताः} %1-43-36

\twolineshloka
{पूजयन्ति महात्मानं जह्नुं पुरुषसत्तमम्}
{गङ्गां चापि नयन्ति स्म दुहितृत्वे महात्मनः} %1-43-37

\twolineshloka
{ततस्तुष्टो महातेजाः श्रोत्राभ्यामसृजत् प्रभुः}
{तस्माज्जह्नुसुता गङ्गा प्रोच्यते जाह्नवीति च} %1-43-38

\twolineshloka
{जगाम च पुनर्गङ्गा भगीरथरथानुगा}
{सागरं चापि सम्प्रप्ता सा सरित्प्रवरा तदा} %1-43-39

\twolineshloka
{रसातलमुपागच्छत् सिद्ध्यर्थं तस्य कर्मणः}
{भगीरथोऽपि राजार्षिर्गङ्गामादाय यत्नतः} %1-43-40

\threelineshloka
{पितमहान् भस्मकृतानपश्यद् गतचेतनः}
{अथ तद्भस्मनां राशिं गङ्गासलिलमुत्तमम्}
{प्लावयत् पूतपाप्मानः स्वर्गं प्राप्ता रघूत्तम} %1-43-41


॥इत्यार्षे श्रीमद्रामायणे वाल्मीकीये आदिकाव्ये बालकाण्डे गङ्गावतरणम् नाम त्रिचत्वारिशः सर्गः ॥१-४३॥
