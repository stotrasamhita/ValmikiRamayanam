\sect{द्वितीयः सर्गः — ब्रह्मागमनम्}

\twolineshloka
{नारदस्य तु तद्वाक्यं श्रुत्वा वाक्यविशारदः}
{पूजयामास धर्मात्मा सहशिष्यो महामुनिम्} %1-2-1

\twolineshloka
{यथावत्पूजितस्तेन देवर्षिर्नारदस्तदा}
{आपृष्ट्वैवाभ्यनुज्ञातः स जगाम विहायसम्} %1-2-2

\twolineshloka
{स मुहूर्तं गते तस्मिन् देवलोकं मुनिस्तदा}
{जगाम तमसातीरं जाह्नव्यास्त्वविदूरतः} %1-2-3

\twolineshloka
{स तु तीरं समासाद्य तमसाया मुनिस्तदा}
{शिष्यमाह स्थितं पार्श्वे दृष्ट्वा तीर्थमकर्दमम्} %1-2-4

\twolineshloka
{अकर्दममिदं तीर्थं भरद्वाज निशामय}
{रमणीयं प्रसन्नाम्बु सन्मनुष्यमनो यथा} %1-2-5

\twolineshloka
{न्यस्यतां कलशस्तात दीयतां वल्कलं मम}
{इदमेवावगाहिष्ये तमसातीर्थमुत्तमम्} %1-2-6

\twolineshloka
{एवमुक्तो भरद्वाजो वाल्मीकेन महात्मना}
{प्रायच्छत मुनेस्तस्य वल्कलं नियतो गुरोः} %1-2-7

\twolineshloka
{स शिष्यहस्तादादाय वल्कलं नियतेन्द्रियः}
{विचचार ह पश्यंस्तत् सर्वतो विपुलं वनम्} %1-2-8

\twolineshloka
{तस्याभ्याशे तु मिथुनं चरन्तमनपायिनम्}
{ददर्श भगवांस्तत्र क्रौञ्चयोश्चारुनिःस्वनम्} %1-2-9

\twolineshloka
{तस्मात् तु मिथुनादेकं पुमांसं पापनिश्चयः}
{जघान वैरनिलयो निषादस्तस्य पश्यतः} %1-2-10

\twolineshloka
{तं शोणितपरीताङ्गं चेष्टमानं महीतले}
{भार्या तु निहतं दृष्ट्वा रुराव करुणां गिरम्} %1-2-11

\twolineshloka
{वियुक्ता पतिना तेन द्विजेन सहचारिणा}
{ताम्रशीर्षेण मत्तेन पत्रिणा सहितेन वै} %1-2-12

\twolineshloka
{तथाविधं द्विजं दृष्ट्वा निषादेन निपातितम्}
{ऋषेर्धर्मात्मनस्तस्य कारुण्यं समपद्यत} %1-2-13

\twolineshloka
{ततः करुणवेदित्वादधर्मोऽयमिति द्विजः}
{निशाम्य रुदतीं क्रौञ्चीमिदं वचनमब्रवीत्} %1-2-14

\twolineshloka
{मा निषाद प्रतिष्ठां त्वमगमः शाश्वतीः समाः}
{यत् क्रौञ्चमिथुनादेकमवधीः काममोहितम्} %1-2-15

\twolineshloka
{तस्यैवं ब्रुवतश्चिन्ता बभूव हृदि वीक्षतः}
{शोकार्तेनास्य शकुनेः किमिदं व्याहृतं मया} %1-2-16

\twolineshloka
{चिन्तयन् स महाप्राज्ञश्चकार मतिमान्मतिम्}
{शिष्यं चैवाब्रवीद् वाक्यमिदं स मुनिपुङ्गवः} %1-2-17

\twolineshloka
{पादबद्धोऽक्षरसमस्तन्त्रीलयसमन्वितः}
{शोकार्तस्य प्रवृत्तो मे श्लोको भवतु नान्यथा} %1-2-18

\twolineshloka
{शिष्यस्तु तस्य ब्रुवतो मुनेर्वाक्यमनुत्तमम्}
{प्रतिजग्राह संहृष्टस्तस्य तुष्टोऽभवद्गुरुः} %1-2-19

\twolineshloka
{सोऽभिषेकं ततः कृत्वा तीर्थे तस्मिन् यथाविधि}
{तमेव चिन्तयन्नर्थमुपावर्तत वै मुनिः} %1-2-20

\twolineshloka
{भरद्वाजस्ततः शिष्यो विनीतः श्रुतवान् गुरोः}
{कलशं पूर्णमादाय पृष्ठतोऽनुजगाम ह} %1-2-21

\twolineshloka
{स प्रविश्याश्रमपदं शिष्येण सह धर्मवित्}
{उपविष्टः कथाश्चान्याश्चकार ध्यानमास्थितः} %1-2-22

\twolineshloka
{आजगाम ततो ब्रह्मा लोककर्ता स्वयं प्रभुः}
{चतुर्मुखो महातेजा द्रष्टुं तं मुनिपुङ्गवम्} %1-2-23

\twolineshloka
{वाल्मीकिरथ तं दृष्ट्वा सहसोत्थाय वाग्यतः}
{प्राञ्जलिः प्रयतो भूत्वा तस्थौ परमविस्मितः} %1-2-24

\twolineshloka
{पूजयामास तं देवं पाद्यार्घ्यासनवन्दनैः}
{प्रणम्य विधिवच्चैनं पृष्ट्वा चैव निरामयम्} %1-2-25

\twolineshloka
{अथोपविश्य भगवानासने परमार्चिते}
{वाल्मीकये च ऋषये सन्दिदेशासनं ततः} %1-2-26

\twolineshloka
{ब्रह्मणा समनुज्ञातः सोऽप्युपाविशदासने}
{उपविष्टे तदा तस्मिन् साक्षाल्लोकपितामहे} %1-2-27

\twolineshloka
{तद्गतेनैव मनसा वाल्मीकिर्ध्यानमास्थितः}
{पापात्मना कृतं कष्टं वैरग्रहणबुद्धिना} %1-2-28

\twolineshloka
{यत् तादृशं चारुरवं क्रौञ्चं हन्यादकारणात्}
{शोचन्नेव पुनः क्रौञ्चीमुपश्लोकमिमं जगौ} %1-2-29

\twolineshloka
{पुनरन्तर्गतमना भूत्वा शोकपरायणः}
{तमुवाच ततो ब्रह्मा प्रहसन् मुनिपुङ्गवम्} %1-2-30

\twolineshloka
{श्लोक एवास्त्वयं बद्धो नात्र कार्या विचारणा}
{मच्छन्दादेव ते ब्रह्मन् प्रवृत्तेयं सरस्वती} %1-2-31

\twolineshloka
{रामस्य चरितं कृत्स्नं कुरु त्वमृषिसत्तम}
{धर्मात्मनो भगवतो लोके रामस्य धीमतः} %1-2-32

\twolineshloka
{वृत्तं कथय धीरस्य यथा ते नारदाच्छ्रुतम्}
{रहस्यं च प्रकाशं च यद् वृत्तं तस्य धीमतः} %1-2-33

\twolineshloka
{रामस्य सहसौमित्रे राक्षसानां च सर्वशः}
{वैदेह्याश्चैव यद् वृत्तं प्रकाशं यदि वा रहः} %1-2-34

\twolineshloka
{तच्चाप्यविदितं सर्वं विदितं ते भविष्यति}
{न ते वागनृता काव्ये काचिदत्र भविष्यति} %1-2-35

\twolineshloka
{कुरु रामकथां पुण्यां श्लोकबद्धां मनोरमाम्}
{यावत् स्थास्यन्ति गिरयः सरितश्च महीतले} %1-2-36

\twolineshloka
{तावद् रामायणकथा लोकेषु प्रचरिष्यति}
{यावद् रामस्य च कथा त्वत्कृता प्रचरिष्यति} %1-2-37

\threelineshloka
{तावदूर्ध्वमधश्च त्वं मल्लोकेषु निवत्स्यसि}
{इत्युक्त्वा भगवान् ब्रह्मा तत्रैवान्तरधीयत}
{ततः सशिष्यो भगवान् र्मुनिर्विस्मयमाययौ} %1-2-38

\twolineshloka
{तस्य शिष्यास्ततः सर्वे जगुः श्लोकमिमं पुनः}
{मुहुर्मुहुः प्रीयमाणाः प्राहुश्च भृशविस्मिताः} %1-2-39

\twolineshloka
{समाक्षरैश्चतुर्भिर्यः पादैर्गीतो महर्षिणा}
{सोऽनुव्याहरणाद् भूयः शोकः श्लोकत्वमागतः} %1-2-40

\twolineshloka
{तस्य बुद्धिरियं जाता वाल्मीकेर्भावितात्मनः}
{कृत्स्नं रामायणं काव्यमीदृशैः करवाण्यहम्} %1-2-41

\twolineshloka
{उदारवृत्तार्थपदैर्मनोरमैस्तदास्य रामस्य चकार कीर्तिमान्}
{समाक्षरैः श्लोकशतैर्यशस्विनो यशस्करं काव्यमुदारदर्शनः} %1-2-42

\fourlineindentedshloka
{तदुपगतसमाससन्धियोगं}
{सममधुरोपनतार्थवाक्यबद्धम्}
{रघुवरचरितं मुनिप्रणीतं}
{दशशिरसश्च वधं निशामयध्वम्} %1-2-43


॥इत्यार्षे श्रीमद्रामायणे वाल्मीकीये आदिकाव्ये बालकाण्डे ब्रह्मागमनम् नाम द्वितीयः सर्गः ॥१-२॥
