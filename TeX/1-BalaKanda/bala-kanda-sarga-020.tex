\sect{विंशः सर्गः — दशरथवाक्यम्}

\twolineshloka
{तच्छ्रुत्वा राजशार्दूलो विश्वामित्रस्य भाषितम्}
{मुहूर्तमिव निःसंज्ञः संज्ञावानिदमब्रवीत्} %1-20-1

\twolineshloka
{ऊनषोडशवर्षो मे रामो राजीवलोचनः}
{न युद्धयोग्यतामस्य पश्यामि सह राक्षसैः} %1-20-2

\twolineshloka
{इयमक्षौहिणी सेना यस्याहं पतिरीश्वरः}
{अनया सहितो गत्वा योद्धाहं तैर्निशाचरैः} %1-20-3

\twolineshloka
{इमे शूराश्च विक्रान्ता भृत्या मेऽस्त्रविशारदाः}
{योग्या रक्षोगणैर्योद्धुं न रामं नेतुमर्हसि} %1-20-4

\twolineshloka
{अहमेव धनुष्पाणिर्गोप्ता समरमूर्धनि}
{यावत् प्राणान् धरिष्यामि तावद्योत्स्ये निशाचरैः} %1-20-5

\twolineshloka
{निर्विघ्ना व्रतचर्या सा भविष्यति सुरक्षिता}
{अहं तत्र गमिष्यामि न रामं नेतुमर्हसि} %1-20-6

\twolineshloka
{बालो ह्यकृतविद्यश्च न च वेत्ति बलाबलम्}
{न चास्त्रबलसंयुक्तो न च युद्धविशारदः} %1-20-7

\twolineshloka
{न चासौ रक्षसां योग्यः कूटयुद्धा हि राक्षसाः}
{विप्रयुक्तो हि रामेण मुहूर्तमपि नोत्सहे} %1-20-8

\twolineshloka
{जीवितुं मुनिशार्दूल न रामं नेतुमर्हसि}
{यदि वा राघवं ब्रह्मन् नेतुमिच्छसि सुव्रत} %1-20-9

\twolineshloka
{चतुरङ्गसमायुक्तं मया सह च तं नय}
{षष्टिर्वर्षसहस्राणि जातस्य मम कौशिक} %1-20-10

\twolineshloka
{कृच्छ्रेणोत्पादितश्चायं न रामं नेतुमर्हसि}
{चतुर्णामात्मजानां हि प्रीतिः परमिका मम} %1-20-11

\twolineshloka
{ज्येष्ठे धर्मप्रधानेच न रामं नेतुमर्हसि}
{किंवीर्या राक्षसास्ते च कस्य पुत्राश्च के च ते} %1-20-12

\twolineshloka
{कथं प्रमाणाः के चैतान् रक्षन्ति मुनिपुंगव}
{कथं च प्रतिकर्तव्यं तेषां रामेण रक्षसाम्} %1-20-13

\twolineshloka
{मामकैर्वा बलैर्ब्रह्मन् मया वा कूटयोधिनाम्}
{सर्वं मे शंस भगवन् कथं तेषां मया रणे} %1-20-14

\twolineshloka
{स्थातव्यं दुष्टभावानां वीर्योत्सिक्ता हि राक्षसाः}
{तस्य तद् वचनं श्रुत्वा विश्वामित्रोऽभ्यभाषत} %1-20-15

\twolineshloka
{पौलस्त्यवंशप्रभवो रावणो नाम राक्षसः}
{स ब्रह्मणा दत्तवरस्त्रैलोक्यं बाधते भृशम्} %1-20-16

\twolineshloka
{महाबलो महावीर्यो राक्षसैर्बहुभिर्वृतः}
{श्रूयते च महाराज रावणो राक्षसाधिपः} %1-20-17

\twolineshloka
{साक्षाद्वैश्रवणभ्राता पुत्रो विश्रवसो मुनेः}
{यदा न खलु यज्ञस्य विघ्नकर्ता महाबलः} %1-20-18

\twolineshloka
{तेन संचोदितौ तौ तु राक्षसौ सुमहाबलौ}
{मारीचश्च सुबाहुश्च यज्ञविघ्नं करिष्यतः} %1-20-19

\twolineshloka
{इत्युक्तो मुनिना तेन राजोवाच मुनिं तदा}
{नहि शक्तोऽस्मि संग्रामे स्थातुं तस्य दुरात्मनः} %1-20-20

\twolineshloka
{स त्वं प्रसादं धर्मज्ञ कुरुष्व मम पुत्रके}
{मम चैवाल्पभाग्यस्य दैवतं हि भवान् गुरुः} %1-20-21

\twolineshloka
{देवदानवगन्धर्वा यक्षाः पतगपन्नगाः}
{न शक्ता रावणं सोढुं किं पुनर्मानवा युधि} %1-20-22

\twolineshloka
{स तु वीर्यवतां वीर्यमादत्ते युधि रावणः}
{तेन चाहं न शक्तोऽस्मि संयोद्धुं तस्य वा बलैः} %1-20-23

\twolineshloka
{सबलो वा मुनिश्रेष्ठ सहितो वा ममात्मजैः}
{कथमप्यमरप्रख्यं संग्रामाणामकोविदम्} %1-20-24

\twolineshloka
{बालं मे तनयं ब्रह्मन् नैव दास्यामि पुत्रकम्}
{अथ कालोपमौ युद्धे सुतौ सुन्दोपसुन्दयोः} %1-20-25

\twolineshloka
{यज्ञविघ्नकरौ तौ ते नैव दास्यामि पुत्रकम्}
{मारीचश्च सुबाहुश्च वीर्यवन्तौ सुशिक्षितौ} %1-20-26

\twolineshloka
{तयोरन्यतरं योद्धुं यास्यामि ससुहृद्गणः}
{अन्यथा त्वनुनेष्यामि भवन्तं सहबान्धवः} %1-20-27

\twolineshloka
{इति नरपतिजल्पनात् द्विजेन्द्रं कुशिकसुतं सुमहान् विवेश मन्युः}
{सुहुत इव मखेऽग्निराज्यसिक्तः समभवदुज्ज्वलितो महर्षिवह्निः} %1-20-28


॥इत्यार्षे श्रीमद्रामायणे वाल्मीकीये आदिकाव्ये बालकाण्डे दशरथवाक्यम् नाम विंशः सर्गः ॥१-२०॥
