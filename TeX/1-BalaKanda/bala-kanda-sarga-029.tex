\sect{एकोनत्रिंशः सर्गः — सिद्धाश्रमः}

\twolineshloka
{अथ तस्याप्रमेयस्य वचनं परिपृच्छतः}
{विश्वामित्रो महातेजा व्याख्यातुमुपचक्रमे} %1-29-1

\twolineshloka
{इह राम महाबाहो विष्णुर्देवनमस्कृतः}
{वर्षाणि सुबहूनीह तथा युगशतानि च} %1-29-2

\twolineshloka
{तपश्चरणयोगार्थमुवास सुमहातपाः}
{एष पूर्वाश्रमो राम वामनस्य महात्मनः} %1-29-3

\twolineshloka
{सिद्धाश्रम इति ख्यातः सिद्धो ह्यत्र महातपाः}
{एतस्मिन्नेव काले तु राजा वैरोचनिर्बलिः} %1-29-4

\twolineshloka
{निर्जित्य दैवतगणान् सेन्द्रान् सहमरुद्गणान्}
{कारयामास तद्राज्यं त्रिषु लोकेषु विश्रुतः} %1-29-5

\threelineshloka
{यज्ञं चकार सुमहानसुरेन्द्रो महाबलः}
{बलेस्तु यजमानस्य देवाः साग्निपुरोगमाः}
{समागम्य स्वयं चैव विष्णुमूचुरिहाश्रमे} %1-29-6

\twolineshloka
{बलिर्वैरोचनिर्विष्णो यजते यज्ञमुत्तमम्}
{असमाप्तव्रते तस्मिन् स्वकार्यमभिपद्यताम्} %1-29-7

\twolineshloka
{ये चैनमभिवर्तन्ते याचितार इतस्ततः}
{यच्च यत्र यथावच्च सर्वं तेभ्यः प्रयच्छति} %1-29-8

\twolineshloka
{स त्वं सुरहितार्थाय मायायोगमुपाश्रितः}
{वामनत्वं गतो विष्णो कुरु कल्याणमुत्तमम्} %1-29-9

\twolineshloka
{एतस्मिन्नन्तरे राम कश्यपोग्निसमप्रभः}
{अदित्या सहितो राम दीप्यमान इवौजसा} %1-29-10

\twolineshloka
{देवीसहायो भगवान् दिव्यं वर्षसहस्रकम्}
{व्रतं समाप्य वरदं तुष्टाव मधुसूदनम्} %1-29-11

\twolineshloka
{तपोमयं तपोराशिं तपोमूर्तिं तपात्मकम्}
{तपसा त्वां सुतप्तेन पश्यामि पुरुषोत्तमम्} %1-29-12

\twolineshloka
{शरीरे तव पश्यामि जगत् सर्वमिदं प्रभो}
{त्वमनादिरनिर्देश्यस्त्वामहं शरणं गतः} %1-29-13

\twolineshloka
{तमुवाच हरिः प्रीतः कश्यपं गतकल्मषम्}
{वरं वरय भद्रं ते वरार्होऽसि मतो मम} %1-29-14

\twolineshloka
{तच्छ्रुत्वा वचनं तस्य मारीचः कश्यपोऽब्रवीत्}
{अदित्या देवतानां च मम चैवानुयाचितम्} %1-29-15

\twolineshloka
{वरं वरद सुप्रीतो दातुमर्हसि सुव्रत}
{पुत्रत्वं गच्छ भगवन्नदित्या मम चानघ} %1-29-16

\twolineshloka
{भ्राता भव यवीयांस्त्वं शक्रस्यासुरसूदन}
{शोकार्तानां तु देवानां साहाय्यं कर्तुमर्हसि} %1-29-17

\twolineshloka
{अयं सिद्धाश्रमो नाम प्रसादात् ते भविष्यति}
{सिद्धे कर्मणि देवेश उत्तिष्ठ भगवन्नितः} %1-29-18

\twolineshloka
{अथ विष्णुर्महातेजा अदित्यां समजायत}
{वामनं रूपमास्थाय वैरोचनिमुपागमत्} %1-29-19

\twolineshloka
{त्रीन् पदानथ भिक्षित्वा प्रतिगृह्य च मेदिनीम्}
{आक्रम्य लोकान् लोकार्थी सर्वलोकहिते रतः} %1-29-20

\twolineshloka
{महेन्द्राय पुनः प्रादान्नियम्य बलिमोजसा}
{त्रैलोक्यं स महातेजाश्चक्रे शक्रवशं पुनः} %1-29-21

\twolineshloka
{तेनैव पूर्वमाक्रान्त आश्रमः श्रमनाशनः}
{मयापि भक्त्या तस्यैव वामनस्योपभुज्यते} %1-29-22

\twolineshloka
{एनमाश्रममायान्ति राक्षसा विघ्नकारिणः}
{अत्र ते पुरुषव्याघ्र हन्तव्या दुष्टचारिणः} %1-29-23

\twolineshloka
{अद्य गच्छामहे राम सिद्धाश्रममनुत्तमम्}
{तदाश्रमपदं तात तवाप्येतद् यथा मम} %1-29-24

\threelineshloka
{इत्युक्त्वा परमप्रीतो गृह्य रामं सलक्ष्मणम्}
{प्रविशन्नाश्रमपदं व्यरोचत महामुनिः}
{शशीव गतनीहारः पुनर्वसुसमन्वितः} %1-29-25

\twolineshloka
{तं दृष्ट्वा मुनयः सर्वे सिद्धाश्रमनिवासिनः}
{उत्पत्योत्पत्य सहसा विश्वामित्रमपूजयन्} %1-29-26

\twolineshloka
{यथार्हं चक्रिरे पूजां विश्वामित्राय धीमते}
{तथैव राजपुत्राभ्यामकुर्वन्नतिथिक्रियाम्} %1-29-27

\twolineshloka
{मुहूर्तमथ विश्रान्तौ राजपुत्रावरिन्दमौ}
{प्राञ्जली मुनिशार्दूलमूचतू रघुनन्दनौ} %1-29-28

\twolineshloka
{अद्यैव दीक्षां प्रविश भद्रं ते मुनिपुङ्गव}
{सिद्धाश्रमोऽयं सिद्धः स्यात् सत्यमस्तु वचस्तव} %1-29-29

\twolineshloka
{एवमुक्तो महातेजा विश्वामित्रो महानृषिः}
{प्रविवेश तदा दीक्षां नियतो नियतेन्द्रियः} %1-29-30

\twolineshloka
{कुमारावपि तां रात्रिमुषित्वा सुसमाहितौ}
{प्रभातकाले चोत्थाय पूर्वां सन्ध्यामुपास्य च} %1-29-31

\twolineshloka
{प्रशुची परमं जाप्यं समाप्य नियमेन च}
{हुताग्निहोत्रमासीनं विश्वामित्रमवन्दताम्} %1-29-32


॥इत्यार्षे श्रीमद्रामायणे वाल्मीकीये आदिकाव्ये बालकाण्डे सिद्धाश्रमः नाम एकोनत्रिंशः सर्गः ॥१-२९॥
