\sect{एकोनचत्वारिंशः सर्गः — पृथिवीविदारणम्}

\twolineshloka
{विश्वामित्रवचः श्रुत्वा कथान्ते रघुनन्दनः}
{उवाच परमप्रीतो मुनिं दीप्तमिवानलम्} %1-39-1

\twolineshloka
{श्रोतुमिच्छामि भद्रं ते विस्तरेण कथामिमाम्}
{पूर्वजो मे कथं ब्रह्मन् यज्ञं वै समुपाहरत्} %1-39-2

\twolineshloka
{तस्य तद् वचनं श्रुत्वा कौतूहलसमन्वितः}
{विश्वामित्रस्तु काकुत्स्थमुवाच प्रहसन्निव} %1-39-3

\twolineshloka
{श्रूयतां विस्तरो राम सगरस्य महात्मनः}
{शङ्करश्वशुरो नाम्ना हिमवानिति विश्रुतः} %1-39-4

\twolineshloka
{विन्ध्यपर्वतमासाद्य निरीक्षेते परस्परम्}
{तयोर्मध्ये समभवद् यज्ञः स पुरुषोत्तम} %1-39-5

\twolineshloka
{स हि देशो नरव्याघ्र प्रशस्तो यज्ञकर्मणि}
{तस्याश्वचर्यां काकुत्स्थ दृढधन्वा महारथः} %1-39-6

\twolineshloka
{अंशुमानकरोत् तात सगरस्य मते स्थितः}
{तस्य पर्वणि तं यज्ञं यजमानस्य वासवः} %1-39-7

\twolineshloka
{राक्षसीं तनुमास्थाय यज्ञियाश्वमपाहरत्}
{ह्रियमाणे तु काकुत्स्थ तस्मिन्नश्वे महात्मनः} %1-39-8

\twolineshloka
{उपाध्यायगणाः सर्वे यजमानमथाब्रुवन्}
{अयं पर्वणि वेगेन यज्ञियाश्वोऽपनीयते} %1-39-9

\twolineshloka
{हर्तारं जहि काकुत्स्थ हयश्चैवोपनीयताम्}
{यज्ञच्छिद्रं भवत्येतत् सर्वेषामशिवाय नः} %1-39-10

\twolineshloka
{तत् तथा क्रियतां राजन् यज्ञोच्छिद्रः कृतो भवेत्}
{सोपाध्यायवचः श्रुत्वा तस्मिन् सदसि पार्थिवः} %1-39-11

\twolineshloka
{षष्टिं पुत्रसहस्राणि वाक्यमेतदुवाच ह}
{गतिं पुत्रा न पश्यामि रक्षसां पुरुषर्षभाः} %1-39-12

\twolineshloka
{मन्त्रपूतैर्महाभागैरास्थितो हि महाक्रतुः}
{तद् गच्छत विचिन्वध्वं पुत्रका भद्रमस्तु वः} %1-39-13

\twolineshloka
{समुद्रमालिनीं सर्वां पृथिवीमनुगच्छथ}
{एकैकं योजनं पुत्रा विस्तारमभिगच्छत} %1-39-14

\twolineshloka
{यावत् तुरगसन्दर्शस्तावत् खनत मेदिनीम्}
{तमेव हयहर्तारं मार्गमाणा ममाज्ञया} %1-39-15

\twolineshloka
{दीक्षितः पौत्रसहितः सोपाध्यायगणस्त्वहम्}
{इह स्थास्यामि भद्रं वो यावत् तुरगदर्शनम्} %1-39-16

\twolineshloka
{ते सर्वे हृष्टमनसो राजपुत्रा महाबलाः}
{जग्मुर्महीतलं राम पितुर्वचनयन्त्रिताः} %1-39-17

\threelineshloka
{गत्वा तु पृथिवीं सर्वामदृष्ट्वा तं महाबलाः}
{योजनायामविस्तारमेकैको धरणीतलम्}
{बिभिदुः पुरुषव्याघ्रा वज्रस्पर्शसमैर्भुजैः} %1-39-18

\twolineshloka
{शूलैरशनिकल्पैश्च हलैश्चापि सुदारुणैः}
{भिद्यमाना वसुमती ननाद रघुनन्दन} %1-39-19

\twolineshloka
{नागानां वध्यमानानामसुराणां च राघव}
{राक्षसानां दुराधर्षं सत्त्वानां निनदोऽभवत्} %1-39-20

\twolineshloka
{योजनानां सहस्राणि षष्टिं तु रघुनन्दन}
{बिभिदुर्धरणीं राम रसातलमनुत्तमम्} %1-39-21

\twolineshloka
{एवं पर्वतसम्बाधं जम्बूद्वीपं नृपात्मजाः}
{खनन्तो नृपशार्दूल सर्वतः परिचक्रमुः} %1-39-22

\twolineshloka
{ततो देवाः सगन्धर्वाः सासुराः सहपन्नगाः}
{सम्भ्रान्तमनसः सर्वे पितामहमुपागमन्} %1-39-23

\twolineshloka
{ते प्रसाद्य महात्मानं विषण्णवदनास्तदा}
{ऊचुः परमसन्त्रस्ताः पितामहमिदं वचः} %1-39-24

\twolineshloka
{भगवन् पृथिवी सर्वा खन्यते सगरात्मजैः}
{बहवश्च महात्मानो वध्यन्ते जलचारिणः} %1-39-25

\twolineshloka
{अयं यज्ञहरोऽस्माकमनेनाश्वोऽपनीयते}
{इति ते सर्वभूतानि हिंसन्ति सगरात्मजाः} %1-39-26


॥इत्यार्षे श्रीमद्रामायणे वाल्मीकीये आदिकाव्ये बालकाण्डे पृथिवीविदारणम् नाम एकोनचत्वारिंशः सर्गः ॥१-३९॥
