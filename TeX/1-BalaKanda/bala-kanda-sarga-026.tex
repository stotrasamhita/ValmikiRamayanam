\sect{षड्विंशः सर्गः — ताटकावधः}

\twolineshloka
{मुनेर्वचनमक्लीबं श्रुत्वा नरवरात्मजः}
{राघवः प्राञ्जलिर्भूत्वा प्रत्युवाच दृढव्रतः} %1-26-1

\twolineshloka
{पितुर्वचननिर्देशात् पितुर्वचनगौरवात्}
{वचनं कौशिकस्येति कर्तव्यमविशङ्कया} %1-26-2

\twolineshloka
{अनुशिष्टोऽस्म्ययोध्यायां गुरुमध्ये महात्मना}
{पित्रा दशरथेनाहं नावज्ञेयं हि तद्वचः} %1-26-3

\twolineshloka
{सोऽहं पितुर्वचः श्रुत्वा शासनाद् ब्रह्मवादिनः}
{करिष्यामि न सन्देहस्ताटकावधमुत्तमम्} %1-26-4

\twolineshloka
{गोब्राह्मणहितार्थाय देशस्य च हिताय च}
{तव चैवाप्रमेयस्य वचनं कर्तुमुद्यतः} %1-26-5

\twolineshloka
{एवमुक्त्वा धनुर्मध्ये बद्ध्वा मुष्टिमरिन्दमः}
{ज्याघोषमकरोत् तीव्रं दिशः शब्देन नादयन्} %1-26-6

\twolineshloka
{तेन शब्देन वित्रस्तास्ताटकावनवासिनः}
{ताटका च सुसङ्क्रुद्धा तेन शब्देन मोहिता} %1-26-7

\twolineshloka
{तं शब्दमभिनिध्याय राक्षसी क्रोधमूर्छिता}
{श्रुत्वा चाभ्यद्रवत् क्रुद्धा यत्र शब्दो विनिस्सृतः} %1-26-8

\twolineshloka
{तां दृष्ट्वा राघवः क्रुद्धां विकृतां विकृताननाम्}
{प्रमाणेनातिवृद्धां च लक्ष्मणं सोऽभ्यभाषत} %1-26-9

\twolineshloka
{पश्य लक्ष्मण यक्षिण्या भैरवं दारुणं वपुः}
{भिद्येरन् दर्शनादस्या भीरूणां हृदयानि च} %1-26-10

\twolineshloka
{एतां पश्य दुराधर्षां मायाबलसमन्विताम्}
{विनिवृत्तां करोम्यद्य हृतकर्णाग्रनासिकाम्} %1-26-11

\twolineshloka
{न ह्येनामुत्सहे हन्तुं स्त्रीस्वभावेन रक्षिताम्}
{वीर्यं चास्या गतिं चैव हन्यामिति हि मे मतिः} %1-26-12

\twolineshloka
{एवं ब्रुवाणे रामे तु ताटका क्रोधमूर्छिता}
{उद्यम्य बाहू गर्जन्ती राममेवाभ्यधावत} %1-26-13

\twolineshloka
{विश्वामित्रस्तु ब्रह्मर्षिर्हुङ्कारेणाभिभर्त्स्य ताम्}
{स्वस्ति राघवयोरस्तु जयं चैवाभ्यभाषत} %1-26-14

\twolineshloka
{उद्धुन्वाना रजो घोरं ताटका राघवावुभौ}
{रजोमेघेन महता मुहूर्तं सा व्यमोहयत्} %1-26-15

\twolineshloka
{ततो मायां समास्थाय शिलावर्षेण राघवौ}
{अवाकिरत् सुमहता ततश्चुक्रोध राघवः} %1-26-16

\twolineshloka
{शिलावर्षं महत् तस्याः शरवर्षेण राघवः}
{प्रतिवार्योपधावन्त्याः करौ चिच्छेद पत्रिभिः} %1-26-17

\twolineshloka
{ततश्च्छिन्नभुजां श्रान्तामभ्याशे परिगर्जतीम्}
{सौमित्रिरकरोत् क्रोधाद्धृतकर्णाग्रनासिकाम्} %1-26-18

\twolineshloka
{कामरूपधरा सा तु कृत्वा रूपाण्यनेकशः}
{अन्तर्धानं गता यक्षी मोहयन्ती स्वमायया} %1-26-19

\twolineshloka
{अश्मवर्षं विमुञ्चन्ती भैरवं विचचार सा}
{ततस्तावश्मवर्षेण कीर्यमाणौ समन्ततः} %1-26-20

\twolineshloka
{दृष्ट्वा गाधिसुतः श्रीमानिदं वचनमब्रवीत्}
{अलं ते घृणया राम पापैषा दुष्टचारिणी} %1-26-21

\twolineshloka
{यज्ञविघ्नकरी यक्षी पुरा वर्धेत मायया}
{वध्यतां तावदेवैषा पुरा सन्ध्या प्रवर्तते} %1-26-22

\twolineshloka
{रक्षांसि सन्ध्याकाले तु दुर्धर्षाणि भवन्ति हि}
{इत्युक्तः स तु तां यक्षीमश्मवृष्ट्याभिवर्षिणीम्} %1-26-23

\twolineshloka
{दर्शयन् शब्दवेधित्वं तां रुरोध स सायकैः}
{सा रुद्धा बाणजालेन मायाबलसमन्विता} %1-26-24

\twolineshloka
{अभिदुद्राव काकुत्स्थं लक्ष्मणं च विनेषुदी}
{तामापतन्तीं वेगेन विक्रान्तामशनीमिव} %1-26-25

\twolineshloka
{शरेणोरसि विव्याध सा पपात ममार च}
{तां हतां भीमसङ्काशां दृष्ट्वा सुरपतिस्तदा} %1-26-26

\twolineshloka
{साधु साध्विति काकुत्स्थं सुराश्चाप्यभिपूजयन्}
{उवाच परमप्रीतः सहस्राक्षः पुरन्दरः} %1-26-27

\twolineshloka
{सुराश्च सर्वे संहृष्टा विश्वामित्रमथाब्रुवन्}
{मुने कौशिक भद्रं ते सेन्द्राः सर्वे मरुद्गणाः} %1-26-28

\twolineshloka
{तोषिताः कर्मणानेन स्नेहं दर्शय राघवे}
{प्रजापतेः कृशाश्वस्य पुत्रान् सत्यपराक्रमान्} %1-26-29

\twolineshloka
{तपोबलभृतो ब्रह्मन् राघवाय निवेदय}
{पात्रभूतश्च ते ब्रह्मंस्तवानुगमने रतः} %1-26-30

\twolineshloka
{कर्तव्यं सुमहत् कर्म सुराणां राजसूनुना}
{एवमुक्त्वा सुराः सर्वे जग्मुर्हृष्टा विहायसम्} %1-26-31

\twolineshloka
{विश्वामित्रं पूजयन्तस्ततः सन्ध्या प्रवर्तते}
{ततो मुनिवरः प्रीतस्ताटकावधतोषितः} %1-26-32

\twolineshloka
{मूर्ध्नि राममुपाघ्राय इदं वचनमब्रवीत्}
{इहाद्य रजनीं राम वसाम शुभदर्शन} %1-26-33

\twolineshloka
{श्वः प्रभाते गमिष्यामस्तदाश्रमपदं मम}
{विश्वामित्रवचः श्रुत्वा हृष्टो दशरथात्मजः} %1-26-34

\threelineshloka
{उवास रजनीं तत्र ताटकाया वने सुखम्}
{मुक्तशापं वनं तच्च तस्मिन्नेव तदाहनि}
{रमणीयं विबभ्राज यथा चैत्ररथं वनम्} %1-26-35

\twolineshloka
{निहत्य तां यक्षसुतां स रामः प्रशस्यमानः सुरसिद्धसङ्घैः}
{उवास तस्मिन् मुनिना सहैव प्रभातवेलां प्रतिबोध्यमानः} %1-26-36


॥इत्यार्षे श्रीमद्रामायणे वाल्मीकीये आदिकाव्ये बालकाण्डे ताटकावधः नाम षड्विंशः सर्गः ॥१-२६॥
