\sect{त्रिपञ्चाशः सर्गः — शबलानिष्क्रयः}

\twolineshloka
{एवमुक्ता वसिष्ठेन शबला शत्रुसूदन}
{विदधे कामधुक् कामान् यस्य यस्येप्सितं यथा} %1-53-1

\twolineshloka
{इक्षून् मधूंस्तथा लाजान् मैरेयांश्च वरासवान्}
{पानानि च महार्हाणि भक्ष्यांश्चोच्चावचानपि} %1-53-2

\twolineshloka
{उष्णाढ्यस्यौदनस्यात्र राशयः पर्वतोपमाः}
{मृष्टान्यन्नानि सूपांश्च दधिकुल्यास्तथैव च} %1-53-3

\twolineshloka
{नानास्वादुरसानां च खाण्डवानां तथैव च}
{भोजनानि सुपूर्णानि गौडानि च सहस्रशः} %1-53-4

\twolineshloka
{सर्वमासीत् सुसन्तुष्टं हृष्टपुष्टजनायुतम्}
{विश्वामित्रबलं राम वसिष्ठेन सुतर्पितम्} %1-53-5

\twolineshloka
{विश्वामित्रो हि राजर्षिर्हृष्टपुष्टस्तदाभवत्}
{सान्तःपुरवरो राजा सब्राह्मणपुरोहितः} %1-53-6

\twolineshloka
{सामात्यो मन्त्रिसहितः सभृत्यः पूजितस्तदा}
{युक्तः परमहर्षेण वसिष्ठमिदमब्रवीत्} %1-53-7

\twolineshloka
{पूजितोऽहं त्वया ब्रह्मन् पूजार्हेण सुसत्कृतः}
{श्रूयतामभिधास्यामि वाक्यं वाक्यविशारद} %1-53-8

\twolineshloka
{गवां शतसहस्रेण दीयतां शबला मम}
{रत्नं हि भगवन्नेतद् रत्नहारी च पार्थिवः} %1-53-9

\twolineshloka
{तस्मान्मे शबलां देहि ममैषा धर्मतो द्विज}
{एवमुक्तस्तु भगवान् वसिष्ठो मुनिपुङ्गवः} %1-53-10

\twolineshloka
{विश्वामित्रेण धर्मात्मा प्रत्युवाच महीपतिम्}
{नाहं शतसहस्रेण नापि कोटिशतैर्गवाम्} %1-53-11

\twolineshloka
{राजन् दास्यामि शबलां राशिभी रजतस्य वा}
{न परित्यागमर्हेयं मत्सकाशादरिन्दम} %1-53-12

\twolineshloka
{शाश्वती शबला मह्यं कीर्तिरात्मवतो यथा}
{अस्यां हव्यं च कव्यं च प्राणयात्रा तथैव च} %1-53-13

\twolineshloka
{आयत्तमग्निहोत्रं च बलिर्होमस्तथैव च}
{स्वाहाकारवषट्कारौ विद्याश्च विविधास्तथा} %1-53-14

\twolineshloka
{आयत्तमत्र राजर्षे सर्वमेतन्न संशयः}
{सर्वस्वमेतत् सत्येन मम तुष्टिकरी तथा} %1-53-15

\twolineshloka
{कारणैर्बहुभी राजन् न दास्ये शबलां तव}
{वसिष्ठेनैवमुक्तस्तु विश्वामित्रोऽब्रवीत् तदा} %1-53-16

\twolineshloka
{संरब्धतरमत्यर्थं वाक्यं वाक्यविशारदः}
{हैरण्यकक्षग्रैवेयान् सुवर्णाङ्कुशभूषितान्} %1-53-17

\twolineshloka
{ददामि कुञ्जराणां ते सहस्राणि चतुर्दश}
{हैरण्यानां रथानां च श्वेताश्वानां चतुर्युजाम्} %1-53-18

\threelineshloka
{ददामि ते शतान्यष्टौ किङ्किणीकविभूषितान्}
{हयानां देशजातानां कुलजानां महौजसाम्}
{सहस्रमेकं दश च ददामि तव सुव्रत} %1-53-19

\twolineshloka
{नानावर्णविभक्तानां वयःस्थानां तथैव च}
{ददाम्येकां गवां कोटिं शबला दीयतां मम} %1-53-20

\twolineshloka
{यावदिच्छसि रत्नानि हिरण्यं वा द्विजोत्तम}
{तावद् ददामि ते सर्वं दीयतां शबला मम} %1-53-21

\twolineshloka
{एवमुक्तस्तु भगवान् विश्वामित्रेण धीमता}
{न दास्यामीति शबलां प्राह राजन् कथञ्चन} %1-53-22

\twolineshloka
{एतदेव हि मे रत्नमेतदेव हि मे धनम्}
{एतदेव हि सर्वस्वमेतदेव हि जीवितम्} %1-53-23

\twolineshloka
{दर्शश्च पौर्णमासश्च यज्ञाश्चैवाप्तदक्षिणाः}
{एतदेव हि मे राजन् विविधाश्च क्रियास्तथा} %1-53-24

\twolineshloka
{अतोमूलाः क्रियाः सर्वा मम राजन् न संशयः}
{बहुना किं प्रलापेन न दास्ये कामदोहिनीम्} %1-53-25


॥इत्यार्षे श्रीमद्रामायणे वाल्मीकीये आदिकाव्ये बालकाण्डे शबलानिष्क्रयः नाम त्रिपञ्चाशः सर्गः ॥१-५३॥
