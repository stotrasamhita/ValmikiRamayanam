\sect{दशमः सर्गः — ऋष्यशृङ्गस्याङ्गदेशानयनप्रकारः}

\threelineshloka
{सुमन्त्रश्चोदितो राज्ञा प्रोवाचेदं वचस्तदा}
{यथर्ष्यशृङ्गस्त्वानीतो येनोपायेन मन्त्रिभिः}
{तन्मे निगदितं सर्वं शृणु मे मन्त्रिभिः सह} %1-10-1

\twolineshloka
{रोमपादमुवाचेदं सहामात्यः पुरोहितः}
{उपायो निरपायोऽयमस्माभिरभिचिन्तितः} %1-10-2

\twolineshloka
{ऋष्यशृङ्गो वनचरस्तपःस्वाध्यायसंयुतः}
{अनभिज्ञस्तु नारीणां विषयाणां सुखस्य च} %1-10-3

\twolineshloka
{इन्द्रियार्थैरभिमतैर्नरचित्तप्रमाथिभिः}
{पुरमानाययिष्यामः क्षिप्रं चाध्यवसीयताम्} %1-10-4

\twolineshloka
{गणिकास्तत्र गच्छन्तु रूपवत्यः स्वलङ्कृताः}
{प्रलोभ्य विविधोपायैरानेष्यन्तीह सत्कृताः} %1-10-5

\twolineshloka
{श्रुत्वा तथेति राजा च प्रत्युवाच पुरोहितम्}
{पुरोहितो मन्त्रिणश्च तथा चक्रुश्च ते तथा} %1-10-6

\twolineshloka
{वारमुख्यास्तु तच्छ्रुत्वा वनं प्रविविशुर्महत्}
{आश्रमस्याविदूरेऽस्मिन् यत्नं कुर्वन्ति दर्शने} %1-10-7

\twolineshloka
{ऋषेः पुत्रस्य धीरस्य नित्यमाश्रमवासिनः}
{पितुः स नित्यसन्तुष्टो नातिचक्राम चाश्रमात्} %1-10-8

\twolineshloka
{न तेन जन्मप्रभृति दृष्टपूर्वं तपस्विना}
{स्त्री वा पुमान्वा यच्चान्यत् सत्त्वं नगरराष्ट्रजम्} %1-10-9

\twolineshloka
{ततः कदाचित् तं देशमाजगाम यदृच्छया}
{विभाण्डकसुतस्तत्र ताश्चापश्यद् वराङ्गनाः} %1-10-10

\twolineshloka
{ताश्चित्रवेषाः प्रमदा गायन्त्यो मधुरस्वरम्}
{ऋषिपुत्रमुपागम्य सर्वा वचनमब्रुवन्} %1-10-11

\twolineshloka
{कस्त्वं किं वर्तसे ब्रह्मन् ज्ञातुमिच्छामहे वयम्}
{एकस्त्वं विजने घोरे वने चरसि शंस नः} %1-10-12

\twolineshloka
{अदृष्टरूपास्तास्तेन काम्यरूपा वने स्त्रियः}
{हार्दात्तस्य मतिर्जाता आख्यातुं पितरं स्वकम्} %1-10-13

\twolineshloka
{पिता विभाण्डकोऽस्माकं तस्याहं सुत औरसः}
{ऋष्यशृङ्ग इति ख्यातं नाम कर्म च मे भुवि} %1-10-14

\twolineshloka
{इहाश्रमपदोऽस्माकं समीपे शुभदर्शनाः}
{करिष्ये वोऽत्र पूजां वै सर्वेषां विधिपूर्वकम्} %1-10-15

\twolineshloka
{ऋषिपुत्रवचः श्रुत्वा सर्वासां मतिरास वै}
{तदाश्रमपदं द्रष्टुं जग्मुः सर्वास्ततोऽङ्गनाः} %1-10-16

\twolineshloka
{गतानां तु ततः पूजामृषिपुत्रश्चकार ह}
{इदमर्घ्यमिदं पाद्यमिदं मूलं फलं च नः} %1-10-17

\twolineshloka
{प्रतिगृह्य तु तां पूजां सर्वा एव समुत्सुकाः}
{ऋषेर्भीताश्च शीघ्रं तु गमनाय मतिं दधुः} %1-10-18

\twolineshloka
{अस्माकमपि मुख्यानि फलानीमानि हे द्विज}
{गृहाण विप्र भद्रं ते भक्षयस्व च मा चिरम्} %1-10-19

\twolineshloka
{ततस्तास्तं समालिङ्ग्य सर्वा हर्षसमन्विताः}
{मोदकान् प्रददुस्तस्मै भक्ष्यांश्च विविधाञ्छुभान्} %1-10-20

\twolineshloka
{तानि चास्वाद्य तेजस्वी फलानीति स्म मन्यते}
{अनास्वादितपूर्वाणि वने नित्यनिवासिनाम्} %1-10-21

\twolineshloka
{आपृच्छ्य च तदा विप्रं व्रतचर्यां निवेद्य च}
{गच्छन्ति स्मापदेशात्ता भीतास्तस्य पितुः स्त्रियः} %1-10-22

\twolineshloka
{गतासु तासु सर्वासु काश्यपस्यात्मजो द्विजः}
{अस्वस्थहृदयश्चासीद् दुःखाच्च परिवर्तते} %1-10-23

\twolineshloka
{ततोऽपरेद्युस्तं देशमाजगाम स वीर्यवान्}
{विभाण्डकसुतः श्रीमान् मनसाचिन्तयन्मुहुः} %1-10-24

\twolineshloka
{मनोज्ञा यत्र ता दृष्टा वारमुख्याः स्वलङ्कृताः}
{दृष्ट्वैव च ततो विप्रमायान्तं हृष्टमानसाः} %1-10-25

\twolineshloka
{उपसृत्य ततः सर्वास्तास्तमूचुरिदं वचः}
{एह्याश्रमपदं सौम्य अस्माकमिति चाब्रुवन्} %1-10-26

\twolineshloka
{चित्राण्यत्र बहूनि स्युर्मूलानि च फलानि च}
{तत्राप्येष विशेषेण विधिर्हि भविता ध्रुवम्} %1-10-27

\twolineshloka
{श्रुत्वा तु वचनं तासां सर्वासां हृदयङ्गमम्}
{गमनाय मतिं चक्रे तं च निन्युस्तदा स्त्रियः} %1-10-28

\twolineshloka
{तत्र चानीयमाने तु विप्रे तस्मिन् महात्मनि}
{ववर्ष सहसा देवो जगत् प्रह्लादयंस्तदा} %1-10-29

\twolineshloka
{वर्षेणैवागतं विप्रं तापसं स नराधिपः}
{प्रत्युद्गम्य मुनिं प्रह्वः शिरसा च महीं गतः} %1-10-30

\twolineshloka
{अर्घ्यं च प्रददौ तस्मै न्यायतः सुसमाहितः}
{वव्रे प्रसादं विप्रेन्द्रान्मा विप्रं मन्युराविशेत्} %1-10-31

\twolineshloka
{अन्तःपुरं प्रवेश्यास्मै कन्यां दत्त्वा यथाविधि}
{शान्तां शान्तेन मनसा राजा हर्षमवाप सः} %1-10-32

\twolineshloka
{एवं स न्यवसत् तत्र सर्वकामैः सुपूजितः}
{ऋष्यशृङ्गो महातेजाः शान्तया सह भार्यया} %1-10-33


॥इत्यार्षे श्रीमद्रामायणे वाल्मीकीये आदिकाव्ये बालकाण्डे ऋष्यशृङ्गस्याङ्गदेशानयनप्रकारः नाम दशमः सर्गः ॥१-१०॥
