\sect{चतुःपञ्चाशः सर्गः — पप्लवादिसृष्टिः}

\twolineshloka
{कामधेनुं वसिष्ठोऽपि यदा न त्यजते मुनिः}
{तदास्य शबलां राम विश्वामित्रोऽन्वकर्षत} %1-54-1

\twolineshloka
{नीयमाना तु शबला राम राज्ञा महात्मना}
{दुःखिता चिन्तयामास रुदन्ती शोककर्शिता} %1-54-2

\twolineshloka
{परित्यक्ता वसिष्ठेन किमहं सुमहात्मना}
{याहं राजभृतैर्दीना ह्रियेय भृशदुःखिता} %1-54-3

\twolineshloka
{किं मयापकृतं तस्य महर्षेर्भावितात्मनः}
{यन्मामनागसं दृष्ट्वा भक्तां त्यजति धार्मिकः} %1-54-4

\twolineshloka
{इति सञ्चिन्तयित्वा तु निःश्वस्य च पुनः पुनः}
{जगाम वेगेन तदा वसिष्ठं परमौजसम्} %1-54-5

\twolineshloka
{निर्धूय तांस्तदा भृत्यान् शतशः शत्रुसूदन}
{जगामानिलवेगेन पादमूलं महात्मनः} %1-54-6

\twolineshloka
{शबला सा रुदन्ती च क्रोशन्ती चेदमब्रवीत्}
{वसिष्ठस्याग्रतः स्थित्वा रुदन्ती मेघनिःस्वना} %1-54-7

\twolineshloka
{भगवन् किं परित्यक्ता त्वयाहं ब्रह्मणः सुत}
{यस्माद् राजभटा मां हि नयन्ते त्वत्सकाशतः} %1-54-8

\twolineshloka
{एवमुक्तस्तु ब्रह्मर्षिरिदं वचनमब्रवीत्}
{शोकसन्तप्तहृदयां स्वसारमिव दुःखिताम्} %1-54-9

\twolineshloka
{न त्वां त्यजामि शबले नापि मेऽपकृतं त्वया}
{एष त्वां नयते राजा बलान्मत्तो महाबलः} %1-54-10

\twolineshloka
{नहि तुल्यं बलं मह्यं राजा त्वद्य विशेषतः}
{बली राजा क्षत्रियश्च पृथिव्याः पतिरेव च} %1-54-11

\twolineshloka
{इयमक्षौहिणी पूर्णा गजवाजिरथाकुला}
{हस्तिध्वजसमाकीर्णा तेनासौ बलवत्तरः} %1-54-12

\twolineshloka
{एवमुक्ता वसिष्ठेन प्रत्युवाच विनीतवत्}
{वचनं वचनज्ञा सा ब्रह्मर्षिमतुलप्रभम्} %1-54-13

\twolineshloka
{न बलं क्षत्रियस्याहुर्ब्राह्मणा बलवत्तराः}
{ब्रह्मन् ब्रह्मबलं दिव्यं क्षात्राच्च बलवत्तरम्} %1-54-14

\twolineshloka
{अप्रमेयं बलं तुभ्यं न त्वया बलवत्तरः}
{विश्वामित्रो महावीर्यस्तेजस्तव दुरासदम्} %1-54-15

\twolineshloka
{नियुङ्क्ष्व मां महातेजस्त्वं ब्रह्मबलसम्भृताम्}
{तस्य दर्पं बलं यत्नं नाशयामि दुरात्मनः} %1-54-16

\twolineshloka
{इत्युक्तस्तु तया राम वसिष्ठस्तु महायशाः}
{सृजस्वेति तदोवाच बलं परबलार्दनम्} %1-54-17

\twolineshloka
{तस्य तद् वचनं श्रुत्वा सुरभिः सासृजत् तदा}
{तस्या हुम्भारवोत्सृष्टाः पह्लवाः शतशो नृप} %1-54-18

\twolineshloka
{नाशयन्ति बलं सर्वं विश्वामित्रस्य पश्यतः}
{स राजा परमक्रुद्धः क्रोधविस्फारितेक्षणः} %1-54-19

\twolineshloka
{पह्लवान् नाशयामास शस्त्रैरुच्चावचैरपि}
{विश्वामित्रार्दितान् दृष्ट्वा पह्लवान् शतशस्तदा} %1-54-20

\twolineshloka
{भूय एवासृजद् घोरान् शकान् यवनमिश्रितान्}
{तैरासीत् संवृता भूमिः शकैर्यवनमिश्रितैः} %1-54-21

\twolineshloka
{प्रभावद्भिर्महावीर्यैर्हेमकिञ्जल्कसन्निभैः}
{तीक्ष्णासिपट्टिशधरैर्हेमवर्णाम्बरावृतैः} %1-54-22

\threelineshloka
{निर्दग्धं तद्बलं सर्वं प्रदीप्तैरिव पावकैः}
{ततोऽस्त्राणि महातेजा विश्वामित्रो मुमोच ह}
{तैस्ते यवनकाम्बोजा बर्बराश्चाकुलीकृताः} %1-54-23


॥इत्यार्षे श्रीमद्रामायणे वाल्मीकीये आदिकाव्ये बालकाण्डे पप्लवादिसृष्टिः नाम चतुःपञ्चाशः सर्गः ॥१-५४॥
