\sect{षड्त्रिंशः सर्गः — उमामाहात्म्यम्}

\twolineshloka
{उक्तवाक्ये मुनौ तस्मिन्नुभौ राघवलक्ष्मणौ}
{प्रतिनन्द्य कथां वीरावूचतुर्मुनिपुङ्गवम्} %1-36-1

\threelineshloka
{धर्मयुक्तमिदं ब्रह्मन् कथितं परमं त्वया}
{दुहितुः शैलराजस्य ज्येष्ठाया वक्तुमर्हसि}
{विस्तरं विस्तरज्ञोऽसि दिव्यमानुषसम्भवम्} %1-36-2

\twolineshloka
{त्रीन् पथो हेतुना केन प्लावयेल्लोकपावनी}
{कथं गङ्गा त्रिपथगा विश्रुता सरिदुत्तमा} %1-36-3

\twolineshloka
{त्रिषु लोकेषु धर्मज्ञ कर्मभिः कैः समन्विता}
{तथा ब्रुवति काकुत्स्थे विश्वामित्रस्तपोधनः} %1-36-4

\twolineshloka
{निखिलेन कथां सर्वामृषिमध्ये न्यवेदयत्}
{पुरा राम कृतोद्वाहो शितिकण्ठो महातपाः} %1-36-5

\threelineshloka
{दृष्ट्वा च भगवान् देवीं मैथुनायोपचक्रमे}
{तस्य सङ्क्रीडमानस्य महादेवस्य धीमतः}
{शितिकण्ठस्य देवस्य दिव्यं वर्षशतं गतम्} %1-36-6

\twolineshloka
{न चापि तनयो राम तस्यामासीत् परन्तप}
{सर्वे देवाः समुद्युक्ताः पितामहपुरोगमाः} %1-36-7

\twolineshloka
{यदिहोत्पद्यते भूतं कस्तत् प्रतिसहिष्यते}
{अभिगम्य सुराः सर्वे प्रणिपत्येदमब्रुवन्} %1-36-8

\twolineshloka
{देवदेव महादेव लोकस्यास्य हिते रत}
{सुराणां प्रणिपातेन प्रसादं कर्तुमर्हसि} %1-36-9

\twolineshloka
{न लोका धारयिष्यन्ति तव तेजः सुरोत्तम}
{ब्राह्मेण तपसा युक्तो देव्या सह तपश्चर} %1-36-10

\twolineshloka
{त्रैलोक्यहितकामार्थं तेजस्तेजसि धारय}
{रक्ष सर्वानिमाँल्लोकान् नालोकं कर्तुमर्हसि} %1-36-11

\twolineshloka
{देवतानां वचः श्रुत्वा सर्वलोकमहेश्वरः}
{बाढमित्यब्रवीत् सर्वान् पुनश्चेदमुवाच ह} %1-36-12

\twolineshloka
{धारयिष्याम्यहं तेजस्तेजस्येव सहोमया}
{त्रिदशाः पृथिवी चैव निर्वाणमधिगच्छतु} %1-36-13

\twolineshloka
{यदिदं क्षुभितं स्थानान्मम तेजो ह्यनुत्तमम्}
{धारयिष्यति कस्तन्मे ब्रुवन्तु सुरसत्तमाः} %1-36-14

\twolineshloka
{एवमुक्तास्ततो देवाः प्रत्यूचुर्वृषभध्वजम्}
{यत्तेजः क्षुभितं ह्यद्य तद्धरा धारयिष्यति} %1-36-15

\twolineshloka
{एवमुक्तः सुरपतिः प्रमुमोच महाबलः}
{तेजसा पृथिवी येन व्याप्ता सगिरिकानना} %1-36-16

\twolineshloka
{ततो देवाः पुनरिदमूचुश्चापि हुताशनम्}
{आविश त्वं महातेजो रौद्रं वायुसमन्वितः} %1-36-17

\twolineshloka
{तदग्निना पुनर्व्याप्तं सञ्जातं श्वेतपर्वतम्}
{दिव्यं शरवणं चैव पावकादित्यसन्निभम्} %1-36-18

\twolineshloka
{यत्र जातो महातेजाः कार्तिकेयोऽग्निसम्भवः}
{अथोमां च शिवं चैव देवाः सर्षिगणास्तदा} %1-36-19

\twolineshloka
{पूजयामासुरत्यर्थं सुप्रीतमनसस्तदा}
{अथ शैलसुता राम त्रिदशानिदमब्रवीत्} %1-36-20

\twolineshloka
{समन्युरशपत् सर्वान् क्रोधसंरक्तलोचना}
{यस्मान्निवारिता चाहं सङ्गता पुत्रकाम्यया} %1-36-21

\twolineshloka
{अपत्यं स्वेषु दारेषु नोत्पादयितुमर्हथ}
{अद्यप्रभृति युष्माकमप्रजाः सन्तु पत्नयः} %1-36-22

\twolineshloka
{एवमुक्त्वा सुरान् सर्वान् शशाप पृथिवीमपि}
{अवने नैकरूपा त्वं बहुभार्या भविष्यसि} %1-36-23

\twolineshloka
{न च पुत्रकृतां प्रीतिं मत्क्रोधकलुषीकृता}
{प्राप्स्यसि त्वं सुदुर्मेधे मम पुत्रमनिच्छती} %1-36-24

\twolineshloka
{तान् सर्वान् पीडितान् दृष्ट्वा सुरान् सुरपतिस्तदा}
{गमनायोपचक्राम दिशं वरुणपालिताम्} %1-36-25

\twolineshloka
{स गत्वा तप आतिष्ठत् पार्श्वे तस्योत्तरे गिरेः}
{हिमवत्प्रभवे शृङ्गे सह देव्या महेश्वरः} %1-36-26

\twolineshloka
{एष ते विस्तरो राम शैलपुत्र्या निवेदितः}
{गङ्गायाः प्रभवं चैव शृणु मे सहलक्ष्मण} %1-36-27


॥इत्यार्षे श्रीमद्रामायणे वाल्मीकीये आदिकाव्ये बालकाण्डे उमामाहात्म्यम् नाम षड्त्रिंशः सर्गः ॥१-३६॥
