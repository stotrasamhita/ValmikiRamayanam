\sect{षट्सप्ततितमः सर्गः — जामदग्न्यप्रतिष्टम्भः}

\twolineshloka
{श्रुत्वा तु जामदग्न्यस्य वाक्यं दाशरथिस्तदा}
{गौरवाद्यन्त्रितकथः पितू राममथाब्रवीत्} %1-76-1

\twolineshloka
{कृतवानसि यत् कर्म श्रुतवानस्मि भार्गव}
{अनुरुध्यामहे ब्रह्मन् पितुरानृण्यमास्थितः} %1-76-2

\twolineshloka
{वीर्यहीनमिवाशक्तं क्षत्रधर्मेण भार्गव}
{अवजानासि मे तेजः पश्य मेऽद्य पराक्रमम्} %1-76-3

\twolineshloka
{इत्युक्त्वा राघवः क्रुद्धो भार्गवस्य वरायुधम्}
{शरं च प्रतिजग्राह हस्ताल्लघुपराक्रमः} %1-76-4

\twolineshloka
{आरोप्य स धनू रामः शरं सज्यं चकार ह}
{जामदग्न्यं ततो रामं रामः क्रुद्धोऽब्रवीदिदम्} %1-76-5

\twolineshloka
{ब्राह्मणोऽसीति पूज्यो मे विश्वामित्रकृतेन च}
{तस्माच्छक्तो न ते राम मोक्तुं प्राणहरं शरम्} %1-76-6

\twolineshloka
{इमां वा त्वद्गतिं राम तपोबलसमर्जितान्}
{लोकानप्रतिमान् वापि हनिष्यामीति मे मतिः} %1-76-7

\twolineshloka
{न ह्ययं वैष्णवो दिव्यः शरः परपुरंजयः}
{मोघः पतति वीर्येण बलदर्पविनाशनः} %1-76-8

\twolineshloka
{वरायुधधरं रामं द्रष्टुं सर्षिगणाः सुराः}
{पितामहं पुरस्कृत्य समेतास्तत्र सर्वशः} %1-76-9

\twolineshloka
{गन्धर्वाप्सरसश्चैव सिद्धचारणकिन्नराः}
{यक्षराक्षसनागाश्च तद् द्रष्टुं महदद्भुतम्} %1-76-10

\twolineshloka
{जडीकृते तदा लोके रामे वरधनुर्धरे}
{निर्वीर्यो जामदग्न्योऽसौ रामो राममुदैक्षत} %1-76-11

\twolineshloka
{तेजोभिर्गतवीर्यत्वाज्जामदग्न्यो जडीकृतः}
{रामं कमलपत्राक्षं मन्दं मन्दमुवाच ह} %1-76-12

\twolineshloka
{काश्यपाय मया दत्ता यदा पूर्वं वसुंधरा}
{विषये मे न वस्तव्यमिति मां काश्यपोऽब्रवीत्} %1-76-13

\twolineshloka
{सोऽहं गुरुवचः कुर्वन् पृथिव्यां न वसे निशाम्}
{तदाप्रभृति काकुत्स्थ कृता मे काश्यपस्य ह} %1-76-14

\twolineshloka
{तामिमां मद्गतिं वीर हन्तुं नार्हसि राघव}
{मनोजवं गमिष्यामि महेन्द्रं पर्वतोत्तमम्} %1-76-15

\twolineshloka
{लोकास्त्वप्रतिमा राम निर्जितास्तपसा मया}
{जहि ताञ्छरमुख्येन मा भूत् कालस्य पर्ययः} %1-76-16

\twolineshloka
{अक्षय्यं मधुहन्तारं जानामि त्वां सुरेश्वरम्}
{धनुषोऽस्य परामर्शात् स्वस्ति तेऽस्तु परंतप} %1-76-17

\twolineshloka
{एते सुरगणाः सर्वे निरीक्षन्ते समागताः}
{त्वामप्रतिमकर्माणमप्रतिद्वन्द्वमाहवे} %1-76-18

\twolineshloka
{न चेयं मम काकुत्स्थ व्रीडा भवितुमर्हति}
{त्वया त्रैलोक्यनाथेन यदहं विमुखीकृतः} %1-76-19

\twolineshloka
{शरमप्रतिमं राम मोक्तुमर्हसि सुव्रत}
{शरमोक्षे गमिष्यामि महेन्द्रं पर्वतोत्तमम्} %1-76-20

\twolineshloka
{तथा ब्रुवति रामे तु जामदग्न्ये प्रतापवान्}
{रामो दाशरथिः श्रीमांश्चिक्षेप शरमुत्तमम्} %1-76-21

\twolineshloka
{स हतान् दृश्य रामेण स्वाँल्लोकांस्तपसार्जितान्}
{जामदग्न्यो जगामाशु महेन्द्रं पर्वतोत्तमम्} %1-76-22

\twolineshloka
{ततो वितिमिराः सर्वा दिशश्चोपदिशस्तथा}
{सुराः सर्षिगणा रामं प्रशशंसुरुदायुधम्} %1-76-23

\twolineshloka
{रामं दाशरथिं रामो जामदग्न्यः प्रपूजितः}
{ततः प्रदक्षिणीकृत्य जगामात्मगतिं प्रभुः} %1-76-24


॥इत्यार्षे श्रीमद्रामायणे वाल्मीकीये आदिकाव्ये बालकाण्डे जामदग्न्यप्रतिष्टम्भः नाम षट्सप्ततितमः सर्गः ॥१-७६॥
