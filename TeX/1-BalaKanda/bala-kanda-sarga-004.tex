\sect{चतुर्थः सर्गः — अनुक्रमणिका}

\twolineshloka
{प्राप्तराज्यस्य रामस्य वाल्मीकिर्भगवानृषिः}
{चकार चरितं कृत्स्नं विचित्रपदमर्थवत्} %1-4-1

\twolineshloka
{चतुर्विंशत्सहस्राणि श्लोकानामुक्तवानृषिः}
{तथा सर्गशतान् पञ्च षट्काण्डानि तथोत्तरम्} %1-4-2

\twolineshloka
{कृत्वा तु तन्महाप्राज्ञः सभविष्यं सहोत्तरम्}
{चिन्तयामास को न्वेतत् प्रयुञ्जीयादिति प्रभुः} %1-4-3

\twolineshloka
{तस्य चिन्तयमानस्य महर्षेर्भावितात्मनः}
{अगृह्णीतां ततः पादौ मुनिवेषौ कुशीलवौ} %1-4-4

\twolineshloka
{कुशीलवौ तु धर्मज्ञौ राजपुत्रौ यशस्विनौ}
{भ्रातरौ स्वरसंपन्नौ ददर्शाश्रमवासिनौ} %1-4-5

\twolineshloka
{स तु मेधाविनौ दृष्ट्वा वेदेषु परिनिष्ठितौ}
{वेदोपबृंह्मणार्थाय तावग्राहयत प्रभुः} %1-4-6

\twolineshloka
{काव्यं रामायणं कृत्स्नं सीतायाश्चरितं महत्}
{पौलस्त्यवधमित्येवं चकार चरितव्रतः} %1-4-7

\twolineshloka
{पाठ्ये गेये च मधुरं प्रमाणैस्त्रिभिरन्वितम्}
{जातिभिः सप्तभिर्युक्तं तन्त्रीलयसमन्वितम्} %1-4-8

\twolineshloka
{रसैः शृङ्गारकरुणहास्यरौद्रभयानकैः}
{वीरादिभी रसैर्युक्तं काव्यमेतदगायताम्} %1-4-9

\twolineshloka
{तौ तु गान्धर्वतत्त्वज्ञौ स्थानमूर्च्छनकोविदौ}
{भ्रातरौ स्वरसम्पन्नौ गन्धर्वाविव रूपिणौ} %1-4-10

\twolineshloka
{रूपलक्षणसम्पन्नौ मधुरस्वरभाषिणौ}
{बिम्बादिवोत्थितौ बिम्बौ रामदेहात् तथापरौ} %1-4-11

\twolineshloka
{तौ राजपुत्रौ कार्त्स्न्येन धर्म्यमाख्यानमुत्तमम्}
{वाचोविधेयं तत्सर्वं कृत्वा काव्यमनिन्दितौ} %1-4-12

\twolineshloka
{ऋषीणां च द्विजातीनां साधूनां च समागमे}
{यथोपदेशं तत्त्वज्ञौ जगतुः सुसमाहितौ} %1-4-13

\twolineshloka
{महात्मानौ महाभागौ सर्वलक्षणलक्षितौ}
{तौ कदाचित् समेतानामृषीणां भावितात्मनाम्} %1-4-14

\twolineshloka
{मध्ये सभं समीपस्थाविदं काव्यमगायताम्}
{तच्छ्रुत्वा मुनयः सर्वे बाष्पपर्याकुलेक्षणाः} %1-4-15

\twolineshloka
{साधु साध्विति तावूचुः परं विस्मयमागताः}
{ते प्रीतमनसः सर्वे मुनयो धर्मवत्सलाः} %1-4-16

\twolineshloka
{प्रशशंसुः प्रशस्तव्यौ गायमानौ कुशीलवौ}
{अहो गीतस्य माधुर्यं श्लोकानां च विशेषतः} %1-4-17

\twolineshloka
{चिरनिर्वृत्तमप्येतत् प्रत्यक्षमिव दर्शितम्}
{प्रविश्य तावुभौ सुष्ठु तथाभावमगायताम्} %1-4-18

\twolineshloka
{सहितौ मधुरं रक्तं सम्पन्नं स्वरसंपदा}
{एवं प्रशस्यमानौ तौ तपः श्लाघ्यैर्महर्षिभिः} %1-4-19

\twolineshloka
{संरक्ततरमत्यर्थं मधुरं तावगायताम्}
{प्रीतः कश्चिन्मुनिस्ताभ्यां संस्थितः कलशं ददौ} %1-4-20

\twolineshloka
{प्रसन्नो वल्कलं कश्चिद् ददौ ताभ्यां महायशाः}
{अन्यः कृष्णाजिनमदाद् यज्ञसूत्रं तथापरः} %1-4-21

\twolineshloka
{कश्चित् कमण्डलुं प्रादान्मौञ्जीमन्यो महामुनिः}
{बृसीमन्यस्तदा प्रादात् कौपीनमपरो मुनिः} %1-4-22

\twolineshloka
{ताभ्यां ददौ तदा हृष्टः कुठारमपरो मुनिः}
{काषायमपरो वस्त्रं चीरमन्यो ददौ मुनिः} %1-4-23

\twolineshloka
{जटाबन्धनमन्यस्तु काष्ठरज्जुं मुदान्वितः}
{यज्ञभाण्डमृषिः कश्चित् काष्ठभारं तथा परः} %1-4-24

\twolineshloka
{औदुम्बरीं ब्रुसीमन्यः स्वस्ति केचित् तदावदन्}
{आयुष्यमपरे प्राहुर्मुदा तत्र महर्षयः} %1-4-25

\twolineshloka
{ददुश्चैवं वरान् सर्वे मुनयः सत्यवादिनः}
{आश्चर्यमिदमाख्यानं मुनिना सम्प्रकीर्तितम्} %1-4-26

\twolineshloka
{परं कवीनामाधारं समाप्तं च यथाक्रमम्}
{अभिगीतमिदं गीतं सर्वगीतेषु कोविदौ} %1-4-27

\twolineshloka
{आयुष्यं पुष्टिजननं सर्वश्रुतिमनोहरम्}
{प्रशस्यमानौ सर्वत्र कदाचित् तत्र गायकौ} %1-4-28

\twolineshloka
{रथ्यासु राजमार्गेषु ददर्श भरताग्रजः}
{स्ववेश्म चानीय ततो भ्रातरौ स कुशीलवौ} %1-4-29

\twolineshloka
{पूजयामास पूजार्हौ रामः शत्रुनिबर्हणः}
{आसीनः काञ्चने दिव्ये स च सिंहासने प्रभुः} %1-4-30

\twolineshloka
{उपोपविष्टैः सचिवैर्भ्रातृभिश्च समन्वितः}
{दृष्ट्वा तु रूपसम्पन्नौ विनीतौ भ्रातरावुभौ} %1-4-31

\twolineshloka
{उवाच लक्ष्मणं रामः शत्रुघ्नं भरतं तथा}
{श्रूयतामेतदाख्यानमनयोर्देववर्चसोः} %1-4-32

\twolineshloka
{विचित्रार्थपदं सम्यग्गायकौ समचोदयत्}
{तौ चापि मधुरं रक्तं स्वचित्तायतनिःस्वनम्} %1-4-33

\threelineshloka
{तन्त्रीलयवदत्यर्थं विश्रुतार्थमगायताम्}
{ह्लादयत् सर्वगात्राणि मनांसि हृदयानि च}
{श्रोत्राश्रयसुखं गेयं तद् बभौ जनसंसदि} %1-4-34

\fourlineindentedshloka
{इमौ मुनी पार्थिवलक्षणान्वितौ}
{कुशीलवौ चैव महातपस्विनौ}
{ममापि तद् भूतिकरं प्रचक्षते}
{महानुभावं चरितं निबोधत} %1-4-35

\twolineshloka
{ततस्तु तौ रामवचः प्रचोदितावगायतां मार्गविधानसम्पदा}
{स चापि रामः परिषद्गतः शनैर्बुभूषयासक्तमना बभूव ह} %1-4-36


॥इत्यार्षे श्रीमद्रामायणे वाल्मीकीये आदिकाव्ये बालकाण्डे अनुक्रमणिका नाम चतुर्थः सर्गः ॥१-४॥
