\sect{पञ्चत्रिंशः सर्गः — उमागङ्गावृत्तान्तसंक्षेपः}

\twolineshloka
{उपास्य रात्रिशेषं तु शोणाकूले महर्षिभिः}
{निशायां सुप्रभातायां विश्वामित्रोऽभ्यभाषत} %1-35-1

\twolineshloka
{सुप्रभाता निशा राम पूर्वा संध्या प्रवर्तते}
{उत्तिष्ठोत्तिष्ठ भद्रं ते गमनायाभिरोचय} %1-35-2

\twolineshloka
{तच्छ्रुत्वा वचनं तस्य कृतपूर्वाह्णिकक्रियः}
{गमनं रोचयामास वाक्यं चेदमुवाच ह} %1-35-3

\twolineshloka
{अयं शोणः शुभजलोऽगाधः पुलिनमण्डितः}
{कतरेण पथा ब्रह्मन् संतरिष्यामहे वयम्} %1-35-4

\twolineshloka
{एवमुक्तस्तु रामेण विश्वामित्रोऽब्रवीदिदम्}
{एष पन्था मयोद्दिष्टो येन यान्ति महर्षयः} %1-35-5

\twolineshloka
{एवमुक्ता महर्षयो विश्वामित्रेण धीमता}
{पश्यन्तस्ते प्रयाता वै वनानि विविधानि च} %1-35-6

\twolineshloka
{ते गत्वा दूरमध्वानं गतेऽर्धदिवसे तदा}
{जाह्नवीं सरितां श्रेष्ठां ददृशुर्मुनिसेविताम्} %1-35-7

\twolineshloka
{तां दृष्ट्वा पुण्यसलिलां हंससारससेविताम्}
{बभूवुर्मुनयः सर्वे मुदिताः सहराघवाः} %1-35-8

\twolineshloka
{तस्यास्तीरे तदा सर्वे चक्रुर्वासपरिग्रहम्}
{ततः स्नात्वा यथान्यायं संतर्प्य पितृदेवताः} %1-35-9

\twolineshloka
{हुत्वा चैवाग्निहोत्राणि प्राश्य चामृतवद्धविः}
{विविशुर्जाह्नवीतीरे शुभा मुदितमानसाः} %1-35-10

\threelineshloka
{विश्वामित्रं महात्मानं परिवार्य समन्ततः}
{विष्ठिताश्च यथान्यायं राघवौ च यथार्हतः}
{संप्रहृष्टमना रामो विश्वामित्रमथाब्रवीत्} %1-35-11

\twolineshloka
{भगवञ्छ्रोतुमिच्छामि गङ्गां त्रिपथगां नदीम्}
{त्रैलोक्यं कथमाक्रम्य गता नदनदीपतिम्} %1-35-12

\twolineshloka
{चोदितो रामवाक्येन विश्वामित्रो महामुनिः}
{वृद्धिं जन्म च गङ्गाया वक्तुमेवोपचक्रमे} %1-35-13

\twolineshloka
{शैलेन्द्रो हिमवान् राम धातूनामाकरो महान्}
{तस्य कन्याद्वयं राम रूपेणाप्रतिमं भुवि} %1-35-14

\twolineshloka
{या मेरुदुहिता राम तयोर्माता सुमध्यमा}
{नाम्ना मेना मनोज्ञा वै पत्नी हिमवतः प्रिया} %1-35-15

\twolineshloka
{तस्यां गङ्गेयमभवज्ज्येष्ठा हिमवतः सुता}
{उमा नाम द्वितीयाभूत् कन्या तस्यैव राघव} %1-35-16

\twolineshloka
{अथ ज्येष्ठां सुराः सर्वे देवकार्यचिकीर्षया}
{शैलेन्द्रं वरयामासुर्गङ्गां त्रिपथगां नदीम्} %1-35-17

\twolineshloka
{ददौ धर्मेण हिमवांस्तनयां लोकपावनीम्}
{स्वच्छन्दपथगां गङ्गां त्रैलोक्यहितकाम्यया} %1-35-18

\twolineshloka
{प्रतिगृह्य त्रिलोकार्थं त्रिलोकहितकांक्षिणः}
{गङ्गामादाय तेऽगच्छन् कृतार्थेनान्तरात्मना} %1-35-19

\twolineshloka
{या चान्या शैलदुहिता कन्याऽसीद् रघुनन्दन}
{उग्रं सुव्रतमास्थाय तपस्तेपे तपोधना} %1-35-20

\twolineshloka
{उग्रेण तपसा युक्तां ददौ शैलवरः सुताम्}
{रुद्रायाप्रतिरूपाय उमां लोकनमस्कृताम्} %1-35-21

\twolineshloka
{एते ते शैलराजस्य सुते लोकनमस्कृते}
{गङ्गा च सरितां श्रेष्ठा उमादेवी च राघव} %1-35-22

\twolineshloka
{एतत् ते सर्वमाख्यातं यथा त्रिपथगामिनी}
{खं गता प्रथमं तात गतिं गतिमतां वर} %1-35-23

\twolineshloka
{सैषा सुरनदी रम्या शैलेन्द्रतनया तदा}
{सुरलोकं समारूढा विपापा जलवाहिनी} %1-35-24


॥इत्यार्षे श्रीमद्रामायणे वाल्मीकीये आदिकाव्ये बालकाण्डे उमागङ्गावृत्तान्तसंक्षेपः नाम पञ्चत्रिंशः सर्गः ॥१-३५॥
