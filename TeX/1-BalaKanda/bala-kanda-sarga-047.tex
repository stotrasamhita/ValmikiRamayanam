\sect{सप्तचत्वारिंशः सर्गः — विशालागमनम्}

\twolineshloka
{सप्तधा तु कृते गर्भे दितिः परमदुःखिता}
{सहस्राक्षं दुराधर्षं वाक्यं सानुनयाब्रवीत्} %1-47-1

\twolineshloka
{ममापराधाद् गर्भोऽयं सप्तधा शकलीकृतः}
{नापराधो हि देवेश तवात्र बलसूदन} %1-47-2

\twolineshloka
{प्रियं त्वत्कृतमिच्छामि मम गर्भविपर्यये}
{मरुतां सप्त सप्तानां स्थानपाला भवन्तु ते} %1-47-3

\twolineshloka
{वातस्कन्धा इमे सप्त चरन्तु दिवि पुत्रक}
{मारुता इति विख्याता दिव्यरूपा ममात्मजाः} %1-47-4

\twolineshloka
{ब्रह्मलोकं चरत्वेक इन्द्रलोकं तथापरः}
{दिव्यवायुरिति ख्यातस्तृतीयोऽपि महायशाः} %1-47-5

\twolineshloka
{चत्वारस्तु सुरश्रेष्ठ दिशो वै तव शासनात्}
{सञ्चरिष्यन्ति भद्रं ते कालेन हि ममात्मजाः} %1-47-6

\twolineshloka
{त्वत्कृतेनैव नाम्ना वै मारुता इति विश्रुताः}
{तस्यास्तद् वचनं श्रुत्वा सहस्राक्षः पुरन्दरः} %1-47-7

\twolineshloka
{उवाच प्राञ्जलिर्वाक्यमतीदं बलसूदनः}
{सर्वमेतद् यथोक्तं ते भविष्यति न संशयः} %1-47-8

\twolineshloka
{विचरिष्यन्ति भद्रं ते देवरूपास्तवात्मजाः}
{एवं तौ निश्चयं कृत्वा मातापुत्रौ तपोवने} %1-47-9

\twolineshloka
{जग्मतुस्त्रिदिवं राम कृतार्थाविति नः श्रुतम्}
{एष देशः स काकुत्स्थ महेन्द्राध्युषितः पुरा} %1-47-10

\twolineshloka
{दितिं यत्र तपःसिद्धामेवं परिचचार सः}
{इक्ष्वाकोस्तु नरव्याघ्र पुत्रः परमधार्मिकः} %1-47-11

\twolineshloka
{अलम्बुषायामुत्पन्नो विशाल इति विश्रुतः}
{तेन चासीदिह स्थाने विशालेति पुरी कृता} %1-47-12

\twolineshloka
{विशालस्य सुतो राम हेमचन्द्रो महाबलः}
{सुचन्द्र इति विख्यातो हेमचन्द्रादनन्तरः} %1-47-13

\twolineshloka
{सुचन्द्रतनयो राम धूम्राश्व इति विश्रुतः}
{धूम्राश्वतनयश्चापि सृञ्जयः समपद्यत} %1-47-14

\twolineshloka
{सृञ्जयस्य सुतः श्रीमान् सहदेवः प्रतापवान्}
{कुशाश्वः सहदेवस्य पुत्रः परमधार्मिकः} %1-47-15

\twolineshloka
{कुशाश्वस्य महातेजाः सोमदत्तः प्रतापवान्}
{सोमदत्तस्य पुत्रस्तु काकुत्स्थ इति विश्रुतः} %1-47-16

\twolineshloka
{तस्य पुत्रो महातेजाः सम्प्रत्येष पुरीमिमाम्}
{आवसत् परमप्रख्यः सुमतिर्नाम दुर्जयः} %1-47-17

\twolineshloka
{इक्ष्वाकोस्तु प्रसादेन सर्वे वैशालिका नृपाः}
{दीर्घायुषो महात्मानो वीर्यवन्तः सुधार्मिकाः} %1-47-18

\twolineshloka
{इहाद्य रजनीमेकां सुखं स्वप्स्यामहे वयम्}
{श्वः प्रभाते नरश्रेष्ठ जनकं द्रष्टुमर्हसि} %1-47-19

\twolineshloka
{सुमतिस्तु महातेजा विश्वामित्रमुपागतम्}
{श्रुत्वा नरवरश्रेष्ठः प्रत्यागच्छन्महायशाः} %1-47-20

\twolineshloka
{पूजां च परमां कृत्वा सोपाध्यायः सबान्धवः}
{प्राञ्जलिः कुशलं पृष्ट्वा विश्वामित्रमथाब्रवीत्} %1-47-21

\twolineshloka
{धन्योऽस्म्यनुगृहीतोऽस्मि यस्य मे विषयं मुने}
{सम्प्राप्तो दर्शनं चैव नास्ति धन्यतरो मम} %1-47-22


॥इत्यार्षे श्रीमद्रामायणे वाल्मीकीये आदिकाव्ये बालकाण्डे विशालागमनम् नाम सप्तचत्वारिंशः सर्गः ॥१-४७॥
