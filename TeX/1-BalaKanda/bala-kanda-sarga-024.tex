\sect{चतुर्विंशः सर्गः — ताटकावनप्रवेशः}

\twolineshloka
{ततः प्रभाते विमले कृताह्निकमरिन्दमौ}
{विश्वामित्रं पुरस्कृत्य नद्यास्तीरमुपागतौ} %1-24-1

\twolineshloka
{ते च सर्वे महात्मानो मुनयः संशितव्रताः}
{उपस्थाप्य शुभां नावं विश्वामित्रमथाब्रुवन्} %1-24-2

\twolineshloka
{आरोहतु भवान् नावं राजपुत्रपुरस्कृतः}
{अरिष्टं गच्छ पन्थानं मा भूत् कालस्य पर्ययः} %1-24-3

\twolineshloka
{विश्वामित्रस्तथेत्युक्त्वा तानृषीन् प्रतिपूज्य च}
{ततार सहितस्ताभ्यां सरितं सागरङ्गमाम्} %1-24-4

\twolineshloka
{तत्र शुश्राव वै शब्दं तोयसंरम्भवर्धितम्}
{मध्यमागम्य तोयस्य तस्य शब्दस्य निश्चयम्} %1-24-5

\twolineshloka
{ज्ञातुकामो महातेजाः सह रामः कनीयसा}
{अथ रामः सरिन्मध्ये पप्रच्छ मुनिपुङ्गवम्} %1-24-6

\twolineshloka
{वारिणो भिद्यमानस्य किमयं तुमुलो ध्वनिः}
{राघवस्य वचः श्रुत्वा कौतूहलसमन्वितम्} %1-24-7

\twolineshloka
{कथयामास धर्मात्मा तस्य शब्दस्य निश्चयम्}
{कैलासपर्वते राम मनसा निर्मितं परम्} %1-24-8

\twolineshloka
{ब्रह्मणा नरशार्दूल तेनेदं मानसं सरः}
{तस्मात् सुस्राव सरसः सायोध्यामुपगूहते} %1-24-9

\twolineshloka
{सरःप्रवृत्ता सरयूः पुण्या ब्रह्मसरश्च्युता}
{तस्यायमतुलः शब्दो जाह्नवीमभिवर्तते} %1-24-10

\twolineshloka
{वारिसङ्क्षोभजो राम प्रणामं नियतः कुरु}
{ताभ्यां तु तावुभौ कृत्वा प्रणाममतिधार्मिकौ} %1-24-11

\twolineshloka
{तीरं दक्षिणमासाद्य जग्मतुर्लघुविक्रमौ}
{स वनं घोरसङ्काशं दृष्ट्वा नरवरात्मजः} %1-24-12

\twolineshloka
{अविप्रहतमैक्ष्वाकः पप्रच्छ मुनिपुङ्गवम्}
{अहो वनमिदं दुर्गं झिल्लिकागणसंयुतम्} %1-24-13

\twolineshloka
{भैरवैः श्वापदैः कीर्णं शकुन्तैर्दारुणारवैः}
{नानाप्रकारैः शकुनैर्वाश्यद्भिर्भैरवस्वनैः} %1-24-14

\twolineshloka
{सिंहव्याघ्रवराहैश्च वारणैश्चापि शोभितम्}
{धवाश्वकर्णककुभैर्बिल्वतिन्दुकपाटलैः} %1-24-15

\twolineshloka
{सङ्कीर्णं बदरीभिश्च किं न्विदं दारुणं वनम्}
{तमुवाच महातेजा विश्वामित्रो महामुनिः} %1-24-16

\twolineshloka
{श्रूयतां वत्स काकुत्स्थ यस्यैतद् दारुणं वनम्}
{एतौ जनपदौ स्फीतौ पूर्वमास्तां नरोत्तम} %1-24-17

\twolineshloka
{मलदाश्च करूषाश्च देवनिर्माणनिर्मितौ}
{पुरा वृत्रवधे राम मलेन समभिप्लुतम्} %1-24-18

\twolineshloka
{क्षुधा चैव सहस्राक्षं ब्रह्महत्या समाविशत्}
{तमिन्द्रं मलिनं देवा ऋषयश्च तपोधनाः} %1-24-19

\twolineshloka
{कलशैः स्नापयामासुर्मलं चास्य प्रमोचयन्}
{इह भूम्यां मलं दत्त्वा देवाः कारूषमेव च} %1-24-20

\twolineshloka
{शरीरजं महेन्द्रस्य ततो हर्षं प्रपेदिरे}
{निर्मलो निष्करूषश्च शुद्ध इन्द्रो यथाभवत्} %1-24-21

\twolineshloka
{ददौ देशस्य सुप्रीतो वरं प्रादादनुत्तमम्}
{इमौ जनपदौ स्फीतौ ख्यातिं लोके गमिष्यतः} %1-24-22

\twolineshloka
{मलदाश्च करूषाश्च ममाङ्गमलधारिणौ}
{साधु साध्विति तं देवाः पाकशासनमब्रुवन्} %1-24-23

\twolineshloka
{देशस्य पूजां तां दृष्ट्वा कृतां शक्रेण धीमता}
{एतौ जनपदौ स्फीतौ दीर्घकालमरिन्दम} %1-24-24

\twolineshloka
{मलदाश्च करूषाश्च मुदिता धनधान्यतः}
{कस्यचित्त्वथ कालस्य यक्षिणी कामरूपिणी} %1-24-25

\twolineshloka
{बलं नागसहस्रस्य धारयन्ती तदा ह्यभूत्}
{ताटका नाम भद्रं ते भार्या सुन्दस्य धीमतः} %1-24-26

\twolineshloka
{मारीचो राक्षसः पुत्रो यस्याः शक्रपराक्रमः}
{वृत्तबाहुर्महाशीर्षो विपुलास्यतनुर्महान्} %1-24-27

\twolineshloka
{राक्षसो भैरवाकारो नित्यं त्रासयते प्रजाः}
{इमौ जनपदौ नित्यं विनाशयति राघव} %1-24-28

\twolineshloka
{मलदांश्च करूषांश्च ताटका दुष्टचारिणी}
{सेयं पन्थानमावृत्य वसत्यत्यर्धयोजने} %1-24-29

\twolineshloka
{अत एव च गन्तव्यं ताटकाया वनं यतः}
{स्वबाहुबलमाश्रित्य जहीमां दुष्टचारिणीम्} %1-24-30

\twolineshloka
{मन्नियोगादिमं देशं कुरु निष्कण्टकं पुनः}
{नहि कश्चिदिमं देशं शक्तो ह्यागन्तुमीदृशम्} %1-24-31

\threelineshloka
{यक्षिण्या घोरया राम उत्सादितमसह्यया}
{एतत्ते सर्वमाख्यातं यथैतद् दारुणं वनम्}
{यक्ष्या चोत्सादितं सर्वमद्यापि न निवर्तते} %1-24-32


॥इत्यार्षे श्रीमद्रामायणे वाल्मीकीये आदिकाव्ये बालकाण्डे ताटकावनप्रवेशः नाम चतुर्विंशः सर्गः ॥१-२४॥
