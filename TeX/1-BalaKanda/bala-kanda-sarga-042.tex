\sect{द्विचत्वारिंशः सर्गः — भगीरथवरप्रदानम्}

\twolineshloka
{कालधर्मं गते राम सगरे प्रकृतीजनाः}
{राजानं रोचयामासुरंशुमन्तं सुधार्मिकम्} %1-42-1

\twolineshloka
{स राजा सुमहानासीदंशुमान् रघुनन्दन}
{तस्य पुत्रो महानासीद् दिलीप इति विश्रुतः} %1-42-2

\twolineshloka
{तस्मै राज्यं समादिश्य दिलीपे रघुनन्दन}
{हिमवच्छिखरे रम्ये तपस्तेपे सुदारुणम्} %1-42-3

\twolineshloka
{द्वात्रिंशच्छतसाहस्रं वर्षाणि सुमहायशाः}
{तपोवनगतो राजा स्वर्गं लेभे तपोधनः} %1-42-4

\twolineshloka
{दिलीपस्तु महातेजाः श्रुत्वा पैतामहं वधम्}
{दुःखोपहतया बुद्ध्या निश्चयं नाध्यगच्छत} %1-42-5

\twolineshloka
{कथं गङ्गावतरणं कथं तेषां जलक्रिया}
{तारयेयं कथं चैतानिति चिन्तापरोऽभवत्} %1-42-6

\twolineshloka
{तस्य चिन्तयतो नित्यं धर्मेण विदितात्मनः}
{पुत्रो भगीरथो नाम जज्ञे परमधार्मिकः} %1-42-7

\twolineshloka
{दिलीपस्तु महातेजा यज्ञैर्बहुभिरिष्टवान्}
{त्रिंशद्वर्षसहस्राणि राजा राज्यमकारयत्} %1-42-8

\twolineshloka
{अगत्वा निश्चयं राजा तेषामुद्धरणं प्रति}
{व्याधिना नरशार्दूल कालधर्ममुपेयिवान्} %1-42-9

\twolineshloka
{इन्द्रलोकं गतो राजा स्वार्जितेनैव कर्मणा}
{राज्ये भगीरथं पुत्रमभिषिच्य नरर्षभः} %1-42-10

\twolineshloka
{भगीरथस्तु राजर्षिर्धार्मिको रघुनन्दन}
{अनपत्यो महाराजः प्रजाकामः स च प्रजाः} %1-42-11

\twolineshloka
{मन्त्रिष्वाधाय तद् राज्यं गङ्गावतरणे रतः}
{तपो दीर्घं समातिष्ठद् गोकर्णे रघुनन्दन} %1-42-12

\twolineshloka
{ऊर्ध्वबाहुः पञ्चतपा मासाहारो जितेन्द्रियः}
{तस्य वर्षसहस्राणि घोरे तपसि तिष्ठतः} %1-42-13

\twolineshloka
{अतीतानि महाबाहो तस्य राज्ञो महात्मनः}
{सुप्रीतो भगवान् ब्रह्मा प्रजानां प्रभुरीश्वरः} %1-42-14

\twolineshloka
{ततः सुरगणैः सार्धमुपागम्य पितामहः}
{भगीरथं महात्मानं तप्यमानमथाब्रवीत्} %1-42-15

\twolineshloka
{भगीरथ महाराज प्रीतस्तेऽहं जनाधिप}
{तपसा च सुतप्तेन वरं वरय सुव्रत} %1-42-16

\twolineshloka
{तमुवाच महातेजाः सर्वलोकपितामहम्}
{भगीरथो महाबाहुः कृताञ्जलिपुटः स्थितः} %1-42-17

\twolineshloka
{यदि मे भगवान् प्रीतो यद्यस्ति तपसः फलम्}
{सगरस्यात्मजाः सर्वे मत्तः सलिलमाप्नुयुः} %1-42-18

\twolineshloka
{गङ्गायाः सलिलक्लिन्ने भस्मन्येषां महात्मनाम्}
{स्वर्गं गच्छेयुरत्यन्तं सर्वे च प्रपितामहाः} %1-42-19

\twolineshloka
{देव याचे ह संतत्यै नावसीदेत् कुलं च नः}
{इक्ष्वाकूणां कुले देव एष मेऽस्तु वरः परः} %1-42-20

\twolineshloka
{उक्तवाक्यं तु राजानं सर्वलोकपितामहः}
{प्रत्युवाच शुभां वाणीं मधुरां मधुराक्षराम्} %1-42-21

\twolineshloka
{मनोरथो महानेष भगीरथ महारथ}
{एवं भवतु भद्रं ते इक्ष्वाकुकुलवर्धन} %1-42-22

\twolineshloka
{इयं हैमवती ज्येष्ठा गङ्गा हिमवतः सुता}
{तां वै धारयितुं राजन् हरस्तत्र नियुज्यताम्} %1-42-23

\twolineshloka
{गङ्गायाः पतनं राजन् पृथिवी न सहिष्यते}
{तां वै धारयितुं राजन् नान्यं पश्यामि शूलिनः} %1-42-24

\twolineshloka
{तमेवमुक्त्वा राजानं गङ्गां चाभाष्य लोककृत्}
{जगाम त्रिदिवं देवैः सर्वैः सह मरुद्गणैः} %1-42-25


॥इत्यार्षे श्रीमद्रामायणे वाल्मीकीये आदिकाव्ये बालकाण्डे भगीरथवरप्रदानम् नाम द्विचत्वारिंशः सर्गः ॥१-४२॥
