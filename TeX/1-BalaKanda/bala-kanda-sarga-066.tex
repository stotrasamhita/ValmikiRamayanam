\sect{षट्षष्ठितमः सर्गः — धनुःप्रसङ्गः}

\twolineshloka
{ततः प्रभाते विमले कृतकर्मा नराधिपः}
{विश्वामित्रं महात्मानमाजुहाव सराघवम्} %1-66-1

\twolineshloka
{तमर्चयित्वा धर्मात्मा शास्त्रदृष्टेन कर्मणा}
{राघवौ च महात्मानौ तदा वाक्यमुवाच ह} %1-66-2

\twolineshloka
{भगवन् स्वागतं तेऽस्तु किं करोमि तवानघ}
{भवानाज्ञापयतु मामाज्ञाप्यो भवता ह्यहम्} %1-66-3

\twolineshloka
{एवमुक्तः स धर्मात्मा जनकेन महात्मना}
{प्रत्युवाच मुनिश्रेष्ठो वाक्यं वाक्यविशारदः} %1-66-4

\twolineshloka
{पुत्रौ दशरथस्येमौ क्षत्रियौ लोकविश्रुतौ}
{द्रष्टुकामौ धनुःश्रेष्ठं यदेतत्त्वयि तिष्ठति} %1-66-5

\twolineshloka
{एतद् दर्शय भद्रं ते कृतकामौ नृपात्मजौ}
{दर्शनादस्य धनुषो यथेष्टं प्रतियास्यतः} %1-66-6

\twolineshloka
{एवमुक्तस्तु जनकः प्रत्युवाच महामुनिम्}
{श्रूयतामस्य धनुषो यदर्थमिह तिष्ठति} %1-66-7

\twolineshloka
{देवरात इति ख्यातो निमेर्ज्येष्ठो महीपतिः}
{न्यासोऽयं तस्य भगवन् हस्ते दत्तो महात्मनः} %1-66-8

\twolineshloka
{दक्षयज्ञवधे पूर्वं धनुरायम्य वीर्यवान्}
{विध्वंस्य त्रिदशान् रोषात् सलीलमिदमब्रवीत्} %1-66-9

\twolineshloka
{यस्माद् भागार्थिनो भागं नाकल्पयत मे सुराः}
{वरांगानि महार्हाणि धनुषा शातयामि वः} %1-66-10

\twolineshloka
{ततो विमनसः सर्वे देवा वै मुनिपुंगव}
{प्रसादयन्त देवेशं तेषां प्रीतोऽभवद् भवः} %1-66-11

\twolineshloka
{प्रीतियुक्तस्तु सर्वेषां ददौ तेषां महात्मनाम्}
{तदेतद् देवदेवस्य धनूरत्नं महात्मनः} %1-66-12

\twolineshloka
{न्यासभूतं तदा न्यस्तमस्माकं पूर्वजे विभौ}
{अथ मे कृषतः क्षेत्रं लांगलादुत्थिता ततः} %1-66-13

\twolineshloka
{क्षेत्रं शोधयता लब्धा नाम्ना सीतेति विश्रुता}
{भूतलादुत्थिता सा तु व्यवर्धत ममात्मजा} %1-66-14

\twolineshloka
{वीर्यशुल्केति मे कन्या स्थापितेयमयोनिजा}
{भूतलादुत्थितां तां तु वर्धमानां ममात्मजाम्} %1-66-15

\twolineshloka
{वरयामासुरागत्य राजानो मुनिपुंगव}
{तेषां वरयतां कन्यां सर्वेषां पृथिवीक्षिताम्} %1-66-16

\twolineshloka
{वीर्यशुल्केति भगवन् न ददामि सुतामहम्}
{ततः सर्वे नृपतयः समेत्य मुनिपुंगव} %1-66-17

\twolineshloka
{मिथिलामप्युपागम्य वीर्यं जिज्ञासवस्तदा}
{तेषां जिज्ञासमानानां शैवं धनुरुपाहृतम्} %1-66-18

\twolineshloka
{न शेकुर्ग्रहणे तस्य धनुषस्तोलनेऽपि वा}
{तेषां वीर्यवतां वीर्यमल्पं ज्ञात्वा महामुने} %1-66-19

\twolineshloka
{प्रत्याख्याता नृपतयस्तन्निबोध तपोधन}
{ततः परमकोपेन राजानो मुनिपुंगव} %1-66-20

\twolineshloka
{अरुन्धन् मिथिलां सर्वे वीर्यसंदेहमागताः}
{आत्मानमवधूतं मे विज्ञाय नृपपुंगवाः} %1-66-21

\twolineshloka
{रोषेण महताविष्टाः पीडयन् मिथिलां पुरीम्}
{ततः संवत्सरे पूर्णे क्षयं यातानि सर्वशः} %1-66-22

\twolineshloka
{साधनानि मुनिश्रेष्ठ ततोऽहं भृशदुःखितः}
{ततो देवगणान् सर्वांस्तपसाहं प्रसादयम्} %1-66-23

\twolineshloka
{ददुश्च परमप्रीताश्चतुरंगबलं सुराः}
{ततो भग्ना नृपतयो हन्यमाना दिशो ययुः} %1-66-24

\twolineshloka
{अवीर्या वीर्यसंदिग्धाः सामात्याः पापकारिणः}
{तदेतन्मुनिशार्दूल धनुः परमभास्वरम्} %1-66-25

\threelineshloka
{रामलक्ष्मणयोश्चापि दर्शयिष्यामि सुव्रत}
{यद्यस्य धनुषो रामः कुर्यादारोपणं मुने}
{सुतामयोनिजां सीतां दद्यां दाशरथेरहम्} %1-66-26


॥इत्यार्षे श्रीमद्रामायणे वाल्मीकीये आदिकाव्ये बालकाण्डे धनुःप्रसङ्गः नाम षट्षष्ठितमः सर्गः ॥१-६६॥
