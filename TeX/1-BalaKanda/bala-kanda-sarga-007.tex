\sect{सप्तमः सर्गः — अमात्यवर्णनम्}

\twolineshloka
{तस्यामात्या गुणैरासन्निक्ष्वाकोः सुमहात्मनः}
{मन्त्रज्ञाश्चेङ्गितज्ञाश्च नित्यं प्रियहिते रताः} %1-7-1

\twolineshloka
{अष्टौ बभूवुर्वीरस्य तस्यामात्या यशस्विनः}
{शुचयश्चानुरक्ताश्च राजकृत्येषु नित्यशः} %1-7-2

\twolineshloka
{धृष्टिर्जयन्तो विजयः सुराष्ट्रो राष्ट्र्वर्धनः}
{अकोपो धर्मपालश्च सुमन्त्रश्चाष्टमोऽर्थवित्} %1-7-3

\twolineshloka
{ऋत्विजौ द्वावभिमतौ तस्यास्तामृषिसत्तमौ}
{वसिष्ठो वामदेवश्च मन्त्रिणश्च तथापरे} %1-7-4

\twolineshloka
{सुयज्ञोऽप्यथ जाबालिः काश्यपोऽप्यथ गौतमः}
{मार्कण्डेयस्तु दीर्घायुस्तथा कात्यायनो द्विजः} %1-7-5

\twolineshloka
{एतैर्ब्रह्मर्षिभिर्नित्यमृत्विजस्तस्य पौर्वकाः}
{विद्याविनीता ह्रीमन्तः कुशला नियतेन्द्रियाः} %1-7-6

\twolineshloka
{श्रीमन्तश्च महात्मानः शास्त्रज्ञा दृढविक्रमाः}
{कीर्तिमन्तः प्रणिहिता यथावचनकारिणः} %1-7-7

\twolineshloka
{तेजःक्षमायशःप्राप्ताः स्मितपूर्वाभिभाषिणः}
{क्रोधात् कामार्थहेतोर्वा न ब्रूयुरनृतं वचः} %1-7-8

\twolineshloka
{तेषामविदितं किंचित् स्वेषु नास्ति परेषु वा}
{क्रियमाणं कृतं वापि चारेणापि चिकीर्षितम्} %1-7-9

\twolineshloka
{कुशला व्यवहारेषु सौहृदेषु परीक्षिताः}
{प्राप्तकालं यथा दण्डं धारयेयुः सुतेष्वपि} %1-7-10

\twolineshloka
{कोशसंग्रहणे युक्ता बलस्य च परिग्रहे}
{अहितं चापि पुरुषं न हिंस्युरविदूषकम्} %1-7-11

\twolineshloka
{वीराश्च नियतोत्साहा राजशास्त्रमनुष्ठिताः}
{शुचीनां रक्षितारश्च नित्यं विषयवासिनाम्} %1-7-12

\twolineshloka
{ब्रह्मक्षत्रमहिंसन्तस्ते कोशं समपूरयन्}
{सुतीक्ष्णदण्डाः सम्प्रेक्ष्य पुरुषस्य बलाबलम्} %1-7-13

\twolineshloka
{शुचीनामेकबुद्धीनां सर्वेषां सम्प्रजानताम्}
{नासीत्पुरे वा राष्ट्रे वा मृषावादी नरः क्वचित्} %1-7-14

\twolineshloka
{क्वचिन्न दुष्टस्तत्रासीत् परदाररतिर्नरः}
{प्रशान्तं सर्वमेवासीद् राष्ट्रं पुरवरं च तत्} %1-7-15

\twolineshloka
{सुवाससः सुवेशाश्च ते च सर्वे शुचिव्रताः}
{हितार्थाश्च नरेन्द्रस्य जाग्रतो नयचक्षुषा} %1-7-16

\twolineshloka
{गुरोर्गुणगृहीताश्च प्रख्याताश्च पराक्रमैः}
{विदेशेष्वपि विज्ञाताः सर्वतो बुद्धिनिश्चयाः} %1-7-17

\twolineshloka
{अभितो गुणवन्तश्च न चासन् गुणवर्जिताः}
{सन्धिविग्रहतत्वज्ञाः प्रकृत्या सम्पदान्विताः} %1-7-18

\twolineshloka
{मंत्रसंवरणे शक्ताः शक्ताः सूक्ष्मासु बुद्धिषु}
{नीतिशास्त्रविशेषज्ञाः सततं प्रियवादिनः} %1-7-19

\twolineshloka
{ईदृशैस्तैरमात्यैश्च राजा दशरथोऽनघः}
{उपपन्नो गुणोपेतैरन्वशासद् वसुन्धराम्} %1-7-20

\twolineshloka
{अवेक्ष्यमाणश्चारेण प्रजा धर्मेण रक्षयन्}
{प्रजानां पालनं कुर्वन्नधर्मं परिवर्जयन्} %1-7-21

\twolineshloka
{विश्रुतस्त्रिषु लोकेषु वदान्यः सत्यसंगरः}
{स तत्र पुरुषव्याघ्रः शशास पृथिवीमिमाम्} %1-7-22

\threelineshloka
{नाध्यगच्छद्विशिष्टं वा तुल्यं वा शत्रुमात्मनः}
{मित्रवान्नतसामन्तः प्रतापहतकण्टकः}
{स शशास जगद् राजा दिवि देवपतिर्यथा} %1-7-23

\twolineshloka
{तैर्मन्त्रिभिर्मन्त्रहिते निविष्टैर्वृतोऽनुरक्तैः कुशलैः समर्थैः}
{स पार्थिवो दीप्तिमवाप युक्तस्तेजोमयैर्गोभिरिवोदितोऽर्कः} %1-7-24


॥इत्यार्षे श्रीमद्रामायणे वाल्मीकीये आदिकाव्ये बालकाण्डे अमात्यवर्णनम् नाम सप्तमः सर्गः ॥१-७॥
