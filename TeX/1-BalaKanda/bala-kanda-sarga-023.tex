\sect{त्रयोविंशः सर्गः — कामाश्रमवासः}

\twolineshloka
{प्रभातायां तु शर्वर्यां विश्वामित्रो महामुनिः}
{अभ्यभाषत काकुत्स्थं शयानं पर्णसंस्तरे} %1-23-1

\twolineshloka
{कौसल्या सुप्रजा राम पूर्वा सन्ध्या प्रवर्तते}
{उत्तिष्ठ नरशार्दूल कर्तव्यं दैवमाह्निकम्} %1-23-2

\twolineshloka
{तस्यर्षेः परमोदारं वचः श्रुत्वा नरोत्तमौ}
{स्नात्वा कृतोदकौ वीरौ जेपतुः परमं जपम्} %1-23-3

\twolineshloka
{कृताह्निकौ महावीर्यौ विश्वामित्रं तपोधनम्}
{अभिवाद्यातिसंहृष्टौ गमनायाभितस्थतुः} %1-23-4

\twolineshloka
{तौ प्रयान्तौ महावीर्यौ दिव्यां त्रिपथगां नदीम्}
{ददृशाते ततस्तत्र सरय्वाः सङ्गमे शुभे} %1-23-5

\twolineshloka
{तत्राश्रमपदं पुण्यमृषीणां भावितात्मनाम्}
{बहुवर्षसहस्राणि तप्यतां परमं तपः} %1-23-6

\twolineshloka
{तं दृष्ट्वा परमप्रीतौ राघवौ पुण्यमाश्रमम्}
{ऊचतुस्तं महात्मानं विश्वामित्रमिदं वचः} %1-23-7

\twolineshloka
{कस्यायमाश्रमः पुण्यः को न्वस्मिन्वसते पुमान्}
{भगवन् श्रोतुमिच्छावः परं कौतूहलं हि नौ} %1-23-8

\twolineshloka
{तयोस्तद्वचनं श्रुत्वा प्रहस्य मुनिपुङ्गवः}
{अब्रवीच्छ्रूयतां राम यस्यायं पूर्व आश्रमः} %1-23-9

\twolineshloka
{कन्दर्पो मूर्तिमानासीत्काम इत्युच्यते बुधैः}
{तपस्यन्तमिह स्थाणुं नियमेन समाहितम्} %1-23-10

\twolineshloka
{कृतोद्वाहं तु देवेशं गच्छन्तं समरुद्गणम्}
{धर्षयामास दुर्मेधा हुङ्कृतश्च महात्मना} %1-23-11

\twolineshloka
{अवध्यातश्च रुद्रेण चक्षुषा रघुनन्दन}
{व्यशीर्यन्त शरीरात् स्वात् सर्वगात्राणि दुर्मतेः} %1-23-12

\twolineshloka
{तस्य गात्रं हतं तत्र निर्दग्धस्य महात्मना}
{अशरीरः कृतः कामः क्रोधाद् देवेश्वरेण ह} %1-23-13

\twolineshloka
{अनङ्ग इति विख्यातस्तदा प्रभृति राघव}
{स चाङ्गविषयः श्रीमान् यत्राङ्गं स मुमोच ह} %1-23-14

\twolineshloka
{तस्यायमाश्रमः पुण्यस्तस्येमे मुनयः पुरा}
{शिष्या धर्मपरा वीर तेषां पापं न विद्यते} %1-23-15

\twolineshloka
{इहाद्य रजनीं राम वसेम शुभदर्शन}
{पुण्ययोः सरितोर्मध्ये श्वस्तरिष्यामहे वयम्} %1-23-16

\twolineshloka
{अभिगच्छामहे सर्वे शुचयः पुण्यमाश्रमम्}
{इह वासः परोऽस्माकं सुखं वस्त्यामहे निशाम्} %1-23-17

\twolineshloka
{स्नाताश्च कृतजप्याश्च हुतहव्या नरोत्तम}
{तेषां संवदतां तत्र तपोदीर्घेण चक्षुषा} %1-23-18

\twolineshloka
{विज्ञाय परमप्रीता मुनयो हर्षमागमन्}
{अर्घ्यं पाद्यं तथाऽऽतिथ्यं निवेद्य कुशिकात्मजे} %1-23-19

\twolineshloka
{रामलक्ष्मणयोः पश्चादकुर्वन्नतिथिक्रियाम्}
{सत्कारं समनुप्राप्य कथाभिरभिरञ्जयन्} %1-23-20

\twolineshloka
{यथार्हमजपन् सन्ध्यामृषयस्ते समाहिताः}
{तत्र वासिभिरानीता मुनिभिः सुव्रतैः सह} %1-23-21

\threelineshloka
{न्यवसन् सुसुखं तत्र कामाश्रमपदे तथा}
{कथाभिरभिरामभिरभिरामौ नृपात्मजौ}
{रमयामास धर्मात्मा कौशिको मुनिपुङ्गवः} %1-23-22


॥इत्यार्षे श्रीमद्रामायणे वाल्मीकीये आदिकाव्ये बालकाण्डे कामाश्रमवासः नाम त्रयोविंशः सर्गः ॥१-२३॥
