\sect{चतुःषष्ठितमः सर्गः — रम्भाशापः}

\twolineshloka
{सुरकार्यमिदं रम्भे कर्तव्यं सुमहत् त्वया}
{लोभनं कौशिकस्येह काममोहसमन्वितम्} %1-64-1

\twolineshloka
{तथोक्ता साप्सरा राम सहस्राक्षेण धीमता}
{व्रीडिता प्राञ्जलिर्वाक्यं प्रत्युवाच सुरेश्वरम्} %1-64-2

\twolineshloka
{अयं सुरपते घोरो विश्वामित्रो महामुनिः}
{क्रोधमुत्स्रक्ष्यते घोरं मयि देव न संशयः} %1-64-3

\twolineshloka
{ततो हि मे भयं देव प्रसादं कर्तुुमर्हसि}
{एवमुक्तस्तया राम सभयं भीतया तदा} %1-64-4

\twolineshloka
{तामुवाच सहस्राक्षो वेपमानां कृताञ्जलिम्}
{मा भैषी रम्भे भद्रं ते कुरुष्व मम शासनम्} %1-64-5

\twolineshloka
{कोकिलो हृदयग्राही माधवे रुचिरद्रुमे}
{अहं कन्दर्पसहितः स्थास्यामि तव पार्श्वतः} %1-64-6

\twolineshloka
{त्वं हि रूपं बहुगुणं कृत्वा परमभास्वरम्}
{तमृषिं कौशिकं भद्रे भेदयस्व तपस्विनम्} %1-64-7

\twolineshloka
{सा श्रुत्वा वचनं तस्य कृत्वा रूपमनुत्तमम्}
{लोभयामास ललिता विश्वामित्रं शुचिस्मिता} %1-64-8

\twolineshloka
{कोकिलस्य तु शुश्राव वल्गु व्याहरतः स्वनम्}
{सम्प्रहृष्टेन मनसा स चैनामन्ववैक्षत} %1-64-9

\twolineshloka
{अथ तस्य च शब्देन गीतेनाप्रतिमेन च}
{दर्शनेन च रम्भाया मुनिः संदेहमागतः} %1-64-10

\twolineshloka
{सहस्राक्षस्य तत्सर्वं विज्ञाय मुनिपुंगवः}
{रम्भां क्रोधसमाविष्टः शशाप कुशिकात्मजः} %1-64-11

\twolineshloka
{यन्मां लोभयसे रम्भे कामक्रोधजयैषिणम्}
{दशवर्षसहस्राणि शैली स्थास्यसि दुर्भगे} %1-64-12

\twolineshloka
{ब्राह्मणः सुमहातेजास्तपोबलसमन्वितः}
{उद्धरिष्यति रम्भे त्वां मत्क्रोधकलुषीकृताम्} %1-64-13

\twolineshloka
{एवमुक्त्वा महातेजा विश्वामित्रो महामुनिः}
{अशक्नुवन् धारयितुं कोपं संतापमात्मनः} %1-64-14

\twolineshloka
{तस्य शापेन महता रम्भा शैली तदाभवत्}
{वचः श्रुत्वा च कन्दर्पो महर्षेः स च निर्गतः} %1-64-15

\twolineshloka
{कोपेन च महातेजास्तपोऽपहरणे कृते}
{इन्द्रियैरजितै राम न लेभे शान्तिमात्मनः} %1-64-16

\twolineshloka
{बभूवास्य मनश्चिन्ता तपोऽपहरणे कृते}
{नैवं क्रोधं गमिष्यामि न च वक्ष्ये कथंचन} %1-64-17

\twolineshloka
{अथवा नोच्छ्वसिष्यामि संवत्सरशतान्यपि}
{अहं हि शोषयिष्यामि आत्मानं विजितेन्द्रियः} %1-64-18

\twolineshloka
{तावद् यावद्धि मे प्राप्तं ब्राह्मण्यं तपसार्जितम्}
{अनुच्छ्वसन्नभुञ्जानस्तिष्ठेयं शाश्वतीः समाः} %1-64-19

\threelineshloka
{नहि मे तप्यमानस्य क्षयं यास्यन्ति मूर्तयः}
{एवं वर्षसहस्रस्य दीक्षां स मुनिपुंगवः}
{चकाराप्रतिमां लोके प्रतिज्ञां रघुनन्दन} %1-64-20


॥इत्यार्षे श्रीमद्रामायणे वाल्मीकीये आदिकाव्ये बालकाण्डे रम्भाशापः नाम चतुःषष्ठितमः सर्गः ॥१-६४॥
