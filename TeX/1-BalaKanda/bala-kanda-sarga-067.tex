\sect{सप्तषष्ठितमः सर्गः — धनुर्भङ्गः}

\twolineshloka
{जनकस्य वचः श्रुत्वा विश्वामित्रो महामुनिः}
{धनुर्दर्शय रामाय इति होवाच पार्थिवम्} %1-67-1

\twolineshloka
{ततः स राजा जनकः सचिवान् व्यादिदेश ह}
{धनुरानीयतां दिव्यं गन्धमाल्यानुलेपितम्} %1-67-2

\twolineshloka
{जनकेन समादिष्टाः सचिवाः प्राविशन् पुरम्}
{तद्धनुः पुरतः कृत्वा निर्जग्मुरमितौजसः} %1-67-3

\twolineshloka
{नृणां शतानि पञ्चाशद् व्यायतानां महात्मनाम्}
{मञ्जूषामष्टचक्रां तां समूहुस्ते कथंचन} %1-67-4

\twolineshloka
{तामादाय सुमञ्जूषामायसीं यत्र तद्धनुः}
{सुरोपमं ते जनकमूचुर्नृपतिमन्त्रिणः} %1-67-5

\twolineshloka
{इदं धनुर्वरं राजन् पूजितं सर्वराजभिः}
{मिथिलाधिप राजेन्द्र दर्शनीयं यदीच्छसि} %1-67-6

\twolineshloka
{तेषां नृपो वचः श्रुत्वा कृताञ्जलिरभाषत}
{विश्वामित्रं महात्मानं तावुभौ रामलक्ष्मणौ} %1-67-7

\twolineshloka
{इदं धनुर्वरं ब्रह्मञ्जनकैरभिपूजितम्}
{राजभिश्च महावीर्यैरशक्तैः पूरितं तदा} %1-67-8

\twolineshloka
{नैतत् सुरगणाः सर्वे सासुरा न च राक्षसाः}
{गन्धर्वयक्षप्रवराः सकिन्नरमहोरगाः} %1-67-9

\twolineshloka
{क्व गतिर्मानुषाणां च धनुषोऽस्य प्रपूरणे}
{आरोपणे समायोगे वेपने तोलने तथा} %1-67-10

\twolineshloka
{तदेतद् धनुषां श्रेष्ठमानीतं मुनिपुंगव}
{दर्शयैतन्महाभाग अनयो राजपुत्रयोः} %1-67-11

\twolineshloka
{विश्वामित्रः सरामस्तु श्रुत्वा जनकभाषितम्}
{वत्स राम धनुः पश्य इति राघवमब्रवीत्} %1-67-12

\twolineshloka
{महर्षेर्वचनाद् रामो यत्र तिष्ठति तद्धनुः}
{मञ्जूषां तामपावृत्य दृष्ट्वा धनुरथाब्रवीत्} %1-67-13

\twolineshloka
{इदं धनुर्वरं दिव्यं संस्पृशामीह पाणिना}
{यत्नवांश्च भविष्यामि तोलने पूरणेऽपि वा} %1-67-14

\twolineshloka
{बाढमित्यब्रवीद् राजा मुनिश्च समभाषत}
{लीलया स धनुर्मध्ये जग्राह वचनान्मुनेः} %1-67-15

\twolineshloka
{पश्यतां नृसहस्राणां बहूनां रघुनन्दनः}
{आरोपयत् स धर्मात्मा सलीलमिव तद्धनुः} %1-67-16

\twolineshloka
{आरोपयित्वा मौर्वीं च पूरयामास तद्धनुः}
{तद् बभञ्ज धनुर्मध्ये नरश्रेष्ठो महायशाः} %1-67-17

\twolineshloka
{तस्य शब्दो महानासीन्निर्घातसमनिःस्वनः}
{भूमिकम्पश्च सुमहान् पर्वतस्येव दीर्यतः} %1-67-18

\twolineshloka
{निपेतुश्च नराः सर्वे तेन शब्देन मोहिताः}
{वर्जयित्वा मुनिवरं राजानं तौ च राघवौ} %1-67-19

\twolineshloka
{प्रत्याश्वस्ते जने तस्मिन् राजा विगतसाध्वसः}
{उवाच प्राञ्जलिर्वाक्यं वाक्यज्ञो मुनिपुंगवम्} %1-67-20

\twolineshloka
{भगवन् दृष्टवीर्यो मे रामो दशरथात्मजः}
{अत्यद्भुतमचिन्त्यं च अतर्कितमिदं मया} %1-67-21

\twolineshloka
{जनकानां कुले कीर्तिमाहरिष्यति मे सुता}
{सीता भर्तारमासाद्य रामं दशरथात्मजम्} %1-67-22

\twolineshloka
{मम सत्या प्रतिज्ञा सा वीर्यशुल्केति कौशिक}
{सीता प्राणैर्बहुमता देया रामाय मे सुता} %1-67-23

\twolineshloka
{भवतोऽनुमते ब्रह्मन् शीघ्रं गच्छन्तु मन्त्रिणः}
{मम कौशिक भद्रं ते अयोध्यां त्वरिता रथैः} %1-67-24

\twolineshloka
{राजानं प्रश्रितैर्वाक्यैरानयन्तु पुरं मम}
{प्रदानं वीर्यशुल्कायाः कथयन्तु च सर्वशः} %1-67-25

\twolineshloka
{मुनिगुप्तौ च काकुत्स्थौ कथयन्तु नृपाय वै}
{प्रीतियुक्तं तु राजानमानयन्तु सुशीघ्रगाः} %1-67-26

\threelineshloka
{कौशिकस्तु तथेत्याह राजा चाभाष्य मन्त्रिणः}
{अयोध्यां प्रेषयामास धर्मात्मा कृतशासनान्}
{यथावृत्तं समाख्यातुमानेतुं च नृपं तथा} %1-67-27


॥इत्यार्षे श्रीमद्रामायणे वाल्मीकीये आदिकाव्ये बालकाण्डे धनुर्भङ्गः नाम सप्तषष्ठितमः सर्गः ॥१-६७॥
