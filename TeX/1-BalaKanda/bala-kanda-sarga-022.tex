\sect{द्वाविंशः सर्गः — विद्याप्रदानम्}

\twolineshloka
{तथा वसिष्ठे ब्रुवति राजा दशरथः स्वयम्}
{प्रहृष्टवदनो राममाजुहाव सलक्ष्मणम्} %1-22-1

\twolineshloka
{कृतस्वस्त्ययनं मात्रा पित्रा दशरथेन च}
{पुरोधसा वसिष्ठेन मङ्गलैरभिमन्त्रितम्} %1-22-2

\twolineshloka
{स पुत्रं मूर्ध्न्युपाघ्राय राजा दशरथस्तदा}
{ददौ कुशिकपुत्राय सुप्रीतेनान्तरात्मना} %1-22-3

\twolineshloka
{ततो वायुः सुखस्पर्शो नीरजस्को ववौ तदा}
{विश्वामित्रगतं रामं दृष्ट्वा राजीवलोचनम्} %1-22-4

\twolineshloka
{पुष्पवृष्टिर्महत्यासीद् देवदुन्दुभिनिस्वनैः}
{शङ्खदुन्दुभिनिर्घोषः प्रयाते तु महात्मनि} %1-22-5

\twolineshloka
{विश्वामित्रो ययावग्रे ततो रामो महायशाः}
{काकपक्षधरो धन्वी तं च सौमित्रिरन्वगात्} %1-22-6

\twolineshloka
{कलापिनौ धनुष्पाणी शोभयानौ दिशो दश}
{विश्वामित्रं महात्मानं त्रिशीर्षाविव पन्नगौ} %1-22-7

\twolineshloka
{अनुजग्मतुरक्षुद्रौ पितामहमिवाश्विनौ}
{अनुयातौ श्रिया दीप्तौ शोभयेतावनिन्दितौ} %1-22-8

\twolineshloka
{तदा कुशिकपुत्रं तु धनुष्पाणी स्वलङ्कृतौ}
{बद्धगोधाङ्गुलित्राणौ खड्गवन्तौ महाद्युती} %1-22-9

\twolineshloka
{कुमारौ चारुवपुषौ भ्रातरौ रामलक्ष्मणौ}
{अनुयातौ श्रिया दीप्तौ शोभयेतावनिन्दितौ} %1-22-10

\twolineshloka
{स्थाणुं देवमिवाचिन्त्यं कुमाराविव पावकी}
{अध्यर्धयोजनं गत्वा सरय्वा दक्षिणे तटे} %1-22-11

\twolineshloka
{रामेति मधुरां वाणीं विश्वामित्रोऽभ्यभाषत}
{गृहाण वत्स सलिलं मा भूत्कालस्य पर्ययः} %1-22-12

\twolineshloka
{मन्त्रग्रामं गृहाण त्वं बलामतिबलां तथा}
{न श्रमो न ज्वरो वा ते न रूपस्य विपर्ययः} %1-22-13

\twolineshloka
{न च सुप्तं प्रमत्तं वा धर्षयिष्यन्ति नैर्ऋताः}
{न बाह्वोः सदृशो वीर्ये पृथिव्यामस्ति कश्चन} %1-22-14

\twolineshloka
{त्रिषु लोकेषु वा राम न भवेत्सदृशस्तव}
{बलामतिबलां चैव पठतस्तात राघव} %1-22-15

\twolineshloka
{न सौभाग्ये न दाक्षिण्ये न ज्ञाने बुद्धिनिश्चये}
{नोत्तरे प्रतिवक्तव्ये समो लोके तवानघ} %1-22-16

\twolineshloka
{एतद्विद्याद्वये लब्धे न भवेत् सदृशस्तव}
{बला चातिबला चैव सर्वज्ञानस्य मातरौ} %1-22-17

\twolineshloka
{क्षुत्पिपासे न ते राम भविष्येते नरोत्तम}
{बलामतिबलां चैव पठतस्तात राघव} %1-22-18

\threelineshloka
{गृहाण सर्वलोकस्य गुप्तये रघुनन्दन}
{विद्याद्वयमधीयाने यशश्चाथ भवेद् भुवि}
{पितामहसुते ह्येते विद्ये तेजःसमन्विते} %1-22-19

\twolineshloka
{प्रदातुं तव काकुत्स्थ सदृशस्त्वं हि पार्थिव}
{कामं बहुगुणाः सर्वे त्वय्येते नात्र संशयः} %1-22-20

\twolineshloka
{तपसा सम्भृते चैते बहुरूपे भविष्यतः}
{ततो रामो जलं स्पृष्ट्वा प्रहृष्टवदनः शुचिः} %1-22-21

\twolineshloka
{प्रतिजग्राह ते विद्ये महर्षेर्भावितात्मनः}
{विद्यासमुदितो रामः शुशुभे भूरिविक्रमः} %1-22-22

\threelineshloka
{सहस्ररश्मिर्भगवान् शरदीव दिवाकरः}
{गुरुकार्याणि सर्वाणि नियुज्य कुशिकात्मजे}
{ऊषुस्तां रजनीं तत्र सरय्वां सुसुखं त्रयः} %1-22-23

\twolineshloka
{दशरथनृपसूनुसत्तमाभ्यां तृणशयनेऽनुचिते तदोषिताभ्याम्}
{कुशिकसुतवचोऽनुलालिताभ्याम् सुखमिव सा विबभौ विभावरी च} %1-22-24


॥इत्यार्षे श्रीमद्रामायणे वाल्मीकीये आदिकाव्ये बालकाण्डे विद्याप्रदानम् नाम द्वाविंशः सर्गः ॥१-२२॥
