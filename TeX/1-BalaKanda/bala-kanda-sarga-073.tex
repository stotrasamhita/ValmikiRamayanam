\sect{त्रिसप्ततितमः सर्गः — दशरथपुत्रोद्वाहः}

\twolineshloka
{यस्मिंस्तु दिवसे राजा चक्रे गोदानमुत्तमम्}
{तस्मिंस्तु दिवसे वीरो युधाजित् समुपेयिवान्} %1-73-1

\twolineshloka
{पुत्रः केकयराजस्य साक्षाद्भरतमातुलः}
{दृष्ट्वा पृष्ट्वा च कुशलं राजानमिदमब्रवीत्} %1-73-2

\twolineshloka
{केकयाधिपती राजा स्नेहात् कुशलमब्रवीत्}
{येषां कुशलकामोऽसि तेषां सम्प्रत्यनामयम्} %1-73-3

\twolineshloka
{स्वस्रीयं मम राजेन्द्र द्रष्टुकामो महीपतिः}
{तदर्थमुपयातोऽहमयोध्यां रघुनन्दन} %1-73-4

\twolineshloka
{श्रुत्वा त्वहमयोध्यायां विवाहार्थं तवात्मजान्}
{मिथिलामुपयातांस्तु त्वया सह महीपते} %1-73-5

\twolineshloka
{त्वरयाभ्युपयातोऽहं द्रष्टुकामः स्वसुः सुतम्}
{अथ राजा दशरथः प्रियातिथिमुपस्थितम्} %1-73-6

\twolineshloka
{दृष्ट्वा परमसत्कारैः पूजनार्हमपूजयत्}
{ततस्तामुषितो रात्रिं सह पुत्रैर्महात्मभिः} %1-73-7

\twolineshloka
{प्रभाते पुनरुत्थाय कृत्वा कर्माणि तत्त्ववित्}
{ऋषींस्तदा पुरस्कृत्य यज्ञवाटमुपागमत्} %1-73-8

\twolineshloka
{युक्ते मुहूर्ते विजये सर्वाभरणभूषितैः}
{भ्रातृभिः सहितो रामः कृतकौतुकमंगलः} %1-73-9

\twolineshloka
{वसिष्ठं पुरतः कृत्वा महर्षीनपरानपि}
{वसिष्ठो भगवानेत्य वैदेहमिदमब्रवीत्} %1-73-10

\twolineshloka
{राजा दशरथो राजन् कृतकौतुकमंगलैः}
{पुत्रैर्नरवरश्रेष्ठो दातारमभिकाङ्क्षते} %1-73-11

\twolineshloka
{दातृप्रतिग्रहीतृभ्यां सर्वार्थाः सम्भवन्ति हि}
{स्वधर्मं प्रतिपद्यस्व कृत्वा वैवाह्यमुत्तमम्} %1-73-12

\twolineshloka
{इत्युक्तः परमोदारो वसिष्ठेन महात्मना}
{प्रत्युवाच महातेजा वाक्यं परमधर्मवित्} %1-73-13

\twolineshloka
{कः स्थितः प्रतिहारो मे कस्याज्ञां सम्प्रतीक्षते}
{स्वगृहे को विचारोऽस्ति यथा राज्यमिदं तव} %1-73-14

\twolineshloka
{कृतकौतुकसर्वस्वा वेदिमूलमुपागताः}
{मम कन्या मुनिश्रेष्ठ दीप्ता बह्नेरिवार्चिषः} %1-73-15

\twolineshloka
{सद्योऽहं त्वत्प्रतीक्षोऽस्मि वेद्यामस्यां प्रतिष्ठितः}
{अविघ्नं क्रियतां सर्वं किमर्थं हि विलम्ब्यते} %1-73-16

\twolineshloka
{तद् वाक्यं जनकेनोक्तं श्रुत्वा दशरथस्तदा}
{प्रवेशयामास सुतान् सर्वानृषिगणानपि} %1-73-17

\twolineshloka
{ततो राजा विदेहानां वसिष्ठमिदमब्रवीत्}
{कारयस्व ऋषे सर्वामृषिभिः सह धार्मिक} %1-73-18

\twolineshloka
{रामस्य लोकरामस्य क्रियां वैवाहिकीं प्रभो}
{तथेत्युक्त्वा तु जनकं वसिष्ठो भगवानृषिः} %1-73-19

\twolineshloka
{विश्वामित्रं पुरस्कृत्य शतानन्दं च धार्मिकम्}
{प्रपामध्ये तु विधिवद् वेदिं कृत्वा महातपाः} %1-73-20

\twolineshloka
{अलंचकार तां वेदिं गन्धपुष्पैः समन्ततः}
{सुवर्णपालिकाभिश्च चित्रकुम्भैश्च साङ्कुरैः} %1-73-21

\twolineshloka
{अङ्कुराढ्यैः शरावैश्च धूपपात्रैः सधूपकैः}
{शङ्खपात्रैः स्रुवैः स्रग्भिः पात्रैरर्घ्यादिपूजितैः} %1-73-22

\twolineshloka
{लाजपूर्णैश्च पात्रीभिरक्षतैरपि संस्कृतैः}
{दर्भैः समैः समास्तीर्य विधिवन्मन्त्रपूर्वकम्} %1-73-23

\twolineshloka
{अग्निमाधाय तं वेद्यां विधिमन्त्रपुरस्कृतम्}
{जुहावाग्नौ महातेजा वसिष्ठो मुनिपुंगवः} %1-73-24

\twolineshloka
{ततः सीतां समानीय सर्वाभरणभूषिताम्}
{समक्षमग्नेः संस्थाप्य राघवाभिमुखे तदा} %1-73-25

\twolineshloka
{अब्रवीज्जनको राजा कौसल्यानन्दवर्धनम्}
{इयं सीता मम सुता सहधर्मचरी तव} %1-73-26

\twolineshloka
{प्रतीच्छ चैनां भद्रं ते पाणिं गृह्णीष्व पाणिना}
{पतिव्रता महाभागा छायेवानुगता सदा} %1-73-27

\twolineshloka
{इत्युक्त्वा प्राक्षिपद् राजा मन्त्रपूतं जलं तदा}
{साधुसाध्विति देवानामृषीणां वदतां तदा} %1-73-28

\twolineshloka
{देवदुन्दुभिनिर्घोषः पुष्पवर्षो महानभूत्}
{एवं दत्त्वा सुतां सीतां मन्त्रोदकपुरस्कृताम्} %1-73-29

\twolineshloka
{अब्रवीज्जनको राजा हर्षेणाभिपरिप्लुतः}
{लक्ष्मणागच्छ भद्रं ते ऊर्मिलामुद्यतां मया} %1-73-30

\twolineshloka
{प्रतीच्छ पाणिं गृह्णीष्व मा भूत् कालस्य पर्ययः}
{तमेवमुक्त्वा जनको भरतं चाभ्यभाषत} %1-73-31

\twolineshloka
{गृहाण पाणिं माण्डव्याः पाणिना रघुनन्दन}
{शत्रुघ्नं चापि धर्मात्मा अब्रवीन्मिथिलेश्वरः} %1-73-32

\twolineshloka
{श्रुतकीर्तेर्महाबाहो पाणिं गृह्णीष्व पाणिना}
{सर्वे भवन्तः सौम्याश्च सर्वे सुचरितव्रताः} %1-73-33

\twolineshloka
{पत्नीभिः सन्तु काकुत्स्था मा भूत् कालस्य पर्ययः}
{जनकस्य वचः श्रुत्वा पाणीन् पाणिभिरस्पृशन्} %1-73-34

\twolineshloka
{चत्वारस्ते चतसॄणां वसिष्ठस्य मते स्थिताः}
{अग्निं प्रदक्षिणं कृत्वा वेदिं राजानमेव च} %1-73-35

\twolineshloka
{ऋषींश्चापि महात्मानः सहभार्या रघूद्वहाः}
{यथोक्तेन ततश्चक्रुर्विवाहं विधिपूर्वकम्} %1-73-36

\twolineshloka
{पुष्पवृष्टिर्महत्यासीदन्तरिक्षात् सुभास्वरा}
{दिव्यदुन्दुभिनिर्घोषैर्गीतवादित्रनिःस्वनैः} %1-73-37

\twolineshloka
{ननृतुश्चाप्सरःसङ्घा गन्धर्वाश्च जगुः कलम्}
{विवाहे रघुमुख्यानां तदद्भुतमदृश्यत} %1-73-38

\twolineshloka
{ईदृशे वर्तमाने तु तूर्योद्घुष्टनिनादिते}
{त्रिरग्निं ते परिक्रम्य ऊहुर्भार्या महौजसः} %1-73-39

\twolineshloka
{अथोपकार्यं जग्मुस्ते सभार्या रघुनन्दनाः}
{राजाप्यनुययौ पश्यन् सर्षिसङ्घः सबान्धवः} %1-73-40


॥इत्यार्षे श्रीमद्रामायणे वाल्मीकीये आदिकाव्ये बालकाण्डे दशरथपुत्रोद्वाहः नाम त्रिसप्ततितमः सर्गः ॥१-७३॥
