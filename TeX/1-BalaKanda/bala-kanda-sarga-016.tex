\sect{षोडशः सर्गः — पायसोत्पत्तिः}

\twolineshloka
{ततो नारायणो विष्णुर्नियुक्तः सुरसत्तमैः}
{जानन्नपि सुरानेवं श्लक्ष्णं वचनमब्रवीत्} %1-16-1

\twolineshloka
{उपायः को वधे तस्य राक्षसाधिपतेः सुराः}
{यमहं तं समास्थाय निहन्यामृषिकण्टकम्} %1-16-2

\twolineshloka
{एवमुक्ताः सुराः सर्वे प्रत्यूचुर्विष्णुमव्ययम्}
{मानुषं रूपमास्थाय रावणं जहि संयुगे} %1-16-3

\twolineshloka
{स हि तेपे तपस्तीव्रं दीर्घकालमरिन्दमः}
{येन तुष्टोऽभवद् ब्रह्मा लोककृल्लोकपूर्वजः} %1-16-4

\twolineshloka
{संतुष्टः प्रददौ तस्मै राक्षसाय वरं प्रभुः}
{नानाविधेभ्यो भूतेभ्यो भयं नान्यत्र मानुषात्} %1-16-5

\twolineshloka
{अवज्ञाताः पुरा तेन वरदाने हि मानवाः}
{एवं पितामहात् तस्मात् वरदानेन गर्वितः} %1-16-6

\twolineshloka
{उत्सादयति लोकांस्त्रीन् स्त्रियश्चाप्युपकर्षति}
{तस्मात् तस्य वधो दृष्टो मानुषेभ्यः परंतप} %1-16-7

\twolineshloka
{इत्येतद् वचनं श्रुत्वा सुराणां विष्णुरात्मवान्}
{पितरं रोचयामास तदा दशरथं नृपम्} %1-16-8

\twolineshloka
{स चाप्यपुत्रो नृपतिस्तस्मिन् काले महाद्युतिः}
{अयजत् पुत्रियामिष्टिं पुत्रेप्सुररिसूदनः} %1-16-9

\twolineshloka
{स कृत्वा निश्चयं विष्णुरामन्त्र्य च पितामहम्}
{अन्तर्धानं गतो देवैः पूज्यमानो महर्षिभिः} %1-16-10

\twolineshloka
{ततो वै यजमानस्य पावकादतुलप्रभम्}
{प्रादुर्भूतं महद् भूतं महावीर्यं महाबलम्} %1-16-11

\twolineshloka
{कृष्णं रक्ताम्बरधरं रक्तास्यं दुन्दुभिस्वनम्}
{स्निग्धहर्यक्षतनुजश्मश्रुप्रवरमूर्धजम्} %1-16-12

\twolineshloka
{शुभलक्षणसम्पन्नं दिव्याभरणभूषितम्}
{शैलशृङ्गसमुत्सेधं दृप्तशार्दूलविक्रमम्} %1-16-13

\twolineshloka
{दिवाकरसमाकारं दीप्तानलशिखोपमम्}
{तप्तजाम्बूनदमयीं राजतान्तपरिच्छदाम्} %1-16-14

\twolineshloka
{दिव्यपायससम्पूर्णां पात्रीं पत्नीमिव प्रियाम्}
{प्रगृह्य विपुलां दोर्भ्यां स्वयं मायामयीमिव} %1-16-15

\twolineshloka
{समवेक्ष्याब्रवीद् वाक्यमिदं दशरथं नृपम्}
{प्राजापत्यं नरं विद्धि मामिहाभ्यागतं नृप} %1-16-16

\twolineshloka
{ततः परं तदा राजा प्रत्युवाच कृताञ्जलिः}
{भगवन् स्वागतं तेऽस्तु किमहं करवाणि ते} %1-16-17

\twolineshloka
{अथो पुनरिदं वाक्यं प्राजापत्यो नरोऽब्रवीत्}
{राजन्नर्चयता देवानद्य प्राप्तमिदं त्वया} %1-16-18

\twolineshloka
{इदं तु नृपशार्दूल पायसं देवनिर्मितम्}
{प्रजाकरं गृहाण त्वं धन्यमारोग्यवर्धनम्} %1-16-19

\twolineshloka
{भार्याणामनुरूपाणामश्नीतेति प्रयच्छ वै}
{तासु त्वं लप्स्यसे पुत्रान् यदर्थं यजसे नृप} %1-16-20

\twolineshloka
{तथेति नृपतिः प्रीतः शिरसा प्रतिगृह्य ताम्}
{पात्रीं देवान्नसम्पूर्णां देवदत्तां हिरण्मयीम्} %1-16-21

\twolineshloka
{अभिवाद्य च तद्भूतमद्भुतं प्रियदर्शनम्}
{मुदा परमया युक्तश्चकाराभिप्रदक्षिणम्} %1-16-22

\twolineshloka
{ततो दशरथः प्राप्य पायसं देवनिर्मितम्}
{बभूव परमप्रीतः प्राप्य वित्तमिवाधनः} %1-16-23

\twolineshloka
{ततस्तदद्भुतप्रख्यं भूतं परमभास्वरम्}
{संवर्तयित्वा तत् कर्म तत्रैवान्तरधीयत} %1-16-24

\twolineshloka
{हर्षरश्मिभिरुद्द्योतं तस्यान्तःपुरमाबभौ}
{शारदस्याभिरामस्य चन्द्रस्येव नभोंऽशुभिः} %1-16-25

\twolineshloka
{सोऽन्तःपुरं प्रविश्यैव कौसल्यामिदमब्रवीत्}
{पायसं प्रतिगृह्णीष्व पुत्रीयं त्विदमात्मनः} %1-16-26

\twolineshloka
{कौसल्यायै नरपतिः पायसार्धं ददौ तदा}
{अर्धादर्धं ददौ चापि सुमित्रायै नराधिपः} %1-16-27

\twolineshloka
{कैकेय्यै चावशिष्टार्धं ददौ पुत्रार्थकारणात्}
{प्रददौ चावशिष्टार्धं पायसस्यामृतोपमम्} %1-16-28

\twolineshloka
{अनुचिन्त्य सुमित्रायै पुनरेव महामतिः}
{एवं तासां ददौ राजा भार्याणां पायसं पृथक्} %1-16-29

\twolineshloka
{ताश्चैवं पायसं प्राप्य नरेन्द्रस्योत्तमस्त्रियः}
{सम्मानं मेनिरे सर्वाः प्रहर्षोदितचेतसः} %1-16-30

\twolineshloka
{ततस्तु ताः प्राश्य तमुत्तमस्त्रियो महीपतेरुत्तमपायसं पृथक्}
{हुताशनादित्यसमानतेजसोऽचिरेण गर्भान् प्रतिपेदिरे तदा} %1-16-31

\twolineshloka
{ततस्तु राजा प्रतिवीक्ष्य ताः स्त्रियः प्ररूढगर्भाः प्रतिलब्धमानसः}
{बभूव हृष्टस्त्रिदिवे यथा हरिः सुरेन्द्रसिद्धर्षिगणाभिपूजितः} %1-16-32


॥इत्यार्षे श्रीमद्रामायणे वाल्मीकीये आदिकाव्ये बालकाण्डे पायसोत्पत्तिः नाम षोडशः सर्गः ॥१-१६॥
