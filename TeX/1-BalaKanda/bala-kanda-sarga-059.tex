\sect{एकोनषष्ठितमः सर्गः — वासिष्ठशापः}

\twolineshloka
{उक्तवाक्यं तु राजानं कृपया कुशिकात्मजः}
{अब्रवीन्मधुरं वाक्यं साक्षाच्चण्डालतां गतम्} %1-59-1

\twolineshloka
{इक्ष्वाको स्वागतं वत्स जानामि त्वां सुधार्मिकम्}
{शरणं ते प्रदास्यामि मा भैषीर्नृपपुङ्गव} %1-59-2

\twolineshloka
{अहमामन्त्रये सर्वान् महर्षीन् पुण्यकर्मणः}
{यज्ञसाह्यकरान् राजंस्ततो यक्ष्यसि निर्वृतः} %1-59-3

\twolineshloka
{गुरुशापकृतं रूपं यदिदं त्वयि वर्तते}
{अनेन सह रूपेण सशरीरो गमिष्यसि} %1-59-4

\twolineshloka
{हस्तप्राप्तमहं मन्ये स्वर्गं तव नराधिप}
{यस्त्वं कौशिकमागम्य शरण्यं शरणागतः} %1-59-5

\twolineshloka
{एवमुक्त्वा महातेजाः पुत्रान् परमधार्मिकान्}
{व्यादिदेश महाप्राज्ञान् यज्ञसम्भारकारणात्} %1-59-6

\twolineshloka
{सर्वान् शिष्यान् समाहूय वाक्यमेतदुवाच ह}
{सर्वानृषीन् सवासिष्ठानानयध्वं ममाज्ञया} %1-59-7

\twolineshloka
{सशिष्यान् सुहृदश्चैव सर्त्विजः सुबहुश्रुतान्}
{यदन्यो वचनं ब्रूयान्मद्वाक्यबलचोदितः} %1-59-8

\twolineshloka
{तत् सर्वमखिलेनोक्तं ममाख्येयमनादृतम्}
{तस्य तद् वचनं श्रुत्वा दिशो जग्मुस्तदाज्ञया} %1-59-9

\twolineshloka
{आजग्मुरथ देशेभ्यः सर्वेभ्यो ब्रह्मवादिनः}
{ते च शिष्याः समागम्य मुनिं ज्वलिततेजसम्} %1-59-10

\twolineshloka
{ऊचुश्च वचनं सर्वं सर्वेषां ब्रह्मवादिनाम्}
{श्रुत्वा ते वचनं सर्वे समायान्ति द्विजातयः} %1-59-11

\twolineshloka
{सर्वदेशेषु चागच्छन् वर्जयित्वा महोदयम्}
{वासिष्ठं यच्छतं सर्वं क्रोधपर्याकुलाक्षरम्} %1-59-12

\twolineshloka
{यथाह वचनं सर्वं शृणु त्वं मुनिपुङ्गव}
{क्षत्रियो याजको यस्य चण्डालस्य विशेषतः} %1-59-13

\twolineshloka
{कथं सदसि भोक्तारो हविस्तस्य सुरर्षयः}
{ब्राह्मणा वा महात्मानो भुक्त्वा चाण्डालभोजनम्} %1-59-14

\twolineshloka
{कथं स्वर्गं गमिष्यन्ति विश्वामित्रेण पालिताः}
{एतद् वचननैष्ठुर्यमूचुः संरक्तलोचनाः} %1-59-15

\twolineshloka
{वासिष्ठा मुनिशार्दूल सर्वे सहमहोदयाः}
{तेषां तद् वचनं श्रुत्वा सर्वेषां मुनिपुङ्गवः} %1-59-16

\twolineshloka
{क्रोधसंरक्तनयनः सरोषमिदमब्रवीत्}
{यद् दूषयन्त्यदुष्टं मां तप उग्रं समास्थितम्} %1-59-17

\twolineshloka
{भस्मीभूता दुरात्मानो भविष्यन्ति न संशयः}
{अद्य ते कालपाशेन नीता वैवस्वतक्षयम्} %1-59-18

\twolineshloka
{सप्तजातिशतान्येव मृतपाः सम्भवन्तु ते}
{श्वमांसनियताहारा मुष्टिका नाम निर्घृणाः} %1-59-19

\twolineshloka
{विकृताश्च विरूपाश्च लोकाननुचरन्त्विमान्}
{महोदयश्च दुर्बुद्धिर्मामदूष्यं ह्यदूषयत्} %1-59-20

\twolineshloka
{दूषितः सर्वलोकेषु निषादत्वं गमिष्यति}
{प्राणातिपातनिरतो निरनुक्रोशतां गतः} %1-59-21

\threelineshloka
{दीर्घकालं मम क्रोधाद् दुर्गतिं वर्तयिष्यति}
{एतावदुक्त्वा वचनं विश्वामित्रो महातपाः}
{विरराम महातेजा ऋषिमध्ये महामुनिः} %1-59-22


॥इत्यार्षे श्रीमद्रामायणे वाल्मीकीये आदिकाव्ये बालकाण्डे वासिष्ठशापः नाम एकोनषष्ठितमः सर्गः ॥१-५९॥
