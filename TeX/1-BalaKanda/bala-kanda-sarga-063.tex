\sect{त्रिषष्ठितमः सर्गः — मेनकानिर्वासः}

\twolineshloka
{पूर्णे वर्षसहस्रे तु व्रतस्नातं महामुनिम्}
{अभ्यगच्छन् सुराः सर्वे तपः फलचिकीर्षवः} %1-63-1

\twolineshloka
{अब्रवीत् सुमहातेजा ब्रह्मा सुरुचिरं वचः}
{ऋषिस्त्वमसि भद्रं ते स्वार्जितैः कर्मभिः शुभैः} %1-63-2

\twolineshloka
{तमेवमुक्त्वा देवेशस्त्रिदिवं पुनरभ्यगात्}
{विश्वामित्रो महातेजा भूयस्तेपे महत् तपः} %1-63-3

\twolineshloka
{ततः कालेन महता मेनका परमाप्सराः}
{पुष्करेषु नरश्रेष्ठ स्नातुं समुपचक्रमे} %1-63-4

\twolineshloka
{तां ददर्श महातेजा मेनकां कुशिकात्मजः}
{रूपेणाप्रतिमां तत्र विद्युतं जलदे यथा} %1-63-5

\twolineshloka
{कन्दर्पदर्पवशगो मुनिस्तामिदमब्रवीत्}
{अप्सरः स्वागतं तेऽस्तु वस चेह ममाश्रमे} %1-63-6

\twolineshloka
{अनुगृह्णीष्व भद्रं ते मदनेन विमोहितम्}
{इत्युक्ता सा वरारोहा तत्र वासमथाकरोत्} %1-63-7

\twolineshloka
{तपसो हि महाविघ्नो विश्वामित्रमुपागमत्}
{तस्यां वसन्त्यां वर्षाणि पञ्च पञ्च च राघव} %1-63-8

\twolineshloka
{विश्वामित्राश्रमे सौम्ये सुखेन व्यतिचक्रमुः}
{अथ काले गते तस्मिन् विश्वामित्रो महामुनिः} %1-63-9

\twolineshloka
{सव्रीड इव संवृत्तश्चिन्ताशोकपरायणः}
{बुद्धिर्मुनेः समुत्पन्ना सामर्षा रघुनन्दन} %1-63-10

\twolineshloka
{सर्वं सुराणां कर्मैतत् तपोऽपहरणं महत्}
{अहोरात्रापदेशेन गताः संवत्सरा दश} %1-63-11

\twolineshloka
{काममोहाभिभूतस्य विघ्नोऽयं प्रत्युपस्थितः}
{स निःश्वसन् मुनिवरः पश्चात्तापेन दुःखितः} %1-63-12

\twolineshloka
{भीतामप्सरसं दृष्ट्वा वेपन्तीं प्राञ्जलिं स्थिताम्}
{मेनकां मधुरैर्वाक्यैर्विसृज्य कुशिकात्मजः} %1-63-13

\twolineshloka
{उत्तरं पर्वतं राम विश्वामित्रो जगाम ह}
{स कृत्वा नैष्ठिकीं बुद्धिं जेतुकामो महायशाः} %1-63-14

\twolineshloka
{कौशिकीतीरमासाद्य तपस्तेपे दुरासदम्}
{तस्य वर्षसहस्राणि घोरं तप उपासतः} %1-63-15

\twolineshloka
{उत्तरे पर्वते राम देवतानामभूद् भयम्}
{आमन्त्रयन् समागम्य सर्वे सर्षिगणाः सुराः} %1-63-16

\twolineshloka
{महर्षिशब्दं लभतां साध्वयं कुशिकात्मजः}
{देवतानां वचः श्रुत्वा सर्वलोकपितामहः} %1-63-17

\twolineshloka
{अब्रवीन्मधुरं वाक्यं विश्वामित्रं तपोधनम्}
{महर्षे स्वागतं वत्स तपसोग्रेण तोषितः} %1-63-18

\twolineshloka
{महत्त्वमृषिमुख्यत्वं ददामि तव कौशिक}
{ब्रह्मणस्तु वचः श्रुत्वा विश्वामित्रस्तपोधनः} %1-63-19

\twolineshloka
{प्राञ्जलिः प्रणतो भूत्वा प्रत्युवाच पितामहम्}
{ब्रह्मर्षिशब्दमतुलं स्वार्जितैः कर्मभिः शुभैः} %1-63-20

\twolineshloka
{यदि मे भगवन्नाह ततोऽहं विजितेन्द्रियः}
{तमुवाच ततो ब्रह्मा न तावत् त्वं जितेन्द्रियः} %1-63-21

\twolineshloka
{यतस्व मुनिशार्दूल इत्युक्त्वा त्रिदिवं गतः}
{विप्रस्थितेषु देवेषु विश्वामित्रो महामुनिः} %1-63-22

\twolineshloka
{ऊर्ध्वबाहुर्निरालम्बो वायुभक्षस्तपश्चरन्}
{घर्मे पञ्चतपा भूत्वा वर्षास्वाकाशसंश्रयः} %1-63-23

\twolineshloka
{शिशिरे सलिलेशायी रात्र्यहानि तपोधनः}
{एवं वर्षसहस्रं हि तपो घोरमुपागमत्} %1-63-24

\twolineshloka
{तस्मिन् संतप्यमाने तु विश्वामित्रे महामुनौ}
{संतापः सुमहानासीत् सुराणां वासवस्य च} %1-63-25

\twolineshloka
{रम्भामप्सरसं शक्रः सर्वैः सह मरुद्गणैः}
{उवाचात्महितं वाक्यमहितं कौशिकस्य च} %1-63-26


॥इत्यार्षे श्रीमद्रामायणे वाल्मीकीये आदिकाव्ये बालकाण्डे मेनकानिर्वासः नाम त्रिषष्ठितमः सर्गः ॥१-६३॥
