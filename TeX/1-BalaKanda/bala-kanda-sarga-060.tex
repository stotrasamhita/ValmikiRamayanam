\sect{षष्ठितमः सर्गः — त्रिशंकुस्वर्गः}

\twolineshloka
{तपोबलहतान् ज्ञात्वा वासिष्ठान् समहोदयान्}
{ऋषिमध्ये महातेजा विश्वामित्रोऽभ्यभाषत} %1-60-1

\twolineshloka
{अयमिक्ष्वाकुदायादस्त्रिशङ्कुरिति विश्रुतः}
{धर्मिष्ठश्च वदान्यश्च मां चैव शरणं गतः} %1-60-2

\twolineshloka
{स्वेनानेन शरीरेण देवलोकजिगीषया}
{यथायं स्वशरीरेण देवलोकं गमिष्यति} %1-60-3

\twolineshloka
{तथा प्रवर्त्यतां यज्ञो भवद्भिश्च मया सह}
{विश्वामित्रवचः श्रुत्वा सर्व एव महर्षयः} %1-60-4

\twolineshloka
{ऊचुः समेताः सहसा धर्मज्ञा धर्मसंहितम्}
{अयं कुशिकदायादो मुनिः परमकोपनः} %1-60-5

\twolineshloka
{यदाह वचनं सम्यगेतत् कार्यं न संशयः}
{अग्निकल्पो हि भगवान् शापं दास्यति रोषतः} %1-60-6

\twolineshloka
{तस्मात् प्रवर्त्यतां यज्ञः सशरीरो यथा दिवि}
{गच्छेदिक्ष्वाकुदायादो विश्वामित्रस्य तेजसा} %1-60-7

\twolineshloka
{ततः प्रवर्त्यतां यज्ञः सर्वे समधितिष्ठत}
{एवमुक्त्वा महर्षयः संजह्रुस्ताः क्रियास्तदा} %1-60-8

\twolineshloka
{याजकश्च महातेजा विश्वामित्रोऽभवत् क्रतौ}
{ऋत्विजश्चानुपूर्व्येण मन्त्रवन्मन्त्रकोविदाः} %1-60-9

\twolineshloka
{चक्रुः सर्वाणि कर्माणि यथाकल्पं यथाविधि}
{ततः कालेन महता विश्वामित्रो महातपाः} %1-60-10

\twolineshloka
{चकारावाहनं तत्र भागार्थं सर्वदेवताः}
{नाभ्यागमंस्तदा तत्र भागार्थं सर्वदेवताः} %1-60-11

\twolineshloka
{ततः कोपसमाविष्टो विश्वामित्रो महामुनिः}
{स्रुवमुद्यम्य सक्रोधस्त्रिशङ्कुमिदमब्रवीत्} %1-60-12

\twolineshloka
{पश्य मे तपसो वीर्यं स्वार्जितस्य नरेश्वर}
{एष त्वां स्वशरीरेण नयामि स्वर्गमोजसा} %1-60-13

\twolineshloka
{दुष्प्रापं स्वशरीरेण स्वर्गं गच्छ नरेश्वर}
{स्वार्जितं किंचिदप्यस्ति मया हि तपसः फलम्} %1-60-14

\twolineshloka
{राजंस्त्वं तेजसा तस्य सशरीरो दिवं व्रज}
{उक्तवाक्ये मुनौ तस्मिन् सशरीरो नरेश्वरः} %1-60-15

\twolineshloka
{दिवं जगाम काकुत्स्थ मुनीनां पश्यतां तदा}
{स्वर्गलोकं गतं दृष्ट्वा त्रिशङ्कुं पाकशासनः} %1-60-16

\twolineshloka
{सह सर्वैः सुरगणैरिदं वचनमब्रवीत्}
{त्रिशङ्को गच्छ भूयस्त्वं नासि स्वर्गकृतालयः} %1-60-17

\twolineshloka
{गुरुशापहतो मूढ पत भूमिमवाक्शिराः}
{एवमुक्तो महेन्द्रेण त्रिशङ्कुरपतत् पुनः} %1-60-18

\twolineshloka
{विक्रोशमानस्त्राहीति विश्वामित्रं तपोधनम्}
{तच्छ्रुत्वा वचनं तस्य क्रोशमानस्य कौशिकः} %1-60-19

\twolineshloka
{रोषमाहारयत् तीव्रं तिष्ठ तिष्ठेति चाब्रवीत्}
{ऋषिमध्ये स तेजस्वी प्रजापतिरिवापरः} %1-60-20

\twolineshloka
{सृजन् दक्षिणमार्गस्थान् सप्तर्षीनपरान् पुनः}
{नक्षत्रवंशमपरमसृजत् क्रोधमूर्च्छितः} %1-60-21

\twolineshloka
{दक्षिणां दिशमास्थाय ऋषिमध्ये महायशाः}
{सृष्ट्वा नक्षत्रवंशं च क्रोधेन कलुषीकृतः} %1-60-22

\twolineshloka
{अन्यमिन्द्रं करिष्यामि लोको वा स्यादनिन्द्रकः}
{दैवतान्यपि स क्रोधात् स्रष्टुं समुपचक्रमे} %1-60-23

\twolineshloka
{ततः परमसम्भ्रान्ताः सर्षिसङ्घाः सुरासुराः}
{विश्वामित्रं महात्मानमूचुः सानुनयं वचः} %1-60-24

\twolineshloka
{अयं राजा महाभाग गुरुशापपरिक्षतः}
{सशरीरो दिवं यातुं नार्हत्येव तपोधन} %1-60-25

\twolineshloka
{तेषां तद् वचनं श्रुत्वा देवानां मुनिपुंगवः}
{अब्रवीत् सुमहद् वाक्यं कौशिकः सर्वदेवताः} %1-60-26

\twolineshloka
{सशरीरस्य भद्रं वस्त्रिशङ्कोरस्य भूपतेः}
{आरोहणं प्रतिज्ञातं नानृतं कर्तुमुत्सहे} %1-60-27

\twolineshloka
{स्वर्गोऽस्तु सशरीरस्य त्रिशङ्कोरस्य शाश्वतः}
{नक्षत्राणि च सर्वाणि मामकानि ध्रुवाण्यथ} %1-60-28

\twolineshloka
{यावल्लोका धरिष्यन्ति तिष्ठन्त्वेतानि सर्वशः}
{यत् कृतानि सुराः सर्वे तदनुज्ञातुमर्हथ} %1-60-29

\twolineshloka
{एवमुक्ताः सुराः सर्वे प्रत्यूचुर्मुनिपुंगवम्}
{एवं भवतु भद्रं ते तिष्ठन्त्वेतानि सर्वशः} %1-60-30

\twolineshloka
{गगने तान्यनेकानि वैश्वानरपथाद् बहिः}
{नक्षत्राणि मुनिश्रेष्ठ तेषु ज्योतिःषु जाज्वलन्} %1-60-31

\twolineshloka
{अवाक्शिरास्त्रिशङ्कुश्च तिष्ठत्वमरसंनिभः}
{अनुयास्यन्ति चैतानि ज्योतींषि नृपसत्तमम्} %1-60-32

\twolineshloka
{कृतार्थं कीर्तिमन्तं च स्वर्गलोकगतं यथा}
{विश्वामित्रस्तु धर्मात्मा सर्वदेवैरभिष्टुतः} %1-60-33

\threelineshloka
{ऋषिमध्ये महातेजा बाढमित्येव देवताः}
{ततो देवा महात्मानो ऋषयश्च तपोधनाः}
{जग्मुर्यथागतं सर्वे यज्ञस्यान्ते नरोत्तम} %1-60-34


॥इत्यार्षे श्रीमद्रामायणे वाल्मीकीये आदिकाव्ये बालकाण्डे त्रिशंकुस्वर्गः नाम षष्ठितमः सर्गः ॥१-६०॥
