\sect{एकपञ्चाशः सर्गः — विश्वामित्रवृत्तम्}

\twolineshloka
{तस्य तद् वचनं श्रुत्वा विश्वामित्रस्य धीमतः}
{हृष्टरोमा महातेजाः शतानन्दो महातपाः} %1-51-1

\twolineshloka
{गौतमस्य सुतो ज्येष्ठस्तपसा द्योतितप्रभः}
{रामसन्दर्शनादेव परं विस्मयमागतः} %1-51-2

\twolineshloka
{एतौ निषण्णौ सम्प्रेक्ष्य शतानन्दो नृपात्मजौ}
{सुखासीनौ मुनिश्रेष्ठं विश्वामित्रमथाब्रवीत्} %1-51-3

\twolineshloka
{अपि ते मुनिशार्दूल मम माता यशस्विनी}
{दर्शिता राजपुत्राय तपोदीर्घमुपागता} %1-51-4

\twolineshloka
{अपि रामे महातेजा मम माता यशस्विनी}
{वन्यैरुपाहरत् पूजां पूजार्हे सर्वदेहिनाम्} %1-51-5

\twolineshloka
{अपि रामाय कथितं यद् वृत्तं तत् पुरातनम्}
{मम मातुर्महातेजो देवेन दुरनुष्ठितम्} %1-51-6

\twolineshloka
{अपि कौशिक भद्रं ते गुरुणा मम सङ्गता}
{मम माता मुनिश्रेष्ठ रामसन्दर्शनादितः} %1-51-7

\twolineshloka
{अपि मे गुरुणा रामः पूजितः कुशिकात्मज}
{इहागतो महातेजाः पूजां प्राप्य महात्मनः} %1-51-8

\twolineshloka
{अपि शान्तेन मनसा गुरुर्मे कुशिकात्मज}
{इहागतेन रामेण पूजितेनाभिवादितः} %1-51-9

\twolineshloka
{तच्छ्रुत्वा वचनं तस्य विश्वामित्रो महामुनिः}
{प्रत्युवाच शतानन्दं वाक्यज्ञो वाक्यकोविदम्} %1-51-10

\twolineshloka
{नातिक्रान्तं मुनिश्रेष्ठ यत्कर्तव्यं कृतं मया}
{सङ्गता मुनिना पत्नी भार्गवेणेव रेणुका} %1-51-11

\twolineshloka
{तच्छ्रुत्वा वचनं तस्य विश्वामित्रस्य धीमतः}
{शतानन्दो महातेजा रामं वचनमब्रवीत्} %1-51-12

\twolineshloka
{स्वागतं ते नरश्रेष्ठ दिष्ट्या प्राप्तोऽसि राघव}
{विश्वामित्रं पुरस्कृत्य महर्षिमपराजितम्} %1-51-13

\twolineshloka
{अचिन्त्यकर्मा तपसा ब्रह्मर्षिरमितप्रभः}
{विश्वामित्रो महातेजा वेद्म्येनं परमां गतिम्} %1-51-14

\twolineshloka
{नास्ति धन्यतरो राम त्वत्तोऽन्यो भुवि कश्चन}
{गोप्ता कुशिकपुत्रस्ते येन तप्तं महत्तपः} %1-51-15

\twolineshloka
{श्रूयतां चाभिधास्यामि कौशिकस्य महात्मनः}
{यथाबलं यथातत्त्वं तन्मे निगदतः शृणु} %1-51-16

\twolineshloka
{राजाऽऽसीदेष धर्मात्मा दीर्घकालमरिन्दमः}
{धर्मज्ञः कृतविद्यश्च प्रजानां च हिते रतः} %1-51-17

\twolineshloka
{प्रजापतिसुतस्त्वासीत् कुशो नाम महीपतिः}
{कुशस्य पुत्रो बलवान् कुशनाभः सुधार्मिकः} %1-51-18

\twolineshloka
{कुशनाभसुतस्त्वासीद् गाधिरित्येव विश्रुतः}
{गाधेः पुत्रो महातेजा विश्वामित्रो महामुनिः} %1-51-19

\twolineshloka
{विश्वामित्रो महातेजाः पालयामास मेदिनीम्}
{बहुवर्षसहस्राणि राजा राज्यमकारयत्} %1-51-20

\twolineshloka
{कदाचित् तु महातेजा योजयित्वा वरूथिनीम्}
{अक्षौहिणीपरिवृतः परिचक्राम मेदिनीम्} %1-51-21

\twolineshloka
{नगराणि च राष्ट्राणि सरितश्च महागिरीन्}
{आश्रमान् क्रमशो राजा विचरन्नाजगाम ह} %1-51-22

\twolineshloka
{वसिष्ठस्याश्रमपदं नानापुष्पलताद्रुमम्}
{नानामृगगणाकीर्णं सिद्धचारणसेवितम्} %1-51-23

\twolineshloka
{देवदानवगन्धर्वैः किन्नरैरुपशोभितम्}
{प्रशान्तहरिणाकीर्णं द्विजसङ्घनिषेवितम्} %1-51-24

\twolineshloka
{ब्रह्मर्षिगणसङ्कीर्णं देवर्षिगणसेवितम्}
{तपश्चरणसंसिद्धैरग्निकल्पैर्महात्मभिः} %1-51-25

\twolineshloka
{सततं सङ्कुलं श्रीमद्ब्रह्मकल्पैर्महात्मभिः}
{अब्भक्षैर्वायुभक्षैश्च शीर्णपर्णाशनैस्तथा} %1-51-26

\twolineshloka
{फलमूलाशनैर्दान्तैर्जितदोषैर्जितेन्द्रियैः}
{ऋषिभिर्वालखिल्यैश्च जपहोमपरायणैः} %1-51-27

\threelineshloka
{अन्यैर्वैखानसैश्चैव समन्तादुपशोभितम्}
{वसिष्ठस्याश्रमपदं ब्रह्मलोकमिवापरम्}
{ददर्श जयतां श्रेष्ठो विश्वामित्रो महाबलः} %1-51-28


॥इत्यार्षे श्रीमद्रामायणे वाल्मीकीये आदिकाव्ये बालकाण्डे विश्वामित्रवृत्तम् नाम एकपञ्चाशः सर्गः ॥१-५१॥
