\sect{अष्टपञ्चाशः सर्गः — त्रिशङ्कुशापः}

\twolineshloka
{ततस्त्रिशङ्कोर्वचनं श्रुत्वा क्रोधसमन्वितम्}
{ऋषिपुत्रशतं राम राजानमिदमब्रवीत्} %1-58-1

\twolineshloka
{प्रत्याख्यातोऽसि दुर्मेधो गुरुणा सत्यवादिना}
{तं कथं समतिक्रम्य शाखान्तरमुपेयिवान्} %1-58-2

\twolineshloka
{इक्ष्वाकूणां हि सर्वेषां पुरोधाः परमा गतिः}
{न चातिक्रमितुं शक्यं वचनं सत्यवादिनः} %1-58-3

\twolineshloka
{अशक्यमिति सोवाच वसिष्ठो भगवानृषिः}
{तं वयं वै समाहर्तुं क्रतुं शक्ताः कथञ्चन} %1-58-4

\twolineshloka
{बालिशस्त्वं नरश्रेष्ठ गम्यतां स्वपुरं पुनः}
{याजने भगवान् शक्तस्त्रैलोक्यस्यापि पार्थिव} %1-58-5

\twolineshloka
{अवमानं कथं कर्तुं तस्य शक्ष्यामहे वयम्}
{तेषां तद् वचनं श्रुत्वा क्रोधपर्याकुलाक्षरम्} %1-58-6

\twolineshloka
{स राजा पुनरेवैतानिदं वचनमब्रवीत्}
{प्रत्याख्यातो भगवता गुरुपुत्रैस्तथैव हि} %1-58-7

\twolineshloka
{अन्यां गतिं गमिष्यामि स्वस्ति वोऽस्तु तपोधनाः}
{ऋषिपुत्रास्तु तच्छ्रुत्वा वाक्यं घोराभिसंहितम्} %1-58-8

\twolineshloka
{शेपुः परमसङ्क्रुद्धाश्चण्डालत्वं गमिष्यसि}
{इत्युक्त्वा ते महात्मानो विविशुः स्वं स्वमाश्रमम्} %1-58-9

\twolineshloka
{अथ रात्र्यां व्यतीतायां राजा चण्डालतां गतः}
{नीलवस्त्रधरो नीलः पुरुषो ध्वस्तमूर्धजः} %1-58-10

\twolineshloka
{चित्यमाल्याङ्गरागश्च आयसाभरणोऽभवत्}
{तं दृष्ट्वा मन्त्रिणः सर्वे त्यज्य चण्डालरूपिणम्} %1-58-11

\twolineshloka
{प्राद्रवन् सहिता राम पौरा येऽस्यानुगामिनः}
{एको हि राजा काकुत्स्थ जगाम परमात्मवान्} %1-58-12

\twolineshloka
{दह्यमानो दिवारात्रं विश्वामित्रं तपोधनम्}
{विश्वामित्रस्तु तं दृष्ट्वा राजानं विफलीकृतम्} %1-58-13

\twolineshloka
{चण्डालरूपिणं राम मुनिः कारुण्यमागतः}
{कारुण्यात् स महातेजा वाक्यं परमधार्मिकः} %1-58-14

\twolineshloka
{इदं जगाद भद्रं ते राजानं घोरदर्शनम्}
{किमागमनकार्यं ते राजपुत्र महाबल} %1-58-15

\twolineshloka
{अयोध्याधिपते वीर शापाच्चण्डालतां गतः}
{अथ तद्वाक्यमाकर्ण्य राजा चण्डालतां गतः} %1-58-16

\twolineshloka
{अब्रवीत् प्राञ्जलिर्वाक्यं वाक्यज्ञो वाक्यकोविदम्}
{प्रत्याख्यातोऽस्मि गुरुणा गुरुपुत्रैस्तथैव च} %1-58-17

\twolineshloka
{अनवाप्यैव तं कामं मया प्राप्तो विपर्ययः}
{सशरीरो दिवं यायामिति मे सौम्यदर्शन} %1-58-18

\twolineshloka
{मया चेष्टं क्रतुशतं तच्च नावाप्यते फलम्}
{अनृतं नोक्तपूर्वं मे न च वक्ष्ये कदाचन} %1-58-19

\twolineshloka
{कृच्छ्रेष्वपि गतः सौम्य क्षत्रधर्मेण ते शपे}
{यज्ञैर्बहुविधैरिष्टं प्रजा धर्मेण पालिताः} %1-58-20

\twolineshloka
{गुरवश्च महात्मानः शीलवृत्तेन तोषिताः}
{धर्मे प्रयतमानस्य यज्ञं चाहर्तुमिच्छतः} %1-58-21

\twolineshloka
{परितोषं न गच्छन्ति गुरवो मुनिपुङ्गव}
{दैवमेव परं मन्ये पौरुषं तु निरर्थकम्} %1-58-22

\threelineshloka
{दैवेनाक्रम्यते सर्वं दैवं हि परमा गतिः}
{तस्य मे परमार्तस्य प्रसादमभिकाङ्क्षतः}
{कर्तुमर्हसि भद्रं ते दैवोपहतकर्मणः} %1-58-23

\twolineshloka
{नान्यां गतिं गमिष्यामि नान्यच्छरणमस्ति मे}
{दैवं पुरुषकारेण निवर्तयितुमर्हसि} %1-58-24


॥इत्यार्षे श्रीमद्रामायणे वाल्मीकीये आदिकाव्ये बालकाण्डे त्रिशङ्कुशापः नाम अष्टपञ्चाशः सर्गः ॥१-५८॥
