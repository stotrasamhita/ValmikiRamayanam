\sect{सप्तविंशः सर्गः — अस्त्रग्रामप्रदानम्}

\twolineshloka
{अथ तां रजनीमुष्य विश्वामित्रो महायशाः}
{प्रहस्य राघवं वाक्यमुवाच मधुरस्वरम्} %1-27-1

\twolineshloka
{परितुष्टोऽस्मि भद्रं ते राजपुत्र महायशः}
{प्रीत्या परमया युक्तो ददाम्यस्त्राणि सर्वशः} %1-27-2

\twolineshloka
{देवासुरगणान् वापि सगन्धर्वोरगान् भुवि}
{यैरमित्रान् प्रसह्याजौ वशीकृत्य जयिष्यसि} %1-27-3

\twolineshloka
{तानि दिव्यानि भद्रं ते ददाम्यस्त्राणि सर्वशः}
{दण्डचक्रं महद् दिव्यं तव दास्यामि राघव} %1-27-4

\twolineshloka
{धर्मचक्रं ततो वीर कालचक्रं तथैव च}
{विष्णुचक्रं तथात्युग्रमैन्द्रं चक्रं तथैव च} %1-27-5

\twolineshloka
{वज्रमस्त्रं नरश्रेष्ठ शैवं शूलवरं तथा}
{अस्त्रं ब्रह्मशिरश्चैव ऐषीकमपि राघव} %1-27-6

\twolineshloka
{ददामि ते महाबाहो ब्राह्ममस्त्रमनुत्तमम्}
{गदे द्वे चैव काकुत्स्थ मोदकीशिखरी शुभे} %1-27-7

\twolineshloka
{प्रदीप्ते नरशार्दूल प्रयच्छामि नृपात्मज}
{धर्मपाशमहं राम कालपाशं तथैव च} %1-27-8

\twolineshloka
{वारुणं पाशमस्त्रं च ददाम्यहमनुत्तमम्}
{अशनी द्वे प्रयच्छामि शुष्कार्द्रे रघुनन्दन} %1-27-9

\twolineshloka
{ददामि चास्त्रं पैनाकमस्त्रं नारायणं तथा}
{आग्नेयमस्त्रं दयितं शिखरं नाम नामतः} %1-27-10

\twolineshloka
{वायव्यं प्रथमं नाम ददामि तव चानघ}
{अस्त्रं हयशिरो नाम क्रौञ्चमस्त्रं तथैव च} %1-27-11

\twolineshloka
{शक्तिद्वयं च काकुत्स्थ ददामि तव राघव}
{कङ्कालं मुसलं घोरं कापालमथ किङ्किणीम्} %1-27-12

\twolineshloka
{वधार्थं रक्षसां यानि ददाम्येतानि सर्वशः}
{वैद्याधरं महास्त्रं च नन्दनं नाम नामतः} %1-27-13

\twolineshloka
{असिरत्नं महाबाहो ददामि नृवरात्मज}
{गान्धर्वमस्त्रं दयितं मोहनं नाम नामतः} %1-27-14

\twolineshloka
{प्रस्वापनं प्रशमनं दद्मि सौम्यं च राघव}
{वर्षणं शोषणं चैव सन्तापनविलापने} %1-27-15

\twolineshloka
{मादनं चैव दुर्धर्षं कन्दर्पदयितं तथा}
{गान्धर्वमस्त्रं दयितं मानवं नाम नामतः} %1-27-16

\twolineshloka
{पैशाचमस्त्रं दयितं मोहनं नाम नामतः}
{प्रतीच्छ नरशार्दूल राजपुत्र महायशः} %1-27-17

\twolineshloka
{तामसं नरशार्दूल सौमनं च महाबलम्}
{संवर्तं चैव दुर्धर्षं मौसलं च नृपात्मज} %1-27-18

\twolineshloka
{सत्यमस्त्रं महाबाहो तथा मायामयं परम्}
{सौरं तेजःप्रभं नाम परतेजोऽपकर्षणम्} %1-27-19

\twolineshloka
{सोमास्त्रं शिशिरं नाम त्वाष्ट्रमस्त्रं सुदारुणम्}
{दारुणं च भगस्यापि शीतेषुमथ मानवम्} %1-27-20

\twolineshloka
{एतान् राम महाबाहो कामरूपान् महाबलान्}
{गृहाण परमोदारान् क्षिप्रमेव नृपात्मज} %1-27-21

\twolineshloka
{स्थितस्तु प्राङ्मुखो भूत्वा शुचिर्मुनिवरस्तदा}
{ददौ रामाय सुप्रीतो मन्त्रग्राममनुत्तमम्} %1-27-22

\twolineshloka
{सर्वसङ्ग्रहणं येषां दैवतैरपि दुर्लभम्}
{तान्यस्त्राणि तदा विप्रो राघवाय न्यवेदयत्} %1-27-23

\twolineshloka
{जपतस्तु मुनेस्तस्य विश्वामित्रस्य धीमतः}
{उपतस्थुर्महार्हाणि सर्वाण्यस्त्राणि राघवम्} %1-27-24

\twolineshloka
{ऊचुश्च मुदिता रामं सर्वे प्राञ्जलयस्तदा}
{इमे च परमोदार किङ्करास्तव राघव} %1-27-25

\twolineshloka
{यद्यदिच्छसि भद्रं ते तत्सर्वं करवाम वै}
{ततो रामः प्रसन्नात्मा तैरित्युक्तो महाबलैः} %1-27-26

\twolineshloka
{प्रतिगृह्य च काकुत्स्थः समालभ्य च पाणिना}
{मनसा मे भविष्यध्वमिति तान्यभ्यचोदयत्} %1-27-27

\twolineshloka
{ततः प्रीतमना रामो विश्वामित्रं महामुनिम्}
{अभिवाद्य महातेजा गमनायोपचक्रमे} %1-27-28


॥इत्यार्षे श्रीमद्रामायणे वाल्मीकीये आदिकाव्ये बालकाण्डे अस्त्रग्रामप्रदानम् नाम सप्तविंशः सर्गः ॥१-२७॥
