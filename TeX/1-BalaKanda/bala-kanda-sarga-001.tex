\sect{प्रथमः सर्गः — नारदवाक्यम्}

\twolineshloka
{तपःस्वाध्यायनिरतं तपस्वी वाग्विदां वरम्}
{नारदं परिपप्रच्छ वाल्मीकिर्मुनिपुङ्गवम्} %1-1-1

\twolineshloka
{को न्वस्मिन्साम्प्रतं लोके गुणवान्कश्च वीर्यवान्}
{धर्मज्ञश्च कृतज्ञश्च सत्यवाक्यो दृढव्रतः} %1-1-2

\twolineshloka
{चारित्रेण च को युक्तः सर्वभूतेषु को हितः}
{विद्वान्कः कः समर्थश्च कश्चैकप्रियदर्शनः} %1-1-3

\twolineshloka
{आत्मवान्को जितक्रोधो द्युतिमान्कोऽनसूयकः}
{कस्य बिभ्यति देवाश्च जातरोषस्य संयुगे} %1-1-4

\twolineshloka
{एतदिच्छाम्यहं श्रोतुं परं कौतूहलं हि मे}
{महर्षे त्वं समर्थोऽसि ज्ञातुमेवंविधं नरम्} %1-1-5

\twolineshloka
{श्रुत्वा चैतत्त्रिलोकज्ञो वाल्मीकेर्नारदो वचः}
{श्रूयतामिति चामन्त्र्य प्रहृष्टो वाक्यमब्रवीत्} %1-1-6

\twolineshloka
{बहवो दुर्लभाश्चैव ये त्वया कीर्तिता गुणाः}
{मुने वक्ष्याम्यहं बुद्ध्वा तैर्युक्तः श्रूयतां नरः} %1-1-7

\twolineshloka
{इक्ष्वाकुवंशप्रभवो रामो नाम जनैः श्रुतः}
{नियतात्मा महावीर्यो द्युतिमान्धृतिमान्वशी} %1-1-8

\twolineshloka
{बुद्धिमान्नीतिमान्वाग्मी श्रीमाञ्छत्रुनिबर्हणः}
{विपुलांसो महाबाहुः कम्बुग्रीवो महाहनुः} %1-1-9

\twolineshloka
{महोरस्को महेष्वासो गूढजत्रुररिन्दमः}
{आजानुबाहुः सुशिराः सुललाटः सुविक्रमः} %1-1-10

\twolineshloka
{समः समविभक्ताङ्गः स्निग्धवर्णः प्रतापवान्}
{पीनवक्षा विशालाक्षो लक्ष्मीवाञ्छुभलक्षणः} %1-1-11

\twolineshloka
{धर्मज्ञः सत्यसन्धश्च प्रजानां च हिते रतः}
{यशस्वी ज्ञानसम्पन्नः शुचिर्वश्यः समाधिमान्} %1-1-12

\twolineshloka
{प्रजापतिसमः श्रीमान् धाता रिपुनिषूदनः}
{रक्षिता जीवलोकस्य धर्मस्य परिरक्षिता} %1-1-13

\twolineshloka
{रक्षिता स्वस्य धर्मस्य स्वजनस्य च रक्षिता}
{वेदवेदाङ्गतत्त्वज्ञो धनुर्वेदे च निष्ठितः} %1-1-14

\twolineshloka
{सर्वशास्त्रार्थतत्त्वज्ञो स्मृतिमान् प्रतिभानवान्}
{सर्वलोकप्रियः साधुरदीनात्मा विचक्षणः} %1-1-15

\twolineshloka
{सर्वदाभिगतः सद्भिः समुद्र इव सिन्धुभिः}
{आर्यः सर्वसमश्चैव सदैव प्रियदर्शनः} %1-1-16

\twolineshloka
{स च सर्व गुणोपेतः कौसल्यानन्दवर्धनः}
{समुद्र इव गाम्भीर्ये धैर्येण हिमवानिव} %1-1-17

\twolineshloka
{विष्णुना सदृशो वीर्ये सोमवत्प्रियदर्शनः}
{कालाग्निसदृशः क्रोधे क्षमया पृथिवीसमः} %1-1-18

\twolineshloka
{धनदेन समस्त्यागे सत्ये धर्म इवापरः}
{तमेवंगुणसम्पन्नं रामं सत्यपराक्रमम्} %1-1-19

\twolineshloka
{ज्येष्ठं ज्येष्ठगुणैर्युक्तं प्रियं दशरथस्सुतम्}
{प्रकृतीनां हितैर्युक्तं प्रकृतिप्रियकाम्यया} %1-1-20

\twolineshloka
{यौवराज्येन संयोक्तुम् ऐच्छत्प्रीत्या महीपतिः}
{तस्याभिषेकसम्भारान् दृष्ट्वा भार्याथ कैकयी} %1-1-21

\twolineshloka
{पूर्वं दत्तवरा देवी वरमेनमयाचत}
{विवासनञ्च रामस्य भरतस्याभिषेचनम्} %1-1-22

\twolineshloka
{स सत्यवचनाद्राजा धर्मपाशेन संयतः}
{विवासयामास सुतं रामं दशरथः प्रियम्} %1-1-23

\twolineshloka
{स जगाम वनं वीरः प्रतिज्ञामनुपालयन्}
{पितुर्वचननिर्देशात् कैकेय्याः प्रियकारणात्} %1-1-24

\twolineshloka
{तं व्रजन्तं प्रियो भ्राता लक्ष्मणोऽनुजगाम ह}
{स्नेहाद् विनयसम्पन्नः सुमित्रानन्दवर्धनः} %1-1-25

\twolineshloka
{भ्रातरं दयितो भ्रातुः सौभ्रात्रमनुदर्शयन्}
{रामस्य दयिता भार्या नित्यं प्राणसमा हिता} %1-1-26

\twolineshloka
{जनकस्य कुले जाता देवमायेव निर्मिता}
{सर्वलक्षणसम्पन्ना नारीणामुत्तमा वधूः} %1-1-27

\twolineshloka
{सीताप्यनुगता रामं शशिनं रोहिणी यथा}
{पौरैरनुगतो दूरं पित्रा दशरथेन च} %1-1-28

\twolineshloka
{शृङ्गवेरपुरे सूतं गङ्गाकूले व्यसर्जयत्}
{गुहमासाद्य धर्मात्मा निषादाधिपतिं प्रियम्} %1-1-29

\twolineshloka
{गुहेन सहितो रामो लक्ष्मणेन च सीतया}
{ते वनेन वनङ्गत्वा नदीस्तीर्त्वा बहूदकाः} %1-1-30

\twolineshloka
{चित्रकूटमनुप्राप्य भरद्वाजस्य शासनात्}
{रम्यमावसथं कृत्वा रममाणा वने त्रयः} %1-1-31

\twolineshloka
{देवगन्धर्वसंकाशाः तत्र ते न्यवसन् सुखम्}
{चित्रकूटङ्गते रामे पुत्रशोकातुरस्तथा} %1-1-32

\twolineshloka
{राजा दशरथस्स्वर्गं जगाम विलपन् सुतम्}
{गते तु तस्मिन् भरतो वसिष्ठप्रमुखैर्द्विजैः} %1-1-33

\twolineshloka
{नियुज्यमानो राज्याय नैच्छत् राज्यं महाबलः}
{स जगाम वनं वीरो रामपादप्रसादकः} %1-1-34

\twolineshloka
{गत्वा तु स महात्मानं रामं सत्यपराक्रमम्}
{अयाचद्भ्रातरं रामम् आर्यभावपुरस्कृतः} %1-1-35

\twolineshloka
{त्वमेव राजा धर्मज्ञ इति रामं वचोऽब्रवीत्}
{रामोऽपि परमोदारः सुमुखस्सुमहायशाः} %1-1-36

\twolineshloka
{न चैच्छत् पितुरादेशात् राज्यं रामो महाबलः}
{पादुके चास्य राज्याय न्यासं दत्त्वा पुनः पुनः} %1-1-37

\twolineshloka
{निवर्तयामास ततो भरतं भरताग्रजः}
{स काममनवाप्यैव रामपादावुपस्पृशन्} %1-1-38

\twolineshloka
{नन्दिग्रामेऽकरोद् राज्यं रामागमनकाङ्क्षया}
{गते तु भरते श्रीमान् सत्यसन्धो जितेन्द्रियः} %1-1-39

\twolineshloka
{रामस्तु पुनरालक्ष्य नागरस्य जनस्य च}
{तत्रागमनमेकाग्रो दण्डकान् प्रविवेश ह} %1-1-40

\twolineshloka
{प्रविश्य तु महारण्यं रामो राजीवलोचनः}
{विराधं राक्षसं हत्वा शरभङ्गं ददर्श ह} %1-1-41

\twolineshloka
{सुतीक्ष्णं चाप्यगस्त्यं च अगस्त्यभ्रातरं तथा}
{अगस्त्यवचनाच्चैव जग्राहैन्द्रं शरासनम्} %1-1-42

\twolineshloka
{खड्गञ्च परम प्रीतस्तूणी चाक्षयसायकौ}
{वसतस्तस्य रामस्य वने वनचरैः सह} %1-1-43

\twolineshloka
{ऋषयोऽभ्यागमन् सर्वे वधायासुररक्षसाम्}
{स तेषां प्रतिशुश्राव राक्षसानां तदा वने} %1-1-44

\twolineshloka
{प्रतिज्ञातश्च रामेण वधः संयति रक्षसाम्}
{ऋषीणामग्निकल्पानां दण्डकारण्यवासिनाम्} %1-1-45

\twolineshloka
{तेन तत्रैव वसता जनस्थाननिवासिनी}
{विरूपिता शूर्पणखा राक्षसी कामरूपिणी} %1-1-46

\twolineshloka
{ततः शूर्पणखावाक्यादुद्युक्तान् सर्वराक्षसान्}
{खरं त्रिशिरसं चैव दूषणं चैव राक्षसम्} %1-1-47

\twolineshloka
{निजघान रणे रामस्तेषां चैव पदानुगान्}
{वने तस्मिन् निवसता जनस्थाननिवासिनाम्} %1-1-48

\twolineshloka
{रक्षसां निहतान्यासन् सहस्राणि चतुर्दश}
{ततो ज्ञातिवधं श्रुत्वा रावणः क्रोधमूर्छितः} %1-1-49

\twolineshloka
{सहायं वरयामास मारीचं नाम राक्षसम्}
{वार्यमाणः सुबहुशो मारीचेन स रावणः} %1-1-50

\twolineshloka
{न विरोधो बलवता क्षमो रावण तेन ते}
{अनादृत्य तु तद्वाक्यं रावणः कालचोदितः} %1-1-51

\twolineshloka
{जगाम सहमारीचस्तस्याश्रमपदं तदा}
{तेन मायाविना दूरमपवाह्य नृपात्मजौ} %1-1-52

\twolineshloka
{जहार भार्यां रामस्य गृध्रं हत्वा जटायुषम्}
{गृध्रञ्च निहतं दृष्ट्वा हृतां श्रुत्वा च मैथिलीम्} %1-1-53

\twolineshloka
{राघवः शोकसंतप्तो विललापाकुलेन्द्रियः}
{ततस्तेनैव शोकेन गृध्रं दग्ध्वा जटायुषम्} %1-1-54

\twolineshloka
{मार्गमाणो वने सीतां राक्षसं सन्ददर्श ह}
{कबन्धं नाम रूपेण विकृतं घोरदर्शनम्} %1-1-55

\twolineshloka
{तन्निहत्य महाबाहुर्ददाह स्वर्गतश्च सः}
{स चास्य कथयामास शबरीं धर्मचारिणीम्} %1-1-56

\twolineshloka
{श्रमणां धर्मनिपुणामभिगच्छेति राघव}
{सोऽभ्य गच्छन्महातेजाः शबरीं शत्रुसूदनः} %1-1-57

\twolineshloka
{शबर्या पूजितः सम्यग् रामो दशरथात्मजः}
{पम्पातीरे हनुमता सङ्गतो वानरेण ह} %1-1-58

\twolineshloka
{हनुमद्वचनाच्चैव सुग्रीवेण समागतः}
{सुग्रीवाय च तत्सर्वं शंसद्रामो महाबलः} %1-1-59

\twolineshloka
{आदितस्तद् यथावृत्तं सीतायाश्च विशेषतः}
{सुग्रीवश्चापि तत्सर्वं श्रुत्वा रामस्य वानरः} %1-1-60

\twolineshloka
{चकार सख्यं रामेण प्रीतश्चैवाग्निसाक्षिकम्}
{ततो वानरराजेन वैरानुकथनं प्रति} %1-1-61

\twolineshloka
{रामायावेदितं सर्वं प्रणयात् दुःखितेन च}
{प्रतिज्ञातं च रामेण तदा वालिवधं प्रति} %1-1-62

\twolineshloka
{वालिनश्च बलं तत्र कथयामास वानरः}
{सुग्रीवः शङ्कितश्चासीन्नित्यं वीर्येण राघवे} %1-1-63

\twolineshloka
{राघवप्रत्ययार्थं तु दुन्दुभेः कायमुत्तमम्}
{दर्शयामास सुग्रीवो महापर्वतसन्निभम्} %1-1-64

\twolineshloka
{उत्स्मयित्वा महाबाहुः प्रेक्ष्य चास्ति महाबलः}
{पादाङ्गुष्ठेन चिक्षेप संपूर्णं दशयोजनम्} %1-1-65

\twolineshloka
{बिभेद च पुनस्सालान् सप्तैकेन महेषुणा}
{गिरिं रसातलञ्चैव जनयन् प्रत्ययं तथा} %1-1-66

\twolineshloka
{ततः प्रीतमनास्तेन विश्वस्तस्स महाकपिः}
{किष्किन्धां रामसहितो जगाम च गुहां तदा} %1-1-67

\twolineshloka
{ततोऽगर्जद्धरिवरः सुग्रीवो हेमपिङ्गलः}
{तेन नादेन महता निर्जगाम हरीश्वरः} %1-1-68

\twolineshloka
{अनुमान्य तदा तारां सुग्रीवेण समागतः}
{निजघान च तत्रैनं शरेणैकेन राघवः} %1-1-69

\twolineshloka
{ततः सुग्रीववचनात् हत्वा वालिनमाहवे}
{सुग्रीवमेव तद्राज्ये राघवः प्रत्यपादयत्} %1-1-70

\twolineshloka
{स च सर्वान् समानीय वानरान् वानरर्षभः}
{दिशः प्रस्थापयामास दिदृक्षुर्जनकात्मजाम्} %1-1-71

\twolineshloka
{ततो गृध्रस्य वचनात् संपातेर्हनुमान् बली}
{शतयोजनविस्तीर्णं पुप्लुवे लवणार्णवम्} %1-1-72

\twolineshloka
{तत्र लङ्कां समासाद्य पुरीं रावणपालिताम्}
{ददर्श सीतां ध्यायन्तीम् अशोकवनिकां गताम्} %1-1-73

\twolineshloka
{निवेदयित्वाभिज्ञानं प्रवृत्तिं विनिवेद्य च}
{समाश्वास्य च वैदेहीं मर्दयामास तोरणम्} %1-1-74

\twolineshloka
{पञ्च सेनाग्रगान् हत्वा सप्त मन्त्रिसुतानपि}
{शूरमक्षं च निष्पिष्य ग्रहणं समुपागमत्} %1-1-75

\twolineshloka
{अस्त्रेणोन्मुक्तमात्मानं ज्ञात्वा पैतामहाद् वरात्}
{मर्षयन् राक्षसान् वीरो यन्त्रिणस्तान् यदृच्छया} %1-1-76

\twolineshloka
{ततो दग्ध्वा पुरीं लङ्काम् ऋते सीताञ्च मैथिलीम्}
{रामाय प्रियमाख्यातुं पुनरायान्महाकपिः} %1-1-77

\twolineshloka
{सोऽभिगम्य महात्मानं कृत्वा रामं प्रदक्षिणम्}
{न्यवेदयदमेयात्मा दृष्टा सीतेति तत्त्वतः} %1-1-78

\twolineshloka
{ततः सुग्रीवसहितो गत्वा तीरं महोदधेः}
{समुद्रं क्षोभयामास शरैरादित्यसन्निभैः} %1-1-79

\twolineshloka
{दर्शयामास चात्मानं समुद्रः सरितां पतिः}
{समुद्रवचनाच्चैव नलं सेतुमकारयत्} %1-1-80

\twolineshloka
{तेन गत्वा पुरीं लङ्कां हत्वा रावणमाहवे}
{रामः सीतामनुप्राप्य परां व्रीडामुपागमत्} %1-1-81

\twolineshloka
{तामुवाच ततो रामः परुषं जनसंसदि}
{अमृष्यमाणा सा सीता विवेश ज्वलनं सती} %1-1-82

\twolineshloka
{ततोऽग्निवचनात् सीतां ज्ञात्वा विगतकल्मषाम्}
{कर्मणा तेन महता त्रैलोक्यं सचराचरम्} %1-1-83

\twolineshloka
{सदेवर्षिगणं तुष्टं राघवस्य महात्मनः}
{बभौ रामः सम्प्रहृष्टः पूजितः सर्वदेवतैः} %1-1-84

\twolineshloka
{अभ्यषिच्य च लङ्कायां राक्षसेन्द्रं विभीषणम्}
{कृतकृत्यस्तदा रामो विज्वरः प्रमुमोद ह} %1-1-85

\twolineshloka
{देवताभ्यो वरं प्राप्य समुत्थाप्य च वानरान्}
{अयोध्यां प्रस्थितो रामः पुष्पकेण सुहृद्वृतः} %1-1-86

\twolineshloka
{भरद्वाजाश्रमं गत्वा रामः सत्यपराक्रमः}
{भरतस्यान्तिके रामो हनूमन्तं व्यसर्जयत्} %1-1-87

\twolineshloka
{पुनराख्यायिकां जल्पन् सुग्रीवसहितस्तदा}
{पुष्पकं तत् समारुह्य नन्दिग्रामं ययौ तदा} %1-1-88

\twolineshloka
{नन्दिग्रामे जटां हित्वा भ्रातृभिः सहितोऽनघः}
{रामः सीतामनुप्राप्य राज्यं पुनरवाप्तवान्} %1-1-89

\twolineshloka
{प्रहृष्टमुदितो लोकस्तुष्टः पुष्टः सुधार्मिकः}
{निरामयो ह्यरोगश्च दुर्भिक्षभयवर्जितः} %1-1-90

\twolineshloka
{न पुत्रमरणं केचित् द्रक्ष्यन्ति पुरुषाः क्वचित्}
{नार्यश्चाविधवा नित्यं भविष्यन्ति पतिव्रताः} %1-1-91

\twolineshloka
{न चाग्निजं भयं किञ्चिन्नाप्सु मज्जन्ति जन्तवः}
{न वातजं भयं किञ्चित् नापि ज्वरकृतं तथा} %1-1-92

\twolineshloka
{न चापि क्षुद्भयं तत्र न तस्करभयं तथा}
{नगराणि च राष्ट्राणि धनधान्ययुतानि च} %1-1-93

\twolineshloka
{नित्यं प्रमुदिताः सर्वे यथा कृतयुगे तथा}
{अश्वमेधशतैरिष्ट्वा तथा बहुसुवर्णकैः} %1-1-94

\twolineshloka
{गवां कोट्ययुतं दत्त्वा विद्वद्भ्यो विधिपूर्वकम्}
{असंख्येयं धनं दत्त्वा ब्राह्मणेभ्यो महायशाः} %1-1-95

\twolineshloka
{राजवंशान् शतगुणान् स्थापयिष्यति राघवः}
{चातुर्वर्ण्यं च लोकेऽस्मिन् स्वे स्वे धर्मे नियोक्ष्यति} %1-1-96

\twolineshloka
{दशवर्षसहस्राणि दशवर्षशतानि च}
{रामो राज्यमुपासित्वा ब्रह्मलोकं प्रयास्यति} %1-1-97

\twolineshloka
{इदं पवित्रं पापघ्नं पुण्यं वेदैश्च सम्मितम्}
{यः पठेद् रामचरितं सर्वपापैः प्रमुच्यते} %1-1-98

\twolineshloka
{एतदाख्यानमायुष्यं पठन् रामायणं नरः}
{सपुत्रपौत्रः सगणः प्रेत्य स्वर्गे महीयते} %1-1-99

\fourlineindentedshloka
{पठन् द्विजो वागृषभत्वमीयात्}
{स्यात् क्षत्रियो भूमिपतित्वमीयात्}
{वणिक् जनः पण्यफलत्वमीयात्}
{जनश्च शूद्रोऽपि महत्त्वमीयात्} %1-1-100


॥इत्यार्षे श्रीमद्रामायणे वाल्मीकीये आदिकाव्ये बालकाण्डे नारदवाक्यम् नाम प्रथमः सर्गः ॥१-१॥
