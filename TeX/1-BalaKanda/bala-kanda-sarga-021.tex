\sect{एकविंशः सर्गः — वसिष्ठवाक्यम्}

\twolineshloka
{तच्छ्रुत्वा वचनं तस्य स्नेहपर्याकुलाक्षरम्}
{समन्युः कौशिको वाक्यं प्रत्युवाच महीपतिम्} %1-21-1

\twolineshloka
{पूर्वमर्थं प्रतिश्रुत्य प्रतिज्ञां हातुमिच्छसि}
{राघवाणामयुक्तोऽयं कुलस्यास्य विपर्ययः} %1-21-2

\twolineshloka
{यदीदं ते क्षमं राजन् गमिष्यामि यथागतम्}
{मिथ्याप्रतिज्ञः काकुत्स्थ सुखी भव सुहृद्वृतः} %1-21-3

\twolineshloka
{तस्य रोषपरीतस्य विश्वामित्रस्य धीमतः}
{चचाल वसुधा कृत्स्ना देवानां च भयं महत्} %1-21-4

\twolineshloka
{त्रस्तरूपं तु विज्ञाय जगत्सर्वं महानृषिः}
{नृपतिं सुव्रतो धीरो वसिष्ठो वाक्यमब्रवीत्} %1-21-5

\twolineshloka
{इक्ष्वाकूणां कुले जातः साक्षाद् धर्म इवापरः}
{धृतिमान् सुव्रतः श्रीमान् न धर्मं हातुमर्हसि} %1-21-6

\twolineshloka
{त्रिषु लोकेषु विख्यातो धर्मात्मा इति राघवः}
{स्वधर्मं प्रतिपद्यस्व नाधर्मं वोढुमर्हसि} %1-21-7

\twolineshloka
{प्रतिश्रुत्य करिष्येति उक्तं वाक्यमकुर्वतः}
{इष्टापूर्तवधो भूयात् तस्माद् रामं विसर्जय} %1-21-8

\twolineshloka
{कृतास्त्रमकृतास्त्रं वा नैनं शक्ष्यन्ति राक्षसाः}
{गुप्तं कुशिकपुत्रेण ज्वलनेनामृतं यथा} %1-21-9

\twolineshloka
{एष विग्रहवान् धर्म एष वीर्यवतां वरः}
{एष विद्याधिको लोके तपसश्च परायणम्} %1-21-10

\twolineshloka
{एषोऽस्त्रान् विविधान् वेत्ति त्रैलोक्ये सचराचरे}
{नैनमन्यः पुमान् वेत्ति न च वेत्स्यन्ति केचन} %1-21-11

\twolineshloka
{न देवा नर्षयः केचिन्नामरा न च राक्षसाः}
{गन्धर्वयक्षप्रवराः सकिन्नरमहोरगाः} %1-21-12

\twolineshloka
{सर्वास्त्राणि कृशाश्वस्य पुत्राः परमधार्मिकाः}
{कौशिकाय पुरा दत्ता यदा राज्यं प्रशासति} %1-21-13

\twolineshloka
{तेऽपि पुत्राः कृशाश्वस्य प्रजापतिसुतासुताः}
{नैकरूपा महावीर्या दीप्तिमन्तो जयावहाः} %1-21-14

\twolineshloka
{जया च सुप्रभा चैव दक्षकन्ये सुमध्यमे}
{ते सुवातेऽस्त्रशस्त्राणि शतं परमभास्वरम्} %1-21-15

\twolineshloka
{पञ्चाशतं सुताँल्लेभे जया लब्धवरा वरान्}
{वधायासुरसैन्यानामप्रमेयानुरूपिणः} %1-21-16

\twolineshloka
{सुप्रभाजनयच्चापि पुत्रान् पञ्चाशतं पुनः}
{संहारान् नाम दुर्धर्षान् दुराक्रामान् बलीयसः} %1-21-17

\twolineshloka
{तानि चास्त्राणि वेत्त्येष यथावत् कुशिकात्मजः}
{अपूर्वाणां च जनने शक्तो भूयश्च धर्मवित्} %1-21-18

\twolineshloka
{तेनास्य मुनिमुख्यस्य धर्मज्ञस्य महात्मनः}
{न किञ्चिदस्त्यविदितं भूतं भव्यम् च राघव} %1-21-19

\twolineshloka
{एवंवीर्यो महातेजा विश्वामित्रो महायशाः}
{न रामगमने राजन् संशयं गन्तुमर्हसि} %1-21-20

\twolineshloka
{तेषां निग्रहणे शक्तः स्वयं च कुशिकात्मजः}
{तव पुत्रहितार्थाय त्वामुपेत्याभियाचते} %1-21-21

\twolineshloka
{इति मुनिवचनात् प्रसन्नचित्तो रघुवृषभश्च मुमोद पार्थिवाग्र्यः}
{गमनमभिरुरोच राघवस्य प्रथितयशाः कुशिकात्मजाय बुद्ध्या} %1-21-22


॥इत्यार्षे श्रीमद्रामायणे वाल्मीकीये आदिकाव्ये बालकाण्डे वसिष्ठवाक्यम् नाम एकविंशः सर्गः ॥१-२१॥
