\sect{एकचत्वारिंशः सर्गः — सगरयज्ञसमाप्तिः}

\twolineshloka
{पुत्रांश्चिरगताञ्ज्ञात्वा सगरो रघुनन्दन}
{नप्तारमब्रवीद् राजा दीप्यमानं स्वतेजसा} %1-41-1

\twolineshloka
{शूरश्च कृतविद्यश्च पूर्वैस्तुल्योऽसि तेजसा}
{पितॄणां गतिमन्विच्छ येन चाश्वोऽपवाहितः} %1-41-2

\twolineshloka
{अन्तर्भौमानि सत्त्वानि वीर्यवन्ति महान्ति च}
{तेषां तु प्रतिघातार्थं सासिं गृह्णीष्व कार्मुकम्} %1-41-3

\twolineshloka
{अभिवाद्याभिवाद्यांस्त्वं हत्वा विघ्नकरानपि}
{सिद्धार्थः संनिवर्तस्व मम यज्ञस्य पारगः} %1-41-4

\twolineshloka
{एवमुक्तोंऽशुमान् सम्यक् सगरेण महात्मना}
{धनुरादाय खड्गं च जगाम लघुविक्रमः} %1-41-5

\twolineshloka
{स खातं पितृभिर्मार्गमन्तर्भौमं महात्मभिः}
{प्रापद्यत नरश्रेष्ठ तेन राज्ञाभिचोदितः} %1-41-6

\twolineshloka
{देवदानवरक्षोभिः पिशाचपतगोरगैः}
{पूज्यमानं महातेजा दिशागजमपश्यत} %1-41-7

\twolineshloka
{स तं प्रदक्षिणं कृत्वा पृष्ट्वा चैव निरामयम्}
{पितॄन् स परिपप्रच्छ वाजिहर्तारमेव च} %1-41-8

\twolineshloka
{दिशागजस्तु तच्छ्रुत्वा प्रत्युवाच महामतिः}
{आसमञ्ज कृतार्थस्त्वं सहाश्वः शीघ्रमेष्यसि} %1-41-9

\twolineshloka
{तस्य तद्वचनं श्रुत्वा सर्वानेव दिशागजान्}
{यथाक्रमं यथान्यायं प्रष्टुं समुपचक्रमे} %1-41-10

\twolineshloka
{तैश्च सर्वैर्दिशापालैर्वाक्यज्ञैर्वाक्यकोविदैः}
{पूजितः सहयश्चैवागन्तासीत्यभिचोदितः} %1-41-11

\twolineshloka
{तेषां तद्वचनं श्रुत्वा जगाम लघुविक्रमः}
{भस्मराशीकृता यत्र पितरस्तस्य सागराः} %1-41-12

\twolineshloka
{स दुःखवशमापन्नस्त्वसमञ्जसुतस्तदा}
{चुक्रोश परमार्तस्तु वधात् तेषां सुदुःखितः} %1-41-13

\twolineshloka
{यज्ञियं च हयं तत्र चरन्तमविदूरतः}
{ददर्श पुरुषव्याघ्रो दुःखशोकसमन्वितः} %1-41-14

\twolineshloka
{स तेषां राजपुत्राणां कर्तुकामो जलक्रियाम्}
{स जलार्थी महातेजा न चापश्यज्जलाशयम्} %1-41-15

\twolineshloka
{विसार्य निपुणां दृष्टिं ततोऽपश्यत् खगाधिपम्}
{पितॄणां मातुलं राम सुपर्णमनिलोपमम्} %1-41-16

\twolineshloka
{स चैनमब्रवीद् वाक्यं वैनतेयो महाबलः}
{मा शुचः पुरुषव्याघ्र वधोऽयं लोकसम्मतः} %1-41-17

\twolineshloka
{कपिलेनाप्रमेयेण दग्धा हीमे महाबलाः}
{सलिलं नार्हसि प्राज्ञ दातुमेषां हि लौकिकम्} %1-41-18

\twolineshloka
{गङ्गा हिमवतो ज्येष्ठा दुहिता पुरुषर्षभ}
{तस्यां कुरु महाबाहो पितॄणां तु जल क्रियाम्} %1-41-19

\threelineshloka
{भस्मराशीकृतानेतान् प्लावयेल्लोकपावनी}
{तया क्लिन्नमिदं भस्म गङ्गया लोककान्तया}
{षष्टिं पुत्रसहस्राणि स्वर्गलोकं गमिष्यति} %1-41-20

\twolineshloka
{निर्गच्छाश्वं महाभाग संगृह्य पुरुषर्षभ}
{यज्ञं पैतामहं वीर निर्वर्तयितुमर्हसि} %1-41-21

\twolineshloka
{सुपर्णवचनं श्रुत्वा सोंऽशुमानतिवीर्यवान्}
{त्वरितं हयमादाय पुनरायान्महातपाः} %1-41-22

\twolineshloka
{ततो राजानमासाद्य दीक्षितं रघुनन्दन}
{न्यवेदयद् यथावृत्तं सुपर्णवचनं तथा} %1-41-23

\twolineshloka
{तच्छ्रुत्वा घोरसंकाशं वाक्यमंशुमतो नृपः}
{यज्ञं निर्वर्तयामास यथाकल्पं यथाविधि} %1-41-24

\twolineshloka
{स्वपुरं त्वगमच्छ्रीमानिष्टयज्ञो महीपतिः}
{गङ्गायाश्चागमे राजा निश्चयं नाध्यगच्छत} %1-41-25

\twolineshloka
{अगत्वा निश्चयं राजा कालेन महता महान्}
{त्रिंशद्वर्षसहस्राणि राज्यं कृत्वा दिवं गतः} %1-41-26


॥इत्यार्षे श्रीमद्रामायणे वाल्मीकीये आदिकाव्ये बालकाण्डे सगरयज्ञसमाप्तिः नाम एकचत्वारिंशः सर्गः ॥१-४१॥
