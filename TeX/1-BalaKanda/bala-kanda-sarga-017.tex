\sect{सप्तदशः सर्गः — ऋक्षवानरोत्पत्तिः}

\twolineshloka
{पुत्रत्वं तु गते विष्णौ राज्ञस्तस्य महात्मनः}
{उवाच देवताः सर्वाः स्वयम्भूर्भगवानिदम्} %1-17-1

\twolineshloka
{सत्यसन्धस्य वीरस्य सर्वेषां नो हितैषिणः}
{विष्णोः सहायान् बलिनः सृजध्वं कामरूपिणः} %1-17-2

\twolineshloka
{मायाविदश्च शूरांश्च वायुवेगसमान् जवे}
{नयज्ञान् बुद्धिसम्पन्नान् विष्णुतुल्यपराक्रमान्} %1-17-3

\twolineshloka
{असंहार्यानुपायज्ञान् दिव्यसंहननान्वितान्}
{सर्वास्त्रगुणसम्पन्नानमृतप्राशनानिव} %1-17-4

\twolineshloka
{अप्सरस्सु च मुख्यासु गन्धर्वीणां तनूषु च}
{यक्षपन्नगकन्यासु ऋक्षविद्याधरीषु च} %1-17-5

\twolineshloka
{किन्नरीणां च गात्रेषु वानरीणां तनूषु च}
{सृजध्वं हरिरूपेण पुत्रांस्तुल्यपराक्रमान्} %1-17-6

\twolineshloka
{पूर्वमेव मया सृष्टो जाम्बवान् ऋक्षपुङ्गवः}
{जृम्भमाणस्य सहसा मम वक्त्रादजायत} %1-17-7

\twolineshloka
{ते तथोक्ता भगवता तत् प्रतिश्रुत्य शासनम्}
{जनयामासुरेवं ते पुत्रान् वानररूपिणः} %1-17-8

\twolineshloka
{ऋषयश्च महात्मानः सिद्धविद्याधरोरगाः}
{चारणाश्च सुतान् वीरान् ससृजुर्वनचारिणः} %1-17-9

\twolineshloka
{वानरेन्द्रं महेन्द्राभमिन्द्रो वालिनमात्मजम्}
{सुग्रीवं जनयामास तपनस्तपतां वरः} %1-17-10

\twolineshloka
{बृहस्पतिस्त्वजनयत् तारं नाम महाकपिम्}
{सर्ववानरमुख्यानां बुद्धिमन्तमनुत्तमम्} %1-17-11

\twolineshloka
{धनदस्य सुतः श्रीमान् वानरो गन्धमादनः}
{विश्वकर्मा त्वजनयन्नलं नाम महाकपिम्} %1-17-12

\twolineshloka
{पावकस्य सुतः श्रीमान् नीलोऽग्निसदृशप्रभः}
{तेजसा यशसा वीर्यादत्यरिच्यत वीर्यवान्} %1-17-13

\twolineshloka
{रूपद्रविणसम्पन्नावश्विनौ रूपसम्मतौ}
{मैन्दं च द्विविदं चैव जनयामासतुः स्वयम्} %1-17-14

\twolineshloka
{वरुणो जनयामास सुषेणं नाम वानरम्}
{शरभं जनयामास पर्जन्यस्तु महाबलः} %1-17-15

\twolineshloka
{मारुतस्यौरसः श्रीमान् हनूमान् नाम वानरः}
{वज्रसंहननोपेतो वैनतेयसमो जवे} %1-17-16

\twolineshloka
{सर्ववानरमुख्येषु बुद्धिमान् बलवानपि}
{ते सृष्टा बहुसाहस्रा दशग्रीववधोद्यताः} %1-17-17

\twolineshloka
{अप्रमेयबला वीरा विक्रान्ताः कामरूपिणः}
{ते गजाचलसङ्काशा वपुष्मन्तो महाबलाः} %1-17-18

\twolineshloka
{ऋक्षवानरगोपुच्छाः क्षिप्रमेवाभिजज्ञिरे}
{यस्य देवस्य यद्रूपं वेषो यश्च पराक्रमः} %1-17-19

\twolineshloka
{अजायत समं तेन तस्य तस्य पृथक् पृथक्}
{गोलाङ्गूलीषु चोत्पन्नाः किञ्चिदुन्नतविक्रमाः} %1-17-20

\twolineshloka
{ऋक्षीषु च तथा जाता वानराः किन्नरीषु च}
{देवा महर्षिगन्धर्वास्तार्क्ष्ययक्षा यशस्विनः} %1-17-21

\twolineshloka
{नागाः किम्पुरुषाश्चैव सिद्धविद्याधरोरगाः}
{बहवो जनयामासुर्हृष्टास्तत्र सहस्रशः} %1-17-22

\twolineshloka
{चारणाश्च सुतान् वीरान् ससृजुर्वनचारिणः}
{वानरान् सुमहाकायान् सर्वान् वै वनचारिणः} %1-17-23

\threelineshloka
{अप्सरस्सु च मुख्यासु तथा विद्याधरीषु च}
{नागकन्यासु च तदा गन्धर्वीणां तनूषु च}
{कामरूपबलोपेता यथाकामविचारिणः} %1-17-24

\twolineshloka
{सिंहशार्दूलसदृशा दर्पेण च बलेन च}
{शिलाप्रहरणाः सर्वे सर्वे पादपयोधिनः} %1-17-25

\twolineshloka
{नखदंष्ट्रायुधाः सर्वे सर्वे सर्वास्त्रकोविदाः}
{विचालयेयुः शैलेन्द्रान् भेदयेयुः स्थिरान् द्रुमान्} %1-17-26

\twolineshloka
{क्षोभयेयुश्च वेगेन समुद्रं सरितां पतिम्}
{दारयेयुः क्षितिं पद्भ्यामाप्लवेयुर्महार्णवान्} %1-17-27

\twolineshloka
{नभस्तलं विशेयुश्च गृह्णीयुरपि तोयदान्}
{गृह्णीयुरपि मातङ्गान् मत्तान् प्रव्रजतो वने} %1-17-28

\twolineshloka
{नर्दमानांश्च नादेन पातयेयुर्विहङ्गमान्}
{ईदृशानां प्रसूतानि हरीणां कामरूपिणाम्} %1-17-29

\twolineshloka
{शतं शतसहस्राणि यूथपानां महात्मनाम्}
{ते प्रधानेषु यूथेषु हरीणां हरियूथपाः} %1-17-30

\twolineshloka
{बभूवुर्यूथपश्रेष्ठान् वीरांश्चाजनयन् हरीन्}
{अन्ये ऋक्षवतः प्रस्थानुपतस्थुः सहस्रशः} %1-17-31

\twolineshloka
{अन्ये नानाविधाञ्छैलान् काननानि च भेजिरे}
{सूर्यपुत्रं च सुग्रीवं शक्रपुत्रं च वालिनम्} %1-17-32

\twolineshloka
{भ्रातरावुपतस्थुस्ते सर्वे च हरियूथपाः}
{नलं नीलं हनूमन्तमन्यांश्च हरियूथपान्} %1-17-33

\twolineshloka
{ते तार्क्ष्यबलसम्पन्नाः सर्वे युद्धविशारदाः}
{विचरन्तोऽर्दयन् सर्वान् सिंहव्याघ्रमहोरगान्} %1-17-34

\twolineshloka
{महाबलो महाबाहुर्वाली विपुलविक्रमः}
{जुगोप भुजवीर्येण ऋक्षगोपुच्छवानरान्} %1-17-35

\twolineshloka
{तैरियं पृथिवी शूरैः सपर्वतवनार्णवा}
{कीर्णा विविधसंस्थानैर्नानाव्यञ्जनलक्षणैः} %1-17-36

\twolineshloka
{तैर्मेघवृन्दाचलकूटसन्निभैर्महाबलैर्वानरयूथपाधिपैः}
{बभूव भूर्भीमशरीररूपैः समावृता रामसहायहेतोः} %1-17-37


॥इत्यार्षे श्रीमद्रामायणे वाल्मीकीये आदिकाव्ये बालकाण्डे ऋक्षवानरोत्पत्तिः नाम सप्तदशः सर्गः ॥१-१७॥
