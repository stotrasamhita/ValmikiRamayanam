\sect{षष्ठः सर्गः — राजवर्णनम्}

\twolineshloka
{तस्यां पुर्यामयोध्यायां वेदवित् सर्वसंग्रहः}
{दीर्घदर्शी महातेजाः पौरजानपदप्रियः} %1-6-1

\twolineshloka
{इक्ष्वाकूणामतिरथो यज्वा धर्मपरो वशी}
{महर्षिकल्पो राजर्षिस्त्रिषु लोकेषु विश्रुतः} %1-6-2

\twolineshloka
{बलवान् निहतामित्रो मित्रवान् विजितेन्द्रियः}
{धनैश्च संचयैश्चान्यैः शक्रवैश्रवणोपमः} %1-6-3

\twolineshloka
{यथा मनुर्महातेजा लोकस्य परिरक्षिता}
{तथा दशरथो राजा लोकस्य परिरक्षिता} %1-6-4

\twolineshloka
{तेन सत्याभिसंधेन त्रिवर्गमनुतिष्ठता}
{पालिता सा पुरी श्रेष्ठा इन्द्रेणेवामरावती} %1-6-5

\twolineshloka
{तस्मिन् पुरवरे हृष्टा धर्मात्मानो बहुश्रुताः}
{नरास्तुष्टा धनैः स्वैः स्वैरलुब्धाः सत्यवादिनः} %1-6-6

\twolineshloka
{नाल्पसंनिचयः कश्चिदासीत् तस्मिन् पुरोत्तमे}
{कुटुम्बी यो ह्यसिद्धार्थोऽगवाश्वधनधान्यवान्} %1-6-7

\twolineshloka
{कामी वा न कदर्यो वा नृशंसः पुरुषः क्वचित्}
{द्रष्टुं शक्यमयोध्यायां नाविद्वान् न च नास्तिकः} %1-6-8

\twolineshloka
{सर्वे नराश्च नार्यश्च धर्मशीलाः सुसंयताः}
{मुदिताः शीलवृत्ताभ्यां महर्षय इवामलाः} %1-6-9

\twolineshloka
{नाकुण्डली नामुकुटी नास्रग्वी नाल्पभोगवान्}
{नामृष्टो न नलिप्ताङ्गो नासुगन्धश्च विद्यते} %1-6-10

\twolineshloka
{नामृष्टभोजी नादाता नाप्यनङ्गदनिष्कधृक्}
{नाहस्ताभरणो वापि दृश्यते नाप्यनात्मवान्} %1-6-11

\twolineshloka
{नानाहिताग्निर्नायज्वा न क्षुद्रो वा न तस्करः}
{कश्चिदासीदयोध्यायां न चार्वृत्तो न संकरः} %1-6-12

\twolineshloka
{स्वकर्मनिरता नित्यं ब्राह्मणा विजितेन्द्रियाः}
{दानाध्ययनशीलाश्च संयताश्च प्रतिग्रहे} %1-6-13

\twolineshloka
{नास्तिको नानृती वापि न कश्चिदबहुश्रुतः}
{नासूयको न चाशक्तो नाविद्वान् विद्यते क्वचित्} %1-6-14

\twolineshloka
{नाषडङ्गविदत्रास्ति नाव्रतो नासहस्रदः}
{न दीनः क्षिप्तचित्तो वा व्यथितो वापि कश्चन} %1-6-15

\twolineshloka
{कश्चिन्नरो वा नारी वा नाश्रीमान् नाप्यरूपवान्}
{द्रष्टुं शक्यमयोध्यायां नापि राजन्यभक्तिमान्} %1-6-16

\twolineshloka
{वर्णेष्वग्र्यचतुर्थेषु देवतातिथिपूजकाः}
{कृतज्ञाश्च वदान्याश्च शूरा विक्रमसंयुताः} %1-6-17

\twolineshloka
{दीर्घायुषो नराः सर्वे धर्मं सत्यं च संश्रिताः}
{सहिताः पुत्रपौत्रैश्च नित्यं स्त्रीभिः पुरोत्तमे} %1-6-18

\twolineshloka
{क्षत्रं ब्रह्ममुखं चासीद् वैश्याः क्षत्रमनुव्रताः}
{शूद्राः स्वधर्मनिरतास्त्रीन् वर्णानुपचारिणः} %1-6-19

\twolineshloka
{सा तेनेक्ष्वाकुनाथेन पुरी सुपरिरक्षिता}
{यथा पुरस्तान्मनुना मानवेन्द्रेण धीमता} %1-6-20

\twolineshloka
{योधानामग्निकल्पानां पेशलानाममर्षिणाम्}
{सम्पूर्णा कृतविद्यानां गुहा केसरिणामिव} %1-6-21

\twolineshloka
{काम्बोजविषये जातैर्बाह्लीकैश्च हयोत्तमैः}
{वनायुजैर्नदीजैश्च पूर्णा हरिहयोत्तमैः} %1-6-22

\twolineshloka
{विन्ध्यपर्वतजैर्मत्तैः पूर्णा हैमवतैरपि}
{मदान्वितैरतिबलैर्मातङ्गैः पर्वतोपमैः} %1-6-23

\twolineshloka
{ऐरावतकुलीनैश्च महापद्मकुलैस्तथा}
{अञ्जनादपि निष्क्रान्तैर्वामनादपि च द्विपैः} %1-6-24

\twolineshloka
{भदैर्मन्द्रैर्मृगैश्चैव भद्रमन्द्रमृगैस्तथा}
{भद्रमन्द्रैर्भद्रमृगैर्मृगमन्द्रैश्च सा पुरी} %1-6-25

\threelineshloka
{नित्यमत्तैः सदा पूर्णा नागैरचलसंनिभैः}
{सा योजने द्वे च भूयः सत्यनामा प्रकाशते}
{यस्यां दशरथो राजा वसञ्जगदपालयत्} %1-6-26

\twolineshloka
{तां पुरीं स महातेजा राजा दशरथो महान्}
{शशास शमितामित्रो नक्षत्राणीव चन्द्रमाः} %1-6-27

\twolineshloka
{तां सत्यनामां दृढतोरणार्गलाम् गृहैर्विचित्रैरुपशोभितां शिवाम्}
{पुरीमयोध्यां नृसहस्रसंकुलां शशास वै शक्रसमो महीपतिः} %1-6-28


॥इत्यार्षे श्रीमद्रामायणे वाल्मीकीये आदिकाव्ये बालकाण्डे राजवर्णनम् नाम षष्ठः सर्गः ॥१-६॥
