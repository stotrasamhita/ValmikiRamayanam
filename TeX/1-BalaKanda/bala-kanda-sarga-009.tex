\sect{नवमः सर्गः — ऋष्यशृङ्गोपाख्यानम्}

\twolineshloka
{एतच्छ्रुत्वा रहः सूतो राजानमिदमब्रवीत्}
{श्रूयतां तत् पुरावृत्तं पुराणे च मया श्रुतम्} %1-9-1

\twolineshloka
{ऋत्विग्भिरुपदिष्टोऽयं पुरावृत्तो मया श्रुतः}
{सनत्कुमारो भगवान् पूर्वं कथितवान् कथाम्} %1-9-2

\twolineshloka
{ऋषीणां संनिधौ राजंस्तव पुत्रागमं प्रति}
{काश्यपस्य च पुत्रोऽस्ति विभाण्डक इति श्रुतः} %1-9-3

\twolineshloka
{ऋष्यशृङ्ग इति ख्यातस्तस्य पुत्रो भविष्यति}
{स वने नित्यसंवृद्धो मुनिर्वनचरः सदा} %1-9-4

\twolineshloka
{नान्यं जानाति विप्रेन्द्रो नित्यं पित्रनुवर्तनात्}
{द्वैविध्यं ब्रह्मचर्यस्य भविष्यति महात्मनः} %1-9-5

\twolineshloka
{लोकेषु प्रथितं राजन् विप्रैश्च कथितं सदा}
{तस्यैवं वर्तमानस्य कालः समभिवर्तत} %1-9-6

\twolineshloka
{अग्निं शुश्रूषमाणस्य पितरं च यशस्विनम्}
{एतस्मिन्नेव काले तु रोमपादः प्रतापवान्} %1-9-7

\twolineshloka
{अङ्गेषु प्रथितो राजा भविष्यति महाबलः}
{तस्य व्यतिक्रमाद् राज्ञो भविष्यति सुदारुणा} %1-9-8

\twolineshloka
{अनावृष्टिः सुघोरा वै सर्वलोकभयावहा}
{अनावृष्ट्यां तु वृत्तायां राजा दुःखसमन्वितः} %1-9-9

\twolineshloka
{ब्राह्मणाञ्छ्रुतसंवृद्धान् समानीय प्रवक्ष्यति}
{भवन्तः श्रुतकर्माणो लोकचारित्रवेदिनः} %1-9-10

\twolineshloka
{समादिशन्तु नियमं प्रायश्चित्तं यथा भवेत्}
{इत्युक्तास्ते ततो राज्ञा सर्वे ब्राह्मणसत्तमाः} %1-9-11

\twolineshloka
{वक्ष्यन्ति ते महीपालं ब्राह्मणा वेदपारगाः}
{विभाण्डकसुतं राजन् सर्वोपायैरिहानय} %1-9-12

\threelineshloka
{आनाय्य तु महीपाल ऋष्यशृङ्गं सुसत्कृतम्}
{विभाण्डकसुतं राजन् ब्राह्मणं वेदपारगम्}
{प्रयच्छ कन्यां शान्तां वै विधिना सुसमाहितः} %1-9-13

\twolineshloka
{तेषां तु वचनं श्रुत्वा राजा चिन्तां प्रपत्स्यते}
{केनोपायेन वै शक्यमिहानेतुं स वीर्यवान्} %1-9-14

\twolineshloka
{ततो राजा विनिश्चित्य सह मन्त्रिभिरात्मवान्}
{पुरोहितममात्यांश्च प्रेषयिष्यति सत्कृतान्} %1-9-15

\twolineshloka
{ते तु राज्ञो वचः श्रुत्वा व्यथिता विनताननाः}
{न गच्छेम ऋषेर्भीता अनुनेष्यन्ति तं नृपम्} %1-9-16

\twolineshloka
{वक्ष्यन्ति चिन्तयित्वा ते तस्योपायांश्च तान्क्षमान्}
{आनेष्यामो वयं विप्रं न च दोषो भविष्यति} %1-9-17

\twolineshloka
{एवमङ्गाधिपेनैव गणिकाभिर्ऋषेः सुतः}
{आनीतोऽवर्षयद् देवः शान्ता चास्मै प्रदीयते} %1-9-18

\twolineshloka
{ऋष्यशृङ्गस्तु जामाता पुत्रांस्तव विधास्यति}
{सनत्कुमारकथितमेतावद् व्याहृतं मया} %1-9-19

\twolineshloka
{अथ हृष्टो दशरथः सुमन्त्रं प्रत्यभाषत}
{यथर्ष्यशृङ्गस्त्वानीतो येनोपायेन सोच्यताम्} %1-9-20


॥इत्यार्षे श्रीमद्रामायणे वाल्मीकीये आदिकाव्ये बालकाण्डे ऋष्यशृङ्गोपाख्यानम् नाम नवमः सर्गः ॥१-९॥
