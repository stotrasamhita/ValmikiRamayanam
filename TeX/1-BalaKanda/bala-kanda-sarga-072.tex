\sect{द्विसप्ततितमः सर्गः — गोदानमङ्गलम्}

\twolineshloka
{तमुक्तवन्तं वैदेहं विश्वामित्रो महामुनिः}
{उवाच वचनं वीरं वसिष्ठसहितो नृपम्} %1-72-1

\twolineshloka
{अचिन्त्यान्यप्रमेयाणि कुलानि नरपुङ्गव}
{इक्ष्वाकूणां विदेहानां नैषां तुल्योऽस्ति कश्चन} %1-72-2

\twolineshloka
{सदृशो धर्मसम्बन्धः सदृशो रूपसम्पदा}
{रामलक्ष्मणयो राजन् सीता चोर्मिलया सह} %1-72-3

\twolineshloka
{वक्तव्यं च नरश्रेष्ठ श्रूयतां वचनं मम}
{भ्राता यवीयान् धर्मज्ञ एष राजा कुशध्वजः} %1-72-4

\twolineshloka
{अस्य धर्मात्मनो राजन् रूपेणाप्रतिमं भुवि}
{सुताद्वयं नरश्रेष्ठ पत्न्यर्थं वरयामहे} %1-72-5

\twolineshloka
{भरतस्य कुमारस्य शत्रुघ्नस्य च धीमतः}
{वरये ते सुते राजंस्तयोरर्थे महात्मनोः} %1-72-6

\twolineshloka
{पुत्रा दशरथस्येमे रूपयौवनशालिनः}
{लोकपालसमाः सर्वे देवतुल्यपराक्रमाः} %1-72-7

\twolineshloka
{उभयोरपि राजेन्द्र सम्बन्धेनानुबध्यताम्}
{इक्ष्वाकुकुलमव्यग्रं भवतः पुण्यकर्मणः} %1-72-8

\twolineshloka
{विश्वामित्रवचः श्रुत्वा वसिष्ठस्य मते तदा}
{जनकः प्राञ्जलिर्वाक्यमुवाच मुनिपुङ्गवौ} %1-72-9

\twolineshloka
{कुलं धन्यमिदं मन्ये येषां तौ मुनिपुङ्गवौ}
{सदृशं कुलसम्बन्धं यदाज्ञापयतः स्वयम्} %1-72-10

\twolineshloka
{एवं भवतु भद्रं वः कुशध्वजसुते इमे}
{पत्न्यौ भजेतां सहितौ शत्रुघ्नभरतावुभौ} %1-72-11

\twolineshloka
{एकाह्ना राजपुत्रीणां चतसॄणां महामुने}
{पाणीन् गृह्णन्तु चत्वारो राजपुत्रा महाबलाः} %1-72-12

\twolineshloka
{उत्तरे दिवसे ब्रह्मन् फल्गुनीभ्यां मनीषिणः}
{वैवाहिकं प्रशंसन्ति भगो यत्र प्रजापतिः} %1-72-13

\twolineshloka
{एवमुक्त्वा वचः सौम्यं प्रत्युत्थाय कृताञ्जलिः}
{उभौ मुनिवरौ राजा जनको वाक्यमब्रवीत्} %1-72-14

\twolineshloka
{परो धर्मः कृतो मह्यं शिष्योऽस्मि भवतोस्तथा}
{इमान्यासनमुख्यानि आस्यतां मुनिपुङ्गवौ} %1-72-15

\twolineshloka
{यथा दशरथस्येयं तथायोध्या पुरी मम}
{प्रभुत्वे नास्ति सन्देहो यथार्हं कर्तुमर्हथ} %1-72-16

\twolineshloka
{तथा ब्रुवति वैदेहे जनके रघुनन्दनः}
{राजा दशरथो हृष्टः प्रत्युवाच महीपतिम्} %1-72-17

\twolineshloka
{युवामसङ्ख्येयगुणौ भ्रातरौ मिथिलेश्वरौ}
{ऋषयो राजसङ्घाश्च भवद्भ्यामभिपूजिताः} %1-72-18

\twolineshloka
{स्वस्ति प्राप्नुहि भद्रं ते गमिष्यामः स्वमालयम्}
{श्राद्धकर्माणि विधिवद्विधास्य इति चाब्रवीत्} %1-72-19

\twolineshloka
{तमापृष्ट्वा नरपतिं राजा दशरथस्तदा}
{मुनीन्द्रौ तौ पुरस्कृत्य जगामाशु महायशाः} %1-72-20

\twolineshloka
{स गत्वा निलयं राजा श्राद्धं कृत्वा विधानतः}
{प्रभाते काल्यमुत्थाय चक्रे गोदानमुत्तमम्} %1-72-21

\twolineshloka
{गवां शतसहस्रं च ब्राह्मणेभ्यो नराधिपः}
{एकैकशो ददौ राजा पुत्रानुद्दिश्य धर्मतः} %1-72-22

\twolineshloka
{सुवर्णशृङ्ग्यः सम्पन्नाः सवत्साः कांस्यदोहनाः}
{गवां शतसहस्राणि चत्वारि पुरुषर्षभः} %1-72-23

\twolineshloka
{वित्तमन्यच्च सुबहु द्विजेभ्यो रघुनन्दनः}
{ददौ गोदानमुद्दिश्य पुत्राणां पुत्रवत्सलः} %1-72-24

\twolineshloka
{स सुतैः कृतगोदानैर्वृतः सन्नृपतिस्तदा}
{लोकपालैरिवाभाति वृतः सौम्यः प्रजापतिः} %1-72-25


॥इत्यार्षे श्रीमद्रामायणे वाल्मीकीये आदिकाव्ये बालकाण्डे गोदानमङ्गलम् नाम द्विसप्ततितमः सर्गः ॥१-७२॥
