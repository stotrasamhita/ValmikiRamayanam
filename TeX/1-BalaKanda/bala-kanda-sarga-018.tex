\sect{अष्टादशः सर्गः — श्रीरामाद्यवतारः}

\twolineshloka
{निर्वृत्ते तु क्रतौ तस्मिन् हयमेधे महात्मनः}
{प्रतिगृह्य सुरा भागान् प्रतिजग्मुर्यथागतम्} %1-18-1

\twolineshloka
{समाप्तदीक्षानियमः पत्नीगणसमन्वितः}
{प्रविवेश पुरीं राजा सभृत्यबलवाहनः} %1-18-2

\twolineshloka
{यथार्हं पूजितास्तेन राज्ञा वै पृथिवीश्वराः}
{मुदिताः प्रययुर्देशान् प्रणम्य मुनिपुंगवम्} %1-18-3

\twolineshloka
{श्रीमतां गच्छतां तेषां स्वगृहाणि पुरात् ततः}
{बलानि राज्ञां शुभ्राणि प्रहृष्टानि चकाशिरे} %1-18-4

\twolineshloka
{गतेषु पृथिवीशेषु राजा दशरथः पुनः}
{प्रविवेश पुरीं श्रीमान् पुरस्कृत्य द्विजोत्तमान्} %1-18-5

\twolineshloka
{शान्तया प्रययौ सार्धमृष्यशृङ्गः सुपूजितः}
{अन्वीयमानो राज्ञाथ सानुयात्रेण धीमता} %1-18-6

\twolineshloka
{एवं विसृज्य तान् सर्वान् राजा सम्पूर्णमानसः}
{उवास सुखितस्तत्र पुत्रोत्पत्तिं विचिन्तयन्} %1-18-7

\twolineshloka
{ततो यज्ञे समाप्ते तु ऋतूनां षट् समत्ययुः}
{ततश्च द्वादशे मासे चैत्रे नावमिके तिथौ} %1-18-8

\twolineshloka
{नक्षत्रेऽदितिदैवत्ये स्वोच्चसंस्थेषु पञ्चसु}
{ग्रहेषु कर्कटे लग्ने वाक्पताविन्दुना सह} %1-18-9

\twolineshloka
{प्रोद्यमाने जगन्नाथं सर्वलोकनमस्कृतम्}
{कौसल्याजनयद् रामं दिव्यलक्षणसंयुतम्} %1-18-10

\twolineshloka
{विष्णोरर्धं महाभागं पुत्रमैक्ष्वाकुनन्दनम्}
{लोहिताक्षं महाबाहुं रक्तोष्ठं दुन्दुभिस्वनम्} %1-18-11

\twolineshloka
{कौसल्या शुशुभे तेन पुत्रेणामिततेजसा}
{यथा वरेण देवानामदितिर्वज्रपाणिना} %1-18-12

\twolineshloka
{भरतो नाम कैकेय्यां जज्ञे सत्यपराक्रमः}
{साक्षाद् विष्णोश्चतुर्भागः सर्वैः समुदितो गुणैः} %1-18-13

\twolineshloka
{अथ लक्ष्मणशत्रुघ्नौ सुमित्राजनयत् सुतौ}
{वीरौ सर्वास्त्रकुशलौ विष्णोरर्धसमन्वितौ} %1-18-14

\twolineshloka
{पुष्ये जातस्तु भरतो मीनलग्ने प्रसन्नधीः}
{सार्पे जातौ तु सौमित्री कुलीरेऽभ्युदिते रवौ} %1-18-15

\twolineshloka
{राज्ञः पुत्रा महात्मानश्चत्वारो जज्ञिरे पृथक्}
{गुणवन्तोऽनुरूपाश्च रुच्या प्रोष्ठपदोपमाः} %1-18-16

\twolineshloka
{जगुः कलं च गन्धर्वा ननृतुश्चाप्सरोगणाः}
{देवदुन्दुभयो नेदुः पुष्पवृष्टिश्च खात् पतत्} %1-18-17

\twolineshloka
{उत्सवश्च महानासीदयोध्यायां जनाकुलः}
{रथ्याश्च जनसम्बाधा नटनर्तकसंकुलाः} %1-18-18

\twolineshloka
{गायनैश्च विराविण्यो वादनैश्च तथापरैः}
{विरेजुर्विपुलास्तत्र सर्वरत्नसमन्विताः} %1-18-19

\twolineshloka
{प्रदेयांश्च ददौ राजा सूतमागधवन्दिनाम्}
{ब्राह्मणेभ्यो ददौ वित्तम् गोधनानि सहस्रशः} %1-18-20

\twolineshloka
{अतीत्यैकादशाहं तु नामकर्म तथाकरोत्}
{ज्येष्ठं रामं महात्मानं भरतं कैकयीसुतम्} %1-18-21

\twolineshloka
{सौमित्रिं लक्ष्मणमिति शत्रुघ्नमपरं तथा}
{वसिष्ठः परमप्रीतो नामानि कुरुते तदा} %1-18-22

\twolineshloka
{ब्राह्मणान् भोजयामास पौराजानपदानपि}
{अददद् ब्राह्मणानां च रत्नौघममलं बहु} %1-18-23

\twolineshloka
{तेषां जन्मक्रियादीनि सर्वकर्माण्यकारयत्}
{तेषां केतुरिव ज्येष्ठो रामो रतिकरः पितुः} %1-18-24

\twolineshloka
{बभूव भूयो भूतानां स्वयम्भूरिव सम्मतः}
{सर्वे वेदविदः शूराः सर्वे लोकहिते रताः} %1-18-25

\twolineshloka
{सर्वे ज्ञानोपसम्पन्नाः सर्वे समुदिता गुणैः}
{तेषामपि महातेजा रामः सत्यपराक्रमः} %1-18-26

\twolineshloka
{इष्टः सर्वस्य लोकस्य शशाङ्क इव निर्मलः}
{गजस्कन्धेऽश्वपृष्ठे च रथचर्यासु सम्मतः} %1-18-27

\twolineshloka
{धनुर्वेदे च निरतः पितुः शुश्रूषणे रतः}
{बाल्यात् प्रभृति सुस्निग्धो लक्ष्मणो लक्ष्मिवर्धनः} %1-18-28

\twolineshloka
{रामस्य लोकरामस्य भ्रातुर्ज्येष्ठस्य नित्यशः}
{सर्वप्रियकरस्तस्य रामस्यापि शरीरतः} %1-18-29

\twolineshloka
{लक्ष्मणो लक्ष्मिसम्पन्नो बहिःप्राण इवापरः}
{न च तेन विना निद्रां लभते पुरुषोत्तमः} %1-18-30

\twolineshloka
{मृष्टमन्नमुपानीतमश्नाति न हि तं विना}
{यदा हि हयमारूढो मृगयां याति राघवः} %1-18-31

\twolineshloka
{अथैनं पृष्ठतोऽभ्येति सधनुः परिपालयन्}
{भरतस्यापि शत्रुघ्नो लक्ष्मणावरजो हि सः} %1-18-32

\twolineshloka
{प्राणैः प्रियतरो नित्यं तस्य चासीत् तथा प्रियः}
{स चतुर्भिर्महाभागैः पुत्रैर्दशरथः प्रियैः} %1-18-33

\twolineshloka
{बभूव परमप्रीतो देवैरिव पितामहः}
{ते यदा ज्ञानसंपन्नाः सर्वे समुदिता गुणैः} %1-18-34

\twolineshloka
{ह्रीमन्तः कीर्तिमन्तश्च सर्वज्ञा दीर्घदर्शिनः}
{तेषामेवंप्रभावाणां सर्वेषां दीप्ततेजसाम्} %1-18-35

\twolineshloka
{पिता दशरथो हृष्टो ब्रह्मा लोकाधिपो यथा}
{ते चापि मनुजव्याघ्रा वैदिकाध्ययने रताः} %1-18-36

\twolineshloka
{पितृशुश्रूषणरता धनुर्वेदे च निष्ठिताः}
{अथ राजा दशरथस्तेषां दारक्रियां प्रति} %1-18-37

\twolineshloka
{चिन्तयामास धर्मात्मा सोपाध्यायः सबान्धवः}
{तस्य चिन्तयमानस्य मन्त्रिमध्ये महात्मनः} %1-18-38

\twolineshloka
{अभ्यागच्छन्महातेजा विश्वामित्रो महामुनिः}
{स राज्ञो दर्शनाकाङ्क्षी द्वाराध्यक्षानुवाच ह} %1-18-39

\twolineshloka
{शीघ्रमाख्यात मां प्राप्तं कौशिकं गाधिनः सुतम्}
{तच्छ्रुत्वा वचनं तस्य राज्ञो वेश्म प्रदुद्रुवुः} %1-18-40

\twolineshloka
{संभ्रान्तमनसः सर्वे तेन वाक्येन चोदिताः}
{ते गत्वा राजभवनं विश्वामित्रमृषिं तदा} %1-18-41

\twolineshloka
{प्राप्तमावेदयामासुर्नृपायेक्ष्वाकवे तदा}
{तेषां तद्वचनं श्रुत्वा सपुरोधाः समाहितः} %1-18-42

\twolineshloka
{प्रत्युज्जगाम संहृष्टो ब्रह्माणमिव वासवः}
{स दृष्ट्वा ज्वलितं दीप्त्या तापसं संशितव्रतम्} %1-18-43

\twolineshloka
{प्रहृष्टवदनो राजा ततोऽर्घ्यमुपहारयत्}
{स राज्ञः प्रतिगृह्यार्घ्यं शास्त्रदृष्टेन कर्मणा} %1-18-44

\twolineshloka
{कुशलं चाव्ययं चैव पर्यपृच्छन्नराधिपम्}
{पुरे कोशे जनपदे बान्धवेषु सुहृत्सु च} %1-18-45

\twolineshloka
{कुशलं कौशिको राज्ञः पर्यपृच्छत् सुधार्मिकः}
{अपि ते संनताः सर्वे सामन्तरिपवो जिताः} %1-18-46

\twolineshloka
{दैवं च मानुषं चैव कर्म ते साध्वनुष्ठितम्}
{वसिष्ठं च समागम्य कुशलं मुनिपुङ्गवः} %1-18-47

\twolineshloka
{ऋषींश्च तान् यथान्यायं महाभाग उवाच ह}
{ते सर्वे हृष्टमनसस्तस्य राज्ञो निवेशनम्} %1-18-48

\twolineshloka
{विविशुः पूजितास्तेन निषेदुश्च यथार्हतः}
{अथ हृष्टमना राजा विश्वामित्रं महामुनिम्} %1-18-49

\twolineshloka
{उवाच परमोदारो हृष्टस्तमभिपूजयन्}
{यथामृतस्य सम्प्राप्तिर्यथा वर्षमनूदके} %1-18-50

\twolineshloka
{यथा सदृशदारेषु पुत्रजन्माप्रजस्य वै}
{प्रणष्टस्य यथा लाभो यथा हर्षो महोदये} %1-18-51

\twolineshloka
{तथैवागमनं मन्ये स्वागतं ते महामुने}
{कं च ते परमं कामं करोमि किमु हर्षितः} %1-18-52

\twolineshloka
{पात्रभूतोऽसि मे ब्रह्मन् दिष्ट्या प्राप्तोऽसि मानद}
{अद्य मे सफलं जन्म जीवितं च सुजीवितम्} %1-18-53

\twolineshloka
{यस्माद् विप्रेन्द्रमद्राक्षं सुप्रभाता निशा मम}
{पूर्वं राजर्षिशब्देन तपसा द्योतितप्रभः} %1-18-54

\twolineshloka
{ब्रह्मर्षित्वमनुप्राप्तः पूज्योऽसि बहुधा मया}
{तदद्भुतमभूद् विप्र पवित्रं परमं मम} %1-18-55

\twolineshloka
{शुभक्षेत्रगतश्चाहं तव संदर्शनात् प्रभो}
{ब्रूहि यत् प्रार्थितं तुभ्यं कार्यमागमनं प्रति} %1-18-56

\twolineshloka
{इच्छाम्यनुगृहीतोऽहं त्वदर्थं परिवृद्धये}
{कार्यस्य न विमर्शं च गन्तुमर्हसि सुव्रत} %1-18-57

\threelineshloka
{कर्ता चाहमशेषेण दैवतं हि भवान् मम}
{मम चायमनुप्राप्तो महानभ्युदयो द्विज}
{तवागमनजः कृत्स्नो धर्मश्चानुत्तमो द्विज} %1-18-58

\twolineshloka
{इति हृदयसुखं निशम्य वाक्यं श्रुतिसुखमात्मवता विनीतमुक्तम्}
{प्रथितगुणयशा गुणैर्विशिष्टः परमऋषिः परमं जगाम हर्षम्} %1-18-59


॥इत्यार्षे श्रीमद्रामायणे वाल्मीकीये आदिकाव्ये बालकाण्डे श्रीरामाद्यवतारः नाम अष्टादशः सर्गः ॥१-१८॥
