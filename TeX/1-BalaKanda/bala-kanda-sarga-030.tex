\sect{त्रिंशः सर्गः — यज्ञरक्षणम्}

\twolineshloka
{अथ तौ देशकालज्ञौ राजपुत्रावरिन्दमौ}
{देशे काले च वाक्यज्ञावब्रूतां कौशिकं वचः} %1-30-1

\twolineshloka
{भगवञ्छ्रोतुमिच्छावो यस्मिन् काले निशाचरौ}
{संरक्षणीयौ तौ ब्रूहि नातिवर्तेत तत्क्षणम्} %1-30-2

\twolineshloka
{एवं ब्रुवाणौ काकुत्स्थौ त्वरमाणौ युयुत्सया}
{सर्वे ते मुनयः प्रीताः प्रशशंसुर्नृपात्मजौ} %1-30-3

\twolineshloka
{अद्यप्रभृति षड्रात्रं रक्षतां राघवौ युवाम्}
{दीक्षां गतो ह्येष मुनिर्मौनित्वं च गमिष्यति} %1-30-4

\twolineshloka
{तौ तु तद्वचनं श्रुत्वा राजपुत्रौ यशस्विनौ}
{अनिद्रं षडहोरात्रं तपोवनमरक्षताम्} %1-30-5

\twolineshloka
{उपासाञ्चक्रतुर्वीरौ यत्तौ परमधन्विनौ}
{ररक्षतुर्मुनिवरं विश्वामित्रमरिन्दमौ} %1-30-6

\twolineshloka
{अथ काले गते तस्मिन् षष्ठेऽहनि तदागते}
{सौमित्रिमब्रवीद् रामो यत्तो भव समाहितः} %1-30-7

\twolineshloka
{रामस्यैवं ब्रुवाणस्य त्वरितस्य युयुत्सया}
{प्रजज्वाल ततो वेदिः सोपाध्यायपुरोहिता} %1-30-8

\twolineshloka
{सदर्भचमसस्रुक्का स समित्कुसुमोच्चया}
{विश्वामित्रेण सहिता वेदिर्जज्वाल सर्त्विजा} %1-30-9

\twolineshloka
{मन्त्रवच्च यथान्यायं यज्ञोऽसौ सम्प्रवर्तते}
{आकाशे च महाञ्छ्ब्दः प्रादुरासीद् भयानकः} %1-30-10

\twolineshloka
{आवार्य गगनं मेघो यथा प्रावृषि दृश्यते}
{तथा मायां विकुर्वाणौ राक्षसावभ्यधावताम्} %1-30-11

\twolineshloka
{मारीचश्च सुबाहुश्च तयोरनुचरास्तथा}
{आगम्य भीमसङ्काशा रुधिरौघानवासृजन्} %1-30-12

\twolineshloka
{ताम् तेन रुधिरौघेण वेदीं वीक्ष्य समुक्षिताम्}
{सहसाभिद्रुतो रामस्तानपश्यत् ततो दिवि} %1-30-13

\twolineshloka
{तावापतन्तौ सहसा दृष्ट्वा राजीवलोचनः}
{लक्ष्मणं त्वभिसम्प्रेक्ष्य रामो वचनमब्रवीत्} %1-30-14

\twolineshloka
{पश्य लक्ष्मण दुर्वृत्तान् राक्षसान् पिशिताशनान्}
{मानवास्त्रसमाधूताननिलेन यथा घनान्} %1-30-15

\twolineshloka
{करिष्यामि न सन्देहो नोत्सहे हन्तुमीदृशान्}
{इत्युक्त्वा वचनं रामश्चापे सन्धाय वेगवान्} %1-30-16

\twolineshloka
{मानवं परमोदारमस्त्रं परमभास्वरम्}
{चिक्षेप परमक्रुद्धो मारीचोरसि राघवः} %1-30-17

\twolineshloka
{स तेन परमास्त्रेण मानवेन समाहतः}
{सम्पूर्णं योजनशतं क्षिप्तः सागरसम्प्लवे} %1-30-18

\twolineshloka
{विचेतनं विघूर्णन्तं शीतेषुबलपीडितम्}
{निरस्तं दृश्य मारीचं रामो लक्ष्मणमब्रवीत्} %1-30-19

\twolineshloka
{पश्य लक्ष्मण शीतेषुं मानवं मनुसंहितम्}
{मोहयित्वा नयत्येनं न च प्राणैर्वियुज्यते} %1-30-20

\twolineshloka
{इमानपि वधिष्यामि निर्घृणान् दुष्टचारिणः}
{राक्षसान् पापकर्मस्थान् यज्ञघ्नान् रुधिराशनान्} %1-30-21

\twolineshloka
{इत्युक्त्वा लक्ष्मणं चाशु लाघवं दर्शयन्निव}
{विगृह्य सुमहच्चास्त्रमाग्नेयं रघुनन्दनः} %1-30-22

\threelineshloka
{सुबाहूरसि चिक्षेप स विद्धः प्रापतद् भुवि}
{शेषान् वायव्यमादाय निजघान महायशाः}
{राघवः परमोदारो मुनीनां मुदमावहन्} %1-30-23

\twolineshloka
{स हत्वा राक्षसान् सर्वान् यज्ञघ्नान् रघुनन्दनः}
{ऋषिभिः पूजितस्तत्र यथेन्द्रो विजये पुरा} %1-30-24

\twolineshloka
{अथ यज्ञे समाप्ते तु विश्वामित्रो महामुनिः}
{निरीतिका दिशो दृष्ट्वा काकुत्स्थमिदमब्रवीत्} %1-30-25

\threelineshloka
{कृतार्थोऽस्मि महाबाहो कृतं गुरुवचस्त्वया}
{सिद्धाश्रममिदं सत्यं कृतं वीर महायशः}
{स हि रामं प्रशस्यैवं ताभ्यां सन्ध्यामुपागमत्} %1-30-26


॥इत्यार्षे श्रीमद्रामायणे वाल्मीकीये आदिकाव्ये बालकाण्डे यज्ञरक्षणम् नाम त्रिंशः सर्गः ॥१-३०॥
