\sect{षट्पञ्चाशः सर्गः — ब्रह्मतेजोबलम्}

\twolineshloka
{एवमुक्तो वसिष्ठेन विश्वामित्रो महाबलः}
{आग्नेयमस्त्रमुद्दिश्य तिष्ठ तिष्ठेति चाब्रवीत्} %1-56-1

\twolineshloka
{ब्रह्मदण्डं समुद्यम्य कालदण्डमिवापरम्}
{वसिष्ठो भगवान् क्रोधादिदं वचनमब्रवीत्} %1-56-2

\twolineshloka
{क्षत्रबन्धो स्थितोऽस्म्येष यद् बलं तद् विदर्शय}
{नाशयाम्यद्य ते दर्पं शस्त्रस्य तव गाधिज} %1-56-3

\twolineshloka
{क्व च ते क्षत्रियबलं क्व च ब्रह्मबलं महत्}
{पश्य ब्रह्मबलं दिव्यं मम क्षत्रियपांसन} %1-56-4

\twolineshloka
{तस्यास्त्रं गाधिपुत्रस्य घोरमाग्नेयमुत्तमम्}
{ब्रह्मदण्डेन तच्छान्तमग्नेर्वेग इवाम्भसा} %1-56-5

\twolineshloka
{वारुणं चैव रौद्रं च ऐन्द्रं पाशुपतं तथा}
{ऐषीकं चापि चिक्षेप कुपितो गाधिनन्दनः} %1-56-6

\twolineshloka
{मानवं मोहनं चैव गान्धर्वं स्वापनं तथा}
{जृम्भणं मादनं चैव सन्तापनविलापने} %1-56-7

\twolineshloka
{शोषणं दारणं चैव वज्रमस्त्रं सुदुर्जयम्}
{ब्रह्मपाशं कालपाशं वारुणं पाशमेव च} %1-56-8

\twolineshloka
{पिनाकमस्त्रं दयितं शुष्कार्द्रे अशनी तथा}
{दण्डास्त्रमथ पैशाचं क्रौञ्चमस्त्रं तथैव च} %1-56-9

\twolineshloka
{धर्मचक्रं कालचक्रं विष्णुचक्रं तथैव च}
{वायव्यं मथनं चैव अस्त्रं हयशिरस्तथा} %1-56-10

\twolineshloka
{शक्तिद्वयं च चिक्षेप कङ्कालं मुसलं तथा}
{वैद्याधरं महास्त्रं च कालास्त्रमथ दारुणम्} %1-56-11

\twolineshloka
{त्रिशूलमस्त्रं घोरं च कापालमथ कङ्कणम्}
{एतान्यस्त्राणि चिक्षेप सर्वाणि रघुनन्दन} %1-56-12

\twolineshloka
{वसिष्ठे जपतां श्रेष्ठे तदद्भुतमिवाभवत्}
{तानि सर्वाणि दण्डेन ग्रसते ब्रह्मणः सुतः} %1-56-13

\twolineshloka
{तेषु शान्तेषु ब्रह्मास्त्रं क्षिप्तवान् गाधिनन्दनः}
{तदस्त्रमुद्यतं दृष्ट्वा देवाः साग्निपुरोगमाः} %1-56-14

\twolineshloka
{देवर्षयश्च सम्भ्रान्ता गन्धर्वाः समहोरगाः}
{त्रैलोक्यमासीत् सन्त्रस्तं ब्रह्मास्त्रे समुदीरिते} %1-56-15

\twolineshloka
{तदप्यस्त्रं महाघोरं ब्राह्मं ब्राह्मेण तेजसा}
{वसिष्ठो ग्रसते सर्वं ब्रह्मदण्डेन राघव} %1-56-16

\twolineshloka
{ब्रह्मास्त्रं ग्रसमानस्य वसिष्ठस्य महात्मनः}
{त्रैलोक्यमोहनं रौद्रं रूपमासीत् सुदारुणम्} %1-56-17

\twolineshloka
{रोमकूपेषु सर्वेषु वसिष्ठस्य महात्मनः}
{मरीच्य इव निष्पेतुरग्नेर्धूमाकुलार्चिषः} %1-56-18

\twolineshloka
{प्राज्वलद् ब्रह्मदण्डश्च वसिष्ठस्य करोद्यतः}
{विधूम इव कालाग्नेर्यमदण्ड इवापरः} %1-56-19

\twolineshloka
{ततोऽस्तुवन् मुनिगणा वसिष्ठं जपतां वरम्}
{अमोघं ते बलं ब्रह्मंस्तेजो धारय तेजसा} %1-56-20

\twolineshloka
{निगृहीतस्त्वया ब्रह्मन् विश्वामित्रो महाबलः}
{अमोघं ते बलं श्रेष्ठ लोकाः सन्तु गतव्यथाः} %1-56-21

\twolineshloka
{एवमुक्तो महातेजाः शमं चक्रे महाबलः}
{विश्वामित्रो विनिकृतो विनिःश्वस्येदमब्रवीत्} %1-56-22

\twolineshloka
{धिग् बलं क्षत्रियबलं ब्रह्मतेजोबलं बलम्}
{एकेन ब्रह्मदण्डेन सर्वास्त्राणि हतानि मे} %1-56-23

\twolineshloka
{तदेतत् प्रसमीक्ष्याहं प्रसन्नेन्द्रियमानसः}
{तपो महत् समास्थास्ये यद् वै ब्रह्मत्वकारणम्} %1-56-24


॥इत्यार्षे श्रीमद्रामायणे वाल्मीकीये आदिकाव्ये बालकाण्डे ब्रह्मतेजोबलम् नाम षट्पञ्चाशः सर्गः ॥१-५६॥
