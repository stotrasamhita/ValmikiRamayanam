\sect{पञ्चपञ्चाशः सर्गः — विश्वामित्रधनुर्वेदाधिगमः}

\twolineshloka
{ततस्तानाकुलान् दृष्ट्वा विश्वामित्रास्त्रमोहितान्}
{वसिष्ठश्चोदयामास कामधुक् सृज योगतः} %1-55-1

\twolineshloka
{तस्या हुंकारतो जाताः काम्बोजा रविसंनिभाः}
{ऊधसश्चाथ सम्भूता बर्बराः शस्त्रपाणयः} %1-55-2

\twolineshloka
{योनिदेशाच्च यवनाः शकृद्देशाच्छकाः स्मृताः}
{रोमकूपेषु म्लेच्छाश्च हारीताः सकिरातकाः} %1-55-3

\twolineshloka
{तैस्तन्निषूदितं सर्वं विश्वामित्रस्य तत्क्षणात्}
{सपदातिगजं साश्वं सरथं रघुनन्दन} %1-55-4

\twolineshloka
{दृष्ट्वा निषूदितं सैन्यं वसिष्ठेन महात्मना}
{विश्वामित्रसुतानां तु शतं नानाविधायुधम्} %1-55-5

\twolineshloka
{अभ्यधावत् सुसंक्रुद्धं वसिष्ठं जपतां वरम्}
{हुंकारेणैव तान् सर्वान् निर्ददाह महानृषिः} %1-55-6

\twolineshloka
{ते साश्वरथपादाता वसिष्ठेन महात्मना}
{भस्मीकृता मुहूर्तेन विश्वामित्रसुतास्तथा} %1-55-7

\twolineshloka
{दृष्ट्वा विनाशितान् सर्वान् बलं च सुमहायशाः}
{सव्रीडं चिन्तयाविष्टो विश्वामित्रोऽभवत् तदा} %1-55-8

\twolineshloka
{समुद्र इव निर्वेगो भग्नद्रंष्ट्र इवोरगः}
{उपरक्त इवादित्यः सद्यो निष्प्रभतां गतः} %1-55-9

\twolineshloka
{हतपुत्रबलो दीनो लूनपक्ष इव द्विजः}
{हतसर्वबलोत्साहो निर्वेदं समपद्यत} %1-55-10

\twolineshloka
{स पुत्रमेकं राज्याय पालयेति नियुज्य च}
{पृथिवीं क्षत्रधर्मेण वनमेवाभ्यपद्यत} %1-55-11

\twolineshloka
{स गत्वा हिमवत्पार्श्वे किंनरोरगसेवितम्}
{महादेवप्रसादार्थं तपस्तेपे महातपाः} %1-55-12

\twolineshloka
{केनचित् त्वथ कालेन देवेशो वृषभध्वजः}
{दर्शयामास वरदो विश्वामित्रं महामुनिम्} %1-55-13

\twolineshloka
{किमर्थं तप्यसे राजन् ब्रूहि यत् ते विवक्षितम्}
{वरदोऽस्मि वरो यस्ते कांक्षितः सोऽभिधीयताम्} %1-55-14

\twolineshloka
{एवमुक्तस्तु देवेन विश्वामित्रो महातपाः}
{प्रणिपत्य महादेवं विश्वामित्रोऽब्रवीदिदम्} %1-55-15

\twolineshloka
{यदि तुष्टो महादेव धनुर्वेदो ममानघ}
{सांगोपांगोपनिषदः सरहस्यः प्रदीयताम्} %1-55-16

\twolineshloka
{यानि देवेषु चास्त्राणि दानवेषु महर्षिषु}
{गन्धर्वयक्षरक्षःसु प्रतिभान्तु ममानघ} %1-55-17

\twolineshloka
{तव प्रसादाद् भवतु देवदेव ममेप्सितम्}
{एवमस्त्विति देवेशो वाक्यमुक्त्वा गतस्तदा} %1-55-18

\twolineshloka
{प्राप्य चास्त्राणि देवेशाद् विश्वामित्रो महाबलः}
{दर्पेण महता युक्तो दर्पपूर्णोऽभवत् तदा} %1-55-19

\twolineshloka
{विवर्धमानो वीर्येण समुद्र इव पर्वणि}
{हतं मेने तदा राम वसिष्ठमृषिसत्तमम्} %1-55-20

\twolineshloka
{ततो गत्वाऽऽश्रमपदं मुमोचास्त्राणि पार्थिवः}
{यैस्तत् तपोवनं नाम निर्दग्धं चास्त्रतेजसा} %1-55-21

\twolineshloka
{उदीर्यमाणमस्त्रं तद् विश्वामित्रस्य धीमतः}
{दृष्ट्वा विप्रद्रुता भीता मुनयः शतशो दिशः} %1-55-22

\twolineshloka
{वसिष्ठस्य च ये शिष्या ये च वै मृगपक्षिणः}
{विद्रवन्ति भयाद् भीता नानादिग्भ्यः सहस्रशः} %1-55-23

\twolineshloka
{वसिष्ठस्याश्रमपदं शून्यमासीन्महात्मनः}
{मुहूर्तमिव निःशब्दमासीदीरिणसंनिभम्} %1-55-24

\twolineshloka
{वदतो वै वसिष्ठस्य मा भैरिति मुहुर्मुहुः}
{नाशयाम्यद्य गाधेयं नीहारमिव भास्करः} %1-55-25

\twolineshloka
{एवमुक्त्वा महातेजा वसिष्ठो जपतां वरः}
{विश्वामित्रं तदा वाक्यं सरोषमिदमब्रवीत्} %1-55-26

\twolineshloka
{आश्रमं चिरसंवृद्धं यद् विनाशितवानसि}
{दुराचारो हि यन्मूढस्तस्मात् त्वं न भविष्यसि} %1-55-27

\twolineshloka
{इत्युक्त्वा परमक्रुद्धो दण्डमुद्यम्य सत्वरः}
{विधूम इव कालाग्निर्यमदण्डमिवापरम्} %1-55-28


॥इत्यार्षे श्रीमद्रामायणे वाल्मीकीये आदिकाव्ये बालकाण्डे विश्वामित्रधनुर्वेदाधिगमः नाम पञ्चपञ्चाशः सर्गः ॥१-५५॥
