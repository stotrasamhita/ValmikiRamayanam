\sect{पञ्चसप्ततितमः सर्गः — वैष्णवधनुःप्रशंसा}

\twolineshloka
{राम दाशरथे वीर वीर्यं ते श्रूयतेऽद्भुतम्}
{धनुषो भेदनं चैव निखिलेन मया श्रुतम्} %1-75-1

\twolineshloka
{तदद्भुतमचिन्त्यं च भेदनं धनुषस्तथा}
{तच्छ्रुत्वाहमनुप्राप्तो धनुर्गृह्यापरं शुभम्} %1-75-2

\twolineshloka
{तदिदं घोरसंकाशं जामदग्न्यं महद्धनुः}
{पूरयस्व शरेणैव स्वबलं दर्शयस्व च} %1-75-3

\twolineshloka
{तदहं ते बलं दृष्ट्वा धनुषोऽप्यस्य पूरणे}
{द्वन्द्वयुद्धं प्रदास्यामि वीर्यश्लाघ्यमहं तव} %1-75-4

\twolineshloka
{तस्य तद् वचनं श्रुत्वा राजा दशरथस्तदा}
{विषण्णवदनो दीनः प्राञ्जलिर्वाक्यमब्रवीत्} %1-75-5

\twolineshloka
{क्षत्ररोषात् प्रशान्तस्त्वं ब्राह्मणश्च महातपाः}
{बालानां मम पुत्राणामभयं दातुमर्हसि} %1-75-6

\twolineshloka
{भार्गवाणां कुले जातः स्वाध्यायव्रतशालिनाम्}
{सहस्राक्षे प्रतिज्ञाय शस्त्रं प्रक्षिप्तवानसि} %1-75-7

\twolineshloka
{स त्वं धर्मपरो भूत्वा कश्यपाय वसुंधराम्}
{दत्त्वा वनमुपागम्य महेन्द्रकृतकेतनः} %1-75-8

\twolineshloka
{मम सर्वविनाशाय सम्प्राप्तस्त्वं महामुने}
{न चैकस्मिन् हते रामे सर्वे जीवामहे वयम्} %1-75-9

\twolineshloka
{ब्रुवत्येवं दशरथे जामदग्न्यः प्रतापवान्}
{अनादृत्य तु तद्वाक्यं राममेवाभ्यभाषत} %1-75-10

\twolineshloka
{इमे द्वे धनुषी श्रेष्ठे दिव्ये लोकाभिपूजिते}
{दृढे बलवती मुख्ये सुकृते विश्वकर्मणा} %1-75-11

\twolineshloka
{अनुसृष्टं सुरैरेकं त्र्यम्बकाय युयुत्सवे}
{त्रिपुरघ्नं नरश्रेष्ठ भग्नं काकुत्स्थ यत्त्वया} %1-75-12

\twolineshloka
{इदं द्वितीयं दुर्धर्षं विष्णोर्दत्तं सुरोत्तमैः}
{तदिदं वैष्णवं राम धनुः परपुरंजयम्} %1-75-13

\twolineshloka
{समानसारं काकुत्स्थ रौद्रेण धनुषा त्विदम्}
{तदा तु देवताः सर्वाः पृच्छन्ति स्म पितामहम्} %1-75-14

\twolineshloka
{शितिकण्ठस्य विष्णोश्च बलाबलनिरीक्षया}
{अभिप्रायं तु विज्ञाय देवतानां पितामहः} %1-75-15

\twolineshloka
{विरोधं जनयामास तयोः सत्यवतां वरः}
{विरोधे तु महद् युद्धमभवद् रोमहर्षणम्} %1-75-16

\twolineshloka
{शितिकण्ठस्य विष्णोश्च परस्परजयैषिणोः}
{तदा तु जृम्भितं शैवं धनुर्भीमपराक्रमम्} %1-75-17

\twolineshloka
{हुंकारेण महादेवः स्तम्भितोऽथ त्रिलोचनः}
{देवैस्तदा समागम्य सर्षिसङ्घः सचारणैः} %1-75-18

\twolineshloka
{याचितौ प्रशमं तत्र जग्मतुस्तौ सुरोत्तमौ}
{जृम्भितं तद् धनुर्दृष्ट्वा शैवं विष्णुपराक्रमैः} %1-75-19

\twolineshloka
{अधिकं मेनिरे विष्णुं देवाः सर्षिगणास्तथा}
{धनू रुद्रस्तु संक्रुद्धो विदेहेषु महायशाः} %1-75-20

\twolineshloka
{देवरातस्य राजर्षेर्ददौ हस्ते ससायकम्}
{इदं च वैष्णवं राम धनुः परपुरंजयम्} %1-75-21

\twolineshloka
{ऋचीके भार्गवे प्रादाद् विष्णुः स न्यासमुत्तमम्}
{ऋचीकस्तु महातेजाः पुत्रस्याप्रतिकर्मणः} %1-75-22

\twolineshloka
{पितुर्मम ददौ दिव्यं जमदग्नेर्महात्मनः}
{न्यस्तशस्त्रे पितरि मे तपोबलसमन्विते} %1-75-23

\threelineshloka
{अर्जुनो विदधे मृत्युं प्राकृतां बुद्धिमास्थितः}
{वधमप्रतिरूपं तु पितुः श्रुत्वा सुदारुणम्}
{क्षत्रमुत्सादयं रोषाज्जातं जातमनेकशः} %1-75-24

\twolineshloka
{पृथिवीं चाखिलां प्राप्य कश्यपाय महात्मने}
{यज्ञस्यान्तेऽददं राम दक्षिणां पुण्यकर्मणे} %1-75-25

\twolineshloka
{दत्त्वा महेन्द्रनिलयस्तपोबलसमन्वितः}
{श्रुत्वा तु धनुषो भेदं ततोऽहं द्रुतमागतः} %1-75-26

\twolineshloka
{तदेवं वैष्णवं राम पितृपैतामहं महत्}
{क्षत्रधर्मं पुरस्कृत्य गृह्णीष्व धनुरुत्तमम्} %1-75-27

\twolineshloka
{योजयस्व धनुःश्रेष्ठे शरं परपुरंजयम्}
{यदि शक्तोऽसि काकुत्स्थ द्वन्द्वं दास्यामि ते ततः} %1-75-28


॥इत्यार्षे श्रीमद्रामायणे वाल्मीकीये आदिकाव्ये बालकाण्डे वैष्णवधनुःप्रशंसा नाम पञ्चसप्ततितमः सर्गः ॥१-७५॥
