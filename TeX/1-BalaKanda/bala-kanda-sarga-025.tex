\sect{पञ्चविंशः सर्गः — ताटकावृत्तान्तः}

\twolineshloka
{अथ तस्याप्रमेयस्य मुनेर्वचनमुत्तमम्}
{श्रुत्वा पुरुषशार्दूलः प्रत्युवाच शुभां गिरम्} %1-25-1

\twolineshloka
{अल्पवीर्या यदा यक्षी श्रूयते मुनिपुङ्गव}
{कथं नागसहस्रस्य धारयत्यबला बलम्} %1-25-2

\twolineshloka
{इत्युक्तं वचनं श्रुत्वा राघवस्यामितौजसः}
{हर्षयन् श्लक्ष्णया वाचा सलक्ष्मणमरिन्दमम्} %1-25-3

\twolineshloka
{विश्वामित्रोऽब्रवीद् वाक्यं शृणु येन बलोत्कटा}
{वरदानकृतं वीर्यं धारयत्यबला बलम्} %1-25-4

\twolineshloka
{पूर्वमासीन्महायक्षः सुकेतुर्नाम वीर्यवान्}
{अनपत्यः शुभाचारः स च तेपे महत्तपः} %1-25-5

\twolineshloka
{पितामहस्तु सुप्रीतस्तस्य यक्षपतेस्तदा}
{कन्यारत्नं ददौ राम ताटकां नाम नामतः} %1-25-6

\twolineshloka
{ददौ नागसहस्रस्य बलं चास्याः पितामहः}
{न त्वेव पुत्रं यक्षाय ददौ चासौ महायशाः} %1-25-7

\twolineshloka
{तां तु बालां विवर्धन्तीं रूपयौवनशालिनीम्}
{जम्भपुत्राय सुन्दाय ददौ भार्यां यशस्विनीम्} %1-25-8

\twolineshloka
{कस्यचित्त्वथ कालस्य यक्षी पुत्रं व्यजायत}
{मारीचं नाम दुर्धर्षं यः शापाद् राक्षसोऽभवत्} %1-25-9

\twolineshloka
{सुन्दे तु निहते राम अगस्त्यमृषिसत्तमम्}
{ताटका सहपुत्रेण प्रधर्षयितुमिच्छति} %1-25-10

\twolineshloka
{भक्षार्थं जातसंरम्भा गर्जन्ती साभ्यधावत}
{आपतन्तीं तु तां दृष्ट्वा अगस्त्यो भगवानृषिः} %1-25-11

\twolineshloka
{राक्षसत्वं भजस्वेति मारीचं व्याजहार सः}
{अगस्त्यः परमामर्षस्ताटकामपि शप्तवान्} %1-25-12

\twolineshloka
{पुरुषादी महायक्षी विकृता विकृतानना}
{इदं रूपम् विहायाशु दारुणं रूपमस्तु ते} %1-25-13

\twolineshloka
{सैषा शापकृतामर्षा ताटका क्रोधमूर्छिता}
{देशमुत्सादयत्येनमगस्त्याचरितं शुभम्} %1-25-14

\twolineshloka
{एनां राघव दुर्वृत्तां यक्षीं परमदारुणाम्}
{गोब्राह्मणहितार्थाय जहि दुष्टपराक्रमाम्} %1-25-15

\twolineshloka
{नह्येनां शापसंसृष्टां कश्चिदुत्सहते पुमान्}
{निहन्तुं त्रिषु लोकेषु त्वामृते रघुनन्दन} %1-25-16

\twolineshloka
{नहि ते स्त्रीवधकृते घृणा कार्या नरोत्तम}
{चातुर्वर्ण्यहितार्थाय कर्तव्यं राजसूनुना} %1-25-17

\twolineshloka
{नृशंसमनृशंसं वा प्रजारक्षणकारणात्}
{पातकं वा सदोषं वा कर्तव्यं रक्षता सदा} %1-25-18

\twolineshloka
{राज्यभारनियुक्तानामेष धर्मः सनातनः}
{अधर्म्यां जहि काकुत्स्थ धर्मो ह्यस्यां न विद्यते} %1-25-19

\twolineshloka
{श्रूयते हि पुरा शक्रो विरोचनसुतां नृप}
{पृथिवीं हन्तुमिच्छन्तीं मन्थरामभ्यसूदयत्} %1-25-20

\twolineshloka
{विष्णुना च पुरा राम भृगुपत्नी पतिव्रता}
{अनिन्द्रं लोकमिच्छन्ती काव्यमाता निषूदिता} %1-25-21

\threelineshloka
{एतैश्चान्यैश्च बहुभी राजपुत्रैर्महात्मभिः}
{अधर्मसहिता नार्यो हताः पुरुषसत्तमैः}
{तस्मादेनां घृणां त्यक्त्वा जहि मच्छासनान्नृप} %1-25-22


॥इत्यार्षे श्रीमद्रामायणे वाल्मीकीये आदिकाव्ये बालकाण्डे ताटकावृत्तान्तः नाम पञ्चविंशः सर्गः ॥१-२५॥
