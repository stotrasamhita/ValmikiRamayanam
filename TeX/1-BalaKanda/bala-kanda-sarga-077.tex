\sect{सप्तसप्ततितमः सर्गः — अयोध्याप्रवेशः}

\twolineshloka
{गते रामे प्रशान्तात्मा रामो दाशरथिर्धनुः}
{वरुणायाप्रमेयाय ददौ हस्ते महायशाः} %1-77-1

\twolineshloka
{अभिवाद्य ततो रामो वसिष्ठप्रमुखानृषीन्}
{पितरं विकलं दृष्ट्वा प्रोवाच रघुनन्दनः} %1-77-2

\twolineshloka
{जामदग्न्यो गतो रामः प्रयातु चतुरंगिणी}
{अयोध्याभिमुखी सेना त्वया नाथेन पालिता} %1-77-3

\twolineshloka
{रामस्य वचनं श्रुत्वा राजा दशरथः सुतम्}
{बाहुभ्यां सम्परिष्वज्य मूर्ध्न्युपाघ्राय राघवम्} %1-77-4

\twolineshloka
{गतो राम इति श्रुत्वा हृष्टः प्रमुदितो नृपः}
{पुनर्जातं तदा मेने पुत्रमात्मानमेव च} %1-77-5

\twolineshloka
{चोदयामास तां सेनां जगामाशु ततः पुरीम्}
{पताकाध्वजिनीं रम्यां तूर्योद्घुष्टनिनादिताम्} %1-77-6

\twolineshloka
{सिक्तराजपथारम्यां प्रकीर्णकुसुमोत्कराम्}
{राजप्रवेशसुमुखैः पौरैर्मङ्गलपाणिभिः} %1-77-7

\twolineshloka
{सम्पूर्णां प्राविशद् राजा जनौघैः समलंकृताम्}
{पौरैः प्रत्युद्गतो दूरं द्विजैश्च पुरवासिभिः} %1-77-8

\twolineshloka
{पुत्रैरनुगतः श्रीमान् श्रीमद्भिश्च महायशाः}
{प्रविवेश गृहं राजा हिमवत्सदृशं प्रियम्} %1-77-9

\twolineshloka
{ननन्द स्वजनै राजा गृहे कामैः सुपूजितः}
{कौसल्या च सुमित्रा च कैकेयी च सुमध्यमा} %1-77-10

\twolineshloka
{वधूप्रतिग्रहे युक्ता याश्चान्या राजयोषितः}
{ततः सीतां महाभागामूर्मिलां च यशस्विनीम्} %1-77-11

\twolineshloka
{कुशध्वजसुते चोभे जगृहुर्नृपयोषितः}
{मंगलालापनैर्होमैः शोभिताः क्षौमवाससः} %1-77-12

\twolineshloka
{देवतायतनान्याशु सर्वास्ताः प्रत्यपूजयन्}
{अभिवाद्याभिवाद्यांश्च सर्वा राजसुतास्तदा} %1-77-13

\twolineshloka
{रेमिरे मुदिताः सर्वा भर्तृभिर्मुदिता रहः}
{कृतदाराः कृतास्त्राश्च सधनाः ससुहृज्जनाः} %1-77-14

\twolineshloka
{शुश्रूषमाणाः पितरं वर्तयन्ति नरर्षभाः}
{कस्यचित्त्वथ कालस्य राजा दशरथः सुतम्} %1-77-15

\twolineshloka
{भरतं कैकयीपुत्रमब्रवीद् रघुनन्दनः}
{अयं केकयराजस्य पुत्रो वसति पुत्रक} %1-77-16

\twolineshloka
{त्वां नेतुमागतो वीरो युधाजिन्मातुलस्तव}
{श्रुत्वा दशरथस्यैतद् भरतः कैकयीसुतः} %1-77-17

\twolineshloka
{गमनायाभिचक्राम शत्रुघ्नसहितस्तदा}
{आपृच्छ्य पितरं शूरो रामं चाक्लिष्टकारिणम्} %1-77-18

\twolineshloka
{मातॄश्चापि नरश्रेष्ठः शत्रुघ्नसहितो ययौ}
{युधाजित् प्राप्य भरतं सशत्रुघ्नं प्रहर्षितः} %1-77-19

\twolineshloka
{स्वपुरं प्राविशद् वीरः पिता तस्य तुतोष ह}
{गते च भरते रामो लक्ष्मणश्च महाबलः} %1-77-20

\twolineshloka
{पितरं देवसंकाशं पूजयामासतुस्तदा}
{पितुराज्ञां पुरस्कृत्य पौरकार्याणि सर्वशः} %1-77-21

\twolineshloka
{चकार रामः सर्वाणि प्रियाणि च हितानि च}
{मातृभ्यो मातृकार्याणि कृत्वा परमयन्त्रितः} %1-77-22

\twolineshloka
{गुरूणां गुरुकार्याणि काले कालेऽन्ववैक्षत}
{एवं दशरथः प्रीतो ब्राह्मणा नैगमास्तथा} %1-77-23

\twolineshloka
{रामस्य शीलवृत्तेन सर्वे विषयवासिनः}
{तेषामतियशा लोके रामः सत्यपराक्रमः} %1-77-24

\twolineshloka
{स्वयंभूरिव भूतानां बभूव गुणवत्तरः}
{रामश्च सीतया सार्धं विजहार बहूनृतून्} %1-77-25

\twolineshloka
{मनस्वी तद्गतमनास्तस्या हृदि समर्पितः}
{प्रिया तु सीता रामस्य दाराः पितृकृता इति} %1-77-26

\twolineshloka
{गुणाद्रूपगुणाच्चापि प्रीतिर्भूयोऽभिवर्धते}
{तस्याश्च भर्ता द्विगुणं हृदये परिवर्तते} %1-77-27

\threelineshloka
{अन्तर्गतमपि व्यक्तमाख्याति हृदयं हृदा}
{तस्य भूयो विशेषेण मैथिली जनकात्मजा}
{देवताभिः समा रूपे सीता श्रीरिव रूपिणी} %1-77-28

\twolineshloka
{तया स राजर्षिसुतोऽभिकामया समेयिवानुत्तमराजकन्यया}
{अतीव रामः शुशुभे मुदान्वितो विभुः श्रिया विष्णुरिवामरेश्वरः} %1-77-29


॥इत्यार्षे श्रीमद्रामायणे वाल्मीकीये आदिकाव्ये बालकाण्डे अयोध्याप्रवेशः नाम सप्तसप्ततितमः सर्गः ॥१-७७॥
