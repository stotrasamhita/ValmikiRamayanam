\sect{एकादशः सर्गः — ऋष्यशृङ्गस्यायोध्याप्रवेशः}

\twolineshloka
{भूय एव हि राजेन्द्र शृणु मे वचनं हितम्}
{यथा स देवप्रवरः कथयामास बुद्धिमान्} %1-11-1

\twolineshloka
{इक्ष्वाकूणां कुले जातो भविष्यति सुधार्मिकः}
{नाम्ना दशरथो राजा श्रीमान् सत्यप्रतिश्रवः} %1-11-2

\twolineshloka
{अङ्गराजेन सख्यं च तस्य राज्ञो भविष्यति}
{कन्या चास्य महाभागा शान्ता नाम भविष्यति} %1-11-3

\twolineshloka
{पुत्रस्त्वङ्गस्य राज्ञस्तु रोमपाद इति श्रुतः}
{तं स राजा दशरथो गमिष्यति महायशाः} %1-11-4

\twolineshloka
{अनपत्योऽस्मि धर्मात्मञ्छान्ताभर्ता मम क्रतुम्}
{आहरेत त्वयाज्ञप्तः सन्तानार्थं कुलस्य च} %1-11-5

\twolineshloka
{श्रुत्वा राज्ञोऽथ तद् वाक्यं मनसा स विचिन्त्य च}
{प्रदास्यते पुत्रवन्तं शान्ता भर्तारमात्मवान्} %1-11-6

\twolineshloka
{प्रतिगृह्य च तं विप्रं स राजा विगतज्वरः}
{आहरिष्यति तं यज्ञं प्रहृष्टेनान्तरात्मना} %1-11-7

\twolineshloka
{तं च राजा दशरथो यशस्कामः कृताञ्जलिः}
{ऋष्यशृङ्गं द्विजश्रेष्ठं वरयिष्यति धर्मवित्} %1-11-8

\twolineshloka
{यज्ञार्थं प्रसवार्थं च स्वर्गार्थं च नरेश्वरः}
{लभते च स तं कामं द्विजमुख्याद् विशाम्पतिः} %1-11-9

\twolineshloka
{पुत्राश्चास्य भविष्यन्ति चत्वारोऽमितविक्रमाः}
{वंशप्रतिष्ठानकराः सर्वभूतेषु विश्रुताः} %1-11-10

\twolineshloka
{एवं स देवप्रवरः पूर्वं कथितवान् कथाम्}
{सनत्कुमारो भगवान् पुरा देवयुगे प्रभुः} %1-11-11

\twolineshloka
{स त्वं पुरुषशार्दूल समानय सुसत्कृतम्}
{स्वयमेव महाराज गत्वा सबलवाहनः} %1-11-12

\twolineshloka
{सुमन्त्रस्य वचः श्रुत्वा हृष्टो दशरथोऽभवत्}
{अनुमान्य वसिष्ठं च सूतवाक्यं निशाम्य च} %1-11-13

\twolineshloka
{सान्तःपुरः सहामात्यः प्रययौ यत्र स द्विजः}
{वनानि सरितश्चैव व्यतिक्रम्य शनैः शनैः} %1-11-14

\twolineshloka
{अभिचक्राम तं देशं यत्र वै मुनिपुङ्गवः}
{आसाद्य तं द्विजश्रेष्ठं रोमपादसमीपगम्} %1-11-15

\twolineshloka
{ऋषिपुत्रं ददर्शाथो दीप्यमानमिवानलम्}
{ततो राजा यथायोग्यं पूजां चक्रे विशेषतः} %1-11-16

\twolineshloka
{सखित्वात्तस्य वै राज्ञः प्रहृष्टेनान्तरात्मना}
{रोमपादेन चाख्यातमृषिपुत्राय धीमते} %1-11-17

\twolineshloka
{सख्यं सम्बन्धकं चैव तदा तं प्रत्यपूजयत्}
{एवं सुसत्कृतस्तेन सहोषित्वा नरर्षभः} %1-11-18

\twolineshloka
{सप्ताष्टदिवसान् राजा राजानमिदमब्रवीत्}
{शान्ता तव सुता राजन् सह भर्त्रा विशाम्पते} %1-11-19

\twolineshloka
{मदीयं नगरं यातु कार्यं हि महदुद्यतम्}
{तथेति राजा संश्रुत्य गमनं तस्य धीमतः} %1-11-20

\twolineshloka
{उवाच वचनं विप्रं गच्छ त्वं सह भार्यया}
{ऋषिपुत्रः प्रतिश्रुत्य तथेत्याह नृपं तदा} %1-11-21

\twolineshloka
{स नृपेणाभ्यनुज्ञातः प्रययौ सह भार्यया}
{तावन्योन्याञ्जलिं कृत्वा स्नेहात्संश्लिष्य चोरसा} %1-11-22

\twolineshloka
{ननन्दतुर्दशरथो रोमपादश्च वीर्यवान्}
{ततः सुहृदमापृच्छ्य प्रस्थितो रघुनन्दनः} %1-11-23

\twolineshloka
{पौरेषु प्रेषयामास दूतान् वै शीघ्रगामिनः}
{क्रियतां नगरं सर्वं क्षिप्रमेव स्वलङ्कृतम्} %1-11-24

\twolineshloka
{धूपितं सिक्तसम्मृष्टं पताकाभिरलङ्कृतम्}
{ततः प्रहृष्टाः पौरास्ते श्रुत्वा राजानमागतम्} %1-11-25

\twolineshloka
{तथा चक्रुश्च तत् सर्वं राज्ञा यत् प्रेषितं तदा}
{ततः स्वलङ्कृतं राजा नगरं प्रविवेश ह} %1-11-26

\twolineshloka
{शङ्खदुन्दुभिनिर्ह्रादैः पुरस्कृत्वा द्विजर्षभम्}
{ततः प्रमुदिताः सर्वे दृष्ट्वा वै नागरा द्विजम्} %1-11-27

\twolineshloka
{प्रवेश्यमानं सत्कृत्य नरेन्द्रेणेन्द्रकर्मणा}
{यथा दिवि सुरेन्द्रेण सहस्राक्षेण काश्यपम्} %1-11-28

\twolineshloka
{अन्तःपुरं प्रवेश्यैनं पूजां कृत्वा तु शास्त्रतः}
{कृतकृत्यं तदात्मानं मेने तस्योपवाहनात्} %1-11-29

\twolineshloka
{अन्तःपुराणि सर्वाणि शान्तां दृष्ट्वा तथागताम्}
{सह भर्त्रा विशालाक्षीं प्रीत्यानन्दमुपागमन्} %1-11-30

\twolineshloka
{पूज्यमाना तु ताभिः सा राज्ञा चैव विशेषतः}
{उवास तत्र सुखिता कञ्चित् कालं सहद्विजा} %1-11-31


॥इत्यार्षे श्रीमद्रामायणे वाल्मीकीये आदिकाव्ये बालकाण्डे ऋष्यशृङ्गस्यायोध्याप्रवेशः नाम एकादशः सर्गः ॥१-११॥
