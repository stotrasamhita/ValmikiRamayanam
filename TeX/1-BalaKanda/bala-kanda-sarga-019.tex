\sect{एकोनविंशः सर्गः — विश्वामित्रवाक्यम्}

\twolineshloka
{तच्छ्रुत्वा राजसिंहस्य वाक्यमद्भुतविस्तरम्}
{हृष्टरोमा महातेजा विश्वामित्रोऽभ्यभाषत} %1-19-1

\twolineshloka
{सदृशं राजशार्दूल तवैवं भुवि नान्यतः}
{महावंशप्रसूतस्य वसिष्ठव्यपदेशिनः} %1-19-2

\twolineshloka
{यत्तु मे हृद्गतं वाक्यं तस्य कार्यस्य निश्चयम्}
{कुरुष्व राजशार्दूल भव सत्यप्रतिश्रवः} %1-19-3

\twolineshloka
{अहं नियममातिष्ठे सिद्ध्यर्थं पुरुषर्षभ}
{तस्य विघ्नकरौ द्वौ तु राक्षसौ कामरूपिणौ} %1-19-4

\twolineshloka
{व्रते तु बहुशश्चीर्णे समाप्त्यां राक्षसाविमौ}
{मारीचश्च सुबाहुश्च वीर्यवन्तौ सुशिक्षितौ} %1-19-5

\twolineshloka
{तौ मांसरुधिरौघेण वेदिं तामभ्यवर्षताम्}
{अवधूते तथाभूते तस्मिन् नियमनिश्चये} %1-19-6

\twolineshloka
{कृतश्रमो निरुत्साहस्तस्माद् देशादपाक्रमे}
{न च मे क्रोधमुत्स्रष्टुं बुद्धिर्भवति पार्थिव} %1-19-7

\twolineshloka
{तथाभूता हि सा चर्या न शापस्तत्र मुच्यते}
{स्वपुत्रं राजशार्दूल रामं सत्यपराक्रमम्} %1-19-8

\twolineshloka
{काकपक्षधरं शूरं ज्येष्ठं मे दातुमर्हसि}
{शक्तो ह्येष मया गुप्तो दिव्येन स्वेन तेजसा} %1-19-9

\twolineshloka
{राक्षसा ये विकर्तारस्तेषामपि विनाशने}
{श्रेयश्चास्मै प्रदास्यामि बहुरूपं न संशयः} %1-19-10

\twolineshloka
{त्रयाणामपि लोकानां येन ख्यातिं गमिष्यति}
{न च तौ राममासाद्य शक्तौ स्थातुं कथञ्चन} %1-19-11

\twolineshloka
{न च तौ राघवादन्यो हन्तुमुत्सहते पुमान्}
{वीर्योत्सिक्तौ हि तौ पापौ कालपाशवशं गतौ} %1-19-12

\twolineshloka
{रामस्य राजशार्दूल न पर्याप्तौ महात्मनः}
{न च पुत्रगतं स्नेहं कर्तुमर्हसि पार्थिव} %1-19-13

\twolineshloka
{अहं ते प्रतिजानामि हतौ तौ विद्धि राक्षसौ}
{अहं वेद्मि महात्मानं रामं सत्यपराक्रमम्} %1-19-14

\twolineshloka
{वसिष्ठोऽपि महातेजा ये चेमे तपसि स्थिताः}
{यदि ते धर्मलाभं च यशश्च परमं भुवि} %1-19-15

\twolineshloka
{स्थिरमिच्छसि राजेन्द्र रामं मे दातुमर्हसि}
{यद्यभ्यनुज्ञां काकुत्स्थ ददते तव मन्त्रिणः} %1-19-16

\twolineshloka
{वसिष्ठप्रमुखाः सर्वे ततो रामं विसर्जय}
{अभिप्रेतमसंसक्तमात्मजं दातुमर्हसि} %1-19-17

\twolineshloka
{दशरात्रं हि यज्ञस्य रामं राजीवलोचनम्}
{नात्येति कालो यज्ञस्य यथायं मम राघव} %1-19-18

\twolineshloka
{तथा कुरुष्व भद्रं ते मा च शोके मनः कृथाः}
{इत्येवमुक्त्वा धर्मात्मा धर्मार्थसहितं वचः} %1-19-19

\twolineshloka
{विरराम महातेजा विश्वामित्रो महामतिः}
{स तन्निशम्य राजेन्द्रो विश्वामित्रवचः शुभम्} %1-19-20

\twolineshloka
{शोकेन महताविष्टश्चचाल च मुमोह च}
{लब्धसंज्ञस्तदोत्थाय व्यषीदत भयान्वितः} %1-19-21

\twolineshloka
{इति हृदयमनोविदारणं मुनिवचनं तदतीव शुश्रुवान्}
{नरपतिरभवन्महान् महात्मा व्यथितमनाः प्रचचाल चासनात्} %1-19-22


॥इत्यार्षे श्रीमद्रामायणे वाल्मीकीये आदिकाव्ये बालकाण्डे विश्वामित्रवाक्यम् नाम एकोनविंशः सर्गः ॥१-१९॥
