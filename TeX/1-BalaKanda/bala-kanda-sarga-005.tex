\sect{पञ्चमः सर्गः — अयोध्यावर्णनम्}

\twolineshloka
{सर्वा पूर्वमियं येषामासीत् कृत्स्ना वसुंधरा}
{प्रजापतिमुपादाय नृपाणां जयशालिनाम्} %1-5-1

\twolineshloka
{येषां स सगरो नाम सागरो येन खानितः}
{षष्टिपुत्रसहस्राणि यं यान्तं पर्यवारयन्} %1-5-2

\twolineshloka
{इक्ष्वाकूणामिदं तेषां राज्ञां वंशे महात्मनाम्}
{महदुत्पन्नमाख्यानं रामायणमिति श्रुतम्} %1-5-3

\twolineshloka
{तदिदं वर्तयिष्यावः सर्वं निखिलमादितः}
{धर्मकामार्थसहितं श्रोतव्यमनसूयता} %1-5-4

\twolineshloka
{कोसलो नाम मुदितः स्फीतो जनपदो महान्}
{निविष्टः सरयूतीरे प्रभूतधनधान्यवान्} %1-5-5

\twolineshloka
{अयोध्या नाम नगरी तत्रासील्लोकविश्रुता}
{मनुना मानवेन्द्रेण या पुरी निर्मिता स्वयम्} %1-5-6

\twolineshloka
{आयता दश च द्वे च योजनानि महापुरी}
{श्रीमती त्रीणि विस्तीर्णा सुविभक्तमहापथा} %1-5-7

\twolineshloka
{राजमार्गेण महता सुविभक्तेन शोभिता}
{मुक्तपुष्पावकीर्णेन जलसिक्तेन नित्यशः} %1-5-8

\twolineshloka
{तां तु राजा दशरथो महाराष्ट्रविवर्धनः}
{पुरीमावासयामास दिवि देवपतिर्यथा} %1-5-9

\twolineshloka
{कवाटतोरणवतीं सुविभक्तान्तरापणाम्}
{सर्वयन्त्रायुधवतीमुषितां सर्वशिल्पिभिः} %1-5-10

\twolineshloka
{सूतमागधसम्बाधां श्रीमतीमतुलप्रभाम्}
{उच्चाट्टालध्वजवतीं शतघ्नीशतसंकुलाम्} %1-5-11

\twolineshloka
{वधूनाटकसङ्घैश्च संयुक्तां सर्वतः पुरीम्}
{उद्यानाम्रवणोपेतां महतीं सालमेखलाम्} %1-5-12

\twolineshloka
{दुर्गगम्भीरपरिखां दुर्गामन्यैर्दुरासदाम्}
{वाजिवारणसम्पूर्णां गोभिरुष्ट्रैः खरैस्तथा} %1-5-13

\twolineshloka
{सामन्तराजसङ्घैश्च बलिकर्मभिरावृताम्}
{नानादेशनिवासैश्च वणिग्भिरुपशोभिताम्} %1-5-14

\twolineshloka
{प्रासादैर्रत्नविकृतैः पर्वतैरिव शोभिताम्}
{कूटागारैश्च सम्पूर्णामिन्द्रस्येवामरावतीम्} %1-5-15

\twolineshloka
{चित्रामष्टापदाकारां वरनारीगणायुताम्}
{सर्वरत्नसमाकीर्णां विमानगृहशोभिताम्} %1-5-16

\twolineshloka
{गृहगाढामविच्छिद्रां समभूमौ निवेशिताम्}
{शालितण्डुलसम्पूर्णामिक्षुकाण्डरसोदकाम्} %1-5-17

\twolineshloka
{दुन्दुभीभिर्मृदङ्गैश्च वीणाभिः पणवैस्तथा}
{नादितां भृशमत्यर्थं पृथिव्यां तामनुत्तमाम्} %1-5-18

\twolineshloka
{विमानमिव सिद्धानां तपसाधिगतं दिवि}
{सुनिवेशितवेश्मान्तां नरोत्तमसमावृताम्} %1-5-19

\twolineshloka
{ये च बाणैर्न विध्यन्ति विविक्तमपरापरम्}
{शब्दवेध्यं च विततं लघुहस्ता विशारदाः} %1-5-20

\twolineshloka
{सिंहव्याघ्रवराहाणां मत्तानां नदतां वने}
{हन्तारो निशितैः शस्त्रैर्बलाद् बाहुबलैरपि} %1-5-21

\twolineshloka
{तादृशानां सहस्रैस्तामभिपूर्णां महारथैः}
{पुरीमावासयामास राजा दशरथस्तदा} %1-5-22

\twolineshloka
{तामग्निमद्भिर्गुणवद्भिरावृतां द्विजोत्तमैर्वेदषडङ्गपारगैः}
{सहस्रदैः सत्यरतैर्महात्मभिर्महर्षिकल्पैर्ऋषिभिश्च केवलैः} %1-5-23


॥इत्यार्षे श्रीमद्रामायणे वाल्मीकीये आदिकाव्ये बालकाण्डे अयोध्यावर्णनम् नाम पञ्चमः सर्गः ॥१-५॥
