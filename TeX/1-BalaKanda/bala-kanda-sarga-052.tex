\sect{द्विपञ्चाशः सर्गः — वसिष्ठातिथ्यम्}

\twolineshloka
{तं दृष्ट्वा परमप्रीतो विश्वामित्रो महाबलः}
{प्रणतो विनयाद् वीरो वसिष्ठं जपतां वरम्} %1-52-1

\twolineshloka
{स्वागतं तव चेत्युक्तो वसिष्ठेन महात्मना}
{आसनं चास्य भगवान् वसिष्ठो व्यादिदेश ह} %1-52-2

\twolineshloka
{उपविष्टाय च तदा विश्वामित्राय धीमते}
{यथान्यायं मुनिवरः फलमूलमुपाहरत्} %1-52-3

\twolineshloka
{प्रतिगृह्य तु तां पूजां वसिष्ठाद् राजसत्तमः}
{तपोऽग्निहोत्रशिष्येषु कुशलं पर्यपृच्छत} %1-52-4

\twolineshloka
{विश्वामित्रो महातेजा वनस्पतिगणे तदा}
{सर्वत्र कुशलं प्राह वसिष्ठो राजसत्तमम्} %1-52-5

\twolineshloka
{सुखोपविष्टं राजानं विश्वामित्रं महातपाः}
{पप्रच्छ जपतां श्रेष्ठो वसिष्ठो ब्रह्मणः सुतः} %1-52-6

\twolineshloka
{कच्चित्ते कुशलं राजन् कच्चिद् धर्मेण रञ्जयन्}
{प्रजाः पालयसे राजन् राजवृत्तेन धार्मिक} %1-52-7

\twolineshloka
{कच्चित्ते सम्भृता भृत्याः कच्चित् तिष्ठन्ति शासने}
{कच्चित्ते विजिताः सर्वे रिपवो रिपुसूदन} %1-52-8

\twolineshloka
{कच्चिद् बलेषु कोशेषु मित्रेषु च परन्तप}
{कुशलं ते नरव्याघ्र पुत्रपौत्रे तथानघ} %1-52-9

\twolineshloka
{सर्वत्र कुशलं राजा वसिष्ठं प्रत्युदाहरत्}
{विश्वामित्रो महातेजा वसिष्ठं विनयान्वितम्} %1-52-10

\twolineshloka
{कृत्वा तौ सुचिरं कालं धर्मिष्ठौ ताः कथास्तदा}
{मुदा परमया युक्तौ प्रीयेतां तौ परस्परम्} %1-52-11

\twolineshloka
{ततो वसिष्ठो भगवान् कथान्ते रघुनन्दन}
{विश्वामित्रमिदं वाक्यमुवाच प्रहसन्निव} %1-52-12

\twolineshloka
{आतिथ्यं कर्तुमिच्छामि बलस्यास्य महाबल}
{तव चैवाप्रमेयस्य यथार्हं सम्प्रतीच्छ मे} %1-52-13

\twolineshloka
{सत्क्रियां हि भवानेतां प्रतीच्छतु मया कृताम्}
{राजंस्त्वमतिथिश्रेष्ठः पूजनीयः प्रयत्नतः} %1-52-14

\twolineshloka
{एवमुक्तो वसिष्ठेन विश्वामित्रो महामतिः}
{कृतमित्यब्रवीद् राजा पूजावाक्येन मे त्वया} %1-52-15

\twolineshloka
{फलमूलेन भगवन् विद्यते यत् तवाश्रमे}
{पाद्येनाचमनीयेन भगवद्दर्शनेन च} %1-52-16

\twolineshloka
{सर्वथा च महाप्राज्ञ पूजार्हेण सुपूजितः}
{नमस्तेऽस्तु गमिष्यामि मैत्रेणेक्षस्व चक्षुषा} %1-52-17

\twolineshloka
{एवं ब्रुवन्तं राजानं वसिष्ठं पुनरेव हि}
{न्यमन्त्रयत धर्मात्मा पुनः पुनरुदारधीः} %1-52-18

\twolineshloka
{बाढमित्येव गाधेयो वसिष्ठं प्रत्युवाच ह}
{यथाप्रियं भगवतस्तथास्तु मुनिपुङ्गव} %1-52-19

\twolineshloka
{एवमुक्तस्तथा तेन वसिष्ठो जपतां वरः}
{आजुहाव ततः प्रीतः कल्माषीं धूतकल्मषाम्} %1-52-20

\threelineshloka
{एह्येहि शबले क्षिप्रं शृणु चापि वचो मम}
{सबलस्यास्य राजर्षेः कर्तुं व्यवसितोऽस्म्यहम्}
{भोजनेन महार्हेण सत्कारं संविधत्स्व मे} %1-52-21

\twolineshloka
{यस्य यस्य यथाकामं षड्रसेष्वभिपूजितम्}
{तत् सर्वं कामधुग् दिव्ये अभिवर्ष कृते मम} %1-52-22

\twolineshloka
{रसेनान्नेन पानेन लेह्यचोष्येण संयुतम्}
{अन्नानां निचयं सर्वं सृजस्व शबले त्वर} %1-52-23


॥इत्यार्षे श्रीमद्रामायणे वाल्मीकीये आदिकाव्ये बालकाण्डे वसिष्ठातिथ्यम् नाम द्विपञ्चाशः सर्गः ॥१-५२॥
